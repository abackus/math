\documentclass[12pt]{report}
\usepackage[utf8]{inputenc}
\usepackage[margin=1in]{geometry} 
\usepackage{amsmath,amsthm,amssymb}
\usepackage{mathrsfs}

\usepackage{enumitem}
%\usepackage[shortlabels]{enumerate}
\usepackage{tikz-cd}
\usepackage{mathtools}
\usepackage{amsfonts}
\usepackage{amscd}
\usepackage{makeidx}
\usepackage{enumitem}

\title{Math 185: Complex Analysis}
\author{Michael Klass}
\date{Fall 2019}

\newcommand{\NN}{\mathbf{N}}
\newcommand{\ZZ}{\mathbf{Z}}
\newcommand{\QQ}{\mathbf{Q}}
\newcommand{\RR}{\mathbf{R}}
\newcommand{\CC}{\mathbf{C}}

\newcommand{\pic}{\vspace{30mm}}
\newcommand{\dfn}[1]{\emph{#1}\index{#1}}

\renewcommand{\Re}{\operatorname{Re}}
\renewcommand{\Im}{\operatorname{Im}}

\newtheorem{theorem}{Theorem}[chapter]
\newtheorem{badtheorem}[theorem]{``Theorem"}
\newtheorem{prop}[theorem]{Proposition}
\newtheorem{lemma}[theorem]{Lemma}
\newtheorem{proposition}[theorem]{Proposition}
\newtheorem{corollary}[theorem]{Corollary}
\newtheorem{conjecture}[theorem]{Conjecture}
\newtheorem{axiom}[theorem]{Axiom}

\theoremstyle{definition}
\newtheorem{definition}[theorem]{Definition}
\newtheorem{remark}[theorem]{Remark}
\newtheorem{example}[theorem]{Example}

\theoremstyle{remark}
\newtheorem{exercise}[theorem]{Discussion topic}
\newtheorem{homework}[theorem]{Homework}
\newtheorem{problem}[theorem]{Problem}

\usepackage{color}
\usepackage{hyperref}
\hypersetup{
    colorlinks=true, % make the links colored
    linkcolor=blue, % color TOC links in blue
    urlcolor=red, % color URLs in red
    linktoc=all % 'all' will create links for everything in the TOC
    %Ning added hyperlinks to the table of contents 6/17/19
}

\makeindex
\begin{document}

\maketitle

\tableofcontents

\newpage


\chapter{Special relativity}
We start with the two experimentally verifiable axioms of special relativity.

\begin{definition}
    A \dfn{reference frame} is a coordinate system for $\RR^4 = \RR \times \RR^3$. A reference frame is said to be \dfn{inertial} if the motion of a body without external influence forms a straight line in $\RR^4$. Otherwise, the reference frame is said to be \dfn{accelerated}.
\end{definition}
\begin{axiom}
    All laws of physics are invariant under change of inertial reference frame.
\end{axiom}
\begin{axiom}
    The speed of light in a vacuum is invariant under change of inertial reference frame.
\end{axiom}
	We denote the speed of light in a vacuum by $c$.

\section{Lorentz transformations}
\begin{definition}
    A \dfn{Lorentz transformation} is a smooth transformation which fixes the origin and is homotopic to the identity, which carries an inertial reference frame to an inertial reference frame.
\end{definition}
\begin{definition}
	Let $p = (t, x), q = (t', x') \in \RR^4$. The \dfn{spacetime interval} $\Delta s = [p, q]$ between $p, q$ is the distance
	$$\Delta s^2 = \Delta x^2 - c^2\Delta t^2$$
	where $\Delta t = t' - t$ and $\Delta x = x' - x$.
\end{definition}
It is not hard to check that Lorentz transformations are linear (since they preserve straight-line trajectories). Moreover, spacetime intervals are also preserved by Lorentz transformations. By the second axiom of relativity, the quadratic polynomials associated to $\Delta s$ and its Lorentz transform, say $\Delta s'$, have the same roots. So there is an $\alpha \neq 0$ such that
$$(\Delta s)^2 = \alpha (\Delta s')^2.$$
Moreover, this constant appears for any choice of $s, s'$, so by ``reciprocity", $\alpha^2 = 1$. Since Lorentz transformations are homotopic to the identity, which clearly has $\alpha > 0$, we have $\alpha = 1$. Therefore the claim holds.

\begin{example}[twin paradox]
\index{twin paradox}
Let $A, B$ be two twins born in space. They are separated at birth (spacetime $P$), and $B$ moves away from $A$ but then suddenly turns around (at spacetime $Q$) and meets $A$ again (at spacetime $R$). Then it appears that both $A$ is older than $B$ (from the point of view of $A$) and $B$ is older than $A$ (from the point of view of $B$). However, one can check that in fact $A$ is older than $B$, since $B$ had an accelerated reference frame (when he turned around at $Q$) and so has an incorrect perception of the universe. This can be checked using the spacetime interval invariance.
\end{example}

Let us consider the Lorentzian metric
$$ds^2 = dx^2 - c^2 dt^2,$$
where as usual we write $s = (t, x) \in \RR \times \RR^3 = \RR^4$ for a point in spacetime. This is a linear combination of the Riemannian metrics $dx^2$ and $dt^2$. We will also write $m$ for the indefinite quadratic form induced by $ds^2$ on the tangent bundle. On the other hand we will write $\delta$ for the positive-definite quadratic form induced by the Riemannian metric $dx^2$. Therefore $(\RR^3, \delta)$ is just Euclidean space.
\begin{definition}
	A \dfn{Lorentzian manifold} is a smooth manifold equipped with a smoothly varying quadratic form on each tangent space. The Lorentzian manifold $(\RR^4, m)$ is known as \dfn{Minkowski spacetime}.
\end{definition}

Now let $\gamma$ be a curve in $\RR^4$, which we think of as parametrized by $[0, 1]$. We denote the tangent vector by $\dot \gamma$.
\begin{definition}
The \dfn{proper time} of the curve $\gamma$ is
$$\int_0^1 \frac{\sqrt{-m(\dot \gamma(\sigma), \dot \gamma(\sigma))}{c}} ~d\sigma.$$
\begin{definition}
	Let $v$ be a tangent vector over $\RR^4$. If $m(v, v) < 0$, we say that $v$ is \dfn{timelike}. If $m(v, v) = 0$, then $v$ is \dfn{lightlike} or \dfn{null}. Otherwise, $v$ is \dfn{spacelike}. If $v$ is not spacelike, we say that $v$ is \dfn{causal}. If every tangent vector to a curve is timelike (lightlike, etc.), we say that the curve itself is timelike (lightlike, etc.)
\end{definition}
Notice that a vector $v$ has speed $\leq c$ iff $v$ is causal. So causal curves are those trajectories of objects which are allowed by the laws of physics.



\printindex

\end{document}
