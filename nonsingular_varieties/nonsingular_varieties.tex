\documentclass[reqno,12pt,letterpaper]{amsart}
\RequirePackage{amsmath,amssymb,amsthm,graphicx,mathrsfs,url}
\RequirePackage[usenames,dvipsnames]{color}
\RequirePackage[colorlinks=true,linkcolor=Red,citecolor=Green]{hyperref}
\RequirePackage{amsxtra}
\usepackage{tikz-cd}

\setlength{\textheight}{8.50in} \setlength{\oddsidemargin}{0.00in}
\setlength{\evensidemargin}{0.00in} \setlength{\textwidth}{6.08in}
\setlength{\topmargin}{0.00in} \setlength{\headheight}{0.18in}
\setlength{\marginparwidth}{1.0in}
\setlength{\abovedisplayskip}{0.2in}
\setlength{\belowdisplayskip}{0.2in}
\setlength{\parskip}{0.05in}
\renewcommand{\baselinestretch}{1.10}

\title{Nonsingular curves}
\author{Aidan Backus}
\date{October 2021}

\newcommand{\NN}{\mathbf{N}}
\newcommand{\ZZ}{\mathbf{Z}}
\newcommand{\QQ}{\mathbf{Q}}
\newcommand{\RR}{\mathbf{R}}
\newcommand{\CC}{\mathbf{C}}
\newcommand{\DD}{\mathbf{D}}
\newcommand{\PP}{\mathbf P}
\newcommand{\MM}{\mathbf M}
\newcommand{\Aff}{\mathbf A}

\DeclareMathOperator{\card}{card}
\DeclareMathOperator{\ch}{ch}
\DeclareMathOperator{\diag}{diag}
\DeclareMathOperator{\dom}{dom}
\DeclareMathOperator{\Gal}{Gal}
\DeclareMathOperator{\id}{id}
\DeclareMathOperator{\rank}{rank}
\DeclareMathOperator*{\Res}{Res}
\DeclareMathOperator{\sgn}{sgn}
\DeclareMathOperator{\singsupp}{sing~supp}
\DeclareMathOperator{\Spec}{Spec}
\DeclareMathOperator{\supp}{supp}
\newcommand{\tr}{\operatorname{tr}}

\newcommand{\dbar}{\overline \partial}

\DeclareMathOperator{\atanh}{atanh}
\DeclareMathOperator{\csch}{csch}
\DeclareMathOperator{\sech}{sech}

\DeclareMathOperator{\Ell}{Ell}
\DeclareMathOperator{\WF}{WF}

\newcommand{\pic}{\vspace{30mm}}
\newcommand{\dfn}[1]{\emph{#1}\index{#1}}

\renewcommand{\Re}{\operatorname{Re}}
\renewcommand{\Im}{\operatorname{Im}}
\newcommand{\Olo}{\mathscr O}
\newcommand{\Mero}{\mathscr M}
\newcommand{\nonsingular}{\mathscr E}
\newcommand{\Test}{\mathscr D}


\newtheorem{theorem}{Theorem}[section]
\newtheorem{badtheorem}[theorem]{``Theorem"}
\newtheorem{prop}[theorem]{Proposition}
\newtheorem{lemma}[theorem]{Lemma}
\newtheorem{proposition}[theorem]{Proposition}
\newtheorem{corollary}[theorem]{Corollary}
\newtheorem{conjecture}[theorem]{Conjecture}
\newtheorem{axiom}[theorem]{Axiom}

\theoremstyle{definition}
\newtheorem{definition}[theorem]{Definition}
\newtheorem{remark}[theorem]{Remark}
\newtheorem{example}[theorem]{Example}

\newtheorem{exercise}[theorem]{Discussion topic}
\newtheorem{homework}[theorem]{Homework}
\newtheorem{problem}[theorem]{Problem}

\newtheorem*{ack}{Acknowledgements}
\newtheorem*{notate}{Notation}

%\usepackage{color}
%\hypersetup{%
%    colorlinks=true, % make the links colored%
%    linkcolor=blue, % color TOC links in blue
%    urlcolor=red, % color URLs in red
%    linktoc=all % 'all' will create links for everything in the TOC
%Ning added hyperlinks to the table of contents 6/17/19
%}

\usepackage[backend=bibtex,style=alphabetic,maxcitenames=50,maxnames=50]{biblatex}
\addbibresource{nonsingular_varieties.bib}
\renewbibmacro{in:}{}
\DeclareFieldFormat{pages}{#1}

\begin{document}
\begin{abstract}
We show that every curve is birational to a nonsingular projective curve.
This follows Hartshorne, Chapter 1.
\end{abstract}

\maketitle


\section{Introduction}
In complex analysis last semester (not the class with the ``quantum states" nonsense), we proved a weaker form of the following theorem:

\begin{theorem}[classification of nonsingular projective curves]
Let $X$ be a nonsingular projective curve over $k$.
There is an invariant $g \in \NN$ of $X$, called the \emph{genus} of $X$, such that:
\begin{enumerate}
\item If $g = 0$, then $X$ is isomorphic to $\PP^1$.
\item If $g = 1$, then $X$ is an elliptic curve $\{y^2 = f(x)\}$ where $f$ is cubic with nonzero discriminant. The moduli space of elliptic curves is $k$.
\item There is an explicit threefold which is the moduli space of curves such that $g = 2$.
\item etc...
\end{enumerate}
Moreover,
$$g = \dim \Gamma(X, \Omega^1) = \dim H^1(X, \Olo_X) = \dim H^1_{sing}(X^{an}, \CC)/2.$$
\end{theorem}

We're going to define nonsingularity in a second.
See Chapter IV of Hartshorne for a proof.

The goal of this talk is to show:

\begin{theorem}
In every birational class $[X]$ of curves there exists a unique nonsingular projective curve $X$.
In fact, every nonsingular curve $Y$ birational to $X$ embeds in $X$.
\end{theorem}

Assuming this is true for a moment, the classification of nonsingular projective curves is actually a classification of all curves modulo birationality.
Moreover, the classification of nonsingular projective curves is just countably many little classification problems, one for each genus.

\section{Nonsingular varieties}
Here's some general facts about nonsingular varieties.

Let $x_0$ be a point on a variety $X$ and let $A$ be its local ring, with maximal ideal $I$.
Then $I$ is the space of functions on $X$ with a zero at $x_0$ and $I^2$ is the space of functions on $X$ with a double zero at $x_0$.
If $f \in I$ then we can write
$$f(x) = df_{x_0} \cdot (x - x_0) + \frac{f''(x_0)}{2}(x - x_0)^2 + \cdots$$
(where $f''(x)$ is the Hessian tensor)
but the quadratic and up terms are elements of $I/I^2$, so the image of $f$ in $I^2$ is $df_{x_0}$. That is:

\begin{definition}
The \dfn{cotangent space} to $X$ at $x_0$ is the vector space $I/I^2$ defined above.
\end{definition}

So elements of $I/I^2$ are covectors based at $x_0$, thus $I/I^2$ is the cotangent space to $X$ at $x_0$.
According to commutative algebra,
$$\dim \frac{I}{I^2} \geq \dim A$$
where the left-hand side is in the sense of vector spaces and the right hand is in the sense of Krull.

\begin{example}
Suppose we have a self-intersecting curve, such as $\{y^2 = x^2(x + 1)\}$.
Any point has codimension $1$, so every local ring has dimension $1$.
But at the intersection point $(0, 0)$, we get $I = (x, y)/(y^2 - x^2(x + 1))$ and $I^2 = (x^2, xy, y^2)/(y^2 - x^2(x + 1)) = (x^2, xy, x^2(x + 1))$.
Thus $I/I^2$ is spanned by $x, y$ and so has dimension $2$!
Intuitively this means that the tangent space to $(0, 0)$ is the entire plane, which doesn't seem right.
\end{example}

In order to avoid singularities like this, we declare that the cotangent space to a variety should have the same dimension as a variety itself, or in other words:

\begin{definition}
A variety is \dfn{nonsingular} if every local ring is a regular local ring.
\end{definition}

A variety is nonsingular iff it is cut out by a regular map whose Jacobian matrix has maximal rank.

\section{Curves of DVRs}
Let $K$ be the function field of a nonsingular curve $Y$.
Then for every $y \in Y$ there is a natural discrete valuation
$$v(y): K \to \ZZ$$
where $v(y)(f)$ is the order of vanishing of $f$ at $y$ and with discrete valuation ring $\Olo_{Y, y}$.
This is because $y$ has codimension $1$ and $Y$ is nonsingular, so $\Olo_{Y, y}$ is a regular local ring of dimension $1$, that is, a DVR.
We always view this ring $\Olo_{Y, y}$ as a subring of the function field $K$.

Let us now make the identification between points and DVRs intrinsic.

\begin{definition}
An (abstract) \dfn{function field} $K$ (of dimension $1$) is a finitely generated field extension $k \to K$ with transcendence degree $1$.
If $K$ is a function field of dimension $1$, then the \dfn{projective curve of DVRs} $C_K$ is, as a set, the set of all discrete valuation subrings of $K$.
\end{definition}

\begin{lemma}
Let $X$ be a nonsingular curve.
Then the map $x \mapsto \Olo_{X, x}$ is an injective map $X \to C_{K(X)}$.
\end{lemma}
\begin{proof}
If $\Olo_{X,x} \subseteq \Olo_{X,y}$ as subrings of $K$, then we can choose an affine open set $U = \Spec A$ containing $\{x, y\}$ and then find maximal ideals $\mathfrak m, \mathfrak n$ with $\Olo_{X, x} = A_{\mathfrak m}$ and $\Olo_{X, y} = A_{\mathfrak n}$, thus $\mathfrak m \subseteq \mathfrak n$.
By maximality of $\mathfrak m$ we obtain $\mathfrak m = \mathfrak n$ and hence $x = y$.
\end{proof}

Owing to the above lemma, it is natural to identify $C = C_{K(X)}$ with the ``projective completion" of $X$.
To make this rigorous, we need to turn $C$ into a scheme; first we introduce a Zariski topology for $C$.
The closed sets are the sets
$$Z_f = \{x \in C: f \notin x\}.$$
Thinking of $x$ as a point, the assumption is that $f$ does not restrict to an element of the stalk $\Olo_{C, x}$; that is, $f$ has a pole at $x$.
Thus our closed sets will be the sets of poles of rational maps.
But the Zariski topology on a curve is nothing more than the cofinite topology, so we need to check that the closed sets $Z_f$ are finite:

\begin{lemma}
Let $K$ be a function field.
Then:
\begin{enumerate}
\item For every $f \in K$, $Z_f$ is a finite set.
\item For every discrete valuation ring $R$ in $K$ there is a nonsingular affine curve $X$ such that $R$ is isomorphic to a local ring of $X$.
\end{enumerate}
\end{lemma}
\begin{proof}
See Hartshorne, Lemma I.6.5; make sure to review Dedekind domains beforehand.
\end{proof}

Now we introduce the structure sheaf of $C = C_K$.
For every open set $U \subseteq C$, write $\Olo_C(U) = \bigcap_{R \in U} R$.
The stalks of $\Olo_C$ are local rings of affine curves, so $C$ is locally ringed.

\begin{definition}
By a \dfn{curve of DVRs} we mean an open subset of $C_K$ for some function field $K$ of dimension $1$.
The \dfn{category of curves of DVRs} is the relevant full subcategory of the category of locally ringed spaces.
\end{definition}

At this point we have three (four over $\CC$) different notions of nonsingular curve:
\begin{enumerate}
\item Nonsingular quasiprojective varieties with Krull dimension $1$.
\item Curves of DVRs.
\item Nonsingular, integral separated schemes of finite type and Krull dimension $1$.
\item Cofinite subsets of closed Riemann surfaces (if we're over $\CC$).
\end{enumerate}
We have $1 \leftrightarrow 3$ by Hartshorne, Chapter 2.
$1 \leftrightarrow 4$ was argued in complex analysis last semester.
It remains to show $1 \leftrightarrow 2$.

\begin{proposition}
Every nonsingular quasiprojective curve is isomorphic to a curve of DVRs according to the embedding that sends a point to its local ring.
\end{proposition}
\begin{proof}
Let $K$ be the function field of the nonsingular quasiprojective curve $Y$ and let $U$ be the set of local rings of $Y$.
We already showed that we have an isomorphism
$$\varphi: Y \to U$$
in the category of sets and we just need to show that $U$ is a curve of DVRs and upgrade $\varphi$ to a morphism of locally ringed spaces.

First we show that $U$ is open in $C = C_K$.
Since $C$ has the cofinite topology, a superset of any open subset of $C$ is open, so we just need to show that $U$ contains a nonempty open subset of $C$.
Thus we can replace $Y$ by an open subset of $Y$.
But $Y$ is quasiprojective and therefore admits an open cover by affine curves, so we may assume that $Y$ is affine, say $Y = \Spec A$.

Since $Y = \Spec A$, $K$ is the field of fractions of $A$, there exists $f$ such that
$$A = \frac{k[x_1, \dots, x_n]}{f(x_1, \dots, x_n)},$$
and $U$ is the space of all discrete valuation rings of $K$ containing $A$.
So for every $R \in U$, $A \subseteq R$ iff $x_1, \dots, x_n \in R$, and hence
$$U = \bigcap_{i=1}^n \{R \in C: x_i \in R\} = \bigcap_{i=1}^n U_i.$$
Now the complement of $U_i$ is finite by the previous lemma, so $U_i$ is open and hence $U$ is open.

Since $U$ and $Y$ both have the cofinite topology and $\varphi$ is a bijection, it follows that $\varphi$ is a homeomorphism.
For any $V \subseteq Y$ open, $\Olo_Y(V) = \bigcap_{y \in Y} \Olo_{Y,y}$ which is exactly the definition of $\Olo_C(\varphi(V))$, so $\varphi$ is an isomorphism of locally ringed spaces.
\end{proof}

\begin{corollary}
Every curve of DVRs is a scheme.
\end{corollary}
\begin{proof}
Every point $x \in C_K$ is the local ring of a point $y$ in a nonsingular affine curve $Y$.
The above construction shows that an open neighborhood of $y$ in $Y$ is isomorphic to an open neighborhood of $x$.
\end{proof}

Of course, Hartshorne Chapter I just states this result by saying that a curve of DVRs can be covered by affine varieties.

\begin{lemma}
Let $X$ be an abstract nonsingular curve, $x \in X$, $Y$ a projective variety, and $\varphi: X \setminus \{x\} \to Y$ a morphism.
Then $\varphi$ extends uniquely to $\overline \varphi: X \to Y$.
\end{lemma}
\begin{proof}
By assumption $Y \subseteq \PP^n$ is closed, so by continuity if $\varphi$ extends to a map $X \to \PP^n$ then it extends to a map $X \to Y$.
So we may assume that $Y = \PP^n$.

Let $z_0, \dots, z_n$ be coordinates on $\PP^n$ and let
$$U = \{[z] \in \PP^n: z_0, z_1, \dots, z_n \neq 0\}.$$
If $\varphi(X \setminus \{x\}) \cap U$ is empty, then since $X$ is irreducible, there exists a hyperplane $\{x_i = 0\}$ which contains $\varphi(X \setminus \{x\})$, so we may assume that $\varphi$ actually maps into $\PP^{n - 1}$.
However if $n = 0$ then $\PP^{n - 1}$ is empty, so after decreasing $n$ finitely many times we may assume that $\varphi$ meets $U$.

Let
$$f_{ij} = \varphi^*(z_i/z_j),$$
which is regular on an open subset of $X$ (since $z_i/z_j$ is regular on $U$) and hence $f_{ij} \in K$.
Let $R$ be the local ring at $x$ and $v$ its discrete valuation.
Let $r_i = v(f_{i0})$, so that
$$v(f_{ij}) = r_i - r_j.$$
Let $k$ be such that $r_k$ is minimal; then
$$v(f_{ik}) = r_i - r_k \geq r_i - r_i = 0$$
and hence $f_{ik} \in R$; that is, $f_{ik}$ is regular in a neighborhood of $x$.
So we can extend $\varphi$ to $\overline \varphi$ by setting
$$\overline \varphi(x) = [f_{0k}(x), \dots, f_{nk}(x)].$$
Then $\overline \varphi$ is the unique continuous extension of $X$.

To see that $\overline \varphi$ is a morphism of schemes, it suffices to check this on a small open set $V \ni \overline \varphi(x)$.
First, if
$$V_0 = \Spec k[z_0/z_k, \dots, z_n/z_k],$$
then $\overline \varphi(x) \in V_0$ since $f_{kk}(x) = 1$.
Also we defined $\overline \varphi$ so that $\overline \varphi^*(z_i/z_k) = f_{ik}$ which is regular in a neighborhood of $x$.
Any function on $V$ is a localization of a function on $V_0$ and hence pulls back to a function which is regular near $x$.
\end{proof}

Note that the above proof is essentially the same proof that projective morphisms are proper.
You can actually prove the above result using the valuative criterion of properness (see the appendix).

\begin{lemma}
The curve of DVRs $C_K$ is isomorphic to a nonsingular projective curve.
\end{lemma}
\begin{proof}
Since $C = C_K$ has the cofinite topology, it is quasicompact.
So $C$ is a quasicompact scheme and hence can be covered by finitely many affine curves $V_i$, each of which has finite complement.
Let
$$\varphi_i: V_i \to Y_i$$
be the projective completion of $V_i$\footnote{that is, embed $V_i$ in $\Aff^{n_i} \subseteq \PP^{n_i}$ and take the closure of $V_i$ in $\PP^{n_i}$}.
Since $V_i$ has finite complement, the previous lemma allows us to extend $\varphi_i$ to a morphism $\overline \varphi_i: C \to Y_i$.
Thus we obtain a product morphism
$$\varphi: C \to \prod_{i=1}^n Y_i.$$
Let $Y$ be the closure in $\prod_i Y_i$ of the image of $C$, so $Y$ is a projective variety and $\varphi: C \to Y$ is dominant.
Therefore $Y$ is a curve.

We claim that $\varphi$ is an isomorphism.
Let $x \in C$, thus there is $i$ such that $x \in V_i$. Let $\Pi_i: Y \to Y_i$ be the projection map, so that pulling back along $\Pi_i$ and then along $\varphi$, we obtain morphisms of local rings
$$\Olo_{Y_i, \varphi_i(x)} \to \Olo_{Y, \varphi(x)} \to \Olo_{C, x}.$$
But $\Pi_i$ and $\varphi$ are dominant so these morphisms of local rings are monomorphisms.
Their composition is an isomorphism by definition of $Y_i$, so they must all be isomorphisms.
In particular,
$${\varphi_x}^* : \Olo_{Y, \varphi(x)} \to \Olo_{C, x}$$
is an isomorphism.

Let $y \in Y$. Then the localization of the integral closure of $\Olo_{Y, y}$ at a maximal ideal is a discrete valuation ring $R$ of $K$.
So $\Olo_{Y, y}$ is contained in a discrete valuation ring $R \in C$ such that the intersection of the maximal ideal of $R$ with $\Olo_{Y, y}$ is the maximal ideal of $\Olo_{Y, y}$\footnote{In the main lecture,
I believe we said that having a structure sheaf whose stalk consisted of discrete valuation rings was equivalent to nonsingularness.
In particular, passing from $\Olo_{Y, y}$ to the discrete valuation ring $R$ amounts to taking a resolution of singularities of $Y$ at $y$.}.
By definition, $\Olo_{Y, \varphi(R)} = R$. So by injectivity of the map that sends points to their local rings, $y = \varphi(R)$.
Therefore $\varphi$ is surjective, and is the map that sends a point to its local ring, hence is injective.

Since $\varphi$ is a homeomorphism and is stalkwise an isomorphism, $\varphi$ is an isomorphism of schemes.
\end{proof}

\begin{theorem}
The following categories are equivalent:
\begin{enumerate}
\item nonsingular projective curves with dominant morphisms.
\item Quasiprojective curves with dominant rational maps.
\item The opposite category of abstract function fields of dimension $1$ with algebra homomorphisms.
\end{enumerate}
\end{theorem}
\begin{proof}
We have a forgetful functor from nonsingular projective curves to quasiprojective curves, and the contravariant functor that sends quasiprojective varieties to their function fields restricts to a contravariant functor on quasiprojective curves.
The latter is an equivalence of categories.

We have just shown that for every function field $K$, the curve $C$ of discrete valuation rings of $K$ is a nonsingular projective curve.
Now let $F: K_2 \to K_1$ be an algebra homomorphism.
Since quasiprojective curves are equivalent to the opposite of abstract function fields, $F$ induces a uniquely defined dominant rational map $C_1 \to C_2$.
Let $\varphi: U \to C_2$ be a restriction of the rational map to a dominant morphism and let
$$\overline \varphi: C_1 \to C_2$$
by the extension of $\varphi$ to $C_1$.
Then $\overline \varphi$ is dominant. The fact that this extension is unique means that this assignment is functorial; that is, $K \mapsto C$, $F \mapsto \overline \varphi$ defines a contravariant functor from the opposite of abstract function fields to nonsingular projective curves.
\end{proof}

\section{Rational and elliptic curves}
In these examples we assume that $k$ does not have characteristic $2$ or $3$ to save myself a headache.

Let $Y$ be a nonsingular rational curve, for example
$$Y = \{(x, y) \in \Aff^2: xy = 1\}.$$
According to our main theorem, $Y$ embeds in $\PP^1$ and $K(Y) = k(x)$ whenever $x$ is an affine parameter on $\PP^1$ (so $K(Y)$ is a pure transcendental extension of $k$).
If $Y$ is not isomorphic to $\PP^1$ then $Y$ embeds in a proper open subset of $\PP^1$.
But $\PP^1 = \Aff^1 \cup \{\infty\}$, so $Y$ is a closed subset of an open subset $U$ of $\Aff^1$.
Since $Y$ is clearly not finite it follows that $Y = U$.
Thus $\Olo_Y$ is a sheaf of localizations of $k[x]$ and in particular is a sheaf of UFD.

Now suppose that
$$X = \{(x, y) \in \Aff^2: y^2 = x(x - 1)(x - \lambda)\}.$$
If $\lambda = 1$ then the curve has a self-intersection at $(1, 0)$ and therefore is singular. In that case the coordinate ring $A$ is
$$\frac{k[x, y]}{(y^2 - x(x-1)^2)} = k[\sqrt x]$$
so $K(X) = k(\sqrt x)$ is purely transcendental, and thus $X$ is rational. Similarly if $\lambda = 0$.
On the other hand, if $\lambda \neq 0$, then $X$ is nonsingular (since the differential of its defining ideal is nonvanishing) and one can show that $A$ is not a localization of $k[x]$.
At least when $\lambda = -1$, you can do this by showing that $y$ is irreducible, and noting that $X \cap \{y = 0\}$ is $\{\pm 1, 0\}$ and hence $y$ is not prime, therefore $A$ is not a UFD. I'm not sure if this is true in general.
Anyways, since $X$ is nonsingular and $A$ is not a UFD, $X$ is irrational.
In particular, $K(X)$ is not a purely transcendental extension and $X^{an}$ is not simply connected.

\appendix
\section{Alternative proofs}
\begin{lemma}
Let $X$ be an abstract nonsingular curve, $x \in X$, $Y$ a projective variety, and $\varphi: X \setminus \{x\} \to Y$ a morphism.
Then $\varphi$ extends uniquely to $\overline \varphi: X \to Y$.
\end{lemma}
\begin{proof}
Let $\xi$ be the generic point of $X$; then the residue field of $\xi$ is the function field $K$ of $X$, and we can restrict $\varphi$ to a map $\Spec K \to Y$
which factors through the inclusion $\Spec K \to X \setminus \{x\}$.
By the valuative criterion applied to the natural map $X \to \Spec k$, and the fact that $R = \Olo_{X, \{x\}}$ is a discrete valuation ring, $\Spec K \to Y$ extends uniquely from $\Spec K$ to $\Olo_{X,\{x\}}$, thus we have
$$\psi: \Spec R \to \Spec A$$
whenever $\Spec A$ is an affine open neighborhood of $\psi(x)$ (and hence also of $\psi(\xi)$). Hence $\psi^\sharp$ maps $A \to R$.
If $\Spec B$ is an affine open neighborhood of $x$ then we actually get a map $\psi^\sharp: A \to B_x$.
But $A$ is a finitely generated $k$-algebra, so $\psi^\sharp$ only sends finitely many generators of $A$ to units in $B_x$ and hence $\psi^\sharp$ actually corestricts to a map $A \to B_f$ for some $f \in B$, thus
$$\psi: \Spec B_f \to \Spec A.$$
If we take the germs of $\psi,\varphi$ in $\Olo_{X, y} \subset K$ with $y \in \Spec B_f\setminus \{x\}$, then those germs restrict to the same map $\Spec K \to Y$ and so by the valuative criterion, $\psi = \varphi$ in $\Olo_{X, y}$.
But $y$ was arbitrary so $\psi = \varphi$ in
$$\Spec B_f \setminus\{x\} = \Spec B_f \cap (X \setminus \{x\}).$$
But $\{\Spec B_f, X \setminus \{x\}\}$ is an open cover of $X$, so the morphisms glue uniquely to a morphism $\overline \varphi: X \to Y$.
\end{proof}




\printbibliography


\end{document}
