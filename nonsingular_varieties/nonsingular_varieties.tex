\documentclass[reqno,12pt,letterpaper]{amsart}
\RequirePackage{amsmath,amssymb,amsthm,graphicx,mathrsfs,url}
\RequirePackage[usenames,dvipsnames]{color}
\RequirePackage[colorlinks=true,linkcolor=Red,citecolor=Green]{hyperref}
\RequirePackage{amsxtra}
\usepackage{tikz-cd}

\setlength{\textheight}{8.50in} \setlength{\oddsidemargin}{0.00in}
\setlength{\evensidemargin}{0.00in} \setlength{\textwidth}{6.08in}
\setlength{\topmargin}{0.00in} \setlength{\headheight}{0.18in}
\setlength{\marginparwidth}{1.0in}
\setlength{\abovedisplayskip}{0.2in}
\setlength{\belowdisplayskip}{0.2in}
\setlength{\parskip}{0.05in}
\renewcommand{\baselinestretch}{1.10}

\title{Nonsingular curves}
\author{Aidan Backus}
\date{October 2021}

\newcommand{\NN}{\mathbf{N}}
\newcommand{\ZZ}{\mathbf{Z}}
\newcommand{\QQ}{\mathbf{Q}}
\newcommand{\RR}{\mathbf{R}}
\newcommand{\CC}{\mathbf{C}}
\newcommand{\DD}{\mathbf{D}}
\newcommand{\PP}{\mathbf P}
\newcommand{\MM}{\mathbf M}
\newcommand{\Aff}{\mathbf A}

\DeclareMathOperator{\card}{card}
\DeclareMathOperator{\ch}{ch}
\DeclareMathOperator{\diag}{diag}
\DeclareMathOperator{\dom}{dom}
\DeclareMathOperator{\Gal}{Gal}
\DeclareMathOperator{\id}{id}
\DeclareMathOperator{\rank}{rank}
\DeclareMathOperator*{\Res}{Res}
\DeclareMathOperator{\sgn}{sgn}
\DeclareMathOperator{\singsupp}{sing~supp}
\DeclareMathOperator{\Spec}{Spec}
\DeclareMathOperator{\supp}{supp}
\newcommand{\tr}{\operatorname{tr}}

\newcommand{\dbar}{\overline \partial}

\DeclareMathOperator{\atanh}{atanh}
\DeclareMathOperator{\csch}{csch}
\DeclareMathOperator{\sech}{sech}

\DeclareMathOperator{\Ell}{Ell}
\DeclareMathOperator{\WF}{WF}

\newcommand{\pic}{\vspace{30mm}}
\newcommand{\dfn}[1]{\emph{#1}\index{#1}}

\renewcommand{\Re}{\operatorname{Re}}
\renewcommand{\Im}{\operatorname{Im}}
\newcommand{\Olo}{\mathscr O}
\newcommand{\Mero}{\mathscr M}
\newcommand{\Smooth}{\mathscr E}
\newcommand{\Test}{\mathscr D}


\newtheorem{theorem}{Theorem}[section]
\newtheorem{badtheorem}[theorem]{``Theorem"}
\newtheorem{prop}[theorem]{Proposition}
\newtheorem{lemma}[theorem]{Lemma}
\newtheorem{proposition}[theorem]{Proposition}
\newtheorem{corollary}[theorem]{Corollary}
\newtheorem{conjecture}[theorem]{Conjecture}
\newtheorem{axiom}[theorem]{Axiom}

\theoremstyle{definition}
\newtheorem{definition}[theorem]{Definition}
\newtheorem{remark}[theorem]{Remark}
\newtheorem{example}[theorem]{Example}

\newtheorem{exercise}[theorem]{Discussion topic}
\newtheorem{homework}[theorem]{Homework}
\newtheorem{problem}[theorem]{Problem}

\newtheorem*{ack}{Acknowledgements}
\newtheorem*{notate}{Notation}

%\usepackage{color}
%\hypersetup{%
%    colorlinks=true, % make the links colored%
%    linkcolor=blue, % color TOC links in blue
%    urlcolor=red, % color URLs in red
%    linktoc=all % 'all' will create links for everything in the TOC
%Ning added hyperlinks to the table of contents 6/17/19
%}

\usepackage[backend=bibtex,style=alphabetic,maxcitenames=50,maxnames=50]{biblatex}
\addbibresource{nonsingular_varieties.bib}
\renewbibmacro{in:}{}
\DeclareFieldFormat{pages}{#1}

\begin{document}
\begin{abstract}
We show that every curve is birational to a nonsingular projective curve.
This follows Hartshorne, Chapter 1.
\end{abstract}

\maketitle


\section{Introduction}
In complex analysis last semester, we proved a weaker form of the following theorem:

\begin{theorem}[classification of nonsingular projective curves]
Let $X$ be a smooth projective curve over $k$.
There is an invariant $g \in \NN$ of $X$, called the \emph{genus} of $X$, such that:
\begin{enumerate}
\item $g = \dim \Omega^1(X)$ where $\Omega^1$ is the sheaf of $1$-forms with coefficients in $\Olo_X$.
\item $g = \dim H^1(X, \Olo_X)$ (in the sense of sheaf cohomology).
\item If $k = \CC$, then $g = \dim H^1(X^{an}, \CC)/2$ (in the sense of singular cohomology).
\end{enumerate}
Moreover:
\begin{enumerate}
\item If $g = 0$, then $X$ is isomorphic to $\PP^1$.
\item If $g = 1$, then $X$ is an elliptic curve. Moreover, the moduli space of elliptic curves is $k$.
\item There is an explicit threefold which is the moduli space of curves such that $g = 2$.
\item And so on...
\end{enumerate}
\end{theorem}

See also Chapter IV of Hartshorne.

The goal of this talk is to show:

\begin{theorem}
In every birational class $[X]$ of curves there exists a unique smooth projective curve $X$.
In fact, every smooth curve $Y$ birational to $X$ embeds in $X$.
\end{theorem}

Assuming this is true for a moment, then the classification of smooth projective curves is actually a classification of all curves modulo birationality.

\section{Abstract curves}
The definition of a scheme was based on the idea that we can study varieties up to isomorphism by understanding the regular functions on them.
To study varieties up to birationality we need a larger class of functions (so that the notion of isomorphism is weaker), namely the rational functions.

\begin{definition}
An (abstract) \dfn{function field} $K$ (of dimension $1$) is a finitely generated field extension $k \to K$ with transcendence degree $1$.
\end{definition}

Let $K$ be the function field of a smooth curve $Y$.
Recall from the main lecture: for every $y \in Y$ there is a natural discrete valuation
$$v(y): K \to \ZZ$$
where $v(y)(f)$ is the order of vanishing of $f$ at $y$ and with discrete valuation ring $\Olo_{Y, y}$.
We always view this ring $\Olo_{Y, y}$ as a subring of the function field $K$.

\begin{definition}
If $K$ is a function field of dimension $1$, then the \dfn{abstract smooth projective curve} $C_K$ is, as a set, the set of all discrete valuation subrings of $K$.
(We'll turn $C_K$ into a scheme later...)
\end{definition}

\begin{lemma}
Let $X$ be a smooth curve.
Then the map $x \mapsto \Olo_{X, x}$ is an injective map $X \to C_{K(X)}$.
\end{lemma}
\begin{proof}
If $\Olo_{X,x} \subseteq \Olo_{X,y}$ as subrings of $K$, then we can choose an affine open set $U = \Spec A$ containing $\{x, y\}$ and then find maximal ideals $\mathfrak m, \mathfrak n$ with $\Olo_{X, x} = A_{\mathfrak m}$ and $\Olo_{X, y} = A_{\mathfrak n}$, thus $\mathfrak m \subseteq \mathfrak n$.
By maximality of $\mathfrak m$ we obtain $\mathfrak m = \mathfrak n$ and hence $x = y$.
\end{proof}

Owing to the above lemma, it is natural to identify $C = C_{K(X)}$ with the ``projective completion" of $X$.
To make this rigorous, we need to turn $C$ into a scheme; first we introduce a Zariski topology for $C$.
The closed sets are the sets
$$Z_f = \{x \in C: f \notin x\}.$$
Thinking of $x$ as a point, the assumption is that $f$ does not restrict to an element of the stalk $\Olo_{C, x}$; that is, $f$ has a pole at $x$.
Thus our closed sets will be the sets of poles of rational maps.
But the Zariski topology on a curve is nothing more than the cofinite topology, so we need to check that the closed sets $Z_f$ are finite:

\begin{lemma}
Let $K$ be a function field.
Then:
\begin{enumerate}
\item For every $f \in K$, $Z_f$ is a finite set.
\item For every discrete valuation ring $R$ in $K$ there is a smooth affine curve $X$ such that $R$ is isomorphic to a local ring of $X$.
\end{enumerate}
\end{lemma}
\begin{proof}
See Hartshorne, Lemma I.6.5; make sure to review Dedekind domains beforehand.
\end{proof}

Now we introduce the structure sheaf of $C = C_K$.
For every open set $U \subseteq C$, write $\Olo_C(U) = \bigcap_{R \in U} R$.
The stalks of $\Olo_C$ are local rings of affine curves, so $C$ is locally ringed.

\begin{definition}
By a \dfn{curve of DVRs} we mean an open subset of $C_K$ for some function field $K$ of dimension $1$.
\end{definition}

Morphisms of the curves $C_K$ will be morphisms of locally ringed spaces.
Since quasiprojective curves are also locally ringed spaces, it makes sense to ask whether a quasiprojective curve is isomorphic to an abstract curve (in the category of locally ringed spaces).

\begin{proposition}
Every smooth quasiprojective curve is isomorphic to an open subset of a curve of DVRs according to the embedding that sends a point to its local ring.
\end{proposition}
\begin{proof}
Let $K$ be the function field of the smooth quasiprojective curve $Y$ and let $U$ be the set of local rings of $Y$.
We already showed that we have an isomorphism
$$\varphi: Y \to U$$
in the category of sets and we just need to show that $U$ is an abstract smooth curve and upgrade $\varphi$ to a morphism of locally ringed spaces.

First we show that $U$ is open, and hence an abstract smooth curve.
Since $C = C_K$ has the cofinite topology, a superset of any open subset of $C$ is open, so we just need to show that $U$ contains a nonempty open subset of $C$.
Thus we can replace $Y$ by an open subset of $Y$.
But $Y$ is quasiprojective and therefore admits an open cover by affine curves, so we may assume that $Y$ is affine, say $Y = \Spec A$.

Since $Y = \Spec A$, $K$ is the field of fractions of $A$, there are $x_1, \dots, x_n$ and $f$ such that
$$A = \frac{k[x_1, \dots, x_n]}{f(x_1, \dots, x_n)},$$
and $U$ is the space of all discrete valuation rings of $K$ containing $A$.
So for every $R \in U$, $A \subseteq R$ iff $x_1, \dots, x_n \in R$, and hence
$$U = \bigcap_{i=1}^n \{R \in C: x_i \in R\} = \bigcap_{i=1}^n U_i.$$
Now the complement of $U_i$ is finite by the previous lemma, so $U_i$ is open and hence $U$ is open.

Since $U$ and $Y$ both have the cofinite topology and $\varphi$ is a bijection, it follows that $\varphi$ is a homeomorphism.
For any $V \subseteq Y$ open, $\Olo_Y(V) = \bigcap_{y \in Y} \Olo_{Y,y}$ which is exactly the definition of $\Olo_C(\varphi(V))$, so $\varphi$ is an isomorphism of locally ringed spaces.
\end{proof}

At this point we have three different notions of curve:
\begin{enumerate}
\item Smooth quasiprojective varieties with Krull dimension $1$.
\item Open subsets of a curve of DVRs $C_K$.
\item Smooth, integral separated schemes of finite type and Krull dimension $1$.
\end{enumerate}
We have $1 \leftrightarrow 3$ by Hartshorne, Chapter 2, and we just argued $1 \to 2$.
It remains to show $2 \to 3$.

\begin{lemma}
Every open subset of a curve of DVRs is a scheme.
\end{lemma}
\begin{proof}
Let $C = C_K$.
In fact, if $R \in C$ then there is a smooth curve $V = \Spec A$ and a point $x \in V$ such that $R = \Olo_{X,x}$.
So the function field of $V$ is $K$ and hence there is an open embedding
$$\varphi: V \to K$$
with $\varphi(x) = R$. This furnishes an open cover of $C$ by affine schemes.
\end{proof}

You can actually prove our next lemma using the valuative criterion of properness (see the appendix) but I don't think this is easier than just writing down what the extension is.

\begin{lemma}
Let $X$ be an abstract smooth curve, $x \in X$, $Y$ a projective variety, and $\varphi: X \setminus \{x\} \to Y$ a morphism.
Then $\varphi$ extends uniquely to $\overline \varphi: X \to Y$.
\end{lemma}
\begin{proof}
By assumption $Y \subseteq \PP^n$ is closed, so by continuity if $\varphi$ extends to a map $X \to \PP^n$ then it extends to a map $X \to Y$.
So we may assume that $Y = \PP^n$.

Let $z_0, \dots, z_n$ be coordinates on $\PP^n$ and let
$$U = \{[z] \in \PP^n: z_0, z_1, \dots, z_n \neq 0\}.$$
If $\varphi(X \setminus \{x\}) \cap U$ is empty, then since $X$ is irreducible, there exists a hyperplane $\{x_i = 0\}$ which contains $\varphi(X \setminus \{x\})$, so we may assume that $\varphi$ actually maps into $\PP^{n - 1}$.
However if $n = 0$ then $\PP^{n - 1}$ is empty, so after decreasing $n$ finitely many times we may assume that $\varphi$ meets $U$.

Let
$$f_{ij} = \varphi^*(z_i/z_j),$$
which is regular on an open subset of $X$ (since $z_i/z_j$ is regular on $U$) and hence $f_{ij} \in K$.
Let $R$ be the local ring at $x$ and $v$ its discrete valuation.
Let $r_i = v(f_{i0})$, so that
$$v(f_{ij}) = r_i - r_j.$$
Let $k$ be such that $r_k$ is minimal; then
$$v(f_{ik}) = r_i - r_k \geq r_i - r_i = 0$$
and hence $f_{ik} \in R$; that is, $f_{ik}$ is regular in a neighborhood of $x$.
So we can extend $\varphi$ to $\overline \varphi$ by setting
$$\overline \varphi(x) = [f_{0k}(x), \dots, f_{nk}(x)].$$
Then $\overline \varphi$ is the unique continuous extension of $X$.

To see that $\overline \varphi$ is a morphism of locally ringed spaces, it suffices to check this on a small open set $V \ni \overline \varphi(x)$.
First, if
$$V_0 = \Spec k[z_0/z_k, \dots, z_n/z_k],$$
then $\overline \varphi(x) \in V_0$ since $f_{kk}(x) = 1$.
Also we defined $\overline \varphi$ so that $\overline \varphi^*(z_i/z_k) = f_{ik}$ which is regular in a neighborhood of $x$.
Any function on $V$ is a localization of a function on $V_0$ and hence pulls back to a function which is regular near $x$.
\end{proof}

\begin{lemma}
The curve of DVRs $C_K$ is isomorphic in the category of schemes to a smooth projective curve.
\end{lemma}
\begin{proof}
Since $C = C_K$ has the cofinite topology, it is quasicompact.
So $C$ is a quasicompact scheme and hence can be covered by finitely many affine curves $V_i$, each of which has finite complement.
Let
$$\varphi_i: V_i \to Y_i$$
be the projective completion of $V_i$\footnote{that is, embed $V_i$ in $\Aff^{n_i} \subseteq \PP^{n_i}$ and take the closure of $V_i$ in $\PP^{n_i}$}.
Since $V_i$ has finite complement, the previous lemma allows us to extend $\varphi_i$ to a morphism $\overline \varphi_i: C \to Y_i$.
Thus we obtain a product morphism
$$\varphi: C \to \prod_{i=1}^n Y_i.$$
Let $Y$ be the closure in $\prod_i Y_i$ of the image of $C$, so $Y$ is a projective variety and $\varphi: C \to Y$ is dominant.
Therefore $Y$ is a curve.

We claim that $\varphi$ is an isomorphism.
Let $x \in C$, thus there is $i$ such that $x \in V_i$. Let $\Pi_i: Y \to Y_i$ be the projection map, so that pulling back along $\Pi_i$ and then along $\varphi$, we obtain morphisms of local rings
$$\Olo_{Y_i, \varphi_i(x)} \to \Olo_{Y, \varphi(x)} \to \Olo_{C, x}.$$
But $\Pi_i$ and $\varphi$ are dominant so these morphisms of local rings are monomorphisms.
Their composition is an isomorphism by definition of $Y_i$, so they must all be isomorphisms.
In particular,
$${\varphi_x}^* : \Olo_{Y, \varphi(x)} \to \Olo_{C, x}$$
is an isomorphism.

Let $y \in Y$. Then the localization of the integral closure of $\Olo_{Y, y}$ at a maximal ideal is a discrete valuation ring $R$ of $K$.
So $\Olo_{Y, y}$ is contained in a discrete valuation ring $R \in C$ such that the intersection of the maximal ideal of $R$ with $\Olo_{Y, y}$ is the maximal ideal of $\Olo_{Y, y}$\footnote{In the main lecture,
I believe we said that having a structure sheaf whose stalk consisted of discrete valuation rings was equivalent to smoothness.
In particular, passing from $\Olo_{Y, y}$ to the discrete valuation ring $R$ amounts to taking a resolution of singularities of $Y$ at $y$.}.
By definition, $\Olo_{Y, \varphi(R)} = R$. So by injectivity of the map that sends points to their local rings, $y = \varphi(R)$.
Therefore $\varphi$ is surjective, and is the map that sends a point to its local ring, hence is injective.

Since $\varphi$ is a homeomorphism and is stalkwise an isomorphism, $\varphi$ is an isomorphism of schemes.
\end{proof}

\begin{theorem}
The following categories are equivalent:
\begin{enumerate}
\item Smooth projective curves with dominant morphisms.
\item Quasiprojective curves with dominant rational maps.
\item The opposite category of abstract function fields of dimension $1$ with algebra homomorphisms.
\end{enumerate}
\end{theorem}
\begin{proof}
We have a forgetful functor from smooth projective curves to quasiprojective curves, and the contravariant functor that sends quasiprojective varieties to their function fields restricts to a contravariant functor on quasiprojective curves.
The latter is an equivalence of categories.

We have just shown that for every function field $K$, the curve $C$ of discrete valuation rings of $K$ is a smooth projective curve.
Now let $F: K_2 \to K_1$ be an algebra homomorphism.
Since quasiprojective curves are equivalent to the opposite of abstract function fields, $F$ induces a uniquely defined dominant rational map $C_1 \to C_2$.
Let $\varphi: U \to C_2$ be a restriction of the rational map to a dominant morphism and let
$$\overline \varphi: C_1 \to C_2$$
by the extension of $\varphi$ to $C_1$.
Then $\overline \varphi$ is dominant. The fact that this extension is unique means that this assignment is functorial; that is, $K \mapsto C$, $F \mapsto \overline \varphi$ defines a contravariant functor from the opposite of abstract function fields to smooth projective curves.
\end{proof}

\section{Rational and elliptic curves}
In these examples we assume that $k$ does not have characteristic $2$ or $3$ to save myself a headache.

Let $Y$ be a smooth rational curve, for example
$$Y = \{(x, y) \in \Aff^2: xy = 1\}.$$
According to our main theorem, $Y$ embeds in $\PP^1$ and $K(Y) = k(x)$ whenever $x$ is an affine parameter on $\PP^1$ (so $K(Y)$ is a pure transcendental extension of $k$).
Let
$$\iota: Y \to \PP^1$$
be the emebdding.

If $\iota$ is surjective then $Y$ is isomorphic to $\PP^1$.
In that case, symmetries of $Y$ correspond by our theorem to symmetries of $k(x)$.
In complex analysis we showed that the group of symmetries of $k(x)$ is $PGL(k^2)$, namely the matrix
$$T = \begin{bmatrix}a & b\\c & d\end{bmatrix}$$
acts on $k(x)$: identify points of $k(x)$ with rational functions $\PP^1 \to \PP^1$, and then identify those with homogeneous maps $k^2 \to k^2$, on which $PGL(k^2)$ acts by pullback.
Thus the symmetry group of $Y$ is $PGL(k^2)$.

If $Y$ is not isomorphic to $\PP^1$ then $Y$ embeds in a proper open subset of $\PP^1$.
But $\PP^1 = \Aff^1 \cup \{\infty\}$, so $Y$ is a closed subset of an open subset $U$ of $\Aff^1$.
Since $Y$ is clearly not finite it follows that $Y = U$.

Now suppose that instead
$$X = \{(x, y) \in \Aff^2: y^2 = x^3 - x\}.$$
Writing $f(x, y) = y^2 - x^3 + x$,
$$df = (1 - 3x^2) ~dx + 2y ~dy$$
which is nonzero on $X$, so $X$ is smooth; by definition it is affine.
On the other hand, one can show that $y$ is irreducible in $A = k[x, y]/(f(x, y))$.
But $X \cap \{y = 0\}$ consists of the three points $\{-1, 0, 1\}$ and hence is reducible, so $y$ is not prime and hence $A$ cannot be a UFD.

Since $Y$ is an open subset of $\Aff^1$, for every open $U \subseteq Y$, $\Olo_Y(U)$ is a localization of the UFD $k[x]$ and hence is itself a UFD.
But $Y$ was an arbitrary smooth rational curve, so $X$ is not a rational curve.
In addition, $K(Y) = k[x, y]/(xy - 1)$ is an extension field of $K(X)$, so we get a dominant rational map $X \to Y$.
In fact, $X$ is an elliptic curve, but we haven't proven that.

Irrationality has many consequences, including:
\begin{enumerate}
\item There exists a dominant rational function on $X$.
\item $K(X) = k[x, y]/(f(x, y))$ is not a pure transcendental extension of $k$.
\item When combined with the Hodge theorem: $H^1(X^{an}, \CC)$ is nontrivial.
\end{enumerate}

\appendix
\section{Alternative proofs}
\begin{lemma}
Let $X$ be an abstract smooth curve, $x \in X$, $Y$ a projective variety, and $\varphi: X \setminus \{x\} \to Y$ a morphism.
Then $\varphi$ extends uniquely to $\overline \varphi: X \to Y$.
\end{lemma}
\begin{proof}
We give an alternative proof at a later stage.
For now we just give a proof using schemes.

Recall that projective schemes are proper over $\Spec k$.
Let $\xi$ be the generic point of $X$; then the residue field of $\xi$ is the function field $K$ of $X$, and we can restrict $\varphi$ to a map $\Spec K \to Y$
which factors through the inclusion $\Spec K \to X \setminus \{x\}$.
By the valuative criterion and the fact that $R = \Olo_{X, \{x\}}$ is a discrete valuation ring, $\Spec K \to Y$ extends uniquely from $\Spec K$ to $\Olo_{X,\{x\}}$, thus we have
$$\psi: \Spec R \to \Spec A$$
whenever $\Spec A$ is an affine open neighborhood of $\psi(x)$ (and hence also of $\psi(\xi)$). Hence $\psi^\sharp$ maps $A \to R$.
If $\Spec B$ is an affine open neighborhood of $x$ then we actually get a map $\psi^\sharp: A \to B_x$.
But $A$ is a finitely generated $k$-algebra, so $\psi^\sharp$ only sends finitely many generators of $A$ to units in $B_x$ and hence $\psi^\sharp$ actually corestricts to a map $A \to B_f$ for some $f \in B$, thus
$$\psi: \Spec B_f \to \Spec A.$$
If we take the germs of $\psi,\varphi$ in $\Olo_{X, y} \subset K$ with $y \in \Spec B_f\setminus \{x\}$, then those germs restrict to the same map $\Spec K \to Y$ and so by the valuative criterion, $\psi = \varphi$ in $\Olo_{X, y}$.
But $y$ was arbitrary so $\psi = \varphi$ in
$$\Spec B_f \setminus\{x\} = \Spec B_f \cap (X \setminus \{x\}).$$
But $\{\Spec B_f, X \setminus \{x\}\}$ is an open cover of $X$, so the morphisms glue uniquely to a morphism $\overline \varphi: X \to Y$.
\end{proof}




\printbibliography


\end{document}
