\documentclass[reqno,11pt]{amsart}
\usepackage[letterpaper, margin=1in]{geometry}
\RequirePackage{amsmath,amssymb,amsthm,graphicx,mathrsfs,url,slashed,subcaption}
\RequirePackage[usenames,dvipsnames]{xcolor}
\RequirePackage[colorlinks=true,linkcolor=Red,citecolor=Green]{hyperref}
\RequirePackage{amsxtra}
\usepackage{cancel}
\usepackage{tikz-cd}
%\usepackage[T1]{fontenc}

% \setlength{\textheight}{9.3in} \setlength{\oddsidemargin}{-0.25in}
% \setlength{\evensidemargin}{-0.25in} \setlength{\textwidth}{7in}
% \setlength{\topmargin}{-0.25in} \setlength{\headheight}{0.18in}
% \setlength{\marginparwidth}{1.0in}
% \setlength{\abovedisplayskip}{0.2in}
% \setlength{\belowdisplayskip}{0.2in}
% \setlength{\parskip}{0.05in}
%\renewcommand{\baselinestretch}{1.05}

\title{Eikonal equation}
\author{Aidan Backus}
\address{Department of Mathematics, Brown University}
\email{aidan\_backus@brown.edu}
\date{\today}

\newcommand{\NN}{\mathbf{N}}
\newcommand{\ZZ}{\mathbf{Z}}
\newcommand{\QQ}{\mathbf{Q}}
\newcommand{\RR}{\mathbf{R}}
\newcommand{\CC}{\mathbf{C}}
\newcommand{\DD}{\mathbf{D}}
\newcommand{\PP}{\mathbf P}
\newcommand{\MM}{\mathbf M}
\newcommand{\II}{\mathbf I}
\newcommand{\Hyp}{\mathbf H}
\newcommand{\Sph}{\mathbf S}
\newcommand{\Group}{\mathbf G}
\newcommand{\GL}{\mathbf{GL}}
\newcommand{\Orth}{\mathbf{O}}
\newcommand{\SpOrth}{\mathbf{SO}}
\newcommand{\Ball}{\mathbf{B}}

\newcommand*\dif{\mathop{}\!\mathrm{d}}

\DeclareMathOperator{\card}{card}
\DeclareMathOperator{\dist}{dist}
\DeclareMathOperator{\id}{id}
\DeclareMathOperator{\supp}{supp}
\DeclareMathOperator{\Teich}{Teich}
\DeclareMathOperator{\tr}{tr}

\newcommand{\Leaves}{\mathscr L}
\newcommand{\Lagrange}{\mathcal L}
\newcommand{\Hypspace}{\mathscr H}

\newcommand{\Chain}{\underline C}

\newcommand{\Two}{\mathrm{I\!I}}

\newcommand{\normal}{\mathbf n}
\newcommand{\radial}{\mathbf r}
\newcommand{\evect}{\mathbf e}
\newcommand{\vol}{\mathrm{vol}}

\newcommand{\diam}{\mathrm{diam}}
\newcommand{\Ell}{\mathrm{Ell}}
\newcommand{\inj}{\mathrm{inj}}
\newcommand{\Lip}{\mathrm{Lip}}
\newcommand{\MCL}{\mathrm{MCL}}
\newcommand{\Riem}{\mathrm{Riem}}

\newcommand{\Min}{\mathrm{Min}}
\newcommand{\Max}{\mathrm{Max}}

\newcommand{\dfn}[1]{\emph{#1}\index{#1}}

\renewcommand{\Re}{\operatorname{Re}}
\renewcommand{\Im}{\operatorname{Im}}

\newcommand{\loc}{\mathrm{loc}}
\newcommand{\cpt}{\mathrm{cpt}}

\def\Japan#1{\left \langle #1 \right \rangle}

\newtheorem{theorem}{Theorem}[section]
\newtheorem{badtheorem}[theorem]{``Theorem"}
\newtheorem{prop}[theorem]{Proposition}
\newtheorem{lemma}[theorem]{Lemma}
\newtheorem{sublemma}[theorem]{Sublemma}
\newtheorem{proposition}[theorem]{Proposition}
\newtheorem{corollary}[theorem]{Corollary}
\newtheorem{conjecture}[theorem]{Conjecture}
\newtheorem{axiom}[theorem]{Axiom}
\newtheorem{assumption}[theorem]{Assumption}

\newtheorem{mainthm}{Theorem}
\renewcommand{\themainthm}{\Alph{mainthm}}

\newtheorem{claim}{Claim}[theorem]
\renewcommand{\theclaim}{\thetheorem\Alph{claim}}
% \newtheorem*{claim}{Claim}

\theoremstyle{definition}
\newtheorem{definition}[theorem]{Definition}
\newtheorem{remark}[theorem]{Remark}
\newtheorem{example}[theorem]{Example}
\newtheorem{notation}[theorem]{Notation}

\newtheorem{exercise}[theorem]{Discussion topic}
\newtheorem{homework}[theorem]{Homework}
\newtheorem{problem}[theorem]{Problem}

\makeatletter
\newcommand{\proofpart}[2]{%
  \par
  \addvspace{\medskipamount}%
  \noindent\emph{Part #1: #2.}
}
\makeatother



\numberwithin{equation}{section}


% Mean
\def\Xint#1{\mathchoice
{\XXint\displaystyle\textstyle{#1}}%
{\XXint\textstyle\scriptstyle{#1}}%
{\XXint\scriptstyle\scriptscriptstyle{#1}}%
{\XXint\scriptscriptstyle\scriptscriptstyle{#1}}%
\!\int}
\def\XXint#1#2#3{{\setbox0=\hbox{$#1{#2#3}{\int}$ }
\vcenter{\hbox{$#2#3$ }}\kern-.6\wd0}}
\def\ddashint{\Xint=}
\def\dashint{\Xint-}

\usepackage[backend=bibtex,style=alphabetic,giveninits=true]{biblatex}
\renewcommand*{\bibfont}{\normalfont\footnotesize}
\addbibresource{best_curl.bib}
\renewbibmacro{in:}{}
\DeclareFieldFormat{pages}{#1}

\newcommand\todo[1]{\textcolor{red}{TODO: #1}}


\begin{document}
\begin{abstract}
    \todo{}
\end{abstract}

\maketitle

%%%%%%%%%%%%%%%%%%%%%%%%%%%%%%%%%%%%%%%%%%%%%%%%%%%%%%%
\section{Introduction}

\section{Foliation induces eikonal equation}
Let $d = 2, 3, 4$ (we can try to weaken this hypothesis later but I literally don't care.)

Let $\mathscr F$ be a smooth minimal foliation of $M = \tilde M/\Gamma$, where $\tilde M$ is $\RR^d$ or $\Hyp^d$, and $M$ is compact.
In flow box coordinates we can define a function $u(k, y) = k$ where $k$ is the leaf index.
Then $\{u = k\}$ is a minimal surface, so $u$ is $1$-harmonic.
We also have a dual calibration $F$, $F = \star \dif u/|\dif u|$.
In fact, $F = \dif A$ where $A$ solves the eikonal equation
$$|\dif A|^2 - 1 = 0.$$
We wrote it in this way to emphasize the Hamiltonian
$$H(\xi) := |\xi^{\rm as}|^2.$$

\begin{proposition}
Every geodesic foliation of $\RR^2$ which admits a transverse line is a foliation by parallel lines.
\end{proposition}
\begin{proof}
Let $u$ be the induced $1$-harmonic, let $A$ be the dual eikon to $u$, and let $\ell$ be a transverse line.
Without loss of generality, $\ell = \{y = 0\}$.
Consider an infinite rectangle $R$ bounded by $\ell$ and a translation of $\ell$ by some distance $r > 0$.
We will bound $\partial_x^2 A$ on $\ell$ in terms of $r$, and then take $r \to \infty$ to conclude that $A|_\ell$ is linear.
The characteristic equations for $A$ are 
$$\begin{cases}
    \dot x &= 2\xi \\
    \dot y &= 2\eta \\
    \dot A &= 2 \\
    \dot \xi &= 0 \\
    \dot \eta &= 0
\end{cases}$$
which we can read off of the Hamiltonian.
It follows that $\dif A$ calibrates the characteristics.
So the characteristics are lines, are transverse to $\ell$, and there are no caustics in $R$.

Let $L$ be the Lipschitz constant of $\xi_0 := \partial_x A$ on $\ell$, witnessed by $x_1, x_2$.
We can take $x_1, x_2$ as close as we want, actually, since $\ell$ is convex, and in particular we may assume that $\ell$ is uniformly transverse to $\mathscr F$ on $[x_1, x_2]$.
Then if $L > 0$, the characteristics through $x_1, x_2$ cross in $R$ or its reflection about $\ell$, if $R$ is large enough depending on $L$.
But this is a contradiction because there are no caustics in $R$, so we get $L = 0$.
Therefore $\xi_0$ is a constant on $\ell$.

But then $\mathscr F$ is the foliation by all lines of slope $\eta_0/\xi_0$ where $\eta_0 := \pm \sqrt{1 - \xi_0^2}$ and the sign is constant by continuity of $\dif A = \star \normal$.
\end{proof}



\printbibliography

\end{document}
