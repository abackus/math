\documentclass[10pt]{beamer}

\usetheme[progressbar=frametitle, block=fill]{metropolis}
\RequirePackage{amsmath,amssymb,amsthm,graphicx,mathrsfs,url,slashed,subcaption}
\usepackage{appendixnumberbeamer}

\usepackage{booktabs}
\usepackage[scale=2]{ccicons}

\usepackage{pgfplots}
\usepgfplotslibrary{dateplot}

\newcommand{\NN}{\mathbf{N}}
\newcommand{\ZZ}{\mathbf{Z}}
\newcommand{\QQ}{\mathbf{Q}}
\newcommand{\RR}{\mathbf{R}}
\newcommand{\CC}{\mathbf{C}}
\newcommand{\DD}{\mathbf{D}}
\newcommand{\PP}{\mathbf P}
\newcommand{\MM}{\mathbf M}
\newcommand{\II}{\mathbf I}
\newcommand{\Hyp}{\mathbf H}
\newcommand{\Sph}{\mathbf S}
\newcommand{\Group}{\mathbf G}
\newcommand{\GL}{\mathbf{GL}}
\newcommand{\Orth}{\mathbf{O}}
\newcommand{\SpOrth}{\mathbf{SO}}
\newcommand{\Ball}{\mathbf{B}}

\newcommand*\dif{\mathop{}\!\mathrm{d}}
\newcommand{\dfn}[1]{\emph{#1}\index{#1}}

\newcommand{\Teich}{\mathrm{Teich}}
\newcommand{\normal}{\mathbf n}

\newcommand{\dist}{\mathrm{dist}}
\newcommand{\id}{\mathrm{id}}
\newcommand{\Lip}{\mathrm{Lip}}
\newcommand{\loc}{\mathrm{loc}}
\newcommand{\cpt}{\mathrm{cpt}}


\newcommand{\Two}{\mathrm{I\!I}}


\newtheorem{proposition}{Proposition}
\newtheorem{philosophy}{Philosophy}
\newtheorem{question}{Question}
\newtheorem{conjecture}{Conjecture}


\usepackage{xspace}
\newcommand{\themename}{\textbf{\textsc{metropolis}}\xspace}

\title{The \texorpdfstring{$p$-Laplacian}{p-Laplacian} in Teichm\"uller theory}
% \subtitle{A modern beamer theme}
% \date{\today}
\date{28 March, 2023}
\author{Aidan Backus}
\institute{Brown University}
% \titlegraphic{\hfill\includegraphics[height=1.5cm]{logo.pdf}}

\begin{document}

\maketitle

% \begin{frame}{Table of contents}
%   \setbeamertemplate{section in toc}[sections numbered]
%   \tableofcontents%[hideallsubsections]
% \end{frame}

\begin{frame}{Dramatis personæ}
\begin{enumerate}
\item Thurston's Finsler metric on Teichm\"uller space
\item Duality between the $1$- and $\infty$-Laplacians
\end{enumerate}

\begin{philosophy}
If we understand (2), we understand (1)!
\end{philosophy}
\end{frame}

\begin{frame}{Thurston's Finsler metric}
\begin{definition}
Let $S$ be a closed oriented (topological) surface of genus $\geq 2$.
The \dfn{Teichm\"uller space} $\Teich(S)$ is the set of hyperbolic manifolds with underlying topological space $S$ modulo self-diffeomorphisms homotopic to the identity.
\end{definition} \pause

What geometry does Teichm\"uller space have?
\end{frame}

\begin{frame}{Thurston's Finsler metric}
\begin{definition}
Let $M, N$ be metric spaces.
A map $f: M \to N$ is \dfn{best Lipschitz} if it minimizes 
$$\Lip(f) := \sup_{x, y \in M} \frac{\dist(f(x), f(y))}{\dist(x, y)}$$
among all maps homotopic to $f$.
\end{definition} \pause

\begin{definition}[Thurston '86]
Let $M, N \in \Teich(S)$, and let $f: M \to N$ be best Lipschitz and homotopic to $\id_S$.
Then $L(M, N) := \Lip(f)$.
\end{definition} \pause

\begin{theorem}[Thurston '86, Papadopoulos '11]
$(\Teich(S), \log L)$ is an asymmetric metric space.
The topology is the same as the topology arising from the usual (Teichm\"uller, Weil-Petersson) geometries, and with respect to this topology, $\log L$ is a Finsler structure.
\end{theorem}
\end{frame}

\begin{frame}{The $\infty$-Laplacian}
If $u: M \to \RR$ is a scalar field then
$$\Lip(u) = \|\nabla u\|_{L^\infty}.$$
In calculus of variations we learn that for $1 < p < \infty$, minimizers of $\|\nabla u\|_{L^p}$ solve the $p$-Laplace equation 
$$\nabla \cdot (|\nabla u|^{p - 2} \nabla u) = 0.$$
Renormalizing this equation and taking $p \to \infty$ we get the $\infty$-Laplacian
$$\langle \nabla^2 u, \nabla u \otimes \nabla u\rangle = 0.$$

\begin{theorem}[Aronsson '68, Jensen '93]
A scalar field is $\infty$-harmonic (in the viscosity sense) iff it is locally best Lipschitz. 
\end{theorem}
\end{frame}

\begin{frame}{The $\infty$-Laplacian}
\begin{philosophy}[Daskalopolous and Uhlenbeck '20, '22]
Equations similar to the $\infty$-Laplacian give best Lipschitz maps into Riemannian manifolds.
If we can solve them, then we can compute $L(M, N)$.
\end{philosophy} \pause

\begin{theorem}[Daskalopolous and Uhlenbeck '22]
Let $M, N \in \Teich(S)$, and let $\rho$ be a homotopy class $M \to N$.
Then for any $1 < p < \infty$ there exists $u_p$, $[u_p] = \rho$, solving the \dfn{Schatten-von Neumann $p$-Laplace equation}
$$\nabla_u \cdot ((\dif u^\dagger \dif u)^{\frac{p - 2}{2}} \dif u)^\sharp = 0.$$
As $p \to \infty$, $u_p$ converges to a best Lipschitz map in $W^{1, q}$ for any $q$.
\end{theorem}

So that's the $\infty$-Laplacian. What about the $1$-Laplacian?
\end{frame}

\begin{frame}{Minimal and geodesic laminations}
\begin{theorem}[Thurston '86]
Let $M, N \in \Teich(S)$.
Let $\gamma_M$ denote the closed geodesic representative of a homotopy class of curves $\gamma \in \pi_1(S) \setminus 0$ in $M$.
Then
$$L(M, N) = \sup_{\pi_1(S) \setminus 0} \frac{|\gamma_N|}{|\gamma_M|}.$$
\end{theorem}\pause
    
To prove this, Thurston compactifies the space of geodesics.
For a chain $\psi := \sum_i c_i \gamma_i$, introduce the current $T_\psi$ defined by 
$$\int_M T_\psi \wedge \varphi = \sum_i c_i \int_{\gamma_i} \varphi.$$
The topology on currents (of locally finite total variation) is the weak topology of measures.
\end{frame}

\begin{frame}{Minimal and geodesic laminations}
\begin{definition}
A \dfn{measured oriented minimal (resp geodesic) lamination} is a limit of locally area (resp length)-minimizing chains in the weak topology of measures.
\end{definition}

More intrinsic definitions available! Morally, a lamination is an infinite multicurve or multisurface.
Let's draw some pictures of them. \pause

The idea of Thurston's proof: best Lipschitz maps $f$ stretch a certain geodesic lamination by a factor of $L(M, N)$. So 
$$L(M, N) = \max_\lambda \frac{|f_* \lambda|}{|\lambda|}.$$
\end{frame}

\begin{frame}{The $1$-Laplacian}
Minimal and geodesic laminations are closely tied to the $1$-Laplacian.

\begin{definition}[Maz\'on, Rossi, Segura de Le\'on '14]
    A \dfn{weak solution of the $1$-Laplace equation}
    $$\nabla \cdot \frac{\nabla v}{|\nabla v|} = 0$$
    is a $v \in BV_\loc$ such that there exists a divergence-free $L^\infty$ vector field $X$ such that $(\dif v, X) = |\dif v|$.
\end{definition}

Solution does not need to be $W^{1, 1}_\loc$, but it does minimize its total variation
$$\int_M |\dif v| \dif V = \sup_{\|\psi\|_{C^0} \leq 1} \int_M \dif v \wedge \psi.$$
Usually $\dif v$ is a Cantor measure.
\end{frame}

\begin{frame}{The $1$-Laplacian}

\begin{theorem}[Daskalopolous and Uhlenbeck '20]
Let $M$ be a closed hyperbolic surface. The following are true:
\begin{enumerate}
\item In every homotopy class of maps $M \to \Sph^1$ there exists an $\infty$-harmonic (hence best Lipschitz) map $u$.
\item There is a measured oriented geodesic lamination $\lambda$ which is stretched by $u$ by a factor of $\Lip(u)$.
\item Locally there exists a $1$-harmonic function $v$ such that $\dif v$ is (the current associated to) $\lambda$.
\end{enumerate}
\end{theorem}

There is also a formula relating $u, v$ which we will not elaborate on here. \pause 

Main theorem of this talk: (3) holds in much higher generality...
\end{frame}

\begin{frame}{The main theorem for the $1$-Laplacian}
\begin{theorem}[AB]
Let $M$ be a Riemannian manifold, $\dim M \leq 4$, and $T$ a current of locally finite total variation on $M$. The following are equivalent:
\begin{enumerate}
\item $T$ is the current associated to a measured oriented minimal lamination $\lambda$.
\item Locally there exists a $1$-harmonic function $v$ such that $T = \dif v$.
\end{enumerate}
If these conditions are met, then the level sets $\partial \{v > y\}$ are the minimal hypersurfaces in $\lambda$.
\end{theorem}\pause

\begin{philosophy}
The $1$-Laplacian governs the geodesic laminations, and the geodesic laminations govern Thurston's Finsler metric.
\end{philosophy}
\end{frame}

\begin{frame}{Proof of the main theorem}
Outline of the proof:
\begin{enumerate}
\item \begin{enumerate}
\item If $v$ is $1$-harmonic, then level sets of $v$ are stable minimal hypersurfaces.
\item Disjoint stable minimal hypersurfaces form a lamination. \end{enumerate}
\item If $\dif v$ is exact and arises from a minimal lamination, then $v$ is $1$-harmonic.
\end{enumerate}
\end{frame}

\begin{frame}{Proof of the main theorem}
\begin{lemma}
    If $v$ is $1$-harmonic, then level sets of $v$ are stable minimal hypersurfaces.
\end{lemma}

Formally for a level set $S$ of $v$,
$$0 = \nabla \cdot \frac{\nabla v}{|\nabla v|} = \nabla \cdot \normal_S = H_S.$$
But $S$ is not a priori smooth -- it's just a de Giorgi reduced boundary.

We need the de Giorgi lemma (de Giorgi '61, Miranda '66, Giusti '77) to prove that $S$ is smooth.
Folklore on Riemannian manifolds -- oscillation doesn't play nicely with curvature.
I filled in the details separately.

Convexity of total variation implies stable.
\end{frame}

\begin{frame}{Proof of the main theorem}
\begin{lemma}
Disjoint stable minimal hypersurfaces form a lamination.
\end{lemma}
    
Need to construct charts $U$ in which 
$$\mathrm{supp} \dif v \cap U \cong K \times (0, 1)^{d - 1}$$
where $K \subseteq \RR$ is a closed set.
So we have to uniformly represent the level sets of $v$ as graphs (Solomon '86, Colding and Minicozzi '05), which is possible if
$$\sup_{y \in \RR} \|\Two_{\partial \{v > y\}}\|_{C^0(U)} < \infty.$$ \pause

\begin{theorem}[Schoen '83, Chodosh and Li '21]
Let $N \subset M$ be a stable minimal hypersurface with trivial normal bundle.
If $\dim M \leq 4$ and $M$ has bounded geometry, then
$$\|\Two_N\|_{C^0(M)} \leq C_M.$$
\end{theorem}
\end{frame}

% \begin{frame}{Proof of the main theorem}
% \begin{theorem}[stable Bernstein theorem; Schoen '83 ($\dim M = 3$), Chodosh and Li '21 ($\dim M = 4$)]
% Let $N \subset M$ be a stable minimal hypersurface with trivial normal bundle. If $\dim M \leq 4$, then
% $$\|\Two_N\|_{C^0(M)} \leq C_M.$$
% \end{theorem}

% Here $d = 2$ is obvious, $d = 3$ relies on quadratic area growth, and $d = 4$ uses a slicing argument and the fact that $d - 2 = 2$ so we can use Gauß-Bonnet. So not so clear how to extend to $d \leq 7$...

% Anyways, the (necessarily stable and with trivial normal bundle) level sets of a $1$-harmonic function have uniform $\Two$ bounds, hence form a lamination.
% \end{frame}

\begin{frame}{Proof of the main theorem}
\begin{lemma}
If $\dif v$ is exact and arises from a minimal lamination, then $v$ is $1$-harmonic.
\end{lemma}

This follows easily from the coarea formula 
$$\int_M |\dif v| \dif V = \int_{-\infty}^\infty |\partial \{v > y\}| \dif y$$
where the right-hand side is minimized because $\dif v$ comes from a minimal lamination.
So $\dif v$ minimizes the total variation.
\end{frame}

\begin{frame}{A possible application}
Sayeth Thurston '98:

\begin{quote}
I currently think that a characterization of minimal stretch [that is, best Lipschitz] maps should be possible in a considerably more general context [than closed hyperbolic surfaces] ... with a simpler proof based on general principles -- in particular, the max flow min cut principle, convexity, and $L^0 \leftrightarrow L^\infty$ duality.
\end{quote}
    
``Max flow min cut principle'' refers to convex duality probably?
In dimension $2$, the convex dual problem to the $1$-Laplacian is the $\infty$-Laplacian.
\end{frame}

\begin{frame}{A possible application}
At least formally, in dimension $3$ the convex dual problem to finding a local $1$-harmonic function is the problem of finding a connection on a line bundle which minimizes the $L^\infty$ norm of its curvature. \pause

(If $M$ is aspherical) here's a dictionary between the two situations:

\begin{table}[]
\begin{tabular}{ll}
Closed surface $M$                                              & Closed threefold $M$                         \\
Geodesic laminations                                            & Minimal laminations                          \\
Homotopy classes $M \to \Sph^1$ & Line bundles \\
Maps $M \to \Sph^1$                                             & Connections                                  \\
$p$-Laplacian                                                   & (elliptic) $p$-Maxwell equation              \\
Best Lipschitz constant                                         & Minimizing $L^\infty$ norm of the curvature 
\end{tabular}
\end{table} \pause 

The natural analogue of the Daskalopolous--Uhlenbeck theorem is:
\end{frame}

\begin{frame}{A possible application}
    
\begin{conjecture}
    Let $\mathscr L \to M$ be a line bundle over a closed Riemannian threefold.
    \begin{enumerate}
    \item There exists a connection on $\mathscr L$ which minimizes the $L^\infty$ norm of its curvature $F$, and which solves the $\infty$-Maxwell equation 
    $$F^{ij} \nabla_j |F|^2 = 0$$
    in a suitable (limit of renormalized $p$-Maxwell) sense.
    \item $F$ attains its $L^\infty$ norm on a set containing a measured oriented minimal lamination $\lambda$ which only depends on $\mathscr L$.
    \item There locally exists a $1$-harmonic function $v$, such that $\dif v$ is the current associated to $\lambda$, and 
    $$\dif v \wedge F = |\dif v| \dif V.$$
    \end{enumerate}
    \end{conjecture}
\end{frame}

\begin{frame}{Thanks for coming!}
    Questions?
\end{frame}

\end{document}
