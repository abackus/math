\documentclass[reqno,11pt]{amsart}
\usepackage[letterpaper, margin=1in]{geometry}
\RequirePackage{amsmath,amssymb,amsthm,graphicx,mathrsfs,url,slashed,subcaption}
\RequirePackage[usenames,dvipsnames]{xcolor}
\RequirePackage[colorlinks=true,linkcolor=Red,citecolor=Green]{hyperref}
\RequirePackage{amsxtra}
\usepackage{cancel}
\usepackage{tikz, quiver, wrapfig}
%\usepackage[T1]{fontenc}

% \setlength{\textheight}{9.3in} \setlength{\oddsidemargin}{-0.25in}
% \setlength{\evensidemargin}{-0.25in} \setlength{\textwidth}{7in}
% \setlength{\topmargin}{-0.25in} \setlength{\headheight}{0.18in}
% \setlength{\marginparwidth}{1.0in}
% \setlength{\abovedisplayskip}{0.2in}
% \setlength{\belowdisplayskip}{0.2in}
% \setlength{\parskip}{0.05in}
%\renewcommand{\baselinestretch}{1.05}

\title[Laminations and calibrations II]{Minimal laminations and tight calibrations II: The max flow/min cut theorem}
\author{Aidan Backus}
\address{Department of Mathematics, Brown University}
\email{aidan\_backus@brown.edu}
\date{\today}

\newcommand{\NN}{\mathbf{N}}
\newcommand{\ZZ}{\mathbf{Z}}
\newcommand{\QQ}{\mathbf{Q}}
\newcommand{\RR}{\mathbf{R}}
\newcommand{\CC}{\mathbf{C}}
\newcommand{\DD}{\mathbf{D}}
\newcommand{\PP}{\mathbf P}
\newcommand{\MM}{\mathbf M}
\newcommand{\II}{\mathbf I}
\newcommand{\Hyp}{\mathbf H}
\newcommand{\Sph}{\mathbf S}
\newcommand{\Group}{\mathbf G}
\newcommand{\GL}{\mathbf{GL}}
\newcommand{\Orth}{\mathbf{O}}
\newcommand{\SpOrth}{\mathbf{SO}}
\newcommand{\Ball}{\mathbf{B}}

\newcommand*\dif{\mathop{}\!\mathrm{d}}

\DeclareMathOperator{\card}{card}
\DeclareMathOperator{\dist}{dist}
\DeclareMathOperator{\Gal}{Gal}
\DeclareMathOperator{\id}{id}
\DeclareMathOperator{\Hom}{Hom}
\DeclareMathOperator{\coker}{coker}
\DeclareMathOperator{\supp}{supp}
\DeclareMathOperator{\Teich}{Teich}
\DeclareMathOperator{\tr}{tr}

\newcommand{\Leaves}{\mathscr L}
\newcommand{\Lagrange}{\mathscr L}
\newcommand{\Hypspace}{\mathscr H}

\newcommand{\Chain}{\underline C}

\newcommand{\Two}{\mathrm{I\!I}}
\newcommand{\Ric}{\mathrm{Ric}}

\newcommand{\normal}{\mathbf n}
\newcommand{\radial}{\mathbf r}
\newcommand{\evect}{\mathbf e}
\newcommand{\vol}{\mathrm{vol}}

\DeclareMathOperator{\diam}{diam}
\newcommand{\Ell}{\mathrm{Ell}}
\newcommand{\inj}{\mathrm{inj}}
\newcommand{\Lip}{\mathrm{Lip}}
\newcommand{\MCL}{\mathrm{MCL}}
\newcommand{\Riem}{\mathrm{Riem}}

\newcommand{\Mass}{\mathbf M}
\newcommand{\Comass}{\mathbf L}

\newcommand{\Min}{\mathrm{Min}}
\newcommand{\Max}{\mathrm{Max}}

\newcommand{\dfn}[1]{\emph{#1}\index{#1}}

\renewcommand{\Re}{\operatorname{Re}}
\renewcommand{\Im}{\operatorname{Im}}

\newcommand{\loc}{\mathrm{loc}}
\newcommand{\cpt}{\mathrm{cpt}}

\def\Japan#1{\left \langle #1 \right \rangle}

\newtheorem{theorem}{Theorem}[section]
\newtheorem{badtheorem}[theorem]{``Theorem"}
\newtheorem{prop}[theorem]{Proposition}
\newtheorem{lemma}[theorem]{Lemma}
\newtheorem{sublemma}[theorem]{Sublemma}
\newtheorem{proposition}[theorem]{Proposition}
\newtheorem{corollary}[theorem]{Corollary}
\newtheorem{conjecture}[theorem]{Conjecture}
\newtheorem{axiom}[theorem]{Axiom}
\newtheorem{assumption}[theorem]{Assumption}

\newtheorem{mainthm}{Theorem}
\renewcommand{\themainthm}{\Alph{mainthm}}

\newtheorem{claim}{Claim}[theorem]
\renewcommand{\theclaim}{\thetheorem\Alph{claim}}
% \newtheorem*{claim}{Claim}

\theoremstyle{definition}
\newtheorem{definition}[theorem]{Definition}
\newtheorem{remark}[theorem]{Remark}
\newtheorem{example}[theorem]{Example}
\newtheorem{notation}[theorem]{Notation}

\newtheorem{exercise}[theorem]{Discussion topic}
\newtheorem{homework}[theorem]{Homework}
\newtheorem{problem}[theorem]{Problem}

\makeatletter
\newcommand{\proofpart}[2]{%
  \par
  \addvspace{\medskipamount}%
  \noindent\emph{Part #1: #2.}
}
\makeatother



\numberwithin{equation}{section}


% Mean
\def\Xint#1{\mathchoice
{\XXint\displaystyle\textstyle{#1}}%
{\XXint\textstyle\scriptstyle{#1}}%
{\XXint\scriptstyle\scriptscriptstyle{#1}}%
{\XXint\scriptscriptstyle\scriptscriptstyle{#1}}%
\!\int}
\def\XXint#1#2#3{{\setbox0=\hbox{$#1{#2#3}{\int}$ }
\vcenter{\hbox{$#2#3$ }}\kern-.6\wd0}}
\def\ddashint{\Xint=}
\def\dashint{\Xint-}

\usepackage[backend=bibtex,style=alphabetic,giveninits=true]{biblatex}
\renewcommand*{\bibfont}{\normalfont\footnotesize}
\addbibresource{best_curl.bib}
\renewbibmacro{in:}{}
\DeclareFieldFormat{pages}{#1}

\newcommand\todo[1]{\textcolor{red}{TODO: #1}}


\begin{document}
\begin{abstract}
We introduce a family of closed $d-1$-forms on Riemannian $d$-manifolds which minimize their comass (or $L^\infty$ norm) in their cohomology class, which we call \dfn{tight}.
Tight forms have properties similar to (gradients of) $\infty$-harmonic maps between surfaces: they are convex duals of $1$-harmonic functions and attain their comass on a measured oriented minimal lamination $\mu$.
We show that $\mu$ has properties analogous to a measured sublamination of Thurston's canonical lamination.
\todo{Rewrite this}
\end{abstract}

\maketitle

%%%%%%%%%%%%%%%%%%%%%%%%%%%%%%%%%%%%%%%%%%%%%%%%%%%%%%%
\section{Introduction}
The classical \dfn{max flow min cut theorem} asserts that in a discrete flow network $\mathcal G$, the maximal flow through $\mathcal G$ is equal to the \dfn{minimal cut} of $\mathcal G$, namely the minimal total capacity of any set of edges which divide $\mathcal G$ into a source component and a sink component; this theorem is a concrete form of the duality theorem for convex optimization.
As we shall survey below, various continuous versions of the max flow min cut theorem have since appeared, which usually replace the maximal flow by a \dfn{calibration} (a closed $d - 1$-form, or equivalently a divergence-free vector field, of $L^\infty$ norm $1$), and the minimal cut by a minimal submanifold.
In particular, Bangert and Cui have shown that given a continuous calibration $F$ on a closed Riemannian manifold of dimension $\leq 7$, such that $\|F\|_{L^\infty}$ is minimized in its cohomology class, there exists a minimal lamination calibrated by $F$ \cite[Theorem 5.1]{bangert_cui_2017}.

We shall prove a particularly strong version of the Bangert--Cui max flow min cut theorem.
We only need to assume that the calibration $F$ is measurable; moreover, it shall follow from our methods that the calibrated lamination is Lipschitz.
We use the approach of $p$-harmonic duality which was pioneered by Daskalopolous and Uhlenbeck in the context of best Lipschitz maps \cite{daskalopoulos2020transverse,daskalopoulos2022,daskalopoulos2023}.
To be more precise, we pick out a special class of calibrations, which we call \dfn{tight forms}, which are the variational solutions of a suitable generalization of the $\infty$-Laplace equation.
Tight forms are necessarily convex duals of functions of least gradients, and because functions of least gradient induce minimal laminations \cite[Theorem B]
{BackusCML}, the main theorem follows.

Along the way, we establish several auxiliary results that may be of independent interest.
We establish the necessary geometric measure theory to study $L^\infty$ calibrations; we construct a minimal lamination, which we call the \dfn{canonical lamination}, associated to each cohomology class in $H^{d - 1}(M, \RR)$ which analogous to Thurston's canonical lamination of a closed hyperbolic surface \cite[\S8]{Thurston98}; and we study some analytic questions concerning the Euler-Lagrange equation for a $C^1$ tight form.
We have also included several conjectures concerning the canonical lamination and the Euler-Lagrange equation.

%%%%%%%%%%%%%%%
\subsection{Max flow min cut theorems}

\begin{figure}
	\begin{tikzpicture}
\draw[rounded corners=35pt](6,-1)--(4.2,-1)--(2,-2)--(0,0)--(2,2)--(4.2,1)--(7,1)--(9.2,2)--(11,0)
--(9,-2)--(6,-1);
\draw (1.5,0.2) arc (175:315:1cm and 0.5cm);
\draw (3,-0.28) arc (-30:180:0.7cm and 0.3cm);
\draw[blue] (1.8,-0.3) arc (0:360:0.63cm and 0.3cm);
\draw (7.5,0.2) arc (175:315:1cm and 0.5cm);
\draw (9,-0.28) arc (-30:180:0.7cm and 0.3cm);
\end{tikzpicture}

\caption{A closed surface containing a homologically minimizing closed geodesic $N^*$ (in blue).
In Figure \ref{discretized} we show a discretization of a tubular neighborhood $M'$ of $N^*$.
Adapted from \cite{Orduz20}. \label{to discretize}}
\end{figure}

Recall the max flow min cut theorem \cite[Chapter 7]{umesh2006algorithms}.
A \dfn{flow network} $\mathcal G = (V, E, c, s, t)$ consists of a finite directed graph $(V, E)$, a \dfn{capacity} $c: E \to \RR_+$, a \dfn{source} $s \in V$, and a \dfn{sink} $t \in V$.
For $v \in V$, let $I(v), O(v) \subseteq E$ be the sets of edges in and out of $v$, respectively.
A \dfn{flow} in $\mathcal G$ is a function $F: E \to \RR_+$ such that $F \leq c$, and which is \dfn{divergence-free} in the sense that for each $v \in V \setminus \{s, t\}$,
$$\sum_{e \in I(v)} F(e) = \sum_{e \in O(v)} F(e).$$
A \dfn{cut} of $\mathcal G$ is a partition $V = S \sqcup T$ such that $s \in S$, $t \in T$; in that case we write $E(S, T) := \{(v, w) \in E: v \in S, w \in T\}$.

\begin{theorem}[max flow min cut, graphs]
Let $\mathcal G$ be a flow network. Then 
$$\max_{F \in \mathrm{Flows}(\mathcal G)} \sum_{e \in O(s)} F(e) = \min_{(S, T) \in \mathrm{Cuts}(\mathcal G)} \sum_{e \in E(S, T)} c(e).$$
In particular, if $F$ is a maximal flow in $\mathcal G$ and $(S, T)$ is a minimal cut of $\mathcal G$, then for each $e \in E(S, T)$, $F(e) = c(e)$.
\end{theorem}

Suppose that we have a closed Riemannian manifold $M$ and a homology class $\sigma \in H_{d - 1}(M, \RR)$, and we want to compute the minimal area of a hypersurface in $\sigma$.
We propose to discretize this problem, as in \cite[Appendix A]{Freedman_2016}.

Assume that we know that $N$ is contained in a tubular neighborhood $M' = N_0 \times (-1, 1) \subset M$ where $[N_0] = \sigma$.
Then we can discretize $M'$ at a length scale $0 < h \ll 1$ to get a graph $(V, E)$, where between any two adjacent vertices there are edges in both directions, and we have two ``boundary'' vertices $s, t$, one for each component of $N_0 \times \{-1, 1\}$.
Then $\mathcal G := (V, E, 1, s, t)$ is a flow network.
Roughly speaking, flows correspond to closed $d - 1$-forms $F$ (which we view as fluxes of divergence-free vector fields) such that $\|F\|_{L^\infty} = 1$, while cuts correspond to hypersurfaces in $\sigma$, and minimal cuts correspond to homologically minimizing hypersurfaces.
If $N$ is a hypersurface in $\sigma$, $F$ is a closed $d - 1$-form, and $N^*$ is a homologically minimizing hypersurface in $\sigma$, then by Stokes' theorem,
\begin{equation}\label{one sided max flow min cut}
\int_N F = \int_{N^*} F \leq \mathrm{Area}(N^*).
\end{equation}
See Figures \ref{to discretize} and \ref{discretized}.
Ignoring all questions about the validity of our discretization, one expects from the max flow min cut theorem that (\ref{one sided max flow min cut}) is sharp:
We have the \dfn{continuous max flow min cut theorem} that for any closed homologically minimizing hypersurface $N^*$, there exists a closed $d - 1$-form $F$ such that $\|F\|_{L^\infty} = 1$ such that for any hypersurface $N$ in $\sigma$,
\begin{equation}\label{continuous max flow min cut}
\int_N F = \mathrm{Area}(N^*).
\end{equation}

% https://q.uiver.app/#q=WzAsMTksWzAsMywiXFxidWxsZXQiXSxbMSwwLCJcXGJ1bGxldCJdLFsxLDEsIlxcYnVsbGV0Il0sWzEsMiwiXFxidWxsZXQiXSxbMSwzLCJcXGJ1bGxldCJdLFsxLDQsIlxcYnVsbGV0Il0sWzEsNSwiXFxidWxsZXQiXSxbMSw2LCJcXGJ1bGxldCJdLFsyLDIsIlxcYnVsbGV0Il0sWzIsMywiXFxidWxsZXQiXSxbMiw0LCJcXGJ1bGxldCJdLFszLDAsIlxcYnVsbGV0Il0sWzMsMSwiXFxidWxsZXQiXSxbMywyLCJcXGJ1bGxldCJdLFszLDMsIlxcYnVsbGV0Il0sWzMsNCwiXFxidWxsZXQiXSxbMyw1LCJcXGJ1bGxldCJdLFszLDYsIlxcYnVsbGV0Il0sWzQsMywiXFxidWxsZXQiXSxbMCw0LCIiLDAseyJzdHlsZSI6eyJib2R5Ijp7Im5hbWUiOiJkYXNoZWQifSwiaGVhZCI6eyJuYW1lIjoibm9uZSJ9fX1dLFswLDMsIiIsMix7InN0eWxlIjp7ImJvZHkiOnsibmFtZSI6ImRhc2hlZCJ9LCJoZWFkIjp7Im5hbWUiOiJub25lIn19fV0sWzAsMiwiIiwyLHsic3R5bGUiOnsiYm9keSI6eyJuYW1lIjoiZGFzaGVkIn0sImhlYWQiOnsibmFtZSI6Im5vbmUifX19XSxbMCwxLCIiLDIseyJzdHlsZSI6eyJib2R5Ijp7Im5hbWUiOiJkYXNoZWQifSwiaGVhZCI6eyJuYW1lIjoibm9uZSJ9fX1dLFswLDUsIiIsMix7InN0eWxlIjp7ImJvZHkiOnsibmFtZSI6ImRhc2hlZCJ9LCJoZWFkIjp7Im5hbWUiOiJub25lIn19fV0sWzAsNiwiIiwyLHsic3R5bGUiOnsiYm9keSI6eyJuYW1lIjoiZGFzaGVkIn0sImhlYWQiOnsibmFtZSI6Im5vbmUifX19XSxbMCw3LCIiLDIseyJzdHlsZSI6eyJib2R5Ijp7Im5hbWUiOiJkYXNoZWQifSwiaGVhZCI6eyJuYW1lIjoibm9uZSJ9fX1dLFsxLDIsIiIsMix7InN0eWxlIjp7ImJvZHkiOnsibmFtZSI6ImRhc2hlZCJ9LCJoZWFkIjp7Im5hbWUiOiJub25lIn19fV0sWzIsMywiIiwyLHsic3R5bGUiOnsiYm9keSI6eyJuYW1lIjoiZGFzaGVkIn0sImhlYWQiOnsibmFtZSI6Im5vbmUifX19XSxbMyw0LCIiLDIseyJzdHlsZSI6eyJib2R5Ijp7Im5hbWUiOiJkYXNoZWQifSwiaGVhZCI6eyJuYW1lIjoibm9uZSJ9fX1dLFs0LDUsIiIsMix7InN0eWxlIjp7ImJvZHkiOnsibmFtZSI6ImRhc2hlZCJ9LCJoZWFkIjp7Im5hbWUiOiJub25lIn19fV0sWzUsNiwiIiwyLHsic3R5bGUiOnsiYm9keSI6eyJuYW1lIjoiZGFzaGVkIn0sImhlYWQiOnsibmFtZSI6Im5vbmUifX19XSxbNiw3LCIiLDEseyJzdHlsZSI6eyJib2R5Ijp7Im5hbWUiOiJkYXNoZWQifSwiaGVhZCI6eyJuYW1lIjoibm9uZSJ9fX1dLFsxNCwxOCwiIiwxLHsic3R5bGUiOnsiYm9keSI6eyJuYW1lIjoiZGFzaGVkIn0sImhlYWQiOnsibmFtZSI6Im5vbmUifX19XSxbMTUsMTgsIiIsMSx7InN0eWxlIjp7ImJvZHkiOnsibmFtZSI6ImRhc2hlZCJ9LCJoZWFkIjp7Im5hbWUiOiJub25lIn19fV0sWzE2LDE4LCIiLDEseyJzdHlsZSI6eyJib2R5Ijp7Im5hbWUiOiJkYXNoZWQifSwiaGVhZCI6eyJuYW1lIjoibm9uZSJ9fX1dLFsxNywxOCwiIiwxLHsic3R5bGUiOnsiYm9keSI6eyJuYW1lIjoiZGFzaGVkIn0sImhlYWQiOnsibmFtZSI6Im5vbmUifX19XSxbMTMsMTgsIiIsMSx7InN0eWxlIjp7ImJvZHkiOnsibmFtZSI6ImRhc2hlZCJ9LCJoZWFkIjp7Im5hbWUiOiJub25lIn19fV0sWzEyLDE4LCIiLDEseyJzdHlsZSI6eyJib2R5Ijp7Im5hbWUiOiJkYXNoZWQifSwiaGVhZCI6eyJuYW1lIjoibm9uZSJ9fX1dLFsxMSwxOCwiIiwxLHsic3R5bGUiOnsiYm9keSI6eyJuYW1lIjoiZGFzaGVkIn0sImhlYWQiOnsibmFtZSI6Im5vbmUifX19XSxbMTEsMTIsIiIsMSx7InN0eWxlIjp7ImJvZHkiOnsibmFtZSI6ImRhc2hlZCJ9LCJoZWFkIjp7Im5hbWUiOiJub25lIn19fV0sWzEyLDEzLCIiLDEseyJzdHlsZSI6eyJib2R5Ijp7Im5hbWUiOiJkYXNoZWQifSwiaGVhZCI6eyJuYW1lIjoibm9uZSJ9fX1dLFsxMywxNCwiIiwxLHsic3R5bGUiOnsiYm9keSI6eyJuYW1lIjoiZGFzaGVkIn0sImhlYWQiOnsibmFtZSI6Im5vbmUifX19XSxbMTQsMTUsIiIsMSx7InN0eWxlIjp7ImJvZHkiOnsibmFtZSI6ImRhc2hlZCJ9LCJoZWFkIjp7Im5hbWUiOiJub25lIn19fV0sWzE1LDE2LCIiLDEseyJzdHlsZSI6eyJib2R5Ijp7Im5hbWUiOiJkYXNoZWQifSwiaGVhZCI6eyJuYW1lIjoibm9uZSJ9fX1dLFsxNiwxNywiIiwxLHsic3R5bGUiOnsiYm9keSI6eyJuYW1lIjoiZGFzaGVkIn0sImhlYWQiOnsibmFtZSI6Im5vbmUifX19XSxbMTcsMTEsIiIsMSx7ImN1cnZlIjoyLCJzdHlsZSI6eyJib2R5Ijp7Im5hbWUiOiJkYXNoZWQifSwiaGVhZCI6eyJuYW1lIjoibm9uZSJ9fX1dLFsxMCw5LCIiLDEseyJjb2xvdXIiOlsyMzcsMTAwLDQ2XSwic3R5bGUiOnsiaGVhZCI6eyJuYW1lIjoibm9uZSJ9fX1dLFs5LDgsIiIsMSx7ImNvbG91ciI6WzIzNywxMDAsNDZdLCJzdHlsZSI6eyJoZWFkIjp7Im5hbWUiOiJub25lIn19fV0sWzgsMTAsIiIsMSx7ImN1cnZlIjotMiwiY29sb3VyIjpbMjM3LDEwMCw0Nl0sInN0eWxlIjp7ImhlYWQiOnsibmFtZSI6Im5vbmUifX19XSxbMSw3LCIiLDIseyJjdXJ2ZSI6LTIsInN0eWxlIjp7ImJvZHkiOnsibmFtZSI6ImRhc2hlZCJ9fX1dLFs4LDEsIiIsMSx7InN0eWxlIjp7ImJvZHkiOnsibmFtZSI6ImRhc2hlZCJ9LCJoZWFkIjp7Im5hbWUiOiJub25lIn19fV0sWzIsOCwiIiwxLHsic3R5bGUiOnsiYm9keSI6eyJuYW1lIjoiZGFzaGVkIn0sImhlYWQiOnsibmFtZSI6Im5vbmUifX19XSxbMyw4LCIiLDIseyJzdHlsZSI6eyJib2R5Ijp7Im5hbWUiOiJkYXNoZWQifSwiaGVhZCI6eyJuYW1lIjoibm9uZSJ9fX1dLFszLDksIiIsMix7InN0eWxlIjp7ImJvZHkiOnsibmFtZSI6ImRhc2hlZCJ9LCJoZWFkIjp7Im5hbWUiOiJub25lIn19fV0sWzQsOSwiIiwyLHsic3R5bGUiOnsiYm9keSI6eyJuYW1lIjoiZGFzaGVkIn0sImhlYWQiOnsibmFtZSI6Im5vbmUifX19XSxbNCw4LCIiLDIseyJzdHlsZSI6eyJib2R5Ijp7Im5hbWUiOiJkYXNoZWQifSwiaGVhZCI6eyJuYW1lIjoibm9uZSJ9fX1dLFs1LDksIiIsMix7InN0eWxlIjp7ImJvZHkiOnsibmFtZSI6ImRhc2hlZCJ9LCJoZWFkIjp7Im5hbWUiOiJub25lIn19fV0sWzQsMTAsIiIsMix7InN0eWxlIjp7ImJvZHkiOnsibmFtZSI6ImRhc2hlZCJ9LCJoZWFkIjp7Im5hbWUiOiJub25lIn19fV0sWzUsMTAsIiIsMix7InN0eWxlIjp7ImJvZHkiOnsibmFtZSI6ImRhc2hlZCJ9LCJoZWFkIjp7Im5hbWUiOiJub25lIn19fV0sWzYsMTAsIiIsMix7InN0eWxlIjp7ImJvZHkiOnsibmFtZSI6ImRhc2hlZCJ9LCJoZWFkIjp7Im5hbWUiOiJub25lIn19fV0sWzcsMTAsIiIsMix7InN0eWxlIjp7ImJvZHkiOnsibmFtZSI6ImRhc2hlZCJ9LCJoZWFkIjp7Im5hbWUiOiJub25lIn19fV0sWzgsMTEsIiIsMix7InN0eWxlIjp7ImJvZHkiOnsibmFtZSI6ImRhc2hlZCJ9LCJoZWFkIjp7Im5hbWUiOiJub25lIn19fV0sWzgsMTIsIiIsMix7InN0eWxlIjp7ImJvZHkiOnsibmFtZSI6ImRhc2hlZCJ9LCJoZWFkIjp7Im5hbWUiOiJub25lIn19fV0sWzgsMTMsIiIsMix7InN0eWxlIjp7ImJvZHkiOnsibmFtZSI6ImRhc2hlZCJ9LCJoZWFkIjp7Im5hbWUiOiJub25lIn19fV0sWzgsMTQsIiIsMix7InN0eWxlIjp7ImJvZHkiOnsibmFtZSI6ImRhc2hlZCJ9LCJoZWFkIjp7Im5hbWUiOiJub25lIn19fV0sWzksMTMsIiIsMix7InN0eWxlIjp7ImJvZHkiOnsibmFtZSI6ImRhc2hlZCJ9LCJoZWFkIjp7Im5hbWUiOiJub25lIn19fV0sWzksMTQsIiIsMix7InN0eWxlIjp7ImJvZHkiOnsibmFtZSI6ImRhc2hlZCJ9LCJoZWFkIjp7Im5hbWUiOiJub25lIn19fV0sWzksMTUsIiIsMix7InN0eWxlIjp7ImJvZHkiOnsibmFtZSI6ImRhc2hlZCJ9LCJoZWFkIjp7Im5hbWUiOiJub25lIn19fV0sWzEwLDE0LCIiLDIseyJzdHlsZSI6eyJib2R5Ijp7Im5hbWUiOiJkYXNoZWQifSwiaGVhZCI6eyJuYW1lIjoibm9uZSJ9fX1dLFsxMCwxNSwiIiwyLHsic3R5bGUiOnsiYm9keSI6eyJuYW1lIjoiZGFzaGVkIn0sImhlYWQiOnsibmFtZSI6Im5vbmUifX19XSxbMTAsMTYsIiIsMix7InN0eWxlIjp7ImJvZHkiOnsibmFtZSI6ImRhc2hlZCJ9LCJoZWFkIjp7Im5hbWUiOiJub25lIn19fV0sWzEwLDE3LCIiLDIseyJzdHlsZSI6eyJib2R5Ijp7Im5hbWUiOiJkYXNoZWQifSwiaGVhZCI6eyJuYW1lIjoibm9uZSJ9fX1dXQ==

\begin{figure}
\[\begin{tikzcd}
	& \bullet && \bullet \\
	& \bullet && \bullet \\
	& \bullet & \bullet & \bullet \\
	s & \bullet & \bullet & \bullet & t \\
	& \bullet & \bullet & \bullet \\
	& \bullet && \bullet \\
	& \bullet && \bullet
	\arrow[dashed, no head, from=4-1, to=4-2]
	\arrow[dashed, no head, from=4-1, to=3-2]
	\arrow[dashed, no head, from=4-1, to=2-2]
	\arrow[dashed, no head, from=4-1, to=1-2]
	\arrow[dashed, no head, from=4-1, to=5-2]
	\arrow[dashed, no head, from=4-1, to=6-2]
	\arrow[dashed, no head, from=4-1, to=7-2]
	\arrow[dashed, no head, from=1-2, to=2-2]
	\arrow[dashed, no head, from=2-2, to=3-2]
	\arrow[dashed, no head, from=3-2, to=4-2]
	\arrow[dashed, no head, from=4-2, to=5-2]
	\arrow[dashed, no head, from=5-2, to=6-2]
	\arrow[dashed, no head, from=6-2, to=7-2]
	\arrow[dashed, no head, from=4-4, to=4-5]
	\arrow[dashed, no head, from=5-4, to=4-5]
	\arrow[dashed, no head, from=6-4, to=4-5]
	\arrow[dashed, no head, from=7-4, to=4-5]
	\arrow[dashed, no head, from=3-4, to=4-5]
	\arrow[dashed, no head, from=2-4, to=4-5]
	\arrow[dashed, no head, from=1-4, to=4-5]
	\arrow[dashed, no head, from=1-4, to=2-4]
	\arrow[dashed, no head, from=2-4, to=3-4]
	\arrow[dashed, no head, from=3-4, to=4-4]
	\arrow[dashed, no head, from=4-4, to=5-4]
	\arrow[dashed, no head, from=5-4, to=6-4]
	\arrow[dashed, no head, from=6-4, to=7-4]
	\arrow[curve={height=12pt}, dashed, no head, from=7-4, to=1-4]
	\arrow[color={rgb,255:red,0;green,0;blue,235}, no head, from=5-3, to=4-3]
	\arrow[color={rgb,255:red,0;green,0;blue,235}, no head, from=4-3, to=3-3]
	\arrow[color={rgb,255:red,0;green,0;blue,235}, curve={height=-12pt}, no head, from=3-3, to=5-3]
	\arrow[curve={height=-12pt}, dashed, no head, from=1-2, to=7-2]
	\arrow[dashed, no head, from=3-3, to=1-2]
	\arrow[dashed, no head, from=2-2, to=3-3]
	\arrow[dashed, no head, from=3-2, to=3-3]
	\arrow[dashed, no head, from=3-2, to=4-3]
	\arrow[dashed, no head, from=4-2, to=4-3]
	\arrow[dashed, no head, from=4-2, to=3-3]
	\arrow[dashed, no head, from=5-2, to=4-3]
	\arrow[dashed, no head, from=4-2, to=5-3]
	\arrow[dashed, no head, from=5-2, to=5-3]
	\arrow[dashed, no head, from=6-2, to=5-3]
	\arrow[dashed, no head, from=7-2, to=5-3]
	\arrow[dashed, no head, from=3-3, to=1-4]
	\arrow[dashed, no head, from=3-3, to=2-4]
	\arrow[dashed, no head, from=3-3, to=3-4]
	\arrow[dashed, no head, from=3-3, to=4-4]
	\arrow[dashed, no head, from=4-3, to=3-4]
	\arrow[dashed, no head, from=4-3, to=4-4]
	\arrow[dashed, no head, from=4-3, to=5-4]
	\arrow[dashed, no head, from=5-3, to=4-4]
	\arrow[dashed, no head, from=5-3, to=5-4]
	\arrow[dashed, no head, from=5-3, to=6-4]
	\arrow[dashed, no head, from=5-3, to=7-4]
\end{tikzcd}\]
\caption{With $M'$ as in Figure \ref{to discretize}, the discretization of $M'$ is a flow network $\mathcal G$ in which the discretization of $N^*$ is a minimal cut (in blue and solid).
Therefore the maximal flow $F$ through $\mathcal G$ must have flux density $1$ through the discretization of $N^*$, and hence the closed $d - 1$-form corresponding to $F$ must be the area form on $N^*$. \label{discretized}}
\end{figure}

\subsubsection{Federer's version}
Modulo regularity issues, the continuous max flow min cut theorem was proven by Federer \cite[\S4.15]{Federer1974}.
In the modern formulation of Federer's theorem, we introduce the stable norm on homology:

\begin{definition}
The \dfn{stable norm} $\Mass(\sigma)$ of a homology class $\sigma \in H_{d - 1}(M, \RR)$ is the infimum of the mass $\Mass(N)$ among all representative chains $N$.
The \dfn{costable norm} $\Comass(\rho)$ of a cohomology class $\rho \in H^{d - 1}(M, \RR)$ is the infimum of $\|F\|_{L^\infty}$ among all representative forms $F$.
\end{definition}

\begin{theorem}[max flow min cut, Federer's version]\label{Federer}
Let $M$ be a closed Riemannian manifold and $\sigma \in H_{d - 1}(M, \RR)$.
Then there exists a cohomology class $\rho \in H^{d - 1}(M, \RR)$ such that $\Comass(\rho) \leq 1$, and 
\begin{equation}\label{Federer duality}
\Mass(\sigma) = \langle \rho, \sigma\rangle.
\end{equation}
\end{theorem}

Theorem \ref{Federer} has found myriad applications, including in computational geometry \cite{sullivan1990crystalline} and string theory \cite{Freedman_2016}; both references here observe that Federer has essentially proven a max flow min cut theorem.
However, Federer's theorem is nothing more than the Hanh-Banach theorem in the special case of the (finite-dimensional) Banach space $H_{d - 1}(M, \RR)$ equipped with the stable norm.
It does not say anything about the pointwise duality of the chains and cochains which realize the infima in the definitions of the stable and costable norms.\footnote{As we have formulated Theorem \ref{Federer}, it is not clear whether the infima are attained at all, but for sufficiently general notions of chain and cochain, this is not hard to show.}

\subsubsection{Thurston's version}
An analogous result to Federer's appears in Thurston's work on best Lipschitz maps \cite{Thurston98}.
A \dfn{best Lipschitz map} $u$ is a map which minimizes its Lipschitz constant, among all maps homotopic to $u$.
% Thurston's motivation was to define a Finsler metric on the Teichm\"uller space $\widetilde{\mathscr M_g}$: the \dfn{Thurston asymmetric distance} between two hyperbolic structures on a closed surface $S$ of genus $g \geq 2$ is the Lipschitz constant of a best Lipschitz map homotopic to $\id_S$.
% Thurston's asymmetric metric is a particularly appealing geometry on $\widetilde{\mathscr M_g}$ because of its intimate connection with the structure of geodesic laminations on hyperbolic surfaces \cite{Thurston98, Gu_ritaud_2017}:
Such maps are intimately connected to geodesic laminations in hyperbolic surfaces:

\begin{theorem}[max flow min cut, Thurston's version]\label{existence of thurston lamination}
Let $f: M \to N$ be a homeomorphism of closed hyperbolic surfaces.
Then there exists a best Lipschitz map $u$ homotopic to $f$, and a measured geodesic lamination $\mu$ on $M$, such that $\mu$ pushes forward to a measured geodesic lamination on $N$, and
\begin{equation}\label{L is K}
	\Lip(u) = \frac{\Mass(u_* \mu)}{\Mass(\mu)}.
\end{equation}
Moreover, $\mu$ maximizes the ratio $\Mass(u_* \lambda)/\Mass(\lambda)$ among measured laminations on $M$.
\end{theorem}

Thurston established Theorem \ref{existence of thurston lamination} in a tour de force of geometric topology, but he conjectured that a proof of his result ``should be feasible with a simpler proof based on more general principles -- in particular, the max flow min cut principle, convexity, and $L^0 \leftrightarrow L^\infty$ duality'' \cite[Abstract]{Thurston98}.

Theorem \ref{existence of thurston lamination} does not follow from our main theorem; however, the analogous result for maps from closed hyperbolic surfaces to $\Sph^1$, which was proven by Daskalopolous and Uhlenbeck \cite{daskalopoulos2020transverse}, is a special case of our main result.
Our work, like \cite{daskalopoulos2020transverse}, is based on the idea of approximating an $L^\infty$ functional by $L^p$ functionals.
As Daskalopolous and Uhlenbeck have given a proof of a version of Theorem \ref{existence of thurston lamination} using similar methods \cite{daskalopoulos2022,daskalopoulos2023}, it may be possible to adapt our results to manifold-valued maps.

\subsubsection{Bangert and Cui's version}
The statement in the literature that we are aware of, which is closest to our main theorem, is due to Bangert and Cui \cite{bangert_cui_2017}.
To emphasize the analogy with best Lipschitz maps, we call a form \dfn{best comass} if it minimizes its comass:

\begin{definition}
The \dfn{comass} of a closed $d - 1$-form $F$ is $\|F\|_{L^\infty}$.
A form $F$ has \dfn{best comass} if it minimizes its comass among all forms cohomologous to $F$.
\end{definition}

\begin{definition}
A \dfn{calibration} is a measurable closed $d - 1$-form $F$, such that $\|F\|_{L^\infty} = 1$.
Given a calibration $F$, a hypersurface $N$ is $F$-\dfn{calibrated} if the pullback of $F$ to $N$ is the area form on $N$.
\end{definition}

Observe that a best comass form is a calibration iff the costable norm of its cohomology class is $1$.
Moreover, the fundamental theorem of calibrated geometry \cite{Harvey82} asserts that every $F$-calibrated hypersurface $N$ is minimal, and in fact homologically area-minimizing if $N$ is closed.
With these preliminaries in mind, we may now state the Bangert--Cui theorem \cite[Theorem 5.1]{bangert_cui_2017}:

\begin{theorem}[max flow min cut, Bangert and Cui's version]\label{BangertCui}
Let $F$ be a continuous best comass calibration on a closed Riemannian manifold of dimension $\leq 7$.
Then there exists a measured oriented minimal lamination $\lambda$ such that every leaf of $\lambda$ is $F$-calibrated.
In particular, 
$$\Mass(\lambda) = \langle [\lambda], [F]\rangle.$$
\end{theorem}

The idea of Theorem \ref{BangertCui} is to use Theorem \ref{Federer} to find a homology class $\sigma$ dual to $[F]$, and then consider a minimizing representative $\lambda$ of $\sigma$, so that $\Mass(\lambda)$ is the stable norm of $\sigma$.
However, most calibrations which appear in nature are measurable and not necessarily continuous \cite[\S1]{bangert_cui_2017}, and it is natural to restrict attention to \emph{Lipschitz} laminations \cite[Remark 2.3]{bangert_cui_2017}; thus, a more natural formulation of Theorem \ref{BangertCui} would not require that $F$ is continuous, and would assert that $\lambda$ is Lipschitz.

\begin{mainthm}\label{existence of infinity tight forms}
Let $\rho \in H^{d - 1}(M, \RR)$ be a cohomology class.
Let $(F_p, u_q)$ be the family of dual pairs of $p$-tight forms and $q$-harmonic functions, suitably normalized, with $[F_p] = \rho$ and $(p, q)$ ranging over H\"older pairs with $d < p < \infty$.
Then there exists a pair $(F, u)$ such that as $p \to \infty$ along a subsequence, $F_p \to F$ weakly in $L^r$ for any $d < r < \infty$, and $u_q \to u$ weakly in $BV$, with the following properties:
\begin{enumerate}
\item $F$ has best comass.
\item $u$ has least gradient.
\item The product of distributions $\dif u \wedge F$ is well-defined.
\item We have the duality relation
\begin{equation}\label{max flow mean cut}
\Comass(\rho) \star |\dif u| = \dif u \wedge F.
\end{equation}
\end{enumerate}
\end{mainthm}

%%%%%%%%%%%%%%%%%%

\subsection{Calibrated laminations}
Suppose that $\rho \in H^{d - 1}(M, \RR)$ satisfies $\Comass(\rho) = 1$.
If $F$ is a continuous tight representative of $\rho$, then by the Bangert--Cui theorem, $F$ calibrates a minimal lamination.
However, a proof that tight forms are continuous is out of reach.\footnote{For domains in $\RR^2$, tight forms are $C^\alpha$ \cite{Evans08}, but it is unlikely that this argument generalizes.}
On the other hand, we know by \cite[Theorem B]{BackusCML} that the level sets of a function of least gradient form a measured oriented Lipschitz minimal lamination.

\begin{definition}
Let $M$ be a closed Riemannian manifold of dimension $d \leq 7$, and $\rho \in H^{d - 1}(M, \RR)$.
We can define a measured oriented minimal lamination $\mu$, by considering a tight representative $F$ of $\rho$, letting $u$ be a dual function of least gradient to $F$, and letting $\mu$ be the lamination induced by $u$.
We call $\mu$ a \dfn{measured stretch lamination} associated to $\rho$.
\end{definition}

\begin{mainthm}\label{lams are calibrated}
Suppose that $M$ is a closed Riemannian manifold of dimension $d \leq 7$, and let $\rho \in H^{d - 1}(M, \RR)$ be nonzero.
Then there is a measured stretch lamination associated to $\rho$.
Moreover, if $\kappa$ is a measured stretch lamination associated to $\rho$, and $F$ is a best comass representative of $\rho$, then $\kappa$ is $F/\Comass(\rho)$-calibrated, and for $\lambda$ ranging over measured oriented laminations,
\begin{equation}\label{duality between stable and comass}
\Comass(\rho) = \sup_\lambda \frac{\langle \rho, [\lambda]\rangle}{\Mass(\lambda)} = \frac{\langle \rho, [\kappa]\rangle}{\Mass(\kappa)}.
\end{equation}
\end{mainthm}

In order to make connection with Thurston's max flow/min cut theorem, we recall that the maximally stretched measured lamination given by Theorem \ref{existence of thurston lamination} is contained in a larger lamination, called the \dfn{canonical lamination} of the homotopy class $[f]$ \cite{Thurston98,Gu_ritaud_2017}.
The canonical lamination is the largest geodesic lamination whose leaves are maximally stretched by the best Lipschitz maps homotopic to $f$.
We construct an analogous lamination for best comass calibrations; this is Proposition \ref{existence of canonical lamination}.

\begin{mainthm}\label{existence of calibrated lam}
Suppose that $M$ is a closed Riemannian manifold of dimension $d \leq 7$, and let $\rho \in H^{d - 1}(M, \RR)$ satisfy $\Comass(\rho) = 1$.
Then there is a Lipschitz lamination $\lambda_\rho$, the \dfn{canonical lamination} of $\rho$, such that a complete hypersurface $N$ is a leaf of $\lambda_\rho$ iff, for every best comass representative $F$ of $\rho$, $N$ is $F$-calibrated.
The support of $\lambda_\rho$ is contained in the maximum comass locus of every best comass representative $F$.
\end{mainthm}

In addition to Theorem \ref{lams are calibrated}, which is necessary to show that $\lambda_\rho$ is nonempty, and curvature bounds on homologically minimizing hypersurfaces \cite{Schoen75, Schoen81}, which are necessary to construct the Lipschitz flow boxes \cite{BackusCML}, we also have to show that if $v$ is a harmonic function, then almost every zero of $v$ is a single zero.
From this it follows that any two calibrated hypersurfaces which intersect, must somewhere intersect transversely, contradicting the definition of calibration.
The highlight of the proof is an application of the rectifiability of the set $P$ of double zeroes of $v$ \cite{Hardt89} to bound the cohomological dimension of $P$.

We then study the structure of the canonical lamination $\lambda_\rho$.
The canonical lamination is covered by the measured stretch laminations which are associated to $\rho$, along with some leaves which do not admit a transverse measure.
The measured stretch laminations, in turn, are exactly those homologically minimizing laminations which represent homology classes in the convex set 
$$\rho^* := \{\alpha \in H_{d - 1}(M, \RR): \langle \rho, \alpha\rangle = \Mass(\alpha) = 1\}.$$
A special role is played by the extreme points of $\rho^*$, which are represented by measured stretch laminations with no proper sublamination.
We shall not attempt to summarize our results on the structure of $\lambda_\rho$ into a clean theorem here, but refer the reader to \S\ref{canonical structure}.
In \S\ref{Teichmuller}, we explain how these results compare and contrast with the results of Thurston's canonical lamination.
In \S\ref{canonical conjectures}, we give some open problems about the structure of the canonical lamination.

%%%%%%%%%%%%%%%%%%%%
\subsection{Convexity of the stable unit ball}
It is known that convexity properties of the stable unit ball 
$$B := \{\alpha \in H_{d - 1}(M, \RR): \Mass(\alpha) \leq 1\}$$
are intimately related to the structure of homologically minimizing laminations \cite{Thurston98,Auer01}.
From our perspective, this is because $\rho^*$ is a \dfn{flat} of $\partial B$ -- that is, the intersection of $\partial B$ with a supporting hyperplane of $B$.

Though the proofs were never published, Auer and Bangert announced several results on the structure of $B$ using laminations \cite{Auer01}.
In fact, some of these results immediately follow from the structure of the canonical lamination, and we now record them.

Recall that the exterior product on the cohomology ring $H^\bullet(M, \RR)$ induces, by Poincar\'e duality, an \dfn{intersection product}
\begin{align*}
H_{d - k}(M, \RR) \times H_{d - \ell}(M, \RR) &\to H_{d - k - \ell}(M, \RR) \\
(\alpha, \beta) &\mapsto \alpha \cdot \beta.
\end{align*}
Also recall that $B$ is strictly convex iff $\partial B$ has no flats except singletons.
Then we have the following theorem, which is Proposition \ref{flats are nonintersecting}:

\begin{mainthm}\label{convexity summary}
Let $B$ be the stable unit ball of a closed Riemannian manifold $M$ of dimension $\leq 7$, and let $S$ be a flat of $\partial B$.
Then for any $\alpha, \beta \in S$, $\alpha \cdot \beta = 0$.
\end{mainthm}

Taken in the contrapositive, Theorem \ref{convexity summary} gives a condition for strict convexity of $B$.
In particular, if $M$ is homeomorphic to a torus, then $B$ is strictly convex (see Example \ref{torus convex}).


%%%%%%%%%%%%%%%%%%%%%
\subsection{Outline of the paper}



%%%%%%%%%%%%%%%%%%%%%%
\subsection{Acknowledgements}
I would like to thank Georgios Daskalopolous and Karen Uhlenbeck for providing helpful comments and providing me with an early draft of the manuscript \cite{daskalopoulos2023} which was a major source of inspiration for this work.
I would also like to thank Victor Bangert for providing me with a copy of \cite{Auer12}.

This research was supported by the National Science Foundation's Graduate Research Fellowship Program under Grant No. DGE-2040433.


% %%%%%%%%%%%%%%%%%%%%%%%%%%%%%%%%%%%%%%%%%%
\section{Preliminaries}\label{prevResults}

\begin{theorem}[{\cite[Theorem A]{BackusCML}}]\label{disjoint surfaces are lamination}
Let $\mathcal S$ be a set of disjoint complete minimal hypersurfaces in a manifold $M$ of bounded geometry.
Suppose that there exists $C > 0$ such that for every $N \in \mathcal S$, $\|\Two_N\|_{C^0} \leq C$.
Then $\mathcal S$ is the set of leaves of a Lipschitz minimal lamination $\lambda$ of bounded curvature.
In particular, if $\lambda$ is oriented, then there is a Lipschitz vector field on $M$ whose restriction to each $N \in \mathcal S$ is the normal vector to $N$.
\end{theorem}


\begin{proposition}\label{calibration condition}
Let $F$ be a calibration on a closed Riemannian manifold $M$.
Let $T_\lambda$ be the Ruelle-Sullivan current of a measured oriented lamination $\lambda$.
Then the following are equivalent:
\begin{enumerate}
\item One has \begin{equation}\label{calibration by Ruelle Sullivan}
\int_M T_\lambda \wedge F = \Mass(\lambda).
\end{equation}
\item $\lambda$ is $F$-calibrated.
\end{enumerate}
\end{proposition}

\begin{proposition}\label{properties of calibrated laminations}
Suppose that $M$ is a closed Riemannian manifold, $F$ is a calibration, and $\lambda$ is a measured $F$-calibrated lamination.
Then:
\begin{enumerate}
\item $\lambda$ is minimal.
\item If $G$ is a calibration and cohomologous to $F$, then $\lambda$ is $G$-calibrated.
\item $\supp \lambda \subseteq \MCL(F)$.
\end{enumerate}
\end{proposition}


\begin{lemma}\label{dual to u is minimizer}
Suppose that $u: \tilde M \to \RR$ is an $\alpha$-equivariant $q$-harmonic function, and suppose that
$$F := - |\dif u|^{q - 2} \star \dif u.$$
Then $F$ is a closed $d - 1$-form, which minimizes $J_{p, \alpha}$ among all closed $d - 1$-forms.
Moreover, $F$ solves the PDE 
\begin{equation}\label{pMaxwell}
\begin{cases}
	\dif F = 0 \\
	\dif^* (|F|^{p - 2} F) = 0.
\end{cases}
\end{equation}
\end{lemma}

\begin{lemma}\label{L1 convergence preserves pi1}
Let $\tilde M \to M$ be the universal cover, and let $(u_q)$ be a sequence of $\pi_1(M)$-equivariant functions on $\tilde M$ which converge in $L^1_\loc(\tilde M)$ to a function $u$ as $q \to 1$.
Then $u$ is $\pi_1(M)$-equivariant, and $[u_q] \to [u]$.
Moreover, if $\dif u_q \to \dif u$ in the weak topology of measures on $M$ and $\dif u_q \in L^q$, then
\begin{equation}\label{q to 1 Holder}
\Mass(\dif u) \leq \liminf_{q \to 1} \frac{1}{q} \int_M \star |\dif u_q|^q.
\end{equation}
\end{lemma}

\begin{proposition}\label{MCL contains Thurston}
Let $F$ be a best comass representative of $\rho$, where $\Comass(\rho) = 1$, and let $\lambda$ be a measured stretch lamination associated to $\rho$.
Then $F$ calibrates $\lambda$. In particular, $\MCL(F) \supseteq \supp \lambda$.
\end{proposition}

\begin{proposition}\label{calibrated means measured stretch}
Let $F$ be a best comass representative of $\rho$, where $\Comass(\rho) = 1$, and suppose that $F$ calibrates a measured oriented lamination $\lambda$.
Then $\lambda$ is a measured stretch lamination associated to $\rho$.
\end{proposition}





%%%%%%%%%%%%%%%%
\section{The canonical lamination}
Throughout this section, we fix $\rho \in H^{d - 1}(M, \RR)$ in the costable unit sphere $\{\Comass(\rho) = 1\}$.
Motivated by Thurston's approach to Teichm\"uller theory (see \S\ref{Teichmuller}), we construct a lamination which is calibrated by every best comass form in $\rho$, and which only depends on $\rho$: the \dfn{canonical lamination} $\lambda_\rho$.

%%%%%%%%%%%%%%%%%%%%%
\subsection{Construction of the canonical lamination}
We note carefully that if $F$ is a best comass representative of $\rho$, then $\MCL(F)$ need not itself be a lamination \cite[Example 5.4]{bangert_cui_2017}.
In particular, unlike in Thurston's theory, we cannot simply take the intersection of all the maximum comass loci of best comass representatives of $\rho$.
On the other hand, one can use the existence of measured stretch laminations to show that $\MCL(F)$ contains a lamination.
So our strategy is to construct the largest lamination $\lambda_F$ which $F$ calibrates, and take an intersection over all the $\lambda_F$s.

We first rule out intersections of the leaves.
This can be done by showing that the generic intersection point of two minimal hypersurfaces is transverse.
If the dimension of the underlying manifold $M$ is $d = 2$, then this is trivial, and if $d = 3$, then the structure of $N \cap N'$ is completely described by complex-analytic means \cite[Theorem 7.3]{colding2011course}, so the proof we present here is mainly of interest if $d \geq 4$.

\begin{proposition}\label{intersection theory prop}
Let $N, N' \subset M$ be minimal hypersurfaces, and let $S \subseteq N, N'$ be the set of points at which $N, N'$ intersect nontransversely.
Then one of the following holds:
\begin{enumerate}
\item $N \cap N'$ is empty.
\item $\dim(N \cap N') = d - 2$ and $\dim S \leq d - 3$.
\item There exists $p \in S$ such that the germs of $N, N'$ at $p$ are equal.
\end{enumerate}
\end{proposition}
\begin{proof}
Let $p \in S$, and let $P$ be the tangent space of $N, N'$ at $x$.
Then we can view $N, N'$ as the graphs of functions $u, u'$ over $P$, say taken in normal coordinates based at $p$; thus we identify $P$ with $\RR^{d - 1}$.
Reasoning as in the proof of \cite[Theorem 7.3]{colding2011course}, the difference $v := u - u'$ solves a linear elliptic PDE $Qv = 0$, and in a neighborhood $U \ni p$, the exponential map $P \to N$ induces Lipschitz isomorphisms $\{v = 0\} \cap U \cong N \cap N' \cap U$ and $\{v = \dif v = 0\} \cap U \cong S \cap U$.
If $v$ only has zeroes of finite order, then the claim follows from Proposition \ref{nodal set is generically smooth}.
Otherwise, $v$ is identically $0$ by the unique continuation theorem \cite[Theorem 6.1]{colding2011course}, so $N \cap U = N' \cap U$.
\end{proof}

\begin{lemma}
Let $F$ be a calibration, and let $B \subseteq M$ be a sufficiently small ball.
Then for any complete connected $F$-calibrated hypersurface $N$, 
\begin{equation}\label{area bound for calibrated}
\Mass(N \cap B) \leq \Mass(\partial B).
\end{equation}
\end{lemma}
\begin{proof}
By the Thom transversality theorem, we may assume that $N$ meets $\partial B$ transversely. 
Let $S := N \cap \partial B$, which by transversality can be identified with a closed $d - 2$-dimensional submanifold of $\Sph^{d - 1}$.
Since $H_{d - 2}(\Sph^{d - 1}, \RR) = 0$, there exists a relatively open set $U \subseteq \partial B$ which is bounded by $S$.
Since $H^{d - 1}(B, \RR) = 0$, we may write $F = \dif A$ in a neighborhood of $B$, where by Lemma \ref{Hodge theorem} and the Sobolev embedding theorem we may assume that $A$ is continuous. Then
\begin{align*}
\Mass(N \cap B) &= \int_{N \cap B} F = \int_S A = \int_U F \leq \Mass(U) \leq \Mass(\partial B). \qedhere
\end{align*}
\end{proof}

\begin{lemma}
There exists a constant $C > 0$, only depending on $M$, such that for every calibration $F$ and complete $F$-calibrated hypersurface $N$, we have the curvature bound
\begin{equation}\label{curvature bound for calibrated}
\|\Two_N\|_{C^0} \leq C.
\end{equation}
\end{lemma}
\begin{proof}
Let $x \in N$ and let $r > 0$ be small.
Then each component $N'$ of $N \cap B(x, r)$ is absolutely area-minimizing by the fundamental theorem of calibrated geometry, so it is stable.
By (\ref{area bound for calibrated}), $\Mass(N') \lesssim r^{d - 1}$.
So by \cite[pg785, Corollary 1]{Schoen81},\footnote{See also \cite[Theorem 3]{Schoen75} for an easier proof when $M$ has nonpositive curvature and dimension $d \leq 6$, or \cite[Chapter 2, \S\S4-5]{colding2011course} for a textbook treatment of a similar estimate. By \cite[Lemma 2.4]{chodosh2022complete}, we may remove the dependence on the volume bound if $d \leq 4$.}
\begin{align*}
\|\Two_{N'}\|_{B(x, r/2)} \lesssim_{d, \|\Riem_g\|_{C^0(B(x, 2r))}} \frac{1}{r}.
\end{align*}
Since $N'$ was an arbitrary component, the same estimate holds for $N$.
Using the compactness of $M$, we may cover it by finitely many balls in which estimates of this form hold to conclude (\ref{curvature bound for calibrated}).
\end{proof}

\begin{lemma}\label{calibrated implies disjoint}
Let $F$ be a calibration, and let $N, N'$ be complete connected $F$-calibrated hypersurfaces.
If $N \cap N'$ is nonempty, then $N = N'$.
\end{lemma}
\begin{proof}
We first observe that if $x \in N \cap N'$, then $(\star F(x))^\sharp$ is the normal vector to both $N, N'$ at $x$.
Therefore $N \cap N'$ only consists of points of tangency.
By Proposition \ref{intersection theory prop}, it follows that either the germs of $N, N'$ at $x$ are equal.
Since the germs are equal and $N, N'$ are connected, a standard boostrapping argument implies that $N = N'$.
\end{proof}

\begin{proposition}\label{existence of semicanonical lamination}
Let $F$ be a best comass calibration.
Then the set of $F$-calibrated hypersurfaces is the set of leaves of a lamination $\lambda_F$, which contains every measured stretch lamination associated to $[F]$.
\end{proposition}
\begin{proof}
Let $\mathscr L_F$ be the set of connected complete $F$-calibrated hypersurfaces.
By Lemma \ref{calibrated implies disjoint}, $\mathscr L_F$ consists of pairwise disjoint minimal hypersurfaces.
By Proposition \ref{MCL contains Thurston}, there exists a measured stretch lamination $\lambda$ associated to $[F]$, and then by Proposition \ref{properties of calibrated laminations}, $\mathscr L_F$ contains every leaf of $\lambda$.
Since the estimate (\ref{curvature bound for calibrated}) is independent of $N$, it follows by Theorem \ref{disjoint surfaces are lamination} that $\mathscr L_F$ is the set of leaves of some lamination $\lambda_F$.
\end{proof}

\begin{lemma}\label{existence of intersections}
Let $\mathscr S$ be a nonempty set of laminations.
Suppose that there exists a hypersurface which is a leaf of every lamination in $\mathscr S$.
Then there exists a lamination whose set of leaves is the intersection of the sets of leaves of the laminations in $\mathscr S$.
\end{lemma}
\begin{proof}
Let $\lambda \in \mathscr S$, and let $(F_\alpha, K_\alpha)$ be a laminar atlas for $\mathscr S$.
Let $K'_\alpha$ be the set of $k \in K_\alpha$ such that for every $\kappa \in \mathscr S$, there exists a leaf $N$ of $\kappa$ such that
$$(F_\alpha)_*(\{k\} \times J) \subseteq N.$$
It is clear that this property is preserved by transition maps.
Then $K_\alpha'$ is an intersection of compact sets (since the local leaf spaces of each $\kappa \in \mathscr S$ is compact), so $K_\alpha'$ is compact.
The hypersurface which is a common leaf of every lamination in $\mathscr S$ witnesses that for some $\alpha$, $K_\alpha'$ is nonempty.
Therefore $(F_\alpha, K'_\alpha)$ is a laminar atlas for the lamination whose support is $\bigcap_{\kappa \in \mathscr S} \supp \kappa$.
\end{proof}

\begin{proposition}\label{existence of canonical lamination}
The set of hypersurfaces which are calibrated by every best comass representative of $\rho$ is the set of leaves of a lamination $\lambda_\rho$, which contains every measured stretch lamination associated to $\rho$.
\end{proposition}
\begin{proof}
By Proposition \ref{MCL contains Thurston}, there is a (measured stretch) lamination which is calibrated by every best comass representative of $\rho$.
So we may apply Lemma \ref{existence of intersections} to the set $\mathscr S$ of all calibrated laminations $\lambda_F$ produced by Proposition \ref{existence of semicanonical lamination}, where $F$ ranges over best comass representatives of $\rho$.
\end{proof}

\begin{definition}
The lamination $\lambda_\rho$ constructed in Proposition \ref{existence of canonical lamination} is the \dfn{canonical lamination} associated to $\rho$.
\end{definition}

\todo{Show that for surfaces, a best Lipschitz map calibrates ALL of its MCL. This proves a much stronger version of \cite[Theorem 5.2]{daskalopoulos2020transverse}. Not quite obvious, since they're not integrable, but it should be OK using the Carath\'eodory condition.}

%%%%%%%%%%%%%%%%%%%%%%%%%%%%%%%%
\subsection{Structure of the canonical lamination}\label{canonical structure}
We now study the structure of the canonical lamination.
A sticky technical point is that $H_{d - 1}(M, \RR)$ need not be strictly convex, so there may be many $\alpha$ in the stable unit sphere such that 
\begin{equation}\label{flats duality}
\Comass(\rho) = \langle \rho, \alpha\rangle.
\end{equation}
In particular, there may be many measured stretch sublaminations of the canonical lamination which are mutually nonhomologous.
We therefore introduce the dual set 
$$\rho^* := \{\alpha \in H_{d - 1}(M, \RR): \langle \rho, \alpha\rangle = \Mass(\alpha) = 1\}.$$
It is clear that any measured sublamination of the canonical lamination normalized to have mass $1$ represents a member of $\rho^*$.
In fact, this condition completely characterizes $\rho^*$, as we now show.

\begin{lemma}\label{homologically minimizing means measured stretch}
For every $\alpha \in \rho^*$, every measured oriented, homologically minimizing, lamination representing $\alpha$ is a measured stretch lamination associated to $\rho$.
\end{lemma}
\begin{proof}
Let $\dif u$ be the Ruelle-Sullivan current of the measured oriented, homologically minimizing lamination $\lambda$, and let $F$ be a best comass representative of $\rho$.
Since $\lambda$ is homologically minimizing,
$$\int_M \dif u \wedge F = \langle \rho, \alpha\rangle = \Mass(\alpha) = \Mass(\lambda),$$
so by Proposition \ref{calibration condition}, $F$ calibrates $\lambda$.
Therefore by Proposition \ref{calibrated means measured stretch}, $\lambda$ is a measured stretch lamination.
\end{proof}

\begin{lemma}\label{existence for least gradient}
For each $\alpha \in H_{d - 1}(M, \RR)$, there exists an $\alpha$-equivariant function of least gradient $u: \tilde M \to \RR$.
We can take $u$ to be approximated in $L^1$ by $\alpha$-equivariant $q$-harmonic functions if we choose.
\end{lemma}
\begin{proof}
One can take a minimizing sequence $(u_n)$, apply Miranda compactness \todo{cite it once the other paper is done} to obtain a limit of least gradient, and observe that the limit is $\alpha$-equivariant by Lemma \ref{L1 convergence preserves pi1}.
\todo{By a H\"older argument -- move it to the other paper maybe?} we can take the approximate minimizer $u_n$ to be $e^{-n}$-harmonic if we choose.
This is a standard argument (aside from the equivariance) and we omit the details.
\end{proof}

\begin{proposition}\label{enough measures in canonical lamination}
For each $\alpha \in \rho^*$, there exists a measured stretch sublamination of $\lambda_\rho$ with homology class $\alpha$.
\end{proposition}
\begin{proof}
Let $u$ be the function of least gradient furnished by Lemma \ref{existence for least gradient}.
The measured oriented, homologically minimizing, lamination $\kappa_u$ has class $\alpha$.
So by Lemma \ref{homologically minimizing means measured stretch}, it is a measured stretch lamination and hence is a sublamination of $\lambda_\rho$.
\end{proof}

We next use the decomposition of measured laminations \cite[{\S}I.3]{Morgan88} to partition the leaves of $\lambda_\rho$ into various categories.
In this direction we shall need to study measured laminations which are minimal with respect to inclusion; as the word ``minimal'' is overloaded, we shall call such laminations ``indecomposable''.

\begin{definition}
Let $\lambda$ be a lamination.
\begin{enumerate}
\item $\lambda$ is \dfn{indecomposable} if the only sublamination of $\lambda$ is itself.
\item If $\lambda$ is indecomposable, then $\lambda$ is \dfn{exceptional} if $\supp \lambda \neq M$ and $\lambda$ does not consist of a single leaf.
\item $\lambda$ is a \dfn{parallel family of closed leaves} if there exists a closed hypersurface $N \subset M$ with trivial normal bundle, such that every leaf of $\lambda$ is a section of the normal bundle of $N$.
\item A leaf $N$ of $\lambda$ is \dfn{nonmeasurable} if, for every sublamination $\kappa \subset \lambda$ which admits a transverse measure, $N$ is not a leaf of $\kappa$.
\end{enumerate}
\end{definition}

Thus every indecomposable lamination either is a foliation in which every leaf is dense, an exceptional indecomposable lamination, or a closed hypersurface.
Moreover, every local leaf space $K_\alpha$ of an exceptional indecomposable lamination $\lambda$ is a Cantor set \cite[{\S}I.3.1]{Morgan88}, and every leaf of $N$ is noncompact.
Every nonmeasurable leaf is noncompact, for if $N$ is a closed leaf, then $N$ equipped with its Dirac measure is a measured sublamination of $\lambda$.

\begin{theorem}\label{MorganShelan}
Let $\lambda$ be a measured oriented lamination in the closed manifold $M$.
Then either $\lambda$ is a foliation with a dense leaf, or $\lambda$ separates into finite number of clopen sublaminations, each of which is a parallel family of closed leaves or an exceptional indecomposable lamination.
\end{theorem}
\begin{proof}
First observe that the proof of \cite[Theorem I.3.2]{Morgan88} goes through for any lamination $\lambda$ such that no leaf of $\lambda$ is dense in $M$, even if $\lambda$ is a foliation.
Twisted families of closed leaves (that is, families of sections of a nontrivial normal bundle of a closed hypersurface) are excluded by the fact that $\lambda$ is oriented, so its leaves are oriented, and hence the normal bundle of any of its leaves is trivial.
\end{proof}

\begin{proposition}\label{classification of leaves}
For each leaf $N$ of $\lambda_\rho$, one of the following holds:
\begin{enumerate}
\item $N$ is closed.
\item $N$ is a noncompact leaf of an exceptional indecomposable measured stretch lamination associated to $\rho$.
\item $N$ is noncompact and $\lambda_\rho$ is a foliation which admits a transverse measure.
\item $N$ is noncompact and $N$ is a nonmeasurable leaf of $\lambda_\rho$.
\end{enumerate}
\end{proposition}
\begin{proof}
If $N$ is a closed leaf of $\lambda_\rho$, then $N$ equipped with its Dirac measure is a measured lamination, calibrated by any tight representative of $\rho$; hence it is measured stretch for $\rho$.
Otherwise, since $N$ has no boundary, it is noncompact.

If $N$ is noncompact, but is contained in a measured sublamination $\kappa$ of $\lambda_\rho$, then by Theorem \ref{MorganShelan}, either $\kappa$ is a foliation or $N$ is contained in an exceptional indecomposable sublamination.
If $\kappa$ is a foliation, then
$$\supp \kappa \supseteq \supp \lambda_\rho \supseteq \supp \kappa,$$
implying $\kappa = \lambda_\rho$.
Otherwise, the exceptional indecomposable sublamination $\zeta$ of $\kappa$ containing $N$ is calibrated by any tight representative of $\rho$, so $\zeta$ is measured stretch for $\rho$ by Proposition \ref{calibrated means measured stretch}.
\end{proof}

\begin{corollary}\label{measurable leaves are contained in indecomposables}
Let $N$ be a leaf of the canonical lamination $\lambda_\rho$.
Then either $N$ is nonmeasurable, or $N$ is contained in an indecomposable measured stretch lamination associated to $\rho$.
\end{corollary}
\begin{proof}
By Proposition \ref{classification of leaves}, if $N$ is not nonmeasurable, then either $N$ is closed, in which case $N$ is itself an indecomposable measured stretch lamination, or $N$ is noncompact and is contained in an exceptional indecomposable measured stretch lamination.
\end{proof}

\begin{corollary}
Let $F$ be a best comass representative of $\rho$, and $N$ a leaf of the calibrated lamination $\lambda_F$.
Then either $N$ is a leaf of the canonical lamination $\lambda_\rho$, or $N$ is a nonmeasurable leaf of $\lambda_F$.
\end{corollary}
\begin{proof}
Suppose that $N$ is a leaf of a measured sublamination $\kappa$ of $\lambda_F$.
Then, since $\kappa$ is calibrated by $F$, $\kappa$ is measured stretch by Proposition \ref{calibrated means measured stretch}, hence is a sublamination of $\lambda_\rho$.
\end{proof}

Another consequence of the decomposition of laminations is that the extreme points of $\rho^*$ are represented by indecomposable laminations.
Recall that a point $\alpha$ of a convex set $S$ is \dfn{extreme} if $\alpha$ cannot be written as the convex combination of two distinct members of $S$.

\begin{lemma}\label{extreme points are closed under sublaminations}
Let $\alpha$ be an extreme point of $\rho^*$, and let $\kappa$ be a measured stretch lamination in $\alpha$.
Then any sublamination of $\kappa$ represents a scalar multiple of $\alpha$.
\end{lemma}
\begin{proof}
By replacing $\kappa$ with a proper sublamination if necessary, we may assume that $\kappa$ is not a foliation.
Let $\zeta$ be a sublamination of $\kappa$.
By Theorem \ref{MorganShelan} and the fact that the leaves of a parallel family of closed leaves are all homologous, after replacing $\zeta$ with a sublamination of $\zeta$, we may assume that $\zeta$ is a clopen parallel family of closed leaves, or is an exceptional indecomposable sublamination of $\kappa$.
Since $\kappa$ is the linear combination of finitely many such clopen sublaminations, we may write $\alpha$ as a convex combination of $\beta_1, \dots, \beta_k$ where the $\beta_i$ are the (normalized to mass $1$) homology classes of clopen sublaminations of $\lambda$.
But $\beta_i \in \rho^*$, so $\beta_i = \alpha$, hence $[\zeta] = \alpha$.
\end{proof}

\begin{proposition}\label{extreme points are indecomposable}
Let $\alpha$ be an extreme point of $\rho^*$. Then $\alpha \in \rho^*_{\rm exc}$.
\end{proposition}
\begin{proof}
By Proposition \ref{enough measures in canonical lamination}, there exists a measured stretch lamination $\kappa$ representing $\alpha$.
By Theorem \ref{MorganShelan}, $\kappa$ has an indecomposable sublamination $\zeta$.
By Lemma \ref{extreme points are closed under sublaminations}, possibly after rescaling the transverse measure, $\zeta$ is a representative of $\alpha$.
Since any tight representative $F$ of $\rho$ calibrates $\kappa$, $F$ also calibrates $\zeta$, so by Proposition \ref{calibrated means measured stretch}, $\zeta$ is a measured stretch sublamination of $\lambda_\rho$.
\end{proof}

%%%%%%%%%%%%%%%%%%%%%%%%%
\subsection{Motivation for the canonical lamination}\label{Teichmuller}
Our motivation for introducing the canonical lamination arose from an analogy with Thurston's approach to Teichm\"uller theory using best Lipschitz maps \cite{Thurston98}.\footnote{None of this discussion shall be used in the sequel, except as motivation, and to state some conjectures in \S\ref{open problems}.} \todo{Reword some of this}
Given $\gamma \geq 2$, let $\widetilde{\mathscr M}_\gamma$ be the Teichm\"uller space of hyperbolic metrics on the closed surface $S_\gamma$ of genus $\gamma$.
Given $g, h \in \widetilde{\mathscr M}_\gamma$, let $\Lip(g, h)$ be the Lipschitz constant of a best Lipschitz map $(S_\gamma, g) \to (S_\gamma, h)$; then for a tangent vector $v \in T_g(\widetilde{\mathscr M}_\gamma)$, let $\Comass(v)$ be the partial derivative of $\log \Lip(g, \cdot)$ in the direction $v$.
This quantity, the \dfn{Thurston asymmetric norm}, is an asymmetric norm on $T_g(\widetilde{\mathscr M}_\gamma)$ obtained by solving an $L^\infty$ variational problem intimately tied to the structure of minimal laminations, so it is tempted to make an analogy between $T_g(\widetilde{\mathscr M}_\gamma)$ and $H^{d - 1}(M, \RR)$, where both vector spaces are equipped with the norm $\Comass$.
Two particularly salient pieces of evidence for the analogy are:
\begin{enumerate}
\item The unit spheres of the dual spaces of $T_g(\widetilde{\mathscr M}_\gamma)$ and $H^{d - 1}(M, \RR)$ can both be viewed as spaces of projective measured minimal laminations, whose norm is given by an $L^1$ (actually $BV$) variational problem \cite[Theorem 5.1]{Thurston98}.
\item In both cases, we can construct a canonical lamination; in Thurston's case, the canonical lamination is given by those geodesics which are maximally stretched by every best Lipschitz map homotopic to $\id_{S_\gamma}$ \cite[\S8]{Thurston98}. See also Conjecture \ref{chain recurrence}.
\end{enumerate}
However, one should not take this analogy too seriously.
A key feature of Thurston's theory is the Birman-Series theorem: the union of the supports of all geodesic laminations on $(S_\gamma, g)$ has Hausdorff dimension $0$.
As a corollary, for almost every $h \in \widetilde{\mathscr M}_\gamma$, the canonical lamination associated to $(g, h)$ is a closed geodesic \cite[\S10]{Thurston98}.
The analogue of the Birman-Series theorem is clearly not true in our case, and in fact, if $M$ is a square flat torus, then it is easy to see that every canonical lamination covers all of $M$.
But see Conjecture \ref{Karen}.

%%%%%%%%%%%%%%%%%%%%%%%%
\section{Applications to homology}\label{homology sec}
\subsection{Convexity of the stable unit ball}\label{convexity sec}
Let $M$ be a closed manifold of dimension $\leq 7$.
Auer and Bangert \cite{Auer01} claimed certain results concerning the convex structure of the stable unit ball
$$B := \{\alpha \in H_{d - 1}(M, \RR): \Mass(\alpha) \leq 1\}.$$
Here we show that some of these results follow from the structure theory of the canonical lamination.

Recall that a \dfn{flat} $S \subset \partial B$ is a set such that, for some supporting hyperplane $H$ of $B$, $S = H \cap B$.
Thus $B$ is strictly convex iff every flat is a point.

\begin{lemma}
Suppose that $S \subset \partial B$ is a flat of the stable unit ball $B \subset H_{d - 1}(M, \RR)$.
Then there exists $\rho$ in the costable unit sphere of $H^{d - 1}(M, \RR)$ such that $S \subseteq \rho^*$.
\end{lemma}
\begin{proof}
Since $S$ is convex, $\partial S$ is topologically a sphere, so $\partial S$ admits a Borel probability measure $\nu$ of full support.
Then we take the vector-valued integral 
$$\beta := \int_{\partial S} \alpha \dif \nu(\alpha),$$
thus $\beta \in S$ by convexity.
By the Hanh-Banach theorem (Theorem \ref{Federer}), there exists $\rho \in H^{d - 1}(M, \RR)$ such that $\beta \in \rho^*$.

We claim that $\partial S \subseteq \rho^*$.
If not, then by continuity of $\alpha \mapsto \langle \rho, \alpha\rangle$, there is a positive measure set of $\partial S$ on which $\langle \rho, \cdot\rangle < 1$, hence
$$\Comass(\rho) = \langle \rho, \beta\rangle = \int_{\partial S} \langle \rho, \alpha\rangle \dif \nu(\alpha) < \Comass(\rho),$$
a contradiction.
Since $\rho^*$ is convex, it follows that $S \subseteq \rho^*$.
\end{proof}

\begin{proposition}\label{flats are nonintersecting}
Let $B$ be the stable unit ball of $H_{d - 1}(M, \RR)$.
Suppose that $S \subset \partial B$ is a flat, and $\alpha, \beta \in S$. Then $\alpha \cdot \beta = 0$.
\end{proposition}
\begin{proof}
Let $\rho$ be the cohomology class dual to $S$ given by (\ref{flats duality}).
By Proposition \ref{enough measures in canonical lamination}, there exist measured stretch sublaminations $\kappa_\alpha, \kappa_\beta$ of $\lambda_\rho$, of classes $\alpha, \beta$.
Let $\dif u_\alpha, \dif u_\beta$ be their Ruelle-Sullivan currents, and suppose that $x$ is in the union of their supports.
By \todo{previous paper}, if $N$ denotes the leaf of $\lambda_\rho$ containing $x$, then for $\sigma = \alpha, \beta$,
$$\dif u_\sigma(x) = \normal_N^\flat(x) \mu_\sigma(x)$$
where $\mu_\sigma$ is the positive Radon measure induced on $M$ by the transverse measure to $\kappa_\sigma$.
In particular, $\dif u_\alpha|_{\supp \dif u_\beta}$ is a scalar multiple of $\dif u_\beta$, so $\dif u_\alpha \wedge \dif u_\beta = 0$, hence $\alpha \cdot \beta = 0$.
\end{proof}

Although the definitions of the stable unit ball $B$ and the intersection product depend on the Riemannian metric on $M$, we conclude a purely topological condition for strict convexity of $B$.

\begin{corollary}\label{condition for strict convexity}
Suppose that the kernel of the wedge product 
\begin{equation}\label{wedge product}
\wedge: H^1(M, \RR) \otimes H^1(M, \RR) \to H^2(M, \RR)
\end{equation}
is spanned by symmetric tensors $\theta \otimes \theta$, $\theta \in H^1(M, \RR)$.
Then the stable unit ball of $H_{d - 1}(M, \RR)$ is strictly convex.
\end{corollary}
\begin{proof}
We prove the contrapositive.
If $H_{d - 1}(M, \RR)$ does not have a strictly convex unit ball, then by Proposition \ref{flats are nonintersecting} there exist linearly independent $\alpha, \beta \in H_{d - 1}(M, \RR)$ such that $\alpha \cdot \beta = 0$.
Dually, this means that we can find linearly independent $\theta, \omega \in H^1(M, \RR)$ such that $\theta \otimes \omega \in \ker(\wedge)$.
\end{proof}

\begin{example}\label{torus convex}
Suppose that $M$ is homotopic to a torus, so that the cohomology ring $H^\bullet(M, \RR)$ is isomorphic to the exterior algebra of $\RR^d$.
In particular, the kernel of (\ref{wedge product}) is spanned by symmetric tensors, so by Corollary \ref{condition for strict convexity}, the stable unit ball of $H_{d - 1}(M, \RR)$ is strictly convex.
\end{example}




%%%%%%%%%%%%%%%%%%%%%%%%%%%%
% \subsection{Existence of an optimal best comass form}
% \todo{exposit this}

% \begin{definition}
% An \dfn{optimal best comass form} is a best comass form $F$ such that
% $$\MCL(F) = \bigcap_G \MCL(G)$$
% where $G$ ranges over best comass forms cohomologous to $F$.
% \end{definition}

% \begin{proposition}
% Let $\rho \in H^2(M, \RR)$.
% Then there exists an optimal best comass representative of $\rho$.
% \end{proposition}
% \begin{proof}
% Let $\lambda := \bigcap_G \MCL(G)$ where $G$ ranges over best comass representatives of $\rho$.
% For $x \notin \lambda$, we can find a best comass form $F_x$ of class $\rho$ such that $x \notin \MCL(F_x)$.
% In particular, $U_x := \{L(F_x, \cdot) < L\}$ is an open set which contains $x$, so $(U_x)_{x \notin \lambda}$ is an open cover of $M \setminus \lambda$.
% Since $M \setminus \lambda$ is $\sigma$-compact, there exists a countable subcover $(U_{x_i})_{i \in I}$, for some countable set $I \subseteq \NN$.

% We then introduce the closed form 
% $$F := \sum_{i \in I} \alpha_i F_{x_i},$$
% where $\sum_{i \in I} \alpha_i = 1$.
% Here the sum converges in the norm topology of $L^\infty$, even if $I$ is infinite.
% Indeed, if $I_N := I \cap \{1, \dots, N\}$, then the partial sums $\sum_{i \in I_N} \alpha_i F_{x_i}$ satisfy the tail bound
% $$\sum_{i \in I \setminus I_N} \alpha_i \|F_{x_i}\|_{L^\infty} \leq L \sum_{i \in I \setminus I_N} \alpha_i \to 0$$
% since $(\alpha_i) \in \ell^1$, which implies the convergence.
% This convergence implies (by Proposition \ref{crandall}) that $[F] = \rho$, so $L(F) \geq L$.
% On the other hand, 
% $$L(F) \leq \sum_{i \in I} \alpha_i L(F_{x_i}) \leq L \sum_{i \in I} \alpha_i = L.$$
% It follows that $L(F) = L$.
% In particular, $F$ has best comass and $\MCL(F) \supseteq \lambda$.

% To complete the proof, we show that $\MCL(F) \subseteq \lambda$.
% Let $x \notin \lambda$, and let $j$ satisfy $U_{x_j} \ni x$.
% Then by Proposition \ref{crandall},
% $$L(F, x) = \lim_{r \to 0} L_{B_r(x)}(F) \leq \lim_{r \to 0} \sum_{i \in I} \alpha_i L_{B_r(x)}(F_{x_i}).$$
% The summands are dominated by the $\ell^1$ sequence $(L\alpha_i)$, so by dominated convergence, 
% $$\lim_{r \to 0} \sum_{i \in I} \alpha_i L_{B_r(x)}(F_{x_i}) = \sum_{i \in I} \lim_{r \to 0} \alpha_i L_{B_r(x)}(F_{x_i}) = \sum_{i \in I} \alpha_i L(F_{x_i}, x).$$
% By assumption on $j$, $L(F_{x_j}, x) < L$, and besides $L(F_{x_i}, x) \leq L$ for any $i$.
% So 
% $$L(F, x) \leq \sum_{i \in I} \alpha_i L(F_{x_i}, x) < L$$
% and we conclude $x \notin \MCL(F)$.
% \end{proof}

%%%%%%%%%%%%%%%%%%%%%%%%%%%%%%%%
\subsection{Rational points in homology}
\begin{definition}
A homology class $\alpha \in H_{d - 1}(M, \RR)$ has \dfn{rational direction} if there exists $c > 0$ such that $c\alpha \in H_{d - 1}(M, \ZZ)$.
\end{definition}

If $\alpha \in H_{d - 1}(M, \ZZ)$, then by Poincar\'e duality and the Hurcewiz theorem, we may identify $\alpha$ with an integral representation
$$\alpha: \pi_1(M) \to \ZZ,$$
or equivalently (since $\Sph^1$ is a $K(\ZZ, 1)$ space) a homotopy class of maps $M \to \Sph^1$.
Thus it is meaningful to ask if $\tilde u: \tilde M \to \RR$ is $\alpha$-equivariant, and the data of an $\alpha$-equivariant function on $\tilde M$ is equivalent to the data of a map $u: M \to \Sph^1$ of homotopy class $\alpha$ (up to addition of an constant).
We record a frequently useful special case of the Galois correspondence for covering spaces:

\begin{lemma}\label{Galois}
Let $M$ be a closed manifold and let $\alpha \in H_{d - 1}(M, \ZZ)$ be nonzero.
Then there exists a covering space $p: \hat M \to M$ of Galois group $\Gal(\hat M, M) \cong \ZZ$, such that for any map $u: M \to \Sph^1$ of homotopy class $\alpha$, the lift of $u$ is topologically trivial:
$$\begin{tikzcd}
\hat M \arrow[r, "\hat u"] \arrow[d, "p"] & \RR \arrow[d] \\
M \arrow[r, "u"] & \Sph^1
\end{tikzcd}$$
\end{lemma}
\begin{proof}
We view $\alpha$ as an integral representation.
Since $\ker \alpha$ fits into a short exact sequence 
$$0 \to \ker \alpha \to \pi_1(M) \to \ZZ \to 0,$$
the result follows from the Galois correspondence for covering spaces.
\end{proof}

\begin{proposition}
Let $M$ be a closed manifold, let $\alpha \in H_{d - 1}(M, \RR)$ have rational direction, $\alpha \neq 0$, and let $u$ be an $\alpha$-equivariant function of least gradient on $\tilde M$.
Then every leaf of $\lambda_u$ is a closed hypersurface.
\end{proposition}
\begin{proof}
Rescaling $u$ by a constant does not affect whether the leaves of $\lambda_u$ are closed, so after rescaling, we may assume that $\alpha \in H_{d - 1}(M, \ZZ)$.
We view $u$ as a map $M \to \Sph^1$ of homotopy class $\alpha$.\footnote{Note that $u$ is not continuous, so strictly speaking the homotopy class of $u$ is not $\alpha$; but by equivariance, it is profitable to view $u$ in this way.}
Let $\hat u: \hat M \to \RR$ and $p: \hat M \to M$ be as in Lemma \ref{Galois}.
Then it makes sense to take superlevel sets $\{\hat u > y\}$.
Since $\Gal(\hat M, M)$ is a cyclic group, it is generated by a single element $h$ with the mapping property
$$h: \partial \{\hat u > y\} \to \partial \{\hat u > y + D\}$$
for each $y \in \RR$.
In particular, $\hat u$ does not have a global minimum or maximum.
Since $\hat u$ does not have a global minimum or maximum on the complete manifold $\hat M$, by the maximum principle for functions of least gradient \todo{cite paper 2}, $\hat u$ does not have any local minimum or maximum.
Therefore \todo{cite paper 2}, every leaf of $\lambda_{\hat u}$ is a component of $\partial \{u > y\}$ for some $y \in \RR$.

Now let $N$ be a (connected) leaf of $\lambda$, and let $\hat N := p^{-1}(N)$.
Then each component of $\hat N$ is a leaf of $\lambda_{\hat u}$, and hence a component of $\partial \{u > y\}$ for some $y \in \RR$.
Moreover, the deck transformation $h$ has the mapping property
$$h: \hat N \to \hat N.$$
So if $\hat N$ meets $\partial \{u > y\}$, then for every $n \in \ZZ$, a component $K_n$ of $\hat N$ is a component of $\partial \{\hat u > y + nD\}$.
Since $h$ acts on $\lambda_{\hat u}$ by sending $\partial \{\hat u > y\}$ to $\partial \{\hat u > y + D\}$, $h$ acts on $\hat N$ by sending $K_n$ to $K_{n + D}$.
In particular, $\hat N \cong N \times \ZZ$ and $p: \hat N \to N$ is projection onto the first factor. 
So $N \cong K_0$, and $K_0$ is a closed hypersurface in $\hat M$ by the regularity theorem \todo{cite paper 2}.
\end{proof}

\begin{example}
Suppose that $M = \Sph^1_x \times \Sph^1_y$.
Let $u(x, y) = x$, so $\lambda_u$ is itself topologically nontrivial (in the $x$ direction) and its leaves are also topologically nontrivial (in the $y$ direction).
The covering space $\hat M$ is $\RR_x \times \Sph_y^1$, the point is that we unwound $\lambda_u$ without messing with its leaves.
\end{example}

%%%%%%%%%%%%%%%%%%%%%%%%%%%%%%%%
% \subsection{\texorpdfstring{$\infty$-harmonic functions} with many duals}
% We now apply the above theory, and a theorem of Massart \cite{Massart1997StableNO}, to disprove a conjecture of Daskalopolous and Uhlenbeck \cite[Conjecture 9.3]{daskalopoulos2020transverse}.
% To state the conjecture, recall that if $F$ is a tight $1$-form on a closed surface $M$, then its primitive $v: \tilde M \to \RR$ is an $\infty$-harmonic function.
% The conjecture is that the dual of an equivariant $\infty$-harmonic function is unique; we shall show that not even the dual cohomology class is unique.

% \begin{proposition}
% Let $M$ be a closed Riemannian surface of genus $\geq 2$.
% Then there exists a tight $1$-form $\dif v$ on $M$, and $1$-harmonic duals $u_1, u_2$, such that $\dif u_1, \dif u_2$ are not homologous.
% \end{proposition}
% \begin{proof}
% 	Let $\alpha \in H_1(M, \RR)$ have rational direction, and lie on the stable unit sphere: $\Mass(\alpha) = 1$.
% 	By Massart's theorem \cite[Theorem 7]{Massart1997StableNO}, there exists $\rho \in H^1(M, \RR)$ with $\Comass(\rho) = 1$ and $\alpha \in \rho^*$, and $\beta \in \rho^*$ with $\alpha \neq \beta$.
% 	Let $u_1, u_2$ be the functions of least gradient representing $\alpha, \beta$ furnished by Lemma \ref{existence for least gradient},\footnote{Daskalopolous and Uhlenbeck require their dual functions of least gradient to admit $p$-approximations, but as we already noted, $u_j$ can be chosen to admit $p$-approximations.} and note that if $\dif v$ is a tight representative of $\rho$ as constructed by \todo{cite it} then $u_1, u_2$ are dual functions of least gradient to $\dif v$.
% \end{proof}


%%%%%%%%%%%%%%%%%%%%%%%%%%%%%%%%


% \section{Lattice MFMC}
% \begin{definition}
% By a \dfn{closed continuous flow network} we mean a pair $(M, \alpha)$ where $M$ is a closed Riemannian manifold, and $\alpha \in H_{d - 1}(M, \RR)$ is nonzero.
% By a \dfn{solution for the continuous MF/MC problem} with data $(M, \alpha)$ we mean a pair $(F, \lambda)$ where $F$ is a calibration on $M$ and $\lambda$ is a measured oriented $F$-calibrated lamination which represents $\alpha$.
% \end{definition}

% \begin{definition}
% By a \dfn{discrete flow network} we mean a triple $(G, s, t)$ where $G$ is a directed graph and $s, t$ are distinct vertices of $G$.
% By a \dfn{solution for the discrete MF/MC problem} with data $(G, s, t)$ we mean a pair $(f, c)$ where $f$ is a divergence-free $(s, t)$-flow, $c$ is an $(s, t)$-cut, and 
% $$\min_{c' \in \mathrm{Cuts}(\mathcal G_h)} \sum_{e \in c'} f(e) = \sum_{e \in c} f(e) = \max_{f' \in \mathrm{Flows}(\mathcal G_h)} \sum_{e \in c} f'(e).$$
% \end{definition}

% \todo{Why are we doing this?}

% \begin{theorem}[discrete MF/MC theorem]\label{existence of discrete MFMC}
% Let $(G, s, t)$ be a discrete flow network.
% Then there exists a solution $(f, c)$ of the discrete MF/MC problem with data $(G, s, t)$.
% \end{theorem}

% %%%%%%%%%%%%%%%%%
% \subsection{Reduction for manifolds with boundary}
% \begin{definition}
% Let $\overline M = M \cup \partial M$ be a compact Riemannian manifold-with-boundary.
% \begin{enumerate}
% \item We say that $\overline M$ is \dfn{weakly convex} if the mean curvature $H_{\partial M} \geq 0$.
% \item Suppose that $\partial M$ has two connected components $S, T$. We call $(\overline M, S, T)$ a \dfn{continuous flow network with boundary} if $S, T$ are homologous in $H_{d - 1}(\overline M, \RR)$.
% \item By a \dfn{solution for the continuous MF/MC problem} with data $(\overline M, S, T)$ we mean a pair $(F, N)$ where $F$ is a calibration on $M$ and $N \subset M$ is a closed $F$-calibrated hypersurface which is homologous to $S$ in $H_{d - 1}(\overline M, \RR)$.
% \end{enumerate}
% \end{definition}

% \begin{proposition}[continuous MF/MC theorem with boundary]\label{existence of continuous MFMC with boundary}
% Let $(\overline M, S, T)$ be a continuous flow network with boundary.
% If $d \leq 7$ and $\overline M$ is weakly convex, then there is a solution $(F, N)$ to the continuous MF/MC problem with data $(\overline M, S, T)$.
% \end{proposition}

% % We use the discretization proposed by Freedman and Headrick \cite[Appendix A.6]{Freedman_2016} for this purpose.
% % They sketch the idea for flows but not for cuts, and they don't prove it works.

% % Let $(\overline M, S, T)$ be a continuous flow network with boundary.
% % Let $h_* := \min(\|\Riem_M\|_{C^0}^{-2}, \inj(M))/10$.
% % For each $0 < h \leq h_*$, choose a maximal $h^2/2$-separated set $V_h' \subset M$ (so $\{B(x, h^2): x \in V_h\}$ is an open cover of $M$), and let
% % $$E_h' := \left\{(x, y) \in V_h \times V_h: \frac{h}{2} \leq \dist(x, y) < h\right\}.$$
% % We add two new members $s, t$ to $V_h'$ to obtain $V_h$, and new members $(s, x)$ (resp $(x, t)$) to $E_h'$ to obtain $E_h$ if $B(x, h)$ intersects $S$ (resp $T$).
% % Then $G_h := (V_h, E_h)$ is a directed graph, so $(G_h, s, t)$ is a discrete flow network.

% % Given an $(s, t)$-cut $c \subseteq E_h$, which is the set of edges out of some vertex set $W \subseteq V_h$, introduce the open set
% % $$U_c := \bigcup \{B(x, h): x \in W \setminus \{s\}\}.$$
% % If $U_c$ is empty, then $W = \{s\}$ and we set $N_h := S$.
% % Otherwise, $\partial U_c$ is piecewise smooth and contains the cycle $S$, so if we write $\partial U_c = N_c - S$, then $N_c$ is a Lipschitz cycle homologous to $S$.

% % \begin{lemma}
% % Let $c \subseteq E_h$ be an $(s, t)$-cut. Then
% % $$\Mass(N_c) = (h^{d - 1} + O(h^d)) \card c.$$
% % \end{lemma}
% % \begin{proof}
% % For each $(x, y) \in c$, consider the sphere $\partial B(x, h)$. 
% % \end{proof}

% % Conversely, if $N \subset M$ is a Lipschitz cycle homologous to $S$, choose an open set $U \subseteq M$ with $\partial U = N - S$.
% % Let $W_N$ be the union of $\{s\}$ with the set of vertices of $G_h$ in $U$, and let $c_N$ be the set of edges out of $W_N$.
% % Then $c_N$ is an $(s, t)$-cut.

% % \begin{lemma}
% % For any Lipschitz cycle $N \subset M$ homologous to $S$, $N_{c_N}$ is Lipschitz close to $N$ if $0 < h \ll h_*$.
% % In particular,
% % $$\Mass(N_{c_N}) = (1 + O(h^{d - 1})) \Mass(N).$$
% % \end{lemma}
% % \begin{proof}

% % \end{proof}

% % Choosing $N$ to be a hypersurface with 
% % $$\Mass(N) \leq (1 + h^{d - 1}) \inf_{N' \sim S} \Mass(N'),$$
% % where $\sim$ means ``homologous,'' we see that for $h$ arbitrarily small, the 

% % % Moreover, there exists some $W' \subseteq W$ and relatively open $E(x) \subseteq \partial B(x, h^2)$, $x \in W'$, such that 
% % % $$N_c' = \bigcup_{x \in W'} E(x).$$
% % % Let $F(x) \subseteq \overline{B(x, h^2)}$ be the area-minimizing hypersurface in $\overline{B(x, h^2)}$ with $\partial F(x) = \partial E(x)$, and let
% % % $$N_c := \bigcup_{x \in W'} F(x).$$
% % % Then $N_c$ is homologous to $N_c'$, \todo{and is piecewise smooth}, so is a Lipschitz cycle homologous to $S$.


% % \todo{Introduce flows}

% % \todo{This doesn't seem to work just use FEM}


% \subsection{Discretization by finite elements}
% Though there are more direct ways to prove Proposition \ref{existence of continuous MFMC with boundary}, we prove it by reducing to the \emph{discrete} MF/MC problem using the finite element method.

% Let $0 < h \leq 1$.
% By a \dfn{triangulation at scale $h$} of a manifold $L$ we mean a triangulation of $L$ into Lipschitz simplices $\delta$, such that for any simplex $\delta$, $\diam \delta \leq h$.
% Since any smooth manifold admits a triangulation, by iteratively taking barycentric subdivision we may find a triangulation at scale $h$ for any $h$.
% Moreover, if $N \subset M$ is a hypersurface, and we fix a triangulation $\mathcal T$ of scale $h$ of $N$, then we can extend $\mathcal T$ to a triangulation of scale $\mathcal T'$ of $M$ \todo{cite Munkres}
% % https://mathoverflow.net/questions/342503/extending-a-triangulation-of-the-boundary-of-m-times-i/342512#342512
% and then, by iteratively taking barycentric subdivisions in $M$, we may assume that $\mathcal T'$ is of scale $h$.

% If $\mathcal T$ is a triangulation of $M$, we introduce the discrete flow network $(G, s, t)$ associated to $\mathcal T$, as follows.
% Let the vertices of $G$ be the $d$-simplices of $\mathcal T$, along with special vertices $s$ and $t$ corresponding to $S, T$.
% If two $d$-simplices $\delta, \delta'$ have a common $d - 1$-face $\gamma$, let there be edges $(\delta, \delta')$ and $(\delta', \delta)$.
% Furthermore, if $\delta$ is a $d$-simplex with a $d - 1$-face $\gamma \subseteq S$, let there be edges $(s, \delta)$ and $(\delta, s)$; and similarly for $d$-vertices with faces in $T$.
% The weight of an edge associated to a $d - 1$-face $\gamma$ shall be $\Mass(\gamma)$.

% If $c \subseteq \mathrm{Edges}(G)$, we obtain a simplicial $d - 1$-chain $N$ (for the simplicial structure $\mathcal T$) by taking the sum of the $d - 1$-faces associated to the edges in $c$; we call $N$ the chain \dfn{associated} to $c$.
% This is mainly interesting when $c$ is an $(s, t)$-cut, for the following reason:

% \begin{lemma}\label{cuts induce cycles}
% If $(G, s, t)$ is a discrete flow network arising from a triangulation of $M$, $c$ is an $(s, t)$-cut in $G$, and $N$ is the chain associated to $c$, then $N$ is a cycle, homologous to $S$.
% \end{lemma}
% \begin{proof}
% Since $c$ is a $(s, t)$-cut, there exists a partition of the vertex set $V_s(G) \sqcup V_t(G)$ where $s \in V_s$, $t \in V_t$, and any edge from $V_s$ to $V_t$ passes through $c$.
% Let $U \subset \overline M$ be the interior of the union of the simplices in $V_s$.
% Then $U$ is bounded by $N - S$, so $N - S$ is exact.
% Since $S$ is a cycle, so is $N$.
% \end{proof}

% Any $f: \mathrm{Edges}(G) \to \RR$ induces a simplicial $d - 1$-cochain for the simplicial structure $\mathcal T$, which we also denote by $f$: if $e$ is a $d - 1$-simplex in $\mathcal T$, associated to an edge which we also denote $e$, we define $\langle f, e\rangle := f(e)$.
% By the characterization of Whitney forms \cite{dodziuk2022characterization}, associated to the simplicial cochain $f$ there exists a unique $d - 1$-form $F$ such that for each $d$-simplex $\iota_\sigma: \sigma \to M$ of $\mathcal T$, $\iota^*_\sigma F$ has affine coefficients, for each $d - 1$-simplex $\iota_e: e \to M$ of $\mathcal T$, $\iota_e^* F$ has constant coefficients, and for each such $e$,
% $$\int_e F = f(e).$$
% The form $F$ is called a \dfn{Whitney form} associated to $f$. \todo{We need to construct Whitney forms adapted to a Riemannian metric to get the $L^\infty$ property and for ``constant coefficients'' to be well-defined}

% \begin{lemma}\label{flows induce calibrations}
% If $(G, s, t)$ is a discrete flow network arising from a triangulation $\mathcal T$ at scale $h$ of $M$, $f$ is an $(s, t)$-flow on $G$, and $F$ is the Whitney form associated to $f$, then $\dif F = 0$ on $M$ and $\|F\|_{L^\infty} \leq 1 + O(h)$.
% \end{lemma}
% \begin{proof}
% The cochain $f$ is actually a cocycle: if $\delta$ is a $d$-simplex, and $O(\delta), I(\delta)$ denote the set of edges out of, and into, the vertex $\delta$, then
% $$\langle \dif f, \delta\rangle = \langle f, \partial \delta\rangle = \sum_{e \in O(\delta)} f(e) - \sum_{e \in I(\delta)} f(e) = 0$$
% since $f$ is a flow.
% Therefore $\dif F = 0$ \cite[Proposition 4.6]{Lohi21}. \todo{Estimate $\Comass(F)$}
% \end{proof}

% Proposition \ref{existence of continuous MFMC with boundary} follows from Theorem \ref{existence of discrete MFMC} and the following proposition:

% \begin{proposition}\label{discrete implies continuous MFMC}
% Let $(\overline M, S, T)$ be a continuous flow network with boundary.
% If $d \leq 7$ and $\overline M$ is weakly convex, then for each $0 < h \ll 1$, there exist triangulations $\mathcal T_h$ of $M$ at scale $h$ with the following property:

% Let $(G_h, s_h, t_h)$ be the discrete flow network induced by $\mathcal T_h$, and let $(f_h, c_h)$ be a solution of the discrete MF/MC problem with data $(G_h, s_h, t_h)$.
% Let $F_h$ be the Whitney form associated to $f_h$, and let $N_h$ be the simplicial $d - 1$-chain with simplices given by $c_h$.
% Then there is a solution $(F, N)$ to the continuous MF/MC problem with data $(\overline M, S, T)$ such that, after taking a subsequence as $h \to 0$, $F_h \to F$ weakly in $L^p$ for any $1 \leq p < \infty$, and $N_h \to N$ in the flat topology on currents.
% \end{proposition}

% Let $\alpha$ be the homology class of $S$ (hence of $T$) in $\overline M$.
% We say that a closed hypersurface $N$ representing $\alpha$ is \dfn{$\varepsilon$-minimizing} if 
% $$\Mass(N) \leq (1 + \varepsilon) \Mass(\alpha).$$

% \begin{lemma}\label{almost minimizers avoid boundary}
% Suppose that $S, T$ are not homologically minimizing hypersurfaces in $\alpha$.
% Then there exists a tubular neighborhood $E \supset \partial M$ such that for any $0 < \varepsilon \ll 1$, there is a $\varepsilon$-minimizing hypersurface in $\alpha$ which does not meet $E$.
% \end{lemma}
% \todo{Prove it, if this is enough}

% \begin{proof}[Proof of Proposition \ref{discrete implies continuous MFMC}]
% We first construct triangulation in which we shall solve the discrete MF/MC problem.
% In particular, for any $\varepsilon > 0$, there exists a smooth $\varepsilon$-minimizing hypersurface.
% Fix a scale $0 < h \ll 1$.
% By Lemma \ref{almost minimizers avoid boundary}, there is a $h^{d - 1}$-minimizing hypersurface $N_h$ which avoids an $h$-independent tubular neighborhood $F$ of $\partial M$.
% Choose a triangulation $\mathcal T_h$ of $M$ at scale $h$ which contains a triangulation of $N_h'$.
% \todo{Refinement of $\mathcal T_h$}
% Let $(G_h, s, t)$ be the induced flow network.

% Since $N_h' - S$ bounds a region $U$ which is a simplicial subcomplex of $(\overline M, \mathcal T_h)$ and similarly for $T - N_h'$ and a region $V$, the set of $d - 1$-simplices in $N_h'$ is an $(s, t)$-cut $c_h'$ in $G_h$, namely the cut which partitions $G_h$ into the $d$-simplicies in $U$ and the $d$-simplices in $V$.
% By Lemma \ref{cuts induce cycles}, there exists a Lipschitz cycle $N_h \subset M$ induced by $c_h$; moreover,
% $$\Mass(N_h) = \Mass(c_h) \leq \Mass(c_h') = \Mass(N_h') \leq (1 + h^{d - 1}) \Mass(\alpha).$$
% The flow $f_h$ induces a Whitney $d - 1$-form $F_h$ on $M$ such that
% $$\int_{N_h} F_h = \sum_{e \in c_h} f(e) = \sum_{e \in c_h} \Mass(e) = \Mass(N_h)$$
% (where $\int_{N_h} F_h$ is well-defined because $N_h$ misses $E$)
% and by Lemma \ref{flows induce calibrations}, $\dif F_h = 0$ and $\|F_h\|_{L^\infty} \leq 1 + O(h)$.
% By \todo{compactness of integral currents}, $N_h$ converges along a subsequence to some integral current $N$ which is homologous to $S$, misses $E$, and is mass-minimizing:
% $$\Mass(\alpha) \leq \Mass(N) \leq \limsup_{h \to 0} \Mass(N_h) \leq \Mass(\alpha).$$
% By \todo{compactness of $L^p$}, $F_h$ converges along a further subsequence to some $F$ with $\|F\|_{L^\infty} \leq 1$.
% Since $N$ misses $E$, $\int_N F$ is well-defined and by \todo{convergence of integral currents}, $\int_N F = \Mass(N)$.
% Therefore $N$ is $F$-calibrated. Therefore \todo{the leaves of $N$ are disjoint}, so $N$ is an $F$-calibrated chain, as desired.
% \end{proof}



% %%%%%%%%%%%%%%
% \subsection{Reduction to integral classes}
% \begin{lemma}\label{reduction to rational direction}
% Let $(\alpha_n)$ be a sequence of classes of rational direction which converge to $\alpha$.
% Let $(F_n, \lambda_n)$ be a sequence of solutions to the continuous MF/MC problem with data $(M, \alpha_n)$.
% Then there exists a solution $(F, \lambda)$ to the problem with data $(M, \alpha)$ such that, along a subsequence:
% \begin{enumerate}
% \item $(F_n)$ converges in $L^p$, for any $1 \leq p < \infty$, to $F$.
% \item $(\lambda_n)$ converges in the weak topology of measures to $\lambda$.
% \end{enumerate}
% \end{lemma}
% \begin{proof}
% The mass and comass of $\lambda_n, F_n$ are bounded, so we can take limits in the measure topology and $L^p$.
% Convergence  implies convergence in homology which guarantees that the sequences converge.
% \todo{Details}
% \end{proof}

% Given a measured lamination $\lambda$, and $c > 0$, let $c\lambda$ denote $\lambda$ with its transverse measure rescaled by $c$.
% It is clear that if $(M, \alpha)$ is data, with a solution $(F, \lambda)$, then for any $c > 0$, $(F, c\lambda)$ is a solution of the continuous MF/MC problem with data $(M, c\alpha)$.
% By Lemma \ref{reduction to rational direction}, for the purposes of approximating a solution to the continuous MF/MC problem we may assume that $\alpha$ has rational direction.
% By the scale-invariance described above, then, we may assume that $\alpha \in H_{d - 1}(M, \ZZ)$, and henceforth we do.

% %%%%%%%%%%%%%%%%
% \subsection{Reduction to cut covering spaces}
% Let $p: \hat M \to M$ be the covering space attained from Lemma \ref{Galois}.
% Since $\Gal(\hat M, M)$ is infinite, $\hat M$ is not compact, but each fundamental domain $\hat M_0 \subset \hat M$ is compact.

% \begin{lemma}
% The fundamental domain $\hat M_0$ has exactly two boundary components, $\partial \hat M_0 = S_0 \sqcup T_0$, each of which is a cycle.
% \end{lemma}
% \begin{proof}
% \todo{Topology is hard}
% \end{proof}

% Let $h: \hat M \to \hat M$ be a generator of the cyclic group $\Gal(\hat M, M)$.
% Then $\hat M$ consists of biinfinitely many copies $h^n_*(\hat M_0)$, $n \in \ZZ$, of $\hat M_0$, which are glued along their boundary components: we can choose the decomposition $\partial \hat M_0 = S_0 \sqcup T_0$ so that for each $n \in \ZZ$,
% $$h^n_*(T_0) = h^{n + 1}_*(S_0).$$
% Let 
% $$\hat M_n := \bigcup_{k=-n}^n h^n_*(\hat M_0),$$
% so that $\hat M_n$ is bounded by $S_n := h^{-n}_*(S_0)$ and $T_n := h^n_*(T_0)$.
% We call $\hat M_n$ a \dfn{cut covering space} of $M$, since it is obtained by cutting off the covering space $\hat M$ to have only finitely many fundamental domains of $M$.

% \begin{definition}
% Let $n \geq 1$. We say that a pair $(\hat F_n, \hat \lambda_n)$ is a \dfn{solution} of the continuous MF/MC problem with data $(\hat M_n, S_n, T_n)$ concentrated in $\hat M_0$ if:
% \begin{enumerate}
% \item $\hat F_n$ has best comass with respect to compactly supported cohomology on the interior of $\hat M_n$.
% \item $\hat F_n$ is a calibration.
% \item $\hat \lambda_n$ is a measured oriented lamination on $\hat M_0$.
% \item $\hat \lambda_n$ is homologous to $S_n$.
% \item $\hat \lambda_n$ is $\hat F_n$-calibrated.
% \end{enumerate}
% \end{definition}

% \todo{Justify that we can move a solution to one which is concentrated on $\hat M_0$}

% Since $\hat M$ is periodic, it is no loss to consider those solutions concentrated in $\hat M_0$.
% In that case, $\hat \lambda$ will not intersect the boundary $S_n$, nor will it ``escape to infinity'' as we take $n \to \infty$.

% \begin{lemma}\label{reduction to one covering space}
% Suppose that $(\hat F_n, \hat \lambda_n)$ are solutions to the continuous MF/MC problems with data $(\hat M_n, S_n, T_n)$ concentrated in $\hat M_0$.
% Then along a subsequence, $(\hat F_n, \hat \lambda_n)$ converge in \todo{what topology?} to a pair $(\hat F, \hat \lambda)$, such that:
% \begin{enumerate}
% \item $\hat F$ has best comass with respect to compactly supported cohomology on $\hat M$.
% \item $\hat F$ is a calibration.
% \item $\hat \lambda$ is a measured oriented lamination on $\hat M_0$.
% \item $\hat \lambda$ is homologous to $S_0$.
% \item $\hat \lambda$ is $\hat F$-calibrated.
% \end{enumerate}
% \end{lemma}
% \begin{proof}
% \todo{Use compactness but you might need to take translations to keep them from running away to infinity}
% \end{proof}

% Suppose that we can solve the continuous MF/MC problem on each cut covering space.
% Let $(\hat F, \hat \lambda)$ be as in Lemma \ref{reduction to one covering space}.
% Let $\hat u$ be the function of least gradient associated to $\hat \lambda$.
% We introduce the average over pullbacks
% $$F := \lim_{n \to \infty} \frac{1}{2n + 1} \sum_{m=-n}^n (h^{-n})^* F$$
% and the sum of pushforwards 
% $$\dif u := \lim_{n \to \infty} \sum_{m=-n}^n h^n_* \dif u.$$
% Here the limits are taken in \todo{what topology?} and it is clear from the definitions that $F, \dif u$ are closed, with $F$ an invariant best comass calibration with respect to compactly supported cohomology, and $u$ a equivariant function of least gradient.
% Let $\lambda$ be the measured oriented lamination associated to $u$, so $\lambda$ is $F$-calibrated. \todo{Check all this}
% Therefore $(F, \lambda)$ drops to a solution of the MF/MC problem on $M$.

% Therefore it remains to show that given a cut covering space $\hat M_n$, we can discretize the continuous MF/MC problem on $\hat M$.


% \subsection{Runtime and practicality}
% \todo{Write this out} Side note: each simplex has mass $\sim h^d$ so there are $h^{-d}$ simplices.
% Each one gives a vertex and $T_d$ edges where $T_d \sim d^2$ is the $d$th triangle number.
% So $|V| \sim h^{-d}$ and $|E| \sim d^2 h^{-d}$.
% Ford-Fulkerson has $\sim |V| |E|^2 \sim d^4 h^{-3d}$ runtime.
% I guess this is worse than Loisel's algorithm since if $d = 3$ our runtime is $\sim h^{-9}$ while Loisel's is $h^{-8.52} \log(h^{-1})$.
% But maybe under some circumstances it's better, like if you need a max flow and a min cut together.

% \todo{Application to quantum gravity?}

%%%%%%%%%%%%%%%%%%%%%%%%%%
\section{Open problems}\label{open problems}
\subsection{More general Riemannian manifolds}
For simplicity in this paper we have only dealt with closed manifolds of dimension $d \leq 7$.
However, it is likely that the results largely go through for compact manifolds with boundary, as our arguments are mainly local, and for those that are not one can pass to the double of the manifold with boundary, as long as the boundary is not itself a minimal hypersurface.
Moreover, one can most likely recover the results of this paper without the dimension assumption, as long as one is willing to deal with the consequence of having minimal hypersurfaces with singularities, since we mainly deal with phenomena in codimension $\leq 2$, while singularities of minimal hypersurfaces necessarily have codimension $\geq 8$. 
There are genuine differences in the case of manifolds with infinite ends, since if $M$ is a manifold with cusps, the stable (semi)norm of every homology class in $M$ may be zero.
Still, one expects to be able to recover most of the results of this paper.

\begin{problem}\label{generalization}
Formulate the results in this paper for complete Riemannian manifolds with boundary and infinite ends, and arbitrary dimension $d \geq 2$.
\end{problem}


%%%%%%%%%%
\subsection{Laminations and foliations}\label{canonical conjectures}
Thurston's canonical lamination $\lambda$ is chain-recurrent, in the sense that traveling along the geodesics in $\lambda$ defines a chain-recurrent dynamical system.
This makes no sense for higher-dimensional laminations, but is equivalent to assert that Thurston's canonical lamination can be approximated by finite sums of closed geodesics \cite[\S9]{Gu_ritaud_2017}.
We conjecture that the analogous fact should hold for our canonical lamination:

\begin{conjecture}\label{chain recurrence}
Let $\rho \in H^{d - 1}(M, \RR)$, and let $\lambda_\rho$ be the canonical lamination.
Then it is possible to approximate $\lambda_\rho$ in Thurston's geometric topology\footnote{See \cite[\S1]{BackusCML} for the definition of Thurston's geometric topology in this setting.} by finite unions of closed minimal hypersurfaces.
\end{conjecture}

\footnote{It might be possible using the reduction to integral classes!}

The following problem was suggested to me by Karen Uhlenbeck. 
There exist closed hyperbolic $3$-manifolds which admit taut foliations; in that case, \emph{after changing the metric} one may find a minimal foliation.
Thus one cannot rule out minimal foliations by a simple topological argument (as one could rule out geodesic foliations of closed hyperbolic surfaces).
However, if a minimal foliation exists, then it is natural to study the tight form which calibrates it.
This form satisfies a particularly strong form 
\begin{equation}\label{eikonal}
\begin{cases}\dif F = 0 \\ \dif(|F|^2) = 0\end{cases}
\end{equation}
of the Euler-Lagrange equation for tight forms which is analogous to the role of the eikonal equation
$$\dif(|\dif u|^2) = 0$$
in the study of the $\infty$-Laplace equation.
Global solutions of the eikonal equation are rather uncommon (for example, the Dirichlet problem for the eikonal equation on $\Ball^d$ is extremely overdetermined), so this suggests a means to rule out the existence of minimal foliations:

\begin{conjecture}\label{Karen}
Let $\Gamma$ be the fundamental group of a closed hyperbolic $3$-manifold $M$.
Then there does not exist a solution of the eikonal system (\ref{eikonal}) on $\Hyp^3$ which is invariant under $\Gamma$.
In particular, there does not exist a minimal foliation on $M$.
\end{conjecture}

%%%%%%%%%%%%%%%%%%%%%%%%
\subsection{String theory}
Let us highlight a possible application of this work to string theory.
Freedman and Headrick considered the holographic principle in the setting where a theory of quantum gravity lives on a Riemannian manifold $M$, and the dual conformal field theory lives on its boundary $\partial M$ \cite{Freedman_2016}.
We assume that if $\Gamma \subset \partial M$ bounds a domain in $\partial M$, then there exists an area-minimizing and minimal hypersurface in $M$ with boundary $\Gamma$.
In this setting, if $N$ is a smooth subdomain of $\partial M$, the entropy of entanglement $S(\partial N)$ of the CFT through $\partial N$ is given by the \dfn{Ryu-Takayanagi formula}
\begin{equation}\label{RyuTakayanagi}
S(\partial N) = \max_F \int_N F
\end{equation}
where $F$ ranges over calibrations on $\overline M$ \cite[(2.8)]{Freedman_2016}.

If the results of this paper generalize cleanly to manifolds with boundary (see Problem \ref{generalization}) then, by replacing $N$ with an area-minimizing hypersurface in $M$ with the same boundary, we should have:

\begin{conjecture}
There exists a tight calibration $F$ which realizes the maximum in (\ref{RyuTakayanagi}).
\end{conjecture}

On the other hand, we know by recent work of Loisel that it is possible to solve the $\infty$-Laplacian in polynomial time \cite{Loisel_2020}.
We expect the same technique to work for tight forms, and so we expect the following problem to be doable.

\begin{problem}
Given a hypersurface $N \subset \overline M$ which is homologous relative to $\partial N$ to an area-minimizing hypersurface, give an efficient algorithm to numerically find a tight calibration $F$ which realizes the maximum in (\ref{RyuTakayanagi}), and in particular to compute $S(\partial N)$ if $N \subset \partial M$.
\end{problem}


%%%%%%%5


\appendix
\section{Local Hodge theory in \texorpdfstring{$L^p$}{Lp}} \label{local Hodge appendix}
\begin{lemma}\label{Hodge theorem}
Suppose that there is a bi-Lipschitz diffeomorphism $M \cong \Ball^d$.
Let $1 < p < \infty$, and let $F$ be an $L^p$ closed $\ell + 1$-form.
Then there exists an $\ell$-form $A$ such that $F = \dif A$ and
\begin{equation}\label{Hodge theorem estimate}
\|A\|_{W^{1, p}} \lesssim_p \|F\|_{L^p}.
\end{equation}
\end{lemma}

\todo{Whitney forms}

%%%%%%%%%%%%%%%%%
\section{Singularities of nodal sets}\label{nodal appendix}
Suppose that $v$ solves an elliptic PDE.
Then we write $Z(v), Z^{\rm sing}(v)$ for the nodal and singular sets of $v$, namely the sets of zeroes and double zeroes, respectively.
We will show that the generic point of $Z(v)$ is not a singular point. More precisely:

\begin{proposition}\label{nodal set is generically smooth}
Let $Q$ be a linear elliptic operator on $\Ball^{d - 1}$ satisfying the maximum principle.
Suppose that $Qv = 0$ and $v$ has a zero of finite order.
Then the Hausdorff dimensions of the nodal and singular sets of $v$ are
\begin{align}
	\dim(Z(v)) &= d - 2, \label{nodal dimension}\\
	\dim(Z^{\rm sing}(v)) &\leq d - 3. \label{singular nodal dimension}
\end{align}
\end{proposition}

The main idea of the proof is to show that the complement of $Z^{\rm sing}(v)$ is path-connected.
By Alexander duality for singular cohomology, the complement of a submanifold $P$ of codimension $\geq 2$ is path-connected.
The singular set $P = Z^{\rm sing}(v)$ is not in general a manifold, but the proof still works as long as we apply Alexander duality for sheaf cohomology.
Let $\hat H^\bullet(P, \RR)$ denote the cohomology of the constant sheaf $\RR$ on $P$.

\begin{lemma}\label{closed mfld complement}
Let $P \subset \Sph^{d - 1}$ be a closed $d - 3$-rectifiable set.
Then $\Sph^{d - 1} \setminus P$ is path-connected.
\end{lemma}
\begin{proof}
Let $\delta^{\rm Haus}, \delta^{\rm cov}, \delta^{\rm shf}$ be the Hausdorff, covering, and sheaf cohomological dimensions of $P$ respectively.
Then by \cite[{\S}II.5.12]{godement1973topologie} and \cite[Theorem 6.3.10]{edgar2008measure}, we have 
$$\delta^{\rm shf} \leq \delta^{\rm cov} \leq \delta^{\rm Haus} \leq d - 3,$$
hence $\hat H^{d - 2}(P, \RR) = 0$.
By Alexander duality for sheaf cohomology \cite[Theorem 6]{Kaplan47}, it follows that $H_0(\Sph^{d - 1} \setminus P, \RR) \cong \ZZ$, or in other words $\Sph^{d - 1} \setminus P$ is path-connected.
\end{proof}

\begin{lemma}\label{open mfld complement}
Let $P \subset \Ball^{d - 1}$ be a closed $d - 3$-rectifiable set.
Then $\Ball^{d - 1} \setminus P$ is path-connected.
\end{lemma}
\begin{proof}
Embed $\Ball^{d - 1}$ in $\Sph^{d - 1}$ using the one-point compactification, let $\infty$ be the point at infinity, and let $x, y \in \Ball^{d - 1} \setminus P$.
Choose a $d - 3$-sphere $S$ in $\Sph^{d - 1}$ which contains $\infty$ but does not contain $x, y$.
Then $P \cup S$ is a closed $d - 3$-rectifiable set and $x, y \notin P \cup S$, so by Lemma \ref{closed mfld complement}, there exists a curve $\gamma$ from $x$ to $y$ which avoids $P \cup S$.
Therefore $\gamma \subset \Ball^{d - 1} \setminus P$.
\end{proof}

\begin{proof}[Proof of Proposition \ref{nodal set is generically smooth}]
By \cite[Lemma 1.9]{Hardt89}, $Z^{\rm sing}(v)$ is $d - 3$-rectifiable, which implies (\ref{singular nodal dimension}).
If there exists $x \in Z(v) \setminus Z^{\rm sing}(v)$, then by the implicit function theorem, there is a neighborhood $U \ni x$ such that $U \cap Z(v)$ is a $d - 2$-dimensional manifold.
So if (\ref{nodal dimension}) fails, we must have $Z(v) = Z^{\rm sing}(v)$, so $Z(v)$ is $d - 3$-rectifiable.
But then, by Lemma \ref{open mfld complement}, the sets $U_\pm := \{\pm v > 0\}$ satisfy $U_+ \cup U_-$ are connected.
Since $v$ is continuous, one of these sets must be empty; without loss of generality, $U_- = \emptyset$.
Then $v \geq 0$ and $v$ has a zero, so by the maximum principle, $v = 0$ identically.
This contradicts the fact that $v$ has a zero of finite order.
\end{proof}


\printbibliography

\end{document}
