\documentclass[reqno,11pt]{amsart}
\usepackage[letterpaper, margin=1in]{geometry}
\RequirePackage{amsmath,amssymb,amsthm,graphicx,mathrsfs,url,slashed,subcaption}
\RequirePackage[usenames,dvipsnames]{xcolor}
\RequirePackage[colorlinks=true,linkcolor=Red,citecolor=Green]{hyperref}
\RequirePackage{amsxtra}
\usepackage{cancel}
\usepackage{tikz, quiver, wrapfig}

% Add the 2020 MSC
\makeatletter
\@namedef{subjclassname@2020}{\textup{2020} Mathematics Subject Classification}
\makeatother

%\usepackage[T1]{fontenc}

% \setlength{\textheight}{9.3in} \setlength{\oddsidemargin}{-0.25in}
% \setlength{\evensidemargin}{-0.25in} \setlength{\textwidth}{7in}
% \setlength{\topmargin}{-0.25in} \setlength{\headheight}{0.18in}
% \setlength{\marginparwidth}{1.0in}
% \setlength{\abovedisplayskip}{0.2in}
% \setlength{\belowdisplayskip}{0.2in}
% \setlength{\parskip}{0.05in}
%\renewcommand{\baselinestretch}{1.05}

\title{The canonical lamination calibrated by a cohomology class}
\author{Aidan Backus}
\address{Department of Mathematics, Brown University}
\email{aidan\_backus@brown.edu}
\date{\today}
\keywords{laminations, convex duality, minimal hypersurfaces, max flow/min cut theorem, calibrations, functions of least gradient, stable norm}
\subjclass[2020]{primary: 49Q20; secondary: 39J60, 49N15, 53C38}

\newcommand{\NN}{\mathbf{N}}
\newcommand{\ZZ}{\mathbf{Z}}
\newcommand{\QQ}{\mathbf{Q}}
\newcommand{\RR}{\mathbf{R}}
\newcommand{\CC}{\mathbf{C}}
\newcommand{\DD}{\mathbf{D}}
\newcommand{\PP}{\mathbf P}
\newcommand{\MM}{\mathbf M}
\newcommand{\II}{\mathbf I}
\newcommand{\Hyp}{\mathbf H}
\newcommand{\Sph}{\mathbf S}
\newcommand{\Torus}{\mathbf T}
\newcommand{\Group}{\mathbf G}
\newcommand{\GL}{\mathbf{GL}}
\newcommand{\Orth}{\mathbf{O}}
\newcommand{\SpOrth}{\mathbf{SO}}
\newcommand{\Ball}{\mathbf{B}}

\newcommand*\dif{\mathop{}\!\mathrm{d}}

\DeclareMathOperator{\card}{card}
\DeclareMathOperator{\dist}{dist}
\DeclareMathOperator{\id}{id}
\DeclareMathOperator{\Hom}{Hom}
\DeclareMathOperator{\coker}{coker}
\DeclareMathOperator{\PD}{PD}
\DeclareMathOperator{\supp}{supp}
\DeclareMathOperator{\Teich}{Teich}
\DeclareMathOperator{\tr}{tr}

\newcommand{\Leaves}{\mathscr L}
\newcommand{\Lagrange}{\mathscr L}
\newcommand{\Hypspace}{\mathscr H}

\newcommand{\Chain}{\underline C}

\newcommand{\Euler}{\mathbf \chi}

\newcommand{\Two}{\mathrm{I\!I}}
\newcommand{\Ric}{\mathrm{Ric}}


\newcommand{\weakto}{\rightharpoonup}

\newcommand{\normal}{\mathbf n}
\newcommand{\radial}{\mathbf r}
\newcommand{\evect}{\mathbf e}
\newcommand{\vol}{\mathrm{vol}}

\newcommand{\diam}{\mathrm{diam}}
\DeclareMathOperator{\Gal}{Gal}
\DeclareMathOperator{\sech}{sech}
\newcommand{\Ell}{\mathrm{Ell}}
\newcommand{\inj}{\mathrm{inj}}
\newcommand{\Lip}{\mathrm{Lip}}
\newcommand{\MCL}{\mathrm{MCL}}
\newcommand{\Riem}{\mathrm{Riem}}

\newcommand{\Mass}{\mathbf M}
\newcommand{\Comass}{\mathbf L}

\newcommand{\Min}{\mathrm{Min}}
\newcommand{\Max}{\mathrm{Max}}

\newcommand{\dfn}[1]{\emph{#1}\index{#1}}

\renewcommand{\Re}{\operatorname{Re}}
\renewcommand{\Im}{\operatorname{Im}}

\newcommand{\loc}{\mathrm{loc}}
\newcommand{\cpt}{\mathrm{cpt}}

\def\Japan#1{\left \langle #1 \right \rangle}

\newtheorem{theorem}{Theorem}[section]
\newtheorem{badtheorem}[theorem]{``Theorem"}
\newtheorem{prop}[theorem]{Proposition}
\newtheorem{lemma}[theorem]{Lemma}
\newtheorem{sublemma}[theorem]{Sublemma}
\newtheorem{proposition}[theorem]{Proposition}
\newtheorem{corollary}[theorem]{Corollary}
\newtheorem{conjecture}[theorem]{Conjecture}
\newtheorem{axiom}[theorem]{Axiom}
\newtheorem{assumption}[theorem]{Assumption}

\newtheorem{mainthm}{Theorem}
\renewcommand{\themainthm}{\Alph{mainthm}}

\newtheorem{claim}{Claim}[theorem]
\renewcommand{\theclaim}{\thetheorem\Alph{claim}}
% \newtheorem*{claim}{Claim}

\theoremstyle{definition}
\newtheorem{definition}[theorem]{Definition}
\newtheorem{remark}[theorem]{Remark}
\newtheorem{example}[theorem]{Example}
\newtheorem{notation}[theorem]{Notation}

\newtheorem{exercise}[theorem]{Discussion topic}
\newtheorem{homework}[theorem]{Homework}
\newtheorem{problem}[theorem]{Problem}

\makeatletter
\newcommand{\proofpart}[2]{%
  \par
  \addvspace{\medskipamount}%
  \noindent\emph{Part #1: #2.}
}
\makeatother



\numberwithin{equation}{section}


% Mean
\def\Xint#1{\mathchoice
{\XXint\displaystyle\textstyle{#1}}%
{\XXint\textstyle\scriptstyle{#1}}%
{\XXint\scriptstyle\scriptscriptstyle{#1}}%
{\XXint\scriptscriptstyle\scriptscriptstyle{#1}}%
\!\int}
\def\XXint#1#2#3{{\setbox0=\hbox{$#1{#2#3}{\int}$ }
\vcenter{\hbox{$#2#3$ }}\kern-.6\wd0}}
\def\ddashint{\Xint=}
\def\dashint{\Xint-}

\usepackage[backend=bibtex,style=alphabetic,giveninits=true]{biblatex}
\renewcommand*{\bibfont}{\normalfont\footnotesize}
\addbibresource{best_curl.bib}
\renewbibmacro{in:}{}
\DeclareFieldFormat{pages}{#1}

\newcommand\todo[1]{\textcolor{red}{TODO: #1}}


\begin{document}
\begin{abstract}
\todo{Absorb from the general paper}
\end{abstract}

\maketitle

%%%%%%%%%%%%%%%%%%%%%%%%%%%%%%%%%%%%%%%%%%%%%%%%%%%%%%%
\section{Introduction}
Given hyperbolic structures $\rho, \sigma$ on a closed surface $M$, Thurston \cite{Thurston98} and Gu\'eritaud and Kassel \cite{Gu_ritaud_2017} constructed a \dfn{canonical geodesic lamination} $\lambda$ on $(M, \rho)$ such that any map $f: (M, \rho) \to (M, \sigma)$ homotopic to $\id_M$ which minimizes its Lipschitz constant $L$ stretches every leaf of $\lambda$ by a factor of $L$.
The canonical lamination has found use in understanding the duality between the tangent and cotangent spaces to the Teichm\"uller space of $M$.

Independently of the Thurston school, Auer and Bangert \cite{Auer01} proposed to study codimension-$1$ measured oriented laminations $\lambda$ of a closed oriented Riemannian manifold $M$ which minimize their mass in their homology class $[\lambda]$.
The \dfn{stable norm} $\Mass([\lambda])$ is the mass of the mass-minimizing lamination.
Though they deferred many of the proofs to the work \cite{Auer12}, which is unfinished and not publicly available, already in the research announcement \cite{Auer01}, Auer and Bangert realized that the study of laminations provides a deep connection between the intersection theory of $M$ and the geometry of the unit ball of the stable norm on $H_{d - 1}(M, \RR)$.
Essentially this is because in codimension $1$, every homology class contains a mass-minimizing lamination $\lambda$, and the leaves of $\lambda$ cannot intersect.

Recent work of Daskalopoulos and Uhlenbeck \cite{daskalopoulos2020transverse,daskalopoulos2022,daskalopoulos2023} and I \cite{BackusCML,BackusBest1} has taken the perspective that the duality between mass-minimizing laminations $\lambda$ and the calibrations which witness that $\lambda$ are minimizing essentially arises as the limit as $p \to \infty$ of the duality between $L^p$ and $L^q$ variational problems (where $1/p + 1/q = 1$).
The $L^\infty$ variational problems describe the calibration, while the $L^1$ variational problems describe the laminations.
In particular, I showed that in any cohomology class $\rho \in H^{d - 1}(M, \RR)$ on a closed oriented Riemannian manifold of dimension $2 \leq d \leq 7$, one can find a solution $F$ of a certain $L^\infty$ variational problem.
The \dfn{costable norm} $\Comass$ is the dual of the stable norm.
If $\Comass(\rho) = 1$, then $F$ is a calibration and is dual to a mass-minimizing measured oriented lamination $\lambda$ arising from a function of least gradient.
However, the measured lamination $\lambda$ is far from unique.

The above situation is analogous to the situation of Thurston, in which there are many measured oriented maximally stretched geodesic laminations, but they all embed into the canonical lamination.
Given that Thurston's lamination ties into the duality theory of the tangent bundle to Teichm\"uller space, and codimension-$1$ lamination tie into the duality theory of the stable norm, it is natural to expect that a canonical codimension-$1$ lamination also exists.
To be more precise, we show:

\begin{mainthm}\label{existence of calibrated lam}
Let $M$ be a closed Riemannian manifold of dimension $2 \leq d \leq 7$, and let $\rho \in H^{d - 1}(M, \RR)$ satisfy $\Comass(\rho) = 1$.
Consider the set $\mathscr S$ of all complete hypersurfaces $N$ such that for every calibration $F$ representing $\rho$, $N$ is $F$-calibrated.
Then $\mathscr S$ is the (nonempty) set of leaves of a Lipschitz lamination $\lambda_\rho$, the \dfn{canonical lamination} of $\rho$.
\end{mainthm}

In particular, every measured oriented lamination which is calibrated by some calibration in $\rho$ embeds in the canonical lamination $\lambda_\rho$.

We then study the structure of the canonical lamination $\lambda_\rho$.
We decompose its leaves into leaves which can be embedded into a measured sublamination, and those which do not admit a transverse measure.
The latter type of leaf is unavoidable in general, since a leaf may accumulate on itself countably many times.
The set of the homology classes of measured sublaminations of $\lambda_\rho$ turns out to be the dual flat $\rho^*$ of $\rho$.
Since $\rho^*$ is a convex set, we study the role of the extreme points of $\rho^*$ in determining the structure of $\lambda_\rho$ as well.

To round out the paper, we provide proofs of two results about the stable norm ball claimed without proof by Auer and Bangert \cite[Theorems 6 and 7]{Auer01}:

\begin{mainthm}
Let $M$ be a closed Riemannian manifold of dimension $2 \leq d \leq 7$, and let $B$ be the unit ball for the stable norm on $H_{d - 1}(M, \RR)$.
Then:
\begin{enumerate}
\item If there is a line segment $[\alpha, \beta]$ in the stable unit sphere $\partial B$, then the intersection product $\alpha \cdot \beta$ is $0$.
\item If the universal abelian covering space $\tilde M^{\rm ab} \to M$ satisfies $H^1(\tilde M^{\rm ab}, \RR) = 0$, then $B$ is strictly convex.
\end{enumerate}
\end{mainthm}

Thus, for example, the stable unit ball of any manifold homotopic to a torus is strictly convex.
The key idea is already in the research announcement \cite{Auer01}.
Indeed, if $[\alpha, \beta]$ is a line segment in $\partial B$, then Auer and Bangert propose to construct a lamination $\lambda$ such that the mass-minimizing laminations $\kappa_\alpha, \kappa_\beta$ in $\alpha, \beta$ both embed in $\lambda$.
Therefore there can be no intersection between the leaves of $\kappa_\alpha, \kappa_\beta$.
From our perspective, however, the existence of the lamination $\lambda$ is quite clear: it is the canonical lamination of a class $\rho \in H^{d - 1}(M, \RR)$ such that $[\alpha, \beta] \subseteq \rho^*$.
We have made use of certain lemmata in the draft \cite{Auer12} which concern the structure of functions of least gradient on $\tilde M^{\rm ab}$.

Finally we discuss the possibility of constructing a canonical calibrated lamination of a class $\rho \in H^k(M, \RR)$ where $1 \leq k \leq d - 2$:
\begin{enumerate}
\item The case $k = 1$ was essentially resolved by Daskalopoulos and Uhlenbeck \cite{daskalopoulos2020transverse}.
One can find a $1$-form $\dif u$ representing $\rho$ such that $u$ is a locally defined $\infty$-harmonic function.
Then comparison with cones implies that $\dif u$ calibrates a geodesic lamination $\kappa$, and a convex duality argument shows that there is a measured oriented sublamination $\kappa'$ of $\kappa$.
It is then straightforward to show that there is a canonical lamination $\lambda_\rho$, with $\kappa' \subseteq \lambda_\rho \subseteq \kappa$.
\item If $k \geq 2$, then one cannot construct a canonical calibrated lamination, because calibrated submanifolds may intersect each other.
This is a particularly natural phenomenon in the presence of special holonomy.
To give a simple example, consider $M = \PP^1_\CC \times \PP^1_\CC$, with K\"ahler $2$-form $\omega$ induced by the Fubini-Study metric on $\PP^1_\CC$.
Then $\omega$ calibrates both factors of $\PP^1_\CC$, but they have nontrivial intersection.
\end{enumerate}

%%%%%%%%%%%%%%%%%%%%%%%%%
\subsection{Acknowledgements}
I would like to thank Georgios Daskalopoulos and Karen Uhlenbeck for helpful discussions and for providing me with a draft copy of \cite{daskalopoulos2023}.
I would also like to thank Victor Bangert for providing me with a draft copy of \cite{Auer12}.

This research was supported by the National Science Foundation's Graduate Research Fellowship Program under Grant No. DGE-2040433.

%%%%%%%%%%%%%%%%%%%%%%%%%%%%%%%%%%%%%%%%%%
\section{Preliminaries}\label{prevResults}
\subsection{Review of previous papers}
We use \cite{BackusCML, BackusBest1} as a reference.
\todo{Summarize the previous papers once they're on the arxiv. In particular define measured stretch laminations}

\begin{theorem}[{\cite[Theorem A]{BackusCML}}]\label{disjoint surfaces are lamination}
Let $\mathcal S$ be a set of disjoint complete minimal hypersurfaces in a manifold $M$ of bounded geometry.
Suppose that there exists $C > 0$ such that for every $N \in \mathcal S$, $\|\Two_N\|_{C^0} \leq C$.
Then $\mathcal S$ is the set of leaves of a Lipschitz minimal lamination $\lambda$.
In particular, if $\lambda$ is oriented, then there is a Lipschitz vector field on $M$ whose restriction to each $N \in \mathcal S$ is the normal vector to $N$.
\end{theorem}

\begin{proposition}\label{calibration condition}
Let $F$ be a calibration on a closed Riemannian manifold $M$.
Let $T_\lambda$ be the Ruelle-Sullivan current of a measured oriented lamination $\lambda$.
Then the following are equivalent:
\begin{enumerate}
\item One has \begin{equation}\label{calibration by Ruelle Sullivan}
\int_M T_\lambda \wedge F = \Mass(\lambda).
\end{equation}
\item $\lambda$ is $F$-calibrated.
\end{enumerate}
\end{proposition}

\begin{proposition}\label{MCL contains Thurston}
Let $F$ be a best comass representative of $\rho$, and let $\lambda$ be a measured stretch lamination associated to $\rho$.
Then $F$ calibrates $\lambda$.
\end{proposition}

\begin{theorem}[{\cite[Theorem B]{BackusBest1}}]\label{lams are calibrated}\label{calibrated means measured stretch}
Suppose that $M$ is a closed Riemannian manifold of dimension $d \leq 7$.
Let $\rho \in H^{d - 1}(M, \RR)$ satisfy $\Comass(\rho) = 1$.
Let $\kappa$ be a measured stretch lamination for $\rho$.
Let $F$ be a best comass representative of $\rho$.
Then $\kappa$ is $F$-calibrated, and for $\lambda$ ranging over measured oriented laminations,
\begin{equation}\label{duality between stable and comass}
1 = \sup_\lambda \frac{\langle \rho, [\lambda]\rangle}{\Mass(\lambda)} = \frac{\langle \rho, [\kappa]\rangle}{\Mass(\kappa)}.
\end{equation}
Conversely, if a measured oriented lamination $\lambda$ attains the maximum in (\ref{duality between stable and comass}), then $\lambda$ is a measured stretch lamination.
\end{theorem}

\subsection{The maximum comass locus}
\begin{definition}
For an open set $\Omega \subseteq M$, let $\Chain_{d - 1}(\Omega)$ be the set of simplicial $d - 1$-chains in $\Omega$.
For each closed $d - 1$-form $F$ on $\Omega$, let
$$\Comass_\Omega(F) := \sup_{\sigma \in \Chain_{d - 1}(\Omega)} \frac{1}{\Mass(\sigma)} \int_\sigma F.$$
The \dfn{local comass} of a closed $d - 1$-form $F$ at $x \in M$ is 
$$\Comass(F, x) = \limsup_{\varepsilon \to 0} \Comass_{B_\varepsilon(x)}(F).$$
\end{definition}

By the trace theorem, $\Comass_\Omega(F)$ is well-defined (but possibly $+\infty$) for any measurable $F$.
Since $\Comass_{B_\varepsilon(x)}(F)$ is a supremum over a set which grows in $\varepsilon$, it is increasing in $\varepsilon$, so the limit superior is actually a limit and an infimum:
$$\Comass(F, x) = \lim_{\varepsilon \to 0} \Comass_{B_\varepsilon(x)}(F) = \inf_{\varepsilon > 0} \Comass_{B_\varepsilon(x)}(F).$$
In particular, if we write $\Comass(F) := \Comass_M(F)$, then $\Comass(F, x) \leq \Comass(F)$.

The local comass was defined in an analogous manner to the local Lipschitz constant.
As such, it enjoys many of the same properties, including those endowed on the local Lipschitz constant by \cite[Lemma 4.3]{Crandall2008}:

\begin{proposition}\label{crandall}
Let $F \in L^\infty(M, \Omega^{d - 1}_{\rm cl})$. Then:
\begin{enumerate}
\item The local comass $\Comass(F, \cdot)$ is upper semicontinuous. \label{crandall usc}
\item For almost every $x \in M$, \label{crandall LDT}
$$|F(x)| \leq \Comass(F, x).$$
\item The local comass is bounded, and \label{crandall linfinity}
$$\Comass(F) = \sup_{x \in M} \Comass(F, x) = \|F\|_{L^\infty}.$$
\item If $\sigma \in \Chain_{d - 1}(M)$ then \label{crandall best curl is ABC}
$$\int_\sigma F \leq \Mass(\sigma) \sup_{x \in \sigma} \Comass(F, x).$$
\end{enumerate}
\end{proposition}
\begin{proof}
We first prove (\ref{crandall usc}).
Let $x^n \to x$ and $r > 0$. Then eventually $x^n \in B_r(x)$, hence $\Comass(F, x^n) \leq \Comass_{B_r(x)}(F)$ and so
\begin{align*}
\limsup_{n \to \infty} \Comass(F, x^n) &\leq \inf_{r > 0} \Comass_{B_r(x)}(F) = \Comass(F, x).
\end{align*}

We now prove (\ref{crandall LDT}).
We may work locally, and choose coordinates $(y^i)$ in which $\sqrt{\det g} = 1$.
Let $I$ be the increasing $d-1$-index with $d$ removed.
By the Lebesgue differentiation theorem and Fubini's theorem, there exists a null set $Z \subset M$, which does not depend on $(y^i)$ by \cite[Proposition 2.1]{BackusFLG}, such that for every $x \notin Z$,
\begin{align*}
F_I(x) 
&= \lim_{\varepsilon \to 0} \frac{1}{\Mass(B_\varepsilon(x))} \int_{B_\varepsilon(x)} F_I(y) \dif y \\
&= \lim_{\varepsilon \to 0} \frac{1}{\Mass(B_\varepsilon(x))} \int_{-\infty}^\infty \int_{\{y^d = t\} \cap B_\varepsilon(x)} F_I(y) \dif y^1 \wedge \cdots \wedge \dif y^{d - 1} \wedge \dif t
\end{align*}
where we used the fact that $\sqrt{\det g} = 1$.
Now $\partial_{y^1} \wedge \cdots \wedge \partial_{y^{d - 1}}$ is the oriented unit $d - 1$-blade tangent to $\{y^d = t\}$, so as forms on $\{y^d = t\}$,
$$F_I(y) \dif y^1 \wedge \cdots \wedge \dif y^{d - 1} = F.$$
So
\begin{align*}
F_I(x) 
&= \lim_{\varepsilon \to 0} \frac{1}{\Mass(B_\varepsilon(x))} \int_{-\infty}^\infty \int_{\{y^d = t\} \cap B_\varepsilon(x)} F \dif t \\
&\leq \lim_{\varepsilon \to 0} \frac{\Comass_{B_\varepsilon(x)}(F)}{\Mass(B_\varepsilon(x))} \int_{-\infty}^\infty |\{y^d = t\} \cap B_\varepsilon(x)| \dif t.
\end{align*}
By Fubini's theorem,
$$F_I(x) \leq \lim_{\varepsilon \to 0} \frac{\Comass_{B_\varepsilon(x)}(F)}{\Mass(B_\varepsilon(x))} \Mass(B_\varepsilon(x)) = \Comass(F, x).$$
For every $x \in M$ we may select coordinates in which $|F(x)| = F_I(x)$, and then if $x \notin Z$, we conclude that (\ref{crandall LDT}) holds for $x$.

If we combine (\ref{crandall LDT}) with \todo{the fact that the trace is a contraction in $L^\infty$}, then
$$\sup_{x \in M} \Comass(F, x) \leq \Comass(F) \leq \|F\|_{L^\infty} \leq \sup_{x \in M} \Comass(F, x).$$
The inequalities collapse, proving (\ref{crandall linfinity}).
In particular, for each $\sigma \in \Chain_{d - 1}(M)$, we obtain (\ref{crandall best curl is ABC}):
\begin{align*}
\int_\sigma F &\leq \Mass(\sigma) \inf_{\Omega \supset \sigma} \sup_{x \in \Omega} \Comass(F, x) = \Mass(\sigma) \sup_{x \in \sigma} \Comass(F, x). \qedhere
\end{align*}
\end{proof}

\begin{definition}
Let $F \in L^\infty(M, \Omega^{d - 1}_{\rm cl})$.
The \dfn{maximum comass locus} is the set
$$\MCL(F) := \{x \in M: \Comass(F, x) = \Comass(F)\}.$$
\end{definition}

\begin{corollary}
Suppose that $M$ is a closed manifold and $F \in L^\infty(M, \Omega^{d - 1}_{\rm cl})$.
Then $\MCL(F)$ is a nonempty compact set.
\end{corollary}
\begin{proof}
This is immediate from Proposition \ref{crandall}(\ref{crandall usc}) and the extreme value theorem for upper semicontinuous functions.
\end{proof}

\begin{proposition}\label{properties of calibrated laminations}
Suppose that $M$ is a closed Riemannian manifold, $F$ is a calibration, and $\lambda$ is a measured oriented $F$-calibrated lamination.
Then $\supp \lambda \subseteq \MCL(F)$.
\end{proposition}
\begin{proof}
Let $S := \MCL(F)$, $N$ a leaf of $\lambda$, and suppose that $x \in N \setminus S$.
Since $S$ is closed, there exists $\varepsilon > 0$ such that $B_\varepsilon(x)$ does not meet $S$.
Moreover, $\sigma := N \cap B_\varepsilon(x)$ is a $d-1$-chain in $B_\varepsilon(x)$, so by Proposition \ref{crandall}(\ref{crandall best curl is ABC}),
$$\frac{1}{\Mass(\sigma)} \int_\sigma F \leq \sup_{y \in B_\varepsilon(x)} \Comass(F, y) < \Comass(F) = 1.$$
But then 
$$\int_N F = \int_\sigma F + \int_{N \setminus B_\varepsilon(x)} F < \Mass(\sigma) + \Mass(N \setminus B_\varepsilon(x)) = \Mass(N),$$
so $N$ (hence $\lambda$) is not $F$-calibrated.
\end{proof}

%%%%%%%%%%%%%%%%
\section{The canonical lamination}
\label{canonical sec}
Throughout this section, we fix $\rho \in H^{d - 1}(M, \RR)$ in the costable unit sphere $\{\Comass(\rho) = 1\}$.
Motivated by Thurston's approach to Teichm\"uller theory (see \S\ref{Teichmuller}), we construct a lamination which is calibrated by every best comass form in $\rho$, and which only depends on $\rho$: the \dfn{canonical lamination} $\lambda_\rho$.

Even if $F$ is a best comass representative of $\rho$, then $\MCL(F)$ need not itself be a lamination.
There are two essentially unrelated reasons why this can happen:
\begin{enumerate}
\item If $d \geq 3$, then the bundle $\ker(\star F)$ does not need to be involutive near $\MCL(F)$.
This happens even if $F$ is smooth and $M$ is homeomorphic to $\mathbf T^3$ \cite[Example 5.4]{bangert_cui_2017}.
\item Suppose that $d = 2$, and let $F$ be the tight $1$-form representing $\rho$. Then $\MCL(F)$ is a geodesic lamination \cite[Theorem 5.2]{daskalopoulos2020transverse}, so if $M$ is hyperbolic then $\MCL(F)$ must be a null subset of $M$.
However, every closed set $\lambda \subseteq M$ containing $\MCL(F)$ is the maximum comass locus of some tight form \cite{BackusZeAn}, and in particular we can take $\lambda$ to not be a geodesic lamination.
\end{enumerate}
So we must show that there is a lamination $\lambda_F \subseteq \MCL(F)$ which contains every $F$-calibrated hypersurface.
We will then be able to show that the intersection $\bigcap_F \lambda_F$, where $F$ ranges over best comass representatives of $\rho$, is a lamination.

One may wonder why we must take an intersection:
If $F$, $G$ are cohomologous, and $\lambda$ is an $F$-calibrated measured lamination, then $\lambda$ is $G$-calibrated by Proposition \ref{properties of calibrated laminations}.
However, not every lamination admits a transverse measure, and for such a lamination, we do not expect calibration to be a cohomological invariant.

%%%%%%%%%%%%%%%%%%%%%%%%%
\subsection{Sheaf cohomology of singular sets}
We are going to need to estimate the connectivity of sets whose complements are not necessarily manifolds.
We accomplish this using Alexander duality for sheaf cohomology.
Indeed, sheaf cohomology is well-behaved for arbitrary closed sets, even if they are not locally contractible \cite{Kaplan47}.
Let $\hat H^\bullet(P, \RR)$ denote the reduced cohomology of the constant sheaf $\RR$ on the space $P$, and let $\tilde H_\bullet(Q, \RR)$ denote the reduced singular homology of the space $Q$.

\begin{lemma}\label{closed mfld complement}
Let $\delta^{\rm Haus} \in [0, d]$, and let $P \subset \Sph^d$ be a closed set of Hausdorff dimension $\delta^{\rm Haus}$.
Then, if $k \geq \lfloor \delta^{\rm Haus} \rfloor$, then
$$\tilde H_{d - k - 1}(\Sph^d \setminus P, \RR) = 0.$$
\end{lemma}
\begin{proof}
Let $\delta^{\rm cov}, \delta^{\rm shf}$ be the Lebesgue covering, and sheaf cohomological dimensions of $P$ respectively.
Then by \cite[{\S}II.5.12]{godement1973topologie} and \cite[Theorem 6.3.10]{edgar2008measure}, we have 
$$\delta^{\rm shf} \leq \delta^{\rm cov} \leq \lfloor \delta^{\rm Haus} \rfloor \leq k.$$
Here we used the fact that $\delta^{\rm cov} \in \NN$ to replace $\delta^{\rm Haus}$ with $\lfloor \delta^{\rm Haus} \rfloor$.
So $\hat H^k(P, \RR) = 0$, and the result follows from Alexander cohomology for sheaf cohomology \cite[Theorem 6]{Kaplan47}.
\end{proof}

\begin{lemma}\label{open mfld complement}
Let $P \subset \Ball^{d - 1}$ be a closed $d - 3$-rectifiable set.
Then $\Ball^{d - 1} \setminus P$ is path-connected.
\end{lemma}
\begin{proof}
Embed $\Ball^{d - 1}$ in $\Sph^{d - 1}$ using the one-point compactification, let $\infty$ be the point at infinity, and let $x, y \in \Ball^{d - 1} \setminus P$.
Choose a $d - 3$-sphere $S$ in $\Sph^{d - 1}$ which contains $\infty$ but does not contain $x, y$.
Then $P \cup S$ is a closed $d - 3$-rectifiable set and $x, y \notin P \cup S$, so by Lemma \ref{closed mfld complement}, there exists a curve $\gamma$ from $x$ to $y$ which avoids $P \cup S$.
Therefore $\gamma \subset \Ball^{d - 1} \setminus P$.
\end{proof}

%%%%%%%%%%%%%%%%%%%%%%%%%%
\subsection{Intersections of minimal hypersurfaces}\label{nodal appendix}
Let $F$ be a best comass representative of $\rho$.
We begin the construction of the lamination $\lambda_F$ by showing that any two $F$-calibrated hypersurfaces are disjoint.
This can be done by showing that the generic intersection point of two minimal hypersurfaces $N, N'$ is transverse.
If the dimension of the underlying manifold $M$ is $d = 2$, then this is trivial, and if $d = 3$, then the structure of $N \cap N'$ is completely described by complex-analytic means \cite[Theorem 7.3]{colding2011course}, so the proof we present here is mainly of interest if $d \geq 4$.

For a solution $v$ of an elliptic PDE, we write $Z(v), Z^{\rm sing}(v)$ for the nodal and singular sets of $v$, namely the sets of zeroes and double zeroes, respectively.
We show that the generic point of the nodal set is nonsingular:

\begin{lemma}\label{nodal set is generically smooth}
Let $Q$ be a linear elliptic operator on $\Ball^{d - 1}$ satisfying the maximum principle.
Suppose that $Qv = 0$ and $v$ has a zero of finite order.
Then the Hausdorff dimensions of the nodal and singular sets of $v$ are
\begin{align}
	\dim(Z(v)) &= d - 2, \label{nodal dimension}\\
	\dim(Z^{\rm sing}(v)) &\leq d - 3. \label{singular nodal dimension}
\end{align}
\end{lemma}
\begin{proof}
By \cite[Lemma 1.9]{Hardt89}, $Z^{\rm sing}(v)$ is $d - 3$-rectifiable, which implies (\ref{singular nodal dimension}).
If there exists $x \in Z(v) \setminus Z^{\rm sing}(v)$, then by the implicit function theorem, there is a neighborhood $U \ni x$ such that $U \cap Z(v)$ is a $d - 2$-dimensional manifold.
So if (\ref{nodal dimension}) fails, we must have $Z(v) = Z^{\rm sing}(v)$, so $Z(v)$ is $d - 3$-rectifiable.
But then, by Lemma \ref{open mfld complement}, the sets $U_\pm := \{\pm v > 0\}$ satisfy $U_+ \cup U_-$ are connected.
Since $v$ is continuous, one of these sets must be empty; without loss of generality, $U_- = \emptyset$.
Then $v \geq 0$ and $v$ has a zero, so by the maximum principle, $v = 0$ identically.
This contradicts the fact that $v$ has a zero of finite order.
\end{proof}

\begin{proposition}\label{intersection theory prop}
Let $N, N' \subset M$ be minimal hypersurfaces, and let $S \subseteq N, N'$ be the set of points at which $N, N'$ intersect nontransversely.
Then one of the following holds:
\begin{enumerate}
\item $N \cap N'$ is empty.
\item $\dim(N \cap N') = d - 2$ and $\dim S \leq d - 3$.
\item There exists $p \in S$ such that the germs of $N, N'$ at $p$ are equal.
\end{enumerate}
\end{proposition}
\begin{proof}
Let $p \in S$, and let $P$ be the tangent space of $N, N'$ at $x$.
Then we can view $N, N'$ as the graphs of functions $u, u'$ over $P$, say taken in normal coordinates based at $p$; thus we identify $P$ with $\RR^{d - 1}$.
Reasoning as in the proof of \cite[Theorem 7.3]{colding2011course}, the difference $v := u - u'$ solves a linear elliptic PDE $Qv = 0$, and in a neighborhood $U \ni p$, the exponential map $P \to N$ induces Lipschitz isomorphisms $\{v = 0\} \cap U \cong N \cap N' \cap U$ and $\{v = \dif v = 0\} \cap U \cong S \cap U$.
If $v$ only has zeroes of finite order, then the claim follows from Proposition \ref{nodal set is generically smooth}.
Otherwise, $v$ is identically $0$ by the unique continuation theorem \cite[Theorem 6.1]{colding2011course}, so $N \cap U = N' \cap U$.
\end{proof}

%%%%%%%%%%%%%%%%%%%%%
\subsection{Curvature and mass bounds on the leaves}
\begin{lemma}
Let $F$ be a calibration, and let $B \subseteq M$ be a sufficiently small ball.
Then for any complete connected $F$-calibrated hypersurface $N$, 
\begin{equation}\label{area bound for calibrated}
\Mass(N \cap B) \leq \Mass(\partial B).
\end{equation}
\end{lemma}
\begin{proof}
By the Thom transversality theorem, possibly after shrinking $B$ slightly, we may assume that $B$ meets $N$ transversely.
Since both sides of (\ref{area bound for calibrated}) depend continuously on the radius of $B$, this assumption is no loss of generality.

Let $S := N \cap \partial B$, which by transversality can be identified with a closed $d - 2$-dimensional submanifold of $\Sph^{d - 1}$.
Since $H_{d - 2}(\Sph^{d - 1}, \RR) = 0$, there exists a relatively open set $U \subseteq \partial B$ which is bounded by $S$.
Since $H^{d - 1}(B, \RR) = 0$, we may write $F = \dif A$ in a neighborhood of $B$, where by \todo{$L^\infty$ Poincare lemma} we may assume that $A$ is continuous.
Then
\begin{align*}
\Mass(N \cap B) &= \int_{N \cap B} F = \int_S A = \int_U F \leq \Mass(U) \leq \Mass(\partial B). \qedhere
\end{align*}
\end{proof}

\begin{proposition}
There exists a constant $C > 0$, only depending on $M$, such that for every calibration $F$ and complete $F$-calibrated hypersurface $N$, we have the curvature bound
\begin{equation}\label{curvature bound for calibrated}
\|\Two_N\|_{C^0} \leq C.
\end{equation}
\end{proposition}
\begin{proof}
Let $x \in N$ and let $r > 0$ be small.
Then each component $N'$ of $N \cap B(x, r)$ is absolutely area-minimizing by the fundamental theorem of calibrated geometry, so it is stable.
By (\ref{area bound for calibrated}), $\Mass(N') \lesssim r^{d - 1}$.
So by \cite[pg785, Corollary 1]{Schoen81} (see also \cite[Chapter 2, \S\S4-5]{colding2011course}),
\begin{align*}
\|\Two_{N'}\|_{C^0(B(x, r/2))} \lesssim_{d, \|\Riem_g\|_{C^0(B(x, 2r))}} \frac{1}{r}.
\end{align*}
Since $N'$ was an arbitrary component, the same estimate holds for $N$.
Using the compactness of $M$, we may cover it by finitely many balls in which estimates of this form hold to conclude (\ref{curvature bound for calibrated}).
\end{proof}

%%%%%%%%%%%%%%%%%
\subsection{Existence of the canonical lamination}
We are now ready to prove Theorem \ref{existence of calibrated lam}, the existence of the canonical lamination.
\begin{lemma}\label{calibrated implies disjoint}
Let $F$ be a calibration, and let $N, N'$ be complete connected $F$-calibrated hypersurfaces.
If $N \cap N'$ is nonempty, then $N = N'$.
\end{lemma}
\begin{proof}
We first observe that if $x \in N \cap N'$, then $(\star F(x))^\sharp$ is the normal vector to both $N, N'$ at $x$.
Therefore $N \cap N'$ only consists of points of tangency.
By Proposition \ref{intersection theory prop}, it follows that either the germs of $N, N'$ at $x$ are equal.
Since the germs are equal and $N, N'$ are connected, a standard boostrapping argument implies that $N = N'$.
\end{proof}

\begin{proposition}\label{existence of semicanonical lamination}
Let $F$ be a best comass calibration.
Then the set of $F$-calibrated hypersurfaces is the set of leaves of a lamination $\lambda_F$, which contains every measured stretch lamination associated to $[F]$.
\end{proposition}
\begin{proof}
Let $\mathscr L_F$ be the set of connected complete $F$-calibrated hypersurfaces.
By Lemma \ref{calibrated implies disjoint}, $\mathscr L_F$ consists of pairwise disjoint minimal hypersurfaces.
By Proposition \ref{MCL contains Thurston}, there exists a measured stretch lamination $\lambda$ associated to $[F]$, and then by Proposition \ref{properties of calibrated laminations}, $\mathscr L_F$ contains every leaf of $\lambda$.
Since the estimate (\ref{curvature bound for calibrated}) is independent of $N$, it follows by Theorem \ref{disjoint surfaces are lamination} that $\mathscr L_F$ is the set of leaves of some lamination $\lambda_F$.
\end{proof}

\begin{lemma}\label{existence of intersections}
Let $\mathscr S$ be a nonempty set of laminations.
Suppose that there exists a hypersurface which is a leaf of every lamination in $\mathscr S$.
Then there exists a lamination whose set of leaves is the intersection of the sets of leaves of the laminations in $\mathscr S$.
\end{lemma}
\begin{proof}
Let $\lambda \in \mathscr S$, and let $(F_\alpha, K_\alpha)$ be a laminar atlas for $\mathscr S$.
Let $K'_\alpha$ be the set of $k \in K_\alpha$ such that for every $\kappa \in \mathscr S$, there exists a leaf $N$ of $\kappa$ such that
$$(F_\alpha)_*(\{k\} \times J) \subseteq N.$$
It is clear that this property is preserved by transition maps.
Then $K_\alpha'$ is an intersection of compact sets (since the local leaf spaces of each $\kappa \in \mathscr S$ is compact), so $K_\alpha'$ is compact.
The hypersurface which is a common leaf of every lamination in $\mathscr S$ witnesses that for some $\alpha$, $K_\alpha'$ is nonempty.
Therefore $(F_\alpha, K'_\alpha)$ is a laminar atlas for the lamination whose support is $\bigcap_{\kappa \in \mathscr S} \supp \kappa$.
\end{proof}

\begin{proposition}\label{existence of canonical lamination}
The set of hypersurfaces which are calibrated by every best comass representative of $\rho$ is the set of leaves of a lamination $\lambda_\rho$, which contains every measured stretch lamination associated to $\rho$.
\end{proposition}
\begin{proof}
By Proposition \ref{MCL contains Thurston}, there is a (measured stretch) lamination which is calibrated by every best comass representative of $\rho$.
So we may apply Lemma \ref{existence of intersections} to the set $\mathscr S$ of all calibrated laminations $\lambda_F$ produced by Proposition \ref{existence of semicanonical lamination}, where $F$ ranges over best comass representatives of $\rho$.
\end{proof}

\begin{definition}
The lamination $\lambda_\rho$ constructed in Proposition \ref{existence of canonical lamination} is the \dfn{canonical lamination} associated to $\rho$.
\end{definition}


%%%%%%%%%%%%%%%%%%%%%%%%%%%%%%%%
\section{Structure of the canonical lamination}\label{canonical structure}
Let $M$ be a closed oriented Riemannian manifold of dimension $\leq 7$, and let $\rho$ be a point in the costable unit sphere of $H^{d - 1}(M, \RR)$.
We now study the structure of the canonical lamination $\lambda_\rho$.

%%%%%%%%%%%%%%%%%
\subsection{The dual set}
A sticky technical point is that $H_{d - 1}(M, \RR)$ need not be strictly convex, so there may be many $\alpha$ in the stable unit sphere such that 
\begin{equation}\label{flats duality}
\Comass(\rho) = \langle \rho, \alpha\rangle.
\end{equation}
In particular, there may be many measured stretch sublaminations of the canonical lamination which are mutually nonhomologous.
We therefore introduce the dual set 
$$\rho^* := \{\alpha \in H_{d - 1}(M, \RR): \langle \rho, \alpha\rangle = \Mass(\alpha) = 1\}.$$
It is clear that any measured sublamination of the canonical lamination normalized to have mass $1$ represents a member of $\rho^*$.
In fact, this condition completely characterizes $\rho^*$, as we now show.

\begin{lemma}\label{homologically minimizing means measured stretch}
For every $\alpha \in \rho^*$, every measured oriented, homologically minimizing, lamination representing $\alpha$ is a measured stretch lamination associated to $\rho$.
\end{lemma}
\begin{proof}
Let $\dif u$ be the Ruelle-Sullivan current of the measured oriented, homologically minimizing lamination $\lambda$, and let $F$ be a best comass representative of $\rho$.
Since $\lambda$ is homologically minimizing,
$$\int_M \dif u \wedge F = \langle \rho, \alpha\rangle = \Mass(\alpha) = \Mass(\lambda),$$
so by Proposition \ref{calibration condition}, $F$ calibrates $\lambda$.
Therefore by Proposition \ref{calibrated means measured stretch}, $\lambda$ is a measured stretch lamination.
\end{proof}

\begin{lemma}\label{existence for least gradient}
For each $\alpha \in H_{d - 1}(M, \RR)$, there exists an $\alpha$-equivariant function of least gradient $u: \tilde M \to \RR$.
\end{lemma}
\begin{proof}
The space $X$ of $\alpha$-equivariant functions on $\tilde M$ is closed under $L^1_\loc$ limits \todo{because $L^1_\loc$ convergence preserves equivariance}, so the existence of a minimizer in $X$ follows from an argument similar to the solution of the Dirichlet problem for least gradient functions \cite[Theorem 1.20]{Giusti77}.
\end{proof}

\begin{proposition}\label{enough measures in canonical lamination}
For each $\alpha \in \rho^*$, there exists a measured stretch sublamination of $\lambda_\rho$ with homology class $\alpha$.
\end{proposition}
\begin{proof}
Let $u$ be the function of least gradient furnished by Lemma \ref{existence for least gradient}.
The measured oriented, homologically minimizing, lamination $\kappa_u$ has class $\alpha$.
So by Lemma \ref{homologically minimizing means measured stretch}, it is a measured stretch lamination and hence is a sublamination of $\lambda_\rho$.
\end{proof}

%%%%%%%%%%%%%%%%%%%%%%%%
\subsection{Classification of the leaves}
We next use the decomposition of measured laminations \cite[{\S}I.3]{Morgan88} to partition the leaves of $\lambda_\rho$ into various categories.
In this direction we shall need to study measured laminations which are minimal with respect to inclusion; as the word ``minimal'' is overloaded, we shall call such laminations ``indecomposable''.

\begin{definition}
Let $\lambda$ be a lamination.
\begin{enumerate}
\item $\lambda$ is \dfn{indecomposable} if the only sublamination of $\lambda$ is itself.
\item If $\lambda$ is indecomposable, then $\lambda$ is \dfn{exceptional} if $\supp \lambda \neq M$ and $\lambda$ does not consist of a single leaf.
\item $\lambda$ is a \dfn{parallel family of closed leaves} if there exists a closed hypersurface $N \subset M$ with trivial normal bundle, such that every leaf of $\lambda$ is a section of the normal bundle of $N$.
\item A leaf $N$ of $\lambda$ is \dfn{nonmeasurable} if, for every sublamination $\kappa \subset \lambda$ which admits a transverse measure, $N$ is not a leaf of $\kappa$.
\end{enumerate}
\end{definition}

Thus every indecomposable lamination either is a foliation in which every leaf is dense, an exceptional indecomposable lamination, or a closed hypersurface.
Moreover, every local leaf space $K_\alpha$ of an exceptional indecomposable lamination $\lambda$ is a Cantor set \cite[{\S}I.3.1]{Morgan88}, and every leaf of $N$ is noncompact.
Every nonmeasurable leaf is noncompact, for if $N$ is a closed leaf, then $N$ equipped with its Dirac measure is a measured sublamination of $\lambda$.

\begin{theorem}\label{MorganShelan}
Let $\lambda$ be a measured oriented lamination in the closed manifold $M$.
Then either $\lambda$ is a foliation with a dense leaf, or $\lambda$ separates into finite number of clopen sublaminations, each of which is a parallel family of closed leaves or an exceptional indecomposable lamination.
\end{theorem}
\begin{proof}
First observe that the proof of \cite[Theorem I.3.2]{Morgan88} goes through for any lamination $\lambda$ such that no leaf of $\lambda$ is dense in $M$, even if $\lambda$ is a foliation.
Twisted families of closed leaves (that is, families of sections of a nontrivial normal bundle of a closed hypersurface) are excluded by the fact that $\lambda$ is oriented, so its leaves are oriented, and hence the normal bundle of any of its leaves is trivial.
\end{proof}

\begin{proposition}\label{classification of leaves}
For each leaf $N$ of $\lambda_\rho$, one of the following holds:
\begin{enumerate}
\item $N$ is closed.
\item $N$ is a noncompact leaf of an exceptional indecomposable measured stretch lamination associated to $\rho$.
\item $N$ is noncompact and $\lambda_\rho$ is a foliation which admits a transverse measure.
\item $N$ is noncompact and $N$ is a nonmeasurable leaf of $\lambda_\rho$.
\end{enumerate}
\end{proposition}
\begin{proof}
If $N$ is a closed leaf of $\lambda_\rho$, then $N$ equipped with its Dirac measure is a measured lamination, calibrated by any tight representative of $\rho$; hence it is measured stretch for $\rho$.
Otherwise, since $N$ has no boundary, it is noncompact.

If $N$ is noncompact, but is contained in a measured sublamination $\kappa$ of $\lambda_\rho$, then by Theorem \ref{MorganShelan}, either $\kappa$ is a foliation or $N$ is contained in an exceptional indecomposable sublamination.
If $\kappa$ is a foliation, then
$$\supp \kappa \supseteq \supp \lambda_\rho \supseteq \supp \kappa,$$
implying $\kappa = \lambda_\rho$.
Otherwise, the exceptional indecomposable sublamination $\zeta$ of $\kappa$ containing $N$ is calibrated by any tight representative of $\rho$, so $\zeta$ is measured stretch for $\rho$ by Proposition \ref{calibrated means measured stretch}.
\end{proof}

\begin{corollary}
Let $F$ be a best comass representative of $\rho$, and $N$ a leaf of the calibrated lamination $\lambda_F$.
Then either $N$ is a leaf of the canonical lamination $\lambda_\rho$, or $N$ is a nonmeasurable leaf of $\lambda_F$.
\end{corollary}
\begin{proof}
Suppose that $N$ is a leaf of a measured sublamination $\kappa$ of $\lambda_F$.
Then, since $\kappa$ is calibrated by $F$, $\kappa$ is measured stretch by Proposition \ref{calibrated means measured stretch}, hence is a sublamination of $\lambda_\rho$.
\end{proof}

%%%%%%%%%%%%%%%%%%%
\subsection{Structure of the extreme points}
A consequence of the decomposition of laminations is that the extreme points of $\rho^*$ are represented by indecomposable laminations.
Recall that a point $\alpha$ of a convex set $S$ is \dfn{extreme} if $\alpha$ cannot be written as the convex combination of two distinct members of $S$.

\begin{lemma}\label{extreme points are closed under sublaminations}
Let $\alpha$ be an extreme point of $\rho^*$, and let $\kappa$ be a measured stretch lamination in $\alpha$.
Then any sublamination of $\kappa$ represents a scalar multiple of $\alpha$.
\end{lemma}
\todo{Rephrase to make this more precise when $\zeta$ is an infinite family of parallel leaves and we want to look at the sublamination given by a single leaf.}
\begin{proof}
By replacing $\kappa$ with a proper sublamination if necessary, we may assume that $\kappa$ is not a foliation.
Let $\zeta$ be a sublamination of $\kappa$.
By Theorem \ref{MorganShelan} and the fact that the leaves of a parallel family of closed leaves are all homologous, after replacing $\zeta$ with a sublamination of $\zeta$, we may assume that $\zeta$ is a clopen parallel family of closed leaves, or is an exceptional indecomposable sublamination of $\kappa$.
Since $\kappa$ is the linear combination of finitely many such clopen sublaminations, we may write $\alpha$ as a convex combination of $\beta_1, \dots, \beta_k$ where the $\beta_i$ are the (normalized to mass $1$) homology classes of clopen sublaminations of $\lambda$.
But $\beta_i \in \rho^*$, so $\beta_i = \alpha$, hence $[\zeta] = \alpha$.
\end{proof}

\begin{proposition}\label{extreme points are indecomposable}
Let $\alpha$ be an extreme point of $\rho^*$.
Then there is a indecomposable measured stretch lamination $\kappa$ representing $\alpha$.
\end{proposition}
\begin{proof}
\todo{Fix the notation to match the statement} By Proposition \ref{enough measures in canonical lamination}, there exists a measured stretch lamination $\kappa$ representing $\alpha$.
By Theorem \ref{MorganShelan}, $\kappa$ has an indecomposable sublamination $\zeta$.
By Lemma \ref{extreme points are closed under sublaminations}, possibly after rescaling the transverse measure, $\zeta$ is a representative of $\alpha$.
Since any tight representative $F$ of $\rho$ calibrates $\kappa$, $F$ also calibrates $\zeta$, so by Proposition \ref{calibrated means measured stretch}, $\zeta$ is a measured stretch sublamination of $\lambda_\rho$.
\end{proof}

% %%%%%%%%%%%%%%%%%%%%%%%%
% \subsection{Rational points and chain-recurrence}
% \todo{Show that $\lambda_\rho$ has a chain-recurrent sublamination}

% The map $\rho \mapsto \lambda_\rho$ is upper semicontinuous.
% More precisely we have:

% \begin{proposition}
% Let $(\rho_n)$ be a sequence in the costable unit sphere, $\rho_n \to \rho$, and assume that $\lambda_{\rho_n} \to \lambda$ in Thurston's geometric topology.
% Then 
% $$\lambda \subseteq \lambda_\rho.$$
% \end{proposition}
% \begin{proof}
% Let $F_n$ be best comass representatives of $\rho_n$.
% By Alaoglu's theorem, after passing to a subsequence, $F_n \weakto F$ in $L^p$ for any $1 < p < \infty$, where
% $$\|F\|_{L^\infty} \leq \lim_{p \to \infty} \lim_{n \to \infty} \|F_n\|_{L^p} \leq \lim_{p \to \infty} \vol(M)^{1/p} \lim_{n \to \infty} \|F_n\|_{L^\infty} = 1.$$
% Testing against smooth $1$-forms, we see that $F$ is a representative of $\rho$, and in particular $\|F\|_{L^\infty} \geq 1$.

% By (\ref{curvature bound for calibrated}) and \cite[Theorem C]{BackusCML}, we may assume that $\lambda_{\rho_n} \to \lambda$ in the H\"older and tangentially $C^\infty$ flow box topology.
% This in particular implies that if $\sigma$ is a small ball in a leaf of $\lambda$, then $\sigma$ is the H\"older limit of small open sets $\sigma_n$ of leaves of $\lambda_{\rho_n}$.
% By \todo{$L^\infty$ Poincar\'e lemma with continuity!}, we can find H\"older $d - 2$-forms $A_n, A$ defined near $\sigma$ such that $A_n \to A$ uniformly and $\dif A_n = F_n$, $\dif A = F$.
% Then we have 
% $$\int_\sigma F = \int_{\partial \sigma} A = \lim_{n \to \infty} \int_{\partial \sigma_n} A_n = \lim_{n \to \infty} \int_{\sigma_n} F_n = \lim_{n \to \infty} \Mass(\sigma_n) = \Mass(\sigma).$$
% Therefore $\lambda$ is an $F$-calibrated lamination.


% \end{proof}


%%%%%%%%%%%%%%%%%%%%%%%%
\section{Convexity of the stable unit ball}\label{convexity sec}
In the research announcement \cite{Auer01}, Auer and Bangert claimed certain results concerning the convex structure of the stable unit ball
$$B := \{\alpha \in H_{d - 1}(M, \RR): \Mass(\alpha) \leq 1\},$$
and a partial proof was given in their unfinished manuscript \cite{Auer12}.
The main difficulty of the proof is to show that if there is a line segment $\ell \subset \partial B$, then the endpoints of $\ell$ can be represented by laminations which embed in a common lamination $\lambda$.
In fact, we can take $\lambda$ to be the canonical lamination, so we now are able to give a short proof of \cite[Theorems 6 and 7]{Auer01}.

%%%%%%%%%%%%%%%%%%%%%%%%
\subsection{Intersection theory}
We are almost ready to establish \cite[Theorem 6]{Auer01}.
To formulate it, recall that a \dfn{flat} $S \subset \partial B$ in the stable unit sphere is a set such that, for some supporting hyperplane $H$ of $B$, $S = H \cap B$.
Thus $B$ is strictly convex iff every flat is a point.
We also recall the \dfn{intersection product} on homology: for any $\alpha, \beta \in H_\bullet(M, \RR)$,
$$\alpha \cdot \beta := \PD(\PD(\alpha) \wedge \PD(\beta)).$$
By identifying $H_0(M, \RR)$ with $\RR$, we recover the intersection number when $\alpha \cdot \beta \in H_0(M, \RR)$.

\begin{proposition}\label{flats are nonintersecting}
Let $B$ be the stable unit ball of $H_{d - 1}(M, \RR)$.
Suppose that $S \subset \partial B$ is a flat, and $\alpha, \beta \in S$.
Then $\alpha \cdot \beta = 0$.
\end{proposition}

\begin{lemma} \label{flats have dual classes}
Suppose that $S \subset \partial B$ is a flat of the stable unit ball $B \subset H_{d - 1}(M, \RR)$.
Then there exists $\rho$ in the costable unit sphere of $H^{d - 1}(M, \RR)$ such that $S \subseteq \rho^*$.
\end{lemma}
\begin{proof}
Since $S$ is convex, $\partial S$ is topologically a sphere, so $\partial S$ admits a Borel probability measure $\nu$ of full support.
Then we take the vector-valued integral 
$$\beta := \int_{\partial S} \alpha \dif \nu(\alpha),$$
thus $\beta \in S$ by convexity.
By the Hanh-Banach theorem, there exists $\rho \in H^{d - 1}(M, \RR)$ such that $\beta \in \rho^*$.

We claim that $\partial S \subseteq \rho^*$.
If not, then by continuity of $\alpha \mapsto \langle \rho, \alpha\rangle$, there is a positive measure set of $\partial S$ on which $\langle \rho, \cdot\rangle < 1$, hence
$$\Comass(\rho) = \langle \rho, \beta\rangle = \int_{\partial S} \langle \rho, \alpha\rangle \dif \nu(\alpha) < \Comass(\rho),$$
a contradiction.
Since $\rho^*$ is convex, it follows that $S \subseteq \rho^*$.
\end{proof}

\begin{proof}[Proof of Proposition \ref{flats are nonintersecting}]
By Lemma \ref{flats have dual classes}, we may assume that for some $\rho$ in the costable unit sphere, $S = \rho^*$.
By Proposition \ref{enough measures in canonical lamination}, there exist measured stretch sublaminations $\kappa_\alpha, \kappa_\beta$ of $\lambda_\rho$, of classes $\alpha, \beta$.
Let $\dif u_\alpha, \dif u_\beta$ be their Ruelle-Sullivan currents, and suppose that $x$ is in the union of their supports.
If $N$ denotes the leaf of $\lambda_\rho$ containing $x$, then for $\sigma = \alpha, \beta$,
$$\dif u_\sigma(x) = \normal_N^\flat(x) \mu_\sigma(x)$$
where $\mu_\sigma$ is the positive Radon measure induced on $M$ by the transverse measure to $\kappa_\sigma$ \cite[Lemma 3.1]{BackusCML}.
In particular, $\dif u_\alpha|_{\supp \dif u_\beta}$ is a (possibly distributional) scalar field times $\dif u_\beta$, so $\dif u_\alpha \wedge \dif u_\beta = 0$, hence $\alpha \cdot \beta = 0$.
\end{proof}

\begin{corollary}
If $M$ is homotopic to a torus, then the stable unit ball of $H_{d - 1}(M, \RR)$ is strictly convex.
\end{corollary}
\begin{proof}
The cohomology ring of $M$ is isomorphic to the exterior algebra of $\RR^d$.
So if $\alpha, \beta \in H_{d - 1}(M, \RR)$ satisfy $\alpha \cdot \beta = 0$, then $\alpha, \beta$ are linearly dependent and cannot both be in the stable unit sphere.
So by Proposition \ref{flats are nonintersecting}, every flat of $B$ is a point.
\end{proof}

\todo{Check this section to make sure we didn't miss the case that $M$ is a foliation with a dense leaf.}

%%%%%%%%%%%%%%%%%%%%%%
\subsection{The universal abelian cover}
We now establish \cite[Theorem 7]{Auer01}.
To formulate it, if $\hat M \to M$ is a Galois covering space, let $\Gal(\hat M, M)$ be the Galois group of deck transformations of $\hat M \to M$.
Thus $\Gamma = \Gal(\tilde M, M)$.
Since the commutator subgroup $[\Gamma, \Gamma]$ is a normal subgroup, it corresponds to a Galois cover, the \dfn{universal abelian covering space} $\tilde M \to M$, and
$$\Gal(\tilde M, \tilde M^{\rm ab}) = [\Gamma, \Gamma].$$
In particular,
$$\Gal(\tilde M^{\rm ab}, M) = \frac{\Gamma}{[\Gamma, \Gamma]} = H_1(M, \RR).$$

\begin{proposition}\label{abelian cover condition}
Suppose that $H^1(\tilde M^{\rm ab}, \RR) = 0$. 
Then the stable unit ball of $H_{d - 1}(M, \RR)$ is strictly convex.
\end{proposition}

\begin{lemma}[reverse isoperimetric inequality]
Suppose that $U$ is a Riemannian domain of bounded curvature and $E \subseteq U$ is a Borel set such that $1_E$ has least gradient in $U$.
Suppose that $0 < r \ll \|\Riem_U\|_{C^0}^{-1/2}$ and $\dist(x, \partial U) > r$.
Then, if $c_d$ denotes the isoperimetric constant on $\RR^d$,
\begin{equation}\label{reverse isoperimetric inequality}
\vol(E \cap B(x, r)) \geq \frac{r^d}{4dc_d}.
\end{equation}
\end{lemma}
\begin{proof}
We first observe that since $r$ is much smaller than the curvature scale, we can use the euclidean isoperimeteric inequality 
$$\vol(\partial(E \cap B(x, \rho))) \geq \frac{1}{2c_d} \vol(E \cap B(x, \rho))^{\frac{d - 1}{d}}.$$
Reasoning as in \cite[Proposition 5.14]{Giusti77}, we estimate for almost every $0 < \rho < r$ that
$$\frac{\dif}{\dif \rho} \vol(E \cap B(x, \rho)) \geq \frac{1}{2} \vol(\partial(E \cap B(x, \rho))) \geq \frac{1}{4c_d} \vol(E \cap B(x, \rho))^{\frac{d - 1}{d}}.$$
The result follows from Gr\"onwall's lemma.
\end{proof}

\begin{lemma}\label{unbounded implies infinite measure}
Suppose that $M$ be a complete Riemannian manifold of bounded curvature, and $E \subseteq U$ is a Borel set such that $1_E$ has least gradient in $M$.
If $E$ is unbounded, then $\vol(E) = \infty$.
\end{lemma}
\begin{proof}
Choose a sequence $(x_n) \subset E$ such that $\dist(x_n, x_m) \geq 1$ if $n \neq m$.
Since $M$ is of bounded curvature, we can find $0 < r < 1/2$ such that $r \ll \|\Riem_M\|_{C^0}^{-1/2}$.
Then $B(x_n, r) \cap B(x_m, r)$ is empty if $n \neq m$, and since $\partial M$ is empty, $\dist(x_n, \partial M) > r$ trivially.
So by the reverse isoperimetric inequality (\ref{reverse isoperimetric inequality}),
\begin{align*}
\vol(E) &\geq \sum_{n=1}^\infty \vol(E \cap B(x_n, r)) \geq \sum_{n=1}^\infty \frac{r^d}{4dc_d} = \infty. \qedhere 
\end{align*}
\end{proof}

Recall that since $\RR$ is abelian, we have natural isomorphisms
$$\Hom(\Gal(\tilde M, M), \RR) = \Hom(\Gamma, \RR) = \Hom\left(\frac{\Gamma}{[\Gamma, \Gamma]}, \RR\right) = \Hom(\Gal(\tilde M^{\rm ab}, M), \RR).$$
Therefore if $\alpha \in \Hom(\Gamma, \RR)$ and $u$ is an $\alpha$-equivariant function on $\tilde M$, we obtain an $\alpha$-equivariant function $u^{\rm ab}$ on $\tilde M^{\rm ab}$.

The next two lemmata appeared in the draft of Auer and Bangert \cite{Auer12}.
Since this draft is not publicly available, or complete, we reproduce them here with full credit to the original authors.

\begin{lemma}\label{superlevel sets are connected}
Let $u$ be an $\alpha$-equivariant function of least gradient on $\tilde M$.
Then the set $\{u^{\rm ab} > t\}$ is connected.
\end{lemma}
\begin{proof}
We may assume that $\alpha$ is nonzero.
Otherwise, $u$ drops to a function of least gradient on $M$, which is constant since $M$ is closed, and the claim is proven.

Let $F$ be the image of a fundamental domain of $M$ in $\tilde M^{\rm ab}$.
Since $u^{\rm ab} \in L^\infty_\loc$ \cite[Theorem 4.3]{Gorny20} and $F$ is compact, there exists $t_0 \in \RR$ such that $u > t_0$ on $F$.
Let 
$$H := \bigcup_{\substack{\rho \in H_1(M, \ZZ) \\ \langle \alpha, \rho\rangle > t - t_0}} \rho(F).$$
If $x \in H$, then for some $\rho \in H_1(M, \ZZ)$ and $y \in F$, $x = \rho(y)$, and then 
$$u^{\rm ab}(x) = u^{\rm ab}(y) + \langle \alpha, \rho\rangle > t_0 + t - t_0 = t$$
so $H \subseteq \{u^{\rm ab} > y\}$.
Since $H$ is the set of translations of the connected fundamental domain $F$ by the intersection of the lattice of deck transformations and a half-space, $H$ is connected.

Suppose that $\{u^{\rm ab} > t\}$ is disconnected, so it has a component $X$ which does not meet $H$.
For any $\rho \in H_1(M, \ZZ)$ such that $\langle \alpha, \rho\rangle > 0$, $\rho$ sends $\{u^{\rm ab} > y\}$ into itself, hence sends $X$ into a component of $\{u^{\rm ab} > t\}$.
If $\rho(X) \subseteq X$, then we choose $x \in X$ and $\theta \in H_1(M, \ZZ)$ such that $\theta(x) \in F$; then, for $m$ large, 
$$\langle \alpha, m\rho - \theta\rangle > t - t_0,$$
and it follows that $\rho^m(X)$ meets $H$, so $X$ meets $H$.

Therefore we may assume that for every $\rho \in H_1(M, \ZZ)$ such that $\langle \alpha, \rho\rangle \neq 0$, $\rho$ sends $X$ into another component of $\{u^{\rm ab} > t\}$.
Let $\hat M$ be the minimal covering space on which $u$ drops to a function $\hat u: \hat M \to \RR$, thus $\Gal(\hat M, M) = \ker(\alpha)$.
Then $\hat u$ has least gradient, and $X$ drops to a component $\hat X$ of $\{\hat u > y\}$.
In particular, $1_{\{\hat u > y\}}$ has least gradient \cite[Theorem 1]{BOMBIERI1969} \todo{define least gradient when not equivariant}.
Therefore $v := 1_{\hat X}$ also has least gradient.
So $\dif v$ cannot have compact support, and hence $\hat X$ is unbounded.
By Lemma \ref{unbounded implies infinite measure}, it follows that $\|v\|_{L^1} = \infty$.
But by our assumption, and the fact that $\Gal(\hat M, M) = \ker(\alpha)$, the projection map $\hat M \to M$ restricts to an injective map on $\hat X$, so $\|v\|_{L^1} < \infty$, a contradiction.
\end{proof}

\begin{lemma}\label{abelian cover connected}
Let $u$ be an $\alpha$-equivariant function of least gradient on $\tilde M$, and let $\mathscr G$ be a set of curves in $\tilde M^{\rm ab}$ which spans $H_1(M^{\rm ab}, \RR)$.
If $\partial \{u^{\rm ab} > t\}$ misses every curve in $\mathscr G$, then $\partial \{u^{\rm ab} > t\}$ is connected.
\end{lemma}
\begin{proof}
We reason by contrapositive.
Let $N_1, N_2$ be two distinct components of $\partial \{u^{\rm ab} > t\}$.
By Lemma \ref{superlevel sets are connected} (and the analogous result for sublevel sets), $\tilde M^{\rm ab} \setminus \partial \{u^{\rm ab} > t\}$ has two components $E_1, E_2$.
We construct a curve $\gamma$, transverse to $N_1$, which starts at a point $x \in N_1$, passes through $E_1$, crosses $N_2$ into $E_2$, and then returns to $x$.
In particular $\gamma$ meets $N_1$ at a single point, so their intersection number $[\gamma] \cdot [N_1] = 1$ (possibly after reorienting).
Therefore $[\gamma]$ is a nontrivial class in $H_1(M^{\rm ab}, \RR)$.
\end{proof}

\begin{proof}[Proof of Proposition \ref{abelian cover condition}]
We prove the contrapositive.
Let $B$ be the stable unit ball.
By Lemma \ref{flats have dual classes}, if $B$ is not strictly convex, then there exists $\rho$ in the costable unit ball, such that $\rho^*$ is not singleton.
By the Krein-Milman theorem, there are two distinct extreme points $\alpha, \beta$ of $\rho^*$, and by Proposition \ref{extreme points are indecomposable}, we can find distinct indecomposable measured stretch laminations $\kappa_\alpha, \kappa_\beta \subset \lambda_\rho$.
Let $u_\alpha, u_\beta$ be primitives of the Ruelle-Sullivan currents on $\tilde M$; by equivariance, they drop to functions $u^{\rm ab}_\alpha, u^{\rm ab}_\beta$ on the universal abelian cover $\tilde M^{\rm ab}$.

We first suppose that $\kappa_\alpha$ is a closed hypersurface.
Then $\kappa_\alpha$ and $\kappa_\beta$ must have disjoint supports, for otherwise $\kappa_\beta$ is the same closed hypersurface, also with unit mass, so $\alpha = \beta$, a contradiction.
In that case, after adding constants to $u_\alpha, u_\beta$, we may assume that $\partial \{u_\alpha > 0\}$ drops to $\kappa_\alpha$ and $\partial \{u_\beta > 0\}$ drops to a leaf of $\kappa_\beta$.

Otherwise, by Theorem \ref{MorganShelan} and symmetry of the roles of $\alpha, \beta$, we may assume that $\kappa_\alpha, \kappa_\beta$ are both exceptional.
Therefore they have infinitely many leaves, and so we may choose leaves $N_\alpha \subset \kappa_\alpha$, $N_\beta \subset \kappa_\beta$ which are distinct.
We then may add constants to $u_\alpha, u_\beta$ so that $\partial \{u_\alpha > 0\}$ drops to $N_\alpha$ and similarly for $\beta$.

In either case, we have shown that we may assume that $\partial \{u_\alpha > 0\}$ and $\partial \{u_\beta > 0\}$ drop to different leaves of $\lambda_\rho$.
Therefore they define different leaves of the covering lamination $\tilde \lambda^{\rm ab}_\rho$.
Since every leaf of $\tilde \lambda^{\rm ab}_\rho$ is a closed set since they are level sets of a function of least gradient, $\partial \{u_\alpha > 0\}$ and $\partial \{u_\beta > 0\}$ are disjoint and separated by open sets.

But if we set $u := (u_\alpha + u_\beta)/2$, $\langle \rho, \dif u\rangle = 1$, and 
$$\Mass(\dif u) = \Mass(\normal^\flat_\lambda |\dif u|) = \frac{1}{2} \Mass(\normal^\flat_\lambda (|\dif u_\alpha| + |\dif u_\beta|)) = \frac{1}{2} (\Mass(\dif u_\alpha) + \Mass(\dif u_\beta)) = 1.$$
So by Proposition \ref{lams are calibrated}, $u$ is the Ruelle-Sullivan current of a measured stretch lamination $\kappa$ associated to $\rho$.
In particular, $u$ has least gradient.
Moreover,
$$\partial \{u > 0\} = \partial \{u_\alpha > 0\} \cup \partial \{u_\beta > 0\}$$
and since the right hand side is separated by open sets, $\partial \{u > 0\}$ is disconnected. 
So by Proposition \ref{abelian cover connected}, $H_1(\tilde M^{\rm ab}, \RR)$ is nonzero.
\end{proof}


% %%%%%%%%%%%%%%%%%%%%%%%%%
% \section{Worked examples}
\section{The canonical lamination on a surface}
Let $M$ be a closed oriented Riemannian surface and $\rho \in H^1(M, \RR)$ satisfy $\Comass(\rho) = 1$.
Then work of Daskalopoulos and Uhlenbeck \cite{daskalopoulos2020transverse} allows us to say quite a bit more about the canonical lamination.
Essentially this is because we have a good theory of viscosity solutions for the $\infty$-Laplacian, and the derivative of an $\infty$-harmonic function is a tight $1$-form.

%%%%%%%%%%%%%%
\subsection{The \texorpdfstring{$\infty$-Laplacian}{infinity-Laplacian}}
We shall work quite generally, with $d \geq 2$.
The $\infty$-Laplacian is the PDE 
$$\langle \nabla^2 u, \nabla u \otimes \nabla u\rangle = 0.$$
We refer the reader to Crandall \cite{Crandall2008} for the general theory of the $\infty$-Laplacian.
In our application, the key fact is that for every $\rho \in H^1(M, \RR)$, there is a $\rho$-equivariant $\infty$-harmonic function $\psi$ on the universal cover $\tilde M$. This is a straightforward modification of \cite[Theorem 2.4]{daskalopoulos2020transverse}.
Moreover, $\MCL(\dif \psi)$ is a geodesic lamination:

\begin{proposition}[{\cite[[Theorem 5.2]{daskalopoulos2020transverse}]}]
Let $\rho \in H^1(M, \RR)$, and let $\psi$ be a $\rho$-equivariant $\infty$-harmonic function.
Then $\MCL(\dif \psi)$ is a geodesic lamination, and the restriction of $\psi$ to each leaf of $\MCL(\dif \psi)$ is an affine function whose Lipschitz constant is $\Lip(\psi, M)$.
\end{proposition}

\begin{proposition}\label{infinity laplacian is canonical lamination}
Let $\rho \in H^1(M, \RR)$, and let $\psi$ be a $\rho$-equivariant $\infty$-harmonic function.
Then for every best comass $1$-form $F$ representing $\rho$,
$$\MCL(\dif \psi) \subseteq \lambda_F.$$
\end{proposition}
\begin{proof}
\todo{George made this sound obvious but I don't think it is}
\end{proof}

\begin{corollary}
Suppose that $d = 2$, let $\rho \in H^1(M, \RR)$ satisfy $\Comass(\rho) = 1$, and let $\psi$ be a $\rho$-equivariant $\infty$-harmonic function.
Then $\MCL(\dif \psi)$ is the canonical lamination $\lambda_\rho$.
\end{corollary}

Let $M$ be a closed oriented Riemannian manifold of dimension $d \geq 3$.
The definition of costable norm $\Comass(\rho)$ of a class $\rho \in H^1(M, \RR)$ still makes sense in this setting, and it is the Lipschitz constant of any $\rho$-equivariant $\infty$-harmonic function.
Since Proposition \ref{infinity laplacian is canonical lamination} did not use the fact that $d = 2$, we can define a canonical geodesic lamination calibrated by a cohomology class $\rho$ such that $\Comass(\rho) = 1$: it is simply $\MCL(\dif \psi)$ where $\psi$ is any $\rho$-equivariant $\infty$-harmonic function.

%%%%%%%%%%%%%
\subsection{Chain recurrence}
\todo{Can we show that the canonical lamination is chain recurrent when $d = 2$?}

%%%%%%%%%%%%%
\subsection{Massart's theorem on negatively curved surfaces}
Part of a problem of Daskalopoulos and Uhlenbeck \cite[Conjecture 9.3]{daskalopoulos2020transverse} asks one to show that, given an $\rho$-equivariant $\infty$-harmonic function $u$ on a closed hyperbolic surface, there is a unique cohomology class $\alpha$ such tht an $\alpha$-equivariant least gradient function $v$ induces a transverse measure on a sublamination of $\MCL(\dif u)$.
We now falsify this conjecture using Massart's theorem on the stable norm ball of a negatively curved surface:

\begin{theorem}[{\cite{Massart1997StableNO}}]\label{Massart}
Let $M$ be a closed Riemannian surface of genus $\zeta \geq 2$ and stable norm ball $B$.
Then there exists a flat of dimension $\zeta - 1$ in $\partial B$.
\end{theorem}

\begin{corollary}
Let $M$ be a closed Riemannian surface of genus $\zeta \geq 2$.
Then there is an equivariant $\infty$-harmonic function $\psi$ on the universal cover $\tilde M$, such that there are multiple, noncohomologous functions of least gradient $u$ such that $\dif u$ is a Ruelle-Sullivan current for a sublamination of $\MCL(\dif \psi)$.
\end{corollary}
\begin{proof}
By Theorem \ref{Massart} and the Hanh-Banach theorem, there exists $\rho \in H^1(M, \RR)$ such that $\Comass(\rho) = 1$ and $\dim \rho^* \geq \zeta - 1$.
Let $\psi$ be a $\rho$-equivariant $\infty$-harmonic function.
By Proposition \ref{enough measures in canonical lamination}, there is an equivariant least gradient function $u$ such that $\dif u$ is the Ruelle-Sullivan current of measured stretch lamination $\kappa$ associated to $\rho$; in particular, $\kappa$ is a sublamination of $\MCL(\dif \psi)$.
\end{proof}

% %%%%%%%%%%%%%%%%
% \subsection{Fiber bundles over \texorpdfstring{$\Sph^1$}{the circle}}
% As a demonstration that in practice it is sometimes possible to explicitly compute the canonical lamination, let us do so when $M$ is a locally warped product with $\Sph^1$.

% Let $(N, h)$ be a closed Riemannian manifold of dimension $d - 1$, and assume that we have an oriented fiber bundle $N_x \to M \to \Sph^1_\theta$ with a warped product metric 
% $$g = \dif \theta^2 + f(\theta)^2 h.$$
% Let $\theta_0$ be a minimizer of $f$, and let $F := \sqrt{\det h} \dif x$ denote the area form on $(N, h)$.
% By rescaling $h$ and $f$, we may assume that $f(\theta_0) = 1$ and $\vol(N, h) = 1$.
% In particular, the fiber $N_\theta$ of $\theta$ has area
% $$\Mass(N_\theta) = \int_N f(\theta)^{d - 1} F = f(\theta)^{d - 1}.$$
% Since $h$ is a function of $x$ but not $\theta$, $\dif F = 0$.
% Moreover,
% \begin{equation}\label{fiber bundle sample comass}
% |F|(\theta, x) = \frac{1}{f(\theta)^{d - 1}}
% \end{equation}
% which attains its maximum $1$ on $N_{\theta_0}$, so $F$ is a calibration, and $N_{\theta_0}$ is $F$-calibrated.
% In particular, the cohomology class $\rho := [F]$ has costable norm $1$.

% \begin{proposition}
% With notation as above, $F$ is tight, the leaves of $\lambda_\rho$ are exactly the fibers $N_\theta$ such that $f(\theta) = 1$, and $\rho^*$ is the convex hull of homology classes $[N_\theta]$ such that $f(\theta) = 1$.
% \end{proposition}
% \begin{proof}
% We first compute
% $$\star F = \frac{\dif \theta}{f(\theta)^{d - 1}}$$
% and combine this with (\ref{fiber bundle sample comass}) to get
% $$\dif(|F|^{p - 2} \star F) = \dif\left(\frac{\dif \theta}{f(\theta)^{(p - 1)(d - 1)}}\right) = 0$$
% so $F$ is $p$-tight for every $1 < p < \infty$ and hence is tight.
% Since $F$ is the area form for a complete hypersurface $L \subset M$ iff $L = N_\theta$ for some $\theta$ such that $f(\theta) = 1$, we deduce that $\lambda_\rho$ is a sublamination of the lamination $\kappa$ whose leaf set is $\{N_\theta: f(\theta) = 1\}$.
% On the other hand, $\kappa$ admits a transverse probability measure $\mu$, because the set $\{f = 1\}$ is a closed subset of $\Sph^1$ and therefore is the support of some Borel probability measure.
% So $(\kappa, \mu)$ is a measured stretch lamination for $\rho$, and therefore $\kappa$ is a sublamination of $\lambda_\rho$.
% Therefore $\lambda_\rho = \kappa$.

% We next compute $\rho^*$.
% Let $\kappa$ be a measured stretch lamination for $\rho$; then every leaf of $\kappa$ is $N_\theta$ for some $\theta$ such that $f(\theta) = 1$; for such a $\theta$, the stable norm is $\Mass([N_\theta]) = 1$.
% Therefore $[\kappa]$ is a convex combination of the classes $[N_\theta]$.
% Since $N_\theta$ is itself a measured stretch lamination for $\rho$ if $f(\theta) = 1$, $\rho^*$ is the convex hull of $\{[N_\theta]: f(\theta) = 1\}$.
% In particular, if the stable unit ball is strictly convex (so $\rho^*$ is singleton), any two fibers $N_\theta$ with $f(\theta) = 1$ must be homologous.
% \end{proof}

% \todo{Who cares?}


%%%%%%%%%%%%%%%%%%%%%%%%%
\section{Concluding remarks}\label{open problems}
% \subsection{Generalizations}
% In this paper I have only dealt with closed Riemannian manifolds $M$ of dimension $d \leq 7$, and submanifolds of codimension $c := 1$.
% In the prequel paper \cite{BackusCML} I have only studied the interior behavior of functions of least gradient, but moreover, the statements of the main theorems would be significantly more involved in a more general setting.

% We cannot easily weaken the assumption $d \leq 7$, since the double-napped Simons cone defines a function of least gradient on $\RR^8$ which does not admit a lamination structure \cite{BackusCML}.
% Similarly, if the codimension $c \geq 2$, then Liu recently constructed homologically minimizing submanifolds $N$ which do not admit calibrations, even if $d = 3$ or $M = \CC \PP^2$ with a perturbation of the Fubini-Study metric \cite{liu2023homologically}.
% Liu also showed that if $d \geq 8$ and $c = 1$, then there exist smooth homologically minimizing hypersurfaces with no smooth calibration; the regularity of best comass calibrations remains open if $d \leq 7$.

% After replacing Poincar\'e duality with Lefschetz duality and imposing boundary conditions, I expect that the results of this paper go through if $M$ is a compact manifold with strictly mean-convex boundary $\partial M$.
% Under that assumption, a version of the max-flow min-cut theorem has been claimed in the physics literature \cite[Appendix A]{Freedman_2016}.
% To see that convexity is necessary, let $\theta$ be the latitude on $\Sph^2$, and let $M := \{|\theta| \leq \pi/4\}$; then any function of least gradient which extends the boundary data 
% $$h(\theta, \phi) := \begin{cases} 1, \text{ if } \theta = \pi/4 \\ 0, \text{ if } \theta = -\pi/4\end{cases}$$
% is constant on the interior, hence does not induce a lamination.

% By replacing the boundary components of $M = \{|\theta| \leq \pi/4\}$ by cusps, we see that if $M$ has infinite ends, then it is possible for $H_{d - 1}(M, \RR) \neq 0$ but the stable seminorm to be identically $0$.
% Freedman and Headrick conjectured that a max-flow min-cut theorem should hold for a manifold with infinite ends $M$ if there exist compact manifolds $M_i$ with mean-convex boundary, such that $(M_i)$ is a compact exhaustion of $M$ \cite[Appendix A]{Freedman_2016}.
% This assumption clearly rules out the existence of cusps.

% However, even assuming the existence of a compact exhaustion with mean-convex boundaries, I expect that extension of this paper to noncompact manifolds to be quite challenging.
% If $H^{d - 1}(M, \RR)$ is infinite-dimensional, then its completion with respect to the costable norm is unlikely to be reflexive, so arguments involving the Hanh-Banach theorem will become precarious.
% Moreover, already in the case of $\RR^d$, the behavior of functions of least gradient near infinity is rather involved \cite[\S4.4]{górny2021}.

\subsection{Motivation: Analogy with Thurston}\label{Teichmuller} \todo{Reword to focus on Thurston metric}
Our motivation for introducing the canonical lamination arose from an analogy with Thurston's approach to Teichm\"uller theory using best Lipschitz maps \cite{Thurston98}.
Given $\gamma \geq 2$, let $\widetilde{\mathscr M}_\gamma$ be the Teichm\"uller space of hyperbolic metrics on the closed surface $S_\gamma$ of genus $\gamma$.
Given $g, h \in \widetilde{\mathscr M}_\gamma$, let $\Lip(g, h)$ be the Lipschitz constant of a best Lipschitz map homotopic to
$$\id: (S_\gamma, g) \to (S_\gamma, h).$$ 
For a tangent vector $v \in T_g(\widetilde{\mathscr M}_\gamma)$, let $\Comass(v)$ be the partial derivative of $\log \Lip(g, \cdot)$ in the direction $v$.
This quantity, the \dfn{Thurston asymmetric norm}, is an asymmetric norm on $T_g(\widetilde{\mathscr M}_\gamma)$ obtained by solving an $L^\infty$ variational problem intimately tied to the structure of minimal laminations, so it is tempted to make an analogy between $T_g(\widetilde{\mathscr M}_\gamma)$ and $H^{d - 1}(M, \RR)$, where both vector spaces are equipped with the norm $\Comass$.
Two particularly salient pieces of evidence for the analogy are:
\begin{enumerate}
\item The unit spheres of the dual spaces of $T_g(\widetilde{\mathscr M}_\gamma)$ and $H^{d - 1}(M, \RR)$ can both be viewed as spaces of projective measured minimal laminations, whose norm is given by an $L^1$ (actually $BV$) variational problem \cite[Theorem 5.1]{Thurston98}.
\item In both cases, we can construct a canonical lamination. In Thurston's case, the canonical lamination is given by those geodesics which are maximally stretched by every best Lipschitz map homotopic to $\id_{S_\gamma}$ \cite[\S8]{Thurston98}. See also Conjecture \ref{chain recurrence}.
\end{enumerate}
However, one should not take this analogy too seriously.
A key feature of Thurston's theory is the Birman-Series theorem: the union of the supports of all geodesic laminations on $(S_\gamma, g)$ has Hausdorff dimension $0$.
As a corollary, for almost every $h \in \widetilde{\mathscr M}_\gamma$, the canonical lamination associated to $(g, h)$ is a closed geodesic \cite[\S10]{Thurston98}.
The analogue of the Birman-Series theorem is clearly not true in our case, and in fact, if $M$ is a square flat torus, then it is easy to see that every canonical lamination covers all of $M$.

Thurston's canonical lamination $\lambda$ is chain-recurrent, in the sense that traveling along the geodesics in $\lambda$ defines a chain-recurrent dynamical system.
This makes no sense for higher-dimensional laminations, but is equivalent to assert that Thurston's canonical lamination can be approximated by finite sums of closed geodesics \cite[\S9]{Gu_ritaud_2017}.
We conjecture that the analogous fact should hold for our canonical lamination:

\begin{conjecture}\label{chain recurrence}
Let $\rho \in H^{d - 1}(M, \RR)$, and let $\lambda_\rho$ be the canonical lamination.
Then it is possible to approximate $\lambda_\rho$ in Thurston's geometric topology\footnote{See \cite[\S1]{BackusCML} for the definition of Thurston's geometric topology in this setting.} by finite unions of closed minimal hypersurfaces.
\end{conjecture}

%%%%%%%%%%%%%%%%%%%
\subsection{Taut foliations and eikonal calibrations}
The following problem was suggested to me by Karen Uhlenbeck. 
There exist closed hyperbolic $3$-manifolds which admit taut foliations; in that case, \emph{after changing the metric} one may find a minimal foliation.
Thus one cannot rule out minimal foliations by a simple topological argument (as one could rule out geodesic foliations of closed hyperbolic surfaces).
However, if a minimal foliation exists, then it is natural to study the tight form which calibrates it.
This form satisfies a particularly strong form 
\begin{equation}\label{eikonal}
\begin{cases}\dif F = 0 \\ \dif(|F|^2) = 0\end{cases}
\end{equation}
of the Euler-Lagrange equation for tight forms which is analogous to the role of the eikonal equation
$$\dif(|\dif u|^2) = 0$$
in the study of the $\infty$-Laplace equation.
Global solutions of the eikonal equation are rather uncommon (for example, the Dirichlet problem for the eikonal equation on $\Ball^d$ is overdetermined), so this suggests a means to rule out the existence of minimal foliations:

\begin{conjecture}\label{Karen}
Let $\Gamma$ be the fundamental group of a closed hyperbolic $3$-manifold $M$.
Then there does not exist a solution of the eikonal system (\ref{eikonal}) on $\Hyp^3$ which is invariant under $\Gamma$.
In particular, there does not exist a minimal foliation on $M$.
\end{conjecture}

\todo{Conjecture about reduction of continuous MFMC to discrete}




\printbibliography

\end{document}
