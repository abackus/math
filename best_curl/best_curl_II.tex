\documentclass[reqno,11pt]{amsart}
\usepackage[letterpaper, margin=1in]{geometry}
\RequirePackage{amsmath,amssymb,amsthm,graphicx,mathrsfs,url,slashed,subcaption}
\RequirePackage[usenames,dvipsnames]{xcolor}
\RequirePackage[colorlinks=true,linkcolor=Red,citecolor=Green]{hyperref}
\RequirePackage{amsxtra}
\usepackage{cancel}
\usepackage{tikz, quiver, wrapfig}

% Add the 2020 MSC
\makeatletter
\@namedef{subjclassname@2020}{\textup{2020} Mathematics Subject Classification}
\makeatother

%\usepackage[T1]{fontenc}

% \setlength{\textheight}{9.3in} \setlength{\oddsidemargin}{-0.25in}
% \setlength{\evensidemargin}{-0.25in} \setlength{\textwidth}{7in}
% \setlength{\topmargin}{-0.25in} \setlength{\headheight}{0.18in}
% \setlength{\marginparwidth}{1.0in}
% \setlength{\abovedisplayskip}{0.2in}
% \setlength{\belowdisplayskip}{0.2in}
% \setlength{\parskip}{0.05in}
%\renewcommand{\baselinestretch}{1.05}

\title{Convex duality between minimal laminations and tight calibrations}
\author{Aidan Backus}
\address{Department of Mathematics, Brown University}
\email{aidan\_backus@brown.edu}
\date{\today}
\keywords{laminations, convex duality, minimal hypersurfaces, max flow/min cut theorem, calibrations, functions of least gradient, stable norm}
\subjclass[2020]{primary: 49Q20; secondary: 39J60, 49N15, 53C38}

\newcommand{\NN}{\mathbf{N}}
\newcommand{\ZZ}{\mathbf{Z}}
\newcommand{\QQ}{\mathbf{Q}}
\newcommand{\RR}{\mathbf{R}}
\newcommand{\CC}{\mathbf{C}}
\newcommand{\DD}{\mathbf{D}}
\newcommand{\PP}{\mathbf P}
\newcommand{\MM}{\mathbf M}
\newcommand{\II}{\mathbf I}
\newcommand{\Hyp}{\mathbf H}
\newcommand{\Sph}{\mathbf S}
\newcommand{\Torus}{\mathbf T}
\newcommand{\Group}{\mathbf G}
\newcommand{\GL}{\mathbf{GL}}
\newcommand{\Orth}{\mathbf{O}}
\newcommand{\SpOrth}{\mathbf{SO}}
\newcommand{\Ball}{\mathbf{B}}

\newcommand*\dif{\mathop{}\!\mathrm{d}}

\DeclareMathOperator{\card}{card}
\DeclareMathOperator{\dist}{dist}
\DeclareMathOperator{\id}{id}
\DeclareMathOperator{\Hom}{Hom}
\DeclareMathOperator{\coker}{coker}
\DeclareMathOperator{\supp}{supp}
\DeclareMathOperator{\Teich}{Teich}
\DeclareMathOperator{\tr}{tr}

\newcommand{\Leaves}{\mathscr L}
\newcommand{\Lagrange}{\mathscr L}
\newcommand{\Hypspace}{\mathscr H}

\newcommand{\Chain}{\underline C}

\newcommand{\Euler}{\mathbf \chi}

\newcommand{\Two}{\mathrm{I\!I}}
\newcommand{\Ric}{\mathrm{Ric}}

\newcommand{\normal}{\mathbf n}
\newcommand{\radial}{\mathbf r}
\newcommand{\evect}{\mathbf e}
\newcommand{\vol}{\mathrm{vol}}

\newcommand{\diam}{\mathrm{diam}}
\DeclareMathOperator{\Gal}{Gal}
\DeclareMathOperator{\sech}{sech}
\newcommand{\Ell}{\mathrm{Ell}}
\newcommand{\inj}{\mathrm{inj}}
\newcommand{\Lip}{\mathrm{Lip}}
\newcommand{\MCL}{\mathrm{MCL}}
\newcommand{\Riem}{\mathrm{Riem}}

\newcommand{\Mass}{\mathbf M}
\newcommand{\Comass}{\mathbf L}

\newcommand{\Min}{\mathrm{Min}}
\newcommand{\Max}{\mathrm{Max}}

\newcommand{\dfn}[1]{\emph{#1}\index{#1}}

\renewcommand{\Re}{\operatorname{Re}}
\renewcommand{\Im}{\operatorname{Im}}

\newcommand{\loc}{\mathrm{loc}}
\newcommand{\cpt}{\mathrm{cpt}}

\def\Japan#1{\left \langle #1 \right \rangle}

\newtheorem{theorem}{Theorem}[section]
\newtheorem{badtheorem}[theorem]{``Theorem"}
\newtheorem{prop}[theorem]{Proposition}
\newtheorem{lemma}[theorem]{Lemma}
\newtheorem{sublemma}[theorem]{Sublemma}
\newtheorem{proposition}[theorem]{Proposition}
\newtheorem{corollary}[theorem]{Corollary}
\newtheorem{conjecture}[theorem]{Conjecture}
\newtheorem{axiom}[theorem]{Axiom}
\newtheorem{assumption}[theorem]{Assumption}

\newtheorem{mainthm}{Theorem}
\renewcommand{\themainthm}{\Alph{mainthm}}

\newtheorem{claim}{Claim}[theorem]
\renewcommand{\theclaim}{\thetheorem\Alph{claim}}
% \newtheorem*{claim}{Claim}

\theoremstyle{definition}
\newtheorem{definition}[theorem]{Definition}
\newtheorem{remark}[theorem]{Remark}
\newtheorem{example}[theorem]{Example}
\newtheorem{notation}[theorem]{Notation}

\newtheorem{exercise}[theorem]{Discussion topic}
\newtheorem{homework}[theorem]{Homework}
\newtheorem{problem}[theorem]{Problem}

\makeatletter
\newcommand{\proofpart}[2]{%
  \par
  \addvspace{\medskipamount}%
  \noindent\emph{Part #1: #2.}
}
\makeatother



\numberwithin{equation}{section}


% Mean
\def\Xint#1{\mathchoice
{\XXint\displaystyle\textstyle{#1}}%
{\XXint\textstyle\scriptstyle{#1}}%
{\XXint\scriptstyle\scriptscriptstyle{#1}}%
{\XXint\scriptscriptstyle\scriptscriptstyle{#1}}%
\!\int}
\def\XXint#1#2#3{{\setbox0=\hbox{$#1{#2#3}{\int}$ }
\vcenter{\hbox{$#2#3$ }}\kern-.6\wd0}}
\def\ddashint{\Xint=}
\def\dashint{\Xint-}

\usepackage[backend=bibtex,style=alphabetic,giveninits=true]{biblatex}
\renewcommand*{\bibfont}{\normalfont\footnotesize}
\addbibresource{best_curl.bib}
\renewbibmacro{in:}{}
\DeclareFieldFormat{pages}{#1}

\newcommand\todo[1]{\textcolor{red}{TODO: #1}}


\begin{document}
\begin{abstract}
\todo{Absorb from the general paper}
\end{abstract}

\maketitle

%%%%%%%%%%%%%%%%%%%%%%%%%%%%%%%%%%%%%%%%%%%%%%%%%%%%%%%
\section{Introduction}
\todo{Absorb this from the general paper}

%%%%%%%%%%%%%%%%%%%%%%%%%%%%%%%%%%%%%%%%%%
\section{Review of previous papers}\label{prevResults}
We use \cite{BackusCML, BackusBest1} as a reference.
\todo{Summarize the previous papers once they're on the arxiv}


%%%%%%%%%%%%%%%%%%%%%%%%%%%%%%%%%%%%%%%%%%

\section{Preliminaries on laminations}
\subsection{The maximum comass locus}
\begin{definition}
For an open set $\Omega \subseteq M$, let $\Chain_{d - 1}(\Omega)$ be the set of simplicial $d - 1$-chains in $\Omega$.
For each closed $d - 1$-form $F$ on $\Omega$, let
$$\Comass_\Omega(F) := \sup_{\sigma \in \Chain_{d - 1}(\Omega)} \frac{1}{\Mass(\sigma)} \int_\sigma F.$$
The \dfn{local comass} of a closed $d - 1$-form $F$ at $x \in M$ is 
$$\Comass(F, x) = \limsup_{\varepsilon \to 0} \Comass_{B_\varepsilon(x)}(F).$$
\end{definition}

By the trace theorem, $\Comass_\Omega(F)$ is well-defined (but possibly $+\infty$) for any measurable $F$.
Since $\Comass_{B_\varepsilon(x)}(F)$ is a supremum over a set which grows in $\varepsilon$, it is increasing in $\varepsilon$, so the limit superior is actually a limit and an infimum:
$$\Comass(F, x) = \lim_{\varepsilon \to 0} \Comass_{B_\varepsilon(x)}(F) = \inf_{\varepsilon > 0} \Comass_{B_\varepsilon(x)}(F).$$
In particular, if we write $\Comass(F) := \Comass_M(F)$, then $\Comass(F, x) \leq \Comass(F)$.

The local comass was defined in an analogous manner to the local Lipschitz constant.
As such, it enjoys many of the same properties, including those endowed on the local Lipschitz constant by \cite[Lemma 4.3]{Crandall2008}:

\begin{proposition}\label{crandall}
Let $F \in L^\infty(M, \Omega^{d - 1}_{\rm cl})$. Then:
\begin{enumerate}
\item The local comass $\Comass(F, \cdot)$ is upper semicontinuous. \label{crandall usc}
\item For almost every $x \in M$, \label{crandall LDT}
$$|F(x)| \leq \Comass(F, x).$$
\item The local comass is bounded, and \label{crandall linfinity}
$$\Comass(F) = \sup_{x \in M} \Comass(F, x) = \|F\|_{L^\infty}.$$
\item If $\sigma \in \Chain_{d - 1}(M)$ then \label{crandall best curl is ABC}
$$\int_\sigma F \leq \Mass(\sigma) \sup_{x \in \sigma} \Comass(F, x).$$
\end{enumerate}
\end{proposition}
\begin{proof}
We first prove (\ref{crandall usc}).
Let $x^n \to x$ and $r > 0$. Then eventually $x^n \in B_r(x)$, hence $\Comass(F, x^n) \leq \Comass_{B_r(x)}(F)$ and so
\begin{align*}
\limsup_{n \to \infty} \Comass(F, x^n) &\leq \inf_{r > 0} \Comass_{B_r(x)}(F) = \Comass(F, x).
\end{align*}

We now prove (\ref{crandall LDT}).
We may work locally, and choose coordinates $(y^i)$ in which $\sqrt{\det g} = 1$.
Let $I$ be the increasing $d-1$-index with $d$ removed.
By the Lebesgue differentiation theorem and Fubini's theorem, there exists a null set $Z \subset M$, which does not depend on $(y^i)$ by \cite[Proposition 2.1]{BackusFLG}, such that for every $x \notin Z$,
\begin{align*}
F_I(x) 
&= \lim_{\varepsilon \to 0} \frac{1}{\Mass(B_\varepsilon(x))} \int_{B_\varepsilon(x)} F_I(y) \dif y \\
&= \lim_{\varepsilon \to 0} \frac{1}{\Mass(B_\varepsilon(x))} \int_{-\infty}^\infty \int_{\{y^d = t\} \cap B_\varepsilon(x)} F_I(y) \dif y^1 \wedge \cdots \wedge \dif y^{d - 1} \wedge \dif t
\end{align*}
where we used the fact that $\sqrt{\det g} = 1$.
Now $\partial_{y^1} \wedge \cdots \wedge \partial_{y^{d - 1}}$ is the oriented unit $d - 1$-blade tangent to $\{y^d = t\}$, so as forms on $\{y^d = t\}$,
$$F_I(y) \dif y^1 \wedge \cdots \wedge \dif y^{d - 1} = F.$$
So
\begin{align*}
F_I(x) 
&= \lim_{\varepsilon \to 0} \frac{1}{\Mass(B_\varepsilon(x))} \int_{-\infty}^\infty \int_{\{y^d = t\} \cap B_\varepsilon(x)} F \dif t \\
&\leq \lim_{\varepsilon \to 0} \frac{\Comass_{B_\varepsilon(x)}(F)}{\Mass(B_\varepsilon(x))} \int_{-\infty}^\infty |\{y^d = t\} \cap B_\varepsilon(x)| \dif t.
\end{align*}
By Fubini's theorem,
$$F_I(x) \leq \lim_{\varepsilon \to 0} \frac{\Comass_{B_\varepsilon(x)}(F)}{\Mass(B_\varepsilon(x))} \Mass(B_\varepsilon(x)) = \Comass(F, x).$$
For every $x \in M$ we may select coordinates in which $|F(x)| = F_I(x)$, and then if $x \notin Z$, we conclude that (\ref{crandall LDT}) holds for $x$.

If we combine (\ref{crandall LDT}) with (\ref{integral over chain is linfinity}), then
$$\sup_{x \in M} \Comass(F, x) \leq \Comass(F) \leq \|F\|_{L^\infty} \leq \sup_{x \in M} \Comass(F, x).$$
The inequalities collapse, proving (\ref{crandall linfinity}).
In particular, for each $\sigma \in \Chain_{d - 1}(M)$, we obtain (\ref{crandall best curl is ABC}):
\begin{align*}
\int_\sigma F &\leq \Mass(\sigma) \inf_{\Omega \supset \sigma} \sup_{x \in \Omega} \Comass(F, x) = \Mass(\sigma) \sup_{x \in \sigma} \Comass(F, x). \qedhere
\end{align*}
\end{proof}

\begin{definition}
Let $F \in L^\infty(M, \Omega^{d - 1}_{\rm cl})$.
The \dfn{maximum comass locus} is the set
$$\MCL(F) := \{x \in M: \Comass(F, x) = \Comass(F)\}.$$
\end{definition}

\begin{corollary}
Suppose that $M$ is a closed manifold and $F \in L^\infty(M, \Omega^{d - 1}_{\rm cl})$.
Then $\MCL(F)$ is a nonempty compact set.
\end{corollary}
\begin{proof}
This is immediate from Proposition \ref{crandall}(\ref{crandall usc}) and the extreme value theorem for upper semicontinuous functions.
\end{proof}

\begin{proposition}\label{properties of calibrated laminations}
Suppose that $M$ is a closed Riemannian manifold, $F$ is a calibration, and $\lambda$ is a measured oriented $F$-calibrated lamination.
Then $\supp \lambda \subseteq \MCL(F)$.
\end{proposition}
\begin{proof}
Let $S := \MCL(F)$, $N$ a leaf of $\lambda$, and suppose that $x \in N \setminus S$.
Since $S$ is closed, there exists $\varepsilon > 0$ such that $B_\varepsilon(x)$ does not meet $S$.
Moreover, $\sigma := N \cap B_\varepsilon(x)$ is a $d-1$-chain in $B_\varepsilon(x)$, so by Proposition \ref{crandall}(\ref{crandall best curl is ABC}),
$$\frac{1}{\Mass(\sigma)} \int_\sigma F \leq \sup_{y \in B_\varepsilon(x)} \Comass(F, y) < \Comass(F) = 1.$$
But then 
$$\int_N F = \int_\sigma F + \int_{N \setminus B_\varepsilon(x)} F < \Mass(\sigma) + \Mass(N \setminus B_\varepsilon(x)) = \Mass(N),$$
so $N$ (hence $\lambda$) is not $F$-calibrated.
\end{proof}

%%%%%%%%%%%%%%%%%%%
\subsection{Laminations of rational class}
We next give a condition for a measured stretch lamination to have only closed leaves, which appeared in unpublished work of Auer and Bangert \cite{Auer12}.

\begin{definition}
A homology class $\alpha \in H_{d - 1}(M, \RR)$ has \dfn{rational direction} if there exists $c > 0$ such that $c\alpha \in H_{d - 1}(M, \ZZ)$.
\end{definition}

\begin{proposition}
Let $M$ be a closed manifold, let $\alpha \in H_{d - 1}(M, \RR)$ have rational direction, $\alpha \neq 0$, and let $u$ be an $\alpha$-equivariant function of least gradient on $\tilde M$.
Then every leaf of $\kappa_u$ is a closed hypersurface.
\end{proposition}
\begin{proof}
Rescaling $u$ by a constant does not affect whether the leaves of $\kappa_u$ are closed, so after rescaling, we may assume that $\alpha \in H_{d - 1}(M, \ZZ)$.
We view $u$ as a map $M \to \Sph^1$ of homotopy class $\alpha$.\footnote{Note that $u$ is not continuous, so strictly speaking the homotopy class of $u$ is not $\alpha$; but by equivariance, it is profitable to view $u$ in this way.}
Since the representation $\alpha$ is integral, $\ker \alpha$ fits into a short exact sequence 
$$0 \to \ker \alpha \to \pi_1(M) \to \ZZ \to 0.$$
So from the Galois correspondence for covering spaces there exists a covering space $p: \hat M \to M$ such that $\Gal(\hat M, M) \cong \ZZ$, and a function $\hat u$ on $\hat M$, such that the diagram 
$$\begin{tikzcd}
\hat M \arrow[r, "\hat u"] \arrow[d, "p"] & \RR \arrow[d] \\
M \arrow[r, "u"] & \Sph^1
\end{tikzcd}$$
commutes.
In particular, it sense to take superlevel sets $\{\hat u > y\}$.

Since $\Gal(\hat M, M)$ is a cyclic group, it is generated by a single element $h$ with the mapping property
$$h: \partial \{\hat u > y\} \to \partial \{\hat u > y + D\}$$
for each $y \in \RR$.
In particular, $\hat u$ does not have a global minimum or maximum.
Since $\hat u$ does not have a global minimum or maximum on the complete manifold $\hat M$, by the maximum principle for functions of least gradient \cite[Theorem 5.1]{HakkarainenKorteLahtiShanmugalingam+2015}, $\hat u$ does not have any local minimum or maximum.
Therefore every leaf of $\kappa_{\hat u}$ is a component of $\partial \{\hat u > y\}$ for some $y \in \RR$ \cite[Proposition 4.7]{BackusCML}.

Now let $N$ be a (connected) leaf of $\lambda$, and let $\hat N := p^{-1}(N)$.
Then each component of $\hat N$ is a leaf of $\kappa_{\hat u}$, and hence a component of $\partial \{u > y\}$ for some $y \in \RR$.
Moreover, the deck transformation $h$ has the mapping property
$$h: \hat N \to \hat N.$$
So if $\hat N$ meets $\partial \{u > y\}$, then for every $n \in \ZZ$, a component $K_n$ of $\hat N$ is a component of $\partial \{\hat u > y + nD\}$.
Since $h$ acts on $\kappa_{\hat u}$ by sending $\partial \{\hat u > y\}$ to $\partial \{\hat u > y + D\}$, $h$ acts on $\hat N$ by sending $K_n$ to $K_{n + D}$.
In particular, $\hat N \cong N \times \ZZ$ and $p: \hat N \to N$ is projection onto the first factor. 
So $N \cong K_0$, and $K_0$ is a closed hypersurface in $\hat M$ since it is a complete manifold \cite[Theorem 3.3]{BackusCML} and it is compact.
\end{proof}

\begin{example}
Suppose that $M = \Sph^1_x \times \Sph^1_y$.
Let $u(x, y) = x$, so $\kappa_u$ is itself topologically nontrivial (in the $x$ direction) and its leaves are also topologically nontrivial (in the $y$ direction).
The covering space $\hat M$ is $\RR_x \times \Sph_y^1$, the point is that we unwound $\kappa_u$ without messing with its leaves.
\end{example}

%%%%%%%%%%%%%%%%%%%%
\section{Convex duality}
\subsection{Abstract convex duality}
We follow \cite{Ekeland99}.
For a reflexive Banach space $X$, we denote by $\hat X$ its dual.
If $I: X \to \RR \cup \{+\infty\}$ is a convex function, we introduce its \dfn{Legendre transform}, the convex function
\begin{align*}
	\hat I: \hat X &\to \RR \cup \{+\infty\}\\
	\xi &\mapsto \sup_{x \in X} \langle \xi, x\rangle - I(x).
\end{align*}
We identify the cokernel of a linear map with the kernel of its adjoint.
In this setting, we have the following form of the convex duality theorem.

\begin{theorem}[convex duality]\label{abstract convex analysis}
Let $\Lambda : X \to Y$ be a bounded linear map between reflexive Banach spaces.
Let $I: Y \to \RR \cup \{+\infty\}$ satisfy:
\begin{enumerate}
\item $I$ and $\hat I$ are strictly convex,
\item $I$ is lower semicontinuous,
\item if $|y| \to \infty$ in $Y$, then $I(y) \to +\infty$, and 
\item there exists a point $x \in X$ such that $I$ is continuous and finite at $\Lambda(x)$.
\end{enumerate}
Then:
\begin{enumerate}
\item There exists a minimizer $\underline x \in X$ of $I(\Lambda(x))$, unique modulo $\ker \Lambda$.
\item There exists a unique maximizer $\overline \eta$ of $-\hat I(-\eta)$ subject to the constraint $\eta \in \coker \Lambda$.
\item We have \dfn{strong duality}
\begin{equation}\label{abstract strong duality}
I(\Lambda(\underline x)) = -\hat I(-\overline \eta).
\end{equation}
\end{enumerate}
\end{theorem}
\begin{proof}
This is largely a special case of \cite[Chapter IV, Theorem 4.2]{Ekeland99}.
Let $\mathscr P, \mathscr P^*$ be as in the statement of that theorem.
Then $\mathscr P$ is the problem of minimizing $J(x, \Lambda x)$ where $J(x, y) := I(y)$.
The Legendre transform of $J$ satisfies 
$$\hat J(\xi, \eta) = \begin{cases} \hat I(\eta), & \xi = 0, \\
	+\infty, &\xi \neq 0,
\end{cases}$$
and $\mathscr P^*$ is the problem of maximizing
$$-\hat J(\Lambda^* \eta, -\eta) = \begin{cases}
	-\hat I(-\eta), &\eta \in \ker \Lambda^*, \\
	-\infty, &\eta \notin \ker \Lambda^*,
\end{cases}$$
where $\Lambda^*$ is the adjoint of $\Lambda$.
Then most of the various assertions of this theorem follow immediately from \cite[Chapter IV, Theorem 4.2]{Ekeland99}.
The fact that $\overline \eta \in \coker \Lambda$ follows from the facts that $\overline \eta$ is a solution of $\mathscr P^*$, but any solution of $\mathscr P^*$ must be a member of $\ker \Lambda^*$. 
To establish uniqueness, we use \cite[Chapter II, Proposition 1.2]{Ekeland99}, the fact that $\hat I$ is strictly convex, and the fact that we may view $I \circ \Lambda$ as a strictly convex function on the reflexive Banach space $X/\ker \Lambda$.
\end{proof}

%%%%%%%%%%%%%%%%%%%%%%%%
\subsection{Norms on cohomology}
Let $M$ be a closed oriented Riemannian manifold. 
We shall, without mention, use Poincar\'e duality to identify 
$$H^k(M, \RR) = H_{d - k}(M, \RR).$$
We put the stable and costable norms on these cohomology groups into a broader family of norms to which H\"older duality applies, in degrees $1$ and $d - 1$.

\begin{definition}
Let $p \in [1, \infty]$, $k \in \{1, d - 1\}$, and $\alpha \in H^k(M, \RR)$, viewed as a de Rham class.
Then 
$$\|\alpha\|_{L^p} := \inf_{[f] = \alpha} \|f\|_{L^p}.$$
\end{definition}

It is clear that the $L^1$ and stable norms on $H^1$ agree, as do the $L^\infty$ and costable norms on $H^{d - 1}$.
The definition of the $L^p$ norm makes sense even on $H^k$, $k \in \{2, \dots, d - 2\}$, but there we cannot identify the norm with the stable or costable norms, because, for example, on the square torus $\Torus^4$, the costable norm of 
$$\alpha := [\dif x \wedge \dif y + \dif z \wedge \dif w]$$
is $1$ but $\|\alpha\|_{L^\infty} = \sqrt 2$.

\begin{proposition}
Let $(p, q)$ be a H\"older pair with $q < \infty$.
Then the $L^p$ norm on $H^{d - 1}$ and the $L^q$ norm on $H^1$ are dual.
\end{proposition}
\todo{Prove it. This might be easier to do after the next section}

\todo{Massart's inequalities for homology}



%%%%%%%%%%%%%%%%%%%%%%%%
\subsection{Application to the equivariant \texorpdfstring{$p$-Laplacian}{p-Laplacian}}
Let $M$ be a closed oriented Riemannian manifold with fundamental group $\Gamma$ and universal cover $\tilde M \to M$.
Choose a fundamental domain $M_{\rm fun} \subseteq \tilde M$.
Given a representation $\alpha: \Gamma \to \RR$, choose a smooth $1$-form, which we also call $\alpha$, to represent the cohomology class corresponding to $\alpha$ in $H^1(M, \RR)$ given by the natural isomorphism
$$H^1(M, \RR) = \Hom(\Gamma, \RR).$$

Given a H\"older pair $(p, q)$ with $1 < p < \infty$ (thus $\frac{1}{p} + \frac{1}{q} = 1$),
we are interested in the $q$-Laplace equation
$$\dif^* (|\dif u|^{q - 2} \dif u) = 0$$
for an $\alpha$-equivariant function $u$.
Let us express this problem variationally.

Let $X$ be the space of $W^{1, q}_\loc(\tilde M)$ functions $u$ which are $\Gamma$-equivariant in the sense that for $\gamma \in \Gamma$, $\gamma^* \dif u = \dif u$, and let $Y := L^q(M, \Omega^1)$.
We identify $\hat Y$ with $L^p(M, \Omega^{d - 1})$ using the perfect pairing 
\begin{align*}
	L^p(M, \Omega^{d - 1}) \times Y &\to \RR \\
	(F, \varphi) &\mapsto \int_M \varphi \wedge F.
\end{align*}
Then $\Lambda := (\dif: X \to Y)$ is a bounded linear map, and the $\alpha$-equivariant $q$-Laplacian is the Euler-Lagrange equation of $I \circ \Lambda$, where $I$ is the strictly convex functional such that
$$I(\varphi) := \frac{1}{q} \int_M \star |\varphi|^q$$
if the cohomology class of $\varphi$ is $\alpha$, and $I(\varphi) := +\infty$ otherwise.

\begin{proposition}[convex duality for the $q$-Laplacian]\label{mfmc qLaplacian}
Given a representation $\alpha: \Gamma \to \RR$, and a H\"older pair $(p, q)$ with $1 < p < \infty$, there exists an $\alpha$-equivariant $q$-harmonic function $u: \tilde M \to \RR$, unique modulo constants, and a unique minimizer $F$ of 
$$J_{p, \alpha}(F) := \frac{1}{p} \int_M \star |F|^p - \int_M \alpha \wedge F$$
among all closed $d - 1$-forms on $M$.
Moreover, we have
\begin{equation}\label{strong duality}
	\frac{1}{q} \int_M \star |\dif u|^q + \frac{1}{p} \int_M \star |F|^p + \int_M \dif u \wedge F = 0.
\end{equation}
\end{proposition}
\begin{proof}
Let
$$I_\alpha(\psi) := \frac{1}{q} \int_M \star |\psi + \alpha|^q,$$
defined for exact $L^q$ $1$-forms $\psi$, thus $\widehat{I_\alpha}$ is defined on the space of $L^p$ $d - 1$-forms modulo the kernel of the map
$$F \mapsto \left(\psi \mapsto \int_M \psi \wedge F\right)$$
and we lift it to $L^p(M, \Omega^{d - 1})$.

Let $v$ be a primitive of $\alpha$.
Then an $\alpha$-equivariant $u$ minimizes $I$ iff $u - v$ minimizes $I_\alpha \circ \Lambda$, which happens iff $u$ is $\alpha$-equivariant $q$-harmonic.
Since $I_\alpha(\psi) = I_0(\psi + \alpha)$, we can apply \cite[Chapter I, Remark 4.1]{Ekeland99} to see that $I_\alpha$ and $\widehat{I_\alpha}$ are strictly convex and
$$\widehat{I_\alpha}(F) = \widehat{I_0}(F) - \int_M \alpha \wedge F = \frac{1}{p} \int_M \star |F|^p - \int_M \alpha \wedge F.$$
Since $\coker \Lambda$ is the space of closed $L^p$ $d - 1$-forms on $M$, for $F \in \coker \Lambda$, $\widehat{I_\alpha}(F)$ does not depend on the choice of representatives $\alpha, v$, or of the lift of $\widehat{I_\alpha}$ to $L^p(M, \Omega^{d - 1})$.

Finally observe that $\ker \Lambda$ is the space of constants, and for any $\alpha$-equivariant $u$ and closed $F$,
\begin{align*}
I(\Lambda u) + \widehat{I_\alpha}(-F)
&= \frac{1}{q} \int_M \star |\dif u|^q + \frac{1}{p} \int_M \star |F|^p + \int_M \alpha \wedge F \\
&= \frac{1}{q} \int_M \star |\dif u|^q + \frac{1}{p} \int_M \star |F|^p + \int_M \dif u \wedge F.
\end{align*}
All of the assertions of this proposition now follow from Theorem \ref{abstract convex analysis}.
\end{proof}

Let $u$ be $\alpha$-equivariant $q$-harmonic.
Motivated by \cite[\S3.1]{daskalopoulos2020transverse}, it is natural to guess that 
\begin{equation}\label{dual solution}
F := - |\dif u|^{q - 2} \star \dif u
\end{equation}
is the solution of the dual problem of minimizing $J_{p, \alpha}$.
In order to prove that this is true, we shall need that if $(p, q)$ is a H\"older pair, then
\begin{equation}\label{holder cancellation}
	(p - 2)(q - 1) + (q - 2) = 0.
\end{equation}

\begin{lemma}\label{dual to u is minimizer}
Suppose that $u: \tilde M \to \RR$ is an $\alpha$-equivariant $q$-harmonic function, and suppose that $F$ satisfies (\ref{dual solution}).
Then $F$ is a closed $d - 1$-form, which minimizes $J_{p, \alpha}$ among all closed $d - 1$-forms.
Moreover, $F$ solves the PDE 
\begin{equation}\label{pMaxwell}
\begin{cases}
	\dif F = 0 \\
	\dif^* (|F|^{p - 2} F) = 0.
\end{cases}
\end{equation}
\end{lemma}
\begin{proof}
We first show that $\dif F = 0$.
In fact, 
$$\star \dif F = - \star \dif(|\dif u|^{q - 2} \star \dif u) = \pm \dif^*(|\dif u|^{q - 2} \dif u) = 0.$$
By uniqueness, if (\ref{strong duality}) holds, then $F$ must be the minimizer of $J_{p, \alpha}$.
One can easily compute 
$$|F|^p = |\dif u|^{(q - 1)p} = |\dif u|^q,$$
so by Stokes' theorem and the fact that $\alpha$ is cohomologous to $\dif u$,
\begin{align*}
\frac{1}{q} \int_M \star |\dif u|^q + \frac{1}{p} \int_M \star |F|^p&
= \left[\frac{1}{p} + \frac{1}{q}\right] \int_M \star |\dif u|^q
= \int_M \dif u \wedge |\dif u|^{q - 2} \star \dif u \\
&= -\int_M \alpha \wedge F,
\end{align*}
implying (\ref{strong duality}).
Finally, we use (\ref{holder cancellation}) to prove
\begin{align*}
\dif^*(|F|^{p - 2} F) &= - \dif^*(|\dif u|^{(p - 2)(q - 1)} |\dif u|^{q - 2} \star \dif u) = - \dif^*(\star \dif u) \\
&= \pm \star \dif^2 u = 0. \qedhere 
\end{align*}
\end{proof}

We next scrutinize the PDE (\ref{pMaxwell}).
At least at the heuristic level, one expects that as $p \to \infty$, the solutions of (\ref{pMaxwell}) converge to an absolute minimizer of a suitable $L^\infty$ variational problem; minimizers of such problems have been called \dfn{tight} by Sheffield and Smart \cite{Sheffield12}.
This motivates the below terminology:

\begin{definition}
Let $1 < p < \infty$.
A \dfn{$p$-tight form} is a solution of the PDE (\ref{pMaxwell}).
\end{definition}

\begin{proposition}
Suppose that $M$ is a closed oriented Riemannian manifold.
Then there is a unique $p$-tight form in each cohomology class in $H^{d - 1}(M, \RR)$.
Moreover, $p$-tight forms are minimizers of the strictly convex functional
$$J_p(F) := \frac{1}{p} \int_M \star |F|^p$$
among all forms cohomologous to them.
\end{proposition}
\begin{proof}
Strict convexity of $J_p$ on closed $L^p$ $d - 1$-forms is straightforward; since each cohomology class is an affine subspace of $L^p(M, \Omega^{d - 1})$, and hence is convex, the strict convexity on each class follows.
Since $J_p(F) \to \infty$ as $\|F\|_{L^p} \to \infty$, $J_p$ is coercive on $L^p(M, \Omega^{d - 1})$.
Therefore we have existence and uniqueness \cite[Chapter II]{Ekeland99}.
To compute the Euler-Lagrange equations for $J_p$, let $B$ be a $d-2$-form (so $F + t \dif B$ is cohomologous to $F$), so that for a minimizer $F$ of $J_p$,
$$\frac{\dif}{\dif t} J_p(F + t \dif B) = \frac{1}{p} \int_M \star \frac{\partial}{\partial t} |F + t \dif B|^p = \int_M \star |F + t \dif B|^{p - 2} \langle F + t \dif B, \dif B\rangle.$$
Setting $t = 0$, we obtain 
$$0 = \int_M \star |F|^{p - 2} \langle F, \dif B\rangle = \int_M \star \langle \dif^*(|F|^{p - 2} F), B\rangle.$$
Thus the Euler-Lagrange equations for $J_p$ are (\ref{pMaxwell}).
\end{proof}

\begin{definition}
Let $F$ be a $p$-tight form, let
\begin{equation}
\dif u := (-1)^d |F|^{p - 2} \star F, \label{inverse extremality}
\end{equation}
and let $u$ be the primitive of $\dif u$ on the universal cover $\tilde M$, which is normalized to have zero mean on a fundamental domain $M_{\rm fun}$.
Then $u$ is called the \dfn{$q$-harmonic conjugate} of the $p$-tight form $F$, where $\frac{1}{p} + \frac{1}{q} = 1$.
\end{definition}

Let $u$ be the $q$-harmonic conjugate of $F$.
By Poincar\'e's inequality,
$$\|u\|_{W^{1, q}(M_{\rm fun})}^q \lesssim \int_M \star |\dif u|^q = \int_M \star |F|^{(p - 1)q} = \int_M \star |F|^p < \infty$$
since $F$ is $p$-tight; that is, we have $F \in L^p$ and $u \in W^{1, q}_\loc$, justifying any manipulations we shall make with these forms.

\begin{lemma}
Let $1 < p, q < \infty$ and $\frac{1}{p} + \frac{1}{q} = 1$.
Let $F$ be a $p$-tight form, and let $u$ be its $q$-harmonic conjugate.
Then $u$ is $q$-harmonic, $F$ satisfies (\ref{dual solution}), and we have strong duality (\ref{strong duality}).
\end{lemma}
\begin{proof}
We first use (\ref{holder cancellation}) to prove
$$|\dif u|^{q - 2} \star \dif u = (-1)^d |F|^{(q - 2)(p - 1)} \star \star |F|^{p - 2} F = - |F|^{(q - 2)(p - 1) - (p - 2)} F = - F.$$
Thus we have (\ref{dual solution}), and moreover
$$\dif \star (|\dif u|^{q - 2} \dif u) = - \dif F = 0$$
so that $u$ is $q$-harmonic.
Then by Lemma \ref{dual to u is minimizer}, $F$ is the unique minimizer of $J_{p, [\dif u]}$, which implies (\ref{strong duality}).
\end{proof}

\subsection{Application to \texorpdfstring{$p$-harmonic}{p-harmonic} maps to \texorpdfstring{$\Hyp^2$}{hyperbolic space}}


%%%%%%%%%%%%%%%%
\section{The canonical lamination}
\label{canonical sec}
Throughout this section, we fix $\rho \in H^{d - 1}(M, \RR)$ in the costable unit sphere $\{\Comass(\rho) = 1\}$.
Motivated by Thurston's approach to Teichm\"uller theory (see \S\ref{Teichmuller}), we construct a lamination which is calibrated by every best comass form in $\rho$, and which only depends on $\rho$: the \dfn{canonical lamination} $\lambda_\rho$.

Even if $F$ is a best comass representative of $\rho$, then $\MCL(F)$ need not itself be a lamination.
This can happen even if $F$ is smooth, because the bundle $\ker(\star F)$ does not need to be integrable near $\MCL(F)$, even if $M$ is homeomorphic to $\mathbf T^3$ \cite[Example 5.4]{bangert_cui_2017}.
So we must show that there is a lamination $\lambda_F \subseteq \MCL(F)$ which contains every $F$-calibrated hypersurface.
We will then be able to show that the intersection $\bigcap_F \lambda_F$, where $F$ ranges over best comass representatives of $\rho$, is a lamination.

One may wonder why we must take an intersection:
If $F$, $G$ are cohomologous, and $\lambda$ is an $F$-calibrated measured lamination, then $\lambda$ is $G$-calibrated by Proposition \ref{properties of calibrated laminations}.
However, not every lamination admits a transverse measure, and for such a lamination, we do not expect calibration to be a cohomological invariant.


%%%%%%%%%%%%%%%%%%%%%%%%%%
\subsection{Intersections of minimal hypersurfaces}\label{nodal appendix}
Let $F$ be a best comass representative of $\rho$.
We begin the construction of the lamination $\lambda_F$ by showing that any two $F$-calibrated hypersurfaces are disjoint.
This can be done by showing that the generic intersection point of two minimal hypersurfaces $N, N'$ is transverse.
If the dimension of the underlying manifold $M$ is $d = 2$, then this is trivial, and if $d = 3$, then the structure of $N \cap N'$ is completely described by complex-analytic means \cite[Theorem 7.3]{colding2011course}, so the proof we present here is mainly of interest if $d \geq 4$.

For a solution $v$ of an elliptic PDE, we write $Z(v), Z^{\rm sing}(v)$ for the nodal and singular sets of $v$, namely the sets of zeroes and double zeroes, respectively.
We will show that the generic point of $Z(v)$ is not a singular point. More precisely:

\begin{proposition}\label{nodal set is generically smooth}
Let $Q$ be a linear elliptic operator on $\Ball^{d - 1}$ satisfying the maximum principle.
Suppose that $Qv = 0$ and $v$ has a zero of finite order.
Then the Hausdorff dimensions of the nodal and singular sets of $v$ are
\begin{align}
	\dim(Z(v)) &= d - 2, \label{nodal dimension}\\
	\dim(Z^{\rm sing}(v)) &\leq d - 3. \label{singular nodal dimension}
\end{align}
\end{proposition}

The main idea of the proof is to show that the complement of $Z^{\rm sing}(v)$ is path-connected.
By Alexander duality for singular cohomology, the complement of a submanifold $P$ of codimension $\geq 2$ is path-connected.
The singular set $P = Z^{\rm sing}(v)$ is not in general a manifold, but the proof still works as long as we apply Alexander duality for sheaf cohomology.
Let $\hat H^\bullet(P, \RR)$ denote the cohomology of the constant sheaf $\RR$ on $P$.

\begin{lemma}\label{closed mfld complement}
Let $P \subset \Sph^{d - 1}$ be a closed $d - 3$-rectifiable set.
Then $\Sph^{d - 1} \setminus P$ is path-connected.
\end{lemma}
\begin{proof}
Let $\delta^{\rm Haus}, \delta^{\rm cov}, \delta^{\rm shf}$ be the Hausdorff, covering, and sheaf cohomological dimensions of $P$ respectively.
Then by \cite[{\S}II.5.12]{godement1973topologie} and \cite[Theorem 6.3.10]{edgar2008measure}, we have 
$$\delta^{\rm shf} \leq \delta^{\rm cov} \leq \delta^{\rm Haus} \leq d - 3,$$
hence $\hat H^{d - 2}(P, \RR) = 0$.
By Alexander duality for sheaf cohomology \cite[Theorem 6]{Kaplan47}, it follows that $H_0(\Sph^{d - 1} \setminus P, \RR) \cong \ZZ$, or in other words $\Sph^{d - 1} \setminus P$ is path-connected.
\end{proof}

\begin{lemma}\label{open mfld complement}
Let $P \subset \Ball^{d - 1}$ be a closed $d - 3$-rectifiable set.
Then $\Ball^{d - 1} \setminus P$ is path-connected.
\end{lemma}
\begin{proof}
Embed $\Ball^{d - 1}$ in $\Sph^{d - 1}$ using the one-point compactification, let $\infty$ be the point at infinity, and let $x, y \in \Ball^{d - 1} \setminus P$.
Choose a $d - 3$-sphere $S$ in $\Sph^{d - 1}$ which contains $\infty$ but does not contain $x, y$.
Then $P \cup S$ is a closed $d - 3$-rectifiable set and $x, y \notin P \cup S$, so by Lemma \ref{closed mfld complement}, there exists a curve $\gamma$ from $x$ to $y$ which avoids $P \cup S$.
Therefore $\gamma \subset \Ball^{d - 1} \setminus P$.
\end{proof}

\begin{proof}[Proof of Proposition \ref{nodal set is generically smooth}]
By \cite[Lemma 1.9]{Hardt89}, $Z^{\rm sing}(v)$ is $d - 3$-rectifiable, which implies (\ref{singular nodal dimension}).
If there exists $x \in Z(v) \setminus Z^{\rm sing}(v)$, then by the implicit function theorem, there is a neighborhood $U \ni x$ such that $U \cap Z(v)$ is a $d - 2$-dimensional manifold.
So if (\ref{nodal dimension}) fails, we must have $Z(v) = Z^{\rm sing}(v)$, so $Z(v)$ is $d - 3$-rectifiable.
But then, by Lemma \ref{open mfld complement}, the sets $U_\pm := \{\pm v > 0\}$ satisfy $U_+ \cup U_-$ are connected.
Since $v$ is continuous, one of these sets must be empty; without loss of generality, $U_- = \emptyset$.
Then $v \geq 0$ and $v$ has a zero, so by the maximum principle, $v = 0$ identically.
This contradicts the fact that $v$ has a zero of finite order.
\end{proof}

\begin{proposition}\label{intersection theory prop}
Let $N, N' \subset M$ be minimal hypersurfaces, and let $S \subseteq N, N'$ be the set of points at which $N, N'$ intersect nontransversely.
Then one of the following holds:
\begin{enumerate}
\item $N \cap N'$ is empty.
\item $\dim(N \cap N') = d - 2$ and $\dim S \leq d - 3$.
\item There exists $p \in S$ such that the germs of $N, N'$ at $p$ are equal.
\end{enumerate}
\end{proposition}
\begin{proof}
Let $p \in S$, and let $P$ be the tangent space of $N, N'$ at $x$.
Then we can view $N, N'$ as the graphs of functions $u, u'$ over $P$, say taken in normal coordinates based at $p$; thus we identify $P$ with $\RR^{d - 1}$.
Reasoning as in the proof of \cite[Theorem 7.3]{colding2011course}, the difference $v := u - u'$ solves a linear elliptic PDE $Qv = 0$, and in a neighborhood $U \ni p$, the exponential map $P \to N$ induces Lipschitz isomorphisms $\{v = 0\} \cap U \cong N \cap N' \cap U$ and $\{v = \dif v = 0\} \cap U \cong S \cap U$.
If $v$ only has zeroes of finite order, then the claim follows from Proposition \ref{nodal set is generically smooth}.
Otherwise, $v$ is identically $0$ by the unique continuation theorem \cite[Theorem 6.1]{colding2011course}, so $N \cap U = N' \cap U$.
\end{proof}

%%%%%%%%%%%%%%%%%%%%%
\subsection{Construction of the canonical lamination}
We are now ready to prove Theorem \ref{existence of calibrated lam}, the existence of the canonical lamination.

\begin{lemma}
Let $F$ be a calibration, and let $B \subseteq M$ be a sufficiently small ball.
Then for any complete connected $F$-calibrated hypersurface $N$, 
\begin{equation}\label{area bound for calibrated}
\Mass(N \cap B) \leq \Mass(\partial B).
\end{equation}
\end{lemma}
\begin{proof}
By the Thom transversality theorem, possibly after shrinking $B$ slightly, we may assume that $B$ meets $N$ transversely.
Since both sides of (\ref{area bound for calibrated}) depend continuously on the radius of $B$, this assumption is no loss of generality.

Let $S := N \cap \partial B$, which by transversality can be identified with a closed $d - 2$-dimensional submanifold of $\Sph^{d - 1}$.
Since $H_{d - 2}(\Sph^{d - 1}, \RR) = 0$, there exists a relatively open set $U \subseteq \partial B$ which is bounded by $S$.
Since $H^{d - 1}(B, \RR) = 0$, we may write $F = \dif A$ in a neighborhood of $B$, where by Lemma \ref{Hodge theorem} we may assume that $A$ is continuous.
Then
\begin{align*}
\Mass(N \cap B) &= \int_{N \cap B} F = \int_S A = \int_U F \leq \Mass(U) \leq \Mass(\partial B). \qedhere
\end{align*}
\end{proof}

\begin{lemma}
There exists a constant $C > 0$, only depending on $M$, such that for every calibration $F$ and complete $F$-calibrated hypersurface $N$, we have the curvature bound
\begin{equation}\label{curvature bound for calibrated}
\|\Two_N\|_{C^0} \leq C.
\end{equation}
\end{lemma}
\begin{proof}
Let $x \in N$ and let $r > 0$ be small.
Then each component $N'$ of $N \cap B(x, r)$ is absolutely area-minimizing by the fundamental theorem of calibrated geometry, so it is stable.
By (\ref{area bound for calibrated}), $\Mass(N') \lesssim r^{d - 1}$.
So by \cite[pg785, Corollary 1]{Schoen81},\footnote{See also \cite[Theorem 3]{Schoen75} for an easier proof when $M$ has nonpositive curvature and dimension $d \leq 6$, or \cite[Chapter 2, \S\S4-5]{colding2011course} for a textbook treatment of a similar estimate. By \cite[Lemma 2.4]{chodosh2022complete}, we may remove the dependence on the volume bound if $d \leq 4$.}
\begin{align*}
\|\Two_{N'}\|_{B(x, r/2)} \lesssim_{d, \|\Riem_g\|_{C^0(B(x, 2r))}} \frac{1}{r}.
\end{align*}
Since $N'$ was an arbitrary component, the same estimate holds for $N$.
Using the compactness of $M$, we may cover it by finitely many balls in which estimates of this form hold to conclude (\ref{curvature bound for calibrated}).
\end{proof}

\begin{lemma}\label{calibrated implies disjoint}
Let $F$ be a calibration, and let $N, N'$ be complete connected $F$-calibrated hypersurfaces.
If $N \cap N'$ is nonempty, then $N = N'$.
\end{lemma}
\begin{proof}
We first observe that if $x \in N \cap N'$, then $(\star F(x))^\sharp$ is the normal vector to both $N, N'$ at $x$.
Therefore $N \cap N'$ only consists of points of tangency.
By Proposition \ref{intersection theory prop}, it follows that either the germs of $N, N'$ at $x$ are equal.
Since the germs are equal and $N, N'$ are connected, a standard boostrapping argument implies that $N = N'$.
\end{proof}

\begin{proposition}\label{existence of semicanonical lamination}
Let $F$ be a best comass calibration.
Then the set of $F$-calibrated hypersurfaces is the set of leaves of a lamination $\lambda_F$, which contains every measured stretch lamination associated to $[F]$.
\end{proposition}
\begin{proof}
Let $\mathscr L_F$ be the set of connected complete $F$-calibrated hypersurfaces.
By Lemma \ref{calibrated implies disjoint}, $\mathscr L_F$ consists of pairwise disjoint minimal hypersurfaces.
By Proposition \ref{MCL contains Thurston}, there exists a measured stretch lamination $\lambda$ associated to $[F]$, and then by Proposition \ref{properties of calibrated laminations}, $\mathscr L_F$ contains every leaf of $\lambda$.
Since the estimate (\ref{curvature bound for calibrated}) is independent of $N$, it follows by Theorem \ref{disjoint surfaces are lamination} that $\mathscr L_F$ is the set of leaves of some lamination $\lambda_F$.
\end{proof}

\begin{lemma}\label{existence of intersections}
Let $\mathscr S$ be a nonempty set of laminations.
Suppose that there exists a hypersurface which is a leaf of every lamination in $\mathscr S$.
Then there exists a lamination whose set of leaves is the intersection of the sets of leaves of the laminations in $\mathscr S$.
\end{lemma}
\begin{proof}
Let $\lambda \in \mathscr S$, and let $(F_\alpha, K_\alpha)$ be a laminar atlas for $\mathscr S$.
Let $K'_\alpha$ be the set of $k \in K_\alpha$ such that for every $\kappa \in \mathscr S$, there exists a leaf $N$ of $\kappa$ such that
$$(F_\alpha)_*(\{k\} \times J) \subseteq N.$$
It is clear that this property is preserved by transition maps.
Then $K_\alpha'$ is an intersection of compact sets (since the local leaf spaces of each $\kappa \in \mathscr S$ is compact), so $K_\alpha'$ is compact.
The hypersurface which is a common leaf of every lamination in $\mathscr S$ witnesses that for some $\alpha$, $K_\alpha'$ is nonempty.
Therefore $(F_\alpha, K'_\alpha)$ is a laminar atlas for the lamination whose support is $\bigcap_{\kappa \in \mathscr S} \supp \kappa$.
\end{proof}

\begin{proposition}\label{existence of canonical lamination}
The set of hypersurfaces which are calibrated by every best comass representative of $\rho$ is the set of leaves of a lamination $\lambda_\rho$, which contains every measured stretch lamination associated to $\rho$.
\end{proposition}
\begin{proof}
By Proposition \ref{MCL contains Thurston}, there is a (measured stretch) lamination which is calibrated by every best comass representative of $\rho$.
So we may apply Lemma \ref{existence of intersections} to the set $\mathscr S$ of all calibrated laminations $\lambda_F$ produced by Proposition \ref{existence of semicanonical lamination}, where $F$ ranges over best comass representatives of $\rho$.
\end{proof}

\begin{definition}
The lamination $\lambda_\rho$ constructed in Proposition \ref{existence of canonical lamination} is the \dfn{canonical lamination} associated to $\rho$.
\end{definition}

If $d = 2$ and $\rho \in H^1(M, \ZZ)$, then $\lambda_\rho$ is the lamination obtained by taking an intersection over the best Lipschitz maps of homotopy class $\rho$ by Daskalopolous and Uhlenbeck \cite[\S6.2]{daskalopoulos2020transverse}.

%%%%%%%%%%%%%%%%%%%%%%%%%%%%%%%%
\subsection{Structure of the canonical lamination}\label{canonical structure}
We now study the structure of the canonical lamination.
A sticky technical point is that $H_{d - 1}(M, \RR)$ need not be strictly convex, so there may be many $\alpha$ in the stable unit sphere such that 
\begin{equation}\label{flats duality}
\Comass(\rho) = \langle \rho, \alpha\rangle.
\end{equation}
In particular, there may be many measured stretch sublaminations of the canonical lamination which are mutually nonhomologous.
We therefore introduce the dual set 
$$\rho^* := \{\alpha \in H_{d - 1}(M, \RR): \langle \rho, \alpha\rangle = \Mass(\alpha) = 1\}.$$
It is clear that any measured sublamination of the canonical lamination normalized to have mass $1$ represents a member of $\rho^*$.
In fact, this condition completely characterizes $\rho^*$, as we now show.

\begin{lemma}\label{homologically minimizing means measured stretch}
For every $\alpha \in \rho^*$, every measured oriented, homologically minimizing, lamination representing $\alpha$ is a measured stretch lamination associated to $\rho$.
\end{lemma}
\begin{proof}
Let $\dif u$ be the Ruelle-Sullivan current of the measured oriented, homologically minimizing lamination $\lambda$, and let $F$ be a best comass representative of $\rho$.
Since $\lambda$ is homologically minimizing,
$$\int_M \dif u \wedge F = \langle \rho, \alpha\rangle = \Mass(\alpha) = \Mass(\lambda),$$
so by Proposition \ref{calibration condition}, $F$ calibrates $\lambda$.
Therefore by Proposition \ref{calibrated means measured stretch}, $\lambda$ is a measured stretch lamination.
\end{proof}

\begin{lemma}\label{existence for least gradient}
For each $\alpha \in H_{d - 1}(M, \RR)$, there exists an $\alpha$-equivariant function of least gradient $u: \tilde M \to \RR$.
\end{lemma}
\begin{proof}
The space $X$ of $\alpha$-equivariant functions on $\tilde M$ is closed under $L^1_\loc$ limits by Lemma \ref{L1 convergence preserves pi1}, so the existence of a minimizer in $X$ follows from an argument similar to the solution of the Dirichlet problem for least gradient functions \cite[Theorem 1.20]{Giusti77}.
\end{proof}

\begin{proposition}\label{enough measures in canonical lamination}
For each $\alpha \in \rho^*$, there exists a measured stretch sublamination of $\lambda_\rho$ with homology class $\alpha$.
\end{proposition}
\begin{proof}
Let $u$ be the function of least gradient furnished by Lemma \ref{existence for least gradient}.
The measured oriented, homologically minimizing, lamination $\kappa_u$ has class $\alpha$.
So by Lemma \ref{homologically minimizing means measured stretch}, it is a measured stretch lamination and hence is a sublamination of $\lambda_\rho$.
\end{proof}

We next use the decomposition of measured laminations \cite[{\S}I.3]{Morgan88} to partition the leaves of $\lambda_\rho$ into various categories.
In this direction we shall need to study measured laminations which are minimal with respect to inclusion; as the word ``minimal'' is overloaded, we shall call such laminations ``indecomposable''.

\begin{definition}
Let $\lambda$ be a lamination.
\begin{enumerate}
\item $\lambda$ is \dfn{indecomposable} if the only sublamination of $\lambda$ is itself.
\item If $\lambda$ is indecomposable, then $\lambda$ is \dfn{exceptional} if $\supp \lambda \neq M$ and $\lambda$ does not consist of a single leaf.
\item $\lambda$ is a \dfn{parallel family of closed leaves} if there exists a closed hypersurface $N \subset M$ with trivial normal bundle, such that every leaf of $\lambda$ is a section of the normal bundle of $N$.
\item A leaf $N$ of $\lambda$ is \dfn{nonmeasurable} if, for every sublamination $\kappa \subset \lambda$ which admits a transverse measure, $N$ is not a leaf of $\kappa$.
\end{enumerate}
\end{definition}

Thus every indecomposable lamination either is a foliation in which every leaf is dense, an exceptional indecomposable lamination, or a closed hypersurface.
Moreover, every local leaf space $K_\alpha$ of an exceptional indecomposable lamination $\lambda$ is a Cantor set \cite[{\S}I.3.1]{Morgan88}, and every leaf of $N$ is noncompact.
Every nonmeasurable leaf is noncompact, for if $N$ is a closed leaf, then $N$ equipped with its Dirac measure is a measured sublamination of $\lambda$.

\begin{theorem}\label{MorganShelan}
Let $\lambda$ be a measured oriented lamination in the closed manifold $M$.
Then either $\lambda$ is a foliation with a dense leaf, or $\lambda$ separates into finite number of clopen sublaminations, each of which is a parallel family of closed leaves or an exceptional indecomposable lamination.
\end{theorem}
\begin{proof}
First observe that the proof of \cite[Theorem I.3.2]{Morgan88} goes through for any lamination $\lambda$ such that no leaf of $\lambda$ is dense in $M$, even if $\lambda$ is a foliation.
Twisted families of closed leaves (that is, families of sections of a nontrivial normal bundle of a closed hypersurface) are excluded by the fact that $\lambda$ is oriented, so its leaves are oriented, and hence the normal bundle of any of its leaves is trivial.
\end{proof}

\begin{proposition}\label{classification of leaves}
For each leaf $N$ of $\lambda_\rho$, one of the following holds:
\begin{enumerate}
\item $N$ is closed.
\item $N$ is a noncompact leaf of an exceptional indecomposable measured stretch lamination associated to $\rho$.
\item $N$ is noncompact and $\lambda_\rho$ is a foliation which admits a transverse measure.
\item $N$ is noncompact and $N$ is a nonmeasurable leaf of $\lambda_\rho$.
\end{enumerate}
\end{proposition}
\begin{proof}
If $N$ is a closed leaf of $\lambda_\rho$, then $N$ equipped with its Dirac measure is a measured lamination, calibrated by any tight representative of $\rho$; hence it is measured stretch for $\rho$.
Otherwise, since $N$ has no boundary, it is noncompact.

If $N$ is noncompact, but is contained in a measured sublamination $\kappa$ of $\lambda_\rho$, then by Theorem \ref{MorganShelan}, either $\kappa$ is a foliation or $N$ is contained in an exceptional indecomposable sublamination.
If $\kappa$ is a foliation, then
$$\supp \kappa \supseteq \supp \lambda_\rho \supseteq \supp \kappa,$$
implying $\kappa = \lambda_\rho$.
Otherwise, the exceptional indecomposable sublamination $\zeta$ of $\kappa$ containing $N$ is calibrated by any tight representative of $\rho$, so $\zeta$ is measured stretch for $\rho$ by Proposition \ref{calibrated means measured stretch}.
\end{proof}

\begin{corollary}\label{measurable leaves are contained in indecomposables}
Let $N$ be a leaf of the canonical lamination $\lambda_\rho$.
Then either $N$ is nonmeasurable, or $N$ is contained in an indecomposable measured stretch lamination associated to $\rho$.
\end{corollary}
\begin{proof}
By Proposition \ref{classification of leaves}, if $N$ is not nonmeasurable, then either $N$ is closed, in which case $N$ is itself an indecomposable measured stretch lamination, or $N$ is noncompact and is contained in an exceptional indecomposable measured stretch lamination.
\end{proof}

\begin{corollary}
Let $F$ be a best comass representative of $\rho$, and $N$ a leaf of the calibrated lamination $\lambda_F$.
Then either $N$ is a leaf of the canonical lamination $\lambda_\rho$, or $N$ is a nonmeasurable leaf of $\lambda_F$.
\end{corollary}
\begin{proof}
Suppose that $N$ is a leaf of a measured sublamination $\kappa$ of $\lambda_F$.
Then, since $\kappa$ is calibrated by $F$, $\kappa$ is measured stretch by Proposition \ref{calibrated means measured stretch}, hence is a sublamination of $\lambda_\rho$.
\end{proof}

Another consequence of the decomposition of laminations is that the extreme points of $\rho^*$ are represented by indecomposable laminations.
Recall that a point $\alpha$ of a convex set $S$ is \dfn{extreme} if $\alpha$ cannot be written as the convex combination of two distinct members of $S$.

\begin{lemma}\label{extreme points are closed under sublaminations}
Let $\alpha$ be an extreme point of $\rho^*$, and let $\kappa$ be a measured stretch lamination in $\alpha$.
Then any sublamination of $\kappa$ represents a scalar multiple of $\alpha$.
\end{lemma}
\begin{proof}
By replacing $\kappa$ with a proper sublamination if necessary, we may assume that $\kappa$ is not a foliation.
Let $\zeta$ be a sublamination of $\kappa$.
By Theorem \ref{MorganShelan} and the fact that the leaves of a parallel family of closed leaves are all homologous, after replacing $\zeta$ with a sublamination of $\zeta$, we may assume that $\zeta$ is a clopen parallel family of closed leaves, or is an exceptional indecomposable sublamination of $\kappa$.
Since $\kappa$ is the linear combination of finitely many such clopen sublaminations, we may write $\alpha$ as a convex combination of $\beta_1, \dots, \beta_k$ where the $\beta_i$ are the (normalized to mass $1$) homology classes of clopen sublaminations of $\lambda$.
But $\beta_i \in \rho^*$, so $\beta_i = \alpha$, hence $[\zeta] = \alpha$.
\end{proof}

\begin{proposition}\label{extreme points are indecomposable}
Let $\alpha$ be an extreme point of $\rho^*$. Then $\alpha \in \rho^*_{\rm exc}$.
\end{proposition}
\begin{proof}
By Proposition \ref{enough measures in canonical lamination}, there exists a measured stretch lamination $\kappa$ representing $\alpha$.
By Theorem \ref{MorganShelan}, $\kappa$ has an indecomposable sublamination $\zeta$.
By Lemma \ref{extreme points are closed under sublaminations}, possibly after rescaling the transverse measure, $\zeta$ is a representative of $\alpha$.
Since any tight representative $F$ of $\rho$ calibrates $\kappa$, $F$ also calibrates $\zeta$, so by Proposition \ref{calibrated means measured stretch}, $\zeta$ is a measured stretch sublamination of $\lambda_\rho$.
\end{proof}
%%%%%%%%%%%%%%%%%%%%%%%%
\subsection{Convexity of the stable unit ball}\label{convexity sec}
Let $M$ be a closed oriented manifold of dimension $\leq 7$.
Auer and Bangert \cite{Auer01} claimed certain results concerning the convex structure of the stable unit ball
$$B := \{\alpha \in H_{d - 1}(M, \RR): \Mass(\alpha) \leq 1\}.$$
Here we show that some of these results follow from the structure theory of the canonical lamination.

Recall that a \dfn{flat} $S \subset \partial B$ is a set such that, for some supporting hyperplane $H$ of $B$, $S = H \cap B$.
Thus $B$ is strictly convex iff every flat is a point.

\begin{lemma}
Suppose that $S \subset \partial B$ is a flat of the stable unit ball $B \subset H_{d - 1}(M, \RR)$.
Then there exists $\rho$ in the costable unit sphere of $H^{d - 1}(M, \RR)$ such that $S \subseteq \rho^*$.
\end{lemma}
\begin{proof}
Since $S$ is convex, $\partial S$ is topologically a sphere, so $\partial S$ admits a Borel probability measure $\nu$ of full support.
Then we take the vector-valued integral 
$$\beta := \int_{\partial S} \alpha \dif \nu(\alpha),$$
thus $\beta \in S$ by convexity.
By the Hanh-Banach theorem (as in (\ref{Federer duality})), there exists $\rho \in H^{d - 1}(M, \RR)$ such that $\beta \in \rho^*$.

We claim that $\partial S \subseteq \rho^*$.
If not, then by continuity of $\alpha \mapsto \langle \rho, \alpha\rangle$, there is a positive measure set of $\partial S$ on which $\langle \rho, \cdot\rangle < 1$, hence
$$\Comass(\rho) = \langle \rho, \beta\rangle = \int_{\partial S} \langle \rho, \alpha\rangle \dif \nu(\alpha) < \Comass(\rho),$$
a contradiction.
Since $\rho^*$ is convex, it follows that $S \subseteq \rho^*$.
\end{proof}

Recall that the exterior product on the cohomology ring $H^\bullet(M, \RR)$ induces, by Poincar\'e duality, an \dfn{intersection product}
\begin{align*}
H_{d - k}(M, \RR) \times H_{d - \ell}(M, \RR) &\to H_{d - k - \ell}(M, \RR) \\
(\alpha, \beta) &\mapsto \alpha \cdot \beta.
\end{align*}

\begin{proposition}\label{flats are nonintersecting}
Let $B$ be the stable unit ball of $H_{d - 1}(M, \RR)$.
Suppose that $S \subset \partial B$ is a flat, and $\alpha, \beta \in S$. Then $\alpha \cdot \beta = 0$.
\end{proposition}
\begin{proof}
Let $\rho$ be the cohomology class dual to $S$ given by (\ref{flats duality}).
By Proposition \ref{enough measures in canonical lamination}, there exist measured stretch sublaminations $\kappa_\alpha, \kappa_\beta$ of $\lambda_\rho$, of classes $\alpha, \beta$.
Let $\dif u_\alpha, \dif u_\beta$ be their Ruelle-Sullivan currents, and suppose that $x$ is in the union of their supports.
If $N$ denotes the leaf of $\lambda_\rho$ containing $x$, then for $\sigma = \alpha, \beta$,
$$\dif u_\sigma(x) = \normal_N^\flat(x) \mu_\sigma(x)$$
where $\mu_\sigma$ is the positive Radon measure induced on $M$ by the transverse measure to $\kappa_\sigma$ \cite[Lemma 3.1]{BackusCML}.
In particular, $\dif u_\alpha|_{\supp \dif u_\beta}$ is a scalar multiple of $\dif u_\beta$, so $\dif u_\alpha \wedge \dif u_\beta = 0$, hence $\alpha \cdot \beta = 0$.
\end{proof}

Although the definitions of the stable unit ball $B$ and the intersection product depend on the Riemannian metric on $M$, we conclude a purely topological condition for strict convexity of $B$.

\begin{corollary}\label{condition for strict convexity}
Suppose that the kernel of the wedge product 
\begin{equation}\label{wedge product}
\wedge: H^1(M, \RR) \otimes H^1(M, \RR) \to H^2(M, \RR)
\end{equation}
is spanned by symmetric tensors $\theta \otimes \theta$, $\theta \in H^1(M, \RR)$.
Then the stable unit ball of $H_{d - 1}(M, \RR)$ is strictly convex.
\end{corollary}
\begin{proof}
We prove the contrapositive.
If $H_{d - 1}(M, \RR)$ does not have a strictly convex unit ball, then by Proposition \ref{flats are nonintersecting} there exist linearly independent $\alpha, \beta \in H_{d - 1}(M, \RR)$ such that $\alpha \cdot \beta = 0$.
Dually, this means that we can find linearly independent $\theta, \omega \in H^1(M, \RR)$ such that $\theta \otimes \omega \in \ker(\wedge)$.
\end{proof}

%%%%%%%%%%%%%%%%%%%%%%%%%
\section{Worked examples}
\subsection{Fiber bundles over \texorpdfstring{$\Sph^1$}{the circle}}
As a demonstration that in practice it is sometimes possible to explicitly compute the canonical lamination, let us do so when $M$ is a locally warped product with $\Sph^1$.

Let $(N, h)$ be a closed Riemannian manifold of dimension $d - 1$, and assume that we have an oriented fiber bundle $N_x \to M \to \Sph^1_\theta$ with a warped product metric 
$$g = \dif \theta^2 + f(\theta)^2 h.$$
Let $\theta_0$ be a minimizer of $f$, and let $F := \sqrt{\det h} \dif x$ denote the area form on $(N, h)$.
By rescaling $h$ and $f$, we may assume that $f(\theta_0) = 1$ and $\vol(N, h) = 1$.
In particular, the fiber $N_\theta$ of $\theta$ has area
$$\Mass(N_\theta) = \int_N f(\theta)^{d - 1} F = f(\theta)^{d - 1}.$$
Since $h$ is a function of $x$ but not $\theta$, $\dif F = 0$.
Moreover,
\begin{equation}\label{fiber bundle sample comass}
|F|(\theta, x) = \frac{1}{f(\theta)^{d - 1}}
\end{equation}
which attains its maximum $1$ on $N_{\theta_0}$, so $F$ is a calibration, and $N_{\theta_0}$ is $F$-calibrated.
In particular, the cohomology class $\rho := [F]$ has costable norm $1$.

\begin{proposition}
With notation as above, $F$ is tight, the leaves of $\lambda_\rho$ are exactly the fibers $N_\theta$ such that $f(\theta) = 1$, and $\rho^*$ is the convex hull of homology classes $[N_\theta]$ such that $f(\theta) = 1$.
\end{proposition}
\begin{proof}
We first compute
$$\star F = \frac{\dif \theta}{f(\theta)^{d - 1}}$$
and combine this with (\ref{fiber bundle sample comass}) to get
$$\dif(|F|^{p - 2} \star F) = \dif\left(\frac{\dif \theta}{f(\theta)^{(p - 1)(d - 1)}}\right) = 0$$
so $F$ is $p$-tight for every $1 < p < \infty$ and hence is tight.
Since $F$ is the area form for a complete hypersurface $L \subset M$ iff $L = N_\theta$ for some $\theta$ such that $f(\theta) = 1$, we deduce that $\lambda_\rho$ is a sublamination of the lamination $\kappa$ whose leaf set is $\{N_\theta: f(\theta) = 1\}$.
On the other hand, $\kappa$ admits a transverse probability measure $\mu$, because the set $\{f = 1\}$ is a closed subset of $\Sph^1$ and therefore is the support of some Borel probability measure.
So $(\kappa, \mu)$ is a measured stretch lamination for $\rho$, and therefore $\kappa$ is a sublamination of $\lambda_\rho$.
Therefore $\lambda_\rho = \kappa$.

We next compute $\rho^*$.
Let $\kappa$ be a measured stretch lamination for $\rho$; then every leaf of $\kappa$ is $N_\theta$ for some $\theta$ such that $f(\theta) = 1$; for such a $\theta$, the stable norm is $\Mass([N_\theta]) = 1$.
Therefore $[\kappa]$ is a convex combination of the classes $[N_\theta]$.
Since $N_\theta$ is itself a measured stretch lamination for $\rho$ if $f(\theta) = 1$, $\rho^*$ is the convex hull of $\{[N_\theta]: f(\theta) = 1\}$.
In particular, if the stable unit ball is strictly convex (so $\rho^*$ is singleton), any two fibers $N_\theta$ with $f(\theta) = 1$ must be homologous.
\end{proof}

%%%%%%%%%%%%%%
\subsection{Generic hyperbolic surfaces}

%%%%%%%%%%%%%%
\subsection{A canonical lamination with nonmeasurable leaves}

%%%%%%%%%%%%%
\subsection{The injectivity radius of \texorpdfstring{$\Hyp^3/\Gamma$}{a closed hyperbolic threefold}}

%%%%%%%%%%%%
\subsection{The stable norm on a homotopy torus}
The following example is due to Auer and Bangert \cite{Auer01} but we include it for completeness.

\begin{proposition}\label{torus convex}
Suppose that $M$ is homotopic to a torus.
Then the stable unit ball of $H_{d - 1}(M, \RR)$ is strictly convex.
\end{proposition}
\begin{proof}
The cohomology ring $H^\bullet(M, \RR)$ is isomorphic to the exterior algebra of $\RR^d$.
In particular, the kernel of (\ref{wedge product}) is spanned by symmetric tensors, so by Corollary \ref{condition for strict convexity}, the stable unit ball of $H_{d - 1}(M, \RR)$ is strictly convex.
\end{proof}


%%%%%%%%%%%%%%%%%%%%%%%%%
\section{Concluding remarks}\label{open problems}
\subsection{Generalizations}
In this paper I have only dealt with closed Riemannian manifolds $M$ of dimension $d \leq 7$, and submanifolds of codimension $c := 1$.
In the prequel paper \cite{BackusCML} I have only studied the interior behavior of functions of least gradient, but moreover, the statements of the main theorems would be significantly more involved in a more general setting.

We cannot easily weaken the assumption $d \leq 7$, since the double-napped Simons cone defines a function of least gradient on $\RR^8$ which does not admit a lamination structure \cite{BackusCML}.
Similarly, if the codimension $c \geq 2$, then Liu recently constructed homologically minimizing submanifolds $N$ which do not admit calibrations, even if $d = 3$ or $M = \CC \PP^2$ with a perturbation of the Fubini-Study metric \cite{liu2023homologically}.
Liu also showed that if $d \geq 8$ and $c = 1$, then there exist smooth homologically minimizing hypersurfaces with no smooth calibration; the regularity of best comass calibrations remains open if $d \leq 7$.

After replacing Poincar\'e duality with Lefschetz duality and imposing boundary conditions, I expect that the results of this paper go through if $M$ is a compact manifold with strictly mean-convex boundary $\partial M$.
Under that assumption, a version of the max-flow min-cut theorem has been claimed in the physics literature \cite[Appendix A]{Freedman_2016}.
To see that convexity is necessary, let $\theta$ be the latitude on $\Sph^2$, and let $M := \{|\theta| \leq \pi/4\}$; then any function of least gradient which extends the boundary data 
$$h(\theta, \phi) := \begin{cases} 1, \text{ if } \theta = \pi/4 \\ 0, \text{ if } \theta = -\pi/4\end{cases}$$
is constant on the interior, hence does not induce a lamination.

By replacing the boundary components of $M = \{|\theta| \leq \pi/4\}$ by cusps, we see that if $M$ has infinite ends, then it is possible for $H_{d - 1}(M, \RR) \neq 0$ but the stable seminorm to be identically $0$.
Freedman and Headrick conjectured that a max-flow min-cut theorem should hold for a manifold with infinite ends $M$ if there exist compact manifolds $M_i$ with mean-convex boundary, such that $(M_i)$ is a compact exhaustion of $M$ \cite[Appendix A]{Freedman_2016}.
This assumption clearly rules out the existence of cusps.

However, even assuming the existence of a compact exhaustion with mean-convex boundaries, I expect that extension of this paper to noncompact manifolds to be quite challenging.
If $H^{d - 1}(M, \RR)$ is infinite-dimensional, then its completion with respect to the costable norm is unlikely to be reflexive, so arguments involving the Hanh-Banach theorem will become precarious.
Moreover, already in the case of $\RR^d$, the behavior of functions of least gradient near infinity is rather involved \cite[\S4.4]{górny2021}.

%%%%%%%%%%%%%%%%%%%%%%%%%
\subsection{Motivation for the canonical lamination}\label{Teichmuller}
Our motivation for introducing the canonical lamination arose from an analogy with Thurston's approach to Teichm\"uller theory using best Lipschitz maps \cite{Thurston98}.
Given $\gamma \geq 2$, let $\widetilde{\mathscr M}_\gamma$ be the Teichm\"uller space of hyperbolic metrics on the closed surface $S_\gamma$ of genus $\gamma$.
Given $g, h \in \widetilde{\mathscr M}_\gamma$, let $\Lip(g, h)$ be the Lipschitz constant of a best Lipschitz map homotopic to
$$\id: (S_\gamma, g) \to (S_\gamma, h).$$ 
For a tangent vector $v \in T_g(\widetilde{\mathscr M}_\gamma)$, let $\Comass(v)$ be the partial derivative of $\log \Lip(g, \cdot)$ in the direction $v$.
This quantity, the \dfn{Thurston asymmetric norm}, is an asymmetric norm on $T_g(\widetilde{\mathscr M}_\gamma)$ obtained by solving an $L^\infty$ variational problem intimately tied to the structure of minimal laminations, so it is tempted to make an analogy between $T_g(\widetilde{\mathscr M}_\gamma)$ and $H^{d - 1}(M, \RR)$, where both vector spaces are equipped with the norm $\Comass$.
Two particularly salient pieces of evidence for the analogy are:
\begin{enumerate}
\item The unit spheres of the dual spaces of $T_g(\widetilde{\mathscr M}_\gamma)$ and $H^{d - 1}(M, \RR)$ can both be viewed as spaces of projective measured minimal laminations, whose norm is given by an $L^1$ (actually $BV$) variational problem \cite[Theorem 5.1]{Thurston98}.
\item In both cases, we can construct a canonical lamination. In Thurston's case, the canonical lamination is given by those geodesics which are maximally stretched by every best Lipschitz map homotopic to $\id_{S_\gamma}$ \cite[\S8]{Thurston98}. See also Conjecture \ref{chain recurrence}.
\end{enumerate}
However, one should not take this analogy too seriously.
A key feature of Thurston's theory is the Birman-Series theorem: the union of the supports of all geodesic laminations on $(S_\gamma, g)$ has Hausdorff dimension $0$.
As a corollary, for almost every $h \in \widetilde{\mathscr M}_\gamma$, the canonical lamination associated to $(g, h)$ is a closed geodesic \cite[\S10]{Thurston98}.
The analogue of the Birman-Series theorem is clearly not true in our case, and in fact, if $M$ is a square flat torus, then it is easy to see that every canonical lamination covers all of $M$.

Thurston's canonical lamination $\lambda$ is chain-recurrent, in the sense that traveling along the geodesics in $\lambda$ defines a chain-recurrent dynamical system.
This makes no sense for higher-dimensional laminations, but is equivalent to assert that Thurston's canonical lamination can be approximated by finite sums of closed geodesics \cite[\S9]{Gu_ritaud_2017}.
We conjecture that the analogous fact should hold for our canonical lamination:

\begin{conjecture}\label{chain recurrence}
Let $\rho \in H^{d - 1}(M, \RR)$, and let $\lambda_\rho$ be the canonical lamination.
Then it is possible to approximate $\lambda_\rho$ in Thurston's geometric topology\footnote{See \cite[\S1]{BackusCML} for the definition of Thurston's geometric topology in this setting.} by finite unions of closed minimal hypersurfaces.
\end{conjecture}

%%%%%%%%%%%%%%%%%%%
\subsection{Taut foliations and eikonal calibrations}
The following problem was suggested to me by Karen Uhlenbeck. 
There exist closed hyperbolic $3$-manifolds which admit taut foliations; in that case, \emph{after changing the metric} one may find a minimal foliation.
Thus one cannot rule out minimal foliations by a simple topological argument (as one could rule out geodesic foliations of closed hyperbolic surfaces).
However, if a minimal foliation exists, then it is natural to study the tight form which calibrates it.
This form satisfies a particularly strong form 
\begin{equation}\label{eikonal}
\begin{cases}\dif F = 0 \\ \dif(|F|^2) = 0\end{cases}
\end{equation}
of the Euler-Lagrange equation for tight forms which is analogous to the role of the eikonal equation
$$\dif(|\dif u|^2) = 0$$
in the study of the $\infty$-Laplace equation.
Global solutions of the eikonal equation are rather uncommon (for example, the Dirichlet problem for the eikonal equation on $\Ball^d$ is overdetermined), so this suggests a means to rule out the existence of minimal foliations:

\begin{conjecture}\label{Karen}
Let $\Gamma$ be the fundamental group of a closed hyperbolic $3$-manifold $M$.
Then there does not exist a solution of the eikonal system (\ref{eikonal}) on $\Hyp^3$ which is invariant under $\Gamma$.
In particular, there does not exist a minimal foliation on $M$.
\end{conjecture}






\printbibliography

\end{document}
