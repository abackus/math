\documentclass[reqno,11pt]{amsart}
\usepackage[letterpaper, margin=1in]{geometry}
\RequirePackage{amsmath,amssymb,amsthm,graphicx,mathrsfs,url,slashed,subcaption}
\RequirePackage[usenames,dvipsnames]{xcolor}
\RequirePackage[colorlinks=true,linkcolor=Red,citecolor=Green]{hyperref}
\RequirePackage{amsxtra}
\usepackage{cancel}
\usepackage{tikz, quiver, wrapfig}
%\usepackage[T1]{fontenc}

% \setlength{\textheight}{9.3in} \setlength{\oddsidemargin}{-0.25in}
% \setlength{\evensidemargin}{-0.25in} \setlength{\textwidth}{7in}
% \setlength{\topmargin}{-0.25in} \setlength{\headheight}{0.18in}
% \setlength{\marginparwidth}{1.0in}
% \setlength{\abovedisplayskip}{0.2in}
% \setlength{\belowdisplayskip}{0.2in}
% \setlength{\parskip}{0.05in}
%\renewcommand{\baselinestretch}{1.05}

\title{Laminations and calibrations as limiting solutions of $p$-Laplacian systems}
\author{Aidan Backus}
\address{Department of Mathematics, Brown University}
\email{aidan\_backus@brown.edu}
\date{\today}

\newcommand{\NN}{\mathbf{N}}
\newcommand{\ZZ}{\mathbf{Z}}
\newcommand{\QQ}{\mathbf{Q}}
\newcommand{\RR}{\mathbf{R}}
\newcommand{\CC}{\mathbf{C}}
\newcommand{\DD}{\mathbf{D}}
\newcommand{\PP}{\mathbf P}
\newcommand{\MM}{\mathbf M}
\newcommand{\II}{\mathbf I}
\newcommand{\Hyp}{\mathbf H}
\newcommand{\Sph}{\mathbf S}
\newcommand{\Group}{\mathbf G}
\newcommand{\GL}{\mathbf{GL}}
\newcommand{\Orth}{\mathbf{O}}
\newcommand{\SpOrth}{\mathbf{SO}}
\newcommand{\Ball}{\mathbf{B}}

\newcommand*\dif{\mathop{}\!\mathrm{d}}

\DeclareMathOperator{\card}{card}
\DeclareMathOperator{\dist}{dist}
\DeclareMathOperator{\id}{id}
\DeclareMathOperator{\Hom}{Hom}
\DeclareMathOperator{\coker}{coker}
\DeclareMathOperator{\supp}{supp}
\DeclareMathOperator{\sech}{sech}
\DeclareMathOperator{\Teich}{Teich}
\DeclareMathOperator{\tr}{tr}

\newcommand{\Leaves}{\mathscr L}
\newcommand{\Lagrange}{\mathscr L}
\newcommand{\Hypspace}{\mathscr H}

\newcommand{\Chain}{\underline C}

\newcommand{\Two}{\mathrm{I\!I}}
\newcommand{\Ric}{\mathrm{Ric}}

\newcommand{\normal}{\mathbf n}
\newcommand{\radial}{\mathbf r}
\newcommand{\evect}{\mathbf e}
\newcommand{\vol}{\mathrm{vol}}

\newcommand{\diam}{\mathrm{diam}}
\newcommand{\Ell}{\mathrm{Ell}}
\newcommand{\inj}{\mathrm{inj}}
\newcommand{\Lip}{\mathrm{Lip}}
\newcommand{\MCL}{\mathrm{MCL}}
\newcommand{\Riem}{\mathrm{Riem}}

\newcommand{\Mass}{\mathbf M}
\newcommand{\Comass}{\mathbf L}

\newcommand{\Min}{\mathrm{Min}}
\newcommand{\Max}{\mathrm{Max}}

\newcommand{\dfn}[1]{\emph{#1}\index{#1}}

\renewcommand{\Re}{\operatorname{Re}}
\renewcommand{\Im}{\operatorname{Im}}

\newcommand{\loc}{\mathrm{loc}}
\newcommand{\cpt}{\mathrm{cpt}}

\def\Japan#1{\left \langle #1 \right \rangle}

\newtheorem{theorem}{Theorem}[section]
\newtheorem{badtheorem}[theorem]{``Theorem"}
\newtheorem{prop}[theorem]{Proposition}
\newtheorem{lemma}[theorem]{Lemma}
\newtheorem{sublemma}[theorem]{Sublemma}
\newtheorem{proposition}[theorem]{Proposition}
\newtheorem{corollary}[theorem]{Corollary}
\newtheorem{conjecture}[theorem]{Conjecture}
\newtheorem{axiom}[theorem]{Axiom}
\newtheorem{assumption}[theorem]{Assumption}

\newtheorem{mainthm}{Theorem}
\renewcommand{\themainthm}{\Alph{mainthm}}

\newtheorem{claim}{Claim}[theorem]
\renewcommand{\theclaim}{\thetheorem\Alph{claim}}
% \newtheorem*{claim}{Claim}

\theoremstyle{definition}
\newtheorem{definition}[theorem]{Definition}
\newtheorem{remark}[theorem]{Remark}
\newtheorem{example}[theorem]{Example}
\newtheorem{notation}[theorem]{Notation}

\newtheorem{exercise}[theorem]{Discussion topic}
\newtheorem{homework}[theorem]{Homework}
\newtheorem{problem}[theorem]{Problem}

\makeatletter
\newcommand{\proofpart}[2]{%
  \par
  \addvspace{\medskipamount}%
  \noindent\emph{Part #1: #2.}
}
\makeatother



\numberwithin{equation}{section}


% Mean
\def\Xint#1{\mathchoice
{\XXint\displaystyle\textstyle{#1}}%
{\XXint\textstyle\scriptstyle{#1}}%
{\XXint\scriptstyle\scriptscriptstyle{#1}}%
{\XXint\scriptscriptstyle\scriptscriptstyle{#1}}%
\!\int}
\def\XXint#1#2#3{{\setbox0=\hbox{$#1{#2#3}{\int}$ }
\vcenter{\hbox{$#2#3$ }}\kern-.6\wd0}}
\def\ddashint{\Xint=}
\def\dashint{\Xint-}

\usepackage[backend=bibtex,style=alphabetic,giveninits=true]{biblatex}
\renewcommand*{\bibfont}{\normalfont\footnotesize}
\addbibresource{best_curl.bib}
\renewbibmacro{in:}{}
\DeclareFieldFormat{pages}{#1}

\newcommand\todo[1]{\textcolor{red}{TODO: #1}}


\begin{document}
\begin{abstract}
We study the behavior of $q$-harmonic functions and their $p$-harmonic conjugates in the limit as $q \to 1$, where $(p, q)$ is a H\"older conjugate pair.
The $1$-Laplacian is already known to give rise to laminations by minimal hypersurfaces; we show that the limiting $p$-harmonic conjugates converge to calibrations $F$ of the laminations.
Moreover, we show that the laminations which are calibrated by $F$ are exactly those which arise from the $1$-Laplacian.

We also explore the limiting dual problem as a model problem in the $L^\infty$ calculus of variations.
We show that under an integrability condition, smooth solutions of the dual variational problem are equivalent to smooth solutions of a PDE which generalizes the $\infty$-Laplacian.
But we show that in the nonintegrable case, there exist smooth solutions of the PDE which are not minimizers.
\end{abstract}

\maketitle

%%%%%%%%%%%%%%%%%%%%%%%%%%%%%%%%%%%%%%%%%%%%%%%%%%%%%%%
\section{Introduction}
\subsection{Motivation: geodesic laminations on hyperbolic surfaces}
Let $M$ be a closed hyperbolic surface.
Daskalopolous and Uhlenbeck have studied $p$-harmonic maps $v_p: M \to \Sph^1$ in the limit $p \to \infty$; these maps converge to an $\infty$-harmonic map which attains its Lipschitz constant on a geodesic lamination $\lambda$ \cite{daskalopoulos2020transverse}.
Given such a map, we can introduce its \dfn{$q$-harmonic conjugate} $u_q$, which satisfies a renormalized version of the \dfn{$p$-Cauchy-Riemann equation}
$$\dif u_q = |\dif v_p|^{p - 2} \star \dif v_p$$
where $(p, q)$ are a \dfn{H\"older pair}, thus $p^{-1} + q^{-1} = 1$.
As $q \to 1$, $\dif u_q$ converges to a Radon measure $\dif u$ which is transverse to a sublamination of $\lambda$.
Moreover, $\dif u$ is locally the derivative of a function $u$ of \dfn{least gradient}, meaning that whenever $\dif w$ is an exact $1$-form, 
$$\int_M \star |\dif u| \leq \int_M \star |\dif u + \dif w|.$$
The goal of this paper is to generalize the above duality to codimension $1$ laminations of much more general closed Riemannian manifolds.

%%%%%%%%%%%
\subsection{Tight forms and least gradient functions}
Throughout this paper, let $M$ be a closed Riemannian manifold of dimension $d \geq 2$.
Let $(p, q)$ be a H\"older pair such that $1 < p < \infty$.
Motivated by the $p$-Laplace equation $\dif^*(|\dif v|^{p - 2} \dif v) = 0$, we introduce \dfn{$p$-tight} forms, which are $d-1$-forms $F$ which solve the system of PDE
$$\begin{cases}\dif F = 0 \\ \dif^*(|F|^{p - 2} F) = 0\end{cases}.$$
Given a $p$-tight form, the $\pi_1(M)$-equivariant function $u$ on the universal cover $\tilde M$ such that
\begin{equation}\label{harmonic conjugate}
\dif u = (-1)^{d - 1} |F|^{p - 2} \star F
\end{equation}
is $q$-harmonic -- in other words, $u$ is a solution of the $q$-Laplace equation 
$$\dif^*(|\dif u|^{q - 2} \dif u) = 0.$$
The $p$-tight forms can be thought as the Fenchel duals of their $q$-harmonic conjugates; for the Dirichlet problem, this is a standard result \cite[Chapter IV]{Ekeland99}, but for our equivariant problem we defer a careful discussion of the Fenchel duality to a later work \cite{BackusBest2}, in which we shall study the duality in a broader context which ties it into the duality theory of Thurston's asymmetric norm on Teichm\"uller space \cite{Thurston98}.

Our first theorem is a summary of Propositions \ref{existence infinity} and \ref{existence 1}, and describes the limiting behavior of the $p$-tight forms and their $q$-harmonic conjugates as $p \to \infty$.

\begin{mainthm}\label{existence of infinity tight forms}
Let $\rho \in H^{d - 1}(M, \RR)$ be a nonzero cohomology class.
Let $(F_p, u_q)$ be the family of dual pairs of $p$-tight forms and $q$-harmonic functions, suitably normalized, with $[F_p] = \rho$ and $(p, q)$ ranging over H\"older pairs with $1 < p < \infty$.
Then there exists a pair $(F, u)$ such that as $p \to \infty$ along a subsequence, $F_p \to F$ weakly in $L^r$ for any $1 \leq r < \infty$, and $u_q \to u$ weakly in $BV$, with the following properties:
\begin{enumerate}
\item For any Lipschitz $d - 2$-form $B$, 
\begin{equation}\label{best comass}
\|F\|_{L^\infty} \leq \|F + \dif B\|_{L^\infty}.
\end{equation}
\item $u$ has least gradient and is nonconstant.
\item The product of distributions $\dif u \wedge F$ is well-defined, and
\begin{equation}\label{max flow min cut}
\|F\|_{L^\infty} \star |\dif u| = \dif u \wedge F.
\end{equation}
\end{enumerate}
\end{mainthm}

If a closed $d - 1$-form $F$ is a limit of $p$-tight forms $F_p$, as in Theorem \ref{existence of infinity tight forms}, we say that $F$ is a \dfn{tight} form.
Thus the content of Theorem \ref{existence of infinity tight forms} is that every tight $d - 1$-form minimizes its $L^\infty$ norm, and has a dual function $u$ of least gradient, in the sense that $(F, u)$ satisfies the relation (\ref{max flow min cut}).
This relation can be viewed as the assertion that $F$ ``calibrates" $u$ in some sense, and allows us to show that $u$ has least gradient without a careful analysis of the behavior of $u_q$ as $q \to 1$ which appeared in the previous works \cite{daskalopoulos2020transverse,daskalopoulos2022,Mazon14}.

The calibration condition (\ref{max flow min cut}) is implicit in work of Maz\'on, Rossi, and Segura de Le\'on on the Dirichlet problem for functions of least gradient \cite{Mazon14}.
In our language, they showed that every function $u$ of least gradient on a domain in $\RR^d$ admits a closed $d - 1$-form $F$ satisfying (\ref{max flow min cut}).
Their proof actually shows that $F$ is tight, so every function of least gradient on a domain has a dual tight $d - 1$-form.
We stress, however, that the proof of Theorem \ref{existence of infinity tight forms} does not go through for domains with a boundary, as the mass of $\dif u_q$ may concentrate along the boundary as $q \to 1$; see \S\ref{boundaries bad} for an explicit counterexample.

%%%%%%%%%%%%
\subsection{Calibrations of measured laminations}
To give a more geometric interpretation of Theorem \ref{existence of infinity tight forms}, we assume that the ambient dimension $d \leq 7$.
Then every nonconstant function $u$ of least gradient induces a measured oriented lamination $\lambda_u$ by homologically minimizing hypersurfaces \cite{BackusCML}.
The leaves are the level sets of $u$, and $\dif u$ is the transverse measure.

\begin{definition}
Let $\rho \in H^{d - 1}(M, \RR)$, let $F$ be a tight representative of $\rho$, and let $u$ be a function of least gradient satisfying (\ref{max flow min cut}).
We call $\lambda_u$ a \dfn{measured stretch lamination} for $\rho$.
\end{definition}

Let $\rho \in H^{d - 1}(M, \RR)$ be nonzero.
It is straightforward to check that for every $1 < p < \infty$ there exists a unique $p$-tight representative of $\rho$, so by Theorem \ref{existence of infinity tight forms}, there is a measured stretch lamination for $\rho$.

For convenience, we want to normalize $\rho$. 
We introduce its \dfn{costable norm}
$$\Comass(\rho) := \inf_{[F] = \rho} \|F\|_{L^\infty},$$
and say that any $F$ which attains the infimum has \dfn{best comass}.
Every tight form has best comass, and $F$ is a calibration which calibrates some current, then $F$ has best comass and its cohomology class has costable norm $1$.
We then have the following consequence of (\ref{max flow min cut}), which is the content of \S\ref{proof of Theorem B}.

\begin{mainthm}\label{lams are calibrated}
Suppose that $M$ is a closed Riemannian manifold of dimension $d \leq 7$.
Let $\rho \in H^{d - 1}(M, \RR)$ satisfy $\Comass(\rho) = 1$.
Let $\kappa$ be a measured stretch lamination for $\rho$.
Let $F$ be a best comass representative of $\rho$.
Then $\kappa$ is $F$-calibrated, and for $\lambda$ ranging over measured oriented laminations,
\begin{equation}\label{duality between stable and comass}
1 = \sup_\lambda \frac{\langle \rho, [\lambda]\rangle}{\Mass(\lambda)} = \frac{\langle \rho, [\kappa]\rangle}{\Mass(\kappa)}.
\end{equation}
Conversely, if a measured oriented lamination $\lambda$ attains the maximum in (\ref{duality between stable and comass}), then $\lambda$ is a measured stretch lamination.
\end{mainthm}

Bangert and Cui showed that every continuous best comass calibration calibrates some measured oriented lamination \cite{bangert_cui_2017}.
This result improves on their work by allowing $F$ to be merely measurable; it also implies that the lamination is Lipschitz.
Moreover, if one has an approximation of $F$ by $p$-tight forms, it allows one to compute the calibrated lamination $\kappa$ more or less explicitly, by using the formula (\ref{harmonic conjugate}) and taking limits.
In the sequel paper \cite{BackusBest2} we shall interpret (\ref{duality between stable and comass}) as a max flow/min cut principle; a similar max flow/min cut principle was predicted by Thurston for certain geodesic laminations on closed hyperbolic manifolds \cite{Thurston98}.

The pairing $\langle \rho, [\lambda]\rangle$ is defined on the level of homology, but if $\lambda$ has sufficient regularity (which holds for any measured stretch lamination), then it has a more appealing interpretation (\ref{coarea formula on laminations}): it is the integration of any representative $F$ of $\rho$ along the leaves of $\lambda$, which are then summed over using the transverse measure.
Meanwhile $\Mass(\lambda)$ is the total area of the leaves of $\lambda$, summed over using the transverse measure.
In \S\ref{L infinity calibrations} we prepare for the proof of Theorem \ref{lams are calibrated} by establishing several related results concerning integration along laminations which may be of independent interest.

In upcoming work we shall embed the measured stretch laminations into a lamination (without a transverse measure) which we call the \dfn{canonical lamination} for $\rho$ \cite{BackusBest2}.
This lamination is the largest lamination whose leaves are locally calibrated by every best comass representative of $\rho$, and exhibits strongly analogous behavior with Thurston's canonical lamination on a closed hyperbolic surface \cite{Thurston98}.

Theorem \ref{lams are calibrated} is based on the fact that the level sets of a function of least gradient are area-minimizing hypersurfaces, but one can seek to formulate such a result for eigenfunctions of the $1$-Laplacian instead.
The superlevel sets of the first eigenfunction are \dfn{Cheeger sets} -- sets $U$ which maximize the isoperimetric ratio $\Mass(\partial U)/\vol(U)$ \cite{Kawohl2003}.
The dual $d - 1$-forms $F$ are not closed, but instead satisfy $\dif F \geq 0$ \cite{Grieser05}.
It would be interesting to study the $L^\infty$ one-sided variational problem that $F$ solves in this setting, but we shall not attempt to do so.

%%%%%%%%%%%%%%%%%
\subsection{The \texorpdfstring{$L^\infty$ variational}{L-infinity variational} model system}
In \S\ref{infinityMax} of the paper we scrutinize the $L^\infty$ variational problem defining tight forms.
At present there is not a suitable theory of viscosity solutions of PDEs for vector-valued maps, and in general, one does not expect our tight forms to be much more regular than $L^\infty$.
However, we formally derive what analytic properties tight forms \emph{should} have, in the tradition of various other papers \cite{Barron2001,Aronsson67,Sheffield12} on the $L^\infty$ calculus of variations.

To be more precise, we study $C^1$ solutions of the PDE 
\begin{equation}\label{tight Einstein}
\begin{cases}\dif F = 0, \\
	(\nabla_i F_{j_1 \cdots j_{d - 1}}) F^{j_1 \cdots j_{d - 1}} {F^i}_{k_1 \cdots k_{d - 2}} = 0.
\end{cases}
\end{equation}
We derive this equation as the formal limit of the PDE solved by the $p$-tight forms.
Therefore tight forms are variational solutions of (\ref{tight Einstein}).
We are particularly interested in this equation not just because it is formally solved by tight forms, but because it seems to be a model problem for $L^\infty$ variational systems, in particular the PDE solved by the tight maps of Sheffield and Smart \cite{Sheffield12} and the closely related $\infty$-harmonic maps of Daskalopolous and Uhlenbeck \cite{daskalopoulos2022,daskalopoulos2023}.
Such maps are solutions to the optimal Lipschitz extension problem between manifolds.

When $d = 2$, (\ref{tight Einstein}) asserts that $F = \dif v$, where $v$ is $\infty$-harmonic:
$$(\nabla_i \partial_j v) \partial^i v \partial^j v = 0.$$
In that case it is well known that $\ker(\star F)$ integrates to a geodesic foliation \cite[Proof of Theorem 1.5]{Sheffield12}.
In general, (\ref{tight Einstein}) can be interpreted as asserting that the distribution $\ker(\star F)$ is ``calibrated" by $F$, though $\ker(\star F)$ need not be integrable.
If it is integrable, then we show in \S\ref{EL interpretation} the following variational interpretation which generalizes the interpretation of $\infty$-harmonic functions as absolute minimizing Lipschitz.

\begin{mainthm}\label{tight are absolute minimizers}
Let $F$ be a closed $C^1$ $d - 1$-form on a compact Riemannian manifold (possibly with boundary). Then:
\begin{enumerate}
\item If, for every sufficiently ball $B$,
\begin{equation}\label{absolute minimizer}
\|F\|_{C^0(B)} = \|F\|_{C^0(\partial B)},
\end{equation}
then $F$ solves (\ref{tight Einstein}).
\item Suppose that $\ker(\star F)$ is a singular integrable distribution. If $F$ solves (\ref{tight Einstein}), then for every sufficiently small ball $B$, (\ref{absolute minimizer}) holds.
\end{enumerate}
\end{mainthm}

In \S\ref{nonintegrability}, we show that the assumption that $B$ is a small ball cannot be removed.
We also show that the rather surprising fact that if the integrability hypothesis is removed, then $F$ is not a minimizer even globally.
These defects appear to be generic features of the theory of $L^\infty$ variational systems, which are altogether unlike the behavior of the scalar $\infty$-Laplacian.

%%%%%%%%%%%%%%%%%%%%%%
\subsection{Notation}
The operator $\star$ is the Hodge star on $M$, thus $\star 1$ is the Riemannian measure of $M$.
We denote the musical isomorphisms by $\sharp, \flat$.
To avoid confusion, we write $H^\ell$ for de Rham cohomology, but never a Sobolev space, which we instead denote $W^{\ell, p}$.
The manifold $\Ball^d$ is the unit ball in $\RR^d$, $\Sph^d$ is the unit sphere in $\RR^{d + 1}$, and $\Hyp^d$ is the hyperbolic space.

The $\delta$-dimensional Hausdorff measure is $\mathcal H^\delta$, normalized so that if $\delta$ is an integer, then $\mathcal H^\delta$ is $\delta$-dimensional Riemannian measure.
If $\tau$ is a $\delta$-rectifiable set, we write $|\tau| := \mathcal H^\delta(\tau)$ and $\dif S_\tau := \mathcal H^\delta|_\tau$.

The sheaf of $\ell$-forms is denoted $\Omega^\ell$, and the sheaf of closed $\ell$-forms is denoted $\Omega^\ell_{\rm cl}$.
We are often interested in those sections of a sheaf $\mathscr F$ of a certain regularity, so we will write, for example, $L^p(U, \mathscr F)$.
We assume that $\ell$-forms are $L^1_\loc$, but \emph{not} that they are continuous; hence $\dif$ must be meant in the sense of distributions.
To avoid confusion, we write $H^\ell$ for de Rham cohomology, but never a Sobolev space, which we instead denote $W^{\ell, p}$.

In Appendix \ref{GMT appendix} we establish certain (slightly nonstandard) conventions for working with currents.

%%%%%%%%%%%%%%%%%%%%%%
\subsection{Acknowledgements}
I would like to thank Georgios Daskalopolous and Karen Uhlenbeck for suggesting this project and providing helpful comments, and for providing me with a draft copy of \cite{daskalopoulos2023}.
I would also like to thank Bernd Kawohl for suggesting the references \cite{Kawohl2003, Grieser05}, Anatole Gaudin for suggesting the reference \cite{Costabel2010}, and Taylor Klotz for suggesting the Hopf fibration and the reference \cite{Peralta_Salas_2023} in Example \ref{integrability needed}.

This research was supported by the National Science Foundation's Graduate Research Fellowship Program under Grant No. DGE-2040433.

%%%%%%%%%%%%%%%%%
\section{Preliminaries on least gradient functions}\label{prevResults}
Following the references \cite{Mazon14, BackusCML}, we consider the variational problems whose Euler-Lagrange equation is, at least at the formal level, the $1$-Laplacian
\begin{equation}\label{1Laplacian}
\dif^*\left(\frac{\dif u}{|\dif u|}\right) = 0.
\end{equation}
A suitable notion of weak solution for (\ref{1Laplacian}), at least for the Dirichlet problem, was introduced by Maz\'on, Rossi, and Segura de L\'eon \cite{Mazon14}; it essentially asserts that the level sets of $u$ are calibrated.

Traditionally, authors have studied the Dirichlet problem for the $1$-Laplacian.
We will instead be interested in the topological Neumann problem, which we now formulate.
Let $M$ be a closed Riemannian manifold with fundamental group $\Gamma$ and universal cover $\tilde M \to M$.
By Poincar\'e duality and the Hurcewiz theorem, we have canonical isomorphisms
\begin{equation}\label{Poincare Hurcewiz}
H_{d - 1}(M, \RR) = H^1(M, \RR) = \Hom(\Gamma, \RR).
\end{equation}
Given a representation $\alpha: \Gamma \to \RR$, which we identify with a class in $H_{d - 1}(M, \RR)$ using (\ref{Poincare Hurcewiz}), we will be interested in $\alpha$-equivariant functions $f: \tilde M \to \RR$, namely those which satisfy (for each $\gamma \in \Gamma$)
$$f(\gamma x) = f(x) + \alpha(\gamma).$$
If $f$ is $\alpha$-equivariant, then $\dif f$ drops to a current on $M$, which we also call $\dif f$, and has homology class $\alpha$.

In order to justify and explain the notation in the next few definitions, we refer the reader to Appendix \ref{GMT appendix}.

\begin{definition}
Let $\alpha: \Gamma \to \RR$ be a representation.
An $\alpha$-equivariant function $u \in BV_\loc(\tilde M)$ has \dfn{least gradient} if the current $\dif u$ on $M$ satisfies, for every function $v \in BV(M)$, 
$$\Mass(\dif u) \leq \Mass(\dif u + \dif v).$$
The \dfn{stable norm} $\Mass(\alpha)$ of $\alpha$ is the mass of $\dif u$ whenever $u$ has least gradient.
\end{definition}

\begin{definition}
We say that an $\alpha$-equivariant function $u \in BV_\loc(\tilde M)$ is a \dfn{calibrated solution} of (\ref{1Laplacian}) if there exists a $L^\infty$ $d - 1$-form $F$ on $M$ such that
\begin{equation}\label{local calibration}
\begin{cases}
\|F\|_{L^\infty} \leq 1, \\
dF = 0, \\
\dif u \wedge F = \star |\dif u|.
\end{cases}
\end{equation}
\end{definition}

We informally refer to calibrated solutions of (\ref{1Laplacian}) as \dfn{$1$-harmonic functions}, though this terminology is not quite precise.
This formulation of calibrated solution is not worded the same as the formulation for the Dirichlet problem given by \cite{Mazon14}, but it is equivalent; their vector field $X$ is given by $(\star F)^\sharp$.
The quantity $\dif u \wedge F$ is well-defined by the coarea formula, since $\dif F = 0$.
A straightforward modification of \cite[Theorem 1.1]{Mazon14} (originally proven for the Dirichlet problem) gives:

\begin{theorem}\label{MazonRossi}
An $\alpha$-equivariant function $u \in BV_\loc(M)$ is a calibrated solution of (\ref{1Laplacian}) iff $u$ has least gradient.
\end{theorem}

We now give a geometric characterization of $1$-harmonic functions.
Fix an interval $I \subset \RR$ and a box $J \subset \RR^{d - 1}$.

\begin{definition}
A \dfn{laminar flow box} is a $C^0$ coordinate chart $F: I \times J \to M$ and a compact set $K \subseteq I$, such that for every $k \in K$, $F|_{\{k\} \times J}$ is a $C^1$ embedding, and the \dfn{leaf} $F(\{k\} \times J)$ is a $C^1$ complete hypersurface in $F(I \times J)$.
Two laminar flow boxes belong to the same \dfn{laminar atlas} if the transition maps between them send leaves to leaves.
\end{definition}

\begin{definition}
A \dfn{lamination} is a closed subset $S \subseteq M$, called its \dfn{support}, and a maximal laminar atlas $\mathscr A$, such that $S$ is the union of the leaves of $\mathscr A$.
A \dfn{foliation} is a lamination $\lambda$ with $\supp \lambda = M$.
\end{definition}

\begin{definition}
A lamination is
\begin{enumerate}
\item \dfn{Lipschitz} if its flow boxes are Lipschitz isomorphisms,
\item \dfn{oriented} if its transition maps are orientation-preserving, and
\item \dfn{minimal} if its leaves are minimal hypersurfaces.
\end{enumerate}
\end{definition}

\begin{definition}
A lamination $\lambda$ with atlas $(F_\alpha, K_\alpha)$ is \dfn{measured} if it is equipped with positive Radon measures $\mu_\alpha$ with $\supp \mu_\alpha = K_\alpha$, such that the transition maps $F_\beta^{-1} \circ F_\alpha$ are measure-preserving.
The \dfn{Ruelle-Sullivan current} of a measured oriented lamination $\lambda$ with atlas $(F_\alpha, K_\alpha, \mu_\alpha)$ is the $d-1$-current $T_\lambda$ satisfying, for any partition of unity $(\chi_\alpha)$ subordinate to the open cover $(F_\alpha(I \times J))$,
$$\int_M T_\lambda \wedge \varphi = \sum_\alpha \int_{K_\alpha} \int_{\{k\} \times J} F_\alpha^* (\chi_\alpha \varphi) \dif \mu_\alpha(k).$$
The \dfn{homology class} $[\lambda]$ and \dfn{mass} $\Mass(\lambda)$ of a measured oriented lamination $\lambda$ are the homology class and mass of its Ruelle-Sullivan current.
The measured oriented lamination $\lambda$ is \dfn{homologically minimizing} if
$$\Mass(\lambda) = \Mass([\lambda]).$$
\end{definition}

The notion of Ruelle-Sullivan current was introduced by \cite{Ruelle75} and studied in the context of geodesic laminations in \cite[\S8]{daskalopoulos2020transverse}.
The motivation of the definition is that if $\lambda$ is a $d - 1$-chain, then $T_\lambda$ is just integration along $\lambda$.

\begin{theorem}\label{1 harmonic is MOML}
Suppose that $d \leq 7$, and let $u \in BV_\loc(\tilde M)$ be a $\Gamma$-equivariant function of least gradient.
Then $\dif u$ is the Ruelle-Sullivan current of a measured oriented Lipschitz lamination on $M$, which is minimal and homologically minimizing.
\end{theorem}
\begin{proof}
There is a measured oriented Lipschitz minimal lamination $\tilde \lambda$ on $\tilde M$ for which $\dif u$ is Ruelle-Sullivan \cite[Theorem B]{BackusCML}.
Invariance of $\dif u$ under $\pi_1(M)$ implies that $\tilde \lambda$ drops to a lamination $\lambda$ on $M$ for which $\dif u$ is Ruelle-Sullivan.
Moreover, $\Mass(\lambda) = \Mass(\dif u) = \Mass([\dif u]) = \Mass([\lambda])$ since $u$ has least gradient.
\end{proof}


%%%%%%%%%%%%%%%%%%%%%%%%%%%%%
\section{Tight forms and functions of least gradient}\label{tight forms sec}
\subsection{\texorpdfstring{$q$-Cauchy-Riemann}{q-Cauchy-Riemann} equations}
If $u$ is a harmonic function on a Riemann surface, then it has a harmonic conjugate $v$, which is defined by the Cauchy-Riemann equations 
$$\dif v = -\star \dif u.$$
Daskalopolous and Uhlenbeck introduced an analogoue of the Cauchy-Riemann equations for the $p$-Laplacian on a Riemann surface \cite[\S3]{daskalopoulos2020transverse}.
Let $u$ be $q$-harmonic, thus 
$$\dif^*(|\dif u|^{q - 2} \dif u) = 0.$$
Then it can be shown that the function $v$ defined by 
$$\dif v = -|\dif u|^{q - 2} \star \dif u$$
is $p$-harmonic, where $(p, q)$ is a H\"older pair (in particular, we have $\dif \dif v = 0$).

In higher dimensions, $|\dif u|^{q - 2} \star \dif u$ is instead locally the exterior derivative $F$ of a $d - 2$-form $A$.
Since the domain is not assumed to have trivial topology (and in fact we are interested in the cohomology class of $F$), it is convenient to eschew the gauge field $A$ altogether and focus on the form $F$, which is defined by 
\begin{equation}\label{dual solution}
F := -|\dif u|^{q - 2} \star \dif u.
\end{equation}
This $d - 1$-form is clearly closed, since $u$ is $q$-harmonic, and it (or rather its dual vector field $\star(F^\flat)$) has appeared in various other works on the $q$-Laplacian, since it solves a Fenchel dual problem to the minimization of the $q$-Dirichlet energy \cite[Chapter IV]{Ekeland99}.
Since
\begin{equation}\label{holder cancellation}
	(p - 2)(q - 1) + (q - 2) = 0,
\end{equation}
we have 
$$\dif(|F|^{p - 2} \star F) = \pm \dif(|\dif u|^{(p - 2)(q - 1)} |\dif u|^{q - 2} \dif u) = \pm \dif \dif u = 0$$
where the sign depends on the dimension of the ambient domain.
Thus $F$ solves the PDE 
\begin{equation}\label{pMaxwell}
\begin{cases}
	\dif F = 0 \\
	\dif^* (|F|^{p - 2} F) = 0
\end{cases}
\end{equation}
which we shall now scrutinize.

\begin{definition}
Let $1 < p < \infty$.
A \dfn{$p$-tight form} is a solution of the PDE (\ref{pMaxwell}).
\end{definition}

For the remainder of this section we fix a closed oriented Riemannian manifold $M$.
Let $\Gamma$ be the fundamental group of $M$, $\tilde M \to M$ its universal cover, and $M_{\rm fun} \subseteq \tilde M$ some fundamental domain of $\Gamma$.

\begin{proposition}
There is a unique $p$-tight form in each cohomology class in $H^{d - 1}(M, \RR)$.
Moreover, $p$-tight forms are minimizers of the strictly convex functional
$$J_p(F) := \frac{1}{p} \int_M \star |F|^p$$
among all forms cohomologous to them.
\end{proposition}
\begin{proof}
Strict convexity of $J_p$ on closed $L^p$ $d - 1$-forms is straightforward; since each cohomology class is an affine subspace of $L^p(M, \Omega^{d - 1})$, and hence is convex, the strict convexity on each class follows.
Since $J_p(F) \to \infty$ as $\|F\|_{L^p} \to \infty$, $J_p$ is coercive on $L^p(M, \Omega^{d - 1})$.
Therefore we have existence and uniqueness \cite[Chapter II]{Ekeland99}.
To compute the Euler-Lagrange equations for $J_p$, let $B$ be a $d-2$-form (so $F + t \dif B$ is cohomologous to $F$), so that for a minimizer $F$ of $J_p$,
$$\frac{\dif}{\dif t} J_p(F + t \dif B) = \frac{1}{p} \int_M \star \frac{\partial}{\partial t} |F + t \dif B|^p = \int_M \star |F + t \dif B|^{p - 2} \langle F + t \dif B, \dif B\rangle.$$
Setting $t = 0$, we obtain 
$$0 = \int_M \star |F|^{p - 2} \langle F, \dif B\rangle = \int_M \star \langle \dif^*(|F|^{p - 2} F), B\rangle.$$
Thus the Euler-Lagrange equations for $J_p$ are (\ref{pMaxwell}).
\end{proof}

\begin{definition}
Let $F$ be a $p$-tight form, let
\begin{equation}
\dif u := (-1)^d |F|^{p - 2} \star F, \label{inverse extremality}
\end{equation}
and let $u$ be the primitive of $\dif u$ on the universal cover $\tilde M$, which is normalized to have zero mean on a fundamental domain $M_{\rm fun}$.
Then $u$ is called the \dfn{$q$-harmonic conjugate} of the $p$-tight form $F$, where $\frac{1}{p} + \frac{1}{q} = 1$.
\end{definition}

Let $u$ be the $q$-harmonic conjugate of the $p$-tight form $F$.
Then $\dif u$ is well-defined as a $1$-form on $M$, so there is a representation $\alpha: \Gamma \to \RR$ such that $u$ is $\alpha$-equivariant.
We record that 
\begin{equation}\label{q energy is p energy}
\int_M \star |\dif u|^q = \int_M \star |F|^p.
\end{equation}
By Poincar\'e's inequality,
$$\|u\|_{W^{1, q}(M_{\rm fun})}^q \lesssim \int_M \star |\dif u|^q = \int_M \star |F|^p < \infty$$
since $F$ is $p$-tight; that is, we have $F \in L^p(M)$ and $u \in W^{1, q}_\loc(\tilde M, \RR)$, justifying any manipulations we shall make with these forms.

\begin{lemma}
Let $(p, q)$ be a H\"older pair.
Let $F$ be a $p$-tight form, and let $u$ be its $q$-harmonic conjugate.
Then $u$ is $q$-harmonic, $F$ satisfies (\ref{dual solution}), and we have
\begin{equation}\label{strong duality}
\frac{1}{q} \int_M \star |\dif u|^q + \frac{1}{p} \int_M \star |F|^p + \int_M \dif u \wedge F = 0.
\end{equation}
\end{lemma}
\begin{proof}
We first use (\ref{holder cancellation}) to prove
$$|\dif u|^{q - 2} \star \dif u = (-1)^d |F|^{(q - 2)(p - 1)} \star \star |F|^{p - 2} F = - |F|^{(q - 2)(p - 1) - (p - 2)} F = - F.$$
Thus we have (\ref{dual solution}), and moreover
$$\dif \star (|\dif u|^{q - 2} \dif u) = - \dif F = 0$$
so that $u$ is $q$-harmonic.
By (\ref{q energy is p energy}),
$$\frac{1}{q} \int_M \star |\dif u|^q + \frac{1}{p} \int_M \star |F|^p = \left[\frac{1}{q} + \frac{1}{p}\right] \int_M \star |F|^p = \int_M \star |F|^p.$$
But
$$\int_M \dif u \wedge F = (-1)^d \int_M |F|^{p - 2} \star F \wedge F = \int_M \star |F|^p,$$
so both sides of (\ref{strong duality}) are equal to $\int_M \star |F|^p$.
\end{proof}

We now estimate the energy of a $q$-harmonic function in terms of the energy of a $1$-harmonic function of the same topological type.
The argument is essentially due to Massart \cite[\S4.2]{Massart96}, who considered the case that $M$ is a hyperbolic surface.
We introduce the \dfn{maximal intersection number}
$$i_M := \sup_{\substack{\xi \in C^\infty(M, \Omega^1) \\ \eta \in C^\infty(M, \Omega^{d - 1})}} \frac{1}{\|\xi\|_{L^1} \|\eta\|_{L^1}} \int_M \xi \wedge \eta$$
which is a positive, finite constant that only depends on the Riemannian manifold $M$.

\begin{lemma}
Let $p^{-1} + q^{-1}$, $1 < q < \infty$, and let $u_q: \tilde M \to \RR$ be an $\alpha$-equivariant $q$-harmonic function.
Then
\begin{equation}\label{q Laplacian Sobolev regularity estimate}
\vol(M)^{-1/p} \Mass(\alpha) \leq \|\dif u_q\|_{L^q} \leq \vol(M)^{1/q} i_M \Mass(\alpha)
\end{equation}
\end{lemma}
\begin{proof}
Let $u_\infty$ be an $\alpha$-equivariant function which minimizes its Lipschitz constant $\|\dif u_\infty\|_{L^\infty}$ among all $\alpha$-equivariant functions.\footnote{For example, we can take $u_\infty$ to be the $\infty$-harmonic map constructed by Daskalopolous and Uhlenbeck \cite[\S2]{daskalopoulos2020transverse}.}
Then, since $\dif u_q$ minimizes its $L^q$ norm, we obtain from H\"older's inequality
$$\|\dif u_q\|_{L^q} \leq \|\dif u_\infty\|_{L^q} \leq \vol(M)^{1/q} \|\dif u_\infty\|_{L^\infty}.$$
By the converse to H\"older's inequality, 
$$\|\dif u_\infty\|_{L^\infty} = \sup_{\eta \in C^\infty(M, \Omega^{d - 1})} \frac{1}{\|\eta\|_{L^1}} \int_M \dif u_\infty \wedge \eta.$$
If we now let $v_\varepsilon$ be $\alpha$-equivariant with $\|\dif v_\varepsilon\|_{L^1} \leq \Mass(\alpha) + \varepsilon$, we have by Stokes' theorem
\begin{align*}
\sup_{\eta \in C^\infty(M, \Omega^{d - 1})} \frac{1}{\|\eta\|_{L^1}} \int_M \dif u_\infty \wedge \eta 
&= \sup_{\eta \in C^\infty(M, \Omega^{d - 1})} \frac{1}{\|\eta\|_{L^1}} \int_M \dif v_\varepsilon \wedge \eta\\
&\leq i_M \|\dif v_\varepsilon\|_{L^1} \\
&\leq i_M(\Mass(\alpha) + \varepsilon).
\end{align*}
Taking $\varepsilon \to 0$, we deduce one direction of (\ref{q Laplacian Sobolev regularity estimate}).
In the other direction, we estimate using H\"older's inequality
\begin{align*}
\Mass(\alpha) &\leq \|\dif u_q\|_{L^1} \leq \|\dif u_q\|_{L^q} \vol(M)^{1/p}. \qedhere 
\end{align*}
\end{proof}

%%%%%%%%%%%%%%%%%%%%%%%
\subsection{\texorpdfstring{Existence of tight forms}{Existence of tight forms}}
Recall that the \dfn{costable norm} $\Comass(\rho)$ of a class $\rho \in H^{d - 1}(M, \RR)$ is the infimum of the comasses $\Comass(F) = \|F\|_{L^\infty}$ among all $F$ with $[F] = \rho$.

\begin{definition}
A closed $d - 1$-form $F$ has \dfn{best comass} if for every $d - 2$-form $B$,
$$\|F\|_{L^\infty} \leq \|F + \dif B\|_{L^\infty}.$$
\end{definition}

We shall show that the $p$-tight forms converge to a form of best comass.
To do so, we shall need the $p$-tight forms to be uniformly bounded in the following sense.

\begin{lemma}
Let $F_p$ be a $p$-tight form, and let $B$ range over $d - 2$-forms. Then
\begin{equation}\label{infinity magnetic rules p magnetic}
	\|F_p\|_{L^p} \leq \vol(M)^{1/p} \inf_B \|F + \dif B\|_{L^\infty}.
\end{equation}
\end{lemma}
\begin{proof}
By H\"older's inequality and the fact that $F_p$ is $p$-tight,
$$\|F_p\|_{L^p} \leq \|F + \dif B\|_{L^p} \leq \vol(M)^{1/p} \|F + \dif B\|_{L^\infty},$$
hence the same holds for the infimum.
\end{proof}

\begin{proposition}\label{existence infinity}
Let $\rho \in H^{d - 1}(M, \RR)$.
For each $p \geq 2$, let $F_p$ be the $p$-tight form representing $\rho$. Then there exists a closed $d - 1$-form $F$ such that:
\begin{enumerate}
\item $F_p \to F$ weakly in $L^r$ along a subsequence, for any $d < r < \infty$.
\item $F$ is a best comass representative of $\rho$.
\end{enumerate}
\end{proposition}
\begin{proof}
Let $r > d$, and let $G$ be an $L^\infty$ representative of $\rho$.
By H\"older's inequality and (\ref{infinity magnetic rules p magnetic}),
\begin{equation}\label{uniform bounds in p by best curl}
	\|F_p\|_{L^r} \leq \vol(M)^{\frac{1}{r} - \frac{1}{p}} \|F_p\|_{L^p} \leq \vol(M)^{\frac{1}{r}} \|G\|_{L^\infty}.
\end{equation}
Thus a compactness argument gives $F_p \to F$ for some $d - 1$-form $F$, weakly in $L^r$, and 
$$\|F\|_{L^r} \leq \liminf_{p \to \infty} \|F_p\|_{L^r} \leq \vol(M)^{\frac{1}{r}} \|G\|_{L^\infty}.$$
Diagonalizing, we may assume that $F_p \to F$ weakly in $L^r$ for every such $r$, and taking $r \to \infty$, we conclude 
\begin{equation}\label{infinity magnetics have best curl}
	\|F\|_{L^\infty} \leq \|G\|_{L^\infty}.
\end{equation}
Moreover, $[F] = \lim_{p \to \infty} [F_p] = \rho$.
Since $G$ was arbitrary in (\ref{infinity magnetics have best curl}), $F$ has best comass.
\end{proof}

\begin{definition}
The $d - 1$-form $F$ of best comass in Proposition \ref{existence infinity} is called a \dfn{tight form}.
\end{definition}

It is a corollary of Proposition \ref{existence infinity} that every cohomology class is represented by a form of best comass.
This could be shown more directly using Alaoglu's theorem on the weakstar topology of $L^\infty$.
However, since $p$-tight forms are intimately related to $q$-harmonic functions, it is more convenient to work with tight forms than general forms of best comass.
In any case, the existence of best comass representatives of each cohomology class $\rho$ implies the following useful lemma on the costable norm of $\rho$.

\begin{lemma}\label{p tights approximate L}
Let $F_p$ be the $p$-tight representative of $\rho$. Then 
$$\lim_{p \to \infty} \|F_p\|_{L^p} = \Comass(\rho).$$
\end{lemma}
\begin{proof}
We follow \cite[Lemma 2.7]{daskalopoulos2020transverse}.
Let $F$ be a best comass representative of $\rho$, so $\|F\|_{L^\infty} = \Comass(\rho)$.
Since $F_p$ is $p$-tight, H\"older's inequality implies 
$$\|F_p\|_{L^p} \leq \|F\|_{L^p} \leq \vol(M)^{\frac{1}{p}} \Comass(\rho).$$
Therefore 
$$\limsup_{p \to \infty} \|F_p\|_{L^p} \leq \Comass(\rho).$$
To prove the converse, suppose that for some $\varepsilon > 0$,
$$\liminf_{p \to \infty} \|F_p\|_{L^p} \leq \Comass(\rho) - \varepsilon < \Comass(\rho).$$
Along a subsequence which attains the limit inferior, $F_p$ converges weakly in every $L^r$, $d < r < \infty$, to a tight form $\tilde F$ such that (by H\"older's inequality)
$$\|\tilde F\|_{L^r} \leq \liminf_{p \to \infty} \|F_p\|_{L^r} \leq \liminf_{p \to \infty} \vol(M)^{\frac{1}{r}} \|\tilde F\|_{L^\infty} \leq \vol(M)^{\frac{1}{r}} (\Comass(\rho) - \varepsilon).$$
Taking $r \to \infty$, we obtain $\Comass(\tilde F) < \Comass(\rho)$, which contradicts the definition of the costable norm $\Comass(\rho)$.
\end{proof}


%%%%%%%%%%%%%%%%%%%%
\subsection{\texorpdfstring{$1$-harmonic conjugates of tight forms}{One-harmonic conjugates of tight forms}}
We now construct the $1$-harmonic conjugates of a tight form.
In the special case that the tight form $F$ is a calibration, that is $\Comass(F) = 1$, a $1$-harmonic conjugate will be a $1$-harmonic function on the universal cover whose level sets are calibrated by $F$.

\begin{definition}
Let $F$ be a tight form of cohomology class $\rho$.
A nonconstant $\Gamma$-equivariant function of least gradient $u \in BV_\loc(\tilde M)$ is called a \dfn{$1$-harmonic conjugate} of $F$ if
\begin{equation}\label{1 extremality}
\dif u \wedge F = \Comass(\rho) \star |\dif u|.
\end{equation}
\end{definition}

By the coarea formula (\ref{coarea formula}), the product $\dif u \wedge F$ makes sense as a finite Radon measure, because $F \in L^\infty$ and $\dif u$ is a current of finite mass.

\begin{lemma}\label{L1 convergence preserves pi1}
Let $\tilde M \to M$ be the universal cover, and let $(u_q)$ be a sequence of $\Gamma$-equivariant functions on $\tilde M$ which converge in $L^1_\loc(\tilde M)$ to a function $u$ as $q \to 1$.
Then $u$ is $\Gamma$-equivariant, and $[u_q] \to [u]$.
Moreover, if $\dif u_q \to \dif u$ in the weak topology of measures on $M$ and $\dif u_q \in L^q$, then
\begin{equation}\label{q to 1 Holder}
\Mass(\dif u) \leq \liminf_{q \to 1} \frac{1}{q} \int_M \star |\dif u_q|^q.
\end{equation}
\end{lemma}
\begin{proof}
Since $u_q$ is $\Gamma$-equivariant, there exists $\alpha_q \in H^1(M, \RR)$ such that for every $\gamma \in \pi_1(M)$,
\begin{equation}\label{equivariance q}
	\gamma^* u_q = u_q + \langle \alpha_q, \gamma\rangle.
\end{equation}
Let $M_{\rm fun}$ be a fundamental domain and $U_\gamma := M_{\rm fun} \cup \gamma_* (M_{\rm fun})$.

We claim that $(\alpha_q)$ has a convergent subsequence.
To see this, we first recall that $M$ has finite Betti numbers, so $H^1(M, \RR)$ is locally compact.
Therefore, if no convergent subsequence exists, there exists a $\gamma \in \pi_1(M)$ and a subsequence along which $\langle \alpha_q, \gamma\rangle \to \infty$.
Moreover, since $u_q \to u$ in $L^1_\loc$, $\|u_q\|_{L^1(M_{\rm fun})} \leq 2\|u\|_{L^1(M_{\rm fun})}$ if $q - 1$ is small enough.
But then 
$$\|u_q\|_{L^1(\gamma_* M_{\rm fun})} = \|\gamma^* u_q\|_{L^1(M_{\rm fun})} \geq \langle \alpha_q, \gamma\rangle - \|u_q\|_{L^1(M_{\rm fun})} \geq \langle \alpha_q, \gamma\rangle - 2\|u\|_{L^1(M_{\rm fun})}$$
and taking $q \to 1$ we conclude that $(u_q)$ is not compact in $L^1(\gamma_* M_{\rm fun})$, contradicting the convergence in $L^1_\loc(\tilde M)$.
So $\alpha_q \to \alpha$ for some $\alpha \in H^1(M, \RR)$ along a subsequence.

For any $q > 1$,
\begin{align*}
\dashint_{M_{\rm fun}} \star |\gamma^* u - u - \langle \alpha, \gamma\rangle| 
&\leq \dashint_{M_{\rm fun}} \star (|\gamma^* u_q - u_q - \langle \alpha_q, \gamma\rangle| + |\gamma^* u_q - u_q| + |\gamma^* u - u|) \\
&\qquad + |\langle \alpha_q - \alpha, \gamma\rangle|.
\end{align*}
Taking $q \to 1$ and applying (\ref{equivariance q}), we conclude that $\|\gamma^* u - u - \langle \alpha, \gamma\rangle\|_{L^1} = 0$, hence $u$ is $\alpha$-equivariant.
Thus $\alpha$ is uniquely defined and $\alpha_q \to \alpha$ along the entire subsequence.

Finally we prove (\ref{q to 1 Holder}).
Suppose that $\dif u_q \to \dif u$ in the weak topology of measures and $\dif u_q$ in $L^q$.
Then
$$\|\dif u_q\|_{L^1} = \Mass(\dif u_q).$$
So we may use the portmanteau theorem and H\"older's inequality to estimate (where $\frac{1}{p} + \frac{1}{q} = 1$)
\begin{align*}
\Mass(\dif u) &= \lim_{q \to 1} \Mass(\dif u_q) \leq \lim_{q \to 1} \vol(M)^{\frac{1}{p}} \|\dif u_q\|_{L^q} = \lim_{q \to 1} \frac{1}{q} \int_M \star |\dif u_q|^q. \qedhere
\end{align*}
\end{proof}

The duality relation (\ref{inverse extremality}) blows up $p \to \infty$.
We now ``renormalize'' away the divergence of the $q$-harmonic conjugates of $p$-tight forms before taking the limit $q \to 1$, as in \cite[\S3.2]{daskalopoulos2020transverse}.
Suppose that $\rho \in H^{d - 1}(M, \RR)$, and let $k_p$ be defined by 
$$k_p^{1 - p} = \int_M \star |F_p|^p$$
where $F_p$ is the $p$-tight representative of $\rho$.

\begin{lemma}\label{normalizations converge}
As $p \to \infty$, $k_p \to \Comass(\rho)^{-1}$.
\end{lemma}
\begin{proof}
We follow \cite[Lemma 3.4]{daskalopoulos2020transverse}.
By Lemma \ref{p tights approximate L},
$$\lim_{p \to \infty} k_p^{-\frac{1}{q}} = \lim_{p \to \infty} \|F_p\|_{L^p} = \Comass(\rho).$$
Taking logarithms we see that $q^{-1} \log k_p \to -\log \Comass(\rho)$, and since $q \to 1$ the claim follows.
\end{proof}

\begin{proposition}\label{existence 1}
Let $\rho \in H^{d - 1}(M, \RR)$ be nonzero, and let $F$ be its tight representative.
For each H\"older pair $(p, q)$ with $d < p < \infty$, let $F_p$ be the $p$-tight representative of $\rho$, and let $u_q$ be the function on $\tilde M$ with mean zero on $M_{\rm fun}$ and
$$\dif u_q = (-1)^{d - 1} k_p^{p - 1} |F_p|^{p - 2} \star F_p.$$
Then there exists a $1$-harmonic conjugate $u$ of $F$ such that as $q \to 1$ along a subsequence, $u_q \to u$ weakly in $BV_\loc(\tilde M)$ and strongly in $L^r_\loc(\tilde M)$ for $1 \leq r < \frac{d}{d - 1}$.
\end{proposition}
\begin{proof}
Let $L := \Comass(\rho)$.
We first compute using H\"older's inequality and Lemma \ref{normalizations converge}
\begin{align}
\lim_{q \to 1} \|\dif u_q\|_{L^1}
&\leq \lim_{q \to 1} \vol(M)^{\frac{1}{p}} \left[\int_M \star |\dif u_q|^q\right]^{\frac{1}{q}} = \lim_{p \to \infty} \left[k_p^p \int_M \star |F_p|^p\right]^{\frac{1}{q}} \label{Rellich 1}\\
&= \lim_{p \to \infty} k_p^{\frac{1}{q}} = \lim_{p \to \infty} k_p = \frac{1}{L} \label{Rellich 2}.
\end{align}
So by Rellich's theorem, $(u_q)$ is weakly compact in $BV$ and strongly compact in $L^r$ for $1 \leq r < \frac{d}{d - 1}$.
In particular, $\dif u_q \to \dif u$ in the weak topology of measures and $u_q \to u$ weakly in $BV$ and strongly in $L^r$.
As the limit of $\Gamma$-equivariant functions, $u$ is also $\Gamma$-equivariant by Lemma \ref{L1 convergence preserves pi1}.
In particular, $\dif u$ drops to a current on $M$.
Moreover, $[\dif u_q] \to [\dif u]$, and we have the bound (\ref{q to 1 Holder}) on $\int \star |\dif u|$.

We next must check that $u$ is nonconstant.
If $u$ is constant, then it is $\Gamma$-invariant, so $[\dif u_q] \to 0$.
By (\ref{q Laplacian Sobolev regularity estimate}), $\|\dif u_q\|_{L^q} \to 0$, so by (\ref{Rellich 1}, \ref{Rellich 2}), $L = \infty$, which is absurd.
Therefore $u$ is nonconstant.

Renormalizing (\ref{strong duality}), we obtain 
$$\frac{k_p^{-p}}{q} \int_M \star |\dif u_q|^q + \frac{1}{p} \int_M \star |F_p|^p = k_p^{1 - p} \int_M \dif u_q \wedge F_p.$$
Multiplying by $k_p^p$, we have 
\begin{equation}\label{1 strong duality before limits}
	\frac{1}{q} \int_M \star |\dif u_q|^q + \frac{k_p^p}{p} \int_M \star |F_p|^p = k_p \int_M \dif u_q \wedge F_p.
\end{equation}

Let $\mu(U) := \Mass_U(\dif u)$ be the total variation measure of $\dif u$.
We claim that
\begin{equation}\label{1 strong duality}
	L\mu(M) \leq \int_M \dif u \wedge F.
\end{equation}
First, we have from (\ref{q to 1 Holder}) and (\ref{1 strong duality before limits}) that
$$\mu(M) \leq \lim_{q \to 1} \frac{1}{q} \int_M \star |\dif u_q|^q = \lim_{p \to \infty} k_p \int_M \dif u_q \wedge F_p - \lim_{p \to \infty} \frac{k_p^p}{p} \int_M \star |F_p|^p.$$
By Lemma \ref{normalizations converge},
$$\lim_{p \to \infty} \frac{k_p^p}{p} \int_M \star |F_p|^p = \lim_{p \to \infty} \frac{k_p}{p} = \frac{0}{L} = 0,$$
and
$$\lim_{p \to \infty} k_p \int_M \dif u_q \wedge F_p = \frac{1}{L} \lim_{p \to \infty} \int_M [\dif u_q] \wedge \rho.$$
Since $[\dif u_q] \to [\dif u]$, we obtain
$$\lim_{p \to \infty} \int_M [\dif u_q] \wedge \rho = \int_M \alpha \wedge \rho = \int_M \dif u \wedge F,$$
completing the proof of (\ref{1 strong duality}).

By the coarea formula (\ref{coarea formula}), we have for any open set $U$,
$$\int_U \dif u \wedge F = \int_{-\infty}^\infty \int_{U \cap \partial \{u > y\}} F \dif y \leq L \int_{-\infty}^\infty |U \cap \partial \{u > y\}| \dif y = L \mu(U).$$
Since $\mu$ is a Radon measure and $M$ is compact, every Borel set $E$ can be $\mu$-approximated from without by open sets, hence
\begin{equation}\label{one sided extremality}
\int_E \dif u \wedge F \leq L \mu(E).
\end{equation}

Next we deduce (\ref{1 extremality}).
We reason by contradiction: if (\ref{1 extremality}) is false, then there exists an open set $U \subseteq M$ such that 
$$\int_U \dif u \wedge F < L \int_U \star |\dif u|.$$
(Indeed, strict inequality cannot point in the other direction, by (\ref{one sided extremality}).)
However, by (\ref{one sided extremality}), 
$$\int_{M \setminus U} \dif u \wedge F \leq L \int_{M \setminus U} \star |\dif u|.$$
Adding up the integrals of $\dif u \wedge F$ over $U$ and $M \setminus U$, we conclude 
$$\int_M \dif u \wedge F < L \int_M \star |\dif u|,$$
but this contradicts (\ref{1 strong duality}); thus (\ref{1 extremality}) must be true.
In particular, $F/L$ satisfies (\ref{local calibration}), so $u$ has least gradient by Theorem \ref{MazonRossi}.
\end{proof}

%%%%%%%%%%%%%%%%
\subsection{Counterexamples for manifolds with boundary}\label{boundaries bad}
We now observe that Proposition \ref{existence 1} fails on manifolds with boundary.
The proof fails exactly due to the fact that the $q$-Dirichlet energy can ``leak out of the boundary'' as $q \to 1$.

To be more precise, the estimate on $q$-harmonic functions (\ref{q Laplacian Sobolev regularity estimate}) fails on compact domains with boundary.
Since $\Mass(\alpha)$ is the $1$-Dirichlet energy of an $\alpha$-equivariant function of least gradient, the natural analogue of this estimate on a contractible domain $M$ is
\begin{equation}\label{least gradient can be zero}
\|\dif u_q\|_{L^q} \sim \inf_{f|_{\partial M} = h} \Mass(\dif f)
\end{equation}
whenever $h \in L^2(\partial M)$ and $u_q$ are $q$-harmonic functions such that $u_q|_{\partial M} = u|_{\partial M} = h$.
Taking $M$ to be a square and $h$ to be the indicator function of one of the edges $\Sigma$ of $\partial M$ provides an example where the right-hand side of (\ref{least gradient can be zero}) is zero but there are (necessarily nonzero) $q$-harmonic functions $u_q$ of $h$.
This phenomenon occurred because $\Sigma$ was length-minimizing, but more sophisticated constructions based on fat Cantor sets shows that (\ref{least gradient can be zero}) fails even when $M$ is a disk \cite{Spradlin14}.
A similar argument shows that the analogue of Lemma \ref{L1 convergence preserves pi1}, which would assert that a sequence of $q$-harmonic functions $u_q$ converging weakly in $BV$ to some function $u$ would have $u_q|_{\partial M} \to u|_{\partial M}$, cannot hold.

Thus we do not expect a tight form to have a $1$-harmonic conjugate on a manifold with boundary.
Here is an explicit example when $M$ is a square.
This example uses the fact that the square is not strictly mean-convex, so it would be interesting to modify the fat Cantor set example to show that Proposition \ref{existence 1} fails even when $M$ is a disk; we shall not attempt to do so here.

\begin{example}
Let $M = (0, 1)^2$ and $F := (1 - x) \dif y$.
Then
$$\dif(|F|^{p - 2} \star F) = \dif((1 - x)^{p - 1} \star \dif y) = -\dif((1 - x)^{p - 1} \dif x) = 0$$
so $F$ is $p$-tight for every $1 < p < \infty$, hence is tight.
If $u$ satisfies (\ref{1 extremality}), then
$$\supp \dif u \subseteq \{|F| = \|F\|_{C^0}\} = \emptyset,$$
so $u$ is constant.
\end{example}

%%%%%%%%%%%%%%%
\section{Calibrations of laminations}\label{comass sec}
\subsection{\texorpdfstring{$L^\infty$}{L-infinity} calibrations of laminations}\label{L infinity calibrations}
The goal of the present series of papers is to study calibrations of laminations.
If the calibration $F$ is continuous, then the theory of \cite{bangert_cui_2017} of calibrated laminations applies; however, we are not aware of general existence results on continuous calibrations, only $L^\infty$ calibrations, such as \cite[\S4.12]{Federer1974}.
Thus, we now introduce $L^\infty$ calibrations of laminations.

\begin{definition}
A \dfn{calibration} of degree $d - 1$ is a closed $d - 1$-form $F$ such that $\Comass(F) = 1$.
A $d - 1$-current $T$ is $F$-\dfn{calibrated} if 
\begin{equation}\label{calibration of current}
\Mass(T) = \int_M T \wedge F.
\end{equation}
A lamination $\lambda$ is \dfn{$F$-calibrated} if every leaf of $\lambda$ is $F$-calibrated.
\end{definition}

By the coarea formula (\ref{coarea formula}), the integrand $T \wedge F$ in the equation (\ref{calibration of current}) makes sense whenever $T$ is a current\footnote{Recall our convention from Appendix \ref{GMT appendix} that every current has locally finite mass unless explicitly stated otherwise.} such that $\dif T$ is sufficiently smooth, $F \in L^\infty_\loc$, and $\dif F \in L^d_\loc$.

We recall the fundamental theorem of calibrated geometry \cite{Harvey82}: if $T$ is an $F$-calibrated current, then $T$ is homologically minimizing relative to $\dif T$.
Indeed, if $T - S = \dif R$, then 
$$\Mass(T) = \int_M T \wedge F = \int_M S \wedge F \leq \Mass(S)$$
where the integral of $\dif R \wedge F$ vanished by integration by parts and the last integral was estimated using the triangle inequality.
A dual result to this theorem is that if $T$ is an $F$-calibrated current and $\dif T = 0$, then $F$ has best comass: if $F - G = \dif B$, then 
$$\|F\|_{L^\infty} = \frac{1}{\Mass(T)} \int_M T \wedge F = \frac{1}{\Mass(T)} \int_M T \wedge G \leq \|G\|_{L^\infty}$$
where we again used integration by parts and the triangle inequality.

A smooth hypersurface $N$ is $F$-calibrated iff the pullback $\iota_N^* F$ is the area form on $N$.
The pullback map $\iota_N^*$ is well-defined by Proposition \ref{integration is welldefined}.
In particular, $\iota_N^* F$ is smooth and $N$ is oriented.
If $\lambda$ is an $F$-calibrated lamination, then the trace $\iota_\lambda^* F$ is defined to be $\iota_N^* F$ along any leaf $N$.
A priori, $\lambda$ could fail to be orientable, and then $\iota_\lambda^* F$ would be necessarily discontinuous.
However, we assert that a calibrated lamination is orientable:

\begin{proposition}\label{calibrated implies oriented}
Let $F$ be a calibration and $\lambda$ an $F$-calibrated lamination.
Then $\iota_\lambda^* F$ is continuous, and $F$ induces an orientation on $\lambda$.
\end{proposition}
\begin{proof}
Let $N$ be a leaf of $\lambda$ and $x \in N$.
Let $\mathscr O$ be the local orientation of $\lambda$ near $x$ which is compatible with $F(x)$. 

There exists a continuous $d - 1$-form $G$ such that $y \in K$ close enough to $x$, $K$ a leaf of $\lambda$, $G(y) = \dif S_K(y)$ is the area form of $K$ with respect to $\mathscr O$.
In fact, we can fill in the spaces between the leaves by linear interpolation.
Then for any $y \in \supp \lambda$ close to $x$, either $F(y) = G(y)$ or $F(y) = -G(y)$; we claim that $F(y) = G(y)$ if $\dist(x, y)$ is small enough.
If this claim is true, then the proposition follows, since $G$ is continuous at $x$.

To prove the claim, we suppose towards contradiction that there is a sequence $(x_n) \subset \supp \lambda$ with $x_n \to x$ and $F(x_n) = -G(x_n)$.
We may write $F = \dif A$ where $A$ is continuous near $x$ by Proposition \ref{Hodge theorem}.
Let $N_n$ be the leaf of $\lambda$ containing $x_n$; then $N_n$ is minimal, hence smooth, so $F_n|_{N_n}$ is smooth.
Therefore by Stokes' theorem, if $(D_{n, m})$ is a sequence of disks in $N_n$ which shrink down to $x_n$ as $m \to \infty$ and are equipped with the orientation $\mathscr O$,
$$-1 = \lim_{m \to \infty} \frac{1}{\Mass(D_{n, m})} \int_{D_{n, m}} F = \lim_{m \to \infty} \frac{1}{\Mass(D_{n, m})} \int_{\partial D_{n, m}} A.$$
In particular, we can choose $m_n \to \infty$ such that 
\begin{equation}\label{orientation contradiction}
\int_{\partial D_{n, m_n}} A \leq 0.
\end{equation}
We set $D_n := D_{n, m_n}$.

Let $(k, z) \in \RR \times \RR^{d - 1}$ be coordinates near $x$ with respect to a flow box, where $k$ indexes the leaves and $z$ is a parameter on each leaf.
Then $D_n = \{k_n\} \times \Omega_n$ for some $\Omega_n \subset \RR^{d - 1}$.
Let $k$ be the index of $N$ and $D_n' := \{k\} \times \Omega_n$.
Since $A$ is continuous, $A(k, z) = A(k_n, z) + o(1)$ as $n \to \infty$, hence
\begin{equation}\label{orientation contradiction 2}
\frac{1}{\Mass(D_n')} \int_{\partial D_n'} A = \frac{1}{\Mass(D_n)} \int_{\partial D_n} A + o(1).
\end{equation}
However, $D_n'$ shrinks down to $x$, and $F(x) = G(x)$, so by Stokes' theorem, for $n$ large,
$$\int_{\partial D_n'} A \gtrsim \Mass(D_n'),$$
hence by (\ref{orientation contradiction}) and (\ref{orientation contradiction 2}),
$$0 < \Mass(D_n) \lesssim \int_{\partial D_n} A \leq 0,$$
a contradiction.
\end{proof}

We next give a natural condition for a lamination to be calibrated.
To state it, we record the following immediate consequence of the coarea formula (\ref{coarea formula}).

\begin{lemma}
Let $(\lambda, \mu)$ be a measured oriented lamination on a closed Riemannian manifold $M$.
Suppose that $F \in L^\infty(M, \Omega^{d - 1})$ has $\dif F \in L^p(M, \Omega^d)$.
Let $(\chi_\alpha)$ be a locally finite partition of unity subordinate to an open cover $(U_\alpha)$ of flow boxes for $\lambda$.
Let $\sigma_{\alpha, k}$ be the leaf of $\lambda \cap U_\alpha$ with parameter $k \in I$.
Then
\begin{equation}\label{coarea formula on laminations}
\int_M T_\lambda \wedge F = \sum_\alpha \int_I \int_{\sigma_{\alpha, k}} \chi_\alpha F \dif \mu_\alpha(k).
\end{equation}
\end{lemma}

On the other hand, if $F$ is a closed $d - 1$-form and $M$ is closed, then $\int_M T_\lambda \wedge F = \langle [\lambda], [F]\rangle$ is simply the pairing of homology and cohomology.

\begin{proposition}\label{calibration condition}
Let $F$ be a calibration on a closed Riemannian manifold $M$.
Let $T_\lambda$ be the Ruelle-Sullivan current of a measured oriented lamination $\lambda$.
Then the following are equivalent:
\begin{enumerate}
\item One has \begin{equation}\label{calibration by Ruelle Sullivan}
\int_M T_\lambda \wedge F = \Mass(\lambda).
\end{equation}
\item $\lambda$ is $F$-calibrated.
\end{enumerate}
\end{proposition}
\begin{proof}
First suppose that (\ref{calibration by Ruelle Sullivan}) holds.
Let $(\chi_\alpha)$ be a locally finite partition of unity subordinate to an open cover $(U_\alpha)$ of flow boxes for $\lambda$, and let $(\mu_\alpha)$ be the transverse measure.
After refining $(U_\alpha)$ we may assume that $U_\alpha$ is contained in a deformed ball as in Appendix \ref{GMT appendix}. After shrinking $U_\alpha$ we may assume that $\chi_\alpha > 0$ on $U_\alpha$.
Then for some hypersurfaces $\sigma_{\alpha,k}$,
$$\Mass(\lambda) = \int_M T_\lambda \wedge F = \sum_\alpha \int_I \int_{\sigma_{\alpha,k}} \chi_\alpha F \dif \mu_\alpha(k).$$
Let $\dif S_{\alpha,k}$ be the surface measure on $\sigma_{\alpha,k}$.
Then
$$\int_M \chi_\alpha \star |T_\lambda| = \int_I \int_{\sigma_{\alpha,k}} \chi_\alpha \dif S_{\alpha,k} \dif \mu_\alpha(k),$$
so summing in $\alpha$, we obtain 
\begin{equation}\label{calibration condition contr}
\sum_\alpha \int_I \int_{\sigma_{\alpha,k}} \chi_\alpha F \dif \mu_\alpha(k) = \Mass(\lambda) = \sum_\alpha \int_I \int_{\sigma_{\alpha,k}} \chi_\alpha \dif S_{\alpha,k} \dif \mu_\alpha(k).
\end{equation}

We claim that $\lambda$ is \dfn{almost calibrated} in the sense that for every $\alpha$ and $\mu_\alpha$-almost every $k$, $\sigma_{\alpha, k}$ is calibrated.
If this is not true, then we may select $\beta$ and $K \subseteq I$ with $\mu_\beta(K) > 0$, such that for every $k \in K$, $\int_{\sigma_{\beta, k}} F < \Mass(\sigma_{\beta, k})$.
Since $0 < \chi_\beta \leq 1$ and $F/\dif S_{\beta, k} \leq 1$ on $\sigma_{\beta, k}$, this is only possible if 
$$\int_{\sigma_{\beta, k}} \chi_\beta F < \int_{\sigma_{\beta, k}} \chi_\beta \dif S_{\beta, k}.$$
Integrating over $K$, and using the fact that in general we have $\int_{\sigma_{\alpha, k}} \chi_\alpha F \leq \int_{\sigma_{\alpha, k}} \chi_\alpha \dif S_{\alpha, k}$, we conclude that 
$$\sum_\alpha \int_I \int_{\sigma_{\alpha, k}} \chi_\alpha F \dif \mu_\alpha(k) < \sum_\alpha \int_I \int_{\sigma_{\alpha, k}} \chi_\alpha \dif S_{\alpha, k} \dif \mu_\alpha(k)$$
which contradicts (\ref{calibration condition contr}).

To upgrade $\lambda$ from an almost calibrated lamination to a calibrated lamination, we first, given $\sigma_{\alpha, k}$, choose $k_j$ such that $\sigma_{\alpha, k_j}$ is calibrated and $k_j \to k$.
By Proposition \ref{Hodge theorem}, we can find a continuous $d - 2$-form $A$ defined near $\sigma_{\alpha, k}$ with $F = \dif A$.
This justifies the following application of Stokes' theorem: 
$$\int_{\sigma_{\alpha, k}} F = \int_{\partial \sigma_{\alpha, k}} A.$$
Since $k_j \to k$, and $A$ is continuous,
\begin{align*}
\Mass(\sigma_{\alpha, k}) &= \lim_{j \to \infty} \Mass(\sigma_{\alpha, k_j}) = \lim_{j \to \infty} \int_{\sigma_{\alpha, k_j}} F = \lim_{j \to \infty} \int_{\partial \sigma_{\alpha, k_j}} A = \int_{\partial \sigma_{\alpha, k}} A = \int_{\sigma_{\alpha, k}} F.
\end{align*}

To establish the converse, suppose that $\lambda$ is $F$-calibrated, and let notation be as above.
Since $\lambda$ is $F$-calibrated, for every $\alpha$ and every $k$, the area form on $\sigma_{\alpha, k}$ is $F$. Therefore
\begin{align*}
\int_M T_\lambda \wedge F &= \sum_\alpha \int_I \int_{\sigma_{\alpha, k}} \chi_\alpha F \dif \mu_\alpha(k) = \Mass(T_\lambda). \qedhere
\end{align*}
\end{proof}

\begin{proposition}\label{properties of calibrated laminations}
Suppose that $M$ is a closed Riemannian manifold, $F$ is a calibration, and $\lambda$ is a measured $F$-calibrated lamination.
Then:
\begin{enumerate}
\item $\lambda$ is minimal and homologically minimizing.
\item If $G$ is a calibration and cohomologous to $F$, then $\lambda$ is $G$-calibrated.
\end{enumerate}
\end{proposition}
\begin{proof}
Every leaf of $\lambda$ is $F$-calibrated, hence minimal, so $\lambda$ is also minimal.
Since $\lambda$ is oriented by Proposition \ref{calibrated implies oriented}, it has a Ruelle-Sullivan current.
Then, since (\ref{calibration by Ruelle Sullivan}) only depends on the cohomology class of $F$, not $F$ itself, $\lambda$ is $G$-calibrated.
From (\ref{calibration by Ruelle Sullivan}) and the fundamental theorem of calibrated geometry, $\lambda$ is homologically minimizing.
\end{proof}

%%%%%%%%%%%%%%%%%%%
\subsection{Existence of measured stretch laminations}\label{proof of Theorem B}
Let $M$ be a closed Riemannian of dimension $d \leq 7$ equipped with a cohomology class $\rho \in H^{d - 1}(M, \RR)$.
To ease notation, we normalize the costable norm:
$$\Comass(\rho) = 1.$$
By Theorem \ref{1 harmonic is MOML}, every function $u$ of least gradient gives rise to a homologically minimizing lamination $\kappa_u$.
Thus the following definition makes sense:

\begin{definition}
Let $F$ be a tight representative of $\rho$, and let $u$ be a $1$-harmonic conjugate of $F$.
Then we call $\kappa_u$ a \dfn{measured stretch lamination} associated to $\rho$.
\end{definition}

\begin{proposition}\label{MCL contains Thurston}
Let $F$ be a best comass representative of $\rho$, and let $\lambda$ be a measured stretch lamination associated to $\rho$.
Then $F$ calibrates $\lambda$.
\end{proposition}
\begin{proof}
Let $G$ be the tight form which is cohomologous to $F$ whose dual $1$-harmonic function $u$ defines the measured stretch lamination $\lambda$.
Then by (\ref{1 extremality}), 
$$\Mass(\lambda) = \Mass(\dif u) = \int_M \dif u \wedge G$$
so $G$ calibrates $\lambda$ by Proposition \ref{calibration condition}.
\end{proof}

\begin{proposition}\label{L equals K}
	Let $\kappa$ be a measured stretch lamination for $\rho$, and let $\lambda$ range over measured oriented laminations. Then 
	\begin{equation}\label{L equals K formula}
	\sup_\lambda \frac{\langle \rho, [\lambda]\rangle}{\Mass(\lambda)} = \frac{\langle \rho, [\kappa]\rangle}{\Mass(\kappa)} = 1.
	\end{equation}
\end{proposition}
\begin{proof}
Fix a tight form $F$ representing $\rho$, and let $u$ be its $1$-harmonic conjugate.
Let
$$K :=  \sup_\lambda \frac{\langle \rho, [\lambda]\rangle}{|\lambda|}.$$

We first prove $K \leq 1$.
Let $\lambda$ be a measured oriented lamination; then, since $F$ represents $\rho$ and the Ruelle-Sullivan current $T_\lambda$ represents $[\lambda]$,
$$\langle \rho, [\lambda]\rangle = \int_M F \wedge T_\lambda.$$
Let $(\chi_\alpha)$ be a partition of unity subordinate to a laminar atlas for $\lambda$, and let $(\mu_\alpha)$ be the associated transverse measure. Then 
$$\int_M F \wedge T_\lambda = \sum_\alpha \int_I \int_{\{k\} \times J} \chi_\alpha F \dif \mu_\alpha(k).$$
Since $F$ has best comass,
$$\frac{\langle \rho, [\lambda] \rangle}{\Mass(\lambda)}
\leq \frac{\|F\|_{L^\infty}}{\Mass(\lambda)} \sum_\alpha \int_I \int_{\{k\} \times J} \chi_\alpha \dif S_k \dif \mu_\alpha(k) = 1.$$
Since $\lambda$ was arbitrary, it holds that $K \leq 1$.

By (\ref{1 extremality}),
$$\langle \rho, [\kappa]\rangle = \int_M F \wedge \dif u = \Mass(\dif u) = \Mass(\kappa).$$
Dividing both sides by $\Mass(\kappa)$ and applying the direction we already proved,
$$K \leq 1 \leq \frac{\langle \rho, [\kappa]\rangle}{\Mass(\kappa)} \leq K$$
which is only possible if $K = 1$ and $\kappa$ is a maximizer.
\end{proof}

\begin{proposition}\label{calibrated means measured stretch}
Let $F$ be a best comass representative of $\rho$, and suppose that $F$ calibrates a measured oriented lamination $\lambda$.
Then $\lambda$ is a measured stretch lamination associated to $\rho$.
\end{proposition}
\begin{proof}
Let $\dif u$ be the Ruelle-Sullivan current for $\lambda$, and suppose that $f \in C^0(M)$ is supported in a flow box for $\lambda$, with local leaf space $K$ and transverse measure $\mu$.
By Proposition \ref{properties of calibrated laminations}, we may assume wihout loss of generality that $F$ is tight.
Since $F$ calibrates every leaf of $\lambda$,
$$\int_M f \star |\dif u| = \int_K \int_{\{k\} \times J} f \dif S_{\{k\} \times J} \dif \mu(k) = \int_K \int_{\{k\} \times J} fF \dif \mu(k) = \int_M f\dif u \wedge F$$
Thus $\dif u \wedge F = \star |\dif u|$, or in other words $u$ is a $1$-harmonic conjugate of the tight form $F$.
Therefore $\lambda$ is a measured stretch lamination.
\end{proof}

\begin{corollary}
Let $F$ be a best comass representative of $\rho$, and suppose that $F$ calibrates a measured oriented lamination $\lambda$.
Then $\lambda$ is Lipschitz.
\end{corollary}
\begin{proof}
By Theorem \ref{1 harmonic is MOML}, every measured stretch lamination is Lipschitz.
\end{proof}

%%%%%%%%%%%%%%%
\section{The Euler-Lagrange equation for tight forms}\label{infinityMax}
\subsection{Formal derivation of Euler-Lagrange equation}
To state our Euler-Lagrange equation we introduce some notation for derivatives of tensor fields along differential forms.
If $\alpha$ is a $k$-form, $\nabla$ is the Levi-Civita connection, and $T$ is a section of a tensor bundle $E$, we introduce the tensor $\nabla^\alpha T$, a section of $E \otimes \Omega^{k - 1}$, defined as follows: if $X_1, \dots, X_{k - 1}$ are vector fields, and
$$Y := (\iota_{X_1} \cdots \iota_{X_{k - 1}} \alpha)^\sharp$$
is the vector field dual to the contraction of $\alpha$, then
$$\langle \nabla^\alpha T, X_1 \otimes \cdots \otimes X_k\rangle := \nabla_Y T.$$
We think of $\nabla^\alpha T$ as a sort of ``weighted projection'' of $\nabla T$ to the subbundle $\ker \star \alpha \subset TM$.

\begin{proposition}
Suppose that $F_p$ are $C^1$ $p$-tight forms converging to a tight form $F$.
Furthermore suppose that as $p \to \infty$, $\|F_p\|_{C^{1 + \alpha}} \lesssim 1$.
Then $F \in C^1$ and 
\begin{equation}\label{infty Max}
\begin{cases}
\dif F = 0, \\
\langle \nabla^F F, F\rangle = 0.
\end{cases}
\end{equation}
\end{proposition}

We should clarify the PDE (\ref{infty Max}): $\nabla^F F$ is a section of $\Omega^{d - 1} \otimes \Omega^{d - 2}$, so its contraction with $F$ is the contraction of the $\Omega^{d - 1}$ part with $F$; thus $\langle \nabla^F F, F\rangle$ is a $d - 2$-form.
If $d = 2$, and $F = \dif u$, then (\ref{infty Max}) is exactly the $\infty$-Laplace equation 
$$\langle\nabla^2 u, \nabla u \otimes \nabla u\rangle = 0.$$

\begin{proof}
We first compute from (\ref{pMaxwell}) that $\dif F = 0$ and
\begin{align*}
0
&= \dif(|F_p|^{p - 2} \star F_p) \\
&= \dif(|F_p|^{p - 2}) \wedge \star F_p + |F_p|^{p - 2} \dif \star F_p \\
&= (p - 2) |F_p|^{p - 4} \langle \nabla F_p, F_p\rangle \wedge \star F_p + |F_p|^{p - 2} \dif \star F_p.
\end{align*}
If $F_p$ is nonzero, then we can divide through by $(p - 2) |F_p|^{p - 4}$ to get
\begin{equation}\label{intermediate p Max}
0 = \langle\nabla F_p, F_p\rangle \wedge \star F_p + \frac{|F_p|^2}{p - 2} \dif \star F_p.
\end{equation}
At the zeroes of $F_p$, we simply observe that (\ref{intermediate p Max}) holds for trivial reasons.

By assumption $||F_p|^2 \dif \star F_p| = o(p)$, so as $p \to \infty$, the second term of (\ref{intermediate p Max}) drops out.
We can take the limit of (\ref{intermediate p Max}) using the equicontinuity of $\nabla F_p$ to get
$$0 = \langle \nabla F, F \rangle \wedge \star F.$$
Taking the Hodge star of both sides, we get (\ref{infty Max}).
\end{proof}

Like other papers on $L^\infty$ variational problems \cite{Barron08,Sheffield12,daskalopoulos2022}, we are not aware of a suitable notion of weak solution of (\ref{infty Max}).
One possible candidate are Katzourakis' \dfn{contact solutions} \cite{Katzourakis2018OnAV}, but even the basic problem of showing that a tight form is a contact solution of (\ref{infty Max}) seems quite challenging.
We regard the problem of finding a suitable notion of weak solution as the most important problem posed in this work, as it is likely that such a notion would also be a viable candidate notion of weak solution for $\infty$-harmonic maps.\footnote{Throughout this section, we take the convention that $\infty$-harmonic maps are based on the operator norm as in \cite{Sheffield12, daskalopoulos2022}, and not on the Frobenius norm as in \cite{Ou12}. This convention is necessary in order than $\infty$-harmonic maps minimize their Lipschitz constants.}
We caution, however, that (\ref{infty Max}) is significantly easier to work with than the PDE for $\infty$-harmonic maps given by Sheffield and Smart \cite{Sheffield12}, since the PDE for $\infty$-harmonic maps is not even defined on every smooth map.

%%%%%%%%%%%%%%%%%%%%%%%%%
\subsection{Geometric and variational interpretations of the Euler-Lagrange equation}\label{EL interpretation}
The PDE (\ref{infty Max}) has a simple geometric interpretation, which generalizes the interpretation of the $\infty$-Laplace equation as asserting that the gradient curves of an $\infty$-harmonic function are lines.

\begin{proposition}\label{infty Max calibrates}
Let $F$ be a $C^1$ solution of (\ref{infty Max}) with no zeroes, and let $N$ be a connected integral hypersurface of $\ker \star F$.
Then there exists $\lambda > 0$ such that $N$ is an $F/\lambda$-calibrated hypersurface.
\end{proposition}
\begin{proof}
Let $(X_1, \dots, X_{d - 1})$ be an orthonormal frame of vector fields tangent to $N$.
We then introduce the tensor field
$$T_i := X_1 \otimes \cdots \otimes \widehat{X_i} \otimes \cdots \otimes X_{d - 1},$$
where the hat means to remove that factor.
By definition of $N$, and the fact that $F$ has no zeroes, $F$ is a nonzero scalar field $\lambda$ times $\dif S_N$.
So $\iota_{X_1} \cdots \widehat{\iota_{X_i}} \cdots \iota_{X_{d - 1}} F$ is a nonzero scalar field $u_i$ times $X_i^\flat$.
Applying (\ref{infty Max}), we have 
$$0 = \langle \langle \nabla^F F, F\rangle, T_i\rangle = u_i \langle \nabla_{X_i} F, F \rangle = \frac{u_i}{2} \partial_{X_i} (|F|^2).$$
Since $u_i/2$ is nonzero, we see that $|F|^2$ is constant along integral curves of $X_i$.
Since $(X_1, \dots, X_{d - 1})$ spans the tangent bundle of $N$, we conclude that $|F|^2$ is constant along $N$, or equivalently that $\lambda$ is a constant.
Therefore $\dif S_N = F/\lambda$ is closed, hence $N$ is $F/\lambda$-calibrated.
\end{proof}

We now give a variational criterion for (\ref{infty Max}).
Unfortunately, the converse is weaker than we would like, because in order to apply arguments similar to those of \cite{Aronsson67,Sheffield12} we need to assume that $\ker(\star F)$ is a singular integrable distribution.

\begin{proposition}
Let $F$ be a $C^1$ closed $d - 1$-form, and suppose that for every $V \subseteq M$ such that $H^{d - 1}(V, \RR) = 0$, and every $U \Subset V$,
\begin{equation}\label{ABC inequality}
\Comass_U(F) \leq \|F\|_{C^0(\partial U)}.
\end{equation}
Then (\ref{infty Max}) holds.
\end{proposition}
\begin{proof}
It suffices to prove (\ref{infty Max}) locally, so we can cover $M$ by open balls $V$ with $H^{d - 1}(V, \RR) = 0$ and prove that (\ref{infty Max}) holds in slightly smaller balls.
Since $H^{d - 1}(V, \RR) = 0$, we can find a $C^2$ $d - 2$-form $A$ on a neighborhood of $\overline V$ with $\dif A = F$.
Also for $\xi$ a covariant tensor of valence $d - 1$ at $x$, let $\xi^{\rm as}$ be its antisymmetrization, and
$$f(x, \xi) := |\xi^{\rm as}|_{g^{-1}(x)}^2.$$

We first claim that for any $d - 2$-form $B$, $f(\cdot, B) = |\dif B|^2$.
Since $\nabla$ is torsion-free, antisymmetrization annihilates the Christoffel symbols of $\nabla$, so if $\nabla^\flat$ denotes a flat connection, then $(\nabla B)^{\rm as} = (\nabla^\flat B)^{\rm as}$; the latter is of course $\dif B$.
Thus in particular, $f(\cdot, A) = |F|^2$.

Let $W \Subset V$ be obtained by slightly shrinking $V$.
We claim that $A|_W$ is an absolute minimizer of $f(x, \nabla A(x))$ in the sense of \cite[Definition 5.1]{Barron2001}.
In other words, we claim that for each open $U \subseteq W$ with smooth boundary and each covariant tensor field $B$ of valence $d - 2$, such that $A - B$ has compact support in $U$,\footnote{Strictly speaking, the definition of absolute minimizer ranges over all open sets $U$ ($\Omega'$ in the notation of \cite{Barron2001}), not just those with smooth boundary; similarly one requires the competition class to range over traceless (rather than compactly supported) variations. However, it is trivial to modify the proof of \cite[Theorem 5.2]{Barron2001} to only require smooth domains and compactly supported variations.}
$$\sup_{x \in U} f(x, A(x)) \leq \sup_{x \in U} f(x, B(x)).$$
To see this, let $G = (\nabla B)^{\rm as}$, so that by (\ref{ABC inequality}),
\begin{align*}
\sup_{x \in U} f(x, A(x))
&= \Comass_U(F) \leq \|F\|_{C^0(\partial U)} = \|G\|_{C^0(\partial U)} \leq \|G\|_{C^0(U)} = \sup_{x \in U} f(x, B(x)).
\end{align*}

By the above claims and \cite[Theorem 5.2]{Barron2001}, for each $x \in W$, we have the Euler-Lagrange-Aronsson equation that for any $d - 2$-form $\theta$,
\begin{align*}
0 
&= \left\langle \frac{\partial f}{\partial \xi}(x, \nabla A(x)), \nabla \left[f(x, \nabla A(x))\right] \otimes \theta(x)\right\rangle \\
&= 2\langle (\nabla A(x))^{\rm as}, \nabla(|(\nabla A(x))^{\rm as}|^2) \otimes \theta(x)\rangle \\
&= 2\langle F(x), \nabla(|F(x)|^2) \otimes \theta(x)\rangle.
\end{align*}
Since $\nabla$ is a metric connection, $\nabla(|F|^2) = 2\langle \nabla F, F\rangle$.
Thus we have 
$$0 = 4\langle F, \langle \nabla F, F\rangle \otimes \theta\rangle = 4\langle \langle \nabla^F F, F\rangle, \theta\rangle$$
and since $\theta$ was arbitrary we conclude (\ref{infty Max}).
\end{proof}

\begin{proposition}\label{tight and integrable implies infinity maxwell}
Let $F$ be a $C^1$ solution of (\ref{infty Max}) such that $\ker(\star F)|_{F \neq 0}$ is an integrable distribution.
Then for every $x \in M$ and every sufficiently small $r > 0$, (\ref{ABC inequality}) holds for $U = B(x, r)$.
\end{proposition}
\begin{proof}
First observe that if $x \in M$ and $F(x) = 0$, then $|F|$ has a local minimum at $x$, so for any $y$ sufficiently close to $x$, say $y \in B(x, r)$, $|F|$ does not have a local maximum at $y$ (unless $y$ is also a local minimum, hence $F(y) = 0$); therefore (\ref{ABC inequality}) holds.
Henceforth we assume that $F(x) \neq 0$.

Let $r > 0$ be such that $F|_{B(x, r)}$ has no zeroes and for some $s > r$, $H^{d - 1}(B(x, s), \RR) = 0$.
Let $\mathscr F$ be the foliation of $B(x, r)$ obtained by integrating $\ker(\star F)$.
By definition of $s$, we may assume that $F = \dif A$ for some $d - 2$-form $A$ defined on $B(x, s)$.
By Proposition \ref{infty Max calibrates}, for each leaf $N$ of $\mathscr F$ there exists $\lambda_N > 0$ such that for each hypersurface $N'$ with $N \cap \partial B(x, r) = N' \cap \partial B(x, r)$,
$$\lambda_N \Mass(N) = \int_N F = \int_{N \cap \partial B(x, r)} A = \int_{N' \cap \partial B(x, r)} A = \int_{N \cap \partial B(x, r)} F \leq \lambda_N \Mass(N'),$$
so $N$ is absolutely area-minimizing in $B(x, r)$ and hence meets $\partial B(x, r)$, say at some point $x_N$.
Moreover, $F|_N$ has constant comass $\lambda_N$; since $\mathscr F$ is a foliation it holds that for each $x \in V$ there exists a leaf $N \ni x$ of $\mathscr F$, and then 
\begin{align*}
|F(x)| &= |F(x_N)| \leq \|F\|_{C^0(\partial B(x, r))}. \qedhere
\end{align*}
\end{proof}

%%%%%%%%%%%%%%%%%
\subsection{Failure of global absolute minimality}\label{nonintegrability}
Let us show that Proposition \ref{tight and integrable implies infinity maxwell} cannot be improved to hold for every open set (as opposed to simply for small balls), and that minimality fails completely in the absence of integrability.
This is in stark contrast to the case of $\infty$-harmonic functions on euclidean space, which are absolutely minimizing for the Lipschitz constant on every open set \cite{Crandall2008}.

We begin by showing that (\ref{ABC inequality}) can fail on large open sets.
The map we study here has a crucial role in the proof that the generic $\infty$-harmonic map between hyperbolic surfaces only attains its Lipschitz constant on the canonical lamination \cite{daskalopoulos2022}.
However, this defect in the theory seems to be generic to $L^\infty$ variational systems.
One can use the failure of the Kirszbraun-Valentine theorem \cite[Example 9.6]{Gu_ritaud_2017}, for example, to show that any homothetic contraction $u: \Hyp^2 \to \Hyp^2$ is $\infty$-harmonic but is not absolutely minimizing Lipschitz.\footnote{However, $u$ is absolutely minimizing for the closely related functional $\mathcal F_\infty$ defined in \cite{Crandall2008}.}

\begin{example}
Consider the cylinder $M := \Sph^1_\theta \times \RR_x$ with the hyperbolic metric
$$g := \dif x^2 + \frac{\cosh^2 x}{4} \dif \theta^2.$$
Then $M$ has a single closed geodesic $\gamma$, which winds around $\{x = 0\}$, and has circumference $\pi$.
Let $\theta: M \to \Sph^1$ be the projection map, so
$$|\dif \theta|(x, \theta) = 2 \sech x.$$
This attains its maximum on $\gamma$, and if $U$ is any neighborhood of $\gamma$, then $|\dif \theta|_{\overline U}|$ attains its maximum on the interior of $U$, violating (\ref{ABC inequality}).
In particular, $\theta$ is not absolutely minimizing Lipschitz.

Using the Christoffel symbols
\begin{align*}
{\Gamma^\theta}_{\theta \theta} &= {\Gamma^\theta}_{x x} = 0, \\
{\Gamma^\theta}_{x \theta} &= {\Gamma^\theta}_{\theta x} = \tanh x,
\end{align*}
we see that
$$\nabla \dif \theta = \tanh x \dif x \dif \theta,$$
and hence $\Delta_\infty \theta = 0$.
This implies that $\dif \theta$ solves (\ref{infty Max}).\footnote{A similar argument shows that $\theta$ is harmonic, hence is $p$-harmonic for every $p \in [2, \infty]$. This can be used to show that $\dif \theta$ is tight.}
\end{example}

The above example relied on a judicious choice of $F = \dif \theta$ which calibrated a closed geodesic.
In particular, $\mathscr D := \ker(\star \dif \theta)$ was integrable.
By taking $\mathscr D$ to be orthogonal to a Beltrami field with constant eigenvalue, we show that the integrability hypothesis is needed even for global minimality:

\begin{example}\label{integrability needed}
Let $V$ be the unit vertical vector to the Hopf fibration 
$$\Sph^1 \to \Sph^3 \to \Sph^2.$$
Let $\mathscr D$ be the distribution of horizontal vectors.
The vertical and horizontal directions are orthogonal, so if we set $F := \star V^\flat$, and let $X, Y$ be an orthonormal frame for $\mathscr D$, then $F(X, Y) = 1$ and $F$ annihilates vertical vectors.
In particular $|F| = 1$ identically, so $2\langle \nabla F, F\rangle = \dif(|F|^2) = 0$.

But $V$ is a Beltrami field whose eigenvalue is identically $2$ \cite[\S3]{Peralta_Salas_2023}, in the sense that
$$\nabla \times V = 2V.$$
Since $(\nabla \cdot) \circ (\nabla \times) = 0$, it follows that $\nabla \cdot V = 0$, so $\dif F = 0$.
Therefore $F$ solves (\ref{infty Max}) and, since $H^2(\Sph^3, \RR) = 0$, $F$ is exact.

Thus $F$ is a solution which is cohomologous to $0$ and nonzero.
It follows that $F$ cannot have best comass and cannot be tight.
\end{example}

%%%%%%%%%%%%%%%%
\appendix
\section{Geometric measure theory}\label{GMT appendix}
By an $\ell$-\dfn{current of locally finite mass} $T$ we mean a continuous linear functional on the space $C^0_\cpt(M, \Omega^\ell)$ of continuous $\ell$-forms of compact support.
By a \dfn{current} we shall always mean a current of locally finite mass, unless explicitly stated otherwise.
We write $\int_M T \wedge \varphi$ for the dual pairing of a current and a form.
The duality norm of $T$ is its \dfn{mass}, namely for an open set $U \subseteq M$,
$$\Mass_U(T) := \sup_{\substack{\supp \varphi \Subset U \\ \|\varphi\|_{C^0} \leq 1}} \int_U T \wedge \varphi.$$
We write $\Mass(T) := \Mass_M(T)$.
If $T$ represents a simplex $\sigma$, then $\Mass(T)$ is the surface area of $\sigma$.

We call a Riemannian manifold $M$ a \dfn{deformed ball} if there is a bi-Lipschitz diffeomorphism $M \cong \Ball^d$.
We shall appeal to results of Anzellotti \cite{Anzellotti1983} which are carried out on $\Ball^d$.
We can formulate these results in the language of differential forms, which is diffeomorphism-invariant, so the proofs go through unchanged if $\Ball^d$ is replaced by a deformed ball.\footnote{Strictly speaking, the $L^p$ and $\Mass$ norms depend on the metric, but every property used about these norms in \cite{Anzellotti1983} also holds in the presence of a Riemannian metric, with the same proof.}

If $F$ is an $L^1_\loc$ $d - 1$-form and $\dif F$ is a current, we introduce the norms
\begin{align*}
\|F\|_{X^p} &:= \|F\|_{L^\infty} + \|\dif F\|_{L^p}, \\
\|F\|_Y &:= \|F\|_{L^\infty} + \Mass(\dif F),
\end{align*}
which define Banach spaces $X^p, Y$.
If $\chi$ is a smooth cutoff function, then multiplication by $\chi$ defines bounded linear endomorphisms of $X^p, Y$.
So, by covering a Riemannian manifold $M$ by deformed balls and applying a partition of unity, we see that Anzellotti theory holds for $M$.

\begin{proposition}[trace theorem {\cite[Theorem 1.2]{Anzellotti1983}}]\label{integration is welldefined}
Let $\iota: N \to M$ be the inclusion of an oriented Lipschitz hypersurface.
Then the pullback $\iota^*$ of $d - 1$-forms extends to a bounded linear map $\iota^*: Y \to L^\infty(N)$ such that
\begin{equation}\label{integral over chain is linfinity}
	\|\iota^* F\|_{L^\infty(N)} \leq \|F\|_{L^\infty(M)}.
\end{equation}
\end{proposition}

If $U$ is a set of locally finite perimeter (thus $1_U \in BV_\loc$), we recall the \dfn{measure-theoretic boundary}, the set $\partial U$ of $x \in M$ such that for all $\varepsilon > 0$,
$$0 < \Mass(U \cap B(x, \varepsilon)) < \Mass(B(x, \varepsilon)).$$
By \cite[Theorem 4.4]{Giusti77}, $\partial U$ is an integral $d - 1$-current (in particular, the sum of Lipschitz hypersurfaces), with surface measure $\star |\dif 1_U|$; moreover, $\dif 1_U$ is conormal to $\partial U$.
By the coarea formula \cite[Theorem 1.23]{Giusti77}, the superlevel sets $\{u > \lambda\}$ of a function $u \in BV_\loc$ have locally finite perimeter.

\begin{proposition}[coarea formula]\label{coarea theorem}
Let $u \in BV$ and $F \in X^d$.
Define a $0$-current $\dif u \wedge F$ (possibly not of locally finite mass) by declaring that for every $\chi \in C^\infty_\cpt$,
$$\int_M \chi \dif u \wedge F = -\int_M u \dif \chi \wedge F - \int_M \chi u \dif F.$$
Then $\dif u \wedge F$ has finite mass, and for any $\chi \in C^0_\cpt$,
\begin{equation}\label{coarea formula}
\int_M \chi \dif u \wedge F = \int_{-\infty}^\infty \int_{\partial \{u > \lambda\}} \chi F \dif \lambda.
\end{equation}
\end{proposition}
\begin{proof}
Observe that for $p := \frac{d}{d - 1}$, $(p, d)$ is a H\"older pair, and by the isoperimetric inequality, $u \in L^p$.
From these assertions and \cite[Theorem 1.5]{Anzellotti1983}, $\dif u \wedge F$ has finite mass.
Since $\star|\dif 1_{\{u > \lambda\}}|$ is the surface measure on $\partial \{u > \lambda\}$ and $\dif 1_{\{u > \lambda\}}$ is conormal to $\partial \{u > \lambda\}$, we compute using \cite[Proposition 2.7(ii)]{Anzellotti1983} that
\begin{align*}
\int_{-\infty}^\infty \int_{\partial \{u > \lambda\}} \chi F \dif \lambda = \int_{-\infty}^\infty \int_M \chi \dif 1_{\{u > \lambda\}} \wedge F \dif \lambda &= \int_M \chi \dif u \wedge F. \qedhere 
\end{align*}
\end{proof}

\begin{proposition}[Poincar\'e's lemma]\label{Hodge theorem}
Suppose that $U$ is a deformed ball, and let $F \in L^\infty(U, \Omega^\ell_{\rm cl})$.
Then there exists $A \in C^\alpha(U, \Omega^{\ell - 1})$ (for every $0 \leq \alpha < 1$) such that $\dif A = F$.
\end{proposition}
\begin{proof}
There exists a solution operator $T$ of the equation $\dif A = F$ such that for any $1 < p < \infty$, $T$ is a bounded linear map $L^p(U, \Omega^\ell_{\rm cl}) \to W^{1, p}(U, \Omega^{\ell - 1})$ \cite{Costabel2010}.
The result now follows from the Sobolev embedding theorem if we take $p$ large enough.
\end{proof}


\printbibliography

\end{document}
