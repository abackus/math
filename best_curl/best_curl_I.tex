\documentclass[reqno,11pt]{amsart}
\usepackage[letterpaper, margin=1in]{geometry}
\RequirePackage{amsmath,amssymb,amsthm,graphicx,mathrsfs,url,slashed,subcaption}
\RequirePackage[usenames,dvipsnames]{xcolor}
\RequirePackage[colorlinks=true,linkcolor=Red,citecolor=Green]{hyperref}
\RequirePackage{amsxtra}
\usepackage{cancel}
\usepackage{tikz, quiver, wrapfig}
%\usepackage[T1]{fontenc}

% \setlength{\textheight}{9.3in} \setlength{\oddsidemargin}{-0.25in}
% \setlength{\evensidemargin}{-0.25in} \setlength{\textwidth}{7in}
% \setlength{\topmargin}{-0.25in} \setlength{\headheight}{0.18in}
% \setlength{\marginparwidth}{1.0in}
% \setlength{\abovedisplayskip}{0.2in}
% \setlength{\belowdisplayskip}{0.2in}
% \setlength{\parskip}{0.05in}
%\renewcommand{\baselinestretch}{1.05}

\title{Laminations and calibrations as limiting solutions of $p$-Laplacian systems}
\author{Aidan Backus}
\address{Department of Mathematics, Brown University}
\email{aidan\_backus@brown.edu}
\date{\today}

\newcommand{\NN}{\mathbf{N}}
\newcommand{\ZZ}{\mathbf{Z}}
\newcommand{\QQ}{\mathbf{Q}}
\newcommand{\RR}{\mathbf{R}}
\newcommand{\CC}{\mathbf{C}}
\newcommand{\DD}{\mathbf{D}}
\newcommand{\PP}{\mathbf P}
\newcommand{\MM}{\mathbf M}
\newcommand{\II}{\mathbf I}
\newcommand{\Hyp}{\mathbf H}
\newcommand{\Sph}{\mathbf S}
\newcommand{\Group}{\mathbf G}
\newcommand{\GL}{\mathbf{GL}}
\newcommand{\Orth}{\mathbf{O}}
\newcommand{\SpOrth}{\mathbf{SO}}
\newcommand{\Ball}{\mathbf{B}}

\newcommand*\dif{\mathop{}\!\mathrm{d}}

\DeclareMathOperator{\card}{card}
\DeclareMathOperator{\dist}{dist}
\DeclareMathOperator{\id}{id}
\DeclareMathOperator{\Hom}{Hom}
\DeclareMathOperator{\PD}{PD}
\DeclareMathOperator{\coker}{coker}
\DeclareMathOperator{\supp}{supp}
\DeclareMathOperator{\sech}{sech}
\DeclareMathOperator{\Teich}{Teich}
\DeclareMathOperator{\tr}{tr}

\newcommand{\Leaves}{\mathscr L}
\newcommand{\Lagrange}{\mathscr L}
\newcommand{\Hypspace}{\mathscr H}

\newcommand{\Chain}{\underline C}


\newcommand{\weakto}{\rightharpoonup}

\newcommand{\Two}{\mathrm{I\!I}}
\newcommand{\Ric}{\mathrm{Ric}}

\newcommand{\normal}{\mathbf n}
\newcommand{\radial}{\mathbf r}
\newcommand{\evect}{\mathbf e}
\newcommand{\vol}{\mathrm{vol}}

\newcommand{\diam}{\mathrm{diam}}
\newcommand{\Ell}{\mathrm{Ell}}
\newcommand{\inj}{\mathrm{inj}}
\newcommand{\Lip}{\mathrm{Lip}}
\newcommand{\MCL}{\mathrm{MCL}}
\newcommand{\Riem}{\mathrm{Riem}}

\newcommand{\Mass}{\mathbf M}
\newcommand{\Comass}{\mathbf L}

\newcommand{\Min}{\mathrm{Min}}
\newcommand{\Max}{\mathrm{Max}}

\newcommand{\dfn}[1]{\emph{#1}\index{#1}}

\renewcommand{\Re}{\operatorname{Re}}
\renewcommand{\Im}{\operatorname{Im}}

\newcommand{\loc}{\mathrm{loc}}
\newcommand{\cpt}{\mathrm{cpt}}

\DeclareMathOperator*{\essinf}{ess\,inf}
\DeclareMathOperator*{\esssup}{ess\,sup}

\def\Japan#1{\left \langle #1 \right \rangle}

\newtheorem{theorem}{Theorem}[section]
\newtheorem{badtheorem}[theorem]{``Theorem"}
\newtheorem{prop}[theorem]{Proposition}
\newtheorem{lemma}[theorem]{Lemma}
\newtheorem{sublemma}[theorem]{Sublemma}
\newtheorem{proposition}[theorem]{Proposition}
\newtheorem{corollary}[theorem]{Corollary}
\newtheorem{conjecture}[theorem]{Conjecture}
\newtheorem{axiom}[theorem]{Axiom}
\newtheorem{assumption}[theorem]{Assumption}

\newtheorem{mainthm}{Theorem}
\renewcommand{\themainthm}{\Alph{mainthm}}

\newtheorem{claim}{Claim}[theorem]
\renewcommand{\theclaim}{\thetheorem\Alph{claim}}
% \newtheorem*{claim}{Claim}

\theoremstyle{definition}
\newtheorem{definition}[theorem]{Definition}
\newtheorem{remark}[theorem]{Remark}
\newtheorem{example}[theorem]{Example}
\newtheorem{notation}[theorem]{Notation}

\newtheorem{exercise}[theorem]{Discussion topic}
\newtheorem{homework}[theorem]{Homework}
\newtheorem{problem}[theorem]{Problem}

\makeatletter
\newcommand{\proofpart}[2]{%
  \par
  \addvspace{\medskipamount}%
  \noindent\emph{Part #1: #2.}
}
\makeatother



\numberwithin{equation}{section}


% Mean
\def\Xint#1{\mathchoice
{\XXint\displaystyle\textstyle{#1}}%
{\XXint\textstyle\scriptstyle{#1}}%
{\XXint\scriptstyle\scriptscriptstyle{#1}}%
{\XXint\scriptscriptstyle\scriptscriptstyle{#1}}%
\!\int}
\def\XXint#1#2#3{{\setbox0=\hbox{$#1{#2#3}{\int}$ }
\vcenter{\hbox{$#2#3$ }}\kern-.6\wd0}}
\def\ddashint{\Xint=}
\def\dashint{\Xint-}

\usepackage[backend=bibtex,style=alphabetic,giveninits=true]{biblatex}
\renewcommand*{\bibfont}{\normalfont\footnotesize}
\addbibresource{best_curl.bib}
\renewbibmacro{in:}{}
\DeclareFieldFormat{pages}{#1}

\newcommand\todo[1]{\textcolor{red}{TODO: #1}}


\begin{document}
\begin{abstract}
Motivated by Thurston and Daskalopoulos--Uhlenbeck's approach to Teichm\"uller theory, we study the behavior of $q$-harmonic functions and their $p$-harmonic conjugates in the limit as $q \to 1$, where $1/p + 1/q = 1$.
The $1$-Laplacian is already known to give rise to laminations by minimal hypersurfaces; we show that the limiting $p$-harmonic conjugates converge to calibrations $F$ of the laminations.
Moreover, we show that the laminations which are calibrated by $F$ are exactly those which arise from the $1$-Laplacian.
We also explore the limiting dual problem as a model problem for the optimal Lipschitz extension problem, which exhibits behavior rather unlike the scalar $\infty$-Laplacian.
In a companion work, we will apply the main result of this paper to associate to each class in $H^{d - 1}$ a lamination in a canonical way, and study the duality of the stable norm on $H_{d - 1}$.
\end{abstract}

\maketitle

%%%%%%%%%%%%%%%%%%%%%%%%%%%%%%%%%%%%%%%%%%%%%%%%%%%%%%%
\section{Introduction}
\subsection{Motivation: Calibrated laminations and Teichm\"uller theory}
Let $M$ be a closed surface of genus $\gamma \geq 2$.
Thurston introduced a Finsler metric on the Teichm\"uller space $\mathscr T_\gamma$ of hyperbolic structures on $M$ \cite{Thurston98}.
Let $\rho, \sigma$ be two hyperbolic structures on $M$. 
Let $f: M \to M$ minimize its Lipschitz constant
$$\Lip(f) := \sup_{\substack{x, y \in M \\ x \neq y}} \frac{\dist_{(M, \rho)}(f(x), f(y))}{\dist_{(M, \sigma)}(x, y)}$$
among all maps homotopic to $\id_M$, and let $L(\rho, \sigma) := \Lip(f)$.
Then the \dfn{Thurston metric} on $\mathscr T_\gamma$ is $\log L$.

Thurston showed that his metric is intimately related to the structure of the geodesic laminations on $M$.
Any geodesic lamination $\lambda$ on $(M, \rho)$ can be deformed to a geodesic lamination on $(M, \sigma)$; the deformation will stretch each leaf of $\lambda$ by some factor which we call the \dfn{stretch ratio} of $\lambda$.
If $K(\rho, \sigma)$ is the supremum of the stretch ratios over all geodesic laminations, then Thurston showed that
\begin{equation}\label{Thurston MFMC formula}
L(\rho, \sigma) = K(\rho, \sigma).
\end{equation}
Furthermore, he constructed a \dfn{canonical maximally stretched lamination}, which every lamination which realizes the supremum in the definition of $K(\rho, \sigma)$ embeds in.

Thurston's proof of (\ref{Thurston MFMC formula}) is a \emph{tour de force} of geometric topology, but he conjectured that there should be an easier and more general proof \cite[Abstract]{Thurston98}:

\begin{quote}
I currently think that a characterization of minimal stretch maps should be possible in a considerably more general context (in particular, to include some version for all Riemannian surfaces), and it should be feasible with a simpler proof based on more general principles -- in particular, the max flow min cut principle, convexity, and $L^0 \leftrightarrow L^\infty$ duality. \dots
I also expect that this theory fits into a context including $L^p$ comparisons \dots. 
\end{quote}

This conjecture has recently been proven by Daskalopoulos and Uhlenbeck \cite{daskalopoulos2022,daskalopoulos2023,uhlenbeck2023noether}.
For our purposes, the salient parts of the proof can already been seen in the toy problem of optimal Lipschitz maps to $\Sph^1$ \cite{daskalopoulos2020transverse}.
Let $M$ be a closed hyperbolic surface, let $\rho \in H^1(M, \ZZ)$, and let $u: M \to \Sph^1$ be a viscosity solution of the \dfn{$\infty$-Laplacian}
\begin{equation}\label{infinity laplacian}\begin{cases}
\langle \nabla^2 u, \dif u \otimes \dif u\rangle = 0, \\
[\dif u] = \rho.
\end{cases}\end{equation}
Then $u$ minimizes $\Lip(u) = \|\dif u\|_{L^\infty}$ among all maps representing $\rho$, and $u$ is approximated by solutions of the $p$-Laplacian in the limit $p \to \infty$.
The convex dual problem to the $p$-Laplacian on a surface is the $q$-Laplacian, where $1/p + 1/q = 1$, and the solution $v_q$ of the dual problem converges to a function of least gradient $v$ -- that is, $v$ minimizes its total variation $\int_M \star |\dif v|$.
The analogue of (\ref{Thurston MFMC formula}) follows from their main theorem, which asserts that for every solution $u$ of (\ref{infinity laplacian}) and every dual function $v$ of least gradient, there is a geodesic lamination $\lambda$, such that $|\dif u|$ attains its maximum on $\lambda$ and $\dif v$ is a Ruelle-Sullivan current for $\lambda$.

It is is convenient to put the above results in the framework of calibrated geometry.
Recall that the \dfn{comass} $\Comass(F)$ of a $d - 1$-form $F$ is $\|F\|_{L^\infty}$.
The \dfn{mass} $\Mass$ is the dual norm on the $d - 1$-currents.
A \dfn{calibration} is a closed $d - 1$-form $F$ such that $\Comass(F) = 1$, and a $d - 1$-current $T$ is \dfn{$F$-calibrated} if 
$$\int_T F = \Mass(T).$$
In that case, we think of $F$ as a ``certificate'' that $T$ minimizes its mass, for by Stokes' theorem, if $S$ is a $d$-current, then $\int_{\partial S} F = 0$ and so
$$\Mass(T) = \int_T F = \int_{T + \partial S} F \leq \Mass(T + \partial S).$$

Let $u$ be a solution of (\ref{infinity laplacian}) and let $\lambda$ be the geodesic lamination furnished by \cite{daskalopoulos2020transverse}.
After normalizing $\Lip(u) = 1$, one can think of $\dif u$ as a calibration which calibrates the leaves of $\lambda$.
This theorem fits into a circle of general results suggesting a general duality theorem, where on one side we have calibrations arising from an $L^\infty$ variational problem, and on the other side we have Ruelle-Sullivan currents arising from functions of least gradient:
\begin{enumerate}
\item In my previous work \cite{BackusCML}, I showed that every function $u$ of least gradient induces a Ruelle-Sullivan current $\dif u$ for the lamination of area-minimizing hypersurfaces whose leaves are the level sets of $u$.
\item Using approximation by the $q$-Laplacian, Maz\'on, Rossi, and Segura de L\'eon \cite{Mazon14} showed that every function of least gradient on a compact domain is dual to a calibration $d - 1$-form, though they did not use this language.
\item Bangert and Cui \cite{bangert_cui_2017} showed that if $F$ is a continuous calibration $d - 1$-form on a closed manifold of dimension $d \leq 7$, and $F$ minimizes its comass in its cohomology class, then there is a measured lamination $\lambda$ on $M$ such that $F$ calibrates the leaves of $\lambda$.
\end{enumerate}

The present paper unifies the results of \cite{daskalopoulos2020transverse,bangert_cui_2017}; our main result is also complementary to \cite{Mazon14}.
To state our result, recall that the comass and mass induce norms on cohomology and homology.
The \dfn{stable norm} $\Mass(\omega)$ of $\omega \in H_{d - 1}(M, \RR)$ is the infimum over all currents $T$ representing $\omega$ of $\Mass(\omega)$.
Its dual norm is the \dfn{costable norm} $\Comass$ on $H^{d - 1}(M, \RR)$, which is the infimum over forms of their comasses.
These norms were introduced by Federer \cite{Federer1974} and studied further by Gromov \cite{gromov2007metric}.
Work of Auer and Bangert \cite{Auer01}, which was later used by \cite{bangert_cui_2017}, shows that the stable norm has a particularly rich duality theory in codimension $1$.

We shall say that a $d - 1$-form is \dfn{tight} if is a variational solution of a certain PDE which generalizes the $\infty$-Laplacian; we shall make this precise later.
Our main theorem roughly says:

\begin{theorem}
Let $M$ be a closed oriented Riemannian manifold of dimension $2 \leq d \leq 7$, and let $\rho \in H^{d - 1}(M, \RR)$ have costable norm $\Comass(\rho) = 1$.
Then there is a tight calibration $d - 1$-form $F$ which represents $\rho$, and a locally defined nonconstant function of least gradient $u$ such that 
$$\dif u \wedge F = \star |\dif u|.$$
The level sets of $u$ are the leaves of a measured oriented lamination $\kappa$ whose leaves are calibrated by $F$.
Furthermore, if $\lambda$ ranges over measured oriented laminations, 
$$\sup_\lambda \frac{\langle \rho, [\lambda]\rangle}{\Mass(\lambda)} = \frac{\langle \rho, [\kappa]\rangle}{\Mass(\kappa)} = 1.$$
\end{theorem}

In a companion paper \cite{BackusBest2}, we turn to the analogue of Thurston's canonical maximally stretched lamination.
We shall associate every $\rho \in H^{d - 1}(M, \RR)$ such that $\Comass(\rho) = 1$ a lamination $\lambda_\rho$, such that every lamination calibrated by forms in $\rho$ embeds in $\lambda_\rho$.

%%%%%%%%%%%%%%%%%%
\subsection{Motivation: \texorpdfstring{$\infty$-harmonic}{infinity-harmonic} maps}
A basic problem in metric geometry is, given two metric spaces $M, N$ and a space $\mathscr F$ of Lipschitz maps $M \to N$, to find $u \in \mathscr F$ which ``stretches $M$ as little as possible.''
When $M$ is a compact euclidean domain, $N = \RR$, and $\mathscr F$ is the set of Lipschitz extensions of a Lipschitz function on $\partial M$, then this problem is solved by the $\infty$-Laplacian \cite{Crandall2008}.
To be more precise, given a Lipschitz function $f$ on $\partial M$, the unique viscosity solution of
$$\begin{cases}
	\langle \nabla^2 u, \dif u \otimes \dif u\rangle = 0, \\
	u|_{\partial M} = f.
\end{cases}$$
is the unique Lipschitz extension of $f$ to $M$ such that for every open $U \subseteq M$ and every Lipschitz function $v: U \to \RR$ such that $u|_{\partial U} = v|_{\partial U}$, $\Lip(u|_U) \leq \Lip(v)$.

If $N$ is a Riemannian manifold of dimension $\geq 2$, however, the picture changes.
As intimated by the use of viscosity solutions, the study of the $\infty$-Laplacian is based on the maximum principle, which seems useless in this setting.
In the Daskalopolous--Uhlenbeck approach to Teichm\"uller theory \cite{daskalopoulos2022,daskalopoulos2023}, an ``$\infty$-harmonic map'' is a map which can be approximated weakly in $W^{1, r}$ by solutions of a certain generalized $p$-Laplacian.
One has little pointwise control of the limiting map, which causes frequent difficulty.
A proof of the local minimization of the Lipschitz constant for Daskalopolous--Uhlenbeck $\infty$-harmonic maps remains open.

Other definitions of $\infty$-harmonic maps are available \cite{Sheffield12}.\footnote{Caveat lector: There are various definitions of ``$\infty$-harmonic morphism'' in the literature, for example \cite{Ou12}, which have nothing to do with the optimal Lipschitz extension problem.}
These definitions emphasize local minimality of the Lipschitz constant, but the well-posedness theory is not in place.
Furthermore, their relationship with Daskalopolous--Uhlenbeck $\infty$-harmonic maps remains unclear.

It is natural to look at other manifold-valued $L^\infty$ variational problems, besides the optimal Lipschitz extension problem, and hope that they serve as an easier toy problem.
The proof of our main theorem is based on an analogue of the $\infty$-Laplacian for differential forms; a solution, which we call \dfn{tight}, is required to minimize its comass in the same $L^p$-approximation sense that Daskalopolous--Uhlenbeck $\infty$-harmonic maps minimize their Lipschitz constants.
Unlike $\infty$-harmonic maps, however \cite{Sheffield12}, one can write down an Euler-Lagrange-Aronsson equation which is satisfied by every smooth tight form, and conversely, under a suitable integrability assumption, a smooth solution of the equation is tight.
This allows us to prove a rather strong negative result if one hopes to easily generalize the theory of $\infty$-Laplacian to higher-dimensional targets: we show that solutions to the Euler-Lagrange-Aronsson equation are nonunique, and the equation admits smooth solutions which have nothing to do with the minimization problem.

%%%%%%%%%%%%%%%%%%%%%%%%%%%%%
\subsection{Tight forms and least gradient functions}
We now precisely state our main results.
Let $M$ be a closed Riemannian manifold of dimension $d \geq 2$, and let $\tilde M \to M$ be the universal covering.
Let $1 < p < \infty$, $1/p + 1/q = 1$, and $\alpha \in \Hom(\pi_1(M), \RR)$.
If a function $u: \tilde M \to \RR$ is $\alpha$-equivariant, then $\dif u$ drops to a closed $1$-form on $M$ whose cohomology class is (the abelianization of) $\alpha$.
We shall be interested in the $q$-Laplacian
\begin{equation}\label{q harmonic}
\begin{cases}
\dif^*(|\dif u|^{q - 2} \dif u) = 0, \\
[\dif u] = \alpha
\end{cases}
\end{equation}
and its convex dual problem
\begin{equation}\label{p tight}
\begin{cases}
\dif F = 0, \\ 
\dif^*(|F|^{p - 2} F) = 0, \\
[F] = \rho.
\end{cases}
\end{equation}
Here $F$ is a $d - 1$-form on $M$, $[F]$ is the cohomology class of $F$, and $\rho \in H^{d - 1}(M, \RR)$ is a class which depends on $\alpha$.
Recall that solutions of (\ref{q harmonic}) are called \dfn{$q$-harmonic}; we say that solutions of (\ref{p tight}) are \dfn{$p$-tight}.
If $\rho$ was chosen so that (\ref{q harmonic}) and (\ref{p tight}) are convex dual problems, then we have 
\begin{equation}\label{harmonic conjugate}
\dif u = (-1)^{d - 1} |F|^{p - 2} \star |F|
\end{equation}
and we say that $(F, u)$ is a \dfn{dual pair of a $p$-tight form and $q$-harmonic function}.

We now recall that the \dfn{comass} $\Comass(F)$ of a differential $k$-form $F$ is the supremum over all $k$-dimensional submanifolds $\Sigma \subseteq M$ of $\vol(\Sigma)^{-1} \int_\Sigma F$.
When $k = d - 1$, we can view $F$ as a vector field $\star F^\sharp$, and it follows that $\Comass(F) = \|F\|_{L^\infty}$.
The \dfn{mass} $\Mass$ is the dual norm on the $k$-currents.
If $u$ is a function of bounded variation, then $\dif u$ is naturally a $d - 1$-current defined by 
$$\int_M \dif u \wedge \varphi = -\int_M u \dif \varphi$$
for every smooth compactly supported $d - 1$-form $\varphi$.

\begin{definition}
Let $u \in BV(\tilde M, \RR)$ be a $\pi_1(M)$-equivariant function.
Suppose that, for every $\pi_1(M)$-invariant function $v$,
$$\Mass(\dif u) \leq \Mass(\dif u + \dif v).$$
Then $u$ has \dfn{least gradient}.
\end{definition}

Our first theorem is a summary of Propositions \ref{existence infinity} and \ref{existence 1}, and describes the limiting behavior of the $p$-tight forms and their $q$-harmonic conjugates as $p \to \infty$.

\begin{mainthm}\label{existence of infinity tight forms}
Let $\rho \in H^{d - 1}(M, \RR)$ be a nonzero cohomology class.
Let $(F_p, u_q)$ be the family of dual pairs of $p$-tight forms and $q$-harmonic functions, suitably normalized, with $[F_p] = \rho$, $1/p + 1/q = 1$, and $1 < p < \infty$.
Then there exists a pair $(F, u)$ such that as $p \to \infty$ along a subsequence, $F_p \weakto F$ in $L^r$ for any $1 \leq r < \infty$, and $u_q \weakto^* u$ in $BV$, such that
\begin{equation}\label{best comass}
\Comass(F) = \Comass(\rho),
\end{equation}
the product of distributions $\dif u \wedge F$ is well-defined,
\begin{equation}\label{max flow min cut}
\Comass(F) \star |\dif u| = \dif u \wedge F,
\end{equation}
and $u$ is a nonconstant function of least gradient.
\end{mainthm}

If a closed $d - 1$-form $F$ is a limit of $p$-tight forms $F_p$, as in Theorem \ref{existence of infinity tight forms}, we say that $F$ is a \dfn{tight} form.
Thus the content of Theorem \ref{existence of infinity tight forms} is that every cohomology class contains a tight form, every tight form minimizes its $L^\infty$ norm, and every tight form $F$ has a dual function $u$ of least gradient, in the sense that $(F, u)$ satisfies the relation (\ref{max flow min cut}).

Suppose that $\Comass(\rho) = 1$.
Then (\ref{best comass}) asserts that the tight form $F$ is a calibration, and (\ref{max flow min cut}) asserts that every level set of $u$ is $F$-calibrated.
Thus $F$ certificates the fact that $u$ has least gradient.
This allows us, in the proof of Theorem \ref{existence of infinity tight forms}, to show that $u$ has least gradient without a careful analysis of the behavior of $u_q$ as $q \to 1$ which appeared in the previous works \cite{daskalopoulos2020transverse,daskalopoulos2022}.

Since Theorem \ref{existence of infinity tight forms} implies that every comass-minimizing calibration calibrates a function of least gradient, it complements the theorem of Maz\'on, Rossi, and Segura de Le\'on \cite{Mazon14}:

\begin{theorem}\label{Mazon MFMC}
Let $\Omega$ be a compact euclidean domain of dimension $d \geq 2$ and $u \in BV(\Omega, \RR)$.
Then $u$ is a function of least gradient (among all functions with the same boundary values) iff there is a calibration $d - 1$-form $F$ such that 
$$\Mass(\dif u) = \int_\Omega \dif u \wedge F.$$
\end{theorem}

In the proof of Theorem \ref{Mazon MFMC}, and in the applied mathematics literature on total variation more generally, one does not work with $F$, but its dual vector field $\star F^\sharp$, which is known as the \dfn{Cahn-Hoffman field} \cite{giga2024topics}.
The proof of Theorem \ref{Mazon MFMC} applies even for the equivariant problem we studied in Theorem \ref{existence of infinity tight forms}.
However, we shall show in \S\ref{boundaries bad} that the analogue of Theorem \ref{existence of infinity tight forms} is false for the boundary-value problem.

In Appendix \ref{duality derivation}, we make precise our claim above that the $p$-tight problem (\ref{p tight}) is convex dual to the $q$-Laplacian.
Convex duality for the least gradient problem, total variation flow, and similar problems have been studied recently by G\'orny and Maz\'on \cite{górny2022dualitybased}.
However, if one directly applies the convex duality theorem to the least gradient problem, one merely concludes the existence of a dual calibration.
Our approach, which studies the convex duality of the $q$-Laplacian, yields that the calibration is in fact tight.

%%%%%%%%%%%%%%%%%
\subsection{Calibrations of measured laminations}\label{lamination intro sec}
To give a more geometric interpretation of Theorem \ref{existence of infinity tight forms}, we recall the language of laminations.
See \cite{BackusCML} for more details.

\begin{definition}
Let $I \subset \RR$ be an interval and $J \subset \RR^{d - 1}$ a box. Then:
\begin{enumerate}
\item A (codimension-$1$, Lipschitz) \dfn{laminar flow box} is a Lipschitz coordinate chart $\Psi: I \times J \to M$ and a compact set $K \subseteq I$, called the \dfn{local leaf space}, such that for each $k \in K$, $\Psi|_{\{k\} \times J}$ is a $C^1$ embedding, and the leaf $\Psi(\{k\} \times J)$ is a $C^1$ complete hypersurface in $\Psi(I \times J)$.
\item Two laminar flow boxes belong to the same \dfn{laminar atlas} if the transition map preserves the local leaf spaces.
\item A \dfn{lamination} $\lambda$ is a closed nonempty set $\supp \lambda$ and a maximal laminar atlas $\{(\Psi_\alpha, K_\alpha): \alpha \in A\}$ such that
$$\supp \lambda \cap \Psi_\alpha(I \times J) = \Psi_\alpha(K_\alpha \times J).$$
\end{enumerate}
\end{definition}

\begin{definition}
Let $\lambda$ be a lamination with laminar atlas $\{(\Psi_\alpha, K_\alpha): \alpha \in A\}$. Then:
\begin{enumerate}
\item If every leaf of $\lambda$ has zero mean curvature, then $\lambda$ is \dfn{minimal}. If in addition $d = 2$, then $\lambda$ is \dfn{geodesic}.
\item Let $F$ be a calibration. If every leaf of $\lambda$ is $F$-calibrated, then $\lambda$ is \dfn{$F$-calibrated}.
\item If the transition maps $\Psi_\alpha^{-1} \circ \Psi_\beta$ are orientation-preserving, then $\lambda$ is \dfn{oriented}.
\item A \dfn{transverse measure} $\mu$ to $\lambda$ consists of Radon measures $\mu_\alpha$ on each local leaf space $K_\alpha$, such that the transition maps $\Psi_\alpha^{-1} \circ \Psi_\beta$ send $\mu_\beta$ to $\mu_\alpha$, and $\supp \mu_\alpha = K_\alpha$.
The pair $(\lambda, \mu)$ is a \dfn{measured lamination}.
\item Suppose that $\lambda$ is measured oriented, and let $\{\chi_\alpha: \alpha \in A\}$ be a partition of unity subordinate to $\{\Psi_\alpha(I \times J): \alpha \in A\}$. The \dfn{Ruelle-Sullivan current} $T_\lambda$ acts on $\varphi \in C^0_\cpt(M, \Omega^{d - 1})$ by 
$$\int_M T_\lambda \wedge \varphi := \sum_{\alpha \in A} \int_{K_\alpha} \left[\int_{\{k\} \times J} (\Psi_\alpha^{-1})^* (\chi_\alpha \varphi)\right] \dif \mu_\alpha(k).$$
\item Suppose that $\lambda$ is measured oriented. The \dfn{mass} $\Mass(\lambda) := \Mass(T_\lambda)$ and the \dfn{homology class} $[\lambda] := [T_\lambda]$.
\end{enumerate}
\end{definition}

In recent work, I interpreted functions of least gradient in terms of their level sets, which form a lamination:

\begin{theorem}[{\cite{BackusCML}}]\label{1 harmonic is MOML}
Suppose that $d \leq 7$ and $u$ is a $\pi_1(M)$-equivariant function on $\tilde M$.
The following are equivalent:
\begin{enumerate}
\item $u$ has least gradient.
\item There is a measured oriented minimal lamination $\lambda_u$, which minimizes its mass in its homology class, whose leaves are all of the form $\partial \{u > y\}$ or $\partial \{u < y\}$, $y \in \RR$, and whose Ruelle-Sullivan current is $\dif u$.
\end{enumerate}
\end{theorem}

By Theorem \ref{existence of infinity tight forms}, if $d \leq 7$, then every nonzero class $\rho \in H^{d - 1}$ contains a tight form $F$, which is related to a function $u$ of least gradient by (\ref{max flow min cut}).
We say that the lamination $\lambda_u$ is a \dfn{measured stretch lamination} associated to $\rho$.
In general, there may be many measured stretch laminations, but our next theorem, proven in \S\ref{proof of Theorem B}, characterizes the measured stretch laminations.

\begin{mainthm}\label{lams are calibrated}
Suppose that $d \leq 7$.
Let $\rho \in H^{d - 1}(M, \RR)$ be nonzero.
Let $\kappa$ be a measured stretch lamination for $\rho$, and let $[F] = \rho$ satisfy $\Comass(F) = \Comass(\rho)$.
Then $\kappa$ is $F/\Comass(\rho)$-calibrated, and for $\lambda$ ranging over measured oriented laminations,
\begin{equation}\label{duality between stable and comass}
\Comass(\rho) = \sup_\lambda \frac{\langle \rho, [\lambda]\rangle}{\Mass(\lambda)} = \frac{\langle \rho, [\kappa]\rangle}{\Mass(\kappa)}.
\end{equation}
Conversely, if a measured oriented lamination $\lambda$ attains the maximum in (\ref{duality between stable and comass}), then $\lambda$ is a measured stretch lamination.
\end{mainthm}

As a corollary, we show that every measured oriented calibrated lamination is Lipschitz.

Given our previous work \cite{BackusCML} and Theorem \ref{existence of infinity tight forms}, the only difficulty in the proof of Theorem \ref{lams are calibrated} is to verify that it is meaningful to ask if a lamination be $F$-calibrated when $F$ is only measurable, but not continuous.
This technicality aside, the proof of Theorem \ref{lams are calibrated} is based on (\ref{max flow min cut}), which says that if $F$ is tight and $\Comass(\rho) = 1$, then $F$ calibrates $\kappa$.
Since calibration is a cohomological invariant, any other calibration representing $\rho$ must also calibrate $\kappa$.
Conversely, if $F$ calibrates $\lambda$, then $\lambda$ must arise from some function of least gradient satisfying (\ref{max flow min cut}).

Theorem \ref{lams are calibrated} immediately implies the Bangert--Cui theorem \cite{bangert_cui_2017}:

\begin{theorem}\label{Bangert MFMC}
Let $M$ be a closed oriented Riemannian manifold of dimension $2 \leq d \leq 7$, and let $F$ be a continuous calibration $d - 1$-form on $M$ which minimizes its comass in its cohomology class.
Then there is a measured oriented lamination $\lambda$ whose leaves are calibrated by $F$.
\end{theorem}

In addition to clarifying the role of the $p$-Laplacian, Theorem \ref{lams are calibrated} is stronger than Theorem \ref{Bangert MFMC} because we need not assume that $\rho$ contains a continuous calibration.
In general, it seems unlikely that $\Comass(\rho) = 1$ implies that $\rho$ contains a continuous calibration, though this is known to hold when $d = 2$ \cite{Evans08} or when $M$ is Ricci-flat. 

Theorems \ref{existence of infinity tight forms} and \ref{lams are calibrated} together imply the theorem of Daskalopolous and Uhlenbeck \cite{daskalopoulos2020transverse} on $\infty$-harmonic maps to $\Sph^1$:

\begin{theorem}\label{Daskalopolous MFMC}
Let $M$ be a closed hyperbolic surface, and $\rho$ a nontrivial homotopy class of maps $M \to \Sph^1$.
If $v$ is an $\infty$-harmonic representative of $\rho$, then $v$ has a dual locally defined function of least gradient $u$.
Furthermore, there is a geodesic lamination $\lambda$ such that $|\dif v|$ attains its maximum on $\lambda$ and $\dif u$ is a Ruelle-Sullivan current of $\lambda$.
\end{theorem}

Indeed, we can identify homotopy classes with cohomology classes $\rho \in H^1(M, \ZZ)$, and rescale the metric so that $\Comass(\rho) = 1$.
Then $\lambda$ can be taken to be any measured stretch lamination associated to $\rho$.

The duality relation (\ref{duality between stable and comass}) has already been interpreted in the string theory literature as a continuous analogue of the max flow min cut theorem \cite{Freedman_2016}, at least in the case that $\lambda$ has only one leaf.
On the other hand, one can think of $\langle \rho, [\lambda]\rangle/\Mass(\lambda)$ as the amount that $\lambda$ is ``stretched'' by a representative of $\rho$, or in other words we can think of it as analogous to Thurston's functional $K$ \cite[\S5.3]{daskalopoulos2020transverse}.
So (\ref{duality between stable and comass}) is an analogue of Thurston's formula (\ref{Thurston MFMC formula}), and Theorem \ref{lams are calibrated} can be viewed as an explanation for why the max flow min cut theorem should appear in Thurston's conjecture.

The analogy with Thurston's approach to Teichm\"uller theory can be made even stronger.
Thurston used his canonical maximally stretched lamination to study the duality of the tangent and cotangent bundles of Teichm\"uller space \cite[\S10]{Thurston98}.
In our situation, while there may be many measured stretch laminations $\rho$, we will embed them all in a \dfn{canonical calibrated lamination} $\lambda(\rho)$, as I will show in an upcoming companion paper \cite{BackusBest2}.
Moreover, I will show that $\lambda(\rho)$ can be used to prove facts about the duality between the costable and stable norms, some of which were claimed but not proven by Auer and Bangert \cite{Auer01}.

Theorem \ref{lams are calibrated} is based on the fact that the level sets of a function of least gradient are area-minimizing hypersurfaces, but one can seek to formulate such a result for eigenfunctions of the $1$-Laplacian instead.
The superlevel sets of the first eigenfunction are \dfn{Cheeger sets} -- sets $U$ which maximize the isoperimetric ratio $\Mass(\partial U)/\vol(U)$ \cite{Kawohl2003}.
The dual $d - 1$-forms $F$ are not closed, but instead satisfy $\dif F \geq 0$; one can also interpret this fact as a continuous max flow min cut theorem \cite{Grieser05}.
It would be interesting to study the $L^\infty$ one-sided variational problem that $F$ solves in this setting, but we shall not attempt to do so.

%%%%%%%%%%%%%%%%%
\subsection{The \texorpdfstring{$L^\infty$ variational}{L-infinity variational} model system}
In \S\ref{infinityMax} of the paper we scrutinize the $L^\infty$ variational problem defining tight forms.
At present there is not a suitable theory of viscosity solutions of PDEs for vector-valued maps, and in general, one does not expect our tight forms to be much more regular than $L^\infty$.
However, we formally derive what analytic properties tight forms \emph{should} have, in the tradition of various other papers \cite{Barron2001,Aronsson67,Sheffield12} on the $L^\infty$ calculus of variations.

To be more precise, we study $C^1$ solutions of the PDE 
\begin{equation}\label{tight Einstein}
\begin{cases}\dif F = 0, \\
	(\nabla_\alpha F_{\beta_1 \cdots \beta_{d - 1}}) F^{\beta_1 \cdots \beta_{d - 1}} {F^\alpha}_{\gamma_1 \cdots \gamma_{d - 2}} = 0.
\end{cases}
\end{equation}
We derive this equation as the formal limit of the PDE solved by the $p$-tight forms, analogous to the derivation of the $\infty$-Laplacian \cite{Aronsson67}.
Therefore tight forms are variational solutions of (\ref{tight Einstein}).

When $d = 2$, (\ref{tight Einstein}) asserts that $F = \dif v$, where $v$ is $\infty$-harmonic.
In that case it is well known that, if $F$ has no zeroes, then $\ker(\star F)$ integrates to a geodesic foliation \cite[Proof of Theorem 1.5]{Sheffield12}.
In general, (\ref{tight Einstein}) can be interpreted as asserting that the distribution $\ker(\star F)$ is ``calibrated" by $F$, though $\ker(\star F)$ need not be integrable.
If it is integrable, then we show in \S\ref{EL interpretation} the following variational interpretation which generalizes the interpretation of $\infty$-harmonic functions as absolute minimizing Lipschitz.

\begin{mainthm}\label{tight are absolute minimizers}
Let $F$ be a closed $C^1$ $d - 1$-form on a compact Riemannian manifold (possibly with boundary). Then:
\begin{enumerate}
\item If, for every sufficiently small ball $B$, $F|_B$ minimizes its $C^0$ norm among all closed forms of the same trace, then $F$ solves (\ref{tight Einstein}).
\item Suppose that, on the set $\{|F| > 0\}$, $\ker(\star F)$ is an integrable distribution.
If $F$ solves (\ref{tight Einstein}), then for every sufficiently small ball $B$, $F|_B$ minimizes its $C^0$ norm among all closed forms of the same trace.
\end{enumerate}
\end{mainthm}

We also show that the rather surprising fact that if the integrability hypothesis is removed, then $F$ is not a minimizer even globally.
This defect is entirely unlike the behavior of the scalar $\infty$-Laplacian and we predict that similar phenomena will be discovered in many other $L^\infty$ variational systems.

%%%%%%%%%%%%%%%%%%%%%%
\subsection{Acknowledgements}
I would like to thank Georgios Daskalopolous for suggesting this project, providing helpful comments, and for providing me with a draft copy of \cite{daskalopoulos2023}, Taylor Klotz for helpful discussions about contact geometry and suggesting the reference \cite{Peralta_Salas_2023}, Bernd Kawohl for suggesting the references \cite{Kawohl2003, Grieser05}, Trent Lucas for helpful comments on a previous draft, and Anatole Gaudin for suggesting the reference \cite{Costabel2010}.

This research was supported by the National Science Foundation's Graduate Research Fellowship Program under Grant No. DGE-2040433.



%%%%%%%%%%%%%%%%%%%%%%%%%%%%%
\section{Tight forms and functions of least gradient}\label{tight forms sec}
In this section we shall prove Theorem \ref{existence of infinity tight forms}, by studying properties of $p$-tight forms.
Unless noted otherwise, $M$ shall denote a closed oriented Riemannian manifold, $\tilde M \to M$ will be the universal covering.
Let $M_{\rm fun} \subseteq \tilde M$ be a fundamental domain of $M$, and let $\Gamma := \pi_1(M)$.


%%%%%%%%%%%%%%%%%%%%%%%%%%%
\subsection{Homological properties of \texorpdfstring{$q$-harmonic}{q-harmonic} functions}
Let $\alpha \in \Hom(\Gamma, \RR)$ be a representation.
It defines a cohomology class in $H^1(M, \RR)$ by Hurcewiz' theorem, which we also denote by $\alpha$.
We also think of $\alpha$ as a harmonic $1$-form sometimes, using the Hodge theorem.
If $u \in BV_\loc(\tilde M, \RR)$ is $\alpha$-equivariant in the sense that for every $\gamma \in \Gamma$ and $x \in \tilde M$,
$$u(\gamma x) = u(x) + \langle \alpha, \gamma\rangle,$$
then $\dif u$ drops to a $d - 1$-current on $M$, and its homology class is $\PD(\alpha)$.\footnote{If $\dif u$ is actually a $1$-form, then of course its cohomology class is $\alpha$, but it turns out that the homology class will be more useful for our purposes.}

We first estimate the energy of an $\alpha$-equivariant $q$-harmonic function $u$ in terms of the stable norm $\Mass(\PD(\alpha))$ and the geometry of $M$.
The argument is essentially due to Massart \cite[\S4.2]{Massart96}, who considered the case that $M$ is a hyperbolic surface.
We introduce the \dfn{maximal intersection number}
$$i_M := \sup_{\substack{\xi \in C^\infty(M, \Omega^1) \\ \eta \in C^\infty(M, \Omega^{d - 1})}} \frac{1}{\|\xi\|_{L^1} \|\eta\|_{L^1}} \int_M \xi \wedge \eta$$
which is a positive, finite constant that only depends on the Riemannian manifold $M$.

\begin{lemma}
Let $1/p + 1/q = 1$, and let $u_q: \tilde M \to \RR$ be an $\alpha$-equivariant $q$-harmonic function.
Then
\begin{equation}\label{q Laplacian Sobolev regularity estimate}
\vol(M)^{-1/p} \Mass(\alpha) \leq \|\dif u_q\|_{L^q} \leq \vol(M)^{1/q} i_M \Mass(\alpha)
\end{equation}
\end{lemma}
\begin{proof}
Let $u_\infty$ be an $\alpha$-equivariant function which minimizes its Lipschitz constant $\|\dif u_\infty\|_{L^\infty}$ among all $\alpha$-equivariant functions.\footnote{For example, we can take $u_\infty$ to be the $\infty$-harmonic map constructed by Daskalopolous and Uhlenbeck \cite[\S2]{daskalopoulos2020transverse}.}
Then, since $\dif u_q$ minimizes its $L^q$ norm, we obtain from H\"older's inequality
$$\|\dif u_q\|_{L^q} \leq \|\dif u_\infty\|_{L^q} \leq \vol(M)^{1/q} \|\dif u_\infty\|_{L^\infty}.$$
By the converse to H\"older's inequality, 
$$\|\dif u_\infty\|_{L^\infty} = \sup_{\eta \in C^\infty(M, \Omega^{d - 1})} \frac{1}{\|\eta\|_{L^1}} \int_M \dif u_\infty \wedge \eta.$$
If we now let $v_\varepsilon$ be $\alpha$-equivariant with $\|\dif v_\varepsilon\|_{L^1} \leq \Mass(\alpha) + \varepsilon$, we have by Stokes' theorem
\begin{align*}
\sup_{\eta \in C^\infty(M, \Omega^{d - 1})} \frac{1}{\|\eta\|_{L^1}} \int_M \dif u_\infty \wedge \eta 
&= \sup_{\eta \in C^\infty(M, \Omega^{d - 1})} \frac{1}{\|\eta\|_{L^1}} \int_M \dif v_\varepsilon \wedge \eta\\
&\leq i_M \|\dif v_\varepsilon\|_{L^1} \\
&\leq i_M(\Mass(\alpha) + \varepsilon).
\end{align*}
Taking $\varepsilon \to 0$, we deduce one direction of (\ref{q Laplacian Sobolev regularity estimate}).
In the other direction, we estimate using H\"older's inequality
\begin{align*}
\Mass(\alpha) &\leq \|\dif u_q\|_{L^1} \leq \|\dif u_q\|_{L^q} \vol(M)^{1/p}. \qedhere 
\end{align*}
\end{proof}

Since we are interested in the limit $q \to 1$, we must show that convergence of equivariant functions, even in a very weak function space, implies convergence of the representations.

\begin{lemma}\label{L1 convergence preserves pi1}
For each $1 < q \leq 2$, let $\alpha_q$ be a representation, let $u_q$ be an $\alpha_q$-equivariant function on $\tilde M$, and let $u \in L^1_\loc(M, \RR)$.
Suppose that $u_q \to u$ in $L^1_\loc$.
Then:
\begin{enumerate}
\item $\alpha_q \to \alpha$ for some representation $\alpha$.
\item $u$ is $\alpha$-equivariant.
\item If $\dif u_q \weakto^* \dif u$ and $\dif u_q \in L^q$, then
\begin{equation}\label{q to 1 Holder}
\Mass(\dif u) \leq \liminf_{q \to 1} \frac{1}{q} \int_M \star |\dif u_q|^q.
\end{equation}
\end{enumerate}
\end{lemma}
\begin{proof}
For each $\gamma \in \Gamma$, let
$$U_\gamma := M_{\rm fun} \cup \gamma_* (M_{\rm fun}).$$
We claim that $(\alpha_q)$ has a convergent subsequence.
To see this, we first recall that $M$ has finite Betti numbers, so $H^1(M, \RR)$ is locally compact.
Therefore, if no convergent subsequence exists, there exists a $\gamma \in \pi_1(M)$ and a subsequence along which $\langle \alpha_q, \gamma\rangle \to \infty$.
Moreover, since $u_q \to u$ in $L^1_\loc$, $\|u_q\|_{L^1(M_{\rm fun})} \leq 2\|u\|_{L^1(M_{\rm fun})}$ if $q - 1$ is small enough.
But then 
$$\|u_q\|_{L^1(\gamma_* M_{\rm fun})} = \|\gamma^* u_q\|_{L^1(M_{\rm fun})} \geq \langle \alpha_q, \gamma\rangle - \|u_q\|_{L^1(M_{\rm fun})} \geq \langle \alpha_q, \gamma\rangle - 2\|u\|_{L^1(M_{\rm fun})}$$
and taking $q \to 1$ we conclude that $(u_q)$ is not compact in $L^1(\gamma_* M_{\rm fun})$, contradicting the convergence in $L^1_\loc(\tilde M)$.
So $\alpha_q \to \alpha$ for some $\alpha \in H^1(M, \RR)$ along a subsequence.

For any $q > 1$,
\begin{align*}
\dashint_{M_{\rm fun}} \star |\gamma^* u - u - \langle \alpha, \gamma\rangle| 
&\leq \dashint_{M_{\rm fun}} \star (|\gamma^* u_q - u_q - \langle \alpha_q, \gamma\rangle| + |\gamma^* u_q - u_q| + |\gamma^* u - u|) \\
&\qquad + |\langle \alpha_q - \alpha, \gamma\rangle|.
\end{align*}
Taking $q \to 1$, we conclude that $\|\gamma^* u - u - \langle \alpha, \gamma\rangle\|_{L^1} = 0$, hence $u$ is $\alpha$-equivariant.
Thus $\alpha$ is uniquely defined and $\alpha_q \to \alpha$ along the entire subsequence.

Finally we prove (\ref{q to 1 Holder}).
Suppose that $\dif u_q \to \dif u$ in the weak topology of measures and $\dif u_q$ in $L^q$.
Then
$$\|\dif u_q\|_{L^1} = \Mass(\dif u_q).$$
So we may use Proposition \ref{portmanteau} and H\"older's inequality to estimate (where $1/p + 1/q = 1$)
\begin{align*}
\Mass(\dif u) &= \lim_{q \to 1} \Mass(\dif u_q) \leq \lim_{q \to 1} \vol(M)^{\frac{1}{p}} \|\dif u_q\|_{L^q} = \lim_{q \to 1} \frac{1}{q} \int_M \star |\dif u_q|^q. \qedhere
\end{align*}
\end{proof}

%%%%%%%%%%%%%%%%
\subsection{Duality and \texorpdfstring{$p$-tight forms}{p-tight forms}}
Let $1 < p < \infty$ and $1/p + 1/q = 1$.
We now study the $p$-tight forms -- namely, solutions of the PDE (\ref{p tight}).
We restate this PDE for convenience:
$$\begin{cases}\dif F = 0 \\
	\dif^* (|F|^{p - 2} F) = 0.
\end{cases}$$
In Appendix \ref{duality derivation}, we derive (\ref{p tight}) from the duality theorem for convex optimization.

\begin{proposition}
There is a unique $p$-tight form in each cohomology class in $H^{d - 1}(M, \RR)$.
Moreover, $p$-tight forms are minimizers of the strictly convex functional
$$J_p(F) := \frac{1}{p} \int_M \star |F|^p$$
among all forms cohomologous to them.
\end{proposition}
\begin{proof}
Strict convexity of $J_p$ on closed $L^p$ $d - 1$-forms is straightforward; since each cohomology class is an affine subspace of $L^p(M, \Omega^{d - 1})$, and hence is convex, the strict convexity on each class follows.
Since $J_p(F) \to \infty$ as $\|F\|_{L^p} \to \infty$, $J_p$ is coercive on $L^p(M, \Omega^{d - 1})$.
Therefore we have existence and uniqueness \cite[Chapter II]{Ekeland99}.
To compute the Euler-Lagrange equations for $J_p$, let $B$ be a $d-2$-form (so $F + t \dif B$ is cohomologous to $F$), so that for a minimizer $F$ of $J_p$,
$$\frac{\dif}{\dif t} J_p(F + t \dif B) = \frac{1}{p} \int_M \star \frac{\partial}{\partial t} |F + t \dif B|^p = \int_M \star |F + t \dif B|^{p - 2} \langle F + t \dif B, \dif B\rangle.$$
Setting $t = 0$, we obtain 
$$0 = \int_M \star |F|^{p - 2} \langle F, \dif B\rangle = \int_M \star \langle \dif^*(|F|^{p - 2} F), B\rangle.$$
Thus the Euler-Lagrange equations for $J_p$ are (\ref{p tight}).
\end{proof}

In Appendix \ref{duality derivation}, we show that every $q$-harmonic function defines a $p$-tight form by the mapping (\ref{dual solution}).
The mapping in the below definition, which sends a $p$-tight form to a $q$-harmonic function, is the inverse of (\ref{dual solution}).

\begin{definition}
Let $F$ be a $p$-tight form on $M$, let
\begin{equation}
\dif u := (-1)^d |F|^{p - 2} \star F, \label{inverse extremality}
\end{equation}
and let $u$ be the primitive of $\dif u$ on the universal cover $\tilde M$, which is normalized to have zero mean on a fundamental domain $M_{\rm fun}$.
Then $u$ is called the \dfn{$q$-harmonic conjugate} of the $p$-tight form $F$.
\end{definition}

Let $u$ be the $q$-harmonic conjugate of a $p$-tight form $F$.
We record that 
\begin{equation}\label{q energy is p energy}
\int_M \star |\dif u|^q = \int_M \star |F|^p.
\end{equation}
By Poincar\'e's inequality,
$$\|u\|_{W^{1, q}(M_{\rm fun})}^q \lesssim \int_M \star |\dif u|^q = \int_M \star |F|^p < \infty$$
since $F$ is $p$-tight; that is, we have $F \in L^p(M)$ and $u \in W^{1, q}_\loc(\tilde M, \RR)$, justifying any manipulations we shall make with these forms.

Our next lemma actually follows from the derivation of (\ref{p tight}) in Appendix \ref{duality derivation}.
However, it is instructive to give a proof which does not use the duality theorem as a black box.

\begin{lemma}
Let $F$ be a $p$-tight form, and let $u$ be its $q$-harmonic conjugate.
Then $u$ is $q$-harmonic, $F$ satisfies
\begin{equation}\label{dual solution}
F := -|\dif u|^{q - 2} \star \dif u.
\end{equation}
 and we have
\begin{equation}\label{strong duality}
\frac{1}{q} \int_M \star |\dif u|^q + \frac{1}{p} \int_M \star |F|^p + \int_M \dif u \wedge F = 0.
\end{equation}.
\end{lemma}
\begin{proof}
We first use
$$(p - 2)(q - 1) + (q - 2) = 0$$
to prove that
$$|\dif u|^{q - 2} \star \dif u = (-1)^d |F|^{(q - 2)(p - 1)} \star \star |F|^{p - 2} F = - |F|^{(q - 2)(p - 1) - (p - 2)} F = - F.$$
Thus we have (\ref{dual solution}), and moreover
$$\dif \star (|\dif u|^{q - 2} \dif u) = - \dif F = 0$$
so that $u$ is $q$-harmonic.
By (\ref{q energy is p energy}),
$$\frac{1}{q} \int_M \star |\dif u|^q + \frac{1}{p} \int_M \star |F|^p = \left[\frac{1}{q} + \frac{1}{p}\right] \int_M \star |F|^p = \int_M \star |F|^p.$$
But
$$\int_M \dif u \wedge F = (-1)^d \int_M |F|^{p - 2} \star F \wedge F = \int_M \star |F|^p,$$
so both sides of (\ref{strong duality}) are equal to $\int_M \star |F|^p$.
\end{proof}


%%%%%%%%%%%%%%%%%%%%%%%
\subsection{\texorpdfstring{Existence of tight forms}{Existence of tight forms}}
We now show that the $p$-tight forms in a class $\rho \in H^{d - 1}(M, \RR)$ converge to a form $F$ which minimizes its comass in $\rho$.
In other words, $\Comass(F)$ equals the costable norm $\Comass(\rho)$.
Since this is such a useful condition, we define that a closed $d - 1$-form $F$ has \dfn{best comass} if $\Comass(F) = \Comass([F])$.

\begin{lemma}
Let $1 < p < \infty$, let $F_p$ be a $p$-tight form, and let $B$ range over Lipschitz $d - 2$-forms. Then
\begin{equation}\label{infinity magnetic rules p magnetic}
	\|F_p\|_{L^p} \leq \vol(M)^{1/p} \inf_B \Comass(F + \dif B)
\end{equation}
\end{lemma}
\begin{proof}
By H\"older's inequality and the fact that $F_p$ is $p$-tight,
$$\|F_p\|_{L^p} \leq \|F + \dif B\|_{L^p} \leq \vol(M)^{1/p} \|F + \dif B\|_{L^\infty} = \vol(M)^{1/p} \Comass(F + \dif B),$$
hence the same holds for the infimum.
\end{proof}

\begin{proposition}\label{existence infinity}
Let $\rho \in H^{d - 1}(M, \RR)$.
For each $1 < p < \infty$, let $F_p$ be the $p$-tight form representing $\rho$.
Then there exists a closed $d - 1$-form $F$ such that:
\begin{enumerate}
\item $F_p \to F$ weakly in $L^r$ along a subsequence, for any $1 < r < \infty$.
\item $F$ is a best comass representative of $\rho$.
\end{enumerate}
\end{proposition}
\begin{proof}
Let $G$ be an $L^\infty$ representative of $\rho$.
By H\"older's inequality and (\ref{infinity magnetic rules p magnetic}), for every $r$,
\begin{equation}\label{uniform bounds in p by best curl}
	\|F_p\|_{L^r} \leq \vol(M)^{\frac{1}{r} - \frac{1}{p}} \|F_p\|_{L^p} \leq \vol(M)^{\frac{1}{r}} \|G\|_{L^\infty}.
\end{equation}
Thus a compactness argument gives $F_p \to F$ for some $d - 1$-form $F$, weakly in $L^r$, and 
$$\|F\|_{L^r} \leq \liminf_{p \to \infty} \|F_p\|_{L^r} \leq \vol(M)^{\frac{1}{r}} \|G\|_{L^\infty}.$$
Diagonalizing, we may assume that $F_p \to F$ weakly in $L^r$ for every such $r$, and taking $r \to \infty$, we conclude 
\begin{equation}\label{infinity magnetics have best curl}
	\|F\|_{L^\infty} \leq \|G\|_{L^\infty}.
\end{equation}
Moreover, $[F] = \lim_{p \to \infty} [F_p] = \rho$.
Since $G$ was arbitrary in (\ref{infinity magnetics have best curl}), $F$ has best comass.
\end{proof}

\begin{definition}
The $d - 1$-form $F$ of best comass in Proposition \ref{existence infinity} is called a \dfn{tight form}.
\end{definition}

It is a corollary of Proposition \ref{existence infinity} that every cohomology class is represented by a form of best comass.
This could be shown more directly using Alaoglu's theorem on the weakstar topology of $L^\infty$.
However, since $p$-tight forms are intimately related to $q$-harmonic functions, it is more convenient to work with tight forms than general forms of best comass.

\begin{lemma}\label{p tights approximate L}
Let $F_p$ be the $p$-tight representative of $\rho$. Then 
$$\lim_{p \to \infty} \|F_p\|_{L^p} = \Comass(\rho).$$
\end{lemma}
\begin{proof}
We follow \cite[Lemma 2.7]{daskalopoulos2020transverse}.
Let $F$ be a best comass representative of $\rho$, so
$$\|F\|_{L^\infty} = \Comass(F) = \Comass(\rho).$$
Since $F_p$ is $p$-tight, H\"older's inequality implies 
$$\|F_p\|_{L^p} \leq \|F\|_{L^p} \leq \vol(M)^{\frac{1}{p}} \Comass(\rho).$$
Therefore 
$$\limsup_{p \to \infty} \|F_p\|_{L^p} \leq \Comass(\rho).$$
To prove the converse, suppose that for some $\varepsilon > 0$,
$$\liminf_{p \to \infty} \|F_p\|_{L^p} \leq \Comass(\rho) - \varepsilon < \Comass(\rho).$$
Along a subsequence which attains the limit inferior, $F_p$ converges weakly in every $L^r$ to a tight form $\tilde F$ such that (by H\"older's inequality)
$$\|\tilde F\|_{L^r} \leq \liminf_{p \to \infty} \|F_p\|_{L^r} \leq \liminf_{p \to \infty} \vol(M)^{\frac{1}{r}} \|\tilde F\|_{L^\infty} \leq \vol(M)^{\frac{1}{r}} (\Comass(\rho) - \varepsilon).$$
Taking $r \to \infty$, we obtain $\Comass(\tilde F) < \Comass(\rho)$, which contradicts the definition of the costable norm $\Comass(\rho)$.
\end{proof}


%%%%%%%%%%%%%%%%%%%%
\subsection{\texorpdfstring{$1$-harmonic conjugates of tight forms}{One-harmonic conjugates of tight forms}}
We now construct the $1$-harmonic conjugates of a tight form.
In the special case that the tight form $F$ is a calibration, that is $\Comass(F) = 1$, a $1$-harmonic conjugate will be a $1$-harmonic function on the universal cover whose level sets are calibrated by $F$.

\begin{definition}
Let $F$ be a tight form of cohomology class $\rho$.
A nonconstant $\Gamma$-equivariant function of least gradient $u \in BV_\loc(\tilde M)$ is called a \dfn{$1$-harmonic conjugate} of $F$ if
\begin{equation}\label{1 extremality}
\dif u \wedge F = \Comass(\rho) \star |\dif u|.
\end{equation}
\end{definition}

By Proposition \ref{Anzellotti wedge product exists}, the wedge product $\dif u \wedge F$ exists as an Anzellotti wedge product, in particular as a Radon measure, and
$$\Mass(\dif u \wedge F) \leq \Comass(F) \Mass(\dif u) = \Comass(\rho) \Mass(\dif u).$$

\begin{lemma}\label{1 extremality implies least gradient}
Let $F$ be a tight form, and let $u$ be a $1$-harmonic conjugate of $F$.
Then $u$ has least gradient.
\end{lemma}
\begin{proof}
This is essentially a calibration argument.
Let $v$ be $\Gamma$-invariant.
Then, by (\ref{1 extremality}), Stokes' theorem, and the fact that $F$ has best comass,
$$\Mass(\dif u) \Comass(\rho) = \int_M \dif u \wedge F = \int_M (\dif u + \dif v) \wedge F \leq \Mass(\dif u + \dif v) \Comass(F) = \Mass(\dif u + \dif v) \Comass(\rho).$$
We conclude that $\Mass(\dif u) \leq \Mass(\dif v)$.
\end{proof}

We are going to derive (\ref{1 extremality}) as a limit of the duality relation (\ref{strong duality}), which itself was a consequence of the convex duality theorem.
But (\ref{inverse extremality}) blows up as $p \to \infty$, so we must ``renormalize'' it before taking the limit $q \to 1$, as in \cite[\S3.2]{daskalopoulos2020transverse}.
This goal is accomplished by the next lemma.

\begin{lemma}\label{normalizations converge}
Let $\rho \in H^{d - 1}$, and let $k_p$ satisfy 
$$k_p^{1 - p} = \int_M \star |F_p|^p$$
where $F_p$ is the $p$-tight representative of $\rho$.
Then, as $p \to \infty$, $k_p \to \Comass(\rho)^{-1}$.
\end{lemma}
\begin{proof}
We follow \cite[Lemma 3.4]{daskalopoulos2020transverse}.
By Lemma \ref{p tights approximate L},
$$\lim_{p \to \infty} k_p^{-\frac{1}{q}} = \lim_{p \to \infty} \|F_p\|_{L^p} = \Comass(\rho).$$
Taking logarithms we see that $q^{-1} \log k_p \to -\log \Comass(\rho)$, and since $q \to 1$ the claim follows.
\end{proof}

Theorem \ref{existence of infinity tight forms} is the conjunction of the following proposition and Proposition \ref{existence infinity}.

\begin{proposition}\label{existence 1}
Let $\rho \in H^{d - 1}(M, \RR)$ be nonzero, let $F_p$ be its $p$-tight representative, and let 
$$F := \lim_{p \to \infty} F_p$$
be the tight representative of $F$.
Let $1/p + 1/q = 1$, and let $u_q$ be the function on $\tilde M$ with mean zero on $M_{\rm fun}$ and
$$\dif u_q = (-1)^{d - 1} k_p^{p - 1} |F_p|^{p - 2} \star F_p.$$
Then there exists $u \in BV_\loc(\tilde M)$ such that:
\begin{enumerate}
\item Along a subsequence as $q \to 1$, $u_q \weakto^* u$ in $BV_\loc(\tilde M)$, and for every $1 \leq r < \frac{d}{d - 1}$, $u_q \to u$ in $L^r(\tilde M)$.
\item $u$ is a $1$-harmonic conjugate of $F$.
\end{enumerate}
\end{proposition}
\begin{proof}
Let $L := \Comass(\rho)$.
We first compute using H\"older's inequality and Lemma \ref{normalizations converge}
\begin{align}
\lim_{q \to 1} \|\dif u_q\|_{L^1}
&\leq \lim_{q \to 1} \vol(M)^{\frac{1}{p}} \left[\int_M \star |\dif u_q|^q\right]^{\frac{1}{q}} = \lim_{p \to \infty} \left[k_p^p \int_M \star |F_p|^p\right]^{\frac{1}{q}} \label{Rellich 1}\\
&= \lim_{p \to \infty} k_p^{\frac{1}{q}} = \lim_{p \to \infty} k_p = \frac{1}{L} \label{Rellich 2}.
\end{align}
So by Rellich's theorem, $(u_q)$ is weakly compact in $BV$ and strongly compact in $L^r$ for $1 \leq r < \frac{d}{d - 1}$.
In particular, $\dif u_q \to \dif u$ in the weak topology of measures and $u_q \to u$ weakly in $BV$ and strongly in $L^r$.
As the limit of $\Gamma$-equivariant functions, $u$ is also $\Gamma$-equivariant by Lemma \ref{L1 convergence preserves pi1}.
In particular, $\dif u$ drops to a current on $M$.
Moreover, $[\dif u_q] \to [\dif u]$, and we have the bound (\ref{q to 1 Holder}) on $\Mass(\dif u)$.

We next must check that $u$ is nonconstant.
If $u$ is constant, then it is $\Gamma$-invariant, so $[\dif u_q] \to 0$.
By (\ref{q Laplacian Sobolev regularity estimate}), $\|\dif u_q\|_{L^q} \to 0$, so by (\ref{Rellich 1}, \ref{Rellich 2}), $L = \infty$, which is absurd.
Therefore $u$ is nonconstant.

Renormalizing (\ref{strong duality}), we obtain 
$$\frac{k_p^{-p}}{q} \int_M \star |\dif u_q|^q + \frac{1}{p} \int_M \star |F_p|^p = k_p^{1 - p} \int_M \dif u_q \wedge F_p.$$
Multiplying by $k_p^p$, we have 
\begin{equation}\label{1 strong duality before limits}
	\frac{1}{q} \int_M \star |\dif u_q|^q + \frac{k_p^p}{p} \int_M \star |F_p|^p = k_p \int_M \dif u_q \wedge F_p.
\end{equation}

Let $\mu(U) := \Mass_U(\dif u)$ be the total variation measure of $\dif u$.
We claim that
\begin{equation}\label{1 strong duality}
	L\mu(M) \leq \int_M \dif u \wedge F.
\end{equation}
First, we have from (\ref{q to 1 Holder}) and (\ref{1 strong duality before limits}) that
$$\mu(M) \leq \lim_{q \to 1} \frac{1}{q} \int_M \star |\dif u_q|^q = \lim_{p \to \infty} k_p \int_M \dif u_q \wedge F_p - \lim_{p \to \infty} \frac{k_p^p}{p} \int_M \star |F_p|^p.$$
By Lemma \ref{normalizations converge},
$$\lim_{p \to \infty} \frac{k_p^p}{p} \int_M \star |F_p|^p = \lim_{p \to \infty} \frac{k_p}{p} = \frac{0}{L} = 0,$$
and
$$\lim_{p \to \infty} k_p \int_M \dif u_q \wedge F_p = \frac{1}{L} \lim_{p \to \infty} \int_M [\dif u_q] \wedge \rho.$$
Since $[\dif u_q] \to [\dif u]$, we obtain
$$\lim_{p \to \infty} \int_M [\dif u_q] \wedge \rho = \int_M \alpha \wedge \rho = \int_M \dif u \wedge F,$$
completing the proof of (\ref{1 strong duality}).

Localizing (\ref{Anzellotti Holder inequality}) to an open set $E \subseteq M$, we bound
$$\int_E \dif u \wedge F \leq L \mu(E).$$
Since $\mu$ is a Radon measure and $M$ is compact, every Borel set can be $\mu$-approximated from without by open sets, so the same inequality holds if $E$ is merely a Borel set.
So, by Lemma \ref{measurable function is 1} with $\star f := \dif u \wedge F/(L|\dif u|)$, (\ref{1 extremality}) holds.
In particular, by Lemma \ref{1 extremality implies least gradient}, $u$ has least gradient.
\end{proof}

%%%%%%%%%%%%%%%%
\subsection{Counterexample for the Dirichlet problem}\label{boundaries bad}
We now observe that the analogue of Theorem \ref{existence of infinity tight forms} -- specifically, the existence of a dual least gradient function for the tight form -- fails for the Dirichlet problem.
This is particularly remarkable because the proof of Theorem \ref{Mazon MFMC} shows that every least gradient function (for the Dirichlet problem) has a dual tight form.

\begin{example}
Let $M = (0, 1)^2$, and consider the boundary data $f$ which is linear on each edge of $\partial M$ and satisfies $f(0, 0) = 0$, $f(0, 1) = 1$, $f(1, 0) = 1/3$, $f(1, 1) = 2/3$.
Let $v$ be the $\infty$-harmonic extension of $f$, and let $F := \dif v$.
Then $F$ is tight, and $\Comass(F) = 1$ since $\Lip(f) = 1$.

Let $\lambda$ be the set of points where (the upper semicontinuous envelope of) $|F|$ attains its maximum.
Clearly $\lambda$ contains the left edge of $\partial M$, and intersects no other edge of $M$.
But one can use comparison with cones to show that $\lambda$ is a geodesic lamination \cite[\S5]{daskalopoulos2020transverse}, so if $x \in \lambda$, then there is a line in $\lambda$ through $x$, which can intersect $\partial M$ only at its left edge.
So $\lambda$ is the left edge.

If $u$ is a $1$-harmonic conjugate of $F$, in the sense that $u$ has least gradient and $\dif u \wedge F = \star |\dif u|$, then $\supp \dif u \subseteq \lambda \subset \partial M$.
But then $u$ is constant.
\end{example}

Straightforward modifications show that the same phenomenon occurs quite generally, even if $\partial M$ is strictly convex.
The proof of Theorem \ref{existence of infinity tight forms} breaks down at the estimate (\ref{q Laplacian Sobolev regularity estimate}): one cannot compare the energy of a $q$-harmonic function to the total variation of a least gradient function of the same trace, since in general, there could be functions with the same trace and arbitrarily small total variations.




%%%%%%%%%%%%%%%
\section{Calibrations of laminations}\label{comass sec}
\subsection{Measurable calibrations}\label{L infinity calibrations}
Let $M$ be an closed oriented Riemannian manifold of dimension $2 \leq d \leq 7$, and let $\rho \in H^{d - 1}(M, \RR)$ have costable norm $1$.
If $\rho$ contains a continuous calibration $F$, then the Bangert--Cui theorem \cite{bangert_cui_2017} establishes the existence of a measured lamination whose leaves are $F$-calibrated.
In general, however, one needs to develop a theory of laminations calibrated by a measurable calibration.

We work somewhat generally: we shall not need to assume that $M$ is closed or $d \leq 7$.
Let $F \in L^\infty(M, \Omega^{d - 1}_{\rm cl})$ be a closed $L^\infty$ form.
By Proposition \ref{integration is welldefined}, for every hypersurface $N \subset M$, $\int_N F$ is well-defined.
If $\Comass(F) = 1$, so $F$ is a calibration, then $N$ is said to be \dfn{$F$-calibrated} if 
$$\int_N F = \vol(N).$$
In that case, for any domain $\Omega \Subset M$, $\int_{\partial \Omega} F = 0$, so
$$\vol(N) = \int_N F = \int_{N + \partial \Omega} F \leq \vol(N + \partial \Omega).$$
Thus $N$ is area-minimizing among all hypersurfaces homologous to $N$.

Dually, if $M$ is closed and $N$ is $F$-calibrated, then for any $d - 2$-form $A$, 
$$\Comass(F) = 1 = \frac{1}{\vol(N)} \int_N F = \frac{1}{\vol(N)} \int_N F + \dif A \leq \Comass(F + \dif A).$$
Therefore the costable norm of $[F]$ is $1$.

\begin{definition}
Let $\lambda$ be a Lipschitz lamination of codimension $1$, and $F$ a measurable calibration $d - 1$-form.
Then $\lambda$ is \dfn{$F$-calibrated} if every leaf of $\lambda$ is $F$-calibrated.
\end{definition}

Let $\lambda$ be a measured oriented Lipschitz lamination.
Then $\lambda$ can be identified with its Ruelle-Sullivan current $T_\lambda$, which has locally finite mass.
By $\Mass(\lambda)$ we mean $\Mass(T_\lambda)$.
Since $T_\lambda$ is closed, it has a homology class that we denote by $[\lambda]$, and we can write $T_\lambda = \dif u$ for some $u \in BV(M, \RR)$.
Then, by Proposition \ref{Anzellotti wedge product exists}, the Anzellotti wedge product $T_\lambda \wedge F = \dif u \wedge F$ exists as a signed Radon measure.

Let $(\chi_\alpha)$ be a locally finite partition of unity subordinate to a laminar atlas $(U_\alpha)$ for the measured oriented lamination $\lambda$.
Let $K_\alpha$ denote the local leaf space of $U_\alpha$, and $\mu_\alpha$ the transverse measure on $K_\alpha$.
If $\sigma_{\alpha, k}$ denotes the leaf in $U_\alpha$ corresponding to the real number $k \in K_\alpha$, then the definition of the Ruelle-Sullivan current unpacks as
\begin{equation}\label{coarea formula on laminations}
\int_M T_\lambda \wedge F = \sum_\alpha \int_{K_\alpha} \int_{\sigma_{\alpha, k}} \chi_\alpha F \dif \mu_\alpha(k).
\end{equation}
Since $T_\lambda$ and $F$ are closed, if $M$ is closed, then the left-hand side of (\ref{coarea formula on laminations}) is a homological invariant:
\begin{equation}\label{Ruelle Sullivan homology}
\int_M T_\lambda \wedge F = \langle [F], [\lambda]\rangle.
\end{equation}

\begin{proposition}\label{calibration condition}
Let $F$ be a calibration.
Let $T_\lambda$ be the Ruelle-Sullivan current of a measured oriented lamination $\lambda$.
Then the following are equivalent:
\begin{enumerate}
\item One has \begin{equation}\label{calibration by Ruelle Sullivan}
\int_M T_\lambda \wedge F = \Mass(\lambda).
\end{equation}
\item $\lambda$ is $F$-calibrated.
\end{enumerate}
\end{proposition}
\begin{proof}
First suppose that (\ref{calibration by Ruelle Sullivan}) holds.
Let $(\chi_\alpha)$ be a locally finite partition of unity subordinate to an open cover $(U_\alpha)$ of flow boxes for $\lambda$, let $(K_\alpha)$ be the local leaf spaces, and let $(\mu_\alpha)$ be the transverse measure.
After refining $(U_\alpha)$ we may assume that $U_\alpha$ is contained in a deformed euclidean ball as in Appendix \ref{GMT appendix}. After shrinking $U_\alpha$ we may assume that $\chi_\alpha > 0$ on $U_\alpha$.
Then for leaves $\sigma_{\alpha,k}$, we rewrite (\ref{coarea formula on laminations}) as 
$$\Mass(\lambda) = \int_M T_\lambda \wedge F = \sum_\alpha \int_{K_\alpha} \int_{\sigma_{\alpha,k}} \chi_\alpha F \dif \mu_\alpha(k).$$
Let $\dif S_{\alpha,k}$ be the surface measure on $\sigma_{\alpha,k}$.
Then
$$\int_M \chi_\alpha \star |T_\lambda| = \int_{K_\alpha} \int_{\sigma_{\alpha,k}} \chi_\alpha \dif S_{\alpha,k} \dif \mu_\alpha(k),$$
so summing in $\alpha$, we obtain 
\begin{equation}\label{calibration condition contr}
\sum_\alpha \int_{K_\alpha} \int_{\sigma_{\alpha,k}} \chi_\alpha F \dif \mu_\alpha(k) = \Mass(\lambda) = \sum_\alpha \int_{K_\alpha} \int_{\sigma_{\alpha,k}} \chi_\alpha \dif S_{\alpha,k} \dif \mu_\alpha(k).
\end{equation}

We claim that $\lambda$ is \dfn{almost calibrated} in the sense that for every $\alpha$ and $\mu_\alpha$-almost every $k$, $\sigma_{\alpha, k}$ is calibrated.
If this is not true, then we may select $\beta$ and $K \subseteq K_\beta$ with $\mu_\beta(K) > 0$, such that for every $k \in K$, $\int_{\sigma_{\beta, k}} F < \vol(\sigma_{\beta, k})$.
Since $0 < \chi_\beta \leq 1$ and $F/\dif S_{\beta, k} \leq 1$ on $\sigma_{\beta, k}$, this is only possible if 
$$\int_{\sigma_{\beta, k}} \chi_\beta F < \int_{\sigma_{\beta, k}} \chi_\beta \dif S_{\beta, k}.$$
Integrating over $K$, and using the fact that in general we have $\int_{\sigma_{\alpha, k}} \chi_\alpha F \leq \int_{\sigma_{\alpha, k}} \chi_\alpha \dif S_{\alpha, k}$, we conclude that 
$$\sum_\alpha \int_{K_\alpha} \int_{\sigma_{\alpha, k}} \chi_\alpha F \dif \mu_\alpha(k) < \sum_\alpha \int_{K_\alpha} \int_{\sigma_{\alpha, k}} \chi_\alpha \dif S_{\alpha, k} \dif \mu_\alpha(k)$$
which contradicts (\ref{calibration condition contr}).

To upgrade $\lambda$ from an almost calibrated lamination to a calibrated lamination, we first, given $\sigma_{\alpha, k}$, choose $k_j$ such that $\sigma_{\alpha, k_j}$ is calibrated and $k_j \to k$.
By Proposition \ref{Hodge theorem}, we can find a continuous $d - 2$-form $A$ defined near $\sigma_{\alpha, k}$ with $F = \dif A$.
This justifies the following application of Stokes' theorem: 
$$\int_{\sigma_{\alpha, k}} F = \int_{\partial \sigma_{\alpha, k}} A.$$
Since $k_j \to k$, and $A$ is continuous,
\begin{align*}
\Mass(\sigma_{\alpha, k}) &= \lim_{j \to \infty} \Mass(\sigma_{\alpha, k_j}) = \lim_{j \to \infty} \int_{\sigma_{\alpha, k_j}} F = \lim_{j \to \infty} \int_{\partial \sigma_{\alpha, k_j}} A = \int_{\partial \sigma_{\alpha, k}} A = \int_{\sigma_{\alpha, k}} F.
\end{align*}

To establish the converse, suppose that $\lambda$ is $F$-calibrated, and let notation be as above.
Since $\lambda$ is $F$-calibrated, for every $\alpha$ and every $k$, the area form on $\sigma_{\alpha, k}$ is $F$. Therefore
\begin{align*}
\int_M T_\lambda \wedge F &= \sum_\alpha \int_{K_\alpha} \int_{\sigma_{\alpha, k}} \chi_\alpha F \dif \mu_\alpha(k) = \Mass(T_\lambda). \qedhere
\end{align*}
\end{proof}


% \subsection{old}

% A smooth hypersurface $N$ is $F$-calibrated iff the pullback $\iota_N^* F$ is the area form on $N$.
% The pullback map $\iota_N^*$ is well-defined by Proposition \ref{integration is welldefined}.
% In particular, $\iota_N^* F$ is smooth and $N$ is oriented.
% If $\lambda$ is an $F$-calibrated lamination, then the trace $\iota_\lambda^* F$ is defined to be $\iota_N^* F$ along any leaf $N$.
% A priori, $\lambda$ could fail to be orientable, and then $\iota_\lambda^* F$ would be necessarily discontinuous.
% However, we assert that a calibrated lamination is orientable:

% \begin{proposition}\label{calibrated implies oriented}
% Let $F$ be a calibration and $\lambda$ an $F$-calibrated lamination.
% Then $\iota_\lambda^* F$ is continuous, and $F$ induces an orientation on $\lambda$.
% \end{proposition}
% \begin{proof}
% Let $N$ be a leaf of $\lambda$ and $x \in N$.
% Let $\mathscr O$ be the local orientation of $\lambda$ near $x$ which is compatible with $F(x)$. 

% There exists a continuous $d - 1$-form $G$ such that $y \in K$ close enough to $x$, $K$ a leaf of $\lambda$, $G(y) = \dif S_K(y)$ is the area form of $K$ with respect to $\mathscr O$.
% In fact, we can fill in the spaces between the leaves by linear interpolation.
% Then for any $y \in \supp \lambda$ close to $x$, either $F(y) = G(y)$ or $F(y) = -G(y)$; we claim that $F(y) = G(y)$ if $\dist(x, y)$ is small enough.
% If this claim is true, then the proposition follows, since $G$ is continuous at $x$.

% To prove the claim, we suppose towards contradiction that there is a sequence $(x_n) \subset \supp \lambda$ with $x_n \to x$ and $F(x_n) = -G(x_n)$.
% We may write $F = \dif A$ where $A$ is continuous near $x$ by Proposition \ref{Hodge theorem}.
% Let $N_n$ be the leaf of $\lambda$ containing $x_n$; then $N_n$ is minimal, hence smooth, so $F_n|_{N_n}$ is smooth.
% Therefore by Stokes' theorem, if $(D_{n, m})$ is a sequence of disks in $N_n$ which shrink down to $x_n$ as $m \to \infty$ and are equipped with the orientation $\mathscr O$,
% $$-1 = \lim_{m \to \infty} \frac{1}{\Mass(D_{n, m})} \int_{D_{n, m}} F = \lim_{m \to \infty} \frac{1}{\Mass(D_{n, m})} \int_{\partial D_{n, m}} A.$$
% In particular, we can choose $m_n \to \infty$ such that 
% \begin{equation}\label{orientation contradiction}
% \int_{\partial D_{n, m_n}} A \leq 0.
% \end{equation}
% We set $D_n := D_{n, m_n}$.

% Let $(k, z) \in \RR \times \RR^{d - 1}$ be coordinates near $x$ with respect to a flow box, where $k$ indexes the leaves and $z$ is a parameter on each leaf.
% Then $D_n = \{k_n\} \times \Omega_n$ for some $\Omega_n \subset \RR^{d - 1}$.
% Let $k$ be the index of $N$ and $D_n' := \{k\} \times \Omega_n$.
% Since $A$ is continuous, $A(k, z) = A(k_n, z) + o(1)$ as $n \to \infty$, hence
% \begin{equation}\label{orientation contradiction 2}
% \frac{1}{\Mass(D_n')} \int_{\partial D_n'} A = \frac{1}{\Mass(D_n)} \int_{\partial D_n} A + o(1).
% \end{equation}
% However, $D_n'$ shrinks down to $x$, and $F(x) = G(x)$, so by Stokes' theorem, for $n$ large,
% $$\int_{\partial D_n'} A \gtrsim \Mass(D_n'),$$
% hence by (\ref{orientation contradiction}) and (\ref{orientation contradiction 2}),
% $$0 < \Mass(D_n) \lesssim \int_{\partial D_n} A \leq 0,$$
% a contradiction.
% \end{proof}

% We next give a natural condition for a lamination to be calibrated.
% To state it, we record the following immediate consequence of the coarea formula (\ref{coarea formula}).

% \begin{lemma}
% Let $(\lambda, \mu)$ be a measured oriented lamination on a closed Riemannian manifold $M$.
% Suppose that $F \in L^\infty(M, \Omega^{d - 1})$ has $\dif F \in L^p(M, \Omega^d)$.
% Let $(\chi_\alpha)$ be a locally finite partition of unity subordinate to an open cover $(U_\alpha)$ of flow boxes for $\lambda$.
% Let $\sigma_{\alpha, k}$ be the leaf of $\lambda \cap U_\alpha$ with parameter $k \in I$.
% Then
% \begin{equation}\label{coarea formula on laminations}
% \int_M T_\lambda \wedge F = \sum_\alpha \int_I \int_{\sigma_{\alpha, k}} \chi_\alpha F \dif \mu_\alpha(k).
% \end{equation}
% \end{lemma}

% On the other hand, if $F$ is a closed $d - 1$-form and $M$ is closed, then $\int_M T_\lambda \wedge F = \langle [\lambda], [F]\rangle$ is simply the pairing of homology and cohomology.

Suppose that $M$ is closed.
If $T$ is a current on $M$, we say that $T$ is \dfn{homologically minimizing} if $T$ minimizes its mass in its homology class.
Thus every $F$-calibrated current is homologically minimizing.
A measured oriented lamination is \dfn{homologically minimizing} if its Ruelle-Sullivan current is.

\begin{proposition}\label{properties of calibrated laminations}
Suppose that $M$ is closed.
Let $F$ be a calibration, and let $\lambda$ be a measured oriented $F$-calibrated lamination.
Then:
\begin{enumerate}
\item $\lambda$ is minimal and homologically minimizing.
\item If $G$ is a calibration and cohomologous to $F$, then $\lambda$ is $G$-calibrated.
\end{enumerate}
\end{proposition}
\begin{proof}
Every leaf of $\lambda$ is $F$-calibrated, hence minimal, so $\lambda$ is also minimal.
Then, since (\ref{calibration by Ruelle Sullivan}) only depends on the cohomology class of $F$, not $F$ itself, $\lambda$ is $G$-calibrated.
From (\ref{calibration by Ruelle Sullivan}), $T_\lambda$ is $F$-calibrated, so $\lambda$ is homologically minimizing.
\end{proof}

%%%%%%%%%%%%%%%%%%%
\subsection{Measured stretch laminations}\label{proof of Theorem B}
We are now ready to prove Theorem \ref{lams are calibrated}.
Let $M$ be a closed oriented Riemannian manifold of dimension $2 \leq d \leq 7$, and let $\rho \in H^{d - 1}(M, \RR)$ be a nonzero class.
To ease notation, we normalize
$$\Comass(\rho) = 1.$$

If $F$ is a tight representative of $\rho$, and $u$ is a $1$-harmonic conjugate of $F$, then by Theorem \ref{existence of infinity tight forms} and Theorem \ref{1 harmonic is MOML}, $u$ defines a measured oriented minimal lamination $\kappa_u$ which we call \dfn{measured stretch} for $\rho$.
By (\ref{1 extremality}),
$$\Mass(\kappa_u) = \Mass(\dif u) = \int_M \dif u \wedge F,$$
so by Proposition \ref{calibration condition} and Proposition \ref{properties of calibrated laminations}, if $G$ is any best comass representative of $\rho$, then $\kappa_u$ is $G$-calibrated.

The two directions of Theorem \ref{lams are calibrated} follow from each of the following propositions:

\begin{proposition}\label{L equals K}
Let $\kappa$ be a measured stretch lamination for $\rho$, and let $\lambda$ range over measured oriented laminations. Then 
	\begin{equation}\label{L equals K formula}
	\sup_\lambda \frac{\langle \rho, [\lambda]\rangle}{\Mass(\lambda)} = \frac{\langle \rho, [\kappa]\rangle}{\Mass(\kappa)} = 1.
	\end{equation}
\end{proposition}
\begin{proof}
Fix a tight form $F$ representing $\rho$.
Let
$$K :=  \sup_\lambda \frac{\langle \rho, [\lambda]\rangle}{\Mass(\lambda)}.$$
Since $\kappa$ is $F$-calibrated, we obtain from (\ref{Ruelle Sullivan homology}) that
$$\langle \rho, [\kappa]\rangle = \int_M T_\kappa \wedge F = \Mass(\kappa).$$
So it is enough to show that $K \leq 1$.

Let $\lambda$ be a measured oriented lamination.
We can write $T_\lambda = \dif v$ for some $v \in BV(M, \RR)$; then, by (\ref{Ruelle Sullivan homology}) and (\ref{Anzellotti Holder inequality}),
$$\langle \rho, [\lambda]\rangle = \int_M T_\lambda \wedge F = \int_M \dif v \wedge F \leq \Mass(\dif v) = \Mass(\lambda).$$
Taking the supremum in $\lambda$, we deduce $K \leq 1$.
\end{proof}

\begin{proposition}\label{calibrated means measured stretch}
Let $\lambda$ be a measured oriented lamination which attains the supremum in (\ref{L equals K formula}).
Then $\lambda$ is a measured stretch lamination for $\rho$.
\end{proposition}
\begin{proof}
We write $T_\lambda = \dif u$ for some $u \in BV(M, \RR)$, so that, by (\ref{Ruelle Sullivan homology}) and the fact that $\lambda$ attains the supremum,
$$\int_M \star |\dif u| = \Mass(\lambda) = \langle \rho, [\lambda]\rangle = \int_M T_\lambda \wedge F = \int_M \dif u \wedge F.$$
Arguing as in the proof of Proposition \ref{existence 1} (using Lemma \ref{measurable function is 1}), we deduce (\ref{1 extremality}).
\end{proof}

\begin{corollary}
Every measured oriented calibrated lamination on $M$ is Lipschitz.
\end{corollary}
\begin{proof}
If $\lambda$ is calibrated by a calibration $F$, and $\rho := [F]$, then $\Comass(\rho) = 1$.
In particular, $\lambda$ attains the supremum in (\ref{L equals K formula}), hence is measured stretch for $\rho$.
The proof of Theorem \ref{1 harmonic is MOML} shows that every measured stretch lamination is Lipschitz.
\end{proof}

%%%%%%%%%%%%%%%
\section{The Euler-Lagrange equation for tight forms}\label{infinityMax}
\subsection{Formal derivation}
Let us imitate the formal derivation of the $\infty$-Laplacian \cite{Aronsson67}.
Suppose that we have $p$-tight forms $F_p$ which are converging in $C^1$ to a tight form $F$.
Then, $\dif F = 0$ and
\begin{align*}
0
&= \dif(|F_p|^{p - 2} \star F_p) \\
&= \dif(|F_p|^{p - 2}) \wedge \star F_p + |F_p|^{p - 2} \dif \star F_p \\
&= (p - 2) |F_p|^{p - 4} \langle \nabla F_p, F_p\rangle \wedge \star F_p + |F_p|^{p - 2} \dif \star F_p.
\end{align*}
If $F_p$ is nonzero, then we can divide through by $(p - 2) |F_p|^{p - 4}$ to get
\begin{equation}\label{intermediate p Max}
0 = \langle\nabla F_p, F_p\rangle \wedge \star F_p + \frac{|F_p|^2}{p - 2} \dif \star F_p.
\end{equation}
At the zeroes of $F_p$, we simply observe that (\ref{intermediate p Max}) holds for trivial reasons.
Note carefully that (\ref{intermediate p Max}) is an equation of $2$-forms.
Since $F_p \to F$ in $C^1$, $|F_p|^2 \dif \star F_p$ is bounded, so when we take the limit $p \to \infty$, we obtain 
\begin{equation}\label{infty Max}
\begin{cases}
\dif F = 0, \\
\langle \nabla F, F\rangle \wedge \star F = 0.
\end{cases}
\end{equation}

Sometimes it is more convenient to work with the Hodge dual of $\langle \nabla F, F\rangle \wedge \star F$.
We can rewrite (\ref{infty Max}), for any vector fields $X_1, \dots, X_{d - 2}$,
\begin{equation}\label{dual infty Max}
\begin{cases}
\dif F = 0, \\
\langle F, \langle \nabla F, F\rangle^\sharp \otimes X_1 \otimes \cdots \otimes X_{d - 2}\rangle = 0.
\end{cases}
\end{equation}
Rewriting this equation in Einstein notation, we deduce (\ref{tight Einstein}).
Furthermore, when $d = 2$, we recover the $\infty$-Laplacian by writing $F = \dif v$ and computing 
$$0 = \langle F, \langle \nabla F, F\rangle\rangle = \langle \nabla^2 v, \dif v \otimes \dif v\rangle.$$
Since the $\infty$-Laplacian has a good theory of viscosity solutions, if $d = 2$, then the above derivation is valid even when $F_p$ is only converging in $C^0$ to $F$.
In particular, a tight $1$-form is nothing more than the derivative of an $\infty$-harmonic function, a fact that we have already tacitly used.

%%%%%%%%%%%%%%%%%%%%%%%%%
\subsection{Geometric and variational interpretation}\label{EL interpretation}
Throughout this section, let $M$ be a Riemannian manifold, possibly with boundary.
The PDE (\ref{infty Max}) has a simple geometric interpretation, which generalizes the interpretation of the $\infty$-Laplace equation as asserting that the gradient curves of an $\infty$-harmonic function are lines.

\begin{proposition}\label{infty Max calibrates}
Let $F$ be a $C^1$ solution of (\ref{infty Max}) with no zeroes, and let $N$ be a connected integral hypersurface of $\ker(\star F)$.
Then there exists a constant $\psi > 0$ such that $N$ is an $F/\psi$-calibrated hypersurface.
\end{proposition}
\begin{proof}
We work in coordinates $(x^\alpha)$ such that $N = \{x^d = 0\}$ and $\partial_d$ is the unit normal vector field of $N$.
The assumption on $N$ means that (possibly after reversing orientation), for some scalar field $\psi > 0$ defined on $N$, $\star F|_N = \psi \dif x^d$.
Since $F$ solves (\ref{infty Max}), $F$ is closed and
\begin{align*}
0 
&= \partial_\alpha(|F|^2)(\star F)_\beta - \partial_\beta(|F|^2)(\star F)_\alpha \\
&= 2\psi(\delta_{\beta d} \partial_\alpha \psi - \delta_{\alpha d} \partial_\beta \psi).
\end{align*}
Taking $\alpha \in \{1, \dots, d - 1\}$ and $\beta = d$ we deduce that $\partial_\alpha \psi = 0$.
Since $\partial_\alpha$ is an arbitrary tangent vector to $N$, we conclude that $\psi$ is constant and $F/\psi$ is the Riemannian volume form on $N$.
Since $F$ is closed, it follows that $F/\psi$ is a calibration for $N$.
\end{proof}

As intimated by Proposition \ref{infty Max calibrates}, the existence (or nonexistence) of integral hypersurfaces of $\ker(\star F)$ will have a crucial role.
Therefore, we shall say that a solution $F$ of (\ref{infty Max}) is \dfn{involutive} if 
$$\star F \wedge \dif(\star F) = 0,$$
so that the Pfaffian system induced by $\star F$ is completely integrable on the open set $\{|F| > 0\}$.

We now give a variational interpretation for (\ref{infty Max}).
We first recall some terminology from the $L^\infty$ calculus of variations, reworded to be diffomorphism-invariant; see \cite[\S5]{Barron08} for a motivation of the definition of absolute minimality.

\begin{definition}
Let $(E, \nabla)$ be a vector bundle with connection, and let
$$f: E \oplus (\Omega^1 \otimes E) \to \RR_+$$
be a continuous function on the jet bundle of $E$.
We say that a section $u: M \to E$ \dfn{absolutely minimizes} $f$, if for every open set $U \Subset M$, and every $v \in W^{1, \infty}(U, E)$ with zero trace,
$$\esssup_{x \in U} f(x, u(x), \nabla u(x)) \leq \esssup_{x \in U} f(x, u(x) + v(x), \nabla u(x) + \nabla v(x)).$$
We say that $u$ \dfn{locally absolutely minimizes} $f$ if we can cover $M$ by open sets $U$ such that $u|_U$ absolutely minimizes $f|_U$.
\end{definition}

\begin{proposition}
Let $F$ be a $C^1$ closed $d - 1$-form.
If $F$ locally absolutely minimizes $|F|$, then $F$ solves (\ref{infty Max}).
\end{proposition}
\begin{proof}
Since the statement is local, we may (by a suitable modification of Proposition \ref{Hodge theorem}) assume that $F = \dif A$ for some $d - 2$-form $A$, and that $F$ is a (global) absolute minimizer of $|F|$.
For $\xi$ a covariant tensor of valence $d - 1$ at $x$, let $\xi^{\rm as}$ be its antisymmetrization.
Define a function on the jet bundle of $\Omega^{d - 2}$ by
$$f(x, \omega, \xi) := |\xi^{\rm as}|^2.$$
Thus, for any $d - 2$-form $B$, $f(x, B(x), \nabla B(x)) = |\dif B(x)|^2$, and $A$ is a absolute minimizer of $f$.

By the above claims and \cite[Theorem 5.2]{Barron2001}, for each $x \in W$, we have the Euler-Lagrange-Aronsson equation that for any vector fields $X_1, \dots, X_{d - 2}$,
\begin{align*}
0 
&= \left\langle \frac{\partial f}{\partial \xi}(x, A(x), \nabla A(x)), \nabla \left[f(x, A(x), \nabla A(x))\right]^\sharp \otimes X_1(x) \otimes \cdots \otimes X_{d - 2}(x)\right\rangle \\
&= 2\langle (\nabla A(x))^{\rm as}, \nabla(|(\nabla A(x))^{\rm as}|^2)^\sharp \otimes X_1(x) \otimes \cdots \otimes X_{d - 2}(x)\rangle \\
&= 2\langle F(x), \nabla(|F(x)|^2)^\sharp \otimes X_1(x) \otimes \cdots \otimes X_{d - 2}(x)\rangle \\
&= 4\langle F(x), \langle \nabla F(x), F(x)\rangle^\sharp \otimes X_1(x) \otimes \cdots \otimes X_{d - 2}(x)\rangle.
\end{align*}
But this is the dualized equation (\ref{dual infty Max}).
\end{proof}

\begin{proposition}\label{tight and integrable implies infinity maxwell}
Let $F$ be a $C^1$ involutive solution of (\ref{infty Max}).
Then $F$ locally absolutely minimizes $|F|$.
\end{proposition}
\begin{proof}
First observe that it suffices to show that, for every sufficiently small open set $U$,
\begin{equation}\label{ABC inequality}
\|F\|_{C^0(U)} = \|F\|_{C^0(\partial U)}
\end{equation}
(where the right-hand side refers to the full trace, not just the pullback, of $F$).
If this inequality holds, and $G$ is a competitor in $U$, then
$$\|G\|_{C^0(U)} \geq \|G\|_{C^0(\partial U)} = \|F\|_{C^0(\partial U)} = \|F\|_{C^0(\partial U)},$$
establishing that $F$ is a local absolute minimizer.

If $x \in M$ and $F(x) = 0$, then $|F|$ has a local minimum at $x$ and (\ref{ABC inequality}) is vacuous for $U$ a small ball near $x$.
Thus we may assume that $F|_U$ has no zeroes.
Furthermore, we may assume that $U$ is contractible.

Since $F$ is involutive and has no zeroes, $\ker(\star F)$ integrates to a foliation $\mathscr F$ of $U$.
By Proposition \ref{infty Max calibrates}, for each leaf $N$ of $\mathscr F$ there exists $\psi_N > 0$ such that $F/\psi_N$ calibrates $N$.
Since $U$ is contractible, it follows that $N$ is absolutely area-minimizing in $U$, and hence must intersect $\partial U$ at some point $x_N$.
In particular, for $x \in N$,
$$|F|(x) = \psi_N = |F|(x_N) \leq \|F\|_{C^0(\partial U)} \leq \|F\|_{C^0(U)}.$$
Taking a supremum over all leaves $N$ and all $x \in N$, we conclude (\ref{ABC inequality}).
\end{proof}

%%%%%%%%%%%%%%%%%
%\subsection{Counterexample for noninvolutive solutions}\label{nonintegrability}

% We begin by showing that (\ref{ABC inequality}) can fail on large open sets.
% The map we study here has a crucial role in the proof of a theorem of Daskalopolous and Uhlenbeck, that for generic hyperbolic structures $\rho, \sigma$ on a closed surface $M$, the $\infty$-harmonic map $u: (M, \rho) \to (M, \sigma)$ homotopic to $\id_M$ attains its Lipschitz constant exactly on the canonical maximally stretched lamination \cite{daskalopoulos2022}.

% \begin{example}
% Consider the cylinder $M := \Sph^1_\theta \times \RR_x$ with the hyperbolic metric
% $$g := \dif x^2 + \frac{\cosh^2 x}{4} \dif \theta^2.$$
% Then $M$ has a single closed geodesic $\gamma$, which winds around $\{x = 0\}$, and has circumference $\pi$.
% Let $\theta: M \to \Sph^1$ be the projection map, so
% $$|\dif \theta|(x, \theta) = 2 \sech x.$$
% This attains its maximum on $\gamma$, and if $U$ is any neighborhood of $\gamma$, then $|\dif \theta|$ is smaller on $\partial U$ than on $\gamma$.
% \todo{Does $\theta$ actually fail to be AML?}

% Using the Christoffel symbols
% \begin{align*}
% {\Gamma^\theta}_{\theta \theta} &= {\Gamma^\theta}_{x x} = 0, \\
% {\Gamma^\theta}_{x \theta} &= {\Gamma^\theta}_{\theta x} = \tanh x,
% \end{align*}
% we see that
% $$\nabla \dif \theta = \tanh x \dif x \dif \theta,$$
% and hence $\Delta_\infty \theta = 0$.
% This implies that $\dif \theta$ solves (\ref{infty Max}) and is tight.
% \end{example}

% However, this defect in the theory seems to be generic to $L^\infty$ variational systems.
% One can use the failure of the Kirszbraun-Valentine theorem \cite[Example 9.6]{Gu_ritaud_2017}, for example, to show that any homothetic contraction $u: \Hyp^2 \to \Hyp^2$ is $\infty$-harmonic but is not absolutely minimizing Lipschitz. \todo{Does this actually fail to be AML?}

% The above example relied on a judicious choice of $F = \dif \theta$ which calibrated a closed geodesic.
% In particular, $\mathscr D := \ker(\star \dif \theta)$ was integrable.
% By taking $\mathscr D$ to be orthogonal to a Beltrami field with constant eigenvalue, we show that the integrability hypothesis is needed even for global minimality:

It is a rather surprising fact that (\ref{infty Max}) has (noninvolutive) solutions which have nothing to do with tightness or minimality of the comass.
Since one does not expect to be able to interpret solutions of a general $L^\infty$ variational problem as calibrations of a foliation, it is likely that this defect is generic to $L^\infty$ variational problems.

\begin{example}\label{integrability needed}
Let $V$ be the unit vertical vector to the Hopf fibration $\Sph^3 \to \Sph^2$.
Let $\mathscr D$ be the distribution of horizontal vectors.
The vertical and horizontal directions are orthogonal, so if we set $F := \star V^\flat$, and let $X, Y$ be an orthonormal frame for $\mathscr D$, then $F(X, Y) = 1$ and $F$ annihilates vertical vectors.
In particular $|F| = 1$ identically, so $2\langle \nabla F, F\rangle = \dif(|F|^2) = 0$.

But $\nabla \times V = 2V$ \cite[\S3]{Peralta_Salas_2023}.
Since $(\nabla \cdot) \circ (\nabla \times) = 0$, it follows that $\nabla \cdot V = 0$, so $\dif F = 0$.
Therefore $F$ solves (\ref{infty Max}) and, since $H^2(\Sph^3, \RR) = 0$, $F$ is exact.

Thus $F$ is a solution which is cohomologous to $0$ and nonzero.
It follows that $F$ cannot have best comass and cannot be tight.
\end{example}

%%%%%%%%%%%%%%%%%
\subsection{Vector-valued viscosity solutions}
It is possible that a suitable definition of vector-valued viscosity solution would justify the derivation of (\ref{infty Max}) when $d \geq 3$.
A candidate notion is the \dfn{contact solution} notion of Katzourakis \cite{Katzourakis2018OnAV}, but we were not able to show that tight maps are contact solutions of (\ref{infty Max}).
We regard the problem of finding a suitable notion of viscosity solution as the most important problem posed in this work, as it is likely to hint at the correct definition of $\infty$-harmonic map.


%%%%%%%%%%%%%%%%
\appendix
\section{Geometric measure theory}\label{GMT appendix}
\subsection{Notation}
Let $M$ be a Riemannian manifold.
The operator $\star$ is the Hodge star on $M$, thus $\star 1$ is the Riemannian measure of $M$.
We denote the musical isomorphisms by $\sharp, \flat$.
To avoid confusion, we write $H^\ell$ for de Rham cohomology, but never a Sobolev space, which we instead denote $W^{\ell, p}$.
The manifold $\Ball^d$ is the unit ball in $\RR^d$, $\Sph^d$ is the unit sphere in $\RR^{d + 1}$, and $\Hyp^d$ is the hyperbolic space.
The $s$-dimensional Hausdorff measure is $\mathcal H^s$, normalized so that if $s$ is an integer, then $\mathcal H^s$ is $s$-dimensional Riemannian measure.

Let $\mathscr F$ be a subpresheaf of the sheaf of distributional sections of some Riemannian vector bundle $E \to M$ and let $\mathcal X$ be a function space.
We write $\mathcal X(\cdot, \mathscr F)$ for the sheaf of sections $u$ of $\mathscr F$ such that for every smooth unit-length local section $e$ of $E$, $\langle e, u\rangle \in \mathcal X$.
We write $\mathcal X_\cpt$ for the space of compactly supported functions in $\mathcal X$, and $\mathcal X_\loc$ for the space of functions $u$ such that for every $\chi \in C^\infty_\cpt$, $\chi u \in \mathcal X$.

The sheaf of $\ell$-forms is denoted $\Omega^\ell$, and the sheaf of closed $\ell$-forms is denoted $\Omega^\ell_{\rm cl}$.
We assume that $\ell$-forms are $L^1_\loc$, but \emph{not} that they are continuous; hence $\dif$ must be meant in the sense of distributions.

We write $A \lesssim_\theta B$ to mean that $A \leq CB$, where $C > 0$ is a constant that only depends on $\theta$.
We write $A \ll_\theta B$ to mean that, as $B \to 0$, $A \to 0$, where the rate of convergence only depends on $\theta$.

%%%%%%%%%%%%%%%
\subsection{Homological integration}
By an $\ell$-\dfn{current of locally finite mass} $T$ we mean a continuous linear functional on the space $C^0_\cpt(M, \Omega^\ell)$ of continuous $\ell$-forms of compact support.
By a \dfn{current} we shall always mean a current of locally finite mass, unless explicitly stated otherwise.
We write $\int_M T \wedge \varphi$ or $\int_T \varphi$ for the dual pairing of a current $T$ and a form $\varphi$.
If $T$ is a current (of locally finite mass), then the components of $T$ are Radon measures, and conversely a Radon measure is a $0$-current.

A closed $\ell$-current $T$ defines a homology class $[T] \in H_\ell(M, \RR)$.
However, every closed $d - \ell$-form $\varphi$ with compact support defines a compactly supported \emph{cohomology} class $[\varphi] \in H^{d - \ell}_\cpt(M, \RR)$, as well as an $\ell$-current $\varphi \mapsto \int_M \psi \wedge \varphi$, which then defines a homology class in $H_\ell(M, \RR)$.
This can be confusing! 
But, recall that the Poincar\'e duality map 
$$\PD: H^{d - \ell}_\cpt(M, \RR) \to H_\ell(M, \RR)$$
(we also write $\PD$ for the inverse map) is induced by the map which sends the class of a form $\varphi$ to the class of the current induced by $\varphi$, a fact that we shall frequently use without comment.
We shall mainly be interested in closed manifolds, so we have a natural ring isomorphism $H^\bullet_\cpt(M, \RR) = H^\bullet(M, \RR)$.

We say that an open set $U \subseteq M$ is a \dfn{deformed euclidean ball} if there is a bi-Lipschitz diffeomorphism $U \cong \Ball^d$.

\begin{proposition}[Poincar\'e's lemma with elliptic regularity]\label{Hodge theorem}
Suppose that $U$ is a deformed euclidean ball, $\ell \in \{1, \dots, d\}$, and $F \in L^\infty(U, \Omega^\ell_{\rm cl})$.
Then there exists a H\"older continuous $\ell - 1$-form $A$ such that $F = \dif A$.
\end{proposition}
\begin{proof}
By the main theorem of \cite{Costabel2010}, for any $1 < p < \infty$ there exists a solution operator 
$$L^p(U, \Omega^\ell_{\rm cl}) \to W^{1, p}(U, \Omega^{\ell - 1})$$
of the equation $\dif A = F$.
The result now follows from the Sobolev embedding theorem if we take $p > d$.
\end{proof}

\begin{proposition}\label{portmanteau}
Let $(\mu_n)$ be a sequence of positive Radon measures with $\mu_n(X) \lesssim 1$, and $\mu$ a positive Radon measure, on a compact metrizable space $X$.
Then the following are equivalent:
\begin{enumerate}
\item $\mu_n \weakto^* \mu$.
\item $\liminf_{n \to \infty} \mu_n(X) \geq \mu(X)$, and for every closed $Y \subseteq X$, $\limsup_{n \to \infty} \mu_n(Y) \leq \mu(Y)$.
\item $\limsup_{n \to \infty} \mu_n(X) \leq \mu(X)$, and for every open $U \subseteq X$, $\liminf_{n \to \infty} \mu_n(U) \geq \mu(U)$.
\end{enumerate}
\end{proposition}
\begin{proof}
After rescaling so that $\mu_n(X) \leq 1$ for every $n$, this follows from \cite[Theorem 13.16]{klenke2013probability}.
\end{proof}

\begin{lemma}\label{measurable function is 1}
Let $\mu$ be a positive finite measure on a measurable space $X$, and let $f$ be a measurable function with $\|f\|_{L^\infty(X, \mu)} \leq 1$.
If $\int_X f \dif \mu = \mu(X)$, then $f = 1$, $\mu$-almost everywhere.
\end{lemma}
\begin{proof}
Let $E \subseteq X$ be a measurable set.
Then $\int_E f \dif \mu \leq \mu(E)$.
Conversely, since $X \setminus E$ is also measurable,
$$\int_E f \dif \mu = \int_X f \dif \mu - \int_{X \setminus E} f \dif \mu \geq \mu(X) - \mu(X \setminus E) = \mu(E),$$
so $\int_E f \dif \mu = \mu(E)$.
Since $E$ was arbitrary, we conclude $f = 1$, $\mu$-almost everywhere.
\end{proof}

%%%%%%%%%%%%%
\subsection{Anzellotti's pairing}
We are going to appeal to results of Anzellotti \cite{Anzellotti1983}, which were formulated in the gradient-divergence-curl formalism of vector calculus on euclidean space.
In fact, these results can be formulated in terms of differential forms, after which they become manifestly invariant under bi-Lipschitz changes of coordinates.
To demonstrate the reformulation process, we go through the details of the construction of the Anzellotti wedge product in the language of differential forms, and then leave the details of the other reformulations to the reader.

\begin{definition}
Let $u \in BV(M, \RR)$ and $F \in L^\infty(M, \Omega^{d - 1})$.
Assume that $\dif F \in L^d$.
Then the \dfn{Anzellotti wedge product} of $\dif u$ and $F$ is the distribution $\dif u \wedge F$, such that for every test function $\chi \in C^\infty_\cpt(M, \RR)$,
$$\langle \dif u \wedge F, \chi\rangle := -\int_M \chi u \wedge \dif F - \int_M \dif \chi \wedge u \wedge F.$$
\end{definition}

\begin{proposition}\label{Anzellotti wedge product exists}
Let $u \in BV(M, \RR)$, $F \in L^\infty(M, \Omega^{d - 1})$, and $\dif F \in L^d$.
Then the Anzellotti wedge product $\dif u \wedge F$ is well-defined as a distribution.
In fact, $\dif u \wedge F$ is a Radon measure, and 
\begin{equation}\label{Anzellotti Holder inequality}
\Mass(\dif u \wedge F) \leq \Comass(F) \Mass(\dif u).
\end{equation}
\end{proposition}
\begin{proof}
We follow the proof of \cite[Theorem 1.5]{Anzellotti1983}.
By the Sobolev embedding theorem for $BV$ \cite[\S5.6]{evans2015measure}, $u \in L^{\frac{d}{d - 1}}$, the dual space of $L^d$.
Therefore for every $\chi \in C^\infty_\cpt(M)$, $\langle \dif u \wedge F, \chi\rangle$ is finite, so $\dif u \wedge F$ is well-defined as a distribution.

Suppose that $\supp \chi \Subset U$ for some $U \Subset M$.
If $u$ is sufficiently smooth, then an integration by parts gives 
$$|\langle \dif u \wedge F, \chi\rangle| = \left|\int_M \chi \dif u \wedge F\right| \leq \Comass(F) \|\chi\|_{C^0} \Mass_U(\dif u)$$
where $\Mass_U$ denotes the mass \emph{computed in $U$}.
In general, we can find a sequence $(u_n) \subset C^\infty$ such that $u_n \weakto^* u$ in $BV$.
Then $u_n \weakto u$ in $L^{\frac{d}{d - 1}}$ and $\dif u_n \weakto^* \dif u$ as currents of locally finite mass.
Since we are testing $\dif u$ against the $L^d$ form $\chi F$,
$$|\langle \dif u \wedge F, \chi\rangle| \leq \liminf_{n \to \infty} |\langle \dif u_n \wedge F, \chi\rangle| \leq \Comass(F) \|\chi\|_{C^0} \liminf_{n \to \infty} \Mass_U(\dif u_n).$$
But, by Proposition \ref{portmanteau},
$$\liminf_{n \to \infty} \Mass_U(\dif u_n) \leq \liminf_{n \to \infty} \Mass_{\overline U}(\dif u_n) \leq \Mass(\dif u)$$
which gives the desired estimate (\ref{Anzellotti Holder inequality}), since we only used the $C^0$ norm of $\chi$.
\end{proof}

\begin{proposition}[trace theorem {\cite[Theorem 1.2]{Anzellotti1983}}]\label{integration is welldefined}
Let $\iota: N \to M$ be the inclusion of an oriented Lipschitz hypersurface.
Let $\mathcal X$ be the space of $F \in L^\infty(M, \Omega^{d - 1})$ such that the components of $\dif F$ are Radon measures.
Then the pullback $\iota^*$ of $d - 1$-forms extends to a bounded linear map
$$\iota^*: \mathcal X \to L^\infty(N, \Omega^{d - 1})$$
satisfying the estimate
\begin{equation}\label{integral over chain is linfinity}
	\|\iota^* F\|_{L^\infty(N)} \leq \|F\|_{L^\infty(M)}.
\end{equation}
\end{proposition}

% A measurable set $U \subseteq M$ has \dfn{locally finite perimeter} if $1_U \in BV_\loc$.
% The \dfn{measure-theoretic boundary} of $U$ is the set $\partial U$ of $x \in M$ such that for all $\varepsilon > 0$,
% $$0 < \Mass(U \cap B(x, \varepsilon)) < \Mass(B(x, \varepsilon)).$$
% Then $\partial U$ is an integral $d - 1$-current (in particular, the sum of Lipschitz hypersurfaces), with surface measure $\star |\dif 1_U|$; moreover, $\dif 1_U$ is conormal to $\partial U$ \cite[Theorem 4.4]{Giusti77}.
% See also \cite[Appendix A]{BackusCML} for a discussion of related issues in our setting; in particular, we assert that without loss of generality, we may assume that $\partial U$ is closed.
% The superlevel sets $\{u > \lambda\}$ of a function $u \in BV_\loc(M, \RR)$ have locally finite perimeter \cite[Theorem 1.23]{Giusti77}.

% \begin{proposition}[coarea formula]\label{coarea theorem}
% Let $u \in BV_\loc(M, \RR)$, $F \in L^\infty_\loc(M, \Omega^{d - 1})$, and $\chi \in C^0_\cpt(M, \RR)$.
% If $\dif F \in L^d_\loc$, then
% \begin{equation}\label{coarea formula}
% \int_M \chi \dif u \wedge F = \int_{-\infty}^\infty \int_{\partial \{u > \lambda\}} \chi F \dif \lambda.
% \end{equation}
% \end{proposition}
% \begin{proof}
% After identifying $\dif 1_U$ with the integral current $\partial U$, this follows from \cite[Proposition 2.7(ii)]{Anzellotti1983}.
% \end{proof}

%%%%%%%%%%%%%
\section{Derivation of the dual equations}\label{duality derivation}
We are interested in deriving the duality equation (\ref{strong duality}) and the PDE for a $p$-tight form (\ref{p tight}) from the duality theorem for convex optimization.
This is not strictly necessary for the proofs of the main theorems of this paper, but they probably seem unmotivated without it.
Indeed, Thurston's conjecture prescribes the use of ``the max flow min cut principle, convexity, and $L^0 \leftrightarrow L^\infty$ duality'' -- and convex duality is a sort of grand generalization of the max flow min cut principle.

We follow \cite[Chapter IV]{Ekeland99}.
For a reflexive Banach space $X$, we denote by $\hat X$ its dual.
If $I: X \to \RR \cup \{+\infty\}$ is a convex function, we introduce its \dfn{Legendre transform}, the convex function
\begin{align*}
	\hat I: \hat X &\to \RR \cup \{+\infty\}\\
	\xi &\mapsto \sup_{x \in X} \langle \xi, x\rangle - I(x).
\end{align*}
We identify the cokernel of a linear map with the kernel of its adjoint.

\begin{theorem}[Fenchel--Rockafellar duality]\label{abstract convex analysis}
Let $\Lambda : X \to Y$ be a bounded linear map between reflexive Banach spaces.
Let $I: Y \to \RR \cup \{+\infty\}$ satisfy:
\begin{enumerate}
\item $I$ and $\hat I$ are strictly convex,
\item $I$ is lower semicontinuous,
\item if $|y| \to \infty$ in $Y$, then $I(y) \to +\infty$, and 
\item there exists a point $x \in X$ such that $I$ is continuous and finite at $\Lambda(x)$.
\end{enumerate}
Then:
\begin{enumerate}
\item There exists a minimizer $\underline x \in X$ of $I(\Lambda(x))$, unique modulo $\ker \Lambda$.
\item There exists a unique maximizer $\overline \eta$ of $-\hat I(-\eta)$ subject to the constraint $\eta \in \coker \Lambda$.
\item We have \dfn{strong duality}
\begin{equation}\label{abstract strong duality}
I(\Lambda(\underline x)) = -\hat I(-\overline \eta).
\end{equation}
\end{enumerate}
\end{theorem}
\begin{proof}
This is largely a special case of \cite[Chapter IV, Theorem 4.2]{Ekeland99}.
Let $\mathscr P, \mathscr P^*$ be as in the statement of that theorem.
Then $\mathscr P$ is the problem of minimizing $J(x, \Lambda x)$ where $J(x, y) := I(y)$.
The Legendre transform of $J$ satisfies 
$$\hat J(\xi, \eta) = \begin{cases} \hat I(\eta), & \xi = 0, \\
	+\infty, &\xi \neq 0,
\end{cases}$$
and $\mathscr P^*$ is the problem of maximizing
$$-\hat J(\Lambda^* \eta, -\eta) = \begin{cases}
	-\hat I(-\eta), &\eta \in \ker \Lambda^*, \\
	-\infty, &\eta \notin \ker \Lambda^*,
\end{cases}$$
where $\Lambda^*$ is the adjoint of $\Lambda$.
Then most of the various assertions of this theorem follow immediately from \cite[Chapter IV, Theorem 4.2]{Ekeland99}.
The fact that $\overline \eta \in \coker \Lambda$ follows from the facts that $\overline \eta$ is a solution of $\mathscr P^*$, but any solution of $\mathscr P^*$ must be a member of $\ker \Lambda^*$. 
To establish uniqueness, we use \cite[Chapter II, Proposition 1.2]{Ekeland99}, the fact that $\hat I$ is strictly convex, and the fact that we may view $I \circ \Lambda$ as a strictly convex function on the reflexive Banach space $X/\ker \Lambda$.
\end{proof}

Let $M$ be a closed oriented Riemannian manifold with fundamental group $\Gamma$ and universal cover $\tilde M \to M$.
Let $\alpha \in \Hom(\Gamma, \RR)$ be a representation, which, as always, we identify with the harmonic $1$-form representing the image of $\alpha$ under the Hurcewiz map.

\begin{theorem}[convex duality for the $q$-Laplacian]\label{mfmc qLaplacian}
Let $\alpha \in \Hom(\Gamma, \RR)$.
Then there is an $\alpha$-equivariant $q$-harmonic function,
which is unique modulo constants.
Furthermore, there is a minimizer $F$ of 
$$J_{p, \alpha}(F) := \frac{1}{p} \int_M \star |F|^p - \int_M \alpha \wedge F$$
among all closed $d - 1$-forms on $M$.
Moreover, $F$ is the unique closed $d - 1$-form such that
\begin{equation}\label{strong duality appendix}
	\frac{1}{q} \int_M \star |\dif u|^q + \frac{1}{p} \int_M \star |F|^p + \int_M \dif u \wedge F = 0.
\end{equation}
\end{theorem}
\begin{proof}
Let $X$ be the space of $W^{1, q}_\loc(\tilde M)$ functions $u$ which are $\Gamma$-equivariant, and let $Y := L^q(M, \Omega^1)$.
We identify $\hat Y$ with $L^p(M, \Omega^{d - 1})$ using the perfect pairing 
\begin{align*}
	L^p(M, \Omega^{d - 1}) \times Y &\to \RR \\
	(F, \varphi) &\mapsto \int_M \varphi \wedge F.
\end{align*}
Then $\Lambda := (\dif: X \to Y)$ is a bounded linear map, and the $\alpha$-equivariant $q$-Laplacian is the Euler-Lagrange equation of $I \circ \Lambda$, where $I$ is the strictly convex functional such that
$$I(\varphi) := \frac{1}{q} \int_M \star |\varphi|^q$$
if $\varphi$ is cohomologous to the harmonic $1$-form $\alpha$, and $I(\varphi) := +\infty$ otherwise.

Let
$$I_\alpha(\psi) := \frac{1}{q} \int_M \star |\psi + \alpha|^q,$$
defined for exact $L^q$ $1$-forms $\psi$, thus $\widehat{I_\alpha}$ is defined on the space of $L^p$ $d - 1$-forms modulo the kernel of the map
$$F \mapsto \left(\psi \mapsto \int_M \psi \wedge F\right)$$
and we lift it to $L^p(M, \Omega^{d - 1})$.

Let $v$ be a primitive of $\alpha$.
Then an $\alpha$-equivariant $u$ minimizes $I$ iff $u - v$ minimizes $I_\alpha \circ \Lambda$, which happens iff $u$ is $\alpha$-equivariant $q$-harmonic.
Since $I_\alpha(\psi) = I_0(\psi + \alpha)$, we can apply \cite[Chapter I, Remark 4.1]{Ekeland99} to see that $I_\alpha$ and $\widehat{I_\alpha}$ are strictly convex and
$$\widehat{I_\alpha}(F) = \widehat{I_0}(F) - \int_M \alpha \wedge F = \frac{1}{p} \int_M \star |F|^p - \int_M \alpha \wedge F.$$
Since $\coker \Lambda$ is the space of closed $L^p$ $d - 1$-forms on $M$, for $F \in \coker \Lambda$, $\widehat{I_\alpha}(F)$ does not depend on the choice of representatives $\alpha, v$, or of the lift of $\widehat{I_\alpha}$ to $L^p(M, \Omega^{d - 1})$.

Next we observe that $\ker \Lambda$ is the space of constants, and for any $\alpha$-equivariant $u$ and closed $F$,
\begin{align*}
I(\Lambda u) + \widehat{I_\alpha}(-F)
&= \frac{1}{q} \int_M \star |\dif u|^q + \frac{1}{p} \int_M \star |F|^p + \int_M \alpha \wedge F \\
&= \frac{1}{q} \int_M \star |\dif u|^q + \frac{1}{p} \int_M \star |F|^p + \int_M \dif u \wedge F.
\end{align*}
All of the assertions of this proposition now follow from Theorem \ref{abstract convex analysis}.
\end{proof}

Let $u$ be an $\alpha$-equivariant $q$-harmonic function, and let 
\begin{equation}\label{dual solution appendix}
F := -|\dif u|^{q - 2} \star \dif u.
\end{equation}
This $d - 1$-form is clearly closed, since $u$ is $q$-harmonic.
Since
$$(p - 2)(q - 1) + (q - 2) = 0,$$
we have 
$$\dif(|F|^{p - 2} \star F) = \pm \dif(|\dif u|^{(p - 2)(q - 1)} |\dif u|^{q - 2} \dif u) = \pm \dif \dif u = 0$$
where the sign depends on the dimension $d$.
Thus $F$ solves the PDE 
\begin{equation}\label{pMaxwell}
\begin{cases}
	\dif F = 0 \\
	\dif^* (|F|^{p - 2} F) = 0.
\end{cases}
\end{equation}
Moreover, $F$ satisfies (\ref{strong duality appendix}) (since $\dif u$ is cohomologous to $\alpha$) so $F$ is the unique minimizer of $J_{p, \alpha}$.
Thus we have proven:

\begin{corollary}
Let $u, F$ be as in Theorem \ref{mfmc qLaplacian}.
Then $F$ is $p$-tight, and we have the pointwise duality relation (\ref{dual solution appendix}).
\end{corollary}

Finally, let us observe that when $d = 2$, we can locally write $F = \dif v$ for a function $v$.
Then (\ref{dual solution appendix}) is the equation for the conjugate $q$-harmonic that was studied by Daskalopolous and Uhlenbeck \cite[\S3]{daskalopoulos2020transverse}.
If, in addition, $p = q = 2$, then (\ref{dual solution appendix}) is exactly the Cauchy-Riemann equation.
So it is reasonable to view (\ref{dual solution appendix}) as a higher-dimensional and nonlinear generalization of the Cauchy-Riemann equation.



\printbibliography

\end{document}
