\documentclass[reqno,11pt]{amsart}
\usepackage[letterpaper, margin=1in]{geometry}
\RequirePackage{amsmath,amssymb,amsthm,graphicx,mathrsfs,url,slashed,subcaption}
\RequirePackage[usenames,dvipsnames]{xcolor}
\RequirePackage[colorlinks=true,linkcolor=Red,citecolor=Green]{hyperref}
\RequirePackage{amsxtra}
\usepackage{cancel}
\usepackage{tikz, quiver, wrapfig}
%\usepackage[T1]{fontenc}

% \setlength{\textheight}{9.3in} \setlength{\oddsidemargin}{-0.25in}
% \setlength{\evensidemargin}{-0.25in} \setlength{\textwidth}{7in}
% \setlength{\topmargin}{-0.25in} \setlength{\headheight}{0.18in}
% \setlength{\marginparwidth}{1.0in}
% \setlength{\abovedisplayskip}{0.2in}
% \setlength{\belowdisplayskip}{0.2in}
% \setlength{\parskip}{0.05in}
%\renewcommand{\baselinestretch}{1.05}

\title[Laminations and calibrations I]{Minimal laminations and tight calibrations I: The $1$-Laplacian and its dual}
\author{Aidan Backus}
\address{Department of Mathematics, Brown University}
\email{aidan\_backus@brown.edu}
\date{\today}

\newcommand{\NN}{\mathbf{N}}
\newcommand{\ZZ}{\mathbf{Z}}
\newcommand{\QQ}{\mathbf{Q}}
\newcommand{\RR}{\mathbf{R}}
\newcommand{\CC}{\mathbf{C}}
\newcommand{\DD}{\mathbf{D}}
\newcommand{\PP}{\mathbf P}
\newcommand{\MM}{\mathbf M}
\newcommand{\II}{\mathbf I}
\newcommand{\Hyp}{\mathbf H}
\newcommand{\Sph}{\mathbf S}
\newcommand{\Group}{\mathbf G}
\newcommand{\GL}{\mathbf{GL}}
\newcommand{\Orth}{\mathbf{O}}
\newcommand{\SpOrth}{\mathbf{SO}}
\newcommand{\Ball}{\mathbf{B}}

\newcommand*\dif{\mathop{}\!\mathrm{d}}

\DeclareMathOperator{\card}{card}
\DeclareMathOperator{\dist}{dist}
\DeclareMathOperator{\id}{id}
\DeclareMathOperator{\Hom}{Hom}
\DeclareMathOperator{\coker}{coker}
\DeclareMathOperator{\supp}{supp}
\DeclareMathOperator{\Teich}{Teich}
\DeclareMathOperator{\tr}{tr}

\newcommand{\Leaves}{\mathscr L}
\newcommand{\Lagrange}{\mathscr L}
\newcommand{\Hypspace}{\mathscr H}

\newcommand{\Chain}{\underline C}

\newcommand{\Two}{\mathrm{I\!I}}
\newcommand{\Ric}{\mathrm{Ric}}

\newcommand{\normal}{\mathbf n}
\newcommand{\radial}{\mathbf r}
\newcommand{\evect}{\mathbf e}
\newcommand{\vol}{\mathrm{vol}}

\newcommand{\diam}{\mathrm{diam}}
\newcommand{\Ell}{\mathrm{Ell}}
\newcommand{\inj}{\mathrm{inj}}
\newcommand{\Lip}{\mathrm{Lip}}
\newcommand{\MCL}{\mathrm{MCL}}
\newcommand{\Riem}{\mathrm{Riem}}

\newcommand{\Mass}{\mathbf M}
\newcommand{\Comass}{\mathbf L}

\newcommand{\Min}{\mathrm{Min}}
\newcommand{\Max}{\mathrm{Max}}

\newcommand{\dfn}[1]{\emph{#1}\index{#1}}

\renewcommand{\Re}{\operatorname{Re}}
\renewcommand{\Im}{\operatorname{Im}}

\newcommand{\loc}{\mathrm{loc}}
\newcommand{\cpt}{\mathrm{cpt}}

\def\Japan#1{\left \langle #1 \right \rangle}

\newtheorem{theorem}{Theorem}[section]
\newtheorem{badtheorem}[theorem]{``Theorem"}
\newtheorem{prop}[theorem]{Proposition}
\newtheorem{lemma}[theorem]{Lemma}
\newtheorem{sublemma}[theorem]{Sublemma}
\newtheorem{proposition}[theorem]{Proposition}
\newtheorem{corollary}[theorem]{Corollary}
\newtheorem{conjecture}[theorem]{Conjecture}
\newtheorem{axiom}[theorem]{Axiom}
\newtheorem{assumption}[theorem]{Assumption}

\newtheorem{mainthm}{Theorem}
\renewcommand{\themainthm}{\Alph{mainthm}}

\newtheorem{claim}{Claim}[theorem]
\renewcommand{\theclaim}{\thetheorem\Alph{claim}}
% \newtheorem*{claim}{Claim}

\theoremstyle{definition}
\newtheorem{definition}[theorem]{Definition}
\newtheorem{remark}[theorem]{Remark}
\newtheorem{example}[theorem]{Example}
\newtheorem{notation}[theorem]{Notation}

\newtheorem{exercise}[theorem]{Discussion topic}
\newtheorem{homework}[theorem]{Homework}
\newtheorem{problem}[theorem]{Problem}

\makeatletter
\newcommand{\proofpart}[2]{%
  \par
  \addvspace{\medskipamount}%
  \noindent\emph{Part #1: #2.}
}
\makeatother



\numberwithin{equation}{section}


% Mean
\def\Xint#1{\mathchoice
{\XXint\displaystyle\textstyle{#1}}%
{\XXint\textstyle\scriptstyle{#1}}%
{\XXint\scriptstyle\scriptscriptstyle{#1}}%
{\XXint\scriptscriptstyle\scriptscriptstyle{#1}}%
\!\int}
\def\XXint#1#2#3{{\setbox0=\hbox{$#1{#2#3}{\int}$ }
\vcenter{\hbox{$#2#3$ }}\kern-.6\wd0}}
\def\ddashint{\Xint=}
\def\dashint{\Xint-}

\usepackage[backend=bibtex,style=alphabetic,giveninits=true]{biblatex}
\renewcommand*{\bibfont}{\normalfont\footnotesize}
\addbibresource{best_curl.bib}
\renewbibmacro{in:}{}
\DeclareFieldFormat{pages}{#1}

\newcommand\todo[1]{\textcolor{red}{TODO: #1}}


\begin{document}
\begin{abstract}
As part of a larger investigation into the duality between laminations and calibrations, we investigate the $1$-Laplacian and its convex dual problem.
\todo{What do we find?}
\end{abstract}

\maketitle

%%%%%%%%%%%%%%%%%%%%%%%%%%%%%%%%%%%%%%%%%%%%%%%%%%%%%%%
\section{Introduction}
\todo{Write an introduction}

\begin{theorem}[max flow min cut, Bangert and Cui's version]\label{BangertCui}
Let $F$ be a continuous best comass calibration on a closed Riemannian manifold of dimension $\leq 7$.
Then there exists a measured oriented minimal lamination $\lambda$ such that every leaf of $\lambda$ is $F$-calibrated.
In particular, 
$$\Mass(\lambda) = \langle [\lambda], [F]\rangle.$$
\end{theorem}

\subsection{Tight forms and functions of least gradient}
We shall prove a version of Theorem \ref{BangertCui}.
The idea, as in \cite{daskalopoulos2020transverse}, is to consider $d - 1$-forms which behave analogously to $p$-harmonic functions, and take the limit $p \to \infty$ to get a best comass form which behaves analogously to an $\infty$-harmonic function.
At the same time we consider the convex dual problem.
The best comass forms will be ``maximal flows''; their dual ``minimal cuts'' will be functions of least gradient.

To be more precise, let $(p, q)$ be a H\"older pair (thus $1/p + 1/q = 1$) such that $d < p < \infty$.
Motivated by the $p$-Laplace equation $\dif^*(|\dif v|^{p - 2} \dif v) = 0$, we introduce \dfn{$p$-tight} forms, which are closed $d-1$-forms which solve the system of PDE
$$\dif^*(|F|^{p - 2} F) = 0.$$
Given a $p$-tight form, the $\Gamma$-equivariant function $u$ on the universal cover such that
$$\dif u = (-1)^{d - 1} |F|^{p - 2} \star F$$
is $q$-harmonic -- in other words, $u$ is a solution of the $q$-Laplace equation 
$$\dif^*(|\dif u|^{q - 2} \dif u) = 0.$$
A function $u$ has \dfn{least gradient} if it minimizes $\int_M \star |\dif u|$.
Our first theorem constructs a best comass form, and a dual function of least gradient, by taking limits of $p$-tight forms and their dual $q$-harmonic functions.

\begin{mainthm}\label{existence of infinity tight forms}
Let $\rho \in H^{d - 1}(M, \RR)$ be a cohomology class.
Let $(F_p, u_q)$ be the family of dual pairs of $p$-tight forms and $q$-harmonic functions, suitably normalized, with $[F_p] = \rho$ and $(p, q)$ ranging over H\"older pairs with $d < p < \infty$.
Then there exists a pair $(F, u)$ such that as $p \to \infty$ along a subsequence, $F_p \to F$ weakly in $L^r$ for any $d < r < \infty$, and $u_q \to u$ weakly in $BV$, with the following properties:
\begin{enumerate}
\item $F$ has best comass.
\item $u$ has least gradient.
\item The product of distributions $\dif u \wedge F$ is well-defined.
\item We have the duality relation
\begin{equation}\label{max flow mean cut}
\Comass(\rho) \star |\dif u| = \dif u \wedge F.
\end{equation}
\end{enumerate}
\end{mainthm}

This is a combination of Propositions \ref{existence infinity} and \ref{existence 1}.
We call the best comass form $F$ a \dfn{tight} form.

We highlight the duality condition (\ref{max flow mean cut}) as the main point of the theorem.
It is crucial to the proof that $u$ is $1$-harmonic, and allows us to prove this without a careful analysis of the limiting behavior of $q$-harmonic functions as in \cite[Theorem 2.4]{Mazon14}, or of $p$-tight forms as in \cite[\S6]{daskalopoulos2020transverse}.
On the other hand, if $\Comass(\rho) = 1$, then we shall be able to use (\ref{max flow mean cut}) to show that the level sets of $u$ are $F$-calibrated.

In addition to $p$-approximation, another ingredient in the proof of Theorem \ref{existence of infinity tight forms} is the geometric measure theory of closed $L^\infty$ $d - 1$-forms, which we believe to be of independent interest. 
In particular, we show that, given an immersion $\iota: N \to M$ of codimension $1$, the normal trace map $F \mapsto \iota^* F$, defined on closed $d - 1$-forms, sends $L^\infty(M)$ to $L^\infty(N)$.
This allows us to define $\dif u \wedge F$ and $\int_N F$.
The reader may compare to the much better-known $L^2$ normal trace theorem, which asserts that the normal trace map sends $L^2(M)$ to $W^{-1/2, 2}(N)$ \cite[Chapter 2]{cessenat1996mathematical}.

%%%%%%%%%%%%%%%%%%%%%
\subsection{The maximum comass locus of a tight form}
\todo{Stuff about calibrations and the duality condition}

%%%%%%%%%%%%%%%%%%%%%
\subsection{The PDE for a tight form}
Tight forms have best comass, and are obtained by taking limits of solutions of $L^p$ variational problems.
Thus they are solutions of an $L^\infty$ variational problem, and one expects them to solve a PDE analogous to the $\infty$-Laplacian.
However, the theory of the $\infty$-Laplacian is largely built around the maximum principle and viscosity solutions, neither of which have been adequately fleshed out for systems of PDE at present \cite{Katzourakis2018OnAV,Sheffield12}.

\todo{Write this out}

%%%%%%%%%%%%%%%%%%%%%
\subsection{Outline of the paper}
\todo{Write this out}

%%%%%%%%%%%%%%%%%%%%%%
\subsection{Acknowledgements}
I would like to thank Georgios Daskalopolous and Karen Uhlenbeck for suggesting this project and providing helpful comments.

This research was supported by the National Science Foundation's Graduate Research Fellowship Program under Grant No. DGE-2040433.


%%%%%%%%%%%%%%%%%%%%%%%%%%%%%%%%%%%%%%%%%%
\section{Preliminaries}\label{prevResults}
\subsection{Locally normal currents}
In order to fix conventions, we recall some well-known measure theory.
We will mainly use \cite{simon1983GMT} as a reference.

The sheaf of $\ell$-forms is denoted $\Omega^\ell$.
We assume that $\ell$-forms are $L^1_\loc$, but \emph{not} that they are continuous; hence $\dif$ must be meant in the sense of distributions.
To avoid confusion, we write $H^\ell$ for de Rham cohomology, but \emph{never} a Sobolev space (which shall only be denoted by $W^{\ell, p}$), nor a Hausdorff measure (which shall be denoted $\mathcal H^\ell$).
We let $\dif V = \star 1$ be the volume form on $M$, and for an $\ell$-rectifiable set $\tau$, we let $\dif S_\tau := \dif \mathcal H^\ell|_\tau$.

\begin{definition}
An \dfn{$\ell$-blade} is the wedge product $v = v_1 \wedge \cdots \wedge v_\ell$ of vectors $v_1, \dots, v_\ell$.
The \dfn{comass} $|\varphi|$ of an $\ell$-covector $\varphi$ is the supremum of $\langle \varphi, v\rangle/|v|$, taken over all nonzero $\ell$-blades $v$.
\end{definition}

\begin{definition}
By an $\ell$-\dfn{current of locally finite mass} $T$ we mean a continuous linear functional on the space $C^0_\cpt(M, \Omega^\ell)$ of continuous $\ell$-forms of compact support.
If we do not specify otherwise, by a \dfn{current} we shall mean a current of locally finite mass.
We write $\int_M T \wedge \varphi$ for the dual pairing of a current and a form.
The duality norm of $T$ is its \dfn{mass}, namely for an open set $U \subseteq M$,
$$\Mass_U(T) := \sup_{\substack{\supp \varphi \Subset U \\ \||\varphi|\|_{C^0} \leq 1}} \int_U T \wedge \varphi.$$
We write $\Mass(T) := \Mass_M(T)$.
\end{definition}

If $T$ represents a simplex $\sigma$, then $\Mass(T)$ is the surface area of $\sigma$.

We write $\dif T$ for the exterior derivative of a current $T$, which is also known as $-\partial T$ \cite[\S26]{simon1983GMT}.
If $\dif T$ is an $\ell - 1$-current of locally finite mass, we say that $T$ is \dfn{locally normal}.
In particular, a locally normal $d$-current is the same thing as a function of locally bounded variation, and a locally normal $0$-current is the same thing as a signed Radon measure.

\begin{definition}
The \dfn{stable norm} $\Mass(\alpha)$ of $\alpha \in H_{d - 1}(M, \RR)$ is the infimum of $\Mass(T)$ where $T$ is a $d - 1$-current of homology class $\alpha$.
Dually, the \dfn{costable norm} $\Comass(\rho)$ of $\rho \in H^{d - 1}(M, \RR)$ is the infimum of $\|F\|_{L^\infty}$ where $F$ is a $d - 1$-form of cohomology class $\rho$.
We say that a $d - 1$-form $F$ of cohomology class $\rho$ has \dfn{best comass} if 
$$\|F\|_{L^\infty} = \Comass(\rho).$$
\end{definition}


%%%%%%%%%%%%%%%%%%%%%
\subsection{Trace theorem and coarea formula}
We would like to be able to compute the comass of a form $F$ by integration of $F$ along chains, and we would like to take $\dif u \wedge F$ where $u \in BV_\loc(M)$.
Unfortunately, $F$ may only be defined almost everywhere, and then it is not clear that such an integral is well-defined; nor is it clear that the cohomology class of $F$ is well-defined (so that the notion of ``best comass'' makes no sense).
Here we show that for closed forms $F$ in $L^p$, suitable chains $\sigma$, and $u \in BV_\loc(M)$, $\int_\sigma F$ and $\dif u \wedge F$ are well-defined.

\begin{definition}
An $\ell$-current $\sigma$ is \dfn{rectifiable} if there exists an $\ell$-rectifiable set $N$, and a $\dif S_N$-measurable $\ell$-blade field $v$, such that for any $C^0_\cpt$ $\ell$-form $\varphi$,
$$\int_M \sigma \wedge \varphi = \int_N \langle \varphi, v\rangle \dif S_N,$$
and for $\dif S_N$-almost every $x \in N$, $|v(x)| \in \ZZ$.
The rectifiable current $\sigma$ is \dfn{integral} if, in addition, $\dif \sigma$ is rectifiable.
To emphasize that integral currents $\sigma$ can be represented by integration along rectifiable sets, we write $\partial \sigma := -\dif \sigma$ and $\int_\sigma \varphi := \int_M \sigma \wedge \varphi$.
\end{definition}

\begin{lemma}\label{local trace theorem}
Suppose that there is a bi-Lipschitz diffeomorphism $M \cong \Ball^d$.
Let $\tau$ be an integral $d - 1$-current and $\psi \in C^1(M)$.
Then for any $d < p < \infty$, $F \mapsto \int_\tau \psi F$ is a continuous linear function on the space of $L^p$ closed $d - 1$-forms.
\end{lemma}
\begin{proof}
First suppose that $F, G$ are closed $C^1$ $d - 1$-forms, and let $A, B$ be the $d - 2$-forms obtained from Lemma \ref{Hodge theorem}.
By integration by parts and the Sobolev embedding theorem,
\begin{align*}
	\left|\int_\tau \psi(F - G)\right| 
	&\leq \left|\int_{\partial \tau} \psi (A - B)\right| + \left|\int_\tau \dif \psi \wedge (A - B)\right| \\
	&\lesssim_{\tau, \psi} \|A - B\|_{C^0} \lesssim_p \|A - B\|_{L^p} \lesssim_p \|F - G\|_{L^p}.
\end{align*}
If $F, G$ are not necessarily $C^1$, then by Lemma \ref{mollification of closed forms}, the $C^1$ closed $d - 1$-forms are dense in the space of $L^p$ closed $d - 1$-forms, so the desired estimate holds by approximating $F, G$ by $C^1$ closed $d - 1$-forms.
\end{proof}

\begin{proposition}[trace theorem]\label{integration is welldefined}
Let $\tau$ be a compactly supported integral $d-1$-current.
Then:
\begin{enumerate}
\item For any $d < p \leq \infty$ and $\psi \in C^1(M)$, $F \mapsto \int_\tau \psi F$ extends to a continuous linear functional on the space of $L^p$ closed $d-1$-forms.
\item For any $d < p \leq \infty$, the cohomology class of any $L^p$ closed $d - 1$-form is well-defined.
\item Let $F$ be an $L^\infty$ closed $d - 1$-form. Then for every $\psi \in C^0(M)$,
\begin{equation}\label{integral over chain is linfinity}
	\int_\tau \psi F \leq \|F\|_{L^\infty} \|\psi\|_{L^1(\tau)}.
\end{equation}
\end{enumerate}
\end{proposition}
\begin{proof}
We can find a partition of unity $(\chi_\alpha)$ subordinate to an open cover $(U_\alpha)$ such that $U_\alpha$ is bi-Lipschitz diffeomorphic to $\Ball^d$.
Then we may replace $(U_\alpha)$ by a finite subcover $U_1, \dots, U_n$ of a neighborhood of $\supp \tau$.
Applying Lemma \ref{local trace theorem} with $\psi$ replaced by $\psi \chi_k$, we obtain for $F, G$ closed $L^p$ $d - 1$-forms and $p < \infty$,
$$\left|\int_\tau \psi(F - G)\right| \leq \sum_{k = 1}^n \left|\int_\tau \psi \chi_k (F - G)\right| \lesssim_\tau \|F - G\|_{L^p}$$
which gives the continuity in $L^p$.
This also implies that the cohomology is well-defined.

We now handle the case $p = \infty$.
Let $d < q < \infty$ and let $U$ be a small neighborhood of $\supp \tau$; since $\supp \tau$ is compact, if $F, G \in L^\infty$, then $F, G \in L^q(U)$ and 
$$\left|\int_\tau \psi(F - G)\right| \lesssim_\tau \|F - g\|_{L^q(U)} \lesssim_{U, q} \|F - G\|_{L^\infty}.$$
This gives the continuity in $L^\infty$.
We then use Lemma \ref{mollification of closed forms} to find smooth closed $d - 1$-forms $F_{k, m}$ on $U_k$ such that $F_{k, m} \to F|_{U_k}$ in $L^q(U_k)$ and
$$\|F_{k, m}\|_{C^0} \leq \|F\|_{L^\infty}.$$
So by the triangle inequality, if $\psi \in C^1(M)$,
\begin{align*}
\int_\tau \psi F 
&= \sum_{k = 1}^n \int_\tau \psi \chi_k F 
= \sum_{k = 1}^n \lim_{m \to \infty} \int_\tau \psi \chi_k F_{k, m} \\
&\leq \lim_{m \to \infty} \|F_{k, m}\|_{C^0} \int_\tau \psi \sum_{k = 1}^n \chi_k \dif S_\tau 
\leq \|F\|_{L^\infty} \int_\tau \psi \dif S_\tau.
\end{align*}
By approximating a $C^0$ test function $\psi$ by $C^1$ functions, we obtain the result for $\psi \in C^0$.
\end{proof}

\begin{lemma}\label{reduced level sets are integral currents}
Let $u \in BV(M)$.
Then for almost every $\lambda \in \RR$, $\{u > \lambda\}$ is an integral $d - 1$-current; moreover,
$$\int_M \star |\dif u| = \int_{-\infty}^\infty \Mass(\partial \{u > \lambda\}) \dif \lambda.$$
\end{lemma}
\begin{proof}
From the discussion of \todo{Cite \cite{BackusCML}}, we see that almost surely $\partial \{u > \lambda\}$ is a closed rectifiable (hence integral) $d - 1$-current.
So the result follows from the coarea formula.\footnote{For a proof for $BV$ functions in arbitrary metrics, see \cite[Proposition 2.5]{BackusFLG}.}
\end{proof}

\begin{proposition}[coarea formula]
Let $u \in BV(M)$, and for each $\lambda \in \RR$, let $\tau_\lambda := \partial \{u > \lambda\}$.
Then the product $\dif u \wedge F$ with a closed $L^\infty$ $d - 1$-form $F$ can be defined in one and only one way so that for every sequence $(F_n)$ which is bounded in $L^\infty$ and converges in $L^p_\loc$ for some $d < p < \infty$ to a closed $d - 1$-form $F$, $\dif u \wedge F_n \to \dif u \wedge F$ in the weak topology of measures.

Moreover, for every $\chi \in C^0_\cpt(M)$, we have the coarea formula
\begin{equation}\label{coarea formula}
\int_M \chi \dif u \wedge F = \int_{-\infty}^\infty \int_{\tau_\lambda} \chi F \dif \lambda.
\end{equation}
\end{proposition}
\begin{proof}
Motivated by Lemma \ref{reduced level sets are integral currents}, it is natural to \emph{define} $\dif u \wedge F$ to be the Radon measure satisfying (\ref{coarea formula}).
If $F$ is continuous, then this will agree with the usual definition of $\dif u \wedge F$ as a product of a Radon measure and a function.
Even if $F$ is discontinuous, (\ref{coarea formula}) defines a Radon measure.
Indeed, let $U \Subset M$ be an open set containing $\supp \chi$.
Then, by the trace theorem, Lemma \ref{reduced level sets are integral currents}, and Lemma \ref{reduced level sets are integral currents},
$$\left|\int_M \chi \dif u \wedge F\right| \leq \|\chi\|_{C^0} \|F\|_{L^\infty} \int_{-\infty}^\infty \Mass_U(\tau_\lambda) \dif \lambda = \|\chi\|_{C^0} \|F\|_{L^\infty} \Mass_U(\dif u).$$
Therefore $\dif u \wedge F$ is a Radon measure, since it has locally finite mass
$$\Mass_U(\dif u \wedge F) \leq \|F\|_{L^\infty} \Mass_U(\dif u) < \infty.$$

Now we check the convergence on $L^\infty$-bounded sequences in $L^p_\loc$.
Let $\chi \in C^0_\cpt(M)$, let $U \Subset M$ be an open set containing $\supp \chi$, and let $(F_n)$ be an $L^\infty$-bounded sequence converging in $L^p_\loc$ to a closed form $F$.
We introduce the functions
$$f_n(\lambda) := \int_{\tau_\lambda} \chi F_n, \qquad f(\lambda) := \int_{\tau_\lambda} \chi F,$$
which are well-defined by the trace theorem and Lemma \ref{reduced level sets are integral currents}.
Also consider the function $g(\lambda) := \Mass(\tau_\lambda)$, so that for almost every $\lambda \in \RR$, 
$$|f_n(\lambda)| \leq \|F_n\|_{C^0} \|\chi\|_{C^0} g(\lambda) \lesssim \|\chi\|_{C^0} g(\lambda).$$
By Lemma \ref{reduced level sets are integral currents} and the fact that $u \in BV$, $g \in L^1(\RR)$.
On the other hand, since $F_n \to F$ in $L^p(U)$ and $p > d$, $f_n \to f$ almost everywhere.
So by dominated convergence, $f_n \to f$ in $L^1(\RR)$.
In particular, by Lemma \ref{reduced level sets are integral currents},
\begin{align*}
\lim_{n \to \infty} \int_M \chi \dif u \wedge F_n
&= \lim_{n \to \infty} \int_{-\infty}^\infty f_n(\lambda) \dif \lambda
= \int_{-\infty}^\infty f(\lambda) \dif \lambda \\
&= \int_M \chi \dif u \wedge F. \qedhere 
\end{align*}
\end{proof}

%%%%%%%%%%%%%%%%%%%%%
\subsection{\texorpdfstring{$1$-harmonic functions}{One-harmonic functions}}
We shall consider the variational problems whose Euler-Lagrange equation is, at least at the formal level, the $1$-Laplacian
\begin{equation}\label{1Laplacian}
\dif^*\left(\frac{\dif u}{|\dif u|}\right) = 0.
\end{equation}
A suitable notion of weak solution for (\ref{1Laplacian}), at least for the Dirichlet problem, was introduced by Maz\'on, Rossi, and Segura de L\'eon \cite{Mazon14}; it essentially asserts that the level sets of $u$ are calibrated.

Traditionally, authors have studied the Dirichlet problem for the $1$-Laplacian.
We will instead be interested in the topological Neumann problem, which we now formulate.
Let $M$ be a closed Riemannian manifold with fundamental group $\Gamma$ and universal cover $\tilde M \to M$.
By Poincar\'e duality and the Hurcewiz theorem, we have canonical isomorphisms
\begin{equation}\label{Poincare Hurcewiz}
H_{d - 1}(M, \RR) = H^1(M, \RR) = \Hom(\Gamma, \RR).
\end{equation}
Given a representation $\alpha: \Gamma \to \RR$, which we identify with a class in $H_{d - 1}(M, \RR)$ using (\ref{Poincare Hurcewiz}), we will be interested in $\alpha$-equivariant functions $f: \tilde M \to \RR$, namely those which satisfy (for each $\gamma \in \Gamma$)
$$f(\gamma x) = f(x) + \alpha(\gamma).$$
If $f$ is $\alpha$-equivariant, then $\dif f$ drops to a current on $M$, which we also call $\dif f$, and has homology class $\alpha$.

\begin{definition}
An $\alpha$-equivariant function $u \in BV_\loc(\tilde M)$ has \dfn{least gradient} if the current $\dif u$ on $M$ satisfies 
$$\Mass(\dif u) = \Mass(\alpha).$$
\end{definition}

In other words, $u$ has least gradient if $\dif u$ has the smallest possible mass among $\alpha$-equivariant functions.

\begin{definition}
We say that an $\alpha$-equivariant function $u \in BV_\loc(\tilde M)$ is a \dfn{calibrated solution} of (\ref{1Laplacian}) if there exists a $L^\infty$ $d - 1$-form $F$ on $M$ such that
\begin{equation}\label{local calibration}
\begin{cases}
\|F\|_{L^\infty} \leq 1, \\
dF = 0, \\
\dif u \wedge F = \star |\dif u|.
\end{cases}
\end{equation}
\end{definition}

We informally refer to calibrated solutions of (\ref{1Laplacian}) as \dfn{$1$-harmonic functions}, though this terminology is not quite precise.
This formulation of calibrated solution is not worded the same as the formulation for the Dirichlet problem given by \cite{Mazon14}, but it is equivalent; their vector field $X$ is given by $(\star F)^\sharp$.
The quantity $\dif u \wedge F$ is well-defined by the coarea formula, since $\dif F = 0$.
A straightforward modification of \cite[Theorem 1.1]{Mazon14} (originally proven for the Dirichlet problem) gives:

\begin{theorem}\label{MazonRossi}
An $\alpha$-equivariant function $u \in BV_\loc(M)$ is a calibrated solution of (\ref{1Laplacian}) iff $u$ has least gradient.
\end{theorem}

% \begin{theorem}\label{main thm of old paper}
% Let $u \in BV(M)$ have least gradient in $M$. Then $1_{\{u > y\}}$ has least gradient.
% In particular, if $d \leq 7$, then every superlevel set $\{u > y\}$ is bounded by complete disjoint embedded oriented minimal hypersurfaces.
% \end{theorem}
% \begin{proof}
% Let $v := 1_{\{u > y\}}$.
% Then $v$ has least gradient \cite[Theorem 1]{BOMBIERI1969}.
% In particular, $\{u > y\}$ has finite perimeter, so it is bounded by Lipschitz hypersurfaces \cite[Chapter 4]{Giusti77}.
% So by the regularity theorem for minimal hypersurfaces \cite[\S37]{simon1983GMT}\footnote{See also \cite[Exercise 1.6]{DeLellis18} for a discussion on why the theory of \cite[\S37]{simon1983GMT} applies for arbitrary metrics, and \cite{BackusFLG} for a proof for functions of least gradient for arbitrary metrics in the spirit of Miranda \cite{Miranda66}'s original argument for functions of least gradient.} if $d \leq 7$, $\{u > y\}$ is bounded by disjoint embedded minimal hypersurfaces.
% These hypersurfaces inherit an orientation from the current $\dif v$.
% \end{proof}

%%%%%%%%%%%%%%%%%%%%%%%%%%%%%%

\subsection{Minimal laminations}
We now give a geometric characterization of $1$-harmonic functions.
Fix an interval $I \subset \RR$ and a box $J \subset \RR^{d - 1}$.
We write $\Two_N$ for the second fundamental form of a submanifold $N$.

\begin{definition}
A \dfn{laminar flow box} is a $C^0$ coordinate chart $F: I \times J \to M$ and a compact set $K \subseteq I$, such that for every $k \in K$, $F|_{\{k\} \times J}$ is a $C^1$ embedding, and the \dfn{leaf} $F(\{k\} \times J)$ is a $C^1$ complete hypersurface in $F(I \times J)$.
Two laminar flow boxes belong to the same \dfn{laminar atlas} if the transition maps between them send leaves to leaves.
\end{definition}

\begin{definition}
A \dfn{lamination} is a closed subset $S \subseteq M$, called its \dfn{support}, and a maximal laminar atlas $\mathscr A$, such that $S$ is the union of the leaves of $\mathscr A$.
A \dfn{foliation} is a lamination $\lambda$ with $\supp \lambda = M$.
\end{definition}

\begin{definition}
A lamination is
\begin{enumerate}
\item \dfn{Lipschitz} if its flow boxes are Lipschitz isomorphisms,
\item \dfn{oriented} if its transition maps are orientation-preserving, and
\item \dfn{minimal} if its leaves are minimal hypersurfaces.
\end{enumerate}
\end{definition}

% \begin{theorem}[{\cite[Theorem A]{BackusCML}}]\label{disjoint surfaces are lamination}
% Let $\mathcal S$ be a set of disjoint complete minimal hypersurfaces in a manifold $M$ of bounded geometry.
% Suppose that there exists $C > 0$ such that for every $N \in \mathcal S$, $\|\Two_N\|_{C^0} \leq C$.
% Then $\mathcal S$ is the set of leaves of a Lipschitz minimal lamination $\lambda$ of bounded curvature.
% In particular, if $\lambda$ is oriented, then there is a Lipschitz vector field on $M$ whose restriction to each $N \in \mathcal S$ is the normal vector to $N$.
% \end{theorem}

\begin{definition}
A lamination $\lambda$ with atlas $(F_\alpha, K_\alpha)$ is \dfn{measured} if it is equipped with positive Radon measures $\mu_\alpha$ with $\supp \mu_\alpha = K_\alpha$, such that the transition maps $F_\beta^{-1} \circ F_\alpha$ are measure-preserving.
The \dfn{Ruelle-Sullivan current} of a measured oriented lamination $\lambda$ with atlas $(F_\alpha, K_\alpha, \mu_\alpha)$ is the $d-1$-current $T_\lambda$ satisfying, for any partition of unity $(\chi_\alpha)$ subordinate to the open cover $(F_\alpha(I \times J))$,
$$\int_M T_\lambda \wedge \varphi = \sum_\alpha \int_{K_\alpha} \int_{\{k\} \times J} F_\alpha^* (\chi_\alpha \varphi) \dif \mu_\alpha(k).$$
The \dfn{homology class} $[\lambda]$ and \dfn{mass} $\Mass(\lambda)$ of a measured oriented lamination $\lambda$ are the homology class and mass of its Ruelle-Sullivan current.
The measured oriented lamination $\lambda$ is \dfn{homologically minimizing} if
$$\Mass(\lambda) = \Mass([\lambda]).$$
\end{definition}

The notion of Ruelle-Sullivan current  was introduced by \cite{Ruelle75} and studied in the context of geodesic laminations in \cite[\S8]{daskalopoulos2020transverse}.
The motivation of the definition is that if $\lambda$ is a $d - 1$-chain, then $T_\lambda$ is just integration along $\lambda$.

\begin{theorem}[{\cite[Theorem B]{BackusCML}}]\label{1 harmonic is MOML}
Suppose that $d \leq 7$ and let $u \in BV_\loc(\tilde M)$ be a $\Gamma$-equivariant function of least gradient.
Then $\dif u$ is the Ruelle-Sullivan current of a measured oriented Lipschitz lamination, which is minimal and homologically minimizing.
\end{theorem}

%%%%%%%%%%%%%%%%%%%%
\subsection{Convex duality}
We follow \cite{Ekeland99}.
For a reflexive Banach space $X$, we denote by $\hat X$ its dual.
If $I: X \to \RR \cup \{+\infty\}$ is a convex function, we introduce its \dfn{Legendre transform}, the convex function
\begin{align*}
	\hat I: \hat X &\to \RR \cup \{+\infty\}\\
	\xi &\mapsto \sup_{x \in X} \langle \xi, x\rangle - I(x).
\end{align*}
We identify the cokernel of a linear map with the kernel of its adjoint.
In this setting, we have the following form of the convex duality theorem.

\begin{theorem}[convex duality]\label{abstract convex analysis}
Let $\Lambda : X \to Y$ be a bounded linear map between reflexive Banach spaces.
Let $I: Y \to \RR \cup \{+\infty\}$ satisfy:
\begin{enumerate}
\item $I$ and $\hat I$ are strictly convex,
\item $I$ is lower semicontinuous,
\item if $|y| \to \infty$ in $Y$, then $I(y) \to +\infty$, and 
\item there exists a point $x \in X$ such that $I$ is continuous and finite at $\Lambda(x)$.
\end{enumerate}
Then:
\begin{enumerate}
\item There exists a minimizer $\underline x \in X$ of $I(\Lambda(x))$, unique modulo $\ker \Lambda$.
\item There exists a unique maximizer $\overline \eta$ of $-\hat I(-\eta)$ subject to the constraint $\eta \in \coker \Lambda$.
\item We have \dfn{strong duality}
\begin{equation}\label{abstract strong duality}
I(\Lambda(\underline x)) = -\hat I(-\overline \eta).
\end{equation}
\end{enumerate}
\end{theorem}
\begin{proof}
This is largely a special case of \cite[Chapter IV, Theorem 4.2]{Ekeland99}.
Let $\mathscr P, \mathscr P^*$ be as in the statement of that theorem.
Then $\mathscr P$ is the problem of minimizing $J(x, \Lambda x)$ where $J(x, y) := I(y)$.
The Legendre transform of $J$ satisfies 
$$\hat J(\xi, \eta) = \begin{cases} \hat I(\eta), & \xi = 0, \\
	+\infty, &\xi \neq 0,
\end{cases}$$
and $\mathscr P^*$ is the problem of maximizing
$$-\hat J(\Lambda^* \eta, -\eta) = \begin{cases}
	-\hat I(-\eta), &\eta \in \ker \Lambda^*, \\
	-\infty, &\eta \notin \ker \Lambda^*,
\end{cases}$$
where $\Lambda^*$ is the adjoint of $\Lambda$.
Then most of the various assertions of this theorem follow immediately from \cite[Chapter IV, Theorem 4.2]{Ekeland99}.
The fact that $\overline \eta \in \coker \Lambda$ follows from the facts that $\overline \eta$ is a solution of $\mathscr P^*$, but any solution of $\mathscr P^*$ must be a member of $\ker \Lambda^*$. 
To establish uniqueness, we use \cite[Chapter II, Proposition 1.2]{Ekeland99}, the fact that $\hat I$ is strictly convex, and the fact that we may view $I \circ \Lambda$ as a strictly convex function on the reflexive Banach space $X/\ker \Lambda$.
\end{proof}




%%%%%%%%%%%%%%%%%%%%%%%%%%%%%%%%%%%%%%%%%%

\section{\texorpdfstring{Closed $L^\infty$ $d - 1$-forms}{Closed (d - 1)-forms of finite comass}}\label{comass sec}
The goal of the present series of papers is to study calibrations of laminations.
If the calibration $F$ is continuous, then the theory of \cite{bangert_cui_2017} of calibrated laminations applies; however, we are not aware of general existence results on continuous calibrations, only $L^\infty$ calibrations, such as \cite[\S4.12]{Federer1974}.
Thus, we now introduce $L^\infty$ calibrations of laminations.

\begin{definition}
A \dfn{calibration} is a closed $d - 1$-form $F$ such that $\Comass(F) = 1$.
An integral $d - 1$-current $\sigma$ is $F$-\dfn{calibrated} if 
$$\Mass(\sigma) = \int_\sigma F.$$
A lamination $\lambda$ is \dfn{$F$-calibrated} if every leaf of $\lambda$ is $F$-calibrated.
\end{definition}

The fundamental theorem of calibrated geometry \cite{Harvey82} asserts that a calibrated integral current is homologically minimizing relative to its boundary (and hence minimal).
Therefore every calibrated lamination $\lambda$ is minimal, and if $\lambda$ is measured oriented (so the homology class $[\lambda]$ is well-defined), $\lambda$ is homologically minimizing.

\subsection{Local comass}
We will be interested in the points at which the $L^\infty$ calibration $F$ attains its comass.
One could pose this problem as the problem of computing the locus $\{|F| = \|F\|_{L^\infty}\}$.
However, $|F(x)|$, the comass of the $d - 1$-covector $F(x)$, is both only defined for almost every $x$, and $F$ is not norm-approximable by smooth functions.
So as a proxy for $|F|$, which may fail to be defined on a null set, we use the local comass, which is defined everywhere.

\begin{definition}
For an open set $\Omega \subseteq M$, let $\Chain_{d - 1}(\Omega)$ be the set of simplicial $d - 1$-chains in $\Omega$.
For each closed $d - 1$-form $F$ on $\Omega$, let
$$\Comass_\Omega(F) := \sup_{\sigma \in \Chain_{d - 1}(\Omega)} \frac{1}{\Mass(\sigma)} \int_\sigma F.$$
The \dfn{local comass} of a closed $d - 1$-form $F$ at $x \in M$ is 
$$\Comass(F, x) = \limsup_{\varepsilon \to 0} \Comass_{B_\varepsilon(x)}(F).$$
\end{definition}

By the trace theorem, $\Comass_\Omega(F)$ is well-defined (but possibly $+\infty$) for any measurable $F$.
Since $\Comass_{B_\varepsilon(x)}(F)$ is a supremum over a set which grows in $\varepsilon$, it is increasing in $\varepsilon$, so the limit superior is actually a limit and an infimum:
$$\Comass(F, x) = \lim_{\varepsilon \to 0} \Comass_{B_\varepsilon(x)}(F) = \inf_{\varepsilon > 0} \Comass_{B_\varepsilon(x)}(F).$$
In particular, if we write $\Comass(F) := \Comass_M(F)$, then $\Comass(F, x) \leq \Comass(F)$.

The local comass was defined in an analogous manner to the local Lipschitz constant.
As such, it enjoys many of the same properties, including those endowed on the local Lipschitz constant by \cite[Lemma 4.3]{Crandall2008}:

\begin{proposition}\label{crandall}
Let $F$ be a closed $L^\infty$ $d - 1$-form. Then:
\begin{enumerate}
\item The local comass $\Comass(F, \cdot)$ is upper semicontinuous. \label{crandall usc}
\item For almost every $x \in M$, \label{crandall LDT}
$$|F(x)| \leq \Comass(F, x).$$
\item The local comass is bounded, and \label{crandall linfinity}
$$\Comass(F) = \sup_{x \in M} \Comass(F, x) = \|F\|_{L^\infty}.$$
\item If $\sigma \in \Chain_{d - 1}(M)$ then \label{crandall best curl is ABC}
$$\int_\sigma F \leq \Mass(\sigma) \sup_{x \in \sigma} \Comass(F, x).$$
\end{enumerate}
\end{proposition}
\begin{proof}
We first prove (\ref{crandall usc}).
Let $x^n \to x$ and $r > 0$. Then eventually $x^n \in B_r(x)$, hence $\Comass(F, x^n) \leq \Comass_{B_r(x)}(F)$ and so
\begin{align*}
\limsup_{n \to \infty} \Comass(F, x^n) &\leq \inf_{r > 0} \Comass_{B_r(x)}(F) = \Comass(F, x).
\end{align*}

We now prove (\ref{crandall LDT}).
We may work locally, and choose coordinates $(y^i)$ in which $\sqrt{\det g} = 1$.
Let $I$ be the increasing $d-1$-index with $d$ removed.
By the Lebesgue differentiation theorem and Fubini's theorem, there exists a null set $Z \subset M$, which does not depend on $(y^i)$ by \cite[Proposition 2.1]{BackusFLG}, such that for every $x \notin Z$,
\begin{align*}
F_I(x) 
&= \lim_{\varepsilon \to 0} \frac{1}{\Mass(B_\varepsilon(x))} \int_{B_\varepsilon(x)} F_I(y) \dif y \\
&= \lim_{\varepsilon \to 0} \frac{1}{\Mass(B_\varepsilon(x))} \int_{-\infty}^\infty \int_{\{y^d = t\} \cap B_\varepsilon(x)} F_I(y) \dif y^1 \wedge \cdots \wedge \dif y^{d - 1} \wedge \dif t
\end{align*}
where we used the fact that $\sqrt{\det g} = 1$.
Now $\partial_{y^1} \wedge \cdots \wedge \partial_{y^{d - 1}}$ is the oriented unit $d - 1$-blade tangent to $\{y^d = t\}$, so as forms on $\{y^d = t\}$,
$$F_I(y) \dif y^1 \wedge \cdots \wedge \dif y^{d - 1} = F.$$
So
\begin{align*}
F_I(x) 
&= \lim_{\varepsilon \to 0} \frac{1}{\Mass(B_\varepsilon(x))} \int_{-\infty}^\infty \int_{\{y^d = t\} \cap B_\varepsilon(x)} F \dif t \\
&\leq \lim_{\varepsilon \to 0} \frac{\Comass_{B_\varepsilon(x)}(F)}{\Mass(B_\varepsilon(x))} \int_{-\infty}^\infty |\{y^d = t\} \cap B_\varepsilon(x)| \dif t.
\end{align*}
By Fubini's theorem,
$$F_I(x) \leq \lim_{\varepsilon \to 0} \frac{\Comass_{B_\varepsilon(x)}(F)}{\Mass(B_\varepsilon(x))} \Mass(B_\varepsilon(x)) = \Comass(F, x).$$
For every $x \in M$ we may select coordinates in which $|F(x)| = F_I(x)$, and then if $x \notin Z$, we conclude that (\ref{crandall LDT}) holds for $x$.

If we combine (\ref{crandall LDT}) with (\ref{integral over chain is linfinity}), then
$$\sup_{x \in M} \Comass(F, x) \leq \Comass(F) \leq \|F\|_{L^\infty} \leq \sup_{x \in M} \Comass(F, x).$$
The inequalities collapse, proving (\ref{crandall linfinity}).
In particular, for each $\sigma \in \Chain_{d - 1}(M)$, we obtain (\ref{crandall best curl is ABC}):
\begin{align*}
\int_\sigma F &\leq \Mass(\sigma) \inf_{\Omega \supset \sigma} \sup_{x \in \Omega} \Comass(F, x) = \Mass(\sigma) \sup_{x \in \sigma} \Comass(F, x). \qedhere
\end{align*}
\end{proof}

\begin{definition}
Let $F$ be a closed $L^\infty$ $d - 1$-form.
The \dfn{maximum comass locus} is the set
$$\MCL(F) := \{x \in M: \Comass(F, x) = \Comass(F)\}.$$
\end{definition}

\begin{corollary}
Suppose that $M$ is a closed manifold and $F$ is a form of best comass.
Then $\MCL(F)$ is a nonempty compact set.
\end{corollary}
\begin{proof}
This is immediate from Proposition \ref{crandall}(\ref{crandall usc}) and the extreme value theorem for upper semicontinuous functions.
\end{proof}

%%%%%%%%%%%%%%%%%%%%%%%
\subsection{\texorpdfstring{$L^\infty$}{L-infinity} calibrations}
It follows from the definitions that if a smooth hypersurface $N$ is $F$-calibrated, then the trace of $F$ along $N$ is the area form $\dif S_N$.
In particular, $F|_N$ is smooth and $N$ is oriented.
If $\lambda$ is an $F$-calibrated lamination, then the trace $F|_\lambda$ is defined to be $F|_N$ along any leaf $N$.
A priori, $\lambda$ could fail to be orientable, and then $F|_\lambda$ would be necessarily discontinuous.
However, we assert that a calibrated lamination is orientable:

\begin{proposition}\label{calibrated implies oriented}
Let $F$ be a calibration and $\lambda$ an $F$-calibrated lamination.
Then $F|_\lambda$ is continuous, and $F$ induces an orientation on $\lambda$.
\end{proposition}
\begin{proof}
Let $N$ be a leaf of $\lambda$ and $x \in N$.
Let $\mathscr O$ be the local orientation of $\lambda$ near $x$ which is compatible with $F(x)$. 
We define a $d - 1$-form $G$ by declaring that for $y \in K$ close enough to $x$, $K$ a leaf of $\lambda$, $G(y) = \dif S_K(y)$ is the area form of $K$ with respect to $\mathscr O$.
Then for any $y \in \supp \lambda$ close to $x$, either $F(y) = G(y)$ or $F(y) = -G(y)$; we claim that $F(y) = G(y)$ if $\dist(x, y)$ is small enough.
If this claim is true, then the proposition follows, since $G$ is continuous at $x$.

To prove the claim, we suppose towards contradiction that there is a sequence $(x_n) \subset \supp \lambda$ with $x_n \to x$ and $F(x_n) = -G(x_n)$.
We may write $F = \dif A$ where $A$ is continuous near $x$ by Lemma \ref{Hodge theorem} and the Sobolev embedding theorem.
Let $N_n$ be the leaf of $\lambda$ containing $x_n$; then $N_n$ is minimal, hence smooth, so $F_n|_{N_n}$ is smooth.
Therefore by Stokes' theorem, if $(D_{n, m})$ is a sequence of disks in $N_n$ which shrink down to $x_n$ as $m \to \infty$ and are equipped with the orientation $\mathscr O$,
$$-1 = \lim_{m \to \infty} \frac{1}{\Mass(D_{n, m})} \int_{D_{n, m}} F = \lim_{m \to \infty} \frac{1}{\Mass(D_{n, m})} \int_{\partial D_{n, m}} A.$$
In particular, we can choose $m_n \to \infty$ such that 
\begin{equation}\label{orientation contradiction}
\int_{\partial D_{n, m_n}} A \leq 0.
\end{equation}
We set $D_n := D_{n, m_n}$.

Let $(k, z) \in \RR \times \RR^{d - 1}$ be coordinates near $x$ with respect to a flow box, where $k$ indexes the leaves and $z$ is a parameter on each leaf.
Then $D_n = \{k_n\} \times \Omega_n$ for some $\Omega_n \subset \RR^{d - 1}$.
Let $k$ be the index of $N$ and $D_n' := \{k\} \times \Omega_n$.
Since $A$ is continuous, $A(k, z) = A(k_n, z) + o(1)$ as $n \to \infty$, hence
\begin{equation}\label{orientation contradiction 2}
\frac{1}{\Mass(D_n')} \int_{\partial D_n'} A = \frac{1}{\Mass(D_n)} \int_{\partial D_n} A + o(1).
\end{equation}
However, $D_n'$ shrinks down to $x$, and $F(x) = G(x)$, so by Stokes' theorem, for $n$ large,
$$\int_{\partial D_n'} A \gtrsim \Mass(D_n'),$$
hence by (\ref{orientation contradiction}) and (\ref{orientation contradiction 2}),
$$0 < \Mass(D_n) \lesssim \int_{\partial D_n} A \leq 0,$$
a contradiction.
\end{proof}

We next give a natural condition for a lamination to be calibrated.
To make sense of it, observe that if $M$ is closed, $\lambda$ is a lamination, and $F$ is a calibration, then the quantity $\int_M T_\lambda \wedge F$ is well-defined, since it is just the pairing $\langle [F], [\lambda]\rangle$ of cohomology with homology.

\begin{proposition}\label{calibration condition}
Let $F$ be a calibration on a closed Riemannian manifold $M$.
Let $T_\lambda$ be the Ruelle-Sullivan current of a measured oriented lamination $\lambda$.
Then the following are equivalent:
\begin{enumerate}
\item One has \begin{equation}\label{calibration by Ruelle Sullivan}
\int_M T_\lambda \wedge F = \Mass(\lambda).
\end{equation}
\item $\lambda$ is $F$-calibrated.
\end{enumerate}
\end{proposition}
\begin{proof}
First suppose that (\ref{calibration by Ruelle Sullivan}) holds.
Let $(\chi_\alpha)$ be a locally finite partition of unity subordinate to an open cover $(U_\alpha)$ of flow boxes for $\lambda$, and let $(\mu_\alpha)$ be the transverse measure.
After refining $(U_\alpha)$ we may assume that $U_\alpha$ is contained in an open set which is bi-Lipschitz diffeomorphic to $\Ball^d$. After shrinking $U_\alpha$ we may assume that $\chi_\alpha > 0$ on $U_\alpha$.
Then for some hypersurfaces $\sigma_{\alpha,k}$,
$$\Mass(\lambda) = \int_M T_\lambda \wedge F = \sum_\alpha \int_I \int_{\sigma_{\alpha,k}} \chi_\alpha F \dif \mu_\alpha(k).$$
Let $\dif S_{\alpha,k}$ be the surface measure on $\sigma_{\alpha,k}$. Then
$$\int_M \chi_\alpha \star |T_\lambda| = \int_I \int_{\sigma_{\alpha,k}} \chi_\alpha \dif S_{\alpha,k} \dif \mu_\alpha(k),$$
so summing in $\alpha$, we obtain 
\begin{equation}\label{calibration condition contr}
\sum_\alpha \int_I \int_{\sigma_{\alpha,k}} \chi_\alpha F \dif \mu_\alpha(k) = \Mass(\lambda) = \sum_\alpha \int_I \int_{\sigma_{\alpha,k}} \chi_\alpha \dif S_{\alpha,k} \dif \mu_\alpha(k).
\end{equation}

We claim that $\lambda$ is \dfn{almost calibrated} in the sense that for every $\alpha$ and $\mu_\alpha$-almost every $k$, $\sigma_{\alpha, k}$ is calibrated.
If this is not true, then we may select $\beta$ and $K \subseteq I$ with $\mu_\beta(K) > 0$, such that for every $k \in K$, $\int_{\sigma_{\beta, k}} F < \Mass(\sigma_{\beta, k})$.
Since $0 < \chi_\beta \leq 1$ and $F/\dif S_{\beta, k} \leq 1$ on $\sigma_{\beta, k}$, this is only possible if 
$$\int_{\sigma_{\beta, k}} \chi_\beta F < \int_{\sigma_{\beta, k}} \chi_\beta \dif S_{\beta, k}.$$
Integrating over $K$, and using the fact that in general we have $\int_{\sigma_{\alpha, k}} \chi_\alpha F \leq \int_{\sigma_{\alpha, k}} \chi_\alpha \dif S_{\alpha, k}$, we conclude that 
$$\sum_\alpha \int_I \int_{\sigma_{\alpha, k}} \chi_\alpha F \dif \mu_\alpha(k) < \sum_\alpha \int_I \int_{\sigma_{\alpha, k}} \chi_\alpha \dif S_{\alpha, k} \dif \mu_\alpha(k)$$
which contradicts (\ref{calibration condition contr}).

To upgrade $\lambda$ from an almost calibrated lamination to a calibrated lamination, we first 
given $\sigma_{\alpha, k}$ we choose $k_j$ such that $\sigma_{\alpha, k_j}$ is calibrated and $k_j \to k$.
By Lemma \ref{Hodge theorem}, we can find a $d - 2$-form $A$ with $F = \dif A$ satisfying (\ref{Hodge theorem estimate}), which is then continuous by the Sobolev embedding theorem.
This justifies the following application of Stokes' theorem: 
$$\int_{\sigma_{\alpha, k}} F = \int_{\partial \sigma_{\alpha, k}} A.$$
Since $k_j \to k$, and $A$ is continuous,
\begin{align*}
\Mass(\sigma_{\alpha, k}) &= \lim_{j \to \infty} \Mass(\sigma_{\alpha, k_j}) = \lim_{j \to \infty} \int_{\sigma_{\alpha, k_j}} F = \lim_{j \to \infty} \int_{\partial \sigma_{\alpha, k_j}} A = \int_{\partial \sigma_{\alpha, k}} A = \int_{\sigma_{\alpha, k}} F.
\end{align*}

To establish the converse, suppose that $\lambda$ is $F$-calibrated, and let notation be as above.
Since $\lambda$ is $F$-calibrated, for every $\alpha$ and every $k$, the area form on $\sigma_{\alpha, k}$ is $F$. Therefore
\begin{align*}
\int_M T_\lambda \wedge F &= \sum_\alpha \int_I \int_{\sigma_{\alpha, k}} \chi_\alpha F \dif \mu_\alpha(k) = \Mass(T_\lambda). \qedhere
\end{align*}
\end{proof}

\begin{proposition}\label{properties of calibrated laminations}
Suppose that $M$ is a closed Riemannian manifold, $F$ is a calibration, and $\lambda$ is a measured oriented $F$-calibrated lamination.
Then:
\begin{enumerate}
\item $\lambda$ is minimal and homologically minimizing.
\item If $G$ is a calibration and cohomologous to $F$, then $\lambda$ is $G$-calibrated.
\item $\supp \lambda \subseteq \MCL(F)$.
\end{enumerate}
\end{proposition}
\begin{proof}
Every leaf of $\lambda$ is $F$-calibrated, hence minimal, so $\lambda$ is also minimal.
Since $\lambda$ is oriented by Proposition \ref{calibrated implies oriented}, it has a Ruelle-Sullivan current.
Then, since (\ref{calibration by Ruelle Sullivan}) only depends on the cohomology class of $F$, not $F$ itself, $\lambda$ is $G$-calibrated.
From (\ref{calibration by Ruelle Sullivan}) and the fundamental theorem of calibrated geometry, $\lambda$ is homologically minimizing.

Now let $S := \MCL(F)$, $N$ a leaf of $\lambda$, and suppose that $x \in N \setminus S$.
Since $S$ is closed, there exists $\varepsilon > 0$ such that $B_\varepsilon(x)$ does not meet $S$.
Moreover, $\sigma := N \cap B_\varepsilon(x)$ is a $d-1$-chain in $B_\varepsilon(x)$, so by Proposition \ref{crandall}(\ref{crandall best curl is ABC}),
$$\frac{1}{\Mass(\sigma)} \int_\sigma F \leq \sup_{y \in B_\varepsilon(x)} \Comass(F, y) < \Comass(F) = 1.$$
But then 
$$\int_N F = \int_\sigma F + \int_{N \setminus B_\varepsilon(x)} F < \Mass(\sigma) + \Mass(N \setminus B_\varepsilon(x)) = \Mass(N),$$
so $N$ (hence $\lambda$) is not $F$-calibrated.
\end{proof}

%%%%%%%%%%%%%%%%%%%%%%%%%%%%%
\section{\texorpdfstring{Tight forms and the $1$-Laplacian}{Infinity-tight forms and the one-Laplacian}}\label{tight forms sec}
In this section we study the dual problem of the $1$-Laplacian.
Since the Banach spaces $L^q$, $1 < q \leq 2$, are reflexive, but $L^1$ is not, we first study the dual problem to the $q$-Laplacian, and then take the limit as $q \to 1$.

%%%%%%%%%%%%%%%

\subsection{The convex dual of the \texorpdfstring{$q$-Laplacian}{q-Laplacian}}
Let $M$ be a closed Riemannian manifold with fundamental group $\Gamma$ and universal cover $\tilde M \to M$.
Choose a fundamental domain $M_{\rm fun} \subseteq \tilde M$.
Given a representation $\alpha: \Gamma \to \RR$, choose a smooth $1$-form, which we also call $\alpha$, to represent the cohomology class corresponding to $\alpha$ in $H^1(M, \RR)$ given by (\ref{Poincare Hurcewiz}).

Given a H\"older pair $(p, q)$ with $1 < p < \infty$ (thus $\frac{1}{p} + \frac{1}{q} = 1$),
we are interested in the $q$-Laplace equation
$$\dif^* (|\dif u|^{q - 2} \dif u) = 0$$
for an $\alpha$-equivariant function $u$.
Let us express this problem variationally.

Let $X$ be the space of $W^{1, q}_\loc(\tilde M)$ functions $u$ which are $\Gamma$-equivariant in the sense that for $\gamma \in \Gamma$, $\gamma^* \dif u = \dif u$, and let $Y := L^q(M, \Omega^1)$.
We identify $\hat Y$ with $L^p(M, \Omega^{d - 1})$ using the perfect pairing 
\begin{align*}
	L^p(M, \Omega^{d - 1}) \times Y &\to \RR \\
	(F, \varphi) &\mapsto \int_M \varphi \wedge F.
\end{align*}
Then $\Lambda := (\dif: X \to Y)$ is a bounded linear map, and the $\alpha$-equivariant $q$-Laplacian is the Euler-Lagrange equation of $I \circ \Lambda$, where $I$ is the strictly convex functional such that
$$I(\varphi) := \frac{1}{q} \int_M \star |\varphi|^q$$
if the cohomology class of $\varphi$ is $\alpha$, and $I(\varphi) := +\infty$ otherwise.

\begin{proposition}[convex duality for the $q$-Laplacian]\label{mfmc qLaplacian}
Given a representation $\alpha: \Gamma \to \RR$, and a H\"older pair $(p, q)$ with $1 < p < \infty$, there exists an $\alpha$-equivariant $q$-harmonic function $u: \tilde M \to \RR$, unique modulo constants, and a unique minimizer $F$ of 
$$J_{p, \alpha}(F) := \frac{1}{p} \int_M \star |F|^p - \int_M \alpha \wedge F$$
among all closed $d - 1$-forms on $M$.
Moreover, we have
\begin{equation}\label{strong duality}
	\frac{1}{q} \int_M \star |\dif u|^q + \frac{1}{p} \int_M \star |F|^p + \int_M \dif u \wedge F = 0.
\end{equation}
\end{proposition}
\begin{proof}
Let
$$I_\alpha(\psi) := \frac{1}{q} \int_M \star |\psi + \alpha|^q,$$
defined for exact $L^q$ $1$-forms $\psi$, thus $\widehat{I_\alpha}$ is defined on the space of $L^p$ $d - 1$-forms modulo the kernel of the map
$$F \mapsto \left(\psi \mapsto \int_M \psi \wedge F\right)$$
and we lift it to $L^p(M, \Omega^{d - 1})$.

Let $v$ be a primitive of $\alpha$.
Then an $\alpha$-equivariant $u$ minimizes $I$ iff $u - v$ minimizes $I_\alpha \circ \Lambda$, which happens iff $u$ is $\alpha$-equivariant $q$-harmonic.
Since $I_\alpha(\psi) = I_0(\psi + \alpha)$, we can apply \cite[Chapter I, Remark 4.1]{Ekeland99} to see that $I_\alpha$ and $\widehat{I_\alpha}$ are strictly convex and
$$\widehat{I_\alpha}(F) = \widehat{I_0}(F) - \int_M \alpha \wedge F = \frac{1}{p} \int_M \star |F|^p - \int_M \alpha \wedge F.$$
Since $\coker \Lambda$ is the space of closed $L^p$ $d - 1$-forms on $M$, for $F \in \coker \Lambda$, $\widehat{I_\alpha}(F)$ does not depend on the choice of representatives $\alpha, v$, or of the lift of $\widehat{I_\alpha}$ to $L^p(M, \Omega^{d - 1})$.

Finally observe that $\ker \Lambda$ is the space of constants, and for any $\alpha$-equivariant $u$ and closed $F$,
\begin{align*}
I(\Lambda u) + \widehat{I_\alpha}(-F)
&= \frac{1}{q} \int_M \star |\dif u|^q + \frac{1}{p} \int_M \star |F|^p + \int_M \alpha \wedge F \\
&= \frac{1}{q} \int_M \star |\dif u|^q + \frac{1}{p} \int_M \star |F|^p + \int_M \dif u \wedge F.
\end{align*}
All of the assertions of this proposition now follow from Theorem \ref{abstract convex analysis}.
\end{proof}

Let $u$ be $\alpha$-equivariant $q$-harmonic.
Motivated by \cite[\S3.1]{daskalopoulos2020transverse}, it is natural to guess that 
\begin{equation}\label{dual solution}
F := - |\dif u|^{q - 2} \star \dif u
\end{equation}
is the solution of the dual problem of minimizing $J_{p, \alpha}$.
In order to prove that this is true, we shall need that if $(p, q)$ is a H\"older pair, then
\begin{equation}\label{holder cancellation}
	(p - 2)(q - 1) + (q - 2) = 0.
\end{equation}

\begin{lemma}\label{dual to u is minimizer}
Suppose that $u: \tilde M \to \RR$ is an $\alpha$-equivariant $q$-harmonic function, and suppose that $F$ satisfies (\ref{dual solution}).
Then $F$ is a closed $d - 1$-form, which minimizes $J_{p, \alpha}$ among all closed $d - 1$-forms.
Moreover, $F$ solves the PDE 
\begin{equation}\label{pMaxwell}
\begin{cases}
	\dif F = 0 \\
	\dif^* (|F|^{p - 2} F) = 0.
\end{cases}
\end{equation}
\end{lemma}
\begin{proof}
We first show that $\dif F = 0$.
In fact, 
$$\star \dif F = - \star \dif(|\dif u|^{q - 2} \star \dif u) = \pm \dif^*(|\dif u|^{q - 2} \dif u) = 0.$$
By uniqueness, if (\ref{strong duality}) holds, then $F$ must be the minimizer of $J_{p, \alpha}$.
One can easily compute 
$$|F|^p = |\dif u|^{(q - 1)p} = |\dif u|^q,$$
so by Stokes' theorem and the fact that $\alpha$ is cohomologous to $\dif u$,
\begin{align*}
\frac{1}{q} \int_M \star |\dif u|^q + \frac{1}{p} \int_M \star |F|^p&
= \left[\frac{1}{p} + \frac{1}{q}\right] \int_M \star |\dif u|^q
= \int_M \dif u \wedge |\dif u|^{q - 2} \star \dif u \\
&= -\int_M \alpha \wedge F,
\end{align*}
implying (\ref{strong duality}).
Finally, we use (\ref{holder cancellation}) to prove
\begin{align*}
\dif^*(|F|^{p - 2} F) &= - \dif^*(|\dif u|^{(p - 2)(q - 1)} |\dif u|^{q - 2} \star \dif u) = - \dif^*(\star \dif u) \\
&= \pm \star \dif^2 u = 0. \qedhere 
\end{align*}
\end{proof}

We next scrutinize the PDE (\ref{pMaxwell}).
At least at the heuristic level, one expects that as $p \to \infty$, the solutions of (\ref{pMaxwell}) converge to an absolute minimizer of a suitable $L^\infty$ variational problem; minimizers of such problems have been called \dfn{tight} by Sheffield and Smart \cite{Sheffield12}.
This motivates the below terminology:

\begin{definition}
Let $1 < p < \infty$.
A \dfn{$p$-tight form} is a solution of the PDE (\ref{pMaxwell}).
\end{definition}

\begin{proposition}
Suppose that $M$ is a closed oriented Riemannian manifold.
Then there is a unique $p$-tight form in each cohomology class in $H^{d - 1}(M, \RR)$.
Moreover, $p$-tight forms are minimizers of the strictly convex functional
$$J_p(F) := \frac{1}{p} \int_M \star |F|^p$$
among all forms cohomologous to them.
\end{proposition}
\begin{proof}
Strict convexity of $J_p$ on closed $L^p$ $d - 1$-forms is straightforward; since each cohomology class is an affine subspace of $L^p(M, \Omega^{d - 1})$, and hence is convex, the strict convexity on each class follows.
The existence and uniqueness of a minimizer is a standard consequence of strict convexity \cite[Chapter II]{Ekeland99}.
To compute the Euler-Lagrange equations for $J_p$, let $B$ be a $d-2$-form (so $F + t \dif B$ is cohomologous to $F$), so that for a minimizer $F$ of $J_p$,
$$\frac{\dif}{\dif t} J_p(F + t \dif B) = \frac{1}{p} \int_M \star \frac{\partial}{\partial t} |F + t \dif B|^p = \int_M \star |F + t \dif B|^{p - 2} \langle F + t \dif B, \dif B\rangle.$$
Setting $t = 0$, we obtain 
$$0 = \int_M \star |F|^{p - 2} \langle F, \dif B\rangle = \int_M \star \langle \dif^*(|F|^{p - 2} F), B\rangle.$$
Thus the Euler-Lagrange equations for $J_p$ are (\ref{pMaxwell}).
\end{proof}

\begin{definition}
Let $F$ be a $p$-tight form, let
\begin{equation}
\dif u := (-1)^d |F|^{p - 2} \star F, \label{inverse extremality}
\end{equation}
and let $u$ be the primitive of $\dif u$ on the universal cover $\tilde M$, which is normalized to have zero mean on a fundamental domain $M_{\rm fun}$.
Then $u$ is called the \dfn{$q$-harmonic conjugate} of the $p$-tight form $F$, where $\frac{1}{p} + \frac{1}{q} = 1$.
\end{definition}

Let $u$ be the $q$-harmonic conjugate of $F$.
By Poincar\'e's inequality,
$$\|u\|_{W^{1, q}(M_{\rm fun})}^q \lesssim \int_M \star |\dif u|^q = \int_M \star |F|^{(p - 1)q} = \int_M \star |F|^p < \infty$$
since $F$ is $p$-tight; that is, we have $F \in L^p$ and $u \in W^{1, q}_\loc$, justifying any manipulations we shall make with these forms.

\begin{lemma}
Let $1 < p, q < \infty$ and $\frac{1}{p} + \frac{1}{q} = 1$.
Let $F$ be a $p$-tight form, and let $u$ be its $q$-harmonic conjugate.
Then $u$ is $q$-harmonic, $F$ satisfies (\ref{dual solution}), and we have strong duality (\ref{strong duality}).
\end{lemma}
\begin{proof}
We first use (\ref{holder cancellation}) to prove
$$|\dif u|^{q - 2} \star \dif u = (-1)^d |F|^{(q - 2)(p - 1)} \star \star |F|^{p - 2} F = - |F|^{(q - 2)(p - 1) - (p - 2)} F = - F.$$
Thus we have (\ref{dual solution}), and moreover
$$\dif \star (|\dif u|^{q - 2} \dif u) = - \dif F = 0$$
so that $u$ is $q$-harmonic.
Then by Lemma \ref{dual to u is minimizer}, $F$ is the unique minimizer of $J_{p, [\dif u]}$, which implies (\ref{strong duality}).
\end{proof}

% %%%%%%%%%%%%%%%%%%%
% \subsection{Regularity of the \texorpdfstring{$q$-Laplacian}{q-Laplacian}}
% We now pause to consider two regularity results for the $q$-Laplacian.
% The first gives H\"older regularity but is not uniform in $q$; the second is uniform but only gives Sobolev regularity.
% \todo{Do we ever use this?}

% \begin{lemma}[{\cite[Theorem 2]{DIBENEDETTO1983827}}]\label{q Laplacian Holder regularity}
% Let $u: \tilde M \to \RR$ be an $\alpha$-equivariant $q$-harmonic function.
% Then $\dif u$ is H\"older continuous.
% \end{lemma}

% \begin{corollary}
% Every $p$-tight form is H\"older continuous.
% \end{corollary}
% \begin{proof}
% Let $F$ be $p$-tight and let $u$ be its $q$-harmonic conjugate, so $\dif u$ is H\"older continuous by Lemma \ref{q Laplacian Holder regularity}.
% The claim now follows from (\ref{dual solution}) and the fact that a product of H\"older continuous functions is H\"older continuous.
% \todo{Write this out carefully since $q < 2$}
% \end{proof}

\begin{lemma}
Suppose that $1 < q \leq 2$.
Let $u_q: \tilde M \to \RR$ be an $\alpha$-equivariant $q$-harmonic function.
Then (with constant independent of $q$)
\begin{equation}\label{q Laplacian Sobolev regularity estimate}
\|\dif u_q\|_{L^q} \sim \Mass(\alpha).
\end{equation}
\end{lemma}
\begin{proof}
Without loss of generality, $\int_{M_{\rm fun}} \star u_q = 0$.
Let $e_1, \dots, e_k$ be a basis for $H_{d - 1}(M, \RR)$.
This induces a norm $\|\cdot\|$ on $H_{d - 1}(M, \RR)$ by
$$\left\|\sum_i \beta_i e_i\right\| = \sum_i |\beta_i|,$$
which is comparable to the stable norm $\Mass$ since $H_{d - 1}(M, \RR)$ is finite-dimensional.
We write $\alpha = \sum_i \alpha_i e_i$.
Let $v_i$ be the $e_i$-equivariant harmonic function such that $\int_{M_{\rm fun}} \star v_i = 0$.
Then $u_2 = \sum_i \alpha_i v_i$, so
$$\|\dif u_2\|_{L^2} \leq \sum_i |\alpha_i| \|\dif v_i\|_{L^2} \lesssim \|\alpha\| \sim \Mass(\alpha).$$
Since $\dif u_q$ is a minimizer of the $L^q$ norm, we estimate using H\"older's inequality 
\begin{align*}
\|\dif u_q\|_{L^q} &\leq \|\dif u_2\|_{L^q} \leq |M|^{\frac{1}{q} - \frac{1}{2}} \|\dif u_2\|_{L^2} \lesssim \|\dif u_2\|_{L^2} \lesssim \Mass(\alpha).
\end{align*}
In the other direction, we estimate using H\"older's inequality
\begin{align*}
\Mass(\alpha) &\leq \|\dif u_q\|_{L^1} \leq \|\dif u_q\|_{L^q} |M|^{1/p} \lesssim \|\dif u_q\|_{L^q}. \qedhere 
\end{align*}
\end{proof}


%%%%%%%%%%%%%%%%%%%%%%%
\subsection{\texorpdfstring{Existence of tight forms}{Existence of tight forms}}
We now take the limit $p \to \infty$ to obtain a privileged form of best comass.
To do so, we shall need the $p$-tight forms to be uniformly bounded in the following sense.

\begin{lemma}
Let $F_p$ be a $p$-tight form, and let $B$ range over closed $d - 1$-forms cohomologous to $F_p$. Then
\begin{equation}\label{infinity magnetic rules p magnetic}
	\|F_p\|_{L^p} \leq |M|^{1/p} \inf_B \|B\|_{L^\infty}.
\end{equation}
\end{lemma}
\begin{proof}
By H\"older's inequality and the fact that $F_p$ is $p$-tight,
$$\|F_p\|_{L^p} \leq \|B\|_{L^p} \leq |M|^{1/p} \|B\|_{L^\infty},$$
hence the same holds for the infimum.
\end{proof}

\begin{proposition}\label{existence infinity}
Let $\rho \in H^{d - 1}(M, \RR)$.
For each $p \geq 2$, let $F_p$ be the $p$-tight form representing $\rho$. Then there exists a closed $d - 1$-form $F$ such that:
\begin{enumerate}
\item $F_p \to F$ weakly in $L^r$ along a subsequence, for any $d < r < \infty$.
\item $F$ is a best comass representative of $\rho$.
\end{enumerate}
\end{proposition}
\begin{proof}
We roughly follow \cite[\S3]{Lindqvist14}.
Let $r > d$, and let $B$ be an $L^\infty$ representative of $\rho$.
By H\"older's inequality and (\ref{infinity magnetic rules p magnetic}),
\begin{equation}\label{uniform bounds in p by best curl}
	\|F_p\|_{L^r} \leq |M|^{\frac{1}{r} - \frac{1}{p}} \|F_p\|_{L^p} \leq |M|^{\frac{1}{r}} \|B\|_{L^\infty}.
\end{equation}
Thus a compactness argument gives $F_p \to F$ for some $d - 1$-form $F$, weakly in $L^r$, and 
$$\|F\|_{L^r} \leq \liminf_{p \to \infty} \|F_p\|_{L^r} \leq |M|^{\frac{1}{r}} \|B\|_{L^\infty}.$$
Diagonalizing, we may assume that $F_p \to F$ weakly in $L^r$ for every such $r$, and taking $r \to \infty$, we conclude 
\begin{equation}\label{infinity magnetics have best curl}
	\|F\|_{L^\infty} \leq \|B\|_{L^\infty}.
\end{equation}
Moreover, $[F] = \lim_{p \to \infty} [F_p] = \rho$.
Since $B$ was arbitrary in (\ref{infinity magnetics have best curl}), $F$ has best comass.
\end{proof}

\begin{definition}
The $d - 1$-form $F$ of best comass in Proposition \ref{existence infinity} is called a \dfn{tight form}.
\end{definition}

Our next corollary follows immediately from Proposition \ref{existence infinity}.
It can also be proven using Alaoglu's theorem, but we record it because it is extremely useful.

\begin{corollary}
Every cohomology class has a best comass representative.
\end{corollary}

The existence of best comass representatives of each cohomology class $\rho$ implies the following useful lemma on the costable norm of $\rho$.

\begin{lemma}\label{p tights approximate L}
Let $F_p$ be the $p$-tight representative of $\rho$. Then 
$$\lim_{p \to \infty} \|F_p\|_{L^p} = \Comass(\rho).$$
\end{lemma}
\begin{proof}
We follow \cite[Lemma 2.7]{daskalopoulos2020transverse}.
Let $F$ be a best comass representative of $\rho$, so $\|F\|_{L^\infty} = \Comass(\rho)$.
Since $F_p$ is $p$-tight, H\"older's inequality implies 
$$\|F_p\|_{L^p} \leq \|F\|_{L^p} \leq |M|^{\frac{1}{p}} \Comass(\rho).$$
Therefore 
$$\limsup_{p \to \infty} \|F_p\|_{L^p} \leq \Comass(\rho).$$
To prove the converse, suppose that for some $\varepsilon > 0$,
$$\liminf_{p \to \infty} \|F_p\|_{L^p} \leq \Comass(\rho) - \varepsilon < \Comass(\rho).$$
Along a subsequence which attains the limit inferior, $F_p$ converges weakly in every $L^r$, $d < r < \infty$, to a tight form $\tilde F$ such that (by H\"older's inequality)
$$\|\tilde F\|_{L^r} \leq \liminf_{p \to \infty} \|F_p\|_{L^r} \leq \liminf_{p \to \infty} |M|^{\frac{1}{r}} \|\tilde F\|_{L^\infty} \leq |M|^{\frac{1}{r}} (\Comass(\rho) - \varepsilon).$$
Taking $r \to \infty$, we obtain $\Comass(\tilde F) < \Comass(\rho)$, which contradicts the definition of the costable norm $\Comass(\rho)$.
\end{proof}


%%%%%%%%%%%%%%%%%%%%
\subsection{\texorpdfstring{$1$-harmonic conjugates of tight forms}{One-harmonic conjugates of tight forms}}
We now construct the $1$-harmonic conjugates of a tight form.
In the special case that the tight form $F$ is a calibration, that is $\Comass(F) = 1$, a $1$-harmonic conjugate will be a $1$-harmonic function on the universal cover whose level sets are calibrated by $F$.

\begin{definition}
Let $F$ be a tight form of cohomology class $\rho$.
A nonconstant $\Gamma$-equivariant function of least gradient $u \in BV_\loc(\tilde M)$ is called a \dfn{$1$-harmonic conjugate} of $F$ if
\begin{equation}\label{1 extremality}
\dif u \wedge F = \Comass(\rho) \star |\dif u|.
\end{equation}
\end{definition}

We begin by showing that $L^1$ convergence preserves the equivariance properties of functions.

\begin{lemma}\label{L1 convergence preserves pi1}
Let $\tilde M \to M$ be the universal cover, and let $(u_q)$ be a sequence of $\Gamma$-equivariant functions on $\tilde M$ which converge in $L^1_\loc(\tilde M)$ to a function $u$ as $q \to 1$.
Then $u$ is $\Gamma$-equivariant, and $[u_q] \to [u]$.
Moreover, if $\dif u_q \to \dif u$ in the weak topology of measures on $M$ and $\dif u_q \in L^q$, then
\begin{equation}\label{q to 1 Holder}
\Mass(\dif u) \leq \liminf_{q \to 1} \frac{1}{q} \int_M \star |\dif u_q|^q.
\end{equation}
\end{lemma}
\begin{proof}
Since $u_q$ is $\Gamma$-equivariant, there exists $\alpha_q \in H^1(M, \RR)$ such that for every $\gamma \in \pi_1(M)$,
\begin{equation}\label{equivariance q}
	\gamma^* u_q = u_q + \langle \alpha_q, \gamma\rangle.
\end{equation}
Let $M_{\rm fun}$ be a fundamental domain and $U_\gamma := M_{\rm fun} \cup \gamma_* (M_{\rm fun})$.

We claim that $(\alpha_q)$ has a convergent subsequence.
To see this, we first recall that $M$ has finite Betti numbers, so $H^1(M, \RR)$ is locally compact.
Therefore, if no convergent subsequence exists, there exists a $\gamma \in \pi_1(M)$ and a subsequence along which $\langle \alpha_q, \gamma\rangle \to \infty$.
Moreover, since $u_q \to u$ in $L^1_\loc$, $\|u_q\|_{L^1(M_{\rm fun})} \leq 2\|u\|_{L^1(M_{\rm fun})}$ if $q - 1$ is small enough.
But then 
$$\|u_q\|_{L^1(\gamma_* M_{\rm fun})} = \|\gamma^* u_q\|_{L^1(M_{\rm fun})} \geq \langle \alpha_q, \gamma\rangle - \|u_q\|_{L^1(M_{\rm fun})} \geq \langle \alpha_q, \gamma\rangle - 2\|u\|_{L^1(M_{\rm fun})}$$
and taking $q \to 1$ we conclude that $(u_q)$ is not compact in $L^1(\gamma_* M_{\rm fun})$, contradicting the convergence in $L^1_\loc(\tilde M)$.
So $\alpha_q \to \alpha$ for some $\alpha \in H^1(M, \RR)$ along a subsequence.

For any $q > 1$,
\begin{align*}
\dashint_{M_{\rm fun}} \star |\gamma^* u - u - \langle \alpha, \gamma\rangle| 
&\leq \dashint_{M_{\rm fun}} \star (|\gamma^* u_q - u_q - \langle \alpha_q, \gamma\rangle| + |\gamma^* u_q - u_q| + |\gamma^* u - u|) \\
&\qquad + |\langle \alpha_q - \alpha, \gamma\rangle|.
\end{align*}
Taking $q \to 1$ and applying (\ref{equivariance q}), we conclude that $\|\gamma^* u - u - \langle \alpha, \gamma\rangle\|_{L^1} = 0$, hence $u$ is $\alpha$-equivariant.
Thus $\alpha$ is uniquely defined and $\alpha_q \to \alpha$ along the entire subsequence.

Finally we prove (\ref{q to 1 Holder}).
Suppose that $\dif u_q \to \dif u$ in the weak topology of measures and $\dif u_q$ in $L^q$.
Then
$$\|\dif u_q\|_{L^1} = \Mass(\dif u_q).$$
So we may use the portmanteau theorem and H\"older's inequality to estimate (where $\frac{1}{p} + \frac{1}{q} = 1$)
\begin{align*}
\Mass(\dif u) &= \lim_{q \to 1} \Mass(\dif u_q) \leq \lim_{q \to 1} |M|^{\frac{1}{p}} \|\dif u_q\|_{L^q} = \lim_{q \to 1} \frac{1}{q} \int_M \star |\dif u_q|^q. \qedhere
\end{align*}
\end{proof}

The duality relation (\ref{inverse extremality}) blows up $p \to \infty$.
We now ``renormalize'' away the divergence of the $q$-harmonic conjugates of $p$-tight forms before taking the limit $q \to 1$, as in \cite[\S3.2]{daskalopoulos2020transverse}.
Suppose that $\rho \in H^{d - 1}(M, \RR)$, and let $k_p$ be defined by 
$$k_p^{1 - p} = \int_M \star |F_p|^p$$
where $F_p$ is the $p$-tight representative of $\rho$.

\begin{lemma}\label{normalizations converge}
As $p \to \infty$, $k_p \to \Comass(\rho)^{-1}$.
\end{lemma}
\begin{proof}
We follow \cite[Lemma 3.4]{daskalopoulos2020transverse}.
By Lemma \ref{p tights approximate L},
$$\lim_{p \to \infty} k_p^{-\frac{1}{q}} = \lim_{p \to \infty} \|F_p\|_{L^p} = \Comass(\rho).$$
Taking logarithms we see that $q^{-1} \log k_p \to -\log \Comass(\rho)$, and since $q \to 1$ the claim follows.
\end{proof}

\begin{proposition}\label{existence 1}
Let $\rho \in H^{d - 1}(M, \RR)$ be nonzero, and let $F$ be its tight representative.
For each H\"older pair $(p, q)$ with $d < p < \infty$, let $F_p$ be the $p$-tight representative of $\rho$, and let $u_q$ be the function on $\tilde M$ with mean zero on $M_{\rm fun}$ and
$$\dif u_q = (-1)^{d - 1} k_p^{p - 1} |F_p|^{p - 2} \star F_p.$$
Then there exists a $1$-harmonic conjugate $u$ of $F$ such that as $q \to 1$ along a subsequence, $u_q \to u$ weakly in $BV_\loc(\tilde M)$ and strongly in $L^r_\loc(\tilde M)$ for $1 \leq r < \frac{d}{d - 1}$.
\end{proposition}
\begin{proof}
Let $L := \Comass(\rho)$.
We first compute using H\"older's inequality and Lemma \ref{normalizations converge}
\begin{align}
\lim_{q \to 1} \|\dif u_q\|_{L^1}
&\leq \lim_{q \to 1} |M|^{\frac{1}{p}} \left[\int_M \star |\dif u_q|^q\right]^{\frac{1}{q}} = \lim_{p \to \infty} \left[k_p^p \int_M \star |F_p|^p\right]^{\frac{1}{q}} \label{Rellich 1}\\
&= \lim_{p \to \infty} k_p^{\frac{1}{q}} = \lim_{p \to \infty} k_p = \frac{1}{L} \label{Rellich 2}.
\end{align}
So by Rellich's theorem, $(u_q)$ is weakly compact in $BV$ and strongly compact in $L^r$ for $1 \leq r < \frac{d}{d - 1}$.
In particular, $\dif u_q \to \dif u$ in the weak topology of measures and $u_q \to u$ weakly in $BV$ and strongly in $L^r$.
As the limit of $\Gamma$-equivariant functions, $u$ is also $\Gamma$-equivariant by Lemma \ref{L1 convergence preserves pi1}.
In particular, $\dif u$ drops to a current on $M$.
Moreover, $[\dif u_q] \to [\dif u]$, and we have the bound (\ref{q to 1 Holder}) on $\int \star |\dif u|$.

We next must check that $u$ is nonconstant.
If $u$ is constant, then it is $\Gamma$-invariant, so $[\dif u_q] \to 0$.
By (\ref{q Laplacian Sobolev regularity estimate}), $\|\dif u_q\|_{L^q} \to 0$, so by (\ref{Rellich 1}, \ref{Rellich 2}), $L = \infty$, which is absurd.
Therefore $u$ is nonconstant.

Renormalizing (\ref{strong duality}), we obtain 
$$\frac{k_p^{-p}}{q} \int_M \star |\dif u_q|^q + \frac{1}{p} \int_M \star |F_p|^p = k_p^{1 - p} \int_M \dif u_q \wedge F_p.$$
Multiplying by $k_p^p$, we have 
\begin{equation}\label{1 strong duality before limits}
	\frac{1}{q} \int_M \star |\dif u_q|^q + \frac{k_p^p}{p} \int_M \star |F_p|^p = k_p \int_M \dif u_q \wedge F_p.
\end{equation}

Let $\mu(U) := \Mass_U(\dif u)$ be the total variation measure of $\dif u$.
We claim that
\begin{equation}\label{1 strong duality}
	L\mu(M) \leq \int_M \dif u \wedge F.
\end{equation}
First, we have from (\ref{q to 1 Holder}) and (\ref{1 strong duality before limits}) that
$$\mu(M) \leq \lim_{q \to 1} \frac{1}{q} \int_M \star |\dif u_q|^q = \lim_{p \to \infty} k_p \int_M \dif u_q \wedge F_p - \lim_{p \to \infty} \frac{k_p^p}{p} \int_M \star |F_p|^p.$$
By Lemma \ref{normalizations converge},
$$\lim_{p \to \infty} \frac{k_p^p}{p} \int_M \star |F_p|^p = \lim_{p \to \infty} \frac{k_p}{p} = \frac{0}{L} = 0,$$
and
$$\lim_{p \to \infty} k_p \int_M \dif u_q \wedge F_p = \frac{1}{L} \lim_{p \to \infty} \int_M [\dif u_q] \wedge \rho.$$
Since $[\dif u_q] \to [\dif u]$, we obtain
$$\lim_{p \to \infty} \int_M [\dif u_q] \wedge \rho = \int_M \alpha \wedge \rho = \int_M \dif u \wedge F,$$
completing the proof of (\ref{1 strong duality}).

By the coarea formula (\ref{coarea formula}), we have for any open set $U$,
$$\int_U \dif u \wedge F = \int_{-\infty}^\infty \int_{U \cap \partial \{u > y\}} F \dif y \leq L \int_{-\infty}^\infty |U \cap \partial \{u > y\}| \dif y = L \mu(U).$$
Since $\mu$ is a Radon measure and $M$ is compact, every Borel set $E$ can be $\mu$-approximated from without by open sets, hence
\begin{equation}\label{one sided extremality}
\int_E \dif u \wedge F \leq L \mu(E).
\end{equation}

Next we deduce (\ref{1 extremality}).
We reason by contradiction: if (\ref{1 extremality}) is false, then there exists an open set $U \subseteq M$ such that 
$$\int_U \dif u \wedge F < L \int_U \star |\dif u|.$$
(Indeed, strict inequality cannot point in the other direction, by (\ref{one sided extremality}).)
However, by (\ref{one sided extremality}), 
$$\int_{M \setminus U} \dif u \wedge F \leq L \int_{M \setminus U} \star |\dif u|.$$
Adding up the integrals of $\dif u \wedge F$ over $U$ and $M \setminus U$, we conclude 
$$\int_M \dif u \wedge F < L \int_M \star |\dif u|,$$
but this contradicts (\ref{1 strong duality}); thus (\ref{1 extremality}) must be true.
In particular, $F/L$ satisfies (\ref{local calibration}), so $u$ has least gradient by Theorem \ref{MazonRossi}.
\end{proof}




%%%%%%%%%%%%%%%%%%%%


\section{The maximum comass locus of a best comass form}\label{MCL sec}
Let $M$ be a closed Riemannian of dimension $d \leq 7$ equipped with a cohomology class $\rho \in H^{d - 1}(M, \RR)$.
To ease notation, we normalize the costable norm:
$$\Comass(\rho) = 1.$$
We shall study the set $\MCL(F)$ on which a best comass representative $F$ of $\rho$ attains its comass.
In the sequel paper \cite{BackusBest2} we shall scrutinize $\MCL(F)$ more carefully, and show that it contains a lamination, which we call the \dfn{canonical lamination} and only depends on $\rho$.
The canonical lamination encodes certain properties of the duality between the stable and costable norms, and so it has the same role as the canonical lamination in Thurston's approach to Teichm\"uller theory \cite{Thurston98}.

In this paper, however, we content ourselves to showing that $\MCL(F)$ contains the measured laminations given by the $1$-harmonic conjugates of the tight representatives of $\rho$.
Since functions of least gradient give homologically minimizing laminations by Theorem \ref{1 harmonic is MOML}, and the $1$-harmonic conjugate of a tight representative has least gradient, the next definition makes sense.

\begin{definition}
Let $F$ be a tight representative of $\rho$, and let $u$ be a $1$-harmonic conjugate of $F$.
Then we call $\kappa_u$ a \dfn{measured stretch lamination} associated to $\rho$.
\end{definition}

\begin{proposition}\label{MCL contains Thurston}
Let $F$ be a best comass representative of $\rho$, and let $\lambda$ be a measured stretch lamination associated to $\rho$.
Then $F$ calibrates $\lambda$. In particular, $\MCL(F) \supseteq \supp \lambda$.
\end{proposition}
\begin{proof}
Let $G$ be the tight form which is cohomologous to $F$ whose dual $1$-harmonic function $u$ defines the measured stretch lamination $\lambda$.
Then by (\ref{1 extremality}), 
$$\Mass(\lambda) = \Mass(\dif u) = \int_M \dif u \wedge G$$
so $G$ calibrates $\lambda$ by Proposition \ref{calibration condition}.
Then by Proposition \ref{properties of calibrated laminations}, $F$ calibrates $\lambda$ and $\MCL(F) \supseteq \supp \lambda$.
\end{proof}

\begin{proposition}\label{L equals K}
	Let $\kappa$ be a measured stretch lamination for $\rho$, and let $\lambda$ range over measured oriented laminations. Then 
	\begin{equation}\label{L equals K formula}
	\sup_\lambda \frac{\langle \rho, [\lambda]\rangle}{\Mass(\lambda)} = \frac{\langle \rho, [\kappa]\rangle}{\Mass(\kappa)} = 1.
	\end{equation}
\end{proposition}
\begin{proof}
Fix a tight form $F$ representing $\rho$, and let $u$ be its $1$-harmonic conjugate.
Let
$$K :=  \sup_\lambda \frac{\langle \rho, [\lambda]\rangle}{|\lambda|}.$$

We first prove $K \leq 1$.
Let $\lambda$ be a measured oriented lamination; then, since $F$ represents $\rho$ and the Ruelle-Sullivan current $T_\lambda$ represents $[\lambda]$,
$$\langle \rho, [\lambda]\rangle = \int_M F \wedge T_\lambda.$$
Let $(\chi_\alpha)$ be a partition of unity subordinate to a laminar atlas for $\lambda$, and let $(\mu_\alpha)$ be the associated transverse measure. Then 
$$\int_M F \wedge T_\lambda = \sum_\alpha \int_I \int_{\{k\} \times J} \chi_\alpha F \dif \mu_\alpha(k).$$
Since $F$ has best comass,
$$\frac{\langle \rho, [\lambda] \rangle}{\Mass(\lambda)}
\leq \frac{\|F\|_{L^\infty}}{\Mass(\lambda)} \sum_\alpha \int_I \int_{\{k\} \times J} \chi_\alpha \dif S_k \dif \mu_\alpha(k) = 1.$$
Since $\lambda$ was arbitrary, it holds that $K \leq 1$.

By (\ref{1 extremality}),
$$\langle \rho, [\kappa]\rangle = \int_M F \wedge \dif u = \Mass(\dif u) = \Mass(\kappa).$$
Dividing both sides by $\Mass(\kappa)$ and applying the direction we already proved,
$$K \leq 1 \leq \frac{\langle \rho, [\kappa]\rangle}{\Mass(\kappa)} \leq K$$
which is only possible if $K = 1$ and $\kappa$ is a maximizer.
\end{proof}

\begin{proposition}\label{calibrated means measured stretch}
Let $F$ be a best comass representative of $\rho$, and suppose that $F$ calibrates a measured oriented lamination $\lambda$.
Then $\lambda$ is a measured stretch lamination associated to $\rho$.
\end{proposition}
\begin{proof}
Let $\dif u$ be the Ruelle-Sullivan current for $\lambda$, and suppose that $f \in C^0(M)$ is supported in a flow box for $\lambda$, with local leaf space $K$ and transverse measure $\mu$.
By Proposition \ref{properties of calibrated laminations}, we may assume wihout loss of generality that $F$ is tight.
Since $F$ calibrates every leaf of $\lambda$,
$$\int_M f \star |\dif u| = \int_K \int_{\{k\} \times J} f \dif S_{\{k\} \times J} \dif \mu(k) = \int_K \int_{\{k\} \times J} fF \dif \mu(k) = \int_M f\dif u \wedge F$$
(in any case, the Radon measure $\dif u \wedge F$ is well-defined by the coarea formula).
Thus $\dif u \wedge F = \star |\dif u|$, or in other words $u$ is a $1$-harmonic conjugate of the tight form $F$.
Therefore $\lambda$ is a measured stretch lamination.
\end{proof}

\begin{corollary}
Let $F$ be a best comass representative of $\rho$, and suppose that $F$ calibrates a measured oriented lamination $\lambda$.
Then $\lambda$ is Lipschitz.
\end{corollary}
\begin{proof}
By Theorem \ref{1 harmonic is MOML}, every measured stretch lamination is Lipschitz.
\end{proof}





%%%%%%%%%%%%%%%%%%%%%%%%%%%%%%%%
\section{The Euler-Lagrange equation for tight forms}\label{infinityMax}
At present there is not a suitable theory of viscosity solutions of PDEs for vector-valued maps, and in general, one does not expect our tight forms to be much more regular than $L^\infty$.
Therefore there is no sense that we are aware of that tight forms should solve an Euler-Lagrange equation.
However, we shall formally derive what analytic properties tight forms \emph{should} have, in the tradition of various other papers \cite{Barron2001,Aronsson67,Sheffield12} on the $L^\infty$ calculus of variations.

\subsection{Formal derivation of Euler-Lagrange equation}
To state our Euler-Lagrange equation we introduce some notation for derivatives of tensor fields along differential forms.
If $\alpha$ is a $k$-form, $\nabla$ is the Levi-Civita connection, and $T$ is a section of a tensor bundle $E$, we introduce the tensor $\nabla^\alpha T$, a section of $E \otimes \Omega^{k - 1}$, defined as follows: if $X_1, \dots, X_{k - 1}$ are vector fields, and
$$Y := (\iota_{X_1} \cdots \iota_{X_{k - 1}} \alpha)^\sharp$$
is the vector field dual to the contraction of $\alpha$, then
$$\langle \nabla^\alpha T, X_1 \otimes \cdots \otimes X_k\rangle := \nabla_Y T.$$
We think of $\nabla^\alpha T$ as a sort of ``weighted projection'' of $\nabla T$ to the subbundle $\ker \star \alpha \subset TM$.

\begin{proposition}
Suppose that $F_p$ are $C^1$ $p$-tight forms converging to a tight form $F$.
Furthermore suppose that as $p \to \infty$, $\|F_p\|_{C^{1 + \alpha}} \lesssim 1$.
Then $F \in C^1$ and 
\begin{equation}\label{infty Max}
\begin{cases}
\dif F = 0, \\
\langle \nabla^F F, F\rangle = 0.
\end{cases}
\end{equation}
\end{proposition}

We should clarify the PDE (\ref{infty Max}): $\nabla^F F$ is a section of $\Omega^{d - 1} \otimes \Omega^{d - 2}$, so its contraction with $F$ is the contraction of the $\Omega^{d - 1}$ part with $F$; thus $\langle \nabla^F F, F\rangle$ is a $d - 2$-form.
If $d = 2$, and $F = \dif u$, then (\ref{infty Max}) is exactly the $\infty$-Laplace equation 
$$\langle\nabla^2 u, \nabla u \otimes \nabla u\rangle = 0.$$

\begin{proof}
We first compute from (\ref{pMaxwell}) that $\dif F = 0$ and
\begin{align*}
0
&= \dif(|F_p|^{p - 2} \star F_p) \\
&= \dif(|F_p|^{p - 2}) \wedge \star F_p + |F_p|^{p - 2} \dif \star F_p \\
&= (p - 2) |F_p|^{p - 4} \langle \nabla F_p, F_p\rangle \wedge \star F_p + |F_p|^{p - 2} \dif \star F_p.
\end{align*}
If $F_p$ is nonzero, then we can divide through by $(p - 2) |F_p|^{p - 4}$ to get
\begin{equation}\label{intermediate p Max}
0 = \langle\nabla F_p, F_p\rangle \wedge \star F_p + \frac{|F_p|^2}{p - 2} \dif \star F_p.
\end{equation}
At the zeroes of $F_p$, we simply observe that (\ref{intermediate p Max}) holds for trivial reasons.

By assumption $||F_p|^2 \dif \star F_p| = o(p)$, so as $p \to \infty$, the second term of (\ref{intermediate p Max}) drops out.
We can take the limit of (\ref{intermediate p Max}) using the equicontinuity of $\nabla F_p$ to get
$$0 = \langle \nabla F, F \rangle \wedge \star F.$$
Taking the Hodge star of both sides, we get (\ref{infty Max}).
\end{proof}

%%%%%%%%%%%%%%%%%%%%%%%%%
\subsection{Geometric and variational interpretations of the Euler-Lagrange equation}
The PDE (\ref{infty Max}) has a simple geometric interpretation, which generalizes the interpretation of the $\infty$-Laplace equation as asserting that the gradient curves of an $\infty$-harmonic function are lines.

\begin{proposition}\label{infty Max calibrates}
Let $F$ be a $C^1$ solution of (\ref{infty Max}) with no zeroes, and let $N$ be a connected integral hypersurface of $\ker \star F$.
Then there exists $\lambda > 0$ such that $N$ is an $F/\lambda$-calibrated hypersurface.
\end{proposition}
\begin{proof}
Let $(X_1, \dots, X_{d - 1})$ be an orthonormal frame of vector fields tangent to $N$.
We then introduce the tensor field
$$T_i := X_1 \otimes \cdots \otimes \widehat{X_i} \otimes \cdots \otimes X_{d - 1},$$
where the hat means to remove that factor.
By definition of $N$, and the fact that $F$ has no zeroes, $F$ is a nonzero scalar field $\lambda$ times $\dif S_N$.
So $\iota_{X_1} \cdots \widehat{\iota_{X_i}} \cdots \iota_{X_{d - 1}} F$ is a nonzero scalar field $u_i$ times $X_i^\flat$.
Applying (\ref{infty Max}), we have 
$$0 = \langle \langle \nabla^F F, F\rangle, T_i\rangle = u_i \langle \nabla_{X_i} F, F \rangle = \frac{u_i}{2} \partial_{X_i} (|F|^2).$$
Since $u_i/2$ is nonzero, we see that $|F|^2$ is constant along integral curves of $X_i$.
Since $(X_1, \dots, X_{d - 1})$ spans the tangent bundle of $N$, we conclude that $|F|^2$ is constant along $N$, or equivalently that $\lambda$ is a constant.
Therefore $\dif S_N = F/\lambda$ is closed, hence $N$ is $F/\lambda$-calibrated.
\end{proof}

We now give a variational criterion for (\ref{infty Max}).
Unfortunately, the converse is weaker than we would like, because in order to apply arguments similar to those of \cite{Aronsson67,Sheffield12} we need to assume that $\ker(\star F)$ is a singular integrable distribution.

\begin{proposition}
Let $F$ be a $C^1$ closed $d - 1$-form, and suppose that for every $V \subseteq M$ such that $H^{d - 1}(V, \RR) = 0$, and every $U \Subset V$,
\begin{equation}\label{ABC inequality}
\Comass_U(F) \leq \|F\|_{C^0(\partial U)}.
\end{equation}
Then (\ref{infty Max}) holds.
\end{proposition}
\begin{proof}
It suffices to prove (\ref{infty Max}) locally, so we can cover $M$ by open balls $V$ with $H^{d - 1}(V, \RR) = 0$ and prove that (\ref{infty Max}) holds in slightly smaller balls.
Since $H^{d - 1}(V, \RR) = 0$, we can find a $C^2$ $d - 2$-form $A$ on a neighborhood of $\overline V$ with $\dif A = F$.
Also for $\xi$ a covariant tensor of valence $d - 1$ at $x$, let $\xi^{\rm as}$ be its antisymmetrization, and
$$f(x, \xi) := |\xi^{\rm as}|_{g^{-1}(x)}^2.$$

We first claim that for any $d - 2$-form $B$, $f(\cdot, B) = |\dif B|^2$.
Since $\nabla$ is torsion-free, antisymmetrization annihilates the Christoffel symbols of $\nabla$, so if $\nabla^\flat$ denotes a flat connection, then $(\nabla B)^{\rm as} = (\nabla^\flat B)^{\rm as}$; the latter is of course $\dif B$.
Thus in particular, $f(\cdot, A) = |F|^2$.

Let $W \Subset V$ be obtained by slightly shrinking $V$.
We claim that $A|_W$ is an absolute minimizer of $f(x, \nabla A(x))$ in the sense of \cite[Definition 5.1]{Barron2001}.
In other words, we claim that for each open $U \subseteq W$ with smooth boundary and each covariant tensor field $B$ of valence $d - 2$, such that $A - B$ has compact support in $U$,\footnote{Strictly speaking, the definition of absolute minimizer ranges over all open sets $U$ ($\Omega'$ in the notation of \cite{Barron2001}), not just those with smooth boundary; similarly one requires the competition class to range over traceless (rather than compactly supported) variations. However, it is trivial to modify the proof of \cite[Theorem 5.2]{Barron2001} to only require smooth domains and compactly supported variations.}
$$\sup_{x \in U} f(x, A(x)) \leq \sup_{x \in U} f(x, B(x)).$$
To see this, let $G = (\nabla B)^{\rm as}$, so that by (\ref{ABC inequality}),
\begin{align*}
\sup_{x \in U} f(x, A(x))
&= \Comass_U(F) \leq \|F\|_{C^0(\partial U)} = \|G\|_{C^0(\partial U)} \leq \|G\|_{C^0(U)} = \sup_{x \in U} f(x, B(x)).
\end{align*}

By the above claims and \cite[Theorem 5.2]{Barron2001}, for each $x \in W$, we have the Euler-Lagrange-Aronsson equation that for any $d - 2$-form $\theta$,
\begin{align*}
0 
&= \left\langle \frac{\partial f}{\partial \xi}(x, \nabla A(x)), \nabla \left[f(x, \nabla A(x))\right] \otimes \theta(x)\right\rangle \\
&= 2\langle (\nabla A(x))^{\rm as}, \nabla(|(\nabla A(x))^{\rm as}|^2) \otimes \theta(x)\rangle \\
&= 2\langle F(x), \nabla(|F(x)|^2) \otimes \theta(x)\rangle.
\end{align*}
Since $\nabla$ is a metric connection, $\nabla(|F|^2) = 2\langle \nabla F, F\rangle$.
Thus we have 
$$0 = 4\langle F, \langle \nabla F, F\rangle \otimes \theta\rangle = 4\langle \langle \nabla^F F, F\rangle, \theta\rangle$$
and since $\theta$ was arbitrary we conclude (\ref{infty Max}).
\end{proof}

\begin{proposition}
Let $F$ be a $C^1$ solution of (\ref{infty Max}) such that $\star F \wedge \dif(\star F) = 0$.
Then for every $x \in M$ and every sufficiently small $r > 0$, (\ref{ABC inequality}) holds for $U = B(x, r)$.
\end{proposition}
\begin{proof}
First observe that if $x \in M$ and $F(x) = 0$, then $|F|$ has a local minimum at $x$, so for any $y$ sufficiently close to $x$, say $y \in B(x, r)$, $|F|$ does not have a local maximum at $y$ (unless $y$ is also a local minimum, hence $F(y) = 0$); therefore (\ref{ABC inequality}) holds.
Henceforth we assume that $F(x) \neq 0$.

Let $r > 0$ be such that $F|_{B(x, r)}$ has no zeroes and for some $s > r$, $H^{d - 1}(B(x, s), \RR) = 0$.
Since $\star F \wedge \dif(\star F) = 0$, $\ker(\star F)$ integrates to a foliation $\mathscr F$ of $B(x, r)$.
By definition of $s$, we may assume that $F = \dif A$ for some $d - 2$-form $A$ defined on $B(x, s)$.
By Proposition \ref{infty Max calibrates}, for each leaf $N$ of $\mathscr F$ there exists $\lambda_N > 0$ such that for each hypersurface $N'$ with $N \cap \partial B(x, r) = N' \cap \partial B(x, r)$,
$$\lambda_N \Mass(N) = \int_N F = \int_{N \cap \partial B(x, r)} A = \int_{N' \cap \partial B(x, r)} A = \int_{N \cap \partial B(x, r)} F \leq \lambda_N \Mass(N'),$$
so $N$ is absolutely area-minimizing in $B(x, r)$ and hence meets $\partial B(x, r)$, say at some point $x_N$.
Moreover, $F|_N$ has constant comass $\lambda_N$; since $\mathscr F$ is a foliation it holds that for each $x \in V$ there exists a leaf $N \ni x$ of $\mathscr F$, and then 
\begin{align*}
|F(x)| &= |F(x_N)| \leq \|F\|_{C^0(\partial B(x, r))}. \qedhere
\end{align*}
\end{proof}

%%%%%%%%%%%%%%%
% \subsection{Parabolic character and higher regularity}
% Similarly to the $\infty$-Laplace equation, \todo{Cite Evans--Smart ``adjoint methods''} we shall view (\ref{infty Max}) as a degenerate parabolic equation.
% To be more precise, suppose that $A$ is a $d - 2$-form in Coulomb gauge such that $F = \dif A$ solves (\ref{infty Max}), hence 
% \begin{equation}\label{integrated infty Max}
% \begin{cases}
% \langle \nabla^{\dif A} \dif A, \dif A\rangle = 0, \\
% \dif^* A = 0.
% \end{cases}
% \end{equation}
% We consider the linearized equation at $A$
% \begin{equation}\label{linearized infty Max}
% \begin{cases}
% \langle \nabla^{\dif B} \dif A, \dif A\rangle + \langle \nabla^{\dif A} \dif B, \dif A\rangle + \langle \nabla^{\dif A} \dif A, \dif B\rangle = 0, \\
% \dif^* B = 0.
% \end{cases}
% \end{equation}
% Suppose that $\dif A$ has no zeroes.
% Let $\partial_{x^1}, \dots, \partial_{x^{d - 1}}$ be an orthonormal frame for $\ker(\star \dif A)$. \todo{Construct $\partial_t$ in terms of $A$. Use a transform to make $A$ smooth}.












% %%%%%%%%%%%%%%%%%%%%%%%%%%
% \section{Open problems}\label{open problems}
% \subsection{More general Riemannian manifolds}
% For simplicity in this paper we have only dealt with closed manifolds of dimension $d \leq 7$.
% However, it is likely that the results largely go through for compact manifolds with boundary, as our arguments are mainly local, and for those that are not one can pass to the double of the manifold with boundary, as long as the boundary is not itself a minimal hypersurface.
% Moreover, one can most likely recover the results of this paper without the dimension assumption, as long as one is willing to deal with the consequence of having minimal hypersurfaces with singularities, since we mainly deal with phenomena in codimension $\leq 2$, while singularities of minimal hypersurfaces necessarily have codimension $\geq 8$. 
% There are genuine differences in the case of manifolds with infinite ends, since if $M$ is a manifold with cusps, the stable (semi)norm of every homology class in $M$ may be zero.
% Still, one expects to be able to recover most of the results of this paper.

% \begin{problem}\label{generalization}
% Formulate the results in this paper for complete Riemannian manifolds with boundary and infinite ends, and arbitrary dimension $d \geq 2$.
% \end{problem}

%%%%%%%%%%%%%%%%%%%%
\subsection{Open problems}
% \subsection{\texorpdfstring{$p$-elliptic systems}{p-elliptic systems}}
We have carefully sidestepped the lack of a theory of $L^\infty$ variational systems by working with $L^p$ variational systems, $d < p < \infty$, which can be written in divergence form, and then taking a limit.
When we do have to work in the limit, we passed to the dual problem, which was a \emph{scalar} $L^1$ variational problem.
One would therefore like to work with a notion of weak solution for the Euler-Lagrange system (\ref{infty Max}).

\begin{problem}
Introduce a notion of weak solution for (\ref{infty Max}) which generalizes the notion of viscosity solution for the $\infty$-Laplacian.
Show that the following are equivalent for a $d - 1$-form $F$:
\begin{enumerate}
\item $F$ is a weak solution of the Euler-Lagrange equation (\ref{infty Max}).
\item $F$ is tight.
\item $F$ is closed and for every small ball $B \subset M$, the variational condition (\ref{ABC inequality}) holds.
\end{enumerate}
\end{problem}

Another problem caused by the lack of the maximum principle is the apparent failure of tight forms to be unique.
This was already a problem in the work of Daskalopolous and Uhlenbeck on maps to $\Sph^1$ which inspired this work \cite[Conjecture 9.2]{daskalopoulos2020transverse}.

\begin{conjecture}
The tight $d - 1$-form in a cohomology class is unique.
\end{conjecture}

We should also point out that if $F_p$ is $p$-tight and we write $F_p = \dif A_p$, where the gauge potential $A_p$ satisfies $\dif^* A_p = 0$, then $A_p$ solves the $L^p$ analogue 
$$\dif^*(|\dif A_p|^{p - 2} \dif A_p) + \dif(|\dif^* A_p|^{p - 2} \dif^* A_p) = 0$$
of the Laplace-de Rham equation. 
There is a rough principle that those estimates on the Laplace equation which do not depend on the maximum principle (or the closely related comparison principle and Harnack inequality) should also hold for the Laplace-de Rham equation.
Therefore one expects that those estimates on the $p$-Laplacian which do not depend on the maximum principle should hold for the $p$-Maxwell equation.
We did not pursue this line of inquiry here, since it is highly unlikely that such estimates would remain valid in the limit $p \to \infty$, but we believe that they should be true, hence:

\begin{problem}
Formulate and prove estimates on the $p$-Maxwell equation analogous to those estimates on the $p$-Laplacian proved in, for example, \cite[Chapter 11]{kinnunen2021maximal}, which do not rely on the maximum principle.
\end{problem}

\todo{Make explicit the Caccioppoli inequality and anything else that's probably true here?}

%%%%%%%%%%
%%%%%%%5


\appendix
\section{Local Hodge theory in \texorpdfstring{$L^p$}{Lp}} \label{local Hodge appendix}
The geometric measure theory that we use throughout this paper relies on the local elliptic regularity of the Hodge system.
Such estimates are standard when $p = 2$ \cite[\S9.5]{taylor2010partial}, but we are mainly interested in the range $d < p < \infty$, where we may apply the Sobolev embedding $W^{1, p} \Subset C^\alpha$.
We thus give a short proof for every $p$.

\begin{lemma}\label{Hodge theorem}
Suppose that there is a bi-Lipschitz diffeomorphism $M \cong \Ball^d$.
Let $1 < p < \infty$, and let $F$ be an $L^p$ closed $\ell + 1$-form.
Then there exists an $\ell$-form $A$ such that $F = \dif A$ and
\begin{equation}\label{Hodge theorem estimate}
\|A\|_{W^{1, p}} \lesssim_p \|F\|_{L^p}.
\end{equation}
\end{lemma}
\begin{proof}
Without loss of generality, $M = (\Ball^d, g)$ where $g$ is a Riemannian metric of bounded geometry.
Then, possibly after rescaling, the $W^{s, p}(M)$ norm is comparable to the $W^{s, p}(\Ball^d)$ norm.
Everything else in the statement of the lemma is a diffeomorphism invariant, so we may assume that $M = \Ball^d$.
We then solve the elliptic system 
\begin{equation}\label{Hodge Laplacian}
\begin{cases}
	\Delta u = F \\
	u|_{\partial \Ball^d} = 0
\end{cases}
\end{equation}
for an $\ell$-form $u$.
Since $M = \Ball^d$, the Hodge Laplacian commutes with taking components, thus the system (\ref{Hodge Laplacian}) decouples into $\binom d\ell$ elliptic PDE
\begin{equation}\label{decoupled Hodge Laplacian}
\begin{cases}
\Delta(u_I) = F_I \\
u_I|_{\partial \Ball^d} = 0,
\end{cases}
\end{equation}
one for each $\ell$-index $I$.
By \cite[Chapter 3, Theorem 6.3]{chen1998second}, we have a unique solution to (\ref{decoupled Hodge Laplacian}), which satisfies
$$\|u_I\|_{W^{2, p}} \lesssim_p \|F_I\|_{L^p}.$$
We then let $A := \dif^* u$; then (\ref{Hodge theorem estimate}) holds, and $\dif A = \dif \dif^* u$.
On euclidean space, $\Delta$ and $\dif$ commute, so $\dif u = 0$, and it holds that 
\begin{align*}
\dif A &= (\dif \dif^* + \dif^* \dif) u = \Delta u = F. \qedhere 
\end{align*}
\end{proof}

\begin{lemma}\label{mollification of closed forms}
Suppose that there is a bi-Lipschitz diffeomorphism $M \cong \Ball^d$.
Let $1 \leq p < \infty$ and let $F$ be an $L^p$ closed $\ell$-form.
Then there exist smooth closed $\ell$-forms $F_n$ such that $F_n \to F$ in $L^p$.
Moreover, if $F \in L^\infty$, then
\begin{equation}\label{heat kernel contracts sup norm}
\limsup_{n \to \infty} \|F_n\|_{C^0} \leq \|F\|_{L^\infty}.
\end{equation}
If $U$ is an open set such that $\supp F \subseteq U$, then for $n$ large enough, $\supp F_n \subseteq U$.
\end{lemma}
\begin{proof}
Without loss of generality, $M = (\Ball^d, g)$ where $g$ is a Riemannian metric of bounded geometry.
Then, possibly after rescaling, the $L^p(M)$ norm is comparable to the $L^p(\Ball^d)$ norm.
If $\varphi$ is am $\ell$-form and $\chi$ is a function on $\RR^d$, we define the convolution $\chi * \varphi$ by $(\chi * \varphi)_I := \chi * \varphi_I$ for every $\ell$-index $I$.
Every convolution operator commutes with $\dif$, so after convolution with a standard mollifier $\chi_n$ as in \cite[Appendix C, Theorem 6]{evans2010partial}, $F_n := (\chi_n * F)|_{\Ball^d}$ is a closed $\ell$-form and $F_n \to F$ in $L^p(\Ball^d)$ and almost everywhere.
Convolution against a standard mollifier cannot increase the support by more than a small neighborhood, so the support property follows.

Now suppose that $F \in L^\infty$.
We may choose $\chi_n$ so that $\chi_n \geq 0$, $\int_{\RR^d} \chi_n = 1$, and $\supp \chi_n \Subset B_{1/n}$, the euclidean ball of radius $1/n$ around $0$.
Then 
$$|F_n(x)|_{g(x)} \leq \left|\int_{B_{1/n}} \chi_n(y) F(x - y) \dif y\right|_{g(x)},$$
where the integral is a vector-valued integral (which makes sense since we may view $F$ as a map into $\RR^{\binom d\ell})$.
By the triangle inequality and a Taylor expansion of $g$ around $g(x)$,
\begin{align*}
\left|\int_{B_{1/n}} \chi_n(y) F(x - y) \dif y\right|_{g(x)}
&\leq \int_{B_{1/n}} \chi_n(y) |F(x - y)|_{g(x)} \dif y \\
&= \int_{B_{1/n}} \chi_n(y) |F(x - y)|_{g(x - y)}(1 + O(y)) \dif y \\
&\leq \|F\|_{L^\infty(M)} \int_{B_{1/n}} \chi_n(y)(1 + O(y)) \dif y \\
&= \|F\|_{L^\infty(M)}(1 + O(n^{-1})). \qedhere 
\end{align*}
\end{proof}


\printbibliography

\end{document}
