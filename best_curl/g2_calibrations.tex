\documentclass[reqno,11pt]{amsart}
\usepackage[letterpaper, margin=1in]{geometry}
\RequirePackage{amsmath,amssymb,amsthm,graphicx,mathrsfs,url,slashed,subcaption}
\RequirePackage[usenames,dvipsnames]{xcolor}
\RequirePackage[colorlinks=true,linkcolor=Red,citecolor=Green]{hyperref}
\RequirePackage{amsxtra}
\usepackage{cancel}
\usepackage{tikz-cd}
%\usepackage[T1]{fontenc}

% \setlength{\textheight}{9.3in} \setlength{\oddsidemargin}{-0.25in}
% \setlength{\evensidemargin}{-0.25in} \setlength{\textwidth}{7in}
% \setlength{\topmargin}{-0.25in} \setlength{\headheight}{0.18in}
% \setlength{\marginparwidth}{1.0in}
% \setlength{\abovedisplayskip}{0.2in}
% \setlength{\belowdisplayskip}{0.2in}
% \setlength{\parskip}{0.05in}
%\renewcommand{\baselinestretch}{1.05}

\title{Variational construction of calibrated currents}
\author{Aidan Backus}
\address{Department of Mathematics, Brown University}
\email{aidan\_backus@brown.edu}
\date{\today}
\keywords{}
\subjclass[2020]{}

\newcommand{\NN}{\mathbf{N}}
\newcommand{\ZZ}{\mathbf{Z}}
\newcommand{\QQ}{\mathbf{Q}}
\newcommand{\RR}{\mathbf{R}}
\newcommand{\CC}{\mathbf{C}}
\newcommand{\DD}{\mathbf{D}}
\newcommand{\PP}{\mathbf P}
\newcommand{\MM}{\mathbf M}
\newcommand{\II}{\mathbf I}
\newcommand{\Hyp}{\mathbf H}
\newcommand{\Sph}{\mathbf S}
\newcommand{\Group}{\mathbf G}
\newcommand{\GL}{\mathbf{GL}}
\newcommand{\Orth}{\mathbf{O}}
\newcommand{\SpOrth}{\mathbf{SO}}
\newcommand{\Ball}{\mathbf{B}}

\newcommand*\dif{\mathop{}\!\mathrm{d}}

\DeclareMathOperator{\card}{card}
\DeclareMathOperator{\dist}{dist}
\DeclareMathOperator{\id}{id}
\DeclareMathOperator{\End}{End}
\DeclareMathOperator{\Hom}{Hom}
\DeclareMathOperator{\coker}{coker}
\DeclareMathOperator{\supp}{supp}
\DeclareMathOperator{\vol}{vol}
\DeclareMathOperator{\tr}{tr}

\newcommand{\Leaves}{\mathscr L}
\newcommand{\Lagrange}{\mathcal L}
\newcommand{\Hypspace}{\mathscr H}

\newcommand{\Chain}{\underline C}

\newcommand{\Two}{\mathrm{I\!I}}

\newcommand{\normal}{\mathbf n}
\newcommand{\radial}{\mathbf r}
\newcommand{\evect}{\mathbf e}

\newcommand{\Bl}{\operatorname{Bl}}
\newcommand{\diam}{\mathrm{diam}}
\newcommand{\Ell}{\mathrm{Ell}}
\newcommand{\inj}{\mathrm{inj}}
\newcommand{\Lip}{\mathrm{Lip}}
\newcommand{\MCL}{\mathrm{MCL}}
\newcommand{\Riem}{\mathrm{Riem}}

\newcommand{\Mass}{\mathbf M}
\newcommand{\Comass}{\mathbf L}

\newcommand{\weakto}{\rightharpoonup}

\newcommand{\Min}{\mathrm{Min}}
\newcommand{\Max}{\mathrm{Max}}

\newcommand{\dfn}[1]{\emph{#1}\index{#1}}

\renewcommand{\Re}{\operatorname{Re}}
\renewcommand{\Im}{\operatorname{Im}}

\newcommand{\loc}{\mathrm{loc}}
\newcommand{\cpt}{\mathrm{cpt}}

\def\Japan#1{\left \langle #1 \right \rangle}

\newtheorem{theorem}{Theorem}[section]
\newtheorem{badtheorem}[theorem]{``Theorem"}
\newtheorem{prop}[theorem]{Proposition}
\newtheorem{lemma}[theorem]{Lemma}
\newtheorem{sublemma}[theorem]{Sublemma}
\newtheorem{proposition}[theorem]{Proposition}
\newtheorem{corollary}[theorem]{Corollary}
\newtheorem{conjecture}[theorem]{Conjecture}
\newtheorem{axiom}[theorem]{Axiom}
\newtheorem{assumption}[theorem]{Assumption}

\newtheorem{mainthm}{Theorem}
\renewcommand{\themainthm}{\Alph{mainthm}}

\newtheorem{claim}{Claim}[theorem]
\renewcommand{\theclaim}{\thetheorem\Alph{claim}}
% \newtheorem*{claim}{Claim}

\theoremstyle{definition}
\newtheorem{definition}[theorem]{Definition}
\newtheorem{remark}[theorem]{Remark}
\newtheorem{example}[theorem]{Example}
\newtheorem{notation}[theorem]{Notation}

\newtheorem{exercise}[theorem]{Discussion topic}
\newtheorem{homework}[theorem]{Homework}
\newtheorem{problem}[theorem]{Problem}

\makeatletter
\newcommand{\proofpart}[2]{%
  \par
  \addvspace{\medskipamount}%
  \noindent\emph{Part #1: #2.}
}
\makeatother

\DeclareMathOperator*{\essinf}{ess\,inf}
\DeclareMathOperator*{\esssup}{ess\,sup}


\numberwithin{equation}{section}


% Mean
\def\Xint#1{\mathchoice
{\XXint\displaystyle\textstyle{#1}}%
{\XXint\textstyle\scriptstyle{#1}}%
{\XXint\scriptstyle\scriptscriptstyle{#1}}%
{\XXint\scriptscriptstyle\scriptscriptstyle{#1}}%
\!\int}
\def\XXint#1#2#3{{\setbox0=\hbox{$#1{#2#3}{\int}$ }
\vcenter{\hbox{$#2#3$ }}\kern-.6\wd0}}
\def\ddashint{\Xint=}
\def\dashint{\Xint-}

\usepackage[backend=bibtex,style=alphabetic,giveninits=true]{biblatex}
\renewcommand*{\bibfont}{\normalfont\footnotesize}
\addbibresource{best_curl.bib}
\renewbibmacro{in:}{}
\DeclareFieldFormat{pages}{#1}

\newcommand\todo[1]{\textcolor{red}{TODO: #1}}


\begin{document}
\begin{abstract}
We show that the problem of finding a calibration $\varphi$, and a $\varphi$-calibrated current, can be expressed as the limit of first-order degenerate-elliptic PDE.
We also show that on a wide class of manifolds of special holonomy, there exist calibrations which calibrate currents, but do not interact with the special holonomy.
\end{abstract}

\maketitle

\section{Introduction}
Let $M$ be a closed Riemannian manifold of dimension $d \geq 2$, and let $k \in \{1, \dots, d - 1\}$.
A \dfn{$k$-blade} at $x \in M$ is a wedge product of the form $v := v_1 \wedge \dots \wedge v_n$, $v_i \in T_x M$; in this case we write $v \in B_{x, k} M$.
The \dfn{comass} of a $k$-form $\varphi$ is
$$\Comass(\varphi) = \esssup_{x \in M} \max_{v \in B_{x, k} M} \langle \varphi(x), v\rangle.$$
The \dfn{costable norm} of a cohomology class $\rho \in H^k(M, \RR)$ is
$$\Comass(\rho) := \inf_{[\varphi] = \rho} \Comass(\varphi).$$
I think that since $M$ is closed, the costable norm really is a (positive-definite) norm on $H^k(M, \RR)$.

\begin{mainthm}\label{existence of calibrations}
Let $\rho \in H^k(M, \RR)$ satisfy $\Comass(\rho) = 1$.
Then there is a measurable calibration $\varphi$ which represents $\rho$, and a closed $k$-current $T$ which is calibrated by $\varphi$:
$$\langle T, \varphi\rangle = \Mass(T).$$
\end{mainthm}

%%%%%%%%%%%%%%%%%%%%%%
\section{Preliminaries}
\subsection{Norms on the exterior algebra}
Let $V$ be a Hilbert space of dimension $d < \infty$, and let $V^{\wedge k}$ be the $k$th exterior power of $V$.
Thus 
$$\dim(V^{\wedge k}) = \binom dk.$$

A \dfn{$k$-blade} over $V$ is a $k$-vector of the form $v_1 \wedge \cdots \wedge v_k$ for some $v_1, \dots, v_k \in V$.
We denote by $\Bl_k(V) \subseteq V^{\wedge k}$ the set of $k$-blades over $V$.
If $v, w \in \Bl_k(V)$, let
$$\langle v, w\rangle := \det(\langle v_i, w_j\rangle_{ij}),$$
which extends to an inner product on $V^{\wedge k}$ in the natural way.

Let $m(d, k)$ be the smallest number such that any $k$-vector can be written as the sum of at most $m(d, k)$ $k$-blades.
Then $m(d, k) = \todo{???}$ by (\ref{minimal number of blades}); in particular, $m(d, k) = 1$ unless $d \geq 4$ and $k \in \{2, \dots, d - 2\}$.
Otherwise, the below singular value decomposition is trivial, in the sense that for any $k$-vector $\varphi$, $\varphi = \sigma_1(\varphi) \xi_1(\varphi)$ with $\sigma_1(\varphi) = |\varphi|$ and $\xi_1(\varphi) = \varphi/|\varphi|$.
This is why in the previous paper \todo{cite my old paper} we could avoid the use of the singular value decomposition.
In the present paper, we are especially interested in the cases $(d, k) = (4, 2)$ and $(d, k) = (7, 3)$ (arising from the study of K\"ahler surfaces and $G_2$ manifolds, respectively).

Let $\varphi \in V^{\wedge k}$ and $m := m(d, k)$.
We define
\begin{equation}\label{first singular value}
\sigma_1(\varphi) := \max_{\substack{v \in \Bl_k(V) \\ |v| \leq 1}} \langle \varphi, v\rangle,
\end{equation}
and let $\xi_1(\varphi) \in \Bl_k(V)$ realize the maximum in (\ref{first singular value}).
We can then inductively define 
\begin{equation}\label{low singular value}
\sigma_{j + 1}(\varphi) := \max_{\substack{v \in \Bl_k(V) \\ |v| \leq 1 \\ v \perp \xi_1(\varphi), \cdots, \xi_j(\varphi)}} \langle \varphi, v\rangle,
\end{equation}
and let $\xi_{j + 1}(\varphi) \in \Bl_k(V)$ realize the maximum in (\ref{low singular value}).
Then $\sigma_j(\varphi) \geq \sigma_{j + 1}(\varphi)$, $\{\xi_1(\varphi), \dots, \xi_m(\varphi)\}$ is an orthonormal subset of $V^{\wedge k}$, and
\begin{equation}\label{SVD of a form}
\varphi = \sum_{j = 1}^m \sigma_j(\varphi) \xi_j(\varphi).
\end{equation}
Moreover, since $\Bl_k(V)$ spans $V^{\wedge k}$,
\begin{equation}\label{seminorms are norms}
\sigma_1(\varphi) = 0 \implies \varphi = 0.
\end{equation}

\begin{definition}
The \dfn{singular value decomposition} of $\varphi \in V^{\wedge k}$ is defined as (\ref{SVD of a form}), where $\sigma_j(\varphi)$ and $\xi_j(\varphi)$ are defined by (\ref{first singular value}) and (\ref{low singular value}).
\end{definition}

\begin{definition}
Let $p \in [1, \infty)$.
The $p$th \dfn{Schatten-von Neumann norm} on $V^{\wedge k}$ is 
$$|\varphi|_p := \left(\sum_{j = 1}^m \sigma_j(\varphi)^p\right)^{1/p}.$$
We moreover define $|\varphi|_\infty := \sigma_1(\varphi)$.
\end{definition}

Since $|\varphi|_p$ is the $\ell^p$ norm of $(\sigma_1(\varphi), \dots, \sigma_m(\varphi))$, $p \in [1, \infty]$, and $\ell^p$ norms converge to the $\ell^\infty$ norm as $p \to \infty$,
\begin{equation}\label{SvN norms converge}
|\varphi|_\infty = \lim_{p \to \infty} |\varphi|_p.
\end{equation}
Since $\xi_1(\varphi), \dots, \xi_m(\varphi)$ is an orthonormal basis of $V^{\wedge k}$, the norm arising from the inner product on $V^{\wedge k}$ is exactly $|\cdot|_2$.
Applying H\"older's inequality to the atomic measure space $\{1, \dots, m\}$, we have for $1 \leq p \leq q \leq \infty$ that
\begin{equation}\label{relating Schatten norms}
|\varphi|_q \leq |\varphi|_p \leq m^{\frac{1}{p} - \frac{1}{q}} |\varphi|_q.
\end{equation}

In order to prove further properties of $|\cdot|_p$, we denote for a matrix $A$,
$$Q(A) := \sqrt{AA^\dagger}.$$
We then have the \dfn{Schatten-von Neumann norm} $|\cdot|_{M, p}$ on matrices, which is the $\ell^p$ norm of their singular values.
To exploit this norm, let $(e_j)$ be an orthonormal basis of $V^{\wedge k}$, and $\varphi \in V^{\wedge k}$.
For convenience, we extend $\{\xi_i(\varphi): i \leq m\}$ to an orthonormal basis of $V^{\wedge k}$.
We introduce the matrix $A \in \End(\RR^{\binom dk})$,
$$A_{ij} := \sigma_i(\varphi) \langle \xi_i(\varphi), e_j\rangle,$$
so that $\sum_i A_{ij} = \langle \varphi, e_j\rangle$.
We also let $\Sigma$ be the diagonal matrix such that $\Sigma_{ii} := \sigma_i(\varphi)$,
and $\Xi_{ij} := \langle \xi_i(\varphi), e_j\rangle$.
Then $\Sigma$ is a positive semidefinite diagonal matrix with decreasing entries, $\Xi$ is an orthogonal matrix, and $A = \Sigma \Xi$.
Then $Q(A) = \Sigma$, the singular value decomposition of $A$ is 
$$A = I \Sigma \Xi,$$
and the singular values of $A$ equal the first $m$ singular values of $\varphi$.
It follows that $|\cdot|_p$ is a norm.
We have 
\begin{equation}\label{derivative of matrix norm}
\frac{\partial}{\partial A_{ij}} |A|_{M, p}^p = p(Q(A)^{p - 2} A)_{ij} = p(\Sigma^{p - 1} \Xi)_{ij}.
\end{equation}
Multiplying both sides of (\ref{derivative of matrix norm}) by the basis vectors $e_j$, and then summing in $i$,
\begin{equation}\label{derivative of schatten norm}
\frac{\dif}{\dif \varphi} |\varphi|_p^p = p \sum_{i=1}^m \sigma_i(\varphi)^{p - 1} \xi_i(\varphi).
\end{equation}
Since $|\cdot|_{M, p}^p$ is a strictly convex function if $p \in (1, \infty)$, so is $|\cdot|_p^p$.

\begin{lemma}[von Neumann inequality]
Let $\psi, \varphi \in V^{\wedge k}$.
Then
\begin{equation}\label{von Neumann inequality}
|\langle \varphi, \psi\rangle| \leq \sum_{i=1}^m \sigma_i(\varphi) \sigma_i(\psi).
\end{equation}
\end{lemma}
\begin{proof}
Introduce the vectors $x_i := \sigma_i(\varphi)$, $y_i := \sigma_i(\psi)$, introduce the matrix $B_{ij} := \langle \xi_i(\varphi), \xi_j(\psi)\rangle$, and apply (\ref{permute to decreasing}).
\end{proof}

\begin{lemma}
Let $(p, q)$ be a H\"older pair. Then the norms $|\cdot|_p$ and $|\cdot|_q$ are dual.
\end{lemma}
\begin{proof}
Let $\varphi \in V^{\wedge k}$.
From von Neumann's inequality (\ref{von Neumann inequality}) and H\"older's inequality applied to the atomic measure space $\{1, \dots, m\}$, we have for every $\psi \in V^{\wedge k}$ that
$$|\langle \varphi, \psi\rangle| \leq |\varphi|_p |\psi|_q.$$
Thus 
\begin{equation}\label{dual norm inequality}
|\varphi|_p \geq \max_{\substack{\psi \in V^{\wedge k} \\ |\psi|_q \leq 1}} \langle \varphi, \psi\rangle.
\end{equation}
Conversely, if $p \in [1, \infty)$ and we take
$$\psi := |\varphi|_p^{1 - p} \sum_{i=1}^m \sigma_i(\varphi)^{p - 1} \xi_i(\varphi),$$
we see that the converse inequality to (\ref{dual norm inequality}) holds.
By taking limits we see that the result holds even if $p = \infty$.
\end{proof}

%%%%%%%%%%%%%%%%
\subsection{Mass and comass}
Let $M$ be a closed Riemannian manifold of dimension $d$.
We apply the above construction when $V$ is (a typical fiber of) the cotangent bundle of $M$.

If $\mathscr F$ is a sheaf and $\mathcal X$ is a function space that the components of sections of $\mathscr F$ can take values in, we write $\mathcal X(\cdot, \mathscr F)$ for the functor that sends an open set to the local sections of $\mathscr F$ with components in $\mathcal X$.

Let $\Omega^k$ ($\Omega^k_{\rm cl}$) be the sheaf of (closed) $k$-forms on $M$, and let $\Omega_k$ ($\Omega^{\rm cl}_k$) be the sheaf of (closed) $k$-currents on $M$.
We willl tacitly identify $L^q(\cdot, \Omega_k)$ with $L^q(\cdot, \Omega^{d - k})$, by setting, for every $\varphi \in L^p(\cdot, \Omega^k)$,
$$\langle T, \varphi\rangle = \int_M T \wedge \varphi.$$
In fact we write $\int_M T \wedge \varphi$ to mean $\langle T, \varphi\rangle$ even if $T$ cannot be identified with a form.
We write $\mathcal M$ for the function space of Radon measures.

\begin{definition}
Let $\varphi \in L^\infty(M, \Omega^k)$. The \dfn{comass} of $\varphi$ is 
\begin{equation}\label{infinity norm gives comass}
\Comass(\varphi) = \esssup_{x \in M} \sigma_1(\varphi(x)).
\end{equation}
Dually, let $T \in \mathcal M(M, \Omega_k)$. The \dfn{mass} of $T$ is 
$$\Mass(T) = \sup_{\substack{\varphi \in C^\infty(M, \Omega^k) \\ \Comass(\varphi) \leq 1}} \int_M T \wedge \varphi.$$
\end{definition}

Let $T \in \mathcal M(M, \Omega_k)$.
We can introduce the \dfn{mass measure} $\mu_T$ by setting, for every open set $U \subseteq M$,
$$\mu_T(U) = \sup_{\substack{\varphi \in C^\infty_\cpt(U, \Omega^k) \\ \Comass(\varphi) \leq 1}} \int_M T \wedge \varphi.$$

\begin{lemma}\label{formula for mass measure}
Let $T \in L^1(M, \Omega_k)$, viewed as a $d - k$-form. Then 
$$\mu_T(U) = \int_U |T|_1 \star 1.$$
\end{lemma}
\begin{proof}
In this proof we use $|T|_1$ exclusively to mean the first Schatten-von Neumann norm, not the Hodge dual of the mass measure.
In particular we think of $T$ as a $d - k$-form.
Since $|\cdot|_1$ and $|\infty|$ are dual norms,
$$\int_U |T|_1 \star 1 = \int_U \max_{\substack{\psi(x) \in (T^*_x M)^{\wedge (d - k)} \\ |\psi(x)|_\infty \leq 1}} \langle T(x), \psi(x)\rangle \dif \mathcal H^d(x).$$
Since $T$ is measurable and the maximizer $\psi(x)$ depends continuously on $T(x)$, $\psi$ can chosen to be measurable, and it is clear that $\Comass(\psi) = 1$.

For each $\varepsilon > 0$ we select $\varphi_\varepsilon \in C^\infty_\cpt(U, \Omega^k)$ such that $\Comass(\varphi_\varepsilon) = 1$, and in $L^\infty$,
$$\varphi_\varepsilon \weakto^* \star \psi.$$
Then we have
$$\int_U |T|_1 \star 1 = \lim_{\varepsilon \to 0} \int_U T \wedge \varphi_\varepsilon.$$
The claim now follows from the definition of $\mu_T(U)$.
\end{proof}

Henceforth we write $|T|_1 \star 1$ for the mass measure of $T \in \mathcal M(M, \Omega_k)$, even if $T$ is not $L^1$.

%%%%%%%%%%%%%%%%%%%%%%
\section{Existence of calibrations and calibrated currents}
\subsection{The regularized variational problems}
\begin{definition}
Let $p \in (1, \infty)$.
The \dfn{$p$-energy} is defined on $L^p(M, \Omega^k_{\rm cl})$ by 
$$J_p(\varphi) := \frac{1}{p} \int_M |\varphi|_p^p \star 1.$$
We say that $\varphi$ is a \dfn{$p$-tight $k$-form} if for every $\beta \in W^{1, p}(M, \Omega^{k - 1})$,
$$J_p(\varphi) \leq J_p(\varphi + \dif \beta).$$
\end{definition}

\begin{proposition}[Hodge theorem for $p$-tight forms]
Let $\rho \in H^k(M, \RR)$ and $p \in (1, \infty)$.
Then there is a unique $p$-tight $k$-form $\varphi$ representing $\rho$.
Moreover, $\varphi$ is the unique representative of $\rho$ which solves
\begin{equation}\label{first variation}
\dif^*\left[\sum_{i=1}^m \sigma_i(\varphi)^{p - 1} \xi_i(\varphi)\right] = 0.
\end{equation}
\end{proposition}
\begin{proof}
Let $\varphi \in L^p(M, \Omega^k_{\rm cl})$, $\gamma \in W^{1, p}(M, \Omega^{k - 1})$, and consider the one-parameter family of $k$-forms
$$\varphi_t := \varphi + t \dif \gamma.$$
Then, by (\ref{derivative of schatten norm}),
\begin{align*}
\frac{\dif}{\dif t} J_p(\varphi_t)
&= \frac{1}{p} \int_M \frac{\partial}{\partial t} |\varphi_t|_p^p \star 1
= \int_M \left\langle \sum_{i=1}^m \sigma_i(\varphi_t)^{p - 1} \xi_i(\varphi_t), \dif \gamma\right\rangle \star 1.
\end{align*}
Therefore if $\varphi$ is $p$-tight, (\ref{first variation}) holds.
Moreover, since $|\cdot|_p^p$ is strictly convex, 
$$\frac{\dif}{\dif t^2} J_p(\varphi_t) = \int_M \frac{\partial^2}{\partial t^2} |\varphi_t|_p^p \star 1 \geq 0$$
with equality iff $\varphi_t = 0$.
Therefore $J_p$ is a strictly convex functional on each cohomology class, with the Euler-Lagrange equation (\ref{first variation}).
It is moreover coercive on each cohomology class in $L^p(M, \Omega^k_{\rm cl})$, since $J_p(\varphi) \sim_p \|\varphi\|_{L^p}^p$.
Therefore by the direct method in the calculus of variations, the existence and uniqueness claims hold.
\end{proof}

\begin{definition}
Let $(p, q)$ be a H\"older pair, and let $\varphi$ be a $p$-tight $k$-form.
Let
$$T := -\frac{1}{p J_p(\varphi)} \sum_{i=1}^m \sigma_i(\varphi)^{p - 1} \star \xi_i(\varphi).$$
We call $T$ the \dfn{dual $k$-current} to $\varphi$.
\end{definition}

\begin{lemma}
Let $(p, q)$ be a H\"older pair, let $\varphi$ be a $p$-tight $k$-form, and let $T$ be the dual $k$-current to $\varphi$.
Then $T$ is a $q$-tight $d - k$-form such that 
\begin{equation}\label{intersection product is 1}
\int_M T \wedge \varphi = 1.
\end{equation}
\end{lemma}
\begin{proof}
By (\ref{first variation}), $\dif T = 0$.
It is clear from the definitions that
$$\sigma_i(T) = \sigma_i(\varphi)^{p - 1}.$$
Moreover, $(p - 1)(q - 1) = 1$, so 
$$p J_p(\varphi) \dif^*\left[\sum_{i=1}^m \sigma_i(T)^{p - 1} \xi_i(T)\right] = -\dif^*\left[\sum_{i=1}^m \sigma_i(\varphi) \star^{-1} \xi_i(\varphi)\right] = \pm \dif \varphi = 0$$
where the sign depends on the parities of $d$ and $k$.
Finally, we use the fact that $\xi_1(\varphi), \dots, \xi_m(\varphi)$ is an orthonormal basis of $(T^* M)^{\wedge k}$ to compute
\begin{align*}
\int_M T \wedge \varphi
&= -\frac{1}{p J_p(\varphi)} \int_M \sum_{i=1}^m \sigma_i(\varphi)^{p - 1} \star \xi_i(\varphi) \wedge \varphi \\
&= \frac{1}{p J_p(\varphi)} \int_M \sum_{i,j=1}^m \sigma_i(\varphi)^{p - 1} \sigma_j(\varphi) \langle\xi_i(\varphi), \xi_j(\varphi)\rangle \star 1 \\
&= \frac{1}{p J_p(\varphi)} \int_M \sum_{i=1}^m \sigma_i(\varphi)^p \star 1 \\
&= 1. \qedhere 
\end{align*}
\end{proof}

%%%%%%%%%%%%%%%
\subsection{Constructing the calibration}
We now fix a cohomology class $\rho \in H^k(M, \RR)$, with $\Comass(\rho) = 1$.
Thus we have $p$-tight $k$-forms $\varphi_p$ for every $p \in (1, \infty)$.
We will construct a limiting form $\varphi$.
Let $m := m(d, k)$ throughout.

\begin{lemma}
For any $2 \leq r \leq p < \infty$, one has the energy bound
\begin{equation}\label{energy bound}
J_r(\varphi_p) \leq \frac{m}{r} (\vol M)^{r/p}.
\end{equation}
In particular, we have the norm bound 
\begin{equation}\label{norm bound}
\|\varphi_p\|_{L^r} \leq m^{1/2} (\vol M)^{1/p}.
\end{equation}
\end{lemma}
\begin{proof}
Let $\varepsilon > 0$ and choose some $\psi \in C^0(M, \Omega^k_{\rm cl})$ to represent $\rho$.
Since $\Comass(\rho) = 1$, we may assume $\Comass(\psi) \leq 1 + \varepsilon$.
Then, by (\ref{relating Schatten norms}),
$$|\psi|_p^p \leq |\psi|_\infty^p m \leq (1 + \varepsilon)^p m.$$
Therefore 
$$J_p(\varphi_p) \leq J_p(\psi) \leq \frac{m}{p} \int_M (1 + \varepsilon)^p \star 1 = \frac{m \vol M}{p} (1 + \varepsilon)^p.$$
Taking $\varepsilon \to 0$ we conclude (\ref{energy bound}) if $r = p$.
Otherwise, we again apply (\ref{relating Schatten norms}) to estimate 
$$J_r(\varphi_p) \leq \frac{1}{r} m^{1 - r/p} \int_M |\varphi_p|_p^r \star 1 \leq \frac{1}{r} m^{1 - r/p} (pJ_p(\varphi_p))^{r/p}.$$
Plugging in (\ref{energy bound}) in the case $r = p$, we conclude (\ref{energy bound}) in the general case.
We can then prove (\ref{norm bound}) by applying (\ref{relating Schatten norms}) and (\ref{energy bound}):
\begin{align*}
\|\varphi_p\|_{L^r} &= \left(\int_M |\varphi_p|_2^r \star 1\right)^{1/r} \leq m^{1/2 - 1/r} (rJ_r(\varphi_p))^{1/r}. \qedhere 
\end{align*}
\end{proof}

\begin{proposition}
There exists $\varphi \in L^\infty(M, \Omega^k_{\rm cl})$ such that:
\begin{enumerate}
\item There exists a subsequence of $p \to \infty$ such that for every $r \in [1, \infty)$, $\varphi_p \weakto \varphi$.
\item $\Comass(\varphi) = 1$.
\item $[\varphi] = \rho$.
\end{enumerate}
\end{proposition}
\begin{proof}
By the norm bound (\ref{norm bound}) and Alaoglu's theorem, we can choose a subsequential limit of $\varphi$, which converges in the weak topology on $L^r$ for every $r \in [1, \infty)$.
By the lower semicontinuity of $J_r$, we can use the energy bound (\ref{energy bound}) to obtain 
$$J_r(\varphi)^{1/r} \leq (m/r)^{1/r}.$$
Taking $r \to \infty$, we see that $\Comass(\varphi) \leq 1$.

Now let $\sigma \in H_k(M, \RR)$, and choose a representative $T \in C^0(M, \Omega^{\rm cl}_k)$ of $\sigma$.
Since $\varphi_p \weakto \varphi$ in $L^2$,
$$\langle \sigma, [\varphi]\rangle = \int_M T \wedge \varphi = \lim_{p \to \infty} \int_M T \wedge \varphi_p = \langle \sigma, \rho\rangle.$$
Therefore $[\varphi] = \rho$; since $\Comass(\rho) = 1$, it follows that $\Comass(\varphi) \geq 1$.
\end{proof}

\begin{lemma}
For any $1 < q < \infty$, one has the mass bound 
\begin{equation}\label{mass bound}
\Mass(T_q) \leq (m \vol M)^{1/p}.
\end{equation}
\end{lemma}
\begin{proof}
We use Lemma \ref{formula for mass measure}, (\ref{relating Schatten norms}), and H\"older's inequality to bound 
\begin{align*}
\Mass(T_q) 
&= \int_M |T_q|_1 \star 1 
\leq (m \vol M)^{1/p} \left(\int_M |T_q|_q^q \star 1\right)^{1/q} \\
&= (m \vol M)^{1/p} \left(\frac{1}{p J_p(\varphi_p)} \int_M |\varphi_p|_p^p \star 1\right)^{1/q} 
= (m \vol M)^{1/p}. \qedhere
\end{align*}
\end{proof}

\begin{proposition}
There exists $T \in \mathcal M(M, \Omega_k^{\rm cl})$ such that $\varphi$ calibrates $T$:
$$\Mass(T) = \langle [T], [\varphi]\rangle.$$
\end{proposition}
\begin{proof}
Taking $q \to 1$ in (\ref{mass bound}) and applying Alaoglu's theorem, there exists $T$ such that $T_q \weakto^* T$ as Radon measures as $q \to 1$.
In particular, $\Mass(T) \leq 1$.
However, since $\varphi_p$ is cohomologous to $\varphi$, we can apply (\ref{intersection product is 1}):
$$\langle [T], [\varphi]\rangle = \lim_{q \to 1} \langle [T_q], [\varphi]\rangle = \lim_{q \to 1} \int_M T_q \wedge \varphi_p = 1.$$
Therefore, since $\Comass(\varphi) = 1$,
\begin{align*}
\Mass(T) &\leq \langle [T], [\varphi]\rangle = \int_M T \wedge \varphi \leq \Mass(T). \qedhere 
\end{align*}
\end{proof}

\todo{Show that there is a stronger sense in which $\varphi$ calibrates $T$.
Otherwise the above construction is pretty trivial.}

\todo{Otherwise, redo the above argument without $p$-regularization, and just weakstar topology}

%%%%%%%%%%%%%%%%%%%%%%%%%%%%
\section{Breaking special holonomy}
\subsection{K\"ahler surfaces}
The first example of a K\"ahler surface is $\CC \PP^2$.
But $H^2(\CC \PP^2, \RR) \cong \RR$, and the generating class contains the K\"ahler $2$-form (since it is dual to the homology class of the equatorial $\CC \PP^1$).

\begin{proposition}
Let $M$ be a K3 surface.
Then there is a calibration $(1, 1)$-form on $M$ which calibrates a current, but is not covariantly constant.
\end{proposition}
\begin{proof}
The space of covariantly constant forms has dimension $\leq \binom 42 = 6$ \todo{right?}, but the space of calibration $(1, 1)$-forms which calibrate a current, up to cohomology, is identified with the unit sphere of $H^{1, 1}(M)$, which has dimension $19$.
\end{proof}

%%%%%%%%%%%%%%%%%
\appendix 
\section{Elementary computations}
\begin{lemma}
Let $m(d, k)$ be the smallest number such that any $k$-vector over $\RR^d$ can be written as the sum of at most $m(d, k)$ $k$-blades.
Then 
\begin{equation}\label{minimal number of blades}
m(d, k) = ???.
\end{equation}
\end{lemma}

\begin{lemma}
Suppose that $x, y \in \RR^m_+$ have decreasing entries, and $B \in \Orth(\RR^m)$. Then
\begin{equation}\label{permute to decreasing}
|\langle x, By\rangle| \leq \langle x, y\rangle.
\end{equation}
\end{lemma}
\begin{proof}
\todo{If we need it...}
\end{proof}

\printbibliography

\end{document}
