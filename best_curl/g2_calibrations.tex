\documentclass[reqno,11pt]{amsart}
\usepackage[letterpaper, margin=1in]{geometry}
\RequirePackage{amsmath,amssymb,amsthm,graphicx,mathrsfs,url,slashed,subcaption}
\RequirePackage[usenames,dvipsnames]{xcolor}
\RequirePackage[colorlinks=true,linkcolor=Red,citecolor=Green]{hyperref}
\RequirePackage{amsxtra}
\usepackage{cancel}
\usepackage{tikz-cd}
%\usepackage[T1]{fontenc}

% \setlength{\textheight}{9.3in} \setlength{\oddsidemargin}{-0.25in}
% \setlength{\evensidemargin}{-0.25in} \setlength{\textwidth}{7in}
% \setlength{\topmargin}{-0.25in} \setlength{\headheight}{0.18in}
% \setlength{\marginparwidth}{1.0in}
% \setlength{\abovedisplayskip}{0.2in}
% \setlength{\belowdisplayskip}{0.2in}
% \setlength{\parskip}{0.05in}
%\renewcommand{\baselinestretch}{1.05}

\title{Many calibrations}
\author{Aidan Backus}
\address{Department of Mathematics, Brown University}
\email{aidan\_backus@brown.edu}
\date{\today}
\keywords{}
\subjclass[2020]{}

\newcommand{\NN}{\mathbf{N}}
\newcommand{\ZZ}{\mathbf{Z}}
\newcommand{\QQ}{\mathbf{Q}}
\newcommand{\RR}{\mathbf{R}}
\newcommand{\CC}{\mathbf{C}}
\newcommand{\DD}{\mathbf{D}}
\newcommand{\PP}{\mathbf P}
\newcommand{\MM}{\mathbf M}
\newcommand{\II}{\mathbf I}
\newcommand{\Hyp}{\mathbf H}
\newcommand{\Sph}{\mathbf S}
\newcommand{\Group}{\mathbf G}
\newcommand{\GL}{\mathbf{GL}}
\newcommand{\Orth}{\mathbf{O}}
\newcommand{\SpOrth}{\mathbf{SO}}
\newcommand{\Ball}{\mathbf{B}}

\newcommand*\dif{\mathop{}\!\mathrm{d}}

\DeclareMathOperator{\card}{card}
\DeclareMathOperator{\dist}{dist}
\DeclareMathOperator{\id}{id}
\DeclareMathOperator{\End}{End}
\DeclareMathOperator{\Hom}{Hom}
\DeclareMathOperator{\coker}{coker}
\DeclareMathOperator{\supp}{supp}
\DeclareMathOperator{\vol}{vol}
\DeclareMathOperator{\tr}{tr}

\newcommand{\Leaves}{\mathscr L}
\newcommand{\Lagrange}{\mathcal L}
\newcommand{\Hypspace}{\mathscr H}

\newcommand{\Chain}{\underline C}

\newcommand{\Two}{\mathrm{I\!I}}

\newcommand{\normal}{\mathbf n}
\newcommand{\radial}{\mathbf r}
\newcommand{\evect}{\mathbf e}

\newcommand{\Bl}{\operatorname{Bl}}
\newcommand{\diam}{\mathrm{diam}}
\newcommand{\Ell}{\mathrm{Ell}}
\newcommand{\inj}{\mathrm{inj}}
\newcommand{\Lip}{\mathrm{Lip}}
\newcommand{\MCL}{\mathrm{MCL}}
\newcommand{\Riem}{\mathrm{Riem}}

\newcommand{\Mass}{\mathbf M}
\newcommand{\Comass}{\mathbf L}

\newcommand{\weakto}{\rightharpoonup}

\newcommand{\Min}{\mathrm{Min}}
\newcommand{\Max}{\mathrm{Max}}

\newcommand{\dfn}[1]{\emph{#1}\index{#1}}

\renewcommand{\Re}{\operatorname{Re}}
\renewcommand{\Im}{\operatorname{Im}}

\newcommand{\loc}{\mathrm{loc}}
\newcommand{\cpt}{\mathrm{cpt}}

\def\Japan#1{\left \langle #1 \right \rangle}

\newtheorem{theorem}{Theorem}[section]
\newtheorem{badtheorem}[theorem]{``Theorem"}
\newtheorem{prop}[theorem]{Proposition}
\newtheorem{lemma}[theorem]{Lemma}
\newtheorem{sublemma}[theorem]{Sublemma}
\newtheorem{proposition}[theorem]{Proposition}
\newtheorem{corollary}[theorem]{Corollary}
\newtheorem{conjecture}[theorem]{Conjecture}
\newtheorem{axiom}[theorem]{Axiom}
\newtheorem{assumption}[theorem]{Assumption}

\newtheorem{mainthm}{Theorem}
\renewcommand{\themainthm}{\Alph{mainthm}}

\newtheorem{claim}{Claim}[theorem]
\renewcommand{\theclaim}{\thetheorem\Alph{claim}}
% \newtheorem*{claim}{Claim}

\theoremstyle{definition}
\newtheorem{definition}[theorem]{Definition}
\newtheorem{remark}[theorem]{Remark}
\newtheorem{example}[theorem]{Example}
\newtheorem{notation}[theorem]{Notation}

\newtheorem{exercise}[theorem]{Discussion topic}
\newtheorem{homework}[theorem]{Homework}
\newtheorem{problem}[theorem]{Problem}

\makeatletter
\newcommand{\proofpart}[2]{%
  \par
  \addvspace{\medskipamount}%
  \noindent\emph{Part #1: #2.}
}
\makeatother

\DeclareMathOperator*{\essinf}{ess\,inf}
\DeclareMathOperator*{\esssup}{ess\,sup}


\numberwithin{equation}{section}


% Mean
\def\Xint#1{\mathchoice
{\XXint\displaystyle\textstyle{#1}}%
{\XXint\textstyle\scriptstyle{#1}}%
{\XXint\scriptstyle\scriptscriptstyle{#1}}%
{\XXint\scriptscriptstyle\scriptscriptstyle{#1}}%
\!\int}
\def\XXint#1#2#3{{\setbox0=\hbox{$#1{#2#3}{\int}$ }
\vcenter{\hbox{$#2#3$ }}\kern-.6\wd0}}
\def\ddashint{\Xint=}
\def\dashint{\Xint-}

\usepackage[backend=bibtex,style=alphabetic,giveninits=true]{biblatex}
\renewcommand*{\bibfont}{\normalfont\footnotesize}
\addbibresource{best_curl.bib}
\renewbibmacro{in:}{}
\DeclareFieldFormat{pages}{#1}

\newcommand\todo[1]{\textcolor{red}{TODO: #1}}


\begin{document}
\maketitle

\section{Introduction}
Let $M$ be a closed Riemannian manifold of dimension $d \geq 2$, and let $k \in \{1, \dots, d - 1\}$.
A \dfn{$k$-blade} at $x \in M$ is a wedge product of the form $v := v_1 \wedge \dots \wedge v_n$, $v_i \in T_x M$; in this case we write $v \in B_{x, k} M$.
The \dfn{comass} of a $k$-form $\varphi$ is
$$\Comass(\varphi) = \esssup_{x \in M} \max_{v \in B_{x, k} M} \langle \varphi(x), v\rangle.$$
The \dfn{costable norm} of a cohomology class $\rho \in H^k(M, \RR)$ is
$$\Comass(\rho) := \inf_{[\varphi] = \rho} \Comass(\varphi).$$
I think that since $M$ is closed, the costable norm really is a (positive-definite) norm on $H^k(M, \RR)$.

\begin{mainthm}\label{existence of calibrations}
Let $\rho \in H^k(M, \RR)$ satisfy $\Comass(\rho) = 1$.
Then there is a measurable calibration $\varphi$ which represents $\rho$, and a closed $k$-current $T$ which is calibrated by $\varphi$:
$$\langle T, \varphi\rangle = \Mass(T).$$
\end{mainthm}

%%%%%%%%%%%%%%%%%%%%%%
\section{Preliminaries}
\subsection{Singular values of differential forms}
Let $V$ be a Hilbert space of dimension $d < \infty$.
We denote by $\Bl_k(V) \subseteq V^{\wedge k}$ the set of $k$-blades over $V$, and record for future reference that
$$\dim V^{\wedge k} = \binom dk.$$
If $v, w \in \Bl_k(V)$, let
$$\langle v, w\rangle := \det(\langle v_i, w_j\rangle_{ij}),$$
which extends to an inner product on $V^{\wedge k}$ in the natural way.

Let $\varphi \in V^{\wedge k}$.
We define
\begin{equation}\label{first singular value}
\sigma_1(\varphi) := \max_{\substack{v \in \Bl_k(V) \\ |v| \leq 1}} \langle \varphi, v\rangle,
\end{equation}
and let $\xi_1(\varphi) \in \Bl_k(V)$ realize the maximum in (\ref{first singular value}).
We can then inductively define 
\begin{equation}\label{low singular value}
\sigma_{j + 1}(\varphi) := \max_{\substack{v \in \Bl_k(V) \\ |v| \leq 1 \\ v \perp \xi_1(\varphi), \cdots, \xi_j(\varphi)}} \langle \varphi, v\rangle,
\end{equation}
and let $\xi_{j + 1}(\varphi) \in \Bl_k(V)$ realize the maximum in (\ref{low singular value}).
Then $\sigma_j(\varphi) \geq \sigma_{j + 1}(\varphi)$, $\xi_1(\varphi), \dots, \xi_{\binom dk}(\varphi)$ is an orthonormal basis of $V^{\wedge k}$, and
\begin{equation}\label{SVD of a form}
\varphi = \sum_{j = 1}^{\binom dk} \sigma_j(\varphi) \xi_j(\varphi).
\end{equation}
Moreover, since $\Bl_k(V)$ spans $V^{\wedge k}$,
\begin{equation}\label{seminorms are norms}
\sigma_1(\varphi) = 0 \implies \varphi = 0.
\end{equation}

\begin{definition}
The \dfn{singular value decomposition} of $\varphi \in V^{\wedge k}$ is defined as (\ref{SVD of a form}), where $\sigma_j(\varphi)$ and $\xi_j(\varphi)$ are defined by (\ref{first singular value}) and (\ref{low singular value}).
\end{definition}

If $V$ is (a typical fiber of) the cotangent bundle of $M$, and $\varphi$ is a $k$-form on $M$, then
\begin{equation}\label{infinity norm gives comass}
\Comass(\varphi) = \esssup_{x \in M} \sigma_1(\varphi(x)).
\end{equation}
Unfortunately $\sigma_1$ is not a strictly convex norm on $V^{\wedge k}$, so we introduce approximations:

\begin{definition}
Let $p \in [1, \infty)$.
The $p$th \dfn{Schatten-von Neumann norm} on $V^{\wedge k}$ is 
$$|\varphi|_p := \left(\sum_{j = 1}^{\binom dk} \sigma_j(\varphi)^p\right)^{1/p}.$$
We moreover define $|\varphi|_\infty := \sigma_1(\varphi)$.
\end{definition}

Since $|\varphi|_p$ is the $\ell^p$ norm of $(\sigma_1(\varphi), \dots, \sigma_{\binom dk}(\varphi))$, $p \in [1, \infty]$, and $\ell^p$ norms converge to the $\ell^\infty$ norm as $p \to \infty$,
\begin{equation}\label{SvN norms converge}
|\varphi|_\infty = \lim_{p \to \infty} |\varphi|_p.
\end{equation}
Since $\xi_1(\varphi), \dots, \xi_{\binom dk}(\varphi)$ is an orthonormal basis of $V^{\wedge k}$, the norm arising from the inner product on $V^{\wedge k}$ is exactly $|\cdot|_2$.
Applying H\"older's inequality to the atomic measure space $\{1, \dots, \binom dk\}$, we have for $1 \leq p \leq q \leq \infty$ that
\begin{equation}\label{relating Schatten norms}
|\varphi|_q \leq |\varphi|_p \leq \binom dk^{\frac{1}{p} - \frac{1}{q}} |\varphi|_q.
\end{equation}

In order to prove further properties of $|\cdot|_p$, we denote for a matrix $A \in \End(\RR^m)$,
$$Q(A) := \sqrt{AA^\dagger}.$$
We then have the \dfn{Schatten-von Neumann norm} $|\cdot|_{M, p}$ on matrices, which is the $\ell^p$ norm of their singular values.
To exploit this norm, let $(e_j)$ be an orthonormal basis of $V^{\wedge k}$, and $\varphi \in V^{\wedge k}$.
We introduce the matrix $A \in \End(\RR^{\binom dk})$,
$$A_{ij} := \sigma_i(\varphi) \langle \xi_i(\varphi), e_j\rangle,$$
so that $\sum_i A_{ij} = \langle \varphi, e_j\rangle$.
We also let $\Sigma$ be the diagonal matrix such that $\Sigma_{ii} := \sigma_i(\varphi)$,
and $\Xi_{ij} := \langle \xi_i(\varphi), e_j\rangle$.
Then $\Sigma$ is a positive semidefinite diagonal matrix with decreasing entries, $\Xi$ is an orthogonal matrix, and $A = \Sigma \Xi$.
Then $Q(A) = \Sigma$, the singular value decomposition of $A$ is 
$$A = I \Sigma \Xi,$$
and the singular values of $A$ equal the singular values of $\varphi$.
It follows that $|\cdot|_p$ is a norm.
We have 
\begin{equation}\label{derivative of matrix norm}
\frac{\partial}{\partial A_{ij}} |A|_{M, p}^p = p(Q(A)^{p - 2} A)_{ij} = p(\Sigma^{p - 1} \Xi)_{ij}.
\end{equation}
Multiplying both sides of (\ref{derivative of matrix norm}) by the basis vectors $e_j$, and then summing in $i$,
\begin{equation}\label{derivative of schatten norm}
\frac{\dif}{\dif \varphi} |\varphi|_p^p = p \sum_{i=1}^{\binom dk} \sigma_i(\varphi)^{p - 1} \xi_i(\varphi).
\end{equation}
Since $|\cdot|_{M, p}^p$ is a strictly convex function if $p \in (1, \infty)$, so is $|\cdot|_p^p$.

\begin{lemma}[von Neumann inequality]
Let $\psi, \varphi \in V^{\wedge k}$.
Then
\begin{equation}\label{von Neumann inequality}
|\langle \varphi, \psi\rangle| \leq \sum_{i=1}^{\binom dk} \sigma_i(\varphi) \sigma_i(\psi).
\end{equation}
\end{lemma}
\begin{proof}
Introduce the vectors $x_i := \sigma_i(\varphi)$, $y_i := \sigma_i(\psi)$, introduce the matrix $B_{ij} := \langle \xi_i(\varphi), \xi_j(\psi)\rangle$, and apply (\ref{permute to decreasing}).
\end{proof}

\begin{lemma}
Let $(p, q)$ be a H\"older pair. Then the norms $|\cdot|_p$ and $|\cdot|_q$ are dual.
\end{lemma}
\begin{proof}
Let $\varphi \in V^{\wedge k}$.
From von Neumann's inequality (\ref{von Neumann inequality}) and H\"older's inequality applied to the atomic measure space $\{1, \dots, \binom dk\}$, we have for every $\psi \in V^{\wedge k}$ that
$$|\langle \varphi, \psi\rangle| \leq |\varphi|_p |\psi|_q.$$
Thus 
\begin{equation}\label{dual norm inequality}
|\varphi|_p \geq \max_{\substack{\psi \in V^{\wedge k} \\ |\psi|_q \leq 1}} \langle \varphi, \psi\rangle.
\end{equation}
Conversely, if $p \in [1, \infty)$ and we take
$$\psi := |\varphi|_p^{1 - p} \sum_{i=1}^{\binom dk} \sigma_i(\varphi)^{p - 1} \xi_i(\varphi),$$
we see that the converse inequality to (\ref{dual norm inequality}) holds.
By taking limits we see that the result holds even if $p = \infty$.
\end{proof}

%%%%%%%%%%%%%%%%
\subsection{Currents of locally finite mass}
The mass of a $k$-current $T$ satisfies 
$$\Mass(T) = \sup_{\substack{\varphi \in C^\infty(M, \Omega^k) \\ \Comass(\varphi) \leq 1}} \int_M T \wedge \varphi.$$
If $T$ has locally finite mass, then we can view $\sigma_1(T) \star 1, \dots, \sigma_{\binom dk}(T) \star 1$ as positive Radon measures. \todo{Write out their definitions rigorously -- it's based on induction probably}.
Thus their sum $|T|_1 \star 1$ is also a positive Radon measure.
Then we have, by (\ref{infinity norm gives comass}) and the duality between the $|\cdot|_1$ and $|\cdot|_\infty$ norms,
$$\Mass(T) = \int_M |T| \star 1.$$

%%%%%%%%%%%%%%%%%%%%%%%%%%
\subsection{Anzellotti theory}
Let $\Omega^k$ ($\Omega^k_{\rm cl}$) be the sheaf of (closed) $k$-forms on $M$. 
If $\mathscr F$ is a sheaf and $\mathcal X$ is a function space that the components of sections of $\mathscr F$ can take values in, we write $\mathcal X(\cdot, \mathscr F)$ for the functor that sends an open set to the local sections of $\mathscr F$ with components in $\mathcal X$.

Let $\Omega_k$ denote the sheaf of $k$-currents.
We willl tacitly identify $L^q(\cdot, \Omega_k)$ with $L^q(\cdot, \Omega^{d - k})$, by setting, for every $\varphi \in L^p(\cdot, \Omega^k)$,
$$\langle T, \varphi\rangle = \int_M T \wedge \varphi.$$
In fact we write $\int_M T \wedge \varphi$ to mean $\langle T, \varphi\rangle$ even if $T$ cannot be identified with a form.


%%%%%%%%%%%%%%%%%%%%%%
\section{Existence of calibrations and calibrated currents}
\subsection{The \texorpdfstring{case $p < \infty$}{finite case}}
\begin{definition}
Let $p \in (1, \infty)$.
The \dfn{$p$-energy} is defined on $L^p(M, \Omega^k_{\rm cl})$ by 
$$J_p(\varphi) := \frac{1}{p} \int_M |\varphi|_p^p \star 1.$$
We say that $\varphi$ is a \dfn{$p$-tight $k$-form} if for every $\beta \in W^{1, p}(M, \Omega^{k - 1})$,
$$J_p(\varphi) \leq J_p(\varphi + \dif \beta).$$
\end{definition}

\begin{proposition}[Hodge theorem for $p$-tight forms]
Let $\rho \in H^k(M, \RR)$ and $p \in (1, \infty)$.
Then there is a unique $p$-tight $k$-form $\varphi$ representing $\rho$.
Moreover, $\varphi$ is the unique representative of $\rho$ which solves
\begin{equation}\label{first variation}
\dif^*\left[\sum_{i=1}^{\binom dk} \sigma_i(\varphi)^{p - 1} \xi_i(\varphi)\right] = 0.
\end{equation}
\end{proposition}
\begin{proof}
Let $\varphi \in L^p(M, \Omega^k_{\rm cl})$, $\gamma \in W^{1, p}(M, \Omega^{k - 1})$, and consider the one-parameter family of $k$-forms
$$\varphi_t := \varphi + t \dif \gamma.$$
Then, by (\ref{derivative of schatten norm}),
\begin{align*}
\frac{\dif}{\dif t} J_p(\varphi_t)
&= \frac{1}{p} \int_M \frac{\partial}{\partial t} |\varphi_t|_p^p \star 1
= \int_M \left\langle \sum_{i=1}^{\binom dk} \sigma_i(\varphi_t)^{p - 1} \xi_i(\varphi_t), \dif \gamma\right\rangle \star 1.
\end{align*}
Therefore if $\varphi$ is $p$-tight, (\ref{first variation}) holds.
Moreover, since $|\cdot|_p^p$ is strictly convex, 
$$\frac{\dif}{\dif t^2} J_p(\varphi_t) = \int_M \frac{\partial^2}{\partial t^2} |\varphi_t|_p^p \star 1 \geq 0$$
with equality iff $\varphi_t = 0$.
Therefore $J_p$ is a strictly convex functional on each cohomology class, with the Euler-Lagrange equation (\ref{first variation}).
It is moreover coercive on each cohomology class in $L^p(M, \Omega^k_{\rm cl})$, since $J_p(\varphi) \sim_p \|\varphi\|_{L^p}^p$.
Therefore by the direct method in the calculus of variations, the existence and uniqueness claims hold.
\end{proof}

\begin{definition}
Let $(p, q)$ be a H\"older pair, and let $\varphi$ be a $p$-tight $k$-form.
Let
$$T := -\frac{1}{p J_p(\varphi)} \sum_{i=1}^{\binom dk} \sigma_i(\varphi)^{p - 1} \star \xi_i(\varphi).$$
We call $T$ the \dfn{dual $k$-current} to $\varphi$.
\end{definition}

\begin{lemma}
Let $(p, q)$ be a H\"older pair, let $\varphi$ be a $p$-tight $k$-form, and let $T$ be the dual $k$-current to $\varphi$.
Then $T$ is a $q$-tight $d - k$-form such that 
$$\int_M T \wedge \varphi = 1.$$
\end{lemma}
\begin{proof}
By (\ref{first variation}), $\dif T = 0$.
It is clear from the definitions that
$$\sigma_i(T) = \sigma_i(\varphi)^{p - 1}.$$
Moreover, $(p - 1)(q - 1) = 1$, so 
$$p J_p(\varphi) \dif^*\left[\sum_{i=1}^{\binom dk} \sigma_i(T)^{p - 1} \xi_i(T)\right] = -\dif^*\left[\sum_{i=1}^{\binom dk} \sigma_i(\varphi) \star^{-1} \xi_i(\varphi)\right] = \pm \dif \varphi = 0$$
where the sign depends on the parities of $d$ and $k$.
Finally, we use the fact that $\xi_1(\varphi), \dots, \xi_{\binom dk}(\varphi)$ is an orthonormal basis of $(T^* M)^{\wedge k}$ to compute
\begin{align*}
\int_M T \wedge \varphi
&= -\frac{1}{p J_p(\varphi)} \int_M \sum_{i=1}^{\binom dk} \sigma_i(\varphi)^{p - 1} \star \xi_i(\varphi) \wedge \varphi \\
&= \frac{1}{p J_p(\varphi)} \int_M \sum_{i,j=1}^{\binom dk} \sigma_i(\varphi)^{p - 1} \sigma_j(\varphi) \langle\xi_i(\varphi), \xi_j(\varphi)\rangle \star 1 \\
&= \frac{1}{p J_p(\varphi)} \int_M \sum_{i=1}^{\binom dk} \sigma_i(\varphi)^p \star 1 \\
&= 1. \qedhere 
\end{align*}
\end{proof}

%%%%%%%%%%%%%%%
\subsection{The limit \texorpdfstring{$p \to \infty$}{p to infinity}}
We now fix a cohomology class $\rho \in H^k(M, \RR)$, with $\Comass(\rho) = 1$.
Thus we have $p$-tight $k$-forms $\varphi_p$ for every $p \in (1, \infty)$.
We will construct a limiting form $\varphi$.

\begin{lemma}
For any $2 \leq r \leq p < \infty$, one has the energy bound
\begin{equation}\label{energy bound}
J_r(\varphi_p) \leq \frac{1}{r} \binom dk (\vol M)^{r/p}.
\end{equation}
In particular, we have the norm bound 
\begin{equation}\label{norm bound}
\|\varphi_p\|_{L^r} \leq \binom dk^{1/2} (\vol M)^{1/p}.
\end{equation}
\end{lemma}
\begin{proof}
Let $\varepsilon > 0$ and choose some $\psi \in C^0(M, \Omega^k_{\rm cl})$ to represent $\rho$.
Since $\Comass(\rho) = 1$, we may assume $\Comass(\psi) \leq 1 + \varepsilon$.
Then, by (\ref{relating Schatten norms}),
$$|\psi|_p^p \leq \binom dk|\psi|_\infty^p \leq (1 + \varepsilon)^p \binom dk.$$
Therefore 
$$J_p(\varphi_p) \leq J_p(\psi) \leq \frac{1}{p} \binom dk \int_M (1 + \varepsilon)^p \star 1 = \frac{\vol M}{p} \binom dk (1 + \varepsilon)^p.$$
Taking $\varepsilon \to 0$ we conclude (\ref{energy bound}) if $r = p$.
Otherwise, we again apply (\ref{relating Schatten norms}) to estimate 
$$J_r(\varphi_p) \leq \frac{1}{r} \binom dk^{1 - r/p} \int_M |\varphi_p|_p^r \star 1 \leq \frac{1}{r} \binom dk^{1 - r/p} (pJ_p(\varphi_p))^{r/p}.$$
Plugging in (\ref{energy bound}) in the case $r = p$, we conclude (\ref{energy bound}) in the general case.
We can then prove (\ref{norm bound}) by applying (\ref{relating Schatten norms}) and (\ref{energy bound}):
\begin{align*}
\|\varphi_p\|_{L^r} &= \left(\int_M |\varphi_p|_2^r \star 1\right)^{1/r} \leq \binom dk^{1/2 - 1/r} (rJ_r(\varphi_p))^{1/r}. \qedhere 
\end{align*}
\end{proof}


\begin{proposition}
There exists $\varphi \in L^\infty(M, \Omega^k_{\rm cl})$ such that:
\begin{enumerate}
\item There exists a subsequence of $p \to \infty$ such that for every $r \in [1, \infty)$, $\varphi_p \weakto \varphi$.
\item $\Comass(\varphi) = 1$.
\item $[\varphi] = \rho$.
\end{enumerate}
\end{proposition}
\begin{proof}
By the norm bound (\ref{norm bound}) and Alaoglu's theorem, we can choose a subsequential limit of $\varphi$, which converges in the weak topology on $L^r$ for every $r \in [1, \infty)$.
By the lower semicontinuity of $J_r$, we can use the energy bound (\ref{energy bound}) to obtain 
$$J_r(\varphi)^{1/r} \leq r^{-1/r} \binom dk^{1/r}.$$
Taking $r \to \infty$, we see that $\Comass(\varphi) \leq 1$.

Now let $\sigma \in H_k(M, \RR)$, and choose a representative $T \in C^0(M, \Omega^{\rm cl}_k)$ of $\sigma$.
Since $\varphi_p \weakto \varphi$ in $L^2$,
$$\langle \sigma, [\varphi]\rangle = \int_M T \wedge \varphi = \lim_{p \to \infty} \int_M T \wedge \varphi_p = \langle \sigma, \rho\rangle.$$
Therefore $[\varphi] = \rho$; since $\Comass(\rho) = 1$, it follows that $\Comass(\varphi) \geq 1$.
\end{proof}

%%%%%%%%%%%%%%%%
\subsection{The limit \texorpdfstring{$q \to 1$}{q to 1}}

%%%%%%%%%%%%%%%%%%%%%%%
\section{Proving the main theorem}
\begin{proof}
Let $1 < p < \infty$.
We are going to define a strictly convex functional which approximates the comass, following Daskalopoulos and Uhlenbeck's approximation of the Lipschitz constant \cite{daskalopoulos2022}.

We first construct a singular value deomposition of a covector.
Let $\varphi$ be a $k$-covector at $x$ and let 
$$\sigma_1(\varphi) := \max_{v \in B_{x, k} M} \langle \varphi, v\rangle$$
and let $\xi_1(\varphi)$ be a $k$-blade which realizes the maximum.
Define 
$$\sigma_{n + 1}(\varphi) := \max_{\substack{v \in B_{x, k} M \\ v \perp \xi_1(\varphi), \dots, \xi_n(\varphi)}} \langle \varphi, v\rangle$$
and $\xi_{n + 1}(\varphi)$ a $k$-blade which realizes the maximum.
Thus we have $\sigma_n(\varphi) \geq \sigma_{n + 1}(\varphi)$.
We can then define the $p$th Schatten-von Neumann norm 
$$|\varphi|_p := \left(\sum_{n=1}^{\binom dk} \sigma_n(\varphi)^p\right)^{1/p}.$$
I think this quantity is a norm with strictly convex unit ball, since this is true of the Schatten-von Neumann norms on matrices.
This is the key step of the argument: \emph{we need to show that the we can approximate the pointwise comass by a norm with strictly convex unit ball}. \todo{Write it out}

We then introduce the $p$-energy
$$J_p(\varphi) := \int_M |\varphi(x)|_p^p \dif x$$
(where $\dif x$ is the Riemannian measure).
Then $J_p(\varphi) \sim \|\varphi\|_{L^p}^p$, hence $J_p$ is coercive on $L^p(M, \Omega^k_{\rm cl})$.
Moreover, since $|\varphi|_p$ has strictly convex unit ball, $J_p$ is strictly convex on each cohomology class in $L^p(M, \Omega^k_{\rm cl})$.
So by the direct method, for each $\rho \in H^k(M, \RR)$ there is a $\varphi_p \in L^p(M, \Omega^k_{\rm cl})$ representing $\rho$ which minimizes $J_p$.

If $\psi$ is cohomologous to $\varphi_p$ then by H\"older's inequality, applied first to $M$ and then to $\{1, \dots, \binom dk\}$ with its counting measure, we have
$$J_p(\varphi_p) \leq J_p(\psi) \leq \vol(M) \binom dk \esssup_{x \in M} \sigma_1(\psi(x))^p = \vol(M) \binom dk \Comass(\psi)^p.$$
In particular, $\|\varphi_p\|_{L^q}$ is bounded independently of $p \in (1, \infty)$ and $q \in (1, p]$.
Taking $p \to \infty$ and using Alaoglu's theorem, we see that for any $q \in (1, \infty)$, $\varphi_p \weakto \varphi$ along a subsequence in $L^q$.
We can then take the limits and use the fact that energy can only decrease along weak limits to deduce 
$$\Comass(\varphi) \leq \Comass(\psi).$$
Since $\psi$ was arbitrary, $\varphi$ minimizes the comass in $\rho$, hence $\Comass(\varphi) = 1$ and $\varphi$ is a calibration.
\end{proof}

Since $H^2(\CC \PP^2, \RR) \cong \RR$, I suspect that there is only one calibration $2$-form on $\CC \PP^2$, which is the Fubini-Study $2$-form.
This seems atypical for K\"ahler varieties, however, given the following.

\begin{corollary}
Let $M$ be a K3 surface. Then there is a calibration $2$-form $\varphi$ on $M$, which is not a K\"ahler $2$-form.
\end{corollary}
\begin{proof}
By Theorem \ref{existence of calibrations}, we can take $\varphi$ to represent the generator of $H^{2, 0}(M, \RR)$ after scaling it to have costable norm $1$.
\end{proof}


\appendix 
\section{Elementary inequalities}
% \begin{lemma}
% Suppose that $x, y \in \RR^m_+$ have decreasing entries, and for every $j \leq m$,
% \begin{equation}\label{Ky Fan gives SvN hyp}
% \sum_{i=1}^j x_j \leq \sum_{i=1}^j y_j.
% \end{equation}
% Then 
% \begin{equation}\label{Ky Fan gives SvN concl}
% \sum_{i=1}^m x_i^p \leq \sum_{i=1}^m y_i^p.
% \end{equation}
% \end{lemma}

\begin{lemma}
Suppose that $x, y \in \RR^m_+$ have decreasing entries, and $B \in \Orth(\RR^m)$. Then
\begin{equation}\label{permute to decreasing}
|\langle x, By\rangle| \leq \langle x, y\rangle.
\end{equation}
\end{lemma}
% https://math.stackexchange.com/questions/3075288/how-to-show-the-von-neumann-trace-inequality

\printbibliography

\end{document}
