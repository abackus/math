\documentclass[reqno,11pt]{amsart}
\usepackage[letterpaper, margin=1in]{geometry}
\RequirePackage{amsmath,amssymb,amsthm,graphicx,mathrsfs,url,slashed,subcaption}
\RequirePackage[usenames,dvipsnames]{xcolor}
\RequirePackage[colorlinks=true,linkcolor=Red,citecolor=Green]{hyperref}
\RequirePackage{amsxtra}
\usepackage{cancel}
\usepackage{tikz-cd}
%\usepackage[T1]{fontenc}

% \setlength{\textheight}{9.3in} \setlength{\oddsidemargin}{-0.25in}
% \setlength{\evensidemargin}{-0.25in} \setlength{\textwidth}{7in}
% \setlength{\topmargin}{-0.25in} \setlength{\headheight}{0.18in}
% \setlength{\marginparwidth}{1.0in}
% \setlength{\abovedisplayskip}{0.2in}
% \setlength{\belowdisplayskip}{0.2in}
% \setlength{\parskip}{0.05in}
%\renewcommand{\baselinestretch}{1.05}

\title{Many calibrations}
\author{Aidan Backus}
\address{Department of Mathematics, Brown University}
\email{aidan\_backus@brown.edu}
\date{\today}
\keywords{}
\subjclass[2020]{}

\newcommand{\NN}{\mathbf{N}}
\newcommand{\ZZ}{\mathbf{Z}}
\newcommand{\QQ}{\mathbf{Q}}
\newcommand{\RR}{\mathbf{R}}
\newcommand{\CC}{\mathbf{C}}
\newcommand{\DD}{\mathbf{D}}
\newcommand{\PP}{\mathbf P}
\newcommand{\MM}{\mathbf M}
\newcommand{\II}{\mathbf I}
\newcommand{\Hyp}{\mathbf H}
\newcommand{\Sph}{\mathbf S}
\newcommand{\Group}{\mathbf G}
\newcommand{\GL}{\mathbf{GL}}
\newcommand{\Orth}{\mathbf{O}}
\newcommand{\SpOrth}{\mathbf{SO}}
\newcommand{\Ball}{\mathbf{B}}

\newcommand*\dif{\mathop{}\!\mathrm{d}}

\DeclareMathOperator{\card}{card}
\DeclareMathOperator{\dist}{dist}
\DeclareMathOperator{\id}{id}
\DeclareMathOperator{\Hom}{Hom}
\DeclareMathOperator{\coker}{coker}
\DeclareMathOperator{\supp}{supp}
\DeclareMathOperator{\Teich}{Teich}
\DeclareMathOperator{\tr}{tr}

\newcommand{\Leaves}{\mathscr L}
\newcommand{\Lagrange}{\mathcal L}
\newcommand{\Hypspace}{\mathscr H}

\newcommand{\Chain}{\underline C}

\newcommand{\Two}{\mathrm{I\!I}}

\newcommand{\normal}{\mathbf n}
\newcommand{\radial}{\mathbf r}
\newcommand{\evect}{\mathbf e}
\newcommand{\vol}{\mathrm{vol}}

\newcommand{\diam}{\mathrm{diam}}
\newcommand{\Ell}{\mathrm{Ell}}
\newcommand{\inj}{\mathrm{inj}}
\newcommand{\Lip}{\mathrm{Lip}}
\newcommand{\MCL}{\mathrm{MCL}}
\newcommand{\Riem}{\mathrm{Riem}}

\newcommand{\Mass}{\mathbf M}
\newcommand{\Comass}{\mathbf L}

\newcommand{\weakto}{\rightharpoonup}

\newcommand{\Min}{\mathrm{Min}}
\newcommand{\Max}{\mathrm{Max}}

\newcommand{\dfn}[1]{\emph{#1}\index{#1}}

\renewcommand{\Re}{\operatorname{Re}}
\renewcommand{\Im}{\operatorname{Im}}

\newcommand{\loc}{\mathrm{loc}}
\newcommand{\cpt}{\mathrm{cpt}}

\def\Japan#1{\left \langle #1 \right \rangle}

\newtheorem{theorem}{Theorem}
\newtheorem{badtheorem}[theorem]{``Theorem"}
\newtheorem{prop}[theorem]{Proposition}
\newtheorem{lemma}[theorem]{Lemma}
\newtheorem{sublemma}[theorem]{Sublemma}
\newtheorem{proposition}[theorem]{Proposition}
\newtheorem{corollary}[theorem]{Corollary}
\newtheorem{conjecture}[theorem]{Conjecture}
\newtheorem{axiom}[theorem]{Axiom}
\newtheorem{assumption}[theorem]{Assumption}

\newtheorem{mainthm}{Theorem}
\renewcommand{\themainthm}{\Alph{mainthm}}

\newtheorem{claim}{Claim}[theorem]
\renewcommand{\theclaim}{\thetheorem\Alph{claim}}
% \newtheorem*{claim}{Claim}

\theoremstyle{definition}
\newtheorem{definition}[theorem]{Definition}
\newtheorem{remark}[theorem]{Remark}
\newtheorem{example}[theorem]{Example}
\newtheorem{notation}[theorem]{Notation}

\newtheorem{exercise}[theorem]{Discussion topic}
\newtheorem{homework}[theorem]{Homework}
\newtheorem{problem}[theorem]{Problem}

\makeatletter
\newcommand{\proofpart}[2]{%
  \par
  \addvspace{\medskipamount}%
  \noindent\emph{Part #1: #2.}
}
\makeatother

\DeclareMathOperator*{\essinf}{ess\,inf}
\DeclareMathOperator*{\esssup}{ess\,sup}


\numberwithin{equation}{section}


% Mean
\def\Xint#1{\mathchoice
{\XXint\displaystyle\textstyle{#1}}%
{\XXint\textstyle\scriptstyle{#1}}%
{\XXint\scriptstyle\scriptscriptstyle{#1}}%
{\XXint\scriptscriptstyle\scriptscriptstyle{#1}}%
\!\int}
\def\XXint#1#2#3{{\setbox0=\hbox{$#1{#2#3}{\int}$ }
\vcenter{\hbox{$#2#3$ }}\kern-.6\wd0}}
\def\ddashint{\Xint=}
\def\dashint{\Xint-}

\usepackage[backend=bibtex,style=alphabetic,giveninits=true]{biblatex}
\renewcommand*{\bibfont}{\normalfont\footnotesize}
\addbibresource{best_curl.bib}
\renewbibmacro{in:}{}
\DeclareFieldFormat{pages}{#1}

\newcommand\todo[1]{\textcolor{red}{TODO: #1}}


\begin{document}
\maketitle

Let $M$ be a closed Riemannian manifold of dimension $d \geq 2$, and let $k \in \{1, \dots, d - 1\}$.
A \dfn{$k$-blade} at $x \in M$ is a wedge product of the form $v := v_1 \wedge \dots \wedge v_n$, $v_i \in T_x M$; in this case we write $v \in B_{x, k} M$.
The \dfn{comass} of a $k$-form $\varphi$ at $x$ is 
$$\Comass(\varphi) = \esssup_{x \in M} \max_{v \in B_{x, k} M} \langle \varphi(x), v\rangle.$$
The \dfn{costable norm} of a cohomology class $\rho \in H^k(M, \RR)$ is
$$\Comass(\rho) := \inf_{[\varphi] = \rho} \Comass(\varphi).$$
I think that since $M$ is closed, the costable norm really is a (positive-definite) norm on $H^k(M, \RR)$.

\begin{theorem}\label{existence of calibrations}
Let $\rho \in H^k(M, \RR)$ satisfy $\Comass(\rho) = 1$.
Then there is a measurable calibration which represents $\rho$.
\end{theorem}
\begin{proof}
Let $1 < p < \infty$.
We are going to define a strictly convex functional which approximates the comass, following Daskalopoulos and Uhlenbeck's approximation of the Lipschitz constant \cite{daskalopoulos2022}.

We first construct a singular value deomposition of a covector.
Let $\varphi$ be a $k$-covector at $x$ and let 
$$\sigma_1(\varphi) := \max_{v \in B_{x, k} M} \langle \varphi, v\rangle$$
and let $\xi_1(\varphi)$ be a $k$-blade which realizes the maximum.
Define 
$$\sigma_{n + 1}(\varphi) := \max_{\substack{v \in B_{x, k} M \\ v \perp \xi_1(\varphi), \dots, \xi_n(\varphi)}} \langle \varphi, v\rangle$$
and $\xi_{n + 1}(\varphi)$ a $k$-blade which realizes the maximum.
Thus we have $\sigma_n(\varphi) \geq \sigma_{n + 1}(\varphi)$.
We can then define the $p$th Schatten-von Neumann norm 
$$|\varphi|_p := \left(\sum_{n=1}^{\binom dk} \sigma_n(\varphi)^p\right)^{1/p}.$$
I think this quantity is a norm with strictly convex unit ball, since this is true of the Schatten-von Neumann norms on matrices.
This is the key step of the argument: \emph{we need to show that the we can approximate the pointwise comass by a norm with strictly convex unit ball}. \todo{Write it out}

We then introduce the $p$-energy
$$J_p(\varphi) := \int_M |\varphi(x)|_p^p \dif x$$
(where $\dif x$ is the Riemannian measure).
Then $J_p(\varphi) \sim \|\varphi\|_{L^p}^p$, hence $J_p$ is coercive on $L^p(M, \Omega^k_{\rm cl})$.
Moreover, since $|\varphi|_p$ has strictly convex unit ball, $J_p$ is strictly convex on each cohomology class in $L^p(M, \Omega^k_{\rm cl})$.
So by the direct method, for each $\rho \in H^k(M, \RR)$ there is a $\varphi_p \in L^p(M, \Omega^k_{\rm cl})$ representing $\rho$ which minimizes $J_p$.

If $\psi$ is cohomologous to $\varphi_p$ then by H\"older's inequality, applied first to $M$ and then to $\{1, \dots, \binom dk\}$ with its counting measure, we have
$$J_p(\varphi_p) \leq J_p(\psi) \leq \vol(M) \binom dk \esssup_{x \in M} \sigma_1(\psi(x))^p = \vol(M) \binom dk \Comass(\psi)^p.$$
In particular, $\|\varphi_p\|_{L^q}$ is bounded independently of $p \in (1, \infty)$ and $q \in (1, p]$.
Taking $p \to \infty$ and using Alaoglu's theorem, we see that for any $q \in (1, \infty)$, $\varphi_p \weakto \varphi$ along a subsequence in $L^q$.
We can then take the limits and use the fact that energy can only decrease along weak limits to deduce 
$$\Comass(\varphi) \leq \Comass(\psi).$$
Since $\psi$ was arbitrary, $\varphi$ minimizes the comass in $\rho$, hence $\Comass(\varphi) = 1$ and $\varphi$ is a calibration.
\end{proof}

Since $H^2(\CC \PP^2, \RR) \cong \RR$, I suspect that there is only one calibration $2$-form on $\CC \PP^2$, which is the Fubini-Study $2$-form.
This seems atypical for K\"ahler varieties, however, given the following.

\begin{corollary}
Let $M$ be a K3 surface. Then there is a calibration $2$-form $\varphi$ on $M$, which is not a K\"ahler $2$-form.
\end{corollary}
\begin{proof}
By Theorem \ref{existence of calibrations}, we can take $\varphi$ to represent the generator of $H^{2, 0}(M, \RR)$ after scaling it to have costable norm $1$.
\end{proof}


\printbibliography

\end{document}
