\documentclass[reqno,11pt]{amsart}
\usepackage[letterpaper, margin=1in]{geometry}
\RequirePackage{amsmath,amssymb,amsthm,graphicx,mathrsfs,url,slashed,subcaption}
\RequirePackage[usenames,dvipsnames]{xcolor}
\RequirePackage[colorlinks=true,linkcolor=Red,citecolor=Green]{hyperref}
\RequirePackage{amsxtra}
\usepackage{cancel}
\usepackage{tikz-cd}
%\usepackage[T1]{fontenc}

% \setlength{\textheight}{9.3in} \setlength{\oddsidemargin}{-0.25in}
% \setlength{\evensidemargin}{-0.25in} \setlength{\textwidth}{7in}
% \setlength{\topmargin}{-0.25in} \setlength{\headheight}{0.18in}
% \setlength{\marginparwidth}{1.0in}
% \setlength{\abovedisplayskip}{0.2in}
% \setlength{\belowdisplayskip}{0.2in}
% \setlength{\parskip}{0.05in}
%\renewcommand{\baselinestretch}{1.05}

\title{Minimal laminations and tight forms in hyperbolic manifolds}
\author{Aidan Backus}
\address{Department of Mathematics, Brown University}
\email{aidan\_backus@brown.edu}
\date{\today}

\newcommand{\NN}{\mathbf{N}}
\newcommand{\ZZ}{\mathbf{Z}}
\newcommand{\QQ}{\mathbf{Q}}
\newcommand{\RR}{\mathbf{R}}
\newcommand{\CC}{\mathbf{C}}
\newcommand{\DD}{\mathbf{D}}
\newcommand{\PP}{\mathbf P}
\newcommand{\MM}{\mathbf M}
\newcommand{\II}{\mathbf I}
\newcommand{\Hyp}{\mathbf H}
\newcommand{\Sph}{\mathbf S}
\newcommand{\Group}{\mathbf G}
\newcommand{\GL}{\mathbf{GL}}
\newcommand{\Orth}{\mathbf{O}}
\newcommand{\SpOrth}{\mathbf{SO}}
\newcommand{\Ball}{\mathbf{B}}

\newcommand*\dif{\mathop{}\!\mathrm{d}}

\DeclareMathOperator{\card}{card}
\DeclareMathOperator{\dist}{dist}
\DeclareMathOperator{\id}{id}
\DeclareMathOperator{\supp}{supp}
\DeclareMathOperator{\Teich}{Teich}
\DeclareMathOperator{\tr}{tr}

\newcommand{\Leaves}{\mathscr L}
\newcommand{\Lagrange}{\mathcal L}
\newcommand{\Hypspace}{\mathscr H}

\newcommand{\Chain}{\underline C}

\newcommand{\Two}{\mathrm{I\!I}}

\newcommand{\normal}{\mathbf n}
\newcommand{\radial}{\mathbf r}
\newcommand{\evect}{\mathbf e}
\newcommand{\vol}{\mathrm{vol}}

\newcommand{\diam}{\mathrm{diam}}
\newcommand{\Ell}{\mathrm{Ell}}
\newcommand{\inj}{\mathrm{inj}}
\newcommand{\Lip}{\mathrm{Lip}}
\newcommand{\Riem}{\mathrm{Riem}}

\newcommand{\Min}{\mathrm{Min}}
\newcommand{\Max}{\mathrm{Max}}

\newcommand{\dfn}[1]{\emph{#1}\index{#1}}

\renewcommand{\Re}{\operatorname{Re}}
\renewcommand{\Im}{\operatorname{Im}}

\newcommand{\loc}{\mathrm{loc}}
\newcommand{\cpt}{\mathrm{cpt}}

\def\Japan#1{\left \langle #1 \right \rangle}

\newtheorem{theorem}{Theorem}[section]
\newtheorem{badtheorem}[theorem]{``Theorem"}
\newtheorem{prop}[theorem]{Proposition}
\newtheorem{lemma}[theorem]{Lemma}
\newtheorem{sublemma}[theorem]{Sublemma}
\newtheorem{proposition}[theorem]{Proposition}
\newtheorem{corollary}[theorem]{Corollary}
\newtheorem{conjecture}[theorem]{Conjecture}
\newtheorem{axiom}[theorem]{Axiom}
\newtheorem{assumption}[theorem]{Assumption}

\newtheorem{mainthm}{Theorem}
\renewcommand{\themainthm}{\Alph{mainthm}}

\newtheorem{claim}{Claim}[theorem]
\renewcommand{\theclaim}{\thetheorem\Alph{claim}}
% \newtheorem*{claim}{Claim}

\theoremstyle{definition}
\newtheorem{definition}[theorem]{Definition}
\newtheorem{remark}[theorem]{Remark}
\newtheorem{example}[theorem]{Example}
\newtheorem{notation}[theorem]{Notation}

\newtheorem{exercise}[theorem]{Discussion topic}
\newtheorem{homework}[theorem]{Homework}
\newtheorem{problem}[theorem]{Problem}

\makeatletter
\newcommand{\proofpart}[2]{%
  \par
  \addvspace{\medskipamount}%
  \noindent\emph{Part #1: #2.}
}
\makeatother



\numberwithin{equation}{section}


% Mean
\def\Xint#1{\mathchoice
{\XXint\displaystyle\textstyle{#1}}%
{\XXint\textstyle\scriptstyle{#1}}%
{\XXint\scriptstyle\scriptscriptstyle{#1}}%
{\XXint\scriptscriptstyle\scriptscriptstyle{#1}}%
\!\int}
\def\XXint#1#2#3{{\setbox0=\hbox{$#1{#2#3}{\int}$ }
\vcenter{\hbox{$#2#3$ }}\kern-.6\wd0}}
\def\ddashint{\Xint=}
\def\dashint{\Xint-}

\usepackage[backend=bibtex,style=alphabetic,giveninits=true]{biblatex}
\renewcommand*{\bibfont}{\normalfont\footnotesize}
\addbibresource{best_curl.bib}
\renewbibmacro{in:}{}
\DeclareFieldFormat{pages}{#1}

\newcommand\todo[1]{\textcolor{red}{TODO: #1}}


\begin{document}
\begin{abstract}
We introduce a family of closed $d-1$-forms on Riemannian $d$-manifolds which minimize their comass (or $L^\infty$ norm) in their cohomology class, which we call \dfn{tight}.
Tight forms have properties similar to (gradients of) best Lipschitz maps: they are convex duals of $1$-harmonic functions and attain their comass on a measured oriented minimal lamination $\mu$.
We show that $\mu$ has properties analogous to Thurston's canonical lamination.
\end{abstract}

\maketitle

%%%%%%%%%%%%%%%%%%%%%%%%%%%%%%%%%%%%%%%%%%%%%%%%%%%%%%%
\section{Introduction}
In this paper we shall propose a generalization of Thurston's Teichm\"uller theory to a broader class of manifolds than hyperbolic surfaces.
To motivate this, let us first recall Thurston's Teichm\"uller theory.

Let $M$ be a closed Riemannian $d$-fold, $N$ another Riemannian manifold, and $\rho \in [M, N]$ a homotopy class.
A map $f: M \to N$ of class $\rho$ is \dfn{best Lipschitz} if it minimizes its Lipschitz constant, or maximum stretch,
$$\Lip(f) := \sup_{x, y \in M} \frac{\dist(f(x), f(y))}{\dist(x, y)},$$
among all maps in $\rho$.
If $M, N$ are closed hyperbolic surfaces with the same underlying topological space $S$, $\rho$ is the homotopy class of the identity $\id_S$, and $L(M, N)$ the best Lipschitz constant of $\rho$, then $\log L$ is a Finsler metric on Teichm\"uller space $\Teich(S)$, called \dfn{Thurston's asymmetric metric} \cite{Thurston98, Papadopoulos15}.
Thurston's asymmetric metric is a particularly appealing geometry on Teichm\"uller space because of its intimate connection with the structure of geodesic laminations on $M, N$ \cite{Gu_ritaud_2017}:

\begin{theorem}
Let $S$ be a closed oriented surface of genus $\geq 2$, $M, N \in \Teich(S)$, and $\mathcal F$ the set of best Lipschitz maps $M \to N$ homotopic to $\id_S$.
Then there exists a measured oriented geodesic lamination $\mu$ such that:
\begin{enumerate}
\item For $f \in \mathcal F$, let $\lambda_f$ be the set on which $f$ attains its Lipschitz constant. Then $\lambda_f$ is (the support of) a geodesic lamination.
\item Let $\lambda$ ranges over the set of measured laminations. Then
\begin{equation}\label{L is K}
L(M, N) = \sup_\lambda \frac{|\lambda|_N}{|\lambda|_M} = \frac{|\mu|_N}{|\mu|_M}.
\end{equation}
\item For $f \in \mathcal F$, $\mu$ is a sublamination of $\lambda_f$.
\end{enumerate}
\end{theorem}

The lamination $\mu$ is called \dfn{Thurston's canonical lamination} associated to $M, N$, or more generally to the given homotopy class.

Daskalopolous--Uhlenbeck have connected Thurston's Teichm\"uller theory to $p$-elliptic PDE, by identifying a particularly nice class of best Lipschitz maps: the $\infty$-harmonic maps \cite{daskalopoulos2022}.
The highlight of this approach is that there exists a dual $1$-harmonic function $u$ to $f$, which can be either predicted using convex duality or Noether's theorem on conserved currents; the derivative $\dif u$ induces the transverse measure on Thurston's canonical lamination.
This ``duality'' approach was conjectured by Thurston, who wrote \cite{Thurston98}:

\begin{quote}
I currently think that a characterization of minimal stretch\footnote{that is, best Lipschitz} maps should be possible in a considerably more general context ... and it should be feasible with a simpler proof based on more general principles -- in particular, the max flow min cut principle, convexity, and $L^0 \leftrightarrow L^\infty$ duality.
\end{quote}

We should like to generalize Thurston's Teichm\"uller theory to apply not just to geodesic laminations on surfaces, but more general minimal laminations on more general manifolds.
In Thurston's case, the main point is to associate a geodesic lamination in $M$ to each homotopy class of maps $\rho \in [M, N]$, by considering the set on which a best Lipschitz representative of $\rho$ attains its Lipschitz constant.
We can use the Hurcewiz theorem to identify $[M, \Sph^1] \cong H^1(M)$, and so we can view $H^\ell(M)$ as homotopy classes of ``higher $\Sph^1$-valued maps'' in some sense.
This makes for a very nice setting for a warm-up generalization: in this paper, we shall associate to each class in $H^{d - 1}(M)$ a codimension-$1$ minimal lamination, on which a certain differential form attains its $L^\infty$ norm.
One expects from experience that the theory should be easier to develop in this setting than the general case.
Indeed, Daskalopolous--Uhlenbeck developed an analogue of Teichm\"uller theory for $\Sph^1$-valued maps before turning to maps between surfaces \cite{daskalopoulos2020transverse}.
Moreover, it is not quite clear what a ``higher map'' $M \to N$ should mean, except that the topological type of a higher map should induce a minimal lamination on $M$.

In order to construct the minimal lamination associated to a class in $H^{d - 1}(M)$, we recall the following \todo{cite} \cite{Mazon14}

\begin{definition}
A $\pi_1(M)$-equivariant function $u \in BV_\loc(\tilde M)$ is called \dfn{$1$-harmonic}, or \dfn{least gradient}, if
$$\dif^* \left(\frac{\dif u}{|\dif u|}\right) = 0,$$
in the very weak sense that there exists an $L^\infty$ divergence-free vector field $X$ such that $\|X\|_{L^\infty} \leq 1$ and
$$|\dif u| = (\dif u, X)$$
in the sense of Radon measures.
\end{definition}

\begin{theorem}
Let $u$ be a $1$-harmonic function on a Riemannian manifold of constant sectional curvature and dimension $2 \leq d \leq 4$.
Then there exists a measured oriented minimal lamination $\lambda_u$, whose leaves are the level sets $\partial \{u > y\}$ of $u$, and whose Ruelle-Sullivan current is $\dif u$.
\end{theorem}

In particular it will suffice to find a $1$-harmonic function associated to each class in $H^{d - 1}(M)$.
The convex dual problem to the (equivariant) $1$-Laplace equation is locally the problem of minimizing $\|F\|_{L^\infty}$ among all $d-1$-forms $F$ in a particular cohomology class, so it is natural to select a minimizer $F$ and its $1$-harmonic dual $u$.
For $d = 2$, we have $F = \dif f$ for a best Lipschitz map $f: M \to \Sph^1$.

Unfortunately, the $L^\infty$ norm is not well-behaved locally; among other issues, an $L^\infty$ form need only attain its $L^\infty$ norm on a measurable set, while we would like it to attain its $L^\infty$ norm on a closed set (which in our application should contain the support of the lamination).
Thus we introduce the \dfn{comass}
\begin{equation}\label{comass}
L(F) := \sup_{\sigma \in \Chain_{d - 1}(M)} \frac{1}{|\sigma|} \int_\sigma F
\end{equation}
of a closed $d-1$-form $F$ in a closed Riemannian $d$-fold $M$.
Here $\Chain_{d - 1}(M)$ denotes the space of oriented $d - 1$-chains in $M$, and $|\sigma|$ is the $d-1$-area of $\sigma$.
Globally, the comass is equal to the $L^\infty$ norm of $F$, but locally it is better behaved.
We study minimizers of the comass, which we call \dfn{best comass} forms, in a given cohomology class.

\subsection{\texorpdfstring{$\infty$-tight forms and $1$-harmonic functions}{Infinity-tight forms and one-harmonic functions}}
Our first main task is to construct best comass forms which are analogous to $\infty$-harmonic functions, or equivalently to limits of $p$-harmonic functions.
Let $d < p < \infty$ and $\frac{1}{p} + \frac{1}{q} = 1$.
Motivated by the $p$-Laplace equation $\dif^*(|\dif f|^{p - 2} \dif f) = 0$, we introduce \dfn{$p$-tight} forms, which are closed $d-1$-forms which solve the system of PDE
$$\dif^*(|F|^{p - 2} F) = 0.$$
Given a $p$-tight form, the $\pi_1(M)$-equivariant function $u$ on the universal cover such that
$$\dif u = (-1)^{d - 1} |F|^{p - 2} \star F$$
is $q$-harmonic -- in other words, $u$ is a solution of the $q$-Laplace equation 
$$\dif^*(|\dif u|^{q - 2} \dif u) = 0.$$
Our first theorem constructs a best comass form, and a dual $1$-harmonic function, by taking limits of $p$-tight forms and their dual $q$-harmonic functions.

\begin{mainthm}\label{existence of infinity tight forms}
Let $\rho \in H^{d - 1}(M, \RR)$ be a cohomology class.
Let $(F_p, u_q)$ be the family of dual pairs of $p$-tight forms and $q$-harmonic functions, suitably normalized, with $[F_p] = \rho$.
Then there exists a pair $(F, u)$ such that as $p \to \infty$, $F_p \to F$ weakly in $L^r$ for any $d < r < \infty$, and $u_q \to u$ weakly in $BV$.
Moreover, $F$ has best comass in $\rho$, $u$ is $1$-harmonic, and we have the duality relation 
\begin{equation}\label{max flow mean cut}
L|\dif u| = \langle \dif u, \star F\rangle
\end{equation}
in the sense of Radon measures, where $L$ is the best comass of a representative of $\rho$.
\end{mainthm}

This is a combination of Propositions \ref{existence infinity} and \ref{existence 1}.
We call the best comass form $F$ an \dfn{$\infty$-tight} form, or simply a \dfn{tight} form.

Most of this theorem essentially follows from the methods of \cite{Mazon14,daskalopoulos2020transverse}, but we highlight (\ref{max flow mean cut}) as the main point of the theorem.
It has multiple interpretations:
\begin{enumerate}
\item Since (\ref{max flow mean cut}) asserts a form of convex duality between $\infty$-tight forms and $1$-harmonic functions, we can view it as the analogue of the max flow min cut principle alluded to by Thurston. We elaborate on this point in Appendix \ref{Max Flow Min Cut}.
\item Since (\ref{max flow mean cut}) is exactly the assertion that $F/L$ is the Poincar\'e dual to a vector field $X$ which witnesses that $u$ is $1$-harmonic, this duality relation allows us to easily prove that $u$ is $1$-harmonic without carrying out a careful analysis of the limiting behavior of the $q$-Laplacian or $p$-tight forms as in \cite[Theorem 2.4]{Mazon14} or \cite[\S6]{daskalopoulos2020transverse}.
\item Modulo technicalities arising from geometric measure theory, one can interpret (\ref{max flow mean cut}) to mean that $F/L$ is the area form on the level sets of $u$. This last point we shall repreatedly use throughout the remainder of the paper.
\end{enumerate}

%%%%%%%%%%%%%%%%%%

\subsection{Calibration of the Thurston lamination}
Let $F$ be a closed $d-1$-form on the Riemannian $d$-fold $M$.
A hypersurface $N \subset M$ is $F$-\dfn{calibrated} if it has comass $1$, and the pullback of $F$ to $N$ is the area form on $N$ \cite{Harvey82}.
In that case, the mean curvature of $N$ is 
\begin{equation}\label{calibrated surfaces are minimal}
H_N = \nabla \cdot \normal_N = \nabla \cdot (\star F)^\sharp = \star \dif F = 0,
\end{equation}
so that $N$ is minimal. 

If a lamination $\lambda$ is $F$-calibrated (in the sense that its leaves are $F$-calibrated), then it is clear from the definitions that $\supp \lambda$ is contained in the locus on which $F$ attains its comass.
In the other direction, Bangert--Cui showed that continuous minimizers of the comass are calibrations of laminations \cite{bangert_cui_2017}:

\begin{theorem}
Let $F$ be a continuous calibration of best comass on a closed Riemannian manifold of dimension $2 \leq d \leq 7$.
Then there exists a measured oriented lamination $\lambda$ which is $F$-calibrated.
In particular, $\lambda$ is minimal and $\supp \lambda$ is contained in the set on which $F$ attains its comass.
\end{theorem}

The proof of \cite{bangert_cui_2017} requires the use of the Hanh-Banach theorem, and so the $F$-calibrated lamination exists abstractly but lacks a nice characterization. \todo{Make sure they didn't characterize it elsewhere}
We would like to characterize the maximum stretch lamination as arising from the dual $1$-harmonic function; in particular, it should not depend on the calibration, but only on its cohomology class.
Moreover, it should not require the calibration to be continuous, since $\infty$-tight forms need not be continuous (indeed, for $d = 2$, $\infty$-tight forms are gradients of $\infty$-harmonic functions, which need not be $C^1$).
We suspect that these proposed strengthenings can be already carried out by modifying the argument of \cite{bangert_cui_2017}, but they also easily follow from (\ref{max flow mean cut}), so we give a new proof using (\ref{max flow mean cut}).

\begin{definition}
Let $M$ be a closed space form of dimension $2 \leq d \leq 4$, and $\rho \in H^{d - 1}(M, \RR)$.
We can define a measured oriented minimal lamination $\mu$, by considering an $\infty$-tight representative $F$ of $\rho$, letting $u$ be the dual $1$-harmonic function to $F$, and letting $\mu$ be the lamination induced by $u$.
We call $\mu$ a \dfn{Thurston lamination} associated to $\rho$.
\end{definition}

By \cite[Remark 2.8]{Mazon14}, it is likely that as an oriented (nonmeasured) lamination, the Thurston lamination is uniquely determined by $\rho$.
To prove this would require stronger uniqueness theory for the $1$-Laplacian than is currently available, so we shall not attempt to show this here.

Our next theorem is Theorem \ref{MCL contains Thurston}.
It easily follows from (\ref{max flow mean cut}) if we take it as given that $1$-harmonic functions induce minimal laminations.

\begin{mainthm}\label{lams are calibrated}
Suppose that $M$ is a closed space form of dimension $2 \leq d \leq 4$.
Let $\mu$ be a Thurston lamination associated to $\rho \in H^{d - 1}(M, \RR)$, and let $F$ be a form of best comass representing $\rho$, with comass $L$.
Then $\mu$ is $F/L$-calibrated.
\end{mainthm}

Finally, we use (\ref{max flow mean cut}) to characterize the Thurston lamination as a maximally stretched lamination in the sense of (\ref{L is K}).
This is Theorem \ref{L equals K}.

\begin{mainthm}\label{LK}
Suppose that $M$ is a closed space form of dimension $2 \leq d \leq 4$.
Let $\rho \in H^{d - 1}(M, \RR)$, let $L$ be the best comass constant of $\rho$, and let $\mu$ be a Thurston lamination of $\rho$. Then, for $\lambda$ ranging over measured oriented laminations,
$$L = \sup_\lambda \frac{\langle \rho, [\lambda]\rangle}{|\lambda|} = \frac{\langle \rho, [\mu]\rangle}{|\mu|}$$
where $[\cdot]$ denotes the homology class of a lamination.
\end{mainthm}

% %%%%%%%%%%%%%%%%%%%%%%%%
% \subsection{A gauge-theoretic interpretation}
% In order to understand the analogy between best comass $2$-forms on threefolds and best Lipschitz functions on surfaces, it is convenient to introduce the formalism of gauge theory.
% We shall not need this in the sequel, but include it for the reader's interest and intuition.

% We begin with the $\Sph^1$-valued $L = K$ theorem of Daskalopolous--Uhlenbeck \cite[Theorem 5.8]{daskalopoulos2020transverse}.
% To state it, recall that if $\rho \in H^1(M, \ZZ)$, then $\rho$ defines a morphism of groups
% $$\rho: \pi_1(M) \to \ZZ = \pi_1(\Sph^1)$$
% by the Hurcewiz theorem.
% Since $\Sph^1$ is aspherical, it follows that $H^1(M, \ZZ)$ can be viewed as the space $[M, \Sph^1]$ of homotopy classes of maps $M \to \Sph^1$.

% By \cite[\S2.1]{daskalopoulos2020transverse}, if $F$ is a best comass representative of $\rho$, then there exists a map $A: M \to \Sph^1$ such that $F = \dif A$ (and $A$ has homotopy class $\rho$).
% Since $F$ has best comass, $A$ is best Lipschitz.
% There is a gauge ambiguity caused by the symmetry of $\Sph^1$: \cite{daskalopoulos2020transverse} treats two maps $A, A'$ as the same if $A - A'$ is a constant angle.
% Of especial interest among the best Lipschitz maps are the $\infty$-harmonic maps $M \to \Sph^1$, which are obtained by taking the limit of $p$-harmonic functions as $p \to \infty$.

% \begin{theorem}[Daskalopolous--Uhlenbeck]
% Let $M$ be a closed hyperbolic surface and $\rho \in H^1(M, \ZZ)$.
% Let $L$ be the best Lipschitz constant of the homotopy class $\rho$.
% Identify $\gamma \in \pi_1(M) \setminus \{0\}$ with its geodesic representative.
% Let
% $$K := \sup_{\gamma \in \pi_1(M) \setminus \{0\}} \frac{\langle \rho, \gamma\rangle}{|\gamma|}.$$
% Then $L = K$.
% \end{theorem}

% Since geodesics are dense in the space of measured oriented geodesic laminations, we can replace the supremum in the definition of $K$ with a supremum over measured oriented geodesic laminations; then, since the supremum over measured oriented laminations is clearly attained by a \emph{geodesic} such lamination, we can replace $K$ with 
% $$K = \sup_\lambda \frac{\langle \rho, [\lambda]\rangle}{|\lambda|}$$
% where $\lambda$ ranges over measured oriented (but possibly nongeodesic) laminations.
% Thus Theorem \ref{L is K} is a generalization of \cite[Theorem 5.8]{daskalopoulos2020transverse}, where we take $d = 2$ and $\rho$ an integral class.

% If $d = 3$ or $d = 4$, it is not so clear how to identify $H^{d - 1}(M, \ZZ)$ with a homotopy class of maps.
% However, for $d = 3$ it is natural to view $H^2(M, \ZZ)$ as the space of line bundles on $M$, by identifying a line bundle $\mathscr L_\rho$ with its Chern class $\rho = c_1(\mathscr L_\rho)$.
% As such, we fix an integral class $\rho \in H^2(M, \ZZ)$, and let $F$ be a best comass representative of $\rho$.
% Then there locally exists a $1$-form $A$ such that $\dif A = F$.
% As in \cite{daskalopoulos2020transverse}, there is a gauge ambiguity, and we must identify two $1$-forms if they differ by a closed $1$-form.

% If $F$ is $\infty$-tight, then we can find approximations $A_p$ which solve the \dfn{$p$-Maxwell equation}
% $$\dif^* (|\dif A_p|^{p - 2} \dif A_p) = 0.$$
% The $1$-forms $A_p$ and their limit $A$ can be viewed as global objects, namely (the Christoffel symbols of globally defined) connections on $\mathscr L_\rho$.
% We could call $A$ a \dfn{best curvature} connection, since it minimizes the $L^\infty$ norm of the curvature among all connections on $\mathscr L_\rho$.

% \todo{Do we want to write down the equivariance relations? They're the same as for a connection, except that it's not true that the differences in the transition functions are integers.}

% Since the best comass problem can be viewed as the best curvature problem on a line bundle, it is natural to make the following conjecture, which we shall not address in this paper.

% \begin{conjecture}
% For any compact gauge group $\mathbf G$, and $\mathbf G$-bundle $\mathscr E$, there exists a minimizer of the $L^\infty$ norm of the curvature among all connections on $\mathscr E$.
% Moreover, one can find a minimizer by taking suitable weak limits of solutions of the \dfn{$p$-Yang-Mills equation}
% $$\dif_A^* (|\dif A|^{p - 2} \dif A) = 0$$
% where $\dif_A$ is the covariant derivative whose Christoffel symbols are given by $A$. 
% \end{conjecture}




%%%%%%%%%%%%%%%%%%%%%
\subsection{Outline of the paper}
In \S\ref{comass sec}, we outline the basic properties of the comass, and the geometric measure theory needed for the problem at hand.

In \S\ref{tight forms sec}, we construct the $\infty$-tight form in each cohomology class, and its $1$-harmonic conjugate, proving Theorem \ref{existence of infinity tight forms}.
We also state a few natural conjectures about $\infty$-tight and $p$-tight forms which we shall not address here.

In \S\ref{MCL sec}, we study the maximum comass locus of a form of best comass, and the Thurston lamination associated to its cohomology class.
By applying the results of the previous two sections, we prove Theorems \ref{lams are calibrated} and \ref{LK}.

Finally, in Appendix \ref{Max Flow Min Cut}, we explain how the relationship between a $p$-tight form and its $q$-harmonic conjugate can be viewed as a form of the max flow min cut principle.

%%%%%%%%%%%%%%%%%%%%%
\subsection{Notation}
We say that an $\ell$-form $F \in L^1_\loc(M, \Omega^\ell)$ is \dfn{closed} if for every smooth, compactly supported closed $d - \ell$-form $\varphi$, $\int_M F \wedge \varphi = 0$.
We do not assume that $F$ is continuous.

We write $\Omega^\ell$, $Z^\ell$, and $B^\ell$ for the spaces of $\ell$-forms, closed $\ell$-forms ($\ell$-cocycles), and exact $\ell$-forms ($\ell$-coboundaries) respectively.
We reserve $H^\ell$ for cohomology and write $W^{s, p}$ for Sobolev spaces.

We let $\dif V = \star 1$ be the volume form on $N$, and for an $\ell$-rectifiable set $\tau$, we let $\dif S_\tau$ denote the $\ell$-dimensional Hausdorff measure on $\tau$ induced by $\dif V$.
We also write $\mathcal H^\ell$ for $\ell$-dimensional Hausdorff measure.

%%%%%%%%%%%%%%%%%%%%%%
\subsection{Acknowledgements}
I would like to thank Georgios Daskalopolous for suggesting this project, providing helpful comments, and providing me with an early draft of the manuscript \cite{daskalopoulos2023} which was a major source of inspiration for this work.
I also would like to thank Karen Uhlenbeck, Tom Goodwillie, Kaya Ferendo, Tainara Borges, and Haram Ko for helpful discussions.

This research was supported by the National Science Foundation's Graduate Research Fellowship Program under Grant No. DGE-2040433.


%%%%%%%%%%%%%%%%%%%%%%%%%%%%%%%%%%%%%%%%%%

\section{Comass}\label{comass sec}
In this section we study the basic properties of the comass.
We show that it is well-defined, equals the $L^\infty$ norm, and has a local version which is upper semicontinuous.
Along the way we establish some tools from geometric measure theory that we shall need in the sequel.
Throughout, we fix a Riemannian manifold $M$ (possibly not closed) of dimension $d$ and metric $g$.

\begin{definition}
For a closed $d-1$-form $F$ and a subdomain $\Omega \subseteq M$, the \dfn{comass} of $F$ in $\Omega$ is
$$L_\Omega(F) := \sup_{\sigma \in \Chain_{d - 1}(\Omega)} \frac{1}{|\sigma|} \int_\sigma F.$$
We let $L(F) := L_M(F)$.

We say that $F$ has \dfn{best comass} if $F$ is a minimizer of $L$ among all forms cohomologous to $F$.
\end{definition}

%%%%%%%%%%%%%%%%%%%
\subsection{Trace theorem}
The comass $L_\Omega(F)$ is well-defined if $F$ is continuous, or has a trace along every $d - 1$-simplex.
We now prove that we may take traces of an $L^\infty$ $d - 1$ form, provided that it is closed.

We say that a $d-2$-form $A$ is in \dfn{Coulomb gauge} if $\dif^* A = 0$.
If $F = \dif A$, then it follows that $F = (\dif + \dif^*) A$, and since the Dirac operator $\dif + \dif^*$ is elliptic, for $p > d$ we get from Sobolev embedding and elliptic regularity \cite[\S4]{Scott95} that
\begin{equation}\label{Sobolev}
	\|A\|_{C^0} \lesssim \|\nabla A\|_{L^p} \lesssim \|(\dif + \dif^*) A\|_{L^p} = \|F\|_{L^p}.
\end{equation}

Recall that an \dfn{integral current} is an current of integer multiplicity (in the sense of \cite[\S27]{Simon84}) whose exterior derivative has integer multiplicity.
Since integral $\ell$-currents are ``morally'' $\ell$-chains, if $\tau$ is an integral $\ell$-current, we write $\int_\tau F$ for the pairing of $\tau$ with an $\ell$-form $F$.

\begin{proposition}[trace theorem]\label{integration is welldefined}
Let $\tau$ be an integral $d-1$-current, and $\psi \in C^1(M)$.
Then for any $d < p < \infty$, $F \mapsto \int_\tau \psi F$ extends to a continuous linear functional on $L^p(M, Z^{d - 1})$.
Moreover,
\begin{equation}\label{integral over chain is linfinity}
	\int_\tau F \leq \|F\|_{L^\infty} \int_\tau \psi \dif S_\tau.
\end{equation}
\end{proposition}
\begin{proof}
After replacing $\psi$ with $\psi \chi_\alpha$ for $(\chi_\alpha)$ a suitable partition of unity, we may assume that $M$ is contractible.
We show that $F \mapsto \int_\tau \psi F$ is continuous for the $L^p$ norm on smooth closed $d-1$-forms.
If $F, G \in C^\infty(M, Z^{d - 1})$ and we select $A, B$ in Coulomb gauge with $\dif A = F$, $\dif B = G$, we have from integration by parts and (\ref{Sobolev}),
\begin{align*}
	\left|\int_\tau \psi(F - G)\right| 
	&\leq \left|\int_{\partial \tau} \psi (A - B)\right| + \left|\int_\tau \dif \psi \wedge (A - B)\right| \\
	&\lesssim_{\tau, \psi} \|A - B\|_{C^0} \lesssim \|F - G\|_{L^p}.
\end{align*}
So by a mollification argument, $F \mapsto \int_\tau \psi F$ is a continuous linear functional on $L^p(M, Z^{d - 1})$.

Suppose now that $F \in L^\infty$ and choose a sequence of mollifiers $\chi_\varepsilon$ such that $\|\chi_\varepsilon\|_{L^1} = 1$.
Let $F_\varepsilon := F * \chi_\varepsilon$, so that $F_\varepsilon \to F$ in $L^p$ for any $d < p < \infty$.
By Young's inequality, 
$$\|F_\varepsilon\|_{C^0} \leq \|F\|_{L^\infty} \|\chi_\varepsilon\|_{L^1} \leq \|F\|_{L^\infty}.$$
Taking $\varepsilon \to 0$ we see that
\begin{align*}
\int_\tau \psi F 
&= \lim_{\varepsilon \to 0} \int_\tau \psi F_\varepsilon \leq \lim_{\varepsilon \to 0} \|F_\varepsilon\|_{C^0} \int_\tau \psi \dif S_\tau \leq \|F\|_{L^\infty} \int_\tau \psi \dif S_\tau. \qedhere
\end{align*}
\end{proof}

Taking $\psi := 1$, it follows the integral in the definition of comass is well-defined as long as $F \in L^p$, $p > d$, and each such integral depends continuously on $F$ in $L^p$.
It will later be useful to instead take $\psi$ to be an element of a partition of unity.

We shall also need to be able to plug in forms of best comass in the coarea formula.
To state it, we recall that if $u \in BV(M)$, then almost all of its superlevel sets $\{u > y\}$ have locally finite perimeter \cite[Theorem 1.23]{Giusti77}.
In particular, each superlevel set has a reduced boundary $\partial^* \{u > y\}$, which is the set of points for which the polar decomposition of $\dif 1_{\{u > y\}}$ is well-defined \cite[Chapter 3]{Giusti77}.

Let $\mu$ be the total variation measure of $\dif u$, and let $\normal$ be the conormal $1$-form to the level sets of $u$, hence $\dif u = \normal \mu$.
Then for any open set $U$,
\begin{equation}\label{TV coarea formula}
\mu(U) = \int_{-\infty}^\infty |\partial^* \{u > y\} \cap U| \dif y.
\end{equation}
If $M$ is flat then (\ref{TV coarea formula}) follows from \cite[Theorem 1.23]{Giusti77}, and in \todo{Cite laminations paper} we sketched a proof in the Riemannian setting.
Since (for almost every $y$) $\{u > y\}$ has locally finite perimeter, $\partial^* \{u > y\}$ is an exact current of integer multiplicity \cite[Theorem 14.3]{Simon84}, and in particular is integral.

\begin{proposition}[coarea formula]
Let $u \in BV(M)$, and let $F \in L^p(M, Z^{d - 1})$ where $p > d$. Then 
\begin{equation}\label{coarea formula}
\int_M \dif u \wedge F = \int_{-\infty}^\infty \int_{\partial^* \{u > y\}} F \dif y.
\end{equation}
\end{proposition}
\begin{proof}
Let $\tau_y := \partial^* \{u > y\}$.
If $F$ is continuous, then $\langle\star \normal, F\rangle \in L^\infty_\loc(\mu)$, so we can compute using (\ref{TV coarea formula})
\begin{align*}
\int_M \dif u \wedge F = \int_M \langle \star \normal, F\rangle \dif \mu = \int_{-\infty}^\infty \int_{\tau_y} \langle \star \normal, F\rangle \dif S_{\tau_y} \dif y = \int_{-\infty}^\infty \int_{\tau_y} F \dif y.
\end{align*}
By the trace theorem, $\int_{\tau_y} F$ depends continuously on $F$ in $L^p(M, Z^{d - 1})$ for almost every $y$.
So by a mollification argument, (\ref{coarea formula}) holds for every $F$.
\end{proof}



%%%%%%%%%%%%%%%%%%%%%%%
\subsection{Local comass}
We will be interested in the points at which $F$ attains its comass.
One could pose this problem as the problem of computing the locus $\{|F| = \|F\|_{L^\infty}\}$.
However, $F$ is both only defined almost everywhere, and not norm-approximable by smooth functions.
So as a proxy for $|F|$, which may fail to be defined on a null set, we use the local comass, which is defined everywhere.

\begin{definition}
The \dfn{local comass} of a closed $d - 1$-form $F$ at $x \in M$ is 
$$L(F, x) = \limsup_{\varepsilon \to 0} L_{B_\varepsilon(x)}(F).$$
\end{definition}

Since $L_{B_\varepsilon(x)}(F)$ is a supremum over a set which grows in $\varepsilon$, it is increasing in $\varepsilon$, so the limit superior is actually a limit and an infimum:
$$L(F, x) = \lim_{\varepsilon \to 0} L_{B_\varepsilon(x)}(F) = \inf_{\varepsilon > 0} L_{B_\varepsilon(x)}(F).$$
In particular, $L(F, x) \leq L(F)$.

The comass enjoys many of the same properties as the Lipschitz constant, including \cite[Lemma 4.3]{Crandall2008}:

\begin{proposition}\label{crandall}
Let $F \in L^\infty(M, Z^{d - 1})$. Then:
\begin{enumerate}
\item The local comass $L(F, \cdot)$ is upper semicontinuous. \label{crandall usc}
\item For almost every $x \in M$,
$$|F(x)| \leq L(F, x).$$
\label{crandall LDT}
\item The local comass is bounded, and \label{crandall linfinity}
$$L(F) = \sup_{x \in M} L(F, x) = \|F\|_{L^\infty}.$$
\item If $\sigma \in \Chain_{d - 1}(M)$ then \label{crandall best curl is ABC}
$$\frac{1}{|\sigma|} \int_\sigma F \leq \sup_{x \in \sigma} L(F, x).$$
\end{enumerate}
\end{proposition}
\begin{proof}
We first prove (\ref{crandall usc}).
Let $x^n \to x$ and $r > 0$. Then eventually $x^n \in B_r(x)$, hence $L(F, x^n) \leq L_{B_r(x)}(F)$ and so
\begin{align*}
\limsup_{n \to \infty} L(F, x^n) &\leq \inf_{r > 0} L_{B_r(x)}(F) = L(F, x).
\end{align*}

We now prove (\ref{crandall LDT}).
We may work locally, and choose coordinates $(y^i)$ in which $\sqrt{\det g} = 1$.
Let $I$ be the increasing $d-1$-index with $d$ removed.
By the Lebesgue differentiation theorem and Fubini's theorem, there exists a null set $Z \subset M$ such that for every $x \notin Z$,
\begin{align*}
F_I(x) 
&= \lim_{\varepsilon \to 0} \frac{1}{|B_\varepsilon(x)|} \int_{B_\varepsilon(x)} F_I(y) \dif y \\
&= \lim_{\varepsilon \to 0} \frac{1}{|B_\varepsilon(x)|} \int_{-\infty}^\infty \int_{\{y^d = t\} \cap B_\varepsilon(x)} F_I(y) \dif y^1 \cdots \dif y^{d - 1} \dif t
\end{align*}
where we used the fact that $\sqrt{\det g} = 1$.
Now $\partial_{y^1}, \dots, \partial_{y^{d - 1}}$ are tangent to $\{y^d = t\}$, so as forms on $\{y^d = t\}$,
$$F_I(y) \dif y^1 \cdots \dif y^{d - 1} = F.$$
So
\begin{align*}
F_I(x) 
&= \lim_{\varepsilon \to 0} \frac{1}{|B_\varepsilon(x)|} \int_{-\infty}^\infty \int_{\{y^d = t\} \cap B_\varepsilon(x)} F \dif t \\
&\leq \lim_{\varepsilon \to 0} \frac{L_{B_\varepsilon(x)}(F)}{|B_\varepsilon(x)|} \int_{-\infty}^\infty |\{y^d = t\} \cap B_\varepsilon(x)| \dif t.
\end{align*}
By Fubini's theorem,
$$F_I(x) \leq \lim_{\varepsilon \to 0} \frac{L_{B_\varepsilon(x)}(F)}{|B_\varepsilon(x)|} |B_\varepsilon(x)| = L(F, x).$$
Moreover, by \todo{Cite minimal laminations paper}, $Z$ is independent of the choice of coordinates.
For every $x \in M$ we may select coordinates in which $|F(x)| = F_I(x)$, and then if $x \notin Z$, we conclude that (\ref{crandall LDT}) holds for $x$.

If we combine (\ref{crandall LDT}) with (\ref{integral over chain is linfinity}), then
$$\sup_{x \in M} L(F, x) \leq L(F) \leq \|F\|_{L^\infty} \leq \sup_{x \in M} L(F, x).$$
The inequalities collapse, proving (\ref{crandall linfinity}).
In particular, for each $\sigma \in \Chain_{d - 1}(M)$, we obtain (\ref{crandall best curl is ABC}):
\begin{align*}
\frac{1}{|\sigma|} \int_\sigma F &\leq \inf_{\Omega \supset \sigma} \sup_{x \in \Omega} L(F, x) = \sup_{x \in \sigma} L(F, x). \qedhere
\end{align*}
\end{proof}


%%%%%%%%%%%%%%%%%%%%%%%%%%%%%
\section{\texorpdfstring{$\infty$-tight forms}{Infinity-tight forms}}\label{tight forms sec}
\subsection{\texorpdfstring{$p$-tight forms}{p-tight forms}}
As before, let $M$ be a closed Riemannian manifold of dimension $d$, equipped with a cohomology class $\rho \in H^{d - 1}(M, \RR)$.
In this section, we construct a representative $F$ of best comass of $\rho$, and a $1$-harmonic conjugate to $F$.
We begin by constructing a minimizing sequence for the comass as $p \to \infty$.

\begin{definition}
Let $1 < p < \infty$ and let $F_p$ be a closed $d - 1$-form with $[F_p] = \rho$.
We call $F_p$ a \dfn{$p$-tight form} if it is a minimizer of $\|F_p\|_{L^p}$ among all $d - 1$-forms representing $\rho$.
\end{definition}

Taking $p$th powers of the $L^p$ norm, we see that a closed $d - 1$-form is $p$-tight iff it is a minimizer of
$$J_p(F) := \frac{1}{p} \int_M \star |F|^p.$$

\begin{proposition}
A $d - 1$-form $F \in L^p(M, \Omega^{d - 1})$ is $p$-tight iff it solves the system
\begin{equation}\label{pMaxwell}
\begin{cases}
	\dif F = 0, \\
	\dif^*(|F|^{p - 2} F) = 0
\end{cases}
\end{equation}
in the distributional sense. Moreover, the restriction of $J_p$ to any cohomology class in $L^p(M, Z^{d - 1})$ is uniformly strictly convex.
\end{proposition}
\begin{proof}
Let us take variations of $J_p$ in the space $L^p(M, Z^{d - 1})$ of closed $d-1$-forms of finite $J_p$-energy.
Let $G$ be an exact $d - 1$-form (so $F + tG$ is cohomologous to $F$), so that
$$\frac{\dif}{\dif t} J_p(F + tG) = \frac{1}{p} \int_M \star \frac{\partial}{\partial t} |F + tG|^p = \int_M \star |F + tG|^{p - 2} \langle F + tG, G\rangle.$$
Setting $t = 0$, and writing $G = \dif B$, we obtain 
$$0 = \int_M \star |F|^{p - 2} \langle F, \dif B\rangle = \int_M \star \langle \dif^*(|F|^{p - 2} F), B\rangle.$$
This equation holds for every $d - 2$-form $B$.
Thus the Euler-Lagrange equations for $J_p$ are (\ref{pMaxwell}).
The second variation is 
$$\frac{\dif^2}{\dif t^2} J_p(F + tG)\bigg|_{t = 0} = (p - 2) \int_M \star |F|^{p - 4} \langle F, G\rangle^2 + \int_M \star |F|^{p - 2} |G|^2.$$
By the Cauchy-Schwarz inequality,
$$(p - 2) \int_M \star |F|^{p - 4} \langle F, G\rangle^2 + \int_M \star |F|^{p - 2} |G|^2 \geq (p - 1) \int_M \star |F|^{p - 2} |G|^2.$$
This implies uniform strict convexity for $p > 1$.
The restriction to a cohomology class remains strictly convex, since cohomology classes are closed affine subspaces of $L^p(M, Z^{d - 1})$ and hence are themselves closed and convex.
\end{proof}

Arguing as in the proof of existence of $p$-harmonic functions (see \cite[\S8.2]{evans2010partial} for the general theory, or \cite[\S2.1]{daskalopoulos2020transverse} for $p$-harmonic maps $M \to \Sph^1$ which is closely analogous to our case), one can easily check that there exists a unique $p$-tight form in every cohomology class on the closed manifold $M$.
We omit the details, both because the argument is standard, and because we shall give a separate proof in Appendix \ref{Max Flow Min Cut}.

\begin{definition}
Let $F$ be a $p$-tight form, let
\begin{equation}
\dif u := (-1)^{d - 1} |F|^{p - 2} \star F, \label{inverse extremality}
\end{equation}
and let $u$ be the primitive of $\dif u$ on the universal cover $\tilde M$, which is normalized to have zero mean on a fundamental domain $M_{\rm fun}$.
Then $u$ is called the \dfn{$q$-harmonic conjugate} of the $p$-tight form $F$, where $\frac{1}{p} + \frac{1}{q} = 1$.
\end{definition}

Let $u$ be the $q$-harmonic conjugate of $F$.
By Poincar\'e's inequality,
$$\|u\|_{W^{1, q}(M_{\rm fun})}^q \lesssim \int_M \star |\dif u|^q = \int_M \star |F|^{(p - 1)q} = \int_M \star |F|^p < \infty$$
since $F$ is $p$-tight; that is, we have $F \in L^p$ and $u \in W^{1, q}$.
From this, and H\"older's inequality, it follows that all expressions in the next proposition are well-defined.

\begin{proposition}
Let $1 < p, q < \infty$ and $\frac{1}{p} + \frac{1}{q} = 1$.
Let $F$ be a $p$-tight form, and let $u$ be its $q$-harmonic conjugate. Then:
\begin{enumerate}
\item $u$ is $q$-harmonic:
$$\dif^*(|\dif u|^{q - 2} \dif u) = 0$$
in the sense of distributions.
\item One has 
\begin{equation}
F = |\dif u|^{q - 2} \star \dif u. \label{extremality} \\
\end{equation}
\item One has the \dfn{max flow min cut principle}
\begin{equation}\label{strong duality}
\frac{1}{q} \int_M \star |\dif u|^q + \frac{1}{p} \int_M \star |F|^p = \int_M \dif u \wedge F.
\end{equation}
\end{enumerate}
\end{proposition}

We give a more highbrow proof of the same proposition in Appendix \ref{Max Flow Min Cut}, which also justifies why (\ref{strong duality}) should be called the ``max flow min cut principle.''
However, we think there is value in providing a straightforward proof from first principles.

\begin{proof}
We first compute 
$$(q - 2)(p - 1) + p = 2,$$
so that
$$|\dif u|^{q - 2} \star \dif u = (-1)^{d - 1} |F|^{(q - 2)(p - 1)} \star \star |F|^{p - 2} F = |F|^{(q - 2)(p - 1) - (p - 2)} F = F.$$
Thus we have (\ref{extremality}), and moreover,
$$\dif \star (|\dif u|^{q - 2} \dif u) = \dif F = 0$$
so that $u$ is $q$-harmonic.
We also compute 
\begin{align*}
\int_M \star |F|^p &= \int_M \star |\dif u|^{(q - 1)p} = \int_M \star |\dif u|^q = \int_M \dif u \wedge |\dif u|^{q - 2} \star \dif u = \int_M \dif u \wedge F. \qedhere 
\end{align*}
\end{proof}

We conjecture that $p$-tight forms are $C^\infty$ for $d < p < \infty$ on $\{|F| > 0\}$.
Indeed, they are H\"older continuous \cite{Uhlenbeck77}, but we \emph{cannot} carry out a na\"ive elliptic bootstrapping scheme.
For example, if $d = 3$ and we set $F = \dif A$, then
$$\partial^j(|\dif A|^{p - 2} \partial_j A_i) - |\dif A|^{p - 2} \partial_i (\dif^* A) - |\dif A|^{p - 2} [\partial^j, \partial_i] A_j - \partial^j(|\dif A|^{p - 2}) \partial_i A_j = 0.$$
The first term is an elliptic operator with H\"older coefficients, the second can be gauged away, and the third is H\"older since $[\partial^j, \partial_i]$ is a connection coefficient of the metric.
However, the last term is problematic, because $\partial^j(|\dif A|^{p - 2}) \partial_i A_j$ is not manifestly elliptic in $A$.


%%%%%%%%%%%%%%%%%%%%%%%
\subsection{\texorpdfstring{Existence of $\infty$-tight forms}{Existence of infinity-tight forms}}
We now take the limit $p \to \infty$ to obtain a privileged form of best comass.
To do so, we shall need the $p$-tight forms to be uniformly bounded in the following sense.

\begin{lemma}
Let $F_p$ be a $p$-tight form, and let $B$ range over closed $d - 1$-forms cohomologous to $F_p$. Then
\begin{equation}\label{infinity magnetic rules p magnetic}
	\|F_p\|_{L^p} \leq |M|^{1/p} \inf_B \|B\|_{L^\infty}.
\end{equation}
\end{lemma}
\begin{proof}
By H\"older's inequality and the fact that $F_p$ is $p$-tight,
$$\|F_p\|_{L^p} \leq \|B\|_{L^p} \leq |M|^{1/p} \|B\|_{L^\infty},$$
hence the same holds for the infimum.
\end{proof}

\begin{proposition}\label{existence infinity}
Let $\rho \in H^{d - 1}(M, \RR)$.
For each $p \geq 2$, let $F_p$ be the $p$-tight form representing $\rho$. Then there exists a closed $d - 1$-form $F$ such that:
\begin{enumerate}
\item $F_p \to F$ weakly in $L^r$ along a subsequence, for any $d < r < \infty$.
\item $F$ is a best comass representative of $\rho$.
\end{enumerate}
\end{proposition}
\begin{proof}
We roughly follow \cite[\S3]{Lindqvist14}.
Let $r > d$, and let $B$ be an $L^\infty$ representative of $\rho$.
By H\"older's inequality and (\ref{infinity magnetic rules p magnetic}),
\begin{equation}\label{uniform bounds in p by best curl}
	\|F_p\|_{L^r} \leq |M|^{\frac{1}{r} - \frac{1}{p}} \|F_p\|_{L^p} \leq |M|^{\frac{1}{r}} \|B\|_{L^\infty}.
\end{equation}
Thus a compactness argument gives $F_p \to F$ for some $d - 1$-form $F$, weakly in $L^r$, and 
$$\|F\|_{L^r} \leq \liminf_{p \to \infty} \|F_p\|_{L^r} \leq |M|^{\frac{1}{r}} \|B\|_{L^\infty}.$$
Diagonalizing, we may assume that $F_p \to F$ weakly in $L^r$ for every such $r$, and taking $r \to \infty$, we conclude 
\begin{equation}\label{infinity magnetics have best curl}
	\|F\|_{L^\infty} \leq \|B\|_{L^\infty}.
\end{equation}
Moreover, $[F] = \lim_{p \to \infty} [F_p] = \rho$.
So by Proposition \ref{crandall}(\ref{crandall linfinity}) and the fact that $B$ was arbitrary in (\ref{infinity magnetics have best curl}), $F$ has best comass.
\end{proof}

\begin{definition}
The $d - 1$-form $F$ of best comass in Proposition \ref{existence infinity} is called an \dfn{$\infty$-tight form}.
\end{definition}

The existence of $\infty$-tight (or even just best comass) representatives of each cohomology class implies the following useful lemma.

\begin{lemma}\label{p tights approximate L}
Let $F_p$ be the $p$-tight representative of $\rho$, and $L$ the best comass of $\rho$. Then 
$$\lim_{p \to \infty} \|F_p\|_{L^p} = L.$$
\end{lemma}
\begin{proof}
We follow \cite[Lemma 2.7]{daskalopoulos2020transverse}.
Let $F$ be an $\infty$-tight representative of $\rho$, so by Proposition \ref{crandall}(\ref{crandall linfinity}), $\|F\|_{L^\infty} = L$.
Since $F_p$ is $p$-tight, H\"older's inequality implies 
$$\|F_p\|_{L^p} \leq \|F\|_{L^p} \leq |M|^{\frac{1}{p}} L.$$
Therefore 
$$\limsup_{p \to \infty} \|F_p\|_{L^p} \leq L.$$
To prove the converse, suppose that 
$$\liminf_{p \to \infty} \|F_p\|_{L^p} \leq \tilde L < L.$$
Along a subsequence which attains the limit inferior, $F_p$ converges weakly in every $L^r$, $d < r < \infty$, to an $\infty$-tight form $\tilde F$ such that (by H\"older's inequality)
$$\|\tilde F\|_{L^r} \leq \liminf_{p \to \infty} \|F_p\|_{L^r} \leq \liminf_{p \to \infty} |M|^{\frac{1}{r}} \|\tilde F\|_{L^\infty} \leq |M|^{\frac{1}{r}} \tilde L.$$
Taking $r \to \infty$, we obtain $L(\tilde F) < L$, which contradicts the fact that $L$ is the best comass.
\end{proof}

By analogy with $\infty$-harmonic maps, one expects $\infty$-tight forms to be unique and have \dfn{absolutely best comass} in the sense that they have best comass in \emph{every} open subset of $M$ with smooth boundary.
One can show formally using \cite[Theorem 5.2]{Barron2001} that the Euler-Lagrange equations for a smooth form of absolutely best comass of 
\begin{equation}\label{infinity Maxwell}
	\nabla_{(\iota_X F)^\sharp} |F|^2 = 0
\end{equation}
but this equation lacks a good notion of weak solution at present.
Thus we conjecture:

\begin{conjecture}
For every $\rho \in H^{d - 1}(M, \RR)$ there exists a unique $\infty$-tight representative $F$ of $\rho$.
Moreover, $F$ has absolutely best comass, and is a weak solution of (\ref{infinity Maxwell}) in a suitable sense.
\end{conjecture}


%%%%%%%%%%%%%%%%%%%%
\subsection{\texorpdfstring{$1$-harmonic conjugates of $\infty$-tight forms}{One-harmonic conjugates of infinity-tight forms}}
We now construct the $1$-harmonic conjugate of an $\infty$-tight form.
Since (\ref{inverse extremality}) may blow up as $p \to \infty$, we have to renormalize the $q$-harmonic conjugates of $p$-tight forms before taking the limit $q \to 1$, as in \cite[\S3.2]{daskalopoulos2020transverse}.

We begin by showing that $L^1$ convergence preserves the equivariance properties of functions.
To make this precise, let $\tilde M \to M$ be the universal cover of $M$.
Following \cite[\S4]{daskalopoulos2020transverse}, we use the Hurcewiz theorem to identify $1$-dimensional representations
$$\alpha: \pi_1(M) \to \RR$$
of the fundamental group with cohomology classes $H^1(M, \RR)$.
If we have an $\alpha$-equivariant function $u$ on $\tilde M$, thus for every $\gamma \in \pi_1(M)$,
$$\gamma^* u = u + \langle \alpha, \gamma\rangle,$$
we write $[u] = \alpha$.
Taking derivatives, we see that $\gamma^* \dif u = \dif u$, so $\dif u$ drops to a closed $1$-form (or perhaps better a closed $d-1$-current) on $M$ whose cohomology class is $\alpha$.

\begin{lemma}\label{L1 convergence preserves pi1}
Let $\tilde M \to M$ be the universal cover, and let $(u_q)$ be a sequence of $\pi_1(M)$-equivariant functions on $\tilde M$ which converge in $L^1_\loc(\tilde M)$ to a function $u$ as $q \to 1$.
Then $u$ is $\pi_1(M)$-equivariant, and $[u_q] \to [u]$.
Moreover, if $\dif u_q \to \dif u$ in the weak topology of measures on $M$ and $\dif u_q \in L^q$, then
\begin{equation}\label{q to 1 Holder}
\|\dif u\|_{TV} \leq \liminf_{q \to 1} \frac{1}{q} \int_M \star |\dif u_q|^q.
\end{equation}
\end{lemma}
\begin{proof}
Since $u_q$ is $\pi_1(M)$-equivariant, there exists $\alpha_q \in H^1(M, \RR)$ such that for every $\gamma \in \pi_1(M)$,
\begin{equation}\label{equivariance q}
	\gamma^* u_q = u_q + \langle \alpha_q, \gamma\rangle.
\end{equation}
Let $M_{\rm fun}$ be a fundamental domain and $U_\gamma := M_{\rm fun} \cup \gamma_* (M_{\rm fun})$.

We claim that $(\alpha_q)$ has a convergent subsequence.
To see this, we first recall that $M$ has finite Betti numbers, so $H^1(M, \RR)$ is locally compact.
Therefore, if no convergent subsequence exists, there exists a $\gamma \in \pi_1(M)$ and a subsequence along which $\langle \alpha_q, \gamma\rangle \to \infty$.
Moreover, since $u_q \to u$ in $L^1_\loc$, $\|u_q\|_{L^1(M_{\rm fun})} \leq 2\|u\|_{L^1(M_{\rm fun})}$ if $q - 1$ is small enough.
But then 
$$\|u_q\|_{L^1(\gamma_* M_{\rm fun})} = \|\gamma^* u_q\|_{L^1(M_{\rm fun})} \geq \langle \alpha_q, \gamma\rangle - \|u_q\|_{L^1(M_{\rm fun})} \geq \langle \alpha_q, \gamma\rangle - 2\|u\|_{L^1(M_{\rm fun})}$$
and taking $q \to 1$ we conclude that $(u_q)$ is not compact in $L^1(\gamma_* M_{\rm fun})$, contradicting the convergence in $L^1_\loc(\tilde M)$.
So $\alpha_q \to \alpha$ for some $\alpha \in H^1(M, \RR)$ along a subsequence.

For any $q > 1$,
\begin{align*}
\dashint_{M_{\rm fun}} \star |\gamma^* u - u - \langle \alpha, \gamma\rangle| 
&\leq \dashint_{M_{\rm fun}} \star (|\gamma^* u_q - u_q - \langle \alpha_q, \gamma\rangle| + |\gamma^* u_q - u_q| + |\gamma^* u - u|) \\
&\qquad + |\langle \alpha_q - \alpha, \gamma\rangle|.
\end{align*}
Taking $q \to 1$ and applying (\ref{equivariance q}), we conclude that $\|\gamma^* u - u - \langle \alpha, \gamma\rangle\|_{L^1} = 0$, hence $u$ is $\alpha$-equivariant.
Thus $\alpha$ is uniquely defined and $\alpha_q \to \alpha$ along the entire subsequence.

Finally we prove (\ref{q to 1 Holder}).
Suppose that $\dif u_q \to \dif u$ in the weak topology of measures and $\dif u_q$ in $L^q$.
Under those hypotheses, we may use the portmanteau theorem and H\"older's inequality to estimate (where $\frac{1}{p} + \frac{1}{q} = 1$)
\begin{align*}
\|\dif u\|_{TV} &= \lim_{q \to 1} \|\dif u_q\|_{L^1} \leq \lim_{q \to 1} |M|^{\frac{1}{p}} \|\dif u_q\|_{L^q} = \lim_{q \to 1} \frac{1}{q} \int_M \star |\dif u_q|^q. \qedhere
\end{align*}
\end{proof}

Next we address the renormalization.
Suppose that $\rho \in H^{d - 1}(M, \RR)$ and denote by $L$ the comass of a best comass representative of $\rho$.
Also let $k_p$ be defined by 
$$k_p^{1 - p} = \int_M \star |F_p|^p$$
where $F_p$ is the $p$-tight representative of $\rho$.

\begin{definition}
The \dfn{renormalized $q$-harmonic conjugate} of a $p$-tight form $F_p$ is the function $u_q: \tilde M \to \RR$ which has mean zero on $M_{\rm fun}$ and solves
$$\dif u_q = (-1)^{d - 1} k_p^{p - 1} |F_p|^{p - 2} \star F_p.$$
\end{definition}

\begin{lemma}\label{normalizations converge}
As $p \to \infty$, $k_p \to 1/L$.
\end{lemma}
\begin{proof}
We follow \cite[Lemma 3.4]{daskalopoulos2020transverse}.
By Lemma \ref{p tights approximate L},
$$\lim_{p \to \infty} k_p^{-\frac{1}{q}} = \lim_{p \to \infty} \|F_p\|_{L^p} = L.$$
Taking logarithms we see that $q^{-1} \log k_p \to -\log L$, and since $q \to 1$ the claim follows.
\end{proof}

\begin{proposition}\label{existence 1}
Let $\rho \in H^{d - 1}(M, \RR)$ and let $\tilde M \to M$ be the universal cover.
For $2 < p < \infty$ and $\frac{1}{p} + \frac{1}{q} = 1$, let $u_q$ be the renormalized $q$-harmonic conjugate of the $p$-tight representative of $\rho$.
Then there exists a $\pi_1(M)$-equivariant function $u \in BV_\loc(\tilde M)$ such that:
\begin{enumerate}
\item $u$ is $1$-harmonic.
\item As $q \to 1$ along a subsequence, $u_q \to u$ weakly in $BV_\loc(\tilde M)$ and strongly in $L^r_\loc(\tilde M)$ for $1 \leq r < \frac{d}{d - 1}$.
\item Let $F$ be the $\infty$-tight representative of $\rho$, with best comass $L$. We have the \dfn{max flow min cut principle} that, as Radon measures,
\begin{equation}\label{1 extremality}
\dif u \wedge F = L \star |\dif u|.
\end{equation}
\end{enumerate}
\end{proposition}
\begin{proof}
We first compute using H\"older's inequality and Lemma \ref{normalizations converge}
\begin{align*}
\lim_{q \to 1} \|\dif u_q\|_{L^1}
&\leq \lim_{q \to 1} |M|^{\frac{1}{p}} \left[\int_M \star |\dif u_q|^q\right]^{\frac{1}{q}} = \lim_{p \to \infty} \left[k_p^p \int_M \star |F_p|^p\right]^{\frac{1}{q}} \\
&= \lim_{p \to \infty} k_p^{\frac{1}{q}} = \lim_{p \to \infty} k_p = \frac{1}{L}.
\end{align*}
So by Rellich's theorem, $(u_q)$ is weakly compact in $BV$ and strongly compact in $L^r$ for $1 \leq r < \frac{d}{d - 1}$.
In particular, $\dif u_q \to \dif u$ in the weak topology of measures and $u_q \to u$ weakly in $BV$ and strongly in $L^r$.
As the limit of $\pi_1(M)$-equivariant functions, $u$ is also $\pi_1(M)$-equivariant by Lemma \ref{L1 convergence preserves pi1}.
In particular, $\dif u$ drops to a current on $M$.
Moreover, $[\dif u_q] \to [\dif u]$, and we have the bound (\ref{q to 1 Holder}) on $\int \star |\dif u|$.

Renormalizing (\ref{strong duality}), we obtain 
$$\frac{k_p^{-p}}{q} \int_M \star |\dif u_q|^q + \frac{1}{p} \int_M \star |F_p|^p = k_p^{1 - p} \int_M \dif u_q \wedge F_p.$$
Multiplying by $k_p^p$, we have 
\begin{equation}\label{1 strong duality before limits}
	\frac{1}{q} \int_M \star |\dif u_q|^q + \frac{k_p^p}{p} \int_M \star |F_p|^p = k_p \int_M \dif u_q \wedge F_p.
\end{equation}

We next claim that
\begin{equation}\label{1 strong duality}
	L \|\dif u\|_{TV} \leq \int_M \dif u \wedge F.
\end{equation}
First, we have from (\ref{q to 1 Holder}) and (\ref{1 strong duality before limits}) that
$$\|\dif u\|_{TV} \leq \lim_{q \to 1} \frac{1}{q} \int_M \star |\dif u_q|^q = \lim_{p \to \infty} k_p \int_M \dif u_q \wedge F_p - \lim_{p \to \infty} \frac{k_p^p}{p} \int_M \star |F_p|^p.$$
By Lemma \ref{normalizations converge},
$$\lim_{p \to \infty} \frac{k_p^p}{p} \int_M \star |F_p|^p = \lim_{p \to \infty} \frac{k_p}{p} = \frac{0}{L} = 0,$$
and
$$\lim_{p \to \infty} k_p \int_M \dif u_q \wedge F_p = \frac{1}{L} \lim_{p \to \infty} \int_M [\dif u_q] \wedge \rho.$$
Since $[\dif u_q] \to [\dif u]$, we obtain
$$\lim_{p \to \infty} \int_M [\dif u_q] \wedge \rho = \int_M \alpha \wedge \rho = \int_M \dif u \wedge F,$$
completing the proof of (\ref{1 strong duality}).

By the coarea formula (\ref{coarea formula}), we have for any contractible open set $U$,
$$\int_U \dif u \wedge F = \int_{-\infty}^\infty \int_{U \cap \partial \{u > y\}} F \dif y \leq L \int_{-\infty}^\infty |U \cap \partial \{u > y\}| \dif y = L \|\dif u\|_{TV}.$$
Since $U$ was arbitrary, as Radon measures, we obtain
\begin{equation}\label{one sided extremality}
\dif u \wedge F \leq L \star |\dif u|.
\end{equation}

Next we deduce (\ref{1 extremality}).
We reason by contradiction: if (\ref{1 extremality}) is false, then there exists an open set $U \subseteq M$ such that 
$$\int_U \dif u \wedge F < L \int_U \star |\dif u|.$$
(Indeed, strict inequality cannot point in the other direction, by (\ref{one sided extremality}).)
However, by (\ref{one sided extremality}), 
$$\int_{M \setminus U} \dif u \wedge F \leq L \int_{M \setminus U} \star |\dif u|.$$
Adding up the integrals of $\dif u \wedge F$ over $U$ and $M \setminus U$, we conclude 
$$\int_M \dif u \wedge F < L \int_M \star |\dif u|,$$
but this contradicts (\ref{1 strong duality}); thus (\ref{1 extremality}) must be true.

To round out the proof, let $X := (\star F/L)^\sharp$ be the Poincar\'e dual vector field to $F/L$. Then
$$\nabla \cdot X = \star \frac{\dif F}{L} = 0,$$
and $\|X\|_{L^\infty} \leq 1$.
Moreover, by (\ref{1 extremality}), $X$ is normal to the level sets of $u$, and hence is a witness that $u$ is $1$-harmonic \cite{Mazon14}.
\end{proof}

%%%%%%%%%%%%%%%%%%%%


\section{The maximum comass locus}\label{MCL sec}
Throughout this section, let $M$ be a closed space form of dimension $2 \leq d \leq 4$ equipped with a cohomology class $\rho \in H^{d - 1}(M, \RR)$.
We study the set on which a best comass representative of $\rho$ attains its comass, which turns out to contain a measured oriented minimal lamination which only depends on $\rho$, and which is calibrated by any best comass form on $\rho$.

\begin{definition}
Let $F$ be a form of best comass.
The \dfn{maximum comass locus} is the set $\{L(F, \cdot) = L(F)\}$.
\end{definition}

By Proposition \ref{crandall}(\ref{crandall usc}) and the compactness of $M$, the maximum comass locus is a nonempty closed subset of $M$.

%%%%%%%%%%%%%%%%%%%%%%%%
\subsection{\texorpdfstring{$L^\infty$}{L-infinity} calibrations}
Recall that a \dfn{calibration} (in the classical sense) is a closed $d-1$-form $F$ such that $\|F\|_{C^0} \leq 1$.
A $d-1$-surface $N$ is then said to be $F$-\dfn{calibrated} if $\int_N F = |F|$.
In our formulation, it is more natural to study calibrations which are merely $L^\infty$ rather than $C^0$, and require that the comass $L(F) \leq 1$.
By Proposition \ref{integration is welldefined}, it is still meaningful to ask if a $d-1$-surface is $F$-calibrated, even if $F$ is discontinuous.

We shall be interested in $F$-calibrated laminations.
We remind the reader of the basic definitions:

\begin{definition}
A codimension-$1$ \dfn{lamination} $\lambda$ in $M$ consists of a nonempty closed set $S \subseteq M$, the \dfn{support} of $\lambda$, and a $C^0$ local product structure on $M$ which locally identifies $S$ with sets $K \times J$, where $K \subseteq \RR$ is a closed set and $J \subseteq \RR^{d - 1}$ is a box.
Moreover, we assume that the restriction of the local product structure to a \dfn{leaf}, or subset of $S$ which locally looks like a fiber $\{k\} \times J$, is $C^1$.
\end{definition}

\begin{definition}
Let $\lambda$ be a lamination and $F$ a calibration. We say that $\lambda$ is
\begin{enumerate}
\item \dfn{minimal} (or \dfn{geodesic}) if every leaf of $\lambda$ is a minimal hypersurface (or geodesic),
\item \dfn{$F$-calibrated} if every leaf of $\lambda$ is $F$-calibrated,
\item \dfn{oriented} if the transition maps for the local product structure are orientation-preserving, and
\item \dfn{measured} if its equipped with a \dfn{transverse measure}, or a positive Radon measure on each local leaf space, such that the transition maps for the local product structure are measure-preserving, and the support of each Radon measure is the set of leaves of $\lambda$.
\end{enumerate}
\end{definition}

\begin{definition}
If $\lambda$ is a measured oriented lamination, we introduce its \dfn{Ruelle-Sullivan current} $T_\lambda$.
Let
$$F_\alpha: I \times J \to U_\alpha \subseteq M$$
be the coordinate charts for the local product structure, let $(\chi_\alpha)$ be a partition of unity subordinate to the open cover $(U_\alpha)$, and let $(\mu_\alpha)$ be the transverse measure.
For every $\varphi \in C^0(M, \Omega^{d - 1})$,
$$\int_M T_\lambda \wedge \varphi := \sum_\alpha \int_I \int_{\{k\} \times J} F_\alpha^* (\chi_\alpha \varphi) \dif \mu_\alpha(k).$$
The \dfn{mass} $|\lambda|$ and the \dfn{homology class} $[\lambda]$ of the measured oriented lamination $\lambda$ are the mass and homology class of its Ruelle-Sullivan current $T_\lambda$.
\end{definition}

For a more careful exposition of the definitions, see \todo{cite laminations paper} and also \cite[\S\S7--8]{daskalopoulos2020transverse}.

It is clear from (\ref{calibrated surfaces are minimal}) that a $F$-calibrated lamination is minimal.
Moreover, if $F$ is a closed $d-1$ form, then the quantity $\int_M T_\lambda \wedge F$ is well-defined, since it is just $\langle [F], [\lambda]\rangle$.

\begin{proposition}\label{calibration condition}
Let $F$ be a calibration.
Let $T_\lambda$ be the Ruelle-Sullivan current of a measured oriented lamination $\lambda$, and suppose that 
\begin{equation}\label{calibration by Ruelle Sullivan}
\int_M T_\lambda \wedge F = |\lambda|.
\end{equation}
Then $\lambda$ is $F$-calibrated (and in particular minimal).
\end{proposition}
\begin{proof}
Let $(\chi_\alpha)$ be a partition of unity subordinate to the local product structure of $\lambda$, and let $(\mu_\alpha)$ be the transverse measure.
Then 
$$|\lambda| = \int_M T_\lambda \wedge F = \sum_\alpha \int_I \int_{\{k\} \times J} \chi_\alpha F \dif \mu_\alpha(k).$$
Let $\dif S_{\alpha,k}$ be the surface measure on the leaf which is locally $\sigma_{\alpha,k} := \{k\} \times J$. Then
$$\int_M \chi_\alpha \star |T_\lambda| = \int_I \int_{\sigma_{\alpha,k}} \chi_\alpha \dif S_{\alpha,k} \dif \mu_\alpha(k),$$
so summing in $\alpha$, we obtain 
$$\sum_\alpha \int_I \int_{\sigma_{\alpha,k}} \chi_\alpha F \dif \mu_\alpha(k) = |\lambda| = \sum_\alpha \int_I \int_{\sigma_{\alpha,k}} \chi_\alpha \dif S_{\alpha,k} \dif \mu_\alpha(k).$$
If our partition of unity is chosen locally finite, this is only possible if for every $\alpha$ and $\mu_\alpha$-almost every $k$, 
$$\int_{\sigma_{\alpha,k}} \chi_\alpha \dif S_{\alpha,k}  = \int_{\sigma_{\alpha,k}} \chi_\alpha F.$$
Since $(\chi_\alpha)$ was an arbitrary partition of unity, this is only possible if $F$ calibrates $\sigma_{\alpha, k}$.

To upgrade $F$ from a calibration of almost all of $\lambda$ to a calibration of all of $\lambda$, we use the fact that $\supp \mu_\alpha$ is the set of $k \in I$ such that $\sigma_{\alpha, k}$ is contained in a leaf of $\lambda$.
In particular, if $\sigma_{\alpha, k}$ is contained in a leaf of $\lambda$, then we may find $k_j$ such that $k_j \to k$ and $\sigma_{\alpha, k_j}$ is $F$-calibrated and contained in a leaf of $\lambda$.
But then, since we may write $F = \dif A$ near $\sigma_{\alpha, k}$, and 
$$\int_{\sigma_{\alpha, k}} F = \int_{\partial \sigma_{\alpha, k}} A,$$
the facts that $k_j \to k$ and that $A$ is continuous imply 
\begin{align*}
|\sigma_{\alpha, k}| &= \lim_{j \to \infty} |\sigma_{\alpha, k_j}| = \lim_{j \to \infty} \int_{\sigma_{\alpha, k_j}} F = \lim_{j \to \infty} \int_{\partial \sigma_{\alpha, k_j}} A = \int_{\partial \sigma_{\alpha, k}} A = \int_{\sigma_{\alpha, k}} F. \qedhere 
\end{align*}
\end{proof}

\begin{proposition}\label{properties of calibrated laminations}
Suppose that $F$ is a calibration and $\lambda$ is a measured oriented $F$-calibrated lamination.
Then:
\begin{enumerate}
\item $\lambda$ is minimal.
\item If $G$ is a calibration and cohomologous to $F$, then $\lambda$ is $G$-calibrated.
\item The maximum comass locus of $F$ contains $\supp \lambda$.
\end{enumerate}
\end{proposition}
\begin{proof}
Every leaf of $\lambda$ is $F$-calibrated, hence minimal, so $\lambda$ is also minimal.
Moreover, (\ref{calibration by Ruelle Sullivan}) only depends on the cohomology class of $F$, not $F$ itself, so $\lambda$ is $G$-calibrated.
Finally, let $S$ be the maximum comass locus of $F$, $N$ a leaf of $\lambda$, and suppose that $x \in N \setminus S$.
Since $S$ is closed, there exists $\varepsilon > 0$ such that $B_\varepsilon(x)$ does not meet $S$.
Moreover, $\sigma := N \cap B_\varepsilon(x)$ is a $d-1$-chain in $B_\varepsilon(x)$, so by Proposition \ref{crandall}(\ref{crandall best curl is ABC}),
$$\frac{1}{|\sigma|} \int_\sigma F \leq \sup_{y \in B_\varepsilon(x)} L(F, y) < L = 1.$$
But then 
$$\int_N F = \int_\sigma F + \int_{N \setminus B_\varepsilon(x)} F < |\sigma| + |N \cap B_\varepsilon(x)| = |N|,$$
so $N$ (hence $\lambda$) is not $F$-calibrated.
\end{proof}


%%%%%%%%%%%%%%%%%%%%%%%%
\subsection{Calibration of Thurston laminations}
Let $u$ be a $\pi_1(M)$-equivariant $1$-harmonic function on $\tilde M$.
Then $\dif u$ drops to a $d-1$-current on $M$, which is still the Ruelle-Sullivan current of a measured oriented minimal lamination on $M$, which we still call $\lambda_u$.

One may associate a measured oriented geodesic lamination $\lambda$ associated to a homotopy class $\rho \in [M, N]$, where $M, N$ are closed hyperbolic surfaces of the same genus, called the \dfn{Thurston lamination}.
Every best Lipschitz representative of $\rho$ attains its Lipschitz constant on (a superlamination of) the Thurston lamination, which also realizes the supremum in Thurston's asymmetric metric $\log K$ \cite{Thurston98}.

We claim that the lamination induced by the $1$-harmonic conjugate of an $\infty$-tight form enjoys the same boons as the Thurston lamination of a homotopy class of surface-to-surface maps.
Thus we define:

\begin{definition}
Let $\rho \in H^{d - 1}(M, \RR)$, let $F$ be an $\infty$-tight representative of $\rho$, and let $u$ be a $1$-harmonic conjugate of $F$.
Then we call $\lambda_u$ a \dfn{Thurston lamination} associated to $\rho$.
\end{definition}

\begin{theorem}\label{MCL contains Thurston}
Let $F$ be a best comass representative of $\rho \in H^{d - 1}(M, \RR)$, let $L := L(F)$ be the best comass of $\rho$, and let $\lambda$ be a Thurston lamination associated to $\rho$.
Then:
\begin{enumerate}
\item The maximum comass locus of $F$ contains $\lambda$.
\item $F/L$ calibrates $\lambda$.
\end{enumerate}
\end{theorem}
\begin{proof}
Let $G$ be the $\infty$-tight form which is cohomologous to $F$ whose dual $1$-harmonic function $u$ defines the Thurston lamination $\lambda$.
Then by the max flow min cut principle (\ref{1 extremality}), 
$$|\lambda| = \|\dif u\|_{TV} = \frac{1}{L} \int_M \dif u \wedge G$$
so $G/L$ calibrates $\lambda$ by Proposition \ref{calibration condition}.
Then by Proposition \ref{properties of calibrated laminations}, $F/L$ calibrates $\lambda$, and the maximum comass locus of $F/L$ contains $\lambda$.
However, the maximum comass locus is preserved by rescaling.
\end{proof}

\begin{theorem}\label{L equals K}
	Let $\rho \in H^{d - 1}(M, \RR)$, and let 
	$$K := \sup_\lambda \frac{\langle \rho, [\lambda]\rangle}{|\lambda|},$$
	where $\lambda$ ranges over measured oriented laminations. Then:
\begin{enumerate}
	\item The supremum in $K$ is attained by the Thurston lamination associated to $\rho$.
	\item Let $L$ be the best comass of $\rho$. Then $L = K$.
\end{enumerate}
\end{theorem}
\begin{proof}
Fix the $\infty$-tight form $F$ representing $\rho$, and let $u$ be its $1$-harmonic conjugate.

We first prove $K \leq L$.
Let $\lambda$ be a measured oriented lamination; then, since $F$ represents $\rho$ and the Ruelle-Sullivan current $T_\lambda$ represents $[\lambda]$,
$$\langle \rho, [\lambda]\rangle = \int_M F \wedge T_\lambda.$$
Let $(\chi_\alpha)$ be a partition of unity subordinate to a laminar atlas for $\lambda$, and let $(\mu_\alpha)$ be the associated transverse measure. Then 
$$\int_M F \wedge T_\lambda = \sum_\alpha \int_I \int_{\{k\} \times J} \chi_\alpha F \dif \mu_\alpha(k).$$
Since $F$ has best comass,
$$\frac{\langle \rho, [\lambda] \rangle}{|\lambda|}
\leq \frac{\|F\|_{L^\infty}}{|\lambda|} \sum_\alpha \int_I \int_{\{k\} \times J} \chi_\alpha \dif S_k \dif \mu_\alpha(k) = L.$$
Since $\lambda$ was arbitrary, it holds that $K \leq L$.

Let $\lambda$ be the Thurston lamination, so $T_\lambda = \dif u$.
Then by the max flow min cut principle (\ref{1 extremality}),
$$\langle \rho, [\lambda]\rangle = \int_M F \wedge \dif u = L \|\dif u\|_{TV} = L|\lambda|.$$
Dividing both sides by $|\lambda|$ and applying the direction we already proved,
$$K \leq L \leq \frac{\langle \rho, [\lambda]\rangle}{|\lambda|} \leq K$$
which is only possible if $L = K$ and $\lambda$ is a maximizer.
\end{proof}

%%%%%%%%%%%%%%%%%%%%%%%%%%%%%%%%%%
\appendix 
\section{Max flow, min cut}\label{Max Flow Min Cut}
Let us interpret the duality between the $q$-Laplacian and $p$-tight forms as a version of the max flow min cut principle. 
For a reflexive Banach space $X$, we denote by $\hat X$ its dual.
If $J: X \to \RR \cup \{+\infty\}$ is a convex function, we introduce its \dfn{Legendre transform}
\begin{align*}
	\hat J: \hat X &\to \RR \cup \{+\infty\}\\
	\xi &\mapsto \sup_{x \in X} \langle \xi, x\rangle - f(x).
\end{align*}

\begin{proposition}\label{abstract convex analysis}
Let $X, Y$ be reflexive Banach spaces equipped with a bounded linear operator $\Lambda: X \to Y$ of trivial kernel.
Let $J: Y \to \RR \cup \{+\infty\}$ satisfy:
\begin{enumerate}
\item $J$ and $\hat J$ are strictly convex,
\item $J$ is lower semicontinuous and not identically $+\infty$,
\item if $|y| \to \infty$ in $Y$, then $J(y) \to +\infty$, and 
\item there exists a point $x \in X$ such that $J$ is continuous and finite at $\Lambda(x)$.
\end{enumerate}
Then:
\begin{enumerate}
\item There exists a unique minimizer $\underline x \in X$ of $J(\Lambda(x))$, and a unique maximizer $\overline \eta \in \hat Y$ of $-\hat J(-\eta)$.
\item We have \dfn{strong duality}
\begin{equation}\label{abstract strong duality}
J(\Lambda(\underline x)) = -\hat J(-\overline \eta).
\end{equation}
\end{enumerate}
\end{proposition}
\begin{proof}
This is the conjunction of several standard results in convex analysis.
Let $\Gamma_0(Y)$ denote the set introduced in \cite[Chapter I, Definition 3.1]{Ekeland99}.
Then by \cite[Chapter I, Proposition 3.1]{Ekeland99}, $J \in \Gamma_0(Y)$, so by \cite[Chapter III, Theorem 4.2]{Ekeland99} and \cite[Chapter III, (4.23)]{Ekeland99}, there exist minimizers which satisfy (\ref{abstract strong duality}).
By \cite[Chapter II, Proposition 1.2]{Ekeland99}, the minimizers of $J$ and $\hat J$ are unique.
Since $\Lambda$ has trivial kernel, minimizers of $J \circ \Lambda$ are also unique.
\end{proof}

One of the most famous corollaries of (\ref{abstract strong duality}) is the \dfn{max flow min cut theorem} of discrete optimization \cite[Chapter 7]{umesh2006algorithms}.
Since it seems rather unlikely that Thurston meant that $L = K$ should be a graph-theoretic fact, we interpret his assertion that ``a characterization of minimum stretch maps should be possible ... based on more general principles -- in particular the max flow min cut principle ...'' to refer to the more abstract formula (\ref{abstract strong duality}).

We now specialize to the problem at hand.
Let $\alpha \in H^1(M, \RR)$ be a cohomology class.
Recall that we write $[u] = \alpha$ to mean that $u$ is an $\alpha$-equivariant function on $\tilde M$.

We here consider the $q$-Laplacian
\begin{equation}\label{preprimal problem}
	\Min\{\|\dif u\|_{L^q(M_{\rm fun})}: u \in W^{1, q}(M_{\rm fun}), [u] = \alpha\},
\end{equation}
where $1 < q < \infty$; we identify two solutions if they agree by a constant.
To put (\ref{preprimal problem}) in the framework of Proposition \ref{abstract convex analysis}, we shall fix a point $0 \in M_{\rm fun}$, choose a representative $1$-form (which we also call $\alpha$), and find $f \in C^\infty(\tilde M)$ such that $\dif f = \Pi^*\alpha$ and $f(0) = 0$.
Thus (\ref{preprimal problem}) is equivalent to
\begin{equation}\label{primal problem}
	\Min\left\{\frac{1}{q} \int_{M_{\rm fun}} \star|\dif v + \Pi^* \alpha|^q: v \in W^{1, q}_0(M_{\rm fun})\right\}
\end{equation}
where we set $u = v + f$.

Let
$$J(\xi) := \frac{1}{q} \int_M \star|\xi + \alpha|^q,$$
defined for $\xi \in L^q(M, \Omega^1)$.
Since $v$ is traceless in (\ref{primal problem}), it is invariant, so $\dif v$ drops to a $1$-form on $M$, and (\ref{primal problem}) is the problem of minimizing $J(\dif v)$.
It is clear that $J$ meets the hypotheses of Proposition \ref{abstract convex analysis} with regards to the linear map
\begin{equation}\label{derivative on traceless}
\dif: X := W^{1, q}_0(M_{\rm fun}) \to L^q(M, \Omega^1) =: Y.
\end{equation}
By \cite[Chapter I, (4.9)]{Ekeland99} and \cite[Chapter I, Remark 4.1]{Ekeland99}, the Legendre transform
$$\hat J: \hat Y = L^p(M, \Omega^{d - 1}) \to \RR$$
of $J$, where $\frac{1}{p} + \frac{1}{q} = 1$, satisfies
\begin{equation}\label{Legendre transform}
\hat J(F) = \frac{1}{p} \int_M \star |F|^p + \int_M \alpha \wedge F
\end{equation}
and in particular is strictly convex.
Thus the convex dual problem of (\ref{primal problem}), namely the problem of maximizing $-\hat J(-F)$, is the problem
\begin{equation}\label{predual problem}
\Max\left\{- \frac{1}{p} \int_M \star |F|^p + \int_M \alpha \wedge F: F \in L^p(M, \Omega^{d - 1})\right\}.
\end{equation}

\begin{proposition}\label{convex duality}
Let $1 < p, q < \infty$ satisfy $\frac{1}{p} + \frac{1}{q} = 1$.
Then we have an isomorphism of cohomology groups
$$H^1(M, \RR) \ni \alpha \mapsto \rho \in H^{d - 1}(M, \RR),$$
so that the convex dual problem of the $\alpha$-equivariant $q$-Laplacian (\ref{preprimal problem}) with solution $u: \tilde M \to \RR$ is the $p$-tight problem (\ref{pMaxwell}) subject to $[F] = \rho$.
The problems (\ref{preprimal problem}) and (\ref{pMaxwell}) both have unique solutions, and we have the relations (\ref{extremality}) and (\ref{inverse extremality}), and the strong duality theorem (\ref{strong duality}).
\end{proposition}
\begin{proof}
By Proposition \ref{abstract convex analysis}, (\ref{preprimal problem}) and (\ref{predual problem}) both have unique solutions $u, F$ which are related by strong duality (\ref{abstract strong duality}) as (\ref{strong duality}).
By uniqueness, if we exhibit $\tilde F$ defined in terms of $u$ which satisfies (\ref{strong duality}) and is a maximizer of $-\hat J$, then $F = \tilde F$.
Following \cite[Chapter IV, (2.12)]{Ekeland99}, we define $\tilde F$ to satisfy (\ref{extremality}), so that (since $\star |\dif u|^q = \dif u \wedge \tilde F$)
\begin{align*}
\int_M \dif u \wedge \tilde F 
= \frac{1}{q} \int_M \star |\dif u|^q + \frac{1}{p} \int_M \dif u \wedge \tilde F.
\end{align*}
Moreover, if we decompose $\dif u = \dif v + \alpha$, then
\begin{align*}
-\hat J(\tilde F)
&= -\frac{1}{p} \int_M \star |\dif u|^q + \int_M \alpha \wedge |\dif u|^{q - 2} \star \dif u \\
&= -\frac{1}{p} \int_M \star |\dif u|^q + \int_M \star |\dif u|^q - \int_M \dif v \wedge |\dif u|^{q - 2} \star \dif u \\
&= \frac{1}{q} \int_M \star |\dif u|^q = J(\dif u)
\end{align*}
where we used the fact that $u$ solves the $q$-Laplace equation in the sense of distributions.
By strong duality (\ref{abstract strong duality}), it follows that $\tilde F$ is a maximizer of $-\hat J$, and therefore $F = \tilde F$; in particular, $F$ satisfies (\ref{extremality}).
Since $u$ solves the $q$-Laplace equation, (\ref{extremality}) implies that $\dif F = 0$.
We then define $\rho := [F]$.
On the other hand, $(p - 2)(q - 1) + q - 2 = 0$, so
$$|F|^{p - 2} F = |\dif u|^{(p - 2)(q - 1) + q - 2} \star \star \dif u = (-1)^{d - 1} \dif u$$
which gives (\ref{inverse extremality}).
By reflexivity, the above argument may also be reversed, so that the map $\alpha \mapsto \rho$ is invertible on the level of cohomology.
\end{proof}

%%%%%%%%%%%%%%%%%%%%%%%%%%%%%%
% \section{The \texorpdfstring{$\infty$-Maxwell equation}{infinity-Maxwell equation}}\label{EulerLagrange}
% We have the following Euler-Lagrange equation for forms with absolutely best comass.
% Because of the lack of a good analogue for viscosity solutions for $\infty$-elliptic systems, and because we did not show that $\infty$-tight forms have absolutely best comass, the equation can only really be interpreted in a formal sense, at least as far as we are aware.
% As such, we did not use it in the main body of this paper, but only include it as a curiosity item.

% % \todo{If we knew that $p$-Maxwell had good quantitative uniqueness, then we would have}
% % It remains to show that $A$ has absolutely best curl, so let $\Omega$ be a small ball and $B$ a $1$-form with $B|_{\partial \Omega} = A|_{\partial \Omega}$.
% % By a straightforward modification of the existence theorem, there exists a $p$-magnetic potential $B_p$ in Coulomb gauge with $B_p|_{\partial \Omega} = A|_{\partial \Omega}$ and $B \in C^{1 + \alpha}$.
% % By quantitative uniqueness
% % $$\|B_p - A\|_{C^0(\Omega)} \leq \|B_p - A_p\|_{C^0(\Omega)} + o(1) \lesssim \|A - A_p\|_{C^0(\partial \Omega)} + o(1) \ll 1.$$
% % Therefore $B_p \to A$ uniformly, and for $3 < q < p < \infty$ with $p$ dyadic,
% % $$\|\dif B_p\|_{L^q(\Omega)} \leq |\Omega|^{\frac{1}{q} -\frac{1}{p}} \|\dif B_p\|_{L^p(\Omega)} \leq |\Omega|^{\frac{1}{q} -\frac{1}{p}} \|\dif B\|_{L^p(\Omega)} \leq |\Omega|^{\frac{1}{q}} \|\dif B\|_{L^\infty(\Omega)}.$$
% % Then along a subsequence, $\dif B_p \to \dif A$ in $L^q(\Omega)$, so 
% % $$\|\dif A\|_{L^q(\Omega)} \leq |\Omega|^{\frac{1}{q}} \|\dif B\|_{L^\infty(\Omega)}.$$
% % Taking $q \to \infty$ we arrive at the conclusion that $F$ has absolutely best comass.



% The $\infty$-Maxwell equation has the following natural interpretation.

% \begin{corollary}
% Suppose that $F$ has absolutely best comass, regularity $C^1$, and no points with $F = 0$, and $N$ is a surface whose normal vector field is annihilated by $F$.
% Then $N$ is a minimal surface.
% \end{corollary}
% \begin{proof}
% Let $V$ be a tangent vector field to $N$. Then $V(|F|) = 0$, by (\ref{infinityMaxwell}).
% Therefore $|F|$ is constant along $N$, but $F$ is a continuous section of the area bundle of $N$, which is a real line bundle.
% It follows that $F$ is constant along $N$, and $F/|F|$ is the area form on $N$.
% In other words, $N$ is calibrated by $F$, and the claim follows from (\ref{calibrated surfaces are minimal}).
% \end{proof}

\printbibliography

\end{document}
