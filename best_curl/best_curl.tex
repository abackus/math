\documentclass[reqno,11pt]{amsart}
\usepackage[letterpaper, margin=1in]{geometry}
\RequirePackage{amsmath,amssymb,amsthm,graphicx,mathrsfs,url,slashed,subcaption}
\RequirePackage[usenames,dvipsnames]{xcolor}
\RequirePackage[colorlinks=true,linkcolor=Red,citecolor=Green]{hyperref}
\RequirePackage{amsxtra}
\usepackage{cancel}
\usepackage{tikz-cd}

% \setlength{\textheight}{9.3in} \setlength{\oddsidemargin}{-0.25in}
% \setlength{\evensidemargin}{-0.25in} \setlength{\textwidth}{7in}
% \setlength{\topmargin}{-0.25in} \setlength{\headheight}{0.18in}
% \setlength{\marginparwidth}{1.0in}
% \setlength{\abovedisplayskip}{0.2in}
% \setlength{\belowdisplayskip}{0.2in}
% \setlength{\parskip}{0.05in}
%\renewcommand{\baselinestretch}{1.05}

\title{Comass minimizers}
\author{Aidan Backus}
\date{\today}

\newcommand{\NN}{\mathbf{N}}
\newcommand{\ZZ}{\mathbf{Z}}
\newcommand{\QQ}{\mathbf{Q}}
\newcommand{\RR}{\mathbf{R}}
\newcommand{\CC}{\mathbf{C}}
\newcommand{\DD}{\mathbf{D}}
\newcommand{\PP}{\mathbf P}
\newcommand{\MM}{\mathbf M}
\newcommand{\II}{\mathbf I}
\newcommand{\Hyp}{\mathbf H}
\newcommand{\Sph}{\mathbf S}
\newcommand{\Group}{\mathbf G}
\newcommand{\GL}{\mathbf{GL}}
\newcommand{\Orth}{\mathbf{O}}
\newcommand{\SpOrth}{\mathbf{SO}}
\newcommand{\Ball}{\mathbf{B}}

\newcommand*\dif{\mathop{}\!\mathrm{d}}

\DeclareMathOperator{\card}{card}
\DeclareMathOperator{\dist}{dist}
\DeclareMathOperator{\supp}{supp}
\DeclareMathOperator{\tr}{tr}

\newcommand{\Leaves}{\mathscr L}
\newcommand{\Lagrange}{\mathcal L}
\newcommand{\Hypspace}{\mathscr H}

\newcommand{\Chain}{\underline C}

\newcommand{\Two}{\mathrm{I\!I}}

\newcommand{\normal}{\mathbf n}
\newcommand{\radial}{\mathbf r}
\newcommand{\evect}{\mathbf e}
\newcommand{\vol}{\mathrm{vol}}

\newcommand{\diam}{\mathrm{diam}}
\newcommand{\Ell}{\mathrm{Ell}}
\newcommand{\inj}{\mathrm{inj}}
\newcommand{\Lip}{\mathrm{Lip}}
\newcommand{\Riem}{\mathrm{Riem}}

\newcommand{\Min}{\mathrm{Min}}
\newcommand{\Max}{\mathrm{Max}}

\newcommand{\dfn}[1]{\emph{#1}\index{#1}}

\renewcommand{\Re}{\operatorname{Re}}
\renewcommand{\Im}{\operatorname{Im}}

\newcommand{\loc}{\mathrm{loc}}
\newcommand{\cpt}{\mathrm{cpt}}

\def\Japan#1{\left \langle #1 \right \rangle}

\newtheorem{theorem}{Theorem}[section]
\newtheorem{badtheorem}[theorem]{``Theorem"}
\newtheorem{prop}[theorem]{Proposition}
\newtheorem{lemma}[theorem]{Lemma}
\newtheorem{sublemma}[theorem]{Sublemma}
\newtheorem{proposition}[theorem]{Proposition}
\newtheorem{corollary}[theorem]{Corollary}
\newtheorem{conjecture}[theorem]{Conjecture}
\newtheorem{axiom}[theorem]{Axiom}
\newtheorem{assumption}[theorem]{Assumption}

\newtheorem{mainthm}{Theorem}
\renewcommand{\themainthm}{\Alph{mainthm}}

% \newtheorem{claim}{Claim}[theorem]
% \renewcommand{\theclaim}{\thetheorem\Alph{claim}}
\newtheorem*{claim}{Claim}

\theoremstyle{definition}
\newtheorem{definition}[theorem]{Definition}
\newtheorem{remark}[theorem]{Remark}
\newtheorem{example}[theorem]{Example}
\newtheorem{notation}[theorem]{Notation}

\newtheorem{exercise}[theorem]{Discussion topic}
\newtheorem{homework}[theorem]{Homework}
\newtheorem{problem}[theorem]{Problem}

\makeatletter
\newcommand{\proofpart}[2]{%
  \par
  \addvspace{\medskipamount}%
  \noindent\emph{Part #1: #2.}
}
\makeatother



\numberwithin{equation}{section}


% Mean
\def\Xint#1{\mathchoice
{\XXint\displaystyle\textstyle{#1}}%
{\XXint\textstyle\scriptstyle{#1}}%
{\XXint\scriptstyle\scriptscriptstyle{#1}}%
{\XXint\scriptscriptstyle\scriptscriptstyle{#1}}%
\!\int}
\def\XXint#1#2#3{{\setbox0=\hbox{$#1{#2#3}{\int}$ }
\vcenter{\hbox{$#2#3$ }}\kern-.6\wd0}}
\def\ddashint{\Xint=}
\def\dashint{\Xint-}

\usepackage[backend=bibtex,style=alphabetic,giveninits=true]{biblatex}
\renewcommand*{\bibfont}{\normalfont\footnotesize}
\addbibresource{best_curl.bib}
\renewbibmacro{in:}{}
\DeclareFieldFormat{pages}{#1}

\newcommand\todo[1]{\textcolor{red}{TODO: #1}}


\begin{document}
\begin{abstract}
	Minimizers of the comass
\end{abstract}

\maketitle

%%%%%%%%%%%%%%%%%%%%%%%%%%%%%%%%%%%%%%%%%%%%%%%%%%%%%%%

In this paper we study the problem of minimization of the \dfn{comass}
$$L(F) := \sup_{\sigma \in \Chain_2(M)} \frac{1}{|\sigma|} \int_\sigma F$$
of a closed $2$-form in a Riemannian $3$-fold $M$, subject to a constraint on the cohomology of $F$.
Here $\Chain_2(M)$ denotes the space of oriented $2$-chains in $M$, and $|\sigma|$ is the area of $\sigma$.
This problem is the analogue on $3$-manifolds of the problem of finding a best Lipschitz map on a surface.

For a discussion of our interest in this problem, see \S\ref{motivation}; for a more precise statement of our results without fluff, see \S\ref{results}.

% \tableofcontents

\section{Introduction}
\subsection{History and motivation} \label{motivation}

Sayeth Thurston \cite[Abstract]{Thurston98}:
\begin{quote}
I currently think that a characterization of minimal stretch maps should be possible in a considerably more general context ... and it should be feasible with a simpler proof based on more general principles -- in particular, the max flow mean cut principle, convexity, and $L^0 \leftrightarrow L^\infty$ duality.
\end{quote}

\subsection{Main theorems} \label{results}

\subsection{Acknowledgements}
George, Karen, Tom Goodwillie, Kaya Ferendo ...

This research was supported by the National Science Foundation's Graduate Research Fellowship Program under Grant No. DGE-2040433

%%%%%%%%%%%%%%%%%%%%%%%%%%%%%%%%%%%%%%%%%%
\section{Preliminaries}
\subsection{Notation}
We write $\Omega^\ell$, $Z^\ell$, and $B^\ell$ for the spaces of $\ell$-forms, closed $\ell$-forms ($\ell$-cocycles), and exact $\ell$-forms ($\ell$-coboundaries) respectively.
We reserve $H^\ell$ for cohomology and write $W^{s, p}$ for Sobolev spaces.

\subsection{The de Rham--Sobolev complex}
\begin{definition}
For $1 \leq p < \infty$ and $0 \leq \ell \leq d$, let $W^{1, p}_{\rm d}(M, \Omega^\ell)$ denote the space of measurable $\ell$-forms $\alpha$ for which the norm
$$\|\alpha\|_{W^{1, p}_{\rm d}}^p := \|\alpha\|_{L^p}^p + \|\dif \alpha\|_{L^p}^p$$
is finite.
For $p = \infty$ we instead define $W^{1, \infty}_{\rm d}(M, \Omega^\ell)$ by the norm 
$$\|\alpha\|_{W^{1, \infty}_{\rm d}} := \max(\|\alpha\|_{L^\infty}, \|\dif \alpha\|_{L^\infty}).$$
The \dfn{de Rham--Sobolev complex} $W^{1, p}_{\rm d}(M, \Omega^\bullet)$ is the chain complex of such spaces, with boundary maps 
$$\dif: W^{1, p}_{\rm d}(M, \Omega^\ell) \to W^{1, p}_{\rm d}(M, \Omega^{\ell + 1}).$$
\end{definition}

The de Rham--Sobolev complex is a well-defined elliptic complex, since 
$$\|\dif \alpha\|_{W^{1, p}_{\rm d}}^p = \|\dif \alpha\|_{L^p}^p + \|\dif^2 \alpha\|_{L^p}^p = \|\dif \alpha\|_{L^p}^p \leq \|\alpha\|_{W^{1, p}_{\rm d}}^p.$$
For $p = 2$, $d = 3$ this complex is well-known for its applications in electromagnetism and numerical analysis, and its constituent spaces are more commonly known as \cite[Chapter 2]{cessenat1996mathematical}
\begin{align*}
H(M, \text{curl}) &:= W^{1, 2}_{\rm d}(M, \Omega^1)\\
H(M, \text{div}) &:= W^{1, 2}_{\rm d}(M, \Omega^2).
\end{align*}

We write 
$$\|\alpha\|_{W^{1, p}}^p := \|\alpha\|_{L^p}^p + \|\dif \alpha\|_{L^p}^p + \|\dif^* \alpha\|_{L^p}^p$$
for the full $W^{1, p}$ norm, using the subscript $W^{1, p}_{\rm d}$ to emphasize the de Rham--Sobolev norm.

The utility of the de Rham--Sobolev complex is the \dfn{normal trace theorem}, which says for $1 < p < \infty$ that the pullback $\iota^*$ to the boundary is bounded \cite[50]{sohr2001navier}
\begin{equation}\label{pNormalTrace}
	\iota^*: W^{1, p}_{\rm d}(M, \Omega^\ell) \to W^{-\frac{1}{p}, p}(\partial M, \Omega^\ell).
\end{equation}
We shall need a version for $W^{1, \infty}_{\rm d}$, which we prove now.

\begin{proposition}[normal trace theorem]
The pullback is a bounded linear operator 
$$\iota^*: W^{1, \infty}_{\rm d}(M, \Omega^\ell) \to L^\infty(\partial M, \Omega^\ell).$$
In particular, for every $\alpha \in W^{1, \infty}_{\rm d}(M, \Omega^\ell)$ and every $\beta \in W^{1, 1}_{\rm d}(M, \Omega^{d - 1 - \ell})$, we have integration by parts:
\begin{equation}\label{Stokes trace}
	\int_{\partial M} \alpha \wedge \beta = \int_M \dif \alpha \wedge \beta - \alpha \wedge \dif \beta.
\end{equation}
\end{proposition}
\begin{proof}
Define an $\ell$-current $\iota^* \alpha$ on $\partial M$ by setting, for every $\beta \in C^\infty_\cpt(\partial M, \Omega^{d - 1 - \ell})$,
\begin{equation}\label{definition of trace}
\int_{\partial M} \iota^* \alpha \wedge \beta = \int_M \dif \alpha \wedge \beta - \alpha \wedge \dif \beta.
\end{equation}
To keep notation simple we shall just write $\alpha$ for $\iota^* \alpha$ when it is clear enough.

In (\ref{definition of trace}), we chose a smooth extension of $\beta$ to $M$.
However, the definition of $\iota^* \alpha$ is independent of the choice of extension.
First, if we replace $\beta$ by $\beta + \gamma$ where $\supp \gamma$ is compact in the interior of $M$, then by Stokes' theorem,
\begin{align*}
\int_M \dif \alpha \wedge (\beta + \gamma) - \alpha \wedge \dif (\beta + \gamma)
&= \int_M \dif \alpha \wedge \beta - \alpha \wedge \dif \beta + \int_M \dif(\alpha \wedge \gamma) \\
&= \int_M \dif \alpha \wedge \beta - \alpha \wedge \dif \beta.
\end{align*}
If we instead replaced $\beta$ by $\beta + \gamma$ where $\gamma$ is traceless, then we could approximate $\gamma$ in $W^{1, 1}$ by compactly supported replacements, and get the same result by dominated convergence.

In particular, we may, by the inverse trace theorem \cite[Teorema 1.II]{Gagliardo1957}, choose the extension $\beta$ to satisfy 
$$\|\beta\|_{W^{1, 1}(M)} \lesssim \|\beta\|_{L^1(N)}.$$
Therefore by (\ref{definition of trace}), 
\begin{align*}
\|\alpha\|_{L^\infty(N)} 
&= \sup_{\|\beta\|_{L^1(N)} = 1} \int_N \alpha \wedge \beta\\
&\leq \sup_{\|\beta\|_{L^1(N)} = 1} \|\dif \alpha\|_{L^\infty(M)} \|\beta\|_{L^1(M)} + \|\alpha\|_{L^\infty(M)} \|\dif \beta\|_{L^1(M)} \\
&\lesssim \sup_{\|\beta\|_{L^1(N)} = 1} \|\alpha\|_{W^{1, \infty}(M)} \|\beta\|_{W^{1, 1}(M)} \\
&\lesssim \|\alpha\|_{W^{1, \infty}(M)}. \qedhere
\end{align*}
\end{proof}

\begin{corollary}\label{trace on cycles}
Let $N$ be a smooth embedded hypersurface in $M$.
\begin{enumerate}
\item \label{pullback bounded} The pullback is a bounded linear operator
$$\iota^*_N: W^{1, \infty}_{\rm d}(M, \Omega^\ell) \to L^\infty(N, \Omega^\ell).$$
\item \label{integral continuous} For $F \in L^\infty(M, \Omega^{d - 1})$ such that $\dif F = 0$, if $N$ is closed, then the integral $\int_N F$ is well-defined, and depends continuously on $F$ for the weakstar topology on $L^\infty$.
\item \label{cohomology exists} For $F \in L^\infty(M, \Omega^{d - 1})$ such that $\dif F = 0$, the cohomology class of $F$ is well-defined, and depends  continuously on $F$ for the weakstar topology on $L^\infty$.
\end{enumerate}
\end{corollary}
\begin{proof}
To prove (\ref{pullback bounded}) we may use a partition of unity to work in a small ball $U$ in $N$, and then we may realize a collar neighborhood $V$ of $U$ in $M$ as a manifold-with-boundary with $U \subseteq \partial V$, and choose $\beta \in C^\infty_\cpt(M)$ in (\ref{Stokes trace}) to be a cutoff which is zero on $\partial V$ except along $U$.
The definition of
$$\iota^*_U: W^{1, \infty}_{\rm d}(V, \Omega^\ell) \to L^\infty(U, \Omega^\ell)$$
does not depend on the choice of $V$, since by Stokes' theorem the right-hand side of (\ref{Stokes trace}) will be the same.

To obtain (\ref{integral continuous}), we first observe that since $F \in L^\infty$ and $\dif F = 0$,
$$F \in W^{1, \infty}_{\rm d}(M, \Omega^{d - 1})$$
which is mapped to
$$L^\infty(N, \Omega^{d - 1}) \subseteq L^1(N, \Omega^{d - 1})$$
by (\ref{pullback bounded}) and the fact that $N$ is closed.
To obtain the continuity, we again use a collar neighborhood $V$ of $U$ and a cutoff $\beta$.
Since $F$ is closed, (\ref{Stokes trace}) reads 
$$\int_U F = \int_M F \wedge \dif \beta.$$
Since $\beta \in C^\infty_\cpt$, $\dif \beta \in L^1$, so if $F_n \to F$ in the weakstar topology on $L^\infty$, then $\int_M F_n \wedge \dif \beta \to \int_M F \wedge \dif \beta$, as desired.
Letting $N$ range over representatives of every homology class, we conclude (\ref{cohomology exists}) as a consequence of (\ref{integral continuous}).
\end{proof}

%%%%%%%%%%%%%%%%%%%%%%%%%%%%%%%%%%%%%%%%%
\subsection{Convex duality}
Let us, for a reflexive Banach space $X$, denote by $\hat X$ its dual.
If $J: X \to \RR \cup \{+\infty\}$ is a convex function, we introduce its \dfn{Legendre transform}
\begin{align*}
	\hat J: \hat X &\to \RR \cup \{+\infty\}\\
	\xi &\mapsto \sup_{x \in X} \langle \xi, x\rangle - f(x).
\end{align*}

\begin{definition}
Let $Y$ be a reflexive Banach space equipped with a mapping $\Lambda: X \to Y$.
A function $J: Y \to \RR \cup \{+\infty\}$ is said to be a \dfn{suitable convex function} for our purposes if:
\begin{enumerate}
\item $J$ is strictly convex,
\item $J$ is lower semicontinuous and not identically $+\infty$,
\item if $|y| \to \infty$ in $Y$, then $J(y) \to +\infty$, and 
\item there exists a point $x \in X$ such that $J$ is continuous and finite at $\Lambda(x)$.
\end{enumerate}
\end{definition}

\begin{proposition}\label{abstract convex analysis}
Let $X, Y$ be reflexive Banach spaces equipped with a bounded linear operator $\Lambda: X \to Y$ of trivial kernel.
Let $J: Y \to \RR \cup \{+\infty\}$ be a suitable convex function.
Then:
\begin{enumerate}
\item There exists a unique minimizer $\underline x \in X$ of $J(\Lambda(x))$, and a maximizer $\overline \eta \in \hat Y$ of $-\hat J(-\eta)$.
\item If $\hat J$ is strictly convex, then $\overline \eta$ is the unique maximizer of $-\hat J(-\eta)$.
\item We have \dfn{strong duality}
\begin{equation}\label{abstract strong duality}
J(\Lambda(\underline x)) = -\hat J(-\overline \eta).
\end{equation}
\end{enumerate}
\end{proposition}
\begin{proof}
This is the conjunction of several standard results in convex analysis.
Let $\Gamma_0(Y)$ denote the set introduced in \cite[Chapter I, Definition 3.1]{Ekeland99}.
Then by \cite[Chapter I, Proposition 3.1]{Ekeland99}, $J \in \Gamma_0(Y)$, so by \cite[Chapter III, Theorem 4.2]{Ekeland99} and \cite[Chapter III, (4.23)]{Ekeland99}, there exist minimizers which satisfy (\ref{abstract strong duality}).
By \cite[Chapter II, Proposition 1.2]{Ekeland99}, the minimizers of $J$ and $\hat J$ are unique.
Since $\Lambda$ has trivial kernel, minimizers of $J \circ \Lambda$ are also unique.
\end{proof}


%%%%%%%%%%%%%%%%%%%%%%%%%%%%%%%%%%%%%%%%%
\section{Convex duality for the \texorpdfstring{$q$-Laplacian}{q-Laplacian}}
Let $\Pi: \tilde M \to M$ be the universal covering, $M_{\rm fun} \subseteq \tilde M$ a fundamental domain, and
$$\alpha \in H^1(M, \RR)$$
a cohomology class.
Since $H_1(M, \RR)$ is the abelianization of $\pi_1(M)$, $\alpha$ is canonically identified with a representation of the fundamental group, which we also call
$$\alpha: \pi_1(M) \to \RR.$$
If a function $u: \tilde M \to \RR$ is $\alpha$-equivariant, we write $[u] = \alpha$.

We here consider the problem
\begin{equation}\label{preprimal problem}
	\Min\{\|\dif u\|_{L^q(M_{\rm fun})}: u \in W^{1, q}(M_{\rm fun}), [u] = \alpha\},
\end{equation}
where $1 < q < \infty$; we identify two solutions if they agree by a constant.
Taking Euler-Lagrange equations, we see that (\ref{preprimal problem}) is equivalent to the $q$-Laplacian 
\begin{equation}\label{qLaplace}
\begin{cases}
	\dif^*(|\dif u|^{q - 2} \dif u) = 0 \\
	[u] = \alpha.
\end{cases}
\end{equation}

To put (\ref{preprimal problem}) in the framework of Proposition \ref{abstract convex analysis}, we shall fix a point $0 \in M_{\rm fun}$, choose a representative $1$-form (which we also call $\alpha$), and solve 
$$\begin{cases}
\dif f = \Pi^* \alpha \\
v(0) = 0.
\end{cases}$$
Thus (\ref{preprimal problem}) is equivalent to
\begin{equation}\label{primal problem}
	\Min\left\{\frac{1}{q} \int_{M_{\rm fun}} \star|\dif v + \Pi^* \alpha|^q: v \in W^{1, q}_0(M_{\rm fun})\right\}
\end{equation}
where we set $u = v + f$.

Let
$$J(\xi) := \frac{1}{q} \int_M \star|\xi + \alpha|^q,$$
defined for $\xi \in L^q(M, \Omega^1)$.
Since $v$ is traceless in (\ref{primal problem}), it is invariant, so $\dif v$ drops to a $1$-form on $M$, and (\ref{primal problem}) is the problem of minimizing $J(\dif v)$.
It is clear that $J$ is a suitable convex function on $L^q(M, \Omega^1)$ when it is equipped with the map
\begin{equation}\label{derivative on traceless}
\dif: W^{1, q}_0(M_{\rm fun}) \to L^q(M, \Omega^1),
\end{equation}
and moreover the kernel of (\ref{derivative on traceless}) is trivial.
By \cite[Chapter I, (4.9)]{Ekeland99} and \cite[Chapter I, Remark 4.1]{Ekeland99}, the Legendre transform
$$\hat J: L^p(M, \Omega^{d - 1}) \to \RR$$
of $J$, where $\frac{1}{p} + \frac{1}{q} = 1$, satisfies
\begin{equation}\label{Legendre transform}
\hat J(F) = \frac{1}{p} \int_M \star |F|^p + \int_M \alpha \wedge F
\end{equation}
and in particular is strictly convex.
Thus the convex dual problem of (\ref{primal problem}), namely the problem of maximizing $-\hat J(-F)$, is the problem
\begin{equation}\label{predual problem}
\Max\left\{- \frac{1}{p} \int_M \star |F|^p + \int_M \alpha \wedge F: F \in L^p(M, \Omega^{d - 1})\right\}.
\end{equation}

% \begin{lemma}\label{EulerLagrange}
% A form $F \in L^p(M, \Omega^{d - 1})$ is a solution of (\ref{predual problem}) iff
% \begin{equation}\label{EL of hat G}
% |F|^{p - 2} F = (-1)^d \star \alpha.
% \end{equation}
% \end{lemma}
% \begin{proof}
% Let $(F_t)$ be an arbitrary variation and $G := \frac{\partial F_t}{\partial t}|_{t = 0}$. Then 
% $$0 = \frac{\dif}{\dif t} \hat f(F_t)\bigg|_{t = 0} = -\int_M \star |F|^{p - 2} \langle F, G \rangle + \alpha \wedge G.$$
% In particular, for any $G$,
% $$|F|^{p - 2} \langle F, G\rangle = -\star^{-1}(\alpha \wedge G) = \langle -\star^{-1} \alpha, G\rangle.$$
% Recalling that, since $\alpha$ is a $1$-form, $-\star^{-1} \alpha = (-1)^d \star \alpha$, and $G$ is arbitrary, it follows that (\ref{predual problem}) implies (\ref{EL of hat G}).
% The converse follows when one realizes that (\ref{EL of hat G}) has only one solution (this follows from some simple algebra).
% \end{proof}

In our next proposition we shall need the fact that the cohomology class of a form in $L^p(M, Z^{d - 1})$ is well-defined.
This follows from (\ref{pNormalTrace}), the fact that $L^p(M, Z^{d - 1}) \subset W^{1, p}_{\rm d}(M, \Omega^{d - 1})$, and the fact that $1 \in W^{\frac{1}{p}, q}(\sigma)$ on any $d-1$-cycle $\sigma$ (so that one can integrate a $W^{-\frac{1}{p}, p}(\sigma, \Omega^{d - 1}))$ form in the distributional sense).

\begin{proposition}\label{convex duality}
Let $1 < p, q < \infty$ satisfy $\frac{1}{p} + \frac{1}{q} = 1$.
Then we have an isomorphism of cohomology groups
\begin{align*}
H^1(M, \RR) &\to H^{d - 1}(M, \RR) \\
\alpha &\mapsto \rho,
\end{align*}
so that the convex dual problem of the $\alpha$-equivariant $q$-Laplacian (\ref{qLaplace}) with solution $u: \tilde M \to \RR$ is
\begin{equation}\label{pMaxwell}
\begin{cases}
	\dif F = 0, \\
	\dif^*(|F|^{p - 2} F) = 0, \\
	[F] = \rho.
\end{cases}
\end{equation}
The problems (\ref{qLaplace}) and (\ref{pMaxwell}) both have unique solutions, and we have the relations
\begin{align}
F &= |\dif u|^{q - 2} \star \dif u, \label{extremality} \\
\dif u &= (-1)^{d - 1} |F|^{p - 2} \star F, \label{inverse extremality}
\end{align}
and the strong duality theorem 
\begin{equation}\label{strong duality}
\frac{1}{q} \int_M \star |\dif u|^q + \frac{1}{p} \int_M \star |F|^p = \int_M \dif u \wedge F.
\end{equation}
\end{proposition}
\begin{proof}
Since $J$ is a suitable convex function, $\hat J$ is strictly convex, and (\ref{derivative on traceless}) has trivial kernel, we may apply Proposition \ref{abstract convex analysis} to see that (\ref{qLaplace}) and (\ref{predual problem}) both have unique solutions $u, F$ which are related by strong duality (\ref{abstract strong duality}) as (\ref{strong duality}).
By uniqueness, if we exhibit $\tilde F$ defined in terms of $u$ which satisfies (\ref{strong duality}) and is a maximizer of $-\hat J$, then $F = \tilde F$.
Following \cite[Chapter IV, (2.12)]{Ekeland99}, we define $\tilde F$ to satisfy (\ref{extremality}), so that 
\begin{align*}
\int_M \dif u \wedge \tilde F 
&= \int_M \star |\dif u|^q \\
&= \frac{1}{q} \int_M \star |\dif u|^q + \frac{1}{p} \int_M \star |\dif u|^p \\
&= \frac{1}{q} \int_M \star |\dif u|^q + \frac{1}{p} \int_M \dif u \wedge \tilde F.
\end{align*}
Moreover, if we decompose 
$$\dif u = \dif v + \alpha$$
then
\begin{align*}
-\hat J(\tilde F)
&= -\frac{1}{p} \int_M \star |\dif u|^q + \int_M \alpha \wedge |\dif u|^{q - 2} \star \dif u \\
&= -\frac{1}{p} \int_M \star |\dif u|^q + \int_M \star |\dif u|^q - \int_M \dif v \wedge |\dif u|^{q - 2} \star \dif u \\
&= \frac{1}{q} \int_M \star |\dif u|^q + \int_M v \dif(|\dif u|^{q - 2} \star \dif u) \\
&= \frac{1}{q} \int_M \star |\dif u|^q = J(\dif u)
\end{align*}
where we used the fact that $\frac{1}{q} = 1 - \frac{1}{p}$, and the fact that $u$ solves the $q$-Laplace equation.
By strong duality (\ref{abstract strong duality}), it follows that $\tilde F$ is a maximizer of $-\hat J$, and therefore $F = \tilde F$; in particular, $F$ satisfies (\ref{extremality}).
Since $u$ solves the $q$-Laplace equation, (\ref{extremality}) implies that $\dif F = 0$.
We then define $\rho := [F]$.
On the other hand, 
$$(p - 2)(q - 1) + q - 2 = 0,$$
so following \cite[Lemma 3.2]{daskalopoulos2020transverse},
$$|F|^{p - 2} F = |\dif u|^{(p - 2)(q - 1) + q - 2} \star \star \dif u = (-1)^{d - 1} \dif u$$
which gives (\ref{inverse extremality}) and the equation
$$\dif^*(|F|^{p - 2} F) = 0.$$
By reflexivity, the above argument may also be reversed, so that the map $\alpha \mapsto \rho$ is invertible on the level of cohomology.
\end{proof}

\begin{definition}
If $u$ is a solution of the $q$-Laplacian, $\frac{1}{p} + \frac{1}{q} = 1$, and $F$ satisfies (\ref{extremality}), we call $u$ the \dfn{conjugate $q$-harmonic} of $F$.
\end{definition}

If $u_q$ is a conjugate $q$-harmonic for $F_p$ for every $p \gg 1$ and $\frac{1}{p} + \frac{1}{q} = 1$, it may be that in a suitable function space we may take the limit $p \to \infty$ to obtain a $d-1$-form $F$ and a function $u$.
In this situation, we still call $u$ a conjugate $1$-harmonic, though this is only literally true in a formal sense.
In particular, since $L^1, L^\infty$ are not reflexive Banach spaces, we do not have uniqueness or an isomorphism theorem -- at least not with the methods we have considered here.

%%%%%%%%%%%%%%%%%%%%%%%%%%%%%%%%%%%%%%%%%%

\section{Best comass and \texorpdfstring{$\infty$-light forms}{infinity-light forms}}
\subsection{\texorpdfstring{$p$-light forms}{p-light forms}}
Let $M$ be a closed Riemannian threefold, and fix a cohomology class $\rho$.

\begin{definition}
Let $1 < p < \infty$ and let $F_p$ be a closed $2$-form with $[F_p] = \rho$.
We call $F_p$ a \dfn{$p$-light form} if it is a minimizer of $\|F_p\|_{L^p}$ among all $2$-forms representing $\rho$.
\end{definition}

We call these forms ``light'' because as $p \to \infty$ they become ``not massive'', in the sense that, as we shall later show, they will define a minimizing sequence for the comass.
A straightforward differentiation shows that the Euler-Lagrange equations for $p$-light forms are (\ref{pMaxwell}).

\begin{lemma}
Let $F_p$ be a $p$-light form, and let $B$ range over closed $2$-forms cohomologous to $F_p$. Then
\begin{equation}\label{infinity magnetic rules p magnetic}
	\|F_p\|_{L^p} \leq |M|^{1/p} \inf_B \|B\|_{L^\infty}.
\end{equation}
\end{lemma}
\begin{proof}
By H\"older's inequality and the fact that $F_p$ is $p$-light,
$$\|F_p\|_{L^p} \leq \|B\|_{L^p} \leq |M|^{1/p} \|B\|_{L^\infty},$$
hence the same holds for the infimum.
\end{proof}

We put any norm on $H^2(M, \RR)$ (as all are equivalent), hence $|\alpha|$ makes sense for $\alpha \in H^2(M, \RR)$.

\begin{proposition}\label{existence for p}
Assume $p > 2$ and $\rho \in H^2(M, \RR)$.
Then:
\begin{enumerate}
\item There exists a unique $p$-light form $F_p$ whose cohomology class is $\rho$.
\item One has
\begin{equation}\label{Sobolev bounds for p}
	\|F_p\|_{L^p} \lesssim |\rho|.
\end{equation}
The constant is independent of $p$.
\item $F_p$ is H\"older continuous.
\end{enumerate}
\end{proposition}
\begin{proof}
The existence and uniqueness follows from Proposition \ref{convex duality} and the fact that the Euler-Lagrange equations for $p$-light forms are given by (\ref{pMaxwell}).
From (\ref{pMaxwell}), the fact that $p > 2$, and \cite{Uhlenbeck77}, we deduce that $F_p$ is H\"older continuous.

Now select a basis $\xi_1, \dots, \xi_r$ of $H^2(M, \RR)$, and apply the above argument to obtain $p$-light forms $G_i$ representing $\xi_i$ for each $i \in \{1, \dots, r\}$.
Then $\|G_i\|_{L^p}$ is bounded indepdendently of $p$ by (\ref{infinity magnetic rules p magnetic}).
Decompose
$$\alpha = \sum_{i=1}^r \alpha_i \xi_i.$$
Using the fact that $F_p$ is $p$-light and cohomologous to $\sum_i \alpha_i G_i$,
$$\|F_p\|_{L^p} \leq \left\|\sum_{i=1}^r \alpha_i G_i\right\|_{L^p} \leq \sum_{i=1}^r |\alpha_i| \|G_i\|_{L^p} \lesssim |\alpha|$$
which proves (\ref{Sobolev bounds for p}).
\end{proof}

\todo{We do not get $C^\infty$ regularity on sets $\Subset \{|F_p| \neq 0\}$, or so it seems.}
Why?
Expanding out (\ref{pMaxwell}) with $F_{ij} = \partial_i A_j - \partial_j A_i$,
$$0 = \partial^j(|F|^{p - 2} \partial_j A_i) - |\dif A|^{p - 2} \partial_i (\dif^* A) - |\dif A|^{p - 2} [\partial^j, \partial_i] A_j - \partial^j(|\dif A|^{p - 2}) \partial_i A_j.$$
The first term is an elliptic operator with H\"older coefficients, the second can be gauged away, and the third is H\"older since $[\partial^j, \partial_i]$ is a connection coefficient of the metric.
But the last term is bad.
\todo{Maybe we can get rid of it using paraproducts?}

%%%%%%%%%%%%%%%%%%%%%%%%%%%%
\subsection{Forms of best comass}
What follows is the main definition of this paper.
For a closed $2$-form $F$ and a subdomain $\Omega$, we introduce the \dfn{comass} 
$$L_\Omega(F) := \sup_{\sigma \in \Chain_2(\Omega)} \frac{1}{|\sigma|} \int_\sigma F,$$
and let $L(F) := L_M(F)$.
The integral in the definition of comass is well-defined, and depends continuously on $F$ in the weakstar topology on $L^\infty(M)$, by Corollary \ref{trace on cycles}(\ref{integral continuous}) and the fact that $F$ is closed.

\begin{definition}
	Let $\alpha \in H^2(M, \RR)$ and let $F$ be a closed $2$-form representing $\alpha$.
\begin{enumerate}
	\item We say that $F$ has \dfn{best comass} if $F$ is a minimizer of $L$ among all $F$ representing $\alpha$.
	\item We say that $F$ has \dfn{absolutely best comass} if for every small open ball $\Omega$, $F$ is a minimizer of $L_\Omega$ among all closed $2$-forms $B$ with $B|_{\partial \Omega} = F|_{\partial \Omega}$.
\end{enumerate}
\end{definition}

We will be interested in the points at which $F$ attains its comass.
However, $F \in L^\infty$, so $F$ is both only defined almost everywhere, and not norm-approximable by smooth functions.
So as a proxy for $|F|$, which may fail to be defined on a null set, we use the local comass, which is defined everywhere.

\begin{definition}
The \dfn{local comass} of a $2$-form $F$ at $x \in M$ is 
$$L(F, x) := \limsup_{\varepsilon \to 0} \sup_{\sigma \in \Chain_2(B_\varepsilon(x))} \frac{1}{|\sigma|} \int_\sigma F.$$
\end{definition}

One has
$$L(F, x) = \limsup_{\varepsilon \to 0} L_{B_\varepsilon(x)}(F)$$
but $L_{B_\varepsilon(x)}(F)$ is increasing in $\varepsilon$ (since it's a supremum over a set which grows in $\varepsilon$).
So the limit superior is actually a limit and an infimum:
$$L(F, x) = \lim_{\varepsilon \to 0} L_{B_\varepsilon(x)}(F) = \inf_{\varepsilon > 0} L_{B_\varepsilon(x)}(F)$$
and in particular $L(F, x) \leq L(F)$.

It's convenient to test the local comass against a more restrictive class of $2$-chains.
Motivated by lattice gauge theory \cite{Gupta98}, we fix a orthonormal basis $(\partial_i)$ of the tangent space $T_x M$, and define the \dfn{plaquette} $R_{ij}^\varepsilon(x)$ to be the exponential pushforward of a square $[0, \varepsilon \partial_i] \times [0, \varepsilon \partial_j]$ to $M$.

% We write $x + v := \exp_x(v)$ whenever $v \in T_x M$.

% \begin{lemma}
% Let $A$ be a continuous $1$-form and $x \in M$. Then as $\varepsilon \to 0$,
% \begin{equation} \label{riemann plaquette}
% 	\frac{1}{|R_{ij}^\varepsilon(x)|} \int_{\partial R_{ij}^\varepsilon(x)} A = \frac{A_i(x) + A_j(x + \varepsilon \partial_i) - A_i(x + \varepsilon \partial_j) - A_j(x)}{\varepsilon} + o(1).
% \end{equation}
% \end{lemma}
% \begin{proof}
% The metric introduces corrections of size $O(\varepsilon^2)$, so we may discard it and assume that $R_{ij}^\varepsilon$ is a rectangle, bounded by line segments 
% $$\partial R_{ij}^\varepsilon = \gamma_1 + \gamma_2 - \gamma_3 - \gamma_4,$$
% where $\gamma_1$ is the line segment between $x$ and $x + \varepsilon \partial_i$ (say).
% Since $A$ is continuous, 
% $$\int_{\gamma_1} A = \int_0^\varepsilon A_i(x + t\partial_i) \dif t = \varepsilon A_i(x) + o(\varepsilon).$$
% A similar computation holds for the other $\gamma_a$ and implies 
% \begin{align*}
% \int_{\partial R_{ij}^\varepsilon}
% &= \varepsilon A_i(x) + \varepsilon A_j(x + \varepsilon \partial_i) - \varepsilon A_i(x + \varepsilon \partial_j) - \varepsilon A_j(x) + o(\varepsilon) \\
% &= \frac{\varepsilon^2}{\varepsilon}(A_i(x) + A_j(x + \varepsilon \partial_i) - A_i(x + \varepsilon \partial_j) - A_j(x) + o(1)).
% \end{align*}
% Since $|R_{ij}^\varepsilon| = \varepsilon^2$, we are done.
% \end{proof}

% \begin{lemma}
% Let $F \in L^\infty(M, Z^2)$ and fix an orthonormal frame on the tangent bundle. Then for almost every $x \in M$,
% \begin{equation}\label{dA is like Lipschitz constant}
% 	F_{ij}(x) = \lim_{\varepsilon \to 0} \frac{1}{|R^\varepsilon_{ij}(x)|} \int_{R^\varepsilon_{ij}(x)} F.
% \end{equation}
% \end{lemma}
% \begin{proof}
% Let $F^\delta \to F$ be an approximation of $F$ by closed, continuous $2$-forms, which converges in the weakstar topology on $L^\infty(M)$ and satisfies $\|F^\delta - F\|_{L^1} \leq \delta$.
% Let $\kappa > 0$ and consider the set $Z_\kappa$ of all $x \in M$ such that 
% $$\left|F_{ij}(x) - \lim_{\varepsilon \to 0} \frac{1}{|R^\varepsilon_{ij}(x)|} \int_{R^\varepsilon_{ij}(x)} F\right| \geq \kappa.$$
% We can estimate on $Z_\kappa$,
% \begin{align*}
% \kappa
% &\leq |F_{ij}(x) - F_{ij}^\delta(x)| + \lim_{\varepsilon \to 0} \frac{1}{|R^\varepsilon_{ij}(x)|} \left|\int_{R^\varepsilon_{ij}(x)} F - F^\delta\right| \\
% &\qquad + \lim_{\varepsilon \to 0} \left|F_{ij}(x)^\delta - \frac{1}{|R^\varepsilon_{ij}(x)|^2} \int_{R^\varepsilon_{ij}(x)} F^\delta\right| \\
% &=: I_1 + I_2 + I_3.
% \end{align*}
% By Markov's inequality,
% $$\left|\left\{I_1 \geq \frac{\kappa}{3}\right\}\right| \lesssim \frac{\delta}{\kappa}.$$
% We control $I_2$ by selecting $A, A^\delta$ defined near $x$, in Coulomb gauge with $\dif A = F$, $\dif A^\delta = F^\delta$, hence by Stokes' theorem and Poincar\'e's inequality,
% $$\left|\int_{R^\varepsilon_{ij}(x)} F - F^\delta\right| \leq \int_{\partial R^\varepsilon_{ij}(x)} \star |A - A^\delta| \leq |\partial R^\varepsilon_{ij}(x)| \|A - A^\delta\|_{C^0} \lesssim |\partial R^\varepsilon_{ij}(x)| \varepsilon^{-\frac{1}{4}} \|F - F^\delta\|_{L^4} \lesssim \varepsilon^{\frac{3}{4}} \delta.$$
% Since $F^\delta \to F$ in the weakstar topology on $L^\infty(M)$, Corollary \ref{trace on cycles}(\ref{integral continuous}) implies that $I_2 \to 0$ as $\delta \to 0$.
% For $\varepsilon$ small, we can write $F^\delta = \dif A^\delta$ on a neighborhood of $R^\varepsilon_{ij}(x)$.
% Let us write for $(x, v) \in TM$, $x + v := \exp_x(v)$.
% By Stokes' theorem,
% \begin{align*}
% \lim_{\varepsilon \to 0} \frac{1}{|R^\varepsilon_{ij}(x)|} \int_{R^\varepsilon_{ij}(x)} F^\delta
% &= \lim_{\varepsilon \to 0} \frac{1}{|R^\varepsilon_{ij}(x)|} \int_{\partial R^\varepsilon_{ij}(x)} A^\delta \\
% &= \lim_{\varepsilon \to 0} \frac{A_i^\delta(x) + A_j^\delta(x + \varepsilon \partial_i) - A_i^\delta(x + \varepsilon \partial_j) - A_j^\delta(x)}{\varepsilon} \\
% &= F_{ij}^\delta(x).
% \end{align*}
% Taking $\varepsilon \to 0$ we conclude that $I_3 = 0$.
% \end{proof}

We prove the following analogue of \cite[Lemma 4.2]{Crandall2008} for best Lipschitz functions.

\begin{proposition}\label{crandall}
Let $F \in L^\infty(M, Z^2)$. Then:
\begin{enumerate}
\item The local comass $L(F, \cdot)$ is upper semicontinuous. \label{crandall usc}
% \item If $F(x)$ exists then $L(F, x) \geq |\dif A(x)|$. \label{crandall dA bounds LA}
% \item If $L(A, x) = 0$, then $\dif A(x)$ exists and $\dif A(x) = 0$. \label{crandall zero LA implies diffble}
\item If $\sigma \in \Chain_2$ then \label{crandall best curl is ABC}
$$\frac{1}{|\sigma|} \int_\sigma F \leq \sup_{x \in \sigma} L(F, x).$$
\item The local comass is bounded, and \label{crandall linfinity}
$$L(F) = \sup_{x \in M} L(F, x) = \|F\|_{L^\infty}.$$
\end{enumerate}
\end{proposition}
\begin{proof}
First let $x^n \to x$, so for $n$ large, $\Omega_n := B_{r - |x - x^n|}(x^n) \subseteq B_r(x)$, hence
$$L(F, x^n) \leq L_{\Omega_n}(F) \leq L_{B_r(x)}(F).$$
Therefore 
$$\limsup_{n \to \infty} L(F, x^n) \leq \inf_{r > 0} L_{B_r(x)}(F) = L(F, x),$$
which proves (\ref{crandall usc}).
	
% We now bound for a net of plaquettes $R_{ij}^\varepsilon(x)$ using (\ref{riemann plaquette})
% \begin{equation}\label{difference quotients}
% 	\limsup_{\varepsilon \to 0} \frac{|A_j(x + \varepsilon \partial_i) - A_j(x) - A_i(x + \varepsilon \partial_j) + A_i(x)|}{\varepsilon}
% = \limsup_{\varepsilon \to 0} \frac{1}{|R_{ij}^\varepsilon(x)|} \left|\int_{\partial R_{ij}^\varepsilon(x)} A\right| \leq L(A, x).
% \end{equation}
% If the first limit superior is actually a limit, then it is the definition of $|\dif A_{ij}(x)|$.
% So if $\dif A_{ij}(x)$ exists we conclude $|\dif A_{ij}(x)| \leq L(A, x)$, which proves (\ref{crandall dA bounds LA}).
% On the other hand, the corresponding limit \emph{inferior} must be nonnegative, so if $L(A, x) = 0$, the first limit superior in (\ref{difference quotients}) is actually a limit and we obtain $\dif A_{ij}(x) = 0$, proving (\ref{crandall zero LA implies diffble}).

Now let $\sigma$ be a $2$-chain and fix an orthonormal frame $(\partial_i)$.
We may write $\sigma = \sum_{n=1}^N \sigma_n$ where $\sigma_n$ is a $2$-cell of the form $x^3_n = 0$ for some coordinates $(x^1_n, x^2_n, x^3_n)$.
In particular, for any $\varepsilon > 0$ we may write $\sigma_n$ as a sum of plaquettes $R_{12}^\delta(x)$ with respect to the coordinates $(x^1_n, x^2_n, x^3_n)$, where $0 < \delta < \varepsilon$ and $x \in \sigma_n$.
Let us denote such plaquettes as $P_{n1}(\varepsilon), \dots, P_{nK(n, \varepsilon)}(\varepsilon)$. Then for any $\varepsilon > 0$, 
$$\frac{1}{|\sigma|} \left|\int_\sigma F\right| \leq \sum_{n=1}^N \sum_{k=1}^{K(n, \varepsilon)} \frac{|P_{nk}(\varepsilon)|}{|\sigma|} \frac{1}{|P_{nk}(\varepsilon)|} \left|\int_{P_{nk}(\varepsilon)} F\right|.$$
But $P_{nk} \in \Chain_2(B_\varepsilon(x_{nk}(\varepsilon)))$ for some $x_{nk}(\varepsilon) \in \sigma$, so it follows that 
\begin{align*}
\frac{1}{|\sigma|} \left|\int_\sigma F\right|
&\leq \limsup_{\varepsilon \to 0} \sum_{n=1}^N \sum_{k=1}^{K(n, \varepsilon)} \frac{|P_{nk}(\varepsilon)|}{|\sigma|} L(F, x_{nk}(\varepsilon)) \\
&\leq \limsup_{\varepsilon \to 0} \sup_{x \in \sigma} L(A, x)  \sum_{n=1}^N \sum_{k=1}^{K(n, \varepsilon)} \frac{|P_{nk}(\varepsilon)|}{|\sigma|} \\
&= \sup_{x \in \sigma} L(F, x)
\end{align*}
which proves (\ref{crandall best curl is ABC}).

To prove (\ref{crandall linfinity}) we recall that trivially,
$$L(F, x) \leq L(F) \leq \|F\|_{L^\infty}.$$
Moreover, (\ref{crandall best curl is ABC}) shows that
$$L(F) = \sup_{\sigma \in \Chain_2(M)} \frac{1}{|\sigma|} \int_\sigma F \leq \sup_{\sigma \in \Chain_2(M)} \sup_{x \in \sigma} L(F, x) = \sup_{x \in M} L(F, x).$$
Finally we must estimate $\|F\|_{L^\infty} \leq L(F)$.
To this end, let
$$S_{ij}^\varepsilon(x) := \exp_x([0, \varepsilon \partial_i] \times [0, \varepsilon \partial_j] \times [-\varepsilon \partial_i \times \partial_j, \varepsilon \partial_i \times \partial_j]).$$
Then $S_{ij}^\varepsilon(x)$ has uniformly bounded eccentricity as $\varepsilon \to 0$ in normal coordinates, so by the Lebesgue differentiation theorem, for almost every $x \in M$,
$$|F(x)|^2 = \lim_{\varepsilon \to 0} \frac{1}{2\varepsilon} \left|\sum_{i < j} \frac{1}{|R_{ij}^\varepsilon(x)|} \int_{S_{ij}^\varepsilon(x)} F\right|^2.$$
In normal coordinates we have the factorization 
$$S_{ij}^\varepsilon(x) = R_{ij}^\varepsilon(x),$$
so by Fubini's theorem, 
\begin{align*}
|F(x)|^2 
&= \lim_{\varepsilon \to 0} \frac{1}{2\varepsilon} \left|\int_{-\varepsilon}^\varepsilon \sum_{i < j} \frac{1}{|R_{ij}^\varepsilon(x)|} \int_{R_{ij}^\varepsilon(x)} F\right|^2 \leq \frac{1}{2\varepsilon} \left|\int_{-\varepsilon}^\varepsilon L(F)\right|^2 \\
& = L(F)^2. \qedhere
\end{align*}
% By (\ref{dA is like Lipschitz constant}),
% $$\|F\|_{L^\infty}^2 \leq \sup_{x \in M} \limsup_{\varepsilon \to 0} \sum_{i < j} \frac{1}{|R^\varepsilon_{ij}(x)|^2} \left|\int_{R^\varepsilon_{ij}(x)} F\right|^2.$$
% Here the left-hand side is a supremum over almost all of $M$, while the right-hand side is a supremum over all of $M$, including those points for which the limit does not exist as $\varepsilon \to 0$ (but the limit superior does exist).
% Now we select for each $x \in M$, a subsequence of $\varepsilon \to 0$ which attains the limit superior, and choose a frame $(\partial_i)$ so that (possibly along a further subsequence), the averaged values of $F_{ij}(x)$ are $0$ unless $i,j$ are $1,2$.
% Then most of the terms in the sum go away, and we are left with 
% $$\limsup_{\varepsilon \to 0} \sum_{i < j} \frac{1}{|R^\varepsilon_{ij}(x)|^2} \left|\int_{R^\varepsilon_{ij}(x)} F\right|^2 = \limsup_{\varepsilon \to 0} \frac{1}{|R^\varepsilon_{12}(x)|^2} \left|\int_{R^\varepsilon_{12}(x)} F\right|^2.$$
% We then set $\sigma_\varepsilon(x) := R^\varepsilon_{12}(x)$ as defined in this frame.
% In particular, 
% \begin{align*}
% 	\|F\|_{L^\infty} &\leq \sup_{x \in M} \sup_{\varepsilon > 0} \frac{1}{|\sigma_\varepsilon(x)|} \left|\int_{\sigma_\varepsilon(x)} F\right| \leq \sup_{\sigma \in \Chain_2(M)} \frac{1}{|\sigma|} \int_\sigma F = L(F). \qedhere 
% \end{align*}
\end{proof}

%%%%%%%%%%%%%%%%%%%%%%%%%%%%%
\subsection{\texorpdfstring{$p \to \infty$}{The limit as p goes to infinity}}
\begin{theorem}\label{existence infinity}
Let $\rho \in H^2(M, \RR)$.
For each $p \geq 2$, let $F_p$ be the $p$-light form representing $\rho$. Then:
\begin{enumerate}
\item There exists a closed $2$-form $F$ such that $F_p \to F$ weakly in $L^q$ along a subsequence, for any $3 < q < \infty$.
\item $F$ is a best comass representation of $\rho$.
\item One has 
\begin{equation}\label{Sobolev bounds for infinity}
	\|F\|_{L^\infty} \lesssim |\rho|.
\end{equation}
\end{enumerate}
\end{theorem}
\begin{proof}
We roughly follow \cite[\S3]{Lindqvist14}.
Let $q > 3$, and let $B$ be an $L^\infty$ representative of $\rho$.
By H\"older's inequality and (\ref{infinity magnetic rules p magnetic}),
\begin{equation}\label{uniform bounds in p by best curl}
	\|F_p\|_{L^q} \leq |M|^{\frac{1}{q} - \frac{1}{p}} \|F_p\|_{L^p} \leq |M|^{\frac{1}{q} - \frac{1}{p}} \|B\|_{L^\infty}.
\end{equation}
Thus a compactness argument gives $F_p \to F$ for some $2$-form $F$, weakly in $L^q$, and by Fatou's lemma, 
$$\|F\|_{L^q} \leq \liminf_{p \to \infty} \|F_p\|_{L^q} \leq |M|^{\frac{1}{q}} \|B\|_{L^\infty}.$$
Diagonalizing, we may assume that $F_p \to F$ weakly in $L^q$ for every such $q$, and taking $q \to \infty$, we conclude 
\begin{equation}\label{infinity magnetics have best curl}
	\|F\|_{L^\infty} \leq \|B\|_{L^\infty}.
\end{equation}
By Corollary \ref{trace on cycles}(\ref{cohomology exists}), $[F] = \rho$.
So by Proposition \ref{crandall}(\ref{crandall linfinity}) and the fact that $B$ was arbitrary in (\ref{infinity magnetics have best curl}), $F$ has best comass.
Moreover, taking the limit as $p \to \infty$ in (\ref{Sobolev bounds for p}), we obtain (\ref{Sobolev bounds for infinity}).
\end{proof}

\begin{definition}
The limiting $2$-form $F$ in Theorem \ref{existence infinity} is called an \dfn{$\infty$-light form}.
\end{definition}

\todo{If we knew that $p$-Maxwell had good quantitative uniqueness, then we would have}
It remains to show that $A$ has absolutely best curl, so let $\Omega$ be a small ball and $B$ a $1$-form with $B|_{\partial \Omega} = A|_{\partial \Omega}$.
By a straightforward modification of the existence theorem, there exists a $p$-magnetic potential $B_p$ in Coulomb gauge with $B_p|_{\partial \Omega} = A|_{\partial \Omega}$ and $B \in C^{1 + \alpha}$.
By quantitative uniqueness
$$\|B_p - A\|_{C^0(\Omega)} \leq \|B_p - A_p\|_{C^0(\Omega)} + o(1) \lesssim \|A - A_p\|_{C^0(\partial \Omega)} + o(1) \ll 1.$$
Therefore $B_p \to A$ uniformly, and for $3 < q < p < \infty$ with $p$ dyadic,
$$\|\dif B_p\|_{L^q(\Omega)} \leq |\Omega|^{\frac{1}{q} -\frac{1}{p}} \|\dif B_p\|_{L^p(\Omega)} \leq |\Omega|^{\frac{1}{q} -\frac{1}{p}} \|\dif B\|_{L^p(\Omega)} \leq |\Omega|^{\frac{1}{q}} \|\dif B\|_{L^\infty(\Omega)}.$$
Then along a subsequence, $\dif B_p \to \dif A$ in $L^q(\Omega)$, so 
$$\|\dif A\|_{L^q(\Omega)} \leq |\Omega|^{\frac{1}{q}} \|\dif B\|_{L^\infty(\Omega)}.$$
Taking $q \to \infty$ we arrive at the conclusion.

We have the following Euler-Lagrange equation for $\infty$-light forms.
Because of the lack of a good analogue for viscosity solutions for $\infty$-elliptic systems, \todo{and because we did not show that $\infty$-light forms have absolutely best comass}, the equation can only really be interpreted in a formal sense, at least as far as we are aware.
As such, we shall not use it in the sequel, but only include it as a curiosity item.

\begin{proposition}
Suppose that $F$ has absolutely best comass, regularity $C^1$, and no points with $F = 0$. Then
\begin{equation}\label{infinityMaxwell}
	F^{ij} \partial_i |F| = 0.
\end{equation}
\end{proposition}
\begin{proof}
For a covariant $2$-tensor $T$, let $T^{\rm as}$ be its antisymmetrization, and let
$$f(x, T) := |T^{\rm as}|_{g(x)}.$$
Working locally, we may write $F = \dif A$ for some $A$, which we may assume to be Coulomb gauge and therefore $C^2$.
Since $A$ has absolutely best curl and $(\nabla A)^{\rm as} = \dif A$, $A$ is an absolute minimizer (see \cite[Definition 5.1]{Barron2001}) of the essential supremum of $f(\cdot, \nabla A)$.
By the Euler-Lagrange-Aronsson formula \cite[Theorem 5.2]{Barron2001},
\begin{equation}\label{ELA}
	\left\langle \frac{\partial f}{\partial T}(x, \nabla A(x)), \dif (f(x, \nabla A(x))) \right\rangle = 0.
\end{equation}
Now
$$\dif(f(x, \nabla A(x))) = \dif |\dif A(x)|$$
and 
$$\frac{\partial f}{\partial T}(x, \nabla A(x)) = \frac{\nabla A(x)^{\rm as}}{|\nabla A(x)^{\rm as}|} = \frac{\dif A(x)}{|\dif A(x)|}$$
we conclude the claim after multiplying both sides of (\ref{ELA}) by $|\dif A|$.
\end{proof}

\begin{corollary}
Suppose that $F$ has absolutely best comass, regularity $C^1$, and no points with $F = 0$, and $N$ is a surface whose normal vector field is annihilated by $F$.
Then $N$ is a minimal surface.
\end{corollary}
\begin{proof}
Let $V$ be a tangent vector field to $N$. Then $V(|F|) = 0$, by (\ref{infinityMaxwell}).
Therefore $|F|$ is constant along $N$, and so $F$ is a constant times the area form $\star \normal_N^\flat$ along $N$.
So there are constant real numbers $c_1, c_2$ such that the mean curvature $H_N$ satisfies
\begin{align*}
H_N &= \tr(\nabla \normal_N^\flat) = c_1 \dif^* \star F = c_2 \dif F = 0. \qedhere
\end{align*}
\end{proof}


%%%%%%%%%%%%%%%%%%%%
\subsection{\texorpdfstring{$q \to 1$}{The limit as q goes to 1}}
We now construct the $1$-harmonic conjugate of an $\infty$-light form.
However, we cannot naively take limits in Proposition \ref{convex duality}, because $(L^1, L^\infty)$ (or even $(L^\infty, BV)$) is not a dual pair of reflexive Banach spaces, and because (\ref{inverse extremality}) may blow up as $p \to \infty$.
Instead, we have to renormalize the $q$-harmonic conjugates of $p$-light forms, as in \cite[\S3.2]{daskalopoulos2020transverse}.

To this end, fix a class $\rho \in H^2(M, \RR)$ and denote by $L$ the comass of a best comass representative of $\rho$.
Also let $k_p$ be defined by 
$$k_p^{1 - p} = \int_M \star |F_p|^p$$
where $F_p$ is the $p$-light representative of $\rho$.

\begin{definition}
The \dfn{renormalized $q$-harmonic conjugate} of a $p$-light form $F_p$ is the function $u_q: \tilde M \to \RR$ which has mean zero on $M_{\rm fun}$ and solves
$$\dif u_q = (-1)^{d - 1} k_p^{p - 1} |F_p|^{p - 2} \star F_p.$$
\end{definition}

\begin{lemma}\label{normalizations converge}
As $p \to \infty$, $k_p \to 1/L$.
\end{lemma}
\begin{proof}
We follow \cite[Lemma 3.4]{daskalopoulos2020transverse}.
One has 
$$\lim_{p \to \infty} k_p^{-\frac{1}{q}} = \lim_{p \to \infty} \|F_p\|_{L^p}.$$
Since $F_p$ converges in the weakstar topology of $L^\infty(M, Z^2)$ to a form of best comass, this limit is at most $L$.
If it is strictly less than $L$, then there exist $p$-light forms with comass strictly than $L$, a contradiction.
Taking logarithms we see that $q^{-1} \log k_p \to -\log L$, and since $q \to 1$ the claim follows.
\end{proof}

\begin{proposition}
Let $\rho \in H^2(M, \RR)$ and let $\tilde M \to M$ be the universal cover.
For $2 < p < \infty$ and $\frac{1}{p} + \frac{1}{q} = 1$, let $u_q$ be the renormalized $q$-harmonic conjugate of the $p$-light representative of $\rho$.
Then there exists a $\pi_1(M)$-equivariant function $u \in BV_\loc(\tilde M)$ such that:
\begin{enumerate}
\item $u$ is $1$-harmonic.
\item As $q \to 1$ along a subsequence, $u_q \to u$ weakly in $BV_\loc(\tilde M)$ and strongly in $L^r$ for $1 \leq r < \infty$.
\item Let $F$ be the $\infty$-light representative of $\rho$, with best comass $L$. We have the strong duality theorem 
\begin{equation}\label{1 strong duality}
	L \int_M \star |\dif u| = \int_M \dif u \wedge F
\end{equation}
and, $|\dif u|$-almost everywhere,
\begin{equation}\label{1 extremality}
L |\dif u| = \langle \dif u, \star F\rangle.
\end{equation}
\end{enumerate}
\end{proposition}
\begin{proof}
We first compute using Lemma \ref{normalizations converge}
\begin{equation}\label{Lqs of qLaplace converge}
\lim_{q \to 1} \int_M \star |\dif u_q|^q = \lim_{p \to \infty} k_p^p \int_M \star |F_p|^p = \lim_{p \to \infty} k_p = \frac{1}{L}.
\end{equation}
So by H\"older's inequality,
$$\lim_{q \to 1} \int_M \star |\dif u_q| \leq \lim_{q \to 1} |M|^{\frac{1}{p}} \int_M \star |\dif u_q|^q = \frac{1}{L}.$$
By Poincar\'e's inequality, $(u_q)$ is then bounded in $W^{1, 1}_\loc(\tilde M) \cap L^\infty_\loc(\tilde M)$ as $q \to 1$, so it converges along a subsequence to an element $u$ of the double dual $BV_\loc(\tilde M)$ of $W^{1, 1}_\loc(\tilde M)$, and also in $L^r_\loc(\tilde M)$ for any $1 \leq r < \infty$.
As the limit of $\pi_1(M)$-equivariant functions, $u$ is also $\pi_1(M)$-equivariant.
In particular $\dif u$ drops to a current on $M$.

Renormalizing (\ref{strong duality}), we obtain 
$$\frac{k_p^{-p}}{q} \int_M \star |\dif u_q|^q + \frac{1}{p} \int_M \star |F_p|^p = k_p^{1 - p} \int_M \dif u_q \wedge F_p.$$
Multiplying by $k_p^p$ and taking limits, we conclude (\ref{1 strong duality}).
However, $\|F/L\|_{L^\infty} \leq 1$, so this is only possible if $X := (\star F)^\sharp/L$ satisfies, $|\dif u|$-almost everywhere,
$$|\dif u| = (X, \dif u).$$
Therefore (\ref{1 extremality}) holds, and since $\dif F = 0$ implies $\nabla \cdot X = 0$, we conclude that $u$ is $1$-harmonic, c.f. \cite{Mazon14} \todo{or put this in the prelims?}.
\end{proof}

%%%%%%%%%%%%%%%%%%%%

\section{The maximum comass locus}
Throughout this section, let $M$ be a closed space form of dimension $3$.

\begin{definition}
Let $F$ be a form of best comass.
The \dfn{maximum comass locus} is the set $\{L(F, \cdot) = L(F)\}$.
\end{definition}

By Proposition \ref{crandall}(\ref{crandall usc}) and the compactness of $M$, the maximum comass locus is a nonempty closed subset of $M$.

%%%%%%%%%%%%%%%%%%%%%%%%%%%%%%%%%%%
\subsection{The calibrated lamination}
Recall that in the work of Harvey--Lawson on calibrated geometry \cite{Harvey82}, a \dfn{calibration} on $M$ is a smooth closed $2$-form of $C^0$ norm $1$.
Owing to the above scale-invariance, and the fact that we do not enjoy the luxury of such high regularity, we shall just mean by a \dfn{calibration} something much weaker: a $2$-form $F \in L^\infty$, such that $\|F\|_{L^\infty} > 0$ and $\dif F = 0$ as distributions.
This definition is even weaker than the definition of \cite[\S2A]{bangert_cui_2017} which assumed that $F$ is continuous and $\|F\|_{C^0} = 1$.

\begin{definition}
Let $F$ be a calibration and $N \subset M$ an embedded surface with area form $\omega_N$.
We say that $N$ is a \dfn{calibrated surface} if the pullback of $F$ to $N$ is $\|F\|_{L^\infty} \omega_N$.
\end{definition}

By Corollary \ref{trace on cycles}(\ref{integral continuous}), we may integrate a calibration over any closed surface $N$, or any small ball in a possibly nonclosed surface $N$.
In particular, the sentence ``$N$ is a calibrated surface'' is well-defined.
Moreover, every calibrated surface is minimal: if $N'$ is homologous to $N$ relative to $\partial N = \partial N'$, then by Stokes' theorem,
$$|N'| = \frac{1}{\|F\|_{L^\infty}} \int_{N'} F = \frac{1}{\|F\|_{L^\infty}} \int_N F \leq |N|.$$
Moreover, by Corollary \ref{trace on cycles}(\ref{cohomology exists}), the cohomology class of a calibration is well-defined.

\begin{proposition}\label{Bangert Cui}
Let $F$ be a calibration of best comass.
Then there exists a measured oriented minimal lamination on $M$ whose leaves are calibrated surfaces with respect to $F$.
\end{proposition}
\begin{proof}
This essentially follows from \cite[Theorem 5.1]{bangert_cui_2017}; we make a few remarks about how the proof works at this level of generality.
By a duality argument, which is purely on the level of homology, one may find a closed minimal $d-1$-current $T$ which is \dfn{calibrated} by $F$ in the sense that
$$\langle [T], [F]\rangle = \mathbf M(T) \|F\|_{L^\infty}$$
where $\mathbf M(T)$ is the mass of $T$ \cite[Proposition 2.2]{bangert_cui_2017}.
We stress that, since the theory of \cite[\S2C]{bangert_cui_2017} is on the level of homology, we need not worry that any of the involved expressions are well-defined at our extremely low regularity.
Now by \cite[Theorem 1]{AUER20011095} (or perhaps \todo{Cite laminations paper}, since \cite{AUER20011095} appeals to \cite[\S37]{Simon84} which is for euclidean space rather than space forms), the minimal current $T$ induces a measured oriented minimal lamination $\lambda$.
In particular, if $K$ denotes the space of leaves of $\lambda$ and $K$ denotes its transverse measure,
$$\langle [T], [F]\rangle = \int_K \int_{N_k} F \dif \mu(k)$$
but 
$$\mathbf M(T) = \int_K |N_k| \dif \mu(k).$$
This is only possible if the leaves of $\lambda$ are calibrated surfaces.
\todo{Need to justify this more carefully, using properties of the local comass. It could be the case that $\int_{N_k} F$ is discontinuous in $k$!}
\end{proof}

\begin{corollary}\label{best comass lamination}
Let $F$ be $\infty$-light.
Then the maximum comass locus contains a measured oriented minimal lamination.
\end{corollary}

By \cite[Example 5.4]{bangert_cui_2017} the maximum comass locus need not equal a lamination.
This is a more severe defect than in the theory developed in \cite{Thurston98,daskalopoulos2020transverse} where the maximum stretch locus is a lamination, but may not be globally measured and oriented.
The issue is that the kernel bundle of $F$ need not be integrable, since it is a vector bundle of rank $2$.

%%%%%%%%%%%%%%%%%%%%%%
\subsection{The dual \texorpdfstring{$1$-harmonic}{1-harmonic} function}
In this section we interpret the lamination of Corollary \ref{best comass lamination} using convex duality.
By \todo{Cite laminations paper}, to each $1$-harmonic function $u$, we may associated a measured oriented minimal lamination $\lambda_u$, whose Ruelle-Sullivan current is $\dif u$, and whose leaves are the level sets of $u$.

\begin{definition}
Let $\rho \in H^2(M, \RR)$, let $F$ be an $\infty$-light representative of $\rho$, and let $u$ be a $1$-harmonic conjugate of $F$.
The \dfn{Thurston lamination} $\lambda$ associated to $\rho$ is $\lambda := \lambda_u$.
\end{definition}

\begin{proposition}
The Thurston lamination of a cohomology class is well-defined.
\end{proposition}

\todo{Not sure how to prove this, but the point is that our choice of $(u, F)$ isn't allowed to matter.} Maybe it follows because the $p$-lights, which are unique, need to concentrate on a set that doesn't depend on the choice made in the compactness argument. Strictly speaking we don't need this, but I want to talk about THE Thurston lamination...
OTOH it could be false because maybe there are lots of possible dual measures, and we want the ``equidistributed'' one in some sense.

\begin{theorem}\label{MCL contains Thurston}
Let $F$ be a best comass representative of $\rho \in H^2(M, \RR)$.
Then the maximum comass locus of $F$ contains the Thurston lamination associated to $\rho$.
\end{theorem}

\todo{Prove me, the proof is basically the same as \cite[\S6.1, \S6.2]{daskalopoulos2020transverse}}
On the first reading, the reader may wish to take $F$ to be $\infty$-light, as this already captures the essential ideas.
Throughout the proof we fix the $p$-light representative $F_p$ of $\rho$.

\begin{lemma}
Let
$$f_p := 2\langle k_p F_p, k_p F_p - L^{-1} F\rangle.$$
Then 
\begin{equation}\label{MCL contains Thurston 1}
	\lim_{p \to \infty} \int_{\{f_p \geq 0\}} \star |k_p F_p|^{p - 2} f_p = 0.
\end{equation}
\end{lemma}
\begin{proof}
The proof is very similar to \cite[Lemma 6.3]{daskalopoulos2020transverse}.
Since $F_p, F$ are cohomologous, there exists a $1$-form $\xi$ such that $\dif \xi = F_p - F$.
On the other hand, since $F_p$ is $p$-light, an integration by parts gives
\begin{align*}
\int_M |F_p|^{p - 2} \langle F_p, F_p - F \rangle = \int_M \langle \dif^*(|F_p|^{p - 2} F_p), \xi\rangle = 0.
\end{align*}
In particular,
\begin{equation}\label{MCL contains Thurston 2}
	I_p := \int_M \star |k_p F_p|^{p - 2} \langle k_p F_p, k_p F_p - k_p F\rangle = 0.
\end{equation}
On the other hand, by the Cauchy-Schwarz inequality, Lemma \ref{normalizations converge}, and the fact that 
$$\lim_{p \to \infty} \|F_p\|_{L^p} = \|F\|_{L^\infty} = L,$$
we have
\begin{align*}
\lim_{p \to \infty} I_p - \frac{1}{2} \int_M \star |k_p F_p|^{p - 2} f_p 
&= \lim_{p \to \infty} \int_M \star |k_p F_p|^{p - 2} \left\langle k_p F_p, \left(\frac{1}{L} - k_p\right) F\right\rangle \\
&\leq \lim_{p \to \infty} k_p^{p - 1} \left(\frac{1}{L} - k_p\right) \int_M \star |F_p|^{p - 1} |F| \\
&= \lim_{p \to \infty} \frac{1}{L^{p - 1}} \left(\frac{1}{L} - \frac{1}{L}\right)L^p \\
&= \lim_{p \to \infty} 1 - 1 = 0.
\end{align*}
If we plug this into (\ref{MCL contains Thurston 2}), we obtain 
\begin{equation}\label{MCL contains Thurston 3}
	\lim_{p \to \infty} \int_M \star |k_p F_p|^{p - 2} f_p = 0.
\end{equation}
If $f_p(x) \leq 0$, then by the Cauchy-Schwarz inequality,
$$|L^{-1} F|(x) \geq |k_p F_p|(x),$$
so by the Cauchy-Schwarz and Peter-Paul inequalities,
\begin{align*}
f_p(x) &\leq 2|L^{-1} F|(x) |k_p F_p|(x) - 2|k_p F_p|^2(x) \\
&\leq |L^{-1} F|(x)^2 + |k_p F_p|(x)^2 - 2|k_p F_p|^2(x) \\
&= |L^{-1} F|(x)^2 - |k_p F_p|(x).
\end{align*}
So by \cite[Lemma 6.2]{daskalopoulos2020transverse},
$$|k_p F_p|(x)^{p - 2} f_p(x) < \frac{2}{p - 2}.$$
Integrating this inequality, 
$$0 \leq \lim_{p \to \infty} \int_{\{f_p \leq 0\}} |k_p F_p|^{p - 2} f_p \leq \lim_{p \to \infty} \frac{2|M|}{p - 2} = 0.$$
Therefore, by (\ref{MCL contains Thurston 3}), we deduce (\ref{MCL contains Thurston 1}).
\end{proof}

\begin{lemma}\label{MCL contains Thurston lemma}
The set $\{L(F, \cdot) < L\}$ is open, and
$$\lim_{p \to \infty} \int_{\{L(F, \cdot) < L\}} \star |k_p F_p|^p = 0.$$
\end{lemma}
\begin{proof}
Since $L(F, \cdot)$ is upper semicontinuous, $\{L(F, \cdot) < L\}$ is open.
We can rewrite it as 
$$\{L(F, \cdot) < L\} = \bigcup_{0 < \theta < 1} \{L(F, \cdot) < \theta L\}$$
and then it suffices to fix $\theta$ and show that the integral over $\{L(F, \cdot) < \theta L\}$ is zero.
We then use the fact that $|F|(x) \leq L(F, x)$ everywhere \todo{Crandall consequence? Justify itand its consequences more carefully} to split
$$\{L(F, \cdot) < \theta L\} \subseteq A \cup B$$
where 
\begin{align*}
A &:= \{|L^{-1} F|^2 \leq \theta |k_p F_p|^2 + |k_p F_p - L^{-1} F|^2\} \\
B &:= \{|L^{-1} F|^2 \geq \theta |k_p F_p|^2 + |k_p F_p - L^{-1} F|^2\} \cap \{L(F, \cdot) < \theta L\} .
\end{align*}
By (\ref{MCL contains Thurston 1}),
\begin{align*}
0 &= \lim_{p \to \infty} \int_{\{f_p \geq 0\}} |k_p F_p|^{p - 2} f_p \\
&= \lim_{p \to \infty} \int_{\{f_p \geq 0\}} |k_p F_p|^{p - 2}((1 - \theta) |k_p F_p|^2 + \theta |k_p F_p|^2 + |k_p F_p - L^{-1} F|^2 - |L^{-1} F|^2).
\end{align*}
But on $A$,
\begin{align*}
|L^{-1} F|^2 &\leq \theta |k_p F_p|^2 + |k_p F_p - L^{-1} F|^2 \\
&\leq |k_p F_p|^2 + |k_p F_p - L^{-1} F|^2 \\
&= f_p + |L^{-1} F|^2
\end{align*}
which implies $f_p \geq 0$ and 
$$0 \leq \lim_{p \to \infty} \int_A \star |k_p F_p|^p \leq \lim_{p \to \infty} \frac{1}{\theta} \int_{\{f_p \geq 0\}} \star |F_p|^{p - 2} f_p = 0.$$
Meanwhile, on $B$,
$$|k_p F_p|^2 \leq |k_p F_p|^2 + \theta^{-1} |k_p F_p - L^{-1} F|^2 \leq \theta^{-1} |L^{-1} F|^2 < \theta.$$
Therefore $|k_p F_p|^p < \theta^{p/2} = o(1)$ on $B$, as desired.
\end{proof}

\begin{proof}[Proof of Theorem \ref{MCL contains Thurston}]
Let $U := \{L(F, \cdot) < L\}$, which is open by Lemma \ref{MCL contains Thurston lemma}.
So by the portmanteau theorem \todo{Cite it}, H\"older's inequality, and Lemma \ref{MCL contains Thurston lemma},
\begin{align*}
\int_U \star |\dif u|
&\leq \lim_{q \to 1} \int_U \star |\dif u_q| \\
= \lim_{p \to \infty} \int_U \star |k_p F_p|^{p - 1} \\
&\leq \lim_{p \to \infty} |M|^{-\frac{1}{p}} \left[\int_U \star |k_p F_p|^p\right]^{\frac{1}{q}} \\
&= \lim_{p \to \infty} \int_U \star |k_p F_p|^p = 0. \qedhere
\end{align*}
\end{proof}



%%%%%%%%%%%%%%%%%%%%%%
\section{Thurston's \texorpdfstring{$L = K$}{L equals K} theorem}
If $\lambda$ is a measured oriented lamination, then we denote by $[\lambda] \in H_1(M, \RR)$ its homology class, and by $|\lambda|$ its area; namely, if $(\chi_n)$ is a partition of unity subordinate to a measured $\lambda$-laminar atlas $(\Phi_n, K_n)$ \todo{Explain this in the prelims},
$$|\lambda| = \int_{K_n} \int_{\{k\} \times \RR^2} \Phi_n^*(\star \chi_n) \dif \mu_{\lambda, n}(k).$$

\begin{theorem}
	Let $\rho \in H^2(M, \RR)$, and let 
	$$K := \sup_\lambda \frac{\langle \rho, [\lambda]\rangle}{|\lambda|},$$
	where $\lambda$ ranges over measured oriented laminations. Then:
\begin{enumerate}
	\item The supremum in $K$ is attained by the Thurston lamination associated to $\rho$.
	\item Let $L$ be the best comass constant of $\rho$. Then $L = K$.
\end{enumerate}
\end{theorem}

\todo{Prove this. Here's a sketch:}
We can identify the space of measured oriented laminations with the equivariant part of $BV_\loc(\tilde M)/\RR$.
Then for any rep $F$ of $\rho$,
$$K = \sup_v \frac{\int \dif v \wedge F}{\int |\dif v|}$$
and in fact we may assume that $F$ is $\infty$-light.
But $F = \pm L \star \frac{\dif u}{|\dif u|}$ for the $1$-harmonic conjugate $u$.
So 
$$K = L \sup_v \frac{\int \langle \dif u, \dif v/|\dif v|\rangle}{\int |\dif v|}$$
and it's pretty obvious that this is maximized by $v = u$, with $K = L$.
However, there's a technicality, which is that the conjugate formula for $F$ only holds $|\dif u|$-almost everywhere.
So we need to validate all this in a sense which is integrated against $|\dif u|$.

\printbibliography

\end{document}
