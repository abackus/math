\documentclass[reqno,11pt]{amsart}
\usepackage[letterpaper, margin=1in]{geometry}
\RequirePackage{amsmath,amssymb,amsthm,graphicx,mathrsfs,url,slashed,subcaption}
\RequirePackage[usenames,dvipsnames]{xcolor}
\RequirePackage[colorlinks=true,linkcolor=Red,citecolor=Green]{hyperref}
\RequirePackage{amsxtra}
\usepackage{cancel}
\usepackage{tikz-cd}

% \setlength{\textheight}{9.3in} \setlength{\oddsidemargin}{-0.25in}
% \setlength{\evensidemargin}{-0.25in} \setlength{\textwidth}{7in}
% \setlength{\topmargin}{-0.25in} \setlength{\headheight}{0.18in}
% \setlength{\marginparwidth}{1.0in}
% \setlength{\abovedisplayskip}{0.2in}
% \setlength{\belowdisplayskip}{0.2in}
% \setlength{\parskip}{0.05in}
%\renewcommand{\baselinestretch}{1.05}

\title{Comass minimizers}
\author{Aidan Backus}
\date{\today}

\newcommand{\NN}{\mathbf{N}}
\newcommand{\ZZ}{\mathbf{Z}}
\newcommand{\QQ}{\mathbf{Q}}
\newcommand{\RR}{\mathbf{R}}
\newcommand{\CC}{\mathbf{C}}
\newcommand{\DD}{\mathbf{D}}
\newcommand{\PP}{\mathbf P}
\newcommand{\MM}{\mathbf M}
\newcommand{\II}{\mathbf I}
\newcommand{\Hyp}{\mathbf H}
\newcommand{\Sph}{\mathbf S}
\newcommand{\Group}{\mathbf G}
\newcommand{\GL}{\mathbf{GL}}
\newcommand{\Orth}{\mathbf{O}}
\newcommand{\SpOrth}{\mathbf{SO}}
\newcommand{\Ball}{\mathbf{B}}

\newcommand*\dif{\mathop{}\!\mathrm{d}}

\DeclareMathOperator{\card}{card}
\DeclareMathOperator{\dist}{dist}
\DeclareMathOperator{\supp}{supp}
\DeclareMathOperator{\tr}{tr}

\newcommand{\Leaves}{\mathscr L}
\newcommand{\Lagrange}{\mathcal L}
\newcommand{\Hypspace}{\mathscr H}

\newcommand{\Chain}{\underline C}

\newcommand{\Two}{\mathrm{I\!I}}

\newcommand{\normal}{\mathbf n}
\newcommand{\radial}{\mathbf r}
\newcommand{\evect}{\mathbf e}
\newcommand{\vol}{\mathrm{vol}}

\newcommand{\diam}{\mathrm{diam}}
\newcommand{\Ell}{\mathrm{Ell}}
\newcommand{\inj}{\mathrm{inj}}
\newcommand{\Lip}{\mathrm{Lip}}
\newcommand{\Riem}{\mathrm{Riem}}

\newcommand{\Min}{\mathrm{Min}}
\newcommand{\Max}{\mathrm{Max}}

\newcommand{\dfn}[1]{\emph{#1}\index{#1}}

\renewcommand{\Re}{\operatorname{Re}}
\renewcommand{\Im}{\operatorname{Im}}

\newcommand{\loc}{\mathrm{loc}}
\newcommand{\cpt}{\mathrm{cpt}}

\def\Japan#1{\left \langle #1 \right \rangle}

\newtheorem{theorem}{Theorem}[section]
\newtheorem{badtheorem}[theorem]{``Theorem"}
\newtheorem{prop}[theorem]{Proposition}
\newtheorem{lemma}[theorem]{Lemma}
\newtheorem{sublemma}[theorem]{Sublemma}
\newtheorem{proposition}[theorem]{Proposition}
\newtheorem{corollary}[theorem]{Corollary}
\newtheorem{conjecture}[theorem]{Conjecture}
\newtheorem{axiom}[theorem]{Axiom}
\newtheorem{assumption}[theorem]{Assumption}

\newtheorem{mainthm}{Theorem}
\renewcommand{\themainthm}{\Alph{mainthm}}

% \newtheorem{claim}{Claim}[theorem]
% \renewcommand{\theclaim}{\thetheorem\Alph{claim}}
\newtheorem*{claim}{Claim}

\theoremstyle{definition}
\newtheorem{definition}[theorem]{Definition}
\newtheorem{remark}[theorem]{Remark}
\newtheorem{example}[theorem]{Example}
\newtheorem{notation}[theorem]{Notation}

\newtheorem{exercise}[theorem]{Discussion topic}
\newtheorem{homework}[theorem]{Homework}
\newtheorem{problem}[theorem]{Problem}

\makeatletter
\newcommand{\proofpart}[2]{%
  \par
  \addvspace{\medskipamount}%
  \noindent\emph{Part #1: #2.}
}
\makeatother



\numberwithin{equation}{section}


% Mean
\def\Xint#1{\mathchoice
{\XXint\displaystyle\textstyle{#1}}%
{\XXint\textstyle\scriptstyle{#1}}%
{\XXint\scriptstyle\scriptscriptstyle{#1}}%
{\XXint\scriptscriptstyle\scriptscriptstyle{#1}}%
\!\int}
\def\XXint#1#2#3{{\setbox0=\hbox{$#1{#2#3}{\int}$ }
\vcenter{\hbox{$#2#3$ }}\kern-.6\wd0}}
\def\ddashint{\Xint=}
\def\dashint{\Xint-}

\usepackage[backend=bibtex,style=alphabetic,giveninits=true]{biblatex}
\renewcommand*{\bibfont}{\normalfont\footnotesize}
\addbibresource{best_curl.bib}
\renewbibmacro{in:}{}
\DeclareFieldFormat{pages}{#1}

\newcommand\todo[1]{\textcolor{red}{TODO: #1}}


\begin{document}
\begin{abstract}
	Minimizers of the comass
\end{abstract}

\maketitle

%%%%%%%%%%%%%%%%%%%%%%%%%%%%%%%%%%%%%%%%%%%%%%%%%%%%%%%

In this paper we study the problem of minimization of the \dfn{comass}
$$L(F) := \sup_{\sigma \in \Chain_2(M)} \frac{1}{|\sigma|} \int_\sigma F$$
of a closed $2$-form in a Riemannian $3$-fold $M$, subject to a constraint on the cohomology of $F$.
Here $\Chain_2(M)$ denotes the space of oriented $2$-chains in $M$, and $|\sigma|$ is the area of $\sigma$.
This problem is the analogue on $3$-manifolds of the problem of finding a best Lipschitz map on a surface.

For a discussion of our interest in this problem, see \S\ref{motivation}; for a more precise statement of our results without fluff, see \S\ref{results}.

% \tableofcontents

\section{Introduction}
\subsection{History and motivation} \label{motivation}

Sayeth Thurston \cite[Abstract]{Thurston98}:
\begin{quote}
I currently think that a characterization of minimal stretch maps should be possible in a considerably more general context ... and it should be feasible with a simpler proof based on more general principles -- in particular, the max flow mean cut principle, convexity, and $L^0 \leftrightarrow L^\infty$ duality.
\end{quote}

\subsection{Main theorems} \label{results}

\subsection{Acknowledgements}
George, Karen, Tom Goodwillie, Kaya Ferendo ...

This research was supported by the National Science Foundation's Graduate Research Fellowship Program under Grant No. DGE-2040433

%%%%%%%%%%%%%%%%%%%%%%%%%%%%%%%%%%%%%%%%%%
\section{Preliminaries}
\subsection{Notation}
\subsection{The de Rham--Sobolev complex}
\begin{definition}
For $1 \leq p < \infty$ and $0 \leq \ell \leq d$, let $W^{1, p}_{\rm d}(M, \Omega^\ell)$ denote the space of measurable $\ell$-forms $\alpha$ for which the norm
$$\|\alpha\|_{W^{1, p}_{\rm d}}^p := \|\alpha\|_{L^p}^p + \|\dif \alpha\|_{L^p}^p$$
is finite.
For $p = \infty$ we instead define $W^{1, \infty}_{\rm d}(M, \Omega^\ell)$ by the norm 
$$\|\alpha\|_{W^{1, \infty}_{\rm d}} := \max(\|\alpha\|_{L^\infty}, \|\dif \alpha\|_{L^\infty}).$$
The \dfn{de Rham--Sobolev complex} $W^{1, p}_{\rm d}(M, \Omega^\bullet)$ is the chain complex of such spaces, with boundary maps 
$$\dif: W^{1, p}_{\rm d}(M, \Omega^\ell) \to W^{1, p}_{\rm d}(M, \Omega^{\ell + 1}).$$
\end{definition}

The de Rham--Sobolev complex is a well-defined elliptic complex, since 
$$\|\dif \alpha\|_{W^{1, p}_{\rm d}}^p = \|\dif \alpha\|_{L^p}^p + \|\dif^2 \alpha\|_{L^p}^p = \|\dif \alpha\|_{L^p}^p \leq \|\alpha\|_{W^{1, p}_{\rm d}}^p.$$
For $p = 2$, $d = 3$ this complex is well-known for its applications in electromagnetism and numerical analysis, and its constituent spaces are more commonly known as \cite[Chapter 2]{cessenat1996mathematical}
\begin{align*}
H(M, \text{curl}) &:= W^{1, 2}_{\rm d}(M, \Omega^1)\\
H(M, \text{div}) &:= W^{1, 2}_{\rm d}(M, \Omega^2).
\end{align*}

We write 
$$\|\alpha\|_{W^{1, p}}^p := \|\alpha\|_{L^p}^p + \|\dif \alpha\|_{L^p}^p + \|\dif^* \alpha\|_{L^p}^p$$
for the full $W^{1, p}$ norm, using the subscript $W^{1, p}_{\rm d}$ to emphasize the de Rham--Sobolev norm.

The utility of the de Rham--Sobolev complex is the \dfn{normal trace theorem}, which says for $1 < p < \infty$ that the pullback $\iota^*$ to the boundary is bounded \cite[50]{sohr2001navier}
$$\iota^*: W^{1, p}_{\rm d}(M, \Omega^\ell) \to W^{-\frac{1}{p}, p}(\partial M, \Omega^\ell).$$
We shall need a version for $W^{1, \infty}_{\rm d}$, which we prove now.

\begin{proposition}[normal trace theorem]
The pullback is a bounded linear operator 
$$\iota^*: W^{1, \infty}_{\rm d}(M, \Omega^\ell) \to L^\infty(\partial M, \Omega^\ell).$$
In particular, for every $\alpha \in W^{1, \infty}_{\rm d}(M, \Omega^\ell)$ and every $\beta \in W^{1, 1}_{\rm d}(M, \Omega^{d - 1 - \ell})$, we have integration by parts:
\begin{equation}\label{Stokes trace}
	\int_{\partial M} \alpha \wedge \beta = \int_M \dif \alpha \wedge \beta - \alpha \wedge \dif \beta.
\end{equation}
\end{proposition}
\begin{proof}
Define an $\ell$-current $\iota^* \alpha$ on $\partial M$ by setting, for every $\beta \in C^\infty_\cpt(\partial M, \Omega^{d - 1 - \ell})$,
\begin{equation}\label{definition of trace}
\int_{\partial M} \iota^* \alpha \wedge \beta = \int_M \dif \alpha \wedge \beta - \alpha \wedge \dif \beta.
\end{equation}
To keep notation simple we shall just write $\alpha$ for $\iota^* \alpha$ when it is clear enough.

In (\ref{definition of trace}), we chose a smooth extension of $\beta$ to $M$.
However, the definition of $\iota^* \alpha$ is independent of the choice of extension.
First, if we replace $\beta$ by $\beta + \gamma$ where $\supp \gamma$ is compact in the interior of $M$, then by Stokes' theorem,
\begin{align*}
\int_M \dif \alpha \wedge (\beta + \gamma) - \alpha \wedge \dif (\beta + \gamma)
&= \int_M \dif \alpha \wedge \beta - \alpha \wedge \dif \beta + \int_M \dif(\alpha \wedge \gamma) \\
&= \int_M \dif \alpha \wedge \beta - \alpha \wedge \dif \beta.
\end{align*}
If we instead replaced $\beta$ by $\beta + \gamma$ where $\gamma$ is traceless, then we could approximate $\gamma$ in $W^{1, 1}$ by compactly supported replacements, and get the same result by dominated convergence.

In particular, we may, by the inverse trace theorem \cite[Teorema 1.II]{Gagliardo1957}, choose the extension $\beta$ to satisfy 
$$\|\beta\|_{W^{1, 1}(M)} \lesssim \|\beta\|_{L^1(N)}.$$
Therefore by (\ref{definition of trace}), 
\begin{align*}
\|\alpha\|_{L^\infty(N)} 
&= \sup_{\|\beta\|_{L^1(N)} = 1} \int_N \alpha \wedge \beta\\
&\leq \sup_{\|\beta\|_{L^1(N)} = 1} \|\dif \alpha\|_{L^\infty(M)} \|\beta\|_{L^1(M)} + \|\alpha\|_{L^\infty(M)} \|\dif \beta\|_{L^1(M)} \\
&\lesssim \sup_{\|\beta\|_{L^1(N)} = 1} \|\alpha\|_{W^{1, \infty}(M)} \|\beta\|_{W^{1, 1}(M)} \\
&\lesssim \|\alpha\|_{W^{1, \infty}(M)}. \qedhere
\end{align*}
\end{proof}

\begin{corollary}\label{trace on cycles}
Let $N$ be a smooth embedded hypersurface in $M$.
\begin{enumerate}
\item \label{pullback bounded} The pullback is a bounded linear operator
$$\iota^*_N: W^{1, \infty}_{\rm d}(M, \Omega^\ell) \to L^\infty(N, \Omega^\ell).$$
\item \label{integral continuous} For $F \in L^\infty(M, \Omega^{d - 1})$ such that $\dif F = 0$, if $N$ is closed, then the integral $\int_N F$ is well-defined, and depends continously on $F$ in weak $L^q(M, \Omega^{d - 1})$ for any $1 < q < \infty$.
\item \label{cohomology exists} For $F \in L^\infty(M, \Omega^{d - 1})$ such that $\dif F = 0$, the cohomology class of $F$ is well-defined, and depends continuously on $F$ in weak $L^q(M, \Omega^{d - 1})$ for any $1 < q < \infty$.
\end{enumerate}
\end{corollary}
\begin{proof}
To prove (\ref{pullback bounded}) we may use a partition of unity to work in a small ball $U$ in $N$, and then we may realize a collar neighborhood $V$ of $U$ in $M$ as a manifold-with-boundary with $U \subseteq \partial V$, and choose $\beta \in C^\infty_\cpt(M)$ in (\ref{Stokes trace}) to be a cutoff which is zero on $\partial V$ except along $U$.
The definition of
$$\iota^*_U: W^{1, \infty}_{\rm d}(V, \Omega^\ell) \to L^\infty(U, \Omega^\ell)$$
does not depend on the choice of $V$, since by Stokes' theorem the right-hand side of (\ref{Stokes trace}) will be the same.

To obtain (\ref{integral continuous}), we first observe that since $F \in L^\infty$ and $\dif F = 0$,
$$F \in W^{1, \infty}_{\rm d}(M, \Omega^{d - 1})$$
which is mapped to
$$L^\infty(N, \Omega^{d - 1}) \subseteq L^1(N, \Omega^{d - 1})$$
by (\ref{pullback bounded}) and the fact that $N$ is closed.
To obtain the continuity, we again use a collar neighborhood $V$ of $U$ and a cutoff $\beta$.
Since $F$ is closed, (\ref{Stokes trace}) reads 
$$\int_U F = \int_M F \wedge \dif \beta.$$
Since $\beta \in C^\infty_\cpt$, $\dif \beta \in L^p$ where $\frac{1}{p} + \frac{1}{q} = 1$.
So if $F_n \to F$ in weak $L^q$, $\int_M F_n \wedge \dif \beta \to \int_M F \wedge \dif \beta$, as desired.
Letting $N$ range over representatives of every homology class, we conclude (\ref{cohomology exists}) as a consequence of (\ref{integral continuous}).
\end{proof}

%%%%%%%%%%%%%%%%%%%%%%%%%%%%%%%%%%%%%%%%%
\section{Convex duality for the \texorpdfstring{$q$-Laplacian}{q-Laplacian}}
Let $\Pi: \tilde M \to M$ be the universal covering, $M_{\rm fun} \subseteq \tilde M$ a fundamental domain, and
$$\alpha \in H^1(M, \RR)$$
a cohomology class.
Since $H_1(M, \RR)$ is the abelianization of $\pi_1(M)$, $\alpha$ is canonically identified with a representation of the fundamental group, which we also call
$$\alpha: \pi_1(M) \to \RR.$$
If a function $u: \tilde M \to \RR$ is $\alpha$-equivariant, we write $[u] = \alpha$.

We here consider the problem
\begin{equation}\label{preprimal problem}
	\Min\{\|\dif u\|_{L^q(M_{\rm fun})}: u \in W^{1, q}(M_{\rm fun}), [u] = \alpha\},
\end{equation}
where $1 < q < \infty$.
Taking Euler-Lagrange equations, we see that (\ref{preprimal problem}) is equivalent to the $q$-Laplacian 
$$\begin{cases}
	\dif^*(|\dif u|^{q - 2} \dif u) = 0 \\
	[u] = \alpha.
\end{cases}$$

To put (\ref{preprimal problem}) in the framework of \cite[Chapter IV]{Ekeland99}, we shall fix a point $0 \in M_{\rm fun}$, choose a representative $1$-form (which we also call $\alpha$), and solve 
$$\begin{cases}
\dif f = \Pi^* \alpha \\
v(0) = 0.
\end{cases}$$
Thus (\ref{preprimal problem}) is equivalent to
\begin{equation}\label{primal problem}
	\Min\left\{\frac{1}{q} \int_{M_{\rm fun}} \star|\dif v + \Pi^* \alpha|^q: v \in W^{1, q}_0(M_{\rm fun})\right\}
\end{equation}
where we set $u = v + f$.

Let
$$f(\xi) := \frac{1}{q} \int_M \star|\xi + \alpha|^q,$$
defined for $\xi \in L^q(M, \Omega^1)$.
Since $v$ is traceless in (\ref{primal problem}), it is invariant, so $\dif v$ drops to a $1$-form on $M$, and (\ref{primal problem}) is the problem of minimizing $f(\dif v)$.
So the Legendre transform
$$\hat f: L^p(M, \Omega^{d - 1}) \to \RR$$
of $f$, where $\frac{1}{p} + \frac{1}{q} = 1$, satisfies
$$\hat f(F) = \frac{1}{p} \int_M \star |F|^p - \int_M \alpha \wedge F.$$
From \cite[III(4.18) and III(4.23)]{Ekeland99}, we immediately conclude:

\begin{lemma}
The convex dual problem of the $q$-Laplacian (\ref{primal problem}) is the problem 
\begin{equation}\label{predual problem}
\Max\left\{\int_M \alpha \wedge F - \frac{1}{p} \int_M \star |F|^p: F \in L^p(M, \Omega^{d - 1})\right\}.
\end{equation}
Moreover, if $v$ is a solution of (\ref{primal problem}) and $F$ is a solution of the dual problem (\ref{predual problem}), then
\begin{equation}\label{extremality relations}
\frac{1}{q} \int_M \star |\dif v + \alpha|^q + \frac{1}{p} \int_M \star |F|^p = \int_M \alpha \wedge F + \int_M \dif v \wedge F.
\end{equation}
\end{lemma}

\begin{lemma}\label{EulerLagrange}
Let $F$ be a solution of the dual problem (\ref{predual problem}). Then
\begin{equation}\label{EL of hat G}
|F|^{p - 2} F = (-1)^{d - 1} \star \alpha.
\end{equation}
\end{lemma}
\begin{proof}
Let $(F_t)$ be an arbitrary variation and $G := \frac{\partial F_t}{\partial t}|_{t = 0}$. Then 
$$0 = \frac{\dif}{\dif t} \hat f(F_t)\bigg|_{t = 0} = \int_M \star |F|^{p - 2} \langle F, G \rangle - \alpha \wedge G.$$
In particular, for any $G$,
$$|F|^{p - 2} \langle F, G\rangle = \star^{-1}(\alpha \wedge G) = \langle \star^{-1} \alpha, G\rangle.$$
Recalling that, since $\alpha$ is a $1$-form, $\star^{-1} \alpha = (-1)^{d - 1} \star \alpha$, and $G$ is arbitrary, the claim follows.
\end{proof}

\begin{proposition}\label{convex duality}
Let $1 < p, q < \infty$ satisfy $\frac{1}{p} + \frac{1}{q} = 1$, and fix a cohomology class $\alpha \in H^1(M, \RR)$.
Then there exists a cohomology class $\rho \in H^{d - 1}(M, \RR)$, such that the convex dual problem of the $q$-Laplacian (\ref{preprimal problem}) is equivalent to 
\begin{equation}\label{dual problem}
\Min\left\{\frac{1}{p} \int_{M_{\rm fun}} \star |F|^p: F \in L^p(M, \Omega^{d - 1}), [F] = \rho\right\}.
\end{equation}
Now let $u, F$ be solutions of (\ref{preprimal problem}) and (\ref{dual problem}). Then
\begin{equation}\label{pMaxwell}
\dif^*(|F|^{p - 2} F) = 0
\end{equation}
and $u, F$ are related by
\begin{align}
F &= |\dif u|^{q - 2} \star \dif u \label{extremality} \\
\dif u &= -|F|^{p - 2} \star F. \label{inverse extremality}
\end{align}
Finally, the map $\alpha \mapsto \rho$ is an isomorphism.
\end{proposition}
\todo{This needs to be able to go backwards as well}
\begin{proof}
\todo{The uniqueness assertions here are sketchy, also, might've mixed up $\alpha, \rho$ in this proof. Also, we don't know that $\alpha \mapsto \rho$ is an isomorphism yet.}
We first solve the $q$-Laplacian (\ref{preprimal problem}) for $u$.
By convex duality \cite[Chapter III, Theorem 4.2]{Ekeland99} there exists a solution $F$ of (\ref{predual problem}) satisfying (\ref{extremality relations}), hence is characterized by 
$$\frac{1}{p} \int_M \star |F|^p + \frac{1}{q} \int_M \star |\dif u|^q = \int_M \dif u \wedge F.$$
Moreover, $F$ satisfies (\ref{EL of hat G}), which uniquely characterizes it; hence (\ref{predual problem}) is uniquely solvable.
If we instead set
$$\tilde F = |\dif u|^{q - 2} \star \dif u,$$
then 
$$\frac{1}{p} |\tilde F|^p + \frac{1}{q} |\dif u|^q = \frac{1}{p} |\dif u|^{p(q - 1)} + \frac{1}{q} |\dif u|^q = |\dif u|^q$$
and 
$$\int_M \dif u \wedge \tilde F = \int_M |\dif u|^{q - 2} \dif u \wedge \star \dif u = \int_M \star |\dif u|^q,$$
so it follows that 
$$\frac{1}{p} \int_M \star |\tilde F|^p + \frac{1}{q} \int_M \star |\dif u|^q = \int_M \dif u \wedge \tilde F.$$
By the uniqueness of (\ref{predual problem}), it follows that $\tilde F = F$ and hence if we define $\dif A := F$, then (\ref{extremality}) holds.
Here, $A$ is well-defined as a $d-2$-form on $\tilde M$ since we assumed $H^{d - 1}(\tilde M, \RR) = $0.
Moreover, from (\ref{EL of hat G}) and the fact that $\rho$ is closed, (\ref{pMaxwell}) holds.
It is clear that the Euler-Lagrange equation of the minimization problem (\ref{dual problem}) is (\ref{pMaxwell}), where $\alpha := [F]$, so $A$ solves (\ref{dual problem}).
Finally, (\ref{inverse extremality}) follows immediately from (\ref{extremality}) and $\frac{1}{p} + \frac{1}{q} = 1$.
\end{proof}

Proposition \ref{convex duality} has the following interesting interpretation, which is motivated by the central role that Noether's theorem plays in \cite{daskalopoulos2022, daskalopoulos2023}.
If we define $u_t := u + t$, $t \in \RR$, and apply Noether's theorem \cite[\S8.6.2]{evans2010partial} to the variation $(u_t)$, then we obtain the conservation law 
\begin{equation}\label{conservation law}
	\frac{q - 1}{q} \dif^*(|\dif u|^{q - 2} \dif u) = 0.
\end{equation}
Applying the extremality relation (\ref{extremality}), we see that (\ref{conservation law}) is exactly the dual PDE (\ref{pMaxwell}).
That is, we have:

\begin{corollary}
Let $u: M \to \RR$ be a $q$-harmonic function, and let $F$ be a solution of the dual problem (\ref{dual problem}).
Then $F$ is the Noether conserved current for $u$, induced by the translation symmetry of the target $\RR$.
\end{corollary}

If $F_p$ is a Noether current for $u_q$ for every $p \gg 1$ and $\frac{1}{p} + \frac{1}{q} = 1$, it may be that in a suitable function space we may take the limit $p \to \infty$ to obtain a $d-1$-form $F$ and a function $u$.
In this situation, we still call $F$ a Noether current, though this is only literally true in a formal sense.

%%%%%%%%%%%%%%%%%%%%%%%%%%%%%%%%%%%%%%%%%%

\section{Best comass and \texorpdfstring{$\infty$-longitudinal forms}{infinity-longitudinal forms}}
\subsection{\texorpdfstring{$p$-longitudinal forms}{p-longitudinal forms}}
Let $M$ be a Riemannian threefold, and fix a cohomology class $\rho$.

\begin{definition}
Let $p \geq 2$ and let $F_p$ be a closed $2$-form with $[F_p] = \rho$.
We call $F_p$ a \dfn{$p$-longitudinal form} if it is a minimizer of $\|F_p\|_{L^p}$ among all $2$-forms representing $\rho$.
\end{definition}

For $p = 2$, $\dif^* F_2 = 0$, hence $F_2$ is the Poincar\'e dual of a longitudinal vector field, hence the name.
By (\ref{EulerLagrange}), the Euler-Lagrange equation for $p$-longitudinal forms is (\ref{pMaxwell}).

\begin{lemma}
Let $F_p$ be a $p$-longitudinal form, and let $B$ range over closed $2$-forms. Then
Then 
\begin{equation}\label{infinity magnetic rules p magnetic}
	\|F_p\|_{L^p} \leq |M|^{1/p} \inf_B \|B\|_{L^\infty}.
\end{equation}
\end{lemma}
\begin{proof}
By H\"older's inequality and the fact that $F_p$ is $p$-longitudinal,
$$\|F_p\|_{L^p} \leq \|B\|_{L^p} \leq |M|^{1/p} \|B\|_{L^\infty},$$
hence the same holds for the infimum.
\end{proof}

We put any norm on $H^2(M, \RR)$ (as all are equivalent), hence $|\alpha|$ makes sense for $\alpha \in H^2(M, \RR)$.

Let $L^p(M, Z^2)$ be the space of closed $2$-forms; then, by the $L^p$ Hodge decomposition \cite[Proposition 6.5]{Scott95}, $L^p(M, Z^2)$ is a Banach space with 
$$\|F\|_{L^p} = \|F\|_{W^{1, p}_{\rm d}}$$
for $F \in L^p(M, Z^2)$.
We have trace maps for each $2$-cycle $\sigma$,
$$L^p(M, Z^2) \to W^{-\frac{1}{p}, p}(\sigma, \Omega^2),$$
and since $1 \in W^{\frac{1}{p}, q}(\sigma)$, it follows that the cohomology class of any element of $L^p(M, Z^2)$ is well-defined and depends continuously on them in the weak topology of $L^p(M, Z^2)$.
Since a cohomology class is given by the solution of an affine equation in $L^p(M, Z^2)$, it in particular defines a convex subset of $L^p(M, Z^2)$.

\begin{proposition}\label{existence for p}
Assume $p \geq 2$ and $\rho \in H^2(M, \RR)$.
Then:
\begin{enumerate}
\item There exists a unique $p$-longitudinal form $F_p$ whose cohomology class is $\rho$.
\item One has
\begin{equation}\label{Sobolev bounds for p}
	\|F_p\|_{L^p} \lesssim |\rho|.
\end{equation}
The constant is independent of $p$.
\item $F_p$ is H\"older continuous.
\end{enumerate}
\end{proposition}
\begin{proof}
Let $F_p(\varepsilon)$ be a smooth closed $2$-form with $[F_p(\varepsilon)] = \rho$ such that
$$\|F_p(\varepsilon)\|_{L^p} \leq \inf_B \|\dif B\|_{L^p} + \varepsilon$$
where $B$ ranges over all forms cohomologous to $F_p(\varepsilon)$.
By \cite[\S8.2.2]{evans2010partial}, $\|\cdot\|_{L^p}$ is a weakly lower semicontinuous function on $L^p(M, Z^2)$.
Therefore, by a compactness argument, we have $F_p(\varepsilon) \to F_p$ along a subsequence in the weakstar topology on $L^p(M, Z^2)$, where
$$\|F_p\|_{L^p} \leq \liminf_{\varepsilon \to 0} \|F_p(\varepsilon)\|_{L^p}.$$
Moreover,
$$[F_p] = \lim_{\varepsilon \to 0} [F_p(\varepsilon)] = \rho$$
so $F_p$ is a $p$-longitudinal form representing $\rho$.
From (\ref{pMaxwell}) and \cite{Uhlenbeck77}, we deduce that $F_p$ is H\"older continuous.
From the uniform convexity of $\|\cdot\|_{L^p}^p$, and the convexity of the cohomology class in $L^p(M, Z^2)$, we conclude the uniqueness of $F_p$.

Now select a basis $\xi_1, \dots, \xi_r$ of $H^2(M, \RR)$, and apply the above argument to obtain $p$-longitudinal forms $G_i$ representing $\xi_i$ for each $i \in \{1, \dots, r\}$.
Then $\|G_i\|_{L^p}$ is bounded indepdendently of $p$ by (\ref{infinity magnetic rules p magnetic}).
Decompose
$$\alpha = \sum_{i=1}^r \alpha_i \xi_i.$$
Using the fact that $F_p$ is $p$-longitudinal and cohomologous to $\sum_i \alpha_i G_i$,
$$\|F_p\|_{L^p} \leq \left\|\sum_{i=1}^r \alpha_i G_i\right\|_{L^p} \leq \sum_{i=1}^r |\alpha_i| \|G_i\|_{L^p} \lesssim |\alpha|$$
which proves (\ref{Sobolev bounds for p}).
\end{proof}

\todo{We do not get $C^\infty$ regularity on sets $\Subset \{|F_p| \neq 0\}$, or so it seems.}
Why?
Expanding out (\ref{pMaxwell}) with $F_{ij} = \partial_i A_j - \partial_j A_i$,
$$0 = \partial^j(|F|^{p - 2} \partial_j A_i) - |\dif A|^{p - 2} \partial_i (\dif^* A) - |\dif A|^{p - 2} [\partial^j, \partial_i] A_j - \partial^j(|\dif A|^{p - 2}) \partial_i A_j.$$
The first term is an elliptic operator with H\"older coefficients, the second can be gauged away, and the third is H\"older since $[\partial^j, \partial_i]$ is a connection coefficient of the metric.
But the last term is bad.
\todo{Maybe we can get rid of it using paraproducts?}

%%%%%%%%%%%%%%%%%%%%%%%%%%%%
\subsection{Forms of best comass}
What follows is the main definition of this paper.
For a closed $2$-form $F$ and a subdomain $\Omega$, we introduce the \dfn{comass} 
$$L_\Omega(F) := \sup_{\sigma \in \Chain_2(\Omega)} \frac{1}{|\sigma|} \int_\sigma F,$$
and let $L(F) := L_M(F)$.

\begin{definition}
	Let $\alpha \in H^2(M, \RR)$ and let $F$ be a closed $2$-form representing $\alpha$.
\begin{enumerate}
	\item We say that $F$ has \dfn{best comass} if $F$ is a minimizer of $L$ among all $F$ representing $\alpha$.
	\item We say that $F$ has \dfn{absolutely best comass} if for every small open ball $\Omega$, $F$ is a minimizer of $L_\Omega$ among all closed $2$-forms $B$ with $B|_{\partial \Omega} = F|_{\partial \Omega}$.
\end{enumerate}
\end{definition}

We will be interested in the points at which $F$ attains its comass.
However, $F \in L^\infty$, so $F$ is both only defined almost everywhere, and not norm-approximable by smooth functions.
So as a proxy for $|F|$, which may fail to be defined on a null set, we use the local comass, which is defined everywhere.

\begin{definition}
The \dfn{local comass} of a $2$-form $F$ at $x \in M$ is 
$$L(F, x) := \limsup_{\varepsilon \to 0} \sup_{\sigma \in \Chain_2(B_\varepsilon(x))} \frac{1}{|\sigma|} \int_\sigma F.$$
\end{definition}

One has
$$L(F, x) = \limsup_{\varepsilon \to 0} L_{B_\varepsilon(x)}(F)$$
but $L_{B_\varepsilon(x)}(F)$ is increasing in $\varepsilon$ (since it's a supremum over a set which grows in $\varepsilon$).
So the limit superior is actually a limit and an infimum:
$$L(F, x) = \lim_{\varepsilon \to 0} L_{B_\varepsilon(x)}(F) = \inf_{\varepsilon > 0} L_{B_\varepsilon(x)}(F)$$
and in particular $L(F, x) \leq L(F)$.

It's convenient to test the local curl modulus against a more restrictive class of $2$-chains.
Motivated by lattice gauge theory \cite{Gupta98}, we define:

\begin{definition}
Fix an orthonormal basis $(\partial_i)$ of the tangent space $T_x M$.
A \dfn{plaquette} $R_{ij}^\varepsilon(x)$ is the exponential pushforward of a square $[0, \varepsilon \partial_i] \times [0, \varepsilon \partial_j]$ to $M$.
\end{definition}

% We write $x + v := \exp_x(v)$ whenever $v \in T_x M$.

% \begin{lemma}
% Let $A$ be a continuous $1$-form and $x \in M$. Then as $\varepsilon \to 0$,
% \begin{equation} \label{riemann plaquette}
% 	\frac{1}{|R_{ij}^\varepsilon(x)|} \int_{\partial R_{ij}^\varepsilon(x)} A = \frac{A_i(x) + A_j(x + \varepsilon \partial_i) - A_i(x + \varepsilon \partial_j) - A_j(x)}{\varepsilon} + o(1).
% \end{equation}
% \end{lemma}
% \begin{proof}
% The metric introduces corrections of size $O(\varepsilon^2)$, so we may discard it and assume that $R_{ij}^\varepsilon$ is a rectangle, bounded by line segments 
% $$\partial R_{ij}^\varepsilon = \gamma_1 + \gamma_2 - \gamma_3 - \gamma_4,$$
% where $\gamma_1$ is the line segment between $x$ and $x + \varepsilon \partial_i$ (say).
% Since $A$ is continuous, 
% $$\int_{\gamma_1} A = \int_0^\varepsilon A_i(x + t\partial_i) \dif t = \varepsilon A_i(x) + o(\varepsilon).$$
% A similar computation holds for the other $\gamma_a$ and implies 
% \begin{align*}
% \int_{\partial R_{ij}^\varepsilon}
% &= \varepsilon A_i(x) + \varepsilon A_j(x + \varepsilon \partial_i) - \varepsilon A_i(x + \varepsilon \partial_j) - \varepsilon A_j(x) + o(\varepsilon) \\
% &= \frac{\varepsilon^2}{\varepsilon}(A_i(x) + A_j(x + \varepsilon \partial_i) - A_i(x + \varepsilon \partial_j) - A_j(x) + o(1)).
% \end{align*}
% Since $|R_{ij}^\varepsilon| = \varepsilon^2$, we are done.
% \end{proof}

\begin{lemma}
Let $F \in L^\infty(M, Z^2)$. Then for almost every $x \in M$,
\begin{equation}\label{dA is like Lipschitz constant}
	F_{ij}(x) = \lim_{\varepsilon \to 0} \frac{1}{|R^\varepsilon_{ij}(x)|^2} \int_{R^\varepsilon_{ij}(x)} F.
\end{equation}
\end{lemma}
Here the right-hand side is well-defined by the $W^{1, \infty}_{\rm d}$ normal trace theorem.
\begin{proof}
Let $F^\delta \to F$ be an $L^1$ approximation of $F$ by continuous closed $2$-forms.
For $\varepsilon$ small, we can write $F^\delta = \dif A^\delta$ on a neighborhood of $R^\varepsilon_{ij}(x)$.
Let us write for $(x, v) \in TM$, $x + v := \exp_x(v)$.
By Stokes' theorem,
\begin{align*}
\lim_{\varepsilon \to 0} \frac{1}{|R^\varepsilon_{ij}(x)|} \int_{R^\varepsilon_{ij}(x)} F^\delta
&= \lim_{\varepsilon \to 0} \frac{1}{|R^\varepsilon_{ij}(x)|} \int_{\partial R^\varepsilon_{ij}(x)} A^\delta \\
&= \lim_{\varepsilon \to 0} \frac{A_i^\delta(x) + A_j^\delta(x + \varepsilon \partial_i) - A_i^\delta(x + \varepsilon \partial_j) - A_j^\delta(x)}{\varepsilon} \\
&= F_{ij}^\delta(x).
\end{align*}
We then estimate 
\begin{align*}
\left|F_{ij}(x) - \frac{1}{|R^\varepsilon_{ij}(x)|^2} \int_{R^\varepsilon_{ij}(x)} F\right|
&\leq |F_{ij}(x) - F_{ij}^\delta(x)| + \frac{1}{|R^\varepsilon_{ij}(x)|^2} \left|\int_{R^\varepsilon_{ij}(x)} F - F^\delta\right| + \\
&\qquad +  \left|F_{ij}(x)^\delta - \frac{1}{|R^\varepsilon_{ij}(x)|^2} \int_{R^\varepsilon_{ij}(x)} F^\delta\right|.
\end{align*}
By the Hardy-Littlewood maximal inequality, the convergence (\ref{dA is like Lipschitz constant}) follows.
\end{proof}

We prove the following analogue of \cite[Lemma 4.2]{Crandall2008} for best Lipschitz functions.

\begin{proposition}\label{crandall}
Let $F \in L^\infty$ be a closed $2$-form. Then:
\begin{enumerate}
\item The local comass $L(F, \cdot)$ is upper semicontinuous. \label{crandall usc}
% \item If $F(x)$ exists then $L(F, x) \geq |\dif A(x)|$. \label{crandall dA bounds LA}
% \item If $L(A, x) = 0$, then $\dif A(x)$ exists and $\dif A(x) = 0$. \label{crandall zero LA implies diffble}
\item If $\sigma \in \Chain_2$ then \label{crandall best curl is ABC}
$$\frac{1}{|\sigma|} \int_\sigma F \leq \sup_{x \in \sigma} L(F, x).$$
\item The local comass is bounded, and \label{crandall linfinity}
$$L(F) = \sup_{x \in M} L(F, x) = \|F\|_{L^\infty}.$$
\end{enumerate}
\end{proposition}
\begin{proof}
First let $x^n \to x$, so for $n$ large, $\Omega_n := B_{r - |x - x^n|}(x^n) \subseteq B_r(x)$, hence
$$L(F, x^n) \leq L_{\Omega_n}(F) \leq L_{B_r(x)}(F).$$
Therefore 
$$\limsup_{n \to \infty} L(F, x^n) \leq \inf_{r > 0} L_{B_r(x)}(F) = L(F, x),$$
which proves (\ref{crandall usc}).
	
% We now bound for a net of plaquettes $R_{ij}^\varepsilon(x)$ using (\ref{riemann plaquette})
% \begin{equation}\label{difference quotients}
% 	\limsup_{\varepsilon \to 0} \frac{|A_j(x + \varepsilon \partial_i) - A_j(x) - A_i(x + \varepsilon \partial_j) + A_i(x)|}{\varepsilon}
% = \limsup_{\varepsilon \to 0} \frac{1}{|R_{ij}^\varepsilon(x)|} \left|\int_{\partial R_{ij}^\varepsilon(x)} A\right| \leq L(A, x).
% \end{equation}
% If the first limit superior is actually a limit, then it is the definition of $|\dif A_{ij}(x)|$.
% So if $\dif A_{ij}(x)$ exists we conclude $|\dif A_{ij}(x)| \leq L(A, x)$, which proves (\ref{crandall dA bounds LA}).
% On the other hand, the corresponding limit \emph{inferior} must be nonnegative, so if $L(A, x) = 0$, the first limit superior in (\ref{difference quotients}) is actually a limit and we obtain $\dif A_{ij}(x) = 0$, proving (\ref{crandall zero LA implies diffble}).

Now let $\sigma$ be a $2$-chain.
We may write $\sigma = \sum_{n=1}^N \sigma_n$ where $\sigma_n$ is a $2$-cell of the form $x^3_n = 0$ for some coordinates $(x^1_n, x^2_n, x^3_n)$.
In particular, for any $\varepsilon > 0$ we may write $\sigma_n$ as a sum of plaquettes $R_{12}^\delta(x)$ with respect to the coordinates $(x^1_n, x^2_n, x^3_n)$, where $0 < \delta < \varepsilon$ and $x \in \sigma_n$.
Let us denote such plaquettes as $P_{n1}(\varepsilon), \dots, P_{nK(n, \varepsilon)}(\varepsilon)$. Then for any $\varepsilon > 0$, 
$$\frac{1}{|\sigma|} \left|\int_\sigma F\right| \leq \sum_{n=1}^N \sum_{k=1}^{K(n, \varepsilon)} \frac{|P_{nk}(\varepsilon)|}{|\sigma|} \frac{1}{|P_{nk}(\varepsilon)|} \left|\int_{P_{nk}(\varepsilon)} F\right|.$$
But $P_{nk} \in \Chain_2(B_\varepsilon(x_{nk}(\varepsilon)))$ for some $x_{nk}(\varepsilon) \in \sigma$, so it follows that 
\begin{align*}
\frac{1}{|\sigma|} \left|\int_\sigma F\right|
&\leq \limsup_{\varepsilon \to 0} \sum_{n=1}^N \sum_{k=1}^{K(n, \varepsilon)} \frac{|P_{nk}(\varepsilon)|}{|\sigma|} L(F, x_{nk}(\varepsilon)) \\
&\leq \limsup_{\varepsilon \to 0} \sup_{x \in \sigma} L(A, x)  \sum_{n=1}^N \sum_{k=1}^{K(n, \varepsilon)} \frac{|P_{nk}(\varepsilon)|}{|\sigma|} \\
&= \sup_{x \in \sigma} L(F, x)
\end{align*}
which proves (\ref{crandall best curl is ABC}).

To prove (\ref{crandall linfinity}) we recall that trivially,
$$L(F, x) \leq L(F) \leq \|F\|_{L^\infty}.$$
Moreover, (\ref{crandall best curl is ABC}) shows that
$$L(F) = \sup_{\sigma \in \Chain_2(M)} \frac{1}{|\sigma|} \int_\sigma F \leq \sup_{\sigma \in \Chain_2(M)} \sup_{x \in \sigma} L(F, x) = \sup_{x \in M} L(F, x).$$
Finally we must estimate $\|F\|_{L^\infty} \leq L(F)$.
By (\ref{dA is like Lipschitz constant}),
$$\|F\|_{L^\infty}^2 \leq \sup_{x \in M} \limsup_{\varepsilon \to 0} \sum_{i < j} \frac{1}{|R^\varepsilon_{ij}(x)|^2} \left|\int_{R^\varepsilon_{ij}(x)} F\right|^2.$$
Here the left-hand side is a supremum over almost all of $M$, while the right-hand side is a supremum over all of $M$, including those points for which the limit does not exist as $\varepsilon \to 0$ (but the limit superior does exist).
Now we select for each $x \in M$, a subsequence of $\varepsilon \to 0$ which attains the limit superior, and choose a frame $(\partial_i)$ so that (possibly along a further subsequence), the averaged values of $F_{ij}(x)$ are $0$ unless $i,j$ are $1,2$.
Then most of the terms in the sum go away, and we are left with 
$$\limsup_{\varepsilon \to 0} \sum_{i < j} \frac{1}{|R^\varepsilon_{ij}(x)|^2} \left|\int_{R^\varepsilon_{ij}(x)} F\right|^2 = \limsup_{\varepsilon \to 0} \frac{1}{|R^\varepsilon_{12}(x)|^2} \left|\int_{R^\varepsilon_{12}(x)} F\right|^2.$$
We then set $\sigma_\varepsilon(x) := R^\varepsilon_{12}(x)$ as defined in this frame.
In particular, 
\begin{align*}
	\|F\|_{L^\infty} &\leq \sup_{x \in M} \sup_{\varepsilon > 0} \frac{1}{|\sigma_\varepsilon(x)|} \left|\int_{\sigma_\varepsilon(x)} F\right| \leq \sup_{\sigma \in \Chain_2(M)} \frac{1}{|\sigma|} \int_\sigma F = L(F). \qedhere 
\end{align*}
\end{proof}

%%%%%%%%%%%%%%%%%%%%%%%%%%%%%
\subsection{\texorpdfstring{$p \to \infty$}{The limit as p goes to infinity}}
\begin{theorem}\label{existence infinity}
Let $\rho \in H^2(M, \RR)$.
For each $p \geq 2$, let $F_p$ be the $p$-longitudinal form representing $\rho$. Then:
\begin{enumerate}
\item There exists a closed $2$-form $F$ such that $F_p \to F$ weakly in $L^q$ along a subsequence, for any $3 < q < \infty$.
\item $F$ is a best comass representation of $\rho$.
\item One has 
\begin{equation}\label{Sobolev bounds for infinity}
	\|F\|_{L^\infty} \lesssim |\rho|.
\end{equation}
\end{enumerate}
\end{theorem}
\begin{proof}
We roughly follow \cite[\S3]{Lindqvist14}.
Let $q > 3$, and let $B$ be an $L^\infty$ representative of $\rho$.
By H\"older's inequality and (\ref{infinity magnetic rules p magnetic}),
\begin{equation}\label{uniform bounds in p by best curl}
	\|F_p\|_{L^q} \leq |M|^{\frac{1}{q} - \frac{1}{p}} \|F_p\|_{L^p} \leq |M|^{\frac{1}{q} - \frac{1}{p}} \|B\|_{L^\infty}.
\end{equation}
Thus a compactness argument gives $F_p \to F$ for some $2$-form $F$, weakly in $L^q$, and by Fatou's lemma, 
$$\|F\|_{L^q} \leq \liminf_{p \to \infty} \|F_p\|_{L^q} \leq |M|^{\frac{1}{q}} \|B\|_{L^\infty}.$$
Diagonalizing, we may assume that $F_p \to F$ weakly in $L^q$ for every such $q$, and taking $q \to \infty$, we conclude 
\begin{equation}\label{infinity magnetics have best curl}
	\|F\|_{L^\infty} \leq \|B\|_{L^\infty}.
\end{equation}
By Corollary \ref{trace on cycles}(\ref{cohomology exists}), $[F] = \rho$.
So by Proposition \ref{crandall}(\ref{crandall linfinity}) and the fact that $B$ was arbitrary in (\ref{infinity magnetics have best curl}), $F$ has best comass.
Moreover, taking the limit as $p \to \infty$ in (\ref{Sobolev bounds for p}), we obtain (\ref{Sobolev bounds for infinity}).
\end{proof}

\begin{definition}
The limiting $2$-form $F$ in Theorem \ref{existence infinity} is called an \dfn{$\infty$-longitudinal form}.
\end{definition}

\todo{If we knew that $p$-Maxwell had good quantitative uniqueness, then we would have}
It remains to show that $A$ has absolutely best curl, so let $\Omega$ be a small ball and $B$ a $1$-form with $B|_{\partial \Omega} = A|_{\partial \Omega}$.
By a straightforward modification of Proposition \ref{existence for p}, there exists a $p$-magnetic potential $B_p$ in Coulomb gauge with $B_p|_{\partial \Omega} = A|_{\partial \Omega}$ and $B \in C^{1 + \alpha}$.
By quantitative uniqueness
$$\|B_p - A\|_{C^0(\Omega)} \leq \|B_p - A_p\|_{C^0(\Omega)} + o(1) \lesssim \|A - A_p\|_{C^0(\partial \Omega)} + o(1) \ll 1.$$
Therefore $B_p \to A$ uniformly, and for $3 < q < p < \infty$ with $p$ dyadic,
$$\|\dif B_p\|_{L^q(\Omega)} \leq |\Omega|^{\frac{1}{q} -\frac{1}{p}} \|\dif B_p\|_{L^p(\Omega)} \leq |\Omega|^{\frac{1}{q} -\frac{1}{p}} \|\dif B\|_{L^p(\Omega)} \leq |\Omega|^{\frac{1}{q}} \|\dif B\|_{L^\infty(\Omega)}.$$
Then along a subsequence, $\dif B_p \to \dif A$ in $L^q(\Omega)$, so 
$$\|\dif A\|_{L^q(\Omega)} \leq |\Omega|^{\frac{1}{q}} \|\dif B\|_{L^\infty(\Omega)}.$$
Taking $q \to \infty$ we arrive at the conclusion.

We have the following Euler-Lagrange equation for $\infty$-longitudinal forms.
Because of the lack of a good analogue for viscosity solutions for $\infty$-elliptic systems, \todo{and because we did not show that $\infty$-longitudinal forms have absolutely best comass}, the equation can only really be interpreted in a formal sense, at least as far as we are aware.
As such, we shall not use it in the sequel, but only include it as a curiosity item.

\begin{proposition}
Suppose that $F$ has absolutely best comass, regularity $C^1$, and no points with $F = 0$. Then
\begin{equation}\label{infinityMaxwell}
	F^{ij} \partial_i |F| = 0.
\end{equation}
\end{proposition}
\begin{proof}
For a covariant $2$-tensor $T$, let $T^{\rm as}$ be its antisymmetrization, and let
$$f(x, T) := |T^{\rm as}|_{g(x)}.$$
Working locally, we may write $F = \dif A$ for some $A$, which we may assume to be Coulomb gauge and therefore $C^2$.
Since $A$ has absolutely best curl and $(\nabla A)^{\rm as} = \dif A$, $A$ is an absolute minimizer (see \cite[Definition 5.1]{Barron2001}) of the essential supremum of $f(\cdot, \nabla A)$.
By the Euler-Lagrange-Aronsson formula \cite[Theorem 5.2]{Barron2001},
\begin{equation}\label{ELA}
	\left\langle \frac{\partial f}{\partial T}(x, \nabla A(x)), \dif (f(x, \nabla A(x))) \right\rangle = 0.
\end{equation}
Now
$$\dif(f(x, \nabla A(x))) = \dif |\dif A(x)|$$
and 
$$\frac{\partial f}{\partial T}(x, \nabla A(x)) = \frac{\nabla A(x)^{\rm as}}{|\nabla A(x)^{\rm as}|} = \frac{\dif A(x)}{|\dif A(x)|}$$
we conclude the claim after multiplying both sides of (\ref{ELA}) by $|\dif A|$.
\end{proof}

\begin{corollary}
Suppose that $F$ has absolutely best comass, regularity $C^1$, and no points with $F = 0$, and $N$ is a surface whose normal vector field is annihilated by $F$.
Then $N$ is a minimal surface.
\end{corollary}
\begin{proof}
Let $V$ be a tangent vector field to $N$. Then $V(|F|) = 0$, by (\ref{infinityMaxwell}).
Therefore $|F|$ is constant along $N$, and so $F$ is a constant times the area form $\star \normal_N^\flat$ along $N$.
So there are constant real numbers $c_1, c_2$ such that the mean curvature $H_N$ satisfies
\begin{align*}
H_N &= \tr(\nabla \normal_N^\flat) = c_1 \dif^* \star F = c_2 \dif F = 0. \qedhere
\end{align*}
\end{proof}


%%%%%%%%%%%%%%%%%%%%
\subsection{\texorpdfstring{$q \to 1$}{The limit as q goes to 1}}
Our next task is to realize an $\infty$-comass form as the Noether current of a $1$-harmonic function.

\begin{proposition}
Let $\alpha \in H^2(M, \RR)$ and let $\tilde M \to M$.
For $\frac{1}{p} + \frac{1}{q} = 1$ and $p \geq 2$, let $F_p$ be the $p$-longitudinal representative of $\alpha$, and let $u_q$ be the $q$-harmonic function on $\tilde M$ whose Noether current is $F_p$.
Then there exists $\rho \in H^1(M, \RR)$ and a limiting function $u \in BV_\loc(\tilde M)$, such that:
\begin{enumerate}
\item $u$ is $1$-harmonic.
\item $u$ is $\rho$-equivariant.
\item As $q \to 1$ along a subsequence, $u_q \to u$ weakly in $BV_\loc(\tilde M)$ and strongly in $L^r$ for $1 \leq r < \infty$.
\item The homology class $[\dif u]$ of the $d-1$-current $\dif u$ is the Poincar\'e dual of $\rho$.
\end{enumerate}
\end{proposition}

%%%%%%%%%%%%%%%%%%%%

\section{The maximum comass locus}
Throughout this section, let $M$ be a closed space form of dimension $3$.

\begin{definition}
Let $A$ be a connection of best curl on $\mathscr F$.
The \dfn{maximum comass locus} is the set $\{L(A, \cdot) = L(A)\}$.
\end{definition}

By Proposition \ref{crandall}(\ref{crandall usc}) and the compactness of $M$, the maximum curl locus is a nonempty closed subset of $M$.

%%%%%%%%%%%%%%%%%%%%%%%%%%%%%%%%%%%
\subsection{The calibrated lamination}
\begin{remark}\label{scalings}
The two minimization problems in consideration are invariant under two scalings: 
\begin{enumerate}
\item Rescaling the entire manifold, which rescales $\|\dif A\|_{L^\infty}$ but not $c_1(\mathscr F)$.
\item $A \mapsto \lambda A$. This changes both $\|\dif A\|_{L^\infty}$ and $c_1(\mathscr F)$. \todo{Could also rescale the circle inducing the circle bundle. Also this doesn't really have anything to do with the task at hand, and would probably better go elsewhere to explain why we may assume $\dif A \in H^2(M, \ZZ)$}
\end{enumerate}
In particular, by applying both scalings and using Chern-Weil theory, we may replace any best curl connection by a best curl connection $A$ with $\|\dif A\|_{L^\infty} = 1$, if desired.
\end{remark}

Recall that in the work of Harvey--Lawson on calibrated geometry \cite{Harvey82}, a \dfn{calibration} on $M$ is a smooth closed $2$-form of $C^0$ norm $1$.
Owing to the above scale-invariance, and the fact that we do not enjoy the luxury of such high regularity, we shall just mean by a \dfn{calibration} something much weaker: a $2$-form $F \in L^\infty$, such that $\|F\|_{L^\infty} > 0$ and $\dif F = 0$ as distributions.
This definition is even weaker than the definition of \cite[\S2A]{bangert_cui_2017} which assumed that $F$ is continuous and $\|F\|_{L^\infty} = 1$.

\begin{definition}
Let $F$ be a calibration and $N \subset M$ an embedded surface with area form $\omega_N$.
We say that $N$ is a \dfn{calibrated surface} if the pullback of $F$ to $N$ is $\|F\|_{L^\infty} \omega_N$.
\end{definition}

By Corollary \ref{trace on cycles}(\ref{integral continuous}), we may integrate a calibration over any closed surface $N$, or any small ball in a possibly nonclosed surface $N$.
In particular, the sentence ``$N$ is a calibrated surface'' is well-defined, and it just means that for every sufficiently small relatively open ball $U \subseteq N$,
$$\int_U F = \|F\|_{L^\infty} |U|.$$
Even at this generality, every calibrated surface is minimal: if $N'$ is homologous to $N$ relative to $\partial N = \partial N'$, then by Stokes' theorem,
$$|N'| = \frac{1}{\|F\|_{L^\infty}} \int_{N'} F = \frac{1}{\|F\|_{L^\infty}} \int_N F \leq |N|.$$
Moreover, by Corollary \ref{trace on cycles}(\ref{cohomology exists}), the cohomology class of a calibration is well-defined.

\begin{proposition}\label{Bangert Cui}
Let $F$ be a calibration of best comass.
Then there exists a measured oriented minimal lamination on $M$ whose leaves are calibrated surfaces with respect to $F$.
\end{proposition}
\begin{proof}
This essentially follows from \cite[Theorem 5.1]{bangert_cui_2017}; we make a few remarks about how the proof works at this level of generality.
By a duality argument, which is purely on the level of homology, one may find a closed minimal $d-1$-current $T$ which is \dfn{calibrated} by $F$ in the sense that
$$\langle [T], [F]\rangle = \mathbf M(T) \|F\|_{L^\infty}$$
where $\mathbf M(T)$ is the mass of $T$ \cite[Proposition 2.2]{bangert_cui_2017}.
We stress that, since the theory of \cite[\S2C]{bangert_cui_2017} is on the level of homology, we need not worry that any of the involved expressions are well-defined at our extremely low regularity.
Now by \cite[Theorem 1]{AUER20011095} (or perhaps \todo{Cite laminations paper}, since \cite{AUER20011095} appeals to \cite[\S37]{Simon84} which is for euclidean space rather than space forms), the minimal current $T$ induces a measured oriented minimal lamination $\lambda$.
In particular, if $K$ denotes the space of leaves of $\lambda$ and $K$ denotes its transverse measure,
$$\langle [T], [F]\rangle = \int_K \int_{N_k} F \dif \mu(k)$$
but 
$$\mathbf M(T) = \int_K |N_k| \dif \mu(k).$$
This is only possible if the leaves of $\lambda$ are calibrated surfaces.
\end{proof}

\begin{corollary}\label{best comass lamination}
Let $F$ be $\infty$-longitudinal.
Then the maximum comass locus contains a measured oriented minimal lamination.
\end{corollary}
\begin{proof}
By definition, $\dif A$ is a calibration which minimizes $\|\dif A\|_{L^\infty}$ in $c_1(\mathscr F)$.
Let $\lambda$ be the lamination of Proposition \ref{Bangert Cui}.
Then $\dif A$ is smooth in the tangent directions to each leaf of $\lambda$, since it is a multiple of the area form on $\lambda$.
In particular, $|\dif A(x)|$ exists and equals $\|\dif A\|_{L^\infty}$ if $x \in \supp \lambda$.
The assertion now follows from Proposition \ref{crandall}.
\end{proof}

By \cite[Example 5.4]{bangert_cui_2017} the maximum comass locus need not equal a lamination.

%%%%%%%%%%%%%%%%%%%%%%
\subsection{The dual \texorpdfstring{$1$-harmonic}{1-harmonic} function}
In this section we interpret the lamination of Corollary \ref{best comass lamination} using convex duality.

\begin{theorem}
Let $F$ be an $\infty$-longitudinal form, which is the Noether current of a $1$-harmonic function $u$, and let $\lambda_u$ be the measured oriented lamination induced by $u$.
Moreover, let $\tilde F$ be any best comass form cohomologous to $F$.
Then the calibrated lamination of $\tilde F$ contains $\lambda_u$.
\end{theorem}


To begin the proof, let $(A_p)$ be a $p$-approximation to $A$.
Using the scaling techniques of Remark \ref{scalings}, we may assume that $\|\dif A\|_{L^\infty} = 1$; this eases the notational burden somewhat.
We follow \cite[\S6.1]{daskalopoulos2020transverse}.

We begin by constructing the function $u$.
For $3 < p < \infty$ and $\frac{1}{p} + \frac{1}{q} = 1$, let
$$\int_M \star |k_p \dif A_p|^p = k_p$$
just like in \cite[\S3.2]{daskalopoulos2020transverse}.
Let $F_p := k_p \dif A_p$ and $U_q := |F_p|^{p - 2} \star F_p$ which is a $1$-form.
Then 
\begin{equation}\label{dAp wedge Uq}
\int_M \dif A_p \wedge U_q = k_p^{-1} \int_M \star |F_p|^p = 1.
\end{equation}
Also for $F := \dif A$,
$$\lim_{p \to \infty} k_p^{-\frac{1}{q}} = \lim_{p \to \infty} \|\dif A_p\|_{L^p} = \|F\|_{L^\infty} = 1.$$
It follows that $k_p \to 1$.

We now compute
\begin{equation}\label{Uq is closed}
	\dif U_q = \dif(|F_p|^{p - 2} \star F_p) = 0
\end{equation}
since $F_p$ solves the $p$-Maxwell equation (\ref{predual problem}).
So we may choose $u_q$ such that $U_q = \dif u_q$ and $u_q(0) = 0$ for some choice of origin $0$ of the universal cover $\tilde M$.

We have the following analogue of \cite[Theorem 4.3]{daskalopoulos2020transverse} with essentially the same proof.

\begin{lemma}
There exists $u \in BV_\loc(\tilde M)$ and $\rho \in H^1(M, \RR)$ such that:
\begin{enumerate}
\item $u$ is $\rho$-equivariant.
\item $u_q \to u$ weakly in $BV_\loc(\tilde M)$.
\item $u_q \to u$ in $L^r(M)$ for $1 \leq r < \infty$.
\end{enumerate}
\end{lemma}
\begin{proof}
By H\"older's inequality,
$$\int_M \star |U_q| = \int_M \star |F_p|^{p - 1} \leq |M|^{\frac{1}{p}} \left[\int_M \star |F_p|^p\right]^{\frac{1}{q}}.$$
Therefore 
$$\limsup_{q \to 1} \int_M \star |U_q| \leq 1.$$
On the other hand, (\ref{dAp wedge Uq}) is only possible if
$$\liminf_{q \to 1} \int_M \star |U_q| \geq 1$$
and so $\|U_q\|_{L^1} \to 1$.
Taking limits in (\ref{Uq is closed}) and using a compactness argument, we obtain $U_q \to U$ in the weak topology of measures, for some closed $2$-current $U$.
We write $\dif u = U$.

Since $U_q, F_p$ are related by the convex duality relation (\ref{inverse extremality}),
\end{proof}



So by the Maz\'on-Rosser-Segura de Le\'on theorem, $u$ is $1$-harmonic.

%%%%%%%%%%%%%%%%%%%%%
\section{\texorpdfstring{$L = K$}{L equals K}}
Let $\alpha \in H^2(M, \RR)$.
For measured oriented laminations $\lambda$, let 
$$K(\lambda) := \frac{\langle \alpha, [\lambda]\rangle}{|\lambda|}$$
where $[\lambda]$ is the homology class of (the Ruelle-Sullivan current for) $\lambda$ and $|\lambda|$ is the area of $\lambda$,
$$|\lambda| := \int_I |\{k\} \times J| \dif \mu_\lambda(k).$$
Let $K$ be the supremum of $K(\lambda)$ over all $\lambda$.
Let $L$ be the best comass constant.

\begin{theorem}[$L = K$]
Let $F$ be an $\infty$-longitudinal form which is the Noether current of a $1$-harmonic function $u$ with lamination $\lambda_u$.
Then
$$K = K(\lambda_u) = L(F) = L.$$
\end{theorem}

%%%%%%%%%%%%%%%%%%%%%
\appendix 

\section{The dual one-harmonic function}
For simplicity let's assume $H^2(M, \ZZ) = 0$.

Let $A$ be an $\infty$-magnetic potential and let $A_p \to A$ witness that $A$ is $\infty$-magnetic.


Question: Does the Fenchel relation imply that $u$ is nonzero?
What about nonconstant?
\cite{bangert_cui_2017} implies that there exists $U$ which is calibrated by $F$, and then $U = \dif \tilde u$, but it doesn't seem clear that this $\tilde u$ should be the same as $U$, maybe the uniqueness part of the M-R-SdL theorem explains this.

\begin{lemma}
For any $\theta \in (0, 1)$,
	$$\lim_{p \to \infty} \int_{\{|F| \leq \theta\}} \star |F_p|^p = 0.$$
\end{lemma}
\begin{proof}
Integrating the $p$-Maxwell equation 
$$\dif^*(|\dif A_p|^{p - 2} \dif A_p) = 0$$
by parts 
\begin{align*}
	0 &= \int_M \dif \star (|\dif A_p|^{p - 2} \dif A_p) \wedge (A_p - A) \\
	&= \int_{\partial M} \star(|\dif A_p|^{p - 2} \dif A_p) \wedge \iota^* (A_p - A) - \int_M \star (|\dif A_p|^{p - 2} \dif A_p) \wedge \dif (A_p - A).
\end{align*}
Here $\iota^* (A_p - A)$ is well-defined because $A_p - A \in C^0$.
I guess that $|\dif A_p|^{p - 2} \dif A_p$ should have a trace as well, otherwise we need some kind of smooth approximation.
Let $J$ be the Neumann data, then 
$$\dif \iota^*(A_p - A) = J - J = 0$$
so $\iota^*(A_p - A)$ is pure gauge. In particular, since we may modify the $p$-approximations $A_p$ up to gauge without affecting the result, we may assume that $\iota^*(A_p - A) = 0$.
Therefore 
$$\int_M |\dif A_p|^{p - 2} \dif A_p \wedge \star \dif (A_p - A) = 0.$$

We now proceed as in \cite[Lemma 6.3]{daskalopoulos2020transverse}.
Let $f(p) := \langle F_p, F_p - F\rangle$ and $Y_p := \{f(p) \geq 0\}$.
Rescaling by $k_p^p$, 
$$\int_M |F_p|^{p - 2} F_p \wedge \star (F_p - k_p F) = 0$$
so 
\begin{align*}
	\lim_{p \to \infty} \int_M |F_p|^{p - 2} F_p \wedge \star (F_p - k_p F) - \star f(p) 
	&= \lim_{p \to \infty} \int_M |F_p|^{p - 2} F_p \wedge \star (F - k_p F) \\
	&\leq \lim_{p \to \infty} (1 - k_p) \int_M |F_p|^{p - 1} \star |F| = 0.
\end{align*}
Therefore 
$$\lim_{p \to \infty} \int_M |F_p|^{p - 2} \star f(p) = 0.$$
On the other hand, on $M \setminus Y_p$ we have by the elementary \cite[Lemma 6.2]{daskalopoulos2020transverse} that 
$$
	-2|F_p|^{p - 2} f(p) \leq |F_p|^{p - 2} (|F|^2 - |F_p|^2) < \frac{2}{p - 2}
$$
which implies 
$$\lim_{p \to \infty} \int_{M \setminus Y_p} |F_p|^{p - 2} \star f(p) = 0.$$
Therefore 
$$\lim_{p \to \infty} \int_{Y_p} |F_p|^{p - 2} \star f(p) = 0.$$

To finish the proof we look to \cite[Proposition 6.5]{daskalopoulos2020transverse}...
\end{proof}

\begin{corollary}
$\supp \dif v$ is contained in the calibrated lamination associated to any best curl form with the same data as $A$.
\end{corollary}
\begin{proof}
By as in \cite[Theorem 6.1]{daskalopoulos2020transverse}, $\supp \dif v \subseteq \{|F| = \|F\|_{L^\infty}\}$.
Actually, as in \cite[Corollary 6.8]{daskalopoulos2020transverse} the same argument works as long as $F = \dif \tilde A$ where $\tilde A$ is a best curl competitor of $A$.
\end{proof}

\subsection{Thurston's K-L theorem}
\begin{conjecture}
For $\sigma \in H_2(M, \RR)$ nontrivial, let $\lambda_\sigma$ denote the minimal measured lamination of homology class $\sigma$.
Let 
$$K = \sup_\sigma \frac{1}{|\lambda_\sigma|} \int_{\lambda_\sigma} \dif A.$$
Let $L$ be the best curl constant. Then 
$$L = K.$$
\end{conjecture}

This is nontrivial because $K$ is a supremum over cycles while the curl modulus is a supremum over chains.

\begin{lemma}
\begin{equation}\label{pSecondVariation}
	p \int_M (p - 2) |F_p|^{p - 4} \star \langle F_p, \dif B\rangle^2 + \star |F_p|^{p - 2} |\dif B|^2.
	\end{equation}
	\end{lemma}
	\begin{proof}
	Let $A_p(t) := A_p + tB$; then 
	\begin{align*}
		\frac{\dif}{\dif t} \int_M \star |\dif A_p(t)|^p
		&= p \int_M \star |\dif A_p(t)|^{p - 2} \langle\dif A_p(t), \dif B\rangle.
	\end{align*}
	Setting $t = 0$, (\ref{pMaxwell}) follows.
	We now differentiate again in $t$ to deduce (\ref{pSecondVariation}).
	\end{proof}

\printbibliography

\end{document}
