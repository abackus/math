\documentclass[reqno,11pt]{amsart}
\usepackage[letterpaper, margin=1in]{geometry}
\RequirePackage{amsmath,amssymb,amsthm,graphicx,mathrsfs,url,slashed,subcaption}
\RequirePackage[usenames,dvipsnames]{xcolor}
\RequirePackage[colorlinks=true,linkcolor=Red,citecolor=Green]{hyperref}
\RequirePackage{amsxtra}
\usepackage{cancel}
\usepackage{tikz-cd}
%\usepackage[T1]{fontenc}

% \setlength{\textheight}{9.3in} \setlength{\oddsidemargin}{-0.25in}
% \setlength{\evensidemargin}{-0.25in} \setlength{\textwidth}{7in}
% \setlength{\topmargin}{-0.25in} \setlength{\headheight}{0.18in}
% \setlength{\marginparwidth}{1.0in}
% \setlength{\abovedisplayskip}{0.2in}
% \setlength{\belowdisplayskip}{0.2in}
% \setlength{\parskip}{0.05in}
%\renewcommand{\baselinestretch}{1.05}

\title{Convex duality between minimal laminations and tight calibrations}
\author{Aidan Backus}
\address{Department of Mathematics, Brown University}
\email{aidan\_backus@brown.edu}
\date{\today}

\newcommand{\NN}{\mathbf{N}}
\newcommand{\ZZ}{\mathbf{Z}}
\newcommand{\QQ}{\mathbf{Q}}
\newcommand{\RR}{\mathbf{R}}
\newcommand{\CC}{\mathbf{C}}
\newcommand{\DD}{\mathbf{D}}
\newcommand{\PP}{\mathbf P}
\newcommand{\MM}{\mathbf M}
\newcommand{\II}{\mathbf I}
\newcommand{\Hyp}{\mathbf H}
\newcommand{\Sph}{\mathbf S}
\newcommand{\Group}{\mathbf G}
\newcommand{\GL}{\mathbf{GL}}
\newcommand{\Orth}{\mathbf{O}}
\newcommand{\SpOrth}{\mathbf{SO}}
\newcommand{\Ball}{\mathbf{B}}

\newcommand*\dif{\mathop{}\!\mathrm{d}}

\DeclareMathOperator{\card}{card}
\DeclareMathOperator{\dist}{dist}
\DeclareMathOperator{\id}{id}
\DeclareMathOperator{\Hom}{Hom}
\DeclareMathOperator{\coker}{coker}
\DeclareMathOperator{\supp}{supp}
\DeclareMathOperator{\Teich}{Teich}
\DeclareMathOperator{\tr}{tr}

\newcommand{\Leaves}{\mathscr L}
\newcommand{\Lagrange}{\mathcal L}
\newcommand{\Hypspace}{\mathscr H}

\newcommand{\Chain}{\underline C}

\newcommand{\Two}{\mathrm{I\!I}}

\newcommand{\normal}{\mathbf n}
\newcommand{\radial}{\mathbf r}
\newcommand{\evect}{\mathbf e}
\newcommand{\vol}{\mathrm{vol}}

\newcommand{\diam}{\mathrm{diam}}
\newcommand{\Ell}{\mathrm{Ell}}
\newcommand{\inj}{\mathrm{inj}}
\newcommand{\Lip}{\mathrm{Lip}}
\newcommand{\MCL}{\mathrm{MCL}}
\newcommand{\Riem}{\mathrm{Riem}}

\newcommand{\Mass}{\mathbf M}
\newcommand{\Comass}{\mathbf L}

\newcommand{\Min}{\mathrm{Min}}
\newcommand{\Max}{\mathrm{Max}}

\newcommand{\dfn}[1]{\emph{#1}\index{#1}}

\renewcommand{\Re}{\operatorname{Re}}
\renewcommand{\Im}{\operatorname{Im}}

\newcommand{\loc}{\mathrm{loc}}
\newcommand{\cpt}{\mathrm{cpt}}

\def\Japan#1{\left \langle #1 \right \rangle}

\newtheorem{theorem}{Theorem}[section]
\newtheorem{badtheorem}[theorem]{``Theorem"}
\newtheorem{prop}[theorem]{Proposition}
\newtheorem{lemma}[theorem]{Lemma}
\newtheorem{sublemma}[theorem]{Sublemma}
\newtheorem{proposition}[theorem]{Proposition}
\newtheorem{corollary}[theorem]{Corollary}
\newtheorem{conjecture}[theorem]{Conjecture}
\newtheorem{axiom}[theorem]{Axiom}
\newtheorem{assumption}[theorem]{Assumption}

\newtheorem{mainthm}{Theorem}
\renewcommand{\themainthm}{\Alph{mainthm}}

\newtheorem{claim}{Claim}[theorem]
\renewcommand{\theclaim}{\thetheorem\Alph{claim}}
% \newtheorem*{claim}{Claim}

\theoremstyle{definition}
\newtheorem{definition}[theorem]{Definition}
\newtheorem{remark}[theorem]{Remark}
\newtheorem{example}[theorem]{Example}
\newtheorem{notation}[theorem]{Notation}

\newtheorem{exercise}[theorem]{Discussion topic}
\newtheorem{homework}[theorem]{Homework}
\newtheorem{problem}[theorem]{Problem}

\makeatletter
\newcommand{\proofpart}[2]{%
  \par
  \addvspace{\medskipamount}%
  \noindent\emph{Part #1: #2.}
}
\makeatother



\numberwithin{equation}{section}


% Mean
\def\Xint#1{\mathchoice
{\XXint\displaystyle\textstyle{#1}}%
{\XXint\textstyle\scriptstyle{#1}}%
{\XXint\scriptstyle\scriptscriptstyle{#1}}%
{\XXint\scriptscriptstyle\scriptscriptstyle{#1}}%
\!\int}
\def\XXint#1#2#3{{\setbox0=\hbox{$#1{#2#3}{\int}$ }
\vcenter{\hbox{$#2#3$ }}\kern-.6\wd0}}
\def\ddashint{\Xint=}
\def\dashint{\Xint-}

\usepackage[backend=bibtex,style=alphabetic,giveninits=true]{biblatex}
\renewcommand*{\bibfont}{\normalfont\footnotesize}
\addbibresource{best_curl.bib}
\renewbibmacro{in:}{}
\DeclareFieldFormat{pages}{#1}

\newcommand\todo[1]{\textcolor{red}{TODO: #1}}


\begin{document}
\begin{abstract}
We introduce a family of closed $d-1$-forms on Riemannian $d$-manifolds which minimize their comass (or $L^\infty$ norm) in their cohomology class, which we call \dfn{tight}.
Tight forms have properties similar to (gradients of) $\infty$-harmonic maps between surfaces: they are convex duals of $1$-harmonic functions and attain their comass on a measured oriented minimal lamination $\mu$.
We show that $\mu$ has properties analogous to a measured sublamination of Thurston's canonical lamination.
\todo{Rewrite this}
\end{abstract}

\maketitle

%%%%%%%%%%%%%%%%%%%%%%%%%%%%%%%%%%%%%%%%%%%%%%%%%%%%%%%
\section{Introduction}
The classical \dfn{max flow min cut theorem} asserts that in a discrete flow network $\mathcal G$, the maximal flow through $\mathcal G$ is equal to the \dfn{minimal cut} of $\mathcal G$, namely the minimal total capacity of any set of edges which divide $\mathcal G$ into a source component and a sink component; this theorem is a concrete form of the duality theorem for convex optimization.
As we shall survey below, various continuous versions of the max flow min cut theorem have since appeared, which usually replace the maximal flow by a \dfn{calibration} (a closed $d - 1$-form, or equivalently a divergence-free vector field, of $L^\infty$ norm $1$), and the minimal cut by a minimal submanifold.
In particular, Bangert and Cui have shown that given a continuous calibration $F$, there exists a minimal lamination calibrated by $F$ \cite[Theorem 5.1]{bangert_cui_2017}.

We shall prove a particularly strong version of the Bangert--Cui max flow min cut theorem on closed Riemannian manifolds of dimension $\leq 4$.
We only need to assume that the calibration $F$ is measurable; moreover, by it shall follow from our methods that the calibrated lamination is Lipschitz.
We use the approach of $p$-harmonic duality which was pioneered by Daskalopolous and Uhlenbeck in the context of best Lipschitz maps \cite{daskalopoulos2020transverse,daskalopoulos2022,daskalopoulos2023}.
To be more precise, we pick out a special class of calibrations, which we call \dfn{tight forms}, which are the variational solutions of a suitable generalization of the $\infty$-Laplace equation.
Tight forms are necessarily convex duals of $1$-harmonic functions, and as $1$-harmonic functions induce minimal laminations \cite[Theorem C]{BackusCML}, the main theorem follows.

%%%%%%%%%%%%%%%
\subsection{Max flow min cut theorems}
So that we may compare it with its continuous form, we recall the classical max flow min cut theorem \cite[Chapter 7]{umesh2006algorithms}.
A \dfn{flow network} $\mathcal G = (G, c, s, t)$ consists of a finite directed graph $G$, a \dfn{capacity} $c: E(G) \to \RR_+$ on the edges of $G$, a \dfn{source} vertex $s \in V(G)$, and a \dfn{sink} vertex $t \in V(G)$.
A \dfn{flow} through $\mathcal G$ is a function $f: E(G) \to \RR_+$ such that for each vertex except $s, t$, the sum of $f(e)$ over incoming edges $e$ equals the sum of $f(e)$ over the outgoing edges $e$; we write $\|f\|$ for the sum of $f(e)$ over edges $e$ leaving $s$.
A \dfn{cut} of $G$ is a partition $V(G) = S \sqcup T$ such that $s \in S$ and $t \in T$; $\|(S, T)\|$ is the sum over all edges $e$ from $S$ to $T$ of $c(e)$.

\begin{theorem}[max flow min cut, graphs]
Let $\mathcal G$ be a flow network. Then 
$$\max_f \|f\| = \min_{(S, T)} \|(S, T)\|.$$
In particular, if $f$ is a maximal flow and $(S, T)$ is a minimal cut, then $\|(S, T)\|$ equals the sum of $f(e)$ over all edges $e$ from $S$ to $T$.
\end{theorem}

Given a Riemannian manifold $M$ diffeomorphic to $\Sph^{d - 1} \times [-1, 1]$ (for example), and a scale $0 < \varepsilon \ll 1$, we can discretize $M$ at scale $\varepsilon$ to get a flow network, where most of the vertices are tiles at scale $\varepsilon$, but the source $s$ is the entire sphere $\Sph^{d - 1} \times \{-1\}$ and the sink $t$ is the other sphere $\Sph^{d - 1} \times \{+1\}$.
Then a flow corresponds to a divergence-free vector field (which for our purposes will be more convenient to view as a closed $d - 1$-form) of $L^\infty$ norm $1$ and a cut corresponds to a hypersurface which partitions $M$ into two open sets, one containing $s$ and the other $t$.
If $C$ is any cut, $\overline C$ is a minimal cut, and $F$ is a flow, then by Stokes' theorem and the triangle inequality,
\begin{equation}\label{one sided max flow min cut}
\int_C F = \int_{\overline C} F \leq |\overline C|.
\end{equation}
Thus the sum of any flow over any cut is bounded by the size of a minimal cut.
We refer to \cite[Appendix A]{Freedman_2016} for a more precise version of the above discretization.
\todo{Draw a picture of this}

The continuous max flow min cut theorem is the assertion that the bound (\ref{one sided max flow min cut}) is sharp: the sum of the max flow over any cut must equal the area of a minimal cut.
In other words, the continuous max flow min cut theorem asserts that we can find a closed $d - 1$-form $F$ such that $\|F\|_{L^\infty} = 1$ and for any cut $C$,
$$\int_C F = |\overline C|.$$

A form of the above sketchy theorem was established by Federer \cite[\S4.15]{Federer1974}.
In the modern formulation of Federer's theorem, we introduce the stable norm on homology:

\begin{definition}
The \dfn{stable norm} $\Mass(\sigma)$ of a homology class $\sigma \in H_{d - 1}(M, \RR)$ is the infimum of the mass $\Mass(N)$ among all representative chains $N$.
The \dfn{costable norm} $\Comass(\rho)$ of a cohomology class $\rho \in H^{d - 1}(M, \RR)$ is the infimum of $\|F\|_{L^\infty}$ among all representative forms $F$.
\end{definition}

\begin{theorem}[max flow min cut, Federer's version]\label{Federer}
Let $M$ be a closed Riemannian manifold and $\sigma \in H_{d - 1}(M, \RR)$.
Then there exists a cohomology class $\rho \in H^{d - 1}(M, \RR)$ such that $\Comass(\rho) \leq 1$, and 
$$\Mass(\sigma) = \langle \rho, \sigma\rangle.$$
\end{theorem}

In particular, if we interpret representatives of $\sigma$ is ``cuts'', a closed minimal hypersurface representing $\sigma$ is a minimal cut.
Meanwhile, a representative $d - 1$-form $F$ of $\rho$ such that $\|F\|_{L^\infty} = \Comass(\rho)$ is a maximal flow.
By a doubling argument, Theorem \ref{Federer} implies the max flow min cut theorem for $M \cong \Sph^{d - 1} \times [-1, 1]$ outlined above, at least under the nondegeneracy condition that there is a closed minimal hypersurface homologous to a boundary component which does not meet $\partial M$ and has strictly less area than either boundary component.
For this reason, we shall henceforth content ourselves to consider closed manifolds, and leave the doubling arguments for manifolds with boundary to the reader.\footnote{On the other hand, there are honest differences in the theory for manifolds with infinite ends, which we shall not attempt to address here. For example, a manifold with cusps may have a homology class whose stable norm is $0$!}

Federer's theorem has found myriad applications, including in computational geometry \cite{sullivan1990crystalline} and string theory \cite{Freedman_2016}; both references here observe that Federer has essentially proven a max flow min cut theorem.
Under the assumption $d \leq 4$, our main theorem implies Theorem \ref{Federer}.

Another form of the max flow min cut theorem was introduced by Thurston in his work on best Lipschitz maps \cite{Thurston98}.
A \dfn{best Lipschitz map} $u$ is a map which minimizes its Lipschitz constant, among all maps homotopic to $u$.
Thurston's motivation was to define a Finsler metric on the Teichm\"uller space $\widetilde{\mathscr M_g}$: the \dfn{Thurston asymmetric distance} between two hyperbolic structures on a closed surface $S$ of genus $g \geq 2$ is the Lipschitz constant of a best Lipschitz map homotopic to $\id_S$.
Thurston's asymmetric metric is a particularly appealing geometry on $\widetilde{\mathscr M_g}$ because of its intimate connection with the structure of geodesic laminations on hyperbolic surfaces \cite{Thurston98, Gu_ritaud_2017}:

\begin{theorem}[max flow min cut, Thurston's version]\label{existence of thurston lamination}
Let $f: M \to N$ be a homeomorphism of closed hyperbolic surfaces.
Then there exists a best Lipschitz map $u$ homotopic to $f$, and a measured geodesic lamination $\mu$ on $M$, such that $\mu$ pushes forward to a measured geodesic lamination on $N$, and
\begin{equation}\label{L is K}
	\Lip(u) = \frac{\Mass(u_* \mu)}{\Mass(\mu)} = \sup_\lambda \frac{\Mass(u_* \lambda)}{\Mass(\lambda)},
\end{equation}
the supremum taken over all measured laminations $\lambda$ on $M$.
\end{theorem}

Thurston established Theorem \ref{existence of thurston lamination} in a tour de force of geometric topology, but he conjectured that a proof of his result ``should be feasible with a simpler proof based on more general principles -- in particular, the max flow min cut principle, convexity, and $L^0 \leftrightarrow L^\infty$ duality'' \cite[Abstract]{Thurston98}.

Theorem \ref{existence of thurston lamination} does not follow from our main theorem; however, the analogous result for maps from closed hyperbolic surfaces to $\Sph^1$, which was proven by Daskalopolous and Uhlenbeck \cite{daskalopoulos2020transverse}, is a corollary of our main result.
The main idea of \cite{daskalopoulos2020transverse} is that a special class of best Lipschitz maps are the $\infty$-harmonic maps, which in turn are dual to $1$-harmonic functions (or equivalently, geodesic laminations).
In fact, we closely follow \cite{daskalopoulos2020transverse} for much of the proof, as we also heavily use $p$-harmonic maps to $\Sph^1$.
As Daskalopolous and Uhlenbeck have given a proof of a version of Theorem \ref{existence of thurston lamination} using similar methods \cite{daskalopoulos2022,daskalopoulos2023}, it may be possible to adapt our results to manifold-valued maps.

The statement in the literature that we are aware of, which is closest to our main theorem, is due to Bangert and Cui \cite{bangert_cui_2017}.
To emphasize the analogy with best Lipschitz maps, we call a form \dfn{best comass} if it minimizes its comass:

\begin{definition}
The \dfn{comass} of a closed $d - 1$-form $F$ is $\|F\|_{L^\infty}$.
A form $F$ is \dfn{best comass} if it minimizes its comass among all forms cohomologous to $F$.
\end{definition}

\begin{definition}
A \dfn{calibration} is a measurable closed $d - 1$-form $F$, such that $\|F\|_{L^\infty} = 1$.
Given a calibration $F$, a hypersurface $N$ is $F$-\dfn{calibrated} if the pullback of $F$ to $N$ is the area form on $N$.
\end{definition}

Observe that a best comass form is a calibration iff the costable norm of its cohomology class is $1$.
Moreover, the fundamental theorem of calibrated geometry \cite{Harvey82} asserts that every $F$-calibrated hypersurface $N$ is minimal (assuming that the pullback exists in the sense of the trace theorem).
In fact, the mean curvature of $N$ satisfies
\begin{equation}\label{calibrated surfaces are minimal}
H_N = \nabla \cdot \normal_N = \nabla \cdot (\star F)^\sharp = \star \dif F = 0.
\end{equation}
With these preliminaries in mind, we may now state the Bangert--Cui theorem \cite[Theorem 5.1]{bangert_cui_2017}:

\begin{theorem}[max flow min cut, Bangert and Cui's version]\label{BangertCui}
Let $F$ be a continuous best comass calibration on a closed Riemannian manifold of dimension $\leq 7$.
Then there exists a measured oriented minimal lamination $\lambda$ such that every leaf of $\lambda$ is $F$-calibrated.
In particular, 
$$\Mass(\lambda) = \langle [\lambda], [F]\rangle.$$
\end{theorem}

The idea of Theorem \ref{BangertCui} is to use Theorem \ref{Federer} to find a homology class $\sigma$ dual to $[F]$, and then consider a minimizing representative $\lambda$ of $\sigma$, so that $\Mass(\lambda)$ is the stable norm of $\sigma$.
However, most calibrations which appear in nature are measurable and not necessarily continuous \cite[\S1]{bangert_cui_2017}, and it is natural to restrict to \emph{Lipschitz} laminations \cite[Remark 2.3]{bangert_cui_2017}.
Moreover, Bangert and Cui appeal to the regularity theory of minimal hypersurfaces on $\RR^d$ \cite[\S37]{Simon84}; as is common in the literature, they suppress the detail of generalizing the regularity theory to Riemannian manifolds.
\todo{Also Auer and Bangert never appeared??}
Thus, it is natural to strengthen Theorem \ref{BangertCui} to not require that $F$ is continuous, to show that the lamination is Lipschitz \todo{and get rid of the BS}.


%%%%%%%%%%%%%%%%%%%%%%%%%%%%%%%%%%%%%%%%%

\subsection{\texorpdfstring{Tight forms and $1$-harmonic functions}{Tight forms and one-harmonic functions}}
We shall prove a Bangert--Cui-type theorem by Daskalopolous and Uhlenbeck's method of $p$-harmonic maps to $\Sph^1$.
To do this, we construct best comass forms which are analogous to $\infty$-harmonic functions in the same spirit as \cite{daskalopoulos2020transverse,daskalopoulos2022}, and in particular which arise as limits of the $L^p$ analogue of best comass forms.
These will be ``maximal flows''; their dual ``minimal cuts'' will be $1$-harmonic functions.

To be more precise, let $(p, q)$ be a H\"older pair (thus $1/p + 1/q = 1$) such that $d < p < \infty$.
Motivated by the $p$-Laplace equation $\dif^*(|\dif v|^{p - 2} \dif v) = 0$, we introduce \dfn{$p$-tight} forms, which are closed $d-1$-forms which solve the system of PDE
$$\dif^*(|F|^{p - 2} F) = 0.$$
Given a $p$-tight form, the $\pi_1(M)$-equivariant function $u$ on the universal cover such that
$$\dif u = (-1)^{d - 1} |F|^{p - 2} \star F$$
is $q$-harmonic -- in other words, $u$ is a solution of the $q$-Laplace equation 
$$\dif^*(|\dif u|^{q - 2} \dif u) = 0.$$
Our first theorem constructs a best comass form, and a dual $1$-harmonic function, by taking limits of $p$-tight forms and their dual $q$-harmonic functions.

\begin{mainthm}\label{existence of infinity tight forms}
Let $\rho \in H^{d - 1}(M, \RR)$ be a cohomology class.
Let $(F_p, u_q)$ be the family of dual pairs of $p$-tight forms and $q$-harmonic functions, suitably normalized, with $[F_p] = \rho$ and $(p, q)$ ranging over H\"older pairs with $d < p < \infty$.
Then there exists a pair $(F, u)$ such that as $p \to \infty$ along a subsequence, $F_p \to F$ weakly in $L^r$ for any $d < r < \infty$, and $u_q \to u$ weakly in $BV$, with the following properties:
\begin{enumerate}
\item $F$ has best comass.
\item $u$ is $1$-harmonic.
\item The product of distributions $\dif u \wedge F$ is well-defined.
\item We have the duality relation
\begin{equation}\label{max flow mean cut}
\Comass(\rho) \star |\dif u| = \dif u \wedge F.
\end{equation}
\end{enumerate}
\end{mainthm}

This is a combination of Propositions \ref{existence infinity} and \ref{existence 1}.
We call the best comass form $F$ an \dfn{$\infty$-tight} form, or simply a \dfn{tight} form.

We highlight the max flow min cut-type formula (\ref{max flow mean cut}) as the main point of the theorem.
It is crucial to the prooof that $u$ is $1$-harmonic, and allows us to prove this without a careful analysis of the limiting behavior of $q$-harmonic functions as in \cite[Theorem 2.4]{Mazon14}, or of $p$-tight forms as in \cite[\S6]{daskalopoulos2020transverse}.
On the other hand, if $\Comass(\rho) = 1$, then we shall be able to use (\ref{max flow mean cut}) to show that the level sets of $u$ are $F$-calibrated.

%%%%%%%%%%%%%%%%%%

\subsection{Calibration of the measured stretch lamination}
Suppose that $\rho \in H^{d - 1}(M, \RR)$ satisfies $\Comass(\rho) = 1$.
If we knew that every tight representative $F$ of $\rho$ was continuous, then it would follow from the Bangert--Cui theorem that $F$ calibrates a minimal lamination.
However, a proof that tight forms are continuous is out of reach.\footnote{For domains in $\RR^2$, $\infty$-tight forms are $C^\alpha$ \cite{Evans08}, but it is unlikely that this argument generalizes.}
On the other hand, we know by my previous work \cite[Theorem C]{BackusCML} that the level sets of a $1$-harmonic function form a measured oriented Lipschitz minimal lamination.

\begin{definition}
Let $M$ be a closed Riemannian manifold of dimension $2 \leq d \leq 4$, and $\rho \in H^{d - 1}(M, \RR)$.
We can define a measured oriented minimal lamination $\mu$, by considering a tight representative $F$ of $\rho$, letting $u$ be a dual $1$-harmonic function to $F$, and letting $\mu$ be the lamination induced by $u$.
We call $\mu$ a \dfn{measured stretch lamination} associated to $\rho$.
\end{definition}

Our main theorem is the combination of Propositions \ref{MCL contains Thurston} and \ref{L equals K}, and follows easily from (\ref{max flow mean cut}) and the theory of \cite{BackusFLG, BackusCML}.
It characterizes the measured stretch lamination as a lamination which is maximally stretched in the sense of (\ref{L is K}).
To state the theorem, we let $[\lambda]$ denote the homology class of a measured oriented lamination $\lambda$.

\begin{mainthm}\label{lams are calibrated}
Suppose that $M$ is a closed Riemannian manifold of dimension $d \leq 4$.
Let $\mu$ be a measured stretch lamination associated to $\rho \in H^{d - 1}(M, \RR)$, and let $F$ be a form of best comass representing $\rho$.
Then $\mu$ is $F/\Comass(\rho)$-calibrated.
Moreover, for $\lambda$ ranging over measured oriented laminations,
\begin{equation}\label{duality between stable and comass}
\Comass(\rho) = \sup_\lambda \frac{\langle \rho, [\lambda]\rangle}{\Mass(\lambda)} = \frac{\langle \rho, [\mu]\rangle}{\Mass(\mu)}.
\end{equation}
\end{mainthm}

Every measured stretch lamination is homologically area-minimizing, and conversely, every homology class contains a measured stretch lamination.
From these considerations, we see that Theorem \ref{Federer} is a corollary of Theorem \ref{lams are calibrated}.

\todo{Connections to Thurston? What about absolute characterization of tight forms?}


%%%%%%%%%%%%%%%%%%%%%
\subsection{Outline of the paper}
\todo{Fix this}

In \S\ref{comass sec}, we outline the basic properties of the comass, and the geometric measure theory needed for the problem at hand.

In \S\ref{tight forms sec}, we construct the $\infty$-tight form in each cohomology class, and its $1$-harmonic conjugate, proving Theorem \ref{existence of infinity tight forms}.
We also state a few natural conjectures about $\infty$-tight and $p$-tight forms which we shall not address here.

In \S\ref{MCL sec}, we study the maximum comass locus of a form of best comass, and the measured stretch lamination associated to its cohomology class.
By applying the results of the previous two sections, we prove Theorem \ref{lams are calibrated}.


%%%%%%%%%%%%%%%%%%%%%%
\subsection{Acknowledgements}
I would like to thank Georgios Daskalopolous and Karen Uhlenbeck for suggesting this project, providing helpful comments, and providing me with an early draft of the manuscript \cite{daskalopoulos2023} which was a major source of inspiration for this work.
I also would like to thank Tom Goodwillie, Kaya Ferendo, Tainara Borges, and Haram Ko for helpful discussions.

This research was supported by the National Science Foundation's Graduate Research Fellowship Program under Grant No. DGE-2040433.


%%%%%%%%%%%%%%%%%%%%%%%%%%%%%%%%%%%%%%%%%%
\section{Summary of \texorpdfstring{\cite{BackusFLG,BackusCML}}{previous results}}
\subsection{Geometric measure theory}
To fix notation and conventions we recall some well-known geometric measure theory.
We also shall recall some technicalities which we established in \cite{BackusFLG}.

The sheaf of $\ell$-forms is denoted $\Omega^\ell$.
We assume that $\ell$-forms are $L^1_\loc$, but \emph{not} that they are continuous; hence $\dif$ must be meant in the sense of distributions.
To avoid confusion, we write $H^\ell$ for de Rham cohomology, but \emph{never} a Sobolev space (which shall only be denoted by $W^{\ell, p}$), nor a Hausdorff measure (which shall be denoted $\mathcal H^\ell$).
We let $\dif V = \star 1$ be the volume form on $M$, and for an $\ell$-rectifiable set $\tau$, we let $\dif S_\tau := \dif \mathcal H^\ell|_\tau$.

An \dfn{$\ell$-blade} is the wedge product $v = v_1 \wedge \cdots \wedge v_\ell$ of vectors $v_1, \dots, v_\ell$.
The \dfn{comass} $|\varphi|$ of an $\ell$-covector $\varphi$ is the supremum of $\langle \varphi, v\rangle/|v|$, taken over all nonzero $\ell$-blades.

By an $\ell$-\dfn{current of locally finite mass} $T$ we mean a continuous linear functional on the space $C^0_\cpt(M, \Omega^\ell)$ of continuous $\ell$-forms of compact support.
We write $\int_M T \wedge \varphi$ for the dual pairing of a current and a form.
The duality norm of $T$ is its \dfn{mass}, namely for an open set $U \subseteq M$,
$$\Mass_U(T) := \sup_{\substack{\supp \varphi \Subset U \\ \|\varphi\|_{C^0} \leq 1}} \int_U T \wedge \varphi.$$
We write $\Mass(T) := \Mass_M(T)$.
If $T$ represents a simplex $\sigma$, then $\Mass(T)$ is the surface area of $\sigma$.
If $\dif T$ is an $\ell - 1$-current of locally finite mass, we say that $T$ is \dfn{locally normal}.
In particular, a locally normal $d$-current is the same thing as a function of locally bounded variation, and a locally normal $0$-current is the same thing as a signed Radon measure.

We showed that the null set in the statement of the Lebesgue differentiation theorem, when applied to a section of a trivialized tensor bundle, does not depend on the choice of trivialization \cite[Proposition 2.1]{BackusFLG}:

\begin{proposition}\label{invariance of LDT}
Let $F$ be a $L^1_\loc$ section of a tensor bundle over $M$, and let $\mu$ be a Radon measure on $M$.
Then there exists a diffeomorphism-invariant $\mu$-null set $Z \subset M$ such that for any coordinate system on $M$, any multiindex $I$, and any $x \notin Z$,
$$F_I(x) = \lim_{r \to 0} \frac{1}{\mu(B_r(x))} \int_{B_r(x)} F_I(y) \dif \mu(y).$$
\end{proposition}

Suppose that an open set $U \subset M$ has \dfn{locally finite perimeter} in the sense that its indicator function $1_U$ has locally bounded variation.
Then $U$ has a \dfn{reduced boundary} $\partial^* U$, the set of points at which the conormal $1$-form to $\partial^* U$ exists.
We carefully wrote down the coarea formula at this level of regularity on curved domains \cite[Proposition 2.5]{BackusFLG}:

\begin{proposition}\label{coarea}
For any function $u \in BV_\loc(M)$ and open set $U \subset M$ with smooth boundary,
$$\Mass_U(\dif u) = \int_{-\infty}^\infty |\partial^* \{u > \lambda\} \cap U| \dif \lambda.$$
\end{proposition}

%%%%%%%%%%%%%%%%%%%%%
\subsection{\texorpdfstring{$1$-harmonic functions}{One-harmonic functions} and minimal laminations}
We now recall the correct formulation of weak solutions to the $1$-Laplace equation 
$$\dif^*\left(\frac{\dif u}{|\dif u|}\right) = 0$$
which we introduced in \cite{BackusCML}.
The formulation essentially says that $\dif u/|\dif u|$ locally looks like a coclosed $d-1$-current, even if $|\dif u|$ is a singular Radon measure.

\begin{definition}
A \dfn{$1$-harmonic function} on $M$ is a function $u \in BV_\loc(M)$ such that we can cover $\supp u$ by open sets $U_\alpha$ equipped with vector fields $X_\alpha$, such that as Radon measures on $U_\alpha$,
$$(\dif u, X_\alpha) = |\dif u|.$$
\end{definition}

There need not exist a global vector field $X$ with the above properties; the indicator function of the northern hemisphere in $\Sph^2$ is $1$-harmonic, but does not admit such a vector field $X$ by the divergence theorem.
Thus this definition is the ``sheafification'' of the definition due to Maz\'on, Rossi, and Segura de Le\'on \cite{Mazon14}.

\begin{definition}
Let $B$ be a ball in $M$.
A function $u \in BV(B)$ has \dfn{least gradient} in $B$ if $u$ minimizes its total variation in the sense that for every function $v \in BV_\cpt(B)$,
$$\Mass_B(\dif u) \leq \Mass_B(\dif u + \dif v).$$
A function has \dfn{locally least gradient} if one can cover $M$ by balls in which $u$ has least gradient.
\end{definition}

By \cite[Theorem 1.1]{Mazon14}, a function is $1$-harmonic iff it has locally least gradient.
The de Giorgi--Miranda regularity theorem asserts that if $1_E$ has least gradient in a ball in $\RR^d$, $d \leq 7$, then $\partial E$ is smooth \cite[Theorem 10.11]{Giusti77}.
We carefully wrote down the details of the de Giorgi--Miranda theorem on Riemannian manifolds \cite[Theorem 1.1]{BackusCML}.
The one main new idea is an estimate on the parallel propagator, which allows us to define the oscillation of a differential form in an approximately coordinate-invariant way; once this has been established, the proof is similar to \cite{Giusti77}.

\begin{theorem}
Let $E \subset M$ be open, suppose that $d \leq 7$, and suppose that $1_E$ has least gradient.
Then $E$ is bounded by smooth embedded stable minimal hypersurfaces.
\end{theorem} 

We also want a geometric formulation of $1$-harmonic functions.
Thus we turn to the theory of laminations.
Fix an interval $I \subset \RR$ and a box $J \subset \RR^{d - 1}$.

\begin{definition}
A \dfn{laminar flow box} is a $C^0$ coordinate chart $F: I \times J \to M$ and a compact set $K \subseteq I$, such that for every $k \in K$, $F|_{\{k\} \times J}$ is a $C^1$ embedding, and the \dfn{leaf} $F(\{k\} \times J)$ is a $C^1$ complete hypersurface in $F(I \times J)$.
Two laminar flow boxes belong to the same \dfn{laminar atlas} if the transition maps between them send leaves to leaves.
\end{definition}

\begin{definition}
A \dfn{lamination} is a closed subset $S \subseteq M$, called its \dfn{support}, and a maximal laminar atlas $\mathscr A$, such that $S$ is the union of the leaves of $\mathscr A$.
A \dfn{foliation} is a lamination $\lambda$ with $\supp \lambda = M$.
\end{definition}

\begin{definition}
A lamination is
\begin{enumerate}
\item \dfn{Lipschitz} if its flow boxes are Lipschitz isomorphisms,
\item \dfn{oriented} if its transition maps are orientation-preserving, and
\item \dfn{minimal} if its leaves are minimal hypersurfaces.
\end{enumerate}
\end{definition}

Construction of the flow boxes can be quite cumbersome, but it can be done systematically; this is \cite[Theorem A]{BackusCML}.

\begin{theorem}
Let $\mathcal S$ be a set of disjoint complete minimal hypersurfaces in a manifold $M$ of bounded geometry.
Suppose that for every $N \in \mathcal S$, we have a bound on the second fundamental form 
$$\|\Two_N\|_{C^0} \leq C.$$
Then $\mathcal S$ is the set of leaves of a Lipschitz minimal lamination $\lambda$.
In particular, if $\lambda$ is oriented, then there is a Lipschitz vector field on $M$ whose restriction to each $N \in \mathcal S$ is the normal vector to $N$.
\end{theorem}

We now recall the notion of Ruelle-Sullivan current that was introduced by \cite{Ruelle75} and studied in the context of geodesic laminations in \cite[\S8]{daskalopoulos2020transverse}.
The motivation of the definition is that if $\lambda$ is a $d - 1$-chain, then $T_\lambda$ is just integration along $\lambda$.

\begin{definition}
A lamination $\lambda$ with atlas $(F_\alpha, K_\alpha)$ is \dfn{measured} if it is equipped with positive Radon measures $\mu_\alpha$ with $\supp \mu_\alpha = K_\alpha$, such that the transition maps $F_\beta^{-1} \circ F_\alpha$ are measure-preserving.
The \dfn{Ruelle-Sullivan current} of a measured oriented lamination $\lambda$ with atlas $(F_\alpha, K_\alpha, \mu_\alpha)$ is the $d-1$-current $T_\lambda$ satisfying, for any partition of unity $(\chi_\alpha)$ subordinate to the open cover $(F_\alpha(I \times J))$,
$$\int_M T_\lambda \wedge \varphi = \sum_\alpha \int_{K_\alpha} \int_{\{k\} \times J} F_\alpha^* (\chi_\alpha \varphi) \dif \mu_\alpha(k).$$
The \dfn{homology class} $[\lambda]$ and \dfn{mass} $\Mass(\lambda)$ of a measured oriented lamination $\lambda$ are the homology class and mass of its Ruelle-Sullivan current.
\end{definition}

We are now ready to state \cite[Theorem C]{BackusCML}:

\begin{theorem}\label{1 harmonic is MOML}
Suppose that $2 \leq \dim M \leq 4$.
Then a function $u \in BV_\loc(M)$ is $1$-harmonic iff $\dif u$ is the Ruelle-Sullivan current for a measured oriented minimal lamination.
\end{theorem}



%%%%%%%%%%%%%%%%%%%%%%%%%%%%%%%%%%%%%%%%%%

\section{\texorpdfstring{$d - 1$-forms}{(d - 1)-forms} of finite comass}\label{comass sec}
In this section we establish some tools from geometric measure theory adapted to closed $L^\infty$ $d - 1$-forms.
Throughout, we fix a Riemannian manifold $M$ (possibly not closed) of dimension $d$ and metric $g$.

%%%%%%%%%%%%%%%%%%%
\subsection{Local Hodge theory in \texorpdfstring{$L^p$}{Lp}}
The geometric measure theory that we develop in this section relies on the local elliptic regularity of the Hodge system.
Such estimates are standard when $p = 2$ \cite[\S9.5]{taylor2010partial}, but we are mainly interested in the range $d < p < \infty$, where we may apply the Sobolev embedding $W^{1, p} \Subset C^\alpha$.
We thus give a short proof for every $p$.

\begin{lemma}\label{Hodge theorem}
Suppose that there is a bi-Lipschitz diffeomorphism $M \cong \Ball^d$.
Let $1 < p < \infty$, and let $F$ be an $L^p$ closed $\ell + 1$-form.
Then there exists an $\ell$-form $A$ such that $F = \dif A$ and
\begin{equation}\label{Hodge theorem estimate}
\|A\|_{W^{1, p}} \lesssim_p \|F\|_{L^p}.
\end{equation}
\end{lemma}
\begin{proof}
Without loss of generality, $M = (\Ball^d, g)$ where $g$ is a Riemannian metric of bounded geometry.
Then, possibly after rescaling, the $L^p(M)$ norm is comparable to the $L^p(\Ball^d)$ norm.
Everything else in the statement of the lemma is a diffeomorphism invariant, so we may assume that $M = \Ball^d$.
We then solve the elliptic system 
\begin{equation}\label{Hodge Laplacian}
\begin{cases}
	\Delta u = F \\
	u|_{\partial \Ball^d} = 0
\end{cases}
\end{equation}
for an $\ell$-form $u$.
Since $M = \Ball^d$, the Hodge Laplacian commutes with taking components, thus the system (\ref{Hodge Laplacian}) decouples into $\binom d\ell$ elliptic PDE
\begin{equation}\label{decoupled Hodge Laplacian}
\begin{cases}
\Delta(u_I) = F_I \\
u_I|_{\partial \Ball^d} = 0,
\end{cases}
\end{equation}
one for each $\ell$-index $I$.
By \cite[Chapter 3, Theorem 6.3]{chen1998second}, we have a unique solution to (\ref{decoupled Hodge Laplacian}), which satisfies
$$\|u_I\|_{W^{2, p}} \lesssim_p \|F_I\|_{L^p}.$$
We then let $A := \dif^* u$; then (\ref{Hodge theorem estimate}) holds, and $\dif A = \dif \dif^* u$.
On euclidean space, $\Delta$ and $\dif$ commute, so $\dif u = 0$, and it holds that 
\begin{align*}
\dif A &= (\dif \dif^* + \dif^* \dif) u = \Delta u = F. \qedhere 
\end{align*}
\end{proof}

We also need the following result, which essentially asserts that mollification can be chosen to commute with the de Rham differential, at least locally.

\begin{lemma}\label{mollification of closed forms}
Suppose that there is a bi-Lipschitz diffeomorphism $M \cong \Ball^d$.
Let $1 \leq p < \infty$ and let $F$ be an $L^p$ closed $\ell$-form.
Then there exist smooth closed $\ell$-forms $F_n$ such that $F_n \to F$ in $L^p$.
Moreover, if $F \in L^\infty$, then
\begin{equation}\label{heat kernel contracts sup norm}
\limsup_{n \to \infty} \|F_n\|_{C^0} \leq \|F\|_{L^\infty}.
\end{equation}
\end{lemma}
\begin{proof}
Without loss of generality, $M = (\Ball^d, g)$ where $g$ is a Riemannian metric of bounded geometry.
Then, possibly after rescaling, the $L^p(M)$ norm is comparable to the $L^p(\Ball^d)$ norm.
If $\varphi$ is am $\ell$-form and $\chi$ is a function on $\RR^d$, we define the convolution $\chi * \varphi$ by $(\chi * \varphi)_I := \chi * \varphi_I$ for every $\ell$-index $I$.
Every convolution operator commutes with $\dif$, so after convolution with a standard mollifier $\chi_n$ as in \cite[Appendix C, Theorem 6]{evans2010partial}, $F_n := (\chi_n * F)|_{\Ball^d}$ is a closed $\ell$-form and $F_n \to F$ in $L^p(\Ball^d)$ and almost everywhere.

Now suppose that $F \in L^\infty$.
We may choose $\chi_n$ so that $\chi_n \geq 0$, $\int_{\RR^d} \chi_n = 1$, and $\supp \chi_n \Subset B_{1/n}$, the euclidean ball of radius $1/n$ around $0$.
Then 
$$|F_n(x)|_{g(x)} \leq \left|\int_{B_{1/n}} \chi_n(y) F(x - y) \dif y\right|_{g(x)},$$
where the integral is a vector-valued integral (which makes sense since we may view $F$ as a map into $\RR^{\binom d\ell})$.
By the triangle inequality and a Taylor expansion of $g$ around $g(x)$,
\begin{align*}
\left|\int_{B_{1/n}} \chi_n(y) F(x - y) \dif y\right|_{g(x)}
&\leq \int_{B_{1/n}} \chi_n(y) |F(x - y)|_{g(x)} \dif y \\
&= \int_{B_{1/n}} \chi_n(y) |F(x - y)|_{g(x - y)}(1 + O(y)) \dif y \\
&\leq \|F\|_{L^\infty(M)} \int_{B_{1/n}} \chi_n(y)(1 + O(y)) \dif y \\
&= \|F\|_{L^\infty(M)}(1 + O(n^{-1})). \qedhere 
\end{align*}
\end{proof}

%%%%%%%%%%%%%%%%%%%
\subsection{Trace theorem}
We would like to be able to compute the comass of a form $F$ by integration of $F$ along chains.
Unfortunately, $F$ may only be defined almost everywhere, and then it is not clear that such an integral is well-defined; nor is it clear that the cohomology class of $F$ is well-defined (so that the notion of ``best comass'' makes no sense).
Here we show that for closed forms $F$ in $L^p$ and suitable chains $\sigma$, $\int_\sigma F$ is well-defined.



The natural setting for our trace theorem is the integral currents (studied for example in \cite[\S27]{Simon84}), as we shall want to be able to integrate a closed $L^p$ $d - 1$-form along the level sets of a $BV$ function.

\begin{definition}
An $\ell$-current $\sigma$ is \dfn{rectifiable} if there exists an $\ell$-rectifiable set $N$, and a $\dif S_N$-measurable $\ell$-blade field $v$, such that for any $C^0_\cpt$ $\ell$-form $\varphi$,
$$\int_M \sigma \wedge \varphi = \int_N \langle \varphi, v\rangle \dif S_N,$$
and $|v|$ is $\ZZ$-valued.
The rectifiable current $\sigma$ is \dfn{integral} if, in addition, $\dif \sigma$ is rectifiable.
\end{definition}

To emphasize that integral currents $\sigma$ can be represented by rectifiable sets, we write $\partial \sigma := -\dif \sigma$, and $\int_\sigma \varphi := \int_M \sigma \wedge \varphi$.

\begin{lemma}\label{local trace theorem}
Suppose that there is a bi-Lipschitz diffeomorphism $M \cong \Ball^d$.
Let $\tau$ be an integral $d - 1$-current and $\psi \in C^1(M)$.
Then for any $d < p < \infty$, $F \mapsto \int_\tau \psi F$ is a continuous linear function on the space of $L^p$ closed $d - 1$-forms.
\end{lemma}
\begin{proof}
First suppose that $F, G$ are closed $C^1$ $d - 1$-forms, and let $A, B$ be the $d - 2$-forms obtained from Lemma \ref{Hodge theorem}.
By integration by parts and the Sobolev embedding theorem,
\begin{align*}
	\left|\int_\tau \psi(F - G)\right| 
	&\leq \left|\int_{\partial \tau} \psi (A - B)\right| + \left|\int_\tau \dif \psi \wedge (A - B)\right| \\
	&\lesssim_{\tau, \psi} \|A - B\|_{C^0} \lesssim_p \|A - B\|_{L^p} \lesssim_p \|F - G\|_{L^p}.
\end{align*}
If $F, G$ are not necessarily $C^1$, then by Lemma \ref{mollification of closed forms}, the $C^1$ closed $d - 1$-forms are dense in the space of $L^p$ closed $d - 1$-forms, so the desired estimate holds by approximating $F, G$ by $C^1$ closed $d - 1$-forms.
\end{proof}

\begin{proposition}[trace theorem]\label{integration is welldefined}
Let $\tau$ be a compactly supported integral $d-1$-current.
Then:
\begin{enumerate}
\item For any $d < p \leq \infty$ and $\psi \in C^1(M)$, $F \mapsto \int_\tau \psi F$ extends to a continuous linear functional on the space of $L^p$ closed $d-1$-forms.
\item For any $d < p \leq \infty$, the cohomology class of any $L^p$ closed $d - 1$-form is well-defined.
\item Let $F$ be an $L^\infty$ closed $d - 1$-form. Then for every $\psi \in C^0(M)$,
\begin{equation}\label{integral over chain is linfinity}
	\int_\tau \psi F \leq \|F\|_{L^\infty} \|\psi\|_{L^1(\tau)}.
\end{equation}
\end{enumerate}
\end{proposition}
\begin{proof}
We can find a partition of unity $(\chi_\alpha)$ subordinate to an open cover $(U_\alpha)$ such that $U_\alpha$ is bi-Lipschitz diffeomorphic to $\Ball^d$.
Then we may replace $(U_\alpha)$ by a finite subcover $U_1, \dots, U_n$ of a neighborhood of $\supp \tau$.
Applying Lemma \ref{local trace theorem} with $\psi$ replaced by $\psi \chi_k$, we obtain for $F, G$ closed $L^p$ $d - 1$-forms and $p < \infty$,
$$\left|\int_\tau \psi(F - G)\right| \leq \sum_{k = 1}^n \left|\int_\tau \psi \chi_k (F - G)\right| \lesssim_\tau \|F - G\|_{L^p}$$
which gives the continuity in $L^p$.
This also implies that the cohomology is well-defined.

We now handle the case $p = \infty$.
Let $d < q < \infty$ and let $U$ be a small neighborhood of $\supp \tau$; since $\supp \tau$ is compact, if $F, G \in L^\infty$, then $F, G \in L^q(U)$ and 
$$\left|\int_\tau \psi(F - G)\right| \lesssim_\tau \|F - g\|_{L^q(U)} \lesssim_{U, q} \|F - G\|_{L^\infty}.$$
This gives the continuity in $L^\infty$.
We then use Lemma \ref{mollification of closed forms} to find smooth closed $d - 1$-forms $F_{k, m}$ on $U_k$ such that $F_{k, m} \to F|_{U_k}$ in $L^q(U_k)$ and
$$\|F_{k, m}\|_{C^0} \leq \|F\|_{L^\infty}.$$
So by the triangle inequality, if $\psi \in C^1(M)$,
\begin{align*}
\int_\tau \psi F 
&= \sum_{k = 1}^n \int_\tau \psi \chi_k F 
= \sum_{k = 1}^n \lim_{m \to \infty} \int_\tau \psi \chi_k F_{k, m} \\
&\leq \lim_{m \to \infty} \|F_{k, m}\|_{C^0} \int_\tau \psi \sum_{k = 1}^n \chi_k \dif S_\tau 
\leq \|F\|_{L^\infty} \int_\tau \psi \dif S_\tau.
\end{align*}
By approximating a $C^0$ test function $\psi$ by $C^1$ functions, we obtain the result for $\psi \in C^0$.
\end{proof}

%%%%%%%%%%%%%%%%%%%%%%%%%
\subsection{Coarea formula}
We next seek to make sense of the Radon measure (or $0$-current of locally finite mass) $\dif u \wedge F$ when $u \in BV_\loc(M)$ and $F$ is a closed $L^\infty$ $d - 1$-form.
A priori the expression $\dif u \wedge F$ is nonsensical, as $\dif u$ may fail to be absolutely continuous, while $F$ need not be defined on a Lebesgue null set.


\begin{lemma}\label{reduced level sets are integral currents}
Let $u \in BV(M)$.
Then for almost every $\lambda \in \RR$, integration along the reduced level set $\partial^* \{u > \lambda\}$ defines an integral $d - 1$-current.
\end{lemma}
\begin{proof}
By Proposition \ref{coarea}, for almost every $\lambda \in \RR$, $\{u > \lambda\}$ has locally finite perimeter.
So by the de Giorgi structure theorem \cite[Theorem 14.3]{Simon84}, $\partial^* \{u > \lambda\}$ is an oriented rectifiable $d - 1$-cycle.
So integration along $\partial^* \{u > \lambda\}$ defines a closed rectifiable, and hence integral, $d - 1$-current.
\end{proof}

\begin{proposition}[coarea formula]
Let $u \in BV(M)$, and for each $\lambda \in \RR$, let $\tau_\lambda$ be the reduced level set of $u$.
Then the product $\dif u \wedge F$ with a closed $L^\infty$ $d - 1$-form $F$ can be defined in one and only one way so that for every sequence $(F_n)$ which is bounded in $L^\infty$ and converges in $L^p_\loc$ for some $d < p < \infty$ to a closed $d - 1$-form $F$, $\dif u \wedge F_n \to \dif u \wedge F$ in the weak topology of measures.

Moreover, for every $\chi \in C^0_\cpt(M)$, we have the coarea formula
\begin{equation}\label{coarea formula}
\int_M \chi \dif u \wedge F = \int_{-\infty}^\infty \int_{\tau_\lambda} \chi F \dif \lambda.
\end{equation}
\end{proposition}
\begin{proof}
Motivated by Proposition \ref{coarea}, it is natural to \emph{define} $\dif u \wedge F$ to be the Radon measure satisfying (\ref{coarea formula}).
If $F$ is continuous, then this will agree with the usual definition of $\dif u \wedge F$ as a product of a Radon measure and a function.
Even if $F$ is discontinuous, (\ref{coarea formula}) defines a Radon measure.
Indeed, let $U \Subset M$ be an open set containing $\supp \chi$.
Then, by the trace theorem, Lemma \ref{reduced level sets are integral currents}, and Proposition \ref{coarea},
$$\left|\int_M \chi \dif u \wedge F\right| \leq \|\chi\|_{C^0} \|F\|_{L^\infty} \int_{-\infty}^\infty \Mass_U(\tau_\lambda) \dif \lambda = \|\chi\|_{C^0} \|F\|_{L^\infty} \Mass_U(\dif u).$$
Therefore $\dif u \wedge F$ is a Radon measure, since it has locally finite mass
$$\Mass_U(\dif u \wedge F) \leq \|F\|_{L^\infty} \Mass_U(\dif u) < \infty.$$

Now we check the convergence on $L^\infty$-bounded sequences in $L^p_\loc$.
Let $\chi \in C^0_\cpt(M)$, let $U \Subset M$ be an open set containing $\supp \chi$, and let $(F_n)$ be an $L^\infty$-bounded sequence converging in $L^p_\loc$ to a closed form $F$.
We introduce the functions
$$f_n(\lambda) := \int_{\tau_\lambda} \chi F_n, \qquad f(\lambda) := \int_{\tau_\lambda} \chi F,$$
which are well-defined by the trace theorem and Lemma \ref{reduced level sets are integral currents}.
Also consider the function $g(\lambda) := \Mass(\tau_\lambda)$, so that for almost every $\lambda \in \RR$, 
$$|f_n(\lambda)| \leq \|F_n\|_{C^0} \|\chi\|_{C^0} g(\lambda) \lesssim \|\chi\|_{C^0} g(\lambda).$$
By Proposition \ref{coarea} and the fact that $u \in BV$, $g \in L^1(\RR)$.
On the other hand, since $F_n \to F$ in $L^p(U)$ and $p > d$, $f_n \to f$ almost everywhere.
So by dominated convergence, $f_n \to f$ in $L^1(\RR)$.
In particular, by Proposition \ref{coarea},
\begin{align*}
\lim_{n \to \infty} \int_M \chi \dif u \wedge F_n
&= \lim_{n \to \infty} \int_{-\infty}^\infty f_n(\lambda) \dif \lambda
= \int_{-\infty}^\infty f(\lambda) \dif \lambda \\
&= \int_M \chi \dif u \wedge F. \qedhere 
\end{align*}
\end{proof}


%%%%%%%%%%%%%%%%%%%%%%%
\subsection{Local comass}
We will be interested in the points at which $F$ attains its comass.
One could pose this problem as the problem of computing the locus $\{|F| = \|F\|_{L^\infty}\}$.
However, $|F(x)|$, the comass of the $d - 1$-covector $F(x)$, is both only defined for almost every $x$, and not norm-approximable by smooth functions.
So as a proxy for $|F|$, which may fail to be defined on a null set, we use the local comass, which is defined everywhere.

\begin{definition}
For an open set $\Omega \subseteq M$ and a a closed $d - 1$-form $F$, define 
$$\Comass_\Omega(F) := \sup_{\sigma \in \Chain_{d - 1}(\Omega)} \frac{1}{\Mass(\sigma)} \int_\sigma F.$$
The \dfn{local comass} of a closed $d - 1$-form $F$ at $x \in M$ is 
$$\Comass(F, x) = \limsup_{\varepsilon \to 0} \Comass_{B_\varepsilon(x)}(F).$$
\end{definition}

By the trace theorem, $\Comass_\Omega(F)$ is well-defined (but possibly $+\infty$).
Since $\Comass_{B_\varepsilon(x)}(F)$ is a supremum over a set which grows in $\varepsilon$, it is increasing in $\varepsilon$, so the limit superior is actually a limit and an infimum:
$$\Comass(F, x) = \lim_{\varepsilon \to 0} \Comass_{B_\varepsilon(x)}(F) = \inf_{\varepsilon > 0} \Comass_{B_\varepsilon(x)}(F).$$
In particular, if we write $\Comass(F) := \Comass_M(F)$, then $\Comass(F, x) \leq \Comass(F)$.

The local comass was defined in an analogous manner to the local Lipschitz constant.
As such, it enjoys many of the same properties, including those endowed on the local Lipschitz constant by \cite[Lemma 4.3]{Crandall2008}:

\begin{proposition}\label{crandall}
Let $F$ be a closed $L^\infty$ $d - 1$-form. Then:
\begin{enumerate}
\item The local comass $\Comass(F, \cdot)$ is upper semicontinuous. \label{crandall usc}
\item For almost every $x \in M$, \label{crandall LDT}
$$|F(x)| \leq \Comass(F, x).$$
\item The local comass is bounded, and \label{crandall linfinity}
$$\Comass(F) = \sup_{x \in M} \Comass(F, x) = \|F\|_{L^\infty}.$$
\item If $\sigma \in \Chain_{d - 1}(M)$ then \label{crandall best curl is ABC}
$$\int_\sigma F \leq \Mass(\sigma) \sup_{x \in \sigma} \Comass(F, x).$$
\end{enumerate}
\end{proposition}
\begin{proof}
We first prove (\ref{crandall usc}).
Let $x^n \to x$ and $r > 0$. Then eventually $x^n \in B_r(x)$, hence $\Comass(F, x^n) \leq \Comass_{B_r(x)}(F)$ and so
\begin{align*}
\limsup_{n \to \infty} \Comass(F, x^n) &\leq \inf_{r > 0} \Comass_{B_r(x)}(F) = \Comass(F, x).
\end{align*}

We now prove (\ref{crandall LDT}).
We may work locally, and choose coordinates $(y^i)$ in which $\sqrt{\det g} = 1$.
Let $I$ be the increasing $d-1$-index with $d$ removed.
By the Lebesgue differentiation theorem and Fubini's theorem, there exists a null set $Z \subset M$, which does not depend on $(y^i)$ by Proposition \ref{invariance of LDT}, such that for every $x \notin Z$,
\begin{align*}
F_I(x) 
&= \lim_{\varepsilon \to 0} \frac{1}{\Mass(B_\varepsilon(x))} \int_{B_\varepsilon(x)} F_I(y) \dif y \\
&= \lim_{\varepsilon \to 0} \frac{1}{\Mass(B_\varepsilon(x))} \int_{-\infty}^\infty \int_{\{y^d = t\} \cap B_\varepsilon(x)} F_I(y) \dif y^1 \wedge \cdots \wedge \dif y^{d - 1} \wedge \dif t
\end{align*}
where we used the fact that $\sqrt{\det g} = 1$.
Now $\partial_{y^1} \wedge \cdots \wedge \partial_{y^{d - 1}}$ is the oriented unit $d - 1$-blade tangent to $\{y^d = t\}$, so as forms on $\{y^d = t\}$,
$$F_I(y) \dif y^1 \wedge \cdots \wedge \dif y^{d - 1} = F.$$
So
\begin{align*}
F_I(x) 
&= \lim_{\varepsilon \to 0} \frac{1}{\Mass(B_\varepsilon(x))} \int_{-\infty}^\infty \int_{\{y^d = t\} \cap B_\varepsilon(x)} F \dif t \\
&\leq \lim_{\varepsilon \to 0} \frac{\Comass_{B_\varepsilon(x)}(F)}{\Mass(B_\varepsilon(x))} \int_{-\infty}^\infty |\{y^d = t\} \cap B_\varepsilon(x)| \dif t.
\end{align*}
By Fubini's theorem,
$$F_I(x) \leq \lim_{\varepsilon \to 0} \frac{\Comass_{B_\varepsilon(x)}(F)}{\Mass(B_\varepsilon(x))} \Mass(B_\varepsilon(x)) = \Comass(F, x).$$
For every $x \in M$ we may select coordinates in which $|F(x)| = F_I(x)$, and then if $x \notin Z$, we conclude that (\ref{crandall LDT}) holds for $x$.

If we combine (\ref{crandall LDT}) with (\ref{integral over chain is linfinity}), then
$$\sup_{x \in M} \Comass(F, x) \leq \Comass(F) \leq \|F\|_{L^\infty} \leq \sup_{x \in M} \Comass(F, x).$$
The inequalities collapse, proving (\ref{crandall linfinity}).
In particular, for each $\sigma \in \Chain_{d - 1}(M)$, we obtain (\ref{crandall best curl is ABC}):
\begin{align*}
\int_\sigma F &\leq \Mass(\sigma) \inf_{\Omega \supset \sigma} \sup_{x \in \Omega} \Comass(F, x) = \Mass(\sigma) \sup_{x \in \sigma} \Comass(F, x). \qedhere
\end{align*}
\end{proof}

\begin{definition}
Let $F$ be a form of best comass.
The \dfn{maximum comass locus} is the set
$$\MCL(F) := \{L(F, \cdot) = L(F)\}.$$
\end{definition}

By Proposition \ref{crandall}(\ref{crandall usc}), if $M$ is a closed manifold, then $\MCL(F)$ is a nonempty closed subset of $M$.

%%%%%%%%%%%%%%%%%%%%%%%
\subsection{\texorpdfstring{$L^\infty$}{L-infinity} calibrations}
The purpose of this paper is to study calibrated laminations.
If the calibration $F$ is continuous, then the theory of \cite{bangert_cui_2017} of calibrated laminations applies; however, we are not aware of general existence results on continuous calibrations, only measurable calibrations \cite{Federer1974}.
Thus, we now introduce $L^\infty$ calibrations of laminations.

\begin{definition}
A \dfn{calibration} is a closed $d - 1$-form $F$ such that $\Comass(F) = 1$.
An integral $d - 1$-current $\sigma$ is $F$-\dfn{calibrated} if 
$$\Mass(\sigma) = \int_\sigma F.$$
A lamination $\lambda$ is \dfn{$F$-calibrated} if every leaf of $\lambda$ is $F$-calibrated.
\end{definition}

The fundamental theorem of calibrated geometry \cite{Harvey82} asserts that a calibrated integral current is minimal, so every calibrated lamination is minimal.
Moreover, if $M$ is closed, $\lambda$ is a lamination, and $F$ is a calibration, then the quantity $\int_M T_\lambda \wedge F$ is well-defined, since it is just $\langle [F], [\lambda]\rangle$.

\begin{proposition}\label{calibration condition}
Let $F$ be a calibration on a closed Riemannian manifold $M$.
Let $T_\lambda$ be the Ruelle-Sullivan current of a measured oriented lamination $\lambda$, and suppose that 
\begin{equation}\label{calibration by Ruelle Sullivan}
\int_M T_\lambda \wedge F = \Mass(\lambda).
\end{equation}
Then $\lambda$ is $F$-calibrated (and in particular minimal).
\end{proposition}
\begin{proof}
Let $(\chi_\alpha)$ be a locally finite partition of unity subordinate to an open cover $(U_\alpha)$ of flow boxes for $\lambda$, and let $(\mu_\alpha)$ be the transverse measure.
After refining $(U_\alpha)$ we may assume that $U_\alpha$ is contained in an open set which is bi-Lipschitz diffeomorphic to $\Ball^d$. After shrinking $U_\alpha$ we may assume that $\chi_\alpha > 0$ on $U_\alpha$.
Then for some hypersurfaces $\sigma_{\alpha,k}$,
$$\Mass(\lambda) = \int_M T_\lambda \wedge F = \sum_\alpha \int_I \int_{\sigma_{\alpha,k}} \chi_\alpha F \dif \mu_\alpha(k).$$
Let $\dif S_{\alpha,k}$ be the surface measure on $\sigma_{\alpha,k}$. Then
$$\int_M \chi_\alpha \star |T_\lambda| = \int_I \int_{\sigma_{\alpha,k}} \chi_\alpha \dif S_{\alpha,k} \dif \mu_\alpha(k),$$
so summing in $\alpha$, we obtain 
\begin{equation}\label{calibration condition contr}
\sum_\alpha \int_I \int_{\sigma_{\alpha,k}} \chi_\alpha F \dif \mu_\alpha(k) = \Mass(\lambda) = \sum_\alpha \int_I \int_{\sigma_{\alpha,k}} \chi_\alpha \dif S_{\alpha,k} \dif \mu_\alpha(k).
\end{equation}

We claim that $\lambda$ is \dfn{almost calibrated} in the sense that for every $\alpha$ and $\mu_\alpha$-almost every $k$, $\sigma_{\alpha, k}$ is calibrated.
If this is not true, then we may select $\beta$ and $K \subseteq I$ with $\mu_\beta(K) > 0$, such that for every $k \in K$, $\int_{\sigma_{\beta, k}} F < \Mass(\sigma_{\beta, k})$.
Since $0 < \chi_\beta \leq 1$ and $F/\dif S_{\beta, k} \leq 1$ on $\sigma_{\beta, k}$, this is only possible if 
$$\int_{\sigma_{\beta, k}} \chi_\beta F < \int_{\sigma_{\beta, k}} \chi_\beta \dif S_{\beta, k}.$$
Integrating over $K$, and using the fact that in general we have $\int_{\sigma_{\alpha, k}} \chi_\alpha F \leq \int_{\sigma_{\alpha, k}} \chi_\alpha \dif S_{\alpha, k}$, we conclude that 
$$\sum_\alpha \int_I \int_{\sigma_{\alpha, k}} \chi_\alpha F \dif \mu_\alpha(k) < \sum_\alpha \int_I \int_{\sigma_{\alpha, k}} \chi_\alpha \dif S_{\alpha, k} \dif \mu_\alpha(k)$$
which contradicts (\ref{calibration condition contr}).

To upgrade $\lambda$ from an almost calibrated lamination to a calibrated lamination, we first 
given $\sigma_{\alpha, k}$ we choose $k_j$ such that $\sigma_{\alpha, k_j}$ is calibrated and $k_j \to k$.
By Lemma \ref{Hodge theorem}, we can find a $d - 2$-form $A$ with $F = \dif A$, which is necessarily continuous by the Sobolev embedding theorem.
This justifies the following application of Stokes' theorem: 
$$\int_{\sigma_{\alpha, k}} F = \int_{\partial \sigma_{\alpha, k}} A.$$
Since $k_j \to k$, and $A$ is continuous,
\begin{align*}
\Mass(\sigma_{\alpha, k}) &= \lim_{j \to \infty} \Mass(\sigma_{\alpha, k_j}) = \lim_{j \to \infty} \int_{\sigma_{\alpha, k_j}} F = \lim_{j \to \infty} \int_{\partial \sigma_{\alpha, k_j}} A = \int_{\partial \sigma_{\alpha, k}} A = \int_{\sigma_{\alpha, k}} F. \qedhere 
\end{align*}
\end{proof}

\begin{proposition}\label{properties of calibrated laminations}
Suppose that $M$ is a closed Riemannian manifold, $F$ is a calibration, and $\lambda$ is a measured oriented $F$-calibrated lamination.
Then:
\begin{enumerate}
\item $\lambda$ is minimal.
\item If $G$ is a calibration and cohomologous to $F$, then $\lambda$ is $G$-calibrated.
\item $\supp \lambda \subseteq \MCL(F)$.
\end{enumerate}
\end{proposition}
\begin{proof}
Every leaf of $\lambda$ is $F$-calibrated, hence minimal, so $\lambda$ is also minimal.
Moreover, (\ref{calibration by Ruelle Sullivan}) only depends on the cohomology class of $F$, not $F$ itself, so $\lambda$ is $G$-calibrated.
Finally, let $S := \MCL(F)$, $N$ a leaf of $\lambda$, and suppose that $x \in N \setminus S$.
Since $S$ is closed, there exists $\varepsilon > 0$ such that $B_\varepsilon(x)$ does not meet $S$.
Moreover, $\sigma := N \cap B_\varepsilon(x)$ is a $d-1$-chain in $B_\varepsilon(x)$, so by Proposition \ref{crandall}(\ref{crandall best curl is ABC}),
$$\frac{1}{\Mass(\sigma)} \int_\sigma F \leq \sup_{y \in B_\varepsilon(x)} \Comass(F, y) < \Comass(F) = 1.$$
But then 
$$\int_N F = \int_\sigma F + \int_{N \setminus B_\varepsilon(x)} F < \Mass(\sigma) + \Mass(N \setminus B_\varepsilon(x)) = \Mass(N),$$
so $N$ (hence $\lambda$) is not $F$-calibrated.
\end{proof}


%%%%%%%%%%%%%%%%%%%%%%%%%%%%%
\section{\texorpdfstring{$\infty$-tight forms and the $1$-Laplacian}{Infinity-tight forms and the one-Laplacian}}\label{tight forms sec}
\subsection{Convex optimization}
We follow \cite{Ekeland99}.
For a reflexive Banach space $X$, we denote by $\hat X$ its dual.
If $I: X \to \RR \cup \{+\infty\}$ is a convex function, we introduce its \dfn{Legendre transform}, the convex function
\begin{align*}
	\hat I: \hat X &\to \RR \cup \{+\infty\}\\
	\xi &\mapsto \sup_{x \in X} \langle \xi, x\rangle - I(x).
\end{align*}
We identify the cokernel of a linear map $\Lambda$ with the kernel of its adjoint.
In this setting, we have the following form of the convex duality theorem.

\begin{theorem}[convex duality]\label{abstract convex analysis}
Let 
$$\Lambda : X \to Y$$
be a bounded linear map between reflexive Banach spaces.
Let $I: Y \to \RR \cup \{+\infty\}$ satisfy:
\begin{enumerate}
\item $I$ and $\hat I$ are strictly convex,
\item $I$ is lower semicontinuous,
\item if $|y| \to \infty$ in $Y$, then $I(y) \to +\infty$, and 
\item there exists a point $x \in X$ such that $I$ is continuous and finite at $\Lambda(x)$.
\end{enumerate}
Then:
\begin{enumerate}
\item There exists a minimizer $\underline x \in X$ of $I(\Lambda(x))$, unique modulo $\ker \Lambda$.
\item There exists a unique maximizer $\overline \eta$ of $-\hat I(-\eta)$ subject to the constraint $\eta \in \coker \Lambda$.
\item We have \dfn{strong duality}
\begin{equation}\label{abstract strong duality}
I(\Lambda(\underline x)) = -\hat I(-\overline \eta).
\end{equation}
\end{enumerate}
\end{theorem}
\begin{proof}
This is largely a special case of \cite[Chapter IV, Theorem 4.2]{Ekeland99}.
Let $\mathscr P, \mathscr P^*$ be as in the statement of that theorem.
Then $\mathscr P$ is the problem of minimizing $J(x, \Lambda x)$ where $J(x, y) := I(y)$.
The Legendre transform of $J$ satisfies 
$$\hat J(\xi, \eta) = \begin{cases} \hat I(\eta), & \xi = 0, \\
	+\infty, &\xi \neq 0,
\end{cases}$$
and $\mathscr P^*$ is the problem of maximizing
$$-\hat J(\Lambda^* \eta, -\eta) = \begin{cases}
	-\hat I(-\eta), &\eta \in \ker \Lambda^*, \\
	-\infty, &\eta \notin \ker \Lambda^*,
\end{cases}$$
where $\Lambda^*$ is the adjoint of $\Lambda$.
Then most of the various assertions of this theorem follow immediately from \cite[Chapter IV, Theorem 4.2]{Ekeland99}.
The fact that $\overline \eta \in \coker \Lambda$ follows from the facts that $\overline \eta$ is a solution of $\mathscr P^*$, but any solution of $\mathscr P^*$ must be a member of $\ker \Lambda^*$. 
To establish uniqueness, we use \cite[Chapter II, Proposition 1.2]{Ekeland99}, the fact that $\hat I$ is strictly convex, and the fact that we may view $I \circ \Lambda$ as a strictly convex function on the reflexive Banach space $X/\ker \Lambda$.
\end{proof}

%%%%%%%%%%%%%%%

\subsection{Max flow min cut for the \texorpdfstring{$q$-Laplacian}{q-Laplacian}}
We now turn to the problem at hand.
Let $M$ be a closed oriented Riemannian manifold with fundamental group $\Gamma$ and universal abelian covering $\tilde M \to M$, and let $M_{\rm fun} \subset \tilde M$ be a fundamental domain of $\Gamma$.
By Poincar\'e duality and the Hurcewiz theorem, we have canonical isomorphisms
\begin{equation}\label{Poincare Hurcewiz}
H_{d - 1}(M, \RR) = H^1(M, \RR) = \Hom(\Gamma, \RR).
\end{equation}
Sometimes we are interested specifically in integral representations; in that case, we can use the fact that $\Sph^1$ is homotopic to $K(\ZZ, 1)$ to further identify 
\begin{equation}\label{Poincare Hurcewiz 2}
H^1(M, \ZZ) = \Hom(\Gamma, \ZZ) = [M, \Sph^1].
\end{equation}

Given a representation
$$\alpha: \Gamma \to \RR,$$
which we always identify with some smooth $1$-form (which we also call $\alpha$) representing the cohomology class corresponding to the representation $\alpha$, and $q \in (1, \infty)$,
we are interested in the $q$-Laplace equation
$$\dif^* (|\dif u|^{q - 2} \dif u) = 0$$
for an $\alpha$-equivariant function $u$ (thus for any $\gamma \in \Gamma$, $u(\gamma(x)) - u(x) = \alpha(\gamma)$).

Let $X$ be the space of locally $W^{1, q}(\tilde M)$ functions $u$ which are $\Gamma$-equivariant in the sense that for $\gamma \in \Gamma$, $\gamma^* \dif u = \dif u$, let $Y := L^q(M, \Omega^1)$, and let $p$ be the H\"older conjugate exponent to $q$.
Then $\Lambda := (\dif: X \to Y)$ is a bounded linear map, whose cokernel can be identified with the kernel of
$$\dif: L^p(M, \Omega^{d - 1}) \to W^{-1, p}(M)$$
using the perfect pairing 
\begin{align*}
	L^p(M, \Omega^{d - 1}) \times Y &\to \RR \\
	(F, \varphi) &\mapsto \int_M \varphi \wedge F 
\end{align*}
to identify $\hat Y$ with $L^p(M, \Omega^{d - 1})$.

One can easily show that the $q$-Laplace equation is the Euler-Lagrange equation of $I \circ \dif$, where $I: Y \to \RR \cup \{+\infty\}$ is defined to be 
$$I(\varphi) := \frac{1}{q} \int_M \star |\varphi|^q$$
if $\varphi$ is cohomologous to $\alpha$, and otherwise 
$$I(\varphi) := +\infty.$$
Using the methods of \cite[Chapter I, \S4]{Ekeland99} one can show that
$$\hat I(F) = \frac{1}{p} \int_M \star |F|^p + \int_M \alpha \wedge F$$
defined for $F \in \coker \Lambda$, and the dual problem to the $q$-Laplacian is the problem of maximizing $-\hat I(-F)$.

\begin{proposition}[max flow min cut for the $q$-Laplacian]
Given a representation $\alpha: \Gamma \to \RR$, and H\"older conjugate exponents $1 < p, q < \infty$, there exists an $\alpha$-equivariant $q$-harmonic function $u: \tilde M \to \RR$, unique modulo constants, and a unique minimizer $F$ of 
$$J_{p, \alpha}(F) := \frac{1}{p} \int_M \star |F|^p - \int_M \alpha \wedge F$$
among all closed $d - 1$-forms on $M$.
Moreover, we have
\begin{equation}\label{strong duality}
	\frac{1}{q} \int_M \star |\dif u|^q + \frac{1}{p} \int_M \star |F|^p = \int_M \dif u \wedge F.
\end{equation}
\end{proposition}
\begin{proof}
By \cite[Lemma 1]{Loisel_2020}, $I$ and $\hat I$ are both strictly convex.
Moreover, $\coker \Lambda$ is the space of closed $L^p$ $d - 1$-forms on $M$, and $\ker \Lambda$ is the space of constant functions on $\tilde M$.
So the various assertions of this proposition mostly follow from Theorem \ref{abstract convex analysis} and the above discussion.
\end{proof}

Motivated by \cite[\S3.1]{daskalopoulos2020transverse}, it is natural to guess that 
\begin{equation}\label{dual solution}
F := |\dif u|^{q - 2} \star \dif u
\end{equation}
is the solution of the dual problem of minimizing $J_{p, \alpha}$.
In order to prove that this is true, we shall need that for H\"older conjugates $1 \leq p, q \leq \infty$,
\begin{equation}\label{holder cancellation}
	(p - 2)(q - 1) + (q - 2) = 0.
\end{equation}

\begin{lemma}
Suppose that $u: \tilde M \to \RR$ is an $\alpha$-equivariant $q$-harmonic function, and suppose that $F$ satisfies (\ref{dual solution}).
Then $F$ is a closed $d - 1$-form, which minimizes $J_{p, \alpha}$ among all closed $d - 1$-forms.
Moreover, $F$ solves the PDE 
\begin{equation}\label{pMaxwell}
\begin{cases}
	\dif F = 0 \\
	\dif^* (|F|^{p - 2} F) = 0.
\end{cases}
\end{equation}
\end{lemma}
\begin{proof}
We first show that $\dif F = 0$.
In fact, 
$$\star \dif F = \star \dif(|\dif u|^{q - 2} \star \dif u) = \pm \dif^*(|\dif u|^{q - 2} \dif u) = 0.$$
So by the max flow min cut principle, to prove that $F$ is a minimizer, it suffices to show that (\ref{strong duality}) holds.
One can easily compute 
$$|F|^p = |\dif u|^{(q - 1)p} = |\dif u|^q,$$
so by Stokes' theorem and the fact that $\alpha$ is cohomologous to $\dif u$,
\begin{align*}
\frac{1}{q} \int_M \star |\dif u|^q + \frac{1}{p} \int_M \star |F|^p&
= \left[\frac{1}{p} + \frac{1}{q}\right] \int_M \star |\dif u|^q
= \int_M \dif u \wedge |\dif u|^{q - 2} \star \dif u \\
&= \int_M \alpha \wedge F.
\end{align*}
Finally, we use (\ref{holder cancellation}) to prove
\begin{align*}
\dif^*(|F|^{p - 2} F) &= \dif^*(|\dif u|^{(p - 2)(q - 1)} |\dif u|^{q - 2} \star \dif u) = \dif^*(\star \dif u) \\
&= \pm \star \dif^2 u = 0. \qedhere 
\end{align*}
\end{proof}

%%%%%%%%%%%%%%%%

\subsection{\texorpdfstring{$p$-tight forms}{p-tight forms}}
We next scrutinize the PDE (\ref{pMaxwell}).
If $p = 2$ and $d = 3$, then these equations are exactly the Maxwell equations in vacuum for a static magnetic strength tensor, and the functional $J_2$ below can be interpreted as the magnetic potential energy.
On the other hand, as $p \to \infty$, the solutions of these equations converge to a minimizer of the comass; if $d = 2$, a minimizer of the comass is sometimes called \dfn{tight} \cite{farre23}.
This motivates the below terminology:

\begin{definition}
Let $1 < p < \infty$.
We call the equation (\ref{pMaxwell}) the \dfn{$p$-Maxwell equation}.
A \dfn{$p$-tight form} is a solution of the $p$-Maxwell equation.
\end{definition}

\begin{proposition}
Suppose that $M$ is a closed oriented Riemannian manifold.
Then there is a unique $p$-tight form in each cohomology class in $H^{d - 1}(M, \RR)$.
Moreover, $p$-tight forms are minimizers of the strictly convex functional
$$J_p(F) := \frac{1}{p} \int_M \star |F|^p$$
among all forms cohomologous to them.
\end{proposition}
\begin{proof}
Strict convexity of $J_p$ on a cohomology class follows from an argument similar to \cite[Lemma 1]{Loisel_2020} and the existence and uniqueness of a minimizer then is a consequence of the direct method of the calculus of variations as in \cite[Chapter II]{Ekeland99}.
To compute the Euler-Lagrange equations for $J_p$, let $B$ be a $d-2$-form (so $F + t \dif B$ is cohomologous to $F$), so that for a minimizer $F$ of $J_p$,
$$\frac{\dif}{\dif t} J_p(F + t \dif B) = \frac{1}{p} \int_M \star \frac{\partial}{\partial t} |F + t \dif B|^p = \int_M \star |F + t \dif B|^{p - 2} \langle F + t \dif B, \dif B\rangle.$$
Setting $t = 0$, we obtain 
$$0 = \int_M \star |F|^{p - 2} \langle F, \dif B\rangle = \int_M \star \langle \dif^*(|F|^{p - 2} F), B\rangle.$$
This equation holds for every $d - 2$-form $B$.
Thus the Euler-Lagrange equations for $J_p$ are (\ref{pMaxwell}).
\end{proof}

\begin{definition}
Let $F$ be a $p$-tight form, let
\begin{equation}
\dif u := (-1)^{d - 1} |F|^{p - 2} \star F, \label{inverse extremality}
\end{equation}
and let $u$ be the primitive of $\dif u$ on the universal abelian cover $\tilde M$, which is normalized to have zero mean on a fundamental domain $M_{\rm fun}$.
Then $u$ is called the \dfn{$q$-harmonic conjugate} of the $p$-tight form $F$, where $\frac{1}{p} + \frac{1}{q} = 1$.
\end{definition}

Let $u$ be the $q$-harmonic conjugate of $F$.
By Poincar\'e's inequality,
$$\|u\|_{W^{1, q}(M_{\rm fun})}^q \lesssim \int_M \star |\dif u|^q = \int_M \star |F|^{(p - 1)q} = \int_M \star |F|^p < \infty$$
since $F$ is $p$-tight; that is, we have $F \in L^p$ and $u \in W^{1, q}$.

We derived the $p$-Maxwell equation as the dual equation to the $q$-Laplace equation.
We now assert that this process can be inverted.

\begin{lemma}
Let $1 < p, q < \infty$ and $\frac{1}{p} + \frac{1}{q} = 1$.
Let $F$ be a $p$-tight form, and let $u$ be its $q$-harmonic conjugate. Then:
\begin{enumerate}
\item $u$ is $q$-harmonic.
\item One has 
\begin{equation}
F = |\dif u|^{q - 2} \star \dif u. \label{extremality} \\
\end{equation}
\item One has the duality formula (\ref{strong duality}).
\end{enumerate}
\end{lemma}
\begin{proof}
We first use (\ref{holder cancellation}) to prove
$$|\dif u|^{q - 2} \star \dif u = (-1)^{d - 1} |F|^{(q - 2)(p - 1)} \star \star |F|^{p - 2} F = |F|^{(q - 2)(p - 1) - (p - 2)} F = F.$$
Thus we have (\ref{extremality}), and moreover
$$\dif \star (|\dif u|^{q - 2} \dif u) = \dif F = 0$$
so that $u$ is $q$-harmonic.
We then obtain (\ref{strong duality}) as before.
\end{proof}

\begin{corollary}
Every $p$-tight form is locally H\"older continuous.
\end{corollary}
\begin{proof}
Let $F$ be $p$-tight and let $u$ be its $q$-harmonic conjugate.
By \cite[Theorem 2]{DIBENEDETTO1983827}, $\dif u$ is H\"older continuous.
The claim now follows from (\ref{extremality}) and the fact that a product of H\"older continuous functions is H\"older continuous.
\end{proof}


%%%%%%%%%%%%%%%%%%%%%%%
\subsection{\texorpdfstring{Existence of $\infty$-tight forms}{Existence of infinity-tight forms}}
We now take the limit $p \to \infty$ to obtain a privileged form of best comass.
To do so, we shall need the $p$-tight forms to be uniformly bounded in the following sense.

\begin{lemma}
Let $F_p$ be a $p$-tight form, and let $B$ range over closed $d - 1$-forms cohomologous to $F_p$. Then
\begin{equation}\label{infinity magnetic rules p magnetic}
	\|F_p\|_{L^p} \leq |M|^{1/p} \inf_B \|B\|_{L^\infty}.
\end{equation}
\end{lemma}
\begin{proof}
By H\"older's inequality and the fact that $F_p$ is $p$-tight,
$$\|F_p\|_{L^p} \leq \|B\|_{L^p} \leq |M|^{1/p} \|B\|_{L^\infty},$$
hence the same holds for the infimum.
\end{proof}

\begin{proposition}\label{existence infinity}
Let $\rho \in H^{d - 1}(M, \RR)$.
For each $p \geq 2$, let $F_p$ be the $p$-tight form representing $\rho$. Then there exists a closed $d - 1$-form $F$ such that:
\begin{enumerate}
\item $F_p \to F$ weakly in $L^r$ along a subsequence, for any $d < r < \infty$.
\item $F$ is a best comass representative of $\rho$.
\end{enumerate}
\end{proposition}
\begin{proof}
We roughly follow \cite[\S3]{Lindqvist14}.
Let $r > d$, and let $B$ be an $L^\infty$ representative of $\rho$.
By H\"older's inequality and (\ref{infinity magnetic rules p magnetic}),
\begin{equation}\label{uniform bounds in p by best curl}
	\|F_p\|_{L^r} \leq |M|^{\frac{1}{r} - \frac{1}{p}} \|F_p\|_{L^p} \leq |M|^{\frac{1}{r}} \|B\|_{L^\infty}.
\end{equation}
Thus a compactness argument gives $F_p \to F$ for some $d - 1$-form $F$, weakly in $L^r$, and 
$$\|F\|_{L^r} \leq \liminf_{p \to \infty} \|F_p\|_{L^r} \leq |M|^{\frac{1}{r}} \|B\|_{L^\infty}.$$
Diagonalizing, we may assume that $F_p \to F$ weakly in $L^r$ for every such $r$, and taking $r \to \infty$, we conclude 
\begin{equation}\label{infinity magnetics have best curl}
	\|F\|_{L^\infty} \leq \|B\|_{L^\infty}.
\end{equation}
Moreover, $[F] = \lim_{p \to \infty} [F_p] = \rho$.
So by Proposition \ref{crandall}(\ref{crandall linfinity}) and the fact that $B$ was arbitrary in (\ref{infinity magnetics have best curl}), $F$ has best comass.
\end{proof}

\begin{definition}
The $d - 1$-form $F$ of best comass in Proposition \ref{existence infinity} is called a \dfn{tight form}, or an \dfn{$\infty$-tight form}.
\end{definition}

The existence of tight (or even just best comass) representatives of each cohomology class $\rho$ implies the following useful lemma on the costable norm of $\rho$.

\begin{lemma}\label{p tights approximate L}
Let $F_p$ be the $p$-tight representative of $\rho$. Then 
$$\lim_{p \to \infty} \|F_p\|_{L^p} = \Comass(\rho).$$
\end{lemma}
\begin{proof}
We follow \cite[Lemma 2.7]{daskalopoulos2020transverse}.
Let $F$ be a tight representative of $\rho$, so $\|F\|_{L^\infty} = \Comass(\rho)$.
Since $F_p$ is $p$-tight, H\"older's inequality implies 
$$\|F_p\|_{L^p} \leq \|F\|_{L^p} \leq |M|^{\frac{1}{p}} \Comass(\rho).$$
Therefore 
$$\limsup_{p \to \infty} \|F_p\|_{L^p} \leq \Comass(\rho).$$
To prove the converse, suppose that for some $\varepsilon > 0$,
$$\liminf_{p \to \infty} \|F_p\|_{L^p} \leq \Comass(\rho) - \varepsilon < \Comass(\rho).$$
Along a subsequence which attains the limit inferior, $F_p$ converges weakly in every $L^r$, $d < r < \infty$, to a tight form $\tilde F$ such that (by H\"older's inequality)
$$\|\tilde F\|_{L^r} \leq \liminf_{p \to \infty} \|F_p\|_{L^r} \leq \liminf_{p \to \infty} |M|^{\frac{1}{r}} \|\tilde F\|_{L^\infty} \leq |M|^{\frac{1}{r}} (\Comass(\rho) - \varepsilon).$$
Taking $r \to \infty$, we obtain $\Comass(\tilde F) < \Comass(\rho)$, which contradicts the definition of the costable norm $\Comass(\rho)$.
\end{proof}


%%%%%%%%%%%%%%%%%%%%
\subsection{\texorpdfstring{$1$-harmonic conjugates of $\infty$-tight forms}{One-harmonic conjugates of infinity-tight forms}}
We now construct the $1$-harmonic conjugate of an $\infty$-tight form.
Since (\ref{inverse extremality}) may blow up as $p \to \infty$, we have to renormalize the $q$-harmonic conjugates of $p$-tight forms before taking the limit $q \to 1$, as in \cite[\S3.2]{daskalopoulos2020transverse}.
We begin by showing that $L^1$ convergence preserves the equivariance properties of functions.

\begin{lemma}\label{L1 convergence preserves pi1}
Let $\tilde M \to M$ be the universal abelian cover, and let $(u_q)$ be a sequence of $\pi_1(M)$-equivariant functions on $\tilde M$ which converge in $L^1_\loc(\tilde M)$ to a function $u$ as $q \to 1$.
Then $u$ is $\pi_1(M)$-equivariant, and $[u_q] \to [u]$.
Moreover, if $\dif u_q \to \dif u$ in the weak topology of measures on $M$ and $\dif u_q \in L^q$, then
\begin{equation}\label{q to 1 Holder}
\Mass(\dif u) \leq \liminf_{q \to 1} \frac{1}{q} \int_M \star |\dif u_q|^q.
\end{equation}
\end{lemma}
\begin{proof}
Since $u_q$ is $\pi_1(M)$-equivariant, there exists $\alpha_q \in H^1(M, \RR)$ such that for every $\gamma \in \pi_1(M)$,
\begin{equation}\label{equivariance q}
	\gamma^* u_q = u_q + \langle \alpha_q, \gamma\rangle.
\end{equation}
Let $M_{\rm fun}$ be a fundamental domain and $U_\gamma := M_{\rm fun} \cup \gamma_* (M_{\rm fun})$.

We claim that $(\alpha_q)$ has a convergent subsequence.
To see this, we first recall that $M$ has finite Betti numbers, so $H^1(M, \RR)$ is locally compact.
Therefore, if no convergent subsequence exists, there exists a $\gamma \in \pi_1(M)$ and a subsequence along which $\langle \alpha_q, \gamma\rangle \to \infty$.
Moreover, since $u_q \to u$ in $L^1_\loc$, $\|u_q\|_{L^1(M_{\rm fun})} \leq 2\|u\|_{L^1(M_{\rm fun})}$ if $q - 1$ is small enough.
But then 
$$\|u_q\|_{L^1(\gamma_* M_{\rm fun})} = \|\gamma^* u_q\|_{L^1(M_{\rm fun})} \geq \langle \alpha_q, \gamma\rangle - \|u_q\|_{L^1(M_{\rm fun})} \geq \langle \alpha_q, \gamma\rangle - 2\|u\|_{L^1(M_{\rm fun})}$$
and taking $q \to 1$ we conclude that $(u_q)$ is not compact in $L^1(\gamma_* M_{\rm fun})$, contradicting the convergence in $L^1_\loc(\tilde M)$.
So $\alpha_q \to \alpha$ for some $\alpha \in H^1(M, \RR)$ along a subsequence.

For any $q > 1$,
\begin{align*}
\dashint_{M_{\rm fun}} \star |\gamma^* u - u - \langle \alpha, \gamma\rangle| 
&\leq \dashint_{M_{\rm fun}} \star (|\gamma^* u_q - u_q - \langle \alpha_q, \gamma\rangle| + |\gamma^* u_q - u_q| + |\gamma^* u - u|) \\
&\qquad + |\langle \alpha_q - \alpha, \gamma\rangle|.
\end{align*}
Taking $q \to 1$ and applying (\ref{equivariance q}), we conclude that $\|\gamma^* u - u - \langle \alpha, \gamma\rangle\|_{L^1} = 0$, hence $u$ is $\alpha$-equivariant.
Thus $\alpha$ is uniquely defined and $\alpha_q \to \alpha$ along the entire subsequence.

Finally we prove (\ref{q to 1 Holder}).
Suppose that $\dif u_q \to \dif u$ in the weak topology of measures and $\dif u_q$ in $L^q$.
Then
$$\|\dif u_q\|_{L^1} = \Mass(\dif u_q).$$
So we may use the portmanteau theorem and H\"older's inequality to estimate (where $\frac{1}{p} + \frac{1}{q} = 1$)
\begin{align*}
\Mass(\dif u) &= \lim_{q \to 1} \Mass(\dif u_q) \leq \lim_{q \to 1} |M|^{\frac{1}{p}} \|\dif u_q\|_{L^q} = \lim_{q \to 1} \frac{1}{q} \int_M \star |\dif u_q|^q. \qedhere
\end{align*}
\end{proof}

Next we address the renormalization.
Suppose that $\rho \in H^{d - 1}(M, \RR)$ and denote by $L$ the comass of a best comass representative of $\rho$.
Also let $k_p$ be defined by 
$$k_p^{1 - p} = \int_M \star |F_p|^p$$
where $F_p$ is the $p$-tight representative of $\rho$.

\begin{definition}
The \dfn{renormalized $q$-harmonic conjugate} of a $p$-tight form $F_p$ is the function $u_q: \tilde M \to \RR$ which has mean zero on $M_{\rm fun}$ and solves
$$\dif u_q = (-1)^{d - 1} k_p^{p - 1} |F_p|^{p - 2} \star F_p.$$
\end{definition}

\begin{lemma}\label{normalizations converge}
As $p \to \infty$, $k_p \to \Comass(\rho)^{-1}$.
\end{lemma}
\begin{proof}
We follow \cite[Lemma 3.4]{daskalopoulos2020transverse}.
By Lemma \ref{p tights approximate L},
$$\lim_{p \to \infty} k_p^{-\frac{1}{q}} = \lim_{p \to \infty} \|F_p\|_{L^p} = \Comass(\rho).$$
Taking logarithms we see that $q^{-1} \log k_p \to -\log \Comass(\rho)$, and since $q \to 1$ the claim follows.
\end{proof}

\begin{proposition}\label{existence 1}
Let $\rho \in H^{d - 1}(M, \RR)$ and let $\tilde M \to M$ be the universal abelian cover.
For $2 < p < \infty$ and $\frac{1}{p} + \frac{1}{q} = 1$, let $u_q$ be the renormalized $q$-harmonic conjugate of the $p$-tight representative of $\rho$.
Then there exists a $\pi_1(M)$-equivariant function $u \in BV_\loc(\tilde M)$ such that:
\begin{enumerate}
\item $u$ is $1$-harmonic.
\item As $q \to 1$ along a subsequence, $u_q \to u$ weakly in $BV_\loc(\tilde M)$ and strongly in $L^r_\loc(\tilde M)$ for $1 \leq r < \frac{d}{d - 1}$.
\item Let $F$ be a tight representative of $\rho$. We have the \dfn{max flow min cut principle} that
\begin{equation}\label{1 extremality}
\dif u \wedge F = \Comass(\rho) \star |\dif u|.
\end{equation}
\end{enumerate}
\end{proposition}
\begin{proof}
Let $L := \Comass(\rho)$.
We first compute using H\"older's inequality and Lemma \ref{normalizations converge}
\begin{align*}
\lim_{q \to 1} \|\dif u_q\|_{L^1}
&\leq \lim_{q \to 1} |M|^{\frac{1}{p}} \left[\int_M \star |\dif u_q|^q\right]^{\frac{1}{q}} = \lim_{p \to \infty} \left[k_p^p \int_M \star |F_p|^p\right]^{\frac{1}{q}} \\
&= \lim_{p \to \infty} k_p^{\frac{1}{q}} = \lim_{p \to \infty} k_p = \frac{1}{L}.
\end{align*}
So by Rellich's theorem, $(u_q)$ is weakly compact in $BV$ and strongly compact in $L^r$ for $1 \leq r < \frac{d}{d - 1}$.
In particular, $\dif u_q \to \dif u$ in the weak topology of measures and $u_q \to u$ weakly in $BV$ and strongly in $L^r$.
As the limit of $\pi_1(M)$-equivariant functions, $u$ is also $\pi_1(M)$-equivariant by Lemma \ref{L1 convergence preserves pi1}.
In particular, $\dif u$ drops to a current on $M$.
Moreover, $[\dif u_q] \to [\dif u]$, and we have the bound (\ref{q to 1 Holder}) on $\int \star |\dif u|$.

Renormalizing (\ref{strong duality}), we obtain 
$$\frac{k_p^{-p}}{q} \int_M \star |\dif u_q|^q + \frac{1}{p} \int_M \star |F_p|^p = k_p^{1 - p} \int_M \dif u_q \wedge F_p.$$
Multiplying by $k_p^p$, we have 
\begin{equation}\label{1 strong duality before limits}
	\frac{1}{q} \int_M \star |\dif u_q|^q + \frac{k_p^p}{p} \int_M \star |F_p|^p = k_p \int_M \dif u_q \wedge F_p.
\end{equation}

Let $\mu(U) := \Mass_U(\dif u)$ be the total variation measure of $\dif u$.
We claim that
\begin{equation}\label{1 strong duality}
	L\mu(M) \leq \int_M \dif u \wedge F.
\end{equation}
First, we have from (\ref{q to 1 Holder}) and (\ref{1 strong duality before limits}) that
$$\mu(M) \leq \lim_{q \to 1} \frac{1}{q} \int_M \star |\dif u_q|^q = \lim_{p \to \infty} k_p \int_M \dif u_q \wedge F_p - \lim_{p \to \infty} \frac{k_p^p}{p} \int_M \star |F_p|^p.$$
By Lemma \ref{normalizations converge},
$$\lim_{p \to \infty} \frac{k_p^p}{p} \int_M \star |F_p|^p = \lim_{p \to \infty} \frac{k_p}{p} = \frac{0}{L} = 0,$$
and
$$\lim_{p \to \infty} k_p \int_M \dif u_q \wedge F_p = \frac{1}{L} \lim_{p \to \infty} \int_M [\dif u_q] \wedge \rho.$$
Since $[\dif u_q] \to [\dif u]$, we obtain
$$\lim_{p \to \infty} \int_M [\dif u_q] \wedge \rho = \int_M \alpha \wedge \rho = \int_M \dif u \wedge F,$$
completing the proof of (\ref{1 strong duality}).

By the coarea formula (\ref{coarea formula}), we have for any open set $U$,
$$\int_U \dif u \wedge F = \int_{-\infty}^\infty \int_{U \cap \partial \{u > y\}} F \dif y \leq L \int_{-\infty}^\infty |U \cap \partial \{u > y\}| \dif y = L \mu(U).$$
Since $\mu$ is a Radon measure and $M$ is compact, every Borel set $E$ can be $\mu$-approximated from without by open sets, hence
\begin{equation}\label{one sided extremality}
\int_E \dif u \wedge F \leq L \mu(E).
\end{equation}

Next we deduce (\ref{1 extremality}).
We reason by contradiction: if (\ref{1 extremality}) is false, then there exists an open set $U \subseteq M$ such that 
$$\int_U \dif u \wedge F < L \int_U \star |\dif u|.$$
(Indeed, strict inequality cannot point in the other direction, by (\ref{one sided extremality}).)
However, by (\ref{one sided extremality}), 
$$\int_{M \setminus U} \dif u \wedge F \leq L \int_{M \setminus U} \star |\dif u|.$$
Adding up the integrals of $\dif u \wedge F$ over $U$ and $M \setminus U$, we conclude 
$$\int_M \dif u \wedge F < L \int_M \star |\dif u|,$$
but this contradicts (\ref{1 strong duality}); thus (\ref{1 extremality}) must be true.

To round out the proof, let $X := (\star F/L)^\sharp$ be the Poincar\'e dual vector field to $F/L$. Then
$$\nabla \cdot X = \star \frac{\dif F}{L} = 0,$$
and $\|X\|_{L^\infty} \leq 1$.
Moreover, by (\ref{1 extremality}), $X$ is normal to the level sets of $u$, and hence is a witness that $u$ is $1$-harmonic.
\end{proof}

\begin{example}
Let $M$ be a closed surface equipped with a homotopy class $\alpha$ of maps $M \to \Sph^1$ with winding number $1$.
Using (\ref{Poincare Hurcewiz}) and (\ref{Poincare Hurcewiz 2}) we obtain a homology class of curves in $M$, from which we choose the geodesic representative $\gamma$.
Then the $1$-harmonic map $u$ in the homotopy class $\alpha$ lifts to a map $\tilde u: M_{\rm fun} \to \RR$ which is $0$ on one side of the lifted geodesic $\tilde \gamma$, and $1$ on the other side of $\tilde \gamma$.
Identifying $0, 1$ with the north pole of $\Sph^1$, we see that $u$ maps almost all of $M$ to the north pole of $\Sph^1$, but still has winding number $1$ as it wraps arbitrarily small neighborhoods of $\gamma$ around $\Sph^1$.
\end{example}



%%%%%%%%%%%%%%%%%%%%


\section{The maximum comass locus}\label{MCL sec}
Let $M$ be a closed Riemannian of dimension $2 \leq d \leq 4$ equipped with a cohomology class $\rho \in H^{d - 1}(M, \RR)$.
We shall study the set on which a best comass representative of $\rho$ attains its comass.
This set turns out to contain a measured oriented minimal lamination which only depends on $\rho$, and which is calibrated by any best comass form on $\rho$.

%%%%%%%%%%%%%%%%%%%%%
\subsection{Measured stretch laminations}
Let $u$ be a $\pi_1(M)$-equivariant $1$-harmonic function on $\tilde M$.
Then $\dif u$ drops to a $d-1$-current on $M$, which is still the Ruelle-Sullivan current of a measured oriented minimal lamination on $M$, which we still call $\lambda_u$.
Thus the following definition makes sense.

\begin{definition}
Let $\rho \in H^{d - 1}(M, \RR)$, let $F$ be a tight representative of $\rho$, and let $u$ be a $1$-harmonic conjugate of $F$.
Then we call $\lambda_u$ a \dfn{measured stretch lamination} associated to $\rho$.
\end{definition}

\begin{proposition}\label{MCL contains Thurston}
Let $F$ be a best comass representative of $\rho \in H^{d - 1}(M, \RR)$, where $\Comass(\rho) = 1$, and let $\lambda$ be a measured stretch lamination associated to $\rho$.
Then $F$ calibrates $\lambda$. In particular, $\MCL(F) \supseteq \supp \lambda$.
\end{proposition}
\begin{proof}
Let $G$ be the tight form which is cohomologous to $F$ whose dual $1$-harmonic function $u$ defines the measured stretch lamination $\lambda$.
Then by the max flow min cut principle (\ref{1 extremality}), 
$$\Mass(\lambda) = \Mass(\dif u) = \int_M \dif u \wedge G$$
so $G$ calibrates $\lambda$ by Proposition \ref{calibration condition}.
Then by Proposition \ref{properties of calibrated laminations}, $F$ calibrates $\lambda$ and $\MCL(F) \supseteq \supp \lambda$.
\end{proof}

\begin{proposition}\label{L equals K}
	Let $\rho \in H^{d - 1}(M, \RR)$ have measured stretch lamination $\mu$, and let $\lambda$ range over measured oriented laminations. Then 
	$$\Comass(\rho) = \sup_\lambda \frac{\langle \rho, [\lambda]\rangle}{\Mass(\lambda)} = \frac{\langle \rho, [\mu]\rangle}{\Mass(\mu)}.$$
\end{proposition}
\begin{proof}
Fix a tight form $F$ representing $\rho$, and let $u$ be its $1$-harmonic conjugate.
Let $L := \Comass(\rho)$ and
$$K :=  \sup_\lambda \frac{\langle \rho, [\lambda]\rangle}{|\lambda|}.$$

We first prove $K \leq L$.
Let $\lambda$ be a measured oriented lamination; then, since $F$ represents $\rho$ and the Ruelle-Sullivan current $T_\lambda$ represents $[\lambda]$,
$$\langle \rho, [\lambda]\rangle = \int_M F \wedge T_\lambda.$$
Let $(\chi_\alpha)$ be a partition of unity subordinate to a laminar atlas for $\lambda$, and let $(\mu_\alpha)$ be the associated transverse measure. Then 
$$\int_M F \wedge T_\lambda = \sum_\alpha \int_I \int_{\{k\} \times J} \chi_\alpha F \dif \mu_\alpha(k).$$
Since $F$ has best comass,
$$\frac{\langle \rho, [\lambda] \rangle}{\Mass(\lambda)}
\leq \frac{\|F\|_{L^\infty}}{\Mass(\lambda)} \sum_\alpha \int_I \int_{\{k\} \times J} \chi_\alpha \dif S_k \dif \mu_\alpha(k) = L.$$
Since $\lambda$ was arbitrary, it holds that $K \leq L$.

By the max flow min cut principle (\ref{1 extremality}),
$$\langle \rho, [\mu]\rangle = \int_M F \wedge \dif u = L \Mass(\dif u) = L \Mass(\mu).$$
Dividing both sides by $\Mass(\mu)$ and applying the direction we already proved,
$$K \leq L \leq \frac{\langle \rho, [\mu]\rangle}{\Mass(\mu)} \leq K$$
which is only possible if $L = K$ and $\mu$ is a maximizer.
\end{proof}

Propositions \ref{MCL contains Thurston} and \ref{L equals K} imply Theorem \ref{lams are calibrated}.
As a corollary we obtain a special case of Federer's max flow min cut theorem, Theorem \ref{Federer}.

\begin{proof}[Proof of Theorem \ref{Federer} if $d \leq 4$]
It suffices to show that if we view $H_{d - 1}(M, \RR)$ and $H^{d - 1}(M, \RR)$ as Banach spaces, equipped with the stable and costable norms respectively, then $H_{d - 1}$ and $H^{d - 1}$ are mutually dual.
Since $H^{d - 1}(M, \RR)$ is finite-dimensional and therefore reflexive \cite[Theorem 1.13.5]{megginson1998introduction}, it suffices to show that the dual space to $H^{d - 1}(M, \RR)$ is $H_{d - 1}(M, \RR)$.
Since we may view the homologically area-minimizing representative of a homology class as a measured oriented minimal lamination, every homology class contains a measured oriented minimal lamination which is homologically area-minimizing.
Therefore, for any $\rho \in H^{d - 1}(M, \RR)$, we obtain from Proposition \ref{L equals K} that
\begin{align*}
\Comass(\rho) &= \sup_\lambda \frac{\langle \rho, [\lambda] \rangle}{\Mass(\lambda)} = \sup_{\alpha \in H_{d - 1}(M, \RR)} \frac{\langle \rho, \alpha\rangle}{\Mass(\alpha)}. \qedhere 
\end{align*}
\end{proof}

% \begin{corollary}\label{existence of calibrations}
% Every closed area-minimizing hypersurface can be calibrated.
% \end{corollary}
% \begin{proof}
% Let $\alpha$ be the homology class of a closed area-minimizing hypersurface $N$.
% By duality between the stable norm and the best comass, and local compactness of $H^{d - 1}(M, \RR)$, there is a cohomology class $\rho \in H^{d - 1}(M, \RR)$ such that $\Comass(\rho) = 1$ and, if $F$ is a best comass representative of $\rho$,
% $$|N| = \Mass(\alpha) = \langle \rho, \alpha\rangle = \int_N F.$$
% Therefore $F$ calibrates $N$.
% \end{proof}


% %%%%%%%%%%%%%%%%%
% \subsection{The dual Ryu-Takayanagi formula}
% In conformal field theory, one is often interested in the \emph{entanglement entropy} $S(N)$ through a closed oriented hypersurface $N \subset M$.
% This quantity has a physical definition, but according to the Ryu-Takayanagi formula \cite{Ryu_2006}, $S(N)$ is the area of any homologically area-minimizing hypersurface homologous to $N$.
% Owing to the conceptual difficulty of interpreting the Ryu-Takayanagi formula, Freedman and Headrick introduced a dual form of the Ryu-Takayanagi formula, motivated by the max flow min cut principle \cite{Freedman_2016}.
% We would like to highlight the fact that the dual Ryu-Takayanagi formula actually follows immediately from Corollary \ref{existence of calibrations}.

% \begin{corollary}[dual Ryu-Takayanagi formula]
% For any closed oriented hypersurface $N \subset M$, the entanglement entropy through $N$ satisfies
% \begin{equation}\label{dual RT formula}
% S(N) = \max_{\substack{\dif F = 0 \\ L(F) \leq 1}} \int_N F.
% \end{equation}
% \end{corollary}
% \begin{proof}
% By Corollary \ref{existence of calibrations}, there exists a calibration $F$ of the area-minimizing hypersurface $N'$ homologous to $N$, and by Stokes' theorem, 
% $$S(N) = |N'| = \int_{N'} F = \int_N F.$$
% To see that $F$ is the maximum among all calibrations, let $G$ be another such form; then
% \begin{align*}
% \int_N G &= \int_{N'} G \leq |N'| = S(N). \qedhere 
% \end{align*}
% \end{proof}






%%%%%%%%%%%%%%%%%%%%%%%%%%%%
% \subsection{Existence of an optimal best comass form}
% \todo{exposit this}

% \begin{definition}
% An \dfn{optimal best comass form} is a best comass form $F$ such that
% $$\MCL(F) = \bigcap_G \MCL(G)$$
% where $G$ ranges over best comass forms cohomologous to $F$.
% \end{definition}

% \begin{proposition}
% Let $\rho \in H^2(M, \RR)$.
% Then there exists an optimal best comass representative of $\rho$.
% \end{proposition}
% \begin{proof}
% Let $\lambda := \bigcap_G \MCL(G)$ where $G$ ranges over best comass representatives of $\rho$.
% For $x \notin \lambda$, we can find a best comass form $F_x$ of class $\rho$ such that $x \notin \MCL(F_x)$.
% In particular, $U_x := \{L(F_x, \cdot) < L\}$ is an open set which contains $x$, so $(U_x)_{x \notin \lambda}$ is an open cover of $M \setminus \lambda$.
% Since $M \setminus \lambda$ is $\sigma$-compact, there exists a countable subcover $(U_{x_i})_{i \in I}$, for some countable set $I \subseteq \NN$.

% We then introduce the closed form 
% $$F := \sum_{i \in I} \alpha_i F_{x_i},$$
% where $\sum_{i \in I} \alpha_i = 1$.
% Here the sum converges in the norm topology of $L^\infty$, even if $I$ is infinite.
% Indeed, if $I_N := I \cap \{1, \dots, N\}$, then the partial sums $\sum_{i \in I_N} \alpha_i F_{x_i}$ satisfy the tail bound
% $$\sum_{i \in I \setminus I_N} \alpha_i \|F_{x_i}\|_{L^\infty} \leq L \sum_{i \in I \setminus I_N} \alpha_i \to 0$$
% since $(\alpha_i) \in \ell^1$, which implies the convergence.
% This convergence implies (by Proposition \ref{crandall}) that $[F] = \rho$, so $L(F) \geq L$.
% On the other hand, 
% $$L(F) \leq \sum_{i \in I} \alpha_i L(F_{x_i}) \leq L \sum_{i \in I} \alpha_i = L.$$
% It follows that $L(F) = L$.
% In particular, $F$ has best comass and $\MCL(F) \supseteq \lambda$.

% To complete the proof, we show that $\MCL(F) \subseteq \lambda$.
% Let $x \notin \lambda$, and let $j$ satisfy $U_{x_j} \ni x$.
% Then by Proposition \ref{crandall},
% $$L(F, x) = \lim_{r \to 0} L_{B_r(x)}(F) \leq \lim_{r \to 0} \sum_{i \in I} \alpha_i L_{B_r(x)}(F_{x_i}).$$
% The summands are dominated by the $\ell^1$ sequence $(L\alpha_i)$, so by dominated convergence, 
% $$\lim_{r \to 0} \sum_{i \in I} \alpha_i L_{B_r(x)}(F_{x_i}) = \sum_{i \in I} \lim_{r \to 0} \alpha_i L_{B_r(x)}(F_{x_i}) = \sum_{i \in I} \alpha_i L(F_{x_i}, x).$$
% By assumption on $j$, $L(F_{x_j}, x) < L$, and besides $L(F_{x_i}, x) \leq L$ for any $i$.
% So 
% $$L(F, x) \leq \sum_{i \in I} \alpha_i L(F_{x_i}, x) < L$$
% and we conclude $x \notin \MCL(F)$.
% \end{proof}


%%%%%%%%%%%%%%%%%%%%%%%%%%%%%%
% \section{The \texorpdfstring{$\infty$-Maxwell equation}{infinity-Maxwell equation}}\label{EulerLagrange}
% We have the following Euler-Lagrange equation for forms with absolutely best comass.
% Because of the lack of a good analogue for viscosity solutions for $\infty$-elliptic systems, and because we did not show that $\infty$-tight forms have absolutely best comass, the equation can only really be interpreted in a formal sense, at least as far as we are aware.
% As such, we did not use it in the main body of this paper, but only include it as a curiosity item.

% % \todo{If we knew that $p$-Maxwell had good quantitative uniqueness, then we would have}
% % It remains to show that $A$ has absolutely best curl, so let $\Omega$ be a small ball and $B$ a $1$-form with $B|_{\partial \Omega} = A|_{\partial \Omega}$.
% % By a straightforward modification of the existence theorem, there exists a $p$-magnetic potential $B_p$ in Coulomb gauge with $B_p|_{\partial \Omega} = A|_{\partial \Omega}$ and $B \in C^{1 + \alpha}$.
% % By quantitative uniqueness
% % $$\|B_p - A\|_{C^0(\Omega)} \leq \|B_p - A_p\|_{C^0(\Omega)} + o(1) \lesssim \|A - A_p\|_{C^0(\partial \Omega)} + o(1) \ll 1.$$
% % Therefore $B_p \to A$ uniformly, and for $3 < q < p < \infty$ with $p$ dyadic,
% % $$\|\dif B_p\|_{L^q(\Omega)} \leq |\Omega|^{\frac{1}{q} -\frac{1}{p}} \|\dif B_p\|_{L^p(\Omega)} \leq |\Omega|^{\frac{1}{q} -\frac{1}{p}} \|\dif B\|_{L^p(\Omega)} \leq |\Omega|^{\frac{1}{q}} \|\dif B\|_{L^\infty(\Omega)}.$$
% % Then along a subsequence, $\dif B_p \to \dif A$ in $L^q(\Omega)$, so 
% % $$\|\dif A\|_{L^q(\Omega)} \leq |\Omega|^{\frac{1}{q}} \|\dif B\|_{L^\infty(\Omega)}.$$
% % Taking $q \to \infty$ we arrive at the conclusion that $F$ has absolutely best comass.



% The $\infty$-Maxwell equation has the following natural interpretation.

% \begin{corollary}
% Suppose that $F$ has absolutely best comass, regularity $C^1$, and no points with $F = 0$, and $N$ is a surface whose normal vector field is annihilated by $F$.
% Then $N$ is a minimal surface.
% \end{corollary}
% \begin{proof}
% Let $V$ be a tangent vector field to $N$. Then $V(|F|) = 0$, by (\ref{infinityMaxwell}).
% Therefore $|F|$ is constant along $N$, but $F$ is a continuous section of the area bundle of $N$, which is a real line bundle.
% It follows that $F$ is constant along $N$, and $F/|F|$ is the area form on $N$.
% In other words, $N$ is calibrated by $F$, and the claim follows from (\ref{calibrated surfaces are minimal}).
% \end{proof}






\section{The \texorpdfstring{$\infty$-Maxwell equation}{infinity-Maxwell equation}}
By analogy with $\infty$-harmonic functions, one expects tight forms to be unique and have \dfn{absolutely best comass} in the sense that they have best comass (subject to the Dirichlet condition) in every small open ball in $M$.
One can show formally that the Euler-Lagrange equations for a smooth form $F$ of absolutely best comass are
\begin{equation}\label{infinity Maxwell}
\begin{cases}
\dif F = 0\\
\nabla_i F_J F^J {F^i}_K = 0
\end{cases}
\end{equation}
where $J$ is a $d - 1$-index and $K$ is a $d - 2$-index.
Let us formally derive the $\infty$-Maxwell equation (\ref{infinity Maxwell}):

\begin{lemma}
When written in nondivergence form, the $p$-Maxwell equation is 
\begin{equation}\label{p Maxwell nondivergence form}
\begin{cases}
\dif F = 0\\ 
|F|^{p - 4} ((p - 2) \nabla_i F_J {F^i}_K F^J - |F|^2 (\dif^* F)_K) = 0.
\end{cases}
\end{equation}
\end{lemma}
\begin{proof}
Using the metric property of $\nabla$ and the fact that $d(d - 2) + 1 \equiv d - 1$ modulo $2$,
\begin{align*}
0 &= \dif^*(|F|^{p - 2} F) \\
&= (-1)^{d(d - 2) + 1} \star \dif(|F|^{p - 2} \star F) \\
&= (-1)^{d - 1} \star(\dif(|F|^{p - 2}) \wedge \star F - |F|^{p - 2} \dif \star F) \\
&= (-1)^{d - 1} (p - 2) \star(|F|^{p - 4} \langle \nabla F, F\rangle \wedge \star F) - |F|^{p - 2} \dif^* F.
\end{align*}
We then compute 
$$(-1)^{d - 1} \star(\langle \nabla F, F\rangle \wedge \star F)_\ell = \nabla_i F_J F^J {F^i}_K$$
and factor out $|F|^{p - 4}$ to conclude (\ref{p Maxwell nondivergence form}).
\end{proof}

Now, to derive the $\infty$-Maxwell equation, one may proceed as in the derivation of the $\infty$-Laplace equation in \cite{Barron08}.
After formally dividing both sides of (\ref{p Maxwell nondivergence form}) by $(p - 2) |F|^{p - 4}$, one obtains the $\infty$-Maxwell equation, plus a term which vanishes as $p \to \infty$.

Recently Katzourakis has introduced a generalization notion of viscosity solutions, called \dfn{contact solutions}, for systems of degenerate PDE which cannot necessarily be written in divergence form or diagonal form \cite{Katzourakis2018OnAV}.
As the correct notion of solution for the $\infty$-Laplace equation is viscosity solution, it is natural to expect that the natural notion of solution of (\ref{infinity Maxwell}) is contact solution.
Thus it is natural to conjecture:

\begin{conjecture}
For every $\rho \in H^{d - 1}(M, \RR)$ there exists a unique tight representative of $\rho$.
Moreover, the following are equivalent for a closed $d-1$-form $F$:
\begin{enumerate}
\item $F$ has absolutely best comass.
\item $F$ is tight.
\item $F$ is a contact solution of the $\infty$-Maxwell equation (\ref{infinity Maxwell}).
\end{enumerate}
\end{conjecture}

The theory of contact solutions is still nascent, and the \emph{uniqueness} theory of contact solutions appears to be out of reach at the time of writing.
So we shall not attempt to prove this conjecture here.
\todo{Prove part of this conjecture, at least that the $p$-tight forms are contact solutions.}

% %%%%%%%
% \subsection{Integral hypersurfaces of classical tight forms}
% We have the following analogue of \cite[Lemma 1]{Aronsson68}.

% \begin{proposition}
% Let $F$ be a classical tight form with no zeroes, and let $N$ be an integral hypersurface of $F$.
% Then $N$ is a minimal hypersurface.
% \end{proposition}
% \begin{proof}
% Since $F/|F|$ is the area form on $N$, it suffices to show that $|F|$ is constant along $N$; if so, then $\dif(F/|F|) = |F|^{-1} \dif F = 0$.
% However, 
% \end{proof}

% %%%%%%%
% \subsection{Eikonal calibrations}
% Recall that that the \dfn{eikonal equation} on scalar fields $u$ is the PDE 
% $$|\dif u| = 1.$$
% If $u$ solves the eikonal equation, then $u$ also is $\infty$-harmonic \cite[\S]{Aronsson68}; this motivates us to study forms that satisfy the analogous condition.

% \begin{definition}
% An \dfn{eikonal calibration} is a $d-1$-form $F$ solving 
% \begin{equation}\label{eikonal}
% \begin{cases}
% \dif F = 0 \\
% |F| = 1.
% \end{cases}
% \end{equation}
% \end{definition}

% An eikonal calibration is classically tight, since 
% $$0 = \partial_i (|F|^2) = \langle \nabla F, F\rangle_i = \nabla_i F_J F^J.$$


\printbibliography

\end{document}
