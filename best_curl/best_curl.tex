\documentclass[reqno,11pt]{amsart}
\usepackage[letterpaper, margin=1in]{geometry}
\RequirePackage{amsmath,amssymb,amsthm,graphicx,mathrsfs,url,slashed,subcaption}
\RequirePackage[usenames,dvipsnames]{xcolor}
\RequirePackage[colorlinks=true,linkcolor=Red,citecolor=Green]{hyperref}
\RequirePackage{amsxtra}
\usepackage{cancel}
\usepackage{tikz-cd}
%\usepackage[T1]{fontenc}

% \setlength{\textheight}{9.3in} \setlength{\oddsidemargin}{-0.25in}
% \setlength{\evensidemargin}{-0.25in} \setlength{\textwidth}{7in}
% \setlength{\topmargin}{-0.25in} \setlength{\headheight}{0.18in}
% \setlength{\marginparwidth}{1.0in}
% \setlength{\abovedisplayskip}{0.2in}
% \setlength{\belowdisplayskip}{0.2in}
% \setlength{\parskip}{0.05in}
%\renewcommand{\baselinestretch}{1.05}

\title{Minimal laminations calibrated by tight $2$-forms}
\author{Aidan Backus}
\address{Department of Mathematics, Brown University}
\email{aidan\_backus@brown.edu}
\date{\today}

\newcommand{\NN}{\mathbf{N}}
\newcommand{\ZZ}{\mathbf{Z}}
\newcommand{\QQ}{\mathbf{Q}}
\newcommand{\RR}{\mathbf{R}}
\newcommand{\CC}{\mathbf{C}}
\newcommand{\DD}{\mathbf{D}}
\newcommand{\PP}{\mathbf P}
\newcommand{\MM}{\mathbf M}
\newcommand{\II}{\mathbf I}
\newcommand{\Hyp}{\mathbf H}
\newcommand{\Sph}{\mathbf S}
\newcommand{\Group}{\mathbf G}
\newcommand{\GL}{\mathbf{GL}}
\newcommand{\Orth}{\mathbf{O}}
\newcommand{\SpOrth}{\mathbf{SO}}
\newcommand{\Ball}{\mathbf{B}}

\newcommand*\dif{\mathop{}\!\mathrm{d}}

\DeclareMathOperator{\card}{card}
\DeclareMathOperator{\dist}{dist}
\DeclareMathOperator{\id}{id}
\DeclareMathOperator{\Hom}{Hom}
\DeclareMathOperator{\coker}{coker}
\DeclareMathOperator{\supp}{supp}
\DeclareMathOperator{\Teich}{Teich}
\DeclareMathOperator{\tr}{tr}

\newcommand{\Leaves}{\mathscr L}
\newcommand{\Lagrange}{\mathcal L}
\newcommand{\Hypspace}{\mathscr H}

\newcommand{\Chain}{\underline C}

\newcommand{\Two}{\mathrm{I\!I}}

\newcommand{\normal}{\mathbf n}
\newcommand{\radial}{\mathbf r}
\newcommand{\evect}{\mathbf e}
\newcommand{\vol}{\mathrm{vol}}

\newcommand{\diam}{\mathrm{diam}}
\newcommand{\Ell}{\mathrm{Ell}}
\newcommand{\inj}{\mathrm{inj}}
\newcommand{\Lip}{\mathrm{Lip}}
\newcommand{\MCL}{\mathrm{MCL}}
\newcommand{\Riem}{\mathrm{Riem}}

\newcommand{\Min}{\mathrm{Min}}
\newcommand{\Max}{\mathrm{Max}}

\newcommand{\dfn}[1]{\emph{#1}\index{#1}}

\renewcommand{\Re}{\operatorname{Re}}
\renewcommand{\Im}{\operatorname{Im}}

\newcommand{\loc}{\mathrm{loc}}
\newcommand{\cpt}{\mathrm{cpt}}

\def\Japan#1{\left \langle #1 \right \rangle}

\newtheorem{theorem}{Theorem}[section]
\newtheorem{badtheorem}[theorem]{``Theorem"}
\newtheorem{prop}[theorem]{Proposition}
\newtheorem{lemma}[theorem]{Lemma}
\newtheorem{sublemma}[theorem]{Sublemma}
\newtheorem{proposition}[theorem]{Proposition}
\newtheorem{corollary}[theorem]{Corollary}
\newtheorem{conjecture}[theorem]{Conjecture}
\newtheorem{axiom}[theorem]{Axiom}
\newtheorem{assumption}[theorem]{Assumption}

\newtheorem{mainthm}{Theorem}
\renewcommand{\themainthm}{\Alph{mainthm}}

\newtheorem{claim}{Claim}[theorem]
\renewcommand{\theclaim}{\thetheorem\Alph{claim}}
% \newtheorem*{claim}{Claim}

\theoremstyle{definition}
\newtheorem{definition}[theorem]{Definition}
\newtheorem{remark}[theorem]{Remark}
\newtheorem{example}[theorem]{Example}
\newtheorem{notation}[theorem]{Notation}

\newtheorem{exercise}[theorem]{Discussion topic}
\newtheorem{homework}[theorem]{Homework}
\newtheorem{problem}[theorem]{Problem}

\makeatletter
\newcommand{\proofpart}[2]{%
  \par
  \addvspace{\medskipamount}%
  \noindent\emph{Part #1: #2.}
}
\makeatother



\numberwithin{equation}{section}


% Mean
\def\Xint#1{\mathchoice
{\XXint\displaystyle\textstyle{#1}}%
{\XXint\textstyle\scriptstyle{#1}}%
{\XXint\scriptstyle\scriptscriptstyle{#1}}%
{\XXint\scriptscriptstyle\scriptscriptstyle{#1}}%
\!\int}
\def\XXint#1#2#3{{\setbox0=\hbox{$#1{#2#3}{\int}$ }
\vcenter{\hbox{$#2#3$ }}\kern-.6\wd0}}
\def\ddashint{\Xint=}
\def\dashint{\Xint-}

\usepackage[backend=bibtex,style=alphabetic,giveninits=true]{biblatex}
\renewcommand*{\bibfont}{\normalfont\footnotesize}
\addbibresource{best_curl.bib}
\renewbibmacro{in:}{}
\DeclareFieldFormat{pages}{#1}

\newcommand\todo[1]{\textcolor{red}{TODO: #1}}


\begin{document}
\begin{abstract}
We introduce a family of closed $d-1$-forms on Riemannian $d$-manifolds which minimize their comass (or $L^\infty$ norm) in their cohomology class, which we call \dfn{tight}.
Tight forms have properties similar to (gradients of) $\infty$-harmonic maps between surfaces: they are convex duals of $1$-harmonic functions and attain their comass on a measured oriented minimal lamination $\mu$.
We show that $\mu$ has properties analogous to a measured sublamination of Thurston's canonical lamination.
\end{abstract}

\maketitle

%%%%%%%%%%%%%%%%%%%%%%%%%%%%%%%%%%%%%%%%%%%%%%%%%%%%%%%
\section{Introduction}
In this paper we shall propose a generalization of Thurston's Teichm\"uller theory to a broader class of manifolds than hyperbolic surfaces.
To motivate this, let us first recall Thurston's Teichm\"uller theory.

Let $M$ be a closed Riemannian $d$-fold, $N$ another Riemannian manifold, and $\rho \in [M, N]$ a homotopy class.
A map $f: M \to N$ of class $\rho$ is \dfn{best Lipschitz} if it minimizes its Lipschitz constant, or maximum stretch,
$$\Lip(f) := \sup_{x, y \in M} \frac{\dist(f(x), f(y))}{\dist(x, y)},$$
among all maps in $\rho$.
If $M, N$ are closed hyperbolic surfaces with the same underlying topological space $S$, $\rho$ is the homotopy class of the identity $\id_S$, and $L(M, N)$ the best Lipschitz constant of $\rho$, then $\log L$ is a Finsler metric on Teichm\"uller space $\Teich(S)$, called \dfn{Thurston's asymmetric metric} \cite{Thurston98, Papadopoulos15}.
Thurston's asymmetric metric is a particularly appealing geometry on Teichm\"uller space because of its intimate connection with the structure of geodesic laminations on $M, N$ \cite{Gu_ritaud_2017}:

\begin{theorem}\label{existence of Thurston lamination}
Let $S$ be a closed oriented surface of genus $\geq 2$, $M, N \in \Teich(S)$, and $\mathcal F$ the set of best Lipschitz maps $M \to N$ homotopic to $\id_S$.
Then there exists a measured oriented geodesic lamination $\mu$ such that:
\begin{enumerate}
\item For $f \in \mathcal F$, let $\lambda_f$ be the set on which $f$ attains its Lipschitz constant. Then $\lambda_f$ is (the support of) a geodesic lamination.
\item Let $\lambda$ ranges over the set of measured laminations. Then
\begin{equation}\label{L is K}
L(M, N) = \sup_\lambda \frac{|\lambda|_N}{|\lambda|_M} = \frac{|\mu|_N}{|\mu|_M}.
\end{equation}
\item For $f \in \mathcal F$, $\mu$ is a sublamination of $\lambda_f$.
\end{enumerate}
\end{theorem}

The measured lamination $\mu$ is a sublamination of a maximal geodesic lamination $\tilde \mu$ on which $f$ attains its Lipschitz constant.
The lamination $\tilde \mu$, which may contained spiraling geodesics and hence need not equal $\mu$\footnote{If $\mu = \tilde \mu$, then the intersection of every isolated leaf of $\tilde \mu$ with a small ball must have locally finitely many connected components, since the transverse measure is Radon. However, a spiraling isolated geodesic $\gamma$ whose $\omega$-limit set is a geodesic $\Gamma$ must pass through any small ball containing a point of $\Gamma$ infinitely many times.}, is called \dfn{Thurston's canonical lamination} associated to $M, N$.

We should like to generalize Thurston's Teichm\"uller theory to apply not just to geodesic laminations on surfaces, but more general minimal laminations on more general manifolds.
In Thurston's case, the main point is to associate a geodesic lamination in $M$ to each homotopy class of maps $\rho \in [M, N]$, by considering the set on which a best Lipschitz representative of $\rho$ attains its Lipschitz constant.
Hence it is natural to pose the following vague problem:

\begin{problem}
	Let $M$ be a closed Riemannian manifold.
	Find a minimax problem, which takes as data a topological class $\rho$ on $M$, and whose solution attains its maximum on a lamination in $M$.
\end{problem}

If $M$ is a closed hyperbolic surface and $\rho \in H^1(M)$ is a (de Rham) cohomology class, this was accomplished by Daskalopolous and Uhlenbeck \cite{daskalopoulos2020transverse}:

\begin{theorem}[Daskalopolous--Uhlenbeck]\label{DU20}
Let $M$ be a closed hyperbolic surface and $\rho \in H^1(M)$.
Then there exists a best Lipschitz map $f: M \to \Sph^1$ such that $\dif f = \rho$, and a geodesic lamination $\mu$ such that:
\begin{enumerate}
\item $f$ is a variational\footnote{that is, arising as the limit of $p$-harmonic maps} solution of the $\infty$-Laplace equation
\begin{equation}\label{infinity laplacian}
	\langle \nabla^2 f, \nabla f \otimes \nabla f\rangle = 0.
\end{equation}
\item Every best Lipschitz map $\tilde f$ with $\dif \tilde f = \rho$ attains its Lipschitz constant on $\mu$.
\item The Noether current $\dif u$ associated to the invariance of (\ref{infinity laplacian}) under rotation of $\Sph^1$ is a transverse measure to $\mu$ and is locally the derivative of a function of least gradient.
\end{enumerate}
\end{theorem}

Here, a \dfn{function of least gradient} is a critical point of the total variation of the derivative, and can equivalently be viewed as solutions of the $1$-Laplace equation
$$\dif^* \left(\frac{\dif u}{|\dif u|}\right) = 0$$
in a suitable weak sense.
Two highlights of this theorem are the connection between Teichm\"uller theory and the endpoint behavior of the $p$-Laplacian, and the duality between the $1$-Laplacian and the $\infty$-Laplacian (arising here through Noether's theorem).

In fact, the role of duality in the study of best Lipschitz maps was conjectured by Thurston \cite[Abstract]{Thurston98}, who wrote:

\begin{quote}
I currently think that a characterization of minimal stretch\footnote{that is, best Lipschitz} maps should be possible in a considerably more general context ... and it should be feasible with a simpler proof based on more general principles -- in particular, the max flow min cut principle, convexity, and $L^0 \leftrightarrow L^\infty$ duality.
\end{quote}

Motivated by Theorem \ref{DU20} and Thurston's conjecture, Daskalopolous and Uhlenbeck later went on to study $\infty$-harmonic maps between closed hyperbolic surfaces, and Lie algebra-valued $1$-harmonic maps associated to them, and reproved by PDE techniques multiple theorems that Thurston had proven using geometric topology \cite{daskalopoulos2022,daskalopoulos2023}.

In this paper, we generalize Theorem \ref{DU20} to closed Riemannian manifolds of dimension $2 \leq d \leq 4$ (though we are mainly interested in the case $d = 3$).
Again we have a minimax problem -- the problem of finding a $d-1$-form $F$ of ``best comass'' with a prescribed cohomology class -- which is dual to the $1$-Laplacian; the form $F$ calibrates a minimal lamination of codimension $1$.
In previous work, we used the variational characterization of $1$-harmonic functions to show that the level sets of a $1$-harmonic function form a minimal lamination \cite[Theorem C]{BackusCML}.
The convex dual problem to the $1$-Laplace equation is locally the problem of minimizing $\|F\|_{L^\infty}$ among $d-1$-forms $F$, so it is natural to select a minimizer $F$ and its $1$-harmonic dual $u$.
For $d = 2$, we have $F = \dif f$ for a best Lipschitz map $f: M \to \Sph^1$.

%%%%%%%%%%%%%%%%%%%%%%%%%%%%%%%%%%%%%%%%%

\subsection{\texorpdfstring{$\infty$-tight forms and $1$-harmonic functions}{Infinity-tight forms and one-harmonic functions}}
We introduce the \dfn{comass}
\begin{equation}\label{comass}
L(F) := \sup_{\sigma \in \Chain_{d - 1}(M)} \frac{1}{|\sigma|} \int_\sigma F
\end{equation}
of a closed $d-1$-form $F$ in a closed Riemannian $d$-fold $M$.
Here $\Chain_{d - 1}(M)$ denotes the space of oriented $d - 1$-chains in $M$, and $|\sigma|$ is the $d-1$-area of $\sigma$.
In fact, we shall show that $\|F\|_{L^\infty} = L(F)$, but $L(F)$ is more geometrically natural than $\|F\|_{L^\infty}$.

We study minimizers of the comass, which we call \dfn{best comass} forms, in a given cohomology class.
To do this, we construct best comass forms which are analogous to $\infty$-harmonic functions in the same spirit as \cite{daskalopoulos2020transverse,daskalopoulos2022}, and in particular which arise as limits of the $L^p$ analogue of best comass forms.

Let $d < p < \infty$ and $\frac{1}{p} + \frac{1}{q} = 1$.
Motivated by the $p$-Laplace equation $\dif^*(|\dif f|^{p - 2} \dif f) = 0$, we introduce \dfn{$p$-tight} forms, which are closed $d-1$-forms which solve the system of PDE
$$\dif^*(|F|^{p - 2} F) = 0.$$
Given a $p$-tight form, the $\pi_1(M)$-equivariant function $u$ on the universal cover such that
$$\dif u = (-1)^{d - 1} |F|^{p - 2} \star F$$
is $q$-harmonic -- in other words, $u$ is a solution of the $q$-Laplace equation 
$$\dif^*(|\dif u|^{q - 2} \dif u) = 0.$$
Our first theorem constructs a best comass form, and a dual $1$-harmonic function, by taking limits of $p$-tight forms and their dual $q$-harmonic functions.

\begin{mainthm}\label{existence of infinity tight forms}
Let $\rho \in H^{d - 1}(M, \RR)$ be a cohomology class.
Let $(F_p, u_q)$ be the family of dual pairs of $p$-tight forms and $q$-harmonic functions, suitably normalized, with $[F_p] = \rho$.
Then there exists a pair $(F, u)$ such that as $p \to \infty$, $F_p \to F$ weakly in $L^r$ for any $d < r < \infty$, and $u_q \to u$ weakly in $BV$.
Moreover, $F$ has best comass in $\rho$, $u$ is $1$-harmonic, and we have the duality relation 
\begin{equation}\label{max flow mean cut}
L|\dif u| = \langle \dif u, \star F\rangle
\end{equation}
in the sense of Radon measures, where $L$ is the best comass of a representative of $\rho$.
\end{mainthm}

This is a combination of Propositions \ref{existence infinity} and \ref{existence 1}.
We call the best comass form $F$ an \dfn{$\infty$-tight} form, or simply a \dfn{tight} form.

Most of this theorem essentially follows from the methods of \cite{Mazon14,daskalopoulos2020transverse}, but we highlight (\ref{max flow mean cut}) as the main point of the theorem.
It has multiple interpretations:
\begin{enumerate}
\item Since (\ref{max flow mean cut}) asserts a form of convex duality between $\infty$-tight forms and $1$-harmonic functions, we can view it as the analogue of the max flow min cut principle alluded to by Thurston.
\item Since (\ref{max flow mean cut}) is exactly the assertion that $F/L$ is the Poincar\'e dual to a vector field $X$ which witnesses that $u$ is $1$-harmonic, this duality relation allows us to easily prove that $u$ is $1$-harmonic without carrying out a careful analysis of the limiting behavior of the $q$-Laplacian or $p$-tight forms as in \cite[Theorem 2.4]{Mazon14} or \cite[\S6]{daskalopoulos2020transverse}.
\item Modulo technicalities arising from geometric measure theory, one can interpret (\ref{max flow mean cut}) to mean that $F/L$ is the area form on the level sets of $u$. This last point we shall repreatedly use throughout the remainder of the paper.
\end{enumerate}

%%%%%%%%%%%%%%%%%%

\subsection{Calibration of the measured stretch lamination}
A \dfn{calibration} is a closed $d-1$-form $F$ on the Riemannian $d$-fold $M$, such that $L(F) = 1$.
A hypersurface $N \subset M$ is $F$-\dfn{calibrated} the pullback of $F$ to $N$ is the area form on $N$ \cite{Harvey82}.
In that case, the mean curvature of $N$ is 
\begin{equation}\label{calibrated surfaces are minimal}
H_N = \nabla \cdot \normal_N = \nabla \cdot (\star F)^\sharp = \star \dif F = 0,
\end{equation}
so that $N$ is minimal. 

If a lamination $\lambda$ is $F$-calibrated (in the sense that its leaves are $F$-calibrated), then it is clear from the definitions and (\ref{calibrated surfaces are minimal}) that $\lambda$ is minimal and $\supp \lambda$ is contained in the locus on which $F$ attains its comass.
In the other direction, Bangert and Cui showed that continuous minimizers of the comass are calibrations of laminations \cite{bangert_cui_2017}:

\begin{theorem}[Bangert--Cui]
Let $F$ be a continuous calibration of best comass on a closed Riemannian manifold of dimension $2 \leq d \leq 7$.
Then there exists a measured oriented lamination $\lambda$ which is $F$-calibrated.
In particular, $\lambda$ is minimal and $\supp \lambda$ is contained in the set on which $F$ attains its comass.
\end{theorem}

We would like to characterize the $F$-calibrated as arising from the dual $1$-harmonic function.
Moreover, we cannot use the Bangert--Cui theorem directly, since a proof that tight forms are continuous is out of reach\footnote{For domains in $\RR^2$, $\infty$-tight forms are $C^\alpha$ \cite{Evans08}, but it is unlikely that this argument generalizes.}, though it may be possible to modify their argument to work for $L^\infty$ calibrations.
As such, we now state a refinement of the Bangert--Cui theorem.

\begin{definition}
Let $M$ be a closed Riemannian manifold of dimension $2 \leq d \leq 4$, and $\rho \in H^{d - 1}(M, \RR)$.
We can define a measured oriented minimal lamination $\mu$, by considering an $\infty$-tight representative $F$ of $\rho$, letting $u$ be the dual $1$-harmonic function to $F$, and letting $\mu$ be the lamination induced by $u$.
We call $\mu$ a \dfn{measured stretch lamination} associated to $\rho$.
\end{definition}

Our next theorem is the combination of Propositions \ref{MCL contains Thurston} and \ref{L equals K}, and follows easily from (\ref{max flow mean cut}) and the theory of \cite{BackusFLG, BackusCML}.
It characterizes the measured stretch lamination as a lamination which is maximally stretched in the sense of (\ref{L is K}).
To state the theorem, we let $[\lambda]$ denote the homology class of a measured oriented lamination $\lambda$.

\begin{mainthm}[refined Bangert--Cui theorem]\label{lams are calibrated}
Suppose that $M$ is a closed Riemannian manifold of dimension $2 \leq d \leq 4$.
Let $\mu$ be a measured stretch lamination associated to $\rho \in H^{d - 1}(M, \RR)$, and let $F$ be a form of best comass representing $\rho$, with comass $L$.
Then $\mu$ is $F/L$-calibrated.
Moreover, for $\lambda$ ranging over measured oriented laminations,
\begin{equation}\label{duality between stable and comass}
L = \sup_\lambda \frac{\langle \rho, [\lambda]\rangle}{|\lambda|} = \frac{\langle \rho, [\mu]\rangle}{|\mu|}.
\end{equation}
\end{mainthm}

Theorem \ref{lams are calibrated} has a rather useful interpretation.
The \dfn{stable norm} $\|\alpha\|_s$ of a homology class $\alpha \in H_{d - 1}(M, \RR)$ is the area of a homologically area-minimizing representative of $\alpha$.
The best comass $L(\rho)$ of a cohomology class $\rho \in H^{d - 1}(M, \RR)$ also defines a norm on $H^{d - 1}(M, \RR)$.
Then by (\ref{duality between stable and comass}), the dual norm of the stable norm is the best comass.
This was already known \cite[Theorem 3.8]{AUER20011095}, but it is convenient to have somewhat explicit witnesses to the duality.



%%%%%%%%%%%%%%%%%%%%%
\subsection{Outline of the paper}
\todo{Fix this}

In \S\ref{comass sec}, we outline the basic properties of the comass, and the geometric measure theory needed for the problem at hand.

In \S\ref{tight forms sec}, we construct the $\infty$-tight form in each cohomology class, and its $1$-harmonic conjugate, proving Theorem \ref{existence of infinity tight forms}.
We also state a few natural conjectures about $\infty$-tight and $p$-tight forms which we shall not address here.

In \S\ref{MCL sec}, we study the maximum comass locus of a form of best comass, and the measured stretch lamination associated to its cohomology class.
By applying the results of the previous two sections, we prove Theorem \ref{lams are calibrated}.


%%%%%%%%%%%%%%%%%%%%%%
\subsection{Acknowledgements}
I would like to thank Georgios Daskalopolous and Karen Uhlenbeck for suggesting this project, providing helpful comments, and providing me with an early draft of the manuscript \cite{daskalopoulos2023} which was a major source of inspiration for this work.
I also would like to thank Tom Goodwillie, Kaya Ferendo, Tainara Borges, and Haram Ko for helpful discussions.

This research was supported by the National Science Foundation's Graduate Research Fellowship Program under Grant No. DGE-2040433.


%%%%%%%%%%%%%%%%%%%%%%%%%%%%%%%%%%%%%%%%%%
\section{Summary of \texorpdfstring{\cite{BackusFLG,BackusCML}}{previous results}}
\subsection{Geometric measure theory}
To fix notation and conventions we recall some well-known geometric measure theory.
We also shall recall some technicalities which we established in \cite{BackusFLG}.

The sheaf of $\ell$-forms is denoted $\Omega^\ell$.
We assume that $\ell$-forms are $L^1_\loc$, but \emph{not} that they are continuous; hence $\dif$ must be meant in the sense of distributions.
To avoid confusion, we write $H^\ell$ for de Rham cohomology, but \emph{never} a Sobolev space (which shall only be denoted by $W^{\ell, p}$), nor a Hausdorff measure (which shall be denoted $\mathcal H^\ell$).
We let $\dif V = \star 1$ be the volume form on $M$, and for an $\ell$-rectifiable set $\tau$, we let $\dif S_\tau := \dif \mathcal H^\ell|_\tau$.

A \dfn{geometric $\ell$-vector} is a tensor of the form $v_1 \wedge \cdots v_\ell$, where $v_i$ are tangent vectors.
The \dfn{absolute value} $|\varphi|$ of an $\ell$-covector $\varphi$ is the supremum of $F(v)$, taken over all \emph{geometric} $\ell$-vectors $v$.

By an $\ell$-\dfn{current} $T$ we mean a continuous linear functional on the space $C^0_\cpt(M, \Omega^\ell)$ of continuous $\ell$-forms of compact support.
Note carefully that this excludes currents which do not have locally finite variation.
We write $\int_M T \wedge \varphi$ for the dual pairing of a current and a form.
The duality norm of $T$ is its \dfn{total variation}, namely for an open set $U \subseteq M$,
$$\|T\|_{TV(U)} := \sup_{\substack{\supp \varphi \Subset U \\ \|\varphi\|_{C^0} \leq 1}} \int_U T \wedge \varphi.$$
A function $u$ has \dfn{locally bounded variation} if $\dif u$ extends to a continuous linear functional on $C^0_\cpt(M, \Omega^\ell)$.
In that case we write $u \in BV_\loc(M)$.

We showed that the null set in the statement of the Lebesgue differentiation theorem, when applied to a section of a trivialized tensor bundle, does not depend on the choice of trivialization \cite[Proposition 2.1]{BackusFLG}:

\begin{proposition}\label{invariance of LDT}
Let $F$ be a $L^1_\loc$ section of a tensor bundle over $M$, and let $\mu$ be a Radon measure on $M$.
Then there exists a diffeomorphism-invariant $\mu$-null set $Z \subset M$ such that for any coordinate system on $M$, and any index $I$, and any $x \notin Z$,
$$F_I(x) = \lim_{r \to 0} \frac{1}{\mu(B_r(x))} \int_{B_r(x)} F_I(y) \dif \mu(y).$$
\end{proposition}

Suppose that an open set $U \subset M$ has \dfn{locally finite perimeter} in the sense that its indicator function $1_U$ has locally bounded variation.
Then $U$ has a \dfn{reduced boundary} $\partial^* U$, which, as a current, is the negative exterior derivative of $1_U$; see \cite[Chapter 3]{Giusti77}.
In particular, the superlevel sets $\{u > y\}$ of a function $u \in BV_\loc(M)$ have locally finite perimeter \cite[Theorem 1.23]{Giusti77}.
We carefully wrote down the coarea formula at this level of regularity on curved domains \cite[Proposition 2.5]{BackusFLG}:

\begin{proposition}\label{coarea}
Let $\mu(U) := \|\dif u\|_{TV(U)}$ be the total variation measure of a function $u \in BV_\loc(M)$.
Then 
$$\mu(U) = \int_{-\infty}^\infty |\partial^* \{u > y\} \cap U| \dif y.$$
\end{proposition}

%%%%%%%%%%%%%%%%%%%%%
\subsection{\texorpdfstring{$1$-harmonic functions}{One-harmonic functions} and minimal laminations}
We now recall the correct formulation of weak solutions to the $1$-Laplace equation 
$$\dif^*\left(\frac{\dif u}{|\dif u|}\right) = 0.$$
The formulation essentially says that $\dif u/|\dif u|$ extends to a coclosed $d-1$-current, even if $|\dif u|$ is a singular Radon measure.
We must allow for this case, because we want the indicator function of a set bounded by a minimal surface to be $1$-harmonic.
The definition we give here is the sheafification of the definition of $1$-harmonic function due to Maz\'on, Rossi, and Segura de Le\'on \cite{Mazon14}, and already appeared in \cite{BackusCML}.

\begin{definition}
A \dfn{$1$-harmonic function} on $M$ is a function $u \in BV_\loc(M)$ such that there exists a divergence-free vector field $X$ on a neighborhood of $\supp \dif u$, with $\|X\|_{L^\infty} \leq 1$, and such that, in the sense of Radon measures,
$$(\dif u, X) = |\dif u|.$$
\end{definition}

\begin{definition}
Let $B$ be a ball in $M$.
A function $u \in BV(B)$ has \dfn{least gradient} in $B$ if $u$ minimizes its total variation in the sense that for every function $v \in BV_\cpt(B)$,
$$\|\dif u\|_{TV(B)} \leq \|\dif u + \dif v\|_{TV(B)}.$$
A function has \dfn{locally least gradient} if one can cover $M$ by balls in which $u$ has least gradient.
\end{definition}

By \cite[Theorem 1.1]{Mazon14}, a function is $1$-harmonic iff it has locally least gradient.

We also want a geometric formulation of $1$-harmonic functions.
Thus we turn to the theory of laminations.
Fix an interval $I \subset \RR$ and a box $J \subset \RR^{d - 1}$.

\begin{definition}
A \dfn{laminar flow box} is a $C^0$ coordinate chart $F: I \times J \to M$ and a compact set $K \subseteq I$, such that for every $k \in K$, $F|_{\{k\} \times J}$ is a $C^1$ embedding, and the \dfn{leaf} $F(\{k\} \times J)$ is a $C^1$ complete hypersurface in $F(I \times J)$.
Two laminar flow boxes belong to the same \dfn{laminar atlas} if the transition maps between them send leaves to leaves.
\end{definition}

\begin{definition}
A \dfn{lamination} is a closed subset $S \subseteq M$, called its \dfn{support}, and a maximal laminar atlas $\mathscr A$, such that $S$ is the union of the leaves of $\mathscr A$.
A \dfn{foliation} is a lamination $\lambda$ with $\supp \lambda = M$.
\end{definition}

\begin{definition}
A lamination is
\begin{enumerate}
\item \dfn{Lipschitz} if its flow boxes are Lipschitz isomorphisms,
\item \dfn{oriented} if its transition maps are orientation-preserving, and
\item \dfn{minimal} if its leaves are minimal hypersurfaces.
\end{enumerate}
\end{definition}

Construction of the flow boxes can be quite cumbersome, but it can be done systematically; this is \cite[Theorem A]{BackusCML}.

\begin{theorem}
Let $\mathcal S$ be a set of disjoint complete minimal hypersurfaces in a manifold $M$ of bounded geometry.
Suppose that for every $N \in \mathcal S$, we have a bound on the second fundamental form 
$$\|\Two_N\|_{C^0} \leq C.$$
Then $\mathcal S$ is the set of leaves of a Lipschitz minimal lamination $\lambda$.
In particular, if $\lambda$ is oriented, then there is a Lipschitz vector field on $M$ whose restriction to each $N \in \mathcal S$ is the normal vector to $N$.
\end{theorem}

\begin{definition}
A lamination $\lambda$ with atlas $(F_\alpha, K_\alpha)$ is \dfn{measured} if it is equipped with positive Radon measures $\mu_\alpha$ with $\supp \mu_\alpha = K_\alpha$, such that the transition maps $F_\beta^{-1} \circ F_\alpha$ are measure-preserving.
The \dfn{Ruelle-Sullivan current} of a measured oriented lamination $\lambda$ with atlas $(F_\alpha, K_\alpha, \mu_\alpha)$ is the $d-1$-current $T_\lambda$ satisfying, for any partition of unity $(\chi_\alpha)$ subordinate to the open cover $(F_\alpha(I \times J))$,
$$\int_M T_\lambda \wedge \varphi = \sum_\alpha \int_{K_\alpha} \int_{\{k\} \times J} F_\alpha^* (\chi_\alpha \varphi) \dif \mu_\alpha(k).$$
The \dfn{homology class} $[\lambda]$ and \dfn{mass} $|\lambda|$ of a measured oriented lamination $\lambda$ are the homology class and mass of its Ruelle-Sullivan current.
\end{definition}

In particular, if $\lambda$ is simply a $d-1$-chain of finitely many hypersurfaces, then $T_\lambda$ is the current defined by integration of a $d-1$-form along the chain.

We are now ready to state \cite[Theorem C]{BackusCML}:

\begin{theorem}\label{1 harmonic is MOML}
Suppose that $2 \leq \dim M \leq 4$.
Then a function $u \in BV_\loc(M)$ is $1$-harmonic iff $\dif u$ is the Ruelle-Sullivan current for a measured oriented minimal lamination.
\end{theorem}



%%%%%%%%%%%%%%%%%%%%%%%%%%%%%%%%%%%%%%%%%%

\section{Comass}\label{comass sec}
In this section we study the basic properties of the comass.
We show that it is well-defined, equals the $L^\infty$ norm, and has a local version which is upper semicontinuous.
Along the way we establish some tools from geometric measure theory that we shall need in the sequel.
Throughout, we fix a Riemannian manifold $M$ (possibly not closed) of dimension $d$ and metric $g$.

\begin{definition}
For a closed $d-1$-form $F$ and a subdomain $\Omega \subseteq M$, the \dfn{comass} of $F$ in $\Omega$ is
$$L_\Omega(F) := \sup_{\sigma \in \Chain_{d - 1}(\Omega)} \frac{1}{|\sigma|} \int_\sigma F.$$
We let $L(F) := L_M(F)$.

We say that $F$ has \dfn{best comass} if $F$ is a minimizer of $L$ among all forms cohomologous to $F$.
\end{definition}

%%%%%%%%%%%%%%%%%%%
\subsection{Trace theorem}
The comass $L_\Omega(F)$ is well-defined if $F$ is continuous, or has a trace along every $d - 1$-simplex.
We now prove that we may take traces of an $L^\infty$ $d - 1$ form, provided that it is closed.

We say that a $d-2$-form $A$ is in \dfn{Coulomb gauge} if $\dif^* A = 0$.
If $F = \dif A$, then it follows that $F = (\dif + \dif^*) A$, and since the Dirac operator $\dif + \dif^*$ is elliptic, for $p > d$ we get from Sobolev embedding and elliptic regularity \cite[\S4]{Scott95} that
\begin{equation}\label{Sobolev}
	\|A\|_{C^0} \lesssim \|\nabla A\|_{L^p} \lesssim \|(\dif + \dif^*) A\|_{L^p} = \|F\|_{L^p}.
\end{equation}

Recall that an \dfn{integral current} is an current of integer multiplicity (in the sense of \cite[\S27]{Simon84}) whose exterior derivative has integer multiplicity.
Since integral $\ell$-currents are ``morally'' $\ell$-chains, if $\tau$ is an integral $\ell$-current, we write $\int_\tau F$ for the pairing of $\tau$ with an $\ell$-form $F$.

\begin{proposition}[trace theorem]\label{integration is welldefined}
Let $\tau$ be an integral $d-1$-current, and $\psi \in C^1(M)$.
Then for any $d < p \leq \infty$, $F \mapsto \int_\tau \psi F$ extends to a continuous linear functional on the space of $L^p$ closed $d-1$-forms.
Moreover,
\begin{equation}\label{integral over chain is linfinity}
	\int_\tau F \leq \|F\|_{L^\infty} \int_\tau \psi \dif S_\tau.
\end{equation}
\end{proposition}
\begin{proof}
After replacing $\psi$ with $\psi \chi_\alpha$ for $(\chi_\alpha)$ a suitable partition of unity, we may assume that $M$ is contractible.
We show that for $d < p < \infty$, $F \mapsto \int_\tau \psi F$ is continuous for the $L^p$ norm on smooth closed $d-1$-forms.
If $F, G$ are closed and smooth and we select $A, B$ in Coulomb gauge with $\dif A = F$, $\dif B = G$, we have from integration by parts and (\ref{Sobolev}),
\begin{align*}
	\left|\int_\tau \psi(F - G)\right| 
	&\leq \left|\int_{\partial \tau} \psi (A - B)\right| + \left|\int_\tau \dif \psi \wedge (A - B)\right| \\
	&\lesssim_{\tau, \psi} \|A - B\|_{C^0} \lesssim \|F - G\|_{L^p}.
\end{align*}
So by a mollification argument, $F \mapsto \int_\tau \psi F$ is a continuous linear functional on the $L^p$ closed forms.

Suppose now that $F \in L^\infty$.
The estimate (\ref{integral over chain is linfinity}) is diffeomorphism-invariant, so we may work in coordinates and hence assume that $M$ is an open subset of $\RR^d$.
Choose a sequence of mollifiers $\chi_\varepsilon$ such that $\|\chi_\varepsilon\|_{L^1} = 1$.
Introduce the convolution $F_\varepsilon := F * \chi_\varepsilon$, so that $F_\varepsilon \to F$ in $L^p$ for any $d < p < \infty$.
By Young's inequality, 
$$\|F_\varepsilon\|_{C^0} \leq \|F\|_{L^\infty} \|\chi_\varepsilon\|_{L^1} \leq \|F\|_{L^\infty}.$$
Taking $\varepsilon \to 0$ we see that
\begin{align*}
\int_\tau \psi F 
&= \lim_{\varepsilon \to 0} \int_\tau \psi F_\varepsilon \leq \lim_{\varepsilon \to 0} \|F_\varepsilon\|_{C^0} \int_\tau \psi \dif S_\tau \leq \|F\|_{L^\infty} \int_\tau \psi \dif S_\tau. \qedhere
\end{align*}
\end{proof}

Taking $\psi := 1$, it follows the integral in the definition of comass is well-defined as long as $F \in L^p$, $p > d$, and each such integral depends continuously on $F$ in $L^p$.
It will later be useful to instead take $\psi$ to be an element of a partition of unity.

Let $u \in BV(M)$; then the reduced level sets $\partial^* \{u > y\}$ are exact currents of integer multiplicity, since they are reduced boundaries of sets of locally finite perimeter \cite[Theorem 14.3]{Simon84}; therefore the reduced level sets are integral currents.

\begin{proposition}[coarea formula]
Let $u \in BV(M)$, let $U \subseteq M$ be an open set, and let $F \in L^\infty(M, \Omega^{d - 1})$ satisfy $\dif F = 0$.
Then for almost every $y \in \RR$, $\int_{\partial^* \{u > y\} \cap U} F$ depends continuously on $F$ in $L^p$.
In particular,
\begin{equation}\label{coarea formula}
\int_U \dif u \wedge F = \int_{-\infty}^\infty \int_{\partial^* \{u > y\} \cap U} F \dif y.
\end{equation}
\end{proposition}
\begin{proof}
Let $\tau_y := \partial^* \{u > y\}$, let $\mu$ be the total variation measure of $\dif u$, and let $\normal$ be the conormal $1$-form to $\partial^* \{u > y\}$, hence $\dif 1_{\{u > y\}} = \normal \mu$.
If $F$ is continuous, then $\langle\star \normal, 1_U F\rangle \in L^\infty_\loc(\mu)$, so we can compute using Proposition \ref{coarea}:
\begin{align*}
\int_U \dif u \wedge F = \int_U \langle \star \normal, F\rangle \dif \mu = \int_{-\infty}^\infty \int_{\tau_y} \langle \star \normal, 1_U F\rangle \dif S_{\tau_y} \dif y = \int_{-\infty}^\infty \int_{\tau_y} 1_U F \dif y.
\end{align*}
By the trace theorem, $\int_{\tau_y} F$ depends continuously on $F$ in $L^p$ for almost every $y$, so by a mollification argument, (\ref{coarea formula}) holds for every $F$.
\end{proof}



%%%%%%%%%%%%%%%%%%%%%%%
\subsection{Local comass}
We will be interested in the points at which $F$ attains its comass.
One could pose this problem as the problem of computing the locus $\{|F| = \|F\|_{L^\infty}\}$.
However, $F$ is both only defined almost everywhere, and not norm-approximable by smooth functions.
So as a proxy for $|F|$, which may fail to be defined on a null set, we use the local comass, which is defined everywhere.

\begin{definition}
The \dfn{local comass} of a closed $d - 1$-form $F$ at $x \in M$ is 
$$L(F, x) = \limsup_{\varepsilon \to 0} L_{B_\varepsilon(x)}(F).$$
\end{definition}

Since $L_{B_\varepsilon(x)}(F)$ is a supremum over a set which grows in $\varepsilon$, it is increasing in $\varepsilon$, so the limit superior is actually a limit and an infimum:
$$L(F, x) = \lim_{\varepsilon \to 0} L_{B_\varepsilon(x)}(F) = \inf_{\varepsilon > 0} L_{B_\varepsilon(x)}(F).$$
In particular, $L(F, x) \leq L(F)$.

The comass enjoys many of the same properties as the Lipschitz constant, including \cite[Lemma 4.3]{Crandall2008}:

\begin{proposition}\label{crandall}
Let $F \in L^\infty(M, \Omega^{d - 1})$ satisfy $\dif F = 0$. Then:
\begin{enumerate}
\item The local comass $L(F, \cdot)$ is upper semicontinuous. \label{crandall usc}
\item For almost every $x \in M$,
$$|F(x)| \leq L(F, x).$$
\label{crandall LDT}
\item The local comass is bounded, and \label{crandall linfinity}
$$L(F) = \sup_{x \in M} L(F, x) = \|F\|_{L^\infty}.$$
\item If $\sigma \in \Chain_{d - 1}(M)$ then \label{crandall best curl is ABC}
$$\frac{1}{|\sigma|} \int_\sigma F \leq \sup_{x \in \sigma} L(F, x).$$
\end{enumerate}
\end{proposition}
\begin{proof}
We first prove (\ref{crandall usc}).
Let $x^n \to x$ and $r > 0$. Then eventually $x^n \in B_r(x)$, hence $L(F, x^n) \leq L_{B_r(x)}(F)$ and so
\begin{align*}
\limsup_{n \to \infty} L(F, x^n) &\leq \inf_{r > 0} L_{B_r(x)}(F) = L(F, x).
\end{align*}

We now prove (\ref{crandall LDT}).
We may work locally, and choose coordinates $(y^i)$ in which $\sqrt{\det g} = 1$.
Let $I$ be the increasing $d-1$-index with $d$ removed.
By the Lebesgue differentiation theorem and Fubini's theorem, there exists a null set $Z \subset M$, which does not depend on $(y^i)$ by Proposition \ref{invariance of LDT}, such that for every $x \notin Z$,
\begin{align*}
F_I(x) 
&= \lim_{\varepsilon \to 0} \frac{1}{|B_\varepsilon(x)|} \int_{B_\varepsilon(x)} F_I(y) \dif y \\
&= \lim_{\varepsilon \to 0} \frac{1}{|B_\varepsilon(x)|} \int_{-\infty}^\infty \int_{\{y^d = t\} \cap B_\varepsilon(x)} F_I(y) \dif y^1 \cdots \dif y^{d - 1} \dif t
\end{align*}
where we used the fact that $\sqrt{\det g} = 1$.
Now $\partial_{y^1}, \dots, \partial_{y^{d - 1}}$ are tangent to $\{y^d = t\}$, so as forms on $\{y^d = t\}$,
$$F_I(y) \dif y^1 \cdots \dif y^{d - 1} = F.$$
So
\begin{align*}
F_I(x) 
&= \lim_{\varepsilon \to 0} \frac{1}{|B_\varepsilon(x)|} \int_{-\infty}^\infty \int_{\{y^d = t\} \cap B_\varepsilon(x)} F \dif t \\
&\leq \lim_{\varepsilon \to 0} \frac{L_{B_\varepsilon(x)}(F)}{|B_\varepsilon(x)|} \int_{-\infty}^\infty |\{y^d = t\} \cap B_\varepsilon(x)| \dif t.
\end{align*}
By Fubini's theorem,
$$F_I(x) \leq \lim_{\varepsilon \to 0} \frac{L_{B_\varepsilon(x)}(F)}{|B_\varepsilon(x)|} |B_\varepsilon(x)| = L(F, x).$$
For every $x \in M$ we may select coordinates in which $|F(x)| = F_I(x)$, and then if $x \notin Z$, we conclude that (\ref{crandall LDT}) holds for $x$.

If we combine (\ref{crandall LDT}) with (\ref{integral over chain is linfinity}), then
$$\sup_{x \in M} L(F, x) \leq L(F) \leq \|F\|_{L^\infty} \leq \sup_{x \in M} L(F, x).$$
The inequalities collapse, proving (\ref{crandall linfinity}).
In particular, for each $\sigma \in \Chain_{d - 1}(M)$, we obtain (\ref{crandall best curl is ABC}):
\begin{align*}
\frac{1}{|\sigma|} \int_\sigma F &\leq \inf_{\Omega \supset \sigma} \sup_{x \in \Omega} L(F, x) = \sup_{x \in \sigma} L(F, x). \qedhere
\end{align*}
\end{proof}

%%%%%%%%%%%%%%%%%%%%%%%
\subsection{\texorpdfstring{$L^\infty$}{L-infinity} calibrations}
In our formulation, it is more natural to study calibrations which are merely $L^\infty$ rather than $C^0$, and require that the comass $L(F) \leq 1$.
By Proposition \ref{integration is welldefined}, it is still meaningful to ask if a $d-1$-surface is $F$-calibrated, even if $F$ is discontinuous.
We extend this definition to laminations:

\begin{definition}
Let $\lambda$ be a lamination and $F$ a calibration. We say that $\lambda$ is \dfn{$F$-calibrated} if every leaf of $\lambda$ is $F$-calibrated.
\end{definition}

It is clear from (\ref{calibrated surfaces are minimal}) that a $F$-calibrated lamination is minimal.
Moreover, if $M$ is closed and $F$ is a closed $d-1$ form, then the quantity $\int_M T_\lambda \wedge F$ is well-defined, since it is just $\langle [F], [\lambda]\rangle$.

\begin{proposition}\label{calibration condition}
Let $F$ be a calibration on a closed Riemannian manifold $M$.
Let $T_\lambda$ be the Ruelle-Sullivan current of a measured oriented lamination $\lambda$, and suppose that 
\begin{equation}\label{calibration by Ruelle Sullivan}
\int_M T_\lambda \wedge F = |\lambda|.
\end{equation}
Then $\lambda$ is $F$-calibrated (and in particular minimal).
\end{proposition}
\begin{proof}
Let $(\chi_\alpha)$ be a locally finite partition of unity subordinate to an open cover $(U_\alpha)$ of flow boxes for $\lambda$, and let $(\mu_\alpha)$ be the transverse measure.
After refining $(U_\alpha)$ we may assume that $U_\alpha$ is contractible, and after shrinking $U_\alpha$ we may assume that $\chi_\alpha > 0$ on $U_\alpha$.
Then for some hypersurfaces $\sigma_{\alpha,k}$,
$$|\lambda| = \int_M T_\lambda \wedge F = \sum_\alpha \int_I \int_{\sigma_{\alpha,k}} \chi_\alpha F \dif \mu_\alpha(k).$$
Let $\dif S_{\alpha,k}$ be the surface measure on $\sigma_{\alpha,k}$. Then
$$\int_M \chi_\alpha \star |T_\lambda| = \int_I \int_{\sigma_{\alpha,k}} \chi_\alpha \dif S_{\alpha,k} \dif \mu_\alpha(k),$$
so summing in $\alpha$, we obtain 
\begin{equation}\label{calibration condition contr}
\sum_\alpha \int_I \int_{\sigma_{\alpha,k}} \chi_\alpha F \dif \mu_\alpha(k) = |\lambda| = \sum_\alpha \int_I \int_{\sigma_{\alpha,k}} \chi_\alpha \dif S_{\alpha,k} \dif \mu_\alpha(k).
\end{equation}

We claim that $\lambda$ is \dfn{almost calibrated} in the sense that for every $\alpha$ and $\mu_\alpha$-almost every $k$, $\sigma_{\alpha, k}$ is calibrated.
If this is not true, then we may select $\beta$ and $K \subseteq I$ with $\mu_\beta(K) > 0$, such that for every $k \in K$, $\int_{\sigma_{\beta, k}} F < |\sigma_{\beta, k}|$.
Since $0 < \chi_\beta \leq 1$ and $F/\dif S_{\beta, k} \leq 1$ on $\sigma_{\beta, k}$, this is only possible if 
$$\int_{\sigma_{\beta, k}} \chi_\beta F < \int_{\sigma_{\beta, k}} \chi_\beta \dif S_{\beta, k}.$$
Integrating over $K$, and using the fact that in general we have $\int_{\sigma_{\alpha, k}} \chi_\alpha F \leq \int_{\sigma_{\alpha, k}} \chi_\alpha \dif S_{\alpha, k}$, we conclude that 
$$\sum_\alpha \int_I \int_{\sigma_{\alpha, k}} \chi_\alpha F \dif \mu_\alpha(k) < \sum_\alpha \int_I \int_{\sigma_{\alpha, k}} \chi_\alpha \dif S_{\alpha, k} \dif \mu_\alpha(k)$$
which contradicts (\ref{calibration condition contr}).

To upgrade $\lambda$ from an almost calibrated lamination to a calibrated lamination, given $\sigma_{\alpha, k}$ we choose $k_j$ such that $\sigma_{\alpha, k_j}$ is calibrated and $k_j \to k$.
Since we may write $F = \dif A$ near $\sigma_{\alpha, k}$ for some $A$ in Coulomb gauge, and 
$$\int_{\sigma_{\alpha, k}} F = \int_{\partial \sigma_{\alpha, k}} A,$$
the facts that $k_j \to k$ and that $A$ is continuous by (\ref{Sobolev}) imply 
\begin{align*}
|\sigma_{\alpha, k}| &= \lim_{j \to \infty} |\sigma_{\alpha, k_j}| = \lim_{j \to \infty} \int_{\sigma_{\alpha, k_j}} F = \lim_{j \to \infty} \int_{\partial \sigma_{\alpha, k_j}} A = \int_{\partial \sigma_{\alpha, k}} A = \int_{\sigma_{\alpha, k}} F. \qedhere 
\end{align*}
\end{proof}

\begin{definition}
Let $F$ be a form of best comass.
The \dfn{maximum comass locus} is the set
$$\MCL(F) := \{L(F, \cdot) = L(F)\}.$$
\end{definition}

By Proposition \ref{crandall}(\ref{crandall usc}), if $M$ is a closed manifold, then the maximum comass locus is a nonempty closed subset of $M$.

\begin{proposition}\label{properties of calibrated laminations}
Suppose that $M$ is a closed Riemannian manifold, $F$ is a calibration, and $\lambda$ is a measured oriented $F$-calibrated lamination.
Then:
\begin{enumerate}
\item $\lambda$ is minimal.
\item If $G$ is a calibration and cohomologous to $F$, then $\lambda$ is $G$-calibrated.
\item $\supp \lambda \subseteq \MCL(F)$.
\end{enumerate}
\end{proposition}
\begin{proof}
Every leaf of $\lambda$ is $F$-calibrated, hence minimal, so $\lambda$ is also minimal.
Moreover, (\ref{calibration by Ruelle Sullivan}) only depends on the cohomology class of $F$, not $F$ itself, so $\lambda$ is $G$-calibrated.
Finally, let $S$ be the maximum comass locus of $F$, $N$ a leaf of $\lambda$, and suppose that $x \in N \setminus S$.
Since $S$ is closed, there exists $\varepsilon > 0$ such that $B_\varepsilon(x)$ does not meet $S$.
Moreover, $\sigma := N \cap B_\varepsilon(x)$ is a $d-1$-chain in $B_\varepsilon(x)$, so by Proposition \ref{crandall}(\ref{crandall best curl is ABC}),
$$\frac{1}{|\sigma|} \int_\sigma F \leq \sup_{y \in B_\varepsilon(x)} L(F, y) < L = 1.$$
But then 
$$\int_N F = \int_\sigma F + \int_{N \setminus B_\varepsilon(x)} F < |\sigma| + |N \cap B_\varepsilon(x)| = |N|,$$
so $N$ (hence $\lambda$) is not $F$-calibrated.
\end{proof}


%%%%%%%%%%%%%%%%%%%%%%%%%%%%%
\section{\texorpdfstring{$\infty$-tight forms and the $1$-Laplacian}{Infinity-tight forms and the one-Laplacian}}\label{tight forms sec}
\subsection{Convex optimization}
We follow \cite{Ekeland99}.
For a reflexive Banach space $X$, we denote by $\hat X$ its dual.
If $I: X \to \RR \cup \{+\infty\}$ is a convex function, we introduce its \dfn{Legendre transform}, the convex function
\begin{align*}
	\hat I: \hat X &\to \RR \cup \{+\infty\}\\
	\xi &\mapsto \sup_{x \in X} \langle \xi, x\rangle - I(x).
\end{align*}
We identify the cokernel of a linear map $\Lambda$ with the kernel of its adjoint.
In this setting, we have the following form of the convex duality theorem.

\begin{theorem}[convex duality]\label{abstract convex analysis}
Let 
$$\Lambda : X \to Y$$
be a bounded linear map between reflexive Banach spaces.
Let $I: Y \to \RR \cup \{+\infty\}$ satisfy:
\begin{enumerate}
\item $I$ and $\hat I$ are strictly convex,
\item $I$ is lower semicontinuous,
\item if $|y| \to \infty$ in $Y$, then $I(y) \to +\infty$, and 
\item there exists a point $x \in X$ such that $I$ is continuous and finite at $\Lambda(x)$.
\end{enumerate}
Then:
\begin{enumerate}
\item There exists a minimizer $\underline x \in X$ of $I(\Lambda(x))$, unique modulo $\ker \Lambda$.
\item There exists a unique maximizer $\overline \eta$ of $-\hat I(-\eta)$ subject to the constraint $\eta \in \coker \Lambda$.
\item We have \dfn{strong duality}
\begin{equation}\label{abstract strong duality}
I(\Lambda(\underline x)) = -\hat I(-\overline \eta).
\end{equation}
\end{enumerate}
\end{theorem}
\begin{proof}
This is largely a special case of \cite[Chapter IV, Theorem 4.2]{Ekeland99}.
Let $\mathscr P, \mathscr P^*$ be as in the statement of that theorem.
Then $\mathscr P$ is the problem of minimizing $J(x, \Lambda x)$ where $J(x, y) := I(y)$.
The Legendre transform of $J$ satisfies 
$$\hat J(\xi, \eta) = \begin{cases} \hat I(\eta), & \xi = 0, \\
	+\infty, &\xi \neq 0,
\end{cases}$$
and $\mathscr P^*$ is the problem of maximizing
$$-\hat J(\Lambda^* \eta, -\eta) = \begin{cases}
	-\hat I(-\eta), &\eta \in \ker \Lambda^*, \\
	-\infty, &\eta \notin \ker \Lambda^*,
\end{cases}$$
where $\Lambda^*$ is the adjoint of $\Lambda$.
Then most of the various assertions of this theorem follow immediately from \cite[Chapter IV, Theorem 4.2]{Ekeland99}.
The fact that $\overline \eta \in \coker \Lambda$ follows from the facts that $\overline \eta$ is a solution of $\mathscr P^*$, but any solution of $\mathscr P^*$ must be a member of $\ker \Lambda^*$. 
To establish uniqueness, we use \cite[Chapter II, Proposition 1.2]{Ekeland99}, the fact that $\hat I$ is strictly convex, and the fact that we may view $I \circ \Lambda$ as a strictly convex function on the reflexive Banach space $X/\ker \Lambda$.
\end{proof}

One of the most famous corollaries of Theorem \ref{abstract convex analysis} is the \dfn{max flow min cut theorem} of discrete convex optimization \cite[Chapter 7]{umesh2006algorithms}, and we sometimes call consequences of Theorem \ref{abstract convex analysis} ``max flow min cut theorems,'' even if they are not combinatorial in nature.

%%%%%%%%%%%%%%%

\subsection{Max flow min cut for the \texorpdfstring{$q$-Laplacian}{q-Laplacian}}
We now turn to the problem at hand.
Let $M$ be a closed oriented Riemannian manifold with fundamental group $\Gamma$ and universal covering $\tilde M \to M$, and let $M_{\rm fun} \subset \tilde M$ be a fundamental domain of $\Gamma$.
By Poincar\'e duality and the Hurcewiz theorem, we have canonical isomorphisms
\begin{equation}\label{Poincare Hurcewiz}
H_{d - 1}(M, \RR) = H^1(M, \RR) = \Hom(\Gamma, \RR).
\end{equation}
Sometimes we are interested specifically in integral representations; in that case, we can use the fact that $\Sph^1$ is homotopic to $K(\ZZ, 1)$ to further identify 
\begin{equation}\label{Poincare Hurcewiz 2}
H^1(M, \ZZ) = \Hom(\Gamma, \ZZ) = [M, \Sph^1].
\end{equation}

Given a representation
$$\alpha: \Gamma \to \RR,$$
which we always identify with some smooth $1$-form (which we also call $\alpha$) representing the cohomology class corresponding to the representation $\alpha$, and $q \in (1, \infty)$,
we are interested in the $q$-Laplace equation
$$\dif^* (|\dif u|^{q - 2} \dif u) = 0$$
for an $\alpha$-equivariant function $u$ (thus for any $\gamma \in \Gamma$, $u(\gamma(x)) - u(x) = \alpha(\gamma)$).

Let $X$ be the space of locally $W^{1, q}(\tilde M)$ functions $u$ which are $\Gamma$-equivariant in the sense that for $\gamma \in \Gamma$, $\gamma^* \dif u = \dif u$, let $Y := L^q(M, \Omega^1)$, and let $p$ be the H\"older conjugate exponent to $q$.
Then $\Lambda := (\dif: X \to Y)$ is a bounded linear map, whose cokernel can be identified with the kernel of
$$\dif: L^p(M, \Omega^{d - 1}) \to W^{-1, p}(M)$$
using the perfect pairing 
\begin{align*}
	L^p(M, \Omega^{d - 1}) \times Y &\to \RR \\
	(F, \varphi) &\mapsto \int_M \varphi \wedge F 
\end{align*}
to identify $\hat Y$ with $L^p(M, \Omega^{d - 1})$.

One can easily show that the $q$-Laplace equation is the Euler-Lagrange equation of $I \circ \dif$, where $I: Y \to \RR \cup \{+\infty\}$ is defined to be 
$$I(\varphi) := \frac{1}{q} \int_M \star |\varphi|^q$$
if $\varphi$ is cohomologous to $\alpha$, and otherwise 
$$I(\varphi) := +\infty.$$
Using the methods of \cite[Chapter I, \S4]{Ekeland99} one can show that
$$\hat I(F) = \frac{1}{p} \int_M \star |F|^p + \int_M \alpha \wedge F$$
defined for $F \in \coker \Lambda$, and the dual problem to the $q$-Laplacian is the problem of maximizing $-\hat I(-F)$.

\begin{proposition}[max flow min cut for the $q$-Laplacian]
Given a representation $\alpha: \Gamma \to \RR$, and H\"older conjugate exponents $1 < p, q < \infty$, there exists an $\alpha$-equivariant $q$-harmonic function $u: \tilde M \to \RR$, unique modulo constants, and a unique minimizer $F$ of 
$$J_{p, \alpha}(F) := \frac{1}{p} \int_M \star |F|^p - \int_M \alpha \wedge F$$
among all closed $d - 1$-forms on $M$.
Moreover, we have
\begin{equation}\label{strong duality}
	\frac{1}{q} \int_M \star |\dif u|^q + \frac{1}{p} \int_M \star |F|^p = \int_M \dif u \wedge F.
\end{equation}
\end{proposition}
\begin{proof}
By \cite[Lemma 1]{Loisel_2020}, $I$ and $\hat I$ are both strictly convex.
Moreover, $\coker \Lambda$ is the space of closed $L^p$ $d - 1$-forms on $M$, and $\ker \Lambda$ is the space of constant functions on $\tilde M$.
So the various assertions of this proposition mostly follow from Theorem \ref{abstract convex analysis} and the above discussion.
\end{proof}

Motivated by \cite[\S3.1]{daskalopoulos2020transverse}, it is natural to guess that 
\begin{equation}\label{dual solution}
F := |\dif u|^{q - 2} \star \dif u
\end{equation}
is the solution of the dual problem of minimizing $J_{p, \alpha}$.
In order to prove that this is true, we shall need that for H\"older conjugates $1 \leq p, q \leq \infty$,
\begin{equation}\label{holder cancellation}
	(p - 2)(q - 1) + (q - 2) = 0.
\end{equation}

\begin{lemma}
Suppose that $u: \tilde M \to \RR$ is an $\alpha$-equivariant $q$-harmonic function, and suppose that $F$ satisfies (\ref{dual solution}).
Then $F$ is a closed $d - 1$-form, which minimizes $J_{p, \alpha}$ among all closed $d - 1$-forms.
Moreover, $F$ solves the PDE 
\begin{equation}\label{pMaxwell}
\begin{cases}
	\dif F = 0 \\
	\dif^* (|F|^{p - 2} F) = 0.
\end{cases}
\end{equation}
\end{lemma}
\begin{proof}
We first show that $\dif F = 0$.
In fact, 
$$\star \dif F = \star \dif(|\dif u|^{q - 2} \star \dif u) = \pm \dif^*(|\dif u|^{q - 2} \dif u) = 0.$$
So by the max flow min cut principle, to prove that $F$ is a minimizer, it suffices to show that (\ref{strong duality}) holds.
One can easily compute 
$$|F|^p = |\dif u|^{(q - 1)p} = |\dif u|^q,$$
so by Stokes' theorem and the fact that $\alpha$ is cohomologous to $\dif u$,
\begin{align*}
\frac{1}{q} \int_M \star |\dif u|^q + \frac{1}{p} \int_M \star |F|^p&
= \left[\frac{1}{p} + \frac{1}{q}\right] \int_M \star |\dif u|^q
= \int_M \dif u \wedge |\dif u|^{q - 2} \star \dif u \\
&= \int_M \alpha \wedge F.
\end{align*}
Finally, we use (\ref{holder cancellation}) to prove
\begin{align*}
\dif^*(|F|^{p - 2} F) &= \dif^*(|\dif u|^{(p - 2)(q - 1)} |\dif u|^{q - 2} \star \dif u) = \dif^*(\star \dif u) \\
&= \pm \star \dif^2 u = 0. \qedhere 
\end{align*}
\end{proof}

%%%%%%%%%%%%%%%%

\subsection{\texorpdfstring{$p$-tight forms}{p-tight forms}}
We next scrutinize the PDE (\ref{pMaxwell}).
If $p = 2$ and $d = 3$, then these equations are those satisfied by a steady-state electromagnetic tensor in vacuum.
On the other hand, as $p \to \infty$, the solutions of these equations converge to a minimizer of the comass.
This motivates the below terminology:

\begin{definition}
Let $1 < p < \infty$.
We call the equation (\ref{pMaxwell}) the \dfn{$p$-Maxwell equation}.
A \dfn{$p$-tight form} is a solution of the $p$-Maxwell equation.
\end{definition}

\begin{proposition}
Suppose that $M$ is a closed oriented Riemannian manifold.
Then there is a unique $p$-tight form in each cohomology class in $H^{d - 1}(M, \RR)$.
Moreover, $p$-tight forms are minimizers of the strictly convex functional
$$J_p(F) := \int_M \star |F|^p$$
among all forms cohomologous to them.
\end{proposition}
\begin{proof}
Strict convexity of $J_p$ on a cohomology class follows from an argument similar to \cite[Lemma 1]{Loisel_2020} and the existence and uniqueness of a minimizer then is a consequence of the direct method of the calculus of variations as in \cite[Chapter II]{Ekeland99}.
To compute the Euler-Lagrange equations for $J_p$, let $B$ be a $d-2$-form (so $F + t \dif B$ is cohomologous to $F$), so that for a minimizer $F$ of $J_p$,
$$\frac{\dif}{\dif t} J_p(F + t \dif B) = \frac{1}{p} \int_M \star \frac{\partial}{\partial t} |F + t \dif B|^p = \int_M \star |F + t \dif B|^{p - 2} \langle F + t \dif B, \dif B\rangle.$$
Setting $t = 0$, we obtain 
$$0 = \int_M \star |F|^{p - 2} \langle F, \dif B\rangle = \int_M \star \langle \dif^*(|F|^{p - 2} F), B\rangle.$$
This equation holds for every $d - 2$-form $B$.
Thus the Euler-Lagrange equations for $J_p$ are (\ref{pMaxwell}).
\end{proof}

\begin{definition}
Let $F$ be a $p$-tight form, let
\begin{equation}
\dif u := (-1)^{d - 1} |F|^{p - 2} \star F, \label{inverse extremality}
\end{equation}
and let $u$ be the primitive of $\dif u$ on the universal cover $\tilde M$, which is normalized to have zero mean on a fundamental domain $M_{\rm fun}$.
Then $u$ is called the \dfn{$q$-harmonic conjugate} of the $p$-tight form $F$, where $\frac{1}{p} + \frac{1}{q} = 1$.
\end{definition}

Let $u$ be the $q$-harmonic conjugate of $F$.
By Poincar\'e's inequality,
$$\|u\|_{W^{1, q}(M_{\rm fun})}^q \lesssim \int_M \star |\dif u|^q = \int_M \star |F|^{(p - 1)q} = \int_M \star |F|^p < \infty$$
since $F$ is $p$-tight; that is, we have $F \in L^p$ and $u \in W^{1, q}$.

We derived the $p$-Maxwell equation as the dual equation to the $q$-Laplace equation.
We now assert that this process can be inverted.

\begin{lemma}
Let $1 < p, q < \infty$ and $\frac{1}{p} + \frac{1}{q} = 1$.
Let $F$ be a $p$-tight form, and let $u$ be its $q$-harmonic conjugate. Then:
\begin{enumerate}
\item $u$ is $q$-harmonic.
\item One has 
\begin{equation}
F = |\dif u|^{q - 2} \star \dif u. \label{extremality} \\
\end{equation}
\item One has the duality formula (\ref{strong duality}).
\end{enumerate}
\end{lemma}
\begin{proof}
We first use (\ref{holder cancellation}) to prove
$$|\dif u|^{q - 2} \star \dif u = (-1)^{d - 1} |F|^{(q - 2)(p - 1)} \star \star |F|^{p - 2} F = |F|^{(q - 2)(p - 1) - (p - 2)} F = F.$$
Thus we have (\ref{extremality}), and moreover
$$\dif \star (|\dif u|^{q - 2} \dif u) = \dif F = 0$$
so that $u$ is $q$-harmonic.
We then obtain (\ref{strong duality}) as before.
\end{proof}

\begin{corollary}
Every $p$-tight form is locally H\"older continuous.
\end{corollary}
\begin{proof}
Let $F$ be $p$-tight and let $u$ be its $q$-harmonic conjugate.
By \cite[Theorem 2]{DIBENEDETTO1983827}, $\dif u$ is H\"older continuous.
The claim now follows from (\ref{extremality}) and the fact that a product of H\"older continuous functions is H\"older continuous.
\end{proof}


%%%%%%%%%%%%%%%%%%%%%%%
\subsection{\texorpdfstring{Existence of $\infty$-tight forms}{Existence of infinity-tight forms}}
We now take the limit $p \to \infty$ to obtain a privileged form of best comass.
To do so, we shall need the $p$-tight forms to be uniformly bounded in the following sense.

\begin{lemma}
Let $F_p$ be a $p$-tight form, and let $B$ range over closed $d - 1$-forms cohomologous to $F_p$. Then
\begin{equation}\label{infinity magnetic rules p magnetic}
	\|F_p\|_{L^p} \leq |M|^{1/p} \inf_B \|B\|_{L^\infty}.
\end{equation}
\end{lemma}
\begin{proof}
By H\"older's inequality and the fact that $F_p$ is $p$-tight,
$$\|F_p\|_{L^p} \leq \|B\|_{L^p} \leq |M|^{1/p} \|B\|_{L^\infty},$$
hence the same holds for the infimum.
\end{proof}

\begin{proposition}\label{existence infinity}
Let $\rho \in H^{d - 1}(M, \RR)$.
For each $p \geq 2$, let $F_p$ be the $p$-tight form representing $\rho$. Then there exists a closed $d - 1$-form $F$ such that:
\begin{enumerate}
\item $F_p \to F$ weakly in $L^r$ along a subsequence, for any $d < r < \infty$.
\item $F$ is a best comass representative of $\rho$.
\end{enumerate}
\end{proposition}
\begin{proof}
We roughly follow \cite[\S3]{Lindqvist14}.
Let $r > d$, and let $B$ be an $L^\infty$ representative of $\rho$.
By H\"older's inequality and (\ref{infinity magnetic rules p magnetic}),
\begin{equation}\label{uniform bounds in p by best curl}
	\|F_p\|_{L^r} \leq |M|^{\frac{1}{r} - \frac{1}{p}} \|F_p\|_{L^p} \leq |M|^{\frac{1}{r}} \|B\|_{L^\infty}.
\end{equation}
Thus a compactness argument gives $F_p \to F$ for some $d - 1$-form $F$, weakly in $L^r$, and 
$$\|F\|_{L^r} \leq \liminf_{p \to \infty} \|F_p\|_{L^r} \leq |M|^{\frac{1}{r}} \|B\|_{L^\infty}.$$
Diagonalizing, we may assume that $F_p \to F$ weakly in $L^r$ for every such $r$, and taking $r \to \infty$, we conclude 
\begin{equation}\label{infinity magnetics have best curl}
	\|F\|_{L^\infty} \leq \|B\|_{L^\infty}.
\end{equation}
Moreover, $[F] = \lim_{p \to \infty} [F_p] = \rho$.
So by Proposition \ref{crandall}(\ref{crandall linfinity}) and the fact that $B$ was arbitrary in (\ref{infinity magnetics have best curl}), $F$ has best comass.
\end{proof}

\begin{definition}
The $d - 1$-form $F$ of best comass in Proposition \ref{existence infinity} is called an \dfn{$\infty$-tight form}, or simply a \dfn{tight form}.
\end{definition}

The existence of $\infty$-tight (or even just best comass) representatives of each cohomology class implies the following useful lemma.

\begin{lemma}\label{p tights approximate L}
Let $F_p$ be the $p$-tight representative of $\rho$, and $L$ the best comass of $\rho$. Then 
$$\lim_{p \to \infty} \|F_p\|_{L^p} = L.$$
\end{lemma}
\begin{proof}
We follow \cite[Lemma 2.7]{daskalopoulos2020transverse}.
Let $F$ be an $\infty$-tight representative of $\rho$, so by Proposition \ref{crandall}(\ref{crandall linfinity}), $\|F\|_{L^\infty} = L$.
Since $F_p$ is $p$-tight, H\"older's inequality implies 
$$\|F_p\|_{L^p} \leq \|F\|_{L^p} \leq |M|^{\frac{1}{p}} L.$$
Therefore 
$$\limsup_{p \to \infty} \|F_p\|_{L^p} \leq L.$$
To prove the converse, suppose that 
$$\liminf_{p \to \infty} \|F_p\|_{L^p} \leq \tilde L < L.$$
Along a subsequence which attains the limit inferior, $F_p$ converges weakly in every $L^r$, $d < r < \infty$, to an $\infty$-tight form $\tilde F$ such that (by H\"older's inequality)
$$\|\tilde F\|_{L^r} \leq \liminf_{p \to \infty} \|F_p\|_{L^r} \leq \liminf_{p \to \infty} |M|^{\frac{1}{r}} \|\tilde F\|_{L^\infty} \leq |M|^{\frac{1}{r}} \tilde L.$$
Taking $r \to \infty$, we obtain $L(\tilde F) < L$, which contradicts the fact that $L$ is the best comass.
\end{proof}


%%%%%%%%%%%%%%%%%%%%
\subsection{\texorpdfstring{$1$-harmonic conjugates of $\infty$-tight forms}{One-harmonic conjugates of infinity-tight forms}}
We now construct the $1$-harmonic conjugate of an $\infty$-tight form.
Since (\ref{inverse extremality}) may blow up as $p \to \infty$, we have to renormalize the $q$-harmonic conjugates of $p$-tight forms before taking the limit $q \to 1$, as in \cite[\S3.2]{daskalopoulos2020transverse}.

We begin by showing that $L^1$ convergence preserves the equivariance properties of functions.
To make this precise, let $\tilde M \to M$ be the universal cover of $M$.
Following \cite[\S4]{daskalopoulos2020transverse}, we use the Hurcewiz theorem to identify $1$-dimensional representations
$$\alpha: \pi_1(M) \to \RR$$
of the fundamental group with cohomology classes $H^1(M, \RR)$.
If we have an $\alpha$-equivariant function $u$ on $\tilde M$, thus for every $\gamma \in \pi_1(M)$,
$$\gamma^* u = u + \langle \alpha, \gamma\rangle,$$
we write $[u] = \alpha$.
Taking derivatives, we see that $\gamma^* \dif u = \dif u$, so $\dif u$ drops to a closed $1$-form (or perhaps better a closed $d-1$-current) on $M$ whose cohomology class is $\alpha$.

\begin{lemma}\label{L1 convergence preserves pi1}
Let $\tilde M \to M$ be the universal cover, and let $(u_q)$ be a sequence of $\pi_1(M)$-equivariant functions on $\tilde M$ which converge in $L^1_\loc(\tilde M)$ to a function $u$ as $q \to 1$.
Then $u$ is $\pi_1(M)$-equivariant, and $[u_q] \to [u]$.
Moreover, if $\dif u_q \to \dif u$ in the weak topology of measures on $M$ and $\dif u_q \in L^q$, then
\begin{equation}\label{q to 1 Holder}
\|\dif u\|_{TV} \leq \liminf_{q \to 1} \frac{1}{q} \int_M \star |\dif u_q|^q.
\end{equation}
\end{lemma}
\begin{proof}
Since $u_q$ is $\pi_1(M)$-equivariant, there exists $\alpha_q \in H^1(M, \RR)$ such that for every $\gamma \in \pi_1(M)$,
\begin{equation}\label{equivariance q}
	\gamma^* u_q = u_q + \langle \alpha_q, \gamma\rangle.
\end{equation}
Let $M_{\rm fun}$ be a fundamental domain and $U_\gamma := M_{\rm fun} \cup \gamma_* (M_{\rm fun})$.

We claim that $(\alpha_q)$ has a convergent subsequence.
To see this, we first recall that $M$ has finite Betti numbers, so $H^1(M, \RR)$ is locally compact.
Therefore, if no convergent subsequence exists, there exists a $\gamma \in \pi_1(M)$ and a subsequence along which $\langle \alpha_q, \gamma\rangle \to \infty$.
Moreover, since $u_q \to u$ in $L^1_\loc$, $\|u_q\|_{L^1(M_{\rm fun})} \leq 2\|u\|_{L^1(M_{\rm fun})}$ if $q - 1$ is small enough.
But then 
$$\|u_q\|_{L^1(\gamma_* M_{\rm fun})} = \|\gamma^* u_q\|_{L^1(M_{\rm fun})} \geq \langle \alpha_q, \gamma\rangle - \|u_q\|_{L^1(M_{\rm fun})} \geq \langle \alpha_q, \gamma\rangle - 2\|u\|_{L^1(M_{\rm fun})}$$
and taking $q \to 1$ we conclude that $(u_q)$ is not compact in $L^1(\gamma_* M_{\rm fun})$, contradicting the convergence in $L^1_\loc(\tilde M)$.
So $\alpha_q \to \alpha$ for some $\alpha \in H^1(M, \RR)$ along a subsequence.

For any $q > 1$,
\begin{align*}
\dashint_{M_{\rm fun}} \star |\gamma^* u - u - \langle \alpha, \gamma\rangle| 
&\leq \dashint_{M_{\rm fun}} \star (|\gamma^* u_q - u_q - \langle \alpha_q, \gamma\rangle| + |\gamma^* u_q - u_q| + |\gamma^* u - u|) \\
&\qquad + |\langle \alpha_q - \alpha, \gamma\rangle|.
\end{align*}
Taking $q \to 1$ and applying (\ref{equivariance q}), we conclude that $\|\gamma^* u - u - \langle \alpha, \gamma\rangle\|_{L^1} = 0$, hence $u$ is $\alpha$-equivariant.
Thus $\alpha$ is uniquely defined and $\alpha_q \to \alpha$ along the entire subsequence.

Finally we prove (\ref{q to 1 Holder}).
Suppose that $\dif u_q \to \dif u$ in the weak topology of measures and $\dif u_q$ in $L^q$.
Under those hypotheses, we may use the portmanteau theorem and H\"older's inequality to estimate (where $\frac{1}{p} + \frac{1}{q} = 1$)
\begin{align*}
\|\dif u\|_{TV} &= \lim_{q \to 1} \|\dif u_q\|_{L^1} \leq \lim_{q \to 1} |M|^{\frac{1}{p}} \|\dif u_q\|_{L^q} = \lim_{q \to 1} \frac{1}{q} \int_M \star |\dif u_q|^q. \qedhere
\end{align*}
\end{proof}

Next we address the renormalization.
Suppose that $\rho \in H^{d - 1}(M, \RR)$ and denote by $L$ the comass of a best comass representative of $\rho$.
Also let $k_p$ be defined by 
$$k_p^{1 - p} = \int_M \star |F_p|^p$$
where $F_p$ is the $p$-tight representative of $\rho$.

\begin{definition}
The \dfn{renormalized $q$-harmonic conjugate} of a $p$-tight form $F_p$ is the function $u_q: \tilde M \to \RR$ which has mean zero on $M_{\rm fun}$ and solves
$$\dif u_q = (-1)^{d - 1} k_p^{p - 1} |F_p|^{p - 2} \star F_p.$$
\end{definition}

\begin{lemma}\label{normalizations converge}
As $p \to \infty$, $k_p \to 1/L$.
\end{lemma}
\begin{proof}
We follow \cite[Lemma 3.4]{daskalopoulos2020transverse}.
By Lemma \ref{p tights approximate L},
$$\lim_{p \to \infty} k_p^{-\frac{1}{q}} = \lim_{p \to \infty} \|F_p\|_{L^p} = L.$$
Taking logarithms we see that $q^{-1} \log k_p \to -\log L$, and since $q \to 1$ the claim follows.
\end{proof}

\begin{proposition}\label{existence 1}
Let $\rho \in H^{d - 1}(M, \RR)$ and let $\tilde M \to M$ be the universal cover.
For $2 < p < \infty$ and $\frac{1}{p} + \frac{1}{q} = 1$, let $u_q$ be the renormalized $q$-harmonic conjugate of the $p$-tight representative of $\rho$.
Then there exists a $\pi_1(M)$-equivariant function $u \in BV_\loc(\tilde M)$ such that:
\begin{enumerate}
\item $u$ is $1$-harmonic.
\item As $q \to 1$ along a subsequence, $u_q \to u$ weakly in $BV_\loc(\tilde M)$ and strongly in $L^r_\loc(\tilde M)$ for $1 \leq r < \frac{d}{d - 1}$.
\item Let $F$ be the $\infty$-tight representative of $\rho$, with best comass $L$. We have the \dfn{max flow min cut principle} that, as Radon measures,
\begin{equation}\label{1 extremality}
\dif u \wedge F = L \star |\dif u|.
\end{equation}
\end{enumerate}
\end{proposition}
\begin{proof}
We first compute using H\"older's inequality and Lemma \ref{normalizations converge}
\begin{align*}
\lim_{q \to 1} \|\dif u_q\|_{L^1}
&\leq \lim_{q \to 1} |M|^{\frac{1}{p}} \left[\int_M \star |\dif u_q|^q\right]^{\frac{1}{q}} = \lim_{p \to \infty} \left[k_p^p \int_M \star |F_p|^p\right]^{\frac{1}{q}} \\
&= \lim_{p \to \infty} k_p^{\frac{1}{q}} = \lim_{p \to \infty} k_p = \frac{1}{L}.
\end{align*}
So by Rellich's theorem, $(u_q)$ is weakly compact in $BV$ and strongly compact in $L^r$ for $1 \leq r < \frac{d}{d - 1}$.
In particular, $\dif u_q \to \dif u$ in the weak topology of measures and $u_q \to u$ weakly in $BV$ and strongly in $L^r$.
As the limit of $\pi_1(M)$-equivariant functions, $u$ is also $\pi_1(M)$-equivariant by Lemma \ref{L1 convergence preserves pi1}.
In particular, $\dif u$ drops to a current on $M$.
Moreover, $[\dif u_q] \to [\dif u]$, and we have the bound (\ref{q to 1 Holder}) on $\int \star |\dif u|$.

Renormalizing (\ref{strong duality}), we obtain 
$$\frac{k_p^{-p}}{q} \int_M \star |\dif u_q|^q + \frac{1}{p} \int_M \star |F_p|^p = k_p^{1 - p} \int_M \dif u_q \wedge F_p.$$
Multiplying by $k_p^p$, we have 
\begin{equation}\label{1 strong duality before limits}
	\frac{1}{q} \int_M \star |\dif u_q|^q + \frac{k_p^p}{p} \int_M \star |F_p|^p = k_p \int_M \dif u_q \wedge F_p.
\end{equation}

Let $\mu$ be the total variation measure of $\dif u$.
We claim that as Radon measures,
\begin{equation}\label{1 strong duality}
	L\mu(M) \leq \int_M \dif u \wedge F.
\end{equation}
First, we have from (\ref{q to 1 Holder}) and (\ref{1 strong duality before limits}) that
$$\mu(M) \leq \lim_{q \to 1} \frac{1}{q} \int_M \star |\dif u_q|^q = \lim_{p \to \infty} k_p \int_M \dif u_q \wedge F_p - \lim_{p \to \infty} \frac{k_p^p}{p} \int_M \star |F_p|^p.$$
By Lemma \ref{normalizations converge},
$$\lim_{p \to \infty} \frac{k_p^p}{p} \int_M \star |F_p|^p = \lim_{p \to \infty} \frac{k_p}{p} = \frac{0}{L} = 0,$$
and
$$\lim_{p \to \infty} k_p \int_M \dif u_q \wedge F_p = \frac{1}{L} \lim_{p \to \infty} \int_M [\dif u_q] \wedge \rho.$$
Since $[\dif u_q] \to [\dif u]$, we obtain
$$\lim_{p \to \infty} \int_M [\dif u_q] \wedge \rho = \int_M \alpha \wedge \rho = \int_M \dif u \wedge F,$$
completing the proof of (\ref{1 strong duality}).

By the coarea formula (\ref{coarea formula}), we have for any open set $U$,
$$\int_U \dif u \wedge F = \int_{-\infty}^\infty \int_{U \cap \partial \{u > y\}} F \dif y \leq L \int_{-\infty}^\infty |U \cap \partial \{u > y\}| \dif y = L \mu(U).$$
Since $\mu$ is a finite Borel measure, every Borel set $E$ can be $\mu$-approximated from without open sets, hence
\begin{equation}\label{one sided extremality}
\int_E \dif u \wedge F \leq L \mu(E).
\end{equation}

Next we deduce (\ref{1 extremality}).
We reason by contradiction: if (\ref{1 extremality}) is false, then there exists an open set $U \subseteq M$ such that 
$$\int_U \dif u \wedge F < L \int_U \star |\dif u|.$$
(Indeed, strict inequality cannot point in the other direction, by (\ref{one sided extremality}).)
However, by (\ref{one sided extremality}), 
$$\int_{M \setminus U} \dif u \wedge F \leq L \int_{M \setminus U} \star |\dif u|.$$
Adding up the integrals of $\dif u \wedge F$ over $U$ and $M \setminus U$, we conclude 
$$\int_M \dif u \wedge F < L \int_M \star |\dif u|,$$
but this contradicts (\ref{1 strong duality}); thus (\ref{1 extremality}) must be true.

To round out the proof, let $X := (\star F/L)^\sharp$ be the Poincar\'e dual vector field to $F/L$. Then
$$\nabla \cdot X = \star \frac{\dif F}{L} = 0,$$
and $\|X\|_{L^\infty} \leq 1$.
Moreover, by (\ref{1 extremality}), $X$ is normal to the level sets of $u$, and hence is a witness that $u$ is $1$-harmonic.
\end{proof}

\begin{example}
Let $M$ be a closed surface equipped with a homotopy class $\alpha$ of maps $M \to \Sph^1$ with winding number $1$.
Using (\ref{Poincare Hurcewiz}) and (\ref{Poincare Hurcewiz 2}) we obtain a homology class of curves in $M$, from which we choose the geodesic representative $\gamma$.
Then the $1$-harmonic map $u$ in the homotopy class $\alpha$ lifts to a map $\tilde u: M_{\rm fun} \to \RR$ which is $0$ on one side of the lifted geodesic $\tilde \gamma$, and $1$ on the other side of $\tilde \gamma$.
Identifying $0, 1$ with the north pole of $\Sph^1$, we see that $u$ maps almost all of $M$ to the north pole of $\Sph^1$, but still has winding number $1$ as it wraps arbitrarily small neighborhoods of $\gamma$ around $\Sph^1$.
\end{example}



%%%%%%%%%%%%%%%%%%%%


\section{The maximum comass locus}\label{MCL sec}
Let $M$ be a closed Riemannian of dimension $2 \leq d \leq 4$ equipped with a cohomology class $\rho \in H^{d - 1}(M, \RR)$.
We shall study the set on which a best comass representative of $\rho$ attains its comass.
This set turns out to contain a measured oriented minimal lamination which only depends on $\rho$, and which is calibrated by any best comass form on $\rho$.

%%%%%%%%%%%%%%%%%%%%%
\subsection{Measured stretch laminations}
Let $u$ be a $\pi_1(M)$-equivariant $1$-harmonic function on $\tilde M$.
Then $\dif u$ drops to a $d-1$-current on $M$, which is still the Ruelle-Sullivan current of a measured oriented minimal lamination on $M$, which we still call $\lambda_u$.
Thus the following definition makes sense.

\begin{definition}
Let $\rho \in H^{d - 1}(M, \RR)$, let $F$ be an $\infty$-tight representative of $\rho$, and let $u$ be a $1$-harmonic conjugate of $F$.
Then we call $\lambda_u$ a \dfn{measured stretch lamination} associated to $\rho$.
\end{definition}

\begin{proposition}\label{MCL contains Thurston}
Let $F$ be a best comass representative of $\rho \in H^{d - 1}(M, \RR)$, let $L := L(F)$ be the best comass of $\rho$, and let $\lambda$ be a measured stretch lamination associated to $\rho$.
Then $F/L$ calibrates $\lambda$. In particular, the maximum comass locus of $F$ contains $\lambda$.
\end{proposition}
\begin{proof}
Let $G$ be the $\infty$-tight form which is cohomologous to $F$ whose dual $1$-harmonic function $u$ defines the measured stretch lamination $\lambda$.
Then by the max flow min cut principle (\ref{1 extremality}), 
$$|\lambda| = \|\dif u\|_{TV} = \frac{1}{L} \int_M \dif u \wedge G$$
so $G/L$ calibrates $\lambda$ by Proposition \ref{calibration condition}.
Then by Proposition \ref{properties of calibrated laminations}, $F/L$ calibrates $\lambda$, and the maximum comass locus of $F/L$ contains $\lambda$.
However, the maximum comass locus is preserved by rescaling.
\end{proof}

\begin{proposition}\label{L equals K}
	Let $\rho \in H^{d - 1}(M, \RR)$ have measured stretch lamination $\mu$ and best comass constant $L$, and let $\lambda$ range over measured oriented laminations. Then 
	$$L = \sup_\lambda \frac{\langle \rho, [\lambda]\rangle}{|\lambda|} = \frac{\langle \rho, [\mu]\rangle}{|\mu|}.$$
\end{proposition}
\begin{proof}
Fix the $\infty$-tight form $F$ representing $\rho$, and let $u$ be its $1$-harmonic conjugate.

We first prove $K \leq L$.
Let $\lambda$ be a measured oriented lamination; then, since $F$ represents $\rho$ and the Ruelle-Sullivan current $T_\lambda$ represents $[\lambda]$,
$$\langle \rho, [\lambda]\rangle = \int_M F \wedge T_\lambda.$$
Let $(\chi_\alpha)$ be a partition of unity subordinate to a laminar atlas for $\lambda$, and let $(\mu_\alpha)$ be the associated transverse measure. Then 
$$\int_M F \wedge T_\lambda = \sum_\alpha \int_I \int_{\{k\} \times J} \chi_\alpha F \dif \mu_\alpha(k).$$
Since $F$ has best comass,
$$\frac{\langle \rho, [\lambda] \rangle}{|\lambda|}
\leq \frac{\|F\|_{L^\infty}}{|\lambda|} \sum_\alpha \int_I \int_{\{k\} \times J} \chi_\alpha \dif S_k \dif \mu_\alpha(k) = L.$$
Since $\lambda$ was arbitrary, it holds that $K \leq L$.

By the max flow min cut principle (\ref{1 extremality}),
$$\langle \rho, [\mu]\rangle = \int_M F \wedge \dif u = L \|\dif u\|_{TV} = L|\mu|.$$
Dividing both sides by $|\mu|$ and applying the direction we already proved,
$$K \leq L \leq \frac{\langle \rho, [\mu]\rangle}{|\mu|} \leq K$$
which is only possible if $L = K$ and $\mu$ is a maximizer.
\end{proof}

We now interpret Proposition \ref{L equals K} in terms of duality of norms.
The fact that the dual norm to the best comass is the stable norm was already well-known \cite[Theorem 3.8]{AUER20011095} but it is interesting to note that it immediately follows from Proposition \ref{L equals K}.

\begin{definition}
The \dfn{stable norm} of a homology class $\alpha$ is
$$\|\alpha\|_s := \inf_{[\sigma] = \alpha} |\sigma|.$$
\end{definition}

\begin{corollary}
View $H_{d - 1}(M, \RR)$ and $H^{d - 1}(M, \RR)$ as Banach spaces, equipped with the stable norm and the best comass respectively.
Then $H_{d - 1}$ and $H^{d - 1}$ are mutually dual.
\end{corollary}
\begin{proof}
Since $H^{d - 1}(M, \RR)$ is finite-dimensional and therefore reflexive \cite[Theorem 1.13.5]{megginson1998introduction}, it suffices to show that the dual space to $H^{d - 1}(M, \RR)$ is $H_{d - 1}(M, \RR)$.
Since we may view the homologically area-minimizing representative of a homology class as a measured oriented minimal lamination, every homology class contains a measured oriented minimal lamination which is homologically area-minimizing.
Therefore, for any $\rho \in H^{d - 1}(M, \RR)$, we obtain from Proposition \ref{L equals K} that
\begin{align*}
L(\rho) &= \sup_\lambda \frac{\langle \rho, [\lambda] \rangle}{|\lambda|} = \sup_{\alpha \in H_{d - 1}(M, \RR)} \frac{\langle \rho, \alpha\rangle}{\|\alpha\|_s}. \qedhere 
\end{align*}
\end{proof}

\begin{corollary}\label{existence of calibrations}
Every closed area-minimizing hypersurface can be calibrated.
\end{corollary}
\begin{proof}
Let $\alpha$ be the homology class of a closed area-minimizing hypersurface $N$.
By duality between the stable norm and the best comass, and local compactness of $H^{d - 1}(M, \RR)$, there is a cohomology class $\rho \in H^{d - 1}(M, \RR)$ such that $L(\rho) = 1$ and, if $F$ is a best comass representative of $\rho$,
$$|N| = \|\alpha\|_s = \langle \rho, \alpha\rangle = \int_N F.$$
Therefore $F$ calibrates $N$.
\end{proof}


%%%%%%%%%%%%%%%%%
\subsection{The dual Ryu-Takayanagi formula}
In conformal field theory, one is often interested in the \emph{entanglement entropy} $S(N)$ through a closed oriented hypersurface $N \subset M$.
This quantity has a physical definition, but according to the Ryu-Takayanagi formula \cite{Ryu_2006}, $S(N)$ is the area of any homologically area-minimizing hypersurface homologous to $N$.
Owing to the conceptual difficulty of interpreting the Ryu-Takayanagi formula, Freedman and Headrick introduced a dual form of the Ryu-Takayanagi formula, motivated by the max flow min cut principle \cite{Freedman_2016}.
We would like to highlight the fact that the dual Ryu-Takayanagi formula actually follows immediately from Corollary \ref{existence of calibrations}.

\begin{corollary}[dual Ryu-Takayanagi formula]
For any closed oriented hypersurface $N \subset M$, the entanglement entropy through $N$ satisfies
\begin{equation}\label{dual RT formula}
S(N) = \max_{\substack{\dif F = 0 \\ L(F) \leq 1}} \int_N F.
\end{equation}
\end{corollary}
\begin{proof}
By Corollary \ref{existence of calibrations}, there exists a calibration $F$ of the area-minimizing hypersurface $N'$ homologous to $N$, and by Stokes' theorem, 
$$S(N) = |N'| = \int_{N'} F = \int_N F.$$
To see that $F$ is the maximum among all forms with $\dif F = 0$ and $L(F) \leq 1$, let $G$ be another such form; then
\begin{align*}
\int_N G &= \int_{N'} G \leq |N'| = S(N). \qedhere 
\end{align*}
\end{proof}






%%%%%%%%%%%%%%%%%%%%%%%%%%%%
% \subsection{Existence of an optimal best comass form}
% \todo{exposit this}

% \begin{definition}
% An \dfn{optimal best comass form} is a best comass form $F$ such that
% $$\MCL(F) = \bigcap_G \MCL(G)$$
% where $G$ ranges over best comass forms cohomologous to $F$.
% \end{definition}

% \begin{proposition}
% Let $\rho \in H^2(M, \RR)$.
% Then there exists an optimal best comass representative of $\rho$.
% \end{proposition}
% \begin{proof}
% Let $\lambda := \bigcap_G \MCL(G)$ where $G$ ranges over best comass representatives of $\rho$.
% For $x \notin \lambda$, we can find a best comass form $F_x$ of class $\rho$ such that $x \notin \MCL(F_x)$.
% In particular, $U_x := \{L(F_x, \cdot) < L\}$ is an open set which contains $x$, so $(U_x)_{x \notin \lambda}$ is an open cover of $M \setminus \lambda$.
% Since $M \setminus \lambda$ is $\sigma$-compact, there exists a countable subcover $(U_{x_i})_{i \in I}$, for some countable set $I \subseteq \NN$.

% We then introduce the closed form 
% $$F := \sum_{i \in I} \alpha_i F_{x_i},$$
% where $\sum_{i \in I} \alpha_i = 1$.
% Here the sum converges in the norm topology of $L^\infty$, even if $I$ is infinite.
% Indeed, if $I_N := I \cap \{1, \dots, N\}$, then the partial sums $\sum_{i \in I_N} \alpha_i F_{x_i}$ satisfy the tail bound
% $$\sum_{i \in I \setminus I_N} \alpha_i \|F_{x_i}\|_{L^\infty} \leq L \sum_{i \in I \setminus I_N} \alpha_i \to 0$$
% since $(\alpha_i) \in \ell^1$, which implies the convergence.
% This convergence implies (by Proposition \ref{crandall}) that $[F] = \rho$, so $L(F) \geq L$.
% On the other hand, 
% $$L(F) \leq \sum_{i \in I} \alpha_i L(F_{x_i}) \leq L \sum_{i \in I} \alpha_i = L.$$
% It follows that $L(F) = L$.
% In particular, $F$ has best comass and $\MCL(F) \supseteq \lambda$.

% To complete the proof, we show that $\MCL(F) \subseteq \lambda$.
% Let $x \notin \lambda$, and let $j$ satisfy $U_{x_j} \ni x$.
% Then by Proposition \ref{crandall},
% $$L(F, x) = \lim_{r \to 0} L_{B_r(x)}(F) \leq \lim_{r \to 0} \sum_{i \in I} \alpha_i L_{B_r(x)}(F_{x_i}).$$
% The summands are dominated by the $\ell^1$ sequence $(L\alpha_i)$, so by dominated convergence, 
% $$\lim_{r \to 0} \sum_{i \in I} \alpha_i L_{B_r(x)}(F_{x_i}) = \sum_{i \in I} \lim_{r \to 0} \alpha_i L_{B_r(x)}(F_{x_i}) = \sum_{i \in I} \alpha_i L(F_{x_i}, x).$$
% By assumption on $j$, $L(F_{x_j}, x) < L$, and besides $L(F_{x_i}, x) \leq L$ for any $i$.
% So 
% $$L(F, x) \leq \sum_{i \in I} \alpha_i L(F_{x_i}, x) < L$$
% and we conclude $x \notin \MCL(F)$.
% \end{proof}


%%%%%%%%%%%%%%%%%%%%%%%%%%%%%%
% \section{The \texorpdfstring{$\infty$-Maxwell equation}{infinity-Maxwell equation}}\label{EulerLagrange}
% We have the following Euler-Lagrange equation for forms with absolutely best comass.
% Because of the lack of a good analogue for viscosity solutions for $\infty$-elliptic systems, and because we did not show that $\infty$-tight forms have absolutely best comass, the equation can only really be interpreted in a formal sense, at least as far as we are aware.
% As such, we did not use it in the main body of this paper, but only include it as a curiosity item.

% % \todo{If we knew that $p$-Maxwell had good quantitative uniqueness, then we would have}
% % It remains to show that $A$ has absolutely best curl, so let $\Omega$ be a small ball and $B$ a $1$-form with $B|_{\partial \Omega} = A|_{\partial \Omega}$.
% % By a straightforward modification of the existence theorem, there exists a $p$-magnetic potential $B_p$ in Coulomb gauge with $B_p|_{\partial \Omega} = A|_{\partial \Omega}$ and $B \in C^{1 + \alpha}$.
% % By quantitative uniqueness
% % $$\|B_p - A\|_{C^0(\Omega)} \leq \|B_p - A_p\|_{C^0(\Omega)} + o(1) \lesssim \|A - A_p\|_{C^0(\partial \Omega)} + o(1) \ll 1.$$
% % Therefore $B_p \to A$ uniformly, and for $3 < q < p < \infty$ with $p$ dyadic,
% % $$\|\dif B_p\|_{L^q(\Omega)} \leq |\Omega|^{\frac{1}{q} -\frac{1}{p}} \|\dif B_p\|_{L^p(\Omega)} \leq |\Omega|^{\frac{1}{q} -\frac{1}{p}} \|\dif B\|_{L^p(\Omega)} \leq |\Omega|^{\frac{1}{q}} \|\dif B\|_{L^\infty(\Omega)}.$$
% % Then along a subsequence, $\dif B_p \to \dif A$ in $L^q(\Omega)$, so 
% % $$\|\dif A\|_{L^q(\Omega)} \leq |\Omega|^{\frac{1}{q}} \|\dif B\|_{L^\infty(\Omega)}.$$
% % Taking $q \to \infty$ we arrive at the conclusion that $F$ has absolutely best comass.



% The $\infty$-Maxwell equation has the following natural interpretation.

% \begin{corollary}
% Suppose that $F$ has absolutely best comass, regularity $C^1$, and no points with $F = 0$, and $N$ is a surface whose normal vector field is annihilated by $F$.
% Then $N$ is a minimal surface.
% \end{corollary}
% \begin{proof}
% Let $V$ be a tangent vector field to $N$. Then $V(|F|) = 0$, by (\ref{infinityMaxwell}).
% Therefore $|F|$ is constant along $N$, but $F$ is a continuous section of the area bundle of $N$, which is a real line bundle.
% It follows that $F$ is constant along $N$, and $F/|F|$ is the area form on $N$.
% In other words, $N$ is calibrated by $F$, and the claim follows from (\ref{calibrated surfaces are minimal}).
% \end{proof}






\section{The \texorpdfstring{$\infty$-Maxwell equation}{infinity-Maxwell equation}}
By analogy with $\infty$-harmonic functions, one expects tight forms to be unique and have \dfn{absolutely best comass} in the sense that they have best comass (subject to the Dirichlet condition) in every small open ball in $M$.
One can show formally that the Euler-Lagrange equations for a smooth form $F$ of absolutely best comass are
\begin{equation}\label{infinity Maxwell}
\begin{cases}
\dif F = 0\\
\nabla_i F_J F^J {F^i}_K = 0
\end{cases}
\end{equation}
where $J$ is a $d - 1$-index and $K$ is a $d - 2$-index.
Let us formally derive the $\infty$-Maxwell equation (\ref{infinity Maxwell}):

\begin{lemma}
When written in nondivergence form, the $p$-Maxwell equation is 
\begin{equation}\label{p Maxwell nondivergence form}
\begin{cases}
\dif F = 0\\ 
|F|^{p - 4} ((p - 2) \nabla_i F_J {F^i}_K F^J - |F|^2 (\dif^* F)_K) = 0.
\end{cases}
\end{equation}
\end{lemma}
\begin{proof}
Using the metric property of $\nabla$ and the fact that $d(d - 2) + 1 \equiv d - 1$ modulo $2$,
\begin{align*}
0 &= \dif^*(|F|^{p - 2} F) \\
&= (-1)^{d(d - 2) + 1} \star \dif(|F|^{p - 2} \star F) \\
&= (-1)^{d - 1} \star(\dif(|F|^{p - 2}) \wedge \star F - |F|^{p - 2} \dif \star F) \\
&= (-1)^{d - 1} (p - 2) \star(|F|^{p - 4} \langle \nabla F, F\rangle \wedge \star F) - |F|^{p - 2} \dif^* F.
\end{align*}
We then compute 
$$(-1)^{d - 1} \star(\langle \nabla F, F\rangle \wedge \star F)_\ell = \nabla_i F_J F^J {F^i}_K$$
and factor out $|F|^{p - 4}$ to conclude (\ref{p Maxwell nondivergence form}).
\end{proof}

Now, to derive the $\infty$-Maxwell equation, one may proceed as in the derivation of the $\infty$-Laplace equation in \cite{Barron08}.
After formally dividing both sides of (\ref{p Maxwell nondivergence form}) by $(p - 2) |F|^{p - 4}$, one obtains the $\infty$-Maxwell equation, plus a term which vanishes as $p \to \infty$.

Recently Katzourakis has introduced a generalization notion of viscosity solutions, called \dfn{contact solutions}, for systems of degenerate PDE which cannot necessarily be written in divergence form or diagonal form \cite{Katzourakis2018OnAV}.
As the correct notion of solution for the $\infty$-Laplace equation is viscosity solution, it is natural to expect that the natural notion of solution of (\ref{infinity Maxwell}) is contact solution.
Thus it is natural to conjecture:

\begin{conjecture}
For every $\rho \in H^{d - 1}(M, \RR)$ there exists a unique tight representative of $\rho$.
Moreover, the following are equivalent for a closed $d-1$-form $F$:
\begin{enumerate}
\item $F$ has absolutely best comass.
\item $F$ is tight.
\item $F$ solves the $\infty$-Maxwell equation (\ref{infinity Maxwell}).
\end{enumerate}
\end{conjecture}

The theory of contact solutions is still nascent, and the \emph{uniqueness} theory of contact solutions appears to be out of reach at the time of writing.
So we shall not attempt to prove this conjecture here.
Instead, we shall explore consequences of the $\infty$-Maxwell equation for \dfn{classical tight forms} -- that is $C^2$ solutions of the $\infty$-Maxwell equation.
In this we follow the footsteps of Aronsson in his celebrated paper on classical $\infty$-harmonic functions \cite{Aronsson68}.

%%%%%%%
\subsection{Integral hypersurfaces of classical tight forms}
We have the following analogue of \cite[Lemma 1]{Aronsson68}.

\begin{proposition}
Let $F$ be a classical tight form with no zeroes, and let $N$ be an integral hypersurface of $F$.
Then $N$ is a minimal hypersurface.
\end{proposition}
\begin{proof}
Since $F/|F|$ is the area form on $N$, it suffices to show that $|F|$ is constant along $N$; if so, then $\dif(F/|F|) = |F|^{-1} \dif F = 0$.
However, 
\end{proof}

%%%%%%%
\subsection{Eikonal calibrations}
Recall that that the \dfn{eikonal equation} on scalar fields $u$ is the PDE 
$$|\dif u| = 1.$$
If $u$ solves the eikonal equation, then $u$ also is $\infty$-harmonic \cite[\S]{Aronsson68}; this motivates us to study forms that satisfy the analogous condition.

\begin{definition}
An \dfn{eikonal calibration} is a $d-1$-form $F$ solving 
\begin{equation}\label{eikonal}
\begin{cases}
\dif F = 0 \\
|F| = 1.
\end{cases}
\end{equation}
\end{definition}

An eikonal calibration is classically tight, since 
$$0 = \partial_i (|F|^2) = \langle \nabla F, F\rangle_i = \nabla_i F_J F^J.$$


\printbibliography

\end{document}
