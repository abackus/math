\documentclass[reqno,11pt]{amsart}
\usepackage[letterpaper, margin=1in]{geometry}
\RequirePackage{amsmath,amssymb,amsthm,graphicx,mathrsfs,url,slashed,subcaption}
\RequirePackage[usenames,dvipsnames]{xcolor}
\RequirePackage[colorlinks=true,linkcolor=Red,citecolor=Green]{hyperref}
\RequirePackage{amsxtra}
\usepackage{cancel}
\usepackage{tikz-cd}

% \setlength{\textheight}{9.3in} \setlength{\oddsidemargin}{-0.25in}
% \setlength{\evensidemargin}{-0.25in} \setlength{\textwidth}{7in}
% \setlength{\topmargin}{-0.25in} \setlength{\headheight}{0.18in}
% \setlength{\marginparwidth}{1.0in}
% \setlength{\abovedisplayskip}{0.2in}
% \setlength{\belowdisplayskip}{0.2in}
% \setlength{\parskip}{0.05in}
%\renewcommand{\baselinestretch}{1.05}

\title{Best curl functions}
\author{Aidan Backus}
\date{October 2022}

\newcommand{\NN}{\mathbf{N}}
\newcommand{\ZZ}{\mathbf{Z}}
\newcommand{\QQ}{\mathbf{Q}}
\newcommand{\RR}{\mathbf{R}}
\newcommand{\CC}{\mathbf{C}}
\newcommand{\DD}{\mathbf{D}}
\newcommand{\PP}{\mathbf P}
\newcommand{\MM}{\mathbf M}
\newcommand{\II}{\mathbf I}
\newcommand{\Hyp}{\mathbf H}
\newcommand{\Sph}{\mathbf S}
\newcommand{\Group}{\mathbf G}
\newcommand{\GL}{\mathbf{GL}}
\newcommand{\Orth}{\mathbf{O}}
\newcommand{\SpOrth}{\mathbf{SO}}
\newcommand{\Ball}{\mathbf{B}}

\newcommand*\dif{\mathop{}\!\mathrm{d}}

\DeclareMathOperator{\card}{card}
\DeclareMathOperator{\dist}{dist}
\DeclareMathOperator{\supp}{supp}
\DeclareMathOperator{\tr}{tr}

\newcommand{\Leaves}{\mathscr L}
\newcommand{\Lagrange}{\mathcal L}
\newcommand{\Hypspace}{\mathscr H}

\newcommand{\Chain}{\underline C}

\newcommand{\Two}{\mathrm{I\!I}}

\newcommand{\normal}{\mathbf n}
\newcommand{\radial}{\mathbf r}
\newcommand{\evect}{\mathbf e}
\newcommand{\vol}{\mathrm{vol}}

\newcommand{\diam}{\mathrm{diam}}
\newcommand{\Ell}{\mathrm{Ell}}
\newcommand{\inj}{\mathrm{inj}}
\newcommand{\Lip}{\mathrm{Lip}}
\newcommand{\Riem}{\mathrm{Riem}}

\newcommand{\Min}{\mathrm{Min}}
\newcommand{\Max}{\mathrm{Max}}

\newcommand{\dfn}[1]{\emph{#1}\index{#1}}

\renewcommand{\Re}{\operatorname{Re}}
\renewcommand{\Im}{\operatorname{Im}}

\newcommand{\loc}{\mathrm{loc}}
\newcommand{\cpt}{\mathrm{cpt}}

\def\Japan#1{\left \langle #1 \right \rangle}

\newtheorem{theorem}{Theorem}[section]
\newtheorem{badtheorem}[theorem]{``Theorem"}
\newtheorem{prop}[theorem]{Proposition}
\newtheorem{lemma}[theorem]{Lemma}
\newtheorem{sublemma}[theorem]{Sublemma}
\newtheorem{proposition}[theorem]{Proposition}
\newtheorem{corollary}[theorem]{Corollary}
\newtheorem{conjecture}[theorem]{Conjecture}
\newtheorem{axiom}[theorem]{Axiom}
\newtheorem{assumption}[theorem]{Assumption}

\newtheorem{mainthm}{Theorem}
\renewcommand{\themainthm}{\Alph{mainthm}}

% \newtheorem{claim}{Claim}[theorem]
% \renewcommand{\theclaim}{\thetheorem\Alph{claim}}
\newtheorem*{claim}{Claim}

\theoremstyle{definition}
\newtheorem{definition}[theorem]{Definition}
\newtheorem{remark}[theorem]{Remark}
\newtheorem{example}[theorem]{Example}
\newtheorem{notation}[theorem]{Notation}

\newtheorem{exercise}[theorem]{Discussion topic}
\newtheorem{homework}[theorem]{Homework}
\newtheorem{problem}[theorem]{Problem}

\makeatletter
\newcommand{\proofpart}[2]{%
  \par
  \addvspace{\medskipamount}%
  \noindent\emph{Part #1: #2.}
}
\makeatother



\numberwithin{equation}{section}


% Mean
\def\Xint#1{\mathchoice
{\XXint\displaystyle\textstyle{#1}}%
{\XXint\textstyle\scriptstyle{#1}}%
{\XXint\scriptstyle\scriptscriptstyle{#1}}%
{\XXint\scriptscriptstyle\scriptscriptstyle{#1}}%
\!\int}
\def\XXint#1#2#3{{\setbox0=\hbox{$#1{#2#3}{\int}$ }
\vcenter{\hbox{$#2#3$ }}\kern-.6\wd0}}
\def\ddashint{\Xint=}
\def\dashint{\Xint-}

\usepackage[backend=bibtex,style=alphabetic,giveninits=true]{biblatex}
\renewcommand*{\bibfont}{\normalfont\footnotesize}
\addbibresource{best_curl.bib}
\renewbibmacro{in:}{}
\DeclareFieldFormat{pages}{#1}

\newcommand\todo[1]{\textcolor{red}{TODO: #1}}


\begin{document}
\begin{abstract}
	Best curl functions
\end{abstract}

\maketitle

%%%%%%%%%%%%%%%%%%%%%%%%%%%%%%%%%%%%%%%%%%%%%%%%%%%%%%%

In this paper we study the following related minimization problems for a connection $1$-form $A$ in a Riemannian $3$-fold $M$ subject to a homological or boundary constraint:
\begin{enumerate}
\item Minimize the \dfn{$\infty$-Maxwell energy} $\|\dif A\|_{L^\infty}$.
\item Minimize the \dfn{curl modulus}
$$L(A) := \sup_{\sigma \in \Chain_2(M)} \frac{1}{|\sigma|} \left|\int_\sigma \dif A\right|$$
where $\Chain_2(M)$ denotes the space of oriented $2$-chains in $M$.
\end{enumerate}
Here $\dif A$ denotes the curvature of $A$, so that if $\sigma$ is exact, $\int_{\partial \sigma} A = \int_\sigma \dif A$.

These problems are both analogous to familiar problems for scalar fields: first problem corresponds to the $\infty$-Laplace equation, while the second corresponds to best Lipschitz functions.
For a discussion of our interest in this problem, see \S\ref{motivation}; for a more precise statement of our results without fluff, see \S\ref{results}.

\tableofcontents

\section{Introduction}
\subsection{History and motivation} \label{motivation}

\subsection{Main theorems} \label{results}

\subsection{Acknowledgements}
George, Karen, Tom Goodwillie, Kaya Ferendo, NSF-GRFP, ...

%%%%%%%%%%%%%%%%%%%%%%%%%%%%%%%%%%%%%%%%%%
\section{Preliminaries}
\subsection{Notation}
\subsection{The de Rham--Sobolev complex}
\begin{definition}
For $1 < p < \infty$ and $0 \leq \ell \leq d$, let $W^{1, p}\Omega^\ell$ denote the space of $\ell$-forms $\alpha$ for which the norm
$$\|\alpha\|_{W^{1, p} \Omega^\ell}^p := \|\alpha\|_{L^p}^p + \|\dif \alpha\|_{L^p}^p$$
is finite.
The \dfn{de Rham--Sobolev complex} $W^{1, p} \Omega^\bullet$ is the chain complex of such spaces, with boundary maps 
$$\dif: W^{1, p} \Omega^\ell \to W^{1, p} \Omega^{\ell + 1}.$$
\end{definition}

The de Rham--Sobolev complex is a well-defined elliptic complex, since 
$$\|\dif \alpha\|_{W^{1, p} \Omega^{\ell + 1}}^p = \|\dif \alpha\|_{L^p}^p + \|\dif^2 \alpha\|_{L^p}^p = \|\dif \alpha\|_{L^p}^p \leq \|\alpha\|_{W^{1, p} \Omega^p}^p.$$
For $p = 2$, $d = 3$ this complex is well-known for its applications in electromagnetism and numerical analysis, and its constituent spaces are more commonly known as \cite[Chapter 2]{cessenat1996mathematical}
\begin{align*}
H(\text{curl}) &:= W^{1, 2} \Omega^1 \\
H(\text{div}) &:= W^{1, 2} \Omega^2.
\end{align*}

The utility of the de Rham--Sobolev complex is its trace theorem, which is well-known for $p = 2$.
To state it, we recall that for $s \geq 0$, $1 < p < \infty$ and $\frac{1}{p} + \frac{1}{q} = 1$, and $V$ a Riemannian vector bundle, the dual space of $W^{s, q}_0(M, V)$ (the traceless part of $W^{s, q}$) is $W^{-s, p}(M, V')$ where $V'$ denotes the dual bundle.

\begin{proposition}[trace theorem]
Let $N$ be a smooth embedded hypersurface in $M$, $1 < p < \infty$, and $s > \frac{1}{p}$.
The pullback is a bounded linear operator 
$$\iota^*_N: W^{1, p} \Omega^{d - 1}(M) \to W^{-s, p}(N, \Omega^{d - 1}).$$
In particular, for every $\alpha \in W^{1, p} \Omega^{d - 1}(M)$ and every $f \in W^{1, q}(M)$, where $\frac{1}{p} + \frac{1}{q} = 1$, we have integration by parts:
\begin{equation}\label{Stokes trace}
	\int_N f\alpha = \int_M f \dif \alpha + \alpha \wedge \dif f.
\end{equation}
\end{proposition}
\begin{proof}
It suffices to prove that $\iota^*_N$ is bounded on smooth $d - 1$-forms.
Let $\alpha \in C^\infty(M, \Omega^{d - 1})$ and $f \in C^\infty(N)$.
Then by Stokes' theorem, (\ref{Stokes trace}) holds.
By \todo{prove an inverse trace theorem here}, we may extend $f$ to a smooth function on $M$ such that
$$\|f\|_{W^{1, q}} \lesssim \|f\|_{W^{s, q}(N)}.$$
We then estimate using (\ref{Stokes trace}) and H\"older's inequality
\begin{align*}
\left|\int_N f\alpha \right|
&\leq \|\dif \alpha\|_{L^p} \|f\|_{L^q} + \|\alpha\|_{L^p} \|\dif f\|_{L^q} \\
&\leq \|\alpha\|_{W^{1, p} \Omega^\ell} \|f\|_{W^{1, q}} \\
&\lesssim \|\alpha\|_{W^{1, p} \Omega^\ell} \|f\|_W^{{s, q}(N)}. \qedhere
\end{align*}
\end{proof}

As $W^{-s, q}(N)$ is somewhat intractible, we shall only use the following consequences rather than the trace theorem directly.

\begin{corollary}\label{trace on cycles}
Let $N$ be a smooth embedded hypersurface in $M$ and $1 < p \leq \infty$.
\begin{enumerate}
\item \label{pullback bounded} The pullback is a bounded linear operator
$$\iota^*_N: W^{1, p} \Omega^{d - 1} \to C^1_\cpt(N)'.$$
\item \label{integral continuous} For $F \in W^{1, p} \Omega^{d - 1}(M)$, if $N$ is closed, then the integral $\int_N F$ is well-defined as the limit of integrals of smooth approximations of $F$.
\item \label{cohomology exists} For $F \in L^p(M, \Omega^{d - 1})$, if $N$ is closed and $\dif F = 0$ in the sense of distributions, then $\int_N F$ only depends on the homology class of $F$. In particular, the cohomology class of $F$ is well-defined as the limit of cohomology classes of smooth approximations of $F$.
\end{enumerate}
\end{corollary}
\begin{proof}
Since these statements are local, we may use H\"older's inequality to assume that $p < \infty$.
By the trace theorem, to prove (\ref{pullback bounded}) we only need only recall that for $\frac{1}{p} < s < 1$ and $\frac{1}{p} + \frac{1}{q} = 1$, $C^1_\cpt(N) \subseteq W^{s, q}(N)$.
Then (\ref{integral continuous}) follows because if $N$ is closed, then $1 \in C^1_\cpt(N)$.

Finally, to prove (\ref{cohomology exists}), suppose that $\dif F = 0$ as distributions, and choose a sequence $(F_n)$ of smooth approximations of $F$ which converge in $L^p$ and satisfy $\dif F_n = 0$.
Then $\dif F_n = 0 \to 0 = \dif F$ in $L^p$, so $F_n \to F$ in $W^{1, p} \Omega^{d - 1}$.
By (\ref{integral continuous}) and Stokes' theorem, $\int_N F$ only depends on the homology class $[N]$ rather than $N$.
In particular, $\int_N F = \lim_n \langle [N], [F_n]\rangle$, and this limit is well-defined for every homology class $[N] \in H_{d - 1}(M)$.
So $[F] := \lim_n [F_n]$ exists.
\end{proof}

%%%%%%%%%%%%%%%%%%%%%%%%%%%%%%%%%%%%%%%%%
\section{Convex duality for the $q$-Laplacian}
Let $\Pi: \tilde M \to M$ be the universal covering, $M_{\rm fun} \subseteq \tilde M$ a fundamental domain, and
$$\rho \in H^1(M, \RR)$$
a cohomology class.
Since $H_1(M, \RR)$ is the abelianization of $\pi_1(M)$, $\rho$ is canonically identified with a representation of the fundamental group, which we also call
$$\rho: \pi_1(M) \to \RR.$$
If a function $u: \tilde M \to \RR$ is $\rho$-equivariant, we write $[u] = \rho$.

We here consider the problem
\begin{equation}\label{preprimal problem}
	\Min\{\|\dif u\|_{L^q(M_{\rm fun})}: u \in W^{1, q}(M_{\rm fun}), [u] = \rho\},
\end{equation}
where $1 < q < \infty$.
Taking Euler-Lagrange equations, we see that (\ref{preprimal problem}) is equivalent to the $q$-Laplacian 
$$\begin{cases}
	\dif^*(|\dif u|^{q - 2} \dif u) = 0 \\
	[u] = \rho.
\end{cases}$$

To put (\ref{preprimal problem}) in the framework of \cite[Chapter IV]{Ekeland99}, we shall fix a point $0 \in M_{\rm fun}$, choose a representative $1$-form (which we also call $\rho$), and solve 
$$\begin{cases}
\dif f = \Pi^* \rho \\
v(0) = 0.
\end{cases}$$
Thus (\ref{preprimal problem}) is equivalent to
\begin{equation}\label{primal problem}
	\Min\left\{\frac{1}{q} \int_{M_{\rm fun}} \star|\dif v + \Pi^* \rho|^q: v \in W^{1, q}_0(M_{\rm fun})\right\}
\end{equation}
where we set $u = v + f$.

Let
$$f(\xi) := \frac{1}{q} \int_M \star|\xi + \rho|^q,$$
defined for $\xi \in L^q(M, \Omega^1)$.
Since $v$ is traceless in (\ref{primal problem}), it is invariant, so $\dif v$ drops to a $1$-form on $M$, and (\ref{primal problem}) is the problem of minimizing $f(\dif v)$.
So the Legendre transform
$$\hat f: L^p(M, \Omega^{d - 1}) \to \RR$$
of $f$, where $\frac{1}{p} + \frac{1}{q} = 1$, satisfies
$$\hat f(F) = \frac{1}{p} \int_M \star |F|^p - \int_M \rho \wedge F.$$
From \cite[III(4.18) and III(4.23)]{Ekeland99}, we immediately conclude:

\begin{lemma}
The convex dual problem of the $q$-Laplacian (\ref{primal problem}) is the problem 
\begin{equation}\label{predual problem}
\Max\left\{\int_M \rho \wedge F - \frac{1}{p} \int_M \star |F|^p: F \in L^p(M, \Omega^{d - 1})\right\}.
\end{equation}
Moreover, if $v$ is a solution of (\ref{primal problem}) and $F$ is a solution of the dual problem (\ref{predual problem}), then
\begin{equation}\label{extremality relations}
\frac{1}{q} \int_M \star |\dif v + \rho|^q + \frac{1}{p} \int_M \star |F|^p = \int_M \rho \wedge F + \int_M \dif v \wedge F.
\end{equation}
\end{lemma}

\begin{lemma}
Let $F$ be a solution of the dual problem (\ref{predual problem}). Then
\begin{equation}\label{EL of hat G}
|F|^{p - 2} F = (-1)^{d - 1} \star \rho.
\end{equation}
\end{lemma}
\begin{proof}
Let $(F_t)$ be an arbitrary variation and $\alpha := \frac{\partial F_t}{\partial t}|_{t = 0}$. Then 
$$0 = \frac{\dif}{\dif t} \hat f(F_t)\bigg|_{t = 0} = \int_M \star |F|^{p - 2} \langle F, \alpha \rangle - \rho \wedge \alpha.$$
Since $\alpha$ was arbitrary, for every $d-1$-form $\beta$,
$$|F|^{p - 2} \langle F, \beta\rangle = \star^{-1}(\rho \wedge \beta) = \langle \star^{-1} \rho, \beta\rangle.$$
Recalling that, since $\rho$ is a $1$-form, $\star^{-1} \rho = (-1)^{d - 1} \star \rho$, and $\beta$ is arbitrary, the claim follows.
\end{proof}

In the next proposition we write $\dif A \in L^p(M, \Omega^{d - 1})$ for a $d-2$-form $A$ on $\tilde M$ to mean that $\dif A$ is invariant and drops to a member of $L^p(M, \Omega^{d - 1})$.
\todo{If $H_{1, \cpt}(\tilde M, \RR) = 0$ for any universal cover, then by Poincar\'e duality we can drop the assumption on $H^{d - 1}$}

\begin{proposition}\label{convex duality}
Let $1 < p, q < \infty$ satisfy $\frac{1}{p} + \frac{1}{q} = 1$, and fix a cohomology class $\rho \in H^1(M, \RR)$.
Suppose that $H^{d - 1}(\tilde M, \RR) = 0$.
Then there exists a cohomology class $\alpha \in H^{d - 1}(M, \RR)$ such that the convex dual problem of the $q$-Laplacian (\ref{preprimal problem}) is equivalent to 
\begin{equation}\label{dual problem}
\Min\left\{\frac{1}{p} \int_{M_{\rm fun}} \star |\dif A|^p: \dif A \in L^p(M, \Omega^{d - 1}), [\dif A] = \alpha\right\}.
\end{equation}
Now let $u, A$ be solutions of (\ref{preprimal problem}) and (\ref{dual problem}). Then
\begin{equation}\label{pMaxwell}
\dif^*(|\dif A|^{p - 2} \dif A) = 0
\end{equation}
and $u, A$ are related by
\begin{align}
\dif A &= |\dif u|^{q - 2} \star \dif u \label{extremality} \\
\dif u &= -|\dif A|^{p - 2} \star \dif A. \label{inverse extremality}
\end{align}
\end{proposition}
\begin{proof}
We first solve the $q$-Laplacian (\ref{preprimal problem}) for $u$.
By convex duality \cite[Chapter III, Theorem 4.2]{Ekeland99} there exists a solution $F$ of (\ref{predual problem}) satisfying (\ref{extremality relations}), hence is characterized by 
$$\frac{1}{p} \int_M \star |F|^p + \frac{1}{q} \int_M \star |\dif u|^q = \int_M \dif u \wedge F.$$
Moreover, $F$ satisfies (\ref{EL of hat G}), which uniquely characterizes it; hence (\ref{predual problem}) is uniquely solvable.
If we instead set
$$\tilde F = |\dif u|^{q - 2} \star \dif u,$$
then 
$$\frac{1}{p} |\tilde F|^p + \frac{1}{q} |\dif u|^q = \frac{1}{p} |\dif u|^{p(q - 1)} + \frac{1}{q} |\dif u|^q = |\dif u|^q$$
and 
$$\int_M \dif u \wedge \tilde F = \int_M |\dif u|^{q - 2} \dif u \wedge \star \dif u = \int_M \star |\dif u|^q,$$
so it follows that 
$$\frac{1}{p} \int_M \star |\tilde F|^p + \frac{1}{q} \int_M \star |\dif u|^q = \int_M \dif u \wedge \tilde F.$$
By the uniqueness of (\ref{predual problem}), it follows that $\tilde F = F$ and hence if we define $\dif A := F$, then (\ref{extremality}) holds.
Here, $A$ is well-defined as a $d-2$-form on $\tilde M$ since we assumed $H^{d - 1}(\tilde M, \RR) = $0.
Moreover, from (\ref{EL of hat G}) and the fact that $\rho$ is closed, (\ref{pMaxwell}) holds.
It is clear that the Euler-Lagrange equation of the minimization problem (\ref{dual problem}) is (\ref{pMaxwell}), where $\alpha := [F]$, so $A$ solves (\ref{dual problem}).
Finally, (\ref{inverse extremality}) follows immediately from (\ref{extremality}) and $\frac{1}{p} + \frac{1}{q} = 1$.
\end{proof}

Proposition \ref{convex duality} has the following interesting interpretation, which is motivated by the central role that Noether's theorem plays in 0\cite{daskalopoulos2022, daskalopoulos2023}.
If we define $u_t := u + t$, $t \in \RR$, and apply Noether's theorem \cite[\S8.6.2]{evans2010partial} to the variation $(u_t)$, then we obtain the conservation law 
\begin{equation}\label{conservation law}
	\frac{q - 1}{q} \dif^*(|\dif u|^{q - 2} \dif u) = 0.
\end{equation}
Applying the extremality relation (\ref{extremality}), we see that (\ref{conservation law}) is exactly the dual PDE (\ref{pMaxwell}).
This motivates the following definition:

\begin{definition}
	Let $A$ be a solution of (\ref{dual problem}) and let $u$ be a solution of the $q$-Laplacian (\ref{preprimal problem}). 
	We call $A$ a \dfn{Noetherian potential} for $u$ with respect to target translation.
\end{definition}

If $A_p$ is a Noetherian potential for $u_q$ for every $p \gg 1$ and $\frac{1}{p} + \frac{1}{q} = 1$, it may be that in a suitable function space we may take the limit $p \to \infty$ to obtain a $d-2$-form $A$ and a function $u$.
In this situation, we still call $A$ a Noetherian potential, though this is only literally true in a formal sense.

%%%%%%%%%%%%%%%%%%%%%%%%%%%%%%%%%%%%%%%%%%

\section{The $p$-Maxwell equation}
\subsection{p finite}
Let $M$ be a Riemannian threefold, and fix a $U(1)$ line bundle $\mathscr F \to M$.
\todo{Most of this is just general theory once you change the $p$s, only once we get to calibrations do we need $d = 3$}.
Motivated by Proposition \ref{convex duality}, we define:

\begin{definition}
Let $A_p$ be a connection on $\mathscr F$ and $1 < p < \infty$.
We call $A_p$ a \dfn{$p$-magnetic potential} if it is a minimizer of the \dfn{$p$-Maxwell energy} $\|\dif A_p\|_{L^p}$, $p \geq 2$. among all connections on $\mathscr F$.
\end{definition}

\todo{Instead of connections we could look at $1$-forms on the universal cover such that $\gamma^* A - A$ is closed for every deck transformation $\gamma$.}

\begin{lemma}[$p$-Maxwell equation]
Every $p$-magnetic potential $A_p$ satisfies
\begin{equation}
	\dif^*(|\dif A_p|^{p - 2} \dif A_p) = 0
\end{equation}
in the weak sense that for every $1$-form $B$,
$$\int_M |\dif A_p|^{p - 2} \dif A_p \wedge \dif B = 0.$$
Moreover, the second variation of the $p$-Maxwell energy at $A_p$, in the direction $B$, is
\begin{equation}\label{pSecondVariation}
p \int_M (p - 2) |\dif A_p|^{p - 4} \star \langle \dif A_p, \dif B\rangle^2 + \star |\dif A_p|^{p - 2} |\dif B|^2.
\end{equation}
\end{lemma}
\begin{proof}
Let $A_p(t) := A_p + tB$; then 
\begin{align*}
	\frac{\dif}{\dif t} \int_M \star |\dif A_p(t)|^p
	&= p \int_M \star |\dif A_p(t)|^{p - 2} \dif A_p(t) \wedge \dif B.
\end{align*}
Setting $t = 0$, (\ref{pMaxwell}) follows.
We now differentiate again in $t$, and exploit the fact that
$$\star \langle \dif A_p, \dif B\rangle = \dif A_p \wedge \dif B,$$
to deduce (\ref{pSecondVariation}).
\end{proof}

If $p = 2$, then (\ref{pMaxwell}) reduces to the Maxwell equation (or equivalently the Yang-Mills equation with a $U(1)$ gauge group), which motivates the ``electromagnetic'' terminology that we use.
It would be interesting to consider the analogous equation for more general gauge groups, but we shall not do so here.
Since the gauge group in question is $U(1)$, we simply mean by a \dfn{gauge transformation} the addition of a closed $1$-form.
We as usual say that $A_p$ is in \dfn{Coulomb gauge} if $\dif^* A_p = 0$.

\begin{lemma}
Let $A_p$ be $p$-magnetic potentials on a $U(1)$ line bundle $\mathscr F$, and let $B$ range over all connections on $\mathscr F$.
Then 
\begin{equation}\label{infinity magnetic rules p magnetic}
	\|\dif A_p\|_{L^p} \leq |M|^{1/p} \inf_B \|\dif B\|_{L^\infty}.
\end{equation}
\end{lemma}
\begin{proof}
By H\"older's inequality and the fact that $A_p$ is $p$-magnetic, for any $B$,
$$\|\dif A_p\|_{L^p} \leq \|\dif B\|_{L^p} \leq |M|^{1/p} \|\dif B\|_{L^\infty},$$
hence the same holds for the infimum.
\end{proof}

We put any norm on $H^2(M, \RR)$ (as all are equivalent), hence $|\xi|$ makes sense for $\xi \in H^2(M, \RR)$.

\begin{proposition}\label{existence for p}
Assume $p = 2^n$ for some integer $n \geq 2$, and let $\mathscr F$ be a $U(1)$ line bundle.
Then:
\begin{enumerate}
\item There exists a $p$-magnetic potential $A_p$ which is a connection on $\mathscr F$.
\item $A_p$ is in Coulomb gauge, and every $p$-magnetic potential on $\mathscr F$ is a gauge transformation of $A_p$.
\item One has (for $A_p$ in Coulomb gauge)
\begin{equation}\label{Sobolev bounds for p}
	\|A_p\|_{W^{1, p}} \lesssim |c_1(\mathscr F)|.
\end{equation}
The constant is independent of $p$.
\item There exists $\alpha > 0$ such that $A_p \in C^{1 + \alpha}$.
\end{enumerate}
\end{proposition}
\begin{proof}
Let $A_p(\varepsilon)$ be a smooth connection satisfying
$$\|\dif A_p(\varepsilon)\|_{L^p} \leq \inf_B \|\dif B\|_{L^p} + \varepsilon$$
where $B$ ranges over all connections on $L$. 
We may assume that $A_p(\varepsilon)$ is in Coulomb gauge, for there exists a function $f$ such that $\Delta f = \dif^* A_p(\varepsilon)$, and then we may replace $A_p(\varepsilon)$ with $A_p(\varepsilon) - \dif f$.

By \cite[\S8.2.2]{evans2010partial} and the fact that $p = 2^n$ implies that $x \mapsto |x|^p$ is smooth, the $p$-Maxwell energy is a $W^{1, p}$-weakly lower semicontinuous function.
Therefore, by a compactness argument, we have $A_p(\varepsilon) \to A_p$ along a subsequence in the weakstar topology on $W^{1, p}$, for some $W^{1, p}$ connection $A_p$ which minimizes the $p$-Maxwell energy.
By the Sobolev embedding theorem, $A_p(\varepsilon) \to A_p$ uniformly, so $A_p$ is in Coulomb gauge.

By weak convergence
$$\|A_p\|_{W^{1, p}} \leq \limsup_{\varepsilon \to 0} \|A_p(\varepsilon)\|_{W^{1, p}}$$
and by Poincar\'e's inequality and the facts that $A_p(\varepsilon)$ is in Coulomb gauge and $A_p$ is $p$-magnetic,
$$\|A_p(\varepsilon)\|_{W^{1, p}} \lesssim \|\dif A_p(\varepsilon)\|_{L^p} \leq \|\dif A_p\|_{L^p} + \varepsilon.$$
To bound this, we select a basis $c_1(\mathscr F_1), \dots, c_1(\mathscr F_r)$ for (the torsion-free part of) $H^2(M, \ZZ)$, and apply the above argument to obtain $q$-magnetic potentials $B_i$ on $\mathscr F_i$ for any dyadic $q \in [p, \infty)$.
For $q$ large enough, $\|\dif B_i\|_{L^q}$ is bounded independently of $q$ by (\ref{infinity magnetic rules p magnetic}).
Fix such a $q$, and then consider
$$c_1(\mathscr F) = \sum_{i=1}^r a_i c_1(\mathscr F_i).$$
Using the fact that $A_p$ is $p$-magnetic,
\begin{align*}
\|\dif A_p\|_{L^p}
&\leq \left\|\sum_{i=1}^r a_i B_i\right\|_{L^p} \leq \sum_i |a_i| \|\dif B_i\|_{L^p} \leq \sum_i |a_i| \|\dif B_i\|_{L^q} \lesssim \sum_i |a_i| \\
&\lesssim c_1(\mathscr F).
\end{align*}

We now prove uniqueness.
Assume that $A_p + B$ is another $p$-magnetic potential, so that $B$ is a $1$-form.
By the Cauchy-Schwarz inequality, the second variation (\ref{pSecondVariation}) is bounded from below by 
$$p (p - 1) \int_M \star |\dif A_p|^2 |\dif B|^2.$$
But up to a gauge transformation of $A_p$, we may assume that $B$ is orthogonal to all closed $1$-forms, so that the second variation becomes strictly positive, as desired.

Finally, we deduce the $C^{1 + \alpha}$ elliptic regularity from \cite{Uhlenbeck77}.
\end{proof}

We do NOT get $C^\infty$ regular on sets $\Subset \{|\dif A_p| = 0\}$, at least not in Coulomb gauge.
Why?
Expanding out the $p$-Maxwell equation for $A = A_p$,
$$0 = \partial^j(|\dif A|^{p - 2} \partial_j A_i) - |\dif A|^{p - 2} \partial_i (\dif^* A) - |\dif A|^{p - 2} [\partial^j, \partial_i] A_j - \partial^j(|\dif A|^{p - 2}) \partial_i A_j.$$
The first term is an elliptic operator with H\"older coefficients, the second can be gauged away, and the third is H\"older since $[\partial^j, \partial_i]$ is a connection coefficient of the metric.
But the last term is bad.
\todo{Maybe we can get rid of it using paraproducts?}

%%%%%%%%%%%%%%%%%%%%%%%%%%%%
\subsection{best curl forms}
What follows is the main definition of this paper.
Let $L(A)$ denote the curl modulus of a $C^0$ connection $1$-form $A$, and let $L_\Omega(A)$ be the curl modulus of the restriction of $A$ to a subdomain $\Omega$.

\begin{definition}
	Let $\mathscr F$ be a $U(1)$ line bundle, and let $A$ be a $C^0$ connection $1$-form on $\mathscr F$.
\begin{enumerate}
	\item We say that $A$ has \dfn{best curl} if $A$ is a minimizer of its curl modulus among all connections on $\mathscr F$.
	\item We say that $A$ has \dfn{absolutely best curl} if for every small open ball $\Omega$, $A$ is a minimizer of $L_\Omega$ among all $1$-forms $B$ with $B|_{\partial \Omega} = A|_{\partial \Omega}$.
\end{enumerate}
\end{definition}

We will be interested in the points at which $A$ attains its curl modulus, which can be tested using a localized version of the curl modulus:

\begin{definition}
The \dfn{local curl modulus} of a $1$-form $A$ at $x \in M$ is 
$$L(A, x) := \limsup_{\varepsilon \to 0} \sup_{\sigma \in \Chain_2(B_\varepsilon(x))} \frac{1}{|\sigma|} \int_{\partial \sigma} A.$$
\end{definition}

One has
$$L(A, x) = \limsup_{\varepsilon \to 0} L_{B_\varepsilon(x)}(A)$$
but $L_{B_\varepsilon(x)}(A)$ is increasing in $\varepsilon$ (since it's a supremum over a set which grows in $\varepsilon$).
So the limit superior is actually a limit and an infimum:
$$L(A, x) = \lim_{\varepsilon \to 0} L_{B_\varepsilon(x)}(A) = \inf_{\varepsilon > 0} L_{B_\varepsilon(x)}(A)$$
and in particular $L(A, x) \leq L(A)$.

It's convenient to test the local curl modulus against a more restrictive class of $2$-chains.
Motivated by lattice gauge theory \cite{Gupta98}, we define:

\begin{definition}
Fix an orthonormal basis $(\partial_i)$ of the tangent space $T_x M$.
A \dfn{plaquette} $R_{ij}^\varepsilon(x)$ is the exponential pushforward of a square $[0, \varepsilon \partial_i] \times [0, \varepsilon \partial_j]$ to $M$.
\end{definition}

We write $x + v := \exp_x(v)$ whenever $v \in T_x M$.

\begin{lemma}
Let $A$ be a $C^0$ connection $1$-form and $x \in M$. Then as $\varepsilon \to 0$,
\begin{equation} \label{riemann plaquette}
	\frac{1}{|R_{ij}^\varepsilon(x)|} \int_{\partial R_{ij}^\varepsilon(x)} A = \frac{A_i(x) + A_j(x + \varepsilon \partial_i) - A_i(x + \varepsilon \partial_j) - A_j(x)}{\varepsilon} + o(1).
\end{equation}
\end{lemma}
\begin{proof}
The metric introduces corrections of size $O(\varepsilon^2)$, so we may discard it and assume that $R_{ij}^\varepsilon$ is a rectangle, bounded by line segments 
$$\partial R_{ij}^\varepsilon = \gamma_1 + \gamma_2 - \gamma_3 - \gamma_4,$$
where $\gamma_1$ is the line segment between $x$ and $x + \varepsilon \partial_i$ (say).
Since $A$ is continuous, 
$$\int_{\gamma_1} A = \int_0^\varepsilon A_i(x + t\partial_i) \dif t = \varepsilon A_i(x) + o(\varepsilon).$$
A similar computation holds for the other $\gamma_a$ and implies 
\begin{align*}
\int_{\partial R_{ij}^\varepsilon}
&= \varepsilon A_i(x) + \varepsilon A_j(x + \varepsilon \partial_i) - \varepsilon A_i(x + \varepsilon \partial_j) - \varepsilon A_j(x) + o(\varepsilon) \\
&= \frac{\varepsilon^2}{\varepsilon}(A_i(x) + A_j(x + \varepsilon \partial_i) - A_i(x + \varepsilon \partial_j) - A_j(x) + o(1)).
\end{align*}
Since $|R_{ij}^\varepsilon| = \varepsilon^2$, we are done.
\end{proof}

\begin{lemma}
Let $A$ be a $C^0$ connection $1$-form. Then
\begin{equation}\label{dA is like Lipschitz constant}
	\|\dif A\|_{L^\infty}^2 = \sup_{x \in M} \lim_{\varepsilon \to 0} \frac{1}{|\partial R^\varepsilon_{ij}(x)|^2} \sum_{i < j} \left|\int_{\partial R^\varepsilon_{ij}(x)} B\right|^2.
\end{equation}
\end{lemma}
\begin{proof}
The proof is similar to \cite[\S5.8.2b]{evans2010partial}, so we just sketch the idea.
Let $(\chi_\delta)$ be a standard mollifier based at $x$, and
$$A^\delta_i := A_i * \chi_\delta.$$
Then $A^\delta$ is a smooth connection $1$-form, $A^\delta \to A$ in $C^0$, and a computation gives 
$$\limsup_{\delta \to 0} \|\dif A^\delta\|_{C^0} \leq \|\dif A\|_{L^\infty}.$$
Denoting for a continuous connection $1$-form $B$
$$|\dif^\varepsilon B(x)|^2 := \frac{1}{|\partial R^\varepsilon_{ij}(x)|} \sum_{i < j} \left|\int_{\partial R^\varepsilon_{ij}(x)} B\right|,$$
we have
$$\|\dif^\varepsilon A^\delta\|_{C^0} \leq L(A^\delta) \leq \|\dif A^\delta\|_{C^0}$$
since $A^\delta$ is smooth. Therefore 
$$\|\dif^\varepsilon A\|_{L^\infty} \leq \lim_{\delta \to 0} \|\dif^\varepsilon A^\delta\|_{C^0} \leq \limsup_{\delta \to 0} \|\dif A^\delta\|_{C^0} \leq \|\dif A\|_{L^\infty}.$$
Conversely, $\|\dif^\varepsilon A\|_{C^0} \leq L(A)$, so by a compactness argument we obtain $\dif^\varepsilon A \to F$ in weak $L^2$ for some $2$-form $F$.
Testing $F$ against a smooth $1$-form $B$, we see that $F = \dif A$.
\end{proof}

We prove the following analogue of \cite[Lemma 4.2]{Crandall2008} for best Lipschitz functions.

\begin{proposition}\label{crandall}
Let $A$ be a $C^0$ connection $1$-form. Then:
\begin{enumerate}
\item $L(A, \cdot)$ is upper semicontinuous. \label{crandall usc}
\item If $\dif A(x)$ exists then $L(A, x) \geq |\dif A(x)|$. \label{crandall dA bounds LA}
\item If $L(A, x) = 0$, then $\dif A(x)$ exists and $\dif A(x) = 0$. \label{crandall zero LA implies diffble}
\item If $\sigma \in \Chain_2$ then
$$\frac{1}{|\sigma|} \left|\int_{\partial \sigma} A\right| \leq \sup_{x \in \sigma} L(A, x).$$
In particular, $L(A) \leq \sup_{x \in M} L(A, x)$. \label{crandall best curl is ABC}
\item $\dif A \in L^\infty$ as distributions iff $\sup_{x \in A} L(A, x) < \infty$, in which case \label{crandall linfinity}
$$\sup_{x \in M} L(A, x) = \|\dif A\|_{L^\infty}, \qquad L(A, x) = \lim_{\varepsilon \to 0} \|\dif A\|_{L^\infty(B_\varepsilon(x))}.$$
\end{enumerate}
\end{proposition}
\begin{proof}
First let $x^n \to x$, so for $n$ large, $\Omega_n := B_{r - |x - x^n|}(x^n) \subseteq B_r(x)$, hence
$$L(A, x^n) \leq L_{\Omega_n}(A) \leq L_{B_r(x)}(A).$$
Therefore 
$$\limsup_{n \to \infty} L(A, x^n) \leq \inf_{r > 0} L_{B_r(x)}(A) = L(A, x),$$
which proves (\ref{crandall usc}).
	
We now bound for a net of plaquettes $R_{ij}^\varepsilon(x)$ using (\ref{riemann plaquette})
\begin{equation}\label{difference quotients}
	\limsup_{\varepsilon \to 0} \frac{|A_j(x + \varepsilon \partial_i) - A_j(x) - A_i(x + \varepsilon \partial_j) + A_i(x)|}{\varepsilon}
= \limsup_{\varepsilon \to 0} \frac{1}{|R_{ij}^\varepsilon(x)|} \left|\int_{\partial R_{ij}^\varepsilon(x)} A\right| \leq L(A, x).
\end{equation}
If the first limit superior is actually a limit, then it is the definition of $|\dif A_{ij}(x)|$.
So if $\dif A_{ij}(x)$ exists we conclude $|\dif A_{ij}(x)| \leq L(A, x)$, which proves (\ref{crandall dA bounds LA}).
On the other hand, the corresponding limit \emph{inferior} must be nonnegative, so if $L(A, x) = 0$, the first limit superior in (\ref{difference quotients}) is actually a limit and we obtain $\dif A_{ij}(x) = 0$, proving (\ref{crandall zero LA implies diffble}).

Now let $\sigma$ be a $2$-chain.
We may write $\sigma = \sum_{n=1}^N \sigma_n$ where $\sigma_n$ is a $2$-cell of the form $x^3_n = 0$ for some coordinates $(x^1_n, x^2_n, x^3_n)$.
In particular, for any $\varepsilon > 0$ we may write $\sigma_n$ as a sum of plaquettes $R_{12}^\delta(x)$ with respect to the coordinates $(x^1_n, x^2_n, x^3_n)$, where $0 < \delta < \varepsilon$ and $x \in \sigma_n$.
Let us denote such plaquettes as $P_{n1}(\varepsilon), \dots, P_{nK(n, \varepsilon)}(\varepsilon)$. Then for any $\varepsilon > 0$, 
$$\frac{1}{|\sigma|} \left|\int_{\partial \sigma} A\right| \leq \sum_{n=1}^N \sum_{k=1}^{K(n, \varepsilon)} \frac{|P_{nk}(\varepsilon)|}{|\sigma|} \frac{1}{|P_{nk}(\varepsilon)|} \left|\int_{\partial P_{nk}(\varepsilon)} A\right|.$$
But $P_{nk} \in \Chain_2(B_\varepsilon(x_{nk}(\varepsilon)))$ for some $x_{nk}(\varepsilon) \in \sigma$, so it follows that 
\begin{align*}
\frac{1}{|\sigma|} \left|\int_{\partial \sigma} A\right|
&\leq \limsup_{\varepsilon \to 0} \sum_{n=1}^N \sum_{k=1}^{K(n, \varepsilon)} \frac{|P_{nk}(\varepsilon)|}{|\sigma|} L(A, x_{nk}(\varepsilon)) \\
&\leq \limsup_{\varepsilon \to 0} \sup_{x \in \sigma} L(A, x)  \sum_{n=1}^N \sum_{k=1}^{K(n, \varepsilon)} \frac{|P_{nk}(\varepsilon)|}{|\sigma|} \\
&= \sup_{x \in \sigma} L(A, x)
\end{align*}
which proves (\ref{crandall best curl is ABC}).

Finally, (\ref{crandall linfinity}) follows from (\ref{dA is like Lipschitz constant}) and (\ref{crandall best curl is ABC}).
\end{proof}

\begin{corollary}\label{local curl mod gives global}
For a $C^0$ connnection $1$-form $A$, $\sup_{x \in M} L(A, x) = L(A)$.
\end{corollary}

%%%%%%%%%%%%%%%%%%%%%%%%%%%%%
\subsection{p goes to infinity}
\begin{theorem}\label{existence infinity}
For each dyadic $4 \leq p < \infty$, let $A_p$ be the $p$-magnetic potential on a $U(1)$ line bundle $\mathscr F$, which we assume to be in Coulomb gauge.
Then there exists a connection $A$ in Coulomb gauge on $\mathscr F$ such that:
\begin{enumerate}
\item Along a subsequence, $A_p \to A$ uniformly and $\dif A_p \to \dif A$ weakly in $L^q$ for any $3 < q < \infty$.
\item $A$ has best curl.
\item One has 
\begin{equation}\label{Sobolev bounds for infinity}
	\|A\|_{W^{1, \infty}} \lesssim |c_1(\mathscr F)|.
\end{equation}
\end{enumerate}
\end{theorem}
\begin{proof}
We roughly follow \cite[\S3]{Lindqvist14}.
Let $q > 3$, and let $B$ be a connection on $\mathscr F$.
By H\"older's inequality and (\ref{infinity magnetic rules p magnetic}),
\begin{equation}\label{uniform bounds in p by best curl}
	\|\dif A_p\|_{L^q} \leq |M|^{\frac{1}{q} - \frac{1}{p}} \|\dif A_p\|_{L^p} \leq |M|^{\frac{1}{q} - \frac{1}{p}} \|\dif B\|_{L^\infty}.
\end{equation}
Then a compactness argument gives $\dif A_p \to F$ for some $2$-form $F$, weakly in $L^q$, and by Fatou's lemma, 
$$\|F\|_{L^q} \leq |M|^{\frac{1}{q}} \|\dif B\|_{L^\infty}.$$
Diagonalizing, we may assume that $\dif A_p \to F$ weakly in $L^q$ for every such $q$, and taking $q \to \infty$, we conclude 
\begin{equation}\label{infinity magnetics have best curl}
	\|F\|_{L^\infty} \leq \|\dif B\|_{L^\infty}.
\end{equation}

We now compute the cohomology class of $F$.
Since $\dif F = 0$ and
$$\|F\|_{L^4} \leq |M|^{\frac{1}{4}} \|F\|_{L^\infty},$$
$\dif A_p \to F$ weakly in $W^{1, 4} \Omega^2$.
So by Corollary \ref{trace on cycles}, for any $2$-cycle $S$,
$$\int_S F = \lim_{p \to \infty} 2\pi \langle c_1(\mathscr F), [S] \rangle = 2\pi \langle c_1(\mathscr F), [S]\rangle,$$
where $[S] \in H_2(M, \ZZ)$ denotes the homology class of $S$.
Therefore $F$ is the curvature of a connection $A$ on $\mathscr F$, which must have best curl by Proposition \ref{crandall}(\ref{crandall linfinity}) since $B$ was arbitrary in (\ref{infinity magnetics have best curl}).
Moreover, taking the limit as $p \to \infty$ in (\ref{Sobolev bounds for p}), we obtain (\ref{Sobolev bounds for infinity}).

We now estimate using the Sobolev embedding theorem, (\ref{uniform bounds in p by best curl}), and H\"older's inequality,
$$\|A_p\|_{C^{\frac{1}{4}}} \lesssim \|A_p\|_{W^{1, 4}} \lesssim \|\dif A_p\|_{L^4} \leq |M|^{\frac{1}{4} - \frac{1}{p}} \|\dif A\|_{L^\infty}$$
for $p \geq 4$.
Passing to a further subsequence, we obtain $A_p \to A$ uniformly, and in particular that $A$ is in Coulomb gauge.
\end{proof}

\begin{definition}
Let $A$ be the limiting connection $1$-form in Theorem \ref{existence infinity} for a $U(1)$ line bundle $\mathscr F$.
By an \dfn{$\infty$-magnetic potential}, we mean a gauge transformation of $A$.
\end{definition}

\todo{If we knew that $p$-Maxwell had good quantitative uniqueness, then we would have}
It remains to show that $A$ has absolutely best curl, so let $\Omega$ be a small ball and $B$ a $1$-form with $B|_{\partial \Omega} = A|_{\partial \Omega}$.
By a straightforward modification of Proposition \ref{existence for p}, there exists a $p$-magnetic potential $B_p$ in Coulomb gauge with $B_p|_{\partial \Omega} = A|_{\partial \Omega}$ and $B \in C^{1 + \alpha}$.
By quantitative uniqueness
$$\|B_p - A\|_{C^0(\Omega)} \leq \|B_p - A_p\|_{C^0(\Omega)} + o(1) \lesssim \|A - A_p\|_{C^0(\partial \Omega)} + o(1) \ll 1.$$
Therefore $B_p \to A$ uniformly, and for $3 < q < p < \infty$ with $p$ dyadic,
$$\|\dif B_p\|_{L^q(\Omega)} \leq |\Omega|^{\frac{1}{q} -\frac{1}{p}} \|\dif B_p\|_{L^p(\Omega)} \leq |\Omega|^{\frac{1}{q} -\frac{1}{p}} \|\dif B\|_{L^p(\Omega)} \leq |\Omega|^{\frac{1}{q}} \|\dif B\|_{L^\infty(\Omega)}.$$
Then along a subsequence, $\dif B_p \to \dif A$ in $L^q(\Omega)$, so 
$$\|\dif A\|_{L^q(\Omega)} \leq |\Omega|^{\frac{1}{q}} \|\dif B\|_{L^\infty(\Omega)}.$$
Taking $q \to \infty$ we arrive at the conclusion.

We have the following Euler-Lagrange equation for $\infty$-magnetic potentials.
Because of the lack of a good analogue for viscosity solutions for $\infty$-elliptic systems, \todo{and because we did not show that $\infty$-magnetic potentials have absolutely best curl}, the equation can only really be interpreted in a formal sense, at least as far as we are aware.
As such, we shall not use it in the sequel, but only include it as a curiosity item.

\begin{proposition}[$\infty$-Maxwell equation]
Suppose that $A$ has absolutely best curl, regularity $C^2$, and no points with $\dif A = 0$. Then for $F := \dif A$,
\begin{equation}\label{infinityMaxwell}
	F^{ij} \partial_i |F| = 0.
\end{equation}
\end{proposition}
\begin{proof}
For a covariant $2$-tensor $T$, let $T^{\rm as}_{ij} := T_{ij} - T_{ji}$ be its antisymmetrization, and let
$$f(x, T) := |T^{\rm as}|_{g(x)}.$$
Since $A$ has absolutely best curl and $(\nabla A)^{\rm as} = \dif A$, $A$ is an absolute minimizer (see \cite[Definition 5.1]{Barron2001}) of the essential supremum of $f(\cdot, \nabla A)$.
By the Euler-Lagrange-Aronsson formula \cite[Theorem 5.2]{Barron2001},
\begin{equation}\label{ELA}
	\left\langle \frac{\partial f}{\partial T}(x, \nabla A(x)), \dif (f(x, \nabla A(x))) \right\rangle = 0.
\end{equation}
Now
$$\dif(f(x, \nabla A(x))) = \dif |\dif A(x)|$$
and 
$$\frac{\partial f}{\partial T}(x, \nabla A(x)) = \frac{\nabla A(x)^{\rm as}}{|\nabla A(x)^{\rm as}|} = \frac{\dif A(x)}{|\dif A(x)|}$$
we conclude the claim after multiplying both sides of (\ref{ELA}) by $|\dif A|$.
\end{proof}

\begin{corollary}
Suppose that $A$ has absolutely best curl, regularity $C^2$, and no points with $\dif A = 0$, and $N$ is a surface whose normal vector field is annihilated by $\dif A$.
Then $N$ is a minimal surface.
\end{corollary}
\begin{proof}
Let $V$ be a tangent vector field to $N$. Then $V|\dif A| = 0$, by the $\infty$-Maxwell equation (\ref{infinityMaxwell}).
Therefore $|\dif A|$ is constant along $N$, and so $\dif A$ is a constant times the area form $\star \normal_N^\flat$ along $N$.
So there are constant real numbers $c_1, c_2$ such that the mean curvature $H_N$ satisfies
\begin{align*}
H_N &= \tr(\nabla \normal_N^\flat) = c_1 \dif^* \star \dif A = c_2 \dif^2 A = 0. \qedhere
\end{align*}
\end{proof}


%%%%%%%%%%%%%%%%%%%%
\subsection{q goes to 1}
Our next task is to realize an $\infty$-magnetic potential as the Noetherian potential of a $1$-harmonic function.


%%%%%%%%%%%%%%%%%%%%

\section{The maximum curl locus}
Throughout this section, let $M$ be a closed space form of dimension $3$.

\begin{definition}
Let $A$ be a connection of best curl on $\mathscr F$.
The \dfn{maximum curl locus} is the set $\{L(A, \cdot) = L(A)\}$.
\end{definition}

By Proposition \ref{crandall}(\ref{crandall usc}) and the compactness of $M$, the maximum curl locus is a nonempty closed subset of $M$.

%%%%%%%%%%%%%%%%%%%%%%%%%%%%%%%%%%%
\subsection{The calibrated lamination}
\begin{remark}\label{scalings}
The two minimization problems in consideration are invariant under two scalings: 
\begin{enumerate}
\item Rescaling the entire manifold, which rescales $\|\dif A\|_{L^\infty}$ but not $c_1(\mathscr F)$.
\item $A \mapsto \lambda A$. This changes both $\|\dif A\|_{L^\infty}$ and $c_1(\mathscr F)$. \todo{Could also rescale the circle inducing the circle bundle. Also this doesn't really have anything to do with the task at hand, and would probably better go elsewhere to explain why we may assume $\dif A \in H^2(M, \ZZ)$}
\end{enumerate}
In particular, by applying both scalings and using Chern-Weil theory, we may replace any best curl connection by a best curl connection $A$ with $\|\dif A\|_{L^\infty} = 1$, if desired.
\end{remark}

Recall that in the work of Harvey--Lawson on calibrated geometry \cite{Harvey82}, a \dfn{calibration} on $M$ is a smooth closed $2$-form of $C^0$ norm $1$.
Owing to the above scale-invariance, and the fact that we do not enjoy the luxury of such high regularity, we shall just mean by a \dfn{calibration} something much weaker: a $2$-form $F \in L^\infty$, such that $\|F\|_{L^\infty} > 0$ and $\dif F = 0$ as distributions.
This definition is even weaker than the definition of \cite[\S2A]{bangert_cui_2017} which assumed that $F$ is continuous and $\|F\|_{L^\infty} = 1$.

\begin{definition}
Let $F$ be a calibration and $N \subset M$ an embedded surface with area form $\omega_N$.
We say that $N$ is a \dfn{calibrated surface} if the pullback of $F$ to $N$ is $\|F\|_{L^\infty} \omega_N$.
\end{definition}

It is clear that a calibration has regularity $H(\text{div}) = W^{1, 2} \Omega^2$.
So by Corollary \ref{trace on cycles}(\ref{integral continuous}), we may integrate a calibration over any closed surface $N$, or any small ball in a possibly nonclosed surface $N$.
In particular, the sentence ``$N$ is a calibrated surface'' is well-defined, and it just means that for every sufficiently small relatively open ball $U \subseteq N$,
$$\int_U F = \|F\|_{L^\infty} |U|.$$
Even at this generality, every calibrated surface is minimal: if $N'$ is homologous to $N$ relative to $\partial N = \partial N'$, then by Stokes' theorem,
$$|N'| = \frac{1}{\|F\|_{L^\infty}} \int_{N'} F = \frac{1}{\|F\|_{L^\infty}} \int_N F \leq |N|.$$
Finally, before stating the next proposition, we need to recall that by Corollary \ref{trace on cycles}(\ref{cohomology exists}), the cohomology class of a calibration is well-defined.

\begin{proposition}\label{Bangert Cui}
Let $F$ be a calibration which minimizes $\|F\|_{L^\infty}$ in its cohomology class.
Then there exists a measured oriented minimal lamination on $M$ whose leaves are calibrated surfaces with respect to $F$.
\end{proposition}
\begin{proof}
This essentially follows from \cite[Theorem 5.1]{bangert_cui_2017}; we make a few remarks about how the proof works at this level of generality.
By a duality argument, which is purely on the level of homology, one may find a closed minimal $d-1$-current $T$ which is \dfn{calibrated} by $F$ in the sense that
$$\langle [T], [F]\rangle = \mathbf M(T) \|F\|_{L^\infty}$$
where $\mathbf M(T)$ is the mass of $T$ \cite[Proposition 2.2]{bangert_cui_2017}.
We stress that, since the theory of \cite[\S2C]{bangert_cui_2017} is on the level of homology, we need not worry that any of the involved expressions are well-defined at our extremely low regularity.
Now by \cite[Theorem 1]{AUER20011095} (or perhaps \todo{Cite laminations paper}, since \cite{AUER20011095} appeals to \cite[\S37]{Simon84} which is for euclidean space rather than space forms), the minimal current $T$ induces a measured oriented minimal lamination $\lambda$.
In particular, if $K$ denotes the space of leaves of $\lambda$ and $K$ denotes its transverse measure,
$$\langle [T], [F]\rangle = \int_K \int_{N_k} F \dif \mu(k)$$
but 
$$\mathbf M(T) = \int_K |N_k| \dif \mu(k).$$
This is only possible if the leaves of $\lambda$ are calibrated surfaces.
\end{proof}

\begin{corollary}\label{best curl lamination}
Let $A$ be a connection of best curl.
Then the maximum curl locus contains a measured oriented minimal lamination.
\end{corollary}
\begin{proof}
By definition, $\dif A$ is a calibration which minimizes $\|\dif A\|_{L^\infty}$ in $c_1(\mathscr F)$.
Let $\lambda$ be the lamination of Proposition \ref{Bangert Cui}.
Then $\dif A$ is smooth in the tangent directions to each leaf of $\lambda$, since it is a multiple of the area form on $\lambda$.
In particular, $|\dif A(x)|$ exists and equals $\|\dif A\|_{L^\infty}$ if $x \in \supp \lambda$.
The assertion now follows from Proposition \ref{crandall}(\ref{crandall linfinity}) and Corollary \ref{local curl mod gives global}.
\end{proof}

By \cite[Example 5.4]{bangert_cui_2017} the maximum curl locus need not equal a lamination.

%%%%%%%%%%%%%%%%%%%%%%
\subsection{The dual $1$-harmonic function}
In this section we interpret the lamination of Corollary \ref{best curl lamination} using convex duality.

\begin{theorem}
Let $A$ be an $\infty$-magnetic potential on $\mathscr F$.
Then there exists a $1$-harmonic function $u$ on the universal cover $\tilde M$ such that:
\begin{enumerate}
\item $A$ is a Noetherian potential for $u$.
\item $\dif u$ is a Ruelle-Sullivan current for a minimal lamination in the maximum curl locus of $A$.
\end{enumerate}
\end{theorem}

\todo{In fact, it's probably true that $\dif u$ gives a minimal lamination which is contained in EVERY maximum curl locus of a best curl connection on $\mathscr F$}

To begin the proof, let $(A_p)$ be a $p$-approximation to $A$.
Using the scaling techniques of Remark \ref{scalings}, we may assume that $\|\dif A\|_{L^\infty} = 1$; this eases the notational burden somewhat.
We follow \cite[\S6.1]{daskalopoulos2020transverse}.

We begin by constructing the function $u$.
For $3 < p < \infty$ and $\frac{1}{p} + \frac{1}{q} = 1$, let
$$\int_M \star |k_p \dif A_p|^p = k_p$$
just like in \cite[\S3.2]{daskalopoulos2020transverse}.
Let $F_p := k_p \dif A_p$ and $U_q := |F_p|^{p - 2} \star F_p$ which is a $1$-form.
Then 
\begin{equation}\label{dAp wedge Uq}
\int_M \dif A_p \wedge U_q = k_p^{-1} \int_M \star |F_p|^p = 1.
\end{equation}
Also for $F := \dif A$,
$$\lim_{p \to \infty} k_p^{-\frac{1}{q}} = \lim_{p \to \infty} \|\dif A_p\|_{L^p} = \|F\|_{L^\infty} = 1.$$
It follows that $k_p \to 1$.

We now compute
\begin{equation}\label{Uq is closed}
	\dif U_q = \dif(|F_p|^{p - 2} \star F_p) = 0
\end{equation}
since $F_p$ solves the $p$-Maxwell equation (\ref{predual problem}).
So we may choose $u_q$ such that $U_q = \dif u_q$ and $u_q(0) = 0$ for some choice of origin $0$ of the universal cover $\tilde M$.

We have the following analogue of \cite[Theorem 4.3]{daskalopoulos2020transverse} with essentially the same proof.

\begin{lemma}
There exists $u \in BV_\loc(\tilde M)$ and $\rho \in H^1(M, \RR)$ such that:
\begin{enumerate}
\item $u$ is $\rho$-equivariant.
\item $u_q \to u$ weakly in $BV_\loc(\tilde M)$.
\item $u_q \to u$ in $L^r(M)$ for $1 \leq r < \infty$.
\end{enumerate}
\end{lemma}
\begin{proof}
By H\"older's inequality,
$$\int_M \star |U_q| = \int_M \star |F_p|^{p - 1} \leq |M|^{\frac{1}{p}} \left[\int_M \star |F_p|^p\right]^{\frac{1}{q}}.$$
Therefore 
$$\limsup_{q \to 1} \int_M \star |U_q| \leq 1.$$
On the other hand, (\ref{dAp wedge Uq}) is only possible if
$$\liminf_{q \to 1} \int_M \star |U_q| \geq 1$$
and so $\|U_q\|_{L^1} \to 1$.
Taking limits in (\ref{Uq is closed}) and using a compactness argument, we obtain $U_q \to U$ in the weak topology of measures, for some closed $2$-current $U$.
We write $\dif u = U$.

Since $U_q, F_p$ are related by the convex duality relation (\ref{inverse extremality}),
\end{proof}



So by the Maz\'on-Rosser-Segura de Le\'on theorem, $u$ is $1$-harmonic.

%%%%%%%%%%%%%%%%%%%%%
\appendix 

\section{The dual one-harmonic function}
For simplicity let's assume $H^2(M, \ZZ) = 0$.

Let $A$ be an $\infty$-magnetic potential and let $A_p \to A$ witness that $A$ is $\infty$-magnetic.


Question: Does the Fenchel relation imply that $u$ is nonzero?
What about nonconstant?
\cite{bangert_cui_2017} implies that there exists $U$ which is calibrated by $F$, and then $U = \dif \tilde u$, but it doesn't seem clear that this $\tilde u$ should be the same as $U$, maybe the uniqueness part of the M-R-SdL theorem explains this.

\begin{lemma}
For any $\theta \in (0, 1)$,
	$$\lim_{p \to \infty} \int_{\{|F| \leq \theta\}} \star |F_p|^p = 0.$$
\end{lemma}
\begin{proof}
Integrating the $p$-Maxwell equation 
$$\dif^*(|\dif A_p|^{p - 2} \dif A_p) = 0$$
by parts 
\begin{align*}
	0 &= \int_M \dif \star (|\dif A_p|^{p - 2} \dif A_p) \wedge (A_p - A) \\
	&= \int_{\partial M} \star(|\dif A_p|^{p - 2} \dif A_p) \wedge \iota^* (A_p - A) - \int_M \star (|\dif A_p|^{p - 2} \dif A_p) \wedge \dif (A_p - A).
\end{align*}
Here $\iota^* (A_p - A)$ is well-defined because $A_p - A \in C^0$.
I guess that $|\dif A_p|^{p - 2} \dif A_p$ should have a trace as well, otherwise we need some kind of smooth approximation.
Let $J$ be the Neumann data, then 
$$\dif \iota^*(A_p - A) = J - J = 0$$
so $\iota^*(A_p - A)$ is pure gauge. In particular, since we may modify the $p$-approximations $A_p$ up to gauge without affecting the result, we may assume that $\iota^*(A_p - A) = 0$.
Therefore 
$$\int_M |\dif A_p|^{p - 2} \dif A_p \wedge \star \dif (A_p - A) = 0.$$

We now proceed as in \cite[Lemma 6.3]{daskalopoulos2020transverse}.
Let $f(p) := \langle F_p, F_p - F\rangle$ and $Y_p := \{f(p) \geq 0\}$.
Rescaling by $k_p^p$, 
$$\int_M |F_p|^{p - 2} F_p \wedge \star (F_p - k_p F) = 0$$
so 
\begin{align*}
	\lim_{p \to \infty} \int_M |F_p|^{p - 2} F_p \wedge \star (F_p - k_p F) - \star f(p) 
	&= \lim_{p \to \infty} \int_M |F_p|^{p - 2} F_p \wedge \star (F - k_p F) \\
	&\leq \lim_{p \to \infty} (1 - k_p) \int_M |F_p|^{p - 1} \star |F| = 0.
\end{align*}
Therefore 
$$\lim_{p \to \infty} \int_M |F_p|^{p - 2} \star f(p) = 0.$$
On the other hand, on $M \setminus Y_p$ we have by the elementary \cite[Lemma 6.2]{daskalopoulos2020transverse} that 
$$
	-2|F_p|^{p - 2} f(p) \leq |F_p|^{p - 2} (|F|^2 - |F_p|^2) < \frac{2}{p - 2}
$$
which implies 
$$\lim_{p \to \infty} \int_{M \setminus Y_p} |F_p|^{p - 2} \star f(p) = 0.$$
Therefore 
$$\lim_{p \to \infty} \int_{Y_p} |F_p|^{p - 2} \star f(p) = 0.$$

To finish the proof we look to \cite[Proposition 6.5]{daskalopoulos2020transverse}...
\end{proof}

\begin{corollary}
$\supp \dif v$ is contained in the calibrated lamination associated to any best curl form with the same data as $A$.
\end{corollary}
\begin{proof}
By as in \cite[Theorem 6.1]{daskalopoulos2020transverse}, $\supp \dif v \subseteq \{|F| = \|F\|_{L^\infty}\}$.
Actually, as in \cite[Corollary 6.8]{daskalopoulos2020transverse} the same argument works as long as $F = \dif \tilde A$ where $\tilde A$ is a best curl competitor of $A$.
\end{proof}

\subsection{Thurston's K-L theorem}
\begin{conjecture}
For $\sigma$ a closed $2$-chain, let
$$K(\sigma) := \frac{1}{|\sigma|} \int_\sigma \dif A$$
and let $K := \sup_\sigma K(\sigma)$, where $A$ has best curl (so is only locally a $1$-form).
Let $L$ be the best curl constant. Then $K = L$.
\end{conjecture}

This holds if the calibrated lamination has a closed leaf.
If it doesn't, how can we deform it to have a closed leaf?



\printbibliography

\end{document}
