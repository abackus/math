\documentclass[reqno,11pt]{amsart}
\usepackage[letterpaper, margin=1in]{geometry}
\RequirePackage{amsmath,amssymb,amsthm,graphicx,mathrsfs,url,slashed,subcaption}
\RequirePackage[usenames,dvipsnames]{xcolor}
\RequirePackage[colorlinks=true,linkcolor=Red,citecolor=Green]{hyperref}
\RequirePackage{amsxtra}
\usepackage{cancel}
\usepackage{tikz, quiver, wrapfig}
%\usepackage[T1]{fontenc}

% \setlength{\textheight}{9.3in} \setlength{\oddsidemargin}{-0.25in}
% \setlength{\evensidemargin}{-0.25in} \setlength{\textwidth}{7in}
% \setlength{\topmargin}{-0.25in} \setlength{\headheight}{0.18in}
% \setlength{\marginparwidth}{1.0in}
% \setlength{\abovedisplayskip}{0.2in}
% \setlength{\belowdisplayskip}{0.2in}
% \setlength{\parskip}{0.05in}
%\renewcommand{\baselinestretch}{1.05}

\title{Convex duality between minimal laminations and tight calibrations}
\author{Aidan Backus}
\address{Department of Mathematics, Brown University}
\email{aidan\_backus@brown.edu}
\date{\today}

\newcommand{\NN}{\mathbf{N}}
\newcommand{\ZZ}{\mathbf{Z}}
\newcommand{\QQ}{\mathbf{Q}}
\newcommand{\RR}{\mathbf{R}}
\newcommand{\CC}{\mathbf{C}}
\newcommand{\DD}{\mathbf{D}}
\newcommand{\PP}{\mathbf P}
\newcommand{\MM}{\mathbf M}
\newcommand{\II}{\mathbf I}
\newcommand{\Hyp}{\mathbf H}
\newcommand{\Sph}{\mathbf S}
\newcommand{\Group}{\mathbf G}
\newcommand{\GL}{\mathbf{GL}}
\newcommand{\Orth}{\mathbf{O}}
\newcommand{\SpOrth}{\mathbf{SO}}
\newcommand{\Ball}{\mathbf{B}}

\newcommand*\dif{\mathop{}\!\mathrm{d}}

\DeclareMathOperator{\card}{card}
\DeclareMathOperator{\dist}{dist}
\DeclareMathOperator{\id}{id}
\DeclareMathOperator{\Hom}{Hom}
\DeclareMathOperator{\coker}{coker}
\DeclareMathOperator{\supp}{supp}
\DeclareMathOperator{\Teich}{Teich}
\DeclareMathOperator{\tr}{tr}

\newcommand{\Leaves}{\mathscr L}
\newcommand{\Lagrange}{\mathscr L}
\newcommand{\Hypspace}{\mathscr H}

\newcommand{\Chain}{\underline C}

\newcommand{\Two}{\mathrm{I\!I}}
\newcommand{\Ric}{\mathrm{Ric}}

\newcommand{\normal}{\mathbf n}
\newcommand{\radial}{\mathbf r}
\newcommand{\evect}{\mathbf e}
\newcommand{\vol}{\mathrm{vol}}

\newcommand{\diam}{\mathrm{diam}}
\newcommand{\Ell}{\mathrm{Ell}}
\newcommand{\inj}{\mathrm{inj}}
\newcommand{\Lip}{\mathrm{Lip}}
\newcommand{\MCL}{\mathrm{MCL}}
\newcommand{\Riem}{\mathrm{Riem}}

\newcommand{\Mass}{\mathbf M}
\newcommand{\Comass}{\mathbf L}

\newcommand{\Min}{\mathrm{Min}}
\newcommand{\Max}{\mathrm{Max}}

\newcommand{\dfn}[1]{\emph{#1}\index{#1}}

\renewcommand{\Re}{\operatorname{Re}}
\renewcommand{\Im}{\operatorname{Im}}

\newcommand{\loc}{\mathrm{loc}}
\newcommand{\cpt}{\mathrm{cpt}}

\def\Japan#1{\left \langle #1 \right \rangle}

\newtheorem{theorem}{Theorem}[section]
\newtheorem{badtheorem}[theorem]{``Theorem"}
\newtheorem{prop}[theorem]{Proposition}
\newtheorem{lemma}[theorem]{Lemma}
\newtheorem{sublemma}[theorem]{Sublemma}
\newtheorem{proposition}[theorem]{Proposition}
\newtheorem{corollary}[theorem]{Corollary}
\newtheorem{conjecture}[theorem]{Conjecture}
\newtheorem{axiom}[theorem]{Axiom}
\newtheorem{assumption}[theorem]{Assumption}

\newtheorem{mainthm}{Theorem}
\renewcommand{\themainthm}{\Alph{mainthm}}

\newtheorem{claim}{Claim}[theorem]
\renewcommand{\theclaim}{\thetheorem\Alph{claim}}
% \newtheorem*{claim}{Claim}

\theoremstyle{definition}
\newtheorem{definition}[theorem]{Definition}
\newtheorem{remark}[theorem]{Remark}
\newtheorem{example}[theorem]{Example}
\newtheorem{notation}[theorem]{Notation}

\newtheorem{exercise}[theorem]{Discussion topic}
\newtheorem{homework}[theorem]{Homework}
\newtheorem{problem}[theorem]{Problem}

\makeatletter
\newcommand{\proofpart}[2]{%
  \par
  \addvspace{\medskipamount}%
  \noindent\emph{Part #1: #2.}
}
\makeatother



\numberwithin{equation}{section}


% Mean
\def\Xint#1{\mathchoice
{\XXint\displaystyle\textstyle{#1}}%
{\XXint\textstyle\scriptstyle{#1}}%
{\XXint\scriptstyle\scriptscriptstyle{#1}}%
{\XXint\scriptscriptstyle\scriptscriptstyle{#1}}%
\!\int}
\def\XXint#1#2#3{{\setbox0=\hbox{$#1{#2#3}{\int}$ }
\vcenter{\hbox{$#2#3$ }}\kern-.6\wd0}}
\def\ddashint{\Xint=}
\def\dashint{\Xint-}

\usepackage[backend=bibtex,style=alphabetic,giveninits=true]{biblatex}
\renewcommand*{\bibfont}{\normalfont\footnotesize}
\addbibresource{best_curl.bib}
\renewbibmacro{in:}{}
\DeclareFieldFormat{pages}{#1}

\newcommand\todo[1]{\textcolor{red}{TODO: #1}}


\begin{document}
\begin{abstract}
We introduce a family of closed $d-1$-forms on Riemannian $d$-manifolds which minimize their comass (or $L^\infty$ norm) in their cohomology class, which we call \dfn{tight}.
Tight forms have properties similar to (gradients of) $\infty$-harmonic maps between surfaces: they are convex duals of $1$-harmonic functions and attain their comass on a measured oriented minimal lamination $\mu$.
We show that $\mu$ has properties analogous to a measured sublamination of Thurston's canonical lamination.
\todo{Rewrite this}
\end{abstract}

\maketitle

%%%%%%%%%%%%%%%%%%%%%%%%%%%%%%%%%%%%%%%%%%%%%%%%%%%%%%%
\section{Introduction}
The classical \dfn{max flow min cut theorem} asserts that in a discrete flow network $\mathcal G$, the maximal flow through $\mathcal G$ is equal to the \dfn{minimal cut} of $\mathcal G$, namely the minimal total capacity of any set of edges which divide $\mathcal G$ into a source component and a sink component; this theorem is a concrete form of the duality theorem for convex optimization.
As we shall survey below, various continuous versions of the max flow min cut theorem have since appeared, which usually replace the maximal flow by a \dfn{calibration} (a closed $d - 1$-form, or equivalently a divergence-free vector field, of $L^\infty$ norm $1$), and the minimal cut by a minimal submanifold.
In particular, Bangert and Cui have shown that given a continuous calibration $F$ on a closed Riemannian manifold of dimension $\leq 7$, such that $\|F\|_{L^\infty}$ is minimized in its cohomology class, there exists a minimal lamination calibrated by $F$ \cite[Theorem 5.1]{bangert_cui_2017}.

We shall prove a particularly strong version of the Bangert--Cui max flow min cut theorem.
We only need to assume that the calibration $F$ is measurable; moreover, it shall follow from our methods that the calibrated lamination is Lipschitz.
We use the approach of $p$-harmonic duality which was pioneered by Daskalopolous and Uhlenbeck in the context of best Lipschitz maps \cite{daskalopoulos2020transverse,daskalopoulos2022,daskalopoulos2023}.
To be more precise, we pick out a special class of calibrations, which we call \dfn{tight forms}, which are the variational solutions of a suitable generalization of the $\infty$-Laplace equation.
Tight forms are necessarily convex duals of functions of least gradients, and because functions of least gradient induce minimal laminations \cite[Theorem B]
{BackusCML}, the main theorem follows.

Along the way, we establish several auxiliary results that may be of independent interest.
We establish the necessary geometric measure theory to study $L^\infty$ calibrations; we construct a minimal lamination, which we call the \dfn{canonical lamination}, associated to each cohomology class in $H^{d - 1}(M, \RR)$ which analogous to Thurston's canonical lamination of a closed hyperbolic surface \cite[\S8]{Thurston98}; and we study some analytic questions concerning the Euler-Lagrange equation for a $C^1$ tight form.
We have also included several conjectures concerning the canonical lamination and the Euler-Lagrange equation.

%%%%%%%%%%%%%%%
\subsection{Max flow min cut theorems}

\begin{figure}
	\begin{tikzpicture}
\draw[rounded corners=35pt](6,-1)--(4.2,-1)--(2,-2)--(0,0)--(2,2)--(4.2,1)--(7,1)--(9.2,2)--(11,0)
--(9,-2)--(6,-1);
\draw (1.5,0.2) arc (175:315:1cm and 0.5cm);
\draw (3,-0.28) arc (-30:180:0.7cm and 0.3cm);
\draw[blue] (1.8,-0.3) arc (0:360:0.63cm and 0.3cm);
\draw (7.5,0.2) arc (175:315:1cm and 0.5cm);
\draw (9,-0.28) arc (-30:180:0.7cm and 0.3cm);
\end{tikzpicture}

\caption{A closed surface containing a homologically minimizing closed geodesic $N^*$ (in blue).
In Figure \ref{discretized} we show a discretization of a tubular neighborhood $M'$ of $N^*$.
Adapted from \cite{Orduz20}. \label{to discretize}}
\end{figure}

Recall the max flow min cut theorem \cite[Chapter 7]{umesh2006algorithms}.
A \dfn{flow network} $\mathcal G = (V, E, c, s, t)$ consists of a finite directed graph $(V, E)$, a \dfn{capacity} $c: E \to \RR_+$, a \dfn{source} $s \in V$, and a \dfn{sink} $t \in V$.
For $v \in V$, let $I(v), O(v) \subseteq E$ be the sets of edges in and out of $v$, respectively.
A \dfn{flow} in $\mathcal G$ is a function $F: E \to \RR_+$ such that $F \leq c$, and which is \dfn{divergence-free} in the sense that for each $v \in V \setminus \{s, t\}$,
$$\sum_{e \in I(v)} F(e) = \sum_{e \in O(v)} F(e).$$
A \dfn{cut} of $\mathcal G$ is a partition $V = S \sqcup T$ such that $s \in S$, $t \in T$; in that case we write $E(S, T) := \{(v, w) \in E: v \in S, w \in T\}$.

\begin{theorem}[max flow min cut, graphs]
Let $\mathcal G$ be a flow network. Then 
$$\max_{F \in \mathrm{Flows}(\mathcal G)} \sum_{e \in O(s)} F(e) = \min_{(S, T) \in \mathrm{Cuts}(\mathcal G)} \sum_{e \in E(S, T)} c(e).$$
In particular, if $F$ is a maximal flow in $\mathcal G$ and $(S, T)$ is a minimal cut of $\mathcal G$, then for each $e \in E(S, T)$, $F(e) = c(e)$.
\end{theorem}

Suppose that we have a closed Riemannian manifold $M$ and a homology class $\sigma \in H_{d - 1}(M, \RR)$, and we want to compute the minimal area of a hypersurface in $\sigma$.
We propose to discretize this problem, as in \cite[Appendix A]{Freedman_2016}.

Assume that we know that $N$ is contained in a tubular neighborhood $M' = N_0 \times (-1, 1) \subset M$ where $[N_0] = \sigma$.
Then we can discretize $M'$ at a length scale $0 < h \ll 1$ to get a graph $(V, E)$, where between any two adjacent vertices there are edges in both directions, and we have two ``boundary'' vertices $s, t$, one for each component of $N_0 \times \{-1, 1\}$.
Then $\mathcal G := (V, E, 1, s, t)$ is a flow network.
Roughly speaking, flows correspond to closed $d - 1$-forms $F$ (which we view as fluxes of divergence-free vector fields) such that $\|F\|_{L^\infty} = 1$, while cuts correspond to hypersurfaces in $\sigma$, and minimal cuts correspond to homologically minimizing hypersurfaces.
If $N$ is a hypersurface in $\sigma$, $F$ is a closed $d - 1$-form, and $N^*$ is a homologically minimizing hypersurface in $\sigma$, then by Stokes' theorem,
\begin{equation}\label{one sided max flow min cut}
\int_N F = \int_{N^*} F \leq \mathrm{Area}(N^*).
\end{equation}
See Figures \ref{to discretize} and \ref{discretized}.
Ignoring all questions about the validity of our discretization, one expects from the max flow min cut theorem that (\ref{one sided max flow min cut}) is sharp:
We have the \dfn{continuous max flow min cut theorem} that for any closed homologically minimizing hypersurface $N^*$, there exists a closed $d - 1$-form $F$ such that $\|F\|_{L^\infty} = 1$ such that for any hypersurface $N$ in $\sigma$,
\begin{equation}\label{continuous max flow min cut}
\int_N F = \mathrm{Area}(N^*).
\end{equation}

% https://q.uiver.app/#q=WzAsMTksWzAsMywiXFxidWxsZXQiXSxbMSwwLCJcXGJ1bGxldCJdLFsxLDEsIlxcYnVsbGV0Il0sWzEsMiwiXFxidWxsZXQiXSxbMSwzLCJcXGJ1bGxldCJdLFsxLDQsIlxcYnVsbGV0Il0sWzEsNSwiXFxidWxsZXQiXSxbMSw2LCJcXGJ1bGxldCJdLFsyLDIsIlxcYnVsbGV0Il0sWzIsMywiXFxidWxsZXQiXSxbMiw0LCJcXGJ1bGxldCJdLFszLDAsIlxcYnVsbGV0Il0sWzMsMSwiXFxidWxsZXQiXSxbMywyLCJcXGJ1bGxldCJdLFszLDMsIlxcYnVsbGV0Il0sWzMsNCwiXFxidWxsZXQiXSxbMyw1LCJcXGJ1bGxldCJdLFszLDYsIlxcYnVsbGV0Il0sWzQsMywiXFxidWxsZXQiXSxbMCw0LCIiLDAseyJzdHlsZSI6eyJib2R5Ijp7Im5hbWUiOiJkYXNoZWQifSwiaGVhZCI6eyJuYW1lIjoibm9uZSJ9fX1dLFswLDMsIiIsMix7InN0eWxlIjp7ImJvZHkiOnsibmFtZSI6ImRhc2hlZCJ9LCJoZWFkIjp7Im5hbWUiOiJub25lIn19fV0sWzAsMiwiIiwyLHsic3R5bGUiOnsiYm9keSI6eyJuYW1lIjoiZGFzaGVkIn0sImhlYWQiOnsibmFtZSI6Im5vbmUifX19XSxbMCwxLCIiLDIseyJzdHlsZSI6eyJib2R5Ijp7Im5hbWUiOiJkYXNoZWQifSwiaGVhZCI6eyJuYW1lIjoibm9uZSJ9fX1dLFswLDUsIiIsMix7InN0eWxlIjp7ImJvZHkiOnsibmFtZSI6ImRhc2hlZCJ9LCJoZWFkIjp7Im5hbWUiOiJub25lIn19fV0sWzAsNiwiIiwyLHsic3R5bGUiOnsiYm9keSI6eyJuYW1lIjoiZGFzaGVkIn0sImhlYWQiOnsibmFtZSI6Im5vbmUifX19XSxbMCw3LCIiLDIseyJzdHlsZSI6eyJib2R5Ijp7Im5hbWUiOiJkYXNoZWQifSwiaGVhZCI6eyJuYW1lIjoibm9uZSJ9fX1dLFsxLDIsIiIsMix7InN0eWxlIjp7ImJvZHkiOnsibmFtZSI6ImRhc2hlZCJ9LCJoZWFkIjp7Im5hbWUiOiJub25lIn19fV0sWzIsMywiIiwyLHsic3R5bGUiOnsiYm9keSI6eyJuYW1lIjoiZGFzaGVkIn0sImhlYWQiOnsibmFtZSI6Im5vbmUifX19XSxbMyw0LCIiLDIseyJzdHlsZSI6eyJib2R5Ijp7Im5hbWUiOiJkYXNoZWQifSwiaGVhZCI6eyJuYW1lIjoibm9uZSJ9fX1dLFs0LDUsIiIsMix7InN0eWxlIjp7ImJvZHkiOnsibmFtZSI6ImRhc2hlZCJ9LCJoZWFkIjp7Im5hbWUiOiJub25lIn19fV0sWzUsNiwiIiwyLHsic3R5bGUiOnsiYm9keSI6eyJuYW1lIjoiZGFzaGVkIn0sImhlYWQiOnsibmFtZSI6Im5vbmUifX19XSxbNiw3LCIiLDEseyJzdHlsZSI6eyJib2R5Ijp7Im5hbWUiOiJkYXNoZWQifSwiaGVhZCI6eyJuYW1lIjoibm9uZSJ9fX1dLFsxNCwxOCwiIiwxLHsic3R5bGUiOnsiYm9keSI6eyJuYW1lIjoiZGFzaGVkIn0sImhlYWQiOnsibmFtZSI6Im5vbmUifX19XSxbMTUsMTgsIiIsMSx7InN0eWxlIjp7ImJvZHkiOnsibmFtZSI6ImRhc2hlZCJ9LCJoZWFkIjp7Im5hbWUiOiJub25lIn19fV0sWzE2LDE4LCIiLDEseyJzdHlsZSI6eyJib2R5Ijp7Im5hbWUiOiJkYXNoZWQifSwiaGVhZCI6eyJuYW1lIjoibm9uZSJ9fX1dLFsxNywxOCwiIiwxLHsic3R5bGUiOnsiYm9keSI6eyJuYW1lIjoiZGFzaGVkIn0sImhlYWQiOnsibmFtZSI6Im5vbmUifX19XSxbMTMsMTgsIiIsMSx7InN0eWxlIjp7ImJvZHkiOnsibmFtZSI6ImRhc2hlZCJ9LCJoZWFkIjp7Im5hbWUiOiJub25lIn19fV0sWzEyLDE4LCIiLDEseyJzdHlsZSI6eyJib2R5Ijp7Im5hbWUiOiJkYXNoZWQifSwiaGVhZCI6eyJuYW1lIjoibm9uZSJ9fX1dLFsxMSwxOCwiIiwxLHsic3R5bGUiOnsiYm9keSI6eyJuYW1lIjoiZGFzaGVkIn0sImhlYWQiOnsibmFtZSI6Im5vbmUifX19XSxbMTEsMTIsIiIsMSx7InN0eWxlIjp7ImJvZHkiOnsibmFtZSI6ImRhc2hlZCJ9LCJoZWFkIjp7Im5hbWUiOiJub25lIn19fV0sWzEyLDEzLCIiLDEseyJzdHlsZSI6eyJib2R5Ijp7Im5hbWUiOiJkYXNoZWQifSwiaGVhZCI6eyJuYW1lIjoibm9uZSJ9fX1dLFsxMywxNCwiIiwxLHsic3R5bGUiOnsiYm9keSI6eyJuYW1lIjoiZGFzaGVkIn0sImhlYWQiOnsibmFtZSI6Im5vbmUifX19XSxbMTQsMTUsIiIsMSx7InN0eWxlIjp7ImJvZHkiOnsibmFtZSI6ImRhc2hlZCJ9LCJoZWFkIjp7Im5hbWUiOiJub25lIn19fV0sWzE1LDE2LCIiLDEseyJzdHlsZSI6eyJib2R5Ijp7Im5hbWUiOiJkYXNoZWQifSwiaGVhZCI6eyJuYW1lIjoibm9uZSJ9fX1dLFsxNiwxNywiIiwxLHsic3R5bGUiOnsiYm9keSI6eyJuYW1lIjoiZGFzaGVkIn0sImhlYWQiOnsibmFtZSI6Im5vbmUifX19XSxbMTcsMTEsIiIsMSx7ImN1cnZlIjoyLCJzdHlsZSI6eyJib2R5Ijp7Im5hbWUiOiJkYXNoZWQifSwiaGVhZCI6eyJuYW1lIjoibm9uZSJ9fX1dLFsxMCw5LCIiLDEseyJjb2xvdXIiOlsyMzcsMTAwLDQ2XSwic3R5bGUiOnsiaGVhZCI6eyJuYW1lIjoibm9uZSJ9fX1dLFs5LDgsIiIsMSx7ImNvbG91ciI6WzIzNywxMDAsNDZdLCJzdHlsZSI6eyJoZWFkIjp7Im5hbWUiOiJub25lIn19fV0sWzgsMTAsIiIsMSx7ImN1cnZlIjotMiwiY29sb3VyIjpbMjM3LDEwMCw0Nl0sInN0eWxlIjp7ImhlYWQiOnsibmFtZSI6Im5vbmUifX19XSxbMSw3LCIiLDIseyJjdXJ2ZSI6LTIsInN0eWxlIjp7ImJvZHkiOnsibmFtZSI6ImRhc2hlZCJ9fX1dLFs4LDEsIiIsMSx7InN0eWxlIjp7ImJvZHkiOnsibmFtZSI6ImRhc2hlZCJ9LCJoZWFkIjp7Im5hbWUiOiJub25lIn19fV0sWzIsOCwiIiwxLHsic3R5bGUiOnsiYm9keSI6eyJuYW1lIjoiZGFzaGVkIn0sImhlYWQiOnsibmFtZSI6Im5vbmUifX19XSxbMyw4LCIiLDIseyJzdHlsZSI6eyJib2R5Ijp7Im5hbWUiOiJkYXNoZWQifSwiaGVhZCI6eyJuYW1lIjoibm9uZSJ9fX1dLFszLDksIiIsMix7InN0eWxlIjp7ImJvZHkiOnsibmFtZSI6ImRhc2hlZCJ9LCJoZWFkIjp7Im5hbWUiOiJub25lIn19fV0sWzQsOSwiIiwyLHsic3R5bGUiOnsiYm9keSI6eyJuYW1lIjoiZGFzaGVkIn0sImhlYWQiOnsibmFtZSI6Im5vbmUifX19XSxbNCw4LCIiLDIseyJzdHlsZSI6eyJib2R5Ijp7Im5hbWUiOiJkYXNoZWQifSwiaGVhZCI6eyJuYW1lIjoibm9uZSJ9fX1dLFs1LDksIiIsMix7InN0eWxlIjp7ImJvZHkiOnsibmFtZSI6ImRhc2hlZCJ9LCJoZWFkIjp7Im5hbWUiOiJub25lIn19fV0sWzQsMTAsIiIsMix7InN0eWxlIjp7ImJvZHkiOnsibmFtZSI6ImRhc2hlZCJ9LCJoZWFkIjp7Im5hbWUiOiJub25lIn19fV0sWzUsMTAsIiIsMix7InN0eWxlIjp7ImJvZHkiOnsibmFtZSI6ImRhc2hlZCJ9LCJoZWFkIjp7Im5hbWUiOiJub25lIn19fV0sWzYsMTAsIiIsMix7InN0eWxlIjp7ImJvZHkiOnsibmFtZSI6ImRhc2hlZCJ9LCJoZWFkIjp7Im5hbWUiOiJub25lIn19fV0sWzcsMTAsIiIsMix7InN0eWxlIjp7ImJvZHkiOnsibmFtZSI6ImRhc2hlZCJ9LCJoZWFkIjp7Im5hbWUiOiJub25lIn19fV0sWzgsMTEsIiIsMix7InN0eWxlIjp7ImJvZHkiOnsibmFtZSI6ImRhc2hlZCJ9LCJoZWFkIjp7Im5hbWUiOiJub25lIn19fV0sWzgsMTIsIiIsMix7InN0eWxlIjp7ImJvZHkiOnsibmFtZSI6ImRhc2hlZCJ9LCJoZWFkIjp7Im5hbWUiOiJub25lIn19fV0sWzgsMTMsIiIsMix7InN0eWxlIjp7ImJvZHkiOnsibmFtZSI6ImRhc2hlZCJ9LCJoZWFkIjp7Im5hbWUiOiJub25lIn19fV0sWzgsMTQsIiIsMix7InN0eWxlIjp7ImJvZHkiOnsibmFtZSI6ImRhc2hlZCJ9LCJoZWFkIjp7Im5hbWUiOiJub25lIn19fV0sWzksMTMsIiIsMix7InN0eWxlIjp7ImJvZHkiOnsibmFtZSI6ImRhc2hlZCJ9LCJoZWFkIjp7Im5hbWUiOiJub25lIn19fV0sWzksMTQsIiIsMix7InN0eWxlIjp7ImJvZHkiOnsibmFtZSI6ImRhc2hlZCJ9LCJoZWFkIjp7Im5hbWUiOiJub25lIn19fV0sWzksMTUsIiIsMix7InN0eWxlIjp7ImJvZHkiOnsibmFtZSI6ImRhc2hlZCJ9LCJoZWFkIjp7Im5hbWUiOiJub25lIn19fV0sWzEwLDE0LCIiLDIseyJzdHlsZSI6eyJib2R5Ijp7Im5hbWUiOiJkYXNoZWQifSwiaGVhZCI6eyJuYW1lIjoibm9uZSJ9fX1dLFsxMCwxNSwiIiwyLHsic3R5bGUiOnsiYm9keSI6eyJuYW1lIjoiZGFzaGVkIn0sImhlYWQiOnsibmFtZSI6Im5vbmUifX19XSxbMTAsMTYsIiIsMix7InN0eWxlIjp7ImJvZHkiOnsibmFtZSI6ImRhc2hlZCJ9LCJoZWFkIjp7Im5hbWUiOiJub25lIn19fV0sWzEwLDE3LCIiLDIseyJzdHlsZSI6eyJib2R5Ijp7Im5hbWUiOiJkYXNoZWQifSwiaGVhZCI6eyJuYW1lIjoibm9uZSJ9fX1dXQ==

\begin{figure}
\[\begin{tikzcd}
	& \bullet && \bullet \\
	& \bullet && \bullet \\
	& \bullet & \bullet & \bullet \\
	s & \bullet & \bullet & \bullet & t \\
	& \bullet & \bullet & \bullet \\
	& \bullet && \bullet \\
	& \bullet && \bullet
	\arrow[dashed, no head, from=4-1, to=4-2]
	\arrow[dashed, no head, from=4-1, to=3-2]
	\arrow[dashed, no head, from=4-1, to=2-2]
	\arrow[dashed, no head, from=4-1, to=1-2]
	\arrow[dashed, no head, from=4-1, to=5-2]
	\arrow[dashed, no head, from=4-1, to=6-2]
	\arrow[dashed, no head, from=4-1, to=7-2]
	\arrow[dashed, no head, from=1-2, to=2-2]
	\arrow[dashed, no head, from=2-2, to=3-2]
	\arrow[dashed, no head, from=3-2, to=4-2]
	\arrow[dashed, no head, from=4-2, to=5-2]
	\arrow[dashed, no head, from=5-2, to=6-2]
	\arrow[dashed, no head, from=6-2, to=7-2]
	\arrow[dashed, no head, from=4-4, to=4-5]
	\arrow[dashed, no head, from=5-4, to=4-5]
	\arrow[dashed, no head, from=6-4, to=4-5]
	\arrow[dashed, no head, from=7-4, to=4-5]
	\arrow[dashed, no head, from=3-4, to=4-5]
	\arrow[dashed, no head, from=2-4, to=4-5]
	\arrow[dashed, no head, from=1-4, to=4-5]
	\arrow[dashed, no head, from=1-4, to=2-4]
	\arrow[dashed, no head, from=2-4, to=3-4]
	\arrow[dashed, no head, from=3-4, to=4-4]
	\arrow[dashed, no head, from=4-4, to=5-4]
	\arrow[dashed, no head, from=5-4, to=6-4]
	\arrow[dashed, no head, from=6-4, to=7-4]
	\arrow[curve={height=12pt}, dashed, no head, from=7-4, to=1-4]
	\arrow[color={rgb,255:red,0;green,0;blue,235}, no head, from=5-3, to=4-3]
	\arrow[color={rgb,255:red,0;green,0;blue,235}, no head, from=4-3, to=3-3]
	\arrow[color={rgb,255:red,0;green,0;blue,235}, curve={height=-12pt}, no head, from=3-3, to=5-3]
	\arrow[curve={height=-12pt}, dashed, no head, from=1-2, to=7-2]
	\arrow[dashed, no head, from=3-3, to=1-2]
	\arrow[dashed, no head, from=2-2, to=3-3]
	\arrow[dashed, no head, from=3-2, to=3-3]
	\arrow[dashed, no head, from=3-2, to=4-3]
	\arrow[dashed, no head, from=4-2, to=4-3]
	\arrow[dashed, no head, from=4-2, to=3-3]
	\arrow[dashed, no head, from=5-2, to=4-3]
	\arrow[dashed, no head, from=4-2, to=5-3]
	\arrow[dashed, no head, from=5-2, to=5-3]
	\arrow[dashed, no head, from=6-2, to=5-3]
	\arrow[dashed, no head, from=7-2, to=5-3]
	\arrow[dashed, no head, from=3-3, to=1-4]
	\arrow[dashed, no head, from=3-3, to=2-4]
	\arrow[dashed, no head, from=3-3, to=3-4]
	\arrow[dashed, no head, from=3-3, to=4-4]
	\arrow[dashed, no head, from=4-3, to=3-4]
	\arrow[dashed, no head, from=4-3, to=4-4]
	\arrow[dashed, no head, from=4-3, to=5-4]
	\arrow[dashed, no head, from=5-3, to=4-4]
	\arrow[dashed, no head, from=5-3, to=5-4]
	\arrow[dashed, no head, from=5-3, to=6-4]
	\arrow[dashed, no head, from=5-3, to=7-4]
\end{tikzcd}\]
\caption{With $M'$ as in Figure \ref{to discretize}, the discretization of $M'$ is a flow network $\mathcal G$ in which the discretization of $N^*$ is a minimal cut (in blue and solid).
Therefore the maximal flow $F$ through $\mathcal G$ must have flux density $1$ through the discretization of $N^*$, and hence the closed $d - 1$-form corresponding to $F$ must be the area form on $N^*$. \label{discretized}}
\end{figure}

\subsubsection{Federer's version}
Modulo regularity issues, the continuous max flow min cut theorem was proven by Federer \cite[\S4.15]{Federer1974}.
In the modern formulation of Federer's theorem, we introduce the stable norm on homology:

\begin{definition}
The \dfn{stable norm} $\Mass(\sigma)$ of a homology class $\sigma \in H_{d - 1}(M, \RR)$ is the infimum of the mass $\Mass(N)$ among all representative chains $N$.
The \dfn{costable norm} $\Comass(\rho)$ of a cohomology class $\rho \in H^{d - 1}(M, \RR)$ is the infimum of $\|F\|_{L^\infty}$ among all representative forms $F$.
\end{definition}

\begin{theorem}[max flow min cut, Federer's version]\label{Federer}
Let $M$ be a closed Riemannian manifold and $\sigma \in H_{d - 1}(M, \RR)$.
Then there exists a cohomology class $\rho \in H^{d - 1}(M, \RR)$ such that $\Comass(\rho) \leq 1$, and 
\begin{equation}\label{Federer duality}
\Mass(\sigma) = \langle \rho, \sigma\rangle.
\end{equation}
\end{theorem}

Theorem \ref{Federer} has found myriad applications, including in computational geometry \cite{sullivan1990crystalline} and string theory \cite{Freedman_2016}; both references here observe that Federer has essentially proven a max flow min cut theorem.
However, Federer's theorem is nothing more than the Hanh-Banach theorem in the special case of the (finite-dimensional) Banach space $H_{d - 1}(M, \RR)$ equipped with the stable norm.
It does not say anything about the pointwise duality of the chains and cochains which realize the infima in the definitions of the stable and costable norms.\footnote{As we have formulated Theorem \ref{Federer}, it is not clear whether the infima are attained at all, but for sufficiently general notions of chain and cochain, this is not hard to show.}

\subsubsection{Thurston's version}
An analogous result to Federer's appears in Thurston's work on best Lipschitz maps \cite{Thurston98}.
A \dfn{best Lipschitz map} $u$ is a map which minimizes its Lipschitz constant, among all maps homotopic to $u$.
% Thurston's motivation was to define a Finsler metric on the Teichm\"uller space $\widetilde{\mathscr M_g}$: the \dfn{Thurston asymmetric distance} between two hyperbolic structures on a closed surface $S$ of genus $g \geq 2$ is the Lipschitz constant of a best Lipschitz map homotopic to $\id_S$.
% Thurston's asymmetric metric is a particularly appealing geometry on $\widetilde{\mathscr M_g}$ because of its intimate connection with the structure of geodesic laminations on hyperbolic surfaces \cite{Thurston98, Gu_ritaud_2017}:
Such maps are intimately connected to geodesic laminations in hyperbolic surfaces:

\begin{theorem}[max flow min cut, Thurston's version]\label{existence of thurston lamination}
Let $f: M \to N$ be a homeomorphism of closed hyperbolic surfaces.
Then there exists a best Lipschitz map $u$ homotopic to $f$, and a measured geodesic lamination $\mu$ on $M$, such that $\mu$ pushes forward to a measured geodesic lamination on $N$, and
\begin{equation}\label{L is K}
	\Lip(u) = \frac{\Mass(u_* \mu)}{\Mass(\mu)}.
\end{equation}
Moreover, $\mu$ maximizes the ratio $\Mass(u_* \lambda)/\Mass(\lambda)$ among measured laminations on $M$.
\end{theorem}

Thurston established Theorem \ref{existence of thurston lamination} in a tour de force of geometric topology, but he conjectured that a proof of his result ``should be feasible with a simpler proof based on more general principles -- in particular, the max flow min cut principle, convexity, and $L^0 \leftrightarrow L^\infty$ duality'' \cite[Abstract]{Thurston98}.

Theorem \ref{existence of thurston lamination} does not follow from our main theorem; however, the analogous result for maps from closed hyperbolic surfaces to $\Sph^1$, which was proven by Daskalopolous and Uhlenbeck \cite{daskalopoulos2020transverse}, is a special case of our main result.
Our work, like \cite{daskalopoulos2020transverse}, is based on the idea of approximating an $L^\infty$ functional by $L^p$ functionals.
As Daskalopolous and Uhlenbeck have given a proof of a version of Theorem \ref{existence of thurston lamination} using similar methods \cite{daskalopoulos2022,daskalopoulos2023}, it may be possible to adapt our results to manifold-valued maps.

\subsubsection{Bangert and Cui's version}
The statement in the literature that we are aware of, which is closest to our main theorem, is due to Bangert and Cui \cite{bangert_cui_2017}.
To emphasize the analogy with best Lipschitz maps, we call a form \dfn{best comass} if it minimizes its comass:

\begin{definition}
The \dfn{comass} of a closed $d - 1$-form $F$ is $\|F\|_{L^\infty}$.
A form $F$ has \dfn{best comass} if it minimizes its comass among all forms cohomologous to $F$.
\end{definition}

\begin{definition}
A \dfn{calibration} is a measurable closed $d - 1$-form $F$, such that $\|F\|_{L^\infty} = 1$.
Given a calibration $F$, a hypersurface $N$ is $F$-\dfn{calibrated} if the pullback of $F$ to $N$ is the area form on $N$.
\end{definition}

Observe that a best comass form is a calibration iff the costable norm of its cohomology class is $1$.
Moreover, the fundamental theorem of calibrated geometry \cite{Harvey82} asserts that every $F$-calibrated hypersurface $N$ is minimal, and in fact homologically area-minimizing if $N$ is closed.
With these preliminaries in mind, we may now state the Bangert--Cui theorem \cite[Theorem 5.1]{bangert_cui_2017}:

\begin{theorem}[max flow min cut, Bangert and Cui's version]\label{BangertCui}
Let $F$ be a continuous best comass calibration on a closed Riemannian manifold of dimension $\leq 7$.
Then there exists a measured oriented minimal lamination $\lambda$ such that every leaf of $\lambda$ is $F$-calibrated.
In particular, 
$$\Mass(\lambda) = \langle [\lambda], [F]\rangle.$$
\end{theorem}

The idea of Theorem \ref{BangertCui} is to use Theorem \ref{Federer} to find a homology class $\sigma$ dual to $[F]$, and then consider a minimizing representative $\lambda$ of $\sigma$, so that $\Mass(\lambda)$ is the stable norm of $\sigma$.
However, most calibrations which appear in nature are measurable and not necessarily continuous \cite[\S1]{bangert_cui_2017}, and it is natural to restrict attention to \emph{Lipschitz} laminations \cite[Remark 2.3]{bangert_cui_2017}; thus, a more natural formulation of Theorem \ref{BangertCui} would not require that $F$ is continuous, and would assert that $\lambda$ is Lipschitz.

A more serious flaw with Theorem \ref{BangertCui} is that a key ingredient in its proof is the structure theorem for locally minimal $d - 1$-currents.
This theorem was announced by Auer and Bangert \cite{Auer01} but its proof has not appeared, though Victor Bangert has kindly allowed me to view a draft proof.
I recently established a similar result to the structure theorem \cite[Theorem B]{BackusCML} by somewhat different methods, which also gives the Lipschitz regularity; this motivated me to prove a stronger version of Theorem \ref{BangertCui} without any gaps.

%%%%%%%%%%%%%%%%%%%%%%%%%%%%%%%%%%%%%%%%%

\subsection{Tight forms and functions of least gradient}
We shall prove a version of Theorem \ref{BangertCui}.
The idea, as in \cite{daskalopoulos2020transverse}, is to consider $d - 1$-forms which behave analogously to $p$-harmonic functions, and take the limit $p \to \infty$ to get a best comass form which behaves analogously to an $\infty$-harmonic function.
At the same time we consider the convex dual problem.
The best comass forms will be ``maximal flows''; their dual ``minimal cuts'' will be functions of least gradient.

To be more precise, let $(p, q)$ be a H\"older pair (thus $1/p + 1/q = 1$) such that $d < p < \infty$.
Motivated by the $p$-Laplace equation $\dif^*(|\dif v|^{p - 2} \dif v) = 0$, we introduce \dfn{$p$-tight} forms, which are closed $d-1$-forms which solve the system of PDE
$$\dif^*(|F|^{p - 2} F) = 0.$$
Given a $p$-tight form, the $\pi_1(M)$-equivariant function $u$ on the universal cover such that
$$\dif u = (-1)^{d - 1} |F|^{p - 2} \star F$$
is $q$-harmonic -- in other words, $u$ is a solution of the $q$-Laplace equation 
$$\dif^*(|\dif u|^{q - 2} \dif u) = 0.$$
A function $u$ has \dfn{least gradient} if it minimizes $\int_M \star |\dif u|$.
Our first theorem constructs a best comass form, and a dual function of least gradient, by taking limits of $p$-tight forms and their dual $q$-harmonic functions.

\begin{mainthm}\label{existence of infinity tight forms}
Let $\rho \in H^{d - 1}(M, \RR)$ be a cohomology class.
Let $(F_p, u_q)$ be the family of dual pairs of $p$-tight forms and $q$-harmonic functions, suitably normalized, with $[F_p] = \rho$ and $(p, q)$ ranging over H\"older pairs with $d < p < \infty$.
Then there exists a pair $(F, u)$ such that as $p \to \infty$ along a subsequence, $F_p \to F$ weakly in $L^r$ for any $d < r < \infty$, and $u_q \to u$ weakly in $BV$, with the following properties:
\begin{enumerate}
\item $F$ has best comass.
\item $u$ has least gradient.
\item The product of distributions $\dif u \wedge F$ is well-defined.
\item We have the duality relation
\begin{equation}\label{max flow mean cut}
\Comass(\rho) \star |\dif u| = \dif u \wedge F.
\end{equation}
\end{enumerate}
\end{mainthm}

This is a combination of Propositions \ref{existence infinity} and \ref{existence 1}.
We call the best comass form $F$ a \dfn{tight} form.

We highlight the max flow min cut-type formula (\ref{max flow mean cut}) as the main point of the theorem.
It is crucial to the proof that $u$ is $1$-harmonic, and allows us to prove this without a careful analysis of the limiting behavior of $q$-harmonic functions as in \cite[Theorem 2.4]{Mazon14}, or of $p$-tight forms as in \cite[\S6]{daskalopoulos2020transverse}.
On the other hand, if $\Comass(\rho) = 1$, then we shall be able to use (\ref{max flow mean cut}) to show that the level sets of $u$ are $F$-calibrated.
This will give us a pointwise version of (\ref{Federer duality}): the minimizing cochain calibrates the minimizing chain!

In addition to $p$-approximation, another ingredient in the proof of Theorem \ref{existence of infinity tight forms} is the geometric measure theory of closed $L^\infty$ $d - 1$-forms, which we believe to be of independent interest. 
In particular, we show that, given an immersion $\iota: N \to M$ of codimension $1$, the normal trace map $F \mapsto \iota^* F$, defined on closed $d - 1$-forms, sends $L^\infty(M)$ to $L^\infty(N)$.
This allows us to define $\dif u \wedge F$ and $\int_N F$.
The reader may compare to the much better-known $L^2$ normal trace theorem, which asserts that the normal trace map sends $L^2(M)$ to $W^{-1/2, 2}(N)$ \cite[Chapter 2]{cessenat1996mathematical}.

%%%%%%%%%%%%%%%%%%

\subsection{Calibrated laminations}
Suppose that $\rho \in H^{d - 1}(M, \RR)$ satisfies $\Comass(\rho) = 1$.
If $F$ is a continuous tight representative of $\rho$, then by the Bangert--Cui theorem, $F$ calibrates a minimal lamination.
However, a proof that tight forms are continuous is out of reach.\footnote{For domains in $\RR^2$, tight forms are $C^\alpha$ \cite{Evans08}, but it is unlikely that this argument generalizes.}
On the other hand, we know by \cite[Theorem B]{BackusCML} that the level sets of a function of least gradient form a measured oriented Lipschitz minimal lamination.

\begin{definition}
Let $M$ be a closed Riemannian manifold of dimension $d \leq 7$, and $\rho \in H^{d - 1}(M, \RR)$.
We can define a measured oriented minimal lamination $\mu$, by considering a tight representative $F$ of $\rho$, letting $u$ be a dual function of least gradient to $F$, and letting $\mu$ be the lamination induced by $u$.
We call $\mu$ a \dfn{measured stretch lamination} associated to $\rho$.
\end{definition}

Our main theorem is the combination of Propositions \ref{MCL contains Thurston} and \ref{L equals K}, and follows quickly from Theorem \ref{existence of infinity tight forms} and the theory surrounding it.
As with Thurston's theorem, we shall characterize the measured stretch lamination as a lamination which is maximally stretched in the sense of (\ref{L is K}).
To state the theorem, we let $[\lambda]$ denote the homology class of a measured oriented lamination $\lambda$.

\begin{mainthm}\label{lams are calibrated}
Suppose that $M$ is a closed Riemannian manifold of dimension $d \leq 7$, and let $\rho \in H^{d - 1}(M, \RR)$ be nonzero.
Then there is a measured stretch lamination associated to $\rho$.
Moreover, if $\kappa$ is a measured stretch lamination associated to $\rho$, and $F$ is a best comass representative of $\rho$, then $\kappa$ is $F/\Comass(\rho)$-calibrated, and for $\lambda$ ranging over measured oriented laminations,
\begin{equation}\label{duality between stable and comass}
\Comass(\rho) = \sup_\lambda \frac{\langle \rho, [\lambda]\rangle}{\Mass(\lambda)} = \frac{\langle \rho, [\kappa]\rangle}{\Mass(\kappa)}.
\end{equation}
\end{mainthm}

In order to make connection with Thurston's max flow/min cut theorem, we recall that the maximally stretched measured lamination given by Theorem \ref{existence of thurston lamination} is contained in a larger lamination, called the \dfn{canonical lamination} of the homotopy class $[f]$ \cite{Thurston98,Gu_ritaud_2017}.
The canonical lamination is the largest geodesic lamination whose leaves are maximally stretched by the best Lipschitz maps homotopic to $f$.
We construct an analogous lamination for best comass calibrations; this is Proposition \ref{existence of canonical lamination}.

\begin{mainthm}\label{existence of calibrated lam}
Suppose that $M$ is a closed Riemannian manifold of dimension $d \leq 7$, and let $\rho \in H^{d - 1}(M, \RR)$ satisfy $\Comass(\rho) = 1$.
Then there is a Lipschitz lamination $\lambda_\rho$, the \dfn{canonical lamination} of $\rho$, such that a complete hypersurface $N$ is a leaf of $\lambda_\rho$ iff, for every best comass representative $F$ of $\rho$, $N$ is $F$-calibrated.
The support of $\lambda_\rho$ is contained in the maximum comass locus of every best comass representative $F$.
\end{mainthm}

In addition to Theorem \ref{lams are calibrated}, which is necessary to show that $\lambda_\rho$ is nonempty, and curvature bounds on homologically minimizing hypersurfaces \cite{Schoen75, Schoen81}, which are necessary to construct the Lipschitz flow boxes \cite{BackusCML}, we also have to show that if $v$ is a harmonic function, then almost every zero of $v$ is a single zero.
From this it follows that any two calibrated hypersurfaces which intersect, must somewhere intersect transversely, contradicting the definition of calibration.
The highlight of the proof is an application of the rectifiability of the set $P$ of double zeroes of $v$ \cite{Hardt89} to bound the cohomological dimension of $P$.

We then study the structure of the canonical lamination $\lambda_\rho$.
The canonical lamination is covered by the measured stretch laminations which are associated to $\rho$, along with some leaves which do not admit a transverse measure.
The measured stretch laminations, in turn, are exactly those homologically minimizing laminations which represent homology classes in the convex set 
$$\rho^* := \{\alpha \in H_{d - 1}(M, \RR): \langle \rho, \alpha\rangle = \Mass(\alpha) = 1\}.$$
A special role is played by the extreme points of $\rho^*$, which are represented by measured stretch laminations with no proper sublamination.
We shall not attempt to summarize our results on the structure of $\lambda_\rho$ into a clean theorem here, but refer the reader to \S\ref{canonical structure}.
In \S\ref{Teichmuller}, we explain how these results compare and contrast with the results of Thurston's canonical lamination.
In \S\ref{canonical conjectures}, we give some open problems about the structure of the canonical lamination.

%%%%%%%%%%%%%%%%%%%%
\subsection{Convexity of the stable unit ball}
It is known that convexity properties of the stable unit ball 
$$B := \{\alpha \in H_{d - 1}(M, \RR): \Mass(\alpha) \leq 1\}$$
are intimately related to the structure of homologically minimizing laminations \cite{Thurston98,Auer01}.
From our perspective, this is because $\rho^*$ is a \dfn{flat} of $\partial B$ -- that is, the intersection of $\partial B$ with a supporting hyperplane of $B$.

Though the proofs were never published, Auer and Bangert announced several results on the structure of $B$ using laminations \cite{Auer01}.
In fact, some of these results immediately follow from the structure of the canonical lamination, and we now record them.

Recall that the exterior product on the cohomology ring $H^\bullet(M, \RR)$ induces, by Poincar\'e duality, an \dfn{intersection product}
\begin{align*}
H_{d - k}(M, \RR) \times H_{d - \ell}(M, \RR) &\to H_{d - k - \ell}(M, \RR) \\
(\alpha, \beta) &\mapsto \alpha \cdot \beta.
\end{align*}
Also recall that $B$ is strictly convex iff $\partial B$ has no flats except singletons.
Then we have the following theorem, which is Proposition \ref{flats are nonintersecting}:

\begin{mainthm}\label{convexity summary}
Let $B$ be the stable unit ball of a closed Riemannian manifold $M$ of dimension $\leq 7$, and let $S$ be a flat of $\partial B$.
Then for any $\alpha, \beta \in S$, $\alpha \cdot \beta = 0$.
\end{mainthm}

Taken in the contrapositive, Theorem \ref{convexity summary} gives a condition for strict convexity of $B$.
In particular, if $M$ is homeomorphic to a torus, then $B$ is strictly convex (see Corollary \ref{torus convex}).

%%%%%%%%%%%%%%%%%%%%%
\subsection{The PDE for a tight form}
Tight forms have best comass, and are obtained by taking limits of solutions of $L^p$ variational problems.
Thus they are solutions of an $L^\infty$ variational problem, and one expects them to solve a PDE analogous to the $\infty$-Laplacian.
However, the theory of the $\infty$-Laplacian is largely built around the maximum principle and viscosity solutions, neither of which have been adequately fleshed out for systems of PDE at present \cite{Katzourakis2018OnAV,Sheffield12}.

\todo{Write this out}

%%%%%%%%%%%%%%%%%%%%%
\subsection{Outline of the paper}
We begin by discussing preliminaries:
\begin{itemize}
\item In \S\ref{prevResults}, we recall some well-known geometric measure theory, and results from \cite{BackusCML,Mazon14} on $1$-harmonic functions and minimal laminations.
\item In \S\ref{comass sec}, we study closed $L^\infty$ $d - 1$-forms. We introduce their local comass, an analogous quantity to the local Lipschitz constant. We also study laminations calibrated by such forms.
\end{itemize}
We then study the duality between minimal laminations and tight calibrations without any assumptions on the regularity of the tight calibration:
\begin{itemize}
\item In \S\ref{tight forms sec}, after a quick review of convex duality, we construct the tight form in each cohomology class, and its $1$-harmonic conjugate, proving Theorem \ref{existence of infinity tight forms}.
\item In \S\ref{MCL sec}, we study the maximum comass locus of a form of best comass, the measured stretch laminations associated to its cohomology class $\rho$, and the canonical lamination of $\rho$.
We prove Theorems \ref{lams are calibrated}, \ref{existence of calibrated lam}, and \ref{convexity summary}.
We also discuss the structure of the canonical lamination, its connection to Thurston's chain-recurrent lamination, and intersections of minimal hypersurfaces.
\end{itemize}
We have then exhausted what seems to be possible without an appeal to the Euler-Lagrange equation for a tight form:
\begin{itemize}
\item In \S\ref{infinityMax}, we study the Euler-Lagrange equation for a tight form under rather strong regularity assumptions. Of course we expect that with a suitable generalization of the notion of viscosity solution, the results in this section should hold without the regularity assumption. This section is independent of \S\ref{MCL sec}.
\item In \S\ref{open problems}, we state some open problems and conjectures concerning the variational problems and laminations studied here.
\end{itemize}
We include appendices proving technical facts, presumably well-known to experts, that we needed but could not find suitable references for:
\begin{itemize}
\item In Appendix \ref{local Hodge appendix}, we prove some estimates in local $L^p$ Hodge theory.
\item In Appendix \ref{nodal appendix}, we show that the generic zero of a harmonic function is a single zero.
\end{itemize}


%%%%%%%%%%%%%%%%%%%%%%
\subsection{Acknowledgements}
I would like to thank Georgios Daskalopolous and Karen Uhlenbeck for suggesting this project, providing helpful comments, and providing me with an early draft of the manuscript \cite{daskalopoulos2023} which was a major source of inspiration for this work.
I would also like to thank Victor Bangert for allowing me to view a draft of the proof announced in \cite{Auer01}.

This research was supported by the National Science Foundation's Graduate Research Fellowship Program under Grant No. DGE-2040433.


%%%%%%%%%%%%%%%%%%%%%%%%%%%%%%%%%%%%%%%%%%
\section{Preliminaries}\label{prevResults}
\subsection{Locally normal currents}
In order to fix conventions, we recall some well-known measure theory.
We will mainly use \cite{simon1983GMT} as a reference.

The sheaf of $\ell$-forms is denoted $\Omega^\ell$.
We assume that $\ell$-forms are $L^1_\loc$, but \emph{not} that they are continuous; hence $\dif$ must be meant in the sense of distributions.
To avoid confusion, we write $H^\ell$ for de Rham cohomology, but \emph{never} a Sobolev space (which shall only be denoted by $W^{\ell, p}$), nor a Hausdorff measure (which shall be denoted $\mathcal H^\ell$).
We let $\dif V = \star 1$ be the volume form on $M$, and for an $\ell$-rectifiable set $\tau$, we let $\dif S_\tau := \dif \mathcal H^\ell|_\tau$.

\begin{definition}
An \dfn{$\ell$-blade} is the wedge product $v = v_1 \wedge \cdots \wedge v_\ell$ of vectors $v_1, \dots, v_\ell$.
The \dfn{comass} $|\varphi|$ of an $\ell$-covector $\varphi$ is the supremum of $\langle \varphi, v\rangle/|v|$, taken over all nonzero $\ell$-blades $v$.
\end{definition}

\begin{definition}
By an $\ell$-\dfn{current of locally finite mass} $T$ we mean a continuous linear functional on the space $C^0_\cpt(M, \Omega^\ell)$ of continuous $\ell$-forms of compact support.
If we do not specify otherwise, by a \dfn{current} we shall mean a current of locally finite mass.
We write $\int_M T \wedge \varphi$ for the dual pairing of a current and a form.
The duality norm of $T$ is its \dfn{mass}, namely for an open set $U \subseteq M$,
$$\Mass_U(T) := \sup_{\substack{\supp \varphi \Subset U \\ \||\varphi|\|_{C^0} \leq 1}} \int_U T \wedge \varphi.$$
We write $\Mass(T) := \Mass_M(T)$.
\end{definition}

If $T$ represents a simplex $\sigma$, then $\Mass(T)$ is the surface area of $\sigma$.

We write $\dif T$ for the exterior derivative of a current $T$, which is also known as $-\partial T$ \cite[\S26]{simon1983GMT}.
If $\dif T$ is an $\ell - 1$-current of locally finite mass, we say that $T$ is \dfn{locally normal}.
In particular, a locally normal $d$-current is the same thing as a function of locally bounded variation, and a locally normal $0$-current is the same thing as a signed Radon measure.


%%%%%%%%%%%%%%%%%%%%%
\subsection{Trace theorem and coarea formula}
We would like to be able to compute the comass of a form $F$ by integration of $F$ along chains, and we would like to take $\dif u \wedge F$ where $u \in BV_\loc(M)$.
Unfortunately, $F$ may only be defined almost everywhere, and then it is not clear that such an integral is well-defined; nor is it clear that the cohomology class of $F$ is well-defined (so that the notion of ``best comass'' makes no sense).
Here we show that for closed forms $F$ in $L^p$, suitable chains $\sigma$, and $u \in BV_\loc(M)$, $\int_\sigma F$ and $\dif u \wedge F$ are well-defined.

\begin{definition}
An $\ell$-current $\sigma$ is \dfn{rectifiable} if there exists an $\ell$-rectifiable set $N$, and a $\dif S_N$-measurable $\ell$-blade field $v$, such that for any $C^0_\cpt$ $\ell$-form $\varphi$,
$$\int_M \sigma \wedge \varphi = \int_N \langle \varphi, v\rangle \dif S_N,$$
and for $\dif S_N$-almost every $x \in N$, $|v(x)| \in \ZZ$.
The rectifiable current $\sigma$ is \dfn{integral} if, in addition, $\dif \sigma$ is rectifiable.
To emphasize that integral currents $\sigma$ can be represented by integration along rectifiable sets, we write $\partial \sigma := -\dif \sigma$ and $\int_\sigma \varphi := \int_M \sigma \wedge \varphi$.
\end{definition}

\begin{lemma}\label{local trace theorem}
Suppose that there is a bi-Lipschitz diffeomorphism $M \cong \Ball^d$.
Let $\tau$ be an integral $d - 1$-current and $\psi \in C^1(M)$.
Then for any $d < p < \infty$, $F \mapsto \int_\tau \psi F$ is a continuous linear function on the space of $L^p$ closed $d - 1$-forms.
\end{lemma}
\begin{proof}
First suppose that $F, G$ are closed $C^1$ $d - 1$-forms, and let $A, B$ be the $d - 2$-forms obtained from Lemma \ref{Hodge theorem}.
By integration by parts and the Sobolev embedding theorem,
\begin{align*}
	\left|\int_\tau \psi(F - G)\right| 
	&\leq \left|\int_{\partial \tau} \psi (A - B)\right| + \left|\int_\tau \dif \psi \wedge (A - B)\right| \\
	&\lesssim_{\tau, \psi} \|A - B\|_{C^0} \lesssim_p \|A - B\|_{L^p} \lesssim_p \|F - G\|_{L^p}.
\end{align*}
If $F, G$ are not necessarily $C^1$, then by Lemma \ref{mollification of closed forms}, the $C^1$ closed $d - 1$-forms are dense in the space of $L^p$ closed $d - 1$-forms, so the desired estimate holds by approximating $F, G$ by $C^1$ closed $d - 1$-forms.
\end{proof}

\begin{proposition}[trace theorem]\label{integration is welldefined}
Let $\tau$ be a compactly supported integral $d-1$-current.
Then:
\begin{enumerate}
\item For any $d < p \leq \infty$ and $\psi \in C^1(M)$, $F \mapsto \int_\tau \psi F$ extends to a continuous linear functional on the space of $L^p$ closed $d-1$-forms.
\item For any $d < p \leq \infty$, the cohomology class of any $L^p$ closed $d - 1$-form is well-defined.
\item Let $F$ be an $L^\infty$ closed $d - 1$-form. Then for every $\psi \in C^0(M)$,
\begin{equation}\label{integral over chain is linfinity}
	\int_\tau \psi F \leq \|F\|_{L^\infty} \|\psi\|_{L^1(\tau)}.
\end{equation}
\end{enumerate}
\end{proposition}
\begin{proof}
We can find a partition of unity $(\chi_\alpha)$ subordinate to an open cover $(U_\alpha)$ such that $U_\alpha$ is bi-Lipschitz diffeomorphic to $\Ball^d$.
Then we may replace $(U_\alpha)$ by a finite subcover $U_1, \dots, U_n$ of a neighborhood of $\supp \tau$.
Applying Lemma \ref{local trace theorem} with $\psi$ replaced by $\psi \chi_k$, we obtain for $F, G$ closed $L^p$ $d - 1$-forms and $p < \infty$,
$$\left|\int_\tau \psi(F - G)\right| \leq \sum_{k = 1}^n \left|\int_\tau \psi \chi_k (F - G)\right| \lesssim_\tau \|F - G\|_{L^p}$$
which gives the continuity in $L^p$.
This also implies that the cohomology is well-defined.

We now handle the case $p = \infty$.
Let $d < q < \infty$ and let $U$ be a small neighborhood of $\supp \tau$; since $\supp \tau$ is compact, if $F, G \in L^\infty$, then $F, G \in L^q(U)$ and 
$$\left|\int_\tau \psi(F - G)\right| \lesssim_\tau \|F - g\|_{L^q(U)} \lesssim_{U, q} \|F - G\|_{L^\infty}.$$
This gives the continuity in $L^\infty$.
We then use Lemma \ref{mollification of closed forms} to find smooth closed $d - 1$-forms $F_{k, m}$ on $U_k$ such that $F_{k, m} \to F|_{U_k}$ in $L^q(U_k)$ and
$$\|F_{k, m}\|_{C^0} \leq \|F\|_{L^\infty}.$$
So by the triangle inequality, if $\psi \in C^1(M)$,
\begin{align*}
\int_\tau \psi F 
&= \sum_{k = 1}^n \int_\tau \psi \chi_k F 
= \sum_{k = 1}^n \lim_{m \to \infty} \int_\tau \psi \chi_k F_{k, m} \\
&\leq \lim_{m \to \infty} \|F_{k, m}\|_{C^0} \int_\tau \psi \sum_{k = 1}^n \chi_k \dif S_\tau 
\leq \|F\|_{L^\infty} \int_\tau \psi \dif S_\tau.
\end{align*}
By approximating a $C^0$ test function $\psi$ by $C^1$ functions, we obtain the result for $\psi \in C^0$.
\end{proof}

\begin{lemma}\label{reduced level sets are integral currents}
Let $u \in BV(M)$.
Then for almost every $\lambda \in \RR$, $\{u > \lambda\}$ is an integral $d - 1$-current; moreover,
$$\int_M \star |\dif u| = \int_{-\infty}^\infty \Mass(\partial \{u > \lambda\}) \dif \lambda.$$
\end{lemma}
\begin{proof}
Recall from \cite[Chapter 3]{Giusti77} the definition of the reduced boundary $\partial^* \{u > \lambda\}$.
By the coarea formula\footnote{For a proof for $BV$ functions in arbitrary metrics, see \cite[Proposition 2.5]{BackusFLG}.}, for almost every $\lambda \in \RR$, $\{u > \lambda\}$ has locally finite perimeter.
So by the de Giorgi structure theorem \cite[Theorem 14.3]{Simon84}, $\partial^* \{u > \lambda\}$ is an oriented rectifiable $d - 1$-cycle.
So integration along $\partial^* \{u > \lambda\}$ defines a closed rectifiable $d - 1$-current, and by the coarea formula the result holds.
\end{proof}

\begin{proposition}[coarea formula]
Let $u \in BV(M)$, and for each $\lambda \in \RR$, let $\tau_\lambda := \partial \{u > \lambda\}$.
Then the product $\dif u \wedge F$ with a closed $L^\infty$ $d - 1$-form $F$ can be defined in one and only one way so that for every sequence $(F_n)$ which is bounded in $L^\infty$ and converges in $L^p_\loc$ for some $d < p < \infty$ to a closed $d - 1$-form $F$, $\dif u \wedge F_n \to \dif u \wedge F$ in the weak topology of measures.

Moreover, for every $\chi \in C^0_\cpt(M)$, we have the coarea formula
\begin{equation}\label{coarea formula}
\int_M \chi \dif u \wedge F = \int_{-\infty}^\infty \int_{\tau_\lambda} \chi F \dif \lambda.
\end{equation}
\end{proposition}
\begin{proof}
Motivated by Lemma \ref{reduced level sets are integral currents}, it is natural to \emph{define} $\dif u \wedge F$ to be the Radon measure satisfying (\ref{coarea formula}).
If $F$ is continuous, then this will agree with the usual definition of $\dif u \wedge F$ as a product of a Radon measure and a function.
Even if $F$ is discontinuous, (\ref{coarea formula}) defines a Radon measure.
Indeed, let $U \Subset M$ be an open set containing $\supp \chi$.
Then, by the trace theorem, Lemma \ref{reduced level sets are integral currents}, and Lemma \ref{reduced level sets are integral currents},
$$\left|\int_M \chi \dif u \wedge F\right| \leq \|\chi\|_{C^0} \|F\|_{L^\infty} \int_{-\infty}^\infty \Mass_U(\tau_\lambda) \dif \lambda = \|\chi\|_{C^0} \|F\|_{L^\infty} \Mass_U(\dif u).$$
Therefore $\dif u \wedge F$ is a Radon measure, since it has locally finite mass
$$\Mass_U(\dif u \wedge F) \leq \|F\|_{L^\infty} \Mass_U(\dif u) < \infty.$$

Now we check the convergence on $L^\infty$-bounded sequences in $L^p_\loc$.
Let $\chi \in C^0_\cpt(M)$, let $U \Subset M$ be an open set containing $\supp \chi$, and let $(F_n)$ be an $L^\infty$-bounded sequence converging in $L^p_\loc$ to a closed form $F$.
We introduce the functions
$$f_n(\lambda) := \int_{\tau_\lambda} \chi F_n, \qquad f(\lambda) := \int_{\tau_\lambda} \chi F,$$
which are well-defined by the trace theorem and Lemma \ref{reduced level sets are integral currents}.
Also consider the function $g(\lambda) := \Mass(\tau_\lambda)$, so that for almost every $\lambda \in \RR$, 
$$|f_n(\lambda)| \leq \|F_n\|_{C^0} \|\chi\|_{C^0} g(\lambda) \lesssim \|\chi\|_{C^0} g(\lambda).$$
By Lemma \ref{reduced level sets are integral currents} and the fact that $u \in BV$, $g \in L^1(\RR)$.
On the other hand, since $F_n \to F$ in $L^p(U)$ and $p > d$, $f_n \to f$ almost everywhere.
So by dominated convergence, $f_n \to f$ in $L^1(\RR)$.
In particular, by Lemma \ref{reduced level sets are integral currents},
\begin{align*}
\lim_{n \to \infty} \int_M \chi \dif u \wedge F_n
&= \lim_{n \to \infty} \int_{-\infty}^\infty f_n(\lambda) \dif \lambda
= \int_{-\infty}^\infty f(\lambda) \dif \lambda \\
&= \int_M \chi \dif u \wedge F. \qedhere 
\end{align*}
\end{proof}

%%%%%%%%%%%%%%%%%%%%%
\subsection{\texorpdfstring{$1$-harmonic functions}{One-harmonic functions}}
We shall consider the variational problems whose Euler-Lagrange equation is, at least at the formal level, the $1$-Laplacian
\begin{equation}\label{1Laplacian}
\dif^*\left(\frac{\dif u}{|\dif u|}\right) = 0.
\end{equation}
A suitable notion of weak solution for (\ref{1Laplacian}), at least for the Dirichlet problem, was introduced by Maz\'on, Rossi, and Segura de L\'eon \cite{Mazon14}; it essentially asserts that the level sets of $u$ are calibrated.
\todo{Do we ever need locally least gradient functions? If so exposit why we need them}
\todo{Do this for equivariant functions instead}

\begin{definition}
Let $B$ be a ball in $M$.
A function $u \in BV(B)$ has \dfn{least gradient} in $B$ if $u$ minimizes its total variation in the sense that for every function $v \in BV_\cpt(B)$,
$$\Mass_B(\dif u) \leq \Mass_B(\dif u + \dif v).$$
A function has \dfn{locally least gradient} if one can cover $M$ by balls in which $u$ has least gradient.
\end{definition}

\begin{definition}
Let $B$ be a ball in $M$.
We say that $u \in BV(B)$ is a \dfn{calibrated solution} of (\ref{1Laplacian}) if there exists a $L^\infty$ $d - 1$-form $F$ such that
\begin{equation}\label{local calibration}
\begin{cases}
\|F\|_{L^\infty} \leq 1, \\
dF = 0, \\
\dif u \wedge F = \star |\dif u|.
\end{cases}
\end{equation}
We say that $u$ is a \dfn{locally calibrated solution} of (\ref{1Laplacian}) if we can cover $M$ by balls $B$ such that $u|_B$ is a calibrated solution.
We also call locally calibrated solutions \dfn{$1$-harmonic functions}.
\end{definition}

This formulation of calibrated solution is not worded the same as in \cite{Mazon14}, but it is equivalent; their vector field $X$ is given by $(\star F)^\sharp$.
The quantity $\dif u \wedge F$ is well-defined by the coarea formula, since $\dif F = 0$.

\begin{theorem}[{\cite[Theorem 1.1]{Mazon14}}]
A function $u \in BV(B)$ is a calibrated solution of the $1$-Laplacian iff $u$ has least gradient in $B$.
\end{theorem}

\begin{corollary}
A function $u \in BV_\loc(M)$ is $1$-harmonic iff $u$ has locally least gradient.
\end{corollary}

\begin{theorem}\label{main thm of old paper}
Let $u \in BV(M)$ have least gradient in $M$. Then $1_{\{u > y\}}$ has least gradient.
In particular, if $d \leq 7$, then every superlevel set $\{u > y\}$ is bounded by complete disjoint embedded oriented minimal hypersurfaces.
\end{theorem}
\begin{proof}
Let $v := 1_{\{u > y\}}$.
Then $v$ has least gradient \cite[Theorem 1]{BOMBIERI1969}.
In particular, $\{u > y\}$ has finite perimeter, so it is bounded by Lipschitz hypersurfaces \cite[Chapter 4]{Giusti77}.
So by the regularity theorem for minimal hypersurfaces \cite[\S37]{simon1983GMT}\footnote{See also \cite[Exercise 1.6]{DeLellis18} for a discussion on why the theory of \cite[\S37]{simon1983GMT} applies for arbitrary metrics, and \cite{BackusFLG} for a proof for functions of least gradient for arbitrary metrics in the spirit of Miranda \cite{Miranda66}'s original argument for functions of least gradient.} if $d \leq 7$, $\{u > y\}$ is bounded by disjoint embedded minimal hypersurfaces.
These hypersurfaces inherit an orientation from the current $\dif v$.
\end{proof}

%%%%%%%%%%%%%%%%%%%%%%%%%%%%%%

\subsection{Minimal laminations}
We now give a geometric characterization of $1$-harmonic functions.
Fix an interval $I \subset \RR$ and a box $J \subset \RR^{d - 1}$.
We write $\Two_N$ for the second fundamental form of a submanifold $N$.

\begin{definition}
A \dfn{laminar flow box} is a $C^0$ coordinate chart $F: I \times J \to M$ and a compact set $K \subseteq I$, such that for every $k \in K$, $F|_{\{k\} \times J}$ is a $C^1$ embedding, and the \dfn{leaf} $F(\{k\} \times J)$ is a $C^1$ complete hypersurface in $F(I \times J)$.
Two laminar flow boxes belong to the same \dfn{laminar atlas} if the transition maps between them send leaves to leaves.
\end{definition}

\begin{definition}
A \dfn{lamination} is a closed subset $S \subseteq M$, called its \dfn{support}, and a maximal laminar atlas $\mathscr A$, such that $S$ is the union of the leaves of $\mathscr A$.
A \dfn{foliation} is a lamination $\lambda$ with $\supp \lambda = M$.
\end{definition}

\begin{definition}
A lamination is
\begin{enumerate}
\item \dfn{Lipschitz} if its flow boxes are Lipschitz isomorphisms,
\item \dfn{oriented} if its transition maps are orientation-preserving,
\item \dfn{minimal} if its leaves are minimal hypersurfaces, and 
\item of \dfn{bounded curvature} if there exists $C > 0$ such that for every leaf $N$, $\|\Two_N\|_{C^0} \leq C$.
\end{enumerate}
\end{definition}

\begin{theorem}[{\cite[Theorem A]{BackusCML}}]\label{disjoint surfaces are lamination}
Let $\mathcal S$ be a set of disjoint complete minimal hypersurfaces in a manifold $M$ of bounded geometry.
Suppose that there exists $C > 0$ such that for every $N \in \mathcal S$, $\|\Two_N\|_{C^0} \leq C$.
Then $\mathcal S$ is the set of leaves of a Lipschitz minimal lamination $\lambda$ of bounded curvature.
In particular, if $\lambda$ is oriented, then there is a Lipschitz vector field on $M$ whose restriction to each $N \in \mathcal S$ is the normal vector to $N$.
\end{theorem}

\begin{definition}
A lamination $\lambda$ with atlas $(F_\alpha, K_\alpha)$ is \dfn{measured} if it is equipped with positive Radon measures $\mu_\alpha$ with $\supp \mu_\alpha = K_\alpha$, such that the transition maps $F_\beta^{-1} \circ F_\alpha$ are measure-preserving.
The \dfn{Ruelle-Sullivan current} of a measured oriented lamination $\lambda$ with atlas $(F_\alpha, K_\alpha, \mu_\alpha)$ is the $d-1$-current $T_\lambda$ satisfying, for any partition of unity $(\chi_\alpha)$ subordinate to the open cover $(F_\alpha(I \times J))$,
$$\int_M T_\lambda \wedge \varphi = \sum_\alpha \int_{K_\alpha} \int_{\{k\} \times J} F_\alpha^* (\chi_\alpha \varphi) \dif \mu_\alpha(k).$$
The \dfn{homology class} $[\lambda]$ and \dfn{mass} $\Mass(\lambda)$ of a measured oriented lamination $\lambda$ are the homology class and mass of its Ruelle-Sullivan current.
\end{definition}

The notion of Ruelle-Sullivan current  was introduced by \cite{Ruelle75} and studied in the context of geodesic laminations in \cite[\S8]{daskalopoulos2020transverse}.
The motivation of the definition is that if $\lambda$ is a $d - 1$-chain, then $T_\lambda$ is just integration along $\lambda$.

\begin{theorem}[{\cite[Theorem B]{BackusCML}}]\label{1 harmonic is MOML}
Suppose that $d \leq 7$.
Then a function $u \in BV_\loc(M)$ is $1$-harmonic iff $\dif u$ is the Ruelle-Sullivan current for a measured oriented minimal lamination of locally bounded curvature.
\end{theorem}



%%%%%%%%%%%%%%%%%%%%%%%%%%%%%%%%%%%%%%%%%%

\section{\texorpdfstring{Closed $L^\infty$ $d - 1$-forms}{Closed (d - 1)-forms of finite comass}}\label{comass sec}
\subsection{Local comass}
We will be interested in the points at which a closed $L^\infty$ $d - 1$-form $F$ attains its comass.
One could pose this problem as the problem of computing the locus $\{|F| = \|F\|_{L^\infty}\}$.
However, $|F(x)|$, the comass of the $d - 1$-covector $F(x)$, is both only defined for almost every $x$, and $F$ is not norm-approximable by smooth functions.
So as a proxy for $|F|$, which may fail to be defined on a null set, we use the local comass, which is defined everywhere.

\begin{definition}
For an open set $\Omega \subseteq M$ and a a closed $d - 1$-form $F$, define 
$$\Comass_\Omega(F) := \sup_{\sigma \in \Chain_{d - 1}(\Omega)} \frac{1}{\Mass(\sigma)} \int_\sigma F.$$
The \dfn{local comass} of a closed $d - 1$-form $F$ at $x \in M$ is 
$$\Comass(F, x) = \limsup_{\varepsilon \to 0} \Comass_{B_\varepsilon(x)}(F).$$
\end{definition}

By the trace theorem, $\Comass_\Omega(F)$ is well-defined (but possibly $+\infty$).
Since $\Comass_{B_\varepsilon(x)}(F)$ is a supremum over a set which grows in $\varepsilon$, it is increasing in $\varepsilon$, so the limit superior is actually a limit and an infimum:
$$\Comass(F, x) = \lim_{\varepsilon \to 0} \Comass_{B_\varepsilon(x)}(F) = \inf_{\varepsilon > 0} \Comass_{B_\varepsilon(x)}(F).$$
In particular, if we write $\Comass(F) := \Comass_M(F)$, then $\Comass(F, x) \leq \Comass(F)$.

The local comass was defined in an analogous manner to the local Lipschitz constant.
As such, it enjoys many of the same properties, including those endowed on the local Lipschitz constant by \cite[Lemma 4.3]{Crandall2008}:

\begin{proposition}\label{crandall}
Let $F$ be a closed $L^\infty$ $d - 1$-form. Then:
\begin{enumerate}
\item The local comass $\Comass(F, \cdot)$ is upper semicontinuous. \label{crandall usc}
\item For almost every $x \in M$, \label{crandall LDT}
$$|F(x)| \leq \Comass(F, x).$$
\item The local comass is bounded, and \label{crandall linfinity}
$$\Comass(F) = \sup_{x \in M} \Comass(F, x) = \|F\|_{L^\infty}.$$
\item If $\sigma \in \Chain_{d - 1}(M)$ then \label{crandall best curl is ABC}
$$\int_\sigma F \leq \Mass(\sigma) \sup_{x \in \sigma} \Comass(F, x).$$
\end{enumerate}
\end{proposition}
\begin{proof}
We first prove (\ref{crandall usc}).
Let $x^n \to x$ and $r > 0$. Then eventually $x^n \in B_r(x)$, hence $\Comass(F, x^n) \leq \Comass_{B_r(x)}(F)$ and so
\begin{align*}
\limsup_{n \to \infty} \Comass(F, x^n) &\leq \inf_{r > 0} \Comass_{B_r(x)}(F) = \Comass(F, x).
\end{align*}

We now prove (\ref{crandall LDT}).
We may work locally, and choose coordinates $(y^i)$ in which $\sqrt{\det g} = 1$.
Let $I$ be the increasing $d-1$-index with $d$ removed.
By the Lebesgue differentiation theorem and Fubini's theorem, there exists a null set $Z \subset M$, which does not depend on $(y^i)$ by \cite[Proposition 2.1]{BackusFLG}, such that for every $x \notin Z$,
\begin{align*}
F_I(x) 
&= \lim_{\varepsilon \to 0} \frac{1}{\Mass(B_\varepsilon(x))} \int_{B_\varepsilon(x)} F_I(y) \dif y \\
&= \lim_{\varepsilon \to 0} \frac{1}{\Mass(B_\varepsilon(x))} \int_{-\infty}^\infty \int_{\{y^d = t\} \cap B_\varepsilon(x)} F_I(y) \dif y^1 \wedge \cdots \wedge \dif y^{d - 1} \wedge \dif t
\end{align*}
where we used the fact that $\sqrt{\det g} = 1$.
Now $\partial_{y^1} \wedge \cdots \wedge \partial_{y^{d - 1}}$ is the oriented unit $d - 1$-blade tangent to $\{y^d = t\}$, so as forms on $\{y^d = t\}$,
$$F_I(y) \dif y^1 \wedge \cdots \wedge \dif y^{d - 1} = F.$$
So
\begin{align*}
F_I(x) 
&= \lim_{\varepsilon \to 0} \frac{1}{\Mass(B_\varepsilon(x))} \int_{-\infty}^\infty \int_{\{y^d = t\} \cap B_\varepsilon(x)} F \dif t \\
&\leq \lim_{\varepsilon \to 0} \frac{\Comass_{B_\varepsilon(x)}(F)}{\Mass(B_\varepsilon(x))} \int_{-\infty}^\infty |\{y^d = t\} \cap B_\varepsilon(x)| \dif t.
\end{align*}
By Fubini's theorem,
$$F_I(x) \leq \lim_{\varepsilon \to 0} \frac{\Comass_{B_\varepsilon(x)}(F)}{\Mass(B_\varepsilon(x))} \Mass(B_\varepsilon(x)) = \Comass(F, x).$$
For every $x \in M$ we may select coordinates in which $|F(x)| = F_I(x)$, and then if $x \notin Z$, we conclude that (\ref{crandall LDT}) holds for $x$.

If we combine (\ref{crandall LDT}) with (\ref{integral over chain is linfinity}), then
$$\sup_{x \in M} \Comass(F, x) \leq \Comass(F) \leq \|F\|_{L^\infty} \leq \sup_{x \in M} \Comass(F, x).$$
The inequalities collapse, proving (\ref{crandall linfinity}).
In particular, for each $\sigma \in \Chain_{d - 1}(M)$, we obtain (\ref{crandall best curl is ABC}):
\begin{align*}
\int_\sigma F &\leq \Mass(\sigma) \inf_{\Omega \supset \sigma} \sup_{x \in \Omega} \Comass(F, x) = \Mass(\sigma) \sup_{x \in \sigma} \Comass(F, x). \qedhere
\end{align*}
\end{proof}

\begin{definition}
Let $F$ be a closed $L^\infty$ $d - 1$-form.
The \dfn{maximum comass locus} is the set
$$\MCL(F) := \{x \in M: \Comass(F, x) = \Comass(F)\}.$$
\end{definition}

\begin{corollary}
Suppose that $M$ is a closed manifold and $F$ is a form of best comass.
Then $\MCL(F)$ is a nonempty compact set.
\end{corollary}
\begin{proof}
This is immediate from Proposition \ref{crandall}(\ref{crandall usc}) and the extreme value theorem for upper semicontinuous functions.
\end{proof}

%%%%%%%%%%%%%%%%%%%%%%%
\subsection{\texorpdfstring{$L^\infty$}{L-infinity} calibrations}
The purpose of this paper is to study calibrated laminations.
If the calibration $F$ is continuous, then the theory of \cite{bangert_cui_2017} of calibrated laminations applies; however, we are not aware of general existence results on continuous calibrations, only $L^\infty$ calibrations, such as \cite[\S4.12]{Federer1974}.
Thus, we now introduce $L^\infty$ calibrations of laminations.

\begin{definition}
A \dfn{calibration} is a closed $d - 1$-form $F$ such that $\Comass(F) = 1$.
An integral $d - 1$-current $\sigma$ is $F$-\dfn{calibrated} if 
$$\Mass(\sigma) = \int_\sigma F.$$
A lamination $\lambda$ is \dfn{$F$-calibrated} if every leaf of $\lambda$ is $F$-calibrated.
\end{definition}

The fundamental theorem of calibrated geometry \cite{Harvey82} asserts that a calibrated integral current is homologically area-minimizing (and in particular minimal), so every calibrated lamination is minimal.

It follows from the definitions that if a smooth hypersurface $N$ is $F$-calibrated, then the trace of $F$ along $N$ is the area form $\dif S_N$.
In particular, $F|_N$ is smooth and $N$ is oriented.
If $\lambda$ is an $F$-calibrated lamination, then the trace $F|_\lambda$ is defined to be $F|_N$ along any leaf $N$.
A priori, $\lambda$ could fail to be orientable, and then $F|_\lambda$ would be necessarily discontinuous.
However, we assert that a calibrated lamination is orientable:

\begin{proposition}\label{calibrated implies oriented}
Let $F$ be a calibration and $\lambda$ an $F$-calibrated lamination.
Then $F|_\lambda$ is continuous, and $F$ induces an orientation on $\lambda$.
\end{proposition}
\begin{proof}
Let $N$ be a leaf of $\lambda$ and $x \in N$.
Let $\mathscr O$ be the local orientation of $\lambda$ near $x$ which is compatible with $F(x)$. 
We define a $d - 1$-form $G$ by declaring that for $y \in K$ close enough to $x$, $K$ a leaf of $\lambda$, $G(y) = \dif S_K(y)$ is the area form of $K$ with respect to $\mathscr O$.
Then for any $y \in \supp \lambda$ close to $x$, either $F(y) = G(y)$ or $F(y) = -G(y)$; we claim that $F(y) = G(y)$ if $\dist(x, y)$ is small enough.
If this claim is true, then the proposition follows, since $G$ is continuous at $x$.

To prove the claim, we suppose towards contradiction that there is a sequence $(x_n) \subset \supp \lambda$ with $x_n \to x$ and $F(x_n) = -G(x_n)$.
We may write $F = \dif A$ where $A$ is continuous near $x$ by Lemma \ref{Hodge theorem} and the Sobolev embedding theorem.
Let $N_n$ be the leaf of $\lambda$ containing $x_n$; then $N_n$ is minimal, hence smooth, so $F_n|_{N_n}$ is smooth.
Therefore by Stokes' theorem, if $(D_{n, m})$ is a sequence of disks in $N_n$ which shrink down to $x_n$ as $m \to \infty$ and are equipped with the orientation $\mathscr O$,
$$-1 = \lim_{m \to \infty} \frac{1}{\Mass(D_{n, m})} \int_{D_{n, m}} F = \lim_{m \to \infty} \frac{1}{\Mass(D_{n, m})} \int_{\partial D_{n, m}} A.$$
In particular, we can choose $m_n \to \infty$ such that 
\begin{equation}\label{orientation contradiction}
\int_{\partial D_{n, m_n}} A \leq 0.
\end{equation}
We set $D_n := D_{n, m_n}$.

Let $(k, z) \in \RR \times \RR^{d - 1}$ be coordinates near $x$ with respect to a flow box, where $k$ indexes the leaves and $z$ is a parameter on each leaf.
Then $D_n = \{k_n\} \times \Omega_n$ for some $\Omega_n \subset \RR^{d - 1}$.
Let $k$ be the index of $N$ and $D_n' := \{k\} \times \Omega_n$.
Since $A$ is continuous, $A(k, z) = A(k_n, z) + o(1)$ as $n \to \infty$, hence
\begin{equation}\label{orientation contradiction 2}
\frac{1}{\Mass(D_n')} \int_{\partial D_n'} A = \frac{1}{\Mass(D_n)} \int_{\partial D_n} A + o(1).
\end{equation}
However, $D_n'$ shrinks down to $x$, and $F(x) = G(x)$, so by Stokes' theorem, for $n$ large,
$$\int_{\partial D_n'} A \gtrsim \Mass(D_n'),$$
hence by (\ref{orientation contradiction}) and (\ref{orientation contradiction 2}),
$$0 < \Mass(D_n) \lesssim \int_{\partial D_n} A \leq 0,$$
a contradiction.
\end{proof}

We next give a natural condition for a lamination to be calibrated.
To make sense of it, observe that if $M$ is closed, $\lambda$ is a lamination, and $F$ is a calibration, then the quantity $\int_M T_\lambda \wedge F$ is well-defined, since it is just the pairing $\langle [F], [\lambda]\rangle$ of cohomology with homology.

\begin{proposition}\label{calibration condition}
Let $F$ be a calibration on a closed Riemannian manifold $M$.
Let $T_\lambda$ be the Ruelle-Sullivan current of a measured oriented lamination $\lambda$.
Then the following are equivalent:
\begin{enumerate}
\item One has \begin{equation}\label{calibration by Ruelle Sullivan}
\int_M T_\lambda \wedge F = \Mass(\lambda).
\end{equation}
\item $\lambda$ is $F$-calibrated.
\end{enumerate}
\end{proposition}
\begin{proof}
First suppose that (\ref{calibration by Ruelle Sullivan}) holds.
Let $(\chi_\alpha)$ be a locally finite partition of unity subordinate to an open cover $(U_\alpha)$ of flow boxes for $\lambda$, and let $(\mu_\alpha)$ be the transverse measure.
After refining $(U_\alpha)$ we may assume that $U_\alpha$ is contained in an open set which is bi-Lipschitz diffeomorphic to $\Ball^d$. After shrinking $U_\alpha$ we may assume that $\chi_\alpha > 0$ on $U_\alpha$.
Then for some hypersurfaces $\sigma_{\alpha,k}$,
$$\Mass(\lambda) = \int_M T_\lambda \wedge F = \sum_\alpha \int_I \int_{\sigma_{\alpha,k}} \chi_\alpha F \dif \mu_\alpha(k).$$
Let $\dif S_{\alpha,k}$ be the surface measure on $\sigma_{\alpha,k}$. Then
$$\int_M \chi_\alpha \star |T_\lambda| = \int_I \int_{\sigma_{\alpha,k}} \chi_\alpha \dif S_{\alpha,k} \dif \mu_\alpha(k),$$
so summing in $\alpha$, we obtain 
\begin{equation}\label{calibration condition contr}
\sum_\alpha \int_I \int_{\sigma_{\alpha,k}} \chi_\alpha F \dif \mu_\alpha(k) = \Mass(\lambda) = \sum_\alpha \int_I \int_{\sigma_{\alpha,k}} \chi_\alpha \dif S_{\alpha,k} \dif \mu_\alpha(k).
\end{equation}

We claim that $\lambda$ is \dfn{almost calibrated} in the sense that for every $\alpha$ and $\mu_\alpha$-almost every $k$, $\sigma_{\alpha, k}$ is calibrated.
If this is not true, then we may select $\beta$ and $K \subseteq I$ with $\mu_\beta(K) > 0$, such that for every $k \in K$, $\int_{\sigma_{\beta, k}} F < \Mass(\sigma_{\beta, k})$.
Since $0 < \chi_\beta \leq 1$ and $F/\dif S_{\beta, k} \leq 1$ on $\sigma_{\beta, k}$, this is only possible if 
$$\int_{\sigma_{\beta, k}} \chi_\beta F < \int_{\sigma_{\beta, k}} \chi_\beta \dif S_{\beta, k}.$$
Integrating over $K$, and using the fact that in general we have $\int_{\sigma_{\alpha, k}} \chi_\alpha F \leq \int_{\sigma_{\alpha, k}} \chi_\alpha \dif S_{\alpha, k}$, we conclude that 
$$\sum_\alpha \int_I \int_{\sigma_{\alpha, k}} \chi_\alpha F \dif \mu_\alpha(k) < \sum_\alpha \int_I \int_{\sigma_{\alpha, k}} \chi_\alpha \dif S_{\alpha, k} \dif \mu_\alpha(k)$$
which contradicts (\ref{calibration condition contr}).

To upgrade $\lambda$ from an almost calibrated lamination to a calibrated lamination, we first 
given $\sigma_{\alpha, k}$ we choose $k_j$ such that $\sigma_{\alpha, k_j}$ is calibrated and $k_j \to k$.
By Lemma \ref{Hodge theorem}, we can find a $d - 2$-form $A$ with $F = \dif A$ satisfying (\ref{Hodge theorem estimate}), which is then continuous by the Sobolev embedding theorem.
This justifies the following application of Stokes' theorem: 
$$\int_{\sigma_{\alpha, k}} F = \int_{\partial \sigma_{\alpha, k}} A.$$
Since $k_j \to k$, and $A$ is continuous,
\begin{align*}
\Mass(\sigma_{\alpha, k}) &= \lim_{j \to \infty} \Mass(\sigma_{\alpha, k_j}) = \lim_{j \to \infty} \int_{\sigma_{\alpha, k_j}} F = \lim_{j \to \infty} \int_{\partial \sigma_{\alpha, k_j}} A = \int_{\partial \sigma_{\alpha, k}} A = \int_{\sigma_{\alpha, k}} F.
\end{align*}

To establish the converse, suppose that $\lambda$ is $F$-calibrated, and let notation be as above.
Since $\lambda$ is $F$-calibrated, for every $\alpha$ and every $k$, the area form on $\sigma_{\alpha, k}$ is $F$. Therefore
\begin{align*}
\int_M T_\lambda \wedge F &= \sum_\alpha \int_I \int_{\sigma_{\alpha, k}} \chi_\alpha F \dif \mu_\alpha(k) = \Mass(T_\lambda). \qedhere
\end{align*}
\end{proof}

\begin{proposition}\label{properties of calibrated laminations}
Suppose that $M$ is a closed Riemannian manifold, $F$ is a calibration, and $\lambda$ is a measured $F$-calibrated lamination.
Then:
\begin{enumerate}
\item $\lambda$ is minimal.
\item If $G$ is a calibration and cohomologous to $F$, then $\lambda$ is $G$-calibrated.
\item $\supp \lambda \subseteq \MCL(F)$.
\end{enumerate}
\end{proposition}
\begin{proof}
Every leaf of $\lambda$ is $F$-calibrated, hence minimal, so $\lambda$ is also minimal.
Since $\lambda$ is oriented by Proposition \ref{calibrated implies oriented}, it has a Ruelle-Sullivan current.
Then, since (\ref{calibration by Ruelle Sullivan}) only depends on the cohomology class of $F$, not $F$ itself, $\lambda$ is $G$-calibrated.

Now let $S := \MCL(F)$, $N$ a leaf of $\lambda$, and suppose that $x \in N \setminus S$.
Since $S$ is closed, there exists $\varepsilon > 0$ such that $B_\varepsilon(x)$ does not meet $S$.
Moreover, $\sigma := N \cap B_\varepsilon(x)$ is a $d-1$-chain in $B_\varepsilon(x)$, so by Proposition \ref{crandall}(\ref{crandall best curl is ABC}),
$$\frac{1}{\Mass(\sigma)} \int_\sigma F \leq \sup_{y \in B_\varepsilon(x)} \Comass(F, y) < \Comass(F) = 1.$$
But then 
$$\int_N F = \int_\sigma F + \int_{N \setminus B_\varepsilon(x)} F < \Mass(\sigma) + \Mass(N \setminus B_\varepsilon(x)) = \Mass(N),$$
so $N$ (hence $\lambda$) is not $F$-calibrated.
\end{proof}

%%%%%%%%%%%%%%%%%%%%%%%%%%%%%
\section{\texorpdfstring{Tight forms and the $1$-Laplacian}{Infinity-tight forms and the one-Laplacian}}\label{tight forms sec}
\subsection{Convex optimization}
We follow \cite{Ekeland99}.
For a reflexive Banach space $X$, we denote by $\hat X$ its dual.
If $I: X \to \RR \cup \{+\infty\}$ is a convex function, we introduce its \dfn{Legendre transform}, the convex function
\begin{align*}
	\hat I: \hat X &\to \RR \cup \{+\infty\}\\
	\xi &\mapsto \sup_{x \in X} \langle \xi, x\rangle - I(x).
\end{align*}
We identify the cokernel of a linear map with the kernel of its adjoint.
In this setting, we have the following form of the convex duality theorem.

\begin{theorem}[convex duality]\label{abstract convex analysis}
Let $\Lambda : X \to Y$ be a bounded linear map between reflexive Banach spaces.
Let $I: Y \to \RR \cup \{+\infty\}$ satisfy:
\begin{enumerate}
\item $I$ and $\hat I$ are strictly convex,
\item $I$ is lower semicontinuous,
\item if $|y| \to \infty$ in $Y$, then $I(y) \to +\infty$, and 
\item there exists a point $x \in X$ such that $I$ is continuous and finite at $\Lambda(x)$.
\end{enumerate}
Then:
\begin{enumerate}
\item There exists a minimizer $\underline x \in X$ of $I(\Lambda(x))$, unique modulo $\ker \Lambda$.
\item There exists a unique maximizer $\overline \eta$ of $-\hat I(-\eta)$ subject to the constraint $\eta \in \coker \Lambda$.
\item We have \dfn{strong duality}
\begin{equation}\label{abstract strong duality}
I(\Lambda(\underline x)) = -\hat I(-\overline \eta).
\end{equation}
\end{enumerate}
\end{theorem}
\begin{proof}
This is largely a special case of \cite[Chapter IV, Theorem 4.2]{Ekeland99}.
Let $\mathscr P, \mathscr P^*$ be as in the statement of that theorem.
Then $\mathscr P$ is the problem of minimizing $J(x, \Lambda x)$ where $J(x, y) := I(y)$.
The Legendre transform of $J$ satisfies 
$$\hat J(\xi, \eta) = \begin{cases} \hat I(\eta), & \xi = 0, \\
	+\infty, &\xi \neq 0,
\end{cases}$$
and $\mathscr P^*$ is the problem of maximizing
$$-\hat J(\Lambda^* \eta, -\eta) = \begin{cases}
	-\hat I(-\eta), &\eta \in \ker \Lambda^*, \\
	-\infty, &\eta \notin \ker \Lambda^*,
\end{cases}$$
where $\Lambda^*$ is the adjoint of $\Lambda$.
Then most of the various assertions of this theorem follow immediately from \cite[Chapter IV, Theorem 4.2]{Ekeland99}.
The fact that $\overline \eta \in \coker \Lambda$ follows from the facts that $\overline \eta$ is a solution of $\mathscr P^*$, but any solution of $\mathscr P^*$ must be a member of $\ker \Lambda^*$. 
To establish uniqueness, we use \cite[Chapter II, Proposition 1.2]{Ekeland99}, the fact that $\hat I$ is strictly convex, and the fact that we may view $I \circ \Lambda$ as a strictly convex function on the reflexive Banach space $X/\ker \Lambda$.
\end{proof}

%%%%%%%%%%%%%%%

\subsection{Max flow min cut for the \texorpdfstring{$q$-Laplacian}{q-Laplacian}}
We now turn to the problem at hand.
Let $M$ be a closed oriented Riemannian manifold with fundamental group $\Gamma$ and universal covering $\tilde M \to M$, and let $M_{\rm fun} \subset \tilde M$ be a fundamental domain of $\Gamma$.
By Poincar\'e duality and the Hurcewiz theorem, we have canonical isomorphisms
\begin{equation}\label{Poincare Hurcewiz}
H_{d - 1}(M, \RR) = H^1(M, \RR) = \Hom(\Gamma, \RR).
\end{equation}
Given a representation $\alpha: \Gamma \to \RR$, which we always identify with some smooth $1$-form (which we also call $\alpha$) representing the cohomology class corresponding to the representation $\alpha$, and a H\"older pair $(p, q)$ (with $1 < p < \infty$),
we are interested in the $q$-Laplace equation
$$\dif^* (|\dif u|^{q - 2} \dif u) = 0$$
for an $\alpha$-equivariant function $u$ (thus for any $\gamma \in \Gamma$, $u(\gamma(x)) - u(x) = \alpha(\gamma)$).
Let us express this problem variationally.

Let $X$ be the space of $W^{1, q}_\loc(\tilde M)$ functions $u$ which are $\Gamma$-equivariant in the sense that for $\gamma \in \Gamma$, $\gamma^* \dif u = \dif u$, and let $Y := L^q(M, \Omega^1)$.
We identify $\hat Y$ with $L^p(M, \Omega^{d - 1})$ using the perfect pairing 
\begin{align*}
	L^p(M, \Omega^{d - 1}) \times Y &\to \RR \\
	(F, \varphi) &\mapsto \int_M \varphi \wedge F.
\end{align*}
Then $\Lambda := (\dif: X \to Y)$ is a bounded linear map, and the $\alpha$-equivariant $q$-Laplacian is the Euler-Lagrange equation of $I \circ \Lambda$, where $I$ is the strictly convex functional such that
$$I(\varphi) := \frac{1}{q} \int_M \star |\varphi|^q$$
if the cohomology class of $\varphi$ is $\alpha$, and $I(\varphi) := +\infty$ otherwise.

\begin{proposition}[max flow min cut for the $q$-Laplacian]\label{mfmc qLaplacian}
Given a representation $\alpha: \Gamma \to \RR$, and a H\"older pair $(p, q)$ with $1 < p < \infty$, there exists an $\alpha$-equivariant $q$-harmonic function $u: \tilde M \to \RR$, unique modulo constants, and a unique minimizer $F$ of 
$$J_{p, \alpha}(F) := \frac{1}{p} \int_M \star |F|^p - \int_M \alpha \wedge F$$
among all closed $d - 1$-forms on $M$.
Moreover, we have
\begin{equation}\label{strong duality}
	\frac{1}{q} \int_M \star |\dif u|^q + \frac{1}{p} \int_M \star |F|^p + \int_M \dif u \wedge F = 0.
\end{equation}
\end{proposition}
\begin{proof}
Let
$$I_\alpha(\psi) := \frac{1}{q} \int_M \star |\psi + \alpha|^q,$$
defined for exact $L^q$ $1$-forms $\psi$, thus $\widehat{I_\alpha}$ is defined on the space of $L^p$ $d - 1$-forms modulo the kernel of the map
$$F \mapsto \left(\psi \mapsto \int_M \psi \wedge F\right)$$
and we lift it to $L^p(M, \Omega^{d - 1})$.

Let $v$ be a primitive of $\alpha$.
Then an $\alpha$-equivariant $u$ minimizes $I$ iff $u - v$ minimizes $I_\alpha \circ \Lambda$, which happens iff $u$ is $\alpha$-equivariant $q$-harmonic.
Since $I_\alpha(\psi) = I_0(\psi + \alpha)$, we can apply \cite[Chapter I, Remark 4.1]{Ekeland99} to see that $I_\alpha$ and $\widehat{I_\alpha}$ are strictly convex and
$$\widehat{I_\alpha}(F) = \widehat{I_0}(F) - \int_M \alpha \wedge F = \frac{1}{p} \int_M \star |F|^p - \int_M \alpha \wedge F.$$
Since $\coker \Lambda$ is the space of closed $L^p$ $d - 1$-forms on $M$, for $F \in \coker \Lambda$, $\widehat{I_\alpha}(F)$ does not depend on the choice of representatives $\alpha, v$, or of the lift of $\widehat{I_\alpha}$ to $L^p(M, \Omega^{d - 1})$.

Finally observe that $\ker \Lambda$ is the space of constants, and for any $\alpha$-equivariant $u$ and closed $F$,
\begin{align*}
I(\Lambda u) + \widehat{I_\alpha}(-F)
&= \frac{1}{q} \int_M \star |\dif u|^q + \frac{1}{p} \int_M \star |F|^p + \int_M \alpha \wedge F \\
&= \frac{1}{q} \int_M \star |\dif u|^q + \frac{1}{p} \int_M \star |F|^p + \int_M \dif u \wedge F.
\end{align*}
All of the assertions of this proposition now follow from Theorem \ref{abstract convex analysis}.
\end{proof}

Let $u$ be $\alpha$-equivariant $q$-harmonic.
Motivated by \cite[\S3.1]{daskalopoulos2020transverse}, it is natural to guess that 
\begin{equation}\label{dual solution}
F := - |\dif u|^{q - 2} \star \dif u
\end{equation}
is the solution of the dual problem of minimizing $J_{p, \alpha}$.
In order to prove that this is true, we shall need that if $(p, q)$ is a H\"older pair, then
\begin{equation}\label{holder cancellation}
	(p - 2)(q - 1) + (q - 2) = 0.
\end{equation}

\begin{lemma}\label{dual to u is minimizer}
Suppose that $u: \tilde M \to \RR$ is an $\alpha$-equivariant $q$-harmonic function, and suppose that $F$ satisfies (\ref{dual solution}).
Then $F$ is a closed $d - 1$-form, which minimizes $J_{p, \alpha}$ among all closed $d - 1$-forms.
Moreover, $F$ solves the PDE 
\begin{equation}\label{pMaxwell}
\begin{cases}
	\dif F = 0 \\
	\dif^* (|F|^{p - 2} F) = 0.
\end{cases}
\end{equation}
\end{lemma}
\begin{proof}
We first show that $\dif F = 0$.
In fact, 
$$\star \dif F = - \star \dif(|\dif u|^{q - 2} \star \dif u) = \pm \dif^*(|\dif u|^{q - 2} \dif u) = 0.$$
By uniqueness, if (\ref{strong duality}) holds, then $F$ must be the minimizer of $J_{p, \alpha}$.
One can easily compute 
$$|F|^p = |\dif u|^{(q - 1)p} = |\dif u|^q,$$
so by Stokes' theorem and the fact that $\alpha$ is cohomologous to $\dif u$,
\begin{align*}
\frac{1}{q} \int_M \star |\dif u|^q + \frac{1}{p} \int_M \star |F|^p&
= \left[\frac{1}{p} + \frac{1}{q}\right] \int_M \star |\dif u|^q
= \int_M \dif u \wedge |\dif u|^{q - 2} \star \dif u \\
&= -\int_M \alpha \wedge F,
\end{align*}
implying (\ref{strong duality}).
Finally, we use (\ref{holder cancellation}) to prove
\begin{align*}
\dif^*(|F|^{p - 2} F) &= - \dif^*(|\dif u|^{(p - 2)(q - 1)} |\dif u|^{q - 2} \star \dif u) = - \dif^*(\star \dif u) \\
&= \pm \star \dif^2 u = 0. \qedhere 
\end{align*}
\end{proof}

We next scrutinize the PDE (\ref{pMaxwell}).
At least at the heuristic level, one expects that as $p \to \infty$, the solutions of (\ref{pMaxwell}) converge to an absolute minimizer of a suitable $L^\infty$ variational problem; minimizers of such problems have been called \dfn{tight} by Sheffield and Smart \cite{Sheffield12}.
This motivates the below terminology:

\begin{definition}
Let $1 < p < \infty$.
A \dfn{$p$-tight form} is a solution of the PDE (\ref{pMaxwell}).
\end{definition}

\begin{proposition}
Suppose that $M$ is a closed oriented Riemannian manifold.
Then there is a unique $p$-tight form in each cohomology class in $H^{d - 1}(M, \RR)$.
Moreover, $p$-tight forms are minimizers of the strictly convex functional
$$J_p(F) := \frac{1}{p} \int_M \star |F|^p$$
among all forms cohomologous to them.
\end{proposition}
\begin{proof}
Strict convexity of $J_p$ on closed $L^p$ $d - 1$-forms is straightforward; since each cohomology class is an affine subspace of $L^p(M, \Omega^{d - 1})$, and hence is convex, the strict convexity on each class follows.
The existence and uniqueness of a minimizer is a standard consequence of strict convexity \cite[Chapter II]{Ekeland99}.
To compute the Euler-Lagrange equations for $J_p$, let $B$ be a $d-2$-form (so $F + t \dif B$ is cohomologous to $F$), so that for a minimizer $F$ of $J_p$,
$$\frac{\dif}{\dif t} J_p(F + t \dif B) = \frac{1}{p} \int_M \star \frac{\partial}{\partial t} |F + t \dif B|^p = \int_M \star |F + t \dif B|^{p - 2} \langle F + t \dif B, \dif B\rangle.$$
Setting $t = 0$, we obtain 
$$0 = \int_M \star |F|^{p - 2} \langle F, \dif B\rangle = \int_M \star \langle \dif^*(|F|^{p - 2} F), B\rangle.$$
Thus the Euler-Lagrange equations for $J_p$ are (\ref{pMaxwell}).
\end{proof}

\begin{definition}
Let $F$ be a $p$-tight form, let
\begin{equation}
\dif u := (-1)^d |F|^{p - 2} \star F, \label{inverse extremality}
\end{equation}
and let $u$ be the primitive of $\dif u$ on the universal cover $\tilde M$, which is normalized to have zero mean on a fundamental domain $M_{\rm fun}$.
Then $u$ is called the \dfn{$q$-harmonic conjugate} of the $p$-tight form $F$, where $\frac{1}{p} + \frac{1}{q} = 1$.
\end{definition}

Let $u$ be the $q$-harmonic conjugate of $F$.
By Poincar\'e's inequality,
$$\|u\|_{W^{1, q}(M_{\rm fun})}^q \lesssim \int_M \star |\dif u|^q = \int_M \star |F|^{(p - 1)q} = \int_M \star |F|^p < \infty$$
since $F$ is $p$-tight; that is, we have $F \in L^p$ and $u \in W^{1, q}_\loc$, justifying any manipulations we shall make with these forms.

\begin{lemma}
Let $1 < p, q < \infty$ and $\frac{1}{p} + \frac{1}{q} = 1$.
Let $F$ be a $p$-tight form, and let $u$ be its $q$-harmonic conjugate.
Then $u$ is $q$-harmonic, $F$ satisfies (\ref{dual solution}), and we have strong duality (\ref{strong duality}).
\end{lemma}
\begin{proof}
We first use (\ref{holder cancellation}) to prove
$$|\dif u|^{q - 2} \star \dif u = (-1)^d |F|^{(q - 2)(p - 1)} \star \star |F|^{p - 2} F = - |F|^{(q - 2)(p - 1) - (p - 2)} F = - F.$$
Thus we have (\ref{dual solution}), and moreover
$$\dif \star (|\dif u|^{q - 2} \dif u) = - \dif F = 0$$
so that $u$ is $q$-harmonic.
Then by Lemma \ref{dual to u is minimizer}, $F$ is the unique minimizer of $J_{p, [\dif u]}$, which implies (\ref{strong duality}).
\end{proof}

% %%%%%%%%%%%%%%%%%%%
% \subsection{Regularity of the \texorpdfstring{$q$-Laplacian}{q-Laplacian}}
% We now pause to consider two regularity results for the $q$-Laplacian.
% The first gives H\"older regularity but is not uniform in $q$; the second is uniform but only gives Sobolev regularity.
% \todo{Do we ever use this?}

% \begin{lemma}[{\cite[Theorem 2]{DIBENEDETTO1983827}}]\label{q Laplacian Holder regularity}
% Let $u: \tilde M \to \RR$ be an $\alpha$-equivariant $q$-harmonic function.
% Then $\dif u$ is H\"older continuous.
% \end{lemma}

% \begin{corollary}
% Every $p$-tight form is H\"older continuous.
% \end{corollary}
% \begin{proof}
% Let $F$ be $p$-tight and let $u$ be its $q$-harmonic conjugate, so $\dif u$ is H\"older continuous by Lemma \ref{q Laplacian Holder regularity}.
% The claim now follows from (\ref{dual solution}) and the fact that a product of H\"older continuous functions is H\"older continuous.
% \todo{Write this out carefully since $q < 2$}
% \end{proof}

\begin{lemma}
Suppose that $1 < q \leq 2$.
Let $u_q: \tilde M \to \RR$ be an $\alpha$-equivariant $q$-harmonic function.
Then (with constant independent of $q$)
\begin{equation}\label{q Laplacian Sobolev regularity estimate}
\|\dif u_q\|_{L^q} \sim \Mass(\alpha).
\end{equation}
\end{lemma}
\begin{proof}
Without loss of generality, $\int_{M_{\rm fun}} \star u_q = 0$.
Let $e_1, \dots, e_k$ be a basis for $H_{d - 1}(M, \RR)$.
This induces a norm $\|\cdot\|$ on $H_{d - 1}(M, \RR)$ by
$$\left\|\sum_i \beta_i e_i\right\| = \sum_i |\beta_i|,$$
which is comparable to the stable norm $\Mass$ since $H_{d - 1}(M, \RR)$ is finite-dimensional.
We write $\alpha = \sum_i \alpha_i e_i$.
Let $v_i$ be the $e_i$-equivariant harmonic function such that $\int_{M_{\rm fun}} \star v_i = 0$.
Then $u_2 = \sum_i \alpha_i v_i$, so
$$\|\dif u_2\|_{L^2} \leq \sum_i |\alpha_i| \|\dif v_i\|_{L^2} \lesssim \|\alpha\| \sim \Mass(\alpha).$$
Since $\dif u_q$ is a minimizer of the $L^q$ norm, we estimate using H\"older's inequality 
\begin{align*}
\|\dif u_q\|_{L^q} &\leq \|\dif u_2\|_{L^q} \leq |M|^{\frac{1}{q} - \frac{1}{2}} \|\dif u_2\|_{L^2} \lesssim \|\dif u_2\|_{L^2} \lesssim \Mass(\alpha).
\end{align*}
In the other direction, we estimate using H\"older's inequality
\begin{align*}
\Mass(\alpha) &\leq \|\dif u_q\|_{L^1} \leq \|\dif u_q\|_{L^q} |M|^{1/p} \lesssim \|\dif u_q\|_{L^q}. \qedhere 
\end{align*}
\end{proof}


%%%%%%%%%%%%%%%%%%%%%%%
\subsection{\texorpdfstring{Existence of $\infty$-tight forms}{Existence of infinity-tight forms}}
We now take the limit $p \to \infty$ to obtain a privileged form of best comass.
To do so, we shall need the $p$-tight forms to be uniformly bounded in the following sense.

\begin{lemma}
Let $F_p$ be a $p$-tight form, and let $B$ range over closed $d - 1$-forms cohomologous to $F_p$. Then
\begin{equation}\label{infinity magnetic rules p magnetic}
	\|F_p\|_{L^p} \leq |M|^{1/p} \inf_B \|B\|_{L^\infty}.
\end{equation}
\end{lemma}
\begin{proof}
By H\"older's inequality and the fact that $F_p$ is $p$-tight,
$$\|F_p\|_{L^p} \leq \|B\|_{L^p} \leq |M|^{1/p} \|B\|_{L^\infty},$$
hence the same holds for the infimum.
\end{proof}

\begin{proposition}\label{existence infinity}
Let $\rho \in H^{d - 1}(M, \RR)$.
For each $p \geq 2$, let $F_p$ be the $p$-tight form representing $\rho$. Then there exists a closed $d - 1$-form $F$ such that:
\begin{enumerate}
\item $F_p \to F$ weakly in $L^r$ along a subsequence, for any $d < r < \infty$.
\item $F$ is a best comass representative of $\rho$.
\end{enumerate}
\end{proposition}
\begin{proof}
We roughly follow \cite[\S3]{Lindqvist14}.
Let $r > d$, and let $B$ be an $L^\infty$ representative of $\rho$.
By H\"older's inequality and (\ref{infinity magnetic rules p magnetic}),
\begin{equation}\label{uniform bounds in p by best curl}
	\|F_p\|_{L^r} \leq |M|^{\frac{1}{r} - \frac{1}{p}} \|F_p\|_{L^p} \leq |M|^{\frac{1}{r}} \|B\|_{L^\infty}.
\end{equation}
Thus a compactness argument gives $F_p \to F$ for some $d - 1$-form $F$, weakly in $L^r$, and 
$$\|F\|_{L^r} \leq \liminf_{p \to \infty} \|F_p\|_{L^r} \leq |M|^{\frac{1}{r}} \|B\|_{L^\infty}.$$
Diagonalizing, we may assume that $F_p \to F$ weakly in $L^r$ for every such $r$, and taking $r \to \infty$, we conclude 
\begin{equation}\label{infinity magnetics have best curl}
	\|F\|_{L^\infty} \leq \|B\|_{L^\infty}.
\end{equation}
Moreover, $[F] = \lim_{p \to \infty} [F_p] = \rho$.
Since $B$ was arbitrary in (\ref{infinity magnetics have best curl}), $F$ has best comass.
\end{proof}

\begin{definition}
The $d - 1$-form $F$ of best comass in Proposition \ref{existence infinity} is called a \dfn{tight form}, or an \dfn{$\infty$-tight form}.
\end{definition}

Our next corollary follows immediately from Proposition \ref{existence infinity}, though it can also be proven using Alaoglu's theorem.

\begin{corollary}
Every cohomology class has a best comass representative.
\end{corollary}

The existence of best comass representatives of each cohomology class $\rho$ implies the following useful lemma on the costable norm of $\rho$.

\begin{lemma}\label{p tights approximate L}
Let $F_p$ be the $p$-tight representative of $\rho$. Then 
$$\lim_{p \to \infty} \|F_p\|_{L^p} = \Comass(\rho).$$
\end{lemma}
\begin{proof}
We follow \cite[Lemma 2.7]{daskalopoulos2020transverse}.
Let $F$ be a best comass representative of $\rho$, so $\|F\|_{L^\infty} = \Comass(\rho)$.
Since $F_p$ is $p$-tight, H\"older's inequality implies 
$$\|F_p\|_{L^p} \leq \|F\|_{L^p} \leq |M|^{\frac{1}{p}} \Comass(\rho).$$
Therefore 
$$\limsup_{p \to \infty} \|F_p\|_{L^p} \leq \Comass(\rho).$$
To prove the converse, suppose that for some $\varepsilon > 0$,
$$\liminf_{p \to \infty} \|F_p\|_{L^p} \leq \Comass(\rho) - \varepsilon < \Comass(\rho).$$
Along a subsequence which attains the limit inferior, $F_p$ converges weakly in every $L^r$, $d < r < \infty$, to a tight form $\tilde F$ such that (by H\"older's inequality)
$$\|\tilde F\|_{L^r} \leq \liminf_{p \to \infty} \|F_p\|_{L^r} \leq \liminf_{p \to \infty} |M|^{\frac{1}{r}} \|\tilde F\|_{L^\infty} \leq |M|^{\frac{1}{r}} (\Comass(\rho) - \varepsilon).$$
Taking $r \to \infty$, we obtain $\Comass(\tilde F) < \Comass(\rho)$, which contradicts the definition of the costable norm $\Comass(\rho)$.
\end{proof}


%%%%%%%%%%%%%%%%%%%%
\subsection{\texorpdfstring{$1$-harmonic conjugates of tight forms}{One-harmonic conjugates of tight forms}}
We now construct the $1$-harmonic conjugates of a tight form.
In the special case that the tight form $F$ is a calibration, that is $\Comass(F) = 1$, a $1$-harmonic conjugate will be a $1$-harmonic function on the universal cover whose level sets are calibrated by $F$.

\begin{definition}
Let $F$ be a tight form of cohomology class $\rho$.
A nonconstant $\pi_1(M)$-equivariant $1$-harmonic function $u \in BV_\loc(\tilde M)$ is called a \dfn{$1$-harmonic conjugate} of $F$ if we have the \dfn{max flow min cut principle} that
\begin{equation}\label{1 extremality}
\dif u \wedge F = \Comass(\rho) \star |\dif u|.
\end{equation}
\end{definition}

We begin by showing that $L^1$ convergence preserves the equivariance properties of functions.

\begin{lemma}\label{L1 convergence preserves pi1}
Let $\tilde M \to M$ be the universal cover, and let $(u_q)$ be a sequence of $\pi_1(M)$-equivariant functions on $\tilde M$ which converge in $L^1_\loc(\tilde M)$ to a function $u$ as $q \to 1$.
Then $u$ is $\pi_1(M)$-equivariant, and $[u_q] \to [u]$.
Moreover, if $\dif u_q \to \dif u$ in the weak topology of measures on $M$ and $\dif u_q \in L^q$, then
\begin{equation}\label{q to 1 Holder}
\Mass(\dif u) \leq \liminf_{q \to 1} \frac{1}{q} \int_M \star |\dif u_q|^q.
\end{equation}
\end{lemma}
\begin{proof}
Since $u_q$ is $\pi_1(M)$-equivariant, there exists $\alpha_q \in H^1(M, \RR)$ such that for every $\gamma \in \pi_1(M)$,
\begin{equation}\label{equivariance q}
	\gamma^* u_q = u_q + \langle \alpha_q, \gamma\rangle.
\end{equation}
Let $M_{\rm fun}$ be a fundamental domain and $U_\gamma := M_{\rm fun} \cup \gamma_* (M_{\rm fun})$.

We claim that $(\alpha_q)$ has a convergent subsequence.
To see this, we first recall that $M$ has finite Betti numbers, so $H^1(M, \RR)$ is locally compact.
Therefore, if no convergent subsequence exists, there exists a $\gamma \in \pi_1(M)$ and a subsequence along which $\langle \alpha_q, \gamma\rangle \to \infty$.
Moreover, since $u_q \to u$ in $L^1_\loc$, $\|u_q\|_{L^1(M_{\rm fun})} \leq 2\|u\|_{L^1(M_{\rm fun})}$ if $q - 1$ is small enough.
But then 
$$\|u_q\|_{L^1(\gamma_* M_{\rm fun})} = \|\gamma^* u_q\|_{L^1(M_{\rm fun})} \geq \langle \alpha_q, \gamma\rangle - \|u_q\|_{L^1(M_{\rm fun})} \geq \langle \alpha_q, \gamma\rangle - 2\|u\|_{L^1(M_{\rm fun})}$$
and taking $q \to 1$ we conclude that $(u_q)$ is not compact in $L^1(\gamma_* M_{\rm fun})$, contradicting the convergence in $L^1_\loc(\tilde M)$.
So $\alpha_q \to \alpha$ for some $\alpha \in H^1(M, \RR)$ along a subsequence.

For any $q > 1$,
\begin{align*}
\dashint_{M_{\rm fun}} \star |\gamma^* u - u - \langle \alpha, \gamma\rangle| 
&\leq \dashint_{M_{\rm fun}} \star (|\gamma^* u_q - u_q - \langle \alpha_q, \gamma\rangle| + |\gamma^* u_q - u_q| + |\gamma^* u - u|) \\
&\qquad + |\langle \alpha_q - \alpha, \gamma\rangle|.
\end{align*}
Taking $q \to 1$ and applying (\ref{equivariance q}), we conclude that $\|\gamma^* u - u - \langle \alpha, \gamma\rangle\|_{L^1} = 0$, hence $u$ is $\alpha$-equivariant.
Thus $\alpha$ is uniquely defined and $\alpha_q \to \alpha$ along the entire subsequence.

Finally we prove (\ref{q to 1 Holder}).
Suppose that $\dif u_q \to \dif u$ in the weak topology of measures and $\dif u_q$ in $L^q$.
Then
$$\|\dif u_q\|_{L^1} = \Mass(\dif u_q).$$
So we may use the portmanteau theorem and H\"older's inequality to estimate (where $\frac{1}{p} + \frac{1}{q} = 1$)
\begin{align*}
\Mass(\dif u) &= \lim_{q \to 1} \Mass(\dif u_q) \leq \lim_{q \to 1} |M|^{\frac{1}{p}} \|\dif u_q\|_{L^q} = \lim_{q \to 1} \frac{1}{q} \int_M \star |\dif u_q|^q. \qedhere
\end{align*}
\end{proof}

The duality relation (\ref{inverse extremality}) blows up $p \to \infty$.
We now ``renormalize'' the divergence of the $q$-harmonic conjugates of $p$-tight forms before taking the limit $q \to 1$, as in \cite[\S3.2]{daskalopoulos2020transverse}.
Suppose that $\rho \in H^{d - 1}(M, \RR)$, and let $k_p$ be defined by 
$$k_p^{1 - p} = \int_M \star |F_p|^p$$
where $F_p$ is the $p$-tight representative of $\rho$.

\begin{lemma}\label{normalizations converge}
As $p \to \infty$, $k_p \to \Comass(\rho)^{-1}$.
\end{lemma}
\begin{proof}
We follow \cite[Lemma 3.4]{daskalopoulos2020transverse}.
By Lemma \ref{p tights approximate L},
$$\lim_{p \to \infty} k_p^{-\frac{1}{q}} = \lim_{p \to \infty} \|F_p\|_{L^p} = \Comass(\rho).$$
Taking logarithms we see that $q^{-1} \log k_p \to -\log \Comass(\rho)$, and since $q \to 1$ the claim follows.
\end{proof}

\begin{proposition}\label{existence 1}
Let $\rho \in H^{d - 1}(M, \RR)$ be nonzero, and let $F$ be its tight representative.
For each H\"older pair $(p, q)$ with $d < p < \infty$, let $F_p$ be the $p$-tight representative of $\rho$, and let $u_q$ be the function on $\tilde M$ with mean zero on $M_{\rm fun}$ and
$$\dif u_q = (-1)^{d - 1} k_p^{p - 1} |F_p|^{p - 2} \star F_p.$$
Then there exists a $1$-harmonic conjugate $u$ of $F$ such that as $q \to 1$ along a subsequence, $u_q \to u$ weakly in $BV_\loc(\tilde M)$ and strongly in $L^r_\loc(\tilde M)$ for $1 \leq r < \frac{d}{d - 1}$.
\end{proposition}
\begin{proof}
Let $L := \Comass(\rho)$.
We first compute using H\"older's inequality and Lemma \ref{normalizations converge}
\begin{align}
\lim_{q \to 1} \|\dif u_q\|_{L^1}
&\leq \lim_{q \to 1} |M|^{\frac{1}{p}} \left[\int_M \star |\dif u_q|^q\right]^{\frac{1}{q}} = \lim_{p \to \infty} \left[k_p^p \int_M \star |F_p|^p\right]^{\frac{1}{q}} \label{Rellich 1}\\
&= \lim_{p \to \infty} k_p^{\frac{1}{q}} = \lim_{p \to \infty} k_p = \frac{1}{L} \label{Rellich 2}.
\end{align}
So by Rellich's theorem, $(u_q)$ is weakly compact in $BV$ and strongly compact in $L^r$ for $1 \leq r < \frac{d}{d - 1}$.
In particular, $\dif u_q \to \dif u$ in the weak topology of measures and $u_q \to u$ weakly in $BV$ and strongly in $L^r$.
As the limit of $\pi_1(M)$-equivariant functions, $u$ is also $\pi_1(M)$-equivariant by Lemma \ref{L1 convergence preserves pi1}.
In particular, $\dif u$ drops to a current on $M$.
Moreover, $[\dif u_q] \to [\dif u]$, and we have the bound (\ref{q to 1 Holder}) on $\int \star |\dif u|$.

We next must check that $u$ is nonconstant.
If $u$ is constant, then it is $\pi_1(M)$-invariant, so $[\dif u_q] \to 0$.
By (\ref{q Laplacian Sobolev regularity estimate}), $\|\dif u_q\|_{L^q} \to 0$, so by (\ref{Rellich 1}, \ref{Rellich 2}), $L = \infty$, which is absurd.
Therefore $u$ is nonconstant.

Renormalizing (\ref{strong duality}), we obtain 
$$\frac{k_p^{-p}}{q} \int_M \star |\dif u_q|^q + \frac{1}{p} \int_M \star |F_p|^p = k_p^{1 - p} \int_M \dif u_q \wedge F_p.$$
Multiplying by $k_p^p$, we have 
\begin{equation}\label{1 strong duality before limits}
	\frac{1}{q} \int_M \star |\dif u_q|^q + \frac{k_p^p}{p} \int_M \star |F_p|^p = k_p \int_M \dif u_q \wedge F_p.
\end{equation}

Let $\mu(U) := \Mass_U(\dif u)$ be the total variation measure of $\dif u$.
We claim that
\begin{equation}\label{1 strong duality}
	L\mu(M) \leq \int_M \dif u \wedge F.
\end{equation}
First, we have from (\ref{q to 1 Holder}) and (\ref{1 strong duality before limits}) that
$$\mu(M) \leq \lim_{q \to 1} \frac{1}{q} \int_M \star |\dif u_q|^q = \lim_{p \to \infty} k_p \int_M \dif u_q \wedge F_p - \lim_{p \to \infty} \frac{k_p^p}{p} \int_M \star |F_p|^p.$$
By Lemma \ref{normalizations converge},
$$\lim_{p \to \infty} \frac{k_p^p}{p} \int_M \star |F_p|^p = \lim_{p \to \infty} \frac{k_p}{p} = \frac{0}{L} = 0,$$
and
$$\lim_{p \to \infty} k_p \int_M \dif u_q \wedge F_p = \frac{1}{L} \lim_{p \to \infty} \int_M [\dif u_q] \wedge \rho.$$
Since $[\dif u_q] \to [\dif u]$, we obtain
$$\lim_{p \to \infty} \int_M [\dif u_q] \wedge \rho = \int_M \alpha \wedge \rho = \int_M \dif u \wedge F,$$
completing the proof of (\ref{1 strong duality}).

By the coarea formula (\ref{coarea formula}), we have for any open set $U$,
$$\int_U \dif u \wedge F = \int_{-\infty}^\infty \int_{U \cap \partial \{u > y\}} F \dif y \leq L \int_{-\infty}^\infty |U \cap \partial \{u > y\}| \dif y = L \mu(U).$$
Since $\mu$ is a Radon measure and $M$ is compact, every Borel set $E$ can be $\mu$-approximated from without by open sets, hence
\begin{equation}\label{one sided extremality}
\int_E \dif u \wedge F \leq L \mu(E).
\end{equation}

Next we deduce (\ref{1 extremality}).
We reason by contradiction: if (\ref{1 extremality}) is false, then there exists an open set $U \subseteq M$ such that 
$$\int_U \dif u \wedge F < L \int_U \star |\dif u|.$$
(Indeed, strict inequality cannot point in the other direction, by (\ref{one sided extremality}).)
However, by (\ref{one sided extremality}), 
$$\int_{M \setminus U} \dif u \wedge F \leq L \int_{M \setminus U} \star |\dif u|.$$
Adding up the integrals of $\dif u \wedge F$ over $U$ and $M \setminus U$, we conclude 
$$\int_M \dif u \wedge F < L \int_M \star |\dif u|,$$
but this contradicts (\ref{1 strong duality}); thus (\ref{1 extremality}) must be true.

To round out the proof, let $X := (\star F/L)^\sharp$ be the Poincar\'e dual vector field to $F/L$. Then
$$\nabla \cdot X = \star \frac{\dif F}{L} = 0,$$
and $\|X\|_{L^\infty} \leq 1$.
Moreover, by (\ref{1 extremality}), $X$ is normal to the level sets of $u$, and hence is a witness that $u$ is $1$-harmonic.
\end{proof}




%%%%%%%%%%%%%%%%%%%%


\section{The maximum comass locus of a best comass form}\label{MCL sec}
Let $M$ be a closed Riemannian of dimension $d \leq 7$ equipped with a cohomology class $\rho \in H^{d - 1}(M, \RR)$.
We shall study the set $\MCL(F)$ on which a best comass representative $F$ of $\rho$ attains its comass.
The main result of this section is that this set contains a lamination, analogous to the canonical lamination of Thurston \cite{Thurston98}, which only depends on $\rho$.

%%%%%%%%%%%%%%%%%%%%%
\subsection{Measured stretch laminations}
Before constructing the canonical lamination, we begin by constructing its measured sublaminations.

Let $u$ be an $\alpha$-equivariant function of least gradient on $\tilde M$.
If $\alpha$ is nonzero, then $\dif u$ is nonconstant, so that $\dif u$ is the Ruelle-Sullivan current of a measured oriented minimal lamination $\widetilde{\kappa_u}$ on $\tilde M$ by Theorem \ref{1 harmonic is MOML}.
The lamination $\widetilde{\kappa_u}$ drops to a measured oriented minimal lamination $\kappa_u$ on $M$ by equivariance.
Moreover, $\kappa_u$ is \dfn{homologically minimizing} in the sense that
$$\Mass(\kappa_u) = \Mass(\alpha);$$
this is because $u$ has least gradient and hence $\Mass(\dif u) = \Mass(\alpha)$.

Since functions of least gradient give homologically minimizing laminations, and the $1$-harmonic conjugate of a tight representative has least gradient, the next definition makes sense.

\begin{definition}
Let $\rho \in H^{d - 1}(M, \RR)$ be nonzero, let $F$ be a tight representative of $\rho$, and let $u$ be a $1$-harmonic conjugate of $F$.
Then we call $\kappa_u$ a \dfn{measured stretch lamination} associated to $\rho$.
\end{definition}

\begin{proposition}\label{MCL contains Thurston}
Let $F$ be a best comass representative of $\rho$, where $\Comass(\rho) = 1$, and let $\lambda$ be a measured stretch lamination associated to $\rho$.
Then $F$ calibrates $\lambda$. In particular, $\MCL(F) \supseteq \supp \lambda$.
\end{proposition}
\begin{proof}
Let $G$ be the tight form which is cohomologous to $F$ whose dual $1$-harmonic function $u$ defines the measured stretch lamination $\lambda$.
Then by the max flow min cut principle (\ref{1 extremality}), 
$$\Mass(\lambda) = \Mass(\dif u) = \int_M \dif u \wedge G$$
so $G$ calibrates $\lambda$ by Proposition \ref{calibration condition}.
Then by Proposition \ref{properties of calibrated laminations}, $F$ calibrates $\lambda$ and $\MCL(F) \supseteq \supp \lambda$.
\end{proof}

\begin{proposition}\label{L equals K}
	Let $\kappa$ be a measured stretch lamination for $\rho$, and let $\lambda$ range over measured oriented laminations. Then 
	$$\Comass(\rho) = \sup_\lambda \frac{\langle \rho, [\lambda]\rangle}{\Mass(\lambda)} = \frac{\langle \rho, [\kappa]\rangle}{\Mass(\kappa)}.$$
\end{proposition}
\begin{proof}
Fix a tight form $F$ representing $\rho$, and let $u$ be its $1$-harmonic conjugate.
Let $L := \Comass(\rho)$ and
$$K :=  \sup_\lambda \frac{\langle \rho, [\lambda]\rangle}{|\lambda|}.$$

We first prove $K \leq L$.
Let $\lambda$ be a measured oriented lamination; then, since $F$ represents $\rho$ and the Ruelle-Sullivan current $T_\lambda$ represents $[\lambda]$,
$$\langle \rho, [\lambda]\rangle = \int_M F \wedge T_\lambda.$$
Let $(\chi_\alpha)$ be a partition of unity subordinate to a laminar atlas for $\lambda$, and let $(\mu_\alpha)$ be the associated transverse measure. Then 
$$\int_M F \wedge T_\lambda = \sum_\alpha \int_I \int_{\{k\} \times J} \chi_\alpha F \dif \mu_\alpha(k).$$
Since $F$ has best comass,
$$\frac{\langle \rho, [\lambda] \rangle}{\Mass(\lambda)}
\leq \frac{\|F\|_{L^\infty}}{\Mass(\lambda)} \sum_\alpha \int_I \int_{\{k\} \times J} \chi_\alpha \dif S_k \dif \mu_\alpha(k) = L.$$
Since $\lambda$ was arbitrary, it holds that $K \leq L$.

By the max flow min cut principle (\ref{1 extremality}),
$$\langle \rho, [\kappa]\rangle = \int_M F \wedge \dif u = L \Mass(\dif u) = L \Mass(\kappa).$$
Dividing both sides by $\Mass(\kappa)$ and applying the direction we already proved,
$$K \leq L \leq \frac{\langle \rho, [\kappa]\rangle}{\Mass(\kappa)} \leq K$$
which is only possible if $L = K$ and $\kappa$ is a maximizer.
\end{proof}

Propositions \ref{MCL contains Thurston} and \ref{L equals K} imply Theorem \ref{lams are calibrated}.
Proposition \ref{MCL contains Thurston} also has a partial converse:

\begin{proposition}\label{calibrated means measured stretch}
Let $F$ be a best comass representative of $\rho$, where $\Comass(\rho) = 1$, and suppose that $F$ calibrates a measured oriented lamination $\lambda$.
Then $\lambda$ is a measured stretch lamination associated to $\rho$.
\end{proposition}
\begin{proof}
Let $\dif u$ be the Ruelle-Sullivan current for $\lambda$, and suppose that $f \in C^0(M)$ is supported in a flow box for $\lambda$, with local leaf space $K$ and transverse measure $\mu$.
By Proposition \ref{properties of calibrated laminations}, we may assume wihout loss of generality that $F$ is tight.
Since $F$ calibrates every leaf of $\lambda$,
$$\int_M f \star |\dif u| = \int_K \int_{\{k\} \times J} f \dif S_{\{k\} \times J} \dif \mu(k) = \int_K \int_{\{k\} \times J} fF \dif \mu(k) = \int_M f\dif u \wedge F$$
(in any case, the Radon measure $\dif u \wedge F$ is well-defined by the coarea formula).
Thus $\dif u \wedge F = \star |\dif u|$, or in other words $u$ is a $1$-harmonic conjugate of the tight form $F$.
Therefore $\lambda$ is a measured stretch lamination.
\end{proof}







%%%%%%%%%%%%%%%%
\subsection{The canonical lamination}
Throughout this section, we fix $\rho \in H^{d - 1}(M, \RR)$ in the costable unit sphere $\{\Comass(\rho) = 1\}$.
Motivated by Thurston's approach to Teichm\"uller theory (see \S\ref{Teichmuller}), we construct a lamination which is calibrated by every best comass form in $\rho$, and which only depends on $\rho$: the \dfn{canonical lamination} $\lambda_\rho$.

%%%%%%%%%%%%%%%%%%%%%
\subsubsection{Construction of the canonical lamination}
We note carefully that if $F$ is a best comass representative of $\rho$, then $\MCL(F)$ need not itself be a lamination \cite[Example 5.4]{bangert_cui_2017}.
In particular, unlike in Thurston's theory, we cannot simply take the intersection of all the maximum comass loci of best comass representatives of $\rho$.
On the other hand, one can use the existence of measured stretch laminations to show that $\MCL(F)$ contains a lamination.
So our strategy is to construct the largest lamination $\lambda_F$ which $F$ calibrates, and take an intersection over all the $\lambda_F$s.

We first rule out intersections of the leaves.
This can be done by showing that the generic intersection point of two minimal hypersurfaces is transverse.
If the dimension of the underlying manifold $M$ is $d = 2$, then this is trivial, and if $d = 3$, then the structure of $N \cap N'$ is completely described by complex-analytic means \cite[Theorem 7.3]{colding2011course}, so the proof we present here is mainly of interest if $d \geq 4$.

\begin{proposition}\label{intersection theory prop}
Let $N, N' \subset M$ be minimal hypersurfaces, and let $S \subseteq N, N'$ be the set of points at which $N, N'$ intersect nontransversely.
Then one of the following holds:
\begin{enumerate}
\item $N \cap N'$ is empty.
\item $\dim(N \cap N') = d - 2$ and $\dim S \leq d - 3$.
\item There exists $p \in S$ such that the germs of $N, N'$ at $p$ are equal.
\end{enumerate}
\end{proposition}
\begin{proof}
Let $p \in S$, and let $P$ be the tangent space of $N, N'$ at $x$.
Then we can view $N, N'$ as the graphs of functions $u, u'$ over $P$, say taken in normal coordinates based at $p$; thus we identify $P$ with $\RR^{d - 1}$.
Reasoning as in the proof of \cite[Theorem 7.3]{colding2011course}, the difference $v := u - u'$ solves a linear elliptic PDE $Qv = 0$, and in a neighborhood $U \ni p$, the exponential map $P \to N$ induces Lipschitz isomorphisms $\{v = 0\} \cap U \cong N \cap N' \cap U$ and $\{v = \dif v = 0\} \cap U \cong S \cap U$.
If $v$ only has zeroes of finite order, then the claim follows from Lemma \ref{nodal set is generically smooth}.
Otherwise, $v$ is identically $0$ by the unique continuation theorem \cite[Theorem 6.1]{colding2011course}, so $N \cap U = N' \cap U$.
\end{proof}

\begin{lemma}
Let $F$ be a calibration, and let $B \subseteq M$ be a sufficiently small ball.
Then for any complete connected $F$-calibrated hypersurface $N$, 
\begin{equation}\label{area bound for calibrated}
\Mass(N \cap B) \leq \Mass(\partial B).
\end{equation}
\end{lemma}
\begin{proof}
By the Thom transversality theorem, we may assume that $N$ meets $\partial B$ transversely. 
Let $S := N \cap \partial B$, which by transversality can be identified with a closed $d - 2$-dimensional submanifold of $\Sph^{d - 1}$.
Since $H_{d - 2}(\Sph^{d - 1}, \RR) = 0$, there exists a relatively open set $U \subseteq \partial B$ which is bounded by $S$.
Since $H^{d - 1}(B, \RR) = 0$, we may write $F = \dif A$ in a neighborhood of $B$, where by Lemma \ref{Hodge theorem} and the Sobolev embedding theorem we may assume that $A$ is continuous. Then
\begin{align*}
\Mass(N \cap B) &= \int_{N \cap B} F = \int_S A = \int_U F \leq \Mass(U) \leq \Mass(\partial B). \qedhere
\end{align*}
\end{proof}

\begin{lemma}
There exists a constant $C > 0$, only depending on $M$, such that for every calibration $F$ and complete $F$-calibrated hypersurface $N$, we have the curvature bound
\begin{equation}\label{curvature bound for calibrated}
\|\Two_N\|_{C^0} \leq C.
\end{equation}
\end{lemma}
\begin{proof}
Let $x \in N$ and let $r > 0$ be small.
Then each component $N'$ of $N \cap B(x, r)$ is absolutely area-minimizing by the fundamental theorem of calibrated geometry, so it is stable.
By (\ref{area bound for calibrated}), $\Mass(N') \lesssim r^{d - 1}$.
So by \cite[pg785, Corollary 1]{Schoen81},\footnote{See also \cite[Theorem 3]{Schoen75} for an easier proof when $M$ has nonpositive curvature and dimension $d \leq 6$, or \cite[Chapter 2, \S\S4-5]{colding2011course} for a textbook treatment of a similar estimate. By \cite[Lemma 2.4]{chodosh2022complete}, we may remove the dependence on the volume bound if $d \leq 4$.}
\begin{align*}
\|\Two_{N'}\|_{B(x, r/2)} \lesssim_{d, \|\Riem_g\|_{C^0(B(x, 2r))}} \frac{1}{r}.
\end{align*}
Since $N'$ was an arbitrary component, the same estimate holds for $N$.
Using the compactness of $M$, we may cover it by finitely many balls in which estimates of this form hold to conclude (\ref{curvature bound for calibrated}).
\end{proof}

\begin{lemma}\label{calibrated implies disjoint}
Let $F$ be a calibration, and let $N, N'$ be complete connected $F$-calibrated hypersurfaces.
If $N \cap N'$ is nonempty, then $N = N'$.
\end{lemma}
\begin{proof}
We first observe that if $x \in N \cap N'$, then $(\star F(x))^\sharp$ is the normal vector to both $N, N'$ at $x$.
Therefore $N \cap N'$ only consists of points of tangency.
By Proposition \ref{intersection theory prop}, it follows that either the germs of $N, N'$ at $x$ are equal.
Since the germs are equal and $N, N'$ are connected, a standard boostrapping argument implies that $N = N'$.
\end{proof}

\begin{proposition}\label{existence of semicanonical lamination}
Let $F$ be a best comass calibration.
Then the set of $F$-calibrated hypersurfaces is the set of leaves of a lamination $\lambda_F$, which contains every measured stretch lamination associated to $[F]$.
\end{proposition}
\begin{proof}
Let $\mathscr L_F$ be the set of connected complete $F$-calibrated hypersurfaces.
By Lemma \ref{calibrated implies disjoint}, $\mathscr L_F$ consists of pairwise disjoint minimal hypersurfaces.
By Proposition \ref{MCL contains Thurston}, there exists a measured stretch lamination $\lambda$ associated to $[F]$, and then by Proposition \ref{properties of calibrated laminations}, $\mathscr L_F$ contains every leaf of $\lambda$.
Since the estimate (\ref{curvature bound for calibrated}) is independent of $N$, it follows by Theorem \ref{disjoint surfaces are lamination} that $\mathscr L_F$ is the set of leaves of some lamination $\lambda_F$.
\end{proof}

\begin{lemma}\label{existence of intersections}
Let $\mathscr S$ be a nonempty set of laminations.
Suppose that there exists a hypersurface which is a leaf of every lamination in $\mathscr S$.
Then there exists a lamination whose set of leaves is the intersection of the sets of leaves of the laminations in $\mathscr S$.
\end{lemma}
\begin{proof}
Let $\lambda \in \mathscr S$, and let $(F_\alpha, K_\alpha)$ be a laminar atlas for $\mathscr S$.
Let $K'_\alpha$ be the set of $k \in K_\alpha$ such that for every $\kappa \in \mathscr S$, there exists a leaf $N$ of $\kappa$ such that
$$(F_\alpha)_*(\{k\} \times J) \subseteq N.$$
It is clear that this property is preserved by transition maps.
Then $K_\alpha'$ is an intersection of compact sets (since the local leaf spaces of each $\kappa \in \mathscr S$ is compact), so $K_\alpha'$ is compact.
The hypersurface which is a common leaf of every lamination in $\mathscr S$ witnesses that for some $\alpha$, $K_\alpha'$ is nonempty.
Therefore $(F_\alpha, K'_\alpha)$ is a laminar atlas for the lamination whose support is $\bigcap_{\kappa \in \mathscr S} \supp \kappa$.
\end{proof}

\begin{proposition}\label{existence of canonical lamination}
The set of hypersurfaces which are calibrated by every best comass representative of $\rho$ is the set of leaves of a lamination $\lambda_\rho$, which contains every measured stretch lamination associated to $\rho$.
\end{proposition}
\begin{proof}
By Proposition \ref{MCL contains Thurston}, there is a (measured stretch) lamination which is calibrated by every best comass representative of $\rho$.
So we may apply Lemma \ref{existence of intersections} to the set $\mathscr S$ of all calibrated laminations $\lambda_F$ produced by Proposition \ref{existence of semicanonical lamination}, where $F$ ranges over best comass representatives of $\rho$.
\end{proof}

\begin{definition}
The lamination $\lambda_\rho$ constructed in Proposition \ref{existence of canonical lamination} is the \dfn{canonical lamination} associated to $\rho$.
\end{definition}

%%%%%%%%%%%%%%%%%%%%%%%%%%%%%%%%
\subsubsection{Structure of the canonical lamination}\label{canonical structure}
We now study the structure of the canonical lamination.
A sticky technical point is that $H_{d - 1}(M, \RR)$ need not be strictly convex, so there may be many $\alpha$ in the stable unit sphere such that 
\begin{equation}\label{flats duality}
\Comass(\rho) = \langle \rho, \alpha\rangle.
\end{equation}
In particular, there may be many measured stretch sublaminations of the canonical lamination which are mutually nonhomologous.
We therefore introduce the dual set 
$$\rho^* := \{\alpha \in H_{d - 1}(M, \RR): \langle \rho, \alpha\rangle = \Mass(\alpha) = 1\}.$$
It is clear that any measured sublamination of the canonical lamination normalized to have mass $1$ represents a member of $\rho^*$.
In fact, this condition completely characterizes $\rho^*$, as we now show.

\begin{lemma}\label{homologically minimizing means measured stretch}
For every $\alpha \in \rho^*$, every measured oriented, homologically minimizing, lamination representing $\alpha$ is a measured stretch lamination associated to $\rho$.
\end{lemma}
\begin{proof}
Let $\dif u$ be the Ruelle-Sullivan current of the measured oriented, homologically minimizing lamination $\lambda$, and let $F$ be a best comass representative of $\rho$.
Since $\lambda$ is homologically minimizing,
$$\int_M \dif u \wedge F = \langle \rho, \alpha\rangle = \Mass(\alpha) = \Mass(\lambda),$$
so by Proposition \ref{calibration condition}, $F$ calibrates $\lambda$.
Therefore by Proposition \ref{calibrated means measured stretch}, $\lambda$ is a measured stretch lamination.
\end{proof}

\begin{proposition}\label{enough measures in canonical lamination}
For each $\alpha \in \rho^*$, there exists a measured stretch sublamination of $\lambda_\rho$ with homology class $\alpha$.
\end{proposition}
\begin{proof}
We first observe that there exists an $\alpha$-equivariant function $u$ of least gradient.
In fact, one can take a minimizing sequence, apply Miranda compactness \todo{cite it once the other paper is done} to obtain a limit of least gradient, and observe that the limit is $\alpha$-equivariant by Lemma \ref{L1 convergence preserves pi1}.
This is a standard argument (aside from the equivariance) and we omit the details.

The measured oriented, homologically minimizing, lamination $\kappa_u$ has class $\alpha$.
So by Lemma \ref{homologically minimizing means measured stretch}, it is a measured stretch lamination and hence is a sublamination of $\lambda_\rho$.
\end{proof}

We next use the decomposition of measured laminations \cite[{\S}I.3]{Morgan88} to partition the leaves of $\lambda_\rho$ into various categories.
In this direction we shall need to study measured laminations which are minimal with respect to inclusion; as the word ``minimal'' is overloaded, we shall call such laminations ``indecomposable''.

\begin{definition}
Let $\lambda$ be a lamination.
\begin{enumerate}
\item $\lambda$ is \dfn{indecomposable} if the only sublamination of $\lambda$ is itself.
\item If $\lambda$ is indecomposable, then $\lambda$ is \dfn{exceptional} if $\supp \lambda \neq M$ and $\lambda$ does not consist of a single leaf.
\item $\lambda$ is a \dfn{parallel family of closed leaves} if there exists a closed hypersurface $N \subset M$ with trivial normal bundle, such that every leaf of $\lambda$ is a section of the normal bundle of $N$.
\item A leaf $N$ of $\lambda$ is \dfn{nonmeasurable} if, for every sublamination $\kappa \subset \lambda$ which admits a transverse measure, $N$ is not a leaf of $\kappa$.
\end{enumerate}
\end{definition}

Thus every indecomposable lamination either is a foliation in which every leaf is dense, an exceptional indecomposable lamination, or a closed hypersurface.
Moreover, every local leaf space $K_\alpha$ of an exceptional indecomposable lamination $\lambda$ is a Cantor set \cite[{\S}I.3.1]{Morgan88}, and every leaf of $N$ is noncompact.
Every nonmeasurable leaf is noncompact, for if $N$ is a closed leaf, then $N$ equipped with its Dirac measure is a measured sublamination of $\lambda$.

\begin{theorem}\label{MorganShelan}
Let $\lambda$ be a measured oriented lamination in the closed manifold $M$.
Then either $\lambda$ is a foliation with a dense leaf, or $\lambda$ separates into finite number of clopen sublaminations, each of which is a parallel family of closed leaves or an exceptional indecomposable lamination.
\end{theorem}
\begin{proof}
First observe that the proof of \cite[Theorem I.3.2]{Morgan88} goes through for any lamination $\lambda$ such that no leaf of $\lambda$ is dense in $M$, even if $\lambda$ is a foliation.
Twisted families of closed leaves (that is, families of sections of a nontrivial normal bundle of a closed hypersurface) are excluded by the fact that $\lambda$ is oriented, so its leaves are oriented, and hence the normal bundle of any of its leaves is trivial.
\end{proof}

\begin{proposition}\label{classification of leaves}
For each leaf $N$ of $\lambda_\rho$, one of the following holds:
\begin{enumerate}
\item $N$ is closed.
\item $N$ is a noncompact leaf of an exceptional indecomposable measured stretch lamination associated to $\rho$.
\item $N$ is noncompact and $\lambda_\rho$ is a foliation which admits a transverse measure.
\item $N$ is noncompact and $N$ is a nonmeasurable leaf of $\lambda_\rho$.
\end{enumerate}
\end{proposition}
\begin{proof}
% First suppose that there are infinitely many distinct homology classes $(\alpha_j)$ which are represented by an indecomposable measured sublamination $\eta_j$ of $\lambda_\rho$, such that $\Mass(\eta_j) = 1$.
% If $\eta_j, \eta_k$ have a common leaf, then by Lemma \ref{existence of intersections}, $\eta_j \cap \eta_k$ is a well-defined sublamination of $\eta_j$, hence $j = k$ by indecomposability.
% Thus the $\eta_j$ are pairwise disjoint, and we may let $\kappa_j$ be the union of $\eta_1, \dots, \eta_j$, where $\eta_k$ has weight $2^{-k}$.
% Then $\Mass(\kappa_j) = 1 - 2^{-j}$, and by \todo{compactness of the weak topology} and \todo{limsup of the measures}, $\kappa_j$ converges to some measured sublamination $\kappa$ of $\lambda_\rho$.
% By Theorem \ref{MorganShelan}, at least one of the following holds:
% \begin{enumerate}
% \item $\kappa$ is a foliation, and there exist infinitely many nonhomologous closed leaves of $\kappa$. \todo{By c}ompactness of $M$ must all be sections of the normal bundle of some leaf $N$.
% By orientability of $\kappa$, $N$ must have trivial bundle, so that $\kappa$ contains a parallel family of nonhomologous closed leaves, an absurdity.
% \item $\kappa$ is a foliation, and there exist infinitely many nonhomologous exceptional indecomposable sublaminations of $\kappa$. \todo{Why not?}
% \item There are finitely many parallel families of closed leaves $\zeta_1, \dots, \zeta_m$ such that all but finitely many $\eta_j$ are leaves of some $\zeta_i$. So by the infinite pigeonhole principle, there exists a parallel family of closed leaves $\zeta_i$ with two leaves which are nonhomologous, which is absurd.
% \end{enumerate}
% Thus we deduce that there are only finitely many homology classes which are represented by mass-$1$ indecomposable measured sublaminations of $\lambda_\rho$.
% So we may let $\rho^*_{\rm cl}$ be the set of homology classes in $\rho^*$ which contain a closed leaf of $\lambda_\rho$, and $\rho^*_{\rm exc}$ be the set of homology classes in $\rho^*$ which contain an exceptional indecomposable measured sublamination of $\lambda_\rho$.
If $N$ is a closed leaf of $\lambda_\rho$, then $N$ equipped with its Dirac measure is a measured lamination, calibrated by any tight representative of $\rho$; hence it is measured stretch for $\rho$.
Otherwise, since $N$ has no boundary, it is noncompact.

If $N$ is noncompact, but is contained in a measured sublamination $\kappa$ of $\lambda_\rho$, then by Theorem \ref{MorganShelan}, either $\kappa$ is a foliation or $N$ is contained in an exceptional indecomposable sublamination.
If $\kappa$ is a foliation, then
$$\supp \kappa \supseteq \supp \lambda_\rho \supseteq \supp \kappa,$$
implying $\kappa = \lambda_\rho$.
Otherwise, the exceptional indecomposable sublamination $\zeta$ of $\kappa$ containing $N$ is calibrated by any tight representative of $\rho$, so $\zeta$ is measured stretch for $\rho$ by Proposition \ref{calibrated means measured stretch}.
\end{proof}

\begin{corollary}\label{measurable leaves are contained in indecomposables}
Let $N$ be a leaf of the canonical lamination $\lambda_\rho$.
Then either $N$ is nonmeasurable, or $N$ is contained in an indecomposable measured stretch lamination associated to $\rho$.
\end{corollary}
\begin{proof}
By Proposition \ref{classification of leaves}, if $N$ is not nonmeasurable, then either $N$ is closed, in which case $N$ is itself an indecomposable measured stretch lamination, or $N$ is noncompact and is contained in an exceptional indecomposable measured stretch lamination.
\end{proof}

\begin{corollary}
Let $F$ be a best comass representative of $\rho$, and $N$ a leaf of the calibrated lamination $\lambda_F$.
Then either $N$ is a leaf of the canonical lamination $\lambda_\rho$, or $N$ is a nonmeasurable leaf of $\lambda_F$.
\end{corollary}
\begin{proof}
Suppose that $N$ is a leaf of a measured sublamination $\kappa$ of $\lambda_F$.
Then, since $\kappa$ is calibrated by $F$, $\kappa$ is measured stretch by Proposition \ref{calibrated means measured stretch}, hence is a sublamination of $\lambda_\rho$.
\end{proof}

Another consequence of the decomposition of laminations is that the extreme points of $\rho^*$ are represented by indecomposable laminations.
Recall that a point $\alpha$ of a convex set $S$ is \dfn{extreme} if $\alpha$ cannot be written as the convex combination of two distinct members of $S$.

\begin{lemma}\label{extreme points are closed under sublaminations}
Let $\alpha$ be an extreme point of $\rho^*$, and let $\kappa$ be a measured stretch lamination in $\alpha$.
Then any sublamination of $\kappa$ represents a scalar multiple of $\alpha$.
\end{lemma}
\begin{proof}
By replacing $\kappa$ with a proper sublamination if necessary, we may assume that $\kappa$ is not a foliation.
Let $\zeta$ be a sublamination of $\kappa$.
By Theorem \ref{MorganShelan} and the fact that the leaves of a parallel family of closed leaves are all homologous, after replacing $\zeta$ with a sublamination of $\zeta$, we may assume that $\zeta$ is a clopen parallel family of closed leaves, or is an exceptional indecomposable sublamination of $\kappa$.
Since $\kappa$ is the linear combination of finitely many such clopen sublaminations, we may write $\alpha$ as a convex combination of $\beta_1, \dots, \beta_k$ where the $\beta_i$ are the (normalized to mass $1$) homology classes of clopen sublaminations of $\lambda$.
But $\beta_i \in \rho^*$, so $\beta_i = \alpha$, hence $[\zeta] = \alpha$.
\end{proof}

\begin{proposition}\label{extreme points are indecomposable}
Let $\alpha$ be an extreme point of $\rho^*$. Then $\alpha \in \rho^*_{\rm exc}$.
\end{proposition}
\begin{proof}
By Proposition \ref{enough measures in canonical lamination}, there exists a measured stretch lamination $\kappa$ representing $\alpha$.
By Theorem \ref{MorganShelan}, $\kappa$ has an indecomposable sublamination $\zeta$.
By Lemma \ref{extreme points are closed under sublaminations}, possibly after rescaling the transverse measure, $\zeta$ is a representative of $\alpha$.
Since any tight representative $F$ of $\rho$ calibrates $\kappa$, $F$ also calibrates $\zeta$, so by Proposition \ref{calibrated means measured stretch}, $\zeta$ is a measured stretch sublamination of $\lambda_\rho$.
\end{proof}

%%%%%%%%%%%%%%%%%%%%%%%%%
\subsubsection{Motivation for the canonical lamination}\label{Teichmuller}
Our motivation for introducing the canonical lamination arose from an analogy with Thurston's approach to Teichm\"uller theory using best Lipschitz maps \cite{Thurston98}.\footnote{None of this discussion shall be used in the sequel, except as motivation, and to state some conjectures in \S\ref{open problems}.} \todo{Reword some of this}
Given $\gamma \geq 2$, let $\widetilde{\mathscr M}_\gamma$ be the Teichm\"uller space of hyperbolic metrics on the closed surface $S_\gamma$ of genus $\gamma$.
Given $g, h \in \widetilde{\mathscr M}_\gamma$, let $\Lip(g, h)$ be the Lipschitz constant of a best Lipschitz map $(S_\gamma, g) \to (S_\gamma, h)$; then for a tangent vector $v \in T_g(\widetilde{\mathscr M}_\gamma)$, let $\Comass(v)$ be the partial derivative of $\log \Lip(g, \cdot)$ in the direction $v$.
This quantity, the \dfn{Thurston asymmetric norm}, is an asymmetric norm on $T_g(\widetilde{\mathscr M}_\gamma)$ obtained by solving an $L^\infty$ variational problem intimately tied to the structure of minimal laminations, so it is tempted to make an analogy between $T_g(\widetilde{\mathscr M}_\gamma)$ and $H^{d - 1}(M, \RR)$, where both vector spaces are equipped with the norm $\Comass$.
Two particularly salient pieces of evidence for the analogy are:
\begin{enumerate}
\item The unit spheres of the dual spaces of $T_g(\widetilde{\mathscr M}_\gamma)$ and $H^{d - 1}(M, \RR)$ can both be viewed as spaces of projective measured minimal laminations, whose norm is given by an $L^1$ (actually $BV$) variational problem \cite[Theorem 5.1]{Thurston98}.
\item In both cases, we can construct a canonical lamination; in Thurston's case, the canonical lamination is given by those geodesics which are maximally stretched by every best Lipschitz map homotopic to $\id_{S_\gamma}$ \cite[\S8]{Thurston98}. See also Conjecture \ref{chain recurrence}.
\end{enumerate}
However, one should not take this analogy too seriously.
A key feature of Thurston's theory is the Birman-Series theorem: the union of the supports of all geodesic laminations on $(S_\gamma, g)$ has Hausdorff dimension $0$.
As a corollary, for almost every $h \in \widetilde{\mathscr M}_\gamma$, the canonical lamination associated to $(g, h)$ is a closed geodesic \cite[\S10]{Thurston98}.
The analogue of the Birman-Series theorem is clearly not true in our case, and in fact, if $M$ is a square flat torus, then it is easy to see that every canonical lamination covers all of $M$.
But see Conjecture \ref{Karen}.

%%%%%%%%%%%%%%%%%%%%%%%%
\subsection{Convexity of the stable unit ball}\label{convexity sec}
Auer and Bangert \cite{Auer01} claimed certain results concerning the convex structure of the stable unit ball
$$B := \{\alpha \in H_{d - 1}(M, \RR): \Mass(\alpha) \leq 1\}.$$
Here we show that some of these results follow from the structure theory of the canonical lamination.

Recall that a \dfn{flat} $S \subset \partial B$ is a set such that, for some supporting hyperplane $H$ of $B$, $S = H \cap B$.
Thus $B$ is strictly convex iff every flat is a point.

\begin{lemma}
Suppose that $S \subset \partial B$ is a flat.
Then there exists $\rho$ in the costable unit sphere such that $S \subseteq \rho^*$.
\end{lemma}
\begin{proof}
Since $S$ is convex, $\partial S$ is topologically a sphere, so $\partial S$ admits a Borel probability measure $\nu$ of full support.
Then we take the vector-valued integral 
$$\beta := \int_{\partial S} \alpha \dif \nu(\alpha),$$
thus $\beta \in S$ by convexity.
By the Hanh-Banach theorem (Theorem \ref{Federer}), there exists $\rho \in H^{d - 1}(M, \RR)$ such that $\beta \in \rho^*$.

We claim that $\partial S \subseteq \rho^*$.
If not, then by continuity of $\alpha \mapsto \langle \rho, \alpha\rangle$, there is a positive measure set of $\partial S$ on which $\langle \rho, \cdot\rangle < 1$, hence
$$\Comass(\rho) = \langle \rho, \beta\rangle = \int_{\partial S} \langle \rho, \alpha\rangle \dif \nu(\alpha) < \Comass(\rho),$$
a contradiction.
Since $\rho^*$ is convex, it follows that $S \subseteq \rho^*$.
\end{proof}

% We identify homology classes $\alpha \in H_\ell(M, \RR)$ with their Poincar\'e duals in $H^{d - \ell}(M, \RR)$.
% Then the wedge product 
% $$H^1(M, \RR) \times H^1(M, \RR) \to H^2(M, \RR)$$
% induces a bilinear map 
% \begin{align*}
% H_{d - 1}(M, \RR) \times H_{d - 1}(M, \RR) &\to H_{d - 2}(M, \RR) \\
% (\alpha, \beta) &\mapsto \alpha \cdot \beta.
% \end{align*}
% Recall that the product $\alpha \cdot \beta$ is called the \dfn{intersection product} of $\alpha$ and $\beta$.

\begin{proposition}\label{flats are nonintersecting}
Suppose that $S \subset \partial B$ is a flat, and $\alpha, \beta \in S$. Then $\alpha \cdot \beta = 0$.
\end{proposition}
\begin{proof}
Let $\rho$ be the cohomology class dual to $S$ given by (\ref{flats duality})
By Proposition \ref{enough measures in canonical lamination}, there exist measured stretch sublaminations $\kappa_\alpha, \kappa_\beta$ of $\lambda_\rho$, of classes $\alpha, \beta$.
Let $\dif u_\alpha, \dif u_\beta$ be their Ruelle-Sullivan currents, and suppose that $x$ is in the union of their supports.
By \todo{previous paper}, if $N$ denotes the leaf of $\lambda_\rho$ containing $x$, then for $\sigma = \alpha, \beta$,
$$\dif u_\sigma(x) = \normal_N^\flat(x) \mu_\sigma(x)$$
where $\mu_\sigma$ is the positive Radon measure induced on $M$ by the transverse measure to $\kappa_\sigma$.
In particular, $\dif u_\alpha|_{\supp \dif u_\beta}$ is a scalar multiple of $\dif u_\beta$, so $\dif u_\alpha \wedge \dif u_\beta = 0$, hence $\alpha \cdot \beta = 0$.
\end{proof}

\begin{corollary}\label{condition for strict convexity}
Suppose that for every for every pair of linearly independent homology classes $\alpha, \beta \in H_{d - 1}(M, \RR)$, $\alpha \cdot \beta \neq 0$.
Then the stable unit ball of $H_{d - 1}(M, \RR)$ is strictly convex.
\end{corollary}

\begin{corollary}\label{torus convex}
Let $M$ be homeomorphic to $\RR^d/\ZZ^d$.
Then the stable unit ball of $H_{d - 1}(M, \RR)$ is strictly convex.
\end{corollary}
\begin{proof}
By Corollary \ref{condition for strict convexity}, it suffices to show that if $\theta, \omega \in H^1(M, \RR)$ are linearly independent, then $\theta \wedge \omega \neq 0$.
However, the cohomology ring $H^\bullet(M, \RR)$ is isomorphic to the exterior algebra of $\RR^d$, and the result follows.
\end{proof}

% If $B$ is not strictly convex, then we instead show that its stable unit ball has polytopes for flats.
% Using estimates on the number of components of the decomposition of a measured lamination, it is possible to bound the number of the vertices of such a polytope \cite{Auer01}, but we shall not attempt to do so here.

% \begin{proposition}\label{flats are polytopes}
% Let $S \subset \partial B$ be a flat. Then $S$ is a convex polytope.
% \end{proposition}
% \begin{proof}
% By Proposition \ref{extreme points are indecomposable}, every extreme point of $S$ has an indecomposable sublamination of the canonical lamination.
% By Proposition \ref{classification of leaves}, there are finitely many exceptional indecomposable laminations, and finitely many homology classes of closed leaves (since they fit into finitely many parallel families).
% \todo{Why are there only finitely many in the canonical lamination? What if it has infinitely many disjoint measured sublaminations? It's the same proof that $\rho^*_{\rm cl}$ is finite}
% Therefore $S$ only has finitely many extreme points, hence is a convex polytope.
% \end{proof}

%%%%%%%%%%%%%%%%%%%
% \subsection{The canonical lamination under rationality assumptions}
% \begin{definition}
% A homology class $\alpha \in H_{d - 1}(M, \RR)$ has \dfn{rational direction} if there exists $c > 0$ such that $c\alpha \in H_{d - 1}(M, \QQ)$.
% A cohomology class $\rho \in H^{d - 1}(M, \RR)$ with $\Comass(\rho) = 1$ has \dfn{corational direction} if $\rho^*$ is a singleton, whose member has rational direction.
% \end{definition}

% \begin{proposition}
% Suppose that $\alpha \in H_{d - 1}(M, \RR)$ has rational direction.
% Then every measured oriented lamination representing $\alpha$ decomposes into finitely many parallel families of closed leaves.
% \end{proposition}
% \begin{proof}
% 	\todo{This is known, fill out the details later}
% \end{proof}

% \begin{corollary}
% Suppose that $\rho \in H^{d - 1}(M, \RR)$ has corational direction.
% Then the canonical lamination $\lambda_\rho$ decomposes into finitely many parallel families of closed leaves.
% \end{corollary}

% \begin{corollary} \label{closed families are dense}
% The set of $\rho$ such that the canonical lamination $\lambda_\rho$ decomposes into finitely many parallel families of closed leaves is dense in the costable unit sphere of $H^{d - 1}(M, \RR)$.
% \end{corollary}
% \begin{proof}
% Suppose not, so there exists a nonempty relatively open subset $U$ of the costable unit sphere such that for $\rho \in U$, $\lambda_\rho$ does not have the desired property.

% \end{proof}

%%%%%%%%%%%%%%%%%%%
% %%%%%%%%%%%%%%%%
% \subsection{Global structure of calibrated laminations}
% \todo{All this is pretty speculative and informally written.}

% \begin{definition}
% A lamination $\lambda$ is \dfn{transitive} if every leaf of $\lambda$ is dense in $\supp \lambda$.
% \end{definition}

% I called it this because minimal set is taken and it reminds me of topologically transitive.

% \begin{lemma}
% Let $\lambda$ be a lamination. Then:
% \begin{enumerate}
% \item The closure of any leaf of $\lambda$ is the support of a sublamination of $\lambda$.
% \item $\lambda$ is transitive iff $\lambda$ has no proper sublaminations.
% \item $\lambda$ has a transitive sublamination.
% \end{enumerate}
% \end{lemma}
% \begin{proof}
% We first recall that $\supp \lambda$ is closed, so if $N$ is a leaf of $\lambda$, then $\overline N \subseteq \supp \lambda$.
% So let $N'$ be a leaf of $\lambda$ and $P \in (\overline N \setminus N) \cap N'$.
% Choose coordinates $(x, y)$ based at $P$ where $N' = \{y = 0\}$.
% Then in a neighborhood of $P$, we can write $N$ as consisting of sheets $\{y = f_n(x)\}$, and after reordering the sheets and reorienting the coordinate system, we may assume that $0 < f_{n + 1} < f_n$.
% Thus $f_n(x)$ decreases to some $f(x)$ such that $f(0) = 0$.
% In particular,
% $$\{y = f(x)\} \subseteq \overline N \setminus N \subseteq \supp \lambda$$
% so $\{y = f(x)\}$ must be a sheet of a leaf of $\lambda$; since $f(0) = 0$ and $(0, 0) \in N'$, it follows that $\{y = f(x)\}$ is a sheet of $N'$.
% Therefore $\overline N \cap N'$ is an open subset of $N'$; since it is clearly a closed subset of $N'$ and $N'$ is connected, it follows that $N' = \overline N \cap N'$, or in other words $N' \subseteq \overline N$.
% The set $\mathscr L$ of leaves of $\lambda$ which meet $\overline N$ is therefore equal to the set of leaves contained in $\overline N$, hence the union of $\mathscr L$ is closed.
% So $\mathscr L$ is the set of leaves of a sublamination of $\lambda$, whose support is obviously $\overline N$.

% If $\lambda$ is transitive and $\mu \preceq \lambda$, then the closure of any leaf of $\mu$ must contain $\supp \lambda$, so $\lambda \preceq \mu$, hence $\lambda = \mu$.
% Conversely, if $\lambda$ is nontransitive, then there is a leaf $N$ of $\lambda$ such that $\overline N \subset \supp \lambda$, so that $\overline N$ is the support of a proper sublamination of $\lambda$.

% Finally, let $\mathscr P$ be the poset of sublaminations of $\lambda$ with the ordering $\preceq$.
% If $\mathscr C \subseteq \mathscr P$ is a chain, then the laminations in $\mathscr C$ have a common leaf, so the meet of $\mathscr C$ is well-defined and therefore is a lower bound of $\mathscr C$.
% So by Zorn's lemma, $\mathscr P$ has a minimal element, which has no proper sublaminations and therefore is transitive.
% \end{proof}


% \todo{Does a transitive calibrated lamination admit a transverse measure? Krylov-Bogoliubov}







%%%%%%%%%%%%%%%%%%%%%%%%%%%%
% \subsection{Existence of an optimal best comass form}
% \todo{exposit this}

% \begin{definition}
% An \dfn{optimal best comass form} is a best comass form $F$ such that
% $$\MCL(F) = \bigcap_G \MCL(G)$$
% where $G$ ranges over best comass forms cohomologous to $F$.
% \end{definition}

% \begin{proposition}
% Let $\rho \in H^2(M, \RR)$.
% Then there exists an optimal best comass representative of $\rho$.
% \end{proposition}
% \begin{proof}
% Let $\lambda := \bigcap_G \MCL(G)$ where $G$ ranges over best comass representatives of $\rho$.
% For $x \notin \lambda$, we can find a best comass form $F_x$ of class $\rho$ such that $x \notin \MCL(F_x)$.
% In particular, $U_x := \{L(F_x, \cdot) < L\}$ is an open set which contains $x$, so $(U_x)_{x \notin \lambda}$ is an open cover of $M \setminus \lambda$.
% Since $M \setminus \lambda$ is $\sigma$-compact, there exists a countable subcover $(U_{x_i})_{i \in I}$, for some countable set $I \subseteq \NN$.

% We then introduce the closed form 
% $$F := \sum_{i \in I} \alpha_i F_{x_i},$$
% where $\sum_{i \in I} \alpha_i = 1$.
% Here the sum converges in the norm topology of $L^\infty$, even if $I$ is infinite.
% Indeed, if $I_N := I \cap \{1, \dots, N\}$, then the partial sums $\sum_{i \in I_N} \alpha_i F_{x_i}$ satisfy the tail bound
% $$\sum_{i \in I \setminus I_N} \alpha_i \|F_{x_i}\|_{L^\infty} \leq L \sum_{i \in I \setminus I_N} \alpha_i \to 0$$
% since $(\alpha_i) \in \ell^1$, which implies the convergence.
% This convergence implies (by Proposition \ref{crandall}) that $[F] = \rho$, so $L(F) \geq L$.
% On the other hand, 
% $$L(F) \leq \sum_{i \in I} \alpha_i L(F_{x_i}) \leq L \sum_{i \in I} \alpha_i = L.$$
% It follows that $L(F) = L$.
% In particular, $F$ has best comass and $\MCL(F) \supseteq \lambda$.

% To complete the proof, we show that $\MCL(F) \subseteq \lambda$.
% Let $x \notin \lambda$, and let $j$ satisfy $U_{x_j} \ni x$.
% Then by Proposition \ref{crandall},
% $$L(F, x) = \lim_{r \to 0} L_{B_r(x)}(F) \leq \lim_{r \to 0} \sum_{i \in I} \alpha_i L_{B_r(x)}(F_{x_i}).$$
% The summands are dominated by the $\ell^1$ sequence $(L\alpha_i)$, so by dominated convergence, 
% $$\lim_{r \to 0} \sum_{i \in I} \alpha_i L_{B_r(x)}(F_{x_i}) = \sum_{i \in I} \lim_{r \to 0} \alpha_i L_{B_r(x)}(F_{x_i}) = \sum_{i \in I} \alpha_i L(F_{x_i}, x).$$
% By assumption on $j$, $L(F_{x_j}, x) < L$, and besides $L(F_{x_i}, x) \leq L$ for any $i$.
% So 
% $$L(F, x) \leq \sum_{i \in I} \alpha_i L(F_{x_i}, x) < L$$
% and we conclude $x \notin \MCL(F)$.
% \end{proof}

%%%%%%%%%%%%%%%%%%%%%%%%%%%%%%%%
\section{The Euler-Lagrange equation for tight forms}\label{infinityMax}
At present there is not a suitable theory of viscosity solutions of PDEs for vector-valued maps, and in general, one does not expect our tight forms to be much more regular than $L^\infty$.
Therefore there is no sense that we are aware of that tight forms should solve an Euler-Lagrange equation.
However, we shall formally derive what analytic properties tight forms \emph{should} have, in the tradition of various other papers \cite{Barron2001,Aronsson67,Sheffield12} on the $L^\infty$ calculus of variations.

\subsection{Formal derivation of Euler-Lagrange equation}
To state our Euler-Lagrange equation we introduce some notation for derivatives of tensor fields along differential forms.
If $\alpha$ is a $k$-form, $\nabla$ is the Levi-Civita connection, and $T$ is a section of a tensor bundle $E$, we introduce the tensor $\nabla^\alpha T$, a section of $E \otimes \Omega^{k - 1}$, defined as follows: if $X_1, \dots, X_{k - 1}$ are vector fields, and
$$Y := (\iota_{X_1} \cdots \iota_{X_{k - 1}} \alpha)^\sharp$$
is the vector field dual to the contraction of $\alpha$, then
$$\langle \nabla^\alpha T, X_1 \otimes \cdots \otimes X_k\rangle := \nabla_Y T.$$
We think of $\nabla^\alpha T$ as a sort of ``weighted projection'' of $\nabla T$ to the subbundle $\ker \star \alpha \subset TM$.

\begin{proposition}
Suppose that $F_p$ are $C^1$ $p$-tight forms converging to a tight form $F$.
Furthermore suppose that as $p \to \infty$, $\|F_p\|_{C^{1 + \alpha}} \lesssim 1$.
Then $F \in C^1$ and 
\begin{equation}\label{infty Max}
\begin{cases}
\dif F = 0, \\
\langle \nabla^F F, F\rangle = 0.
\end{cases}
\end{equation}
\end{proposition}

We should clarify the PDE (\ref{infty Max}): $\nabla^F F$ is a section of $\Omega^{d - 1} \otimes \Omega^{d - 2}$, so its contraction with $F$ is the contraction of the $\Omega^{d - 1}$ part with $F$; thus $\langle \nabla^F F, F\rangle$ is a $d - 2$-form.
If $d = 2$, and $F = \dif u$, then (\ref{infty Max}) is exactly the $\infty$-Laplace equation 
$$\langle\nabla^2 u, \nabla u \otimes \nabla u\rangle = 0.$$

\begin{proof}
We first compute from (\ref{pMaxwell}) that $\dif F = 0$ and
\begin{align*}
0
&= \dif(|F_p|^{p - 2} \star F_p) \\
&= \dif(|F_p|^{p - 2}) \wedge \star F_p + |F_p|^{p - 2} \dif \star F_p \\
&= (p - 2) |F_p|^{p - 4} \langle \nabla F_p, F_p\rangle \wedge \star F_p + |F_p|^{p - 2} \dif \star F_p.
\end{align*}
If $F_p$ is nonzero, then we can divide through by $(p - 2) |F_p|^{p - 4}$ to get
\begin{equation}\label{intermediate p Max}
0 = \langle\nabla F_p, F_p\rangle \wedge \star F_p + \frac{|F_p|^2}{p - 2} \dif \star F_p.
\end{equation}
At the zeroes of $F_p$, we simply observe that (\ref{intermediate p Max}) holds for trivial reasons.

By assumption $||F_p|^2 \dif \star F_p| = o(p)$, so as $p \to \infty$, the second term of (\ref{intermediate p Max}) drops out.
We can take the limit of (\ref{intermediate p Max}) using the equicontinuity of $\nabla F_p$ to get
$$0 = \langle \nabla F, F \rangle \wedge \star F.$$
Taking the Hodge star of both sides, we get (\ref{infty Max}).
\end{proof}

%%%%%%%%%%%%%%%%%%%%%%%%%
\subsection{Geometric and variational interpretations of the Euler-Lagrange equation}
The PDE (\ref{infty Max}) has a simple geometric interpretation, which generalizes the interpretation of the $\infty$-Laplace equation as asserting that the gradient curves of an $\infty$-harmonic function are lines.

\begin{proposition}\label{infty Max calibrates}
Let $F$ be a $C^1$ solution of (\ref{infty Max}) with no zeroes, and let $N$ be a connected integral hypersurface of $\ker \star F$.
Then there exists $\lambda > 0$ such that $N$ is an $F/\lambda$-calibrated hypersurface.
\end{proposition}
\begin{proof}
Let $(X_1, \dots, X_{d - 1})$ be an orthonormal frame of vector fields tangent to $N$.
We then introduce the tensor field
$$T_i := X_1 \otimes \cdots \otimes \widehat{X_i} \otimes \cdots \otimes X_{d - 1},$$
where the hat means to remove that factor.
By definition of $N$, and the fact that $F$ has no zeroes, $F$ is a nonzero scalar field $\lambda$ times $\dif S_N$.
So $\iota_{X_1} \cdots \widehat{\iota_{X_i}} \cdots \iota_{X_{d - 1}} F$ is a nonzero scalar field $u_i$ times $X_i^\flat$.
Applying (\ref{infty Max}), we have 
$$0 = \langle \langle \nabla^F F, F\rangle, T_i\rangle = u_i \langle \nabla_{X_i} F, F \rangle = \frac{u_i}{2} \partial_{X_i} (|F|^2).$$
Since $u_i/2$ is nonzero, we see that $|F|^2$ is constant along integral curves of $X_i$.
Since $(X_1, \dots, X_{d - 1})$ spans the tangent bundle of $N$, we conclude that $|F|^2$ is constant along $N$, or equivalently that $\lambda$ is a constant.
Therefore $\dif S_N = F/\lambda$ is closed, hence $N$ is $F/\lambda$-calibrated.
\end{proof}

We now give a variational criterion for (\ref{infty Max}).
Unfortunately, the converse is weaker than we would like, because in order to apply arguments similar to those of \cite{Aronsson67,Sheffield12} we need to assume that $\ker(\star F)$ is a singular integrable distribution.

\begin{proposition}
Let $F$ be a $C^1$ closed $d - 1$-form, and suppose that for every $V \subseteq M$ such that $H^{d - 1}(V, \RR) = 0$, and every $U \Subset V$,
\begin{equation}\label{ABC inequality}
\Comass_U(F) \leq \|F\|_{C^0(\partial U)}.
\end{equation}
Then (\ref{infty Max}) holds.
\end{proposition}
\begin{proof}
It suffices to prove (\ref{infty Max}) locally, so we can cover $M$ by open balls $V$ with $H^{d - 1}(V, \RR) = 0$ and prove that (\ref{infty Max}) holds in slightly smaller balls.
Since $H^{d - 1}(V, \RR) = 0$, we can find a $C^2$ $d - 2$-form $A$ on a neighborhood of $\overline V$ with $\dif A = F$.
Also for $\xi$ a covariant tensor of valence $d - 1$ at $x$, let $\xi^{\rm as}$ be its antisymmetrization, and
$$f(x, \xi) := |\xi^{\rm as}|_{g^{-1}(x)}^2.$$

We first claim that for any $d - 2$-form $B$, $f(\cdot, B) = |\dif B|^2$.
Since $\nabla$ is torsion-free, antisymmetrization annihilates the Christoffel symbols of $\nabla$, so if $\nabla^\flat$ denotes a flat connection, then $(\nabla B)^{\rm as} = (\nabla^\flat B)^{\rm as}$; the latter is of course $\dif B$.
Thus in particular, $f(\cdot, A) = |F|^2$.

Let $W \Subset V$ be obtained by slightly shrinking $V$.
We claim that $A|_W$ is an absolute minimizer of $f(x, \nabla A(x))$ in the sense of \cite[Definition 5.1]{Barron2001}.
In other words, we claim that for each open $U \subseteq W$ with smooth boundary and each covariant tensor field $B$ of valence $d - 2$, such that $A - B$ has compact support in $U$,\footnote{Strictly speaking, the definition of absolute minimizer ranges over all open sets $U$ ($\Omega'$ in the notation of \cite{Barron2001}), not just those with smooth boundary; similarly one requires the competition class to range over traceless (rather than compactly supported) variations. However, it is trivial to modify the proof of \cite[Theorem 5.2]{Barron2001} to only require smooth domains and compactly supported variations.}
$$\sup_{x \in U} f(x, A(x)) \leq \sup_{x \in U} f(x, B(x)).$$
To see this, let $G = (\nabla B)^{\rm as}$, so that by (\ref{ABC inequality}),
\begin{align*}
\sup_{x \in U} f(x, A(x))
&= \Comass_U(F) \leq \|F\|_{C^0(\partial U)} = \|G\|_{C^0(\partial U)} \leq \|G\|_{C^0(U)} = \sup_{x \in U} f(x, B(x)).
\end{align*}

By the above claims and \cite[Theorem 5.2]{Barron2001}, for each $x \in W$, we have the Euler-Lagrange-Aronsson equation that for any $d - 2$-form $\theta$,
\begin{align*}
0 
&= \left\langle \frac{\partial f}{\partial \xi}(x, \nabla A(x)), \nabla \left[f(x, \nabla A(x))\right] \otimes \theta(x)\right\rangle \\
&= 2\langle (\nabla A(x))^{\rm as}, \nabla(|(\nabla A(x))^{\rm as}|^2) \otimes \theta(x)\rangle \\
&= 2\langle F(x), \nabla(|F(x)|^2) \otimes \theta(x)\rangle.
\end{align*}
Since $\nabla$ is a metric connection, $\nabla(|F|^2) = 2\langle \nabla F, F\rangle$.
Thus we have 
$$0 = 4\langle F, \langle \nabla F, F\rangle \otimes \theta\rangle = 4\langle \langle \nabla^F F, F\rangle, \theta\rangle$$
and since $\theta$ was arbitrary we conclude (\ref{infty Max}).
\end{proof}

\begin{proposition}
Let $F$ be a $C^1$ solution of (\ref{infty Max}) such that $\star F \wedge \dif(\star F) = 0$.
Then for every $x \in M$ and every sufficiently small $r > 0$, (\ref{ABC inequality}) holds for $U = B(x, r)$.
\end{proposition}
\begin{proof}
First observe that if $x \in M$ and $F(x) = 0$, then $|F|$ has a local minimum at $x$, so for any $y$ sufficiently close to $x$, say $y \in B(x, r)$, $|F|$ does not have a local maximum at $y$ (unless $y$ is also a local minimum, hence $F(y) = 0$); therefore (\ref{ABC inequality}) holds.
Henceforth we assume that $F(x) \neq 0$.

Let $r > 0$ be such that $F|_{B(x, r)}$ has no zeroes and for some $s > r$, $H^{d - 1}(B(x, s), \RR) = 0$.
Since $\star F \wedge \dif(\star F) = 0$, $\ker(\star F)$ integrates to a foliation $\mathscr F$ of $B(x, r)$.
By definition of $s$, we may assume that $F = \dif A$ for some $d - 2$-form $A$ defined on $B(x, s)$.
By Proposition \ref{infty Max calibrates}, for each leaf $N$ of $\mathscr F$ there exists $\lambda_N > 0$ such that for each hypersurface $N'$ with $N \cap \partial B(x, r) = N' \cap \partial B(x, r)$,
$$\lambda_N \Mass(N) = \int_N F = \int_{N \cap \partial B(x, r)} A = \int_{N' \cap \partial B(x, r)} A = \int_{N \cap \partial B(x, r)} F \leq \lambda_N \Mass(N'),$$
so $N$ is absolutely area-minimizing in $B(x, r)$ and hence meets $\partial B(x, r)$, say at some point $x_N$.
Moreover, $F|_N$ has constant comass $\lambda_N$; since $\mathscr F$ is a foliation it holds that for each $x \in V$ there exists a leaf $N \ni x$ of $\mathscr F$, and then 
\begin{align*}
|F(x)| &= |F(x_N)| \leq \|F\|_{C^0(\partial B(x, r))}. \qedhere
\end{align*}
\end{proof}

%%%%%%%%%%%%%%%
% \subsection{Parabolic character and higher regularity}
% Similarly to the $\infty$-Laplace equation, \todo{Cite Evans--Smart ``adjoint methods''} we shall view (\ref{infty Max}) as a degenerate parabolic equation.
% To be more precise, suppose that $A$ is a $d - 2$-form in Coulomb gauge such that $F = \dif A$ solves (\ref{infty Max}), hence 
% \begin{equation}\label{integrated infty Max}
% \begin{cases}
% \langle \nabla^{\dif A} \dif A, \dif A\rangle = 0, \\
% \dif^* A = 0.
% \end{cases}
% \end{equation}
% We consider the linearized equation at $A$
% \begin{equation}\label{linearized infty Max}
% \begin{cases}
% \langle \nabla^{\dif B} \dif A, \dif A\rangle + \langle \nabla^{\dif A} \dif B, \dif A\rangle + \langle \nabla^{\dif A} \dif A, \dif B\rangle = 0, \\
% \dif^* B = 0.
% \end{cases}
% \end{equation}
% Suppose that $\dif A$ has no zeroes.
% Let $\partial_{x^1}, \dots, \partial_{x^{d - 1}}$ be an orthonormal frame for $\ker(\star \dif A)$. \todo{Construct $\partial_t$ in terms of $A$. Use a transform to make $A$ smooth}.












%%%%%%%%%%%%%%%%%%%%%%%%%%
\section{Open problems}\label{open problems}
\subsection{More general Riemannian manifolds}
For simplicity in this paper we have only dealt with closed manifolds of dimension $d \leq 7$.
However, it is likely that the results largely go through for compact manifolds with boundary, as our arguments are mainly local, and for those that are not one can pass to the double of the manifold with boundary, as long as the boundary is not itself a minimal hypersurface.
Moreover, one can most likely recover the results of this paper without the dimension assumption, as long as one is willing to deal with the consequence of having minimal hypersurfaces with singularities, since we mainly deal with phenomena in codimension $\leq 2$, while singularities of minimal hypersurfaces necessarily have codimension $\geq 8$. 
There are genuine differences in the case of manifolds with infinite ends, since if $M$ is a manifold with cusps, the stable (semi)norm of every homology class in $M$ may be zero.
Still, one expects to be able to recover most of the results of this paper.

\begin{problem}\label{generalization}
Formulate the results in this paper for complete Riemannian manifolds with boundary and infinite ends, and arbitrary dimension $d \geq 2$.
\end{problem}

%%%%%%%%%%%%%%%%%%%%

\subsection{\texorpdfstring{$p$-elliptic systems}{p-elliptic systems}}
We have carefully sidestepped the lack of a theory of $L^\infty$ variational systems by working with $L^p$ variational systems, $d < p < \infty$, which can be written in divergence form, and then taking a limit.
When we do have to work in the limit, we passed to the dual problem, which was a \emph{scalar} $L^1$ variational problem.
One would therefore like to work with a notion of weak solution for the Euler-Lagrange system (\ref{infty Max}).

\begin{problem}
Introduce a notion of weak solution for (\ref{infty Max}) which generalizes the notion of viscosity solution for the $\infty$-Laplacian.
Show that the following are equivalent for a $d - 1$-form $F$:
\begin{enumerate}
\item $F$ is a weak solution of the Euler-Lagrange equation (\ref{infty Max}).
\item $F$ is tight.
\item $F$ is closed and for every small ball $B \subset M$, the variational condition (\ref{ABC inequality}) holds.
\end{enumerate}
\end{problem}

Another problem caused by the lack of the maximum principle is the apparent failure of tight forms to be unique.
This was already a problem in the work of Daskalopolous and Uhlenbeck on maps to $\Sph^1$ which inspired this work \cite[Conjecture 9.2]{daskalopoulos2020transverse}.

\begin{conjecture}
The tight $d - 1$-form in a cohomology class is unique.
\end{conjecture}

We should also point out that if $F_p$ is $p$-tight and we write $F_p = \dif A_p$, where the gauge potential $A_p$ satisfies $\dif^* A_p = 0$, then $A_p$ solves the $L^p$ analogue 
$$\dif^*(|\dif A_p|^{p - 2} \dif A_p) + \dif(|\dif^* A_p|^{p - 2} \dif^* A_p) = 0$$
of the Laplace-de Rham equation. 
There is a rough principle that those estimates on the Laplace equation which do not depend on the maximum principle (or the closely related comparison principle and Harnack inequality) should also hold for the Laplace-de Rham equation.
Therefore one expects that those estimates on the $p$-Laplacian which do not depend on the maximum principle should hold for the $p$-Maxwell equation.
We did not pursue this line of inquiry here, since it is highly unlikely that such estimates would remain valid in the limit $p \to \infty$, but we believe that they should be true, hence:

\begin{problem}
Formulate and prove estimates on the $p$-Maxwell equation analogous to those estimates on the $p$-Laplacian proved in, for example, \cite[Chapter 11]{kinnunen2021maximal}, which do not rely on the maximum principle.
\end{problem}

\todo{Make explicit the Caccioppoli inequality and anything else that's probably true here?}

\todo{Lattice MFMC Theorem: can we prove MFMC by reducing to lattices? Maybe don't state this as a conjecture, just prove it}


%%%%%%%%%%
\subsection{Laminations and foliations}\label{canonical conjectures}
Thurston's canonical lamination $\lambda$ is chain-recurrent, in the sense that traveling along the geodesics in $\lambda$ defines a chain-recurrent dynamical system.
This makes no sense for higher-dimensional laminations, but is equivalent to assert that Thurston's canonical lamination can be approximated by finite sums of closed geodesics \cite[\S9]{Gu_ritaud_2017}.
We conjecture that the analogous fact should hold for our canonical lamination:

\begin{conjecture}\label{chain recurrence}
Let $\rho \in H^{d - 1}(M, \RR)$, and let $\lambda_\rho$ be the canonical lamination.
Then it is possible to approximate $\lambda_\rho$ in Thurston's geometric topology\footnote{See \cite[\S1]{BackusCML} for the definition of Thurston's geometric topology in this setting.} by finite unions of closed minimal hypersurfaces.
\end{conjecture}

The following problem was suggested to me by Karen Uhlenbeck. 
There exist closed hyperbolic $3$-manifolds which admit taut foliations; in that case, \emph{after changing the metric} one may find a minimal foliation.
Thus one cannot rule out minimal foliations by a simple topological argument (as one could rule out geodesic foliations of closed hyperbolic surfaces).
However, if a minimal foliation exists, then it is natural to study the tight form which calibrates it.
This form satisfies a particularly strong form 
\begin{equation}\label{eikonal}
\begin{cases}\dif F = 0 \\ \dif(|F|^2) = 0\end{cases}
\end{equation}
of the Euler-Lagrange equation (\ref{infty Max}) which is analogous to the role of the eikonal equation
$$\dif(|\dif u|^2) = 0$$
in the study of the $\infty$-Laplace equation.
Global solutions of the eikonal equation are rather uncommon (for example, the Dirichlet problem for the eikonal equation on $\Ball^d$ is extremely overdetermined), so this suggests a means to rule out the existence of minimal foliations:

\begin{conjecture}\label{Karen}
Let $\Gamma$ be the fundamental group of a closed hyperbolic $3$-manifold $M$.
Then there does not exist a solution of the eikonal system (\ref{eikonal}) on $\Hyp^3$ which is invariant under $\Gamma$.
In particular, there does not exist a minimal foliation on $M$.
\end{conjecture}

%%%%%%%%%%%%%%%%%%%%%%%%
\subsection{String theory}
Let us highlight a possible application of this work to string theory.
Freedman and Headrick considered the holographic principle in the setting where a theory of quantum gravity lives on a Riemannian manifold $M$, and the dual conformal field theory lives on its boundary $\partial M$ \cite{Freedman_2016}.
We assume that if $\Gamma \subset \partial M$ bounds a domain in $\partial M$, then there exists an area-minimizing and minimal hypersurface in $M$ with boundary $\Gamma$.
In this setting, if $N$ is a smooth subdomain of $\partial M$, the entropy of entanglement $S(\partial N)$ of the CFT through $\partial N$ is given by the \dfn{Ryu-Takayanagi formula}
\begin{equation}\label{RyuTakayanagi}
S(\partial N) = \max_F \int_N F
\end{equation}
where $F$ ranges over calibrations on $\overline M$ \cite[(2.8)]{Freedman_2016}.

If the results of this paper generalize cleanly to manifolds with boundary (see Problem \ref{generalization}) then, by replacing $N$ with an area-minimizing hypersurface in $M$ with the same boundary, we should have:

\begin{conjecture}
There exists a tight calibration $F$ which realizes the maximum in (\ref{RyuTakayanagi}).
\end{conjecture}

On the other hand, we know by recent work of Loisel that it is possible to solve the $\infty$-Laplacian in polynomial time \cite{Loisel_2020}.
We expect the same technique to work for (\ref{infty Max}), and so we expect the following problem to be doable.

\begin{problem}
Given a hypersurface $N \subset \overline M$ which is homologous relative to $\partial N$ to an area-minimizing hypersurface, give an efficient algorithm to numerically find a tight calibration $F$ which realizes the maximum in (\ref{RyuTakayanagi}), and in particular to compute $S(\partial N)$ if $N \subset \partial M$.
\end{problem}


%%%%%%%5


\appendix
\section{Local Hodge theory in \texorpdfstring{$L^p$}{Lp}} \label{local Hodge appendix}
The geometric measure theory that we use throughout this paper relies on the local elliptic regularity of the Hodge system.
Such estimates are standard when $p = 2$ \cite[\S9.5]{taylor2010partial}, but we are mainly interested in the range $d < p < \infty$, where we may apply the Sobolev embedding $W^{1, p} \Subset C^\alpha$.
We thus give a short proof for every $p$.

\begin{lemma}\label{Hodge theorem}
Suppose that there is a bi-Lipschitz diffeomorphism $M \cong \Ball^d$.
Let $1 < p < \infty$, and let $F$ be an $L^p$ closed $\ell + 1$-form.
Then there exists an $\ell$-form $A$ such that $F = \dif A$ and
\begin{equation}\label{Hodge theorem estimate}
\|A\|_{W^{1, p}} \lesssim_p \|F\|_{L^p}.
\end{equation}
\end{lemma}
\begin{proof}
Without loss of generality, $M = (\Ball^d, g)$ where $g$ is a Riemannian metric of bounded geometry.
Then, possibly after rescaling, the $W^{s, p}(M)$ norm is comparable to the $W^{s, p}(\Ball^d)$ norm.
Everything else in the statement of the lemma is a diffeomorphism invariant, so we may assume that $M = \Ball^d$.
We then solve the elliptic system 
\begin{equation}\label{Hodge Laplacian}
\begin{cases}
	\Delta u = F \\
	u|_{\partial \Ball^d} = 0
\end{cases}
\end{equation}
for an $\ell$-form $u$.
Since $M = \Ball^d$, the Hodge Laplacian commutes with taking components, thus the system (\ref{Hodge Laplacian}) decouples into $\binom d\ell$ elliptic PDE
\begin{equation}\label{decoupled Hodge Laplacian}
\begin{cases}
\Delta(u_I) = F_I \\
u_I|_{\partial \Ball^d} = 0,
\end{cases}
\end{equation}
one for each $\ell$-index $I$.
By \cite[Chapter 3, Theorem 6.3]{chen1998second}, we have a unique solution to (\ref{decoupled Hodge Laplacian}), which satisfies
$$\|u_I\|_{W^{2, p}} \lesssim_p \|F_I\|_{L^p}.$$
We then let $A := \dif^* u$; then (\ref{Hodge theorem estimate}) holds, and $\dif A = \dif \dif^* u$.
On euclidean space, $\Delta$ and $\dif$ commute, so $\dif u = 0$, and it holds that 
\begin{align*}
\dif A &= (\dif \dif^* + \dif^* \dif) u = \Delta u = F. \qedhere 
\end{align*}
\end{proof}

\begin{lemma}\label{mollification of closed forms}
Suppose that there is a bi-Lipschitz diffeomorphism $M \cong \Ball^d$.
Let $1 \leq p < \infty$ and let $F$ be an $L^p$ closed $\ell$-form.
Then there exist smooth closed $\ell$-forms $F_n$ such that $F_n \to F$ in $L^p$.
Moreover, if $F \in L^\infty$, then
\begin{equation}\label{heat kernel contracts sup norm}
\limsup_{n \to \infty} \|F_n\|_{C^0} \leq \|F\|_{L^\infty}.
\end{equation}
If $U$ is an open set such that $\supp F \subseteq U$, then for $n$ large enough, $\supp F_n \subseteq U$.
\end{lemma}
\begin{proof}
Without loss of generality, $M = (\Ball^d, g)$ where $g$ is a Riemannian metric of bounded geometry.
Then, possibly after rescaling, the $L^p(M)$ norm is comparable to the $L^p(\Ball^d)$ norm.
If $\varphi$ is am $\ell$-form and $\chi$ is a function on $\RR^d$, we define the convolution $\chi * \varphi$ by $(\chi * \varphi)_I := \chi * \varphi_I$ for every $\ell$-index $I$.
Every convolution operator commutes with $\dif$, so after convolution with a standard mollifier $\chi_n$ as in \cite[Appendix C, Theorem 6]{evans2010partial}, $F_n := (\chi_n * F)|_{\Ball^d}$ is a closed $\ell$-form and $F_n \to F$ in $L^p(\Ball^d)$ and almost everywhere.
Convolution against a standard mollifier cannot increase the support by more than a small neighborhood, so the support property follows.

Now suppose that $F \in L^\infty$.
We may choose $\chi_n$ so that $\chi_n \geq 0$, $\int_{\RR^d} \chi_n = 1$, and $\supp \chi_n \Subset B_{1/n}$, the euclidean ball of radius $1/n$ around $0$.
Then 
$$|F_n(x)|_{g(x)} \leq \left|\int_{B_{1/n}} \chi_n(y) F(x - y) \dif y\right|_{g(x)},$$
where the integral is a vector-valued integral (which makes sense since we may view $F$ as a map into $\RR^{\binom d\ell})$.
By the triangle inequality and a Taylor expansion of $g$ around $g(x)$,
\begin{align*}
\left|\int_{B_{1/n}} \chi_n(y) F(x - y) \dif y\right|_{g(x)}
&\leq \int_{B_{1/n}} \chi_n(y) |F(x - y)|_{g(x)} \dif y \\
&= \int_{B_{1/n}} \chi_n(y) |F(x - y)|_{g(x - y)}(1 + O(y)) \dif y \\
&\leq \|F\|_{L^\infty(M)} \int_{B_{1/n}} \chi_n(y)(1 + O(y)) \dif y \\
&= \|F\|_{L^\infty(M)}(1 + O(n^{-1})). \qedhere 
\end{align*}
\end{proof}

%%%%%%%%%%%%%%%%%
\section{Singularities of nodal sets}\label{nodal appendix}
Suppose that $v$ solves an elliptic PDE.
Then we write $Z(v), Z^{\rm sing}(v)$ for the nodal and singular sets of $v$, namely the sets of zeroes and double zeroes, respectively.
We will show that the generic point of $Z(v)$ is not a singular point.

It is well-known that the complement of a submanifold $P$ of codimension $\geq 2$ is path-connected; this follows from Alexander duality for singular cohomology.
However, we shall need this fact when $P = Z^{\rm sing}(v)$, which is not in general a manifold, but the proof still works as long as we apply Alexander duality for sheaf cohomology.
Let $\hat H^\bullet(P, \RR)$ denote the cohomology of the constant sheaf $\RR$ on $P$.

\begin{lemma}\label{closed mfld complement}
Let $P \subset \Sph^{d - 1}$ be a closed $d - 3$-rectifiable set.
Then $\Sph^{d - 1} \setminus P$ is path-connected.
\end{lemma}
\begin{proof}
Let $\delta^{\rm Haus}, \delta^{\rm cov}, \delta^{\rm shf}$ be the Hausdorff, covering, and sheaf cohomological dimensions of $P$ respectively.
Then by \cite[{\S}II.5.12]{godement1973topologie} and \cite[Theorem 6.3.10]{edgar2008measure}, we have 
$$\delta^{\rm shf} \leq \delta^{\rm cov} \leq \delta^{\rm Haus} \leq d - 3,$$
hence $\hat H^{d - 2}(P, \RR) = 0$.
By Alexander duality for sheaf cohomology \cite[Theorem 6]{Kaplan47}, it follows that $H_0(\Sph^{d - 1} \setminus P, \RR) = 0$, or in other words $\Sph^{d - 1} \setminus P$ is path-connected.
\end{proof}

\begin{lemma}\label{open mfld complement}
Let $P \subset \Ball^{d - 1}$ be a closed $d - 3$-rectifiable set.
Then $\Ball^{d - 1} \setminus P$ is path-connected.
\end{lemma}
\begin{proof}
Embed $\Ball^{d - 1}$ in $\Sph^{d - 1}$ using the one-point compactification, let $\infty$ be the point at infinity, and let $x, y \in \Ball^{d - 1} \setminus P$.
Choose a $d - 3$-sphere $S$ in $\Sph^{d - 1}$ which contains $\infty$ but does not contain $x, y$.
Then $P \cup S$ is a closed $d - 3$-rectifiable set and $x, y \notin P \cup S$, so by Lemma \ref{closed mfld complement}, there exists a curve $\gamma$ from $x$ to $y$ which avoids $P \cup S$.
Therefore $\gamma \subset \Ball^{d - 1} \setminus P$.
\end{proof}

\begin{lemma}\label{nodal set is generically smooth}
Let $Q$ be a linear elliptic operator on $\Ball^{d - 1}$ satisfying the maximum principle.
Suppose that $Qv = 0$ and $v$ has a zero of finite order.
Then the Hausdorff dimensions of the nodal and singular sets of $v$ are
\begin{align}
	\dim(Z(v)) &= d - 2, \label{nodal dimension}\\
	\dim(Z^{\rm sing}(v)) &\leq d - 3. \label{singular nodal dimension}
\end{align}
\end{lemma}
\begin{proof}
By \cite[Lemma 1.9]{Hardt89}, $Z^{\rm sing}(v)$ is $d - 3$-rectifiable, which implies (\ref{singular nodal dimension}).
If there exists $x \in Z(v) \setminus Z^{\rm sing}(v)$, then by the implicit function theorem, there is a neighborhood $U \ni x$ such that $U \cap Z(v)$ is a $d - 2$-dimensional manifold.
So if (\ref{nodal dimension}) fails, we must have $Z(v) = Z^{\rm sing}(v)$, so $Z(v)$ is $d - 3$-rectifiable.
But then, by Lemma \ref{open mfld complement}, the sets $U_\pm := \{\pm v > 0\}$ satisfy $U_+ \cup U_-$ are connected.
Since $v$ is continuous, one of these sets must be empty; without loss of generality, $U_- = \emptyset$.
Then $v \geq 0$ and $v$ has a zero, so by the maximum principle, $v = 0$ identically.
This contradicts the fact that $v$ has a zero of finite order.
\end{proof}


\printbibliography

\end{document}
