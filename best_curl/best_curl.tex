\documentclass[reqno,11pt]{amsart}
\usepackage[letterpaper, margin=1in]{geometry}
\RequirePackage{amsmath,amssymb,amsthm,graphicx,mathrsfs,url,slashed,subcaption}
\RequirePackage[usenames,dvipsnames]{xcolor}
\RequirePackage[colorlinks=true,linkcolor=Red,citecolor=Green]{hyperref}
\RequirePackage{amsxtra}
\usepackage{cancel}
\usepackage{tikz-cd}

% \setlength{\textheight}{9.3in} \setlength{\oddsidemargin}{-0.25in}
% \setlength{\evensidemargin}{-0.25in} \setlength{\textwidth}{7in}
% \setlength{\topmargin}{-0.25in} \setlength{\headheight}{0.18in}
% \setlength{\marginparwidth}{1.0in}
% \setlength{\abovedisplayskip}{0.2in}
% \setlength{\belowdisplayskip}{0.2in}
% \setlength{\parskip}{0.05in}
%\renewcommand{\baselinestretch}{1.05}

\title{Comass minimizers}
\author{Aidan Backus}
\date{\today}

\newcommand{\NN}{\mathbf{N}}
\newcommand{\ZZ}{\mathbf{Z}}
\newcommand{\QQ}{\mathbf{Q}}
\newcommand{\RR}{\mathbf{R}}
\newcommand{\CC}{\mathbf{C}}
\newcommand{\DD}{\mathbf{D}}
\newcommand{\PP}{\mathbf P}
\newcommand{\MM}{\mathbf M}
\newcommand{\II}{\mathbf I}
\newcommand{\Hyp}{\mathbf H}
\newcommand{\Sph}{\mathbf S}
\newcommand{\Group}{\mathbf G}
\newcommand{\GL}{\mathbf{GL}}
\newcommand{\Orth}{\mathbf{O}}
\newcommand{\SpOrth}{\mathbf{SO}}
\newcommand{\Ball}{\mathbf{B}}

\newcommand*\dif{\mathop{}\!\mathrm{d}}

\DeclareMathOperator{\card}{card}
\DeclareMathOperator{\dist}{dist}
\DeclareMathOperator{\supp}{supp}
\DeclareMathOperator{\tr}{tr}

\newcommand{\Leaves}{\mathscr L}
\newcommand{\Lagrange}{\mathcal L}
\newcommand{\Hypspace}{\mathscr H}

\newcommand{\Chain}{\underline C}

\newcommand{\Two}{\mathrm{I\!I}}

\newcommand{\normal}{\mathbf n}
\newcommand{\radial}{\mathbf r}
\newcommand{\evect}{\mathbf e}
\newcommand{\vol}{\mathrm{vol}}

\newcommand{\diam}{\mathrm{diam}}
\newcommand{\Ell}{\mathrm{Ell}}
\newcommand{\inj}{\mathrm{inj}}
\newcommand{\Lip}{\mathrm{Lip}}
\newcommand{\Riem}{\mathrm{Riem}}

\newcommand{\Min}{\mathrm{Min}}
\newcommand{\Max}{\mathrm{Max}}

\newcommand{\dfn}[1]{\emph{#1}\index{#1}}

\renewcommand{\Re}{\operatorname{Re}}
\renewcommand{\Im}{\operatorname{Im}}

\newcommand{\loc}{\mathrm{loc}}
\newcommand{\cpt}{\mathrm{cpt}}

\def\Japan#1{\left \langle #1 \right \rangle}

\newtheorem{theorem}{Theorem}[section]
\newtheorem{badtheorem}[theorem]{``Theorem"}
\newtheorem{prop}[theorem]{Proposition}
\newtheorem{lemma}[theorem]{Lemma}
\newtheorem{sublemma}[theorem]{Sublemma}
\newtheorem{proposition}[theorem]{Proposition}
\newtheorem{corollary}[theorem]{Corollary}
\newtheorem{conjecture}[theorem]{Conjecture}
\newtheorem{axiom}[theorem]{Axiom}
\newtheorem{assumption}[theorem]{Assumption}

\newtheorem{mainthm}{Theorem}
\renewcommand{\themainthm}{\Alph{mainthm}}

\newtheorem{claim}{Claim}[theorem]
\renewcommand{\theclaim}{\thetheorem\Alph{claim}}
% \newtheorem*{claim}{Claim}

\theoremstyle{definition}
\newtheorem{definition}[theorem]{Definition}
\newtheorem{remark}[theorem]{Remark}
\newtheorem{example}[theorem]{Example}
\newtheorem{notation}[theorem]{Notation}

\newtheorem{exercise}[theorem]{Discussion topic}
\newtheorem{homework}[theorem]{Homework}
\newtheorem{problem}[theorem]{Problem}

\makeatletter
\newcommand{\proofpart}[2]{%
  \par
  \addvspace{\medskipamount}%
  \noindent\emph{Part #1: #2.}
}
\makeatother



\numberwithin{equation}{section}


% Mean
\def\Xint#1{\mathchoice
{\XXint\displaystyle\textstyle{#1}}%
{\XXint\textstyle\scriptstyle{#1}}%
{\XXint\scriptstyle\scriptscriptstyle{#1}}%
{\XXint\scriptscriptstyle\scriptscriptstyle{#1}}%
\!\int}
\def\XXint#1#2#3{{\setbox0=\hbox{$#1{#2#3}{\int}$ }
\vcenter{\hbox{$#2#3$ }}\kern-.6\wd0}}
\def\ddashint{\Xint=}
\def\dashint{\Xint-}

\usepackage[backend=bibtex,style=alphabetic,giveninits=true]{biblatex}
\renewcommand*{\bibfont}{\normalfont\footnotesize}
\addbibresource{best_curl.bib}
\renewbibmacro{in:}{}
\DeclareFieldFormat{pages}{#1}

\newcommand\todo[1]{\textcolor{red}{TODO: #1}}


\begin{document}
\begin{abstract}
	Minimizers of the comass
\end{abstract}

\maketitle

%%%%%%%%%%%%%%%%%%%%%%%%%%%%%%%%%%%%%%%%%%%%%%%%%%%%%%%
\section{Introduction}
In this paper we study the problem of minimization of the \dfn{comass}
\begin{equation}\label{comass}
L(F) := \sup_{\sigma \in \Chain_{d - 1}(M)} \frac{1}{|\sigma|} \int_\sigma F
\end{equation}
of a closed $d-1$-form in a closed Riemannian $d$-fold $M$, subject to a constraint on the cohomology of $F$.
Here $\Chain_{d - 1}(M)$ denotes the space of oriented $d - 1$-chains in $M$, and $|\sigma|$ is the $d-1$-area of $\sigma$.

Many of our results shall be local, and so one could just as well view much of this paper as concerned with the Dirichlet problem for minimizers of (\ref{comass}).
To be more concrete, one could view the results of this paper as concerned with properties with vector fields $X$ on a domain in $\RR^3$ which minimize $\|\nabla \times X\|_{L^\infty}$ subject to a Dirichlet condition.
However, because of the topological applications that we will elaborate upon below, we will stick to the more abstract formulation of the problem at hand, namely that of forms of best comass:

\begin{definition}
A closed $d-1$-form $F$ has \dfn{best comass} if it minimizes its comass (\ref{comass}) in its cohomology class.
\end{definition}

Forms of best comass may be viewed as generalizations of (derivatives of) best Lipschitz maps from a surface to $\Sph^1$, which were recently studied by Daskalopolous--Uhlenbeck \cite{daskalopoulos2020transverse}.
Daskalopolous--Uhlenbeck were motivated by Thurston's asymmetric metric on Teichm\"uller space \cite{Thurston98}, which we now recall.

\subsection{Review of Thurston's Teichm\"uller theory}
Given a closed oriented surface $S$, we view $S$ as the underlying topological space of two hyperbolic manifolds $M, N$.
Thurston considers \dfn{best Lipschitz} maps $f: M \to N$ -- that is, maps $f: M \to N$ which are homotopic to the identiy on $S$ and which minimize the Lipschitz constant in their homotopy class. 
The best Lipschitz constant is denoted $L$, and \dfn{Thurston's asymmetric metric} is the Finsler metric $\log L$ on the Teichm\"uller space of hyperbolic manifolds with underlying topological space $S$.
There is another characterization of Thurston's asymmetric metric, which originally motivated Thurston's work on Teichm\"uller theory \cite[Theorem 8.5]{Thurston98}:

\begin{theorem}[Thurston]
Let $M, N$ be closed hyperbolic surfaces, with underlying topological space $S$.
Given a nontrivial homotopy class of loops $\gamma \in \pi_1(S) \setminus \{0\}$, let $\gamma_M$ be a geodesic representative of $\gamma$ in $M$.
Let
$$K := \sup_{\gamma \in \pi_1(S) \setminus \{0\}} \frac{|\gamma_N|}{|\gamma_M|}.$$
Then $L = K$.
\end{theorem}

Thurston's proof of $L = K$ is purely one of geometric topology.
However, solutions of the $\infty$-Laplace equation 
$$\nabla^2_{\nabla v, \nabla v} v = 0$$
are best Lipschitz \cite{Crandall2008}, so $L$ can be viewed as concerned with the $\infty$-Laplacian.
Dually, $K$ can be viewed as a maximization over measured geodesic laminations (rather than geodesics), which are naturally identified with solutions of the $1$-Laplace equation \todo{cite me}
$$\dif^* \left(\frac{\dif u}{|\dif u|}\right) = 0.$$
Thus it is natural to think that there should be a PDE proof of $L = K$.
Thurston himself conjectured this \cite[Abstract]{Thurston98}:

\begin{quote}
I currently think that a characterization of minimal stretch\footnote{that is, best Lipschitz} maps should be possible in a considerably more general context ... and it should be feasible with a simpler proof based on more general principles -- in particular, the max flow min cut principle, convexity, and $L^0 \leftrightarrow L^\infty$\footnote{Here, $L^0$ denotes the space of Radon measures. However, the natural norm to put on $L^0$ is the total variation norm, which scales like $L^1$; this is why $L^0$ can be viewed as ``dual'' in some sense to $L^\infty$.} duality.
\end{quote}

Inspired by Thurston's conjecture, Daskalopolous--Uhlenbeck used PDE methods to prove a version of $L = K$ for maps from a surface to $\Sph^1$ \cite{daskalopoulos2020transverse} and then generalized these methods to prove $L = K$ for maps between surfaces \cite{daskalopoulos2022,daskalopoulos2023}.

%%%%%%%%%%%%%%

\subsection{Main results}
We seek to apply the methods of Daskalopolous--Uhlenbeck's proof of $L = K$ \cite{daskalopoulos2020transverse, daskalopoulos2022, daskalopoulos2023} to prove a version of $L = K$ for forms of best comass.
We will especially stress the role of the ``max flow min cut principle'' alluded to by Thurston.

\subsubsection{Existence of $\infty$-light forms}
In \cite{daskalopoulos2020transverse}, Daskalopolous--Uhlenbeck view $\infty$-harmonic functions as a special kind of best Lipschitz functions.
So we begin by constructing a privileged class of forms of best comass, called ``$\infty$-light forms'', which are analogous to $\infty$-harmonic functions.

Fix a closed Riemannian manifold $M$ of dimension $d$, let $d < p < \infty$, and $\frac{1}{p} + \frac{1}{q} = 1$.
We call a closed $2$-form $F$ \dfn{$p$-light} if it minimizes the $L^p$ norm, among all cohomologous forms.
We call such forms ``light'' to emphasize that they are a minimizing sequence for the comass (thus ``not heavy" or ``not massive'').
Given a $p$-light form, the $\pi_1(M)$-equivariant function $u$ on the universal cover such that
$$\dif u = (-1)^{d - 1} |F|^{p - 2} \star F$$
is $q$-harmonic -- in other words, $u$ is a solution of the $q$-Laplace equation 
$$\dif^*(|\dif u|^{q - 2} \dif u) = 0.$$

A $\pi_1(M)$-equivariant function $u \in BV_\loc(\tilde M)$ on the universal cover is \dfn{$1$-harmonic} if it is a minimizer of its total variation 
$$\int_M \star |\dif u| := \sup_{\|\varphi\|_{C^0} \leq 1} \int_M \dif u \wedge \varphi$$
among all perturbations of $u$ by a $\pi_1(M)$-invariant function.
Equivalently, $u$ is $1$-harmonic if 
$$\dif^* \left(\frac{\dif u}{|\dif u|}\right) = 0$$
in the very weak sense that there exists an $L^\infty$ divergence-free vector field $X$ such that $\|X\|_{L^\infty} \leq 1$ and
$$|\dif u| = (\dif u, X)$$
in the sense of Radon measures \cite{Mazon14}.

\begin{mainthm}\label{existence of infinity light forms}
Let $\rho \in H^{d - 1}(M, \RR)$ be a cohomology class.
Let $(F_p, u_q)$ be the family of dual pairs of $p$-light forms and $q$-harmonic functions, suitably normalized, with $[F_p] = \rho$.
Then there exists a pair $(F, u)$ such that as $p \to \infty$, $F_p \to F$ weakly in $L^r$ for any $d < r < \infty$, and $u_q \to u$ weakly in $BV$.
Moreover, $F$ has best comass in $\rho$, $u$ is $1$-harmonic, and we have the duality relation 
\begin{equation}\label{max flow mean cut}
L|\dif u| = \langle \dif u, \star F\rangle
\end{equation}
in the sense of Radon measures, where $L$ is the best comass constant.
\end{mainthm}

This is a combination of Propositions \ref{existence infinity} and \ref{existence 1}, and allows us to take limits as $p \to \infty$, $q \to 1$.

\begin{definition}
We call the best comass form $F$ in Theorem \ref{existence of infinity light forms} an \dfn{$\infty$-light form}.
\end{definition}

Since (\ref{max flow mean cut}) asserts a form of convex duality between $\infty$-light forms and $1$-harmonic functions, we view it as the analogue of the max flow min cut principle alluded to by Thurston.
We elaborate on this point in Appendix \ref{Max Flow Min Cut}.

Most of the technical work in this paper are preliminaries for the proof of Theorem \ref{existence of infinity light forms}.
To be precise, we have two technical preliminaries:
\begin{enumerate}
\item We need to identify the comass with $\|F\|_{L^\infty}$, just as one needs to identify the Lipschitz constant with the $L^\infty$ norm of the derivative in \cite{Crandall2008,daskalopoulos2020transverse}. For continuous $F$, this is clear; in general, we have to reduce to the continuous case using the Hardy-Littlewood maximal inequality.
\item It is easy to deduce an analogue of (\ref{max flow mean cut}) for $q$-harmonic functions where $q > 1$, but it is not so easy to take the limit as $q \to 1$. To justify taking the limit, we avoid working directly with $q$-harmonic functions, and instead work with solutions of the heat equation whose Cauchy data is $q$-harmonic.
\end{enumerate}
We should stress that we do not want to work with the $L^\infty$ norm $\|F\|_{L^\infty}$ directly.
The reason is that we are interested at the points where $F$ attains its comass, but $\{|F| = \|F\|_{L^\infty}\}$ is just a measurable set, which may even be Lebesgue null.
Thus on the level of measure theory, $\{|F| = \|F\|_{L^\infty}\}$ is indistinguishable from the empty set!
However, as in the case of Lipschitz functions \cite{Crandall2008,daskalopoulos2020transverse}, one can define the set of points where $F$ attains its comass to be a certain closed set which is invariant under modification of $F$ on a Lebesgue null set.

%%%%%%%%%%%%%%%%%%

\subsubsection{Forms of best comass calibrate Thurston laminations}
A surface $N \subseteq M$ is \dfn{calibrated} by a closed $2$-form $F$ if $\|F\|_{L^\infty} \leq 1$ and the pullback of $F$ to $N$ is the area form on $N$ \cite{Harvey82}.
In that case, the mean curvature of $N$ is 
\begin{equation}\label{calibrated surfaces are minimal}
H_N = \nabla \cdot \normal_N = \nabla \cdot (\star F)^\sharp = \star \dif F = 0,
\end{equation}
so that $N$ is minimal. 
In particular, if a lamination $\lambda$ is calibrated by $F$ (in the sense that every leaf of $\lambda$ is calibrated), then $\lambda$ is minimal.

If $F$ calibrates a surface $N$, then it is clear from the definitions that $N$ is contained in the locus on which $F$ attains its comass.
If $F$ is a continuous form of best comass, and $L$ is the best comass constant, then $F/L$ calibrates a minimal lamination \cite{bangert_cui_2017}.
On the other hand, if $M$ is a space form of dimension $2 \leq d \leq 4$, then every $1$-harmonic function $u$ defines a minimal lamination $\lambda$ with Ruelle-Sullivan current $\dif u$ \todo{Cite me}.
The lamination $\lambda$ induced by the $1$-harmonic conjugate of an $\infty$-light form $F$ plays the role of Thurston's canonical lamination \cite{Thurston98}, so we call it a \dfn{Thurston lamination} associated to $[F]$.

Our next theorem is similar to \cite[Theorem 5.1]{bangert_cui_2017}.
However, we do not require that $F$ is continuous, and we identify the calibrated lamination $\lambda$ as a familiar object -- it is Thurston's canonical lamination!

\begin{mainthm}
Suppose that $M$ is a closed space form of dimension $2 \leq d \leq 4$.
Let $\lambda$ be a Thurston lamination associated to $\rho \in H^{d - 1}(M, \RR)$, and let $F$ be a form of best comass representing $\rho$.
Then $F/L$ calibrates $\lambda$. 
\end{mainthm}

This is Theorem \ref{MCL contains Thurston}.
Since the \dfn{best comass locus} where $F$ attains its comass is a well-defined closed set, and in \todo{cite me} we already worked out the details of the proof that $\lambda$ is well-defined, all that is left to do is to show that $\lambda$ is contained in the best comass locus.
\todo{Explain why this is easier than in \cite{daskalopoulos2020transverse}}

%%%%%%%%%%%%%%%%
\subsubsection{\texorpdfstring{$L = K$}{L equals K}}
Let $\lambda$ be a measured oriented lamination.
Then $\lambda$ defines a closed $d-1$-current -- the \dfn{Ruelle-Sullivan current} $T_\lambda$ -- by 
$$\int_M T_\lambda \wedge \varphi := \int_{\mathscr L_\lambda} \int_N \varphi \dif \mu_\lambda(N).$$
Here $\mathscr L_\lambda$ is the space of leaves of $\lambda$, and $\mu_\lambda$ is the transverse measure on the space of leaves.
Since $\lambda$ defines a closed $d-1$-current, it has a homology class $[\lambda] \in H_{d - 1}(M, \RR)$.
It also has an \dfn{area} $|\lambda|$, which is the total mass of $T_\lambda$:
$$|\lambda| = \sup_{\|\varphi\|_{C^0} \leq 1} \int_M T_\lambda \wedge \varphi.$$

The next theorem is Theorem \ref{L equals K}, and is an analogue of the $L = K$ theorem of Thurston.
It is essentially immediate from the above results and the duality statement (\ref{max flow mean cut}).

\begin{mainthm}\label{L is K}
Suppose that $M$ is a closed space form of dimension $2 \leq d \leq 4$.
Let $\rho \in H^{d - 1}(M, \RR)$, let $L$ be the best comass constant of $\rho$, and let $K$ be the supremum of $\langle \rho, [\lambda]\rangle/|\lambda|$, taken over all measured oriented laminations $\lambda$.
Then $L = K$, and the supremum in $K$ is attained by any Thurston lamination for $\rho$.
\end{mainthm}


%%%%%%%%%%%%%%%%%%%%%%%%
\subsection{A gauge-theoretic interpretation}
In order to understand the analogy between best comass $2$-forms on threefolds and best Lipschitz functions on surfaces, it is convenient to introduce the formalism of gauge theory.
We shall not need this in the sequel, but include it for the reader's interest and intuition.

We begin with the $\Sph^1$-valued $L = K$ theorem of Daskalopolous--Uhlenbeck \cite[Theorem 5.8]{daskalopoulos2020transverse}.
To state it, recall that if $\rho \in H^1(M, \ZZ)$, then $\rho$ defines a morphism of groups
$$\rho: \pi_1(M) \to \ZZ = \pi_1(\Sph^1)$$
by the Hurcewiz theorem.
Since $\Sph^1$ is aspherical, it follows that $H^1(M, \ZZ)$ can be viewed as the space $[M, \Sph^1]$ of homotopy classes of maps $M \to \Sph^1$.

By \cite[\S2.1]{daskalopoulos2020transverse}, if $F$ is a best comass representative of $\rho$, then there exists a map $A: M \to \Sph^1$ such that $F = \dif A$ (and $A$ has homotopy class $\rho$).
Since $F$ has best comass, $A$ is best Lipschitz.
There is a gauge ambiguity caused by the symmetry of $\Sph^1$: \cite{daskalopoulos2020transverse} treats two maps $A, A'$ as the same if $A - A'$ is a constant angle.
Of especial interest among the best Lipschitz maps are the $\infty$-harmonic maps $M \to \Sph^1$, which are obtained by taking the limit of $p$-harmonic functions as $p \to \infty$.

\begin{theorem}[Daskalopolous--Uhlenbeck]
Let $M$ be a closed hyperbolic surface and $\rho \in H^1(M, \ZZ)$.
Let $L$ be the best Lipschitz constant of the homotopy class $\rho$.
Identify $\gamma \in \pi_1(M) \setminus \{0\}$ with its geodesic representative.
Let
$$K := \sup_{\gamma \in \pi_1(M) \setminus \{0\}} \frac{\langle \rho, \gamma\rangle}{|\gamma|}.$$
Then $L = K$.
\end{theorem}

Since geodesics are dense in the space of measured oriented geodesic laminations, we can replace the supremum in the definition of $K$ with a supremum over measured oriented geodesic laminations; then, since the supremum over measured oriented laminations is clearly attained by a \emph{geodesic} such lamination, we can replace $K$ with 
$$K = \sup_\lambda \frac{\langle \rho, [\lambda]\rangle}{|\lambda|}$$
where $\lambda$ ranges over measured oriented (but possibly nongeodesic) laminations.
Thus Theorem \ref{L is K} is a generalization of \cite[Theorem 5.8]{daskalopoulos2020transverse}, where we take $d = 2$ and $\rho$ an integral class.

If $d = 3$ or $d = 4$, it is not so clear how to identify $H^{d - 1}(M, \ZZ)$ with a homotopy class of maps.
However, for $d = 3$ it is natural to view $H^2(M, \ZZ)$ as the space of line bundles on $M$, by identifying a line bundle $\mathscr L_\rho$ with its Chern class $\rho = c_1(\mathscr L)$.
As such, we fix an integral class $\rho \in H^2(M, \ZZ)$, and let $F$ be a best comass representative of $\rho$.
Then there locally exists a $1$-form $A$ such that $\dif A = F$.
As in \cite{daskalopoulos2020transverse}, there is a gauge ambiguity, and we must identify two $1$-forms if they differ by a closed $1$-form.

If $F$ is $\infty$-light, then we can find approximations $A_p$ which solve the \dfn{$p$-Maxwell equation}
$$\dif^* (|\dif A_p|^{p - 2} \dif A_p) = 0.$$
The $1$-forms $A_p$ and their limit $A$ can be viewed as global objects, namely (the Christoffel symbols of globally defined) connections on $\mathscr L_\rho$.
We could call $A$ a \dfn{best curvature} connection, since it minimizes the $L^\infty$ norm of the curvature among all connections on $\mathscr L$.

\todo{Do we want to write down the equivariance relations? They're the same as for a connection, except that it's not true that the differences in the transition functions are integers.}

Since the best comass problem can be viewed as the best curvature problem on a line bundle, it is natural to make the following conjecture, which we shall not address in this paper.

\begin{conjecture}
For any compact gauge group $\mathbf G$, and $\mathbf G$-bundle $\mathscr E$, there exists a minimizer of the $L^\infty$ norm of the curvature among all connections on $\mathscr E$.
Moreover, one can find a minimizer by taking suitable weak limits of solutions of the \dfn{$p$-Yang-Mills equation}
$$\dif_A^* (|\dif A|^{p - 2} \dif A) = 0$$
where $\dif_A$ is the covariant derivative whose Christoffel symbols are given by $A$. 
\end{conjecture}



%%%%%%%%%%%%%%%%%%%%%
\subsection{Analytic ideas} \todo{Do this later}

%%%%%%%%%%%%%%%%%%%%%
\subsection{Outline of the paper} \todo{Do this later}

%%%%%%%%%%%%%%%%%%%%%
\subsection{Notation and preliminaries}
We write $\Omega^\ell$, $Z^\ell$, and $B^\ell$ for the spaces of $\ell$-forms, closed $\ell$-forms ($\ell$-cocycles), and exact $\ell$-forms ($\ell$-coboundaries) respectively.
We reserve $H^\ell$ for cohomology and write $W^{s, p}$ for Sobolev spaces. \todo{Do this later}

We say that a $d-2$-form $A$ is in \dfn{Coulomb gauge} if $\dif^* A = 0$.
If $F := \dif A$, then this is equivalent to the assertion that $\Delta A = \dif^* F = \nabla F$.
In particular, if $p > d$, we get from Sobolev embedding and the boundedness of the Riesz transform that
\begin{equation}\label{Sobolev}
	\|A\|_{C^0} \lesssim \|\nabla A\|_{L^p} \lesssim \|\nabla F\|_{W^{-1, p}} \lesssim \|F\|_{L^p}.
\end{equation}
 
%%%%%%%%%%%%%%%%%%%%%%
\subsection{Acknowledgements}
I would like to thank Georgios Daskalopolous and Karen Uhlenbeck for suggesting this project, providing helpful comments, and providing me with an early draft of the manuscript \cite{daskalopoulos2023} which was a major source of inspiration for this work.
I also would like to thank Tom Goodwillie, Kaya Ferendo, Tainara Borges, and Haram Ko for helpful discussions.

This research was supported by the National Science Foundation's Graduate Research Fellowship Program under Grant No. DGE-2040433.


% \subsection{Normal trace theorem}
% We follow \cite[Chapter I, Theorem 1.2]{temam2016navier}, which handles the $p = 2$ case of the normal trace theorem.

% \begin{proposition}[normal trace theorem]
% Let $\iota: \Sph^{d - 1} \to \Ball^d$ be the inclusion map.
% The pullback is a bounded linear operator 
% $$\iota^*: L^\infty(\Ball^d, Z^{d - 1}) \to L^\infty(\Sph^{d - 1}, Z^{d - 1}).$$
% In particular, for every $\alpha \in L^\infty(\Ball^d, Z^{d - 1})$ and every $f \in W^{1, 1}(\Ball^d)$, we have integration by parts:
% \begin{equation}\label{Stokes trace}
% 	\int_{\Sph^{d - 1}} f\alpha = \int_{\Ball^d} \dif f \wedge \alpha.
% \end{equation}
% \end{proposition}
% \begin{proof}
% By the inverse trace theorem \cite[Teorema 1.II]{Gagliardo1957}, there exists a (possibly nonlinear and discontinuous) right inverse $T$ of the trace map $W^{1, 1}(\Ball^d) \to L^1(\Sph^{d - 1})$ which satisfies 
% $$\|Tf\|_{W^{1, 1}(\Ball^d)} \lesssim \|f\|_{L^1(\Sph^{d - 1})}.$$
% We use $T$ to formally define an $\ell$-current $\iota^* \alpha$ on $\Sph^{d - 1}$ by setting, for every $f \in C^\infty(\Sph^{d - 1})$,
% $$\int_{\Sph^{d - 1}} f\iota^* \alpha := \int_{\Ball^d} \dif(Tf) \wedge \alpha.$$
% To show that this current is well-defined, it suffices to show that for any $f$, 
% $$\left|\int_{\Sph^{d - 1}} f\iota^* \alpha\right| \lesssim_\alpha \|f\|_{L^1(\Sph^{d - 1})}.$$
% In fact, 
% \begin{align*}
% \left|\int_{\Sph^{d - 1}} f \iota^*\alpha\right|
% &= \left|\int_{\Ball^d} \dif(Tf) \wedge \alpha\right| \leq \|Tf\|_{W^{1, 1}(\Ball^d)} \|\alpha\|_{L^\infty(\Ball^d)} \\
% &\lesssim \|f\|_{L^1(\Sph^{d - 1})} \|\alpha\|_{L^\infty(\Ball^d)}.
% \end{align*}
% It follows that $\iota^* \alpha$ is well-defined.
% Moreover, by Stokes' theorem, that $\iota^* \alpha$ agrees with the usual definition of pullback on continuous forms $\alpha \in C^0(M, Z^{d - 1})$.
% Finally, we estimate 
% \begin{align*}
% \|\iota^* \alpha\|_{L^\infty(\Sph^{d - 1})}
% &= \sup_{\|f\|_{L^1(\Sph^{d - 1})} = 1} \int_{\Sph^{d - 1}} f \iota^*\alpha \\
% &\lesssim \sup_{\|f\|_{L^1(\Sph^{d - 1})} = 1} \|f\|_{L^1(\Sph^{d - 1})} \|\alpha\|_{L^\infty(\Ball^d)} \\
% &= \|\alpha\|_{L^\infty(\Ball^d)}. \qedhere 
% \end{align*}
% \end{proof}

% \begin{corollary}\label{trace on cycles}
% Let $N$ be a smooth embedded hypersurface in $M$ and $1 < p < \infty$.
% \begin{enumerate}
% \item \label{pullback bounded} The pullback is a bounded linear operator
% $$\iota^*_N: L^\infty(M, Z^{d - 1}) \to L^\infty(N, Z^{d - 1}).$$
% \item \label{integral continuous} For $F \in L^\infty(M, Z^{d - 1})$, and $N$ closed, the integral $\int_N F$ is well-defined, and depends continuously on $F$ for the weak topology on $L^p$.
% \item \label{cohomology exists} For $F \in L^\infty(M, Z^{d - 1})$, the cohomology class of $F$ is well-defined, and depends continuously on $F$ for the weak topology on $L^p$.
% \end{enumerate}
% \end{corollary}
% \begin{proof}
% To prove (\ref{pullback bounded}) we may use a partition of unity to work in a small ball $U$ in $N$, and then we may realize a collar neighborhood $V$ of $U$ in $M$ as a manifold-with-boundary with $U \subseteq \partial V$, and choose $\chi \in C^\infty_\cpt(M)$ in (\ref{Stokes trace}) to be a cutoff which is zero on $\partial V$ except along $U$.
% The definition of
% $$\iota^*_U: L^\infty(M, Z^{d - 1}) \to L^\infty(U, \Omega^\ell)$$
% does not depend on the choice of $V$, since by Stokes' theorem the right-hand side of (\ref{Stokes trace}) will be the same.

% To obtain (\ref{integral continuous}), we first use the fact that $N$ is closed to obtain $L^\infty(N, Z^{d - 1}) \subseteq L^1(N, Z^{d - 1})$.
% To obtain the continuity, we again use a collar neighborhood $V$ of $U$ and a cutoff $\chi$.
% Then (\ref{Stokes trace}) reads 
% $$\int_U F = \int_M F \wedge \dif \chi.$$
% Since $\chi \in C^\infty_\cpt$, $\dif \chi \in L^q$ where $\frac{1}{p} + \frac{1}{q} = 1$, so if $F_n \to F$ in the weak topology on $L^p$, then $\int_M F_n \wedge \dif \chi \to \int_M F \wedge \dif \chi$, as desired.
% Letting $N$ range over representatives of every homology class, we conclude (\ref{cohomology exists}) as a consequence of (\ref{integral continuous}).
% \end{proof}


%%%%%%%%%%%%%%%%%%%%%%%%%%%%%%%%%%%%%%%%%%

\section{Comass}
In this section we study the basic properties of the comass.
We show that it is well-defined, equals the $L^\infty$ norm, and is attained on a nonempty closed set.
Throughout, we fix a closed Riemannian manifold $M$ of dimension $d$ and metric $g$.

\begin{definition}
For a closed $d-1$-form $F$ and a subdomain $\Omega \subseteq M$, the \dfn{comass} of $F$ in $\Omega$ is
$$L_\Omega(F) := \sup_{\sigma \in \Chain_{d - 1}(\Omega)} \frac{1}{|\sigma|} \int_\sigma F.$$
We let $L(F) := L_M(F)$.
\end{definition}

The comass $L_\Omega(F)$ is well-defined if $F$ is continuous, or has a trace along every $d - 1$-simplex.
We use the following proposition to take traces of $F$ whenever $F \in L^\infty$:

\begin{proposition}\label{integration is welldefined}
Let $\Sigma$ be the standard $d$-simplex, $\tau$ a $d-1$-face of $\Sigma$, and $d < p < \infty$.
Then $F \mapsto \int_\tau F$ extends to a continuous linear functional on $L^p(\Sigma, Z^{d - 1})$.
Moreover,
\begin{equation}\label{integral over chain is linfinity}
	\frac{1}{|\tau|} \int_\tau F \leq \|F\|_{L^\infty}.
\end{equation}
\end{proposition}
\begin{proof}
We first show that $F \mapsto \int_\tau F$ is continuous for the $L^p$ norm on smooth closed $d-1$-forms.
Let $F, F' \in C^\infty(\Sigma, Z^{d - 1})$.
Since $H^{d - 1}(\Sigma, \RR) = 0$, there exist $A, A'$ in Coulomb gauge such that $\dif A = F$ and $\dif A' = F'$, and then by Stokes' theorem and (\ref{Sobolev}),
$$\left|\int_\tau F - F'\right| = \left|\int_{\partial \tau} A - A'\right| \lesssim \|A - A'\|_{C^0} \lesssim \|F - F'\|_{L^p}.$$
By a mollification argument, $F \mapsto \int_\tau F$ is a continuous linear functional on $L^p(\Delta, Z^{d - 1})$.

Suppose now that $F \in L^\infty$ and choose a sequence of mollifiers $\chi_\varepsilon$ such that $\|\chi_\varepsilon\|_{L^1} = 1$.
Let $F_\varepsilon := F * \chi_\varepsilon$, so that $F_\varepsilon \to F$ in $L^p$ for any $d < p < \infty$.
By Young's inequality, 
$$\|F_\varepsilon\|_{C^0} \leq \|F\|_{L^\infty} \|\chi_\varepsilon\|_{L^1} \leq \|F\|_{L^\infty}.$$
Taking $\varepsilon \to 0$ we see that
\begin{align*}
\frac{1}{|\tau|} \int_\tau F 
&= \lim_{\varepsilon \to 0} \frac{1}{|\tau|} \int_\tau F_\varepsilon \leq \lim_{\varepsilon \to 0} \|F_\varepsilon\|_{C^0} \leq \|F\|_{L^\infty}. \qedhere
\end{align*}
\end{proof}

In particular, since any $d-1$-chain $\sigma$ can be written as the sums of $d-1$-simplices $\tau$ which can then be viewed as faces of contractible $d$-simplices $\Sigma$, the integral in the definition of comass is well-defined as long as $F \in L^p$, $p > d$, and each such integral depends continuously on $F$ in $L^p$.

\begin{definition}
	Let $\rho \in H^{d - 1}(M, \RR)$ and let $F$ be a closed $d-1$-form representing $\rho$.
\begin{enumerate}
	\item We say that $F$ has \dfn{best comass} if $F$ is a minimizer of $L$ among all $F$ representing $\rho$.
	\item We say that $F$ has \dfn{absolutely best comass} if it has best comass, and for every small open ball $\Omega$, $F$ is a minimizer of $L_\Omega$ among all closed $2$-forms $B$ with $B|_{\partial \Omega} = F|_{\partial \Omega}$.
\end{enumerate}
\end{definition}

We will be interested in the points at which $F$ attains its comass.
One could pose this problem as the problem of computing the locus $\{|F| = \|F\|_{L^\infty}\}$.
However, $F$ is both only defined almost everywhere, and not norm-approximable by smooth functions.
So as a proxy for $|F|$, which may fail to be defined on a null set, we use the local comass, which is defined everywhere.

\begin{definition}
The \dfn{local comass} of a closed $d - 1$-form $F$ at $x \in M$ is 
$$L(F, x) := \limsup_{\varepsilon \to 0} \sup_{\sigma \in \Chain_{d - 1}(B_\varepsilon(x))} \frac{1}{|\sigma|} \int_\sigma F.$$
\end{definition}

One has
$$L(F, x) = \limsup_{\varepsilon \to 0} L_{B_\varepsilon(x)}(F)$$
but $L_{B_\varepsilon(x)}(F)$ is increasing in $\varepsilon$ (since it's a supremum over a set which grows in $\varepsilon$).
So the limit superior is actually a limit and an infimum:
$$L(F, x) = \lim_{\varepsilon \to 0} L_{B_\varepsilon(x)}(F) = \inf_{\varepsilon > 0} L_{B_\varepsilon(x)}(F)$$
and in particular $L(F, x) \leq L(F)$. Thus we have the following analogue of \cite[Lemma 4.3(a)]{Crandall2008}:

\begin{proposition}
For $F \in L^\infty(M, Z^{d - 1})$, the local comass $L(F, \cdot)$ is upper semicontinuous. \label{crandall usc}
\end{proposition}
\begin{proof}
Let $x^n \to x$ and $r > 0$. Then eventually $x^n \in B_r(x)$, hence $L(F, x^n) \leq L_{B_r(x)}(F)$ and so
\begin{align*}
\limsup_{n \to \infty} L(F, x^n) &\leq \inf_{r > 0} L_{B_r(x)}(F) = L(F, x). \qedhere 
\end{align*}
\end{proof}

It's convenient to test the local comass against a more restrictive class of $d- 1$-chains.
For a $d-1$-plane $S \subseteq T_x M$ and $\varepsilon > 0$, let
$$P_{S, \varepsilon} := (\exp_x)_*(S) \cap B_\varepsilon(x).$$
We refer to $P_{S, \varepsilon}$ as a \dfn{diskette} at scale $\varepsilon$.\footnote{The terminology serves to remind ourselves to how we want to test the $d - 1$-form $F$ against small disks, analogous to how in lattice gauge theory one tests $2$-forms against small squares (or plaques) called plaquettes \cite{Gupta98}.}

The utility of diskettes is that, for a covector $\xi \in T_x' M$, we obtain a foliation of the tangent space $T_x M$ into planes, namely $(\ker \xi + t\xi^\sharp)_{t \in \RR}$.
Pushing forward this foliation by the exponential map, we obtain a foliation
$$\lambda_{\xi, \varepsilon} := (P_{\ker \xi + t\xi^\sharp, \varepsilon})_{t \in \RR}$$
of $B_\varepsilon(x)$ into diskettes which are approximately conormal to $\xi$ in the limit $\varepsilon \to 0$.
Using this foliation and Fubini's theorem, we can turn integrals of $F$ over $d-1$-surfaces, which are Lebesgue null, into integrals over balls, which satisfy the Lebesgue differentiation theorem.
From these considerations and the Hardy-Littlewood maximal inequality, we deduce that diskettes satisfy their own version of the Lebesgue differentiation theorem, at least on average.

\begin{proposition}[diskette Lebesgue differentiation theorem]\label{PLDT}
Let $F \in L^\infty(M, Z^{d - 1})$.
Then for almost every $x \in M$ there exists a unit covector $\xi \in T_x' M$, which depends measurably on $x$, such that
\begin{equation}\label{diskette LDT}
|F(x)| = \lim_{\varepsilon \to 0} \frac{1}{|B_\varepsilon(x)|} \int_{-\infty}^\infty \left[\int_{P_{\ker \xi + t\xi^\sharp, \varepsilon}} F\right] \dif t
\end{equation}
where $\dif t$ is the disintegration of the Riemannian measure with respect to $\lambda_{\xi, \varepsilon}$.
\end{proposition}

To set up the proof, we define for any closed $d - 1$-form $G$, covector $\eta \in T'_x M$, and $\varepsilon > 0$,
$$A(G, \eta, \varepsilon) := \frac{1}{|B_\varepsilon(x)|} \int_{-\infty}^\infty \left[\int_{P_{\ker \eta + t\eta^\sharp, \varepsilon}} G\right] \dif t.$$
Also let 
$$\Pi_\eta: T' B_\varepsilon(x)^{\wedge (d - 1)} \to T' \lambda_{\eta, \varepsilon}^{\wedge (d - 1)}$$
be the projection from the bundle of antisymmetric $d - 1$-tensors on $B_\varepsilon(x)$ to those which are cotangent to $\lambda_{\eta, \varepsilon}$.
Thus, if $G$ and $P := P_{\ker \eta + t\eta^\sharp, \varepsilon}$ are cooriented, then
\begin{equation}\label{cooriented integral is area integral of norm}
	\int_P G = \int_P |\Pi_\eta G| \dif A
\end{equation}
where $\dif A$ is the area element on the leaves of the foliation.

\begin{lemma}\label{PLDT nonzero}
Proposition \ref{PLDT} holds on almost all of $\{F = 0\}$ for any measurable section $\xi$ of the cosphere bundle of $M$.
\end{lemma}
\begin{proof}
We bound using (\ref{cooriented integral is area integral of norm}) and Fubini's theorem
$$|A(F, \xi(x), \varepsilon)| \leq \frac{1}{|B_\varepsilon(x)|} \int_{-\infty}^\infty \int_{P_{\ker \xi + t\xi^\sharp, \varepsilon}} |F| \dif A \dif t = \dashint_{B_\varepsilon(x)} \star |F|.$$
Almost every $x$ such that $F(x) = 0$ is a Lebesgue point of $|F|$, and for such $x$,
$$0 \leq \limsup_{\varepsilon \to 0} |A(F, \xi(x), \varepsilon)| \leq \lim_{\varepsilon \to 0} \dashint_{B_\varepsilon(x)} \star |F| = |F(x)| = 0.$$
This implies (\ref{diskette LDT}) on almost all of $\{F \neq 0\}$.
\end{proof}

\begin{lemma}\label{PLDT continuous}
Proposition \ref{PLDT} holds if $F$ is continuous and everywhere nonzero, and $\xi := \star F/F$.
\end{lemma}
\begin{proof}
For $\varepsilon > 0$, $\xi$ is conormal to $P_{\ker \xi, \varepsilon}$, and hence $\lambda_{\xi, \varepsilon}$, at $x$.
Thus, by definition of $\xi$, $\Pi_\xi F(x) = F(x)$, so by continuity of $F$,
\begin{equation}\label{continuous forms are almost their projections}
	\|\Pi_\xi F - F\|_{C^0(B_\varepsilon(x))} \ll 1
\end{equation}
as $\varepsilon \to 0$.
By (\ref{cooriented integral is area integral of norm}), the fact that the Hodge star coorients $F$ and $\xi$, and Fubini's theorem,
\begin{align*}
A(F, \xi, \varepsilon)
&= \frac{1}{|B_\varepsilon(x)|} \int_{-\infty}^\infty \int_{P_{\ker \xi + t\xi^\sharp, \varepsilon}} |\Pi_\xi F| \dif A \dif t
= \dashint_{B_\varepsilon(x)} \star |\Pi_\xi F|.
\end{align*}
By (\ref{continuous forms are almost their projections}) and the fact that $|F|$ is continuous (so $x$ is a Lebesgue point of $|F|$),
\begin{align*}
\dashint_{B_\varepsilon(x)} \star |\Pi_\xi F|
&= \dashint_{B_\varepsilon(x)} \star |F| + o(1) = |F(x)| + o(1).
\end{align*}
Taking the limit $\varepsilon \to 0$, we deduce (\ref{diskette LDT}) for $x$.
\end{proof}

\begin{lemma}\label{PLDT bound mass}
Suppose that $F \neq 0$ almost everywhere.
Let $\xi := \star F/|F|$, and let $G$ be a continuous, nowhere zero, closed $d - 1$-form.
For every $\alpha > 0$,
$$\left|\left\{\liminf_{\varepsilon \to 0} |A(F, \xi, \varepsilon) - |F|| > 4\alpha\right\} \cap \{F \neq 0\}\right| \lesssim \frac{\|F - G\|_{L^1}}{\alpha}.$$
\end{lemma}
\begin{proof}
Let $\eta := \star G/|G|$.
We split up
\begin{align*}
|A(F, \xi(x), \varepsilon) - |F(x)||
&\leq |A(F, \xi(x), \varepsilon) - A(G, \xi(x), \varepsilon)| + |A(G, \xi(x), \varepsilon) - A(G, \eta(x), \varepsilon)| \\
&\qquad + |A(G, \eta(x), \varepsilon) - |G(x)|| + ||F(x)| - |G(x)||\\
&=: I_1 + I_2 + I_3 + I_4.
\end{align*}

To treat $I_1$, we apply (\ref{cooriented integral is area integral of norm}):
\begin{align*}
I_1
&= \left|\frac{1}{|B_\varepsilon(x)|} \int_{-\infty}^\infty \int_{P_{\ker \xi + t\xi^\sharp, \varepsilon}} F - G \dif t\right| 
\leq \frac{1}{|B_\varepsilon(x)|} \int_{-\infty}^\infty \int_{P_{\ker \xi + t\xi^\sharp, \varepsilon}} |\Pi_\xi(F - G)| \dif A \dif t.
\end{align*}
By Fubini's theorem, 
\begin{align*}
\frac{1}{|B_\varepsilon(x)|} \int_{-\infty}^\infty \int_{P_{\ker \xi + t\xi^\sharp, \varepsilon}} |\Pi_\xi(F - G)| \dif A \dif t
&= \dashint_{B_\varepsilon(x)} \star|\Pi_\xi(F - G)| \leq \dashint_{B_\varepsilon(x)} \star |F - G|.
\end{align*}
If we let $\mathcal M$ denote the Hardy-Littlewood maximal operator, then we have just proven $I_1 \leq \mathcal M|F - G|(x)$.
So by the Hardy-Littlewood maximal inequality, for any $\varepsilon > 0$,
$$|\{I_1 > \alpha\}| \leq |\{\mathcal M |F - G| > \alpha\}| \lesssim \frac{\|F - G\|_{L^1}}{\alpha}.$$

Next, to bound $I_2$, we pull back everything into the tangent space at $x$.
We put tildes over balls or Hodge stars to indicate that they are taken with respect to the euclidean metric on $T_x M$.
After accepting a correction of size $O(\varepsilon)$ arising from the metric on $M$, 
$$I_2 = \frac{1}{\varepsilon^d |\Ball^d|} \left|\int_{-\varepsilon}^\varepsilon \left[\int_{\ker \xi + t\xi^\sharp \cap \tilde B_\varepsilon} - \int_{\ker \eta + t\eta^\sharp \cap \tilde B_\varepsilon}\right] G \dif t\right| + O(\varepsilon)$$
which we can rewrite as
$$I_2 = \left|\dashint_{\tilde B_\varepsilon(0)} \tilde \star [(G, \xi(x)^\sharp) - (G, \eta(x)^\sharp)]\right| + O(\varepsilon) \leq \dashint_{\tilde B_\varepsilon(0)} \tilde \star |G| \cdot |\xi(x) - \eta(x)| + O(\varepsilon).$$
Since $G$ is continuous, on $\tilde B_\varepsilon(0)$, $|G| = |G(x)| + o(1)$ as $\varepsilon \to 0$.
Therefore by the reverse triangle inequality,
\begin{align*}
I_2
&\leq |G(x)| |\xi(x) - \eta(x)| + o(1) \\
&= |G(x)| \left|\frac{G(x)}{|G(x)|} - \frac{F(x)}{|F(x)|}\right| + o(1) \\
&= \left|G(x) - \frac{|G(x)|}{|F(x)|} F(x)\right| + o(1) \\
&= |G(x) - F(x)| + \left|1 - \frac{|G(x)|}{|F(x)|}\right| |F(x)| + o(1)\\
&= |G(x) - F(x)| + ||F(x)| - |G(x)|| + o(1) \\
&\leq 2|G(x) - F(x)| + o(1).
\end{align*}
By Markov's inequality, we conclude that 
$$\left|\left\{\liminf_{\varepsilon \to 0} I_2 > \alpha\right\}\right| \leq \left|\left\{|F - G| > \frac{\alpha}{2}\right\}\right| \leq \frac{2\|F - G\|_{L^1}}{\alpha}.$$

By Lemma \ref{PLDT continuous}, $I_3 = o(1)$, or in other words,
$$\left|\left\{\liminf_{\varepsilon \to 0} I_3 > \alpha\right\}\right| = 0.$$
We bound $I_4$ for any $\varepsilon > 0$ using the reverse triangle inequality and Markov's inequality as 
$$|\{I_4 > \alpha\}| \leq |\{|F(x) - G(x)| > \alpha\}| \leq \frac{\|F - G\|_{L^1}}{\alpha}.$$
Adding up $I_1, \dots, I_4$, we conclude that 
\begin{align*}
\left|\left\{\liminf_{\varepsilon \to 0} |A(F, \xi, \varepsilon) - |F|| > 4\alpha\right\}\right| \leq \sum_{i=1}^4 \left|\left\{\liminf_{\varepsilon \to 0} I_i > \alpha\right\}\right| &\lesssim \frac{\|F - G\|_{L^1}}{\alpha}. \qedhere 
\end{align*}
\end{proof}

\begin{proof}[Proof of Proposition \ref{PLDT}]
We write $M = \{F = 0\} \sqcup \{F \neq 0\}$.
By Lemma \ref{PLDT nonzero}, the claim holds on almost all of $\{F = 0\}$, so we may assume that $F \neq 0$ almost everywhere.
In particular, $\xi := \star F/|F|$ is well-defined almost everywhere.
We choose a continuous, everywhere nonzero, closed $d - 1$-form $G$ so that $\|F - G\|_{L^1} \leq \delta \alpha^3$ for any $\delta > 0$, so that, after applying Lemma \ref{PLDT bound mass} with $\alpha = \frac{1}{m}$,
$$\left|\left\{\lim_{\varepsilon \to 0} |A(F, \xi, \varepsilon) - |F|| > 0\right\}\right| \leq \sum_{m=1}^\infty \left|\left\{\lim_{\varepsilon \to 0} |A(F, \xi, \varepsilon) - |F|| > \frac{1}{m}\right\}\right| \lesssim \sum_{m=1}^\infty \frac{\delta}{m^2} \lesssim \delta.$$
The claim follows when we take $\delta \to 0$.
\end{proof}

As a consequence of the diskette Lebesgue differentiation theorem, we obtain an analogue for the comass of a familiar result about the Lipschitz seminorm \cite[Lemma 4.3]{Crandall2008}.

\begin{proposition}\label{crandall}
Let $F \in L^\infty(M, Z^{d - 1})$. Then:
\begin{enumerate}
% \item If $F(x)$ exists then $L(F, x) \geq |\dif A(x)|$. \label{crandall dA bounds LA}
% \item If $L(A, x) = 0$, then $\dif A(x)$ exists and $\dif A(x) = 0$. \label{crandall zero LA implies diffble}
\item For almost every $x \in M$,
$$|F(x)| \leq L(F, x).$$
\label{crandall LDT}
\item The local comass is bounded, and \label{crandall linfinity}
$$L(F) = \max_{x \in M} L(F, x) = \|F\|_{L^\infty}.$$
% \item Let $\iota^* F$ be the pullback of $F$ to a $2$-chain $\sigma$. Then
% $$\|\iota^* F\|_{L^\infty(\sigma)} \leq \|F\|_{L^\infty}.$$
% \label{crandall normal trace is contraction}
% \item If $\sigma \in \Chain_2$ then \label{crandall best curl is ABC}
% $$\frac{1}{|\sigma|} \int_\sigma F \leq \max_{x \in \sigma} L(F, x).$$
\end{enumerate}
\end{proposition}
\begin{proof}
% We now bound for a net of diskettes $R_{ij}^\varepsilon(x)$ using (\ref{riemann diskette})
% \begin{equation}\label{difference quotients}
% 	\limsup_{\varepsilon \to 0} \frac{|A_j(x + \varepsilon \partial_i) - A_j(x) - A_i(x + \varepsilon \partial_j) + A_i(x)|}{\varepsilon}
% = \limsup_{\varepsilon \to 0} \frac{1}{|R_{ij}^\varepsilon(x)|} \left|\int_{\partial R_{ij}^\varepsilon(x)} A\right| \leq L(A, x).
% \end{equation}
% If the first limit superior is actually a limit, then it is the definition of $|\dif A_{ij}(x)|$.
% So if $\dif A_{ij}(x)$ exists we conclude $|\dif A_{ij}(x)| \leq L(A, x)$, which proves (\ref{crandall dA bounds LA}).
% On the other hand, the corresponding limit \emph{inferior} must be nonnegative, so if $L(A, x) = 0$, the first limit superior in (\ref{difference quotients}) is actually a limit and we obtain $\dif A_{ij}(x) = 0$, proving (\ref{crandall zero LA implies diffble}).

% Now let $\sigma$ be a $2$-chain and $\varepsilon > 0$.
% Then we may write $\sigma = \sum_{n=1}^N \sigma_n(\varepsilon)$ where $\sigma_n(\varepsilon)$ is a $2$-simplex such that $\sigma_n(\varepsilon) \in \Chain_2(B_\varepsilon(x_n(\varepsilon)))$ for some $x_n(\varepsilon) \in \sigma$, and the $\sigma_n(\varepsilon)$ are almost disjoint.
% Then 
% $$\frac{1}{|\sigma|} \left|\int_\sigma F\right| \leq \sum_{n=1}^{N_\varepsilon} \frac{|\sigma_n(\varepsilon)|}{|\sigma|} \frac{1}{|\sigma_n(\varepsilon)|} \left|\int_{\sigma_n(\varepsilon)} F\right| \leq \sum_{n=1}^{N_\varepsilon} \frac{|\sigma_n(\varepsilon)|}{|\sigma|} L_{B_{\varepsilon}(x_n(\varepsilon))}(F).$$
% Since this inequality is true for every $\varepsilon$, it remains true in the limit superior as $\varepsilon \to 0$:
% $$\frac{1}{|\sigma|} \left|\int_\sigma F\right| \leq \limsup_{\varepsilon \to 0} \sum_{n=1}^{N_\varepsilon} \frac{|\sigma_n(\varepsilon)|}{|\sigma|} L_{B_{\varepsilon}(x_n(\varepsilon))}(F).$$
% Let $(y_m(\varepsilon))_{m \in \NN}$ be a maximizing sequence for $L_{B_\varepsilon(\cdot)}(F)$ subject to $y_m(\varepsilon) \in \sigma$.
% Such a sequence exists, since we trivially have
% $$\sup_{x \in \sigma} L_{B_\varepsilon(x)}(F) \leq \|F\|_{L^\infty} < \infty.$$
% After passing to a subsequence in $m$, we may assume that $y_m(\varepsilon) \to y(\varepsilon)$ for some $y(\varepsilon) \in \sigma$.
% Then for $m$ large depending on $\varepsilon$, $y_m(\varepsilon) \in B_{2\varepsilon}(y(\varepsilon))$, and it follows that
% \begin{align*}
% \limsup_{\varepsilon \to 0} \sum_{n=1}^{N_\varepsilon} \frac{|\sigma_n(\varepsilon)|}{|\sigma|} L_{B_{\varepsilon}(x_n(\varepsilon))}(F)
% &\leq \limsup_{\varepsilon \to 0} \sum_{n=1}^{N_\varepsilon} \frac{|\sigma_n(\varepsilon)|}{|\sigma|} \lim_{m \to \infty} L_{B_\varepsilon(y_m(\varepsilon))}(F) \\
% &\leq \limsup_{\varepsilon \to 0} \lim_{m \to \infty} L_{B_\varepsilon(y_m(\varepsilon))}(F) \\
% &\leq \limsup_{\varepsilon \to 0} L_{B_{2\varepsilon}(y(\varepsilon))}(F).
% \end{align*}
% We then pass to a subsequence in $\varepsilon$ along which the limit superior is obtained, and then a further subsequence along which $y(\varepsilon) \to y$ for some $y \in \sigma$.
% Let $\delta > 0$; then $B_{2\varepsilon}(y(\varepsilon)) \subseteq B_\delta(y)$ for any sufficiently small $\varepsilon$, hence 
% $$\limsup_{\varepsilon \to 0} L_{B_{2\varepsilon}(y(\varepsilon))}(F) \leq L_{B_\delta(y)}(F).$$
% Plugging everything in and taking a limit superior in $\delta$,
% $$\frac{1}{|\sigma|} \left|\int_\sigma F\right| \leq \limsup_{\delta \to 0} L_{B_\delta(y)}(F) = L(F, y),$$
% implying (\ref{crandall best curl is ABC}).

We bound for almost every $x \in M$ using the diskette Lebesgue differentiation theorem
\begin{align*}
|F(x)|
&= \lim_{\varepsilon \to 0} \frac{1}{|B_\varepsilon(x)|} \int_{-\infty}^\infty \int_{P_{\ker \xi + t\xi^\sharp, \varepsilon}} F \dif t \\
&\leq \lim_{\varepsilon \to 0} L_{B_\varepsilon(x)}(F) \frac{1}{|B_\varepsilon(x)|} \int_{-\infty}^\infty |P_{\ker \xi + t\xi^\sharp, \varepsilon}| \dif t.
\end{align*}
Since $\dif t$ is the disintegration of the Riemannian measure with respect to the foliation $\lambda_{\xi, \varepsilon}$ of $B_\varepsilon(x)$,  it follows that 
$$\int_{-\infty}^\infty |P_{\ker \xi + t\xi^\sharp, \varepsilon}| \dif t = |B_\varepsilon(x)|.$$
Therefore
$$|F(x)| \leq \lim_{\varepsilon \to 0} L_{B_\varepsilon(x)}(F) = L(F, x).$$
This proves (\ref{crandall LDT}). In particular, if we combine this estimate with (\ref{integral over chain is linfinity}), then
$$\sup_{x \in M} L(F, x) \leq L(F) \leq \|F\|_{L^\infty} \leq \sup_{x \in M} L(F, x).$$
The inequalities collapse; moreover, $L(F, \cdot)$ attains its supremum since it is upper semicontinuous.
This proves (\ref{crandall linfinity}).
\end{proof}

%%%%%%%%%%%%%%%%%%%%%%%%%%%%%
\section{Analysis of the \texorpdfstring{$q$-Laplacian as $q \to 1$}{q-Laplacian}}\label{qLaplace theory}
It is well-known that one can obtain a $1$-harmonic function $u$ by taking limits of $q$-harmonic functions $u_q$ as $q \to 1$ \cite{Mazon14}.
However, it does not immediately follow that $\|\dif u_q\|_{L^q}$ converges to the total variation of $u$, because it is not clear that $|\dif u_q|^q - |\dif u_q|$ is small in $L^1$ as $q \to 1$; essentially, this is because we know that $\dif u_q \in L^q$ but not that $\dif u_q$ lies in the Orcliz space $L^q \log L$.
In this section, we regularize the $q$-harmonic function $u_q$ using a truncated heat kernel, and use this to estimate the total variation of $u$ in terms of $\|\dif u_q\|_{L^q}$.

%%%%%%%%%%%%%%%%%%%%%%%%%%%%
\subsection{A truncated heat kernel}\label{truncated heat kernel}
Let
$$\tilde H_t(x) = (4\pi t)^{-\frac{d}{2}} \exp\left(-\frac{|x|^2}{4t}\right)$$
be the heat kernel on $\RR^d$.
Choose a smooth radial function $H_t$ on $\RR^d$ such that $H_t = \tilde H_t$ on $|x| \leq t^{1/4}$, $0 \leq H_t \leq \tilde H_t$ for $|x| \in [t^{1/4}, 2t^{1/4}]$, and $H_t = 0$ for $|x| \geq 2 t^{1/4}$.
For $f \in L^2(\RR^d)$, we let $f(t) := H_t * f$, so that $t \mapsto f(t)$ is an approximate solution of the heat equation.

We shall use the truncated heat kernel $H_t$ as a mollifier with two special properties:
\begin{enumerate}
\item The truncated heat equation preserves compact support: if $\supp f \subseteq B_\delta(0)$, then
\begin{equation}\label{support bounds on heat kernel}
\supp f(t) \subseteq B_{\delta + 2t^{1/4}}(0).
\end{equation}
\item The truncated heat equation satisfies similar $L^p$ estimates to the heat equation.
\end{enumerate}
It is very important that implied constants in this section do not depend on the parameter $q$, because we are eventually going to take $q \to 1$.

\begin{lemma}\label{approximation by heat kernels}
Uniformly in $0 < t \ll 1$,
$$\|H_t - \tilde H_t\|_{L^q} \lesssim t^{\frac{2 - d}{4q}} \exp\left(-\frac{1}{4 \sqrt t}\right).$$
\end{lemma}
\begin{proof}
From the definitions,
$$|H_t - \tilde H_t| \leq 1_{r^4 > t} \tilde H_t,$$
so we integrate 
\begin{align*}
\|H_t - \tilde H_t\|_{L^q}^q
&\leq \int_{\{|x|^4 > t\}} \tilde H_t(x)^q \dif x \sim t^{-\frac{d}{2}} \int_{\{|x|^4 > t\}} \exp\left(-\frac{q|x|^2}{4t}\right) \dif x.
\end{align*}
With the changes of variables $y = x \sqrt{\frac{q}{4t}}$ and $y = (r, \theta)$, and putting $a^2 = \frac{q}{4 \sqrt t}$, we obtain 
\begin{align*}
t^{-\frac{d}{2}} \int_{\{|x|^4 > t\}} \exp\left(-\frac{q|x|^2}{4t}\right) \dif x
&= \left(\frac{4}{q}\right)^{\frac{d}{2}} \int_{\{|y|^2 > \frac{q}{4\sqrt t}\}} e^{-|y|^2} \dif y \sim q^{-\frac{d}{2}} \int_a^\infty e^{-r^2} r^{d - 1} \dif r.
\end{align*}
To evaluate this last integral, let
$$\Gamma(w, z) := \int_z^\infty s^{w - 1} e^{-s} \dif s$$
be the upper incomplete $\Gamma$-function, so that \todo{cite me} $\Gamma(w, z) \sim z^{w - 1} e^{-z}$ as $z \to +\infty$.
With the change of variable $s = r^2$, we obtain 
$$\int_a^\infty e^{-r^2} r^{d - 1} \dif r = \frac{1}{2} \int_{a^2}^\infty e^{-s} s^{\frac{d}{2} - 1} \dif s = \frac{1}{2} \Gamma\left(\frac{d}{2}, \frac{q}{4 \sqrt t}\right) \sim \left(\frac{q}{\sqrt t}\right)^{\frac{d}{2} - 1} \exp\left(-\frac{q}{4 \sqrt t}\right).$$
In particular,
$$\|H_t - \tilde H_t\|_{L^q} \lesssim q^{-\frac{1}{q}} t^{\frac{2 - d}{4q}} \exp\left(-\frac{1}{4 \sqrt t}\right).$$
The claim follows from $q^{1/q} \sim 1$.
\end{proof}

\begin{lemma}
For $\frac{1}{p} + \frac{1}{q} = 1$, $q \in [1, \infty]$, and $0 < t \ll 1$,
\begin{equation}\label{Lq norm of truncated heat kernel}
	\|H_t\|_{L^q} \sim t^{-\frac{d}{2p}}.
\end{equation}
\end{lemma}
\begin{proof}
We have $t^{-\frac{d}{2p}} \gg t^{-\frac{2 - d}{4q}} e^{-1/4 \sqrt t}$, so by Lemma \ref{approximation by heat kernels}, it suffices to prove the same estimate for $\tilde H_t$. But this is standard:
\begin{align*}
\|\tilde H_t\|_{L^q} &\sim t^{-\frac{d}{2}} \left[\int_{\RR^d} e^{-\frac{q|x|^2}{4t}} \dif x\right]^{\frac{1}{q}} \sim \frac{t^{d/2q}}{q^{d/2q}t^{d/2}} \sim t^{\frac{d}{2q} - \frac{d}{2}} = t^{-\frac{d}{2p}}. \qedhere
\end{align*}
\end{proof}

\begin{lemma}
For $f \in L^1$,
\begin{equation}\label{heat flow is contraction}
	\|f(t)\|_{L^1} \leq \|f\|_{L^1}.
\end{equation}
\end{lemma}
\begin{proof}
We first bound 
$$|f(t, x)| = \left|\int_{\RR^d} H_t(x - y) f(y) \dif y\right| \leq \int_{\RR^d} H_t(x - y) |f(y)| \dif y.$$
So by Fubini's theorem and the fact that $\|H_t\|_{L^1} \leq \|\tilde H_t\|_{L^1} = 1$,
\begin{align*}
\int_{\RR^d} |f(t, x)| \dif x
&\leq \iint_{\RR^{d + d}} H_t(x - y) |f(y)| \dif y \dif x \\
&= \int_{\RR^d} |f(y)| \int_{\RR^d} H_t(x - y) \dif x \dif y \\
&= \int_{\RR^d} |f(y)| \dif y. \qedhere
\end{align*}
\end{proof}

We now cover $M$ by coordinate charts $B(x_\alpha, 2r_\alpha)$ in which $\sqrt{\det g} = 1$, and such that $(B(x_\alpha, r_\alpha))$ is an open cover of $M$.
Let $(\chi_\alpha)$ be a subordinate partition of unity to $(B(x_\alpha, r_\alpha))$.
Then for $f \in L^2(M)$, we let
$$f(t) = \sum_\alpha (\chi_\alpha f)(t).$$
Here the convolution in the definition of $(\chi_\alpha f)(t)$ is taken in the coordinates on $B(x_\alpha, r_\alpha)$.
If $t$ is smaller than some $t_0$ which only depends on $M$, then (\ref{support bounds on heat kernel}) implies that $(\chi_\alpha f)(t)$ is supported on $B(x_\alpha, 2r_\alpha)$, hence is a well-defined function of compact support on $M$.
In the sequel, we shall use without comment that the $L^1$ norm of a function can be computed in each coordinate patch and summing using the inclusion-exclusion principle, by hypothesis on $\sqrt{\det g}$.

%%%%%%%%%%%%%%%%%%%%%%%%%%%%

\subsection{Convergence of representations of \texorpdfstring{$\pi_1(M)$}{the fundamental group}}
Let $\tilde M \to M$ be the universal cover of $M$.
Following \cite[\S4]{daskalopoulos2020transverse}, we use the Hurcewiz theorem to identify $1$-dimensional representations
$$\alpha: \pi_1(M) \to \RR$$
of the fundamental group with cohomology classes $H^1(M, \RR)$.
If we have an $\alpha$-equivariant function $u$ on $\tilde M$, thus for every $\gamma \in \pi_1(M)$,
$$\gamma^* u = u + \langle \alpha, \gamma\rangle,$$
we write $[u] = \alpha$.
Taking derivatives, we see that $\gamma^* \dif u = \dif u$, so $\dif u$ drops to a closed $1$-form (or perhaps better a closed $d-1$-current) on $M$ whose cohomology class is $\alpha$.

\begin{lemma}\label{L1 convergence preserves pi1}
Let $\tilde M \to M$ be the universal cover, and let $(u_q)$ be a sequence of $\pi_1(M)$-equivariant functions on $\tilde M$ which converge in $L^1_\loc(\tilde M)$ to a function $u$ as $q \to 1$.
Then $u$ is $\pi_1(M)$-equivariant, and $[u_q] \to [u]$.
\end{lemma}
\begin{proof}
Since $u_q$ is $\pi_1(M)$-equivariant, there exists $\alpha_q \in H^1(M, \RR)$ such that for every $\gamma \in \pi_1(M)$,
\begin{equation}\label{equivariance q}
	\gamma^* u_q = u_q + \langle \alpha_q, \gamma\rangle.
\end{equation}
Let $M_{\rm fun}$ be a fundamental domain and $U_\gamma := M_{\rm fun} \cup \gamma_* (M_{\rm fun})$.

We claim that $(\alpha_q)$ has a convergent subsequence.
To see this, we first recall that $M$ has finite Betti numbers, so $H^1(M, \RR)$ is locally compact.
Therefore, if no convergent subsequence exists, there exists a $\gamma \in \pi_1(M)$ and a subsequence along which $\langle \alpha_q, \gamma\rangle \to \infty$.
Moreover, since $u_q \to u$ in $L^1_\loc$, $\|u_q\|_{L^1(M_{\rm fun})} \leq 2\|u\|_{L^1(M_{\rm fun})}$ if $q - 1$ is small enough.
But then 
$$\|u_q\|_{L^1(\gamma_* M_{\rm fun})} = \|\gamma^* u_q\|_{L^1(M_{\rm fun})} \geq \langle \alpha_q, \gamma\rangle - \|u_q\|_{L^1(M_{\rm fun})} \geq \langle \alpha_q, \gamma\rangle - 2\|u\|_{L^1(M_{\rm fun})}$$
and taking $q \to 1$ we conclude that $(u_q)$ is not compact in $L^1(\gamma_* M_{\rm fun})$, contradicting the convergence in $L^1_\loc(\tilde M)$.
So $\alpha_q \to \alpha$ for some $\alpha \in H^1(M, \RR)$.

For any $q > 1$,
\begin{align*}
\dashint_{M_{\rm fun}} \star |\gamma^* u - u - \langle \alpha, \gamma\rangle| 
&\leq \dashint_{M_{\rm fun}} \star (|\gamma^* u_q - u_q - \langle \alpha_q, \gamma\rangle| + |\gamma^* u_q - u_q| + |\gamma^* u - u|) \\
&\qquad + |\langle \alpha_q - \alpha, \gamma\rangle|.
\end{align*}
Taking $q \to 1$ and applying (\ref{equivariance q}), we conclude that $\|\gamma^* u - u - \langle \alpha, \gamma\rangle\|_{L^1} = 0$.
The claim follows when we recall $\alpha = [u]$, $\alpha_q = [u_q]$.
\end{proof}

%%%%%%%%%%%%%%%%%%%%%%
\subsection{Convergence of \texorpdfstring{$q$-harmonic functions}{q-harmonic functions}}
\begin{proposition}\label{qharmonics converge}
For each $1 < q < 2$, let $u_q$ be a $q$-harmonic, $\pi_1(M)$-equivariant function on $\tilde M$, so that
$$\|u_q\|_{W^{1, q}} \lesssim 1.$$
Then there exists a $\pi_1(M)$-equivariant function $u \in BV_\loc(M)$ such that, as $q \to 1$ along a subsequence, $u_q \to u$ weakly in $BV_\loc(\tilde M)$ and strongly in $L^r$ for $1 \leq r < \frac{d}{d - 1}$.
Moreover, $[u_q] \to [u]$ and
\begin{equation}\label{convergence of Lq norms to TV}
\lim_{q \to 1} \frac{1}{q} \int_M \star |\dif u_q|^q = \int_M \star |\dif u|.
\end{equation}
\end{proposition}
\begin{proof}
We first apply the assumption that $(u_q)$ is bounded in $W^{1, q}$.
By H\"older's inequality,
$$\lim_{q \to 1} \int_M \star |\dif u_q| \leq \lim_{q \to 1} |M|^{\frac{1}{p}} \int_M \star |\dif u_q|^q \lesssim 1.$$
So by Rellich's theorem, $(u_q)$ is weakly compact in $BV$ and strongly compact in $L^r$ for $1 \leq r < \frac{d}{d - 1}$.
In particular, $\dif u_q \to \dif u$ in the weak topology of measures and $u_q \to u$ in $BV$ and $L^r$.
As the limit of $\pi_1(M)$-equivariant functions, $u$ is also $\pi_1(M)$-equivariant by Lemma \ref{L1 convergence preserves pi1}.
In particular, $\dif u$ drops to a current on $M$.

It remains to show (\ref{convergence of Lq norms to TV}).
In one direction we use the portmanteau theorem and H\"older's inequality to control
\begin{align*}
\int_M \star |\dif u|& = \lim_{q \to 1} \int_M \int_M \star |\dif u_q| 
\leq \liminf_{q \to 1} |M|^{\frac{1}{p}} \|\dif u_q\|_{L^q} = \liminf_{q \to 1} \frac{1}{q} \int_M \star |\dif u_q|^q.
\end{align*}
Now let $u_q(t)$ be as \S\ref{truncated heat kernel}.
Recall that convolution operators and constant-coefficient differential operators commute, hence in each coordinate patch,
\begin{equation}\label{derivative solves the heat equation too}
(\dif u_q)(t) = \dif (u_q(t)).
\end{equation}

Since $u_q$ is $q$-harmonic, for any $0 < t < t_0$,
$$\int_M \star |\dif u_q|^q \leq \int_M \star |\dif u_q(t)|^q.$$
By the fundamental theorem of calculus, 
$$\int_M \star |\dif u_q(t)|^q = \int_M \star |\dif u_q(t)| + \int_M \int_1^q \frac{\partial}{\partial r} |\dif u_q(t, x)|^r \dif r \dif x.$$
We then have by (\ref{heat flow is contraction}) that
$$\int_M \star |\dif u_q(t)| \leq \int_M \star |\dif u_q|.$$
Recall that for $1 \leq r < 2$ and $t \geq 0$, $t^r \log t \leq t^2$.
We bound the integral in $r$ as
\begin{align*}
\int_1^q \frac{\partial}{\partial r} |\dif u_q(t, x)|^r \dif r
&= \int_1^q |\dif u_q(t, x)|^r \log |\dif u_q(t, x)|^r \dif r 
\leq (q - 1) |\dif u_q(t, x)|^2.
\end{align*}
Define $r, s \in [1, \infty]$ by 
$$\frac{1}{q} + \frac{1}{r} = \frac{3}{2}, \quad \frac{1}{r} + \frac{1}{s} = 1.$$
Then, since $q \geq 1$, $r \geq 2$, so $s \leq 2$.
By Young's inequality and (\ref{Lq norm of truncated heat kernel}),
$$\|\dif u_q(t)\|_{L^2} \leq \|H_t\|_{L^r} \|\dif u_q\|_{L^q} \sim t^{-\frac{d}{2s}} \|\dif u_q\|_{L^q} \leq t^{-d/4} \|\dif u_q\|_{L^q}.$$
We now set $t_q := (q - 1)^{1/d}$.
By the above considerations, there exist $C > 0$ and $1 < q_0 < 2$ which only depend on $M$ such that for any $1 < q \leq q_0$,
\begin{align*}
\int_M \star |\dif u_q|^q 
&\leq \int_M \star |\dif u_q(t_q)| + C(q - 1)t_q^{-d/2} \|\dif u_q\|_{L^q}^2 \\
&\leq \int_M \star |\dif u_q| + C\sqrt{q - 1} \|\dif u_q\|_{L^q}^2.
\end{align*}
Since $C$ is independent of $q$, we may take the limit $q \to 1$ and apply the portmanteau theorem to deduce that 
\begin{align*}
\limsup_{q \to 1} \frac{1}{q} \int_M \star |\dif u_q|^q &\leq \lim_{q \to 1} \int_M \star |\dif u_q| = \int_M \star |\dif u|. \qedhere 
\end{align*}
\end{proof}

%%%%%%%%%%%%%%%%%%%%%%%%%%%%%
\section{\texorpdfstring{$\infty$-light forms}{Infinity-light forms}}
\subsection{\texorpdfstring{$p$-light forms}{p-light forms}}
As before, let $M$ be a closed Riemannian manifold of dimension $d$, equipped with a cohomology class $\rho \in H^{d - 1}(M, \RR)$.
In this section, we construct a representative $F$ of best comass of $\rho$, and a $1$-harmonic conjugate to $F$.
We begin by constructing a minimizing sequence for the comass as $p \to \infty$.

\begin{definition}
Let $1 < p < \infty$ and let $F_p$ be a closed $d - 1$-form with $[F_p] = \rho$.
We call $F_p$ a \dfn{$p$-light form} if it is a minimizer of $\|F_p\|_{L^p}$ among all $d - 1$-forms representing $\rho$.
\end{definition}

Taking $p$th powers of the $L^p$ norm, we see that a closed $d - 1$-form is $p$-light iff it is a minimizer of
$$J_p(F) := \frac{1}{p} \int_M \star |F|^p.$$

\begin{proposition}
A $d - 1$-form $F \in L^p(M, \Omega^{d - 1})$ is $p$-light iff it solves the system
\begin{equation}\label{pMaxwell}
\begin{cases}
	\dif F = 0, \\
	\dif^*(|F|^{p - 2} F) = 0
\end{cases}
\end{equation}
in the distributional sense. Moreover, the restriction of $J_p$ to any cohomology class in $L^p(M, Z^{d - 1})$ is uniformly strictly convex.
\end{proposition}
\begin{proof}
Let us take variations of $J_p$ in the space $L^p(M, Z^{d - 1})$ of closed $d-1$-forms of finite $J_p$-energy.
Let $G$ be an exact $d - 1$-form (so $F + tG$ is cohomologous to $F$), so that
$$\frac{\dif}{\dif t} J_p(F + tG) = \frac{1}{p} \int_M \star \frac{\partial}{\partial t} |F + tG|^p = \int_M \star |F + tG|^{p - 2} \langle F + tG, G\rangle.$$
Setting $t = 0$, and writing $G = \dif B$, we obtain 
$$0 = \int_M \star |F|^{p - 2} \langle F, \dif B\rangle = \int_M \star \langle \dif^*(|F|^{p - 2} F), B\rangle.$$
This equation holds for every $d - 2$-form $B$.
Thus the Euler-Lagrange equations for $J_p$ are (\ref{pMaxwell}).
The second variation is 
$$\frac{\dif^2}{\dif t^2} J_p(F + tG)\bigg|_{t = 0} = (p - 2) \int_M \star |F|^{p - 4} \langle F, G\rangle^2 + \int_M \star |F|^{p - 2} |G|^2.$$
By the Cauchy-Schwarz inequality,
$$(p - 2) \int_M \star |F|^{p - 4} \langle F, G\rangle^2 + \int_M \star |F|^{p - 2} |G|^2 \geq (p - 1) \int_M \star |F|^{p - 2} |G|^2.$$
This implies uniform strict convexity for $p > 1$.
The restriction to a cohomology class remains strictly convex, since cohomology classes are closed affine subspaces of $L^p(M, Z^{d - 1})$ and hence are themselves closed and convex.
\end{proof}

Arguing as in the proof of existence of $p$-harmonic functions (see \cite[\S8.2]{evans2010partial} for the general theory, or \cite[\S2.1]{daskalopoulos2020transverse} for $p$-harmonic maps $M \to \Sph^1$ which is closely analogous to our case), one can easily check that there exists a unique $p$-light form in every cohomology class on the closed manifold $M$.
We omit the details, both because the argument is standard, and because we shall give a separate proof in Appendix \ref{Max Flow Min Cut}.

By \cite{Uhlenbeck77}, every $p$-light form is H\"older continuous. However, the continuity degenerates as $p \to \infty$, so we shall not be able to apply this fact in the sequel.

\begin{definition}
Let $F$ be a $p$-light form, let
\begin{equation}
\dif u := (-1)^{d - 1} |F|^{p - 2} \star F, \label{inverse extremality}
\end{equation}
and let $u$ be the primitive of $\dif u$ on the universal cover $\tilde M$, which is normalized to have zero mean on a fundamental domain $M_{\rm fun}$.
Then $u$ is called the \dfn{$q$-harmonic conjugate} of the $p$-light form $F$, where $\frac{1}{p} + \frac{1}{q} = 1$.
\end{definition}

Let $u$ be the $q$-harmonic conjugate of $F$.
By Poincar\'e's inequality,
$$\|u\|_{W^{1, q}(M_{\rm fun})}^q \lesssim \int_M \star |\dif u|^q = \int_M \star |F|^{(p - 1)q} = \int_M \star |F|^p < \infty$$
since $F$ is $p$-light; that is, we have $F \in L^p$ and $u \in W^{1, q}$.
From this, and H\"older's inequality, it follows that all expressions in the next proposition are well-defined.

\begin{proposition}
Let $1 < p, q < \infty$ and $\frac{1}{p} + \frac{1}{q} = 1$.
Let $F$ be a $p$-light form, and let $u$ be its $q$-harmonic conjugate. Then:
\begin{enumerate}
\item $u$ is $q$-harmonic.
\item One has 
\begin{equation}
F = |\dif u|^{q - 2} \star \dif u. \label{extremality} \\
\end{equation}
\item One has the \dfn{max flow min cut principle}
\begin{equation}\label{strong duality}
\frac{1}{q} \int_M \star |\dif u|^q + \frac{1}{p} \int_M \star |F|^p = \int_M \dif u \wedge F.
\end{equation}
\end{enumerate}
\end{proposition}

We give a more highbrow proof of the same proposition in Appendix \ref{Max Flow Min Cut}, which also justifies why (\ref{strong duality}) should be called the ``max flow min cut principle.''
However, we think there is value in providing a straightforward proof from first principles.

\begin{proof}
We first compute 
$$(q - 2)(p - 1) + p = 2,$$
so that
$$|\dif u|^{q - 2} \star \dif u = (-1)^{d - 1} |F|^{(q - 2)(p - 1)} \star \star |F|^{p - 2} F = |F|^{(q - 2)(p - 1) - (p - 2)} F = F.$$
Thus we have (\ref{extremality}), and moreover,
$$\dif \star (|\dif u|^{q - 2} \dif u) = \dif F = 0$$
so that $u$ is $q$-harmonic.
We also compute 
\begin{align*}
\int_M \star |F|^p &= \int_M \star |\dif u|^{(q - 1)p} = \int_M \star |\dif u|^q = \int_M \dif u \wedge |\dif u|^{q - 2} \star \dif u = \int_M \dif u \wedge F. \qedhere 
\end{align*}
\end{proof}


% We put any norm on $H^2(M, \RR)$ (as all are equivalent), hence $|\alpha|$ makes sense for $\alpha \in H^2(M, \RR)$.

% \begin{proposition}
% Assume $p > 2$ and $\rho \in H^2(M, \RR)$.
% Then:
% \begin{enumerate}
% \item There exists a unique $p$-light form $F_p$ whose cohomology class is $\rho$.
% \item One has
% \begin{equation}\label{Sobolev bounds for p}
% 	\|F_p\|_{L^p} \lesssim |\rho|.
% \end{equation}
% The constant is independent of $p$.
% \item $F_p$ is H\"older continuous.
% \end{enumerate}
% \end{proposition}
% \begin{proof}
% The existence and uniqueness follows from Proposition \ref{convex duality} and the fact that the Euler-Lagrange equations for $p$-light forms are given by (\ref{pMaxwell}).
% From (\ref{pMaxwell}), the fact that $p > 2$, and \cite{Uhlenbeck77}, we deduce that $F_p$ is H\"older continuous.

% Now select a basis $\xi_1, \dots, \xi_r$ of $H^2(M, \RR)$, and apply the above argument to obtain $p$-light forms $G_i$ representing $\xi_i$ for each $i \in \{1, \dots, r\}$.
% Then $\|G_i\|_{L^p}$ is bounded indepdendently of $p$ by (\ref{infinity magnetic rules p magnetic}).
% Decompose
% $$\alpha = \sum_{i=1}^r \alpha_i \xi_i.$$
% Using the fact that $F_p$ is $p$-light and cohomologous to $\sum_i \alpha_i G_i$,
% $$\|F_p\|_{L^p} \leq \left\|\sum_{i=1}^r \alpha_i G_i\right\|_{L^p} \leq \sum_{i=1}^r |\alpha_i| \|G_i\|_{L^p} \lesssim |\alpha|$$
% which proves (\ref{Sobolev bounds for p}).
% \end{proof}

% \todo{We do not get $C^\infty$ regularity on sets $\Subset \{|F_p| \neq 0\}$, or so it seems.}
% Why?
% Expanding out (\ref{pMaxwell}) with $F_{ij} = \partial_i A_j - \partial_j A_i$,
% $$0 = \partial^j(|F|^{p - 2} \partial_j A_i) - |\dif A|^{p - 2} \partial_i (\dif^* A) - |\dif A|^{p - 2} [\partial^j, \partial_i] A_j - \partial^j(|\dif A|^{p - 2}) \partial_i A_j.$$
% The first term is an elliptic operator with H\"older coefficients, the second can be gauged away, and the third is H\"older since $[\partial^j, \partial_i]$ is a connection coefficient of the metric.
% But the last term is bad.
% \todo{Maybe we can get rid of it using paraproducts?}

%%%%%%%%%%%%%%%%%%%%%%%
\subsection{\texorpdfstring{Existence of $\infty$-light forms}{Existence of infinity-light forms}}
We now take the limit $p \to \infty$ to obtain a privileged form of best comass.
To do so, we shall need the $p$-light forms to be uniformly bounded in the following sense.

\begin{lemma}
Let $F_p$ be a $p$-light form, and let $B$ range over closed $d - 1$-forms cohomologous to $F_p$. Then
\begin{equation}\label{infinity magnetic rules p magnetic}
	\|F_p\|_{L^p} \leq |M|^{1/p} \inf_B \|B\|_{L^\infty}.
\end{equation}
\end{lemma}
\begin{proof}
By H\"older's inequality and the fact that $F_p$ is $p$-light,
$$\|F_p\|_{L^p} \leq \|B\|_{L^p} \leq |M|^{1/p} \|B\|_{L^\infty},$$
hence the same holds for the infimum.
\end{proof}

\begin{proposition}\label{existence infinity}
Let $\rho \in H^{d - 1}(M, \RR)$.
For each $p \geq 2$, let $F_p$ be the $p$-light form representing $\rho$. Then there exists a closed $d - 1$-form $F$ such that:
\begin{enumerate}
\item $F_p \to F$ weakly in $L^r$ along a subsequence, for any $d < r < \infty$.
\item $F$ is a best comass representative of $\rho$.
% \item One has 
% \begin{equation}\label{Sobolev bounds for infinity}
% 	\|F\|_{L^\infty} \lesssim |\rho|.
% \end{equation}
\end{enumerate}
\end{proposition}
\begin{proof}
We roughly follow \cite[\S3]{Lindqvist14}.
Let $r > d$, and let $B$ be an $L^\infty$ representative of $\rho$.
By H\"older's inequality and (\ref{infinity magnetic rules p magnetic}),
\begin{equation}\label{uniform bounds in p by best curl}
	\|F_p\|_{L^r} \leq |M|^{\frac{1}{r} - \frac{1}{p}} \|F_p\|_{L^p} \leq |M|^{\frac{1}{r}} \|B\|_{L^\infty}.
\end{equation}
Thus a compactness argument gives $F_p \to F$ for some $d - 1$-form $F$, weakly in $L^r$, and 
$$\|F\|_{L^r} \leq \liminf_{p \to \infty} \|F_p\|_{L^r} \leq |M|^{\frac{1}{r}} \|B\|_{L^\infty}.$$
Diagonalizing, we may assume that $F_p \to F$ weakly in $L^r$ for every such $r$, and taking $r \to \infty$, we conclude 
\begin{equation}\label{infinity magnetics have best curl}
	\|F\|_{L^\infty} \leq \|B\|_{L^\infty}.
\end{equation}
Moreover, $[F] = \lim_{p \to \infty} [F_p] = \rho$.
So by Proposition \ref{crandall}(\ref{crandall linfinity}) and the fact that $B$ was arbitrary in (\ref{infinity magnetics have best curl}), $F$ has best comass.
%  Moreover, taking the limit as $p \to \infty$ in (\ref{Sobolev bounds for p}), we obtain (\ref{Sobolev bounds for infinity}).
\end{proof}

\begin{definition}
The $d - 1$-form $F$ of best comass in Proposition \ref{existence infinity} is called an \dfn{$\infty$-light form}.
\end{definition}

The existence of $\infty$-light (or even just best comass) representatives of each cohomology class implies the following useful lemma.

\begin{lemma}\label{p lights approximate L}
Let $F_p$ be the $p$-light representative of $\rho$, and $L$ the best comass of $\rho$. Then 
$$\lim_{p \to \infty} \|F_p\|_{L^p} = L.$$
\end{lemma}
\begin{proof}
We follow \cite[Lemma 2.7]{daskalopoulos2020transverse}.
Let $F$ be an $\infty$-light representative of $\rho$, so by Proposition \ref{crandall}(\ref{crandall linfinity}), $\|F\|_{L^\infty} = L$.
Since $F_p$ is $p$-light, H\"older's inequality implies 
$$\|F_p\|_{L^p} \leq \|F\|_{L^p} \leq |M|^{\frac{1}{p}} L.$$
Therefore 
$$\limsup_{p \to \infty} \|F_p\|_{L^p} \leq L.$$
To prove the converse, suppose that 
$$\liminf_{p \to \infty} \|F_p\|_{L^p} \leq \tilde L < L.$$
Along a subsequence which attains the limit inferior, $F_p$ converges weakly in every $L^r$, $d < r < \infty$, to an $\infty$-light form $\tilde F$ such that (by H\"older's inequality)
$$\|\tilde F\|_{L^r} \leq \liminf_{p \to \infty} \|F_p\|_{L^r} \leq \liminf_{p \to \infty} |M|^{\frac{1}{r}} \|\tilde F\|_{L^\infty} \leq |M|^{\frac{1}{r}} \tilde L.$$
Taking $r \to \infty$, we obtain $L(\tilde F) < L$, which contradicts the fact that $L$ is the best comass.
\end{proof}

We now make some remarks on $\infty$-light forms that we shall not need in the sequel.
There is an Euler-Lagrange equation for $C^1$ absolutely best comass forms, which we state in Appendix \ref{EulerLagrange}.
By analogy with $\infty$-harmonic functions, one expects the following:

\begin{conjecture}
For every $\rho \in H^{d - 1}(M, \RR)$ there exists a unique $\infty$-light representative $F$ of $\rho$.
Moreover, $F$ has absolutely best comass.
\end{conjecture}

The proof seems not so obvious.
Since the comass is not a local functional in the sense of \cite{Crandall2008}, and is certainly not strictly convex, the usual lines of attack are ruled out.
This is already a problem for proving that $\infty$-harmonic functions are unique and absolutely best Lipschitz, but the $\infty$-Laplacian has a good notion of viscosity solution, and in particular a maximum principle, enabling proofs of both claims \cite{Lindqvist14}.
Of course, the notions of viscosity solution and maximum principle are of no use in \emph{systems} of PDE, and in particular are useless for studying $\infty$-light forms.


%%%%%%%%%%%%%%%%%%%%
\subsection{\texorpdfstring{$1$-harmonic conjugates of $\infty$-light forms}{One-harmonic conjugates of infinity-light forms}}
We now construct the $1$-harmonic conjugate of an $\infty$-light form.
We cannot naively take limits, however, because (\ref{inverse extremality}) may blow up as $p \to \infty$.
Instead, we have to renormalize the $q$-harmonic conjugates of $p$-light forms, as in \cite[\S3.2]{daskalopoulos2020transverse}, and then apply the theory of \S\ref{qLaplace theory}.

To this end, fix a class $\rho \in H^{d - 1}(M, \RR)$ and denote by $L$ the comass of a best comass representative of $\rho$.
Also let $k_p$ be defined by 
$$k_p^{1 - p} = \int_M \star |F_p|^p$$
where $F_p$ is the $p$-light representative of $\rho$.

\begin{definition}
The \dfn{renormalized $q$-harmonic conjugate} of a $p$-light form $F_p$ is the function $u_q: \tilde M \to \RR$ which has mean zero on $M_{\rm fun}$ and solves
$$\dif u_q = (-1)^{d - 1} k_p^{p - 1} |F_p|^{p - 2} \star F_p.$$
\end{definition}

\begin{lemma}\label{normalizations converge}
As $p \to \infty$, $k_p \to 1/L$.
\end{lemma}
\begin{proof}
We follow \cite[Lemma 3.4]{daskalopoulos2020transverse}.
By Lemma \ref{p lights approximate L},
$$\lim_{p \to \infty} k_p^{-\frac{1}{q}} = \lim_{p \to \infty} \|F_p\|_{L^p} = L.$$
Taking logarithms we see that $q^{-1} \log k_p \to -\log L$, and since $q \to 1$ the claim follows.
\end{proof}

\begin{proposition}\label{existence 1}
Let $\rho \in H^{d - 1}(M, \RR)$ and let $\tilde M \to M$ be the universal cover.
For $2 < p < \infty$ and $\frac{1}{p} + \frac{1}{q} = 1$, let $u_q$ be the renormalized $q$-harmonic conjugate of the $p$-light representative of $\rho$.
Then there exists a $\pi_1(M)$-equivariant function $u \in BV_\loc(\tilde M)$ such that:
\begin{enumerate}
\item $u$ is $1$-harmonic.
\item As $q \to 1$ along a subsequence, $u_q \to u$ weakly in $BV_\loc(\tilde M)$ and strongly in $L^r_\loc(\tilde M)$ for $1 \leq r < \frac{d}{d - 1}$.
\item Let $F$ be the $\infty$-light representative of $\rho$, with best comass $L$. We have the \dfn{max flow min cut principle} that, as Radon measures,
\begin{equation}\label{1 extremality}
\dif u \wedge F = L \star |\dif u|.
\end{equation}
\end{enumerate}
\end{proposition}
\begin{proof}
We first compute using Lemma \ref{normalizations converge}
\begin{equation}\label{Lqs of qLaplace converge}
\lim_{q \to 1} \int_M \star |\dif u_q|^q = \lim_{p \to \infty} k_p^p \int_M \star |F_p|^p = \lim_{p \to \infty} k_p = \frac{1}{L}.
\end{equation}
So by Proposition \ref{qharmonics converge}, there exists a $\pi_1(M)$-equivariant function $u$ such that, along a subsequence, $u_q \to u$ weakly $BV_\loc(\tilde M)$ and strongly in $L^r_\loc(\tilde M)$.
Moreover, $[\dif u_q] \to [\dif u]$, and
\begin{equation}\label{why we used heat}
	\int_M \star |\dif u| = \lim_{q \to 1} \frac{1}{q} \int_M \star |\dif u_q|^q.
\end{equation}

Renormalizing (\ref{strong duality}), we obtain 
$$\frac{k_p^{-p}}{q} \int_M \star |\dif u_q|^q + \frac{1}{p} \int_M \star |F_p|^p = k_p^{1 - p} \int_M \dif u_q \wedge F_p.$$
Multiplying by $k_p^p$, we have 
\begin{equation}\label{1 strong duality before limits}
	\frac{1}{q} \int_M \star |\dif u_q|^q + \frac{k_p}{p} \int_M \star |F_p|^p = k_p \int_M \dif u_q \wedge F_p.
\end{equation}
We want to take the limit as $p \to \infty$ in (\ref{1 strong duality before limits}) to deduce
\begin{equation}\label{1 strong duality}
	L \int_M \star |\dif u| = \int_M \dif u \wedge F.
\end{equation}
We already know that (\ref{why we used heat}) holds.
By Lemma \ref{normalizations converge},
$$\lim_{p \to \infty} \frac{k_p}{p} \int_M \star |F_p|^p = \lim_{p \to \infty} \frac{k_p}{p} = \frac{0}{L} = 0,$$
and
$$\lim_{p \to \infty} k_p \int_M \dif u_q \wedge F_p = \frac{1}{L} \lim_{p \to \infty} \int_M [\dif u_q] \wedge \rho.$$
Since $[\dif u_q] \to [\dif u]$, we obtain
$$\lim_{p \to \infty} \int_M [\dif u_q] \wedge \rho = \int_M \alpha \wedge \rho = \int_M \dif u \wedge F,$$
as desired.
Taking the limit in (\ref{1 strong duality before limits}) completes the proof of (\ref{1 strong duality}).

We next claim that as Radon measures, 
\begin{equation}\label{one sided extremality}
\dif u \wedge F \leq L \star |\dif u|.
\end{equation}
To see this, fix a smooth test function $f$, integrate by parts, and use H\"older's inequality:
$$\int_M f \dif u \wedge F = -\int_M u \dif f \wedge F \leq L \|u \dif f\|_{L^1}.$$
But, if $\psi$ ranges over smooth $d-1$-forms such that $\|\psi\|_{C^0} = 1$, then another application of integration by parts and H\"older's inequality gives
\begin{align*}
\|u \dif f\|_{L^1} &= \sup_\psi \int_M u \dif f \wedge \psi = \sup_\psi \int_M f \dif u \wedge \psi = \int_M \star |f \dif u| \\
&\leq \|f\|_{C^0} \int_M \star |\dif u|.
\end{align*}
Thus we have 
$$\int_M f \dif u \wedge F \leq L \|f\|_{C^0} \int_M \star |\dif u|,$$
and since $f$ is arbitrary we conclude (\ref{one sided extremality}).

Next we deduce (\ref{1 extremality}).
We reason by contradiction: if (\ref{1 extremality}) is false, then there exists an open set $U \subseteq M$ such that 
$$\int_U \dif u \wedge F < L \int_U \star |\dif u|.$$
(Indeed, strict inequality cannot point in the other direction, by (\ref{one sided extremality}).)
However, by (\ref{one sided extremality}), 
$$\int_{M \setminus U} \dif u \wedge F \leq L \int_{M \setminus U} \star |\dif u|.$$
Adding up the integrals of $\dif u \wedge F$ over $U$ and $M \setminus U$, we conclude 
$$\int_M \dif u \wedge F < L \int_M \star |\dif u|,$$
but this contradicts (\ref{1 strong duality}); thus (\ref{1 extremality}) must be true.

To round out the proof, let $X := (\star F/L)^\sharp$ be the Poincar\'e dual vector field to $F/L$. Then
$$\nabla \cdot X = \star \frac{\dif F}{L} = 0,$$
and $\|X\|_{L^\infty} \leq 1$.
Moreover, by (\ref{1 extremality}), $X$ is normal to the level sets of $u$, and hence is a witness that $u$ is $1$-harmonic \cite{Mazon14}.
\end{proof}

%%%%%%%%%%%%%%%%%%%%

\section{The maximum comass locus}
Throughout this section, let $M$ be a closed space form of dimension $2 \leq d \leq 4$ equipped with a cohomology class $\rho \in H^{d - 1}(M, \RR)$.
We study the set on which a best comass representative of $\rho$ attains its comass, which turns out to contain a measured oriented minimal lamination which only depends on $\rho$, and which is calibrated by any best comass form on $\rho$.

\begin{definition}
Let $F$ be a form of best comass.
The \dfn{maximum comass locus} is the set $\{L(F, \cdot) = L(F)\}$.
\end{definition}

By Proposition \ref{crandall usc} and the compactness of $M$, the maximum comass locus is a nonempty closed subset of $M$.

%%%%%%%%%%%%%%%%%%%%%%%%
\subsection{\texorpdfstring{$L^\infty$}{L-infinity} calibrations}
Recall that a \dfn{calibration} (in the classical sense) is a closed $d-1$-form $F$ such that $\|F\|_{C^0} \leq 1$.
A $d-1$-surface $N$ is then said to be $F$-\dfn{calibrated} if $\int_N F = |F|$.
In our formulation, it is more natural to study calibrations which are merely $L^\infty$ rather than $C^0$, and require that the comass $L(F) \leq 1$.
By Proposition \ref{integration is welldefined}, it is still meaningful to ask if a $d-1$-surface is $F$-calibrated, even if $F$ is discontinuous.
However, we shall be interested in calibrated laminations.

\begin{definition}
Let $F$ be a calibration, and let $\lambda$ be a measured oriented lamination in $M$.
Then $\lambda$ is $F$-\dfn{calibrated} if every leaf of $\lambda$ is $F$-calibrated.
\end{definition}

Recall that if $\lambda$ is a measured oriented lamination, and $T_\lambda$ is its Ruelle-Sullivan current, then we define the \dfn{mass} $|\lambda|$ to be the mass of $T_\lambda$, and we define the homology class $[\lambda] \in H_{d - 1}(M, \RR)$ to be the homology class of $T_\lambda$.
Since it has a homology class, we may pair $T_\lambda$ with any closed but possibly discontinuous $d-1$-form $F$, and then $\int_M T_\lambda \wedge F$ is just the pairing of $[\lambda]$ with $[F]$.

\begin{proposition}\label{calibration condition}
Let $F$ be a calibration.
Let $T_\lambda$ be the Ruelle-Sullivan current of a measured oriented lamination $\lambda$, and suppose that 
\begin{equation}\label{calibration by Ruelle Sullivan}
\int_M T_\lambda \wedge F = |\lambda|.
\end{equation}
Then $\lambda$ is $F$-calibrated (and in particular minimal).
\end{proposition}
\begin{proof}
Let \todo{Cite this} $(\chi_\alpha)$ be a partition of unity subordinate to a laminar atlas with charts
$$\Psi_\alpha: I \times J \to M$$
for $\lambda$, where $I$ is an interval in $\RR$ and $J$ is a cube in $\RR^{d - 1}$.
Let $(\mu_\alpha)$ denote the transverse measure with respect to this atlas.
Then 
$$|\lambda| = \int_M T_\lambda \wedge F = \sum_\alpha \int_I \int_{\{k\} \times J} \chi_\alpha F \dif \mu_\alpha(k).$$

Here we should point out that $\int_{\{k\} \times J} \chi_\alpha F$ is well-defined: $\{k\} \times J$ is a cube, so its pushforward by $\Psi_\alpha$ is a contractible $d-1$-cell $\sigma_{\alpha, k}$ in $M$ such that $\chi_\alpha|_{\sigma_{\alpha, k}}$ has compact support on the interior of $\sigma_{\alpha, k}$.
In particular, in a neighborhood of $\sigma_{\alpha, k}$, $F = \dif A$ for some $A$, which we may assume to be in Coulomb gauge, and hence continuous by (\ref{Sobolev}).
Then if $F$ is continuous,
$$\int_{\sigma_{\alpha, k}} \chi_\alpha F = \int_{\sigma_{\alpha, k}} \chi_\alpha \dif A = -\int_{\sigma_{\alpha, k}} \dif \chi_\alpha \wedge A,$$
which clearly does not depend on the choice of $A$, and extends by density to general calibrations $F$.

We know that 
$$\int_M \chi_\alpha \star |T_\lambda| = \int_I \int_{\{k\} \times J} \star_{\{k\} \times J} \chi_\alpha \dif \mu_\alpha(k),$$
so summing in $\alpha$, we obtain 
$$\sum_\alpha \int_I \int_{\{k\} \times J} \chi_\alpha F \dif \mu_\alpha(k) = |\lambda| = \sum_\alpha \int_I \int_{\{k\} \times J} \star_{\{k\} \times J} \chi_\alpha \dif \mu_\alpha(k).$$
If our partition of unity is chosen locally finite, this is only possible if for every $\alpha$ and $\mu_\alpha$-almost every $k$, 
$$\int_{\{k\} \times J} \star_{\{k\} \times J} \chi_\alpha = \int_{\{k\} \times J} \chi_\alpha F,$$
or in other words that $F$ calibrates $\sigma_{\alpha, k}$.

To upgrade $F$ from a calibration of almost all of $\lambda$ to a calibration of all of $\lambda$, we use the fact that $\supp \mu_\alpha$ is the set of $k \in I$ such that $\sigma_{\alpha, k}$ is contained in a leaf of $\lambda$.
In particular, if $\sigma_{\alpha, k}$ is contained in a leaf of $\lambda$, then we may find $k_j$ such that $k_j \to k$ and $\sigma_{\alpha, k_j}$ is $F$-calibrated and contained in a leaf of $\lambda$.
But then, since we may write $F = \dif A$ near $\sigma_{\alpha, k}$, and 
$$\int_{\sigma_{\alpha, k}} F = \int_{\partial \sigma_{\alpha, k}} A,$$
the facts that $k_j \to k$ and that $A$ is continuous imply 
\begin{align*}
|\sigma_{\alpha, k}| &= \lim_{j \to \infty} |\sigma_{\alpha, k_j}| = \lim_{j \to \infty} \int_{\sigma_{\alpha, k_j}} F = \lim_{j \to \infty} \int_{\partial \sigma_{\alpha, k_j}} A = \int_{\partial \sigma_{\alpha, k}} A \\
&= \int_{\sigma_{\alpha, k}} F. \qedhere 
\end{align*}
\end{proof}

\begin{proposition}\label{properties of calibrated laminations}
Suppose that $F$ is a calibration and $\lambda$ is a measured oriented $F$-calibrated lamination.
Then:
\begin{enumerate}
\item $\lambda$ is minimal.
\item If $G$ is a calibration and cohomologous to $F$, then $\lambda$ is $G$-calibrated.
\item The maximum comass locus of $F$ contains $\supp \lambda$.
\end{enumerate}
\end{proposition}
\begin{proof}
Every leaf of $\lambda$ is $F$-calibrated, hence minimal, so $\lambda$ is also minimal.
Moreover, (\ref{calibration by Ruelle Sullivan}) only depends on the cohomology class of $F$, not $F$ itself, so $\lambda$ is $G$-calibrated.
Finally, let $S$ be the maximum comass locus of $F$, $N$ a leaf of $\lambda$, and suppose that $x \in N \setminus S$.
Since $S$ is closed, there exists $\varepsilon > 0$ such that $B_\varepsilon(x)$ does not meet $S$.
Moreover, $\sigma := N \cap B_\varepsilon(x)$ is a $d-1$-chain in $B_\varepsilon(x)$, so by \todo{cite the right part of the Crandall theorem}
$$\frac{1}{|\sigma|} \int_\sigma F \leq \sup_{y \in B_\varepsilon(x)} L(F, y) < L = 1.$$
But then 
$$\int_N F = \int_\sigma F + \int_{N \setminus B_\varepsilon(x)} F < |\sigma| + |N \cap B_\varepsilon(x)| = |N|,$$
so $N$ (hence $\lambda$) is not $F$-calibrated.
\end{proof}


%%%%%%%%%%%%%%%%%%%%%%%%
\subsection{Calibration of Thurston laminations}
We recall the main result of \todo{Cite laminations paper}:

\begin{theorem}
Let $u$ be a $1$-harmonic function on a Riemannian manifold of constant sectional curvature and dimension $2 \leq d \leq 4$.
Then there exists a measured oriented minimal lamination $\lambda_u$, whose leaves are the level sets $\partial \{u > y\}$ of $u$, and whose Ruelle-Sullivan current is $\dif u$.
\end{theorem}

Let $u$ be a $\pi_1(M)$-equivariant $1$-harmonic function on $\tilde M$.
Then $\dif u$ drops to a $d-1$-current on $M$, which is still the Ruelle-Sullivan current of a measured oriented minimal lamination on $M$, which we still call $\lambda_u$.

One may associate a measured oriented geodesic lamination $\lambda$ associated to a homotopy class $\rho \in [M, N]$, where $M, N$ are closed hyperbolic surfaces of the same genus, called the \dfn{Thurston lamination}.
Every best Lipschitz representative of $\rho$ attains its Lipschitz constant on (a superlamination of) the Thurston lamination, which also realizes the supremum in Thurston's asymmetric metric $\log K$ \cite{Thurston98}.

We claim that the lamination induced by the $1$-harmonic conjugate of an $\infty$-light form enjoys the same boons as the Thurston lamination of a homotopy class of surface-to-surface maps.
Thus we define:

\begin{definition}
Let $\rho \in H^{d - 1}(M, \RR)$, let $F$ be an $\infty$-light representative of $\rho$, and let $u$ be a $1$-harmonic conjugate of $F$.
Then we call $\lambda_u$ a \dfn{Thurston lamination} associated to $\rho$.
\end{definition}

\begin{theorem}\label{MCL contains Thurston}
Let $F$ be a best comass representative of $\rho \in H^{d - 1}(M, \RR)$, let $L := L(F)$ be the best comass of $\rho$, and let $\lambda$ be a Thurston lamination associated to $\rho$.
Then:
\begin{enumerate}
\item The maximum comass locus of $F$ contains $\lambda$.
\item $F/L$ calibrates $\lambda$.
\end{enumerate}
\end{theorem}
\begin{proof}
Let $G$ be the $\infty$-light form which is cohomologous to $F$ whose dual $1$-harmonic function $u$ defines the Thurston lamination $\lambda$.
Then by the max flow min cut principle (\ref{1 extremality}), 
$$|\lambda| = \int_M \star |\dif u| = \frac{1}{L} \int_M \dif u \wedge G$$
so $G/L$ calibrates $\lambda$ by Proposition \ref{calibration condition}.
Then by Proposition \ref{properties of calibrated laminations}, $F/L$ calibrates $\lambda$, and the maximum comass locus of $F/L$ contains $\lambda$.
However, the maximum comass locus is preserved by rescaling.
\end{proof}

% On the first reading, the reader may wish to take $F$ to be $\infty$-light, as this already captures the essential ideas.
% Throughout the proof we fix the $p$-light representative $F_p$ of $\rho$, its conjugate $q$-harmonic $u_q$, and the limiting $1$-harmonic $u$.

% \begin{lemma}
% Let
% $$f_p := 2\langle k_p F_p, k_p F_p - L^{-1} F\rangle.$$
% Then 
% \begin{equation}\label{MCL contains Thurston 1}
% 	\lim_{p \to \infty} \int_{\{f_p \geq 0\}} \star |k_p F_p|^{p - 2} f_p = 0.
% \end{equation}
% \end{lemma}
% \begin{proof}
% The proof is very similar to \cite[Lemma 6.3]{daskalopoulos2020transverse}.
% Since $F_p, F$ are cohomologous, there exists a $1$-form $\xi$ such that $\dif \xi = F_p - F$.
% On the other hand, since $F_p$ is $p$-light, an integration by parts gives
% \begin{align*}
% \int_M |F_p|^{p - 2} \langle F_p, F_p - F \rangle = \int_M \langle \dif^*(|F_p|^{p - 2} F_p), \xi\rangle = 0.
% \end{align*}
% In particular,
% \begin{equation}\label{MCL contains Thurston 2}
% 	I_p := \int_M \star |k_p F_p|^{p - 2} \langle k_p F_p, k_p F_p - k_p F\rangle = 0.
% \end{equation}
% On the other hand, by the Cauchy-Schwarz inequality, Lemma \ref{normalizations converge}, and the fact that 
% $$\lim_{p \to \infty} \|F_p\|_{L^p} = \|F\|_{L^\infty} = L,$$
% we have
% \begin{align*}
% \lim_{p \to \infty} I_p - \frac{1}{2} \int_M \star |k_p F_p|^{p - 2} f_p 
% &= \lim_{p \to \infty} \int_M \star |k_p F_p|^{p - 2} \left\langle k_p F_p, \left(\frac{1}{L} - k_p\right) F\right\rangle \\
% &\leq \lim_{p \to \infty} k_p^{p - 1} \left(\frac{1}{L} - k_p\right) \int_M \star |F_p|^{p - 1} |F| \\
% &= \lim_{p \to \infty} \frac{1}{L^{p - 1}} \left(\frac{1}{L} - \frac{1}{L}\right)L^p \\
% &= \lim_{p \to \infty} 1 - 1 = 0.
% \end{align*}
% If we plug this into (\ref{MCL contains Thurston 2}), we obtain 
% \begin{equation}\label{MCL contains Thurston 3}
% 	\lim_{p \to \infty} \int_M \star |k_p F_p|^{p - 2} f_p = 0.
% \end{equation}
% If $f_p(x) \leq 0$, then by the Cauchy-Schwarz inequality,
% $$|L^{-1} F|(x) \geq |k_p F_p|(x),$$
% so by the Cauchy-Schwarz and Peter-Paul inequalities,
% \begin{align*}
% f_p(x) &\leq 2|L^{-1} F|(x) |k_p F_p|(x) - 2|k_p F_p|^2(x) \\
% &\leq |L^{-1} F|(x)^2 + |k_p F_p|(x)^2 - 2|k_p F_p|^2(x) \\
% &= |L^{-1} F|(x)^2 - |k_p F_p|(x).
% \end{align*}
% So by \cite[Lemma 6.2]{daskalopoulos2020transverse},
% $$|k_p F_p|(x)^{p - 2} f_p(x) < \frac{2}{p - 2}.$$
% Integrating this inequality, 
% $$0 \leq \lim_{p \to \infty} \int_{\{f_p \leq 0\}} |k_p F_p|^{p - 2} f_p \leq \lim_{p \to \infty} \frac{2|M|}{p - 2} = 0.$$
% Therefore, by (\ref{MCL contains Thurston 3}), we deduce (\ref{MCL contains Thurston 1}).
% \end{proof}

% \begin{lemma}\label{MCL contains Thurston lemma}
% The set $\{L(F, \cdot) < L\}$ satisfies
% $$\lim_{p \to \infty} \int_{\{L(F, \cdot) < L\}} \star |k_p F_p|^p = 0.$$
% \end{lemma}
% \begin{proof}
% We decompose
% $$\{L(F, \cdot) < L\} = \bigcup_{0 < \theta < 1} \{L(F, \cdot) < \theta L\}$$
% and then it suffices to fix $\theta$ and show that the integral over $\{L(F, \cdot) < \theta L\}$ is zero.
% We then use the fact that $|F|(x) \leq L(F, x)$ almost everywhere, by Proposition \ref{crandall}(\ref{crandall LDT}), to obtain
% $$\{L(F, \cdot) < \theta L\} \subseteq A \cup B \cup Z$$
% where 
% \begin{align*}
% A &:= \{|L^{-1} F|^2 \leq \theta |k_p F_p|^2 + |k_p F_p - L^{-1} F|^2\}, \\
% B &:= \{|L^{-1} F|^2 \geq \theta |k_p F_p|^2 + |k_p F_p - L^{-1} F|^2\} \cap \{L(F, \cdot) < \theta L\},
% \end{align*}
% and $Z$ is null. 
% In particular,
% $$0 \leq \lim_{p \to \infty} \int_{\{L(F, \cdot) < L\}} \star |k_p F_p|^p \leq \lim_{p \to \infty} \int_A \star |k_p F_p|^p + \lim_{p \to \infty} \int_B \star |k_p F_p|^p.$$
% By (\ref{MCL contains Thurston 1}),
% \begin{align*}
% 0 &= \lim_{p \to \infty} \int_{\{f_p \geq 0\}} |k_p F_p|^{p - 2} f_p \\
% &= \lim_{p \to \infty} \int_{\{f_p \geq 0\}} |k_p F_p|^{p - 2}((1 - \theta) |k_p F_p|^2 + \theta |k_p F_p|^2 + |k_p F_p - L^{-1} F|^2 - |L^{-1} F|^2).
% \end{align*}
% But on $A$,
% \begin{align*}
% |L^{-1} F|^2 &\leq \theta |k_p F_p|^2 + |k_p F_p - L^{-1} F|^2 \\
% &\leq |k_p F_p|^2 + |k_p F_p - L^{-1} F|^2 \\
% &= f_p + |L^{-1} F|^2
% \end{align*}
% which implies $f_p \geq 0$ and 
% $$\lim_{p \to \infty} \int_A \star |k_p F_p|^p \leq \lim_{p \to \infty} \frac{1}{\theta} \int_{\{f_p \geq 0\}} \star |F_p|^{p - 2} f_p = 0.$$
% Meanwhile, on $B$,
% $$|k_p F_p|^2 \leq |k_p F_p|^2 + \theta^{-1} |k_p F_p - L^{-1} F|^2 \leq \theta^{-1} |L^{-1} F|^2 < \theta.$$
% Therefore
% \begin{align*}
% 	\lim_{p \to \infty} \int_B |k_p F_p|^p &\leq \lim_{p \to \infty} \theta^{p/2} |M| = 0. \qedhere
% \end{align*}
% \end{proof}

% \begin{proof}[Proof of Theorem \ref{MCL contains Thurston}]
% Let $U := \{L(F, \cdot) < L\}$, which is open since it is the complement of the best comass locus of $F$.
% So by the portmanteau theorem \todo{Cite it}
% \begin{align*}
% \int_U \star |\dif u|
% &\leq \lim_{q \to 1} \int_U \star |\dif u_q|
% = \lim_{p \to \infty} \int_U \star |k_p F_p|^{p - 1}.
% \end{align*}
% Then by H\"older's inequality and Lemma \ref{MCL contains Thurston lemma},
% \begin{align*}
% \lim_{p \to \infty} \int_U \star |k_p F_p|^{p - 1}
% &\leq \lim_{p \to \infty} |M|^{-\frac{1}{p}} \left[\int_U \star |k_p F_p|^p\right]^{\frac{1}{q}} 
% = \lim_{p \to \infty} \int_U \star |k_p F_p|^p = 0.
% \end{align*}
% Since $\dif u$ is the Ruelle-Sullivan current for the Thurston lamination $\lambda$, $\lambda$ is contained in the best comass locus of $F$. 
% Moreover, if $N$ is a leaf of $\lambda$, then the restriction of $F/L$ is, by (\ref{1 extremality}), equal to $\star \normal_N^\flat$, which is the area form of $N$.
% In other words, $F/L$ calibrates $N$.
% \todo{Fix the proof of calibration if $F$ is not $\infty$-light, or add that as a hypothesis}
% \end{proof}



%%%%%%%%%%%%%%%%%%%%%%
% \subsection{Thurston's \texorpdfstring{$L = K$}{L equals K} theorem}
\begin{theorem}\label{L equals K}
	Let $\rho \in H^{d - 1}(M, \RR)$, and let 
	$$K := \sup_\lambda \frac{\langle \rho, [\lambda]\rangle}{|\lambda|},$$
	where $\lambda$ ranges over measured oriented laminations. Then:
\begin{enumerate}
	\item The supremum in $K$ is attained by the Thurston lamination associated to $\rho$.
	\item Let $L$ be the best comass of $\rho$. Then $L = K$.
\end{enumerate}
\end{theorem}
\begin{proof}
Fix the $\infty$-light form $F$ representing $\rho$, and let $u$ be its $1$-harmonic conjugate. \todo{Do this more carefully and in coordinate patches}

We first prove $K \leq L$.
Let $\lambda$ be a measured oriented lamination; then, since $F$ represents $\rho$ and the Ruelle-Sullivan current $T_\lambda$ represents $[\lambda]$,
$$\langle \rho, [\lambda]\rangle = \int_M F \wedge T_\lambda.$$
Let $\mu_\lambda$ be the transverse measure to $\lambda$ and $\mathscr L_\lambda$ the space of leaves of $\lambda$; then, by definition of $T_\lambda$ and the definition of comass,
$$\int_M F \wedge T_\lambda = \int_{\mathscr L_\lambda} \int_N F \dif \mu_\lambda(N) \leq L(F) \int_{\mathscr L_\lambda} |N| \dif \mu_\lambda(N) = L(F) |\lambda|.$$
By Proposition \ref{crandall}(\ref{crandall linfinity}), $F$ has best comass $L(F) = L$.
Since $\lambda$ was arbitrary, it holds that $K \leq L$.

Let $\lambda$ be the Thurston lamination, so $T_\lambda = \dif u$.
Then by the max flow min cut principle (\ref{1 extremality}),
$$\langle \rho, [\lambda]\rangle = \int_M F \wedge \dif u = L \int_M \star |\dif u| = L|\lambda|.$$
Dividing both sides by $|\lambda|$ and applying the direction we already proved,
$$K \leq L \leq \frac{\langle \rho, [\lambda]\rangle}{|\lambda|} \leq K$$
which is only possible if $L = K$ and $\lambda$ is a maximizer.
\end{proof}

%%%%%%%%%%%%%%%%%%%%%%%%%%%%%%%%%%
\appendix 
\section{Max flow, min cut}\label{Max Flow Min Cut}
Let us interpret the duality between the $q$-Laplacian and $p$-light forms as a version of the max flow min cut principle. 
For a reflexive Banach space $X$, we denote by $\hat X$ its dual.
If $J: X \to \RR \cup \{+\infty\}$ is a convex function, we introduce its \dfn{Legendre transform}
\begin{align*}
	\hat J: \hat X &\to \RR \cup \{+\infty\}\\
	\xi &\mapsto \sup_{x \in X} \langle \xi, x\rangle - f(x).
\end{align*}

\begin{definition}
Let $Y$ be a reflexive Banach space equipped with a mapping $\Lambda: X \to Y$.
A function $J: Y \to \RR \cup \{+\infty\}$ is said to be a \dfn{suitable convex function} for our purposes if:
\begin{enumerate}
\item $J$ is strictly convex,
\item $J$ is lower semicontinuous and not identically $+\infty$,
\item if $|y| \to \infty$ in $Y$, then $J(y) \to +\infty$, and 
\item there exists a point $x \in X$ such that $J$ is continuous and finite at $\Lambda(x)$.
\end{enumerate}
\end{definition}

\begin{proposition}\label{abstract convex analysis}
Let $X, Y$ be reflexive Banach spaces equipped with a bounded linear operator $\Lambda: X \to Y$ of trivial kernel.
Let $J: Y \to \RR \cup \{+\infty\}$ be a suitable convex function.
Then:
\begin{enumerate}
\item There exists a unique minimizer $\underline x \in X$ of $J(\Lambda(x))$, and a maximizer $\overline \eta \in \hat Y$ of $-\hat J(-\eta)$.
\item If $\hat J$ is strictly convex, then $\overline \eta$ is the unique maximizer of $-\hat J(-\eta)$.
\item We have \dfn{strong duality}
\begin{equation}\label{abstract strong duality}
J(\Lambda(\underline x)) = -\hat J(-\overline \eta).
\end{equation}
\end{enumerate}
\end{proposition}
\begin{proof}
This is the conjunction of several standard results in convex analysis.
Let $\Gamma_0(Y)$ denote the set introduced in \cite[Chapter I, Definition 3.1]{Ekeland99}.
Then by \cite[Chapter I, Proposition 3.1]{Ekeland99}, $J \in \Gamma_0(Y)$, so by \cite[Chapter III, Theorem 4.2]{Ekeland99} and \cite[Chapter III, (4.23)]{Ekeland99}, there exist minimizers which satisfy (\ref{abstract strong duality}).
By \cite[Chapter II, Proposition 1.2]{Ekeland99}, the minimizers of $J$ and $\hat J$ are unique.
Since $\Lambda$ has trivial kernel, minimizers of $J \circ \Lambda$ are also unique.
\end{proof}

\todo{Talk about the max flow min cut principle. Also shorten the arguments here}

We now specialize to the problem at hand.
Let $\alpha \in H^1(M, \RR)$ be a cohomology class.
Recall that we write $[u] = \alpha$ to mean that $u$ is an $\alpha$-equivariant function on $\tilde M$.

We here consider the problem
\begin{equation}\label{preprimal problem}
	\Min\{\|\dif u\|_{L^q(M_{\rm fun})}: u \in W^{1, q}(M_{\rm fun}), [u] = \alpha\},
\end{equation}
where $1 < q < \infty$; we identify two solutions if they agree by a constant.
Taking Euler-Lagrange equations, we see that (\ref{preprimal problem}) is equivalent to the $q$-Laplacian 
\begin{equation}\label{qLaplace}
\begin{cases}
	\dif^*(|\dif u|^{q - 2} \dif u) = 0 \\
	[u] = \alpha.
\end{cases}
\end{equation}

To put (\ref{preprimal problem}) in the framework of Proposition \ref{abstract convex analysis}, we shall fix a point $0 \in M_{\rm fun}$, choose a representative $1$-form (which we also call $\alpha$), and solve 
$$\begin{cases}
\dif f = \Pi^* \alpha \\
v(0) = 0.
\end{cases}$$
Thus (\ref{preprimal problem}) is equivalent to
\begin{equation}\label{primal problem}
	\Min\left\{\frac{1}{q} \int_{M_{\rm fun}} \star|\dif v + \Pi^* \alpha|^q: v \in W^{1, q}_0(M_{\rm fun})\right\}
\end{equation}
where we set $u = v + f$.

Let
$$J(\xi) := \frac{1}{q} \int_M \star|\xi + \alpha|^q,$$
defined for $\xi \in L^q(M, \Omega^1)$.
Since $v$ is traceless in (\ref{primal problem}), it is invariant, so $\dif v$ drops to a $1$-form on $M$, and (\ref{primal problem}) is the problem of minimizing $J(\dif v)$.
It is clear that $J$ is a suitable convex function on $L^q(M, \Omega^1)$ when it is equipped with the map
\begin{equation}\label{derivative on traceless}
\dif: W^{1, q}_0(M_{\rm fun}) \to L^q(M, \Omega^1),
\end{equation}
and moreover the kernel of (\ref{derivative on traceless}) is trivial.
By \cite[Chapter I, (4.9)]{Ekeland99} and \cite[Chapter I, Remark 4.1]{Ekeland99}, the Legendre transform
$$\hat J: L^p(M, \Omega^{d - 1}) \to \RR$$
of $J$, where $\frac{1}{p} + \frac{1}{q} = 1$, satisfies
\begin{equation}\label{Legendre transform}
\hat J(F) = \frac{1}{p} \int_M \star |F|^p + \int_M \alpha \wedge F
\end{equation}
and in particular is strictly convex.
Thus the convex dual problem of (\ref{primal problem}), namely the problem of maximizing $-\hat J(-F)$, is the problem
\begin{equation}\label{predual problem}
\Max\left\{- \frac{1}{p} \int_M \star |F|^p + \int_M \alpha \wedge F: F \in L^p(M, \Omega^{d - 1})\right\}.
\end{equation}

\begin{proposition}\label{convex duality}
Let $1 < p, q < \infty$ satisfy $\frac{1}{p} + \frac{1}{q} = 1$.
Then we have an isomorphism of cohomology groups
\begin{align*}
H^1(M, \RR) &\to H^{d - 1}(M, \RR) \\
\alpha &\mapsto \rho,
\end{align*}
so that the convex dual problem of the $\alpha$-equivariant $q$-Laplacian (\ref{qLaplace}) with solution $u: \tilde M \to \RR$ is the $p$-light problem (\ref{pMaxwell}) subject to $[F] = \rho$.
The problems (\ref{qLaplace}) and (\ref{pMaxwell}) both have unique solutions, and we have the relations (\ref{extremality}) and (\ref{inverse extremality}), and the strong duality theorem (\ref{strong duality}).
\end{proposition}
\begin{proof}
Since $J$ is a suitable convex function, $\hat J$ is strictly convex, and (\ref{derivative on traceless}) has trivial kernel, we may apply Proposition \ref{abstract convex analysis} to see that (\ref{qLaplace}) and (\ref{predual problem}) both have unique solutions $u, F$ which are related by strong duality (\ref{abstract strong duality}) as (\ref{strong duality}).
By uniqueness, if we exhibit $\tilde F$ defined in terms of $u$ which satisfies (\ref{strong duality}) and is a maximizer of $-\hat J$, then $F = \tilde F$.
Following \cite[Chapter IV, (2.12)]{Ekeland99}, we define $\tilde F$ to satisfy (\ref{extremality}), so that 
\begin{align*}
\int_M \dif u \wedge \tilde F 
&= \int_M \star |\dif u|^q \\
&= \frac{1}{q} \int_M \star |\dif u|^q + \frac{1}{p} \int_M \star |\dif u|^p \\
&= \frac{1}{q} \int_M \star |\dif u|^q + \frac{1}{p} \int_M \dif u \wedge \tilde F.
\end{align*}
Moreover, if we decompose 
$$\dif u = \dif v + \alpha$$
then
\begin{align*}
-\hat J(\tilde F)
&= -\frac{1}{p} \int_M \star |\dif u|^q + \int_M \alpha \wedge |\dif u|^{q - 2} \star \dif u \\
&= -\frac{1}{p} \int_M \star |\dif u|^q + \int_M \star |\dif u|^q - \int_M \dif v \wedge |\dif u|^{q - 2} \star \dif u \\
&= \frac{1}{q} \int_M \star |\dif u|^q + \int_M v \dif(|\dif u|^{q - 2} \star \dif u) \\
&= \frac{1}{q} \int_M \star |\dif u|^q = J(\dif u)
\end{align*}
where we used the fact that $\frac{1}{q} = 1 - \frac{1}{p}$, and the fact that $u$ solves the $q$-Laplace equation.
By strong duality (\ref{abstract strong duality}), it follows that $\tilde F$ is a maximizer of $-\hat J$, and therefore $F = \tilde F$; in particular, $F$ satisfies (\ref{extremality}).
Since $u$ solves the $q$-Laplace equation, (\ref{extremality}) implies that $\dif F = 0$.
We then define $\rho := [F]$.
On the other hand, 
$$(p - 2)(q - 1) + q - 2 = 0,$$
so following \cite[Lemma 3.2]{daskalopoulos2020transverse},
$$|F|^{p - 2} F = |\dif u|^{(p - 2)(q - 1) + q - 2} \star \star \dif u = (-1)^{d - 1} \dif u$$
which gives (\ref{inverse extremality}) and the equation
$$\dif^*(|F|^{p - 2} F) = 0.$$
By reflexivity, the above argument may also be reversed, so that the map $\alpha \mapsto \rho$ is invertible on the level of cohomology.
\end{proof}

\todo{Cite those glaciology papers which already include this duality}

%%%%%%%%%%%%%%%%%%%%%%%%%%%%%%
\section{The \texorpdfstring{$\infty$-Maxwell equation}{infinity-Maxwell equation}}\label{EulerLagrange}
We have the following Euler-Lagrange equation for forms with absolutely best comass.
Because of the lack of a good analogue for viscosity solutions for $\infty$-elliptic systems, and because we did not show that $\infty$-light forms have absolutely best comass, the equation can only really be interpreted in a formal sense, at least as far as we are aware.
As such, we did not use it in the main body of this paper, but only include it as a curiosity item.

% \todo{If we knew that $p$-Maxwell had good quantitative uniqueness, then we would have}
% It remains to show that $A$ has absolutely best curl, so let $\Omega$ be a small ball and $B$ a $1$-form with $B|_{\partial \Omega} = A|_{\partial \Omega}$.
% By a straightforward modification of the existence theorem, there exists a $p$-magnetic potential $B_p$ in Coulomb gauge with $B_p|_{\partial \Omega} = A|_{\partial \Omega}$ and $B \in C^{1 + \alpha}$.
% By quantitative uniqueness
% $$\|B_p - A\|_{C^0(\Omega)} \leq \|B_p - A_p\|_{C^0(\Omega)} + o(1) \lesssim \|A - A_p\|_{C^0(\partial \Omega)} + o(1) \ll 1.$$
% Therefore $B_p \to A$ uniformly, and for $3 < q < p < \infty$ with $p$ dyadic,
% $$\|\dif B_p\|_{L^q(\Omega)} \leq |\Omega|^{\frac{1}{q} -\frac{1}{p}} \|\dif B_p\|_{L^p(\Omega)} \leq |\Omega|^{\frac{1}{q} -\frac{1}{p}} \|\dif B\|_{L^p(\Omega)} \leq |\Omega|^{\frac{1}{q}} \|\dif B\|_{L^\infty(\Omega)}.$$
% Then along a subsequence, $\dif B_p \to \dif A$ in $L^q(\Omega)$, so 
% $$\|\dif A\|_{L^q(\Omega)} \leq |\Omega|^{\frac{1}{q}} \|\dif B\|_{L^\infty(\Omega)}.$$
% Taking $q \to \infty$ we arrive at the conclusion that $F$ has absolutely best comass.

\begin{proposition}
Suppose that $F$ has absolutely best comass, regularity $C^1$, and no points with $F = 0$. Then
\begin{equation}\label{infinityMaxwell}
	F^{ij} \partial_i |F| = 0.
\end{equation}
\end{proposition}
\begin{proof}
For a covariant $2$-tensor $T$, let $T^{\rm as}$ be its antisymmetrization, and let
$$f(x, T) := |T^{\rm as}|_{g(x)}.$$
Working locally, we may write $F = \dif A$ for some $A$, which we may assume to be in Coulomb gauge and therefore $C^2$ by a variant of (\ref{Sobolev}).
Since $A$ has absolutely best curl and $(\nabla A)^{\rm as} = \dif A$, $A$ is an absolute minimizer (see \cite[Definition 5.1]{Barron2001}) of the essential supremum of $f(\cdot, \nabla A)$.
By \cite[Theorem 5.2]{Barron2001},
\begin{equation}\label{ELA}
	\left\langle \frac{\partial f}{\partial T}(x, \nabla A(x)), \dif (f(x, \nabla A(x))) \right\rangle = 0.
\end{equation}
Now
$$\dif(f(x, \nabla A(x))) = \dif |\dif A(x)|$$
and 
$$\frac{\partial f}{\partial T}(x, \nabla A(x)) = \frac{\nabla A(x)^{\rm as}}{|\nabla A(x)^{\rm as}|} = \frac{\dif A(x)}{|\dif A(x)|}.$$
We conclude the claim after multiplying both sides of (\ref{ELA}) by $|\dif A|$.
\end{proof}

The $\infty$-Maxwell equation has the following natural interpretation.

\begin{corollary}
Suppose that $F$ has absolutely best comass, regularity $C^1$, and no points with $F = 0$, and $N$ is a surface whose normal vector field is annihilated by $F$.
Then $N$ is a minimal surface.
\end{corollary}
\begin{proof}
Let $V$ be a tangent vector field to $N$. Then $V(|F|) = 0$, by (\ref{infinityMaxwell}).
Therefore $|F|$ is constant along $N$, but $F$ is a continuous section of the area bundle of $N$, which is a real line bundle.
It follows that $F$ is constant along $N$, and $F/|F|$ is the area form on $N$.
In other words, $N$ is calibrated by $F$, and the claim follows from (\ref{calibrated surfaces are minimal}).
\end{proof}

\printbibliography

\end{document}
