\documentclass[reqno,11pt]{amsart}
\usepackage[letterpaper, margin=1in]{geometry}
\RequirePackage{amsmath,amssymb,amsthm,graphicx,mathrsfs,url,slashed,subcaption}
\RequirePackage[usenames,dvipsnames]{xcolor}
\RequirePackage[colorlinks=true,linkcolor=Red,citecolor=Green]{hyperref}
\RequirePackage{amsxtra}
\usepackage{cancel}
\usepackage{tikz-cd}

% \setlength{\textheight}{9.3in} \setlength{\oddsidemargin}{-0.25in}
% \setlength{\evensidemargin}{-0.25in} \setlength{\textwidth}{7in}
% \setlength{\topmargin}{-0.25in} \setlength{\headheight}{0.18in}
% \setlength{\marginparwidth}{1.0in}
% \setlength{\abovedisplayskip}{0.2in}
% \setlength{\belowdisplayskip}{0.2in}
% \setlength{\parskip}{0.05in}
%\renewcommand{\baselinestretch}{1.05}

\title{Comass minimizers}
\author{Aidan Backus}
\date{\today}

\newcommand{\NN}{\mathbf{N}}
\newcommand{\ZZ}{\mathbf{Z}}
\newcommand{\QQ}{\mathbf{Q}}
\newcommand{\RR}{\mathbf{R}}
\newcommand{\CC}{\mathbf{C}}
\newcommand{\DD}{\mathbf{D}}
\newcommand{\PP}{\mathbf P}
\newcommand{\MM}{\mathbf M}
\newcommand{\II}{\mathbf I}
\newcommand{\Hyp}{\mathbf H}
\newcommand{\Sph}{\mathbf S}
\newcommand{\Group}{\mathbf G}
\newcommand{\GL}{\mathbf{GL}}
\newcommand{\Orth}{\mathbf{O}}
\newcommand{\SpOrth}{\mathbf{SO}}
\newcommand{\Ball}{\mathbf{B}}

\newcommand*\dif{\mathop{}\!\mathrm{d}}

\DeclareMathOperator{\card}{card}
\DeclareMathOperator{\dist}{dist}
\DeclareMathOperator{\supp}{supp}
\DeclareMathOperator{\tr}{tr}

\newcommand{\Leaves}{\mathscr L}
\newcommand{\Lagrange}{\mathcal L}
\newcommand{\Hypspace}{\mathscr H}

\newcommand{\Chain}{\underline C}

\newcommand{\Two}{\mathrm{I\!I}}

\newcommand{\normal}{\mathbf n}
\newcommand{\radial}{\mathbf r}
\newcommand{\evect}{\mathbf e}
\newcommand{\vol}{\mathrm{vol}}

\newcommand{\diam}{\mathrm{diam}}
\newcommand{\Ell}{\mathrm{Ell}}
\newcommand{\inj}{\mathrm{inj}}
\newcommand{\Lip}{\mathrm{Lip}}
\newcommand{\Riem}{\mathrm{Riem}}

\newcommand{\Min}{\mathrm{Min}}
\newcommand{\Max}{\mathrm{Max}}

\newcommand{\dfn}[1]{\emph{#1}\index{#1}}

\renewcommand{\Re}{\operatorname{Re}}
\renewcommand{\Im}{\operatorname{Im}}

\newcommand{\loc}{\mathrm{loc}}
\newcommand{\cpt}{\mathrm{cpt}}

\def\Japan#1{\left \langle #1 \right \rangle}

\newtheorem{theorem}{Theorem}[section]
\newtheorem{badtheorem}[theorem]{``Theorem"}
\newtheorem{prop}[theorem]{Proposition}
\newtheorem{lemma}[theorem]{Lemma}
\newtheorem{sublemma}[theorem]{Sublemma}
\newtheorem{proposition}[theorem]{Proposition}
\newtheorem{corollary}[theorem]{Corollary}
\newtheorem{conjecture}[theorem]{Conjecture}
\newtheorem{axiom}[theorem]{Axiom}
\newtheorem{assumption}[theorem]{Assumption}

\newtheorem{mainthm}{Theorem}
\renewcommand{\themainthm}{\Alph{mainthm}}

% \newtheorem{claim}{Claim}[theorem]
% \renewcommand{\theclaim}{\thetheorem\Alph{claim}}
\newtheorem*{claim}{Claim}

\theoremstyle{definition}
\newtheorem{definition}[theorem]{Definition}
\newtheorem{remark}[theorem]{Remark}
\newtheorem{example}[theorem]{Example}
\newtheorem{notation}[theorem]{Notation}

\newtheorem{exercise}[theorem]{Discussion topic}
\newtheorem{homework}[theorem]{Homework}
\newtheorem{problem}[theorem]{Problem}

\makeatletter
\newcommand{\proofpart}[2]{%
  \par
  \addvspace{\medskipamount}%
  \noindent\emph{Part #1: #2.}
}
\makeatother



\numberwithin{equation}{section}


% Mean
\def\Xint#1{\mathchoice
{\XXint\displaystyle\textstyle{#1}}%
{\XXint\textstyle\scriptstyle{#1}}%
{\XXint\scriptstyle\scriptscriptstyle{#1}}%
{\XXint\scriptscriptstyle\scriptscriptstyle{#1}}%
\!\int}
\def\XXint#1#2#3{{\setbox0=\hbox{$#1{#2#3}{\int}$ }
\vcenter{\hbox{$#2#3$ }}\kern-.6\wd0}}
\def\ddashint{\Xint=}
\def\dashint{\Xint-}

\usepackage[backend=bibtex,style=alphabetic,giveninits=true]{biblatex}
\renewcommand*{\bibfont}{\normalfont\footnotesize}
\addbibresource{best_curl.bib}
\renewbibmacro{in:}{}
\DeclareFieldFormat{pages}{#1}

\newcommand\todo[1]{\textcolor{red}{TODO: #1}}


\begin{document}
\begin{abstract}
	Minimizers of the comass
\end{abstract}

\maketitle

%%%%%%%%%%%%%%%%%%%%%%%%%%%%%%%%%%%%%%%%%%%%%%%%%%%%%%%
\section{Introduction}
In this paper we study the problem of minimization of the \dfn{comass}
\begin{equation}\label{comass}
L(F) := \sup_{\sigma \in \Chain_2(M)} \frac{1}{|\sigma|} \int_\sigma F
\end{equation}
of a closed $2$-form in a Riemannian $3$-fold $M$, subject to a constraint on the cohomology of $F$.
Here $\Chain_2(M)$ denotes the space of oriented $2$-chains in $M$, and $|\sigma|$ is the area of $\sigma$.
This problem is the analogue on $3$-manifolds of the problem of finding a best Lipschitz map on a surface.

Sayeth Thurston \cite[Abstract]{Thurston98}:
\begin{quote}
I currently think that a characterization of minimal stretch maps should be possible in a considerably more general context ... and it should be feasible with a simpler proof based on more general principles -- in particular, the max flow min cut principle, convexity, and $L^0 \leftrightarrow L^\infty$ duality.
\end{quote}

Let $3 < p < \infty$ and $\frac{1}{p} + \frac{1}{q} = 1$.
We call a closed $2$-form \dfn{$p$-light} if it minimizes the $L^p$ norm, among all cohomologous forms.
We say that it has \dfn{best comass} if it minimizes the comass (\ref{comass}).
This problem is one of convex optimization, so it has a dual problem, namely the $q$-Laplacian.
Our first main theorem is a combination of Propositions \ref{existence infinity} and \ref{existence 1}.

\begin{mainthm}
Let $\rho \in H^2(M, \RR)$ be a cohomology class.
Let $(F_p, u_q)$ be the family of dual pairs of $p$-light forms and $q$-harmonic functions, suitably normalized, with $[F_p] = \rho$.
Then there exists a pair $(F, u)$ such that as $p \to \infty$, $F_p \to F$ weakly in $L^r$ for any $3 < r < \infty$, and $u_q \to u$ weakly in $BV$.
Moreover, $F$ has best comass in $\rho$, $u$ is $1$-harmonic, and we have the duality relation 
\begin{equation}\label{max flow mean cut}
L|\dif u| = \langle \dif u, \star F\rangle,
\end{equation}
$|\dif u|$-almost everywhere, where $L$ is the best comass constant.
\end{mainthm}

We call $F$ an \dfn{$\infty$-light form}.
Since (\ref{max flow mean cut}) asserts a form of convex duality between $\infty$-light forms and $1$-harmonic functions, we view it as the analogue of the max flow min cut principle alluded to by Thurston.

A surface $N \subseteq M$ is \dfn{calibrated} by a closed $2$-form $F$ if $\|F\|_{L^\infty} \leq 1$ and the pullback of $F$ to $N$ is the area form on $N$ \cite{Harvey82}.
In that case, the mean curvature of $N$ is 
\begin{equation}\label{calibrated surfaces are minimal}
H_N = \nabla \cdot \normal_N = \nabla \cdot (\star F)^\sharp = \star \dif F = 0,
\end{equation}
so that $N$ is minimal. 
In particular, if a lamination $\lambda$ is calibrated by $F$ (in the sense that every leaf of $\lambda$ is calibrated), then $\lambda$ is minimal.

If $F$ calibrates a surface $N$, then it is clear from the definitions that $N$ is contained in the locus on which $F$ attains its comass.
If $F$ is a continuous form of best comass, then $F/L$ calibrates a minimal lamination \cite{bangert_cui_2017}.
On the other hand, every $1$-harmonic function $u$ defines a minimal lamination $\lambda$ with Ruelle-Sullivan current $\dif u$ \todo{Cite me}.
The lamination $\lambda$ induced by the $1$-harmonic conjugate of an $\infty$-light form $F$ plays the role of Thurston's canonical lamination \cite{Thurston98}, so we call it a \dfn{Thurston lamination} associated to $[F]$.
Our next theorem, Theorem \ref{MCL contains Thurston}, is similar to \cite[Theorem 5.1]{bangert_cui_2017}, but it does not require that $F$ is continuous,shows that the lamination $\lambda$ calibrated by a form of best comass $F$ as only dependent on the cohomology class $[F]$, and identifies $\lambda$ as a familiar object that was already studied in \cite{daskalopoulos2020transverse}.

\begin{mainthm}
Let $\lambda$ be a Thurston lamination associated to $\rho \in H^2(M, \RR)$, and let $F$ be a form of best comass representing $\rho$.
Then $F/L$ calibrates $\lambda$. 
\end{mainthm}

For a measured oriented lamination $\lambda$, we let $[\lambda] \in H_2(M, \RR)$ be the homology class of the associated Ruelle-Sullivan current, and $|\lambda|$ its area -- that is, the mass of the current.
The next theorem is Theorem \ref{L equals K}, and is an analogue of the $L = K$ theorem of Thurston.

\begin{mainthm}
Let $\rho \in H^2(M, \RR)$, let $L$ be the best comass constant of $\rho$, and let $K$ be the supremum of $\langle \rho, [\lambda]\rangle/|\lambda|$, taken over all measured oriented laminations $\lambda$.
Then $L = K$, and the supremum in $K$ is attained by any Thurston lamination for $\rho$.
\end{mainthm}

%%%%%%%%%%%%%%%%%%%%%
\subsection{Outline of the paper}

%%%%%%%%%%%%%%%%%%%%%
\subsection{Notation}
We write $\Omega^\ell$, $Z^\ell$, and $B^\ell$ for the spaces of $\ell$-forms, closed $\ell$-forms ($\ell$-cocycles), and exact $\ell$-forms ($\ell$-coboundaries) respectively.
We reserve $H^\ell$ for cohomology and write $W^{s, p}$ for Sobolev spaces.
 
%%%%%%%%%%%%%%%%%%%%%%
\subsection{Acknowledgements}
George, Karen, Tom Goodwillie, Kaya Ferendo ...

This research was supported by the National Science Foundation's Graduate Research Fellowship Program under Grant No. DGE-2040433.


% \subsection{Normal trace theorem}
% We follow \cite[Chapter I, Theorem 1.2]{temam2016navier}, which handles the $p = 2$ case of the normal trace theorem.

% \begin{proposition}[normal trace theorem]
% Let $\iota: \Sph^{d - 1} \to \Ball^d$ be the inclusion map.
% The pullback is a bounded linear operator 
% $$\iota^*: L^\infty(\Ball^d, Z^{d - 1}) \to L^\infty(\Sph^{d - 1}, Z^{d - 1}).$$
% In particular, for every $\alpha \in L^\infty(\Ball^d, Z^{d - 1})$ and every $f \in W^{1, 1}(\Ball^d)$, we have integration by parts:
% \begin{equation}\label{Stokes trace}
% 	\int_{\Sph^{d - 1}} f\alpha = \int_{\Ball^d} \dif f \wedge \alpha.
% \end{equation}
% \end{proposition}
% \begin{proof}
% By the inverse trace theorem \cite[Teorema 1.II]{Gagliardo1957}, there exists a (possibly nonlinear and discontinuous) right inverse $T$ of the trace map $W^{1, 1}(\Ball^d) \to L^1(\Sph^{d - 1})$ which satisfies 
% $$\|Tf\|_{W^{1, 1}(\Ball^d)} \lesssim \|f\|_{L^1(\Sph^{d - 1})}.$$
% We use $T$ to formally define an $\ell$-current $\iota^* \alpha$ on $\Sph^{d - 1}$ by setting, for every $f \in C^\infty(\Sph^{d - 1})$,
% $$\int_{\Sph^{d - 1}} f\iota^* \alpha := \int_{\Ball^d} \dif(Tf) \wedge \alpha.$$
% To show that this current is well-defined, it suffices to show that for any $f$, 
% $$\left|\int_{\Sph^{d - 1}} f\iota^* \alpha\right| \lesssim_\alpha \|f\|_{L^1(\Sph^{d - 1})}.$$
% In fact, 
% \begin{align*}
% \left|\int_{\Sph^{d - 1}} f \iota^*\alpha\right|
% &= \left|\int_{\Ball^d} \dif(Tf) \wedge \alpha\right| \leq \|Tf\|_{W^{1, 1}(\Ball^d)} \|\alpha\|_{L^\infty(\Ball^d)} \\
% &\lesssim \|f\|_{L^1(\Sph^{d - 1})} \|\alpha\|_{L^\infty(\Ball^d)}.
% \end{align*}
% It follows that $\iota^* \alpha$ is well-defined.
% Moreover, by Stokes' theorem, that $\iota^* \alpha$ agrees with the usual definition of pullback on continuous forms $\alpha \in C^0(M, Z^{d - 1})$.
% Finally, we estimate 
% \begin{align*}
% \|\iota^* \alpha\|_{L^\infty(\Sph^{d - 1})}
% &= \sup_{\|f\|_{L^1(\Sph^{d - 1})} = 1} \int_{\Sph^{d - 1}} f \iota^*\alpha \\
% &\lesssim \sup_{\|f\|_{L^1(\Sph^{d - 1})} = 1} \|f\|_{L^1(\Sph^{d - 1})} \|\alpha\|_{L^\infty(\Ball^d)} \\
% &= \|\alpha\|_{L^\infty(\Ball^d)}. \qedhere 
% \end{align*}
% \end{proof}

% \begin{corollary}\label{trace on cycles}
% Let $N$ be a smooth embedded hypersurface in $M$ and $1 < p < \infty$.
% \begin{enumerate}
% \item \label{pullback bounded} The pullback is a bounded linear operator
% $$\iota^*_N: L^\infty(M, Z^{d - 1}) \to L^\infty(N, Z^{d - 1}).$$
% \item \label{integral continuous} For $F \in L^\infty(M, Z^{d - 1})$, and $N$ closed, the integral $\int_N F$ is well-defined, and depends continuously on $F$ for the weak topology on $L^p$.
% \item \label{cohomology exists} For $F \in L^\infty(M, Z^{d - 1})$, the cohomology class of $F$ is well-defined, and depends continuously on $F$ for the weak topology on $L^p$.
% \end{enumerate}
% \end{corollary}
% \begin{proof}
% To prove (\ref{pullback bounded}) we may use a partition of unity to work in a small ball $U$ in $N$, and then we may realize a collar neighborhood $V$ of $U$ in $M$ as a manifold-with-boundary with $U \subseteq \partial V$, and choose $\chi \in C^\infty_\cpt(M)$ in (\ref{Stokes trace}) to be a cutoff which is zero on $\partial V$ except along $U$.
% The definition of
% $$\iota^*_U: L^\infty(M, Z^{d - 1}) \to L^\infty(U, \Omega^\ell)$$
% does not depend on the choice of $V$, since by Stokes' theorem the right-hand side of (\ref{Stokes trace}) will be the same.

% To obtain (\ref{integral continuous}), we first use the fact that $N$ is closed to obtain $L^\infty(N, Z^{d - 1}) \subseteq L^1(N, Z^{d - 1})$.
% To obtain the continuity, we again use a collar neighborhood $V$ of $U$ and a cutoff $\chi$.
% Then (\ref{Stokes trace}) reads 
% $$\int_U F = \int_M F \wedge \dif \chi.$$
% Since $\chi \in C^\infty_\cpt$, $\dif \chi \in L^q$ where $\frac{1}{p} + \frac{1}{q} = 1$, so if $F_n \to F$ in the weak topology on $L^p$, then $\int_M F_n \wedge \dif \chi \to \int_M F \wedge \dif \chi$, as desired.
% Letting $N$ range over representatives of every homology class, we conclude (\ref{cohomology exists}) as a consequence of (\ref{integral continuous}).
% \end{proof}


%%%%%%%%%%%%%%%%%%%%%%%%%%%%%%%%%%%%%%%%%
\section{Convex duality for the \texorpdfstring{$q$-Laplacian}{q-Laplacian}}

Let us, for a reflexive Banach space $X$, denote by $\hat X$ its dual.
If $J: X \to \RR \cup \{+\infty\}$ is a convex function, we introduce its \dfn{Legendre transform}
\begin{align*}
	\hat J: \hat X &\to \RR \cup \{+\infty\}\\
	\xi &\mapsto \sup_{x \in X} \langle \xi, x\rangle - f(x).
\end{align*}

\begin{definition}
Let $Y$ be a reflexive Banach space equipped with a mapping $\Lambda: X \to Y$.
A function $J: Y \to \RR \cup \{+\infty\}$ is said to be a \dfn{suitable convex function} for our purposes if:
\begin{enumerate}
\item $J$ is strictly convex,
\item $J$ is lower semicontinuous and not identically $+\infty$,
\item if $|y| \to \infty$ in $Y$, then $J(y) \to +\infty$, and 
\item there exists a point $x \in X$ such that $J$ is continuous and finite at $\Lambda(x)$.
\end{enumerate}
\end{definition}

\begin{proposition}\label{abstract convex analysis}
Let $X, Y$ be reflexive Banach spaces equipped with a bounded linear operator $\Lambda: X \to Y$ of trivial kernel.
Let $J: Y \to \RR \cup \{+\infty\}$ be a suitable convex function.
Then:
\begin{enumerate}
\item There exists a unique minimizer $\underline x \in X$ of $J(\Lambda(x))$, and a maximizer $\overline \eta \in \hat Y$ of $-\hat J(-\eta)$.
\item If $\hat J$ is strictly convex, then $\overline \eta$ is the unique maximizer of $-\hat J(-\eta)$.
\item We have \dfn{strong duality}
\begin{equation}\label{abstract strong duality}
J(\Lambda(\underline x)) = -\hat J(-\overline \eta).
\end{equation}
\end{enumerate}
\end{proposition}
\begin{proof}
This is the conjunction of several standard results in convex analysis.
Let $\Gamma_0(Y)$ denote the set introduced in \cite[Chapter I, Definition 3.1]{Ekeland99}.
Then by \cite[Chapter I, Proposition 3.1]{Ekeland99}, $J \in \Gamma_0(Y)$, so by \cite[Chapter III, Theorem 4.2]{Ekeland99} and \cite[Chapter III, (4.23)]{Ekeland99}, there exist minimizers which satisfy (\ref{abstract strong duality}).
By \cite[Chapter II, Proposition 1.2]{Ekeland99}, the minimizers of $J$ and $\hat J$ are unique.
Since $\Lambda$ has trivial kernel, minimizers of $J \circ \Lambda$ are also unique.
\end{proof}

We now specialize to the problem at hand. \todo{Move to an appendix, have cheap versions of the same theorems in the body. In the body make sure to spell out the first and second variations of $p$-light, and the fact that the $p$-light equation makes sense as distributions, and that $u \in W^{1, q}$ is equivalent to $F \in L^p$}
Let $\Pi: \tilde M \to M$ be the universal covering, $M_{\rm fun} \subseteq \tilde M$ a fundamental domain, and
$$\alpha \in H^1(M, \RR)$$
a cohomology class.
Since $H_1(M, \RR)$ is the abelianization of $\pi_1(M)$, $\alpha$ is canonically identified with a representation of the fundamental group, which we also call
$$\alpha: \pi_1(M) \to \RR.$$
If a function $u: \tilde M \to \RR$ is $\alpha$-equivariant, we write $[u] = \alpha$.

We here consider the problem
\begin{equation}\label{preprimal problem}
	\Min\{\|\dif u\|_{L^q(M_{\rm fun})}: u \in W^{1, q}(M_{\rm fun}), [u] = \alpha\},
\end{equation}
where $1 < q < \infty$; we identify two solutions if they agree by a constant.
Taking Euler-Lagrange equations, we see that (\ref{preprimal problem}) is equivalent to the $q$-Laplacian 
\begin{equation}\label{qLaplace}
\begin{cases}
	\dif^*(|\dif u|^{q - 2} \dif u) = 0 \\
	[u] = \alpha.
\end{cases}
\end{equation}

To put (\ref{preprimal problem}) in the framework of Proposition \ref{abstract convex analysis}, we shall fix a point $0 \in M_{\rm fun}$, choose a representative $1$-form (which we also call $\alpha$), and solve 
$$\begin{cases}
\dif f = \Pi^* \alpha \\
v(0) = 0.
\end{cases}$$
Thus (\ref{preprimal problem}) is equivalent to
\begin{equation}\label{primal problem}
	\Min\left\{\frac{1}{q} \int_{M_{\rm fun}} \star|\dif v + \Pi^* \alpha|^q: v \in W^{1, q}_0(M_{\rm fun})\right\}
\end{equation}
where we set $u = v + f$.

Let
$$J(\xi) := \frac{1}{q} \int_M \star|\xi + \alpha|^q,$$
defined for $\xi \in L^q(M, \Omega^1)$.
Since $v$ is traceless in (\ref{primal problem}), it is invariant, so $\dif v$ drops to a $1$-form on $M$, and (\ref{primal problem}) is the problem of minimizing $J(\dif v)$.
It is clear that $J$ is a suitable convex function on $L^q(M, \Omega^1)$ when it is equipped with the map
\begin{equation}\label{derivative on traceless}
\dif: W^{1, q}_0(M_{\rm fun}) \to L^q(M, \Omega^1),
\end{equation}
and moreover the kernel of (\ref{derivative on traceless}) is trivial.
By \cite[Chapter I, (4.9)]{Ekeland99} and \cite[Chapter I, Remark 4.1]{Ekeland99}, the Legendre transform
$$\hat J: L^p(M, \Omega^{d - 1}) \to \RR$$
of $J$, where $\frac{1}{p} + \frac{1}{q} = 1$, satisfies
\begin{equation}\label{Legendre transform}
\hat J(F) = \frac{1}{p} \int_M \star |F|^p + \int_M \alpha \wedge F
\end{equation}
and in particular is strictly convex.
Thus the convex dual problem of (\ref{primal problem}), namely the problem of maximizing $-\hat J(-F)$, is the problem
\begin{equation}\label{predual problem}
\Max\left\{- \frac{1}{p} \int_M \star |F|^p + \int_M \alpha \wedge F: F \in L^p(M, \Omega^{d - 1})\right\}.
\end{equation}

% \begin{lemma}\label{EulerLagrange}
% A form $F \in L^p(M, \Omega^{d - 1})$ is a solution of (\ref{predual problem}) iff
% \begin{equation}\label{EL of hat G}
% |F|^{p - 2} F = (-1)^d \star \alpha.
% \end{equation}
% \end{lemma}
% \begin{proof}
% Let $(F_t)$ be an arbitrary variation and $G := \frac{\partial F_t}{\partial t}|_{t = 0}$. Then 
% $$0 = \frac{\dif}{\dif t} \hat f(F_t)\bigg|_{t = 0} = -\int_M \star |F|^{p - 2} \langle F, G \rangle + \alpha \wedge G.$$
% In particular, for any $G$,
% $$|F|^{p - 2} \langle F, G\rangle = -\star^{-1}(\alpha \wedge G) = \langle -\star^{-1} \alpha, G\rangle.$$
% Recalling that, since $\alpha$ is a $1$-form, $-\star^{-1} \alpha = (-1)^d \star \alpha$, and $G$ is arbitrary, it follows that (\ref{predual problem}) implies (\ref{EL of hat G}).
% The converse follows when one realizes that (\ref{EL of hat G}) has only one solution (this follows from some simple algebra).
% \end{proof}

\begin{proposition}\label{convex duality}
Let $1 < p, q < \infty$ satisfy $\frac{1}{p} + \frac{1}{q} = 1$.
Then we have an isomorphism of cohomology groups
\begin{align*}
H^1(M, \RR) &\to H^{d - 1}(M, \RR) \\
\alpha &\mapsto \rho,
\end{align*}
so that the convex dual problem of the $\alpha$-equivariant $q$-Laplacian (\ref{qLaplace}) with solution $u: \tilde M \to \RR$ is
\begin{equation}\label{pMaxwell}
\begin{cases}
	\dif F = 0, \\
	\dif^*(|F|^{p - 2} F) = 0, \\
	[F] = \rho.
\end{cases}
\end{equation}
The problems (\ref{qLaplace}) and (\ref{pMaxwell}) both have unique solutions, and we have the relations
\begin{align}
F &= |\dif u|^{q - 2} \star \dif u, \label{extremality} \\
\dif u &= (-1)^{d - 1} |F|^{p - 2} \star F, \label{inverse extremality}
\end{align}
and the strong duality theorem 
\begin{equation}\label{strong duality}
\frac{1}{q} \int_M \star |\dif u|^q + \frac{1}{p} \int_M \star |F|^p = \int_M \dif u \wedge F.
\end{equation}
\end{proposition}
\begin{proof}
Since $J$ is a suitable convex function, $\hat J$ is strictly convex, and (\ref{derivative on traceless}) has trivial kernel, we may apply Proposition \ref{abstract convex analysis} to see that (\ref{qLaplace}) and (\ref{predual problem}) both have unique solutions $u, F$ which are related by strong duality (\ref{abstract strong duality}) as (\ref{strong duality}).
By uniqueness, if we exhibit $\tilde F$ defined in terms of $u$ which satisfies (\ref{strong duality}) and is a maximizer of $-\hat J$, then $F = \tilde F$.
Following \cite[Chapter IV, (2.12)]{Ekeland99}, we define $\tilde F$ to satisfy (\ref{extremality}), so that 
\begin{align*}
\int_M \dif u \wedge \tilde F 
&= \int_M \star |\dif u|^q \\
&= \frac{1}{q} \int_M \star |\dif u|^q + \frac{1}{p} \int_M \star |\dif u|^p \\
&= \frac{1}{q} \int_M \star |\dif u|^q + \frac{1}{p} \int_M \dif u \wedge \tilde F.
\end{align*}
Moreover, if we decompose 
$$\dif u = \dif v + \alpha$$
then
\begin{align*}
-\hat J(\tilde F)
&= -\frac{1}{p} \int_M \star |\dif u|^q + \int_M \alpha \wedge |\dif u|^{q - 2} \star \dif u \\
&= -\frac{1}{p} \int_M \star |\dif u|^q + \int_M \star |\dif u|^q - \int_M \dif v \wedge |\dif u|^{q - 2} \star \dif u \\
&= \frac{1}{q} \int_M \star |\dif u|^q + \int_M v \dif(|\dif u|^{q - 2} \star \dif u) \\
&= \frac{1}{q} \int_M \star |\dif u|^q = J(\dif u)
\end{align*}
where we used the fact that $\frac{1}{q} = 1 - \frac{1}{p}$, and the fact that $u$ solves the $q$-Laplace equation.
By strong duality (\ref{abstract strong duality}), it follows that $\tilde F$ is a maximizer of $-\hat J$, and therefore $F = \tilde F$; in particular, $F$ satisfies (\ref{extremality}).
Since $u$ solves the $q$-Laplace equation, (\ref{extremality}) implies that $\dif F = 0$.
We then define $\rho := [F]$.
On the other hand, 
$$(p - 2)(q - 1) + q - 2 = 0,$$
so following \cite[Lemma 3.2]{daskalopoulos2020transverse},
$$|F|^{p - 2} F = |\dif u|^{(p - 2)(q - 1) + q - 2} \star \star \dif u = (-1)^{d - 1} \dif u$$
which gives (\ref{inverse extremality}) and the equation
$$\dif^*(|F|^{p - 2} F) = 0.$$
By reflexivity, the above argument may also be reversed, so that the map $\alpha \mapsto \rho$ is invertible on the level of cohomology.
\end{proof}

\begin{definition}
If $u$ is a solution of the $q$-Laplacian, $\frac{1}{p} + \frac{1}{q} = 1$, and $F$ satisfies (\ref{extremality}), we call $u$ the \dfn{conjugate $q$-harmonic} of $F$.
\end{definition}

If $u_q$ is a conjugate $q$-harmonic for $F_p$ for every $p \gg 1$ and $\frac{1}{p} + \frac{1}{q} = 1$, it may be that in a suitable function space we may take the limit $p \to \infty$ to obtain a $d-1$-form $F$ and a function $u$.
In this situation, we still call $u$ a conjugate $1$-harmonic, though this is only literally true in a formal sense.
In particular, since $L^1, L^\infty$ are not reflexive Banach spaces, we do not have uniqueness or an isomorphism theorem -- at least not with the methods we have considered here.

\todo{Cite those glaciology papers which already include this duality}

%%%%%%%%%%%%%%%%%%%%%%%%%%%%%%%%%%%%%%%%%%

\section{Best comass and \texorpdfstring{$\infty$-light forms}{infinity-light forms}}
%\subsection{\texorpdfstring{$p$-light forms}{p-light forms}}

%%%%%%%%%%%%%%%%%%%%%%%%%%%%
\subsection{Forms of best comass}
What follows is the main definition of this paper.
For a closed $2$-form $F$ and a subdomain $\Omega$, we introduce the \dfn{comass} 
$$L_\Omega(F) := \sup_{\sigma \in \Chain_2(\Omega)} \frac{1}{|\sigma|} \int_\sigma F,$$
and let $L(F) := L_M(F)$.

\begin{proposition}
Let $\Delta$ be the standard $d$-simplex, $\tau$ a $d-1$-face of $\Delta$, and $d < p < \infty$.
Then $F \mapsto \int_\tau F$ extends to a continuous linear functional on $L^p(\Delta, Z^{d - 1})$.
Moreover,
\begin{equation}\label{integral over chain is linfinity}
	\frac{1}{|\tau|} \int_\tau F \leq \|F\|_{L^\infty}.
\end{equation}
\end{proposition}
\begin{proof}
We first show that $F \mapsto \int_\tau F$ is continuous for the $L^p$ norm on smooth closed $d-1$-forms.
Let $F, F' \in C^\infty(\Delta, Z^{d - 1})$.
Since $\Delta$ is contractible, there exist $A, A'$ such that $\dif A = F$ and $\dif A' = F'$, and such that $A, A'$ are in Coulomb gauge (that is, coclosed).
By elliptic regularity and Sobolev embedding ($p > d$),
$$\|A - A'\|_{C^0} \lesssim \|\nabla(A - A')\|_{L^p} \lesssim \|F - F'\|_{L^p}.$$
So by Stokes' theorem,
$$\left|\int_\tau F - F'\right| = \left|\int_{\partial \tau} A - A'\right| \lesssim \|A - A'\|_{C^0} \lesssim \|F - F'\|_{L^p}.$$
By a mollification argument, $F \mapsto \int_\tau F$ is a continuous linear functional on $L^p(\Delta, Z^{d - 1})$.

Suppose now that $F \in L^\infty$ and choose a sequence of mollifiers $\chi_\varepsilon$ such that $\|\chi_\varepsilon\|_{L^1} = 1$.
Let $F_\varepsilon := F * \chi_\varepsilon$, so that $F_\varepsilon \to F$ in $L^p$ for any $d < p < \infty$.
By Young's inequality, 
$$\|F_\varepsilon\|_{C^0} \leq \|F\|_{L^\infty} \|\chi_\varepsilon\|_{L^1} \leq \|F\|_{L^\infty}.$$
Taking $\varepsilon \to 0$ we see that
\begin{align*}
\frac{1}{|\tau|} \int_\tau F 
&= \lim_{\varepsilon \to 0} \frac{1}{|\tau|} \int_\tau F_\varepsilon \leq \lim_{\varepsilon \to 0} \|F_\varepsilon\|_{C^0} \leq \|F\|_{L^\infty}. \qedhere
\end{align*}
\end{proof}

In particular, since any $2$-chain $\sigma$ can be written as the sums of $2$-simplices $\tau$ which can then be viewed as faces of contractible $d$-simplices $\Delta$, the integral in the definition of comass is well-defined as long as $F \in L^p$, $p > 3$, and each such integral depends continuously on $F$ in $L^p$.

\begin{definition}
	Let $\rho \in H^2(M, \RR)$ and let $F$ be a closed $2$-form representing $\rho$.
\begin{enumerate}
	\item We say that $F$ has \dfn{best comass} if $F$ is a minimizer of $L$ among all $F$ representing $\rho$.
	\item We say that $F$ has \dfn{absolutely best comass} if it has best comass, and for every small open ball $\Omega$, $F$ is a minimizer of $L_\Omega$ among all closed $2$-forms $B$ with $B|_{\partial \Omega} = F|_{\partial \Omega}$.
\end{enumerate}
\end{definition}

We will be interested in the points at which $F$ attains its comass.
However, $F \in L^\infty$, so $F$ is both only defined almost everywhere, and not norm-approximable by smooth functions.
So as a proxy for $|F|$, which may fail to be defined on a null set, we use the local comass, which is defined everywhere.

\begin{definition}
The \dfn{local comass} of a closed $2$-form $F$ at $x \in M$ is 
$$L(F, x) := \limsup_{\varepsilon \to 0} \sup_{\sigma \in \Chain_2(B_\varepsilon(x))} \frac{1}{|\sigma|} \int_\sigma F.$$
\end{definition}

One has
$$L(F, x) = \limsup_{\varepsilon \to 0} L_{B_\varepsilon(x)}(F)$$
but $L_{B_\varepsilon(x)}(F)$ is increasing in $\varepsilon$ (since it's a supremum over a set which grows in $\varepsilon$).
So the limit superior is actually a limit and an infimum:
$$L(F, x) = \lim_{\varepsilon \to 0} L_{B_\varepsilon(x)}(F) = \inf_{\varepsilon > 0} L_{B_\varepsilon(x)}(F)$$
and in particular $L(F, x) \leq L(F)$. Thus we have the following analogue of \cite[Lemma 4.3(a)]{Crandall2008}:

\begin{proposition}
For $F \in L^\infty(M, Z^2)$, the local comass $L(F, \cdot)$ is upper semicontinuous. \label{crandall usc}
\end{proposition}
\begin{proof}
Let $x^n \to x$ and $r > 0$. Then eventually $x^n \in B_r(x)$, hence $L(F, x^n) \leq L_{B_r(x)}(F)$ and so
\begin{align*}
\limsup_{n \to \infty} L(F, x^n) &\leq \inf_{r > 0} L_{B_r(x)}(F) = L(F, x). \qedhere 
\end{align*}
\end{proof}

It's convenient to test the local comass against a more restrictive class of $2$-chains.
For a tangent plane $S \subseteq T_x M$ and $\varepsilon > 0$, let
$$P_{S, \varepsilon} := (\exp_x)_*(S) \cap B_\varepsilon(x).$$
Following the terminology of lattice gauge theory (see for example \cite{Gupta98}), we refer to $P_{S, \varepsilon}$ as a \dfn{plaquette} at scale $\varepsilon$.

The utility of plaquettes is that, for a covector $\xi \in T_x' M$, we obtain a foliation of the tangent space $T_x M$ into planes, namely $(\ker \xi + t\xi^\sharp)_{t \in \RR}$.
Pushing forward this foliation by the exponential map, we obtain a foliation
$$\lambda_{\xi, \varepsilon} := (P_{\ker \xi + t\xi^\sharp, \varepsilon})_{t \in \RR}$$
of $B_\varepsilon(x)$ into plaquettes which are approximately conormal to $\xi$ in the limit $\varepsilon \to 0$.
Using this foliation and Fubini's theorem, we can turn integrals of $F$ over surfaces, which are Lebesgue null, into integrals over balls, which satisfy the Lebesgue differentiation theorem.
From these considerations and the Hardy-Littlewood maximal inequality, we deduce that plaquettes satisfy their own version of the Lebesgue differentiation theorem, at least on average.

\begin{lemma}[plaquette Lebesgue differentation theorem]\label{PLDT}
Let $F \in L^\infty(M, Z^2)$.
Then for almost every $x \in M$ there exists a unit covector $\xi \in T_x' M$, which depends measurably on $x$, such that
\begin{equation}\label{plaquette LDT}
|F(x)| = \lim_{\varepsilon \to 0} \frac{1}{|B_\varepsilon(x)|} \int_{-\infty}^\infty \left[\int_{P_{\ker \xi + t\xi^\sharp, \varepsilon}} F\right] \dif t
\end{equation}
where $\dif t$ is the disintegration of the Riemannian measure with respect to $\lambda_{\eta, \varepsilon}$.
\end{lemma}

We defer the proof of Lemma \ref{PLDT} to \S\ref{proof of PLDT}, as it is rather technical.
As a consequence of the plaquette Lebesgue differentiation theorem, we obtain an analogue for the comass of a familiar result about the Lipschitz seminorm \cite[Lemma 4.3]{Crandall2008}.

\begin{proposition}\label{crandall}
Let $F \in L^\infty(M, Z^2)$. Then:
\begin{enumerate}
% \item If $F(x)$ exists then $L(F, x) \geq |\dif A(x)|$. \label{crandall dA bounds LA}
% \item If $L(A, x) = 0$, then $\dif A(x)$ exists and $\dif A(x) = 0$. \label{crandall zero LA implies diffble}
\item For almost every $x \in M$,
$$|F(x)| \leq L(F, x).$$
\label{crandall LDT}
\item The local comass is bounded, and \label{crandall linfinity}
$$L(F) = \max_{x \in M} L(F, x) = \|F\|_{L^\infty}.$$
% \item Let $\iota^* F$ be the pullback of $F$ to a $2$-chain $\sigma$. Then
% $$\|\iota^* F\|_{L^\infty(\sigma)} \leq \|F\|_{L^\infty}.$$
% \label{crandall normal trace is contraction}
% \item If $\sigma \in \Chain_2$ then \label{crandall best curl is ABC}
% $$\frac{1}{|\sigma|} \int_\sigma F \leq \max_{x \in \sigma} L(F, x).$$
\end{enumerate}
\end{proposition}
\begin{proof}
% We now bound for a net of plaquettes $R_{ij}^\varepsilon(x)$ using (\ref{riemann plaquette})
% \begin{equation}\label{difference quotients}
% 	\limsup_{\varepsilon \to 0} \frac{|A_j(x + \varepsilon \partial_i) - A_j(x) - A_i(x + \varepsilon \partial_j) + A_i(x)|}{\varepsilon}
% = \limsup_{\varepsilon \to 0} \frac{1}{|R_{ij}^\varepsilon(x)|} \left|\int_{\partial R_{ij}^\varepsilon(x)} A\right| \leq L(A, x).
% \end{equation}
% If the first limit superior is actually a limit, then it is the definition of $|\dif A_{ij}(x)|$.
% So if $\dif A_{ij}(x)$ exists we conclude $|\dif A_{ij}(x)| \leq L(A, x)$, which proves (\ref{crandall dA bounds LA}).
% On the other hand, the corresponding limit \emph{inferior} must be nonnegative, so if $L(A, x) = 0$, the first limit superior in (\ref{difference quotients}) is actually a limit and we obtain $\dif A_{ij}(x) = 0$, proving (\ref{crandall zero LA implies diffble}).

% Now let $\sigma$ be a $2$-chain and $\varepsilon > 0$.
% Then we may write $\sigma = \sum_{n=1}^N \sigma_n(\varepsilon)$ where $\sigma_n(\varepsilon)$ is a $2$-simplex such that $\sigma_n(\varepsilon) \in \Chain_2(B_\varepsilon(x_n(\varepsilon)))$ for some $x_n(\varepsilon) \in \sigma$, and the $\sigma_n(\varepsilon)$ are almost disjoint.
% Then 
% $$\frac{1}{|\sigma|} \left|\int_\sigma F\right| \leq \sum_{n=1}^{N_\varepsilon} \frac{|\sigma_n(\varepsilon)|}{|\sigma|} \frac{1}{|\sigma_n(\varepsilon)|} \left|\int_{\sigma_n(\varepsilon)} F\right| \leq \sum_{n=1}^{N_\varepsilon} \frac{|\sigma_n(\varepsilon)|}{|\sigma|} L_{B_{\varepsilon}(x_n(\varepsilon))}(F).$$
% Since this inequality is true for every $\varepsilon$, it remains true in the limit superior as $\varepsilon \to 0$:
% $$\frac{1}{|\sigma|} \left|\int_\sigma F\right| \leq \limsup_{\varepsilon \to 0} \sum_{n=1}^{N_\varepsilon} \frac{|\sigma_n(\varepsilon)|}{|\sigma|} L_{B_{\varepsilon}(x_n(\varepsilon))}(F).$$
% Let $(y_m(\varepsilon))_{m \in \NN}$ be a maximizing sequence for $L_{B_\varepsilon(\cdot)}(F)$ subject to $y_m(\varepsilon) \in \sigma$.
% Such a sequence exists, since we trivially have
% $$\sup_{x \in \sigma} L_{B_\varepsilon(x)}(F) \leq \|F\|_{L^\infty} < \infty.$$
% After passing to a subsequence in $m$, we may assume that $y_m(\varepsilon) \to y(\varepsilon)$ for some $y(\varepsilon) \in \sigma$.
% Then for $m$ large depending on $\varepsilon$, $y_m(\varepsilon) \in B_{2\varepsilon}(y(\varepsilon))$, and it follows that
% \begin{align*}
% \limsup_{\varepsilon \to 0} \sum_{n=1}^{N_\varepsilon} \frac{|\sigma_n(\varepsilon)|}{|\sigma|} L_{B_{\varepsilon}(x_n(\varepsilon))}(F)
% &\leq \limsup_{\varepsilon \to 0} \sum_{n=1}^{N_\varepsilon} \frac{|\sigma_n(\varepsilon)|}{|\sigma|} \lim_{m \to \infty} L_{B_\varepsilon(y_m(\varepsilon))}(F) \\
% &\leq \limsup_{\varepsilon \to 0} \lim_{m \to \infty} L_{B_\varepsilon(y_m(\varepsilon))}(F) \\
% &\leq \limsup_{\varepsilon \to 0} L_{B_{2\varepsilon}(y(\varepsilon))}(F).
% \end{align*}
% We then pass to a subsequence in $\varepsilon$ along which the limit superior is obtained, and then a further subsequence along which $y(\varepsilon) \to y$ for some $y \in \sigma$.
% Let $\delta > 0$; then $B_{2\varepsilon}(y(\varepsilon)) \subseteq B_\delta(y)$ for any sufficiently small $\varepsilon$, hence 
% $$\limsup_{\varepsilon \to 0} L_{B_{2\varepsilon}(y(\varepsilon))}(F) \leq L_{B_\delta(y)}(F).$$
% Plugging everything in and taking a limit superior in $\delta$,
% $$\frac{1}{|\sigma|} \left|\int_\sigma F\right| \leq \limsup_{\delta \to 0} L_{B_\delta(y)}(F) = L(F, y),$$
% implying (\ref{crandall best curl is ABC}).

To prove (\ref{crandall LDT}), we bound for almost every $x \in M$ using the plaquette Lebesgue differentiation theorem
\begin{align*}
|F(x)|
&= \lim_{\varepsilon \to 0} \frac{1}{|B_\varepsilon(x)|} \int_{-\infty}^\infty \int_{P_{\ker \xi + t\xi^\sharp, \varepsilon}} F \dif t \\
&\leq \lim_{\varepsilon \to 0} L_{B_\varepsilon(x)}(F) \frac{1}{|B_\varepsilon(x)|} \int_{-\infty}^\infty |P_{\ker \xi + t\xi^\sharp, \varepsilon}| \dif t.
\end{align*}
Since $\dif t$ is the disintegration of the Riemannian measure with respect to the foliation $\lambda_{\xi, \varepsilon}$ of $B_\varepsilon(x)$,  it follows that 
$$\int_{-\infty}^\infty |P_{\ker \xi + t\xi^\sharp, \varepsilon}| \dif t = |B_\varepsilon(x)|.$$
Therefore
$$|F(x)| \leq \lim_{\varepsilon \to 0} L_{B_\varepsilon(x)}(F) = L(F, x).$$

Next we bound using (\ref{integral over chain is linfinity}) and (\ref{crandall LDT})
$$\sup_{x \in M} L(F, x) \leq L(F) \leq \|F\|_{L^\infty} \leq \sup_{x \in M} L(F, x).$$
Therefore the inequalities collapse, proving (\ref{crandall linfinity}).
\end{proof}

%%%%%%%%%%%%%%%%%%%%%%%%%%%%%
\subsection{Proof of the plaquette Lebesgue differentiation theorem}\label{proof of PLDT}

To fix notation, define for any closed $2$-form $G$, covector $\eta \in T'_x M$, and $\varepsilon > 0$,
$$A(G, \eta, \varepsilon) := \frac{1}{|B_\varepsilon(x)|} \int_{-\infty}^\infty \left[\int_{P_{\ker \eta + t\eta^\sharp, \varepsilon}} G\right] \dif t.$$
Also let 
$$\Pi_\eta: T' B_\varepsilon(x) \wedge T' B_\varepsilon(x) \to T' \lambda_{\eta, \varepsilon} \wedge T' \lambda_{\eta, \varepsilon}$$
be the projection from the bundle of antisymmetric $2$-tensors on $B_\varepsilon(x)$ to those which are cotangent to $\lambda_{\eta, \varepsilon}$.
Thus, if $G$ and $P := P_{\ker \eta + t\eta^\sharp, \varepsilon}$ are cooriented, then
\begin{equation}\label{cooriented integral is area integral of norm}
	\int_P G = \int_P |\Pi_\eta G| \dif A
\end{equation}
where $\dif A$ is the area element on the leaves of the foliation.

\begin{lemma}
Lemma \ref{PLDT} holds if $F$ is continuous and everywhere nonzero, and $\xi := \star F/F$.
\end{lemma}
\begin{proof}
For $\varepsilon > 0$, $\xi$ is conormal to $P_{\ker \xi, \varepsilon}$, and hence $\lambda_{\xi, \varepsilon}$, at $x$.
Thus, by definition of $\xi$, $\Pi_\xi F(x) = F(x)$, so by continuity of $F$,
\begin{equation}\label{continuous forms are almost their projections}
	\|\Pi_\xi F - F\|_{C^0(B_\varepsilon(x))} \ll 1
\end{equation}
as $\varepsilon \to 0$.
By (\ref{cooriented integral is area integral of norm}), the fact that the Hodge star coorients $F$ and $\xi$, and Fubini's theorem,
\begin{align*}
A(F, \xi, \varepsilon)
&= \frac{1}{|B_\varepsilon(x)|} \int_{-\infty}^\infty \int_{P_{\ker \xi + t\xi^\sharp, \varepsilon}} |\Pi_\xi F| \dif A \dif t
= \dashint_{B_\varepsilon(x)} \star |\Pi_\xi F|.
\end{align*}
By (\ref{continuous forms are almost their projections}) and the fact that $|F|$ is continuous (so $x$ is a Lebesgue point of $|F|$),
\begin{align*}
\dashint_{B_\varepsilon(x)} \star |\Pi_\xi F|
&= \dashint_{B_\varepsilon(x)} \star |F| + o(1) = |F(x)| + o(1).
\end{align*}
Taking the limit $\varepsilon \to 0$, we deduce (\ref{plaquette LDT}) for $x$.
\end{proof}

Now let $F \in L^\infty(M, Z^2)$, and split $M$ into measurable sets $M = \{F \neq 0\} \sqcup \{F = 0\}$.

\begin{lemma}
Lemma \ref{PLDT} holds on almost all of $\{F = 0\}$ for any measurable section $\xi$ of the cosphere bundle of $M$.
\end{lemma}
\begin{proof}
We bound using (\ref{cooriented integral is area integral of norm}) and Fubini's theorem
$$|A(F, \xi(x), \varepsilon)| \leq \frac{1}{|B_\varepsilon(x)|} \int_{-\infty}^\infty \int_{P_{\ker \xi + t\xi^\sharp, \varepsilon}} |F| \dif A \dif t = \dashint_{B_\varepsilon(x)} \star |F|.$$
Almost every $x$ such that $F(x) = 0$ is a Lebesgue point of $|F|$, and for such $x$,
$$0 \leq \limsup_{\varepsilon \to 0} |A(F, \xi(x), \varepsilon)| \leq \lim_{\varepsilon \to 0} \dashint_{B_\varepsilon(x)} \star |F| = |F(x)| = 0.$$
This implies (\ref{plaquette LDT}) on almost all of $\{F \neq 0\}$.
\end{proof}

So we may assume that $F(x) \neq 0$ almost everywhere.
Let $\xi := \star F/|F|$.
Let $G$ be a continuous, nowhere zero, closed $2$-form to be chosen later, and $\eta := \star G/|G|$.

\begin{lemma}
There exists $G$ such that for every $\alpha > 0$,
$$\left|\left\{\liminf_{\varepsilon \to 0} |A(F, \xi, \varepsilon) - |F|| > 4\alpha\right\}\right| \lesssim \frac{\|F - G\|_{L^1}}{\alpha}.$$
\end{lemma}
\begin{proof}
We split up
\begin{align*}
|A(F, \xi(x), \varepsilon) - |F(x)||
&\leq |A(F, \xi(x), \varepsilon) - A(G, \xi(x), \varepsilon)| + |A(G, \xi(x), \varepsilon) - A(G, \eta(x), \varepsilon)| \\
&\qquad + |A(G, \eta(x), \varepsilon) - |G(x)|| + ||F(x)| - |G(x)||\\
&=: I_1 + I_2 + I_3 + I_4.
\end{align*}

To treat $I_1$, we apply (\ref{cooriented integral is area integral of norm}):
\begin{align*}
I_1
&= \left|\frac{1}{|B_\varepsilon(x)|} \int_{-\infty}^\infty \int_{P_{\ker \xi + t\xi^\sharp, \varepsilon}} F - G \dif t\right| 
\leq \frac{1}{|B_\varepsilon(x)|} \int_{-\infty}^\infty \int_{P_{\ker \xi + t\xi^\sharp, \varepsilon}} |\Pi_\xi(F - G)| \dif A \dif t.
\end{align*}
By Fubini's theorem, 
\begin{align*}
\frac{1}{|B_\varepsilon(x)|} \int_{-\infty}^\infty \int_{P_{\ker \xi + t\xi^\sharp, \varepsilon}} |\Pi_\xi(F - G)| \dif A \dif t
&= \dashint_{B_\varepsilon(x)} \star|\Pi_\xi(F - G)| \leq \dashint_{B_\varepsilon(x)} \star |F - G|.
\end{align*}
Therefore $I_1 \leq \mathcal M|F - G|(x)$.
So by the Hardy-Littlewood maximal inequality, for any $\varepsilon > 0$,
$$|\{I_1 > \alpha\}| \leq |\{\mathcal M |F - G| > \alpha\}| \lesssim \frac{\|F - G\|_{L^1}}{\alpha}.$$

Next, to bound $I_2$, we pull back everything into the tangent space at $x$.
We put tildes over balls or Hodge stars to indicate that they are taken with respect to the euclidean metric on $T_x M$.
After accepting a correction of size $O(\varepsilon)$ arising from the metric on $M$, 
$$I_2 = \frac{1}{\varepsilon^d |\Ball^d|} \left|\int_{-\varepsilon}^\varepsilon \left[\int_{\ker \xi + t\xi^\sharp \cap \tilde B_\varepsilon} - \int_{\ker \eta + t\eta^\sharp \cap \tilde B_\varepsilon}\right] G \dif t\right| + O(\varepsilon)$$
which we can rewrite as
$$I_2 = \left|\dashint_{\tilde B_\varepsilon(0)} \tilde \star [(G, \xi(x)^\sharp) - (G, \eta(x)^\sharp)]\right| + O(\varepsilon) \leq \dashint_{\tilde B_\varepsilon(0)} \tilde \star |G| \cdot |\xi(x) - \eta(x)| + O(\varepsilon).$$
Since $G$ is continuous, on $\tilde B_\varepsilon(0)$, $|G| = |G(x)| + o(1)$ as $\varepsilon \to 0$.
Therefore by the reverse triangle inequality,
\begin{align*}
I_2
&\leq |G(x)| |\xi(x) - \eta(x)| + o(1) \\
&= |G(x)| \left|\frac{G(x)}{|G(x)|} - \frac{F(x)}{|F(x)|}\right| + o(1) \\
&= \left|G(x) - \frac{|G(x)|}{|F(x)|} F(x)\right| + o(1) \\
&= |G(x) - F(x)| + \left|1 - \frac{|G(x)|}{|F(x)|}\right| |F(x)| + o(1)\\
&= |G(x) - F(x)| + ||F(x)| - |G(x)|| + o(1) \\
&\leq 2|G(x) - F(x)| + o(1).
\end{align*}
By Markov's inequality, we conclude that 
$$\left|\left\{\liminf_{\varepsilon \to 0} I_2 > \alpha\right\}\right| \leq \left|\left\{|F - G| > \frac{\alpha}{2}\right\}\right| \leq \frac{2\|F - G\|_{L^1}}{\alpha}.$$

Since we already proved the plaquette Lebesgue differentiation theorem for continuous, nonzero $G$, we know that $I_3 = o(1)$, or in other words 
$$\left|\left\{\liminf_{\varepsilon \to 0} I_3 > \alpha\right\}\right| = 0.$$
Finally we bound $I_4$ for any $\varepsilon > 0$ using the reverse triangle inequality and Markov's inequality as 
$$|\{I_4 > \alpha\}| \leq |\{|F(x) - G(x)| > \alpha\}| \leq \frac{\|F - G\|_{L^1}}{\alpha}.$$

Adding up $I_1, \dots, I_4$, we conclude that 
\begin{align*}
\left|\left\{\liminf_{\varepsilon \to 0} |A(F, \xi, \varepsilon) - |F|| > 4\alpha\right\}\right| \leq \sum_{i=1}^4 \left|\left\{\liminf_{\varepsilon \to 0} I_i > \alpha\right\}\right| &\lesssim \frac{\|F - G\|_{L^1}}{\alpha}. \qedhere 
\end{align*}
\end{proof}

We now choose $G$ so that $\|F - G\|_{L^1} \leq \alpha^2$ and take $\alpha \to 0$ to conclude that
$$\left|\left\{\lim_{\varepsilon \to 0} |A(F, \xi, \varepsilon) - |F|| > 0\right\}\right| = 0$$
as desired.

%%%%%%%%%%%%%%%%%%%%%%%%%%%%%
\subsection{\texorpdfstring{$\infty$-light forms}{Infinity-light forms}}
Let $M$ be a closed Riemannian threefold and fix a cohomology class $\rho$.

\begin{definition}
Let $1 < p < \infty$ and let $F_p$ be a closed $2$-form with $[F_p] = \rho$.
We call $F_p$ a \dfn{$p$-light form} if it is a minimizer of $\|F_p\|_{L^p}$ among all $2$-forms representing $\rho$.
\end{definition}

We call these forms ``light'' because as $p \to \infty$ they become ``not massive'', in the sense that, as we shall later show, they will define a minimizing sequence for the comass.
A straightforward differentiation shows that the Euler-Lagrange equations for $p$-light forms are (\ref{pMaxwell}).
It follows from Proposition \ref{convex duality} that there exists a unique $p$-light form which represents $\rho$.

\begin{lemma}
Let $F_p$ be a $p$-light form, and let $B$ range over closed $2$-forms cohomologous to $F_p$. Then
\begin{equation}\label{infinity magnetic rules p magnetic}
	\|F_p\|_{L^p} \leq |M|^{1/p} \inf_B \|B\|_{L^\infty}.
\end{equation}
\end{lemma}
\begin{proof}
By H\"older's inequality and the fact that $F_p$ is $p$-light,
$$\|F_p\|_{L^p} \leq \|B\|_{L^p} \leq |M|^{1/p} \|B\|_{L^\infty},$$
hence the same holds for the infimum.
\end{proof}

% We put any norm on $H^2(M, \RR)$ (as all are equivalent), hence $|\alpha|$ makes sense for $\alpha \in H^2(M, \RR)$.

% \begin{proposition}
% Assume $p > 2$ and $\rho \in H^2(M, \RR)$.
% Then:
% \begin{enumerate}
% \item There exists a unique $p$-light form $F_p$ whose cohomology class is $\rho$.
% \item One has
% \begin{equation}\label{Sobolev bounds for p}
% 	\|F_p\|_{L^p} \lesssim |\rho|.
% \end{equation}
% The constant is independent of $p$.
% \item $F_p$ is H\"older continuous.
% \end{enumerate}
% \end{proposition}
% \begin{proof}
% The existence and uniqueness follows from Proposition \ref{convex duality} and the fact that the Euler-Lagrange equations for $p$-light forms are given by (\ref{pMaxwell}).
% From (\ref{pMaxwell}), the fact that $p > 2$, and \cite{Uhlenbeck77}, we deduce that $F_p$ is H\"older continuous.

% Now select a basis $\xi_1, \dots, \xi_r$ of $H^2(M, \RR)$, and apply the above argument to obtain $p$-light forms $G_i$ representing $\xi_i$ for each $i \in \{1, \dots, r\}$.
% Then $\|G_i\|_{L^p}$ is bounded indepdendently of $p$ by (\ref{infinity magnetic rules p magnetic}).
% Decompose
% $$\alpha = \sum_{i=1}^r \alpha_i \xi_i.$$
% Using the fact that $F_p$ is $p$-light and cohomologous to $\sum_i \alpha_i G_i$,
% $$\|F_p\|_{L^p} \leq \left\|\sum_{i=1}^r \alpha_i G_i\right\|_{L^p} \leq \sum_{i=1}^r |\alpha_i| \|G_i\|_{L^p} \lesssim |\alpha|$$
% which proves (\ref{Sobolev bounds for p}).
% \end{proof}

% \todo{We do not get $C^\infty$ regularity on sets $\Subset \{|F_p| \neq 0\}$, or so it seems.}
% Why?
% Expanding out (\ref{pMaxwell}) with $F_{ij} = \partial_i A_j - \partial_j A_i$,
% $$0 = \partial^j(|F|^{p - 2} \partial_j A_i) - |\dif A|^{p - 2} \partial_i (\dif^* A) - |\dif A|^{p - 2} [\partial^j, \partial_i] A_j - \partial^j(|\dif A|^{p - 2}) \partial_i A_j.$$
% The first term is an elliptic operator with H\"older coefficients, the second can be gauged away, and the third is H\"older since $[\partial^j, \partial_i]$ is a connection coefficient of the metric.
% But the last term is bad.
% \todo{Maybe we can get rid of it using paraproducts?}

\begin{proposition}\label{existence infinity}
Let $\rho \in H^2(M, \RR)$.
For each $p \geq 2$, let $F_p$ be the $p$-light form representing $\rho$. Then there exists a closed $2$-form $F$ such that:
\begin{enumerate}
\item $F_p \to F$ weakly in $L^r$ along a subsequence, for any $3 < r < \infty$.
\item $F$ is a best comass representative of $\rho$.
% \item One has 
% \begin{equation}\label{Sobolev bounds for infinity}
% 	\|F\|_{L^\infty} \lesssim |\rho|.
% \end{equation}
\end{enumerate}
\end{proposition}
\begin{proof}
We roughly follow \cite[\S3]{Lindqvist14}.
Let $r > 3$, and let $B$ be an $L^\infty$ representative of $\rho$.
By H\"older's inequality and (\ref{infinity magnetic rules p magnetic}),
\begin{equation}\label{uniform bounds in p by best curl}
	\|F_p\|_{L^r} \leq |M|^{\frac{1}{r} - \frac{1}{p}} \|F_p\|_{L^p} \leq |M|^{\frac{1}{r}} \|B\|_{L^\infty}.
\end{equation}
Thus a compactness argument gives $F_p \to F$ for some $2$-form $F$, weakly in $L^r$, and by Fatou's lemma, 
$$\|F\|_{L^r} \leq \liminf_{p \to \infty} \|F_p\|_{L^r} \leq |M|^{\frac{1}{r}} \|B\|_{L^\infty}.$$
Diagonalizing, we may assume that $F_p \to F$ weakly in $L^r$ for every such $r$, and taking $r \to \infty$, we conclude 
\begin{equation}\label{infinity magnetics have best curl}
	\|F\|_{L^\infty} \leq \|B\|_{L^\infty}.
\end{equation}
Moreover, $[F] = \lim_{p \to \infty} [F_p] = \rho$.
So by Proposition \ref{crandall}(\ref{crandall linfinity}) and the fact that $B$ was arbitrary in (\ref{infinity magnetics have best curl}), $F$ has best comass.
%  Moreover, taking the limit as $p \to \infty$ in (\ref{Sobolev bounds for p}), we obtain (\ref{Sobolev bounds for infinity}).
\end{proof}

\begin{definition}
The limiting $2$-form $F$ in Proposition \ref{existence infinity} is called an \dfn{$\infty$-light form}.
\end{definition}


%%%%%%%%%%%%%%%%%%%%
\subsection{\texorpdfstring{Convex duality for the $1$-Laplacian}{Convex duality of for the one-Laplacian}}
We now construct the $1$-harmonic conjugate of an $\infty$-light form.
We cannot naively take limits in Proposition \ref{convex duality}, because $(L^1, L^\infty)$ (or even $(L^\infty, BV)$) is not a dual pair of reflexive Banach spaces, and because (\ref{inverse extremality}) may blow up as $p \to \infty$.
Instead, we have to renormalize the $q$-harmonic conjugates of $p$-light forms, as in \cite[\S3.2]{daskalopoulos2020transverse}.

To this end, fix a class $\rho \in H^2(M, \RR)$ and denote by $L$ the comass of a best comass representative of $\rho$.
Also let $k_p$ be defined by 
$$k_p^{1 - p} = \int_M \star |F_p|^p$$
where $F_p$ is the $p$-light representative of $\rho$.

\begin{definition}
The \dfn{renormalized $q$-harmonic conjugate} of a $p$-light form $F_p$ is the function $u_q: \tilde M \to \RR$ which has mean zero on $M_{\rm fun}$ and solves
$$\dif u_q = (-1)^{d - 1} k_p^{p - 1} |F_p|^{p - 2} \star F_p.$$
\end{definition}

\begin{lemma}\label{normalizations converge}
As $p \to \infty$, $k_p \to 1/L$.
\end{lemma}
\begin{proof}
We follow \cite[Lemma 3.4]{daskalopoulos2020transverse}.
One has 
$$\lim_{p \to \infty} k_p^{-\frac{1}{q}} = \lim_{p \to \infty} \|F_p\|_{L^p}.$$
Since $F_p$ converges in the the weak topology on $L^q$ for any large $q < \infty$ to a form of best comass, this limit is at most $L$.
If it is strictly less than $L$, then there exist $p$-light forms with comass strictly than $L$, a contradiction.
\todo{That is true but it needs more detail.}
Taking logarithms we see that $q^{-1} \log k_p \to -\log L$, and since $q \to 1$ the claim follows.
\end{proof}

\begin{lemma}
Let $\tilde M \to M$ be the universal cover, and let $(u_q)$ be a sequence of $\pi_1(M)$-equivariant functions on $\tilde M$ which converge weakly in $BV_\loc(\tilde M)$ to a function $u$.
Then $u$ is $\pi_1(M)$-equivariant.
\end{lemma}
\begin{proof}
	\todo{Use the equivairance to take limits along curves}
\end{proof}

\begin{proposition}\label{existence 1}
Let $\rho \in H^2(M, \RR)$ and let $\tilde M \to M$ be the universal cover.
For $2 < p < \infty$ and $\frac{1}{p} + \frac{1}{q} = 1$, let $u_q$ be the renormalized $q$-harmonic conjugate of the $p$-light representative of $\rho$.
Then there exists a $\pi_1(M)$-equivariant function $u \in BV_\loc(\tilde M)$ such that:
\begin{enumerate}
\item $u$ is $1$-harmonic.
\item As $q \to 1$ along a subsequence, $u_q \to u$ weakly in $BV_\loc(\tilde M)$ and strongly in $L^r$ for $1 \leq r < \infty$.
\item Let $F$ be the $\infty$-light representative of $\rho$, with best comass $L$. We have the strong duality theorem 
\begin{equation}\label{1 strong duality}
	L \int_M \star |\dif u| = \int_M \dif u \wedge F
\end{equation}
and, $|\dif u|$-almost everywhere,
\begin{equation}\label{1 extremality}
L |\dif u| = \langle \dif u, \star F\rangle.
\end{equation}
\end{enumerate}
\end{proposition}
\begin{proof}
We first compute using Lemma \ref{normalizations converge}
\begin{equation}\label{Lqs of qLaplace converge}
\lim_{q \to 1} \int_M \star |\dif u_q|^q = \lim_{p \to \infty} k_p^p \int_M \star |F_p|^p = \lim_{p \to \infty} k_p = \frac{1}{L}.
\end{equation}
So by H\"older's inequality,
$$\lim_{q \to 1} \int_M \star |\dif u_q| \leq \lim_{q \to 1} |M|^{\frac{1}{p}} \int_M \star |\dif u_q|^q = \frac{1}{L}.$$
By Poincar\'e's inequality, $(u_q)$ is then bounded in $W^{1, 1}_\loc(\tilde M) \cap L^\infty_\loc(\tilde M)$ as $q \to 1$, so it converges along a subsequence to an element $u$ of the double dual $BV_\loc(\tilde M)$ of $W^{1, 1}_\loc(\tilde M)$, and also in $L^r_\loc(\tilde M)$ for any $1 \leq r < \infty$.
As the limit of $\pi_1(M)$-equivariant functions, $u$ is also $\pi_1(M)$-equivariant.
In particular $\dif u$ drops to a current on $M$.

Renormalizing (\ref{strong duality}), we obtain 
$$\frac{k_p^{-p}}{q} \int_M \star |\dif u_q|^q + \frac{1}{p} \int_M \star |F_p|^p = k_p^{1 - p} \int_M \dif u_q \wedge F_p.$$
Multiplying by $k_p^p$ and taking limits, we conclude (\ref{1 strong duality}).
However, $\|F/L\|_{L^\infty} \leq 1$, so this is only possible if $X := (\star F)^\sharp/L$ satisfies, $|\dif u|$-almost everywhere,
$$|\dif u| = (X, \dif u).$$
Therefore (\ref{1 extremality}) holds, and since $\dif F = 0$ implies $\nabla \cdot X = 0$, we conclude from \cite{Mazon14} that $u$ is $1$-harmonic.
\end{proof}

%%%%%%%%%%%%%%%%%%%%%%%%
\subsection{A gauge-theoretic interpretation}
\todo{Explain how to think of $A$ as a connection when $\rho$ is integral and $d = 3$, and the equivariance conditions in general.}


%%%%%%%%%%%%%%%%%%%%%%%%%
\subsection{The \texorpdfstring{$\infty$-light equation}{infinity-light equation}}
We have the following Euler-Lagrange equation for $\infty$-light forms.
Because of the lack of a good analogue for viscosity solutions for $\infty$-elliptic systems, \todo{and because we did not show that $\infty$-light forms have absolutely best comass}, the equation can only really be interpreted in a formal sense, at least as far as we are aware.
As such, we shall not use it in the sequel, but only include it as a curiosity item.


\todo{If we knew that $p$-Maxwell had good quantitative uniqueness, then we would have}
It remains to show that $A$ has absolutely best curl, so let $\Omega$ be a small ball and $B$ a $1$-form with $B|_{\partial \Omega} = A|_{\partial \Omega}$.
By a straightforward modification of the existence theorem, there exists a $p$-magnetic potential $B_p$ in Coulomb gauge with $B_p|_{\partial \Omega} = A|_{\partial \Omega}$ and $B \in C^{1 + \alpha}$.
By quantitative uniqueness
$$\|B_p - A\|_{C^0(\Omega)} \leq \|B_p - A_p\|_{C^0(\Omega)} + o(1) \lesssim \|A - A_p\|_{C^0(\partial \Omega)} + o(1) \ll 1.$$
Therefore $B_p \to A$ uniformly, and for $3 < q < p < \infty$ with $p$ dyadic,
$$\|\dif B_p\|_{L^q(\Omega)} \leq |\Omega|^{\frac{1}{q} -\frac{1}{p}} \|\dif B_p\|_{L^p(\Omega)} \leq |\Omega|^{\frac{1}{q} -\frac{1}{p}} \|\dif B\|_{L^p(\Omega)} \leq |\Omega|^{\frac{1}{q}} \|\dif B\|_{L^\infty(\Omega)}.$$
Then along a subsequence, $\dif B_p \to \dif A$ in $L^q(\Omega)$, so 
$$\|\dif A\|_{L^q(\Omega)} \leq |\Omega|^{\frac{1}{q}} \|\dif B\|_{L^\infty(\Omega)}.$$
Taking $q \to \infty$ we arrive at the conclusion that $F$ has absolutely best comass.

\begin{proposition}
Suppose that $F$ has absolutely best comass, regularity $C^1$, and no points with $F = 0$. Then
\begin{equation}\label{infinityMaxwell}
	F^{ij} \partial_i |F| = 0.
\end{equation}
\end{proposition}
\begin{proof}
For a covariant $2$-tensor $T$, let $T^{\rm as}$ be its antisymmetrization, and let
$$f(x, T) := |T^{\rm as}|_{g(x)}.$$
Working locally, we may write $F = \dif A$ for some $A$, which we may assume to be Coulomb gauge and therefore $C^2$.
Since $A$ has absolutely best curl and $(\nabla A)^{\rm as} = \dif A$, $A$ is an absolute minimizer (see \cite[Definition 5.1]{Barron2001}) of the essential supremum of $f(\cdot, \nabla A)$.
By the Euler-Lagrange-Aronsson formula \cite[Theorem 5.2]{Barron2001},
\begin{equation}\label{ELA}
	\left\langle \frac{\partial f}{\partial T}(x, \nabla A(x)), \dif (f(x, \nabla A(x))) \right\rangle = 0.
\end{equation}
Now
$$\dif(f(x, \nabla A(x))) = \dif |\dif A(x)|$$
and 
$$\frac{\partial f}{\partial T}(x, \nabla A(x)) = \frac{\nabla A(x)^{\rm as}}{|\nabla A(x)^{\rm as}|} = \frac{\dif A(x)}{|\dif A(x)|}.$$
We conclude the claim after multiplying both sides of (\ref{ELA}) by $|\dif A|$.
\end{proof}

\begin{corollary}
Suppose that $F$ has absolutely best comass, regularity $C^1$, and no points with $F = 0$, and $N$ is a surface whose normal vector field is annihilated by $F$.
Then $N$ is a minimal surface.
\end{corollary}
\begin{proof}
Let $V$ be a tangent vector field to $N$. Then $V(|F|) = 0$, by (\ref{infinityMaxwell}).
Therefore $|F|$ is constant along $N$, but $F$ is a continuous section of the area bundle of $N$, which is a real line bundle.
It follows that $F$ is constant along $N$, and $F/|F|$ is the area form on $N$.
In other words, $N$ is calibrated by $F$, and the claim follows from (\ref{calibrated surfaces are minimal}).
\end{proof}

%%%%%%%%%%%%%%%%%%%%

\section{The maximum comass locus}
Throughout this section, let $M$ be a closed space form of dimension $3$.

\begin{definition}
Let $F$ be a form of best comass.
The \dfn{maximum comass locus} is the set $\{L(F, \cdot) = L(F)\}$.
\end{definition}

By Proposition \ref{crandall usc} and the compactness of $M$, the maximum comass locus is a nonempty closed subset of $M$.

By \todo{Cite laminations paper}, to each $1$-harmonic function $u$, we may associated a measured oriented minimal lamination $\lambda_u$, whose Ruelle-Sullivan current is $\dif u$, and whose leaves are the level sets of $u$.

\begin{definition}
Let $\rho \in H^2(M, \RR)$, let $F$ be an $\infty$-light representative of $\rho$, and let $u$ be a $1$-harmonic conjugate of $F$.
A \dfn{Thurston lamination} $\lambda$ associated to $\rho$ is $\lambda := \lambda_u$.
\end{definition}

\subsection{Calibration of Thurston laminations}
\begin{theorem}\label{MCL contains Thurston}
Let $F$ be a best comass representative of $\rho \in H^2(M, \RR)$.
Then the maximum comass locus of $F$ contains any Thurston lamination $\lambda$ associated to $\rho$.
In particular, if $L$ is the best comass of $\rho$, then $F/L$ calibrates $\lambda$.
\end{theorem}

On the first reading, the reader may wish to take $F$ to be $\infty$-light, as this already captures the essential ideas.
Throughout the proof we fix the $p$-light representative $F_p$ of $\rho$, its conjugate $q$-harmonic $u_q$, and the limiting $1$-harmonic $u$.

\begin{lemma}
Let
$$f_p := 2\langle k_p F_p, k_p F_p - L^{-1} F\rangle.$$
Then 
\begin{equation}\label{MCL contains Thurston 1}
	\lim_{p \to \infty} \int_{\{f_p \geq 0\}} \star |k_p F_p|^{p - 2} f_p = 0.
\end{equation}
\end{lemma}
\begin{proof}
The proof is very similar to \cite[Lemma 6.3]{daskalopoulos2020transverse}.
Since $F_p, F$ are cohomologous, there exists a $1$-form $\xi$ such that $\dif \xi = F_p - F$.
On the other hand, since $F_p$ is $p$-light, an integration by parts gives
\begin{align*}
\int_M |F_p|^{p - 2} \langle F_p, F_p - F \rangle = \int_M \langle \dif^*(|F_p|^{p - 2} F_p), \xi\rangle = 0.
\end{align*}
In particular,
\begin{equation}\label{MCL contains Thurston 2}
	I_p := \int_M \star |k_p F_p|^{p - 2} \langle k_p F_p, k_p F_p - k_p F\rangle = 0.
\end{equation}
On the other hand, by the Cauchy-Schwarz inequality, Lemma \ref{normalizations converge}, and the fact that 
$$\lim_{p \to \infty} \|F_p\|_{L^p} = \|F\|_{L^\infty} = L,$$
we have
\begin{align*}
\lim_{p \to \infty} I_p - \frac{1}{2} \int_M \star |k_p F_p|^{p - 2} f_p 
&= \lim_{p \to \infty} \int_M \star |k_p F_p|^{p - 2} \left\langle k_p F_p, \left(\frac{1}{L} - k_p\right) F\right\rangle \\
&\leq \lim_{p \to \infty} k_p^{p - 1} \left(\frac{1}{L} - k_p\right) \int_M \star |F_p|^{p - 1} |F| \\
&= \lim_{p \to \infty} \frac{1}{L^{p - 1}} \left(\frac{1}{L} - \frac{1}{L}\right)L^p \\
&= \lim_{p \to \infty} 1 - 1 = 0.
\end{align*}
If we plug this into (\ref{MCL contains Thurston 2}), we obtain 
\begin{equation}\label{MCL contains Thurston 3}
	\lim_{p \to \infty} \int_M \star |k_p F_p|^{p - 2} f_p = 0.
\end{equation}
If $f_p(x) \leq 0$, then by the Cauchy-Schwarz inequality,
$$|L^{-1} F|(x) \geq |k_p F_p|(x),$$
so by the Cauchy-Schwarz and Peter-Paul inequalities,
\begin{align*}
f_p(x) &\leq 2|L^{-1} F|(x) |k_p F_p|(x) - 2|k_p F_p|^2(x) \\
&\leq |L^{-1} F|(x)^2 + |k_p F_p|(x)^2 - 2|k_p F_p|^2(x) \\
&= |L^{-1} F|(x)^2 - |k_p F_p|(x).
\end{align*}
So by \cite[Lemma 6.2]{daskalopoulos2020transverse},
$$|k_p F_p|(x)^{p - 2} f_p(x) < \frac{2}{p - 2}.$$
Integrating this inequality, 
$$0 \leq \lim_{p \to \infty} \int_{\{f_p \leq 0\}} |k_p F_p|^{p - 2} f_p \leq \lim_{p \to \infty} \frac{2|M|}{p - 2} = 0.$$
Therefore, by (\ref{MCL contains Thurston 3}), we deduce (\ref{MCL contains Thurston 1}).
\end{proof}

\begin{lemma}\label{MCL contains Thurston lemma}
The set $\{L(F, \cdot) < L\}$ satisfies
$$\lim_{p \to \infty} \int_{\{L(F, \cdot) < L\}} \star |k_p F_p|^p = 0.$$
\end{lemma}
\begin{proof}
We decompose
$$\{L(F, \cdot) < L\} = \bigcup_{0 < \theta < 1} \{L(F, \cdot) < \theta L\}$$
and then it suffices to fix $\theta$ and show that the integral over $\{L(F, \cdot) < \theta L\}$ is zero.
We then use the fact that $|F|(x) \leq L(F, x)$ almost everywhere, by Proposition \ref{crandall}(\ref{crandall LDT}), to obtain
$$\{L(F, \cdot) < \theta L\} \subseteq A \cup B \cup Z$$
where 
\begin{align*}
A &:= \{|L^{-1} F|^2 \leq \theta |k_p F_p|^2 + |k_p F_p - L^{-1} F|^2\}, \\
B &:= \{|L^{-1} F|^2 \geq \theta |k_p F_p|^2 + |k_p F_p - L^{-1} F|^2\} \cap \{L(F, \cdot) < \theta L\},
\end{align*}
and $Z$ is null. 
In particular,
$$0 \leq \lim_{p \to \infty} \int_{\{L(F, \cdot) < L\}} \star |k_p F_p|^p \leq \lim_{p \to \infty} \int_A \star |k_p F_p|^p + \lim_{p \to \infty} \int_B \star |k_p F_p|^p.$$
By (\ref{MCL contains Thurston 1}),
\begin{align*}
0 &= \lim_{p \to \infty} \int_{\{f_p \geq 0\}} |k_p F_p|^{p - 2} f_p \\
&= \lim_{p \to \infty} \int_{\{f_p \geq 0\}} |k_p F_p|^{p - 2}((1 - \theta) |k_p F_p|^2 + \theta |k_p F_p|^2 + |k_p F_p - L^{-1} F|^2 - |L^{-1} F|^2).
\end{align*}
But on $A$,
\begin{align*}
|L^{-1} F|^2 &\leq \theta |k_p F_p|^2 + |k_p F_p - L^{-1} F|^2 \\
&\leq |k_p F_p|^2 + |k_p F_p - L^{-1} F|^2 \\
&= f_p + |L^{-1} F|^2
\end{align*}
which implies $f_p \geq 0$ and 
$$\lim_{p \to \infty} \int_A \star |k_p F_p|^p \leq \lim_{p \to \infty} \frac{1}{\theta} \int_{\{f_p \geq 0\}} \star |F_p|^{p - 2} f_p = 0.$$
Meanwhile, on $B$,
$$|k_p F_p|^2 \leq |k_p F_p|^2 + \theta^{-1} |k_p F_p - L^{-1} F|^2 \leq \theta^{-1} |L^{-1} F|^2 < \theta.$$
Therefore
\begin{align*}
	\lim_{p \to \infty} \int_B |k_p F_p|^p &\leq \lim_{p \to \infty} \theta^{p/2} |M| = 0. \qedhere
\end{align*}
\end{proof}

\begin{proof}[Proof of Theorem \ref{MCL contains Thurston}]
Let $U := \{L(F, \cdot) < L\}$, which is open since it is the complement of the best comass locus of $F$.
So by the portmanteau theorem \todo{Cite it}
\begin{align*}
\int_U \star |\dif u|
&\leq \lim_{q \to 1} \int_U \star |\dif u_q|
= \lim_{p \to \infty} \int_U \star |k_p F_p|^{p - 1}.
\end{align*}
Then by H\"older's inequality and Lemma \ref{MCL contains Thurston lemma},
\begin{align*}
\lim_{p \to \infty} \int_U \star |k_p F_p|^{p - 1}
&\leq \lim_{p \to \infty} |M|^{-\frac{1}{p}} \left[\int_U \star |k_p F_p|^p\right]^{\frac{1}{q}} 
= \lim_{p \to \infty} \int_U \star |k_p F_p|^p = 0.
\end{align*}
Since $\dif u$ is the Ruelle-Sullivan current for the Thurston lamination $\lambda$, $\lambda$ is contained in the best comass locus of $F$. 
Moreover, if $N$ is a leaf of $\lambda$, then the restriction of $F/L$ is, by (\ref{1 extremality}), equal to $\star \normal_N^\flat$, which is the area form of $N$.
In other words, $F/L$ calibrates $N$.
\end{proof}



%%%%%%%%%%%%%%%%%%%%%%
\subsection{Thurston's \texorpdfstring{$L = K$}{L equals K} theorem}
If $\lambda$ is a measured oriented lamination, then we denote by $[\lambda] \in H_2(M, \RR)$ its homology class, and by $|\lambda|$ its area; namely, the mass of its Ruelle-Sullivan current $T_\lambda$.

\begin{theorem}\label{L equals K}
	Let $\rho \in H^2(M, \RR)$, and let 
	$$K := \sup_\lambda \frac{\langle \rho, [\lambda]\rangle}{|\lambda|},$$
	where $\lambda$ ranges over measured oriented laminations. Then:
\begin{enumerate}
	\item The supremum in $K$ is attained by the Thurston lamination associated to $\rho$.
	\item Let $L$ be the best comass of $\rho$. Then $L = K$.
\end{enumerate}
\end{theorem}
\begin{proof}
Fix the $\infty$-light form $F$ representing $\rho$, and let $u$ be its $1$-harmonic conjugate.

We first prove $K \leq L$.
Let $\lambda$ be a measured oriented lamination; then, since $F$ represents $\rho$ and the Ruelle-Sullivan current $T_\lambda$ represents $[\lambda]$,
$$\langle \rho, [\lambda]\rangle = \int_M F \wedge T_\lambda.$$
Let $\mu_\lambda$ be the transverse measure to $\lambda$ and $\mathscr L_\lambda$ the space of leaves of $\lambda$; then, by definition of $T_\lambda$ and the definition of comass,
$$\int_M F \wedge T_\lambda = \int_{\mathscr L_\lambda} \int_N F \dif \mu_\lambda(N) \leq L(F) \int_{\mathscr L_\lambda} |N| \dif \mu_\lambda(N) = L(F) |\lambda|.$$
By Proposition \ref{crandall}(\ref{crandall linfinity}), $F$ has best comass $L(F) = L$.
Since $\lambda$ was arbitrary, it holds that $K \leq L$.

Let $\lambda$ be the Thurston lamination, so $T_\lambda = \dif u$.
Then by the max flow min cut principle (\ref{1 strong duality}),
$$\langle \rho, [\lambda]\rangle = \int_M F \wedge \dif u = L \int_M \star |\dif u| = L|\lambda|.$$
Dividing both sides by $|\lambda|$ and applying the direction we already proved,
$$K \leq L \leq \frac{\langle \rho, [\lambda]\rangle}{|\lambda|} \leq K$$
which is only possible if $L = K$ and $\lambda$ is a maximizer.
\end{proof}

\printbibliography

\end{document}
