\documentclass[reqno,11pt]{amsart}
\usepackage[letterpaper, margin=1in]{geometry}
\RequirePackage{amsmath,amssymb,amsthm,graphicx,mathrsfs,url,slashed,subcaption}
\RequirePackage[usenames,dvipsnames]{xcolor}
\RequirePackage[colorlinks=true,linkcolor=Red,citecolor=Green]{hyperref}
\RequirePackage{amsxtra}
\usepackage{cancel}
\usepackage{tikz, quiver, wrapfig}

% Add the 2020 MSC
\makeatletter
\@namedef{subjclassname@2020}{\textup{2020} Mathematics Subject Classification}
\makeatother

%\usepackage[T1]{fontenc}

% \setlength{\textheight}{9.3in} \setlength{\oddsidemargin}{-0.25in}
% \setlength{\evensidemargin}{-0.25in} \setlength{\textwidth}{7in}
% \setlength{\topmargin}{-0.25in} \setlength{\headheight}{0.18in}
% \setlength{\marginparwidth}{1.0in}
% \setlength{\abovedisplayskip}{0.2in}
% \setlength{\belowdisplayskip}{0.2in}
% \setlength{\parskip}{0.05in}
%\renewcommand{\baselinestretch}{1.05}

\title{Convex duality between minimal laminations and tight calibrations}
\author{Aidan Backus}
\address{Department of Mathematics, Brown University}
\email{aidan\_backus@brown.edu}
\date{\today}
\keywords{laminations, convex duality, minimal hypersurfaces, max flow/min cut theorem, calibrations, functions of least gradient}
\subjclass[2020]{primary: 49Q20; secondary: 39J60, 49N15, 53C38}

\newcommand{\NN}{\mathbf{N}}
\newcommand{\ZZ}{\mathbf{Z}}
\newcommand{\QQ}{\mathbf{Q}}
\newcommand{\RR}{\mathbf{R}}
\newcommand{\CC}{\mathbf{C}}
\newcommand{\DD}{\mathbf{D}}
\newcommand{\PP}{\mathbf P}
\newcommand{\MM}{\mathbf M}
\newcommand{\II}{\mathbf I}
\newcommand{\Hyp}{\mathbf H}
\newcommand{\Sph}{\mathbf S}
\newcommand{\Group}{\mathbf G}
\newcommand{\GL}{\mathbf{GL}}
\newcommand{\Orth}{\mathbf{O}}
\newcommand{\SpOrth}{\mathbf{SO}}
\newcommand{\Ball}{\mathbf{B}}

\newcommand*\dif{\mathop{}\!\mathrm{d}}

\DeclareMathOperator{\card}{card}
\DeclareMathOperator{\dist}{dist}
\DeclareMathOperator{\id}{id}
\DeclareMathOperator{\Hom}{Hom}
\DeclareMathOperator{\coker}{coker}
\DeclareMathOperator{\supp}{supp}
\DeclareMathOperator{\Teich}{Teich}
\DeclareMathOperator{\tr}{tr}

\newcommand{\Leaves}{\mathscr L}
\newcommand{\Lagrange}{\mathscr L}
\newcommand{\Hypspace}{\mathscr H}

\newcommand{\Chain}{\underline C}

\newcommand{\Two}{\mathrm{I\!I}}
\newcommand{\Ric}{\mathrm{Ric}}

\newcommand{\normal}{\mathbf n}
\newcommand{\radial}{\mathbf r}
\newcommand{\evect}{\mathbf e}
\newcommand{\vol}{\mathrm{vol}}

\newcommand{\diam}{\mathrm{diam}}
\DeclareMathOperator{\Gal}{Gal}
\newcommand{\Ell}{\mathrm{Ell}}
\newcommand{\inj}{\mathrm{inj}}
\newcommand{\Lip}{\mathrm{Lip}}
\newcommand{\MCL}{\mathrm{MCL}}
\newcommand{\Riem}{\mathrm{Riem}}

\newcommand{\Mass}{\mathbf M}
\newcommand{\Comass}{\mathbf L}

\newcommand{\Min}{\mathrm{Min}}
\newcommand{\Max}{\mathrm{Max}}

\newcommand{\dfn}[1]{\emph{#1}\index{#1}}

\renewcommand{\Re}{\operatorname{Re}}
\renewcommand{\Im}{\operatorname{Im}}

\newcommand{\loc}{\mathrm{loc}}
\newcommand{\cpt}{\mathrm{cpt}}

\def\Japan#1{\left \langle #1 \right \rangle}

\newtheorem{theorem}{Theorem}[section]
\newtheorem{badtheorem}[theorem]{``Theorem"}
\newtheorem{prop}[theorem]{Proposition}
\newtheorem{lemma}[theorem]{Lemma}
\newtheorem{sublemma}[theorem]{Sublemma}
\newtheorem{proposition}[theorem]{Proposition}
\newtheorem{corollary}[theorem]{Corollary}
\newtheorem{conjecture}[theorem]{Conjecture}
\newtheorem{axiom}[theorem]{Axiom}
\newtheorem{assumption}[theorem]{Assumption}

\newtheorem{mainthm}{Theorem}
\renewcommand{\themainthm}{\Alph{mainthm}}

\newtheorem{claim}{Claim}[theorem]
\renewcommand{\theclaim}{\thetheorem\Alph{claim}}
% \newtheorem*{claim}{Claim}

\theoremstyle{definition}
\newtheorem{definition}[theorem]{Definition}
\newtheorem{remark}[theorem]{Remark}
\newtheorem{example}[theorem]{Example}
\newtheorem{notation}[theorem]{Notation}

\newtheorem{exercise}[theorem]{Discussion topic}
\newtheorem{homework}[theorem]{Homework}
\newtheorem{problem}[theorem]{Problem}

\makeatletter
\newcommand{\proofpart}[2]{%
  \par
  \addvspace{\medskipamount}%
  \noindent\emph{Part #1: #2.}
}
\makeatother



\numberwithin{equation}{section}


% Mean
\def\Xint#1{\mathchoice
{\XXint\displaystyle\textstyle{#1}}%
{\XXint\textstyle\scriptstyle{#1}}%
{\XXint\scriptstyle\scriptscriptstyle{#1}}%
{\XXint\scriptscriptstyle\scriptscriptstyle{#1}}%
\!\int}
\def\XXint#1#2#3{{\setbox0=\hbox{$#1{#2#3}{\int}$ }
\vcenter{\hbox{$#2#3$ }}\kern-.6\wd0}}
\def\ddashint{\Xint=}
\def\dashint{\Xint-}

\usepackage[backend=bibtex,style=alphabetic,giveninits=true]{biblatex}
\renewcommand*{\bibfont}{\normalfont\footnotesize}
\addbibresource{best_curl.bib}
\renewbibmacro{in:}{}
\DeclareFieldFormat{pages}{#1}

\newcommand\todo[1]{\textcolor{red}{TODO: #1}}


\begin{document}
\begin{abstract}
We investigate the \emph{continuous max flow/min cut theorem}, namely the convex duality between calibrations and homologically minimizing laminations.
We realize this duality as the limiting behavior of the duality between the $p$-Laplacian and its dual problem, and use this limiting process to extend the duality theorem of Bangert--Cui \cite{bangert_cui_2017} from continuous calibrations to measurable calibrations.
The novel idea in the proof is to study a system of PDE solved by certain calibrations, which we call \dfn{tight}, which generalizes the $\infty$-Laplacian; we suspect that this PDE will be of independent interest as a model problem in the $L^\infty$ calculus of variations.

As an application, given a cohomology class $\rho \in H^{d - 1}(M, \RR)$, we construct a lamination similar to Thurston's maximally stretched lamination \cite{Thurston98} which encodes the interaction of $\rho$ with the stable norm on $H_{d - 1}(M, \RR)$.
\end{abstract}

\maketitle

%%%%%%%%%%%%%%%%%%%%%%%%%%%%%%%%%%%%%%%%%%%%%%%%%%%%%%%
\section{Introduction}
Let $M$ be a closed Riemannian manifold.
The \dfn{stable norm} $\Mass(\sigma)$ of a homology class $\sigma$ is the infimum of the mass $\Mass(N)$ over all cycles $N$ representing $\sigma$.
We call its dual norm the \dfn{costable norm} $\Comass(\rho)$, which for $\rho \in H^{d - 1}(M, \RR)$ is the minimum of the comass $\|F\|_{L^\infty}$ among all differential forms $F$ representing $\rho$.
We say that a minimizing representative of $\rho$ has \dfn{best comass}.

A straightforward application of the Hanh-Banach theorem, due to Federer \cite{Federer1974}, shows that for each $\rho \in H^{d - 1}(M, \RR)$ such that $\Comass(\rho) = 1$, there exists $\sigma \in H^{d - 1}(M, \RR)$ such that
\begin{equation}\label{Federer duality}
\Mass(\sigma) = \langle \rho, \sigma\rangle.
\end{equation}
This result can be viewed somewhat charitably as a continuous analogue of the max flow/min cut theorem of graph theory \cite{sullivan1990crystalline}.
Indeed, if every homology class could be represented by a smooth homologically minimizing cycle, then for any continuous best comass calibration $F$ of degree $d - 1$ (which could be viewed as a ``maximal flow'', since $F$ is the flux of a divergence-free vector field), it would follow from (\ref{Federer duality}) that $F$ would calibrate some minimal hypersurface $N$ (a ``minimal cut''), which would then ``bottleneck'' $F$ in the sense that for any $N'$ homologous to $N$,
$$\Mass(N) = \int_N F.$$
We refer to \cite[Figure 5]{Freedman_2016} for an enlightening visualization of this bottlenecking phenomenon.

In reality, not every homology class has a smooth homologically minimizing cycle.
In light of this fact, Bangert and Cui proved the following result which replaces the hypersurface by a lamination (see \S\S\ref{prevResults}--\ref{comass sec} for the definitions):

\begin{theorem}[{\cite{bangert_cui_2017}}]\label{BangertCui}
Let $F$ be a continuous best comass calibration of degree $d - 1$ on a closed Riemannian manifold of dimension $\leq 7$.
Then there exists a measured oriented minimal lamination $\lambda$ such that every leaf of $\lambda$ is $F$-calibrated.
In particular, 
$$\Mass(\lambda) = \langle [\lambda], [F]\rangle.$$
\end{theorem}

However, most calibrations which appear in nature are $L^\infty$ but not necessarily continuous \cite[\S1]{bangert_cui_2017}, and it is natural to restrict attention to \emph{Lipschitz} laminations \cite[Remark 2.3]{bangert_cui_2017}.
Thus a more natural formulation of Theorem \ref{BangertCui} would not require that $F$ is continuous, and would assert that $\lambda$ is Lipschitz.

The main goal of this paper, then, is an improvement of Theorem \ref{BangertCui} (see Theorem \ref{lams are calibrated} for a precise statement).
There are three steps in this process.
First, we must develop the theory of $L^\infty$ calibrations.
Secondly, we introduce a special class of best comass calibrations, which we call \dfn{tight}, and which can be shown, using convex optimization techniques and the duality of $L^1$ and $L^\infty$, to be ``bottlenecked'' by a function of least gradient.
Finally, we use the fact that the level sets of a function of least gradient form a Lipschitz lamination \cite{BackusCML}.

The role of convexity and H\"older duality was predicted by Thurston, who conjectured that they could be used to prove a similar result in the analogous situation where the best comass calibration is replaced by a best Lipschitz map between hyperbolic surfaces \cite{Thurston98}.
This goal was recently realized by Daskalopolous and Uhlenbeck \cite{daskalopoulos2020transverse,daskalopoulos2022,daskalopoulos2023}.
A key idea of Daskalopolous and Uhlenbeck, which we shall also need, is to regularize the optimization problems by instead studying convex duality between minimization problems in $L^p$ and $L^q$, and take $p \to \infty$, $q \to 1$.

Once we have proven a strong Bangert--Cui theorem, we study the laminations which can be calibrated by some best comass representative of a cohomology class $\rho$ \emph{en masse}.
It turns out that they all are extensions of a lamination $\lambda_\rho$ that only depends on $\rho$, which we call the \dfn{canonical lamination}, and which can be used to prove certain facts about the duality between the stable and costable norms.
We also study the PDE solved by a tight form, which we believe to be of independent interest as a model problem in the $L^\infty$ calculus of variations.

%%%%%%%%%%%%%%%%%%%%
\subsection{Tight forms and functions of least gradient}
Following Daskalopolous and Uhlenbeck \cite{daskalopoulos2020transverse}, we consider $d - 1$-forms which behave analogously to $p$-harmonic functions, and take the limit $p \to \infty$ to get a best comass form which behaves analogously to an $\infty$-harmonic function.
At the same time we consider the convex dual problem.
The best comass forms will be ``maximal flows''; their dual ``minimal cuts'' will be functions of least gradient.

To be more precise, let $(p, q)$ be a H\"older pair (thus $1/p + 1/q = 1$) such that $d < p < \infty$.
Motivated by the $p$-Laplace equation $\dif^*(|\dif v|^{p - 2} \dif v) = 0$, we introduce \dfn{$p$-tight} forms, which are closed $d-1$-forms which solve the system of PDE
$$\dif^*(|F|^{p - 2} F) = 0.$$
Given a $p$-tight form, the $\pi_1(M)$-equivariant function $u$ on the universal cover such that
$$\dif u = (-1)^{d - 1} |F|^{p - 2} \star F$$
is $q$-harmonic -- in other words, $u$ is a solution of the $q$-Laplace equation 
$$\dif^*(|\dif u|^{q - 2} \dif u) = 0.$$
A function $u$ has \dfn{least gradient} if it minimizes $\int_M \star |\dif u|$.

Our first theorem, a combination of Propositions \ref{existence infinity} and \ref{existence 1}, constructs a best comass form, and a dual function of least gradient, by taking limits of $p$-tight forms and their dual $q$-harmonic functions.

\begin{mainthm}\label{existence of infinity tight forms}
Let $\rho \in H^{d - 1}(M, \RR)$ be a nonzero cohomology class.
Let $(F_p, u_q)$ be the family of dual pairs of $p$-tight forms and $q$-harmonic functions, suitably normalized, with $[F_p] = \rho$ and $(p, q)$ ranging over H\"older pairs with $d < p < \infty$.
Then there exists a pair $(F, u)$ such that as $p \to \infty$ along a subsequence, $F_p \to F$ weakly in $L^r$ for any $d < r < \infty$, and $u_q \to u$ weakly in $BV$, with the following properties:
\begin{enumerate}
\item $F$ has best comass; we call $F$ \dfn{tight}.
\item $u$ has least gradient and is nonconstant.
\item The product of distributions $\dif u \wedge F$ is well-defined, and
\begin{equation}\label{max flow mean cut}
\Comass(\rho) \star |\dif u| = \dif u \wedge F.
\end{equation}
\end{enumerate}
\end{mainthm}

We highlight the duality condition (\ref{max flow mean cut}) as the main point of the theorem.
It is crucial to the proof that $u$ is $1$-harmonic, and allows us to prove this without a careful analysis of the limiting behavior of $q$-harmonic functions as in \cite[Theorem 2.4]{Mazon14}, or of $p$-tight forms as in \cite[\S6]{daskalopoulos2020transverse}.
On the other hand, if $\Comass(\rho) = 1$, then we shall be able to use (\ref{max flow mean cut}) to show that the level sets of $u$ are $F$-calibrated.

%%%%%%%%%%%%%%%%%%%%%
\subsection{Calibrated laminations}
Suppose that $\rho \in H^{d - 1}(M, \RR)$ satisfies $\Comass(\rho) = 1$.
If $F$ is a continuous tight representative of $\rho$, then by the Bangert--Cui theorem, $F$ calibrates a minimal lamination.
However, a proof that tight forms are continuous is out of reach.\footnote{For domains in $\RR^2$, tight forms are $C^\alpha$ \cite{Evans08}, but it is unlikely that this argument generalizes.}
On the other hand, we know by \cite[Theorem B]{BackusCML} that the level sets of a function of least gradient form a measured oriented Lipschitz minimal lamination.
In \S\ref{MCL sec} we use Theorem \ref{existence of infinity tight forms} to prove the following.

\begin{definition}
Let $M$ be a closed Riemannian manifold of dimension $d \leq 7$, and $\rho \in H^{d - 1}(M, \RR)$.
We can define a measured oriented minimal lamination $\mu$, by considering a tight representative $F$ of $\rho$, letting $u$ be a dual function of least gradient to $F$, and letting $\mu$ be the lamination induced by $u$.
We call $\mu$ a \dfn{measured stretch lamination} associated to $\rho$.
\end{definition}

\begin{mainthm}\label{lams are calibrated}
Suppose that $M$ is a closed Riemannian manifold of dimension $d \leq 7$, and let $\rho \in H^{d - 1}(M, \RR)$ be nonzero.
Then there is a measured stretch lamination associated to $\rho$.
Moreover, if $\kappa$ is a measured stretch lamination associated to $\rho$, and $F$ is a best comass representative of $\rho$, then $\kappa$ is $F/\Comass(\rho)$-calibrated, and for $\lambda$ ranging over measured oriented laminations,
\begin{equation}\label{duality between stable and comass}
\Comass(\rho) = \sup_\lambda \frac{\langle \rho, [\lambda]\rangle}{\Mass(\lambda)} = \frac{\langle \rho, [\kappa]\rangle}{\Mass(\kappa)}.
\end{equation}
\end{mainthm}

An analogous result to (\ref{duality between stable and comass}), often summarized as ``$L = K$,'' was the main theorem of Thurston's seminal paper on best Lipschitz maps \cite{Thurston98}.
Here $L$ is the Lipschitz constant of a given best Lipschitz map $f$ (which plays the role of $F$) and $K$ is the maximal amount that a geodesic lamination may be stretched by $f$.
In fact, every maximally stretched lamination is contained in a geodesic lamination called the \dfn{canonical lamination} \cite{Thurston98,Gu_ritaud_2017}.
We construct an analogous lamination for best comass calibrations; this is Proposition \ref{existence of canonical lamination}.

\begin{mainthm}\label{existence of calibrated lam}
Suppose that $M$ is a closed Riemannian manifold of dimension $d \leq 7$, and let $\rho \in H^{d - 1}(M, \RR)$ satisfy $\Comass(\rho) = 1$.
Then there is a Lipschitz lamination $\lambda_\rho$, the \dfn{canonical lamination} of $\rho$, such that a complete hypersurface $N$ is a leaf of $\lambda_\rho$ iff, for every best comass representative $F$ of $\rho$, $N$ is $F$-calibrated.
The support of $\lambda_\rho$ is contained in the maximum comass locus of every best comass representative $F$.
\end{mainthm}

In addition to Theorem \ref{lams are calibrated}, which is necessary to show that $\lambda_\rho$ is nonempty, and curvature bounds on homologically minimizing hypersurfaces \cite{Schoen75, Schoen81}, which are necessary to construct the Lipschitz flow boxes \cite{BackusCML}, we also have to show that if $v$ is a harmonic function, then almost every zero of $v$ is a single zero.
From this it follows that any two calibrated hypersurfaces which intersect, must somewhere intersect transversely, contradicting the definition of calibration; this implies Theorem \ref{existence of calibrated lam}.
The highlight of the proof is an application of the rectifiability of the set $P$ of double zeroes of $v$ \cite{Hardt89} to bound the cohomological dimension of $P$.

We then study the structure of the canonical lamination $\lambda_\rho$.
The canonical lamination is covered by the measured stretch laminations which are associated to $\rho$, along with some leaves which do not admit a transverse measure.
The measured stretch laminations, in turn, are exactly those homologically minimizing laminations which represent homology classes in the convex set 
$$\rho^* := \{\alpha \in H_{d - 1}(M, \RR): \langle \rho, \alpha\rangle = \Mass(\alpha) = 1\}.$$
A special role is played by the extreme points of $\rho^*$, which are represented by measured stretch laminations with no proper sublamination.
We shall not attempt to summarize our results on the structure of $\lambda_\rho$ into a clean theorem here, but refer the reader to \S\ref{canonical structure}.

%%%%%%%%%%%%%%%%%%%%
% \subsection{Convexity of the stable unit ball}
% It is known that convexity properties of the stable unit ball 
% $$B := \{\alpha \in H_{d - 1}(M, \RR): \Mass(\alpha) \leq 1\}$$
% are intimately related to the structure of homologically minimizing laminations \cite{Thurston98,Auer01}.
% From our perspective, this is because $\rho^*$ is a \dfn{flat} of $\partial B$ -- that is, the intersection of $\partial B$ with a supporting hyperplane of $B$.

% Though the proofs were never published, Auer and Bangert announced several results on the structure of $B$ using laminations \cite{Auer01}.
% In fact, some of these results immediately follow from the structure of the canonical lamination, and we now record them.

% Recall that the exterior product on the cohomology ring $H^\bullet(M, \RR)$ induces, by Poincar\'e duality, an \dfn{intersection product}
% \begin{align*}
% H_{d - k}(M, \RR) \times H_{d - \ell}(M, \RR) &\to H_{d - k - \ell}(M, \RR) \\
% (\alpha, \beta) &\mapsto \alpha \cdot \beta.
% \end{align*}
% Also recall that $B$ is strictly convex iff $\partial B$ has no flats except singletons.
% Then we have the following theorem, which is Proposition \ref{flats are nonintersecting}:

% \begin{mainthm}\label{convexity summary}
% Let $B$ be the stable unit ball of a closed Riemannian manifold $M$ of dimension $\leq 7$, and let $S$ be a flat of $\partial B$.
% Then for any $\alpha, \beta \in S$, $\alpha \cdot \beta = 0$.
% \end{mainthm}

% Taken in the contrapositive, Theorem \ref{convexity summary} gives a condition for strict convexity of $B$.
% In particular, if $M$ is homeomorphic to a torus, then $B$ is strictly convex (see Example \ref{torus convex}).

%%%%%%%%%%%%%%%%%%%%%
\subsection{The PDE for a tight form}
At present there is not a suitable theory of viscosity solutions of PDEs for vector-valued maps, and in general, one does not expect our tight forms to be much more regular than $L^\infty$.
However, we formally derive what analytic properties tight forms \emph{should} have, in the tradition of various other papers \cite{Barron2001,Aronsson67,Sheffield12} on the $L^\infty$ calculus of variations.

To be more precise, in \S\ref{infinityMax} we study $C^1$ solutions of the PDE 
\begin{equation}\label{tight Einstein}
\begin{cases}\dif F = 0, \\
	(\nabla_i F_{j_1 \cdots j_{d - 1}}) F^{j_1 \cdots j_{d - 1}} {F^i}_{k_1 \cdots k_{d - 2}} = 0.
\end{cases}
\end{equation}
We derive this equation as the formal limit of the PDE solved by the $p$-tight forms.
Therefore tight forms are variational solutions of (\ref{tight Einstein}).

When $d = 2$, (\ref{tight Einstein}) asserts that $F$ has an $\infty$-harmonic scalar potential.
In general, (\ref{tight Einstein}) interpreted as asserting that the distribution $\ker(\star F)$ is ``calibrated" by $F$, though $\ker(\star F)$ need not be integrable.
If it is integrable, then we show in \S\ref{EL interpretation} the following variational interpretation which generalizes the interpretation of $\infty$-harmonic functions as absolute minimizing Lipschitz. \todo{If the Caccioppoli estimates approach pans out, cut this theorem and refer to the other paper!}

\begin{mainthm}\label{tight are absolute minimizers}
Let $F$ be a closed $C^1$ $d - 1$-form on a compact Riemannian manifold (possibly with boundary). Then:
\begin{enumerate}
\item If, for every sufficiently ball $B$,
\begin{equation}\label{absolute minimizer}
\|F\|_{C^0(B)} = \|F\|_{C^0(\partial B)},
\end{equation}
then $F$ solves (\ref{tight Einstein}).
\item Suppose that $\ker(\star F)$ is a singular integrable distribution. If $F$ solves (\ref{tight Einstein}), then for every sufficiently small ball $B$, (\ref{absolute minimizer}) holds.
\end{enumerate}
\end{mainthm}

%%%%%%%%%%%%%%%%%%%%%
\subsection{Outline of the paper}
In \S\ref{prevResults} we review known results on geometric measure theory, convex optimization, functions of least gradient, and laminations.

In \S\ref{comass sec} we study $L^\infty$ calibrations. We also introduce the \dfn{local comass} of a form and show that it has analogous properties to the local Lipschitz constant of a function.

In \S\ref{tight forms sec} we study the $q$-Laplacian and its convex dual problem in the limit $q \to 1$. We construct the tight form in each cohomology class and its dual function of least gradient, proving Theorem \ref{existence of infinity tight forms}.

In \S\ref{MCL sec} we study the homologically minimizing lamination arising from the dual function of least gradient. We show that it is calibrated by the tight form, proving Theorem \ref{lams are calibrated}. We also give a condition for the lamination to only have closed leaves.

In \S\ref{canonical sec} we construct the canonical lamination $\lambda_\rho$ of a cohomology class $\rho$, proving Theorem \ref{existence of calibrated lam}. We then establish various results about the structure of $\lambda_\rho$ and, as an application, the duality between stable and costable norms.

In \S\ref{infinityMax} we study the PDE (\ref{tight Einstein}) and prove Theorem \ref{tight are absolute minimizers}.

In \S\ref{open problems} we make some concluding remarks. We discuss some external motivation and some applications for the canonical lamination and the PDE (\ref{tight Einstein}). We also give some open problems and discuss how the results of this paper could naturally be generalized.

%%%%%%%%%%%%%%%%%%%%%%
\subsection{Acknowledgements}
I would like to thank Georgios Daskalopolous and Karen Uhlenbeck for suggesting this project and providing helpful comments, and for providing me with a draft copy of \cite{daskalopoulos2023}.
I would also like to thank Victor Bangert for providing me with a copy of \cite{Auer12}, Bernd Kawohl for suggesting the references \cite{Kawohl2003, Grieser05}, and Anatole Gaudin for suggesting the reference \cite{Costabel2010}.

This research was supported by the National Science Foundation's Graduate Research Fellowship Program under Grant No. DGE-2040433.


%%%%%%%%%%%%%%%%%%%%%%%%%%%%%%%%%%%%%%%%%%
\section{Preliminaries}\label{prevResults}
\subsection{Locally normal currents}
In order to fix conventions, we recall some well-known measure theory.
We will mainly use \cite{simon1983GMT} as a reference.

The operator $\star$ is the Hodge star on $M$, thus $\star 1$ is the Riemannian measure of $M$.
We denote the musical isomorphisms by $\sharp, \flat$.
To avoid confusion, we write $H^\ell$ for de Rham cohomology, but never a Sobolev space, which we instead denote $W^{\ell, p}$.
The manifold $\Ball^d$ is the unit ball in $\RR^d$, $\Sph^d$ is the unit sphere in $\RR^{d + 1}$, and $\Hyp^d$ is the hyperbolic space.

The $\delta$-dimensional Hausdorff measure is $\mathcal H^\delta$, normalized so that if $\delta$ is an integer, then $\mathcal H^\delta$ is $\delta$-dimensional Riemannian measure.
If $\tau$ is a $\delta$-rectifiable set, we write $|\tau| := \mathcal H^\delta(\tau)$ and $\dif S_\tau := \mathcal H^\delta|_\tau$.

The sheaf of $\ell$-forms is denoted $\Omega^\ell$, and the sheaf of closed $\ell$-forms is denoted $\Omega^\ell_{\rm cl}$.
We are often interested in those sections of a sheaf $\mathscr F$ of a certain regularity, so we will write, for example, $L^p(U, \mathscr F)$.
We assume that $\ell$-forms are $L^1_\loc$, but \emph{not} that they are continuous; hence $\dif$ must be meant in the sense of distributions.
To avoid confusion, we write $H^\ell$ for de Rham cohomology, but never a Sobolev space, which we instead denote $W^{\ell, p}$.

\begin{definition}
An \dfn{$\ell$-blade} is the wedge product $v = v_1 \wedge \cdots \wedge v_\ell$ of vectors $v_1, \dots, v_\ell$.
The \dfn{comass} $|\varphi|$ of an $\ell$-covector $\varphi$ is the supremum of $\langle \varphi, v\rangle/|v|$, taken over all nonzero $\ell$-blades $v$.
\end{definition}

\begin{definition}
By an $\ell$-\dfn{current of locally finite mass} $T$ we mean a continuous linear functional on the space $C^0_\cpt(M, \Omega^\ell)$ of continuous $\ell$-forms of compact support.
If we do not specify otherwise, by a \dfn{current} we shall mean a current of locally finite mass.
We write $\int_M T \wedge \varphi$ for the dual pairing of a current and a form.
The duality norm of $T$ is its \dfn{mass}, namely for an open set $U \subseteq M$,
$$\Mass_U(T) := \sup_{\substack{\supp \varphi \Subset U \\ \||\varphi|\|_{C^0} \leq 1}} \int_U T \wedge \varphi.$$
We write $\Mass(T) := \Mass_M(T)$.
\end{definition}

If $T$ represents a simplex $\sigma$, then $\Mass(T)$ is the surface area of $\sigma$.
We write $\dif T$ for the exterior derivative of a current $T$, which is also known as $-\partial T$ \cite[\S26]{simon1983GMT}.
If $\dif T$ is an $\ell - 1$-current of locally finite mass, we say that $T$ is \dfn{locally normal}.
In particular, a locally normal $d$-current is the same thing as a function of locally bounded variation, and a locally normal $0$-current is the same thing as a signed Radon measure.

\begin{definition}
An $\ell$-current $\sigma$ is \dfn{rectifiable} if there exists an $\ell$-rectifiable set $N$, and a $\dif S_N$-measurable $\ell$-blade field $v$, such that for any $C^0_\cpt$ $\ell$-form $\varphi$,
$$\int_M \sigma \wedge \varphi = \int_N \langle \varphi, v\rangle \dif S_N,$$
and for $\dif S_N$-almost every $x \in N$, $|v(x)| \in \ZZ$.
The rectifiable current $\sigma$ is \dfn{integral} if, in addition, $\dif \sigma$ is rectifiable.
To emphasize that integral currents $\sigma$ can be represented by integration along rectifiable sets, we write $\partial \sigma := -\dif \sigma$ and $\int_\sigma \varphi := \int_M \sigma \wedge \varphi$.
\end{definition}

% \begin{definition}
% The \dfn{stable norm} $\Mass(\alpha)$ of $\alpha \in H_{d - 1}(M, \RR)$ is the infimum of $\Mass(T)$ where $T$ is a $d - 1$-current of homology class $\alpha$.
% Dually, the \dfn{costable norm} $\Comass(\rho)$ of $\rho \in H^{d - 1}(M, \RR)$ is the infimum of $\|F\|_{L^\infty}$ where $F$ is a $d - 1$-form of cohomology class $\rho$.
% We say that a $d - 1$-form $F$ of cohomology class $\rho$ has \dfn{best comass} if 
% $$\|F\|_{L^\infty} = \Comass(\rho).$$
% \end{definition}

%%%%%%%%%%%%%%%%%%%%%
\subsection{Calculus with rough differential forms}
We call a Riemannian manifold $M$ a \dfn{deformed ball} if there is a bi-Lipschitz diffeomorphism $M \cong \Ball^d$.
We shall appeal to results of Anzellotti \cite{Anzellotti1983} which are carried out on $\Ball^d$.
We can formulate these results in the language of differential forms, which is \emph{diffeomorphism-invariant}, so the proofs go through unchanged if $\Ball^d$ is replaced by a deformed ball.\footnote{Strictly speaking, the $L^p$ and $\Mass$ norms depend on the metric, but every property used about these norms in \cite{Anzellotti1983} also holds in the presence of a Riemannian metric, with the same proof.}

If $F$ is an $L^1_\loc$ $d - 1$-form, then $\dif F$ is a $2$-current (possibly not of finite mass), and we introduce the norms
\begin{align*}
\|F\|_{X^p} &:= \|F\|_{L^\infty} + \|\dif F\|_{L^p}, \\
\|F\|_Y &:= \|F\|_{L^\infty} + \Mass(\dif F),
\end{align*}
which define Banach spaces $X^p, Y$.
If $\chi$ is a smooth cutoff function, then multiplication by $\chi$ defines bounded linear endomorphisms of $X^p, Y$.
So, by covering a Riemannian manifold $M$ by deformed balls and applying a partition of unity, we see that Anzellotti theory holds for $M$.

\begin{proposition}[trace theorem {\cite[Theorem 1.2]{Anzellotti1983}}]\label{integration is welldefined}
Let $\iota: N \to M$ be the inclusion of an oriented Lipschitz hypersurface.
Then the pullback $\iota^*$ of $d - 1$-forms extends to a bounded linear map $\iota^*: Y \to L^\infty(N)$ such that
\begin{equation}\label{integral over chain is linfinity}
	\|\iota^* F\|_{L^\infty(N)} \leq \|F\|_{L^\infty(M)}.
\end{equation}
\end{proposition}

If $U$ is a set of locally finite perimeter (thus $1_U \in BV_\loc$), we recall the \dfn{measure-theoretic boundary}, the set $\partial U$ of $x \in M$ such that for all $\varepsilon > 0$,
$$0 < \Mass(U \cap B(x, \varepsilon)) < \Mass(B(x, \varepsilon)).$$
By \cite[Theorem 4.4]{Giusti77}, $\partial U$ is an integral $d - 1$-current (in particular, the sum of Lipschitz hypersurfaces), with surface measure $\star |\dif 1_U|$; moreover, $\dif 1_U$ is conormal to $\partial U$.
By the coarea formula \cite[Theorem 1.23]{Giusti77}, the superlevel sets $\{u > \lambda\}$ of a function $u \in BV_\loc$ have locally finite perimeter.

\begin{proposition}[coarea formula]
Let $u \in BV$ and $F \in X^d$.
Define a $0$-current $\dif u \wedge F$ (possibly not of locally finite mass) by declaring that for every $\chi \in C^\infty_\cpt$,
$$\int_M \chi \dif u \wedge F = -\int_M u \dif \chi \wedge F - \int_M \chi u \dif F.$$
Then $\dif u \wedge F$ has finite mass, and for any $\chi \in C^0_\cpt$,
\begin{equation}\label{coarea formula}
\int_M \chi \dif u \wedge F = \int_{-\infty}^\infty \int_{\partial \{u > \lambda\}} \chi F \dif \lambda.
\end{equation}
\end{proposition}
\begin{proof}
Observe that for $p := \frac{d}{d - 1}$, $(p, d)$ is a H\"older pair, and by the isoperimetric inequality, $u \in L^p$.
From these assertions and \cite[Theorem 1.5]{Anzellotti1983}, $\dif u \wedge F$ has finite mass.
Since $\star|\dif 1_{\{u > \lambda\}}|$ is the surface measure on $\partial \{u > \lambda\}$ and $\dif 1_{\{u > \lambda\}}$ is conormal to $\partial \{u > \lambda\}$, we compute using \cite[Proposition 2.7(ii)]{Anzellotti1983} that
\begin{align*}
\int_{-\infty}^\infty \int_{\partial \{u > \lambda\}} \chi F \dif \lambda = \int_{-\infty}^\infty \int_M \chi \dif 1_{\{u > \lambda\}} \wedge F \dif \lambda &= \int_M \chi \dif u \wedge F. \qedhere 
\end{align*}
\end{proof}

%%%%%%%%%%%%%%%%%%%%%
\subsection{\texorpdfstring{$1$-harmonic functions}{One-harmonic functions}}
We shall consider the variational problems whose Euler-Lagrange equation is, at least at the formal level, the $1$-Laplacian
\begin{equation}\label{1Laplacian}
\dif^*\left(\frac{\dif u}{|\dif u|}\right) = 0.
\end{equation}
A suitable notion of weak solution for (\ref{1Laplacian}), at least for the Dirichlet problem, was introduced by Maz\'on, Rossi, and Segura de L\'eon \cite{Mazon14}; it essentially asserts that the level sets of $u$ are calibrated.

Traditionally, authors have studied the Dirichlet problem for the $1$-Laplacian.
We will instead be interested in the topological Neumann problem, which we now formulate.
Let $M$ be a closed Riemannian manifold with fundamental group $\Gamma$ and universal cover $\tilde M \to M$.
By Poincar\'e duality and the Hurcewiz theorem, we have canonical isomorphisms
\begin{equation}\label{Poincare Hurcewiz}
H_{d - 1}(M, \RR) = H^1(M, \RR) = \Hom(\Gamma, \RR).
\end{equation}
Given a representation $\alpha: \Gamma \to \RR$, which we identify with a class in $H_{d - 1}(M, \RR)$ using (\ref{Poincare Hurcewiz}), we will be interested in $\alpha$-equivariant functions $f: \tilde M \to \RR$, namely those which satisfy (for each $\gamma \in \Gamma$)
$$f(\gamma x) = f(x) + \alpha(\gamma).$$
If $f$ is $\alpha$-equivariant, then $\dif f$ drops to a current on $M$, which we also call $\dif f$, and has homology class $\alpha$.

\begin{definition}
An $\alpha$-equivariant function $u \in BV_\loc(\tilde M)$ has \dfn{least gradient} if the current $\dif u$ on $M$ satisfies 
$$\Mass(\dif u) = \Mass(\alpha).$$
\end{definition}

In other words, $u$ has least gradient if $\dif u$ has the smallest possible mass among $\alpha$-equivariant functions.

\begin{definition}
We say that an $\alpha$-equivariant function $u \in BV_\loc(\tilde M)$ is a \dfn{calibrated solution} of (\ref{1Laplacian}) if there exists a $L^\infty$ $d - 1$-form $F$ on $M$ such that
\begin{equation}\label{local calibration}
\begin{cases}
\|F\|_{L^\infty} \leq 1, \\
dF = 0, \\
\dif u \wedge F = \star |\dif u|.
\end{cases}
\end{equation}
\end{definition}

We informally refer to calibrated solutions of (\ref{1Laplacian}) as \dfn{$1$-harmonic functions}, though this terminology is not quite precise.
This formulation of calibrated solution is not worded the same as the formulation for the Dirichlet problem given by \cite{Mazon14}, but it is equivalent; their vector field $X$ is given by $(\star F)^\sharp$.
The quantity $\dif u \wedge F$ is well-defined by the coarea formula, since $\dif F = 0$.
A straightforward modification of \cite[Theorem 1.1]{Mazon14} (originally proven for the Dirichlet problem) gives:

\begin{theorem}\label{MazonRossi}
An $\alpha$-equivariant function $u \in BV_\loc(M)$ is a calibrated solution of (\ref{1Laplacian}) iff $u$ has least gradient.
\end{theorem}

% \begin{theorem}\label{main thm of old paper}
% Let $u \in BV(M)$ have least gradient in $M$. Then $1_{\{u > y\}}$ has least gradient.
% In particular, if $d \leq 7$, then every superlevel set $\{u > y\}$ is bounded by complete disjoint embedded oriented minimal hypersurfaces.
% \end{theorem}
% \begin{proof}
% Let $v := 1_{\{u > y\}}$.
% Then $v$ has least gradient \cite[Theorem 1]{BOMBIERI1969}.
% In particular, $\{u > y\}$ has finite perimeter, so it is bounded by Lipschitz hypersurfaces \cite[Chapter 4]{Giusti77}.
% So by the regularity theorem for minimal hypersurfaces \cite[\S37]{simon1983GMT}\footnote{See also \cite[Exercise 1.6]{DeLellis18} for a discussion on why the theory of \cite[\S37]{simon1983GMT} applies for arbitrary metrics, and \cite{BackusFLG} for a proof for functions of least gradient for arbitrary metrics in the spirit of Miranda \cite{Miranda66}'s original argument for functions of least gradient.} if $d \leq 7$, $\{u > y\}$ is bounded by disjoint embedded minimal hypersurfaces.
% These hypersurfaces inherit an orientation from the current $\dif v$.
% \end{proof}

%%%%%%%%%%%%%%%%%%%%%%%%%%%%%%

\subsection{Minimal laminations}
We now give a geometric characterization of $1$-harmonic functions.
Fix an interval $I \subset \RR$ and a box $J \subset \RR^{d - 1}$.
We write $\Two_N$ for the second fundamental form of a submanifold $N$.

\begin{definition}
A \dfn{laminar flow box} is a $C^0$ coordinate chart $F: I \times J \to M$ and a compact set $K \subseteq I$, such that for every $k \in K$, $F|_{\{k\} \times J}$ is a $C^1$ embedding, and the \dfn{leaf} $F(\{k\} \times J)$ is a $C^1$ complete hypersurface in $F(I \times J)$.
Two laminar flow boxes belong to the same \dfn{laminar atlas} if the transition maps between them send leaves to leaves.
\end{definition}

\begin{definition}
A \dfn{lamination} is a closed subset $S \subseteq M$, called its \dfn{support}, and a maximal laminar atlas $\mathscr A$, such that $S$ is the union of the leaves of $\mathscr A$.
A \dfn{foliation} is a lamination $\lambda$ with $\supp \lambda = M$.
\end{definition}

\begin{definition}
A lamination is
\begin{enumerate}
\item \dfn{Lipschitz} if its flow boxes are Lipschitz isomorphisms,
\item \dfn{oriented} if its transition maps are orientation-preserving, and
\item \dfn{minimal} if its leaves are minimal hypersurfaces.
\end{enumerate}
\end{definition}

\begin{theorem}[{\cite[Theorem A]{BackusCML}}]\label{disjoint surfaces are lamination}
Let $\mathcal S$ be a set of disjoint complete minimal hypersurfaces in a manifold $M$ of bounded geometry.
Suppose that there exists $C > 0$ such that for every $N \in \mathcal S$, $\|\Two_N\|_{C^0} \leq C$.
Then $\mathcal S$ is the set of leaves of a Lipschitz minimal lamination $\lambda$.
In particular, if $\lambda$ is oriented, then there is a Lipschitz vector field on $M$ whose restriction to each $N \in \mathcal S$ is the normal vector to $N$.
\end{theorem}

\begin{definition}
A lamination $\lambda$ with atlas $(F_\alpha, K_\alpha)$ is \dfn{measured} if it is equipped with positive Radon measures $\mu_\alpha$ with $\supp \mu_\alpha = K_\alpha$, such that the transition maps $F_\beta^{-1} \circ F_\alpha$ are measure-preserving.
The \dfn{Ruelle-Sullivan current} of a measured oriented lamination $\lambda$ with atlas $(F_\alpha, K_\alpha, \mu_\alpha)$ is the $d-1$-current $T_\lambda$ satisfying, for any partition of unity $(\chi_\alpha)$ subordinate to the open cover $(F_\alpha(I \times J))$,
$$\int_M T_\lambda \wedge \varphi = \sum_\alpha \int_{K_\alpha} \int_{\{k\} \times J} F_\alpha^* (\chi_\alpha \varphi) \dif \mu_\alpha(k).$$
The \dfn{homology class} $[\lambda]$ and \dfn{mass} $\Mass(\lambda)$ of a measured oriented lamination $\lambda$ are the homology class and mass of its Ruelle-Sullivan current.
The measured oriented lamination $\lambda$ is \dfn{homologically minimizing} if
$$\Mass(\lambda) = \Mass([\lambda]).$$
\end{definition}

The notion of Ruelle-Sullivan current  was introduced by \cite{Ruelle75} and studied in the context of geodesic laminations in \cite[\S8]{daskalopoulos2020transverse}.
The motivation of the definition is that if $\lambda$ is a $d - 1$-chain, then $T_\lambda$ is just integration along $\lambda$.

\begin{theorem}[{\cite[Theorem B]{BackusCML}}]\label{1 harmonic is MOML}
Suppose that $d \leq 7$ and let $u \in BV_\loc(\tilde M)$ be a $\Gamma$-equivariant function of least gradient.
Then $\dif u$ is the Ruelle-Sullivan current of a measured oriented Lipschitz lamination, which is minimal and homologically minimizing.
\end{theorem}

%%%%%%%%%%%%%%%%%%%%
\subsection{Convex duality}
We follow \cite{Ekeland99}.
For a reflexive Banach space $X$, we denote by $\hat X$ its dual.
If $I: X \to \RR \cup \{+\infty\}$ is a convex function, we introduce its \dfn{Legendre transform}, the convex function
\begin{align*}
	\hat I: \hat X &\to \RR \cup \{+\infty\}\\
	\xi &\mapsto \sup_{x \in X} \langle \xi, x\rangle - I(x).
\end{align*}
We identify the cokernel of a linear map with the kernel of its adjoint.
In this setting, we have the following form of the convex duality theorem.

\begin{theorem}[convex duality]\label{abstract convex analysis}
Let $\Lambda : X \to Y$ be a bounded linear map between reflexive Banach spaces.
Let $I: Y \to \RR \cup \{+\infty\}$ satisfy:
\begin{enumerate}
\item $I$ and $\hat I$ are strictly convex,
\item $I$ is lower semicontinuous,
\item if $|y| \to \infty$ in $Y$, then $I(y) \to +\infty$, and 
\item there exists a point $x \in X$ such that $I$ is continuous and finite at $\Lambda(x)$.
\end{enumerate}
Then:
\begin{enumerate}
\item There exists a minimizer $\underline x \in X$ of $I(\Lambda(x))$, unique modulo $\ker \Lambda$.
\item There exists a unique maximizer $\overline \eta$ of $-\hat I(-\eta)$ subject to the constraint $\eta \in \coker \Lambda$.
\item We have \dfn{strong duality}
\begin{equation}\label{abstract strong duality}
I(\Lambda(\underline x)) = -\hat I(-\overline \eta).
\end{equation}
\end{enumerate}
\end{theorem}
\begin{proof}
This is largely a special case of \cite[Chapter IV, Theorem 4.2]{Ekeland99}.
Let $\mathscr P, \mathscr P^*$ be as in the statement of that theorem.
Then $\mathscr P$ is the problem of minimizing $J(x, \Lambda x)$ where $J(x, y) := I(y)$.
The Legendre transform of $J$ satisfies 
$$\hat J(\xi, \eta) = \begin{cases} \hat I(\eta), & \xi = 0, \\
	+\infty, &\xi \neq 0,
\end{cases}$$
and $\mathscr P^*$ is the problem of maximizing
$$-\hat J(\Lambda^* \eta, -\eta) = \begin{cases}
	-\hat I(-\eta), &\eta \in \ker \Lambda^*, \\
	-\infty, &\eta \notin \ker \Lambda^*,
\end{cases}$$
where $\Lambda^*$ is the adjoint of $\Lambda$.
Then most of the various assertions of this theorem follow immediately from \cite[Chapter IV, Theorem 4.2]{Ekeland99}.
The fact that $\overline \eta \in \coker \Lambda$ follows from the facts that $\overline \eta$ is a solution of $\mathscr P^*$, but any solution of $\mathscr P^*$ must be a member of $\ker \Lambda^*$. 
To establish uniqueness, we use \cite[Chapter II, Proposition 1.2]{Ekeland99}, the fact that $\hat I$ is strictly convex, and the fact that we may view $I \circ \Lambda$ as a strictly convex function on the reflexive Banach space $X/\ker \Lambda$.
\end{proof}



%%%%%%%%%%%%%%%%%%%%%%%%%%%%%%%%%%%%%%%%%%

\section{\texorpdfstring{Closed $L^\infty$ $d - 1$-forms}{Closed (d - 1)-forms of finite comass}}\label{comass sec}
The goal of the present series of papers is to study calibrations of laminations.
If the calibration $F$ is continuous, then the theory of \cite{bangert_cui_2017} of calibrated laminations applies; however, we are not aware of general existence results on continuous calibrations, only $L^\infty$ calibrations, such as \cite[\S4.12]{Federer1974}.
Thus, we now introduce $L^\infty$ calibrations of laminations.

\begin{definition}
A \dfn{calibration} of degree $d - 1$ is a closed $d - 1$-form $F$ such that $\Comass(F) = 1$.
An integral $d - 1$-current $\sigma$ is $F$-\dfn{calibrated} if 
$$\Mass(\sigma) = \int_\sigma F.$$
A lamination $\lambda$ is \dfn{$F$-calibrated} if every leaf of $\lambda$ is $F$-calibrated.
\end{definition}

The fundamental theorem of calibrated geometry \cite{Harvey82} asserts that a calibrated integral current is homologically minimizing relative to its boundary (and hence minimal).
Therefore every calibrated lamination $\lambda$ is minimal, and if $\lambda$ is measured oriented (so the homology class $[\lambda]$ is well-defined), $\lambda$ is homologically minimizing.

\subsection{Local comass}
We will be interested in the points at which the $L^\infty$ calibration $F$ attains its comass.
One could pose this problem as the problem of computing the locus $\{|F| = \|F\|_{L^\infty}\}$.
However, $|F(x)|$, the comass of the $d - 1$-covector $F(x)$, is both only defined for almost every $x$, and $F$ is not norm-approximable by smooth functions.
So as a proxy for $|F|$, which may fail to be defined on a null set, we use the local comass, which is defined everywhere.

\begin{definition}
For an open set $\Omega \subseteq M$, let $\Chain_{d - 1}(\Omega)$ be the set of simplicial $d - 1$-chains in $\Omega$.
For each closed $d - 1$-form $F$ on $\Omega$, let
$$\Comass_\Omega(F) := \sup_{\sigma \in \Chain_{d - 1}(\Omega)} \frac{1}{\Mass(\sigma)} \int_\sigma F.$$
The \dfn{local comass} of a closed $d - 1$-form $F$ at $x \in M$ is 
$$\Comass(F, x) = \limsup_{\varepsilon \to 0} \Comass_{B_\varepsilon(x)}(F).$$
\end{definition}

By the trace theorem, $\Comass_\Omega(F)$ is well-defined (but possibly $+\infty$) for any measurable $F$.
Since $\Comass_{B_\varepsilon(x)}(F)$ is a supremum over a set which grows in $\varepsilon$, it is increasing in $\varepsilon$, so the limit superior is actually a limit and an infimum:
$$\Comass(F, x) = \lim_{\varepsilon \to 0} \Comass_{B_\varepsilon(x)}(F) = \inf_{\varepsilon > 0} \Comass_{B_\varepsilon(x)}(F).$$
In particular, if we write $\Comass(F) := \Comass_M(F)$, then $\Comass(F, x) \leq \Comass(F)$.

The local comass was defined in an analogous manner to the local Lipschitz constant.
As such, it enjoys many of the same properties, including those endowed on the local Lipschitz constant by \cite[Lemma 4.3]{Crandall2008}:

\begin{proposition}\label{crandall}
Let $F \in L^\infty(M, \Omega^{d - 1}_{\rm cl})$. Then:
\begin{enumerate}
\item The local comass $\Comass(F, \cdot)$ is upper semicontinuous. \label{crandall usc}
\item For almost every $x \in M$, \label{crandall LDT}
$$|F(x)| \leq \Comass(F, x).$$
\item The local comass is bounded, and \label{crandall linfinity}
$$\Comass(F) = \sup_{x \in M} \Comass(F, x) = \|F\|_{L^\infty}.$$
\item If $\sigma \in \Chain_{d - 1}(M)$ then \label{crandall best curl is ABC}
$$\int_\sigma F \leq \Mass(\sigma) \sup_{x \in \sigma} \Comass(F, x).$$
\end{enumerate}
\end{proposition}
\begin{proof}
We first prove (\ref{crandall usc}).
Let $x^n \to x$ and $r > 0$. Then eventually $x^n \in B_r(x)$, hence $\Comass(F, x^n) \leq \Comass_{B_r(x)}(F)$ and so
\begin{align*}
\limsup_{n \to \infty} \Comass(F, x^n) &\leq \inf_{r > 0} \Comass_{B_r(x)}(F) = \Comass(F, x).
\end{align*}

We now prove (\ref{crandall LDT}).
We may work locally, and choose coordinates $(y^i)$ in which $\sqrt{\det g} = 1$.
Let $I$ be the increasing $d-1$-index with $d$ removed.
By the Lebesgue differentiation theorem and Fubini's theorem, there exists a null set $Z \subset M$, which does not depend on $(y^i)$ by \cite[Proposition 2.1]{BackusFLG}, such that for every $x \notin Z$,
\begin{align*}
F_I(x) 
&= \lim_{\varepsilon \to 0} \frac{1}{\Mass(B_\varepsilon(x))} \int_{B_\varepsilon(x)} F_I(y) \dif y \\
&= \lim_{\varepsilon \to 0} \frac{1}{\Mass(B_\varepsilon(x))} \int_{-\infty}^\infty \int_{\{y^d = t\} \cap B_\varepsilon(x)} F_I(y) \dif y^1 \wedge \cdots \wedge \dif y^{d - 1} \wedge \dif t
\end{align*}
where we used the fact that $\sqrt{\det g} = 1$.
Now $\partial_{y^1} \wedge \cdots \wedge \partial_{y^{d - 1}}$ is the oriented unit $d - 1$-blade tangent to $\{y^d = t\}$, so as forms on $\{y^d = t\}$,
$$F_I(y) \dif y^1 \wedge \cdots \wedge \dif y^{d - 1} = F.$$
So
\begin{align*}
F_I(x) 
&= \lim_{\varepsilon \to 0} \frac{1}{\Mass(B_\varepsilon(x))} \int_{-\infty}^\infty \int_{\{y^d = t\} \cap B_\varepsilon(x)} F \dif t \\
&\leq \lim_{\varepsilon \to 0} \frac{\Comass_{B_\varepsilon(x)}(F)}{\Mass(B_\varepsilon(x))} \int_{-\infty}^\infty |\{y^d = t\} \cap B_\varepsilon(x)| \dif t.
\end{align*}
By Fubini's theorem,
$$F_I(x) \leq \lim_{\varepsilon \to 0} \frac{\Comass_{B_\varepsilon(x)}(F)}{\Mass(B_\varepsilon(x))} \Mass(B_\varepsilon(x)) = \Comass(F, x).$$
For every $x \in M$ we may select coordinates in which $|F(x)| = F_I(x)$, and then if $x \notin Z$, we conclude that (\ref{crandall LDT}) holds for $x$.

If we combine (\ref{crandall LDT}) with (\ref{integral over chain is linfinity}), then
$$\sup_{x \in M} \Comass(F, x) \leq \Comass(F) \leq \|F\|_{L^\infty} \leq \sup_{x \in M} \Comass(F, x).$$
The inequalities collapse, proving (\ref{crandall linfinity}).
In particular, for each $\sigma \in \Chain_{d - 1}(M)$, we obtain (\ref{crandall best curl is ABC}):
\begin{align*}
\int_\sigma F &\leq \Mass(\sigma) \inf_{\Omega \supset \sigma} \sup_{x \in \Omega} \Comass(F, x) = \Mass(\sigma) \sup_{x \in \sigma} \Comass(F, x). \qedhere
\end{align*}
\end{proof}

\begin{definition}
Let $F \in L^\infty(M, \Omega^{d - 1}_{\rm cl})$.
The \dfn{maximum comass locus} is the set
$$\MCL(F) := \{x \in M: \Comass(F, x) = \Comass(F)\}.$$
\end{definition}

\begin{corollary}
Suppose that $M$ is a closed manifold and $F \in L^\infty(M, \Omega^{d - 1}_{\rm cl})$.
Then $\MCL(F)$ is a nonempty compact set.
\end{corollary}
\begin{proof}
This is immediate from Proposition \ref{crandall}(\ref{crandall usc}) and the extreme value theorem for upper semicontinuous functions.
\end{proof}

%%%%%%%%%%%%%%%%%%%%%%%
\subsection{\texorpdfstring{$L^\infty$}{L-infinity} calibrations}
When working with $L^\infty$ calibrations $F$, it is often convenient to replace $F$ with a continuous ``gauge field'' $A$; this is always possible, as we now assert:

\begin{lemma}\label{Hodge theorem}
Suppose that $U$ is a deformed ball, and let $F \in L^\infty(U, \Omega^\ell_{\rm cl})$.
Then there exists $A \in C^{1/2}(U, \Omega^{\ell - 1})$ such that $\dif A = F$.
\end{lemma}
\begin{proof}
There exists a solution operator $T$ of the equation $\dif A = F$ such that for any $1 < p < \infty$, $T$ is a bounded linear map $L^p(U, \Omega^\ell_{\rm cl}) \to W^{1, p}(U, \Omega^{\ell - 1})$ \cite{Costabel2010}.
The result now follows from the Sobolev embedding theorem if we take $p$ large enough.
\end{proof}

It follows from the definitions that if a smooth hypersurface $N$ is $F$-calibrated, then the pullback of $F$ along $N$ is the area form $\dif S_N$.
In particular, $\iota_N^* F$ is smooth and $N$ is oriented.
If $\lambda$ is an $F$-calibrated lamination, then the trace $\iota_\lambda^* F$ is defined to be $\iota_N^* F$ along any leaf $N$.
A priori, $\lambda$ could fail to be orientable, and then $\iota_\lambda^* F$ would be necessarily discontinuous.
However, we assert that a calibrated lamination is orientable:

\begin{proposition}\label{calibrated implies oriented}
Let $F$ be a calibration and $\lambda$ an $F$-calibrated lamination.
Then $\iota_\lambda^* F$ is continuous, and $F$ induces an orientation on $\lambda$.
\end{proposition}
\begin{proof}
Let $N$ be a leaf of $\lambda$ and $x \in N$.
Let $\mathscr O$ be the local orientation of $\lambda$ near $x$ which is compatible with $F(x)$. 

There exists a continuous $d - 1$-form $G$ such that $y \in K$ close enough to $x$, $K$ a leaf of $\lambda$, $G(y) = \dif S_K(y)$ is the area form of $K$ with respect to $\mathscr O$.
In fact, we can fill in the spaces between the leaves by linear interpolation.
Then for any $y \in \supp \lambda$ close to $x$, either $F(y) = G(y)$ or $F(y) = -G(y)$; we claim that $F(y) = G(y)$ if $\dist(x, y)$ is small enough.
If this claim is true, then the proposition follows, since $G$ is continuous at $x$.

To prove the claim, we suppose towards contradiction that there is a sequence $(x_n) \subset \supp \lambda$ with $x_n \to x$ and $F(x_n) = -G(x_n)$.
We may write $F = \dif A$ where $A$ is continuous near $x$ by Lemma \ref{Hodge theorem}.
Let $N_n$ be the leaf of $\lambda$ containing $x_n$; then $N_n$ is minimal, hence smooth, so $F_n|_{N_n}$ is smooth.
Therefore by Stokes' theorem, if $(D_{n, m})$ is a sequence of disks in $N_n$ which shrink down to $x_n$ as $m \to \infty$ and are equipped with the orientation $\mathscr O$,
$$-1 = \lim_{m \to \infty} \frac{1}{\Mass(D_{n, m})} \int_{D_{n, m}} F = \lim_{m \to \infty} \frac{1}{\Mass(D_{n, m})} \int_{\partial D_{n, m}} A.$$
In particular, we can choose $m_n \to \infty$ such that 
\begin{equation}\label{orientation contradiction}
\int_{\partial D_{n, m_n}} A \leq 0.
\end{equation}
We set $D_n := D_{n, m_n}$.

Let $(k, z) \in \RR \times \RR^{d - 1}$ be coordinates near $x$ with respect to a flow box, where $k$ indexes the leaves and $z$ is a parameter on each leaf.
Then $D_n = \{k_n\} \times \Omega_n$ for some $\Omega_n \subset \RR^{d - 1}$.
Let $k$ be the index of $N$ and $D_n' := \{k\} \times \Omega_n$.
Since $A$ is continuous, $A(k, z) = A(k_n, z) + o(1)$ as $n \to \infty$, hence
\begin{equation}\label{orientation contradiction 2}
\frac{1}{\Mass(D_n')} \int_{\partial D_n'} A = \frac{1}{\Mass(D_n)} \int_{\partial D_n} A + o(1).
\end{equation}
However, $D_n'$ shrinks down to $x$, and $F(x) = G(x)$, so by Stokes' theorem, for $n$ large,
$$\int_{\partial D_n'} A \gtrsim \Mass(D_n'),$$
hence by (\ref{orientation contradiction}) and (\ref{orientation contradiction 2}),
$$0 < \Mass(D_n) \lesssim \int_{\partial D_n} A \leq 0,$$
a contradiction.
\end{proof}

We next give a natural condition for a lamination to be calibrated.
To make sense of it, observe that if $M$ is closed, $\lambda$ is a lamination, and $F$ is a calibration, then the quantity $\int_M T_\lambda \wedge F$ is well-defined, since it is just the pairing $\langle [F], [\lambda]\rangle$ of cohomology with homology.

\begin{proposition}\label{calibration condition}
Let $F$ be a calibration on a closed Riemannian manifold $M$.
Let $T_\lambda$ be the Ruelle-Sullivan current of a measured oriented lamination $\lambda$.
Then the following are equivalent:
\begin{enumerate}
\item One has \begin{equation}\label{calibration by Ruelle Sullivan}
\int_M T_\lambda \wedge F = \Mass(\lambda).
\end{equation}
\item $\lambda$ is $F$-calibrated.
\end{enumerate}
\end{proposition}
\begin{proof}
First suppose that (\ref{calibration by Ruelle Sullivan}) holds.
Let $(\chi_\alpha)$ be a locally finite partition of unity subordinate to an open cover $(U_\alpha)$ of flow boxes for $\lambda$, and let $(\mu_\alpha)$ be the transverse measure.
After refining $(U_\alpha)$ we may assume that $U_\alpha$ is contained in an open set which is bi-Lipschitz diffeomorphic to $\Ball^d$. After shrinking $U_\alpha$ we may assume that $\chi_\alpha > 0$ on $U_\alpha$.
Then for some hypersurfaces $\sigma_{\alpha,k}$,
$$\Mass(\lambda) = \int_M T_\lambda \wedge F = \sum_\alpha \int_I \int_{\sigma_{\alpha,k}} \chi_\alpha F \dif \mu_\alpha(k).$$
Let $\dif S_{\alpha,k}$ be the surface measure on $\sigma_{\alpha,k}$. Then
$$\int_M \chi_\alpha \star |T_\lambda| = \int_I \int_{\sigma_{\alpha,k}} \chi_\alpha \dif S_{\alpha,k} \dif \mu_\alpha(k),$$
so summing in $\alpha$, we obtain 
\begin{equation}\label{calibration condition contr}
\sum_\alpha \int_I \int_{\sigma_{\alpha,k}} \chi_\alpha F \dif \mu_\alpha(k) = \Mass(\lambda) = \sum_\alpha \int_I \int_{\sigma_{\alpha,k}} \chi_\alpha \dif S_{\alpha,k} \dif \mu_\alpha(k).
\end{equation}

We claim that $\lambda$ is \dfn{almost calibrated} in the sense that for every $\alpha$ and $\mu_\alpha$-almost every $k$, $\sigma_{\alpha, k}$ is calibrated.
If this is not true, then we may select $\beta$ and $K \subseteq I$ with $\mu_\beta(K) > 0$, such that for every $k \in K$, $\int_{\sigma_{\beta, k}} F < \Mass(\sigma_{\beta, k})$.
Since $0 < \chi_\beta \leq 1$ and $F/\dif S_{\beta, k} \leq 1$ on $\sigma_{\beta, k}$, this is only possible if 
$$\int_{\sigma_{\beta, k}} \chi_\beta F < \int_{\sigma_{\beta, k}} \chi_\beta \dif S_{\beta, k}.$$
Integrating over $K$, and using the fact that in general we have $\int_{\sigma_{\alpha, k}} \chi_\alpha F \leq \int_{\sigma_{\alpha, k}} \chi_\alpha \dif S_{\alpha, k}$, we conclude that 
$$\sum_\alpha \int_I \int_{\sigma_{\alpha, k}} \chi_\alpha F \dif \mu_\alpha(k) < \sum_\alpha \int_I \int_{\sigma_{\alpha, k}} \chi_\alpha \dif S_{\alpha, k} \dif \mu_\alpha(k)$$
which contradicts (\ref{calibration condition contr}).

To upgrade $\lambda$ from an almost calibrated lamination to a calibrated lamination, we first, given $\sigma_{\alpha, k}$, choose $k_j$ such that $\sigma_{\alpha, k_j}$ is calibrated and $k_j \to k$.
By Lemma \ref{Hodge theorem}, we can find a continuous $d - 2$-form $A$ defined near $\sigma_{\alpha, k}$ with $F = \dif A$.
This justifies the following application of Stokes' theorem: 
$$\int_{\sigma_{\alpha, k}} F = \int_{\partial \sigma_{\alpha, k}} A.$$
Since $k_j \to k$, and $A$ is continuous,
\begin{align*}
\Mass(\sigma_{\alpha, k}) &= \lim_{j \to \infty} \Mass(\sigma_{\alpha, k_j}) = \lim_{j \to \infty} \int_{\sigma_{\alpha, k_j}} F = \lim_{j \to \infty} \int_{\partial \sigma_{\alpha, k_j}} A = \int_{\partial \sigma_{\alpha, k}} A = \int_{\sigma_{\alpha, k}} F.
\end{align*}

To establish the converse, suppose that $\lambda$ is $F$-calibrated, and let notation be as above.
Since $\lambda$ is $F$-calibrated, for every $\alpha$ and every $k$, the area form on $\sigma_{\alpha, k}$ is $F$. Therefore
\begin{align*}
\int_M T_\lambda \wedge F &= \sum_\alpha \int_I \int_{\sigma_{\alpha, k}} \chi_\alpha F \dif \mu_\alpha(k) = \Mass(T_\lambda). \qedhere
\end{align*}
\end{proof}

\begin{proposition}\label{properties of calibrated laminations}
Suppose that $M$ is a closed Riemannian manifold, $F$ is a calibration, and $\lambda$ is a measured oriented $F$-calibrated lamination.
Then:
\begin{enumerate}
\item $\lambda$ is minimal and homologically minimizing.
\item If $G$ is a calibration and cohomologous to $F$, then $\lambda$ is $G$-calibrated.
\item $\supp \lambda \subseteq \MCL(F)$.
\end{enumerate}
\end{proposition}
\begin{proof}
Every leaf of $\lambda$ is $F$-calibrated, hence minimal, so $\lambda$ is also minimal.
Since $\lambda$ is oriented by Proposition \ref{calibrated implies oriented}, it has a Ruelle-Sullivan current.
Then, since (\ref{calibration by Ruelle Sullivan}) only depends on the cohomology class of $F$, not $F$ itself, $\lambda$ is $G$-calibrated.
From (\ref{calibration by Ruelle Sullivan}) and the fundamental theorem of calibrated geometry, $\lambda$ is homologically minimizing.

Now let $S := \MCL(F)$, $N$ a leaf of $\lambda$, and suppose that $x \in N \setminus S$.
Since $S$ is closed, there exists $\varepsilon > 0$ such that $B_\varepsilon(x)$ does not meet $S$.
Moreover, $\sigma := N \cap B_\varepsilon(x)$ is a $d-1$-chain in $B_\varepsilon(x)$, so by Proposition \ref{crandall}(\ref{crandall best curl is ABC}),
$$\frac{1}{\Mass(\sigma)} \int_\sigma F \leq \sup_{y \in B_\varepsilon(x)} \Comass(F, y) < \Comass(F) = 1.$$
But then 
$$\int_N F = \int_\sigma F + \int_{N \setminus B_\varepsilon(x)} F < \Mass(\sigma) + \Mass(N \setminus B_\varepsilon(x)) = \Mass(N),$$
so $N$ (hence $\lambda$) is not $F$-calibrated.
\end{proof}

%%%%%%%%%%%%%%%%%%%%%%%%%%%%%
\section{\texorpdfstring{Tight forms and the $1$-Laplacian}{Infinity-tight forms and the one-Laplacian}}\label{tight forms sec}
In this section we study the dual problem of the $1$-Laplacian.
Since the Banach spaces $L^q$, $1 < q \leq 2$, are reflexive, but $L^1$ is not, we first study the dual problem to the $q$-Laplacian, and then take the limit as $q \to 1$.

%%%%%%%%%%%%%%%

\subsection{The convex dual of the \texorpdfstring{$q$-Laplacian}{q-Laplacian}}
Let $M$ be a closed Riemannian manifold with fundamental group $\Gamma$ and universal cover $\tilde M \to M$.
Choose a fundamental domain $M_{\rm fun} \subseteq \tilde M$.
Given a representation $\alpha: \Gamma \to \RR$, choose a smooth $1$-form, which we also call $\alpha$, to represent the cohomology class corresponding to $\alpha$ in $H^1(M, \RR)$ given by (\ref{Poincare Hurcewiz}).

Given a H\"older pair $(p, q)$ with $1 < p < \infty$ (thus $\frac{1}{p} + \frac{1}{q} = 1$),
we are interested in the $q$-Laplace equation
$$\dif^* (|\dif u|^{q - 2} \dif u) = 0$$
for an $\alpha$-equivariant function $u$.
Let us express this problem variationally.

Let $X$ be the space of $W^{1, q}_\loc(\tilde M)$ functions $u$ which are $\Gamma$-equivariant in the sense that for $\gamma \in \Gamma$, $\gamma^* \dif u = \dif u$, and let $Y := L^q(M, \Omega^1)$.
We identify $\hat Y$ with $L^p(M, \Omega^{d - 1})$ using the perfect pairing 
\begin{align*}
	L^p(M, \Omega^{d - 1}) \times Y &\to \RR \\
	(F, \varphi) &\mapsto \int_M \varphi \wedge F.
\end{align*}
Then $\Lambda := (\dif: X \to Y)$ is a bounded linear map, and the $\alpha$-equivariant $q$-Laplacian is the Euler-Lagrange equation of $I \circ \Lambda$, where $I$ is the strictly convex functional such that
$$I(\varphi) := \frac{1}{q} \int_M \star |\varphi|^q$$
if the cohomology class of $\varphi$ is $\alpha$, and $I(\varphi) := +\infty$ otherwise.

\begin{proposition}[convex duality for the $q$-Laplacian]\label{mfmc qLaplacian}
Given a representation $\alpha: \Gamma \to \RR$, and a H\"older pair $(p, q)$ with $1 < p < \infty$, there exists an $\alpha$-equivariant $q$-harmonic function $u: \tilde M \to \RR$, unique modulo constants, and a unique minimizer $F$ of 
$$J_{p, \alpha}(F) := \frac{1}{p} \int_M \star |F|^p - \int_M \alpha \wedge F$$
among all closed $d - 1$-forms on $M$.
Moreover, we have
\begin{equation}\label{strong duality}
	\frac{1}{q} \int_M \star |\dif u|^q + \frac{1}{p} \int_M \star |F|^p + \int_M \dif u \wedge F = 0.
\end{equation}
\end{proposition}
\begin{proof}
Let
$$I_\alpha(\psi) := \frac{1}{q} \int_M \star |\psi + \alpha|^q,$$
defined for exact $L^q$ $1$-forms $\psi$, thus $\widehat{I_\alpha}$ is defined on the space of $L^p$ $d - 1$-forms modulo the kernel of the map
$$F \mapsto \left(\psi \mapsto \int_M \psi \wedge F\right)$$
and we lift it to $L^p(M, \Omega^{d - 1})$.

Let $v$ be a primitive of $\alpha$.
Then an $\alpha$-equivariant $u$ minimizes $I$ iff $u - v$ minimizes $I_\alpha \circ \Lambda$, which happens iff $u$ is $\alpha$-equivariant $q$-harmonic.
Since $I_\alpha(\psi) = I_0(\psi + \alpha)$, we can apply \cite[Chapter I, Remark 4.1]{Ekeland99} to see that $I_\alpha$ and $\widehat{I_\alpha}$ are strictly convex and
$$\widehat{I_\alpha}(F) = \widehat{I_0}(F) - \int_M \alpha \wedge F = \frac{1}{p} \int_M \star |F|^p - \int_M \alpha \wedge F.$$
Since $\coker \Lambda$ is the space of closed $L^p$ $d - 1$-forms on $M$, for $F \in \coker \Lambda$, $\widehat{I_\alpha}(F)$ does not depend on the choice of representatives $\alpha, v$, or of the lift of $\widehat{I_\alpha}$ to $L^p(M, \Omega^{d - 1})$.

Finally observe that $\ker \Lambda$ is the space of constants, and for any $\alpha$-equivariant $u$ and closed $F$,
\begin{align*}
I(\Lambda u) + \widehat{I_\alpha}(-F)
&= \frac{1}{q} \int_M \star |\dif u|^q + \frac{1}{p} \int_M \star |F|^p + \int_M \alpha \wedge F \\
&= \frac{1}{q} \int_M \star |\dif u|^q + \frac{1}{p} \int_M \star |F|^p + \int_M \dif u \wedge F.
\end{align*}
All of the assertions of this proposition now follow from Theorem \ref{abstract convex analysis}.
\end{proof}

Let $u$ be $\alpha$-equivariant $q$-harmonic.
Motivated by \cite[\S3.1]{daskalopoulos2020transverse}, it is natural to guess that 
\begin{equation}\label{dual solution}
F := - |\dif u|^{q - 2} \star \dif u
\end{equation}
is the solution of the dual problem of minimizing $J_{p, \alpha}$.
In order to prove that this is true, we shall need that if $(p, q)$ is a H\"older pair, then
\begin{equation}\label{holder cancellation}
	(p - 2)(q - 1) + (q - 2) = 0.
\end{equation}

\begin{lemma}\label{dual to u is minimizer}
Suppose that $u: \tilde M \to \RR$ is an $\alpha$-equivariant $q$-harmonic function, and suppose that $F$ satisfies (\ref{dual solution}).
Then $F$ is a closed $d - 1$-form, which minimizes $J_{p, \alpha}$ among all closed $d - 1$-forms.
Moreover, $F$ solves the PDE 
\begin{equation}\label{pMaxwell}
\begin{cases}
	\dif F = 0 \\
	\dif^* (|F|^{p - 2} F) = 0.
\end{cases}
\end{equation}
\end{lemma}
\begin{proof}
We first show that $\dif F = 0$.
In fact, 
$$\star \dif F = - \star \dif(|\dif u|^{q - 2} \star \dif u) = \pm \dif^*(|\dif u|^{q - 2} \dif u) = 0.$$
By uniqueness, if (\ref{strong duality}) holds, then $F$ must be the minimizer of $J_{p, \alpha}$.
One can easily compute 
$$|F|^p = |\dif u|^{(q - 1)p} = |\dif u|^q,$$
so by Stokes' theorem and the fact that $\alpha$ is cohomologous to $\dif u$,
\begin{align*}
\frac{1}{q} \int_M \star |\dif u|^q + \frac{1}{p} \int_M \star |F|^p&
= \left[\frac{1}{p} + \frac{1}{q}\right] \int_M \star |\dif u|^q
= \int_M \dif u \wedge |\dif u|^{q - 2} \star \dif u \\
&= -\int_M \alpha \wedge F,
\end{align*}
implying (\ref{strong duality}).
Finally, we use (\ref{holder cancellation}) to prove
\begin{align*}
\dif^*(|F|^{p - 2} F) &= - \dif^*(|\dif u|^{(p - 2)(q - 1)} |\dif u|^{q - 2} \star \dif u) = - \dif^*(\star \dif u) \\
&= \pm \star \dif^2 u = 0. \qedhere 
\end{align*}
\end{proof}

We next scrutinize the PDE (\ref{pMaxwell}).
At least at the heuristic level, one expects that as $p \to \infty$, the solutions of (\ref{pMaxwell}) converge to an absolute minimizer of a suitable $L^\infty$ variational problem; minimizers of such problems have been called \dfn{tight} by Sheffield and Smart \cite{Sheffield12}.
This motivates the below terminology:

\begin{definition}
Let $1 < p < \infty$.
A \dfn{$p$-tight form} is a solution of the PDE (\ref{pMaxwell}).
\end{definition}

\begin{proposition}
Suppose that $M$ is a closed oriented Riemannian manifold.
Then there is a unique $p$-tight form in each cohomology class in $H^{d - 1}(M, \RR)$.
Moreover, $p$-tight forms are minimizers of the strictly convex functional
$$J_p(F) := \frac{1}{p} \int_M \star |F|^p$$
among all forms cohomologous to them.
\end{proposition}
\begin{proof}
Strict convexity of $J_p$ on closed $L^p$ $d - 1$-forms is straightforward; since each cohomology class is an affine subspace of $L^p(M, \Omega^{d - 1})$, and hence is convex, the strict convexity on each class follows.
The existence and uniqueness of a minimizer is a standard consequence of strict convexity \cite[Chapter II]{Ekeland99}.
To compute the Euler-Lagrange equations for $J_p$, let $B$ be a $d-2$-form (so $F + t \dif B$ is cohomologous to $F$), so that for a minimizer $F$ of $J_p$,
$$\frac{\dif}{\dif t} J_p(F + t \dif B) = \frac{1}{p} \int_M \star \frac{\partial}{\partial t} |F + t \dif B|^p = \int_M \star |F + t \dif B|^{p - 2} \langle F + t \dif B, \dif B\rangle.$$
Setting $t = 0$, we obtain 
$$0 = \int_M \star |F|^{p - 2} \langle F, \dif B\rangle = \int_M \star \langle \dif^*(|F|^{p - 2} F), B\rangle.$$
Thus the Euler-Lagrange equations for $J_p$ are (\ref{pMaxwell}).
\end{proof}

\begin{definition}
Let $F$ be a $p$-tight form, let
\begin{equation}
\dif u := (-1)^d |F|^{p - 2} \star F, \label{inverse extremality}
\end{equation}
and let $u$ be the primitive of $\dif u$ on the universal cover $\tilde M$, which is normalized to have zero mean on a fundamental domain $M_{\rm fun}$.
Then $u$ is called the \dfn{$q$-harmonic conjugate} of the $p$-tight form $F$, where $\frac{1}{p} + \frac{1}{q} = 1$.
\end{definition}

Let $u$ be the $q$-harmonic conjugate of $F$.
By Poincar\'e's inequality,
$$\|u\|_{W^{1, q}(M_{\rm fun})}^q \lesssim \int_M \star |\dif u|^q = \int_M \star |F|^{(p - 1)q} = \int_M \star |F|^p < \infty$$
since $F$ is $p$-tight; that is, we have $F \in L^p$ and $u \in W^{1, q}_\loc$, justifying any manipulations we shall make with these forms.

\begin{lemma}
Let $1 < p, q < \infty$ and $\frac{1}{p} + \frac{1}{q} = 1$.
Let $F$ be a $p$-tight form, and let $u$ be its $q$-harmonic conjugate.
Then $u$ is $q$-harmonic, $F$ satisfies (\ref{dual solution}), and we have strong duality (\ref{strong duality}).
\end{lemma}
\begin{proof}
We first use (\ref{holder cancellation}) to prove
$$|\dif u|^{q - 2} \star \dif u = (-1)^d |F|^{(q - 2)(p - 1)} \star \star |F|^{p - 2} F = - |F|^{(q - 2)(p - 1) - (p - 2)} F = - F.$$
Thus we have (\ref{dual solution}), and moreover
$$\dif \star (|\dif u|^{q - 2} \dif u) = - \dif F = 0$$
so that $u$ is $q$-harmonic.
Then by Lemma \ref{dual to u is minimizer}, $F$ is the unique minimizer of $J_{p, [\dif u]}$, which implies (\ref{strong duality}).
\end{proof}

% %%%%%%%%%%%%%%%%%%%
% \subsection{Regularity of the \texorpdfstring{$q$-Laplacian}{q-Laplacian}}
% We now pause to consider two regularity results for the $q$-Laplacian.
% The first gives H\"older regularity but is not uniform in $q$; the second is uniform but only gives Sobolev regularity.
% \todo{Do we ever use this?}

% \begin{lemma}[{\cite[Theorem 2]{DIBENEDETTO1983827}}]\label{q Laplacian Holder regularity}
% Let $u: \tilde M \to \RR$ be an $\alpha$-equivariant $q$-harmonic function.
% Then $\dif u$ is H\"older continuous.
% \end{lemma}

% \begin{corollary}
% Every $p$-tight form is H\"older continuous.
% \end{corollary}
% \begin{proof}
% Let $F$ be $p$-tight and let $u$ be its $q$-harmonic conjugate, so $\dif u$ is H\"older continuous by Lemma \ref{q Laplacian Holder regularity}.
% The claim now follows from (\ref{dual solution}) and the fact that a product of H\"older continuous functions is H\"older continuous.
% \todo{Write this out carefully since $q < 2$}
% \end{proof}

\begin{lemma}
Suppose that $1 < q \leq 2$.
Let $u_q: \tilde M \to \RR$ be an $\alpha$-equivariant $q$-harmonic function.
Then (with constant independent of $q$)
\begin{equation}\label{q Laplacian Sobolev regularity estimate}
\|\dif u_q\|_{L^q} \sim \Mass(\alpha).
\end{equation}
\end{lemma}
\begin{proof}
Without loss of generality, $\int_{M_{\rm fun}} \star u_q = 0$.
Let $e_1, \dots, e_k$ be a basis for $H_{d - 1}(M, \RR)$.
This induces a norm $\|\cdot\|$ on $H_{d - 1}(M, \RR)$ by
$$\left\|\sum_i \beta_i e_i\right\| = \sum_i |\beta_i|,$$
which is comparable to the stable norm $\Mass$ since $H_{d - 1}(M, \RR)$ is finite-dimensional.
We write $\alpha = \sum_i \alpha_i e_i$.
Let $v_i$ be the $e_i$-equivariant harmonic function such that $\int_{M_{\rm fun}} \star v_i = 0$.
Then $u_2 = \sum_i \alpha_i v_i$, so
$$\|\dif u_2\|_{L^2} \leq \sum_i |\alpha_i| \|\dif v_i\|_{L^2} \lesssim \|\alpha\| \sim \Mass(\alpha).$$
Since $\dif u_q$ is a minimizer of the $L^q$ norm, we estimate using H\"older's inequality 
\begin{align*}
\|\dif u_q\|_{L^q} &\leq \|\dif u_2\|_{L^q} \leq |M|^{\frac{1}{q} - \frac{1}{2}} \|\dif u_2\|_{L^2} \lesssim \|\dif u_2\|_{L^2} \lesssim \Mass(\alpha).
\end{align*}
In the other direction, we estimate using H\"older's inequality
\begin{align*}
\Mass(\alpha) &\leq \|\dif u_q\|_{L^1} \leq \|\dif u_q\|_{L^q} |M|^{1/p} \lesssim \|\dif u_q\|_{L^q}. \qedhere 
\end{align*}
\end{proof}


%%%%%%%%%%%%%%%%%%%%%%%
\subsection{\texorpdfstring{Existence of tight forms}{Existence of tight forms}}
We now take the limit $p \to \infty$ to obtain a privileged form of best comass.
To do so, we shall need the $p$-tight forms to be uniformly bounded in the following sense.

\begin{lemma}
Let $F_p$ be a $p$-tight form, and let $B$ range over closed $d - 1$-forms cohomologous to $F_p$. Then
\begin{equation}\label{infinity magnetic rules p magnetic}
	\|F_p\|_{L^p} \leq |M|^{1/p} \inf_B \|B\|_{L^\infty}.
\end{equation}
\end{lemma}
\begin{proof}
By H\"older's inequality and the fact that $F_p$ is $p$-tight,
$$\|F_p\|_{L^p} \leq \|B\|_{L^p} \leq |M|^{1/p} \|B\|_{L^\infty},$$
hence the same holds for the infimum.
\end{proof}

\begin{proposition}\label{existence infinity}
Let $\rho \in H^{d - 1}(M, \RR)$.
For each $p \geq 2$, let $F_p$ be the $p$-tight form representing $\rho$. Then there exists a closed $d - 1$-form $F$ such that:
\begin{enumerate}
\item $F_p \to F$ weakly in $L^r$ along a subsequence, for any $d < r < \infty$.
\item $F$ is a best comass representative of $\rho$.
\end{enumerate}
\end{proposition}
\begin{proof}
We roughly follow \cite[\S3]{Lindqvist14}.
Let $r > d$, and let $B$ be an $L^\infty$ representative of $\rho$.
By H\"older's inequality and (\ref{infinity magnetic rules p magnetic}),
\begin{equation}\label{uniform bounds in p by best curl}
	\|F_p\|_{L^r} \leq |M|^{\frac{1}{r} - \frac{1}{p}} \|F_p\|_{L^p} \leq |M|^{\frac{1}{r}} \|B\|_{L^\infty}.
\end{equation}
Thus a compactness argument gives $F_p \to F$ for some $d - 1$-form $F$, weakly in $L^r$, and 
$$\|F\|_{L^r} \leq \liminf_{p \to \infty} \|F_p\|_{L^r} \leq |M|^{\frac{1}{r}} \|B\|_{L^\infty}.$$
Diagonalizing, we may assume that $F_p \to F$ weakly in $L^r$ for every such $r$, and taking $r \to \infty$, we conclude 
\begin{equation}\label{infinity magnetics have best curl}
	\|F\|_{L^\infty} \leq \|B\|_{L^\infty}.
\end{equation}
Moreover, $[F] = \lim_{p \to \infty} [F_p] = \rho$.
Since $B$ was arbitrary in (\ref{infinity magnetics have best curl}), $F$ has best comass.
\end{proof}

\begin{definition}
The $d - 1$-form $F$ of best comass in Proposition \ref{existence infinity} is called a \dfn{tight form}.
\end{definition}

Our next corollary follows immediately from Proposition \ref{existence infinity}.
It can also be proven using Alaoglu's theorem, but we record it because it is extremely useful.

\begin{corollary}
Every cohomology class has a best comass representative.
\end{corollary}

The existence of best comass representatives of each cohomology class $\rho$ implies the following useful lemma on the costable norm of $\rho$.

\begin{lemma}\label{p tights approximate L}
Let $F_p$ be the $p$-tight representative of $\rho$. Then 
$$\lim_{p \to \infty} \|F_p\|_{L^p} = \Comass(\rho).$$
\end{lemma}
\begin{proof}
We follow \cite[Lemma 2.7]{daskalopoulos2020transverse}.
Let $F$ be a best comass representative of $\rho$, so $\|F\|_{L^\infty} = \Comass(\rho)$.
Since $F_p$ is $p$-tight, H\"older's inequality implies 
$$\|F_p\|_{L^p} \leq \|F\|_{L^p} \leq |M|^{\frac{1}{p}} \Comass(\rho).$$
Therefore 
$$\limsup_{p \to \infty} \|F_p\|_{L^p} \leq \Comass(\rho).$$
To prove the converse, suppose that for some $\varepsilon > 0$,
$$\liminf_{p \to \infty} \|F_p\|_{L^p} \leq \Comass(\rho) - \varepsilon < \Comass(\rho).$$
Along a subsequence which attains the limit inferior, $F_p$ converges weakly in every $L^r$, $d < r < \infty$, to a tight form $\tilde F$ such that (by H\"older's inequality)
$$\|\tilde F\|_{L^r} \leq \liminf_{p \to \infty} \|F_p\|_{L^r} \leq \liminf_{p \to \infty} |M|^{\frac{1}{r}} \|\tilde F\|_{L^\infty} \leq |M|^{\frac{1}{r}} (\Comass(\rho) - \varepsilon).$$
Taking $r \to \infty$, we obtain $\Comass(\tilde F) < \Comass(\rho)$, which contradicts the definition of the costable norm $\Comass(\rho)$.
\end{proof}


%%%%%%%%%%%%%%%%%%%%
\subsection{\texorpdfstring{$1$-harmonic conjugates of tight forms}{One-harmonic conjugates of tight forms}}
We now construct the $1$-harmonic conjugates of a tight form.
In the special case that the tight form $F$ is a calibration, that is $\Comass(F) = 1$, a $1$-harmonic conjugate will be a $1$-harmonic function on the universal cover whose level sets are calibrated by $F$.

\begin{definition}
Let $F$ be a tight form of cohomology class $\rho$.
A nonconstant $\Gamma$-equivariant function of least gradient $u \in BV_\loc(\tilde M)$ is called a \dfn{$1$-harmonic conjugate} of $F$ if
\begin{equation}\label{1 extremality}
\dif u \wedge F = \Comass(\rho) \star |\dif u|.
\end{equation}
\end{definition}

We begin by showing that $L^1$ convergence preserves the equivariance properties of functions.

\begin{lemma}\label{L1 convergence preserves pi1}
Let $\tilde M \to M$ be the universal cover, and let $(u_q)$ be a sequence of $\Gamma$-equivariant functions on $\tilde M$ which converge in $L^1_\loc(\tilde M)$ to a function $u$ as $q \to 1$.
Then $u$ is $\Gamma$-equivariant, and $[u_q] \to [u]$.
Moreover, if $\dif u_q \to \dif u$ in the weak topology of measures on $M$ and $\dif u_q \in L^q$, then
\begin{equation}\label{q to 1 Holder}
\Mass(\dif u) \leq \liminf_{q \to 1} \frac{1}{q} \int_M \star |\dif u_q|^q.
\end{equation}
\end{lemma}
\begin{proof}
Since $u_q$ is $\Gamma$-equivariant, there exists $\alpha_q \in H^1(M, \RR)$ such that for every $\gamma \in \pi_1(M)$,
\begin{equation}\label{equivariance q}
	\gamma^* u_q = u_q + \langle \alpha_q, \gamma\rangle.
\end{equation}
Let $M_{\rm fun}$ be a fundamental domain and $U_\gamma := M_{\rm fun} \cup \gamma_* (M_{\rm fun})$.

We claim that $(\alpha_q)$ has a convergent subsequence.
To see this, we first recall that $M$ has finite Betti numbers, so $H^1(M, \RR)$ is locally compact.
Therefore, if no convergent subsequence exists, there exists a $\gamma \in \pi_1(M)$ and a subsequence along which $\langle \alpha_q, \gamma\rangle \to \infty$.
Moreover, since $u_q \to u$ in $L^1_\loc$, $\|u_q\|_{L^1(M_{\rm fun})} \leq 2\|u\|_{L^1(M_{\rm fun})}$ if $q - 1$ is small enough.
But then 
$$\|u_q\|_{L^1(\gamma_* M_{\rm fun})} = \|\gamma^* u_q\|_{L^1(M_{\rm fun})} \geq \langle \alpha_q, \gamma\rangle - \|u_q\|_{L^1(M_{\rm fun})} \geq \langle \alpha_q, \gamma\rangle - 2\|u\|_{L^1(M_{\rm fun})}$$
and taking $q \to 1$ we conclude that $(u_q)$ is not compact in $L^1(\gamma_* M_{\rm fun})$, contradicting the convergence in $L^1_\loc(\tilde M)$.
So $\alpha_q \to \alpha$ for some $\alpha \in H^1(M, \RR)$ along a subsequence.

For any $q > 1$,
\begin{align*}
\dashint_{M_{\rm fun}} \star |\gamma^* u - u - \langle \alpha, \gamma\rangle| 
&\leq \dashint_{M_{\rm fun}} \star (|\gamma^* u_q - u_q - \langle \alpha_q, \gamma\rangle| + |\gamma^* u_q - u_q| + |\gamma^* u - u|) \\
&\qquad + |\langle \alpha_q - \alpha, \gamma\rangle|.
\end{align*}
Taking $q \to 1$ and applying (\ref{equivariance q}), we conclude that $\|\gamma^* u - u - \langle \alpha, \gamma\rangle\|_{L^1} = 0$, hence $u$ is $\alpha$-equivariant.
Thus $\alpha$ is uniquely defined and $\alpha_q \to \alpha$ along the entire subsequence.

Finally we prove (\ref{q to 1 Holder}).
Suppose that $\dif u_q \to \dif u$ in the weak topology of measures and $\dif u_q$ in $L^q$.
Then
$$\|\dif u_q\|_{L^1} = \Mass(\dif u_q).$$
So we may use the portmanteau theorem and H\"older's inequality to estimate (where $\frac{1}{p} + \frac{1}{q} = 1$)
\begin{align*}
\Mass(\dif u) &= \lim_{q \to 1} \Mass(\dif u_q) \leq \lim_{q \to 1} |M|^{\frac{1}{p}} \|\dif u_q\|_{L^q} = \lim_{q \to 1} \frac{1}{q} \int_M \star |\dif u_q|^q. \qedhere
\end{align*}
\end{proof}

The duality relation (\ref{inverse extremality}) blows up $p \to \infty$.
We now ``renormalize'' away the divergence of the $q$-harmonic conjugates of $p$-tight forms before taking the limit $q \to 1$, as in \cite[\S3.2]{daskalopoulos2020transverse}.
Suppose that $\rho \in H^{d - 1}(M, \RR)$, and let $k_p$ be defined by 
$$k_p^{1 - p} = \int_M \star |F_p|^p$$
where $F_p$ is the $p$-tight representative of $\rho$.

\begin{lemma}\label{normalizations converge}
As $p \to \infty$, $k_p \to \Comass(\rho)^{-1}$.
\end{lemma}
\begin{proof}
We follow \cite[Lemma 3.4]{daskalopoulos2020transverse}.
By Lemma \ref{p tights approximate L},
$$\lim_{p \to \infty} k_p^{-\frac{1}{q}} = \lim_{p \to \infty} \|F_p\|_{L^p} = \Comass(\rho).$$
Taking logarithms we see that $q^{-1} \log k_p \to -\log \Comass(\rho)$, and since $q \to 1$ the claim follows.
\end{proof}

\begin{proposition}\label{existence 1}
Let $\rho \in H^{d - 1}(M, \RR)$ be nonzero, and let $F$ be its tight representative.
For each H\"older pair $(p, q)$ with $d < p < \infty$, let $F_p$ be the $p$-tight representative of $\rho$, and let $u_q$ be the function on $\tilde M$ with mean zero on $M_{\rm fun}$ and
$$\dif u_q = (-1)^{d - 1} k_p^{p - 1} |F_p|^{p - 2} \star F_p.$$
Then there exists a $1$-harmonic conjugate $u$ of $F$ such that as $q \to 1$ along a subsequence, $u_q \to u$ weakly in $BV_\loc(\tilde M)$ and strongly in $L^r_\loc(\tilde M)$ for $1 \leq r < \frac{d}{d - 1}$.
\end{proposition}
\begin{proof}
Let $L := \Comass(\rho)$.
We first compute using H\"older's inequality and Lemma \ref{normalizations converge}
\begin{align}
\lim_{q \to 1} \|\dif u_q\|_{L^1}
&\leq \lim_{q \to 1} |M|^{\frac{1}{p}} \left[\int_M \star |\dif u_q|^q\right]^{\frac{1}{q}} = \lim_{p \to \infty} \left[k_p^p \int_M \star |F_p|^p\right]^{\frac{1}{q}} \label{Rellich 1}\\
&= \lim_{p \to \infty} k_p^{\frac{1}{q}} = \lim_{p \to \infty} k_p = \frac{1}{L} \label{Rellich 2}.
\end{align}
So by Rellich's theorem, $(u_q)$ is weakly compact in $BV$ and strongly compact in $L^r$ for $1 \leq r < \frac{d}{d - 1}$.
In particular, $\dif u_q \to \dif u$ in the weak topology of measures and $u_q \to u$ weakly in $BV$ and strongly in $L^r$.
As the limit of $\Gamma$-equivariant functions, $u$ is also $\Gamma$-equivariant by Lemma \ref{L1 convergence preserves pi1}.
In particular, $\dif u$ drops to a current on $M$.
Moreover, $[\dif u_q] \to [\dif u]$, and we have the bound (\ref{q to 1 Holder}) on $\int \star |\dif u|$.

We next must check that $u$ is nonconstant.
If $u$ is constant, then it is $\Gamma$-invariant, so $[\dif u_q] \to 0$.
By (\ref{q Laplacian Sobolev regularity estimate}), $\|\dif u_q\|_{L^q} \to 0$, so by (\ref{Rellich 1}, \ref{Rellich 2}), $L = \infty$, which is absurd.
Therefore $u$ is nonconstant.

Renormalizing (\ref{strong duality}), we obtain 
$$\frac{k_p^{-p}}{q} \int_M \star |\dif u_q|^q + \frac{1}{p} \int_M \star |F_p|^p = k_p^{1 - p} \int_M \dif u_q \wedge F_p.$$
Multiplying by $k_p^p$, we have 
\begin{equation}\label{1 strong duality before limits}
	\frac{1}{q} \int_M \star |\dif u_q|^q + \frac{k_p^p}{p} \int_M \star |F_p|^p = k_p \int_M \dif u_q \wedge F_p.
\end{equation}

Let $\mu(U) := \Mass_U(\dif u)$ be the total variation measure of $\dif u$.
We claim that
\begin{equation}\label{1 strong duality}
	L\mu(M) \leq \int_M \dif u \wedge F.
\end{equation}
First, we have from (\ref{q to 1 Holder}) and (\ref{1 strong duality before limits}) that
$$\mu(M) \leq \lim_{q \to 1} \frac{1}{q} \int_M \star |\dif u_q|^q = \lim_{p \to \infty} k_p \int_M \dif u_q \wedge F_p - \lim_{p \to \infty} \frac{k_p^p}{p} \int_M \star |F_p|^p.$$
By Lemma \ref{normalizations converge},
$$\lim_{p \to \infty} \frac{k_p^p}{p} \int_M \star |F_p|^p = \lim_{p \to \infty} \frac{k_p}{p} = \frac{0}{L} = 0,$$
and
$$\lim_{p \to \infty} k_p \int_M \dif u_q \wedge F_p = \frac{1}{L} \lim_{p \to \infty} \int_M [\dif u_q] \wedge \rho.$$
Since $[\dif u_q] \to [\dif u]$, we obtain
$$\lim_{p \to \infty} \int_M [\dif u_q] \wedge \rho = \int_M \alpha \wedge \rho = \int_M \dif u \wedge F,$$
completing the proof of (\ref{1 strong duality}).

By the coarea formula (\ref{coarea formula}), we have for any open set $U$,
$$\int_U \dif u \wedge F = \int_{-\infty}^\infty \int_{U \cap \partial \{u > y\}} F \dif y \leq L \int_{-\infty}^\infty |U \cap \partial \{u > y\}| \dif y = L \mu(U).$$
Since $\mu$ is a Radon measure and $M$ is compact, every Borel set $E$ can be $\mu$-approximated from without by open sets, hence
\begin{equation}\label{one sided extremality}
\int_E \dif u \wedge F \leq L \mu(E).
\end{equation}

Next we deduce (\ref{1 extremality}).
We reason by contradiction: if (\ref{1 extremality}) is false, then there exists an open set $U \subseteq M$ such that 
$$\int_U \dif u \wedge F < L \int_U \star |\dif u|.$$
(Indeed, strict inequality cannot point in the other direction, by (\ref{one sided extremality}).)
However, by (\ref{one sided extremality}), 
$$\int_{M \setminus U} \dif u \wedge F \leq L \int_{M \setminus U} \star |\dif u|.$$
Adding up the integrals of $\dif u \wedge F$ over $U$ and $M \setminus U$, we conclude 
$$\int_M \dif u \wedge F < L \int_M \star |\dif u|,$$
but this contradicts (\ref{1 strong duality}); thus (\ref{1 extremality}) must be true.
In particular, $F/L$ satisfies (\ref{local calibration}), so $u$ has least gradient by Theorem \ref{MazonRossi}.
\end{proof}




%%%%%%%%%%%%%%%%%%%%


\section{Measured stretch laminations}\label{MCL sec}
\subsection{Existence of measured stretch laminations}
Let $M$ be a closed Riemannian of dimension $d \leq 7$ equipped with a cohomology class $\rho \in H^{d - 1}(M, \RR)$.
To ease notation, we normalize the costable norm:
$$\Comass(\rho) = 1.$$
Here we show that the maximum comass locus $\MCL(F)$ of a best comass representative of $\rho$ contains the measured laminations given by the $1$-harmonic conjugates of the tight representatives of $\rho$.

By Theorem \ref{1 harmonic is MOML}, every function $u$ of least gradient gives rise to a homologically minimizing lamination $\kappa_u$.
Thus the following definition makes sense:

\begin{definition}
Let $F$ be a tight representative of $\rho$, and let $u$ be a $1$-harmonic conjugate of $F$.
Then we call $\kappa_u$ a \dfn{measured stretch lamination} associated to $\rho$.
\end{definition}

\begin{proposition}\label{MCL contains Thurston}
Let $F$ be a best comass representative of $\rho$, and let $\lambda$ be a measured stretch lamination associated to $\rho$.
Then $F$ calibrates $\lambda$. In particular, $\MCL(F) \supseteq \supp \lambda$.
\end{proposition}
\begin{proof}
Let $G$ be the tight form which is cohomologous to $F$ whose dual $1$-harmonic function $u$ defines the measured stretch lamination $\lambda$.
Then by (\ref{1 extremality}), 
$$\Mass(\lambda) = \Mass(\dif u) = \int_M \dif u \wedge G$$
so $G$ calibrates $\lambda$ by Proposition \ref{calibration condition}.
Then by Proposition \ref{properties of calibrated laminations}, $F$ calibrates $\lambda$ and $\MCL(F) \supseteq \supp \lambda$.
\end{proof}

\begin{proposition}\label{L equals K}
	Let $\kappa$ be a measured stretch lamination for $\rho$, and let $\lambda$ range over measured oriented laminations. Then 
	\begin{equation}\label{L equals K formula}
	\sup_\lambda \frac{\langle \rho, [\lambda]\rangle}{\Mass(\lambda)} = \frac{\langle \rho, [\kappa]\rangle}{\Mass(\kappa)} = 1.
	\end{equation}
\end{proposition}
\begin{proof}
Fix a tight form $F$ representing $\rho$, and let $u$ be its $1$-harmonic conjugate.
Let
$$K :=  \sup_\lambda \frac{\langle \rho, [\lambda]\rangle}{|\lambda|}.$$

We first prove $K \leq 1$.
Let $\lambda$ be a measured oriented lamination; then, since $F$ represents $\rho$ and the Ruelle-Sullivan current $T_\lambda$ represents $[\lambda]$,
$$\langle \rho, [\lambda]\rangle = \int_M F \wedge T_\lambda.$$
Let $(\chi_\alpha)$ be a partition of unity subordinate to a laminar atlas for $\lambda$, and let $(\mu_\alpha)$ be the associated transverse measure. Then 
$$\int_M F \wedge T_\lambda = \sum_\alpha \int_I \int_{\{k\} \times J} \chi_\alpha F \dif \mu_\alpha(k).$$
Since $F$ has best comass,
$$\frac{\langle \rho, [\lambda] \rangle}{\Mass(\lambda)}
\leq \frac{\|F\|_{L^\infty}}{\Mass(\lambda)} \sum_\alpha \int_I \int_{\{k\} \times J} \chi_\alpha \dif S_k \dif \mu_\alpha(k) = 1.$$
Since $\lambda$ was arbitrary, it holds that $K \leq 1$.

By (\ref{1 extremality}),
$$\langle \rho, [\kappa]\rangle = \int_M F \wedge \dif u = \Mass(\dif u) = \Mass(\kappa).$$
Dividing both sides by $\Mass(\kappa)$ and applying the direction we already proved,
$$K \leq 1 \leq \frac{\langle \rho, [\kappa]\rangle}{\Mass(\kappa)} \leq K$$
which is only possible if $K = 1$ and $\kappa$ is a maximizer.
\end{proof}

\begin{proposition}\label{calibrated means measured stretch}
Let $F$ be a best comass representative of $\rho$, and suppose that $F$ calibrates a measured oriented lamination $\lambda$.
Then $\lambda$ is a measured stretch lamination associated to $\rho$.
\end{proposition}
\begin{proof}
Let $\dif u$ be the Ruelle-Sullivan current for $\lambda$, and suppose that $f \in C^0(M)$ is supported in a flow box for $\lambda$, with local leaf space $K$ and transverse measure $\mu$.
By Proposition \ref{properties of calibrated laminations}, we may assume wihout loss of generality that $F$ is tight.
Since $F$ calibrates every leaf of $\lambda$,
$$\int_M f \star |\dif u| = \int_K \int_{\{k\} \times J} f \dif S_{\{k\} \times J} \dif \mu(k) = \int_K \int_{\{k\} \times J} fF \dif \mu(k) = \int_M f\dif u \wedge F$$
(in any case, the Radon measure $\dif u \wedge F$ is well-defined by the coarea formula).
Thus $\dif u \wedge F = \star |\dif u|$, or in other words $u$ is a $1$-harmonic conjugate of the tight form $F$.
Therefore $\lambda$ is a measured stretch lamination.
\end{proof}

\begin{corollary}
Let $F$ be a best comass representative of $\rho$, and suppose that $F$ calibrates a measured oriented lamination $\lambda$.
Then $\lambda$ is Lipschitz.
\end{corollary}
\begin{proof}
By Theorem \ref{1 harmonic is MOML}, every measured stretch lamination is Lipschitz.
\end{proof}

%%%%%%%%%%%%%%%%%%%
\subsection{Laminations of rational class}
We next give a condition for a measured stretch lamination to have only closed leaves.
Note that this condition appears in unpublished work of Auer and Bangert \cite{Auer12}, so we do not take credit for it.
We simply include this result because it is interesting enough that we would like it to be publicly available.

\begin{definition}
A homology class $\alpha \in H_{d - 1}(M, \RR)$ has \dfn{rational direction} if there exists $c > 0$ such that $c\alpha \in H_{d - 1}(M, \ZZ)$.
\end{definition}

\begin{proposition}
Let $M$ be a closed manifold, let $\alpha \in H_{d - 1}(M, \RR)$ have rational direction, $\alpha \neq 0$, and let $u$ be an $\alpha$-equivariant function of least gradient on $\tilde M$.
Then every leaf of $\kappa_u$ is a closed hypersurface.
\end{proposition}
\begin{proof}
Rescaling $u$ by a constant does not affect whether the leaves of $\kappa_u$ are closed, so after rescaling, we may assume that $\alpha \in H_{d - 1}(M, \ZZ)$.
We view $u$ as a map $M \to \Sph^1$ of homotopy class $\alpha$.\footnote{Note that $u$ is not continuous, so strictly speaking the homotopy class of $u$ is not $\alpha$; but by equivariance, it is profitable to view $u$ in this way.}
Since the representation $\alpha$ is integral, $\ker \alpha$ fits into a short exact sequence 
$$0 \to \ker \alpha \to \pi_1(M) \to \ZZ \to 0.$$
So from the Galois correspondence for covering spaces there exists a covering space $p: \hat M \to M$ such that $\Gal(\hat M, M) \cong \ZZ$, and a function $\hat u$ on $\hat M$, such that the diagram 
$$\begin{tikzcd}
\hat M \arrow[r, "\hat u"] \arrow[d, "p"] & \RR \arrow[d] \\
M \arrow[r, "u"] & \Sph^1
\end{tikzcd}$$
commutes.
In particular, it sense to take superlevel sets $\{\hat u > y\}$.

Since $\Gal(\hat M, M)$ is a cyclic group, it is generated by a single element $h$ with the mapping property
$$h: \partial \{\hat u > y\} \to \partial \{\hat u > y + D\}$$
for each $y \in \RR$.
In particular, $\hat u$ does not have a global minimum or maximum.
Since $\hat u$ does not have a global minimum or maximum on the complete manifold $\hat M$, by the maximum principle for functions of least gradient \cite[Theorem 5.1]{HakkarainenKorteLahtiShanmugalingam+2015}, $\hat u$ does not have any local minimum or maximum.
Therefore every leaf of $\kappa_{\hat u}$ is a component of $\partial \{\hat u > y\}$ for some $y \in \RR$ \cite[Proposition 4.7]{BackusCML}.

Now let $N$ be a (connected) leaf of $\lambda$, and let $\hat N := p^{-1}(N)$.
Then each component of $\hat N$ is a leaf of $\kappa_{\hat u}$, and hence a component of $\partial \{u > y\}$ for some $y \in \RR$.
Moreover, the deck transformation $h$ has the mapping property
$$h: \hat N \to \hat N.$$
So if $\hat N$ meets $\partial \{u > y\}$, then for every $n \in \ZZ$, a component $K_n$ of $\hat N$ is a component of $\partial \{\hat u > y + nD\}$.
Since $h$ acts on $\kappa_{\hat u}$ by sending $\partial \{\hat u > y\}$ to $\partial \{\hat u > y + D\}$, $h$ acts on $\hat N$ by sending $K_n$ to $K_{n + D}$.
In particular, $\hat N \cong N \times \ZZ$ and $p: \hat N \to N$ is projection onto the first factor. 
So $N \cong K_0$, and $K_0$ is a closed hypersurface in $\hat M$ since it is a complete manifold \cite[Theorem 3.3]{BackusCML} and it is compact.
\end{proof}

\begin{example}
Suppose that $M = \Sph^1_x \times \Sph^1_y$.
Let $u(x, y) = x$, so $\kappa_u$ is itself topologically nontrivial (in the $x$ direction) and its leaves are also topologically nontrivial (in the $y$ direction).
The covering space $\hat M$ is $\RR_x \times \Sph_y^1$, the point is that we unwound $\kappa_u$ without messing with its leaves.
\end{example}

\todo{Can we show that the Federer norm is basically the Thurston norm?}

%%%%%%%%%%%%%%%%
\section{The canonical lamination}
\label{canonical sec}
Throughout this section, we fix $\rho \in H^{d - 1}(M, \RR)$ in the costable unit sphere $\{\Comass(\rho) = 1\}$.
Motivated by Thurston's approach to Teichm\"uller theory (see \S\ref{Teichmuller}), we construct a lamination which is calibrated by every best comass form in $\rho$, and which only depends on $\rho$: the \dfn{canonical lamination} $\lambda_\rho$.

\subsection{Singularities of nodal sets}\label{nodal appendix}
For a solution $v$ of an elliptic PDE, we write $Z(v), Z^{\rm sing}(v)$ for the nodal and singular sets of $v$, namely the sets of zeroes and double zeroes, respectively.
We will show that the generic point of $Z(v)$ is not a singular point. More precisely:

\begin{proposition}\label{nodal set is generically smooth}
Let $Q$ be a linear elliptic operator on $\Ball^{d - 1}$ satisfying the maximum principle.
Suppose that $Qv = 0$ and $v$ has a zero of finite order.
Then the Hausdorff dimensions of the nodal and singular sets of $v$ are
\begin{align}
	\dim(Z(v)) &= d - 2, \label{nodal dimension}\\
	\dim(Z^{\rm sing}(v)) &\leq d - 3. \label{singular nodal dimension}
\end{align}
\end{proposition}

The main idea of the proof is to show that the complement of $Z^{\rm sing}(v)$ is path-connected.
By Alexander duality for singular cohomology, the complement of a submanifold $P$ of codimension $\geq 2$ is path-connected.
The singular set $P = Z^{\rm sing}(v)$ is not in general a manifold, but the proof still works as long as we apply Alexander duality for sheaf cohomology.
Let $\hat H^\bullet(P, \RR)$ denote the cohomology of the constant sheaf $\RR$ on $P$.

\begin{lemma}\label{closed mfld complement}
Let $P \subset \Sph^{d - 1}$ be a closed $d - 3$-rectifiable set.
Then $\Sph^{d - 1} \setminus P$ is path-connected.
\end{lemma}
\begin{proof}
Let $\delta^{\rm Haus}, \delta^{\rm cov}, \delta^{\rm shf}$ be the Hausdorff, covering, and sheaf cohomological dimensions of $P$ respectively.
Then by \cite[{\S}II.5.12]{godement1973topologie} and \cite[Theorem 6.3.10]{edgar2008measure}, we have 
$$\delta^{\rm shf} \leq \delta^{\rm cov} \leq \delta^{\rm Haus} \leq d - 3,$$
hence $\hat H^{d - 2}(P, \RR) = 0$.
By Alexander duality for sheaf cohomology \cite[Theorem 6]{Kaplan47}, it follows that $H_0(\Sph^{d - 1} \setminus P, \RR) \cong \ZZ$, or in other words $\Sph^{d - 1} \setminus P$ is path-connected.
\end{proof}

\begin{lemma}\label{open mfld complement}
Let $P \subset \Ball^{d - 1}$ be a closed $d - 3$-rectifiable set.
Then $\Ball^{d - 1} \setminus P$ is path-connected.
\end{lemma}
\begin{proof}
Embed $\Ball^{d - 1}$ in $\Sph^{d - 1}$ using the one-point compactification, let $\infty$ be the point at infinity, and let $x, y \in \Ball^{d - 1} \setminus P$.
Choose a $d - 3$-sphere $S$ in $\Sph^{d - 1}$ which contains $\infty$ but does not contain $x, y$.
Then $P \cup S$ is a closed $d - 3$-rectifiable set and $x, y \notin P \cup S$, so by Lemma \ref{closed mfld complement}, there exists a curve $\gamma$ from $x$ to $y$ which avoids $P \cup S$.
Therefore $\gamma \subset \Ball^{d - 1} \setminus P$.
\end{proof}

\begin{proof}[Proof of Proposition \ref{nodal set is generically smooth}]
By \cite[Lemma 1.9]{Hardt89}, $Z^{\rm sing}(v)$ is $d - 3$-rectifiable, which implies (\ref{singular nodal dimension}).
If there exists $x \in Z(v) \setminus Z^{\rm sing}(v)$, then by the implicit function theorem, there is a neighborhood $U \ni x$ such that $U \cap Z(v)$ is a $d - 2$-dimensional manifold.
So if (\ref{nodal dimension}) fails, we must have $Z(v) = Z^{\rm sing}(v)$, so $Z(v)$ is $d - 3$-rectifiable.
But then, by Lemma \ref{open mfld complement}, the sets $U_\pm := \{\pm v > 0\}$ satisfy $U_+ \cup U_-$ are connected.
Since $v$ is continuous, one of these sets must be empty; without loss of generality, $U_- = \emptyset$.
Then $v \geq 0$ and $v$ has a zero, so by the maximum principle, $v = 0$ identically.
This contradicts the fact that $v$ has a zero of finite order.
\end{proof}


%%%%%%%%%%%%%%%%%%%%%
\subsection{Construction of the canonical lamination}
We note carefully that if $F$ is a best comass representative of $\rho$, then $\MCL(F)$ need not itself be a lamination \cite[Example 5.4]{bangert_cui_2017}.
In particular, unlike in Thurston's theory, we cannot simply take the intersection of all the maximum comass loci of best comass representatives of $\rho$.
On the other hand, one can use the existence of measured stretch laminations to show that $\MCL(F)$ contains a lamination.
So our strategy is to construct the largest lamination $\lambda_F$ which $F$ calibrates, and take an intersection over all the $\lambda_F$s.

We first rule out intersections of the leaves.
This can be done by showing that the generic intersection point of two minimal hypersurfaces is transverse.
If the dimension of the underlying manifold $M$ is $d = 2$, then this is trivial, and if $d = 3$, then the structure of $N \cap N'$ is completely described by complex-analytic means \cite[Theorem 7.3]{colding2011course}, so the proof we present here is mainly of interest if $d \geq 4$.

\begin{proposition}\label{intersection theory prop}
Let $N, N' \subset M$ be minimal hypersurfaces, and let $S \subseteq N, N'$ be the set of points at which $N, N'$ intersect nontransversely.
Then one of the following holds:
\begin{enumerate}
\item $N \cap N'$ is empty.
\item $\dim(N \cap N') = d - 2$ and $\dim S \leq d - 3$.
\item There exists $p \in S$ such that the germs of $N, N'$ at $p$ are equal.
\end{enumerate}
\end{proposition}
\begin{proof}
Let $p \in S$, and let $P$ be the tangent space of $N, N'$ at $x$.
Then we can view $N, N'$ as the graphs of functions $u, u'$ over $P$, say taken in normal coordinates based at $p$; thus we identify $P$ with $\RR^{d - 1}$.
Reasoning as in the proof of \cite[Theorem 7.3]{colding2011course}, the difference $v := u - u'$ solves a linear elliptic PDE $Qv = 0$, and in a neighborhood $U \ni p$, the exponential map $P \to N$ induces Lipschitz isomorphisms $\{v = 0\} \cap U \cong N \cap N' \cap U$ and $\{v = \dif v = 0\} \cap U \cong S \cap U$.
If $v$ only has zeroes of finite order, then the claim follows from Proposition \ref{nodal set is generically smooth}.
Otherwise, $v$ is identically $0$ by the unique continuation theorem \cite[Theorem 6.1]{colding2011course}, so $N \cap U = N' \cap U$.
\end{proof}

\begin{lemma}
Let $F$ be a calibration, and let $B \subseteq M$ be a sufficiently small ball.
Then for any complete connected $F$-calibrated hypersurface $N$, 
\begin{equation}\label{area bound for calibrated}
\Mass(N \cap B) \leq \Mass(\partial B).
\end{equation}
\end{lemma}
\begin{proof}
By the Thom transversality theorem, we may assume that $N$ meets $\partial B$ transversely. 
Let $S := N \cap \partial B$, which by transversality can be identified with a closed $d - 2$-dimensional submanifold of $\Sph^{d - 1}$.
Since $H_{d - 2}(\Sph^{d - 1}, \RR) = 0$, there exists a relatively open set $U \subseteq \partial B$ which is bounded by $S$.
Since $H^{d - 1}(B, \RR) = 0$, we may write $F = \dif A$ in a neighborhood of $B$, where by Lemma \ref{Hodge theorem} and the Sobolev embedding theorem we may assume that $A$ is continuous. Then
\begin{align*}
\Mass(N \cap B) &= \int_{N \cap B} F = \int_S A = \int_U F \leq \Mass(U) \leq \Mass(\partial B). \qedhere
\end{align*}
\end{proof}

\begin{lemma}
There exists a constant $C > 0$, only depending on $M$, such that for every calibration $F$ and complete $F$-calibrated hypersurface $N$, we have the curvature bound
\begin{equation}\label{curvature bound for calibrated}
\|\Two_N\|_{C^0} \leq C.
\end{equation}
\end{lemma}
\begin{proof}
Let $x \in N$ and let $r > 0$ be small.
Then each component $N'$ of $N \cap B(x, r)$ is absolutely area-minimizing by the fundamental theorem of calibrated geometry, so it is stable.
By (\ref{area bound for calibrated}), $\Mass(N') \lesssim r^{d - 1}$.
So by \cite[pg785, Corollary 1]{Schoen81},\footnote{See also \cite[Theorem 3]{Schoen75} for an easier proof when $M$ has nonpositive curvature and dimension $d \leq 6$, or \cite[Chapter 2, \S\S4-5]{colding2011course} for a textbook treatment of a similar estimate. By \cite[Lemma 2.4]{chodosh2022complete}, we may remove the dependence on the volume bound if $d \leq 4$.}
\begin{align*}
\|\Two_{N'}\|_{B(x, r/2)} \lesssim_{d, \|\Riem_g\|_{C^0(B(x, 2r))}} \frac{1}{r}.
\end{align*}
Since $N'$ was an arbitrary component, the same estimate holds for $N$.
Using the compactness of $M$, we may cover it by finitely many balls in which estimates of this form hold to conclude (\ref{curvature bound for calibrated}).
\end{proof}

\begin{lemma}\label{calibrated implies disjoint}
Let $F$ be a calibration, and let $N, N'$ be complete connected $F$-calibrated hypersurfaces.
If $N \cap N'$ is nonempty, then $N = N'$.
\end{lemma}
\begin{proof}
We first observe that if $x \in N \cap N'$, then $(\star F(x))^\sharp$ is the normal vector to both $N, N'$ at $x$.
Therefore $N \cap N'$ only consists of points of tangency.
By Proposition \ref{intersection theory prop}, it follows that either the germs of $N, N'$ at $x$ are equal.
Since the germs are equal and $N, N'$ are connected, a standard boostrapping argument implies that $N = N'$.
\end{proof}

\begin{proposition}\label{existence of semicanonical lamination}
Let $F$ be a best comass calibration.
Then the set of $F$-calibrated hypersurfaces is the set of leaves of a lamination $\lambda_F$, which contains every measured stretch lamination associated to $[F]$.
\end{proposition}
\begin{proof}
Let $\mathscr L_F$ be the set of connected complete $F$-calibrated hypersurfaces.
By Lemma \ref{calibrated implies disjoint}, $\mathscr L_F$ consists of pairwise disjoint minimal hypersurfaces.
By Proposition \ref{MCL contains Thurston}, there exists a measured stretch lamination $\lambda$ associated to $[F]$, and then by Proposition \ref{properties of calibrated laminations}, $\mathscr L_F$ contains every leaf of $\lambda$.
Since the estimate (\ref{curvature bound for calibrated}) is independent of $N$, it follows by Theorem \ref{disjoint surfaces are lamination} that $\mathscr L_F$ is the set of leaves of some lamination $\lambda_F$.
\end{proof}

\begin{lemma}\label{existence of intersections}
Let $\mathscr S$ be a nonempty set of laminations.
Suppose that there exists a hypersurface which is a leaf of every lamination in $\mathscr S$.
Then there exists a lamination whose set of leaves is the intersection of the sets of leaves of the laminations in $\mathscr S$.
\end{lemma}
\begin{proof}
Let $\lambda \in \mathscr S$, and let $(F_\alpha, K_\alpha)$ be a laminar atlas for $\mathscr S$.
Let $K'_\alpha$ be the set of $k \in K_\alpha$ such that for every $\kappa \in \mathscr S$, there exists a leaf $N$ of $\kappa$ such that
$$(F_\alpha)_*(\{k\} \times J) \subseteq N.$$
It is clear that this property is preserved by transition maps.
Then $K_\alpha'$ is an intersection of compact sets (since the local leaf spaces of each $\kappa \in \mathscr S$ is compact), so $K_\alpha'$ is compact.
The hypersurface which is a common leaf of every lamination in $\mathscr S$ witnesses that for some $\alpha$, $K_\alpha'$ is nonempty.
Therefore $(F_\alpha, K'_\alpha)$ is a laminar atlas for the lamination whose support is $\bigcap_{\kappa \in \mathscr S} \supp \kappa$.
\end{proof}

\begin{proposition}\label{existence of canonical lamination}
The set of hypersurfaces which are calibrated by every best comass representative of $\rho$ is the set of leaves of a lamination $\lambda_\rho$, which contains every measured stretch lamination associated to $\rho$.
\end{proposition}
\begin{proof}
By Proposition \ref{MCL contains Thurston}, there is a (measured stretch) lamination which is calibrated by every best comass representative of $\rho$.
So we may apply Lemma \ref{existence of intersections} to the set $\mathscr S$ of all calibrated laminations $\lambda_F$ produced by Proposition \ref{existence of semicanonical lamination}, where $F$ ranges over best comass representatives of $\rho$.
\end{proof}

\begin{definition}
The lamination $\lambda_\rho$ constructed in Proposition \ref{existence of canonical lamination} is the \dfn{canonical lamination} associated to $\rho$.
\end{definition}

If $d = 2$ and $\rho \in H^1(M, \ZZ)$, then $\lambda_\rho$ is the lamination obtained by taking an intersection over the best Lipschitz maps of homotopy class $\rho$ by Daskalopolous and Uhlenbeck \cite[\S6.2]{daskalopoulos2020transverse}.

%%%%%%%%%%%%%%%%%%%%%%%%%%%%%%%%
\subsection{Structure of the canonical lamination}\label{canonical structure}
We now study the structure of the canonical lamination.
A sticky technical point is that $H_{d - 1}(M, \RR)$ need not be strictly convex, so there may be many $\alpha$ in the stable unit sphere such that 
\begin{equation}\label{flats duality}
\Comass(\rho) = \langle \rho, \alpha\rangle.
\end{equation}
In particular, there may be many measured stretch sublaminations of the canonical lamination which are mutually nonhomologous.
We therefore introduce the dual set 
$$\rho^* := \{\alpha \in H_{d - 1}(M, \RR): \langle \rho, \alpha\rangle = \Mass(\alpha) = 1\}.$$
It is clear that any measured sublamination of the canonical lamination normalized to have mass $1$ represents a member of $\rho^*$.
In fact, this condition completely characterizes $\rho^*$, as we now show.

\begin{lemma}\label{homologically minimizing means measured stretch}
For every $\alpha \in \rho^*$, every measured oriented, homologically minimizing, lamination representing $\alpha$ is a measured stretch lamination associated to $\rho$.
\end{lemma}
\begin{proof}
Let $\dif u$ be the Ruelle-Sullivan current of the measured oriented, homologically minimizing lamination $\lambda$, and let $F$ be a best comass representative of $\rho$.
Since $\lambda$ is homologically minimizing,
$$\int_M \dif u \wedge F = \langle \rho, \alpha\rangle = \Mass(\alpha) = \Mass(\lambda),$$
so by Proposition \ref{calibration condition}, $F$ calibrates $\lambda$.
Therefore by Proposition \ref{calibrated means measured stretch}, $\lambda$ is a measured stretch lamination.
\end{proof}

\begin{lemma}\label{existence for least gradient}
For each $\alpha \in H_{d - 1}(M, \RR)$, there exists an $\alpha$-equivariant function of least gradient $u: \tilde M \to \RR$.
\end{lemma}
\begin{proof}
The space $X$ of $\alpha$-equivariant functions on $\tilde M$ is closed under $L^1_\loc$ limits by Lemma \ref{L1 convergence preserves pi1}, so the existence of a minimizer in $X$ follows from an argument similar to the solution of the Dirichlet problem for least gradient functions \cite[Theorem 1.20]{Giusti77}.
\end{proof}

\begin{proposition}\label{enough measures in canonical lamination}
For each $\alpha \in \rho^*$, there exists a measured stretch sublamination of $\lambda_\rho$ with homology class $\alpha$.
\end{proposition}
\begin{proof}
Let $u$ be the function of least gradient furnished by Lemma \ref{existence for least gradient}.
The measured oriented, homologically minimizing, lamination $\kappa_u$ has class $\alpha$.
So by Lemma \ref{homologically minimizing means measured stretch}, it is a measured stretch lamination and hence is a sublamination of $\lambda_\rho$.
\end{proof}

We next use the decomposition of measured laminations \cite[{\S}I.3]{Morgan88} to partition the leaves of $\lambda_\rho$ into various categories.
In this direction we shall need to study measured laminations which are minimal with respect to inclusion; as the word ``minimal'' is overloaded, we shall call such laminations ``indecomposable''.

\begin{definition}
Let $\lambda$ be a lamination.
\begin{enumerate}
\item $\lambda$ is \dfn{indecomposable} if the only sublamination of $\lambda$ is itself.
\item If $\lambda$ is indecomposable, then $\lambda$ is \dfn{exceptional} if $\supp \lambda \neq M$ and $\lambda$ does not consist of a single leaf.
\item $\lambda$ is a \dfn{parallel family of closed leaves} if there exists a closed hypersurface $N \subset M$ with trivial normal bundle, such that every leaf of $\lambda$ is a section of the normal bundle of $N$.
\item A leaf $N$ of $\lambda$ is \dfn{nonmeasurable} if, for every sublamination $\kappa \subset \lambda$ which admits a transverse measure, $N$ is not a leaf of $\kappa$.
\end{enumerate}
\end{definition}

Thus every indecomposable lamination either is a foliation in which every leaf is dense, an exceptional indecomposable lamination, or a closed hypersurface.
Moreover, every local leaf space $K_\alpha$ of an exceptional indecomposable lamination $\lambda$ is a Cantor set \cite[{\S}I.3.1]{Morgan88}, and every leaf of $N$ is noncompact.
Every nonmeasurable leaf is noncompact, for if $N$ is a closed leaf, then $N$ equipped with its Dirac measure is a measured sublamination of $\lambda$.

\begin{theorem}\label{MorganShelan}
Let $\lambda$ be a measured oriented lamination in the closed manifold $M$.
Then either $\lambda$ is a foliation with a dense leaf, or $\lambda$ separates into finite number of clopen sublaminations, each of which is a parallel family of closed leaves or an exceptional indecomposable lamination.
\end{theorem}
\begin{proof}
First observe that the proof of \cite[Theorem I.3.2]{Morgan88} goes through for any lamination $\lambda$ such that no leaf of $\lambda$ is dense in $M$, even if $\lambda$ is a foliation.
Twisted families of closed leaves (that is, families of sections of a nontrivial normal bundle of a closed hypersurface) are excluded by the fact that $\lambda$ is oriented, so its leaves are oriented, and hence the normal bundle of any of its leaves is trivial.
\end{proof}

\begin{proposition}\label{classification of leaves}
For each leaf $N$ of $\lambda_\rho$, one of the following holds:
\begin{enumerate}
\item $N$ is closed.
\item $N$ is a noncompact leaf of an exceptional indecomposable measured stretch lamination associated to $\rho$.
\item $N$ is noncompact and $\lambda_\rho$ is a foliation which admits a transverse measure.
\item $N$ is noncompact and $N$ is a nonmeasurable leaf of $\lambda_\rho$.
\end{enumerate}
\end{proposition}
\begin{proof}
If $N$ is a closed leaf of $\lambda_\rho$, then $N$ equipped with its Dirac measure is a measured lamination, calibrated by any tight representative of $\rho$; hence it is measured stretch for $\rho$.
Otherwise, since $N$ has no boundary, it is noncompact.

If $N$ is noncompact, but is contained in a measured sublamination $\kappa$ of $\lambda_\rho$, then by Theorem \ref{MorganShelan}, either $\kappa$ is a foliation or $N$ is contained in an exceptional indecomposable sublamination.
If $\kappa$ is a foliation, then
$$\supp \kappa \supseteq \supp \lambda_\rho \supseteq \supp \kappa,$$
implying $\kappa = \lambda_\rho$.
Otherwise, the exceptional indecomposable sublamination $\zeta$ of $\kappa$ containing $N$ is calibrated by any tight representative of $\rho$, so $\zeta$ is measured stretch for $\rho$ by Proposition \ref{calibrated means measured stretch}.
\end{proof}

\begin{corollary}\label{measurable leaves are contained in indecomposables}
Let $N$ be a leaf of the canonical lamination $\lambda_\rho$.
Then either $N$ is nonmeasurable, or $N$ is contained in an indecomposable measured stretch lamination associated to $\rho$.
\end{corollary}
\begin{proof}
By Proposition \ref{classification of leaves}, if $N$ is not nonmeasurable, then either $N$ is closed, in which case $N$ is itself an indecomposable measured stretch lamination, or $N$ is noncompact and is contained in an exceptional indecomposable measured stretch lamination.
\end{proof}

\begin{corollary}
Let $F$ be a best comass representative of $\rho$, and $N$ a leaf of the calibrated lamination $\lambda_F$.
Then either $N$ is a leaf of the canonical lamination $\lambda_\rho$, or $N$ is a nonmeasurable leaf of $\lambda_F$.
\end{corollary}
\begin{proof}
Suppose that $N$ is a leaf of a measured sublamination $\kappa$ of $\lambda_F$.
Then, since $\kappa$ is calibrated by $F$, $\kappa$ is measured stretch by Proposition \ref{calibrated means measured stretch}, hence is a sublamination of $\lambda_\rho$.
\end{proof}

Another consequence of the decomposition of laminations is that the extreme points of $\rho^*$ are represented by indecomposable laminations.
Recall that a point $\alpha$ of a convex set $S$ is \dfn{extreme} if $\alpha$ cannot be written as the convex combination of two distinct members of $S$.

\begin{lemma}\label{extreme points are closed under sublaminations}
Let $\alpha$ be an extreme point of $\rho^*$, and let $\kappa$ be a measured stretch lamination in $\alpha$.
Then any sublamination of $\kappa$ represents a scalar multiple of $\alpha$.
\end{lemma}
\begin{proof}
By replacing $\kappa$ with a proper sublamination if necessary, we may assume that $\kappa$ is not a foliation.
Let $\zeta$ be a sublamination of $\kappa$.
By Theorem \ref{MorganShelan} and the fact that the leaves of a parallel family of closed leaves are all homologous, after replacing $\zeta$ with a sublamination of $\zeta$, we may assume that $\zeta$ is a clopen parallel family of closed leaves, or is an exceptional indecomposable sublamination of $\kappa$.
Since $\kappa$ is the linear combination of finitely many such clopen sublaminations, we may write $\alpha$ as a convex combination of $\beta_1, \dots, \beta_k$ where the $\beta_i$ are the (normalized to mass $1$) homology classes of clopen sublaminations of $\lambda$.
But $\beta_i \in \rho^*$, so $\beta_i = \alpha$, hence $[\zeta] = \alpha$.
\end{proof}

\begin{proposition}\label{extreme points are indecomposable}
Let $\alpha$ be an extreme point of $\rho^*$. Then $\alpha \in \rho^*_{\rm exc}$.
\end{proposition}
\begin{proof}
By Proposition \ref{enough measures in canonical lamination}, there exists a measured stretch lamination $\kappa$ representing $\alpha$.
By Theorem \ref{MorganShelan}, $\kappa$ has an indecomposable sublamination $\zeta$.
By Lemma \ref{extreme points are closed under sublaminations}, possibly after rescaling the transverse measure, $\zeta$ is a representative of $\alpha$.
Since any tight representative $F$ of $\rho$ calibrates $\kappa$, $F$ also calibrates $\zeta$, so by Proposition \ref{calibrated means measured stretch}, $\zeta$ is a measured stretch sublamination of $\lambda_\rho$.
\end{proof}
%%%%%%%%%%%%%%%%%%%%%%%%
\subsection{Convexity of the stable unit ball}\label{convexity sec}
Let $M$ be a closed manifold of dimension $\leq 7$.
Auer and Bangert \cite{Auer01} claimed certain results concerning the convex structure of the stable unit ball
$$B := \{\alpha \in H_{d - 1}(M, \RR): \Mass(\alpha) \leq 1\}.$$
Here we show that some of these results follow from the structure theory of the canonical lamination.

Recall that a \dfn{flat} $S \subset \partial B$ is a set such that, for some supporting hyperplane $H$ of $B$, $S = H \cap B$.
Thus $B$ is strictly convex iff every flat is a point.

\begin{lemma}
Suppose that $S \subset \partial B$ is a flat of the stable unit ball $B \subset H_{d - 1}(M, \RR)$.
Then there exists $\rho$ in the costable unit sphere of $H^{d - 1}(M, \RR)$ such that $S \subseteq \rho^*$.
\end{lemma}
\begin{proof}
Since $S$ is convex, $\partial S$ is topologically a sphere, so $\partial S$ admits a Borel probability measure $\nu$ of full support.
Then we take the vector-valued integral 
$$\beta := \int_{\partial S} \alpha \dif \nu(\alpha),$$
thus $\beta \in S$ by convexity.
By the Hanh-Banach theorem (as in (\ref{Federer duality})), there exists $\rho \in H^{d - 1}(M, \RR)$ such that $\beta \in \rho^*$.

We claim that $\partial S \subseteq \rho^*$.
If not, then by continuity of $\alpha \mapsto \langle \rho, \alpha\rangle$, there is a positive measure set of $\partial S$ on which $\langle \rho, \cdot\rangle < 1$, hence
$$\Comass(\rho) = \langle \rho, \beta\rangle = \int_{\partial S} \langle \rho, \alpha\rangle \dif \nu(\alpha) < \Comass(\rho),$$
a contradiction.
Since $\rho^*$ is convex, it follows that $S \subseteq \rho^*$.
\end{proof}

Recall that the exterior product on the cohomology ring $H^\bullet(M, \RR)$ induces, by Poincar\'e duality, an \dfn{intersection product}
\begin{align*}
H_{d - k}(M, \RR) \times H_{d - \ell}(M, \RR) &\to H_{d - k - \ell}(M, \RR) \\
(\alpha, \beta) &\mapsto \alpha \cdot \beta.
\end{align*}

\begin{proposition}\label{flats are nonintersecting}
Let $B$ be the stable unit ball of $H_{d - 1}(M, \RR)$.
Suppose that $S \subset \partial B$ is a flat, and $\alpha, \beta \in S$. Then $\alpha \cdot \beta = 0$.
\end{proposition}
\begin{proof}
Let $\rho$ be the cohomology class dual to $S$ given by (\ref{flats duality}).
By Proposition \ref{enough measures in canonical lamination}, there exist measured stretch sublaminations $\kappa_\alpha, \kappa_\beta$ of $\lambda_\rho$, of classes $\alpha, \beta$.
Let $\dif u_\alpha, \dif u_\beta$ be their Ruelle-Sullivan currents, and suppose that $x$ is in the union of their supports.
If $N$ denotes the leaf of $\lambda_\rho$ containing $x$, then for $\sigma = \alpha, \beta$,
$$\dif u_\sigma(x) = \normal_N^\flat(x) \mu_\sigma(x)$$
where $\mu_\sigma$ is the positive Radon measure induced on $M$ by the transverse measure to $\kappa_\sigma$ \cite[Lemma 3.1]{BackusCML}.
In particular, $\dif u_\alpha|_{\supp \dif u_\beta}$ is a scalar multiple of $\dif u_\beta$, so $\dif u_\alpha \wedge \dif u_\beta = 0$, hence $\alpha \cdot \beta = 0$.
\end{proof}

Although the definitions of the stable unit ball $B$ and the intersection product depend on the Riemannian metric on $M$, we conclude a purely topological condition for strict convexity of $B$.

\begin{corollary}\label{condition for strict convexity}
Suppose that the kernel of the wedge product 
\begin{equation}\label{wedge product}
\wedge: H^1(M, \RR) \otimes H^1(M, \RR) \to H^2(M, \RR)
\end{equation}
is spanned by symmetric tensors $\theta \otimes \theta$, $\theta \in H^1(M, \RR)$.
Then the stable unit ball of $H_{d - 1}(M, \RR)$ is strictly convex.
\end{corollary}
\begin{proof}
We prove the contrapositive.
If $H_{d - 1}(M, \RR)$ does not have a strictly convex unit ball, then by Proposition \ref{flats are nonintersecting} there exist linearly independent $\alpha, \beta \in H_{d - 1}(M, \RR)$ such that $\alpha \cdot \beta = 0$.
Dually, this means that we can find linearly independent $\theta, \omega \in H^1(M, \RR)$ such that $\theta \otimes \omega \in \ker(\wedge)$.
\end{proof}

\begin{example}\label{torus convex}
Suppose that $M$ is homotopic to a torus, so that the cohomology ring $H^\bullet(M, \RR)$ is isomorphic to the exterior algebra of $\RR^d$.
In particular, the kernel of (\ref{wedge product}) is spanned by symmetric tensors, so by Corollary \ref{condition for strict convexity}, the stable unit ball of $H_{d - 1}(M, \RR)$ is strictly convex.
\end{example}






%%%%%%%%%%%%%%%%%%%%%%%%%%%%%%%%
\section{The Euler-Lagrange equation for tight forms}\label{infinityMax}
\subsection{Formal derivation of Euler-Lagrange equation}
To state our Euler-Lagrange equation we introduce some notation for derivatives of tensor fields along differential forms.
If $\alpha$ is a $k$-form, $\nabla$ is the Levi-Civita connection, and $T$ is a section of a tensor bundle $E$, we introduce the tensor $\nabla^\alpha T$, a section of $E \otimes \Omega^{k - 1}$, defined as follows: if $X_1, \dots, X_{k - 1}$ are vector fields, and
$$Y := (\iota_{X_1} \cdots \iota_{X_{k - 1}} \alpha)^\sharp$$
is the vector field dual to the contraction of $\alpha$, then
$$\langle \nabla^\alpha T, X_1 \otimes \cdots \otimes X_k\rangle := \nabla_Y T.$$
We think of $\nabla^\alpha T$ as a sort of ``weighted projection'' of $\nabla T$ to the subbundle $\ker \star \alpha \subset TM$.

\begin{proposition}
Suppose that $F_p$ are $C^1$ $p$-tight forms converging to a tight form $F$.
Furthermore suppose that as $p \to \infty$, $\|F_p\|_{C^{1 + \alpha}} \lesssim 1$.
Then $F \in C^1$ and 
\begin{equation}\label{infty Max}
\begin{cases}
\dif F = 0, \\
\langle \nabla^F F, F\rangle = 0.
\end{cases}
\end{equation}
\end{proposition}

We should clarify the PDE (\ref{infty Max}): $\nabla^F F$ is a section of $\Omega^{d - 1} \otimes \Omega^{d - 2}$, so its contraction with $F$ is the contraction of the $\Omega^{d - 1}$ part with $F$; thus $\langle \nabla^F F, F\rangle$ is a $d - 2$-form.
If $d = 2$, and $F = \dif u$, then (\ref{infty Max}) is exactly the $\infty$-Laplace equation 
$$\langle\nabla^2 u, \nabla u \otimes \nabla u\rangle = 0.$$

\begin{proof}
We first compute from (\ref{pMaxwell}) that $\dif F = 0$ and
\begin{align*}
0
&= \dif(|F_p|^{p - 2} \star F_p) \\
&= \dif(|F_p|^{p - 2}) \wedge \star F_p + |F_p|^{p - 2} \dif \star F_p \\
&= (p - 2) |F_p|^{p - 4} \langle \nabla F_p, F_p\rangle \wedge \star F_p + |F_p|^{p - 2} \dif \star F_p.
\end{align*}
If $F_p$ is nonzero, then we can divide through by $(p - 2) |F_p|^{p - 4}$ to get
\begin{equation}\label{intermediate p Max}
0 = \langle\nabla F_p, F_p\rangle \wedge \star F_p + \frac{|F_p|^2}{p - 2} \dif \star F_p.
\end{equation}
At the zeroes of $F_p$, we simply observe that (\ref{intermediate p Max}) holds for trivial reasons.

By assumption $||F_p|^2 \dif \star F_p| = o(p)$, so as $p \to \infty$, the second term of (\ref{intermediate p Max}) drops out.
We can take the limit of (\ref{intermediate p Max}) using the equicontinuity of $\nabla F_p$ to get
$$0 = \langle \nabla F, F \rangle \wedge \star F.$$
Taking the Hodge star of both sides, we get (\ref{infty Max}).
\end{proof}

%%%%%%%%%%%%%%%%%%%%%%%%%
\subsection{Geometric and variational interpretations of the Euler-Lagrange equation}\label{EL interpretation}
The PDE (\ref{infty Max}) has a simple geometric interpretation, which generalizes the interpretation of the $\infty$-Laplace equation as asserting that the gradient curves of an $\infty$-harmonic function are lines.

\begin{proposition}\label{infty Max calibrates}
Let $F$ be a $C^1$ solution of (\ref{infty Max}) with no zeroes, and let $N$ be a connected integral hypersurface of $\ker \star F$.
Then there exists $\lambda > 0$ such that $N$ is an $F/\lambda$-calibrated hypersurface.
\end{proposition}
\begin{proof}
Let $(X_1, \dots, X_{d - 1})$ be an orthonormal frame of vector fields tangent to $N$.
We then introduce the tensor field
$$T_i := X_1 \otimes \cdots \otimes \widehat{X_i} \otimes \cdots \otimes X_{d - 1},$$
where the hat means to remove that factor.
By definition of $N$, and the fact that $F$ has no zeroes, $F$ is a nonzero scalar field $\lambda$ times $\dif S_N$.
So $\iota_{X_1} \cdots \widehat{\iota_{X_i}} \cdots \iota_{X_{d - 1}} F$ is a nonzero scalar field $u_i$ times $X_i^\flat$.
Applying (\ref{infty Max}), we have 
$$0 = \langle \langle \nabla^F F, F\rangle, T_i\rangle = u_i \langle \nabla_{X_i} F, F \rangle = \frac{u_i}{2} \partial_{X_i} (|F|^2).$$
Since $u_i/2$ is nonzero, we see that $|F|^2$ is constant along integral curves of $X_i$.
Since $(X_1, \dots, X_{d - 1})$ spans the tangent bundle of $N$, we conclude that $|F|^2$ is constant along $N$, or equivalently that $\lambda$ is a constant.
Therefore $\dif S_N = F/\lambda$ is closed, hence $N$ is $F/\lambda$-calibrated.
\end{proof}

We now give a variational criterion for (\ref{infty Max}).
Unfortunately, the converse is weaker than we would like, because in order to apply arguments similar to those of \cite{Aronsson67,Sheffield12} we need to assume that $\ker(\star F)$ is a singular integrable distribution.

\begin{proposition}
Let $F$ be a $C^1$ closed $d - 1$-form, and suppose that for every $V \subseteq M$ such that $H^{d - 1}(V, \RR) = 0$, and every $U \Subset V$,
\begin{equation}\label{ABC inequality}
\Comass_U(F) \leq \|F\|_{C^0(\partial U)}.
\end{equation}
Then (\ref{infty Max}) holds.
\end{proposition}
\begin{proof}
It suffices to prove (\ref{infty Max}) locally, so we can cover $M$ by open balls $V$ with $H^{d - 1}(V, \RR) = 0$ and prove that (\ref{infty Max}) holds in slightly smaller balls.
Since $H^{d - 1}(V, \RR) = 0$, we can find a $C^2$ $d - 2$-form $A$ on a neighborhood of $\overline V$ with $\dif A = F$.
Also for $\xi$ a covariant tensor of valence $d - 1$ at $x$, let $\xi^{\rm as}$ be its antisymmetrization, and
$$f(x, \xi) := |\xi^{\rm as}|_{g^{-1}(x)}^2.$$

We first claim that for any $d - 2$-form $B$, $f(\cdot, B) = |\dif B|^2$.
Since $\nabla$ is torsion-free, antisymmetrization annihilates the Christoffel symbols of $\nabla$, so if $\nabla^\flat$ denotes a flat connection, then $(\nabla B)^{\rm as} = (\nabla^\flat B)^{\rm as}$; the latter is of course $\dif B$.
Thus in particular, $f(\cdot, A) = |F|^2$.

Let $W \Subset V$ be obtained by slightly shrinking $V$.
We claim that $A|_W$ is an absolute minimizer of $f(x, \nabla A(x))$ in the sense of \cite[Definition 5.1]{Barron2001}.
In other words, we claim that for each open $U \subseteq W$ with smooth boundary and each covariant tensor field $B$ of valence $d - 2$, such that $A - B$ has compact support in $U$,\footnote{Strictly speaking, the definition of absolute minimizer ranges over all open sets $U$ ($\Omega'$ in the notation of \cite{Barron2001}), not just those with smooth boundary; similarly one requires the competition class to range over traceless (rather than compactly supported) variations. However, it is trivial to modify the proof of \cite[Theorem 5.2]{Barron2001} to only require smooth domains and compactly supported variations.}
$$\sup_{x \in U} f(x, A(x)) \leq \sup_{x \in U} f(x, B(x)).$$
To see this, let $G = (\nabla B)^{\rm as}$, so that by (\ref{ABC inequality}),
\begin{align*}
\sup_{x \in U} f(x, A(x))
&= \Comass_U(F) \leq \|F\|_{C^0(\partial U)} = \|G\|_{C^0(\partial U)} \leq \|G\|_{C^0(U)} = \sup_{x \in U} f(x, B(x)).
\end{align*}

By the above claims and \cite[Theorem 5.2]{Barron2001}, for each $x \in W$, we have the Euler-Lagrange-Aronsson equation that for any $d - 2$-form $\theta$,
\begin{align*}
0 
&= \left\langle \frac{\partial f}{\partial \xi}(x, \nabla A(x)), \nabla \left[f(x, \nabla A(x))\right] \otimes \theta(x)\right\rangle \\
&= 2\langle (\nabla A(x))^{\rm as}, \nabla(|(\nabla A(x))^{\rm as}|^2) \otimes \theta(x)\rangle \\
&= 2\langle F(x), \nabla(|F(x)|^2) \otimes \theta(x)\rangle.
\end{align*}
Since $\nabla$ is a metric connection, $\nabla(|F|^2) = 2\langle \nabla F, F\rangle$.
Thus we have 
$$0 = 4\langle F, \langle \nabla F, F\rangle \otimes \theta\rangle = 4\langle \langle \nabla^F F, F\rangle, \theta\rangle$$
and since $\theta$ was arbitrary we conclude (\ref{infty Max}).
\end{proof}

\begin{proposition}
Let $F$ be a $C^1$ solution of (\ref{infty Max}) such that $\ker(\star F)|_{F \neq 0}$ is an integrable distribution.
Then for every $x \in M$ and every sufficiently small $r > 0$, (\ref{ABC inequality}) holds for $U = B(x, r)$.
\end{proposition}
\begin{proof}
First observe that if $x \in M$ and $F(x) = 0$, then $|F|$ has a local minimum at $x$, so for any $y$ sufficiently close to $x$, say $y \in B(x, r)$, $|F|$ does not have a local maximum at $y$ (unless $y$ is also a local minimum, hence $F(y) = 0$); therefore (\ref{ABC inequality}) holds.
Henceforth we assume that $F(x) \neq 0$.

Let $r > 0$ be such that $F|_{B(x, r)}$ has no zeroes and for some $s > r$, $H^{d - 1}(B(x, s), \RR) = 0$.
Let $\mathscr F$ be the foliation of $B(x, r)$ obtained by integrating $\ker(\star F)$.
By definition of $s$, we may assume that $F = \dif A$ for some $d - 2$-form $A$ defined on $B(x, s)$.
By Proposition \ref{infty Max calibrates}, for each leaf $N$ of $\mathscr F$ there exists $\lambda_N > 0$ such that for each hypersurface $N'$ with $N \cap \partial B(x, r) = N' \cap \partial B(x, r)$,
$$\lambda_N \Mass(N) = \int_N F = \int_{N \cap \partial B(x, r)} A = \int_{N' \cap \partial B(x, r)} A = \int_{N \cap \partial B(x, r)} F \leq \lambda_N \Mass(N'),$$
so $N$ is absolutely area-minimizing in $B(x, r)$ and hence meets $\partial B(x, r)$, say at some point $x_N$.
Moreover, $F|_N$ has constant comass $\lambda_N$; since $\mathscr F$ is a foliation it holds that for each $x \in V$ there exists a leaf $N \ni x$ of $\mathscr F$, and then 
\begin{align*}
|F(x)| &= |F(x_N)| \leq \|F\|_{C^0(\partial B(x, r))}. \qedhere
\end{align*}
\end{proof}

%%%%%%%%%%%%%%%
% \subsection{Parabolic character and higher regularity}
% Similarly to the $\infty$-Laplace equation, \todo{Cite Evans--Smart ``adjoint methods''} we shall view (\ref{infty Max}) as a degenerate parabolic equation.
% To be more precise, suppose that $A$ is a $d - 2$-form in Coulomb gauge such that $F = \dif A$ solves (\ref{infty Max}), hence 
% \begin{equation}\label{integrated infty Max}
% \begin{cases}
% \langle \nabla^{\dif A} \dif A, \dif A\rangle = 0, \\
% \dif^* A = 0.
% \end{cases}
% \end{equation}
% We consider the linearized equation at $A$
% \begin{equation}\label{linearized infty Max}
% \begin{cases}
% \langle \nabla^{\dif B} \dif A, \dif A\rangle + \langle \nabla^{\dif A} \dif B, \dif A\rangle + \langle \nabla^{\dif A} \dif A, \dif B\rangle = 0, \\
% \dif^* B = 0.
% \end{cases}
% \end{equation}
% Suppose that $\dif A$ has no zeroes.
% Let $\partial_{x^1}, \dots, \partial_{x^{d - 1}}$ be an orthonormal frame for $\ker(\star \dif A)$. \todo{Construct $\partial_t$ in terms of $A$. Use a transform to make $A$ smooth}.













\section{Concluding remarks}\label{open problems}
\subsection{Generalizations}
In this paper I have only dealt with closed Riemannian manifolds $M$ of dimension $d \leq 7$, and submanifolds of codimension $c := 1$.
In the prequel paper \cite{BackusCML} I have only studied the interior behavior of functions of least gradient, but moreover, the statements of the main theorems would be significantly more involved in a more general setting.

We cannot easily weaken the assumption $d \leq 7$, since the double-napped Simons cone defines a function of least gradient on $\RR^8$ which does not admit a lamination structure \cite{BackusCML}.
Similarly, if the codimension $c \geq 2$, then Liu recently constructed homologically minimizing submanifolds $N$ which do not admit calibrations, even if $d = 3$ or $M = \CC \PP^2$ with a perturbation of the Fubini-Study metric \cite{liu2023homologically}.
Liu also showed that if $d \geq 8$ and $c = 1$, then homologically minimizing hypersurfaces may only admit \emph{smooth} calibrations; the regularity of best comass calibrations remains open if $d \leq 7$ \todo{Look for more literature on this}.

After replacing Poincar\'e duality with Lefschetz duality and imposing boundary conditions, I expect that the results of this paper go through if $M$ is a compact manifold with strictly mean-convex boundary $\partial M$.
Under that assumption, a version of the max-flow min-cut theorem has been claimed in the physics literature \cite[Appendix A]{Freedman_2016}.
To see that convexity is necessary, let $\theta$ be the latitude on $\Sph^2$, and let $M := \{|\theta| \leq \pi/4\}$; then any function of least gradient which extends the boundary data 
$$h(\theta, \phi) := \begin{cases} 1, \text{ if } \theta = \pi/4 \\ 0, \text{ if } \theta = -\pi/4\end{cases}$$
is constant on the interior, hence does not induce a lamination.

By replacing the boundary components of $M = \{|\theta| \leq \pi/4\}$ by cusps, we see that if $M$ has infinite ends, then it is possible for $H_{d - 1}(M, \RR) \neq 0$ but the stable seminorm to be identically $0$.
Freedman and Headrick conjectured that a max-flow min-cut theorem should hold for a manifold with infinite ends $M$ if there exist compact manifolds $M_i$ with mean-convex boundary, such that $(M_i)$ is a compact exhaustion of $M$ \cite[Appendix A]{Freedman_2016}.
This assumption clearly rules out the existence of cusps.

However, even assuming the existence of a compact exhaustion with mean-convex boundaries, I expect that extension of this paper to noncompact manifolds to be quite challenging.
If $H^{d - 1}(M, \RR)$ is infinite-dimensional, then its completion with respect to the costable norm is unlikely to be reflexive, so arguments involving the Hanh-Banach theorem will become precarious.
Moreover, already in the case of $\RR^d$, the behavior of functions of least gradient near infinity is rather involved \cite[\S4.4]{górny2021}.

%%%%%%%%%%%%%%%%%%%%%%%%%
\subsection{Motivation for the canonical lamination}\label{Teichmuller}
Our motivation for introducing the canonical lamination arose from an analogy with Thurston's approach to Teichm\"uller theory using best Lipschitz maps \cite{Thurston98}.
Given $\gamma \geq 2$, let $\widetilde{\mathscr M}_\gamma$ be the Teichm\"uller space of hyperbolic metrics on the closed surface $S_\gamma$ of genus $\gamma$.
Given $g, h \in \widetilde{\mathscr M}_\gamma$, let $\Lip(g, h)$ be the Lipschitz constant of a best Lipschitz map homotopic to
$$\id: (S_\gamma, g) \to (S_\gamma, h).$$ 
For a tangent vector $v \in T_g(\widetilde{\mathscr M}_\gamma)$, let $\Comass(v)$ be the partial derivative of $\log \Lip(g, \cdot)$ in the direction $v$.
This quantity, the \dfn{Thurston asymmetric norm}, is an asymmetric norm on $T_g(\widetilde{\mathscr M}_\gamma)$ obtained by solving an $L^\infty$ variational problem intimately tied to the structure of minimal laminations, so it is tempted to make an analogy between $T_g(\widetilde{\mathscr M}_\gamma)$ and $H^{d - 1}(M, \RR)$, where both vector spaces are equipped with the norm $\Comass$.
Two particularly salient pieces of evidence for the analogy are:
\begin{enumerate}
\item The unit spheres of the dual spaces of $T_g(\widetilde{\mathscr M}_\gamma)$ and $H^{d - 1}(M, \RR)$ can both be viewed as spaces of projective measured minimal laminations, whose norm is given by an $L^1$ (actually $BV$) variational problem \cite[Theorem 5.1]{Thurston98}.
\item In both cases, we can construct a canonical lamination. In Thurston's case, the canonical lamination is given by those geodesics which are maximally stretched by every best Lipschitz map homotopic to $\id_{S_\gamma}$ \cite[\S8]{Thurston98}. See also Conjecture \ref{chain recurrence}.
\end{enumerate}
However, one should not take this analogy too seriously.
A key feature of Thurston's theory is the Birman-Series theorem: the union of the supports of all geodesic laminations on $(S_\gamma, g)$ has Hausdorff dimension $0$.
As a corollary, for almost every $h \in \widetilde{\mathscr M}_\gamma$, the canonical lamination associated to $(g, h)$ is a closed geodesic \cite[\S10]{Thurston98}.
The analogue of the Birman-Series theorem is clearly not true in our case, and in fact, if $M$ is a square flat torus, then it is easy to see that every canonical lamination covers all of $M$.

Thurston's canonical lamination $\lambda$ is chain-recurrent, in the sense that traveling along the geodesics in $\lambda$ defines a chain-recurrent dynamical system.
This makes no sense for higher-dimensional laminations, but is equivalent to assert that Thurston's canonical lamination can be approximated by finite sums of closed geodesics \cite[\S9]{Gu_ritaud_2017}.
We conjecture that the analogous fact should hold for our canonical lamination:

\begin{conjecture}\label{chain recurrence}
Let $\rho \in H^{d - 1}(M, \RR)$, and let $\lambda_\rho$ be the canonical lamination.
Then it is possible to approximate $\lambda_\rho$ in Thurston's geometric topology\footnote{See \cite[\S1]{BackusCML} for the definition of Thurston's geometric topology in this setting.} by finite unions of closed minimal hypersurfaces.
\end{conjecture}

%%%%%%%%%%%%%%%%%%%
\subsection{Taut foliations and eikonal calibrations}
The following problem was suggested to me by Karen Uhlenbeck. 
There exist closed hyperbolic $3$-manifolds which admit taut foliations; in that case, \emph{after changing the metric} one may find a minimal foliation.
Thus one cannot rule out minimal foliations by a simple topological argument (as one could rule out geodesic foliations of closed hyperbolic surfaces).
However, if a minimal foliation exists, then it is natural to study the tight form which calibrates it.
This form satisfies a particularly strong form 
\begin{equation}\label{eikonal}
\begin{cases}\dif F = 0 \\ \dif(|F|^2) = 0\end{cases}
\end{equation}
of the Euler-Lagrange equation for tight forms which is analogous to the role of the eikonal equation
$$\dif(|\dif u|^2) = 0$$
in the study of the $\infty$-Laplace equation.
Global solutions of the eikonal equation are rather uncommon (for example, the Dirichlet problem for the eikonal equation on $\Ball^d$ is overdetermined), so this suggests a means to rule out the existence of minimal foliations:

\begin{conjecture}\label{Karen}
Let $\Gamma$ be the fundamental group of a closed hyperbolic $3$-manifold $M$.
Then there does not exist a solution of the eikonal system (\ref{eikonal}) on $\Hyp^3$ which is invariant under $\Gamma$.
In particular, there does not exist a minimal foliation on $M$.
\end{conjecture}

% %%%%%%%%%%%%%%%%%%%%%%%%
% \subsection{String theory}
% Suppose that $\overline M$ is a Riemannian manifold-with-boundary such that if $\Gamma \subset \partial M$ bounds a domain in $\partial M$, then there exists an area-minimizing and minimal hypersurface in $M$ with boundary $\Gamma$.
% One can interpret $M$ as a spacelike slice of a domain of a string theory, and $\partial M$ as a spacelike slice of the domain of the conformal field theory predicted by the gauge theory/gravity duality \cite{Freedman_2016}.
% In this setting, if $N$ is a smooth subdomain of $\partial M$, the entropy of entanglement $S(\partial N)$ of the CFT through $\partial N$ is given by the \dfn{Ryu-Takayanagi formula}
% \begin{equation}\label{RyuTakayanagi}
% S(\partial N) = \max_F \int_N F
% \end{equation}
% where $F$ ranges over calibrations on $\overline M$ \cite[(2.8)]{Freedman_2016}.
% If the results of this paper generalize cleanly to manifolds with boundary, then, by replacing $N$ with an area-minimizing hypersurface in $M$ with the same boundary, we should have:

% \begin{conjecture}\label{tight RyuTakayanagi}
% There exists a tight calibration $F$ which realizes the maximum in (\ref{RyuTakayanagi}).
% \end{conjecture}

% We know by recent work of Loisel that it is possible to solve the $\infty$-Laplacian in polynomial time \cite{Loisel_2020}.
% We expect the same technique to work for tight forms, so Conjecture \ref{tight RyuTakayanagi} could give an efficient algorithm for computing the entanglement entropy of $\partial N$.

%%%%%%%%%%%%%%%%%%%%
\subsection{The \texorpdfstring{$p$-Laplacian}{p-Laplacian}}
If $F_p$ is $p$-tight and we write $F_p = \dif A_p$, where the gauge potential $A_p$ satisfies $\dif^* A_p = 0$, then $A_p$ solves the $L^p$ analogue 
$$\dif^*(|\dif A_p|^{p - 2} \dif A_p) + \dif(|\dif^* A_p|^{p - 2} \dif^* A_p) = 0$$
of the Hodge Laplacian.
There is a rough principle that those estimates on the Laplace equation which do not depend on the maximum principle (or the closely related comparison principle and Harnack inequality) should also hold for the Hodge Laplacian.
Therefore one expects that those estimates on $p$-harmonic functions which do not depend on the maximum principle should hold for $p$-tight forms; see \cite[Chapter 11]{kinnunen2021maximal} for a reference on such estimates.

The first eigenfunction of the $1$-Laplacian, in a suitable $p$-regularized sense, is the indicator function of a Cheeger set \cite{Kawohl2003}.
An analogue of calibrations, given by vector fields whose divergence is bounded from below, is available for Cheeger sets \cite{Grieser05}.
It would be very interesting to establish an analogue of tight forms, and their duality with the first eigenfunction of the $1$-Laplacian, in the setting of Cheeger sets. 

%%%%%%%%%%%%%%%%%%%%
\subsection{The tight PDE}
We have carefully sidestepped the lack of a theory of $L^\infty$ variational systems by working with $L^p$ variational systems, $d < p < \infty$, which can be written in divergence form, and then taking a limit.
When we do have to work in the limit, we passed to the dual problem, which was a \emph{scalar} $L^1$ variational problem.
One would therefore like to work with a notion of weak solution for the Euler-Lagrange system (\ref{infty Max}).

\begin{problem}
Introduce a notion of weak solution for (\ref{infty Max}) which generalizes the notion of viscosity solution for the $\infty$-Laplacian.
Show that the following are equivalent for a $d - 1$-form $F$:
\begin{enumerate}
\item $F$ is a weak solution of the Euler-Lagrange equation (\ref{infty Max}).
\item $F$ is tight.
\item $F$ is closed and for every small ball $B \subset M$, the variational condition (\ref{ABC inequality}) holds.
\end{enumerate}
\end{problem}

A candidate notion of weak solution is the \dfn{contact solution} of Katzourakis \cite{Katzourakis2018OnAV}, though this theory is not fleshed out at the time of writing.

Another problem caused by the lack of the maximum principle is the apparent failure of tight forms to be unique.
This was already a problem in the work of Daskalopolous and Uhlenbeck on maps to $\Sph^1$ which inspired this work \cite[Conjecture 9.2]{daskalopoulos2020transverse}.

\begin{conjecture}
The tight $d - 1$-form in a cohomology class is unique.
\end{conjecture}


%%%%%%%5





\printbibliography

\end{document}
