\documentclass[reqno,11pt]{amsart}
\usepackage[letterpaper, margin=1in]{geometry}
\RequirePackage{amsmath,amssymb,amsthm,graphicx,mathrsfs,url,slashed,subcaption}
\RequirePackage[usenames,dvipsnames]{xcolor}
\RequirePackage[colorlinks=true,linkcolor=Red,citecolor=Green]{hyperref}
\RequirePackage{amsxtra}
\usepackage{cancel}
\usepackage{tikz, quiver, wrapfig}
%\usepackage[T1]{fontenc}

% \setlength{\textheight}{9.3in} \setlength{\oddsidemargin}{-0.25in}
% \setlength{\evensidemargin}{-0.25in} \setlength{\textwidth}{7in}
% \setlength{\topmargin}{-0.25in} \setlength{\headheight}{0.18in}
% \setlength{\marginparwidth}{1.0in}
% \setlength{\abovedisplayskip}{0.2in}
% \setlength{\belowdisplayskip}{0.2in}
% \setlength{\parskip}{0.05in}
%\renewcommand{\baselinestretch}{1.05}

\title{The Katzourakis correction}
\author{Aidan Backus}
\address{Department of Mathematics, Brown University}
\email{aidan\_backus@brown.edu}
\date{\today}

\newcommand{\NN}{\mathbf{N}}
\newcommand{\ZZ}{\mathbf{Z}}
\newcommand{\QQ}{\mathbf{Q}}
\newcommand{\RR}{\mathbf{R}}
\newcommand{\CC}{\mathbf{C}}
\newcommand{\DD}{\mathbf{D}}
\newcommand{\PP}{\mathbf P}
\newcommand{\MM}{\mathbf M}
\newcommand{\II}{\mathbf I}
\newcommand{\Hyp}{\mathbf H}
\newcommand{\Sph}{\mathbf S}
\newcommand{\Group}{\mathbf G}
\newcommand{\GL}{\mathbf{GL}}
\newcommand{\Orth}{\mathbf{O}}
\newcommand{\SpOrth}{\mathbf{SO}}
\newcommand{\Ball}{\mathbf{B}}

\newcommand*\dif{\mathop{}\!\mathrm{d}}
\newcommand*\Dif{\mathop{}\!\mathrm{D}}

\DeclareMathOperator{\card}{card}
\DeclareMathOperator{\dist}{dist}
\DeclareMathOperator{\id}{id}
\DeclareMathOperator{\Hom}{Hom}
\DeclareMathOperator{\PD}{PD}
\DeclareMathOperator{\coker}{coker}
\DeclareMathOperator{\supp}{supp}
\DeclareMathOperator{\sech}{sech}
\DeclareMathOperator{\Teich}{Teich}
\DeclareMathOperator{\tr}{tr}

\newcommand{\Leaves}{\mathscr L}
\newcommand{\Lagrange}{\mathscr L}
\newcommand{\Hypspace}{\mathscr H}

\newcommand{\Chain}{\underline C}


\newcommand{\weakto}{\rightharpoonup}

\newcommand{\Two}{\mathrm{I\!I}}
\newcommand{\Ric}{\mathrm{Ric}}

\newcommand{\normal}{\mathbf n}
\newcommand{\radial}{\mathbf r}
\newcommand{\evect}{\mathbf e}
\newcommand{\vol}{\mathrm{vol}}

\newcommand{\diam}{\mathrm{diam}}
\newcommand{\Ell}{\mathrm{Ell}}
\newcommand{\inj}{\mathrm{inj}}
\newcommand{\Lip}{\mathrm{Lip}}
\newcommand{\MCL}{\mathrm{MCL}}
\newcommand{\Riem}{\mathrm{Riem}}

\newcommand{\Mass}{\mathbf M}
\newcommand{\Comass}{\mathbf L}

\newcommand{\Min}{\mathrm{Min}}
\newcommand{\Max}{\mathrm{Max}}

\newcommand{\dfn}[1]{\emph{#1}\index{#1}}

\renewcommand{\Re}{\operatorname{Re}}
\renewcommand{\Im}{\operatorname{Im}}

\newcommand{\loc}{\mathrm{loc}}
\newcommand{\cpt}{\mathrm{cpt}}

\DeclareMathOperator*{\essinf}{ess\,inf}
\DeclareMathOperator*{\esssup}{ess\,sup}

\def\Japan#1{\left \langle #1 \right \rangle}

\newtheorem{theorem}{Theorem}[section]
\newtheorem{badtheorem}[theorem]{``Theorem"}
\newtheorem{prop}[theorem]{Proposition}
\newtheorem{lemma}[theorem]{Lemma}
\newtheorem{sublemma}[theorem]{Sublemma}
\newtheorem{proposition}[theorem]{Proposition}
\newtheorem{corollary}[theorem]{Corollary}
\newtheorem{conjecture}[theorem]{Conjecture}
\newtheorem{axiom}[theorem]{Axiom}
\newtheorem{assumption}[theorem]{Assumption}

\newtheorem{mainthm}{Theorem}
\renewcommand{\themainthm}{\Alph{mainthm}}

\newtheorem{claim}{Claim}[theorem]
\renewcommand{\theclaim}{\thetheorem\Alph{claim}}
% \newtheorem*{claim}{Claim}

\theoremstyle{definition}
\newtheorem{definition}[theorem]{Definition}
\newtheorem{remark}[theorem]{Remark}
\newtheorem{warning}[theorem]{Warning}
\newtheorem{example}[theorem]{Example}
\newtheorem{notation}[theorem]{Notation}

\newtheorem{exercise}[theorem]{Discussion topic}
\newtheorem{homework}[theorem]{Homework}
\newtheorem{problem}[theorem]{Problem}

\makeatletter
\newcommand{\proofpart}[2]{%
  \par
  \addvspace{\medskipamount}%
  \noindent\emph{Part #1: #2.}
}
\makeatother



\numberwithin{equation}{section}


% Mean
\def\Xint#1{\mathchoice
{\XXint\displaystyle\textstyle{#1}}%
{\XXint\textstyle\scriptstyle{#1}}%
{\XXint\scriptstyle\scriptscriptstyle{#1}}%
{\XXint\scriptscriptstyle\scriptscriptstyle{#1}}%
\!\int}
\def\XXint#1#2#3{{\setbox0=\hbox{$#1{#2#3}{\int}$ }
\vcenter{\hbox{$#2#3$ }}\kern-.6\wd0}}
\def\ddashint{\Xint=}
\def\dashint{\Xint-}

\usepackage[backend=bibtex,style=alphabetic,giveninits=true]{biblatex}
\renewcommand*{\bibfont}{\normalfont\footnotesize}
\addbibresource{best_curl.bib}
\renewbibmacro{in:}{}
\DeclareFieldFormat{pages}{#1}

\newcommand\todo[1]{\textcolor{red}{TODO: #1}}


\begin{document}
\begin{abstract}
We derive the PDE for a tight form using the Katzourakis correction.
\end{abstract}

\maketitle

%%%%%%%%%%%%%%%%%%%%%%%%%%%%%%%%%%%%%%%%%%%%%%%%%%%%%%%
\section{On differential forms}
For $1 < p < \infty$, we say that $F$ is a \dfn{$p$-tight form} if
$$\begin{cases} 
    \dif F = 0 \\
    \dif^* (|F|^{p - 2} F) = 0.
\end{cases}$$
We rewrite this equation in nondivergence form:

\begin{lemma}
Suppose that $2 < p < \infty$, and let $F$ be a $C^1$ $p$-tight form.
Then
\begin{equation}\label{pMax nondivergence}
\begin{cases}
    \dif F = 0 \\
    \star F \wedge \dif(\star F) = 0 \\
    (p - 2) |F|^{p - 4} \langle \nabla F, F\rangle \wedge \star F + |F|^{p - 2} \dif \star F.
\end{cases}
\end{equation}
\end{lemma}
\begin{proof}
We first write 
\begin{align*} 
0 
&= \dif(|F|^{p - 2} \star F) \\
&= \dif(|F|^{p - 2}) \wedge \star F + |F|^{p - 2} \dif \star F.
\end{align*}
We first address the first term:
$$\dif(|F|^{p - 2}) \wedge \star F = (p - 2) |F|^{p - 4} \langle \nabla F, F\rangle \wedge \star F.$$
In order to take the Hodge star of this equation, we write $\iota(X, G)$ for the contraction of a vector field $X$ with a differential form $G$.
We get 
$$\star(\langle \nabla F, F\rangle \wedge \star F) = \iota(\langle \nabla F, F\rangle^\sharp, F).$$
It is more convenient to rewrite this in Einstein notation as 
$$\iota(\langle \nabla F, F\rangle^\sharp, F)_{\alpha_1 \cdots \alpha_{d - 2}} = \nabla_\beta F_{\gamma_1 \cdots \gamma_{d - 1}} F^{\gamma_1 \cdots \gamma_{d - 1}} {F^\beta}_{\alpha_1 \cdots \alpha_{d - 2}}.$$
Notice that when we raise the first index, we are thinking of $F$ as a linear map from the $d - 2$th tensor power $TM^{d - 2}$ of $TM$ to $TM$.
Let $\Pi$ be the projection onto the kernel of this map.
So by definition of $\Pi$,
$$\nabla_\beta F_{\gamma_1 \cdots \gamma_{d - 1}} F^{\gamma_1 \cdots \gamma_{d - 1}} {F^\beta}_{\delta_1 \cdots \delta_{d - 2}} {\Pi^{\delta_1 \cdots \delta_{d - 2}}}_{\alpha_1 \cdots \alpha_{d - 2}} = 0.$$
Returning to differential forms notation, we conclude that
$$\star(\dif(|F|^{p - 2}) \wedge \star F) \Pi = 0.$$
Taking $\star$ of 
$$\dif(|F|^{p - 2}) \wedge \star F + |F|^{p - 2} \dif \star F = 0$$
and then multiplying on the right by $\Pi$, we conclude 
\begin{equation}\label{incomplete Katzourakis correction}
|F|^{p - 2} \dif^* F \Pi = 0.
\end{equation}

In order to simplify this equation further, we must compute $\Pi$.
The value of $\Pi$ does not matter at $x \in M$ if $F(x) = 0$, so we may assume that $F$ has no zeroes.
Then, if $d = 2$, then $TM^{d - 2}$ is the trivial line bundle, and for any $x \in M$, $\Pi(x)$ has nontrivial kernel, hence is zero; thus the equation (\ref{incomplete Katzourakis correction}) is simply the equation $0 = 0$.
So we may assume that $d \geq 3$.

\begin{claim}\label{exterior splitting lemma}
Let $\Lambda^m V$ denote the $m$th exterior power of a finite-dimensional vector space $V$.
Let $X \in \Lambda^1 V$ and $v \in \Lambda^k V$.
Assume that $X \wedge v = 0$, $X \neq 0$, and $k \geq 1$.
Then there is a $k - 1$-form $\varphi$ such that $v = X \wedge \varphi$.
\end{claim}
\begin{proof}[Proof of claim]
Let $[n]^m$ be the set of increasing multiindices of length $m$ taking values in $\{1, \dots, n\}$.
Let $n := \dim V$.
Since $X$ is nonzero, we can find a basis $\{e^j: 1 \leq j \leq n\}$ of $V$ such that $X = e^n$.
Given $J := (j_1, \dots, j_m) \in [n]^m$, we write $e^J := \bigwedge_{i=1}^m e^{j_i}$.
Then $\{e^I: I \in [n]^m\}$ is a basis of $\Lambda^m V$, so there are coefficients $c_J \in [n]^k$ such that
$$v = \sum_{J \in [n]^k} c_J e^J = \sum_{i=1}^{n - 1} \sum_{I \in [n - 1]^{k - 1}} c_{In} e^{In} + \sum_{I \in [n - 1]^{k - 1}} c_{Ii} e^{Ii}.$$
Therefore 
$$0 = v \wedge X = \sum_{i=1}^{n - 1} \sum_{I \in [n - 1]^{k - 1}} c_{Ii} e^{Iin} + \sum_{I \in [n - 1]^{k - 1}} c_{In} e^{Inn} = \sum_{J \in [n - 1]^k} c_J e^{Jn}.$$
Now $\{e^{Jn}: J \in [n-1]^k\}$ is a subset of the basis $\{e^I: I \in [n]^{k + 1}\}$ of $\Lambda^{k + 1} V$, and therefore is linearly independent.
So $c_J = 0$ for every $J \in [n - 1]^k$; in other words, if $J \in [n]^k$ and $c_J \neq 0$, then there exists $I \in [n - 1]^{k - 1}$ such that $J = (I, n)$.
Now we set
$$\varphi := -\sum_{I \in [n - 1]^{k - 1}} c_{In} e^I,$$
so that 
\begin{align*} 
X \wedge \varphi &= \sum_{I \in [n - 1]^{k - 1}} c_{In} e^{In} = \sum_{J \in [n]^k} c_J e^J = v. \qedhere 
\end{align*}
\end{proof}

Let $X := \star F$.
Let $X \wedge \Omega^{d - 3}$ be the space of $d - 2$-forms of the form $X \wedge \varphi$ for some $\varphi \in \Omega^{d - 3}$.
In the next claim we identify $\Omega^{d - 2}$ with the antisymmetric subbundle of $TM^{d - 2}$, by raising indices.

\begin{claim}
$\Pi$ is the orthogonal projection
% https://q.uiver.app/#q=WzAsMixbMCwwLCJcXE9tZWdhXntkIC0gMn0iXSxbMSwwLCJYIFxcd2VkZ2UgXFxPbWVnYV57ZCAtIDN9Il0sWzAsMSwiIiwwLHsic3R5bGUiOnsiaGVhZCI6eyJuYW1lIjoiZXBpIn19fV1d
\[\begin{tikzcd}
	{\Omega^{d - 2}} & {X \wedge \Omega^{d - 3}}
	\arrow["\Pi", two heads, from=1-1, to=1-2]
\end{tikzcd}\]
\end{claim}
\begin{proof}[Proof of claim]
We first observe that $\dif^* F \Pi$ is a $d - 2$-form.
If $d = 3$, this is clear, since then $\Pi$ is a section of $\Hom(TM, TM)$ and $\dif^* F$ is a section of $\Hom(TM, \RR) = \Omega^1$.
Otherwise, $d \geq 4$, and it is easy to see that $\Pi$ preserves the splitting of $TM^{d - 2}$ into its symmetric and antisymmetric subbundles.
Moreover, $\dif^* F$ annihilates the symmetric subbundle.
So $\dif^* F \Pi$ annihilates the symmetric subbundle of $TM^{d - 2}$, as desired.
Henceforth, we think of $F$ (or perhaps better $F^\sharp$) as a linear map $\Omega^{d - 2} \to \Omega^1$.

Let $v$ be a $d - 2$-form.
Each of the below statements is equivalent to the statement immediately below it:
\begin{enumerate}
\item $v \in \ker F$.
\item For every $1$-form $\psi$, $\langle F, v \wedge \psi\rangle = 0$.
\item For every $1$-form $\psi$, $X \wedge v \wedge \psi = 0$.
\item $X \wedge v = 0$.
\item $v \in X \wedge \Omega^{d - 3}$.
\end{enumerate}
In the last equivalence we used Claim \ref{exterior splitting lemma}.
In summary, $\Pi$ (or perhaps better $\Pi^\flat$) is the orthogonal projection onto the space $X \wedge \Omega^{d - 3}$.
\end{proof}

\begin{claim}
We have
\begin{equation}\label{involutivity on X}
    X \wedge \dif X = 0.
\end{equation}
\end{claim}
\begin{proof}[Proof of claim]
Let $\varphi$ be a $d - 3$-form, so that $\Pi(X \wedge \varphi) = X \wedge \varphi$.
Therefore 
$$0 = |F|^{p - 2} \langle \dif^* F, \Pi(X \wedge \varphi)\rangle = |F|^{p - 2} \langle \dif^* F, X \wedge \varphi\rangle.$$
Since $F$ has no zeroes, we divide through by $|F|^{p - 2}$ to see that for every $\varphi$,
$$0 = \star \langle \star \dif X, X \wedge \varphi\rangle = \langle \dif X, \star(X \wedge \varphi)\rangle.$$
This can be rewritten more suggestively as 
$$0 = \langle \dif X, \iota(\varphi^\sharp, \star X)\rangle = \dif X \wedge \iota_{\varphi^\sharp} X.$$
Since $\varphi$ was arbitrary we conclude (\ref{involutivity on X}).
\end{proof}

From (\ref{involutivity on X}) we conclude the second equation in (\ref{pMax nondivergence}), which was the only equation that we still had to show.
\end{proof}

\begin{proposition}
For each $2 < p < \infty$, let $F_p$ be a $p$-tight form, such that $F_p \to F$ in $C^1$.
Then
$$\begin{cases}
    \dif F = 0 \\
    \star F \wedge \dif(\star F) = 0 \\
    \langle \nabla F, F\rangle \wedge \star F = 0.
\end{cases}$$
\end{proposition}
\begin{proof}
We renormalize the third equation in (\ref{pMax nondivergence}) by dividing through by $(p - 2) |F|^{p - 4}$ (and observing that the equation holds vacuously whenever $F = 0$) to obtain 
$$\langle \nabla F_p, F_p\rangle \wedge \star F_p + \frac{|F_p|^2}{p - 2} \dif \star F_p = 0.$$
Since $F_p \to F$ in $C^1$, $|F_p|^2 \cdot |\dif \star F_p|$ is bounded.
Therefore the result follows by taking $p \to \infty$.
\end{proof}

It follows from the above derivation that we can rewrite the third equation in Einstein notation as
$$\nabla_\alpha F_{\beta_1 \cdots \beta_{d - 1}} F^{\beta_1 \cdots \beta_{d - 1}} {F^\alpha}_{\gamma_1 \cdots \gamma_{d - 2}} = 0$$
and it is sometimes convenient to use this formulation.

\begin{corollary}
Suppose that the $p$-tight forms $F_p$ converge in $C^1$ to a tight form $F$.
Then $F$ absolutely minimizes its comass.
\end{corollary}
\begin{proof}
Since $\star F \wedge \dif(\star F) = 0$, the kernel bundle of $\star F$ is involutive on the set $\{|F| > 0\}$.
\end{proof}


% \section{On Sheffield--Smart maps}
% Let $u$ be a Schatten $p$-harmonic map, and let $\Dif(u)$ be the connection on the pullback bundle $u^*(TN)$.
% Let $Q(A) := \sqrt{AA^\dagger}$ for any linear map $A$.
% Then 
% $$\Dif(u) (Q(\dif u)^{p - 2} \star \dif u) = 0.$$
% We write $\Dif$ and $Q$ instead of $\Dif(u)$ and $Q(\dif u)$ sometimes.

% To write the $p$-Laplacian in nondivergence form, we first compute 
% \begin{align*}
% \Dif(Q^{p - 2})
% &= \Dif((Q^2)^{\frac{p - 2}{2}}) \\
% &= \frac{p - 2}{2} Q^{p - 4} \Dif(\dif u \dif u^\dagger) \\
% &= \frac{p - 2}{2} Q^{p - 4} \Dif \dif u \cdot \dif u^\dagger + \frac{p - 2}{2} Q^{p - 4} \dif u \cdot \Dif \dif u^\dagger\\
% &= (p - 2) Q^{p - 4} \Dif \dif u \cdot \dif u^\dagger.
% \end{align*}
% As a sanity check, we observe that $\Dif \dif u \cdot \dif u^\dagger$ is a $\Hom(u^*(TN), u^*(TN))$-valued $1$-form, so it can be multiplied with $Q^{p - 4}$ to obtain a $\Hom(u^*(TN), u^*(TN))$-valued $1$-form.
% In particular, it can be wedged with $\star \dif u$, a $u^*(TN)$-valued $d - 1$-form, to obtain a $u^*(TN)$-valued $d$-form.
% Carrying out this computation,
% \begin{align*}
% 0 
% &= \Dif(Q^{p - 2} \star \dif u) \\
% &= (p - 2) Q^{p - 4} \Dif \dif u \cdot \dif u^\dagger \wedge \star \dif u - Q^{p - 2} \Dif \star \dif u.
% \end{align*}
% Thus both terms are $u^*(TN)$-valued $d$-forms.

% Let $\sigma_i$ denote the $i$th singular value of $\dif u$, so that $\sigma_i \geq \sigma_{i + 1}$.
% We assume that there exists $\theta < 1$ independent of $x$, such that $\sigma_2 \leq \theta \sigma_1$.
% Let $\xi$ be the first singular vector of $\dif u$ on the target space.
% Let $\Pi$ be the projection to the orthogonal complement of $\xi$.
% Then (where Roman letters refer to the domain indices and Greek letters refer to the target indices)
% $${(\dif u)_i}^\alpha = \xi^\alpha \xi_\beta {(\dif u)_i}^\beta + {\Pi^\alpha}_\beta {(\dif u)_i}^\beta$$
% which we summarize as 
% $$\dif u = \xi \langle \xi, \dif u\rangle + \Pi \dif u.$$
% We then expand
% \begin{align*} 
% \Dif \dif u \cdot \dif u^\dagger \wedge \star \dif u 
% &= \Dif \dif u \cdot \dif u^\dagger \wedge \star \xi \langle \xi, \dif u\rangle + \Dif \dif u \cdot \dif u^\dagger \wedge \star \Pi \dif u \\
% &= \Dif \dif u \cdot \langle \dif u^\dagger, \xi\rangle \wedge \star \langle \xi, \dif u\rangle + \Dif \dif u \cdot \dif u^\dagger \wedge \star \Pi \dif u
% \end{align*}
% To rewrite the first term, let $X$ be the first singular vector of $\dif u$ on the domain, thus
% $$\langle \dif u, X\rangle = \sigma_1 \xi.$$
% Then $\langle \dif u^\dagger, \xi\rangle = \sigma_1 X$, and we have 
% $$\Dif \dif u \cdot \langle \dif u^\dagger, \xi\rangle \wedge \star \langle \xi, \dif u\rangle = \sigma_1 \Dif \Dif_X u \wedge \star \langle \xi, \dif u\rangle.$$
% Since $Q$ preserves the kernel and cokernel of $\Pi$, we can split
% $$Q^{p - 4} \eta = \sigma_1^{p - 4} \xi \langle \xi, \eta\rangle + Q^{p - 4} \Pi \eta.$$
% Therefore
% \begin{align*} 
% &Q^{p - 4} \Dif \dif u \cdot \dif u^\dagger \wedge \star \dif u \\
% &\qquad = \sigma_1^{p - 3} \xi \langle \xi, \Dif \Dif_X u\rangle \wedge \star \langle \xi, \dif u\rangle + \sigma_1 Q^{p - 4} \Pi \Dif \Dif_X u \wedge \star \langle \xi, \dif u\rangle  \\
% &\qquad \qquad + \sigma_1^{p - 4} \xi \langle \xi, \Dif \dif u \cdot \dif u^\dagger\rangle \wedge \star \Pi \dif u + Q^{p - 4} \Pi \Dif \dif u \cdot \dif u^\dagger \wedge \star \Pi \dif u \\
% &\qquad =: \mathbf I + \mathbf{II} + \mathbf{III} + \mathbf{IV}.
% \end{align*}
% Since $|Q^{p - 4} \Pi| \leq \theta^{p - 4} \sigma_1^{p - 4}$ and $|\dif u| \leq \sigma_1$,
% $$|\mathbf{II}| + |\mathbf{IV}| \leq 2\sigma_1^{p - 3} \theta^{p - 4} |\Dif \dif u|$$
% hence if we assume that $\sigma_1 |\Dif \dif u|$ is bounded,
% $$Q^{p - 4} \Dif \dif u \cdot \dif u^\dagger \wedge \star \dif u = \sigma_1^{p - 4}(\sigma_1 \xi \langle \xi, \Dif \Dif_X u\rangle \wedge \star \langle \xi, \dif u\rangle + \xi \langle \xi, \Dif \dif u \cdot \dif u^\dagger\rangle \wedge \star \Pi \dif u + O(\theta^{p - 4})).$$


\printbibliography

\end{document}
