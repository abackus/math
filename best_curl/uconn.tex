\documentclass[reqno,11pt]{amsart}
\usepackage[letterpaper, margin=1in]{geometry}
\RequirePackage{amsmath,amssymb,amsthm,graphicx,mathrsfs,url,slashed,subcaption}
\RequirePackage[usenames,dvipsnames]{xcolor}
\RequirePackage[colorlinks=true,linkcolor=Red,citecolor=Green]{hyperref}
\RequirePackage{amsxtra}
\usepackage{cancel}
\usepackage{tikz, quiver, wrapfig}
%\usepackage[T1]{fontenc}

% \setlength{\textheight}{9.3in} \setlength{\oddsidemargin}{-0.25in}
% \setlength{\evensidemargin}{-0.25in} \setlength{\textwidth}{7in}
% \setlength{\topmargin}{-0.25in} \setlength{\headheight}{0.18in}
% \setlength{\marginparwidth}{1.0in}
% \setlength{\abovedisplayskip}{0.2in}
% \setlength{\belowdisplayskip}{0.2in}
% \setlength{\parskip}{0.05in}
%\renewcommand{\baselinestretch}{1.05}

\title{The $p$-Laplacian, minimal laminations, and the max flow/min cut theorem}
\author{Aidan Backus}
\address{Department of Mathematics, Brown University}
\email{aidan\_backus@brown.edu}
\date{\today}

\newcommand{\NN}{\mathbf{N}}
\newcommand{\ZZ}{\mathbf{Z}}
\newcommand{\QQ}{\mathbf{Q}}
\newcommand{\RR}{\mathbf{R}}
\newcommand{\CC}{\mathbf{C}}
\newcommand{\DD}{\mathbf{D}}
\newcommand{\PP}{\mathbf P}
\newcommand{\MM}{\mathbf M}
\newcommand{\II}{\mathbf I}
\newcommand{\Hyp}{\mathbf H}
\newcommand{\Sph}{\mathbf S}
\newcommand{\Group}{\mathbf G}
\newcommand{\GL}{\mathbf{GL}}
\newcommand{\Orth}{\mathbf{O}}
\newcommand{\SpOrth}{\mathbf{SO}}
\newcommand{\Ball}{\mathbf{B}}

\newcommand*\dif{\mathop{}\!\mathrm{d}}

\DeclareMathOperator{\card}{card}
\DeclareMathOperator{\dist}{dist}
\DeclareMathOperator{\id}{id}
\DeclareMathOperator{\Hom}{Hom}
\DeclareMathOperator{\coker}{coker}
\DeclareMathOperator{\supp}{supp}
\DeclareMathOperator{\Teich}{Teich}
\DeclareMathOperator{\tr}{tr}

\newcommand{\Leaves}{\mathscr L}
\newcommand{\Lagrange}{\mathscr L}
\newcommand{\Hypspace}{\mathscr H}

\newcommand{\Chain}{\underline C}

\newcommand{\Two}{\mathrm{I\!I}}
\newcommand{\Ric}{\mathrm{Ric}}

\newcommand{\normal}{\mathbf n}
\newcommand{\radial}{\mathbf r}
\newcommand{\evect}{\mathbf e}
\newcommand{\vol}{\mathrm{vol}}

\newcommand{\diam}{\mathrm{diam}}
\newcommand{\Ell}{\mathrm{Ell}}
\newcommand{\inj}{\mathrm{inj}}
\newcommand{\Lip}{\mathrm{Lip}}
\newcommand{\MCL}{\mathrm{MCL}}
\newcommand{\Riem}{\mathrm{Riem}}

\newcommand{\Mass}{\mathbf M}
\newcommand{\Comass}{\mathbf L}

\newcommand{\Min}{\mathrm{Min}}
\newcommand{\Max}{\mathrm{Max}}

\newcommand{\dfn}[1]{\emph{#1}\index{#1}}

\renewcommand{\Re}{\operatorname{Re}}
\renewcommand{\Im}{\operatorname{Im}}

\newcommand{\loc}{\mathrm{loc}}
\newcommand{\cpt}{\mathrm{cpt}}

\def\Japan#1{\left \langle #1 \right \rangle}

\newtheorem{theorem}{Theorem}[section]
\newtheorem{badtheorem}[theorem]{``Theorem"}
\newtheorem{prop}[theorem]{Proposition}
\newtheorem{lemma}[theorem]{Lemma}
\newtheorem{sublemma}[theorem]{Sublemma}
\newtheorem{proposition}[theorem]{Proposition}
\newtheorem{corollary}[theorem]{Corollary}
\newtheorem{conjecture}[theorem]{Conjecture}
\newtheorem{axiom}[theorem]{Axiom}
\newtheorem{assumption}[theorem]{Assumption}

\newtheorem{mainthm}{Theorem}
\renewcommand{\themainthm}{\Alph{mainthm}}

\newtheorem{claim}{Claim}[theorem]
\renewcommand{\theclaim}{\thetheorem\Alph{claim}}
% \newtheorem*{claim}{Claim}

\theoremstyle{definition}
\newtheorem{definition}[theorem]{Definition}
\newtheorem{remark}[theorem]{Remark}
\newtheorem{example}[theorem]{Example}
\newtheorem{notation}[theorem]{Notation}

\newtheorem{exercise}[theorem]{Discussion topic}
\newtheorem{homework}[theorem]{Homework}
\newtheorem{problem}[theorem]{Problem}

\makeatletter
\newcommand{\proofpart}[2]{%
  \par
  \addvspace{\medskipamount}%
  \noindent\emph{Part #1: #2.}
}
\makeatother



\numberwithin{equation}{section}


% Mean
\def\Xint#1{\mathchoice
{\XXint\displaystyle\textstyle{#1}}%
{\XXint\textstyle\scriptstyle{#1}}%
{\XXint\scriptstyle\scriptscriptstyle{#1}}%
{\XXint\scriptscriptstyle\scriptscriptstyle{#1}}%
\!\int}
\def\XXint#1#2#3{{\setbox0=\hbox{$#1{#2#3}{\int}$ }
\vcenter{\hbox{$#2#3$ }}\kern-.6\wd0}}
\def\ddashint{\Xint=}
\def\dashint{\Xint-}

\usepackage[backend=bibtex,style=alphabetic,giveninits=true]{biblatex}
\renewcommand*{\bibfont}{\normalfont\footnotesize}
\addbibresource{best_curl.bib}
\renewbibmacro{in:}{}
\DeclareFieldFormat{pages}{#1}

\newcommand\todo[1]{\textcolor{red}{TODO: #1}}


\begin{document}
\begin{abstract}
Motivated by an application to Teichm\"uller theory, we investigate the limit of the $p$-Laplacian and its convex dual problem as $p \to 1$.
On one side of the duality we get Lipschitz laminations of minimal hypersurfaces; on the other side, we get calibrations which solve a generalization of the $\infty$-Laplacian.
The laminations act as a ``bottleneck'' on the calibrations, giving a continuous analogue of the max flow/min cut theorem on closed manifolds which was predicted by Thurston.
\end{abstract}

\maketitle

%%%%%%%%%%%%%%%%%%%%%%%%%%%%%%%%%%%%%%%%%%%%%%%%%%%%%%
\section{Functions of least gradient and minimal laminations}
Thanks to seminar organizers etc; talk appeals to geometers and PDE; other niceties.
Going to introduce objects from PDE, Teichm\"uller theory, and graph theory, and I'd rather not lose people even if I only get through half the talk; so stop me if one of the definitions doesn't make much sense to you.

\begin{definition}
A function $u: U \to \RR$ is \dfn{$p$-harmonic}, $1 \leq p < \infty$, if for any $v$ with $v|_{\partial U} = u|_{\partial U}$,
$$\frac{1}{p} \int_U |\dif u|^p \dif V \leq \frac{1}{p} \int_U |\dif v|^p \dif V.$$
A function is said to have \dfn{least gradient} if it is $1$-harmonic.
\end{definition}

% The Euler-Lagrange equation is 
% $$\nabla \cdot (|\nabla u|^{p - 2} \nabla u) = 0$$
% though for $p = 1$ one needs to be careful about what this means since $\nabla u/|\nabla u|$ makes no sense at critical points of $u$ \cite{Mazon14}.
% We will not elaborate more on this point here.

Functions of least gradient are motivated by the coarea formula 
$$\int_U |\dif u| \dif V = \int_{-\infty}^\infty |\partial \{u > y\} \cap U| \dif y.$$
In particular, the superlevel sets $\{u > y\}$ must minimizer their perimeter \cite{BOMBIERI1969}.
In fact they were crucial in the codimension-$1$ regularity theory of minimal submanifolds in the 1960s \cite{BOMBIERI1969,Miranda66}.

Note carefully: $\int_U |\dif u| \dif V < \infty$ does not mean that $\dif u \in L^1(U)$, but only that $|\dif u| \dif V$ is a finite Borel measure (so $u \in BV(U)$).
Indeed, $W^{1, 1}(U)$ is not weakly locally compact, so we cannot find minimizers in that space.
Our minimizers will have jumps and flat parts, and can even be indicator functions!
But this is OK, it means that $\int_U |\dif u| \dif V$ does not penalizes jumps or near-jumps which will be important for us.
(This turns out to also be important in image denoising.)

\begin{definition}
A (Lipschitz) \dfn{lamination} in a manifold $M$ is a closed set $\lambda \subseteq M$ equipped with charts in which $\lambda$ becomes $K \times (-1, 1)^{d - 1}$ with $K \subseteq \RR$ closed (and the transition maps are Lipschitz).
The fibers, locally of the form $\{k\} \times (-1, 1)^d$, are called \dfn{leaves}.
\end{definition}

Obviously a foliation is a lamination, as is a single closed leaf.
A much more interesting case is when the local leaf space $K$ is a Cantor set.
Then we get a family of disjoint hypersurfaces, parametrized by a Cantor set.

\begin{definition}
The lamination $\lambda$ is \dfn{geodesic} or \dfn{minimal} if every leaf is.
A \dfn{transverse measure} to a lamination $\lambda$ consists of Radon measures on each of the local leaf spaces $K$, which are compatible.
\end{definition}

If $\lambda$ is oriented and $\mu$ is a transverse measure, we get a $d - 1$-current, the \dfn{Ruelle-Sullivan current},
$$\varphi \mapsto \sum_\alpha \int_{K_\alpha} \int_{\{k\} \times (-1, 1)^{d - 1}} \chi_\alpha \varphi \dif \mu_\alpha(k)$$
where $(\chi_\alpha)$ is a partition of unity.
The compatibility condition means that the choice of $(\chi_\alpha)$ is irrelevant.

Geodesic laminations are crucial in Teichm\"uller theory, one is interested in how they deform as one deforms a Riemann surface $M$ (which we always assume to have genus $\geq 2$).
While investigating this, DU noticed that certain geodesic laminations always arise as the level sets of a function of least gradient $u$ on $\Hyp^2$ and you get a transverse measure from $\dif u$ \cite{daskalopoulos2020transverse}.
Once I saw this theorem I knew the following had to be true:

\begin{theorem}\label{main thm 1}
Let $M$ be a Riemannian manifold of dimension $2 \leq d \leq 7$.
\begin{enumerate}
\item Given a function $u$ of locally least gradient on $M$ there exists a measured oriented Lipschitz minimal lamination $\lambda$ such that:
\begin{enumerate}
\item The level sets of $u$ are exactly the leaves of $\lambda$.
\item $\dif u$ is the Ruelle-Sullivan current of a transverse measure on $\lambda$.
\end{enumerate}
\item Suppose that $H^1(M, \RR) = 0$. Given a measured oriented minimal lamination $\lambda$, if it is possible to cover $M$ by open sets $U$ in which every leaf $N$ satisfies 
$$\|\Two_N\|_{C^0(N \cap U)} \leq C(U) < \infty,$$
then $\lambda$ arises from a function of locally least gradient as above.
\end{enumerate}
\end{theorem}

That the level sets were disjoint (but no Lipschitz structure) was known \cite{Auer12}.
A useful immediate consequence is that (by passing to the Lipschitz coordinates) $u$ is basically just a glorified $BV$ function in 1D, which has some applications.
% So we can use the theory of $BV(\RR)$ to show:

% \begin{corollary}[{\cite{HakkarainenKorteLahtiShanmugalingam+2015, górny2017planar}}]
% Suppose $H^1(U, \RR) = 0$ and $u: U \to \RR$ has locally least gradient.
% Then $u = u_{ac} + u_C + u_j$ where $u_{ac}$ is continuous and absolutely continuous, $u_C$ is continuous and Cantor, $u_j$ is jump, and they all have locally least gradient.
% \end{corollary}

\begin{lemma}\label{main thm 2}
Let $\mathscr T$ be a set of disjoint minimal hypersurfaces such that $\bigcup \mathscr T$ is closed, and for some $L > 0$ and every $N \in \mathscr T$,
$$\|\Two_N\|_{C^0} \leq L.$$
Then $\mathscr T$ is the set of leaves of a Lipschitz lamination.
\end{lemma}
% \begin{proof}
% Working at length scales $\ll L^{-1/2}$ and using the exponential map, these hypersurfaces are basically hyperplanes which are very close to parallel (this is made rigorous by Schauder and Harnack inequalities).
% We can then deform slightly the normal coordinates to make them exactly parallel.
% \end{proof}
 
\begin{proof}[Proof of Theorem \ref{main thm 1}]
From the regularity theorem, the level sets are smooth and disjoint.
(A junction point of the level sets would give a singularity of the hypersurfaces -- not allowed for the Dirichlet problem in codimension $1$!)
The level sets are absolutely area-minimizing so they are stable, and their area in a ball $B_\varepsilon$ makes the monotonicity formula sharp: it is $\sim \varepsilon^{d - 1}$.
A stable minimal hypersurface for which the monotonicity formula is sharp has bounded curvature \cite{Schoen81} so we can use Lemma \ref{main thm 2}.
\end{proof}

%%%%%%%%%%%%%%%%%%%%%
\section{Convex duality}
\begin{definition}
A \dfn{calibration} is a closed $d - 1$-form $F$ such that $\|F\|_{L^\infty} = 1$.
A $F$-\dfn{calibrated hypersurface} is a hypersurface $N$ such that $F$ restricts to the area form on $N$.
\end{definition}

Calibrated hypersurfaces are minimal.
We can show that ``$F$-calibrated hypersurface'' makes sense for any \emph{measurable} calibration $F$ by working with continuous vector potentials $A$, $F = \dif A$.

We are interested in the $\infty$-Laplacian
$$\nabla_i \partial_j v \partial^i v \partial^j v = 0$$
for a nonconstant function $v$ on $M$ (note that the natural space for $v$ to live in is Lip).
Using the good convergence properties of viscosity solutions, we can regularize this PDE to get the $p$-Laplacian, which is easy to solve for such a map using variational methods, then take the limit as $p \to \infty$.
Possibly by rescaling $M$ we can normalize everything so that the Lipschitz constant of $v$ is $1$, in other words $\dif v$ is a (measurable) calibration.

\begin{theorem}[{\cite{daskalopoulos2020transverse}}]
Let $v$ be an $\infty$-harmonic function such that $\dif v$ is a calibration, and let $\lambda$ be the set of points such that $|\dif v| = 1$.
Then $\lambda$ is a lamination whose leaves are $\dif v$-calibrated, in particular $\lambda$ is a geodesic lamination.
\end{theorem}

DU studied the geodesic laminations coming from the $\infty$-Laplacian.
They realized the transverse measures as coming from a least gradient function they found by inspection.

I wanted to explain the existence of the least gradient function.
Recall that for $1/p + 1/q = 1$, the convex dual of the $q$-Laplacian is the problem of minimizing $1/p \int_M |F|^p \dif V$.
This means that it is possible to choose constraints for the dual problem so that there is a unique solution $F$ with 
$$\frac{1}{q} \int_M |\dif u|^q \dif V + \frac{1}{p} \int_M |F|^p \dif V = \int_M \dif u \wedge F.$$
If $d = 2$, then we can write $F = \dif v$ and see that $v$ solves the $p$-Laplacian.
Taking the limit we see that the convex dual problem to the least gradient problem if $d = 2$ is the $\infty$-Laplacian.
In general, one can show that the limiting dual problem is 
\begin{equation}\label{tight equation}
\begin{cases}
    (\nabla_i F_{j_1 \cdots j_{d - 1}}) F^{j_1 \cdots j_{d - 1}} {F^i}_{k_1 \cdots k_{d - 2}} = 0 \\
    \dif F = 0.
\end{cases}
\end{equation}

Let's digress to talk about a problem purely in the realm of PDE.
The system of PDE (\ref{tight equation}) is a model example in the calculus of variations in $L^\infty$.
There's just one problem: that theory doesn't actually exist yet. 
The natural space for $F$ to live in is $L^\infty$, and so we want to say that $F$ is a viscosity solution of (\ref{tight equation}), but viscosity solutions are defined by weaponizing the maximum principle, which makes no sense for systems.

\begin{problem}
What is the natural generalization of viscosity solutions to systems of PDE from the calculus of variations in $L^\infty$?
\end{problem}

Some work was done in this direction by \cite{Katzourakis2018OnAV} but it remains very speculative.

For now, we proceed by taking a $p$-regularization of (\ref{tight equation}).
1D reps of $\pi_1(M)$ are parametrized by $H^1(M, \RR)$, so instead of fixing a rep we're going to fix a cohomology class.

\begin{definition}
A closed $d - 1$-form $F$ is a \dfn{variational solution} of (\ref{tight equation}), or a \dfn{tight} form, if there are minimizers $F_p$ of $\int_M |F_p|^p \dif V$ subject to the same boundary condition as $F$, which converge as $p \to \infty$, weakly in $L^r$ for every $1 < r < \infty$ to $F$.
\end{definition}

Let $M$ be a closed Riemannian manifold and $\rho \in H^{d - 1}(M, \RR)$.
By solving the $p$-regularized problem and its dual, and taking limits, we see that we can find a tight representative of any class $\rho$, and it comes with a dual $1$-form $\dif u$, which locally integrates to a function of least gradient.
They satisfy the duality relation 
\begin{equation}\label{duality}
\dif u \wedge F = |\dif u| \|F\|_{L^\infty} \dif V.
\end{equation}
We always normalize so that $\|F\|_{L^\infty} = 1$, in other words $F$ is a calibration.

\section{Max flow min cut}
Let $d \leq 7$.
Let $\kappa$ be the lamination with Ruelle-Sullivan current $\dif u$.
Then every leaf of $\kappa$ is calibrated by $F$, according to the duality relation (\ref{duality}).
In fact, on $\kappa$, it can be rewritten using $\dif u = \normal^\flat |\dif u|$
$$F = \iota_\normal \dif V$$
where $\normal$ is the normal vector field to $\kappa$.

This gives us a \emph{geometric} interpretation of the duality relation.
We can picture $F$ by the divergence-free vector field $(\star F)^\sharp$, and for simplicity let's assume that $\kappa$ is a single closed hypersurface.
Then $F$ gets bottlenecked by the hypersurface $N$.
We can draw a picture of this.

Recall the max flow/min cut theorem from graph theory: for any maximal flow $f$ there is a minimal cut $C$ such that for any other cut $C'$,
$$1 = \frac{\sum_{e \in C} f(e)}{|C|} \geq \frac{\sum_{e \in C'} f(e)}{|C'|}.$$

In our case, we might not be able to find any hypersurfaces in the homology class of $\kappa$, but we can find a lamination, and it plays the role of the minimal cut.
Let $T_\lambda$ be the Ruelle-Sullivan current of a measured oriented lamination.

\begin{theorem}[max flow/min cut]
Let $F$ be a tight calibration.
Then there is a measured oriented $F$-calibrated lamination $\kappa$ such that for any measured oriented lamination $\lambda$,
$$1 = \frac{\int_M T_\kappa \wedge F}{\int_M |T_\kappa| \dif V} \geq \frac{\int_M T_\lambda \wedge F}{\int_M |T_\lambda| \dif V}.$$
\end{theorem}
\begin{proof}
We can realize $\kappa$ as coming from a function of least gradient on $\tilde M$, which then is dual to a tight calibration $F$, so $\kappa$ is $F$-calibrated by (\ref{duality}).
This gives the first equality.
The second inequality follows from the fact that, since $F$ is a calibration, $|T_\lambda \wedge F| \leq |T_\lambda|$, so the right-hand side is $\leq 1$.
\end{proof}

In the context of Teichm\"uller theory, the interpretation of geodesic laminations as minimal cuts was due to Thurston \cite{Thurston98}, and he also conjectured that you could prove this using convex duality.




\printbibliography

\end{document}
