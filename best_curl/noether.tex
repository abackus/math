\documentclass[reqno,11pt]{amsart}
\usepackage[letterpaper, margin=1in]{geometry}
\RequirePackage{amsmath,amssymb,amsthm,graphicx,mathrsfs,url,slashed,subcaption}
\RequirePackage[usenames,dvipsnames]{xcolor}
\RequirePackage[colorlinks=true,linkcolor=Red,citecolor=Green]{hyperref}
\RequirePackage{amsxtra}
\usepackage{cancel, longtable}
\usepackage{wrapfig, quiver}

% Add the 2020 MSC
\makeatletter
\@namedef{subjclassname@2020}{\textup{2020} Mathematics Subject Classification}
\makeatother

%\usepackage[T1]{fontenc}

% \setlength{\textheight}{9.3in} \setlength{\oddsidemargin}{-0.25in}
% \setlength{\evensidemargin}{-0.25in} \setlength{\textwidth}{7in}
% \setlength{\topmargin}{-0.25in} \setlength{\headheight}{0.18in}
% \setlength{\marginparwidth}{1.0in}
% \setlength{\abovedisplayskip}{0.2in}
% \setlength{\belowdisplayskip}{0.2in}
% \setlength{\parskip}{0.05in}
%\renewcommand{\baselinestretch}{1.05}

\title{Convex duality as a conservation law}
\author{Aidan Backus}
\address{}
\email{}
\date{\today}
\keywords{}
\subjclass[2020]{primary: }

\newcommand{\NN}{\mathbf{N}}
\newcommand{\ZZ}{\mathbf{Z}}
\newcommand{\QQ}{\mathbf{Q}}
\newcommand{\RR}{\mathbf{R}}
\newcommand{\CC}{\mathbf{C}}
\newcommand{\DD}{\mathbf{D}}
\newcommand{\PP}{\mathbf P}
\newcommand{\MM}{\mathbf M}
\newcommand{\II}{\mathbf I}
\newcommand{\Hyp}{\mathbf H}
\newcommand{\Sph}{\mathbf S}
\newcommand{\Torus}{\mathbf T}
\newcommand{\Group}{\mathbf G}
\newcommand{\GL}{\mathbf{GL}}
\newcommand{\Orth}{\mathbf{O}}
\newcommand{\SpOrth}{\mathbf{SO}}
\newcommand{\Ball}{\mathbf{B}}

\newcommand*\dif{\mathop{}\!\mathrm{d}}

\DeclareMathOperator{\card}{card}
\DeclareMathOperator{\dist}{dist}
\DeclareMathOperator{\id}{id}
\DeclareMathOperator{\len}{len}
\DeclareMathOperator{\Lex}{Lex}
\DeclareMathOperator{\Hom}{Hom}
\DeclareMathOperator{\mesh}{mesh}
\DeclareMathOperator{\coker}{coker}
\DeclareMathOperator{\supp}{supp}
\DeclareMathOperator{\tr}{tr}

\newcommand{\Two}{\mathrm{I\!I}}
\newcommand{\weakto}{\rightharpoonup}

\newcommand{\normal}{\mathbf n}
\newcommand{\vol}{\mathrm{vol}}

\DeclareMathOperator{\hull}{hull}
\newcommand{\diam}{\mathrm{diam}}
\DeclareMathOperator{\sech}{sech}
\newcommand{\inj}{\mathrm{inj}}
\newcommand{\Lip}{\mathrm{Lip}}
\newcommand{\Riem}{\mathrm{Riem}}

\newcommand{\Lagrange}{\mathscr L}

\newcommand{\Bl}{\mathrm{Bl}}
\newcommand{\Comass}{\mathbf L}
\newcommand{\Mass}{\mathbf M}

\DeclareMathOperator*{\essinf}{ess\,inf}
\DeclareMathOperator*{\esssup}{ess\,sup}

\newcommand{\dfn}[1]{\emph{#1}\index{#1}}

\renewcommand{\Re}{\operatorname{Re}}
\renewcommand{\Im}{\operatorname{Im}}

\newcommand{\loc}{\mathrm{loc}}
\newcommand{\cpt}{\mathrm{cpt}}

\def\Japan#1{\left \langle #1 \right \rangle}

\newtheorem{theorem}{Theorem}[section]
\newtheorem{badtheorem}[theorem]{``Theorem"}
\newtheorem{prop}[theorem]{Proposition}
\newtheorem{lemma}[theorem]{Lemma}
\newtheorem{sublemma}[theorem]{Sublemma}
\newtheorem{invariant}[theorem]{Invariant}
\newtheorem{proposition}[theorem]{Proposition}
\newtheorem{corollary}[theorem]{Corollary}
\newtheorem{conjecture}[theorem]{Conjecture}
\newtheorem{axiom}[theorem]{Axiom}
\newtheorem{assumption}[theorem]{Assumption}

\newtheorem{mainthm}{Theorem}
\renewcommand{\themainthm}{\Alph{mainthm}}

%\newtheorem{mainthm}{Theorem}
%\renewcommand{\themainthm}{}

\newtheorem{claim}{Claim}[theorem]
\renewcommand{\theclaim}{\thetheorem\Alph{claim}}
% \newtheorem*{claim}{Claim}

\theoremstyle{definition}
\newtheorem{definition}[theorem]{Definition}
\newtheorem{remark}[theorem]{Remark}
\newtheorem{example}[theorem]{Example}
\newtheorem{notation}[theorem]{Notation}

\newtheorem{exercise}[theorem]{Discussion topic}
\newtheorem{homework}[theorem]{Homework}
\newtheorem{problem}[theorem]{Problem}

\makeatletter
\newcommand{\proofpart}[2]{%
  \par
  \addvspace{\medskipamount}%
  \noindent\emph{Part #1: #2.}
}
\makeatother



\numberwithin{equation}{section}


% Mean
\def\Xint#1{\mathchoice
{\XXint\displaystyle\textstyle{#1}}%
{\XXint\textstyle\scriptstyle{#1}}%
{\XXint\scriptstyle\scriptscriptstyle{#1}}%
{\XXint\scriptscriptstyle\scriptscriptstyle{#1}}%
\!\int}
\def\XXint#1#2#3{{\setbox0=\hbox{$#1{#2#3}{\int}$ }
\vcenter{\hbox{$#2#3$ }}\kern-.6\wd0}}
\def\ddashint{\Xint=}
\def\dashint{\Xint-}

\usepackage[backend=bibtex,style=alphabetic,giveninits=true]{biblatex}
\renewcommand*{\bibfont}{\normalfont\footnotesize}
\addbibresource{best_curl.bib}
\renewbibmacro{in:}{}
\DeclareFieldFormat{pages}{#1}

\newcommand\todo[1]{\textcolor{red}{TODO: #1}}


\begin{document}
\begin{abstract}
    
\end{abstract}

\maketitle

%%%%%%%%%%%%%%%%%%%%%%%%%%%%%%%%%%%%%%%%%%%%%%%%%%%%%%%
\section{Introduction}
One of the fundamental theorems of the calculus of variations, and indeed of differential geometry and mathematical physics, is \dfn{Noether's theorem}: to every symmetry of a calculus of variations problem is associated a conservation law.
It is well-known that we can interpret invariance under symmetries of the domain as the classical conservation laws of physics: invariance under temporal translation corresponds to conservation of energy, invariance under spatial translation corresponds to conservation of momentum, and invariance under spatial rotation corresponds to conservation of angular momentum.

Let us give an interpretation of the conservation laws associated to invariance of symmetries of the target.
Roughly speaking, our main theorem is as follows:

\begin{mainthm}
Let $M$ be a Riemannian manifold and $N$ a Riemannian locally symmetric space.
Let $I$ be the action functional of a calculus of variations problem whose solution is a map $u: M \to N$.
Assume that $I$ is strictly convex and invariant under the infinitesimal symmetries of $M$.
Let $u$ be the minimizer of $I$ and let $Q$ be the solution of the dual calculus of variations problem to $I$.
Then $Q$ is the conserved current of $u$ associated to the symmetries of $N$.
\end{mainthm}

We shall later make the statement of this theorem, most importantly the meanings of the terms ``dual calculus of variations problem'' and ``conserved current'', more precise in the body of the paper.
After proving this theorem, we illustrate its application to certain $p$-Laplacian systems.
This explains how I was able to construct the dual transverse measure to certain problems in the $L^\infty$ calculus of variations using convex duality \cite{BackusBest1} while Daskalopolous and Uhlenbeck constructed a closely related dual transverse measure using Noether's theorem \cite{daskalopoulos2022}.


\subsection{Notation}
Here is a brief summary of the notation used throughout this paper.

\begin{longtable}{@{\extracolsep{\fill}}lp{0.75\textwidth}}
\multicolumn{2}{c}{\textbf{General notation}}\\[4pt]
$J^1(M, N)$ & bundle of $1$-jets on $M$ with values in $N$ \\
$V^*$ & dual vector space to a vector space $V$ \\
$\hat I$ & Legendre transform of the convex function $I$ \\
\\
\end{longtable}

%%%%%%%%%%%%%%%%%%
\subsection{Acknowlegments}
This research was supported by the National Science Foundation's Graduate Research Fellowship Program under Grant No. DGE-2040433.

\section{Proof of main theorem}
\subsection{Application of Noether's theorem}
Let $M$ be a Riemannian manifold of dimension $d$ and $N$ a manifold of dimension $D$.
Let $J^1(M, N)$ be the bundle of $1$-jets on $M$ with values in $N$; a typical point $(x, z, \xi) \in J^1(M, N)$ satisfies $x \in M$, $z \in N$, and $\xi \in \Hom(T_x M, T_z N)$.
Thus $J^1(M, N)$ is a fiber bundle over $M$.
By a \dfn{Lagrangian} (of order $1$), we shall always mean a smooth function on $J^1(M, N)$.

Suppose that $N = \Gamma \setminus G/H$ is a locally symmetric space.
Here $G$ is a connected Lie group, $H$ is a compact subgroup, and $\Gamma$ is a discrete subgroup.
\todo{We ignore the twisting by $\Gamma$ by now but this can be resolved later easily enough.}
The action of $G$ on its tangent bundle induces an action of $G$ on each fiber of $J^1(M, N)$.
Let $\mathfrak g$ be the Lie algebra of $G$.

\begin{definition}
A Lagrangian $\Lagrange$ is \dfn{$G$-invariant} if there exists a smooth map $\Lambda: M \to \mathfrak g^*$ such that for every $X \in \mathfrak g$ and $(x, z, \xi) \in J^1(M, N)$,
$$\frac{\partial}{\partial t} \Lagrange(x, e^{tX}z, e^{tX} \xi)\bigg|_{t = 0} = \nabla \cdot \langle \Lambda, X\rangle.$$
\end{definition}

Suppose that $\Lagrange$ is $G$-invariant, and consider the action
$$I(u) := \int_M \Lagrange(x, u(x), \dif u(x)) \dif V(x).$$
If $I'(u) = 0$ and $X$ is a constant element of the Lie algebra, then $I$ is constant along the variation $u_t := e^{tX} u$.
So by Noether's theorem, the $d - 1$-form 
$$\langle J, X\rangle := \star \left\langle\frac{\partial \Lagrange}{\partial \xi}(x, u(x), \dif u(x)), X\right\rangle$$
is closed.
Here, $X$ is viewed as a vector field on $N$, and $\partial \Lagrange/\partial \xi(x, u(x), \dif u(x))$ is viewed as a $1$-form on $M$ with values in the cotangent bundle of $N$.
In particular, $\partial \Lagrange/\partial \xi$ can be viewed as a $\mathfrak g^*$-valued $1$-form on $M$.
This motivates the following definition, and proposition which sums up the above discussion:

\begin{definition}
Suppose that $\Lagrange$ is $G$-invariant and $I'(u) = 0$.
The \dfn{conserved current associated to $G$} is 
$$Q_u := \star \frac{\partial \Lagrange}{\partial \xi}(x, u(x), \dif u(x)).$$
\end{definition}

\begin{proposition}
Suppose that $\Lagrange$ is $G$-invariant and $I'(u) = 0$.
Then the conserved current $Q_u$ associated to $G$ is a $\mathfrak g^*$-valued $d - 1$-form.
Furthermore, $\dif Q_u = 0$.
\end{proposition}

It is convenient to rewrite the action as a function on $C^\infty_\cpt(M, \mathfrak g)$.
To be more precise, if $I'(u) = 0$, then we introduce a new action
\begin{align*}
J: C^\infty_\cpt(M, \mathfrak g) &\to \RR \\
X &\mapsto I(e^X u).
\end{align*}
If $(X_t)$ is a variation of $0$, then we obtain a variation $u_t := e^{X_t} u$ of $u$, and 
$$\frac{\dif}{\dif t} J(X_t)\bigg|_{t = 0} = \frac{\dif}{\dif t} I(u_t)\bigg|_{t = 0} = 0,$$
so $J'(0) = 0$.
Conversely, if $(u_t)$ is a variation of $u$, then we can use the symmetry of $N$ to find $X \in \mathfrak g$ such that 
$$u_t = e^{tX}u + O(t^2),$$
and we have 
$$I(u_t) = J(tX) + O(t^2),$$
so knowledge of the fact that $0$ is a critical point of $J$ is enough to recover that $I'(u) = 0$.


\subsection{Application of convex duality}
Throughout this section, fix a $G$-invariant Lagrangian $\Lagrange$ with action $I$.

\begin{definition}
The action $I$ is \dfn{convex} if, whenever $(u_t)_{t \in [0, 1]}$ is a geodesic homotopy between maps $u_0, u_1: M \to N$,
$$tI(u_0) + (1 - t)I(u_1) \leq I(u_t).$$
\end{definition}

\begin{lemma}
If $I$ is convex, and $u$ is a minimizer of $I$, then the action
$$J(X) := I(e^X u)$$
on the Lie algebra is also convex, with a minimum at $X = 0$.
\end{lemma}

\todo{This might not literally be true, since $\mathfrak g$ includes rotations.
But I guess it is true on the positive cone or something, at least when $I$ is strictly convex and $X$ is close to $0$ (since then we can Taylor expand the Lie exponential map and it's basically just euclidean translations).}

Since $J$ is convex, we can introduce its Legendre transform \todo{Need to formulate $J$ in such a way that $\partial \Lagrange/\partial \xi$ is naturally an element of the domain of $\hat J$. Look at the $p$-Laplacian for examples}
Here we're supposed to conclude that $Q_u$ satisfies strong duality:
$$J(X) = -\hat J(-Q_u).$$
This completes the proof.

\section{Application to \texorpdfstring{$p$-Laplacian}{p-Laplacian} systems}


\printbibliography

\end{document}
