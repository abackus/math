  \documentclass[12pt]{report}
\usepackage[utf8]{inputenc}
\usepackage[margin=1in]{geometry}
\usepackage{amsmath,amsthm,amssymb}
\usepackage{mathrsfs}

\usepackage{enumitem}
%\usepackage[shortlabels]{enumerate}
\usepackage{tikz-cd}
\usepackage{mathtools}
\usepackage{amsfonts}
\usepackage{amscd}
\usepackage{makeidx}
\usepackage{enumitem}
\title{Logic}
\author{Aidan Backus}
\date{December 2019}


\newcommand{\NN}{\mathbb{N}}
\newcommand{\ZZ}{\mathbb{Z}}
\newcommand{\QQ}{\mathbb{Q}}
\newcommand{\RR}{\mathbb{R}}
\newcommand{\CC}{\mathbb{C}}
\newcommand{\CP}{\mathbb{CP}}
\newcommand{\PP}{\mathbb{P}}
\newcommand{\DD}{\mathbb{D}}

\newcommand{\AAA}{\mathcal A}
\newcommand{\BB}{\mathcal B}
\newcommand{\HH}{\mathcal H}

\newcommand{\CVect}{\mathbf{Vect}_\CC}
\newcommand{\Grp}{\mathbf{Grp}}
\newcommand{\Open}{\mathbf{Open}}
\newcommand{\Set}{\mathbf{Set}}

\newcommand{\A}{\mathcal A}
\newcommand{\Aut}{\operatorname{Aut}}
\newcommand{\Cantor}{\mathcal{C}}
\DeclareMathOperator{\ch}{ch}
\DeclareMathOperator*{\chsupp}{ch\,supp}
\DeclareMathOperator*{\coker}{coker}
\newcommand{\D}{\mathcal{D}}
\newcommand{\dbar}{\overline \partial}
\newcommand{\card}{\operatorname{card}}
\newcommand{\diam}{\operatorname{diam}}
\newcommand{\End}{\operatorname{End}}
\DeclareMathOperator*{\esssup}{ess\,sup}
\newcommand{\FF}{\mathcal{F}}
\newcommand{\GL}{\operatorname{GL}}
\newcommand{\Hom}{\operatorname{Hom}}
\newcommand{\id}{\operatorname{id}}
\newcommand{\ind}{\operatorname{ind}}
\newcommand{\interior}{\operatorname{int}}
\newcommand{\lcm}{\operatorname{lcm}}
\newcommand{\Lip}{\operatorname{Lip}}
\newcommand{\MM}{\mathcal M}
\newcommand{\OO}{\mathcal{O}}
\newcommand{\PGL}{\operatorname{PGL}}
\newcommand{\pic}{\vspace{30mm}}
\newcommand{\pset}{\mathcal{P}}
\newcommand{\rank}{\operatorname{rank}}
\newcommand{\Rep}{\operatorname{Rep}}
\newcommand{\Res}{\operatorname{Res}}
\newcommand{\Riem}{\mathcal{R}}
\newcommand{\RVect}{\RR\operatorname{-Vect}}
\newcommand{\Sch}{\mathcal{S}}
\newcommand{\SL}{\operatorname{SL}}
\newcommand{\Spec}{\operatorname{Spec}}
\newcommand{\spn}{\operatorname{span}}
\newcommand{\supp}{\operatorname{supp}}

\newcommand{\altrep}{\rho_{\text{alt}}}
\newcommand{\trivrep}{\rho_{\text{triv}}}
\newcommand{\regrep}{\rho_{\text{reg}}}
\newcommand{\stdrep}{\rho_{\text{std}}}

\newcommand{\Card}{\mathbf{Card}}
\newcommand{\code}{\texttt}
\DeclareMathOperator{\Der}{Der}
\DeclareMathOperator{\Halt}{\mathbf{Halt}}
\newcommand{\halts}{\downarrow}
\renewcommand{\iff}{\leftrightarrow}
\newcommand{\nohalts}{\uparrow}
\newcommand{\HOD}{\mathbf{HOD}}
\DeclareMathOperator{\Hull}{Hull}
\newcommand{\Ord}{\mathbf{Ord}}
\newcommand{\PA}{\mathbf{PA}}
\newcommand{\PR}{\mathbf{PR}}
\newcommand{\proves}{\vdash}

\DeclareMathOperator{\Add}{Add}
\DeclareMathOperator{\cof}{cof}
\DeclareMathOperator{\comp}{comp}
\DeclareMathOperator{\Con}{Con}
\DeclareMathOperator{\crt}{crt}
\DeclareMathOperator{\Def}{Def}
\DeclareMathOperator{\dom}{dom}
\DeclareMathOperator{\ext}{ext}
\DeclareMathOperator{\Fin}{Fin}
\DeclareMathOperator{\Name}{Name}
\DeclareMathOperator{\pred}{pred}
\DeclareMathOperator{\Sat}{Sat}
\DeclareMathOperator{\TC}{TC}
\DeclareMathOperator{\Th}{Th}
\DeclareMathOperator{\Tot}{Tot}
\DeclareMathOperator{\Ult}{Ult}

\newcommand{\forces}{\Vdash}

\def\Xint#1{\mathchoice
{\XXint\displaystyle\textstyle{#1}}%
{\XXint\textstyle\scriptstyle{#1}}%
{\XXint\scriptstyle\scriptscriptstyle{#1}}%
{\XXint\scriptscriptstyle\scriptscriptstyle{#1}}%
\!\int}
\def\XXint#1#2#3{{\setbox0=\hbox{$#1{#2#3}{\int}$ }
\vcenter{\hbox{$#2#3$ }}\kern-.6\wd0}}
\def\ddashint{\Xint=}
\def\dashint{\Xint-}

\renewcommand{\Re}{\operatorname{Re}}
\renewcommand{\Im}{\operatorname{Im}}
\newcommand{\dfn}[1]{\emph{#1}\index{#1}}

\usepackage{color}
\usepackage{hyperref}
\hypersetup{
    colorlinks=true, % make the links colored
    linkcolor=blue, % color TOC links in blue
    urlcolor=red, % color URLs in red
    linktoc=all % 'all' will create links for everything in the TOC
    %Ning added hyperlinks to the table of contents 6/17/19
}

\theoremstyle{definition}
\newtheorem{theorem}{Theorem}[chapter]
\newtheorem{lemma}[theorem]{Lemma}
\newtheorem{proposition}[theorem]{Proposition}
\newtheorem{corollary}[theorem]{Corollary}
\newtheorem{conjecture}[theorem]{Conjecture}
\newtheorem{axiom}[theorem]{Axiom}
\newtheorem{axiomZFC}{Axiom of ZFC}
\newtheorem{definition}[theorem]{Definition}
\newtheorem{remark}[theorem]{Remark}
\newtheorem{example}[theorem]{Example}
\newtheorem{exercise}[theorem]{Exercise}
\newtheorem{problem}[theorem]{Problem}

\makeindex
\begin{document}

\maketitle

\tableofcontents

\chapter{Elementary recursion theory}
Let $V$ be the set of validities; i.e. first-order formulae $\psi$ such that for every structure $M$, $M \models \psi$.

\begin{theorem}[completeness]
For every first-order formula $\psi$, $\psi \in V$ iff $\proves \psi$.
\end{theorem}
\begin{corollary}
There is an algorithm which enumerates $V$.
\end{corollary}
\begin{proof}
Iterate over all strings in the language. For each such string check if it is a proof; if it is, yield its conclusion.
\end{proof}
So it is possible to check that a formula is in $V$; but the above procedure does not allow us to check that a formula is not in $V$.
\begin{theorem}[Church-Turing]
There is no algorithm to determine membership in $V$.
\end{theorem}
The proof is easy once we have a rigorous definition of ``algorithm."
\begin{corollary}
There is no complete finite axiomatization of the theory of algorithms.
\end{corollary}
\begin{proof}
Suppose there were finitely many axioms from which we could deduce the theory of algorithms. Then we could decide membership in $V$, by considering the algorithm for $V$.
\end{proof}
Note that we prove that mathematical statements are undecidable in the same way every time.
\begin{example}
Suppose that $P$ is an NP-complete problem. Then the satisfaction relation embeds in $P$ (in fact, this is how we always prove that $P$ is NP-complete!) So there is really only one reason that a problem is ``hard" in complexity theory.
\end{example}

\section{Examples of the diaagonal argument}
\begin{example}
For every real $x$, there is a number which is not algebraic in $x$. Enumerate the algebraic numbers in $x$ as $q_i$ and let $r$ disagree with $q_i$ at the $i$th digit. In fact, when Liouville proved that there was a transcendental number, his proof used a difficult Diophantine approximation argument that does not generalize to the case of numbers which are actually algebraic in $x$.
\end{example}
\begin{example}[halting problem]
\index{halting problem}
There is a set which is not computable. Enumerate the programs as $P_i$. Let $X$ be the set of $i$ such that $P_i(i)$ returns False. Thus the halting problem is not decidable.
\end{example}
\begin{example}[Tarski's undefinability theorem]
\index{Tarski's undefinability theorem}
There is an undefinable set. Enumerate the first-order formulae with one free variables as $\psi_i$ and let $X$ be the set of $i$ such that $\NN \models \neg\psi_i(i)$. Thus the satisfaction predicate is not definable.
\end{example}

\section{A model for computation}
We now make the above diagonal arguments rigorous.

Let $A$ be an alphabet, so words in $A$ are elements of $A^{<\omega}$, which we will also write as $A^*$. By a \dfn{procedure} in $A$ we mean an algorithm which takes words as inputs and outputs.
\begin{definition}
A \dfn{decision procedure} for $W \subseteq A^*$ is a procedure which takes a word $w$ as input and returns the empty word $\Box$ if $w \in W$ and a nonempty word if $w \notin W$.
\end{definition}
If $A$ is a finite and $W$ has a decision procedure, then so does $W^c$. Similarly for finite unions and intersections, and set differences. (So the set of sets which admit a decision procedure forms a boolean algebra.)
\begin{definition}
An \dfn{enumeration procedure} is a procedure which on $\Box$ input returns a sequence of words.
\end{definition}
So a set $W$ is enumerable iff there is an enumeration procedure whose image is $W$, and $W$ has a decision procedure iff $W$ and $W^c$ are enumerable.
\begin{definition}
Any $f: A^* \to A^*$ is a \dfn{computable function} iff there is a procedure which takes $w \in A^*$ and returns $f(w)$.
\end{definition}
So a function is computable iff its graph is enumerable iff its graph has a decision procedure.

We now rigorously define ``procedure" using the notion of a register machine. Let $A = \{a_1, \dots, a_r\}$. A \dfn{register machine} has finitely many \dfn{register}s $R_1, ..., R_N$, each of which can hold a word in $A^*$.
\begin{definition}
A \dfn{register machine program} is a finite list of instructions of the following forms:
\begin{enumerate}
\item Let $R_i = R_i + a_j$.
\item If the last letter of $R_i$ is $a_j$ then let $R_i = R_i - a_j$.
\item If $R_i = \Box$ goto line $L$. Elif the last letter of $R_i$ is $a_j$ goto line $L_j$.
\item Print $R_1$.
\item Halt.
\end{enumerate}
Here $+$ denotes concatenation. We will assume that if $\alpha_1,\dots, \alpha_k$ are the instructions, then $\alpha_j$ says ``Halt" if and only if $j = k$.
\end{definition}
Note that if we only cared about computable sets rather than enumerable sets we would not need ``Print". This is analogous to the ``yield" command in Python and lazy returns in Lisp.
\begin{example}
Let $A = \{a\}$. Consider the register machine program:
\begin{enumerate}
\item If $R_2 = \Box$ then goto $5$, else goto $2$
\item Let $R_1 = R_1 + a$
\item If the last letter of $R_2$ is $a$ then let $R_2 = R_2 - a$
\item If $R_2 = \Box$ then goto $5$, else goto $2$
\item Print $R_1$
\item Halt
\end{enumerate}
This program takes $R_1$ and $R_2$ as input, and replaces $R_1$ with $R_1 + R_2$ and $R_2$ with $\Box$. So this program witnesses that addition is computable.
\end{example}
If $P$ is a register machine program, then we write
$$P: w \mapsto \eta$$
if $P$ halts on input $w$ and prints $\eta$. We write $P: w \mapsto \infty$ if $P$ does not halt.
\begin{axiom}[Church-Turing thesis]
\index{Church-Turing thesis}
  \label{church and turing}
Assume that $A$ is finite. If there is a procedure to decide $W \subseteq A^*$, then there is a register machine program which also decides $W$.
\end{axiom}
So without loss of generality we may assume that any program is a register machine program. Note that the Church-Turing thesis only applies to idealized programs. For example, if Planck's constant is uncomputable, one could in principle use the digits of Planck's constant in some machine, but it would not be necessarily computable using a register machine program.

One might also hope to extend the Church-Turing thesis to nondeterministic algorithms. However, it is a theorem that if $X$ is a measurable subset of $[0, 1]$ whose measure is positive, then any function that can be computed by a program which can take elements of $[0, 1]$ as parameters and returns the same as long as those elements lie in $X$ is actually computable in the deterministic sense.
\begin{definition}
A set $W$ is \dfn{register-decidable} if there is a register machine program which decides $W$.
\end{definition}
\begin{example}
Let $A = \{a\}$. Consider the register machine program $p$:
\begin{enumerate}
\item If $R_1 = \Box$ then goto 2 else goto 3
\item Let $R_2 = R_2 + a$
\item If the last letter of $R_1$ is $a$ then let $R_1 = R_1 - a$
\item If $R_1 = \Box$ then goto 5 else goto 3
\item If $R_2 = \Box$ then goto 7 else goto 6
\item Let $R_1 = R_1 + a$
\item Print
\item Halt
\end{enumerate}
Suppose that $R_0 = w$ and $R_1 = \Box$ initially. Then this program outputs $a$ if $w = \Box$ and $\Box$ otherwise. Therefore the set of nonempty words, $\{\Box\}^c$, is register-decidable. In particular it is decidable.
\end{example}

\section{The halting problem}
Let $P$ be a register machine program. Then $P$ is a finite sequence of instructions, each of which consists of finitely many symbols. Therefore $P$ can be viewed a finite word in a finite alphabet $A$.

Let $\Halt$ be the set of register machine programs $P$ such that $P$ halts on input $P$.
\begin{theorem}
$\Halt$ is not register-decidable by any machine in the alphabet $A$.
\end{theorem}
\begin{proof}
Suppose that $\Halt$ is register-decidable by some program $P$ in the alphabet $A$. Then $P$ has a halt instruction, say on line $k$. Define a program $Q$ by taking the code for $P$ and replacing line $k$ with the instruction ``goto $k$", and append line $k+1$ with a halt instruction. In addition, for all lines $\ell$ in $P$ with a print instruction, replace $\ell$ in $Q$ with ``if $R_1 = \Box$ then goto $\ell + 1$ else goto $k+1$."

Let $\sigma$ be a string. If $P: \sigma \mapsto \eta$ and $\eta \neq \Box$, then the change to lines $\ell$ and $k+1$ causes $Q$ to halt. That is, if $P$ thinks that $\sigma$ does not halt, then $Q$ halts on input $\sigma$. Conversely, if $P: \sigma \mapsto \Box$, then $Q$ does not print anything due to the change to line $\ell$, and then loops forever due to the change to line $k$. So if $P$ thinks that $\sigma$ halts, then $Q$ does not halt on input $\sigma$.

We call $Q$ on input $Q$. If $P: Q \mapsto \eta$ and $\eta \neq \Box$, then $Q$ halts when called with input $Q$, so $P: Q \mapsto \Box$, a contradiction. Otherwise, if $P: Q \mapsto \Box$, then $Q$ does not halt when called with input $Q$, so $P: Q \mapsto \eta$ for some $\eta \neq \Box$, which is also a contradiction.
\end{proof}

Let $A \subset B$ be alphabets. Clearly we cannot solve the halting problem in $A$ with a program in the alphabet $A$, but one could consider choosing $B$ so big that a program in $B$ can decide the halting problem in $A$. However, this turns out to be impossible, and we can always collapse $A$ to an alphabet of just one letter without affecting the halting problem.
\begin{exercise}
Show that if $P$ is a program in any nonempty alphabet $A$ and $a \in A$ is a symbol, there is a program $Q$ in the alphabet $\{a\}$, such that if $P: Q \mapsto \Box$ then $Q$ does not halt, and if $P: Q \mapsto \eta$ for some $\eta \neq \Box$ then $Q$ halts. (Hint: Show that $\{a\}$ can be used to describe an arbitrarily large alphabet using Godel codes.)
\end{exercise}
\begin{corollary}
$\Halt$ is not register-decidable by any register machine written in any alphabet.
\end{corollary}
\begin{corollary}
There is a definable set which is not register-decidable.
\end{corollary}


\section{Primitive recursion}
We now introduce a stronger notion of recursive function, those which are especially easy to analyze because we have a priori control over their runtime.
\begin{definition}
A loop in a computer program is said to be a \dfn{bounded loop} if it can be expressed as a \code{for} loop, such that the set that the \code{for} loop indexes over is fixed at the start of the loop and cannot be edited from within the loop.
\end{definition}

\begin{definition}
A \dfn{primitive recursive function} is a partial recursive fuction $f$ such that there is a program which computes $f$, for which we are only allowed the operations $+$, $-$, $=$, $<$, \code{if}, and bounded loops. The set of primitive recursive functions is denoted $\PR$.
\end{definition}

\begin{theorem}
Every primitive recursive function is recursive.
\end{theorem}
\begin{proof}
Let $f$ be primitive recursive and suppose that $P$ witnesses that $f$ is primitive recursive. We fix a notion of runtime where the primitive operations $+$, $-$, $=$, $<$, and \code{if} take one unit of time. Those blocks of code in $P$ which are $s$ lines long and have no bounded loops have runtime at most $s < \infty$. Those blocks of code in $P$ which have $s$ lines of code and $N$ bounded loops, each with at most $n_k$ iterations and internal runtime at most $s_k < \infty$, have runtime at most $s + \sum_{k=1}^N n_ks_k < \infty$. By Noetherian induction, the outermost code block in $P$ has finite runtime. Therefore $f(n) \halts$.
\end{proof}

We now give a primitive recursive function that is not recursive by a diagonal argument. So it is not so easy to use Noetherian induction to bound the runtime of a program.
\begin{definition}
The \dfn{Ackermann function} $A: \NN^2 \to \NN$ is defined recursively by
$$
A(m, n) = \begin{cases}
  n + 1, &m = 0\\
  A(m-1,1), &m > 0, \quad n = 0\\
  A(m-1, A(m,n-1)) &(m,n) \neq 0.
\end{cases}
$$
\end{definition}
\begin{example}
Let $\alpha(n)$ denote the least $m$ such that $A(n, n) \leq m$, the \dfn{inverse Ackermann function}. Then $\alpha(n) \leq 5$ for any $n$ which would ``normally appear" in applied complexity theory (in fact for any $n$ less than the number of particles in the universe). This is because $A(4, 3) = 2^{2^{65536}} - 3$. In fact, the disjoint-set data structure, implemented with path compression and union-by-rank, has amortized runtime $O(\alpha(n))$, hence $O(1)$ for any physically existing machine.
\end{example}

\begin{theorem}
The Ackermann function is not primitive recursive.
\end{theorem}
\begin{proof}
Let $\mathcal A$ be the class of all multivariable functions $f: \NN^n \to \NN$ such that there is a $t \in \NN$, such that for every $x \in \NN^n$,
$$f(x_1, \dots, x_n) < A(t, \max_i x_i).$$
Clearly $\mathcal A$ contains the constant functions, and is closed under the successor function and projections. If $g: \NN^k \to \NN^m$ and $h: \NN^m \to \NN$ have $g_j, h \in \mathcal A$, then $g_i(x) < A(r_i, \max x)$, $h(y) < A(s, \max y)$ for some $r,s$. If $f = h \circ g$ and $g_j(x) = \max_j g(x)$ then
$$f(x) < A(s, g_j(x)) < A(s, A(r_j, x)) < A(s + r_j + 2, x)$$
so setting $t = s + r_j + 2$ we see that $f \in \mathcal A$. Therefore $\mathcal A$ is closed under function compsition.

Now suppose that $g: \NN^k \to \NN$, $h: \NN^{k+2} \to \NN$, $g, h \in \mathcal A$. Suppose that $g(x) < A(r, \max x)$ and $h(y) < A(s, \max y)$. Suppose that $f: \NN^k \times \NN \to \NN$ is recoverable from $g, h$ using a primitive recursive program. We claim $f \in \mathcal A$.

Let $q = 1 + \max(r,s)$ and induct on $n$ to see that
$$f(x, n) < A(q, n + x).$$
If $n = 0$ then $f(x, 0) < A(q, x)$. So
$$f(x, n + 1) = h(x, n, f(x, n)) < A(s, z)$$
where $z = \max(\max x, n, f(x, n))$. Then $z < A(q, n +x)$ by the inductive hypothesis. Thus
$$f(x, n + 1) < A(s, z) < A(s, A(q, n + x)) \leq A(q, n + 1 + x)$$
so $f \in \mathcal A$.

Therefore $A$ strictly dominates $\PR$, so $A \notin \PR$.
\end{proof}
The Ackermann function is used to test the ability of compilers to optimize deep recursion, which is difficult because of the limited height of the stack and the inability to tail-recurse, among other problems.


\chapter{Undecidability in first-order logic}
\section{Undecidability of the theory of algorithms}
\begin{definition}
A \dfn{validity} is a sentence $\varphi$ in a language $\mathcal L$ such that for every $\mathcal L$-structure $M$, $M \models \varphi$.
\end{definition}
In other words, $\varphi$ is a validity iff $\Box \varphi$ (here $\Box$ is the modal operator, not the empty string). For example, ``$\forall x ~x = x$" is a validity. Hilbert asked for an algorithm to determine whether a sentence is a validity. This wasn't completely unreasonable, because zeroth-order logic has an obvious algorithm to determine validity (namely truth table calculus), though this is NP-complete. However, it was unreasonable.

\begin{theorem}[Church-Turing]
The set of validities is not register-decidable.
\end{theorem}
We set up the proof. Let us assume that programs are given by register-machine programs in the alphabet $\{a\}$, words in which we identify with natural numbers: $(a, a, \dots, a)$ with $k$ entries is identified with $k$.
\begin{definition}
Let $P$ be a program with $n$ registers. A \dfn{configuration} for $P$ is a vector
$$(S, L, m_1, \dots, m_n) \in \omega^{n+3}$$
such that $S$ denotes the time step that we are on (so $S \mapsto S + 1$ every time we carry out an instruction), $L$ is the label of the instruction to be done next, and $m_j$ is the value of the register $R_j$ at the current time stage. The \dfn{initial configuration} with input $k$ is the configuration $(0, 0, k, 0, \dots, 0)$.
\end{definition}
The set $X$ of configurations which appear in the evaluation of $P$ with input $k$ clearly satisfies a recursion condition. In fact, the initial configuration is in $X$. Moreover, for every $S$, if $(S, L, m_1, \dots, m_n) \in X$ and the instruction at label $L$ is not ``Halt", then $(S+1, L^*, m_1^*, \dots, m_n^*) \in X$ where $L^*,m_1^*,\dots,m_n^*$ are the values that result at the next time stage.
\begin{definition}
We will say that a set of configurations $X$ is \dfn{closed under $P$} if for every $(S, L, m_1, \dots, m_n) \in X$ and the instruction at label $L$ is not ``Halt", then the vector $(s+1, L^*, m_1^*, \dots, m_n^*)$ defined by the instruction in $P$ at label $L$ with registers $m_1, \dots, m_n$, where $L^*$ is the label of the next instructions and $m_1^*,\dots,m_n^*$ are the new values of the registers, is also in $X$.
\end{definition}
So a set which is closed under $X$ is a generalization of the set of configurations that appear in the evaluation of $P$ with input $k$. This gives us the following useful fact:
\begin{lemma}
\label{halting closure}
The program $P$ halts on input $k$ if and only if for every set of configurations $X \subseteq \omega^{k+3}$ containing the initial configuration $(0, 0, k, 0, \dots, 0)$ which is closed under $P$, there is a configuration in $X$ at label $L$, such that $P$ has a ``Halt" instruction at label $L$.
\end{lemma}

We now construct a first-order language $\mathcal L$ to discuss algorithms. Let $c$ be a constant symbol, $f$ a unary function symbol, and $R$ an $(n+3)$-ary relation symbol. We will interpret $\mathcal L$ on $\omega$, interpreting $c$ as $0$, $f$ the successor function, and $R$ as the set of configurations which appear in the evaluation of a program $P$ with input $k$, viewed as a subset of $\omega^{n+3}$.

We now let $\varphi_0$ be the sentence ``$f$ is injective and $c$ is not in the image of $f$", i.e.
$$\varphi_0: \forall x_1\forall x_2 (f(x_1) = f(x_2) \iff x_1=x_2) \wedge \forall x_1 (f(x_1) \neq c).$$
In particular, a model of $\varphi_0$ contains a copy of $\omega$, namely the set generated by the intepretations of $c$ and $f$. Therefore, for any $m \in \omega$, we may let $\overline m$ denote the symbol $f(f(\cdots f(c)))$ ($m$ copies of $f$), which corresponds to $m$ in the copy of $\omega$.
Given an input string $k$, let $\varphi_{init}(\alpha)$ be the formula ``$\alpha$ is the initial configuration vector for $k$", i.e. if $\alpha = (\overline 0, \overline 0, \overline k, \overline 0, \dots, \overline 0)$ then
$$\varphi_{init}(\alpha): R(\overline 0, \overline 0, \overline k, \overline 0, \dots, \overline 0).$$
Let $\varphi_\alpha$ be the formula ``if a configuration $\alpha$ is in $R$ then so is the instruction obtained from $\alpha$"; i.e.
\begin{enumerate}
\item If $\alpha$ is at label $L$ and says ``Let $R_i = R_i + a$" then
$$\varphi_\alpha: \forall x \forall y_1 \cdots \forall y_n R(x, \overline L, y) \to R(f(x), \overline{L+1}, y_1, \dots, y_{i-1}, f(y_i), y_{i+1}, \cdots, y_n).$$
\item If $\alpha$ is at label $L$ and says ``Let $R_i = R_i - a$" then
\begin{align*}\varphi_\alpha: \forall x \forall y_1 \cdots \forall y_n R(x, \overline L, y) \to &((y_i = \overline 0) \wedge (R(f(x), \overline{L+1}, y)) \\
&\vee ((y_i \neq \overline 0) \wedge (\exists z(y = f(z)) \wedge R(f(x), \overline{L+1}, y_1, \dots, y_{i-1}, z, y_{i+1}, \dots, y_n)))).
\end{align*}
\item If $\alpha$ is at label $L$ and says ``If $R_i = \Box$ then goto $L'$ else goto $L_0$" then
\begin{align*}\varphi_\alpha: \forall x \forall y_1 \cdots \forall y_n R(x, \overline L, y) \to& ((y_i = \overline 0) \wedge R(f(x), \overline{L'}, y)) \\
&\vee (y_i \neq \overline 0) \wedge (R(f(x), \overline{L_0}, y)).
\end{align*}
\item If $\alpha$ is at label $L$ and says ``Print" then
\begin{align*}
\varphi_\alpha: \forall x \forall y_1 \cdots \forall y_n R(x, \overline L, y) \to R(x, \overline{L+1}, y).
\end{align*}
\item If $\alpha$ is at label $L$ and says ``Halt" then $\varphi_\alpha$ is not defined.
\end{enumerate}
Finally, we let $\varphi_{halt}$ mean ``$P$ halts", i.e. if $L$ is the label of the halting instruction in $P$ then
$$\varphi_{halt}: \exists x \exists y_1 \cdots \exists y_n R(x, \overline{L_{halt}}, y).$$
\begin{proof}[Proof of the Church-Turing theorem]
For every program $P$ and input string $k$, we define a sentence $\varphi_{P,k}$ such that $(P, k) \mapsto \varphi_{P,k}$ is a computable function and $P$ halts on input $k$ if and only if $\varphi_{P,k}$ is a validity. Then deciding validity in $\{\varphi_{P,P}\}$ ranging over all programs $P$ is equivalent to deciding $\Halt$, which is impossible.

In fact we let
$$\varphi_{P,k}: \left(\varphi_0 \wedge \varphi_{init} \wedge \bigwedge_\alpha \varphi_\alpha\right) \to \varphi_{halt}.$$
where the conjunction ranges over all $\alpha$ such that $P$ contains $\alpha$ as an instruction and $\alpha$ is not the halting instruction. Clearly $\varphi_{P,k}$ is computable from $(P,k)$; we must show that $P$ halts on input $k$ iff $\varphi_{P,k}$ is valid.

Suppose that $P$ diverges on $k$. Define a structure $M$ on $\omega$ by giving $c$ and $f$ their correct interpretations, and letting $R \subseteq \omega^{n+3}$ be the set of $(s, t, m) \in \omega \times \omega \times \omega^n$ such that $(s, t, m)$ is a configuration in the evaluation of $P$ on $k$.
Then $M$ clearly satisfies the antecedent of $\varphi_{P,k}$ but violates its consequent, so $M \models \neg \varphi_{P,k}$. Therefore $\varphi_{P,k}$ is invalid.

Conversely, if $P$ halts on $k$, then suppose that $M = (M, c^M, f^M, R^M)$ is a model of the antecedent of $\varphi_{P,k}$. We will prove that $M$ is a model of the consequent, so $\varphi_{P,k}$ is valid. In fact, we induct to show that if $(s, t, m)$ is a configuration in the evaluation of $P$ on $k$, then
$$M \models R(\overline s, \overline t, \overline m).$$
Since $M \models \varphi_{init}$, this is true when $s = t = 0$ and $m = (k, 0, \dots, 0)$. Now suppose that $(s, t, m)$ appears and $M \models R(\overline s, \overline t, \overline m)$. Let $\alpha$ be the instruction at label $t$ and let $t^*$ and $m^*$ be the result of the computation at time $s$.
Since $M \models \varphi_\alpha$ it follows that $M \models R(\overline{s+1}, \overline t^*, \overline m^*)$.
Since $P$ eventually halts, it follows by induction that $M \models \varphi_{halt}$.
\end{proof}
Notice that a key property of the theory of algorithms is that for every model $M$ (i.e. set of configurations), there is an embedding of $\omega$ in $M$ defined by $n \mapsto f^n(c)$. Since $M \models \varphi_0$ this mapping is actually injective. We can think of $M$ as a (possibly nonstandard) model of arithmetic (since there may be $k \in M$ which was not obtained from $n$ by applying the successor operation $f$). However, the nonstandard part of $M$ is irrelevant to the validity of $\varphi_{P,\eta}$.

\section{Undecidability of true arithmetic}
We now prove something stronger than the Church-Turing theorem, namely that the theory of arithmetic is undecidable.
\begin{theorem}
\label{true arithmetic is undecidable}
The theory $T$ of $\omega$ with the language $(0, 1, +, \times)$ is not decidable.
\end{theorem}
By definition, $\varphi \in T$ iff $\omega \models \varphi$. We sometimes call $T$ \dfn{true arithmetic}. In fact, $T$ is not even recursively axiomatizable.

To show that $T$ is not decidable using the above argument, we need a way to refer to configuration vectors using just natural numbers. Therefore we introduce a Godel coding. This will be challenging because we cannot refer directly to exponentiation in the language $(0, 1, +, \times)$. In fact the function $(p, i) \mapsto p^i$ is not definable in $\omega$.
\begin{lemma}[$\beta$-lemma]
\index{$\beta$-lemma}
There is a function $\beta: \omega^3 \to \omega$ such that:
\begin{enumerate}
\item For every sequence $(a_0, \dots, a_r) \in \omega^{r+1}$ there exist $t, p \in \omega$ such that for every $i \leq r$, $\beta(t, p, i) = a_i$.
\item $\beta$ is definable in $\omega$.
\end{enumerate}
\end{lemma}
In other words, if we view $(t, p)$ as coding the vector $a$, then $\beta(\cdot, \cdot, i)$ is the projection of $a$ onto the $i$th factor. Obviously many such $\beta$ exist, but it is important that $\beta$ can be easily described.
Definability in $\omega$ means that there is a formula $\varphi$ in the language $(0, 1, +, \times)$ with four free variables such that for all $t, p, i, a$, $\omega \models \varphi(t, p, i, a)$ if and only if $\beta(t, p, i) = a_i$.

The idea of the proof of the $\beta$-lemma is to use the $p$-adic expansion of $t$. This is not unique, but that is OK.

Before the proof, we note that $<$ is definable in $(0, 1, +, \times)$ by noting that $a < b$ iff
$$\exists c((c \neq 0) \wedge (a + c = b)).$$
Therefore we use $<$ freely.
\begin{proof}
Given $a_0, \dots, a_r$, let $p$ be a prime such that $p > r + 1$ and for each $i$, $p > a_i$, which is possible by Euclid's theorem on primes. Let
$$t = 1 + a_0p + 2p^2 + a_1p^3 + 3p^4 + a_2p^5 + \cdots + (r+1)p^{2r} + a_r p^{2r+1}.$$
Then the coefficient of $p^{2k}$ indiates the number $k+1$ and the coefficient of $p^{2k+1}$ indicates $a_k$. In particular, the $p$-adic expansion of $t$ is unique.
\begin{lemma}
For all $a$ and $i \leq r$, $a = a_i$ if and only if there exist $b_0, b_1,b_2$ such that:
\begin{enumerate}
\item $t = b_0 + b_1((i+1) + ap + b_2p^2)$.
\item $a < p$.
\item $b_0 < b_1$.
\item There exists $m$ such that $b_1 = p^{2m}$.
\end{enumerate}
\end{lemma}
\begin{proof}[Proof of sublemma]
Suppose $a = a_i$. Then the claim immediately follows from the definition of $t, p$: $b_0$ is given by lower powers of $p$, and then
$$t = b_0 + p^{2i}((i+1) + a_ip + p^2((i+2) + a_{i+1}p + \cdots + a_rp^{2r+1 - (2i +2)} )).$$
Then $b_1 = p^{2i}$ and $b_2$ consists of the higher powers of $p$, divided by $p^{2i}$, namely $b_2 = (i+2) + a_{i+1}p + \cdots$. Since $p > a_i$ it follows that $a < p$. The $a_i < p$ and $p > r + 1$, so $b_0 < p^{2i}$.

Conversely, suppose that $b_0, b_1, b_2$ are given and $b_1 = p^{2m}$. Then by the assumptions, $b_0$ is the remainder of $t$ by dividing by $p^{2m}$. Since $i + 1 < p$, $i + 1$ is the remainder of dividing $(t - b_0)p^{-2m}$ by $p$. So $i + 1$ is the $2m$th coefficient of the $p$-adic expansion of $t$. By definition of $t$, $m = i$.
Since $a < p$, $a$ is the $2i+1$th coefficient in the $p$-adic expansion of $t$. Therefore $a = a_i$.
\end{proof}
The fourth claim in the sublemma is not obviously definable in $\omega$. But it is definable by the formula
$$\exists x((x \times x = b_1) \wedge P(p, x))$$
where $P(q, y)$ means that $q$ is the unique prime divisor of $y$. Therefore the sublemma implies that the statement $a = a_i$ can be expressed in $(0, 1, +, \times)$.
In fact, if we choose $p$ to be minimal possible, then $(a_0, \dots, a_r) \mapsto (p, t)$ is a well-defined, computable function.

Now let $\beta(u, q, j)$ by the smallest $a \in \omega$ such that there are $b_0, b_1, b_2$ such that:
\begin{enumerate}
\item $u = b_0 + b_1((j+1) + aq + b_2q^2)$.
\item $a < q$.
\item $b_0 < b_1$.
\item There is an $m$ such that $b_1 = q^{2m}$.
\item $q$ is prime.
\end{enumerate}
If no such $b_0, b_1, b_2$ exist, then let $\beta(u, q, j) = 0$.

Since $<$ is definable, so is ``the smallest ...", so the sublemma implies that $\beta$ is definable.
\end{proof}
Note that if $p$ is composite then $\beta(\cdot, p, \cdot) = 0$.

As we will show, the coding of finite sequences (by any computable means, but in particular by the $\beta$-lemma) is sufficient to emulate a register machine within $\omega$. In particular, the theory of arithmetic can decide the halting problem. Trouble comes when we realize that we need to implement recursion in $(0, 1, +, \times)$. In fact, the exponential map $(x, y) \mapsto x^y$ is computable, but cannot be defined in the obvious way because it grows faster than any polynomial.

Given a register machine program $P$ and input $m \in \omega$, we define a first-order formula $\chi_{P,m}$ such that $\omega \models \chi_{P,m}$ iff $P$ halts on input $m$. Intuitively, $\chi_{P,m}$ says ``There is a finite sequence
$$0, m, 0, \dots, 0, L_1, x_{1,0}, \dots, x_{1,n}, L_2, x_{2,0}, \dots, x_{2,n}, L_3, \dots, L_z, x_{z,0}, \dots, x_{z,n}$$
such that for all $i$, the $(i+1)$th block $L_{i+1}, x_{i+1,0}, \dots, x_{i+1,n}$ is obtained from the $i$th block $L_i, x_{i,0}, \dots, x_{i,n}$ by applying the instruction $L_i$ to the registers $x_{i,0}, \dots, x_{i,n}$ to obtain registers $x_{i+1,0}, \dots, x_{i+1,n}$, and such that $L_z$ is the Halt instruction."
Thus $\omega \models \chi_{P,m}$ iff we can code the configuration sequence which witnesses that $P$ halts on input $m$ in $\omega$ (which will be possible provided that such a configuration sequence exists, by the $\beta$-lemma).

\begin{proof}[Proof of Theorem \ref{true arithmetic is undecidable}]
Let $\varphi_\beta$ be the definition of $\beta$. So $\omega \models \varphi_\beta(x, y, z, w)$ iff $\beta(x, y, z) = w$. Let $H$ denote the label for the halting instruction.

Then $\chi_{P,m}$ says that there are $t,p,z$ such that the following conjunction holds:
\begin{enumerate}
\item $\varphi_\beta(t, p, 0, 0)$
\item $\varphi_\beta(t, p, 1, \overline m)$
\item For every $j \in \{2, \dots, n+2\}$, $\varphi_\beta(t, p, \overline j, \overline 0)$
\item $\forall i < z \forall L \forall x_0 \cdots \forall x_n \forall L' \forall x_0' \cdots \forall x_n'$, if the conjunction
\begin{enumerate}
\item $\varphi_\beta(t, p, i \times \overline{n+2}, L)$
\item For every $j \in \{1, \dots, n+1\}$, $\varphi_\beta(t, p, i \times \overline{n+2} + \overline j, x_0)$
\item For every $j \in \{0, \dots, n+1\}$, $\varphi_\beta(t, p, (i+1) \times \overline{n+2} + \overline j, L')$
\end{enumerate}
then ``$(L', x_0', \dots, x_n')$ is obtained from $(L, x_0, \dots, x_n)$ by application of the instruction of $P$ with label $L$ on registers with contents in $x_0, \dots, x_n$."
\item $\varphi_\beta(t, p, \overline z \times \overline{n+2} + \overline 1, \overline H)$
\end{enumerate}
The clause $\psi$ in quotes (``$(L', x_0', \dots, x_n')$ is obtained...") is tedious but easy to define. It is a conjunction of $H$ many clauses, one for each line of $P$ that is not a halting instruction.

We do not construct all conjuncts of $\psi$ but give an example of just one clause. Suppose that at label $\ell$ we have the instruction ``Let $R_g = R_g + a$." Then the $\ell$th conjunct of $\psi$ is
$$(L = \overline \ell) \to \left(L' = \overline{\ell + 1} \wedge (x_g' = x_g + 1) \wedge \bigwedge_{j \neq g} (x_j' = x_j + 1)\right).$$
The other conjuncts are defined similarly.

Therefore if we can determine if $\chi_{P,P} \in T$, we can decide whether $P \in \Halt$.
\end{proof}
Note that $\chi_{P,m}$ is of the form $\exists p \exists t \exists z \psi_{P,m}$ where $\psi_{P,m}$ has bounded quantifiers; $p,t$ control how large we have to check for each quantifier in $\psi_{P,m}$. Thus $\chi_{P,m}$ has a very low quantifier complexity, but even this is undecidable, so we have actually proven something stronger than the claimed theorem:
\begin{corollary}
The set of true $\Sigma_1^0$ sentences in $\omega$ is not decidable.
\end{corollary}
Stronger than this, the set of multivariable polynomials $p$ over $\ZZ$ such that $p$ has an integer zero is not decidable. This theorem does not even have a bounded quantifier, but just a single unbounded $\exists$.
\begin{corollary}[representability theorem]
\index{representability theorem}
Suppose $D$ is a decidable predicate on $\omega$. Then then the definition of $D$ has quantifier complexity $\Sigma_1^0$.
\end{corollary}
To see this, simply use the formula $\chi_{P,m}$ where $P$ is defined to halt iff $D$ returns True. In particular, every computable function $\omega^n \to \omega$ has a $\Sigma_1^0$ definition. The converse is not true, as we already proved.

\section{Goedel's completeness theorem}
\begin{theorem}[completeness]
\index{completeness theorem}
For every set of formula $\Gamma$ and every formula $\varphi$, if $\Gamma \models \varphi$, then $\Gamma \proves \varphi$.
\end{theorem}
Here $\Gamma \models \varphi$ means that for every $(M, \nu)$, if $(M, \nu) \models \Gamma$, then $(M, \nu) \models \varphi$. Meanwhile, $\Gamma \proves \varphi$ simply means that there is a proof $P$ of $\varphi$ from $\Gamma$, namely a finite sequence of formulae, each of which follows from those before it and elements of $\Gamma$ by modus ponens, such that the final formula in $P$ is $\varphi$.

In particular, if $\Gamma$ is decidable, then the set of $\varphi$ such that $\Gamma \proves \varphi$ is computably enumerable; namely, iterate over all finite sequences of formulae which follow from elements of $\Gamma$ by modus ponens, and check if each one is a proof.
\begin{definition}
A \dfn{theory} is a set of formulae which is closed under the relation $\proves$. A \dfn{complete theory} is a theory $T$ such that for every sentence $\varphi$, either $\varphi \in T$ or $\neg\varphi \in T$.
\end{definition}
In other words, if $T$ is complete then for every $\varphi$, $T \proves \varphi$ or $T \proves \neg \varphi$.
\begin{example}
Let $\Th(M)$ be the set of $\varphi$ such that $M \models \varphi$. Then $\Th(M)$ is a complete theory.
\end{example}
Now note that every proof has a finite length, and in particular there is an algorithm that decides whether each line in a sequence of formulae follows from those before it, whence the set of proofs is decidable. Therefore we have the following theorem.
\begin{theorem}[compactness]
\index{compactness theorem}
Let $\Gamma$ be a set of sentences and $\varphi$ a formula. If $\Gamma \proves \varphi$ then there is a finite subset $\Gamma_0 \subseteq \Gamma$ such that $\Gamma_0 \proves \varphi$.
\end{theorem}
We start by assuming that the language of $\Gamma$ has no function symbols. In fact every function is a relation, so any formula that is true about a function is also true about some relation. If $F$ is a function let $G_F$ be the graph of $F$; then for any $\varphi$,
$$\varphi(F(x)) \iff \exists y(G_F(x, y) \wedge \varphi(y)).$$

We now carry out a modified Henkin construction. Fix a formula $\varphi$. Expand the language of $\Gamma$ by adding a countable set of constant symbols $\{c_n\}_n$ and fix an enumeration $\{\theta_n\}_n$ of the sentences in the expanded language such that for every $j < i$, $c_i$ does not appear in $\theta_j$. (This is possible by choosing a sentence that has no $c_n$'s, then a sentence that has $c_0$, then a sentence with no $c_n$'s, then a sentence that has $c_0$, then a sentence that has $c_0$ or $c_1$, then a sentence that has no $c_n$'s, et cetra. Then before we get to a sentence that has $c_n$ we have already taken $n$ sentences that had no $c_i$'s at all.)

Now consider the binary tree of all possible attempts to satisfy all the formulae in $\Gamma$ and $\neg \varphi$. The root of the tree is $\neg\varphi$. Then include two branches: $\theta_0$ and $\neg\theta_0$. Now $c_0$ does not appear in $\theta_0$, so if $\theta_0 = \exists x \psi_0(x)$, we can define $c_0$ to be the witness to $\theta_0$. In this case, $\theta_0$ only has one child: $\psi_0(c_0)$.

We inductively assume that we have constructed a finite binary tree $G$, and that certain nodes have been ``terminated." At each stage we give each nonterminated leaf two children, namely $\theta_n$ and $\neg\theta_n$, and if $\theta_n = \exists x \psi_n(x)$, we give $\theta_n$ the child $\psi_n(c_n)$. We then also add $\psi_k(c_n)$ for every $k$ such that $\forall x \psi_k(x)$ has already appeared. Then, we terminate one of the new leaves $L$ if there is no finite structure which satisfies all the quantifier-free sentences which appear in the path from $\neg\varphi$ to $L$.

In fact, since such sentences are quantifier-free, they only refer to the $c_k$, and so it suffices to check finitely many sentences on the set of all relational structures defined on the set $\{c_j: j < k\}$, which can be done computably! Thus one could concievably build the infinite path by breadth-first search.

Suppose that $M$ is a model of $\Gamma \cup \{\neg\varphi\}$. Then we construct an infinite path through $G$: at each $\theta_n$, if $M \models \theta_n$, go down the path $\theta_n$ (and conversely for $\neg\theta_n$) and if necessary interpret $c_n$ to be the least witness to $\exists x \psi_n(x)$ after choosing some well-ordering of $M$ (which is possible with just countable choice by the Lowenheim-Skolem theorem).

Conversely, if $P$ is an infinite path through $G$, we define a structure $M_P$. Elements of $M_P$ will be the constant symbols $\{c_n\}_n$ modulo the equivalence relation that if a formula that appear in $P$ is of the form $c_i = c_j$ then $c_i \sim c_j$. (If a formula implies $c_i = c_j$ then termination implies that $c_i = c_j$ appears at a later stage. So it suffices to just check formulae of the form $c_i = c_j$.)
We interpret the constants $c$ not of the form $c_i$ to simply map to the equivalence class of $c_j$ if $c = c_j$ appears (and it must appear for some $j$). Then we interpret $R([c_{i_1}], \dots, [c_{i_k}])$ to be true if $R(c_{i_1}, \dots, c_{i_k})$ appears in $P$.

\begin{lemma}
For every sentence $\chi$, $M \models \chi$ iff $\chi \in P$.
\end{lemma}
\begin{proof}
By induction on quantifier complexity. This is true by definition for sentences of the form $c_i = c_j$. Moreover if $a,b$ are constants and $a = b$ appears in $P$, then there are some $i,j$ such that $a = c_i$ and $b = c_j$, but then $P$ must contain $c_i = c_j$ whence $M \models (a=b)$, and the converse follows from the same argument in reverse. A similar argument checks relational sentences as well as boolean operations.

Finally we consider existential sentences. Suppose $M \models \exists x \psi(x)$; then we can find a minimal $i$ such that $M \models \psi([c_i])$, so by induction $\psi([c_i]) \in P$ and hence $\theta_i = \exists x \psi(x)$ and $\theta_i \in P$. The converse is similar.
\end{proof}

We now appeal to weak Koenig's lemma:
\begin{lemma}[weak Koenig's lemma]
\index{weak Koenig's lemma}
Every infinite binary tree has an infinite path.
\end{lemma}
\begin{proof}
The first $n$ levels have at most $2^n$ nodes, so there must be more than $n$ nodes for every $n$.
\end{proof}
The decision tree $G$ is a binary tree, so either $G$ is finite and every path terminates (in which case $\Gamma \cup \{\neg\varphi\}$ is inconsistent, and we have a finite proof of this fact, namely $G$), or $G$ is infinite and has an infinite path $P$, which gives rise to a model $M_P$ of $\Gamma \cup \{\neg\varphi\}$.

In fact it is tempting to define $\Gamma \proves \varphi$ if there is a terminated binary tree $G$, satisfying the properties above, whose root is $\neg\varphi$. It is computable to check that the tree has terminated, so $\proves$ is a finitistic object. With this interpretation the above construction is a proof of the completeness theorem. It is also a proof that weak Koenig's lemma is not constructive, since first-order logic is not decidable.

\section{Motivating Peano arithmetic}
\begin{definition}
A theory $\Gamma$ is \dfn{recursively axiomatizable} if there is a subset of $\Gamma$ which is computable and which generates $\Gamma$.
\end{definition}
\begin{corollary}
The theory of true arithmetic is not recursively axiomatizable.
\end{corollary}
\begin{proof}
If it was, then we could use the completeness theorem to decide truth in $\omega$.
\end{proof}
We now prove something much stronger.
\begin{theorem}[Goedel's first incompleteness theorem]
\index{Goedel's first incompleteness theorem}
There is a fragment $T$ of true arithmetic such that for every $T^* \supseteq T$ such that if $T^*$ is consistent and computable, then $T^*$ is incomplete.
\end{theorem}
\begin{theorem}[Goedel's second incompleteness theorem]
For every $T^* \supseteq T$ which is consistent and computable, $T^*$ does not prove $\Con T^*$.
\end{theorem}

We prove these theorems after setting up the formal theory of incompleteness. Let $\Phi$ be a set of sentences in the language $\{0, 1, +, \times\}$.
\begin{definition}
An $n$-ary relation $D$ on $\omega$ is a \dfn{representable relation} in $\Phi$ if there is a formula $\varphi$ such that for all $m_0,\dots,m_{n-1} \in \omega$,
if $D(m_0, \dots, m_{n-1})$ then $\Phi \proves \varphi(\overline{m_0}, \dots, \overline{m_{n-1}})$, and otherwise, $\Phi \proves \neg\varphi(\overline{m_0}, \dots,\overline{m_{n-1}})$.
A function $f: \omega^n \to \omega$ is a \dfn{representable function} in $\Phi$ if the graph of $f$ is representable in $\Phi$ by a formula $\varphi$, and for every $m_0,\dots,m_{n-1}$, $\Phi \proves \exists!y ~\varphi(\overline{m_0}, \dots, \overline{m_{n-1}}, y)$.
\end{definition}
Here $\exists!$ is ``there is a unique".
\begin{example}
Every computable function and every computable relation is representable in $\Th \NN$. This follows from the $\beta$-lemma. On the other hand, $\Phi$ is inconsistent if and only if every relation and every function is representable in $\Phi$, by a cardinality argument.

If $\Phi$ is consistent and computable then only relations and functions representable in $\Phi$ must also be computable, since if $\Phi$ is computable then we can decide if $\Phi \proves \varphi$ or not.
\end{example}

\section{Peano arithmetic}
We now give a possible recursive axiomatization of arithmetic.
\begin{definition}
\dfn{Peano arithmetic} $\PA$ is the theory consisting of the universal closures of $\neg((x+1)=0)$, $(x+0)=x$, $x\times 0 = 0$, $((x+1)=(y+1)) \to (x=y)$, $(x+(y+1))=((x+y)+1)$, and $(x \times (y+1)) = ((x \times y) + x)$, as well as, for every formula $\varphi$ with a free variable $x$, the axiom
$$(\varphi(0) \wedge \forall x (\varphi(x) \to \varphi(x+1))) \to \forall x \varphi(x).$$
\end{definition}
So $\PA$ implies the usual, set-theoretic principle of induction for definable sets.

Suppose $M \models \PA$. Then we will show that $M$ admits the relation $<$, and there is a $<$-initial segment which is isomorphic in $\{0, 1, +, \times\}$ to $\NN$, the \dfn{standard part} of $M$. If $M$ has further elements beyond its standard part we will refer to it as the \dfn{nonstandard part}. Therefore any existential statement in $\Th \NN$ is already true in $\PA$. We will then show that every computable function and computable relation is representable in $\PA$.

We now prove a bunch of theorems inside $\PA$.
\begin{theorem}[PA]
$\forall x \forall y\forall z~((x+y)+z)=(x+(y+z))$.
\end{theorem}
\begin{proof}
By induction on $z$. If $z = 0$ we have $(x+y)+0= x+y=x+(y+0)$. Now suppose $(x+y)+z = x+(y+z)$, then $(x+y)+(z+1) = ((x+y)+z)+1 = (x+(y+z))+1 = x+((y+z)+1) = x+(y+(z+1))$. Therefore the theorem.
\end{proof}
\begin{theorem}[PA]
$\forall x \forall y ~x+y=y+x$.
\end{theorem}
\begin{proof}
We first prove $\forall x~(x+0=0+x)$ by induction on $x$. $0 + 0 = 0 + 0$ and if $x+0=0+x$ then $(x+1)+0=x+1 = (0+x)+1 = 0+(x+1)$. Therefore the claim.

We now prove $\forall x~(x+1=1+x)$ by induction on $x$. We already know $0+1=1+0$, and if $x+1=1+x$ then $(x+1)+1 = (1+x)+1 = 1+(x+1)$, therefore the claim.

Finally we prove the theorem by induction on $x$. We already know $0+y=y+0$. If $x+y=y+x$ then $(x+1)+y = x+(1+y) = x+(y+1) = (x+y)+1 = (y+x)+1 = y+(x+1)$. Therefore the theorem.
\end{proof}
Henceforth I will omit the proofs of basic facts in $\PA$ unless some trick was involved.
\begin{theorem}[PA]
Multiplication is commutative and associative and distributes over addition. Addition and multiplication are injective.
\end{theorem}
We now begin the proof of the following theorem:
\begin{theorem}
\label{structure theorem for PA}
A certain order relation $<$ is definable in $\PA$ with the following property: Let $M$ be a model of $\PA$. Then a $<$-initial segment of $M$ is $(0, 1, +, \times, <)$-isomorphic to $\omega$.
\end{theorem}
We first show that if $\varphi$ is an equality of arithmetic terms which is true in $\NN$ (e.g. $\varphi: ((1+1+1+1+1+1)=((1+1)\times(1+1+1)))$), then $\PA \proves \varphi$. This requires us to go outside $\PA$ -- we are proving a fact about $\PA$ itself, not proving something from $\PA$.
\begin{lemma}
For any $i,j \in \NN$, if $i \neq j$ then $\PA \proves \overline i \neq \overline j$.
\end{lemma}
\begin{proof}
Without loss of generality we may assume $i < j$ and then induct on $i$. If $i = 0$ then there is a $k$ such that $j = k + 1$, and $\PA \proves \neq(\overline k + 1 = 0)$ so we're done. Otherwise, assume $\PA \proves \overline{i-1} \neq \overline{j-1}$. Therefore $\PA \proves \overline{i-1} + 1 \neq \overline{j-1} + 1$ since $\PA$ proves that addition is injective.
But $\overline i = \overline{i-1} + 1$ by definition and similarly for $\overline j$ so $\PA \proves \overline i \neq \overline j$.
\end{proof}
On the other hand obviously $\PA \proves \overline i = \overline i$. So $\PA$ gets facts about equality of natural numbers correct.
\begin{lemma}
For all $i, j \in \NN$, $\PA \proves \overline i + \overline j = \overline{i+j}$.
\end{lemma}
\begin{proof}
Fix $\NN$ and induct on $j$. If $j = 0$ this follows immediately.
If $\PA \proves \overline i + \overline j = \overline{i+j}$, then $\overline i + \overline{j+1} = \overline i + (\overline j + 1)$ so by associativity $\PA \proves \overline i + \overline{j+1} = (\overline i + \overline j) + 1$ so the claim follows by induction.
\end{proof}
Similarly $\PA$ proves facts about multiplication on $\NN$ correctly.
\begin{definition}[PA]
Let $x_1 < x_2$ be the formula
$$\exists y ~x_1 + (y+1) = x_2.$$
\end{definition}
In other words there is a nonzero number $y + 1$ such that $x_1 + y + 1 = x_2$.
\begin{lemma}[PA]
$\forall y ~(y \neq 0) \to (\exists z ~y = (z+1))$.
\end{lemma}
\begin{proof}
The formula
$$y = 0 \wedge (\exists z ~y = (z+1))$$
satisfies the hypotheses of an induction axiom.
\end{proof}
\begin{lemma}
\label{initial segments in PA}
For every $j \in \NN$,
$$\PA \proves \forall x ~(x < \overline j) \to (x = 0 \vee x = 1 \vee \cdots \vee x = \overline{j-1}).$$
\end{lemma}
\begin{proof}
By induction on $j$. If $j = 0$ we are proving $(x < 0) \to (0 = 1)$ which follows from $\neg(x < 0)$. In fact if $x < 0$ then $\PA \proves \exists y ~x + (y + 1) = 0$ which contradicts an axiom.

If $\PA$ proves the claim for $j$, suppose $x < \overline{j+1}$.
Therefore $\exists y~x+(y+1) = \overline{j+1}$, and fixing that $y$ we have $x + y = \overline j$. If $y = 0$ then $x = \overline j$.
Otherwise $\exists z ~y = z + 1$, so by induction $\PA \proves x = 0 \wedge \cdots \wedge x = \overline{j-1}$.
\end{proof}
\begin{lemma}
For all $i, j \in \NN$, $\PA \overline i < \overline j$, iff $i < j$.
\end{lemma}
\begin{proof}
Suppose $i < j$. Then $j - i \in \NN$, $j - i \neq 0$, and because $\PA$ gets facts about addition correct, $\PA \proves \overline i + (\overline{j-i-1} + 1) = \overline j$.
In particular $\PA \proves \exists y \overline i + (y + 1) = \overline j$.

Conversely, we use Lemma \ref{initial segments in PA}.
\end{proof}
Finally to prove Theorem \ref{structure theorem for PA}, we just need to show that in every model $M$, $(M, <)$ is a chain; then Lemma \ref{initial segments in PA} immediately gives a bijection between a $<$-initial segment and $\NN$, and the other lemmata imply that this bijection preserves $(0, 1, +, \times, <)$.
\begin{lemma}[PA]
$\forall x \forall y ~(x < y) \vee (x = y) \vee (y < x)$.
\end{lemma}
\begin{proof}
By induction. If $x = 0$, $y = 0$, done.
Otherwise if $x = 0$ and $y \neq 0$ then $\exists z~(z+1)=y$, so $\exists z~((0+z)+1) = y$ so $y > x$.

Reasoning inductively, we know $(x < y) \vee (x = y) \vee (y < x)$.
If $y < x$ then $\exists z ~(y+(z+1)) = x$ and then $y < x + 1$.
If $y = x$ then $y + (0 + 1) = x + 1$ so $y < x + 1$.
If $x < y$ then $\exists t ~(x + (t+1))=y$. If $t = 0$ then $y = x + 1$. Else $\exists s ~((x+ 1) + (s+1))=y$ so $y > x + 1$.
\end{proof}
\begin{lemma}[PA]
$\forall x \forall y ~((x < y) \to \neg(x=y)) \wedge (x=y)\to\neg(x < y)$.
\end{lemma}
\begin{proof}
Suppose $x < y$ and $x = y$, then $\exists z~(x+(z+1)) = x$ so $z + 1 = 0$ so $0 = 1$. Similarly for the other conjunct.
\end{proof}
Therefore we have proven Theorem \ref{structure theorem for PA}.

We now state a few more facts about $\PA$ before discussing its models in greater detail.
\begin{lemma}[PA]
$(y < x + 1) \iff ((y < x) \vee (y = x))$.
\end{lemma}
\begin{theorem}[least-number principle]
\index{least-number principle}
For any formula $\varphi$,
$$\PA \proves (\exists x~\varphi(x)) \to (\exists x ~\varphi(x) \wedge \forall y < x ~\neg \varphi(y)).$$
\end{theorem}
\begin{proof}
Assume the consequent fails and work inside $\PA$. Then
$$\forall x(\forall y < x ~\neg \varphi(y)) \to \neg \varphi(x).$$
Let $\psi(x) = \forall y \leq x~\neg \varphi(y)$.
Then trivially $\forall y < 0 ~\neg \varphi(y)$ so $\neg \varphi(0)$ and hence $\psi(0)$.
Suppose $\psi(x)$. Then $y \leq x + 1$ iff $y = x + 1$ or $y \leq x$.
If $y \leq x$ then $\neg\varphi(x+1)$.
So by induction $\forall x ~\psi(x)$, and hence $\forall x~\neg\varphi(x)$.
\end{proof}

Recall that a sentence is $\Sigma_1^0$ if there is an $n$ and a formula $\theta$ with bounded quantifiers (i.e. a $\Delta_0^0$ formula) such that the sentence is
$$\exists x_1 \cdots \exists x_n \theta.$$
Now if $M \subset M'$ are structures in the language of a $\Sigma_1^0$ sentence $\varphi$, $M \models \varphi$, and $M$ is an initial segment of $M'$, then $M' \models \varphi$:
any witnesses to the truth of $\theta$ must appear in $M$ and hence also in $M'$, since all quantifiers are existential or bounded by existential quantifiers.
\begin{corollary}
Let $\varphi$ be a $\Sigma_1^0$ sentence and suppose that $\NN \models \varphi$. Then $\PA \proves \varphi$.
\end{corollary}
\begin{proof}
Let $M$ be a model of $\varphi = \exists x_1 \cdots \exists x_n \theta$, so $\NN$ is an initial segment of $M$. Let $w_1, \dots, w_n$ be witnesses to the truth of $\varphi$ in $\NN$, so
$$\NN \models \theta(w_1, \dots, w_n).$$
Since $\theta$ has bounded quantifiers we also have $M \models \theta(w_1, \dots, w_n)$ whence the claim.
\end{proof}
Now we have already proven that there is a $\Sigma_1^0$ sentence which is not decidable in a computably axiomatizable theory, in particular not in $\PA$.
This corollary therefore implies that any such sentence must be false in $\NN$, so while we cannot disprove that sentence from $\PA$, we can disprove it by reasoning externally about $\PA$.

\begin{corollary}
$\PA$ does not prove the full induction schema.
\end{corollary}
\begin{proof}
Let $M$ be a nonstandard model. Then $\omega \subset M$ and $\omega$ is closed under $0$ and successor.
\end{proof}
\begin{corollary}
$\omega$ is not definable from $\PA$.
\end{corollary}
\begin{proof}
If it was definable, then $\omega$ would be equal to the entire model by the schema of definable induction.
\end{proof}
In particular, second-order logic allows us to consider the full induction schema rather than just the definable schema, and then reasoning internally, unique model of arithmetic up to isomorphism.
But reasoning externally, there are in fact multiple models of second-order arithmetic, because the power set of $\omega$ is not uniquely determined (by Cohen's theorem, for example).
\begin{corollary}
Let $M$ be a nonstandard model, and let $m \in M$ be nonstandard. Then there is a $m' < m$ such that $m'$ is nonstandard. In fact, $m$ is contained in a copy of $\ZZ$ whose elements are nonstandard in $M$, but $m'$ is not in that copy of $\ZZ$.
\end{corollary}
\begin{proof}
Either $m$ or $m+1$ is even. Then either $m/2$ or $(m+1)/2$ exists, so let $m'$ be the one that exists. Then $m' \notin \omega$ because if it was then so would be $m$.
Moreover $m,m/2$ are successors of something, and $m-m/2$ is nonstandard, hence the claim about $\ZZ$.
\end{proof}
In fact, if $M$ is a countable nonstandard model, then there are $\QQ$ many copies of $\ZZ$ after $\omega$. This can be shown using the uniqueness of countable dense linear orders without endpoints.
However, the fact that $M$ has ordertype $\omega + \QQ\ZZ$ does not uniquely determine the arithmetic of $M$.
On the other hand, a theorem of Friedman implies that every nonstandard model $M$ contains an initial segment which is isomorphic to $M$.

In spite of this anomaly, most of elementary number theory, at least those facts which can be established without complex analysis, is provable from $\PA$, because it is just given by algebraic manipulation.

\section{Goedel codes in Peano arithmetic}
Recall that the $\beta$-lemma implies that if $\PA$ can encode pairs, then $\PA$ can code arbitrarily long vectors.
We now show that in fact $\PA$ can code pairs.
\begin{definition}
Define the \dfn{pairing function}
$$J(p, t) = \frac{(p+t)(p+t+1)}{2} + p.$$
\end{definition}
Then $J$ is a bijection $\omega^2 \to \omega$, because $J$ computes the position of $(p, t)$ in the ordering on $\omega^2$ defined by $(a, b) < (c, d)$ iff $a + b < c + d$ or $(a + b) = (c + d) \wedge a < c$.
Moreover, $J$ and the projections of $J^{-1}$ are recursive; and $\PA$ in fact proves that $J$ is a bijection.
\begin{definition}
For every $n \in \omega$, if $J(p, t) = n$, then we let $(n)_0 = p$ and $(n)_1 = t$.
\end{definition}

Given a recursive $D \subseteq \omega^n$, let $\psi(p, t, x)$ be the formula that means that $(p, t)$ encodes, in the sense of the $\beta$-lemma, a sequence of configurations to compute whether $x \in D$, and that in fact this computation proves that $x \in D$.
Then let $\varphi(x)$ be the formula
\begin{align*}\exists i (&\exists p < i \exists t < i~i = J(p,t) \wedge \psi(p, t, x)) \\
(&\wedge \forall j < i \neg(\exists q < j \exists s < j~j=J(s,q) \wedge \psi(q, s, x)))\end{align*}
so $\varphi(x)$ says there is a \emph{minimal} $i$ such that $(i)_0,(i)_1$ codes, in the sense of the $\beta$-lemma, a sequence of configurations to determine whether $x \in D$.

If $m \in D$ then $\varphi(m)$ is true in $\omega$, and $\varphi(m)$ is $\Sigma_0^1$, so $\PA \proves \varphi(\overline m)$.
Conversely, if $m \notin D$, then $\PA$ proves that there is a nonaccepting computation for $\overline m$, coded by $j$, then $\PA$ can prove that for every $i \leq j$, $i$ does not code an accepting computation for $\overline m$. Nothing bigger than $j$ satisfies $\varphi$ so in fact $\PA \proves \neg \varphi(\overline m)$.
Therefore we have proven the following result.
\begin{theorem}
If $D \subseteq \omega^n$ is recursive, then $D$ is $\Sigma_0^1$-definable.
\end{theorem}

To show that a recursive function $f$ is $\Sigma_0^1$-definable, we not only show that the graph of $f$ is $\Sigma_0^1$-definable (which follows from the above) but also must show that $\PA$ proves that the graph satisfies the vertical line test.
Let $f$ be given and let $\varphi(x, y) = \exists w ~\theta(x, y, w)$ be the definition of the graph $\{(x, y) \in \omega^{n-1} \times \omega: f(x) = y\}$ as given by the pairing function.
Then $\theta$ has bounded quantifiers and we define $\varphi^*(x, y)$ by
$$\exists w ~\theta(x, y, w) \wedge \forall z < J(y, w) ~\neg\theta(x, (z)_0, (z)_1).$$
Thus $\varphi^*(x, y)$ iff $y$ is the least $y'$ such that $f(x) = y'$. So the relation defined by $\varphi^*(x, y)$ must be the graph of a function.
By the least number principle, $\PA$ can then prove that there is a $(y, w) \in \omega^2$ such that $\theta(m, y, w)$ and $J(y, w)$ is minimal possible.
That is, $\PA \proves \exists!y~\varphi^*(m, y)$, and it is easy to check that $\PA$ proves that the unique such $y$ is $f(m)$.
Therefore recursive functions are $\Sigma_0^1$-definable.

\section{Goedel's first incompleteness theorem}
\begin{theorem}[Goedel's first incompletess theorem]
\index{Goedel's first incompleteness theorem}
Let $\Phi$ be a computable set of sentences in the language $(0, 1, +, \times)$. If $\Phi$ allows representation of recursive functions and relations, then $\Phi$ is incomplete. In fact, we can explicitly give a sentence $\varphi$ such that $\Phi$ cannot decide $\varphi$.
\end{theorem}
Note that the assumption that the language is $(0, 1, +, \times)$ can be dropped. In fact, the theorem holds for the language of set theory, since the set of Zermelo-Fraenkel axioms without infinity is bi-interpretable with $\PA$.

To prove this theorem, we must internalize self-reference in $\Phi$. Fix a computable enumeration of the set of all formulas in the language $(0, 1, +, \times)$, say $\{\varphi_n\}_n$, and assume that this enumeration is a bijection. If $\varphi$ is a formula, let $n_\varphi$ be such that $\varphi = \varphi_{n_\varphi}$.
We note that since syntatic operations are computable, they correspond to operations on indices. For example, finding $m$ such that $\varphi_{n_1} \to \varphi_{n_2}$ is $\varphi_m$ gives a computable map $(n_1, n_2) \mapsto m$.

\begin{theorem}[Goedel's fixed point theorem]
\index{Goedel's fixed point theorem}
Let $\Phi$ be a set of sentences which allows for representations of computable functions and relations. Then for any formula $\psi$ there is a sentence $\varphi$ such that
$$\Phi \proves (\varphi \iff \psi(\overline{n_\varphi})).$$
\end{theorem}
\begin{proof}
Let $\psi$ be given. Let $F: \omega^2 \to \omega$ be defined by
$$F(m_1, m_2) = \begin{cases}
  n_{\varphi_{m_1}(\overline{m_2})}&\text{, if $\varphi_{m_1}$ has one free variable}\\
  0&\text{ else.}
\end{cases}$$
So if $\varphi_{m_1}$ has one free variable then $F(m_1, m_2)$ is the index of $\varphi_{m_1}(\overline{m_2})$. By assumption $F$ is computable, therefore representable in $\Phi$.
Let $\theta(x)$ represent the graph of $F$, so $\Phi \proves \theta(x)$ iff $F(x_0, x_1) = x_2$.
Let $\beta(x) = \forall z ~\theta(x, x, z) \to \psi(z)$. Thus $\beta(x)$ says that $\psi(F(x, x))$; in other words, the index of $\varphi_x(\overline x)$ satisfies $\psi$.
Let
$$\varphi(x) = \beta(n_\beta).$$
Then $\varphi$ asserts that $\psi$ holds at the index of $\varphi_{n_\beta}(\overline{n_\beta})$.

We first show $\Phi, \varphi \proves \psi(\overline{n_\varphi})$. In fact,
$$\Phi, \varphi \proves \theta(n_\beta, n_\beta, n_{\beta(\overline{n_\beta})}) \to \psi(\overline{n_{\beta(\overline{n_\beta})}}).$$
This is true because if $F(n_\beta, n_\beta) = n_{\beta(\overline{n_\beta})}$ and $\beta(\overline{n_\beta})$ (i.e. $\varphi$) then $\psi(n_{\beta(\overline{n_\beta})})$.
But by definition of $\theta$, and because $F(n_\beta, n_\beta) = n_{\beta(\overline{n_\beta})}$, we have
$$\Phi, \varphi \proves \theta(n_\beta, n_\beta, n_{\beta(\overline{n_\beta})}).$$
Therefore the claim.

We finally show $\Phi, \psi(\overline{n_\varphi}) \proves \varphi$. Now $n_\varphi = n_{\beta(\overline{n_\beta})}$. By definition of $\theta$,
$$\Phi \proves \exists!y~\theta(\overline{n_\beta}, \overline{n_\beta}, y)$$
but also $F(n_\beta, n_\beta) = n_{\beta(\overline{n_\beta})}$. Therefore
$$\Phi \proves \forall z~\theta(\overline{n_\beta}, \overline{n_\beta}, z) \to (z = \overline{n_{\beta(\overline{n_\beta})}})$$
or in other words
$$\Phi \proves \forall z~\theta(\overline{n_\beta}, \overline{n_\beta}, z) \to (z = \overline{n_\varphi}).$$
Hence
$$\Phi, \psi(\overline{n_\varphi}) \proves \forall z~\theta(\overline{n_\beta}, \overline{n_\beta}, z) \to \psi(z)$$
and $\varphi = \forall z~\theta(\overline{n_\beta}, \overline{n_\beta}, z) \to \psi(z)$.
\end{proof}
To give an idea of how this theorem will be used, we will construct $\psi(n)$ to mean ``There is no proof of $\varphi_n$."
The proof seems a little mysterious until we recall the proof of Banach's contraction fixed point theorem: one iterates the contraction until we have found a fixed point. Similarly we iterate $F$ until we find a fixed-point.
We note that in the proof of $\Phi, \psi(\overline{n_\varphi}) \proves \varphi$, we used the fact that $\Phi \proves \exists!y~\theta(\overline{n_\beta}, \overline{n_\beta}, y)$, which is why we went to the pain of requiring that $\Phi$ can actually prove that function representations are single-valued -- something that was not easy to do when $\Phi = \PA$.
Doing this allowed us to replace the $\forall z$ with $z = n_{\beta(\overline{n_\beta})}$ since that is the only $z$ meeting the antecedent.

\begin{corollary}
Suppose $\Phi$ meets the hypotheses of Goedel's fixed point theorem. Let $\Phi^\proves \subseteq \omega$ encode the theory generated by $\Phi$, i.e.
$$\Phi^\proves = \{n_\varphi: \Phi \proves \varphi\}.$$
If $\Phi$ is consistent then $\Phi^\proves$ is not representable in $\Phi$.
\end{corollary}
\begin{proof}
Suppose that $\Phi^\proves$ is representable by $\psi$. Let $\varphi$ be such that
$$\Phi \proves \varphi \iff \neg \psi(\overline{n_\varphi});$$
i.e. $\varphi$ says ``I am not provable from $\Phi$."
Then $\Phi \proves \varphi$ iff $n_\varphi \in \Phi^\proves$ iff $\Phi \proves \psi(\overline{n_\varphi})$ iff $\Phi \proves \neg \varphi$, so $\Phi$ is inconsistent.
\end{proof}
\begin{proof}[Proof of Goedel's first incompleteness theorem]
Suppose that $\Phi$ is complete. Then $\Phi^\proves$ is computable, since $\Phi$ is computable, $\Phi^\proves$ is computably enumerable, and we can just run two programs in parallel to check whether $\varphi \in \Phi^\proves$ or $\neg\varphi \in \Phi^\proves$ since $\Phi$ is complete.
So $\Phi^\proves$ is representable in $\Phi$, which by the previous corollary implies that $\Phi$ is inconsistent.
\end{proof}

\section{Goedel's second incompleteness theorem}
Goedel's second incompleteness theorem is much harder to prove, because to do so requires us to fill in lots of details about how certain proof systems behave. We will not do this, but sketch the proof.

To prepare for the proof, fix a recursively axiomatized extension $\Phi$ of $\PA$. We might as well assume that $\Phi$ is consistent (if not, use the principle of explosion). We will formalize the proof of the first incompleteness theorem within $\Phi$.
Fix an recursive enumeration of all valid proofs from $\Phi$ and let $H(n, m)$ be the recursive predicate which checks whether $m$ encodes the proof of $\varphi_n$.
Let $\varphi_H$ be the formula that represents $H$ in $\Phi$. Let $\Der_\Phi(x)$ be the formula $\exists y~\varphi_H(x, y)$; intuitively $\Der_\Phi(n)$ means there really is a proof of $\varphi_n$.

Fix a formula $\varphi$ such that
$$\Phi \proves \varphi \iff \neg \Der_\Phi(\overline{n_\varphi}).$$
This formula exists by Goedel's fixed point theorem.
\begin{lemma}
$\Phi \not\proves \varphi$.
\end{lemma}
\begin{proof}
If $\Phi \proves \varphi$, then let $m$ encode the proof of $\varphi$; then $H(n_\varphi, m)$. So $\varphi_H(\overline{n_\varphi}, \overline m)$ holds, so does $\Der_\Phi(\overline{n_\varphi})$, so $\Phi \proves \neg\varphi$, a contradiction.
\end{proof}

Now define $\Con \Phi$ to be the sentence $\neg \Der_\Phi(\overline{n_{0=1}})$.
\begin{theorem}[Goedel's second incompleteness theorem]
\index{Goedel's second incompleteness theorem}
If $\Phi$ is consistent then $\Phi \neg \proves \Con \Phi$.
\end{theorem}
By the lemma,
$$\omega \models (\Con \Phi \to \neg \Der_\Phi(\overline{n_\varphi})).$$
Here we need to work in the model $\omega$ to be sure that the definition of a ``proof" is what we intend it to be; there are no proofs where the number of formulae appearing in the proof is nonstandard.
We discard this assumption below.

We now study properties of $\Der_\Phi$.
\begin{lemma}
One has:
\begin{enumerate}
\item If $\Phi \proves \varphi_n$ then $\Phi \proves \Der_\Phi(n)$.
\item $\Phi \proves (\Der_\Phi(\overline{n_{\varphi_{m_1} \to \varphi_{m_2}}}) \to (\Der_\Phi(\overline{m_1}) \to \Der_\Phi(\overline{m_2})))$.
\item $\Phi \proves (\Der_\Phi(\overline{n_\psi}) \wedge \Der_\Phi(\overline{\neg n_\psi})) \to \Der_\Phi(\overline{n_{0=1}})$.
\item $\Phi \proves \Der_\Phi(\overline n) \to \Der_\Phi(\overline{n_{\Der_\Phi(\overline n)}}))$
\end{enumerate}
\end{lemma}
We prove this lemma by inducting on the length of a proof.

\begin{proof}[Proof of Goedel's second incompleteness theorem]
We need to prove that $\Phi \neg\proves \Con \Phi$. We know that $\Phi \proves \varphi \iff \neg \Der_\Phi(\overline{n_\varphi})$.
Then:
\begin{align*}
\Phi \proves &\Der_\Phi(\overline n_{\varphi \to (\Der_\Phi(\overline{n_\varphi}) \to (0=1))})\\
\Phi \proves &\Der_\Phi(\overline{n_\varphi}) \to (\Der_\Phi(\overline{n_{\Der_\Phi(\overline{n_\varphi})}}) \to \Der_\Phi(\overline{n_{0=1}}))\\
\Phi \proves &\Der_\Phi(\overline{n_\varphi}) \to \Der_\Phi(\overline{n_{0=1}})\\
\Phi \proves &\varphi \iff \neg \Der_\Phi(\overline{n_\varphi})\\
\Phi \proves &\varphi \to \neg \Der_\Phi(\overline{n_\varphi})\\
\Phi \proves &\Der_\Phi(\overline{n_{\varphi \to \Der_\Phi(\overline{n_\varphi})}})\\
\Phi \proves &\Der_\Phi(\overline{n_\varphi}) \to \Der_\Phi(\overline{n_{\neg \Der_\Phi(\overline{n_\varphi})}})\\
\Phi \proves &\neg\Der_\Phi(n_{0=1}) \to \neg \Der_\Phi(\overline{n_\varphi})\\
\Phi \proves &\Con \Phi \to \varphi.
\end{align*}
So if $\Phi \proves \Con \Phi$ then $\Phi \proves \varphi$. We know that if this is true then $\Phi \proves \neg\Con \Phi$, a contradiction.
\end{proof}

\chapter{Kleene's recursion theorem}
\section{Kleene's recursion theorem}
\begin{definition}
Fix a recursive enumeration of the partial recursive functions. Let $\varphi_e$ be the $e$th partial recursive function, and let $W_e$ be the image of $\varphi_e$.
\end{definition}
\begin{theorem}[Kleene]
\index{Kleene's recursion theorem}
For every recursive function $f$ there is a fixed point of the map $\varphi_e \mapsto \varphi_{f(e)}$.
\end{theorem}
\begin{proof}
Define
$$\psi_u(x) = \varphi_{\varphi_u(u)}(x)$$
whenever $\varphi_u(u)$ and $\varphi_{\varphi_u(u)}(x)$ are defined. Now let $t(u)$ be the index for $\psi_u$, i.e.
$$\varphi_{t(u)} = \varphi_{\varphi_u(u)}.$$
Since $\psi_u$ is partial recursive, $t(u)$ must exist. Then $f \circ t$ is recursive, say $f \circ t = \varphi_v$. Then
$$\varphi_{t(v)} = \varphi_{\varphi_v(v)} = \varphi_{f \circ t(v)} = \varphi_{f(t(v))}.$$
So $t(v)$ is a fixed point.
\end{proof}
Kleene's recursion theorem is highly mysterious, essentially the same as the Goedel fixed-point theorem.
Most of the time, the fixed point $e$ is just an index for the program that loops forever on all inputs.
Intuitively, $t(v)$ says ``I want f of myself to talk about me."

Though its behavior is mysterious, Kleene's recursion theorem is constructive, because the map $h$ which sends $e$ to a fixed point to $\varphi_e$, i.e.
$$\varphi_{\varphi_e(h(e))} = \varphi_{h(e)}$$
is even recursive. This follows because the fixed point for $\varphi_e$ was obtained by an explicit syntactic manipulation.

\begin{corollary}
There is an $e$ such that $W_e = \{e\}$.
\end{corollary}
\begin{proof}
Let $f$ be such that $\varphi_{f(x)}$ enumerates $\{x\}$. Then $f$ is clearly recursive, and by Kleene's recursion theorem, we can find an $e$ such that $\varphi_e = \varphi_{f(e)}$, so $W_e = W_{f(e)} = \{e\}$.
\end{proof}

\begin{corollary}
The set $\{e: W_e \neq \emptyset\}$ is not recursive.
\end{corollary}
\begin{proof}
Let $g(e)$ be the indicator function, and assume that $g$ is recursive. Let $\varphi_{f(e)}(x)$ be such that if $g(e) = 0$ then $\varphi_{f(e)}(x) = 1$ and otherwise leave $\varphi_{f(e)}(x)$ undefined.
Since we just need to wait for $\varphi_{f(e)}(x)$ to halt to define $\varphi_{f(e)}(x)$, $\varphi_{f(e)}$ is partial recursive, so $f(e)$ exists and is even recursive.
By Kleene's recursion theorem we may assume $\varphi_e = \varphi_{f(e)}$. But then $g(e) = 0$ iff $g(e) = 1$, which is absurd.
\end{proof}

\section{Rice's theorem}
\begin{definition}
The \dfn{index set} for a set of r.e. reals $C$ is the set $P(C) = \{e: W_e \in C\}$.
\end{definition}
\begin{example}
$\NN$ and $\emptyset$ are recursive index sets. Moreover, $\{e: |W_e| < \infty\}$ and $\{e: W_e \text{ is creative }\}$ are index sets, which, as we will see, are not recursive.
\end{example}
We can think of an index set as a set of programs which compute some semantic property. We can also view the complexity of a semantic property in terms of the complexity of the index set.
\begin{theorem}[Rice]
\index{Rice's theorem}
The only recursive index sets are $\NN$ and $\emptyset$.
\end{theorem}
In other words, there are no recursive nontrivial semantic properties.
\begin{proof}
Suppose $P_C$ is neither $\emptyset$ or $\NN$. Let $W_{e_1} \in C, W_{e_2} \notin C$. Assume $P_C$ is recursive.
Let $f$ be the recursive function such that if $e \in P_C$, then $f(e) = e_2$, and if $e \notin P_C$, then $f(e) = e_1$.
By Kleene's recursion theorem, there is an $e$ such that $W_e = W_{f(e)}$. Therefore $e \in P_C$ iff $e \notin P_C$, which is impossible.
\end{proof}
The point that semantic properties are not recursive can be made more strongly. Let $D_e$ be the finite set coded by $e$.
\begin{theorem}[Rice-Shapiro]
\index{Rice-Shapiro theorem}
If $P_C$ is an r.e. index set and $C$ is nonempty, then there is a recursive function $f$ such that $C = \{W_e: \exists x(D_{f(x)} \subseteq W_e)\}$.
\end{theorem}
\begin{proof}
\begin{lemma}
If $A \subseteq B$ and $A \in C$, then $B \in C$.
\end{lemma}
\begin{proof}[Proof of lemma]
Let $f$ be the recursive function such that $W_{f(e)}$ is enumerated according to the following algorithm: While $e$ has not been enumerated into $P_C$, enumerate $A$ into $W_{f(e)}$; once $e$ has been enumerated into $P_C$, then enumerate $B$ into $W_{f(e)}$.
Since $A \subseteq B$, nothing done in the first loop interferes with the second loop.
So if $e \in P_C$, $W_{f(e)} = B$, and otherwise $W_{f(e)} = A$.

Let $e$ be such that $W_e = W_{f(e)}$. If $e \notin P_C$, then $W_e = A$, so $e \in P_C$, a contradiction. So $e \in P_C$, so $W_e = B$, and so $B \in C$.
\end{proof}
\begin{lemma}
If $A \in C$ then there is a finite set $D \subseteq A$ such that $D \in C$.
\end{lemma}
\begin{proof}[Proof of lemma]
Define $f$ to be recursive and such that $W_{f(e)}$ is enumerated by: if $e$ has not been enumerated into $P_C$, enumerate $A$ into $W_{f(e)}$; once $e$ has been enumerated, halt.
By a similar reasoning as in the previous lemma, $e \in P_C$ and $W_e \subseteq A$, yet $W_e$ is finite.
\end{proof}

Now let $g$ be such that $W_{g(x)} = D_x$ and let $f$ enumerate $\{x: g(x) \in P_C\}$.
\end{proof}



\chapter{Semantic properties of arithmetic}
\section{Undecidability in Ramsey theorey}
Most examples of statements that are undecided by $\PA$ can be proven undecidable in less stupid ways than the proof that $\PA \not\proves \Con \PA$. We give an example.
\begin{definition}
Let $m \to (h)_k^e$ be the sentence which says that for every coloring of the set of all subsets of $\{1, \dots, m\}$ of cardinality $e$ with $k$ colors, there is a $H \subseteq \{1, \dots, m\}$ such that every subset of $H$ of cardinality $e$ is monochromatic, and $\card H \geq k$.
\end{definition}

\begin{theorem}[Ramsey]
\index{Ramsey's theorem}
For every $e,k,h$, there is a $m$ such that $m \to (h)_k^e$.
\end{theorem}
This result was a generalization of the pigeonhole principle.

\begin{definition}
Let $m \to^* (h)_k^e$ be the sentence which says that for every coloring of the set of all subsets of $\{1, \dots, m\}$ of cardinality $e$ with $k$ colors, there is a $H \subseteq \{1, \dots, m\}$ such that every subset of $H$ of cardinality $e$ is monochromatic, and $\card H \geq \max(k, \min H)$.
\end{definition}

\begin{definition}
The \dfn{Paris-Harrington principle} is the sentence $\forall e\forall k \forall h \exists m~(m \to^* (h)_k^e)$.
\end{definition}

\begin{theorem}[Paris-Harrington]
\index{Paris-Harrington theorem}
The Paris-Harrington principle is true in $\omega$ and not provable from $\PA$.
\end{theorem}
The idea behind the proof of truth in $\omega$ is that $\aleph_0 \to (\aleph_0)_k^e$ and hence $\aleph_0 \to^* (\aleph_0)^e_k$, which follows by the existence of nonprincipal ultrafilters on $\omega$; one then uses Koenig's lemma, or Tychonoff's theorem, to drop this to a statement to finite sets. However, the Paris-Harrington principle implies that $\PA$ is consistent.

\section{Overspill}
Fix $M \models \PA$ and suppose that $M$ is nonstandard. Recall that $\omega$ is an initial segment of $M$, which is not definable in $M$.

\begin{definition}
A \dfn{cut} $I \subseteq M$ is a nonempty initial segment of $M$ such that for every $x \in I$, $x + 1 \in I$.
\end{definition}
In particular, $\omega \subseteq I$. Let $\PA^-$ be the axioms of $\PA$ except for induction; then a model $N \models \PA^-$ is a model of $\PA$ iff $N$ does not have a definable proper cut. Therefore we have the following pair of results.

\begin{theorem}[overspill]
\index{overspill}
Let $I \subset M$ be a proper cut and $a \in M^k$ a parameter vector. If for every $b \in I$, $M \models \varphi(b, a)$, then there is a $c \notin I$ such that $M \models \varphi(c, a)$.
\end{theorem}
This is clear; if not, $I$ would be definable by $\varphi$.
\begin{theorem}[underspill]
\index{underspill}
Let $I \subset M$ be a proper cut and $a \in M^k$ a parameter vector. If for every $c \notin I$, $M \models \varphi(c, a)$, then there is a $b \in M$ such that $M \models \forall x \geq b~\varphi(x,a)$.
\end{theorem}

As a consequence, we show that while model theory cannot distinguish between models of $\PA$, recursion theory can.
\begin{theorem}[Tennenbaum]
\index{Tennenbaum's theorem}
Suppose that $M$ is countable and $+,\times$ are computable. Then $M \cong \NN$.
\end{theorem}
To prove Tennenbaum's theorem we will need the following lemma, which shows that the weak Koenig's lemma is not constructive.
\begin{lemma}
There is a computable, infinite binary tree $T$ such that no infinite path through $T$ is computable.
\end{lemma}
\begin{proof}
Let $\{\varphi_i\}_i$ be an enumeration of sentences in the language $\{0, 1, +, \times\}$. If $\sigma$ is a finite binary sequence, then $\sigma$ encodes the sentences $\varphi_i$ such that $\sigma_i = 1$ and $\neg \varphi_i$ such that $\sigma_i = 0$.
Let us say that $\sigma$ is consistent with $\PA$ if there is no proof of $0 = 1$ from the sentences encoded by $\sigma$ and the first $|\sigma|$ axioms of $\PA$.
Then it is computable to check whether $\sigma$ is consistent with $\PA$, so let $T$ be the set of all $\sigma$ such that $\sigma$ is consistent with $\PA$. One can easily check that $T$ is a computable, infinite binary tree.

Let $x$ be a path through $T$; then the set of all initial segments of $x$ encodes a complete, consistent extension of $\PA$, which is not recursive by Goedel's first incompleteness theorem.
\end{proof}

\begin{proof}[Proof of Tennenbaum's theorem]
Suppose $M \not \cong \NN$. Let $T^M$ be the subtree of $2^{<M}$ whose definition agrees with the definition of $T$ (so if $M$ were actually $\NN$ then $T^M = T$).

If $\sigma$ is actually a finite binary sequence then $\sigma \in T$ if $\sigma \in T^M$. Moreover, for every $n \in \NN$ there is a $\sigma \in T^M$ such that $|\sigma| = n$. So by overspill, there is a $\sigma^* \in T^M$ such that $|\sigma^*|$ is nonstandard.

If $+,\times$ were computable, then $\sigma^*$ would be computable, yet restricts to an infinite path through $T$.
\end{proof}

\section{Standard systems}
Let $M$ be a nonstandard model.
\begin{definition}
A set $S \subseteq M$ is an \dfn{$M$-finite set} if there are $p,t \in M$ such that the sequence coded by $(p, t)$ according to the $\beta$-lemma encodes $S$.
\end{definition}
\begin{definition}
The \dfn{standard system} of $M$ is the set of all $X \in 2^\omega$ such that there is an $M$-finite set $S$ where $S \cap \omega = X$.
\end{definition}
\begin{definition}
A \dfn{Scott set} is a nonempty set $S \subseteq 2^\omega$ such that if $Y_1, \dots, Y_k \in S$ and $X$ is computable from $Y_1, \dots, Y_k$ then $X \in S$; and if $T$ is an infinite binary tree computed from $Y_1, \dots, Y_k$, then $T$ has an infinite path in $S$.
\end{definition}
So in other words, Scott sets are closed under computability and under weak Koenig's lemma.
\begin{theorem}[Scott]
\index{Scott's theorem}
Every standard system is a Scott set. Conversely, for every Scott set $S$ of cardinality $\leq \aleph_1$, there is a nonstandard model $M$ such that $S$ is the standard system of $M$.
\end{theorem}
Clearly the restriction on cardinality can be dropped if the continuum hypothesis holds. Whether the restriction can be dropped in general is not known.

\section{The arithmetical hierarchy}
We now consider the complexity of a real number (i.e. a set of natural numbers) in terms of the complexity of the formulae that define it.

Recall that if $R$ is a computable relation, then $\PA$ interprets $R$ as a relation definable using only bounded quantifiers.
\begin{definition}
Let $\Sigma_0^0$ be the set of all formulae (logically equivalent to formulae) with only bounded quantifiers. For every $n$, define sets by induction:
\begin{enumerate}
\item $\Sigma_n^0$ is the set of all formulae of the form $\exists x_{i_1} \cdots \exists x_{i_k} \psi$ where $k \geq 0$ and $\psi \in \Pi_{n-1}^0$.
\item $\Pi_n^0$ is the set of all formulae of the form $\neg \psi$ where $\psi \in \Sigma_n^0$.
\item $\Delta_n^0 = \Sigma_n^0 \cap \Pi_n^0$.
\end{enumerate}
\end{definition}
In particular, $\Sigma_0^0 = \Pi_0^0 = \Delta_0^0$ is the set of formulae defining recursive relations.
We will abuse notation and say that a real is $\Sigma_n^0$ if it is definable by a $\Sigma_n^0$ formula.
So a recursive real is $\Sigma_0^0$.

\begin{lemma}
For every $\Sigma_n^0$ formula $\varphi$ and bounded quantifier $\forall x < \tau$, there is a $\Sigma_n^0$ formula $\psi$ such that
$$\omega \models (\forall x<\tau~\varphi) \iff \psi.$$
\end{lemma}
\begin{proof}
We induct on $n$. The case $n = 0$ is obvious. So suppose the claim holds for $n - 1$. Let
$$\varphi = \exists y_1 \cdots \exists y_k \chi$$
where $\chi$ is $\Pi_{n-1}^0$.

The formula $\forall x < \tau~\varphi$ holds iff there is a $y$ which codes a sequence of length $k\tau$ such that
$$\forall x < \tau~\psi(x, y_{xk}, y_{xk + 1}, \dots, y_{xk+k-1}).$$
Now this formula, say $\theta$, has a bounded quantifier followed by a $\Pi_{n-1}^0$ formula.
By induction, $\theta$ is logically equivalent to $\exists y\theta'$ where $\theta'$ is $\Pi_{n-1}^0$.
Now let $\psi = \exists y\theta'$.
\end{proof}
In particular, if $\varphi$ is $\Sigma_0^n$, then $\forall x < \tau~\varphi$ is also $\Sigma_0^n$. We also have the following theorem, left to an exercise.
\begin{theorem}
Suppose that $\varphi$ is $\Sigma_0^n$. Then there is a $\Sigma_0^0$ formula $\psi$ such that $\varphi$ is logically equivalent to
$$\exists x_1 \forall x_2 \exists x_3 \cdots Q_n x_n ~\psi$$
where $Q_n = \exists$ if $n$ is odd and $Q_n = \forall$ if $n$ is even.
\end{theorem}

\begin{definition}
If $\theta$ is a formula, let $[\theta]$ be the \dfn{Goedel number} of $\theta$, the number which codes $\theta$.
\end{definition}
\begin{definition}
Let $S_n^*$ be the set of all $(\theta, a)$ such that $a$ codes a vector $\vec a$, $\NN \models (\theta, \vec a)$, and $\theta$ is $\Sigma_n^0$.
Let $S_n$ be the set of all $([\theta], a)$ with this property.
We call $S_n$ the \dfn{universal $\Sigma_n^0$ set}.
\end{definition}
\begin{theorem}
The universal $\Sigma_n^0$ set is $\Sigma_n^0$.
\end{theorem}
\begin{proof}
By a previous theorem, for every $\theta \in \Sigma_n^0$ there is a $\theta'$ with bounded quantifiers such that
$$\theta = \exists x_1 \forall x_2 \exists x_3 \cdots Q_n x_n \theta'$$
and the map $[\theta] \mapsto [\theta']$ is recursive.
Let $\gamma$ represent that map; then $\gamma$ is $\Sigma_1^0$. Then $([\theta], a) \in S_n$ iff
$$\exists x_1 \forall x_2 \cdots Q_n x_n \theta'(\vec x, \vec a)$$
which happens iff $\exists x_1 \forall x_2 \cdots Q_n x_n$ such that $([\theta'], y) \in S_0$ where $\vec y = (\vec x, \vec a)$,
which happens iff $\exists w$ such that $\gamma([\theta], w)$ and $\exists x_1 \forall x_2 \cdots Q_n x_n$ such that $(w, y) \in S_0$ and $\vec y = (\vec x, \vec a)$.
The statement $\vec y = (\vec x, \vec a)$ is bounded. Since $\gamma([\theta], w)$ is $\Sigma_1^0$,
the statement ``$\exists w$ such that $\gamma([\theta], w)$ and $\exists x_1 \forall x_2 \cdots Q_n x_n$ such that $(w, y) \in S_0$ and $\vec y = (\vec x, \vec a)$" is $\Sigma_n^0$.
\end{proof}

We now show that $S_n$ is not $\Sigma_{n-1}^0$.
\begin{theorem}[Kleene]
\index{Kleene's theorem}
$S_n$ is not $\Pi_n^0$.
\end{theorem}
\begin{proof}
We use the diagonal argument. Suppose $S_n$ was defined by a $\Pi_n^0$ formula $\varphi$. Define $\psi(x)$ to be the formula $-\neg \varphi(x, [(x)])$, which is $\Sigma_n^0$.
Then $\psi([\psi])$ iff $\neg\varphi([\psi], [([\psi])])$. But $\varphi$ defines $S_n$, so $\psi([\psi])$ iff $\varphi([\psi], [([\psi])])$, a contradiction.
\end{proof}

\section{Turing degrees}
\begin{definition}
Let $X,Y$ be reals. Let $X \leq_T Y$ mean that $Y$ is computable relative to $X$.
\end{definition}
One easily checks that $\leq_T$ is a preorder. Let $X \equiv_T Y$ mean that $X \leq_T Y$ and $Y \leq_T X$; then $\equiv_T$ is an equivalence relation. The equivalence classes under $\equiv_T$ are known as \dfn{Turing degrees}; then $\leq_T$ drops to a partial order on the Turing degrees.

Given Turing degrees $X,Y$ let $X \oplus Y$ be the supremum of $\{X, Y\}$ under $\leq_T$. If we choose representatives, then in fact $X \oplus Y$ is the equivalence class of $\{2n: n \in X\} \cup \{2n + 1: n \in Y\}$.
\begin{definition}
Given a real $X$, let $X'$ be the \dfn{Turing jump} of $X$, the set of $m$ such that the $m$th program relative to $X$ halts on empty input. Let $X^{(n)}$ denote the $n$th iterate of the Turing jump of $X$.
\end{definition}
By a straight diagonalization we see that $X <_T X'$. Of course $X'$ can compute $X$, and for the opposite, we consider that the problem of computing $X$ from $X'$ is the halting problem.
Thus given $X$, we obtain an infinite chain of Turing degrees above $X$. We let $0$ denote the minimal Turing degree.
Let $\mathcal D$ denote the set of Turing degrees.
The equivalence class of any set is countable, so $\mathcal D$ has cardinality $2^{\aleph_0}$.

\begin{theorem}[Post]
\index{Post's theorem}
For every $n$ and every real $X$, $X$ is $\Sigma_{n+1}^0$ iff $X$ is r.e. relative to $0^{(n)}$.
\end{theorem}
\begin{proof}
Proof by induction. We already proved the case $n = 0$. So assume that the theorem is true for $n$.

Suppose that $X$ is r.e. relative to $0^{(n+1)}$. Since $0^{(n+1)}$ is r.e. relative to $0^{(n)}$, by induction, $0^{(n+1)}$ is $\Sigma_{n+1}^0$.
Let $\varphi$ define $0^{(n+1)}$, so $\varphi$ is $\Sigma_{n+1}^0$. Let $P$ be the program which enumerates $X$. Then $m \in X$ iff there is a $c$ such that $c$ codes a computation of $P$ with output $m$ such that whenever $x \in 0^{(n+1)}$, $\varphi(x)$, and else $\neg \varphi(x)$.
This is $\Sigma_{n+2}^0$ since an ``if and only if" statement involving $\varphi$ is $\Delta_{n+2}^0$, $c$ is unbounded, and the predicate ``codes a computation" is $\Sigma_0^0$.

Conversely, if $X$ is $\Sigma_{n+2}^0$, suppose that $m \in X$ iff $\exists w ~\varphi(w, m)$ and $\varphi$ is $\Pi_{n+1}^0$. Then $\varphi$ defines a set of pairs which is co-r.e. relative to $0^{(n)}$.
Let $P = \{(w, y): \omega \models \varphi(w, y)\}$. Then $P \leq_T 0^{(n+1)}$, so $0^{(n+1)}$ can enumerate $m \in X$ by searching for a $w$ such that $(w, m) \in P$.
\end{proof}

Now if $P$ is a preorder of cardinality at most $2^{\aleph_0}$ such that for every $p \in P$, $\{q \in P: q > p\}$ is countable, Sacks asked if there was an embedding $P \to \mathcal D$. This is still an open question.

\section{The arithmetical hierarchy in $\PA$}
We now formalize the above results inside $\PA$. Note that here the superscript $0$ in the symbol $\Sigma_n^0$, which is meant to indicate that we are working in first-order logic, is irrelevant, since $\PA$ is already first-order. So we drop the superscripts for the time being.

Recall that if $a$ is a number we let $\vec a$ be the vector coded by $a$.
\begin{theorem}
For every $n \geq 1$ and $Q \in \{\Sigma, \Pi, \Delta\}$ there is a formula $\Sat_{Q_n}$ which is $Q_n$ such that for every $Q_n$ formulae $\varphi$,
$$\PA \proves \forall a~\varphi(\vec a) \iff \Sat_{Q_n}([\varphi], a).$$
\end{theorem}
The proof is in Chapter 9 of Kaye's book ``Models of Peano Arithmetic", but it is quite tedious. We sketch the ideas of the proof:
\begin{enumerate}
\item The vector $\vec a$ can be coded by a number $a$ using the $\beta$-lemma and the $J$-pairing function.
\item There is a definable bijection between logical symbols and numbers, so a bijection between formulae and vectors. We will call the number $a$ such that $\vec a$ maps to a given formula $\psi$ the ``code" of $\psi$, and write $a = [\psi]$.
\item The formula-building operations are definable operations on codes.
\item The relation ``$x_1, \dots, x_n$ are codes for symbols that make up a formula coded by y" is a definable relation $R(x_1, \dots, x_n, y)$.
\item If $x$ is the Goedel number of a term $\tau$ with $n$ free variables and $\vec y$ has $n$ entries, let $V(x, y)$ be the evaluation of $\tau(\vec y)$.
\item There is a formula $\gamma$ such that
$$\PA \proves \forall y_1 \cdots \forall y_n \forall y ((\forall i < n~(\vec y_i = y_i)) \to (\tau(\vec y) = \gamma([\tau], y)).$$
So $\PA$ proves that $\gamma$ correctly represents $V$. This takes infinitely many sentences to state, one for each $\tau$.
\item We now are in the position to define satisfaction for $\Delta_0$ formulae. Let $S(s, x)$ mean that:
\begin{enumerate}
\item $s$ is the code for a $\Delta_0$ formula,
\item $x$ is the code for a sequence of triples $(i, z, w)$ where $i$ is a formula in the sequence coded by $s$ and $w$ is the truth value of that formula when it is evaluated at $z$, and
\item For each $k$ less than the length of the sequence coded by $x$, if $\vec x_k$ is $([\exists y_\ell<\tau\theta]), z, w)$ then for each $u < V([\tau], z)$ there is a $j < k$ such that $\vec x_j$ is $([\theta], z^*, w)$, $z^*_\ell = u$, and $w$ is the truth value.
\end{enumerate}
\item Let $|x|$ be the length of $\vec x$, which is a first-order term in $x$ and let $1$ be the code for the truth value ``True."
\item Let
$$\Sat_{\Sigma_0}(x, y) = \exists s \exists t S(s, t) \wedge \exists j < |s|(\vec s_j = x) \wedge \exists k < |t|(\vec t_k = (j, y, 1)).$$
Here $x$ codes the formula and $y$ codes the free variables.
\item $\Sat_{\Sigma_0}$ is clearly $\Sigma_1$, but either there is a witness that $x$ is satisfied or there is a witness that $x$ is not satisfied, so in fact $\PA \proves \Sat_{\Sigma_0} \in \Delta_1$.
\item By induction, $\PA \proves \Sat_{\Sigma_0}([\varphi], y) \iff \varphi(\vec y)$.
\item One then inductively defines $\Sat_{\Sigma_n}$ inductively by adding existential quantifiers to the front of $\Sat_{\Pi_{n-1}}$.
\end{enumerate}


\section{Submodels of arithmetic}
We prepare for the proof that $\PA$ does not have a finite axiomatization.
\begin{definition}
Let $M$ be a first-order structure and $A \subseteq M$.
Let $K^n(M, A)$ be the substructure of all $x \in M$ such that $x$ is definable from parameters in $A$ by a $\Sigma_n$ formula.
\end{definition}

Let $N \preceq_{\Sigma_n} M$ mean that a $\Sigma_n$ formula holds in $N$ iff it holds in $M$.
So $N$ is an elementary substructure if $N \preceq_{\Sigma_n} M$ for all $n$.
\begin{theorem}
\label{Sigma_n elementary substructure}
If $M$ is a nonstandard model of arithmetic then $K^n(M, A) \preceq_{\Sigma_n} M$.
\end{theorem}
To prove this theorem, we need the theory of the following axiom.
\begin{axiom}[bounding scheme]
\index{bounding scheme}
For every $\varphi$ we have
$$\forall a(\forall x < a \exists w ~\varphi(x, w, a) \to \exists b\forall x < a\exists w<b~\varphi(x, w, a)).$$
\end{axiom}
In other words, every definable function on a finite set has a bounded image.

We let $B$ denote the bounding scheme. Let $B\Sigma_n$ be the bounding scheme for $\Sigma_n$ formulae; let $I\Sigma_n$ be the induction scheme for $\Sigma_n$ formulae.

\begin{theorem}
In $\PA^- + I\Sigma_0$ we have $I\Sigma_n$ if and only if we have $I\Pi_n$. In fact, $\PA^- + I\Sigma_0 + B$ iff $\PA$.

In $\PA^- + I\Sigma_0$ we have $I\Sigma_n \implies B\Sigma_n \implies I\Sigma_{n-1}$.
\end{theorem}
\begin{proof}[Proof that $I\Sigma_n$ implies $I\Pi_n$]
Suppose $n \geq 1$ and $\PA^- + I\Sigma_n$. Let $\varphi$ be $\Pi_n$ and suppose $\forall x(\varphi(x) \to \varphi(x+1))$.
Suppose $\neg \varphi(a)$, which is $\Sigma_n$. Let $\psi(x) = (x > a) \vee \neg \varphi(a-x)$.
Then $\psi(0) \wedge \forall x(\psi(x) \to \psi(x+1))$. So $\forall x \psi(x)$, so $\neg \varphi(0)$.
Therefore induction holds for $\varphi$.
\end{proof}
The proof that $I\Pi_n$ implies $I\Sigma_n$ is similar.
\begin{proof}[Proof that $I\Sigma_0 + B\Sigma_{n+1}$ implies $I\Sigma_n$]
We prove this by induction on $n$. The base case is given. Assume $I\Sigma_n$ and let $\varphi(x) = \exists w~\theta(x, w)$ is $\Sigma_{n+1}$.
Assume $\varphi(0) \wedge \forall x(\varphi(x) \to \varphi(x+1))$. We will show that $\forall a\forall x < a~\varphi(x)$.

Suppose not, so there is an $a$ such that $a$ is larger than a counterexample to $\varphi$. Then
$$\forall x <a\exists w(\theta(x, w) \vee (\neg\varphi(x) \wedge (w=0))).$$
By $B$, there is a $b$ such that
$$\forall x<a\exists w<b(\theta(x, w) \vee (\neg \varphi(x) \wedge (w=0))).$$
Let
$$\varphi'(x) = \exists w<b(\theta(x, w) \vee x \geq a).$$
Then $\varphi'(x)$ is $\Pi_n$ and $\varphi'(0) \wedge \forall x(\varphi(x)\to\varphi(x+1))$.
Since $I\Sigma_n$ implies $I\Pi_n$, $\forall x~\varphi'(x)$, contradicting the definition of $a$.
\end{proof}
\begin{proof}[Proof that $I\Sigma_n$ implies $B\Sigma_n$]
It suffices to prove $B\Pi_n$. Suppose that $\varphi \in \Pi_n$ and
$$\forall x <a\exists w\varphi(x, w, a).$$
Let
$$\psi(z) = \exists b\forall x <z\exists w<b\varphi(x,w,a) \vee z > a.$$
Then $\psi(0)$. If $\psi(z)$ is true, then if $z \geq a$, $z + 1 > a$ so $\psi(z+1)$.
Otherwise, $z < a$ so $\exists w \varphi(z,w,a)$. Let $b'$ be the largest $b$ that works for $\psi(z)$ and $w+1$ where $\varphi(z,w,a)$. So
$$\forall x<z+1\exists w<b'\varphi(x,w,a)$$
so $b'$ is a witness to $\psi(z+1)$. Therefore $I\Sigma_{n+1}$ implies $\forall z~\psi(z)$.
But $\psi(a)$ implies $B$ for $\varphi$.
\end{proof}

The next thing we need is the following relativized variant of the Tarski-Vaught test.
\begin{lemma}
\index{relativized Tarski-Vaught test}
Assume $M_1$ is a substructure of $M_2$. Then $M_1 \preceq_{\Sigma_n} M_2$ iff for every $m_1, \dots, m_n \in M_1$ and $\Sigma_n$ formula $\exists x \varphi$ such that $M_2 \models \exists x \varphi$, there is a $m \in M_1$ such that $M_2 \models \varphi(m, m_1, \dots, m_n)$.
\end{lemma}
This follows immediately from the usual proof of the Tarski-Vaught test. It can be interpeted to mean that every nonempty $\Sigma_n(M_1)$-subset meets $\Sigma_n$.

\begin{proof}[Proof of Theorem \ref{Sigma_n elementary substructure}]
One first checks $K^n(M, A)$ is a substructure. In fact if $x_1, x_2$ are $\Sigma_n$-definable, say by $\theta_{x_1}, \theta_{x_2}$, then $\exists y_1 \exists y_2 \theta_{x_1}(y_1)\wedge\theta_{x_2}(y_2)\wedge (z=y_1+y_2)$ is a definition (with free variable $z$) of $x_1 + x_2$, which is $\Sigma_n$. Similarly for multiplication and $A$.

Now suppose $\exists w\varphi(x, w)$ is a $\Sigma_n$ formula and $M \models \exists x \exists w\varphi(x, w, b_1, \dots, b_m)$ where the $b_i \in K^n(M, A)$, say $\psi_i(b_i)$. Then
$$M \models \exists x \exists w \exists \vec y(\bigwedge_i \psi_i(y_i) \wedge \varphi(x, w, \vec y) \wedge \forall x^*<x \neg \exists w^*\leq w\varphi(x^*, w^*, \vec y));$$
by the definable well-ordering principle. The sequence $(x, w, \vec y)$ is unique and the formula is in $\Sigma_n$, so $(x, w, \vec y)$ consists of elements of $K^n(M, A)$. Thus the claim follows by the relativized Tarski-Vaught test.
\end{proof}

\begin{theorem}
For every nonstandard model $M$ and $n$, there is a $K \preceq_{\Sigma_n} M$ such that $K \not\models \PA$.
\end{theorem}
\begin{proof}
Let $a \in M$ be nonstandard, $K_0 = \{a\}$. Let $K_{n+1}$ be the $M$-finite set containing $M_i$ and all $z$ such that there is an $x < a$ and a $y$ such that $y$ codes a sequence of $m$ elements from $M_i$, $z$ is unique, $\Sat_{\Sigma_n}(x, (\vec y, z))$.
So $K_{n+1}$ consists of the things that $M$ thinks can be definable from the first $a$ $\Sigma_n$ formulae and parameters from $K_n$.
This is why $M$ thinks that $K_{n+1}$ is finite.
Let $K$ be the limit of the $K_n$ so $K^n(M, K) = K$, so $K \preceq_{\Sigma_n} M$.

Then $K_n$ is definable in $K$ (relative to finite subsets of $K$), since it is definable in $M$ and $K$ and $M$ agree on existence and uniqueness for solutions to $\Sat_{\Sigma_n}$.
Therefore $n \in \NN$ iff $K$ thinks there is a code for $K_n$, so $\NN$ is definable in $K$. (In $M$ you could run this construction for any $M$-finite length, so if $m$ is nonstandard then $K_m$ exists. But this is impossible in $K$, since $K$ consists of only the things we constructed at a finite stage.)
Therefore $K$ does not model $\PA$.
\end{proof}

\begin{theorem}[Ryll-Nardzewski]
\index{Ryll-Nardzewski theorem}
$\PA$ does not have a consistent, finitely axiomatizable extension.
\end{theorem}
\begin{proof}
If $T$ is such a theory, let $n$ be so large that every axiom of $T$ is $\Sigma_n$. This is possible since $T$ is finitely axiomatizable. Let $M \models T$ be nonstandard and let $K \preceq_{\Sigma_n} M$ not model $\PA$. Then $K \models T$, a contradiction.
\end{proof}

Originally we developed the theory of $\Sat_{\Sigma_n}$ to show that the arithmetic hierarchy doesn't collapse at any finite stage. However, the Ryll-Nardzewski theorem is a nice application of this theory that isn't just a statement about the arithmetic hierarchy.

\section{Slaman's theorem}
\begin{axiom}[$I\Delta_n$]
For all $\Sigma_n$ formulas $\varphi,\psi$,
$$\forall x(\varphi(x)\iff\neg\psi(x))\to (\varphi(0)\wedge\forall x(\varphi(x)\to\varphi(x+1)))\to\forall x\varphi(x).$$
\end{axiom}
\begin{axiom}[$EXP$]
$\forall x\forall y\exists z(x^y=z)$.
\end{axiom}
\begin{theorem}[Slaman]
\index{Slaman's theorem}
Work in $\PA^-+I\Sigma_0+EXP$. For all $n \geq 1$, $I\Delta_n$ iff $B\Sigma_n$.
\end{theorem}
The clause $x^y=z$ is a definable, $\Sigma_0$ relation since exponentation is recursive. So $EXP$ is a $\Pi_2$ formula, and in fact follows from $I\Sigma_1$. Thus the very similar theorem that says that if $\PA^-+I\Sigma_1$ implies that for all $n \geq 2$, $I\Delta_n \Leftrightarrow B\Sigma_n$ is accurate. It is an open problem to remove the assumption on $EXP$ in the case $n = 1$.

The idea behind the trick in the proof of $B\Sigma_n \implies I\Delta_n$ is that if we knew the supremum $M(N)$ of the runtimes of all halting programs in the first $N$ programs, we can decide $0' \cap \{0, \dots, N-1\}$ by running each program until time $M(N)$ and seeing if it has halted yet.
\begin{proof}[Proof of $B\Sigma_n \implies I\Delta_n$]
Let $\varphi,\psi$ be $\Sigma_n$ and suppose that $\varphi$ and $\neg\psi$ define the set $J$ from parameters $p$. We must show that $J$ satisfies $I\Delta_n$.

Suppose $\varphi = \exists y\varphi_0$ and similarly for $\psi$. Suppose that $a$ is strictly above an element of $J^c$. Since $\varphi \vee \psi$ is always true,
$$M \models \forall x < a\exists y\varphi_0(x, y, p) \vee \psi_0(x, y, p)$$
so by $B\Sigma_n$, there is a $b \in M$ such that0
$$M \models \forall x < a\exists y<b\varphi_0(x, y, p) \vee \psi_0(x, y, p)$$
Then, under $a$, $J$ is defined by
$$\exists y<b\varphi_0(x, y, p).$$
But $B\Sigma_n$ allows us to absorb the bounded quantifier into $\varphi_0$, so $J$ is defined by a $\Pi_{n-1}$ formula.
Now $B\Sigma_n \implies I\Pi_{n-1}$, so if $J$ satisfies the hypotheses of the principle of induction, $a$ does not exist, a contradiction. Therefore $J$ cannot be a counterexample of $I\Delta_n$.
\end{proof}
Now recall that a function is cofinal if its image is unbounded. To show that $I \Delta_n \implies B\Sigma_n$, we first construct a function that explicitly demonstrates the failure of $B\Sigma_1$, if it is actually false.
\begin{lemma}
Let $M \models \PA^- + I\Delta_1 + EXP + \neg B\Sigma_1$. Then there is a $a \in M$ and $f: [0, a) \to M$ such that $f$ is injective, $f$ is cofinal, and $f$ is $\Sigma_0$ relative to $M$.
\end{lemma}
\begin{proof}
Let $a \in M$, $p \in M$, and $\varphi = \exists w \varphi_0$, where $\varphi_0$ is $\Sigma_0$, and
$$M \models \forall x<a \exists y\exists w\varphi_0(x, y, w, p)$$
 and
$$M \models \forall s \exists x<a\exists y<s\exists w<s\neg\varphi_0(x, y, w, p).$$
This is what it means for $M \not \models B\Sigma_1$.

Given $x<a$, let $f(x)$ be the code of $(x, s_x)$ where $s_x$ is the least $s$ such that
$$M \models \exists y<s\exists w<s\varphi_0(x, y, w, p).$$
The existence of $s$ follows from
$$M \models \forall x<a \exists y\exists w\varphi_0(x, y, w, p)$$
and the existence of $s_x$ follows from $I\Delta_1$.

Now $x$ is determined by the first coordinate of $f(x)$ so $f$ is injective, and because
$$M \models \forall s \exists x<a\exists y<s\exists w<s\neg\varphi_0(x, y, w, p),$$
$f$ is cofinal. Moreover, the definition of $f$ is $\Sigma_0$.
\end{proof}
\begin{lemma}
Let $M \models \PA^- + I\Delta_1 + EXP + \neg B\Sigma_1$. Then there is a $a \in M$, a nonprincipal $\Sigma_1$-cut $I \subset [0, a)$, and a $g: I \to M$ such that $g$ is $\Sigma_1$ from $M$, and:
\begin{enumerate}
\item For every $i \in I$, $g(i)$ is the code for a sequence $\{m_j: j < i\}$ such that $j \mapsto m_j$ is injective and $m_j < a$.
\item For every $i_1 < i_2 \in I$, $g(i_2)$ is an end-extension of $g(i_1)$.
\item For every $m < a$, there is an $i \in I$ such that $m$ appears in $g(i)$.
\end{enumerate}
\end{lemma}
So the sequence $g(i)$ is an enumeration of $i$ many numbers under $a$, and $g(\omega)$ is an enumeration of the numbers under $a$. Recall that a cut is nonprincipal if it has no greatest element.
\begin{proof}
Let $(f, a)$ be given by the previous lemma. Then $[0, a)$ is $M$-finite, so is $M$-r.e. Then $m$ is enumerated into this set at $f(m)$. So let $g$ satisfy the claimed properties, and let the new entry $m$ in the sequence $g(s)$ at stage $s$ be the $m$ such that $f(m) = s$, which is possible since $f$ is injective.
Now let $I = \dom g$. In order to enumerate $s + 1$, we must have already enumerated $s$, so $I$ is an initial segment of $M$.

If $i > a$ and $i \in I$ then we have enumerated $a$ many numbers less than $a$, yet $\PA^- + I\Delta_1 + EXP$ proves the $\Sigma_0$-pigeonhole principle, so this is absurd. So $I \subset [0, a)$.

Since $f$ is cofinal and $I$ has the same ordertype as $f([0, a))$ it follows that $I$ has no greatest element. So $I$ is a proper nonprincipal cut, and is $\Sigma_1$ since it is the domain of the $\Sigma_1$ function $g$.

Finally, if $m < a$ and $i \in I$ large enough, $m$ appears in $g(i)$, since $m \in \dom f$.
\end{proof}
The cut $I$ above is a counterexample to $I\Sigma_1$ but we actually want it to be a counterexample to $I\Delta_1$. Moreover, $M$ thinks that $m_i$ has no finite end extension, by the following lemma.
\begin{lemma}
Let $a$ and $m_i$ be as in the previous lemma. Let $c \in M$, $n$ a sequence of length $c$ of elements $< a$. Then either $c$ is not an upper bound for $I$ or there is an $i \in I$ such that $n_i \neq m_i$.
\end{lemma}
\begin{proof}
Suppose $c$ is an upper bound on $I$. If $m_i = n_i$ for all $i \in I$, let
$$J = \{j < c: \forall i < j \neg(n_i = n_j)\}.$$
Then $J$ is $\Sigma_0$, and if we can show that $J$ is a cut, then $J$ will be a counterexample to $I\Delta_1$, which is a contradiction.

Given $i \in I$, $g(i)$ is an injective sequence, so $I \subseteq J$. But if $x < a$ then $x = m_i$ for some $i \in I$, so it follows that $J \subseteq I$. But $I$ is a proper cut.
\end{proof}
\begin{proof}[Proof of $I\Delta_1+EXP \implies B\Sigma_1$]
Let $M \models \PA^-+I\Delta_1+EXP+\neg B\Sigma_1$. Let $I, g, m$ be as above.

Let $c_0 = 0$ and $d_0 = a^a$. Let $c_i = \sum_{j<i} m_ja^{-(j+1)}$ and $d_i=c_i+a^{a-i}$. Then $[c_{i+1}, d_{i+1}] \subseteq [c_i, d_i]$.
Viewing elements of $M$ as in base $a$, the $i$th interval specifies the $i$th digit of the number in $\bigcap_i [c_i, d_i]$.
Let $J = \{x: \exists i( x\leq c_i)\}$ and $K = \{x: \exists i(x \geq c_i)\}$. Then $J,K$ are $\Sigma_1$, since the intervals are.
Moreover $J$ is closed downwards, $K$ is closed upwards, and $J$ has no greatest element (because neither does $I$, so that the $c_i$ is eventually a strictly increasing sequence). Clearly $J \cap K$ is empty.

We claim $M = J \cup K$. If not, there is a $n \in \bigcap_i [c_i, d_i]$. Then the base-$a$ representation of $n$ is given by the entries of $m$. Therefore $n$ can be computed from $m$, so by $I\Delta_1$, the entries of $n$ are coded by $M$, which contradicts a previous lemma.

So the decomposition $M = J + K$ implies that $J$ is a $\Delta_1$-cut, so by $I\Delta_1$, $J = M$, a contradiction since $K$ is nonempty.
\end{proof}
\begin{conjecture}[Paris-Wilkie]
\index{Paris-Wilkie conjecture}
$\PA^-+I\Sigma_0+\neg B\Sigma_1 \proves EXP$.
\end{conjecture}
This problem has been open for three decades. If the Paris-Wilkie conjecture were true we could remove the assumption on $EXP$ from Slaman's theorem.

\chapter{The Turing degrees}
\section{Myhill's theorem}
\begin{definition}
Let $X, Y$ be reals. Say that $Y \leq_m X$, i.e. $Y$ is \dfn{many-one reducible} to $X$, if there is a recursive function $h$ such that $h(X) = Y$ and $h^{-1}(Y) = X$.
Say that $Y \leq_1 X$, i.e. $Y$ is \dfn{one-one reducible} to $X$, if there is an injective recursive function $h$ such that $h(X) = Y$ and $h^{-1}(Y) = X$.
\end{definition}
Inutitively a many-one reduction $h$ is a recursive homomorphism $X \to Y$.
\begin{definition}
Two reals $X, Y$ are \dfn{recursively isomorphic} if there is a recursive bijection $h: \omega \to \omega$ such that $h(X) = Y$ and $h^{-1}(Y) = X$.
\end{definition}
We now prove a recursive version of the Schroeder-Bernstein theorem.
\begin{theorem}[Myhill]
\index{Myhill's theorem}
If $X \leq_1 Y$ and $Y \leq_1 X$ then $X,Y$ are recursively isomorphic.
\end{theorem}
\begin{proof}
Suppose $f(X) = Y$ and $g(Y) = X$. We construct a bijection $h$ in stages.

At stage $s$, if $s \notin \dom h_s$, look at the sequences $x_0 = s$, $y_n = f(x_n)$, $x_{n+1} = h_s^{-1}(y_n)$. Let $h_{s+1}(s)$ be the first $y_i$ such that $y_i \notin h_s(\dom h_s)$.
A similar argument with $h_s$ and $h_s^{-1}$ swapped, and $f$ and $g$ swapped, allows us to also guarantee $s \in h_{s+1}(\dom h_{s+1})$.

Clearly $h$ is a bijection $\omega \to \omega$. By induction on $s$ we see that $h(X) = Y$ and $h^{-1}(Y) = X$.
\end{proof}

\begin{corollary}
$\Sat_{\Sigma_1}$ is recursively isomorphic to $0'$.
\end{corollary}
\begin{proof}
Here we view $\Sat_{\Sigma_1}$ as $\{([\theta], x)\}$ and $0' = \{(e, m)\}$ where $e$ is a program and $m$ is an input. Given a program $e$, choose $\theta$ so that $\varphi_e(m)$ exists whenever $\NN \models \theta(x)$.
This gives a reduction $ \Sat_{\Sigma_1}\leq_1 0'$. Arguing similarly, we have a reduction $0' \leq \Sat_{\Sigma_1}$.
\end{proof}

\begin{definition}
A r.e. real $X$ is \dfn{$1$-complete} if for all r.e. sets $W$, $W \leq_1 X$. Similarly, $X$ is \dfn{$m$-complete} if for all r.e. sets $X$, $W \leq_m X$.
\end{definition}
By Myhill's theorem, all $1$-complete sets are recursively isomorphic.

\begin{example}
$\Sat_{\Sigma_1}$ is $1$-complete. In fact, given $W$, there is a $\Sigma_1$ definition $\theta$ of $W$, but $\theta(x)$ is equivalent to $([\theta], n) \in \Sat_{\Sigma_1}$. Thus $n \mapsto ([\theta], n)$ is a $1$-$1$ reduction.
\end{example}
So the $1$-complete sets are exactly those which are isomorphic to $\Sat_{\Sigma_1}$.

\begin{theorem}[Matiyasevich-Robinson-Davis-Putnam]
\index{Matiyasevich-Robinson-Davis-Putnam theorem}
\index{Hilbert's tenth problem}
The set of polynomials with solutions in $\ZZ^n$ is $1$-complete.
\end{theorem}
The proof of this is beyond the scope of these notes but it is worth noting. It shows that Hilbert's tenth problem, which asked for an algorithm to find solutions to polynomial equations or decide that they are unsolvable, has no answer.

\begin{example}
$\Fin = \{e: |W_e| < \infty\}$ is $\Sigma_2$-complete.
\end{example}
\begin{proof}
$|W_e| < \infty$ iff $\exists b\forall n\forall s(n \in W_e[s]\implies n < b)$. This is clearly $\Sigma_2$.

Let $\theta$ be bounded and $\Sat_{\Sigma_2}(e, n) = \exists x\forall y\theta(e, n, x, y)$. Let $f$ be recursive and enumerate $W_{f(e, n)}$ such that at stage $s$, if $k < s$, $k \notin W_{f(e)}[s]$ and $\forall x < k\exists y < s \neg\theta(e, n, x, y)$, enumerate $k$ into $W_{f(e, n)}$.
Then $W_{f(e, n)} = \{k: \forall x < k \exists y\neg \theta(e, n, x, y)$. So $W_{f(e, n)}$ is finite iff $\exists x\forall y\theta(e, n, x, y)$. So $f$ reduces $\Sat_{\Sigma_2}$ to $\Fin$.
\end{proof}
\begin{corollary}
$\Tot = \{e: W_e = \NN\}$ is $\Pi_2$-complete.
\end{corollary}

\begin{example}
Being a recursive convergent sequence is $\Pi_3$-complete. More precisely, consider the recursive functions $\varphi_e: A \to \QQ$ where $A \subseteq \NN$. Let $C$ be the set of $e$ such that $\varphi_e$ is total and $\lim_{n \to \infty} \varphi_e(n)$ exists. Then $C$ is $\Pi_3$-complete.
\end{example}
\begin{proof}
Whether $e \in \Tot$ is $\Pi_2$. The limit exists iff the sequence $\varphi_e(n)$ is Cauchy, which is $\Pi_3$.

To reduce $\Sat_{\Pi_3}$ to $C$, let $\theta$ be bounded and $\Sat_{\Sigma_3}(e, n) = \forall x \exists y \forall z \theta(e,n,x,y,z)$.
At stage $s$, let $x < s$. Let $y(x, s)$ be the least $y < s$ such that $\forall z < s\theta(e, n, x, y, z)$ if one such exists, or $y = s$ otherwise. So $y$ is the ``least candidate for a witness that is viable at stage $s$."
Let $p(x, s)$ be the cardinality of $\{t < s: y(x, t) \neq y(x, t + 1)\}$. Then $\Sat_{\Pi_3}(e, n)$ iff  $\forall x \exists y$ such that $\forall z \theta(e, n, x, y, z)$, which would mean that $y(x, s)$ and $p(x, s)$ are Cauchy in $s$, since they converge to the witness $y$ and the number of times we changed our mind about the witness $y$, respectively.

Define $f$ so that
$$\varphi_{f(e, n)}(s) = \sum_{x=1}^s \frac{(-1)^{p(x, s)}}{4^x}.$$
Then $f$ is recursive, and if for every $x$ $p(x, s)$ has a limit, $\varphi_{f(e, n)}(s)$ converges.
But if not, there is an $x_0$ which is least possible and $p(x_0, s)$ is not Cauchy. Let
$$a = \lim_{s \to \infty} \sum_{x = 1}^{x_0 - 1} \frac{(-1)^{p(x, s)}}{4^x}.$$
If $p(x, s)$ is even then $\varphi_{f(e)}(s) \geq a + 2/3\cdot 4^{x_0}$ but otherwise $\varphi_{f(e)}(s) \leq a - 2/3\cdot 4^{x_0}$.
So $f$ reduces $\Sat_{\Pi_3}$ to $C$.
\end{proof}
So the representation of convergence as Cauchyness is optimal in quantifier complexity.

\begin{corollary}
The sets of cofinite, or recursive, r.e. sets, are $\Sigma_3$-complete.
\end{corollary}


\section{Completeness}
\begin{theorem}
For every r.e. real $A$, $A$ is $m$-complete iff $A$ is $1$-complete.
\end{theorem}
\begin{proof}
Let $\{D_i\}_i$ be a recursive enumeration of the finite subsets of $\omega$. Let $f$ be a $m$-reduction from $\Sat_{\Sigma_1}$ to $A$.
We must find a $1$-reduction from $\Sat_{\Sigma_1}$ to $A$.

\begin{lemma}
There is a recursive function $g$ such that if $\emptyset \subset D_x \subset A$, then $g(x) \in A \setminus D_x$, and if $\emptyset \subset D_x \subset A^c$, then $g(x) \in A^c \setminus D_x$.
\end{lemma}
\begin{proof}
Let $\theta_x(y) = (f(y, y) \in D_x) \vee \exists z(z \in D_x \cap A)$. This definition makes sense because $D_x$ is a finite subset of $\omega$. Note if $D_x \cap A$ is nonempty then $\forall y\theta_x(y)$.
The function $x \mapsto [\theta_x]$ is recursive since $x \mapsto D_x$ is. If $f([\theta_x], [\theta_x]) \in D_x$, let $g(x) = f([\theta_x], [\theta_x])$. Otherwise, let $g(x)$ be the first element enumerated into $A \setminus D_x$.
Since $\Sat_{\Sigma_1} \leq_m A$, $A$ is infinite, so $g$ is well-defined, and $g(x) \notin D_x$.

Assume that $D_x$ is nonempty. If $D_x \subset A$, then $\NN \models \theta_x([\theta_x])$ since $\NN \models \forall y \theta_x(y)$. Therefore $([\theta_x], [\theta_x]) \in \Sat_{\Sigma_1}$, so $f([\theta_x], [\theta_x]) \in A$. So $g(x) \in A$.
Otherwise, if $D_x \subset A^c$, either $f([\theta_x], [\theta_x]) \in D_x$ or not. If so, then $\NN \models \theta_x([\theta_x])$, so $f([\theta_x], [\theta_x]) \in A$, which contradicts the assumption.
So $f([\theta_x], [\theta_x]) \notin D_x^c$ and hence $g(x) = f([\theta_x], [\theta_x])$, so $g(x) \in A^c \setminus D_x$.
\end{proof}

Now assume we have an injective reduction $h: \Sat_{\Sigma_1} \leq_1 A$ defined for all $z < x$.
Let $y_0 = f(x)$. If $y_0 \notin \{h(z): z < x\}$, let $h(x) = y_0$. Otherwise, induct.
Let $x_i$ be defined by $D_{x_i} = \{y_j: j \leq i\}$, and let $y_{i+1} = g(x_i)$.
If $y_{i+1} \not \{h(z): z < x\}$, let $h(x) = y_{i+1}$. Else, continue the induction..

This defines an injective recursive function $h$. In fact $h(x)$ is $y_i$ where $i \leq x$, so $h(x)$ must exist. Moreover $x \in \Sat_{\Sigma_1}$ iff $y_i \in A$, by definition of $g$.
\end{proof}
The technical lemma above is quite remarkable because $A$ is not recursive, yet the recursive function $g$ can decide whether $x \in A$ provided that $D_x \neq \emptyset$.

Now we will give an r.e. set which is neither recursive nor $m$-complete.
\begin{definition}
An r.e. set $A$ is \dfn{simple} if $A^c$ is infinite and $A^c$ has no infinite r.e. subset.
\end{definition}
\begin{lemma}
If $A$ is simple then $A$ is neither recursive nor $m$-complete.
\end{lemma}
\begin{proof}
Since $A$ is not co-r.e., $A$ is not recursive. If $A$ were $m$-complete, then by the above theory, $A$ would be isomorphic to $\Sat_{\Sigma_1}$. But $\Sat_{\Sigma_1}^c$ has an infinite recursive subset.
\end{proof}
\begin{theorem}[Post]
\index{Post's simple set theorem}
There is a simple set.
\end{theorem}
\begin{proof}
Let $\{W_e\}_e$ be a recursive enumeration of the r.e. sets. We say that $x \in W_k[s]$ if $x$ is enumerated into $W_k$ by time $s$.

At time $s$, say that $W_k$ requires \dfn{attention} if $k \leq s$ and there is a $x > 2k$ such that $x \in W_k[s]$ and no element of $W_k[s]$ has been enumerated into $A$ before time $s$.
If any r.e. set requires attention, let $k$ be the minimal possible index where $W_k$ requires attention, enumerate the least $x > 2k$ such that $x \in W_k[s]$ into $A$.

By definition, $A$ is r.e. There are at most $k$ many elements of $A$ below $2k$, so $A^c$ is infinite.
If $W_k$ is infinite, then there is an $x > 2k$ and $s$ such that $x \in W_k[s]$. If no element of $W_k[s]$ has been added to $A$ by time $s$, then $W_k$ required attention at time $s$. So som element of $W_k$ will have been enumerated by time $s + k + 1$. Therefore $A \cap W_k$ is nonempty. Therefore $A^c$ does not contain $W_k$.
\end{proof}

So there is a set $A$ such that $0 <_m < A <_m 0'$. But this not does imply that there is a set $A$ such that $0 <_T A <_T 0'$. Post asked whether this was true.

\section{Creative sets}
Recall that $W_x$ is the set enumerated by the $x$th program; that is, $W_x$ is the image of $\varphi_x$. We may assume that $\varphi_x$ is total and injective by ``speeding up" an enumeration, providing that $W_x$ is infinite.
\begin{definition}
A real $P$ is a \dfn{productive set} if there is a partial recursive function $h$ such that if $W_x \subseteq P$ then $x \in \dom h$ and $h(x) \in P \setminus W_x$. We call $h$ a \dfn{production}.
\end{definition}
In other words, productive sets are far from r.e. So in particular, productive sets cannot be recursive.
\begin{definition}
A real $A$ is a \dfn{creative set} if $A$ is r.e. and $A^c$ is productive.
\end{definition}
Therefore creative sets are not recursive. They also cannot be simple: $A^c$ is productive and hence has infinite r.e. subsets.
The intuition is that the production of $A^c$ is a recursive way to show that $A$ is not recursive, which is much stronger than simply not being recursive.
According to Post, the interesting parts of recursion theory are in co-r.e. sets. One would have to be quite creative to understand the complement of a creative set.
\begin{example}
Let $S = \{x: x \in W_x\}$. Then $S$ is creative.
\end{example}
\begin{proof}
The statement $x \in W_x$ can be expressed by saying $\exists s\exists y(\varphi_{s,x}(x) = y)$ which is $\Sigma_1$.

We must show that $S^c$ is productive. Let $h(x) = x$. Then if $x \in W_x$, $W_x$ is not a subset of $S^c$ since $x \in S$. If $x \notin W_x$, then $x \in S^c \setminus W_x$. So $h$ is a production.
\end{proof}
\begin{theorem}[Myhill]
\index{Myhill's theorem}
A real $A$ is creative iff $A$ is $m$-complete.
\end{theorem}
\begin{proof}
Let $A$ be r.e.

First suppose $A$ is $m$-complete. Then the creative set $S$ above satifies $S \leq_m A$. Let $f$ be recursive such that $n \in S$ iff $f(n) \in A$.
Given $W_x$, let
$$W_{g(x)} = \{n: f(n) \in W_x\}.$$
So if $W_x \subseteq A^c$, then $W_{g(x)} \subseteq S^c$. Since the identity is a production for $S^c$, if $W_x \subseteq A^c$ then $W_{g(x)} \subseteq S^c$, so $g(x) \in S^c \setminus W_{g(x)}$, so $f(g(x)) \in A^c \setminus W_x$.
Therefore $f \circ g$ is a production for $A^c$. Therefore $A$ is creative.

If $A$ is creative, and $h$ is a production for $A^c$, and $B$ is r.e., given $e,y$ let $W_{f(e,y)} = \{h(e)\}$ if $y \in B$ or $\emptyset$ otherwise.
Here $\varphi_{f(e, y)}$ is defined by running the enumeration of $B$ until $y$ is enumerated, then halting and returning $h(e)$.
Use the Kleene recursion theorem to find $e_y$ such that $W_{f(e_y, y)} = W_{e_y}$. The function $y \mapsto e_y$ is recursive, and either $W_{e_y} = \emptyset$ or $h(e_y) \in W_{e_y}$.

If $W_{e_y} = \emptyset$ then $W_{e_y} \subseteq A^c$ so $h(e_y) \in A^c$. Otherwise, $W_{e_y}$ is not contained in $A^c$ so $h(e_y) \in A$. Thus $y \in B$ iff $h(e_y) \in A$, so $B \leq_m A$.
\end{proof}
Thus $A$ is creative iff $A$ is $m$-complete iff $A$ is $1$-complete iff $A$ is isomorphic to $\Sat_{\Sigma_1}$.

\begin{definition}
An r.e. set $A$ is \dfn{Turing complete} if $\Sat_{\Sigma_1} \leq_T A$.
\end{definition}
\begin{theorem}
There is a simple set which is Turing complete.
\end{theorem}
\begin{proof}
Fix an enumeration of $\Sat_{\Sigma_1}$ which enumerates exactly one number at each stage and enumerates each number exactly once.
Let $\Sat_{\Sigma_1}[s]$ denote the first $s$ elements of $\Sat_{\Sigma_1}$ when enumerated this way.
Assume $0 \notin \Sat_{\sigma_1}$.

We will find a real $A$ such that $A^c$ is infinite, if $W_k$ is infinite then $W_k \cap A$ is nonempty, and if $n$ is enumerated into $\Sat_{\Sigma_1}$ at $s$ then there is an $x \in [n^2, (n+1)^2]$ such that $x$ is enumerated into $A$ at stage $s$.

At stage $s$, we declare that $k$ requires attention if $k \leq s$ and there is an $x > (k+1)^2$ such that $x \in W_k[s]$ and no element of $W_k[s]$ has been enumerated into $A$ before $s$.
For the least $k$ which requires attention, enumerate its respective $x$ into $A$.
This guarantees that $W_k \cap A$ is nonempty.
Let $n$ be the number enumerated into $\Sat_{\Sigma_1}$ at stage $s$. Enumerate the least element of $[n^2, (n+1)^2)$ which is not in $A$ into $A$.

By definition $A$ is r.e. We enumerated at most $2n$ elements under $(n+1)^2$ so $A$ is coinfinite. By attention, $W_k \cap A$ is nonempty.
Moreover $[n^2, (n+1)^2)$ has $2n+1$ elements and at most $2n$ belong to $A$. Thus if $n$ enters $\Sat_{\Sigma_1}$ at stage $s$, then there is an element of $[n^2, (n+1)^2)$ not yet in $A$ which enters $A$ at stage $s$.

Given $n$ we compute whether $n \in \Sat_{\Sigma_1}$. Run the enumeration of $A$ until the first stage $s_n$ where $A \cap [n^2, (n+1)^2)$ has been totally enumerated. At this point no more elements of $[n^2, (n+1)^2)$ can be added to $A$, which tells us that we could not enumerate $n$ into $\Sat_{\Sigma_1}$ at a later stage.
So $n \in \Sat_{\Sigma_1}$ iff $n \in \Sat_{\Sigma_n}[s_n]$, and the latter is finite.
\end{proof}
So in particular there are Turing complete sets which are not many-one complete, and so Turing completeness is a weaker notion.

\section{Forcing for sets below $0'$}
Throughout this section, if $A$ is a real, we view it as an infinite binary sequence and let $A|w$ denote its truncation to its first $w$ entries.
We let $\Sigma_1(A)$ be the set of all reals defined by $\exists w\theta(n, w, A|w)$ where $\theta$ is bounded. We define $Q_n(A)$ similarly.
Then a real is $\Delta_1(0')$ iff it is $\Delta_2(0)$ iff it is recursive in $0'$.
To do this, we need to use a special type of forcing that was already known to Kleene and Post.

\begin{definition}
A \dfn{forcing condition} is a finite binary sequence. We say $p \geq q$ if $p$ is an initial segment of the forcing condition $q$.
\end{definition}
\begin{definition}
Let $D$ be a set of forcing conditions. Then $D$ is \dfn{closed downwards} if $p \in D$, $p \geq q$, implies $q \in D$; $D$ is \dfn{dense} if for every $p \in 2^{<\omega}$ there is a $q \in D$ such that $p \geq q$.
\end{definition}
Being closed downwards means that we are closed under taking stronger conditions, and being dense means that for every condition we can find a weaker condition in it.
\begin{definition}
If $\theta$ is a bounded sentence and $p$ is a forcing condition, then $p \forces \theta(G)$, that is, $p$ \dfn{forces} $\theta(G)$, if the length of $p$ is greater than the bounds in $\theta$ and $\theta(p)$ is true.
\end{definition}
Intuitively $p \forces \theta(G)$ if we know that $\theta(q)$ will be true for $p \geq q$ such that $q$ is arbitrarily long.
\begin{definition}
\begin{enumerate}
\item $p \forces \exists x \theta(x, G)$ if there is an $n$ such that $p \forces \theta(n, G)$.
\item $p \forces \forall x \theta(x, G)$ if for every $q \leq p$, $q \not \forces \exists x\neg \theta(x, G)$.
\item $p$ \dfn{decides} $\psi$ if $p \forces \psi$ or $p \forces \neg \psi$.
\end{enumerate}
\end{definition}
If $\psi$ is a $\Sigma_1(G)$-sentence for some real $G$, then $p \forces \psi$ is a recursive statement about $p$ and $p \forces \neg \psi$ is a $\Pi_1$ property of $p$.
To see this, note that to check whether $p \forces \exists x\theta(x, G)$ is decided by checking whether $p \forces \theta(n, G)$ for all $n \leq |p|$.

\begin{lemma}
If $\psi$ is $\Sigma_1(G)$, then the set of forcing conditions which decide $\psi$ is dense.
\end{lemma}
\begin{proof}
If $p \forces \theta(G)$ and $\theta$ is bounded, then for any $q \leq p$ we have $q \forces \theta(G)$, since the quantifiers in $\theta$ only referred to numbers under $|p| \leq |q|$.
In particular, $p \forces \theta(G)$ iff for every $q \leq p$, $q \forces \theta(G)$.
Now that we know the claim is true for $\Delta_1(G)$ sentences it is easy to see that it is true for $\Sigma_1(G)$ sentences.
\end{proof}

\begin{definition}
A real $G$ is \dfn{$1$-generic} if for every $\Sigma_1$ set of conditions $S \subseteq 2^{<\omega}$, either there is an $\ell$ such that $G|\ell \in S$, or there is no $q \leq G|\ell$ such that $q \in S$.
\end{definition}
Therefore at any finite stage, $G$ has decided whether $G$ contains an element of $S$. In other words, if $\psi$ is $\Sigma_1$, there is an $\ell$ such that $G|\ell$ decides $\psi$.

We now show that forcing is equivalent to truth for generic reals.
\begin{theorem}
Suppose that $G$ is a $1$-generic real and $\psi(G)$ is a $\Sigma_1$-sentence. Then $\psi(G)$ is true iff there is an $\ell$ such that $G|\ell \forces \psi$.
\end{theorem}
\begin{proof}
Suppose $\psi(G) = \exists x \theta(x, G|x)$ where $\theta$ is bounded. Let $S = \{p: p \forces \psi(G)\}$. Thus $S$ is defined by $\exists x(p\forces \theta(x, G))$, so $S$ is r.e.
Since $G$ is generic, there is an $\ell$ such that $G|\ell \in S$ or $G|\ell$ has no extension in $S$.

If $G|\ell \in S$ then $G|\ell \forces \psi(G)$ and $\NN \models \psi(G)$.

Otherwise, $G|\ell \not \forces \psi(G)$, and if $\exists x\theta(G)$ were true, then there would be an $x$ such that $\theta(x, G|x)$ holds, and $G|\max(x,\ell)$ would extend $G|\ell$ in $S$, a contradiction. Therefore $\neg \psi(G)$ is true.
\end{proof}

\begin{theorem}
There is a $1$-generic real which is recursive in $0'$.
\end{theorem}
\begin{proof}
Let $\{S_n\}$ be an enumeration of all the $\Sigma_1$ subsets of $2^{<\omega}$. Let $p_0$ be the empty condition.

Let $p_{n+1}$ be the lexicographically least $q \leq p_n$ such that $q \in S_n$, if one exists, or $p_n$ otherwise. Computing $p_{n+1}$ from $p_n$ is recursive in $0'$ since it is asking an existential question.

Let $G = \lim_n p_n$. Since $A_k = \{p: |p| \geq k\}$ is dense, and $G \cap A_k$ for all $k$, $G$ is infinite. For every $S_n$, either $G$ meets it or cannot enter into it by extension, so $G$ is $1$-generic.
\end{proof}
This was just the proof of the Baire category theorem.
To see this, let $\{N_n\}$ be an enumeration of a countable set of nowhere dense sets of reals. Let $U_0 = \RR$. Choose $U_n$ to be an open set outside $N_n$ and contained in $U_{n-1}$. Then the $U_n$ have nonempty intersection $U$.
A generic real lies inside $U$.

In fact, the above theorem immediately follows from the Baire category theorem when applied to the topology of the Cantor set $2^\omega$.
The topology of $2^\omega$ is generated by the cylinders
$$B_\sigma = \{x: \exists n(x|n = \sigma)\}.$$
Let $S \subseteq 2^{<\omega}$ be $\Sigma_1$. Then for every generic real $G$, either $\exists \ell(G|\ell \in S)$ or $\exists \ell\forall \tau(G|\ell \geq \tau \to \tau \notin S)$,
and the condition $\exists \ell(G|\ell \in S)$ or $\exists \ell\forall \tau(G|\ell < \tau \to \tau \notin S)$ is dense in the Cantor sets. Since there are only countably many $\Sigma_1$ sets the claim immediately follows.

\begin{theorem}
If $G$ is $1$-generic then $G$ is not computable and does not compute $0'$.
\end{theorem}
\begin{proof}
Suppose $f: \NN \to 2$ is total and recursive. Let $S = \{p: \exists n < |p|(p(n) \neq f(n))\}$. Then $S$ is $\Sigma_1$ and dense, so there is an $\ell$ such that $G|\ell \in S$.
So there is an $n$ such that $G(n) \neq f(n)$, so $G$ is not computable.

Now suppose $G$ compute $0'$. Then $N\setminus 0'$ is $\Sigma_1(G)$, say defined by $\psi = \exists x\theta$.
Let
Then $S$ is $\Sigma_1$ and $G$ has no initial segment in $S$. Let $\ell$ be such that $G|\ell$ has no extension in $S$. Then for any $n$, $n \notin 0'$ iff $\exists q \leq G|\ell$ such that $\exists x<|q|\theta(n, x, q)$.
$$S = \{p: \exists n \in 0' \exists x<|p|\theta(n, x, p)\}.$$
\end{proof}
Intuitively, recursive sets cannot achieve everything that is possible because they never disagree from themselves, so are not generic.

\section{Incomparable reals}
We now use forcing to prove the following theorem of Kleene and Post.
\begin{theorem}[Kleene-Post]
\index{Kleene-Post theorem}
There are reals $A, B \leq_T 0'$ which are Turing incomparable.
\end{theorem}
\begin{proof}
Let $G$ be $1$-generic and recursive in $0'$. Let $A = \{n: 2n \in G\}$ and $B = \{n: 2n + 1 \in G\}$. By symmetry, it suffices to show that $A \not \leq_T B$, and in fact we will show $A \notin \Sigma_1(B)$.

Suppose that $\psi = \exists x \theta$ is $\Sigma_1$. For $p$ a condition, let $p_0$ and $p_1$ be the even and odd parts of $p$, $p_j(n) = p(2n+j)$. Let
$$S = \{p: \exists n(p_0 \forces \psi(n, G) \wedge p_1(n) = 0)\}.$$
Then $S$ is recursive.
If $G$ meets $S$, say $G|\ell \in S$, then there is an $n \notin B$ such that $\psi(n, A)$. So $\psi(A) \neq B$.

Otherwise, there is an $\ell$ such that no extension of $G|\ell$ is in $S$. We claim that $\psi(A)$ is finite. To see this, let $n > \ell$ and $x$ be given.
If $q_0$ is an extension of $p_0$ such that $|q_0| > n$ and $\theta(n, x, q_0)$, extend $p_1$ to $q_1$ such that $|q_1| = |p_1|$ and $p_1(n) = 0$. So $q$ extends $G|\ell$ and $q \in S$, which is impossible. Therefore $|\psi(A)| \leq n$.

But, for any $k$, it is generic that $|\{n: p_1(n)\}|>k$, so $B$ is infinite. Therefore $\psi(A) \neq B$.
\end{proof}
The original Kleene-Post theorem was a bit stronger than this: for any countable poset $P$, there is an embedding of $P$ into the set of Turing degrees below $0'$.
As a corollary, the existential theory of the Turing degrees below $0'$ is decidable.

\begin{definition}
A \dfn{locally countable poset} is a poset $P$ such that for every $x$, $\card \{y: y < x\} = \aleph_0$.
\end{definition}
Clearly the Turing degrees are locally countable. An open problem asks whether this is a universal property of the Turing degrees.
\begin{conjecture}
Suppose $P$ is a locally countable poset such that $\card P = 2^{\aleph_0}$. Then there is an embedding of $P$ into the Turing degrees.
\end{conjecture}
\begin{theorem}
If $\aleph_1 = 2^{\aleph_0}$ then the above conjecture is true.
\end{theorem}

\section{Inverting the jump}
Suppose that $G$ is any real. Then $0' \leq G'$, so the Turing jump maps the set of all reals into the set of reals above $0'$. The converse is also true.
\begin{theorem}[Friedberg jump inversion]
\index{Friedberg jump inversion theorem}
If $0' \leq_T X$, then there is a $1$-generic real $G$ such that $G' \equiv_T X$.
\end{theorem}
\begin{proof}
Let $\{S_n\}_n$ be an enumeration of the $\Sigma_1$-definable subsets of $2^{<\omega}$. Let $p_0$ be the empty condition.
If there is a $q$ which extends an element of $S_n$, let $p_{n+1}^*$ be the lexicographically least $q$. Otherwise let $p_{n+1}^* = p_n$. Then let $p_{n+1} = p_{n+1}^* + X(n)$, where $+$ is concatenation.
Then $G$ is $1$-generic.

We now show that $G' \equiv_T G \oplus 0' \equiv_T X$.
By assumption $X$ can decide $0'$, and then $X$ can decide whether a $q$ exists (and what it is), so $X$ can decide $p_{n+1}$ and hence $G$. Therefore $G \oplus 0' \leq_T X$.
On the other hand, $0'$ can compute $p_n^*$ and hence $G \oplus 0'$ can compute $X$.

Clearly $G' \geq_T G \oplus 0'$. Conversely, we show that $G \oplus 0'$ computes $G'$.
Let $\psi$ be $\Sigma_1$.
Since $G$ is $1$-generic, so there is an $\ell$ such that $G|\ell$ decides $\psi(G)$. What $G|\ell$ decides is computable from $0'$. Therefore $0'$ can compute the existential theory of $G$, hence can compute $G'$.
\end{proof}

\begin{theorem}[Posner-Robinson]
\index{Posner-Robinson theorem}
For every nonrecursive real $X$, there is a $1$-generic real $G$ such that $G' \equiv_T G \oplus X$.
\end{theorem}
\begin{proof}
By replacing $X$ with $X^c$ if necessary, we can assume that $X$ is not r.e.
Let $\{\psi_n\}_n$ be an enumeration of the $\Sigma_1(G)$ formulae.

Let $p_0$ be the empty sequence.
Given $p_n$, let $k$ be the least number such that $k \in X$ and $p_n + \overline k + 0 \forces \neg \psi_n(G)$, or $k \notin X$ and there is a $q$ such that $p_n + \overline k + 0 \geq q$ and $q \forces \psi_n(G)$.
Here $\overline k$ consists of $k$ many $1$'s and $+$ is concatenation.
In the first case, let $p_{n+1} = p_n + \overline k + 0 + X(n)$, and otherwise take $q'$ to be the lexicographically least $q$ and take $p_{n+1} = q' + X(n)$.
Let $G = \lim_{n \to \infty} p_n$.

We need to show that for every $n$, $k$ exists. Suppose not, so $p_n$ exists but not $p_{n+1}$. Then for every $k$, $k \in X$ iff there is a $q$ such that $p_n + \overline k + 0 \geq q$ and $q \forces \psi_n(G)$.
Since $\forces$ for $\Sigma_1$ sentences is $\Sigma_1$, $X$ now has a $\Sigma_1$ definition, namely
$$\exists q((p_n + \overline k + 0 \geq q) \wedge q \forces \psi_n(G)).$$
This is a contradiction.

To see $G \oplus X \leq_T G'$, note that $G'$ knows what our computation steps were, so can compute $p_n$. But if we know every $p_n$, we can compute $G$ and $X$.

Conversely, we will show that $G \oplus X$ can compute the $p_n$ and hence $G'$. Suppose we know $p_n$. To compute $p_{n+1}$, we determine the length $k$ of consecutive $1$'s and then checking whether $k \in X$. Knowing $k$ and $X$ allows us to compute the initial segment $p_{n+1}$ of $X$. Then $\psi_n(G)$ is true iff $p_{n+1}$ thinks it is true.
\end{proof}
According to Ted Slaman, coming up with the coding trick used in the Posner-Robinson theorem probably took a long time, but once one comes up with it it's obvious that it will work.
Slaman learned this theorem in the most poorly motivated talk he had heard in a long time, when he was a graduate student.
However, he frequently found the theorem useful in the years following.

The Posner-Robison theorem also easily relativizes.
\begin{corollary}
Let $B,X$ be real numbers such that $X \not\leq_T B$. Then there is a $1$-generic $G$ such that $B \leq_T G$ and $G \oplus X = G'$.
\end{corollary}
The proof is exactly the same; just whenever we had a computation in the proof of the Posner-Robinson theorem, allow the use of $B$ as a blackbox oracle.

\section{The axiom of determinacy}
The Posner-Robinson theorem has applications to the axiom of determinacy.

Throughout, let $\omega^\omega$ be the Baire space, whose points we will call reals (rather than points in Cantor space, as above.)
We consider the \dfn{standard Gale-Stewart game} $G_A$ on $\omega^\omega$ with victory set $A \subseteq \omega^\omega$: Alice and Bob alternate between playing natural numbers: Alice plays $a_1$, Bob plays $a_2$, Alice plays $a_3$, etc. Then Alice wins iff the sequence is in the victory set $A$.

Since $\omega^\omega$ is a product space, its topology is generated by cylinders of the form $U_\sigma$ for each $\sigma \in \omega^{<\omega}$.

\begin{definition}
A \dfn{strategy} for Alice (Bob) is a function that maps finite sequences of even (odd) length to natural numbers. The strategy \dfn{wins} the game $G_A$ if it gives an algorithm which guarantees that no matter what Bob (Alice) plays, the infinite sequence generated is in $A$.
\end{definition}
\begin{example}
If $A = \omega^\omega$ then any strategy for Alice wins. Similarly if $A = \{x: x(0) = 0\}$, where the strategy ``Always play 0" wins for Alice. If $A = \emptyset$ then any strategy for Bob wins.
\end{example}
\begin{definition}
If there is a winning strategy for victory set $A$, we say that $G_A$ is a \dfn{determined game}.
\end{definition}
\begin{theorem}[Gale-Stewart]
\index{Gale-Stewart theorem}
Every open game is determined. In particular, closed games or games of finite length are determined.
\end{theorem}
\begin{proof}
Suppose we are playing $G_A$, $A$ is open and Alice does not have a winning strategy.
Assume that we have played $a_1, \dots, a_{2n}$ and now Alice plays $a_{2n+1}$. Since Alice does not have a winning strategy,
there is a $a_{2n+2}$ such that there is a sequence beginning with $a_1, \dots, a_{2n+2}$ which is not in $A$.
So let Bob play $a_{2n + 2}$.

Suppose that Bob does not win following the above strategy.
In particular, since Bob did not win, $\{a_n\}_n \in A$, so there is a restriction $\sigma$ of $\{a_n\}_n$ such that $\{a_n\}_n \in U_\sigma$. So Bob lost at the finite time $|\sigma|$, which is impossible since Alice didn't have a winning strategy.

Games of finite length are exactly those of the form $G_{U_\sigma}$ for some $\sigma$, so are open. The complement of a determined game is determined.
\end{proof}
The Gale-Stewart theorem is really just another restatement of the Baire category theorem.
\begin{theorem}
There is a victory set that is not determined.
\end{theorem}
\begin{proof}
First, for any strategy $\sigma$, the set of possible outputs if a player uses strategy $\sigma$ has cardinality $2^{\aleph_0}$. Similarly there are $2^{\aleph_0}$ many strategies.
So let $<$ be a well-ordering of $\omega^\omega$ and $<_s$ a well-ordering of the set of strategies, such that $<$ and $<_s$ have the same ordertype.

We now diagonalize against the set of all strategies. If $\alpha < 2^{\aleph_0}$, we have specified membership in $A$ or its complement for fewer than $2^{\aleph_0}$ many elements of $\omega^\omega$.
Let $\sigma_\alpha$ be the $\alpha$th strategy under ordering $<_s$. If $\sigma_\alpha$ is a strategy for player $P$, let $a_\alpha$ be the $<$-least output such that we have not determined whether $a_\alpha$ is in $A$; if $P = 1$ declare $a_\alpha \in A^c$, otherwise declare $a_\alpha \in A$.

If $a$ was not decided for any $a_\alpha$, let $a \in A$. Then neither player has a winning strategy.
\end{proof}
Note that this theorem uses the axiom of choice in an extremely strong way.
\begin{axiom}[determinacy]
\index{axiom of determinacy}
For every set $A \subseteq \omega^\omega$, $G_A$ is a determined game.
\end{axiom}
This axiom is provably false from ZFC, but it is still very interesting, because of the following theorem.
\begin{definition}
A set $A \subseteq \omega^\omega$ is a \dfn{projective set} if there is a closed set $C \subseteq \omega^\omega$ such that $A$ can be constructed from $C$ using only the operations of complementation and projection $\omega^\omega \times \omega^\omega \to \omega^\omega$.
\end{definition}
Here we are using the fact that $\omega^\omega \times \omega^\omega = \omega^\omega$.
\begin{definition}
A \dfn{Woodin cardinal} is a cardinal $\lambda$ such that for every $f: \lambda \to \lambda$, there is
\begin{enumerate}
\item a cardinal $\kappa < \lambda$ such that $f$ restricts to a function $\kappa \to \kappa$,
\item a transitive class $M \subseteq V$,
\item and an elementary embedding $j: V \to M$ such that $\crt j = \kappa$
\end{enumerate}
such that $j$ restricts to a function $V_{j(f)(\kappa)} \to M$.
\end{definition}
\begin{theorem}[Martin-Steel]
\index{Martin-Steel theorem}
Every Borel game is determined, and if there are infinitely many Woodin cardinals, then every projective game is determined.
\end{theorem}
The assumption that there are infinitely many Woodin cardinals seems ridiculously strong. But the proof of this for Borel sets is not much better: it uses arbitrarily large iterates of the power set operation, and this part of the proof is provably unavoidable.
\begin{theorem}[Friedman-Woodin]
If every projective set is determined then there is a model of ZFC with infinitely many Woodin cardinals. Moreover, there is a model of Zermelo theory with a Borel game which is not determined.
\end{theorem}
Here \dfn{Zermelo theory} is set theory without the axiom schema of replacement (which in particular does not allow one to iterate the power set operation for arbitrarily long.)

Determined sets have very nice properties. This includes applications to measure theory and the Turing degrees.
\begin{theorem}
Every determined set is Lebesgue measurable.
\end{theorem}
Here we identify subsets of $2^\omega$ with subsets of $[0, 1]$ according to the mapping
$$a \mapsto \sum_{n=0}^\infty \frac{a_n}{2^{n+1}}.$$
Thus it makes sense to ask if a subset of $\RR$ is determined.
\begin{proof}
Let $\mu$ denote Lebesgue measure and $\mu^*$ denote Lebesgue outer measure.

We need the following lemma.
\begin{lemma}
Assume that $S$ is a determined set and for every measurable $Z \subseteq S$, $Z$ is null. Then $S$ is null.
\end{lemma}
Once we have proven this lemma, the claim immediately falls out.
Indeed, for any set $E$, there is a $G_\delta$-set $F \supseteq E$ such that $\mu^*(F) = \mu^*(E)$; in particular, for every measurable subset $Z$ of $F \setminus E$, $Z$ is null.
Now by the Gale-Stewart theorem, $F$ is determined; if $E$ is also determined, $F \setminus E$ is determined.
So by lemma, every measurable subset of $F \setminus E$ is null, so $F \setminus E$ is null. Therefore $E$ is measurable.

To prove the lemma, let $\varepsilon > 0$; we claim $\mu^*(S) < \varepsilon$. For any $n$ let $K_n$ be the countable set of all sets $G$ which are finite unions of open intervals with rational endpoints, and $m(G) < \varepsilon/2^{2n+1}$, and let $G_{n,k}$ enumerate $K_n$.

We define a game that Alice wins if she can play a point $a \in S$ such that $a \notin \bigcup_n G_{b_n}^n$, where Bob plays $b$.
Since everything is determined, to show that Bob has a winning strategy, it suffices to show that Alice does not.

Let $\sigma$ be a winning strategy for Alice, which gives a continuous function $\omega^\omega \to \RR$. Then $\sigma$ is continuous, so $Z = \sigma(\omega^\omega)$ is measurable, and so by assumption $Z$ is null. So there are open sets $H_n \in K_n$ which cover $Z$. Therefore Bob wins if he plays optimally against $\sigma$, a contradiction.

Since Bob has a winning strategy, he can cover $S$ by open sets $H_n \in K_n$, and $\bigcup_n \mu(H_n) < \varepsilon$. Therefore $S$ is null.
\end{proof}
\begin{theorem}[Martin cone theorem]
\index{Martin cone theorem}
Let $\mathcal D$ be the set of Turing degrees, and let $A$ be a cofinal subset of $\mathcal D$. Suppose that the set $\{x \in \omega^\omega: [x]_T \in A\}$ is determined. If for every $x \in A$ and every $y \equiv_T x$, $y \in A$, then there is a $b$ such that for all $a$, if $b \leq_T a$ then $a \in A$.
\end{theorem}
\begin{proof}
Let $P$ be the determined set.

Suppose Alice has a winning strategy for $G_P$.
Now a strategy consists of functions $\omega^{2n} \to \omega$ for all $n \in \omega$, each of which can be coded as a point $x_n \in \omega^\omega$.
Clearly the $x_n$ form a Cauchy sequence, so they converge to a point $x \in \omega^\omega$, Alice's \dfn{master strategy}. Let $b = [x]_T$ be the Turing degree of Alice's master strategy.

Suppose that Bob counters Alice's master strategy by playing a sequence $y$, and $a = [y]_T$. If $a \geq_T b$ then the sequence that Alice and Bob generate is of degree $a$, so $a \in A$.

Now suppose Bob has a winning strategy. Running the same argument as above, we conclude that if $a \geq_T b$ then $a \in A^c$. Since $\mathcal D$ does not have a greatest element (since we can always take the Turing jump), and $A$ is cofinal, we find an element $a \in A \cap A^c$, a contradiction.
\end{proof}
\begin{definition}[Martin]
A function $f: \omega^\omega \to \omega^\omega$ is \dfn{degree-invariant} if for any $x, y$, if $x \equiv_T y$ then $f(x) \equiv_T y$.
If $f,g$ are degree-invariant, we say that $f \geq_M g$ if there is a $y$ such that for any $x \geq_T y$, $f(x) \geq_T g(x)$.
\end{definition}
\begin{example}
The Turing jump operation is degree invariant, as is the constant function, the identity function, and iterates of the Turing jump.
\end{example}
\begin{conjecture}[Martin]
Let $I$ be the set of degree-invariant functions. If $f \in I$ and $f \not \geq_M \id$ then there is a $B$ such that for all $A \geq_T B$ $f(A) \equiv_T B$.
Moreover, $I' = \{f \in I: f \geq_M \id\}$ is well-ordered by $\geq_M$ (modulo Martin equivalence $\equiv_M$). For all $f \in I'$, the succesor $f'$ of $f$ is given by
$$f'(x) = f(x)'.$$
\end{conjecture}
Thus Martin's conjecture essentially rules out the existence of $\geq_M$-incomparable functions.

One can use the Posner-Robinson theorem to prove a special case of Martin's conjecture.
\begin{theorem}[Slaman-Steel]
\index{Slaman-Steel theorem}
Assume that $f: \omega^\omega \to \omega^\omega$ is determined.
Let $Jx = x'$ be the Turing jump. Assume $f$ preserves $\geq_T$ and $f >_M \id$. Then $f \geq_M J$.
\end{theorem}
\begin{proof}
We must show that for all $B$ there is an $A \geq_T B$ such that $f(A) \geq A'$. By assumption, we can take $B$ large enough that $f(B) >_T B$.

By the Posner-Robinson theorem, there is a $G \geq_T B$ such that $f(B) \oplus G \equiv_T G'$. But then $f(G) \geq_T G$ since $G \geq_T B$ and $f$ increases there.
Thus $f(G) \geq_T f(B)$. So
$$G' \equiv_T f(B) \oplus G \leq_T f(G).$$
Thus $JG \geq_T f(G)$.
\end{proof}

\section{The Friedberg-Muchnik theorem and the priority argument}
\begin{theorem}[Friedberg-Muchnik]
There are $\Sigma_1$ reals $A,B$ which are Turing incomparable.
\end{theorem}
If $X$ is a real number that we are enumerating, let $X[s]$ denote the set of reals that have been enumerated at stage $s$.
Let $\Phi_i(X)$ denote the $i$th Turing functional (i.e. program) which uses $X$ as an oracle.
We will construct $A,B$ so that for any $i$, $\Phi_i(A) \neq B$ and $\Phi_i(B) \neq A$.
We let $R_i^A$ mean that $\Phi_i(A) \neq B$ and similarly for $R_i^B$.
Let $\Phi_i(n, X)[s]$ denote the result of running the $i$th program with oracle $X$ and input $n$ for at most $s$ stages and returning only numbers less than $s$.

Suppose that we want to satisfy just one requirement, namely $R_i^A$, at stage $s$. We execute the following strategy:
\begin{enumerate}
\item Let $n$ be the stage the strategy is activated in. Suppose $n \notin B$ and go to Step $2$.
\item If $\Phi_i(n, A)[s - 1] = 0$, i.e. $\Phi_i(A)$ thinks that $n$ does not belong in $B$ at stage $s - 1$, set $s_i^A = s$ and go to Step $3$. Otherwise, do not allow $n$ to enter $B$ at stage $s$.
\item Enumerate $n$ into $B$. Restrain any number less than $s_i^A - 1$ from entering $A$ at stage $s$. Expect that $n$ is much smaller than $s$.
\end{enumerate}
At stage $s + 1$ we initialize the above strategy in the Step that we ended it at stage $s$ at.

Now this is not an algorithm to enumerate $A,B$, but if we can prove this existence of an algorithm to enumerate $A,B$ such that for every $s$, $R_i^A$ is satisfied at stage $s$, then $B \not \leq_T A$, regardless of what that algorithm is.
In fact, either as $s \to \infty$ we stay at Step 2, so $B(n) = 0$ and $\Phi_i(n, A) \neq 0$, so $\Phi_i(A) \neq B$, or we end at Step 3, in which case at stage $s_i^A$, $n$ enters $B$ and hence $B(n) = 1$.
But if we ended at Step 3, then $\Phi_i(n, A) = 0$, since the computation that sent us to Step 3 is repeated over and over forever. So $\Phi_i(A) \neq B$.

Unfortunately, it is not true that we can require $R_i^A$ and $R_i^B$ at every stage of computation.
If we cannot carry out some strategy, we say that it is an \dfn{injured strategy}.
Our goal is to show that every strategy is only injured finitely many times.
Consider the enumeration
$$R_0^A, R_0^B, R_1^A, R_1^B, R_2^A, \cdots.$$
We will say that a requirement $S$ has \dfn{higher priority} than $S'$ if $S$ appears earlier in the enumeration than $S'$.
We will not allow a higher priority strategy injure a lower priority strategy. This proof technique is known as the \dfn{finite injury method}.

At stage $0$, let $A[0] = B[0] = \emptyset$ and have no strategies active.

At stage $s$, we are given the sequence of strategies that were active at the end of stage $s - 1$, the record of the numbers they selected, the restraints they posed in Step 3, and the values of $A[s]$ and $B[s]$.

Declare that the strategy for $R_i^X$ \dfn{requires attention} if it is not active, or it is in Step 2 with number $n$ and $\Phi_i(n, X)[s - 1] = 0$. S
Let $R$ be the highest priority argument that requires attention.
If $R$ is not currently active, activate it and follow its instructions in Steps 1, 2. Choose $n$ and assign it to $R$.
Otherwise, if $R$ is already active, follow its instructions as it moves to Step 3. Enumerate the $n$ associated to $R$ into the appropriate set.

At the end of stage $s$, deactivate, or \dfn{injure}, all strategies of lower priority to $R$. If they were injured, they will be restarted at a later stage.

\begin{lemma}
For each requirement $R$, there is a strategy for $R$ that is never deactivated.
\end{lemma}
\begin{proof}
By induction on priority. $R_0^A$ had highest priority, so it is never injured.

Suppose that the claim holds for all requirements of higher priority than $R$, and assume $R^-$ is the requirement of highest priority such is of lower priority than $R$.
Let $s_0$ be the final stage where $R^-$ was injured.
After this, no requirement of higher priority than $R^-$ can require attention.
But then $R^-$ can only require attention at most twice after $s_0$: at $s_0 + 1$ when $R^-$ reactivates, and possibly a later stage when it moves from Step 2 to Step 3.

Therefore there is a $s_1$ such that $R^-$ does not require attention after $s_1$.
Since no other requirements of higher priority to $R$ can require attention after stage $s_1$, after $s_1$, $R$ never deactivates.
\end{proof}

\begin{lemma}
For each requirement $R$, there is a stage $s$ such that a strategy for $R$ is activated at $s$ and the construction implements all constraints that the strategy imposes after $s$.
\end{lemma}
\begin{proof}
Let $t$ be the last stage for which there was no active strategy for $R$. No strategy of higher priority will require attention for any stage after $t$, and $R$ requires attention at stage $s = t + 1$.

Since $R$ is of highest priority among those requiring attention at stage $s$, the construction follows $R$'s instructions at stage $s$. $R$ records $n = s$.
Since strategies can only enumerate numbers less than the current stage, $n \notin A[s]$ and $n \notin B[s]$.
Since activating $R$ deactivates all lower-priority strategies, no such strategy enumerates $n$.
Since strategies of higher priority ignore $n$ entirely, they also do not enumerate $n$.

If $R$ remains at Step 2 forever, then the lemma is verified. Otherwise, suppose that $R$ requires attention at stage $s_1 > s$.
Then $R$ will be the highest priority strategy at $s_1$, and will enumerate $n$ into the appropriate set, and deactivate all strategies of lower priority.
These strategies choose new values for their parameters at later stages, and these parameters will exceed $s_1 > n$, so they will not contradict the requirements $R$ imposed.
The higher priority strategies will never again require attention, so they too will not contradict the requirements imposed by $R$.
Therefore the construction implements the requirements of $R$.
\end{proof}

Therefore the strategies are satisfied at cofinitely many stages and hence $A,B$ are as desired.

One can extend the above techniques to require that $A,B$ are, for example, simple. We would have to add additional strategies that also require that $A,B$ are simple.

\begin{conjecture}[Sacks]
There is an $e$ such that for all $A$:
\begin{enumerate}
\item $A <_T W_e^A <_T A'$
\item For all $B$, if $A \equiv_T B$ then $W_e^A \equiv_T W_e^B$.
\end{enumerate}
\end{conjecture}
This would give a counterexample to Martin's conjecture. Note that the first condition requires that $e$ is a solution to Post's problem relative to all $A$, and the second condition requires that $W_e$ is degree-invariant.

Sacks was the Ph.D. adviser to Ted Slaman, and according to Slaman, Sacks believed that if his conjecture were true he would find the solution extremely interesting.
But if Martin's conjecture were true, Sacks would not find the solution very interesting.
Sacks had an ideal of recursion theory that was based on elaborate constructions like the finite-injury method, while Martin had an ideal of recursion theory as a field with a beautiful global structure.


\chapter{Elementary set theory}
\section{The axioms of ZF without Foundation}
By an \dfn{LST-formula} we mean a first-order formula in the language of set theory, namely the language consisting of a single binary relation symbol $\in$.

The axioms are:
\begin{axiomZFC}[extensionality]
    A set is determined by its elements.
\end{axiomZFC}
\begin{axiomZFC}[empty set]
    There is a set with no elements, denoted $\emptyset$.
\end{axiomZFC}
\begin{axiomZFC}[pairing]
    For any two sets $x,y$, the set $\{x, y\}$ exists.
\end{axiomZFC}
\begin{axiomZFC}[union]
    For any set $x$, the set $\bigcup x$ exists.
\end{axiomZFC}
\begin{axiomZFC}[power set]
    For any set $x$, the set $\pset x$ exists.
\end{axiomZFC}
    By extensionality, all the above sets are unique. Moreover, the class $V_\omega$ of hereditarily finite sets is a model of the above theory. So we need to introduce a new axiom to escape $V_\omega$. This will be the first ``small large-cardinal axiom."
\begin{axiomZFC}[infinity]
    There is a set $x$ such that $\emptyset \in x$ and such that for each $y \in x$, $y \cup \{y\} \in x$.
\end{axiomZFC}
    Then $\omega$ is the intersection of all such sets. (We will call these sets \dfn{inductive}.)

    Now $V_{\omega + \omega}$ is a model of the above theory, but does not contain $\aleph_\omega$. So we introduce the following schema of ``small large-cardinal axioms."
\begin{axiomZFC}[replacement schema]
    For every LST-formula $\varphi$, we have
$$\forall \vec p \forall x(\forall y \in x \exists! z \varphi(y, z, \vec p) \to \exists y \forall z (z \in y \leftrightarrow \exists w \in x \varphi(y, z, \vec p)).$$
\end{axiomZFC}

\section{Transfinite induction}
    Fix a binary relation $R$ on a class $X$.
\begin{definition}
    The $R$-\dfn{extension} of $x \in X$ is
    $$\ext_R(x) = \{x^* \in X: x^*Rx\}.$$

    $R$ is \dfn{transitive} if for every $x_1,x_2,x_3$, $x_1Rx_2$ and $x_2Rx_3$ implies $x_1Rx_3$. The \dfn{transitive closure} of $R$, $\TC(R)$, is the smallest binary relation which is transitive and contains $R$.

    The $R$-\dfn{predecessors} of $x$ are
    $$\pred_R(x) = \ext_{\TC(R)}(x).$$

    $Y \subseteq X$ is $R$-\dfn{transitive} if for every $y \in Y$, $\pred_R(y) \subseteq Y$ (i.e. $Y$ is closed under $\pred$).
\end{definition}
\begin{definition}
    $R$ is \dfn{well-founded} if for every set $Y \subseteq X$ there is a $R$-minimal element, and if for every $x \in X$, $\ext_R(x)$ is a set.
\end{definition}
\begin{axiomZFC}[foundation]
    $\in$ is well-founded.
\end{axiomZFC}
    We will always assume foundation in what follows. Foundation, along with the above axioms, comprise the Zermelo-Frankel axiom system. This is still not ZFC because we have no introduced choice.

    Note that if $X$ is actually a set, then $R$ is well-founded iff every subset has a minimal element, and if $(X, R)$ is a chain, then this happens iff $R$ is a well-order. In fact we take this as a definition.
\begin{definition}
    $R$ is \dfn{strict} if for every $x \in X$, $\neg(xRx)$.

    $R$ is \dfn{linear} if every $x_1,x_2 \in X$, $x_1Rx_2$ or $x_2Rx_1$.
\end{definition}
\begin{definition}
    $R$ is a \dfn{well-ordering} if $R$ is strict, linear, and well-founded.
\end{definition}
\begin{axiom}[well-ordering]
    Every set admits a well-ordering.
\end{axiom}
    We will \emph{not} assume this axiom.

    Now we have set up transfinite induction and recursion.
\begin{theorem}[transfinite induction]
    \index{transfinite induction}
    Let $R$ be well-founded and $Y \subseteq X$. If for every $x \in X$ such that $\pred_R(x) \subseteq Y$, $x \in Y$, then $Y = X$.
\end{theorem}
    We think of $x$ as the ``successor" of its predecessors $\pred_R(x)$. So this is analogous to strong induction.
\begin{proof}
    Assume $x \in X \setminus Y$. Then we can take $x$ to be minimal since $R$ is well-founded. Then if $y \in \pred_R(x)$, $y \notin X \setminus Y$ so $y \in Y$. So $x \in Y$, a contradiction.
\end{proof}
\begin{theorem}[transfinite recursion]
    \index{transfinite recursion}
    Let $R$ be well-founded and $G: X \times V \to V$ is a class function. There is a unique class function $F: X \to V$ such that for every $x \in X$,
    $$F(x) = G(x, F|_{\pred_R(x)}).$$
\end{theorem}
\begin{proof}
    Uniqueness follows immediately by transfinite induction. For existence, let $x$ be minimal and put $f_0(x) = G(x, \emptyset)$ to start the induction.

    Say that $f: D \to X$ is \dfn{good} if $D \subseteq X$, for every $x \in D$, $\pred_R(x) \subseteq D$, and $f(x) = G(x, f|_{\pred_R(x)})$. Clearly $f_0$ is good. By uniqueness, if $f_1: D_1 \to X$ and $f_2: D_2 \to X$ are good, then $f_1\cap f_2: D_1 \cap D_2 \to X$ is defined. By transfinite induction with $f_0$ as the base case, the union of all $D$ such that there is a good $f$ is $X$ itself. So by ``gluing" the good functions, $F$ is uniquely defined.
\end{proof}

\section{Ordinals}
Let $R$ be a well-ordering of a class $X$. For $x \in X$, we let $I_x^R$ be the initial segment $\pred_R(x)$ with its induced well-ordering.

By an easy transfinite induction, we have the following theorem.
\begin{theorem}
    Let $S$ be a well-ordering of a class $Y$. Then either $Y$ is an initial segment of $X$ or $X$ is an initial segment of $X$.
\end{theorem}

\begin{definition}
    A \dfn{transitive set} $X$ is one for which $x \in X$ implies $x \subset X$.
\end{definition}
\begin{lemma}
    If $X$ is transitive then so are $X \cup \{X\}$, $\bigcup X$, and $\pset X$.
\end{lemma}
    Obviously $\emptyset$ is transitive.
\begin{definition}
    An \dfn{ordinal} is a transitive set $\alpha$ such that $\in$ is a well-ordering of $\alpha$.
\end{definition}
    We denote the class of ordinals $\Ord$. Now if $\alpha \in \Ord$ and $\beta \in \alpha$, $\beta \in \Ord$. So $\alpha$ consists of the ordinals under $\alpha$. In particular $\beta \in \alpha$ iff $\beta \subset \alpha$. So it follows that $\Ord$ is well-ordered by $\in$.
\begin{theorem}
    $\Ord$ is a proper class.
\end{theorem}
\begin{proof}
    Suppose not. Then $\in$ is a well-ordering of $\Ord$, so $\Ord$ is an ordinal. So $\Ord \in \Ord$, contradicting foundation.
\end{proof}

Now we see that the only well-ordered sets are the ordinals.
\begin{theorem}
    Let $(X, R)$ be a well-ordered set. Then there is an ordinal $\alpha$ and an isomorphism $X \to \alpha$.
\end{theorem}
\begin{definition}
    The ordinal $\alpha$ is called the \dfn{ordertype} of $(X, R)$.
\end{definition}
    To prove this we need a lemma.
\begin{lemma}
    For any $x \in X$ there is an ordinal $\alpha$ such that $I_x^R \cong \alpha$.
\end{lemma}
\begin{proof}
    Apply replacement from $I_x^R$ into $\Ord$.
\end{proof}
\begin{proof}[Proof of theorem]
    Take the set $\beta$ of all ordinals $\alpha$ such that for some $x \in X$, $I_x^R \cong \alpha$. Then clearly $\beta$ is an ordinal.
\end{proof}

Now we define arithemtic on ordinals by transfinite recursion. Namely, we put $\alpha + 0 = \alpha$, $\alpha\cdot 0 = \alpha$, and $\alpha^0 = 1$. We then put $\alpha + (\beta + 1) = (\alpha + \beta) + 1$, $\alpha(\beta + 1) = \alpha\beta + \alpha$, and $\alpha^{\beta+1} = \alpha^\beta\alpha$. Then we take unions at limit stages. We let $\omega$ denote the smallest infinite ordinal, which exists by the axiom of infinity.

\begin{theorem}[division algorithm]
    Let $\alpha$ and $\beta > 0$ be ordinals. There are unique $\gamma_1$ and $\gamma_2 < \beta$ such that
    $$\alpha = \beta\gamma_1 + \gamma_2.$$
\end{theorem}
The proof is the same as for the classical division algorithm.

\begin{theorem}[Cantor normal form]
    \index{Cantor normal form}
    If $\alpha > 0$ is an ordinal then we can uniquely write
    $$\alpha = \omega^{\beta_1}\kappa_1 + \dots + \omega^{\beta_n}\kappa_n$$
    where $\alpha > \beta_1 > \dots > \beta_n \in \Ord$ and $\kappa_1, \dots, \kappa_n, n \in \omega$.
\end{theorem}
The proof is by greedy transfinite induction.

\begin{definition}
    The \dfn{cumulative hierarchy} is defined by transfinite recursion as:
\begin{enumerate}
    \item $V_0 = \emptyset$.
    \item $V_{\alpha + 1} = \pset V_\alpha$.
    \item $V_\lambda = \bigcup_{\alpha \in \lambda} V_\alpha$ for $\lambda$ a limit ordinal.
\end{enumerate}
    Finally $V$ is the proper class $V = \bigcup_{\alpha \in \Ord} V_\alpha$.
\end{definition}
By transfinite induction, every $V_\alpha$ is transitive. Since $V_\alpha \in V_\beta$ for $\beta > \alpha$, transitivity implies that the $V_\alpha$ form a chain with respect to $\subset$.

By foundation, $\in$ is a well-founded relation on the proper class of all sets. Thus we have the following theorem.
\begin{theorem}
For any set $x$, $x \in V$.
\end{theorem}
\begin{proof}
Suppose not. Then take $x$ to be an $\in$-minimal set which does not appear in $V$, which is possible since $\in$ is well-founded. Then $x \in V$, a contradiction.
\end{proof}

\begin{definition}
The \dfn{rank} $\alpha$ of a set $x$ is the least ordinal such that $x \in V_{\alpha + 1}$.
\end{definition}

\section{The axiom of choice}
We now study the most famous of the axioms of ZFC.
\begin{definition}
    A \dfn{choice function} on a set $X$ is a function $f: X \to \bigcup X$ such that for every $x \in X$, $f(x) \in x$.
\end{definition}
\begin{axiomZFC}[choice]
    Every set admits a choice function.
\end{axiomZFC}
ZF + choice = ZFC. However we will \emph{not} assume choice until a later stage. Here's another famous axiom.
\begin{axiom}[Zorn]
    If $(X, \leq)$ is a nonempty poset such that every subchain of $X$ has an upper bound, then $X$ has a maximal element.
\end{axiom}
\begin{theorem}
    Choice, well-ordering, and Zorn are equivalent.
\end{theorem}
\begin{proof}
    First we prove that choice implies well-ordering. Given $X$ there is a choice function on $\pset X \setminus \{\emptyset\}$. Now let $x_1 \in f(X)$, $x_2 \in f(X \setminus \{x_1\}$, $\dots$. This gives an bijective function $\alpha \to X$, $\beta \mapsto x_\beta$, where $\alpha \in \Ord$, and must stop after transfinitely many steps because if not then $\alpha = \Ord$, a contradiction.

    Now assume well-ordering. To prove Zorn we let $(X, \leq)$ be a poset such that every subchain of $X$ has an upper bound. By well-ordering there is a bijection $f: \alpha \to X$ for some $\alpha \in \Ord$. This gives an injection $g: \beta \to X$ for some $\beta \in \alpha$ by choosing $g(0) = f(0)$ and always letting $g(\beta)$ be the $f$-minimal element in the current chain. This process stops after transfinitely many steps (or else $\beta = \Ord$), and then $g(\beta)$ is maximal.

    To prove choice from Zorn, let $\mathcal F$ be the set of all partial choice functions on $X$. By Zorn, $\mathcal F$ has a maximal element $F$, and it is easy to see that $F$ is a (total) choice function on $X$.
\end{proof}

\section{Cardinals}
We still are working in ZF, i.e. still not assuming choice.

\begin{definition}
    If $\kappa$ is an ordinal such that for every ordinal $\alpha$ which is in bijection with $\kappa$, $\alpha \geq \kappa$, then $\kappa$ is a \dfn{cardinal}, and we write $\kappa \in \Card$.
\end{definition}
\begin{definition}
    Assume the axiom of choice. The \dfn{cardinality} of $X$, $\card X$, is the unique $\kappa \in \Card$ such that $X$ and $\kappa$ are in bijection.
\end{definition}
\begin{lemma}
    If $A$ is a set of cardinals then $\bigcup A$ is a cardinal.
\end{lemma}
\begin{proof}
    Clearly $\bigcup A = \sup A$, so we just have to show that $\sup A \in A$. Assume not. So there is an ordinal $\alpha < \bigcup A$ and a bijection $f: \alpha \to \bigcup A$. Then there is a $\kappa \in \bigcup A$ such that $\alpha < \kappa$ and an injection $\alpha \to \kappa$ obtained by restricting $f$, a contradiction.
\end{proof}
    We let $\aleph_0 = \omega$, $\aleph_{\alpha + 1}$ be the smallest cardinal larger than $\aleph_\alpha$, and let $\aleph_\gamma = \bigcup_{\alpha < \gamma} \aleph_\gamma$ for $\gamma$ a limit ordinal, which is a set by replacement. In particular, $\aleph_\omega$ exists; this is the sense in which replacement is a ``large cardinal axiom."
    Since $\Ord$ is well-ordered, for every cardinal $\kappa$ there is an ordinal $\alpha$ such that $\kappa = \aleph_\alpha$.

    Similarly, we define $\beth_0 = \omega$, $\beth_{\alpha + 1} = \card \pset \beth_\alpha$, $\beth_\gamma = \bigcup_{\alpha < \gamma} \beth_\gamma$.
\begin{lemma}
    If $\kappa \geq \aleph_0$ then there is a $\alpha \in \Ord$ such that $\kappa = \aleph_\alpha$.
\end{lemma}
\begin{proof}
    Let $\alpha$ be minimal among those for which $\aleph_\alpha \geq \kappa$. Assume that $\aleph_\alpha > \kappa$. If $\alpha$ is a limit ordinal, then there is a $\beta < \alpha$ such that $\aleph_\beta > \kappa$, so we might as well assume $\alpha = \gamma + 1$ for some $\gamma$. Then we have $\aleph_\gamma < \kappa < \aleph_\alpha$, a contradiction.
\end{proof}

We now consider the two axioms that will motivate much of the rest of what we do. They are independent of ZFC, but proving this is highly nontrivial.
\begin{axiom}[continuum hypothesis]
    \index{continuum hypothesis}
    $\card \RR = \aleph_1$.
\end{axiom}
\begin{axiom}[generalized continuum hypothesis]
    \index{generalized continuum hypothesis}
    For every $\alpha$, $\aleph_\alpha = \beth_\alpha$.
\end{axiom}
Obviously GCH implies CH. In the absence of choice these are a little silly, because we have no reason to believe that $\card \RR$ is even well-defined.

We now set up cardinal arithmetic. We define $\kappa + \lambda = \card(\kappa \sqcup \lambda)$ and $\kappa\lambda = \card(\kappa \times \lambda)$. This ends up being completely trivial.
\begin{theorem}
If $\kappa$ is infinite, then
$$\kappa + \lambda = \kappa\lambda = \max(\kappa, \lambda).$$
\end{theorem}
\begin{proof}
We have
$$\max(\kappa, \lambda) \leq \kappa + \lambda \leq \kappa\lambda \leq \max(\kappa, \lambda) \max(\kappa, \lambda) = \max(\kappa, \lambda).$$
The only step in this that isn't trivial is $\max(\kappa, \lambda)^2 = \max(\kappa, \lambda)$. But this is true of any infinite set.
\end{proof}

On the other hand, exponentiation is highly nontrivial. We let $\kappa^\lambda$ be the cardinality of the set of all functions $\lambda \to \kappa$.

Let $\kappa$ be an infinite cardinal. Then $\kappa^\kappa = 2^\kappa$, so $\kappa^\kappa > \kappa$ by the diagonal argument. This (and one more theorem) is pretty much all we can prove about cardinal exponentiation. Just about everything can be proven independence using forcing.

More generally, if $2 \leq \kappa \leq \lambda$ and $\lambda \geq \aleph_0$, then $\kappa^\lambda = 2^\lambda$. So we might as well assume $\kappa > \lambda$ unless we are studying $2^\lambda$.

\begin{definition}
A function $f: \alpha \to \kappa$ is \dfn{cofinal} if $f(\alpha)$ is unbounded in $\kappa$. The \dfn{cofinality} of $\kappa$, $\cof\kappa$, is the least $\alpha$ such that there is a cofinal $f: \alpha \to \kappa$.
\end{definition}
Obviously $\cof \kappa \leq \kappa$.
\begin{definition}
    A \dfn{regular cardinal} is a $\kappa \in \Card$ such that $\cof \kappa = \kappa$. Otherwise, $\kappa$ is a \dfn{singular cardinal}.
\end{definition}
It is easy to see that $\cof \aleph_0 = \cof \aleph_\omega = \aleph_0$. So $\aleph_0$ is a regular cardinal, while $\aleph_\omega$ is singular.
\begin{lemma}
    Let $\kappa \in \Card$. There is a strictly increasing cofinal function $\cof \kappa \to \kappa$.
\end{lemma}
\begin{proof}
    There is a cofinal function $f: \cof \kappa \to \kappa$. Now let $g: \cof \kappa \to \kappa$ be given by
    $$\alpha \mapsto \max(f(\alpha), 1+\sup_{\beta<\alpha}g(\beta)).$$
    Then $g$ is strictly increasing and cofinal.
\end{proof}
\begin{lemma}
    Suppose $f:\kappa \to \lambda$ is cofinal and nondecreasing. Then $\cof\kappa = \cof\lambda$.
\end{lemma}
\begin{proof}
    For $\cof \lambda \leq \cof \kappa$, let $g: \cof\kappa \to \kappa$ be cofinal, then $f \circ g$ is cofinal. For the converse, let $g: \cof\lambda\to \lambda$ be cofinal and strictly increasing (possible by the above lemma). We let $h: \cof \lambda \to \kappa$ be given by sending $\alpha$ to the least $\beta$ such that $f(\alpha) > g(\beta)$, which is possible since $f$ is cofinal. Then $h$ is cofinal since $g$ is.
\end{proof}
    In particular, $\cof \cof \kappa = \cof \kappa$. It's not hard to show that $\cof \kappa$ is a cardinal, so $\cof \kappa$ is a regular cardinal.
\begin{theorem}
    Assume choice. If $\kappa$ is a successor cardinal then $\kappa$ is regular.
\end{theorem}
\begin{proof}
    Assume $\kappa$ is the successor of $\lambda$. If $\kappa$ is singular then there is a cofinal function $f: \lambda \to \kappa$. For each $\alpha \in \lambda$ we have $f(\alpha) < \kappa$, so $\card f(\alpha) \leq \lambda$, which is well-defined by choice. So there is a surjective function $g_\alpha: \lambda \to f(\alpha)$. Now let $g(\beta) = \bigcup_\alpha g_\alpha(\beta)$. Then $g$ is surjective and maps onto $\kappa$, a contradiction.
\end{proof}
Now we can prove SOMETHING about cardinal exponentiation.
\begin{theorem}
    If $\kappa \geq \aleph_0$ then $\kappa^{\cof \kappa} > \kappa$.
\end{theorem}
\begin{proof}
Assume $f: \cof \kappa \to \kappa$ is cofinal and $\kappa^{\cof \kappa} = \kappa$. Enumerate $\kappa^{\cof \kappa}$ as $g_\alpha$, for $\alpha < \kappa$. So define $g: \cof \kappa \to \kappa$ by sending $\beta$ to the least element of $\kappa$ such that $g_\alpha(\beta)$ does not have $\alpha < f(\beta)$. Then $g(\beta) \neq g_\alpha(\beta)$ for any $\alpha < f(\beta)$. Since $f$ is cofinal, $g \neq g_\alpha$ for any $\alpha$. So we have diagonalized.
\end{proof}

We now arrive at a large cardinal axiom.
\begin{definition}
A \dfn{weakly inaccessible cardinal} $\kappa$ is a regular limit cardinal such that $\kappa > \aleph_0$. An \dfn{inaccessible cardinal} is a weakly inaccessible cardinal $\kappa$ such that if $\lambda < \kappa$ then $2^\lambda < \kappa$.
\end{definition}
If $\kappa$ is weakly inaccessible, then $\aleph_\kappa = \kappa$. But $\kappa$ is much larger than the least $\aleph$-fixed point, since the limit $\lambda$ of $\aleph_0,\aleph_{\aleph_0},\aleph_{\aleph_{\aleph_0}},\dots$ is an $\aleph$-fixed point, yet $\lambda \leq \kappa$ and $\cof \lambda = \aleph_0$, so $\lambda < \kappa$. Moreover, if $\kappa$ is inaccessible and choice holds, then $V_\kappa$ is closed under every set-theoretic operation, so $(V_\kappa, \in)$ is a model of ZFC. (Weakly inaccessible cardinals already give models of ZFC minus power set, since inaccessibility gives replacement.) So their consistency strength is much stronger than that of ZFC, since they prove the consistency of ZFC.

However, it is often useful (e.g. in the foundations of category theory) to assume the following large cardinal axiom.
\begin{axiom}[Grothendieck]
  \index{Grothendieck's axiom}
There is a proper class of inaccessible cardinals.
\end{axiom}





\chapter{Constructibility}
\section{Definability}
We introduce a notion of quantifier complexity for LST.
\begin{definition}
A formula has \dfn{bounded quantifiers} if every quantification only ranges over sets.
\end{definition}
For example, $\forall y \in x(y \neq y)$ has bounded quantifiers (and defines $\emptyset$).
\begin{definition}
A formula $\varphi$ is $\Sigma_0$ and $\Pi_0$ if it has bounded quantifiers. It is $\Sigma_{n+1}$ if it can be expressed as $\exists x_1 \exists x_2 \cdots \exists x_k \psi$ for some $\psi$ which is $\Pi_n$. It is $\Pi_n$ if it can be expressed as $\neg\psi$ for some $\psi$ which is $\Sigma_n$. It is $\Delta_n$ if it is both $\Sigma_n$ and $\Pi_n$.
\end{definition}

\begin{theorem}
Suppose $M \subseteq N \subseteq V$ is a chain of transitive models. Suppose $\psi$ is a formula and $a \in M^n$. Then:
\begin{enumerate}
\item If $\psi \in \Delta_0$ then $M \models \psi(a)$ iff $N \models \psi(a)$.
\item If $\psi \in \Sigma_1$ and $M \models \psi(a)$ then $N \models \psi(a)$.
\item If $\psi \in \Pi_1$ and $N \models \psi(a)$ then $M \models \psi(a)$.
\end{enumerate}
\end{theorem}
\begin{proof}
In the $\Delta_0$ case, if $a_1, a_2 \in M$ then $M \models a_1 \in a_2$ iff $a_1 \in a_2$ iff $N \models a_1 \in a_2$. Similarly for $a_1 = a_2$. Something similar happens here for $\Sigma_1$ and $\Pi_1$ but obviously it only works in one direction.

Now suppose that the theorem holds for $\varphi$ and $\psi$. Then obviously the theorem holds for $\neg\psi$ and $\varphi \wedge \psi$. Finally, we check the theorem for $\exists$. Note that $M \models \exists x \in a_1 \psi(x, a)$ iff there is a $b \in a_1$ such that $b \in M$ and $M \models \psi(b, a)$. Similarly for $N$. Here upward absoluteness follows because $M \subseteq N$. For downward absoluteness, we know $b \in a_1$, $b \in N$, $a_1 \in M$ and must show $b \in M$. This follows because $M$ is transitive.
\end{proof}

\begin{definition}
Let $M$ be a set. Let $\Def^M$, the \dfn{definable power set} of $M$, be the set of $X \subseteq M$ which are definable with parameters taken from $M$.
\end{definition}
The definable power set exists, by the axiom of power set. Everything in $M$ is definable from $M$; take $x = m$ to define $m \in M$. So $M \subseteq \Def^M \subseteq \pset M$. If $M$ is finite then $\Def^M = \pset M$; if $M$ is infinite then $\card \Def^M = \aleph_0\card M = \card M$.

\section{The reflection principle}
\begin{definition}
Suppose $\kappa$ is a regular uncountable cardinal. A set $S \subseteq \kappa$ is a \dfn{club set} if for every $\alpha < \kappa$ there is a $\beta \in C$ such that $\beta > \alpha$ and if $\alpha = \sup(S \cap \alpha)$ then $\alpha \in S$.
\end{definition}
So $S$ is closed and unbounded in the topology of $\kappa$.

Recall that $M \preceq N$ means that $M$ is an elementary substructure of $N$.

\begin{theorem}
Let $\kappa$ be a regular uncountable cardinal and suppose that we have a chain of LST-models $M_\alpha = (M_\alpha, E_\alpha)$, $\alpha < \kappa$, so $M_\alpha \subseteq M_\beta$ whenever $\alpha < \beta$. Suppose that for every $\alpha < \kappa$, $\card M_\alpha < \kappa$, and for every limit $\beta < \kappa$, $M_\lambda = \bigcup_{\alpha < \beta} M_\alpha$.

Let $M = \bigcup_{\alpha < \kappa} M_\alpha$, $E = \bigcup_{\alpha < \kappa} E_\alpha$, so $M = (M, E)$ is the injective limit of the $M_\alpha$. Let $S \subseteq \kappa$ be the set of $\alpha$ such that $M_\alpha \preceq M$. Then $S$ is a club set.

In particular, for any $\alpha, \beta \in S$, if $\alpha < \beta$ then $M_\alpha \preceq M_\beta$.
\end{theorem}
\begin{proof}
To see that $S$ is closed, let $\beta$ be a limit point of $S$. We must show $M_\beta \preceq M$, so let $\psi$ be a formula, $b \in M_\beta^n$. Then we can find a $\alpha < \beta$ such that $b \in M_\alpha^n$ and $M_\alpha \preceq M$, since $\beta$ is a limit point.

Suppose that there is a $a \in M$ such that $M \models \psi(a, b)$. By the Tarski-Vaught test, there is a $a_\alpha \in M_\alpha$ such that $M_\alpha \models \psi(a_\alpha, b)$. Since $M_\beta$ is a limit, we can take $\alpha$ so large as to guarantee $M_\beta \models \psi(a_\alpha, b)$. Therefore $M_\beta \preceq M$ by the Tarski-Vaught test. Therefore $S$ is closed.

Given $\psi$, let $f_\psi: \kappa \to \kappa$ send $\alpha$ to the least $\beta$ such that for all $b \in M_\alpha^n$, if there is a $a \in M$ such that $M \models \psi(a, b)$ then there is a $a_\beta \in M_\beta$ such that $M_\beta \models \psi(a_\beta, b)$.

To see that $f_\psi$ is well-defined, fix $\psi$ and $\alpha$ and suppose that no such $\beta$ exists. There are $|M_\alpha|^n = |M_\alpha| < \kappa$ choices of $b$. Let $\beta_b$ be a witness to the failure of the theorem for $b$. Then $b \mapsto \beta_b$ is cofinal $M_\alpha^n \to \kappa$, so $\kappa$ is singular.

Let $g: \kappa \to \kappa$ be defined by
$$g(\alpha) = \sup_\psi f_\psi(\alpha).$$
The mapping $\psi \mapsto f_\psi(\alpha)$ is cofinal $\omega \to \kappa$ if it is unbounded; since $\kappa$ is regular, it follows that the mapping is bounded, so $g$ is well-defined.

Suppose that $g$ restricts to a function $\alpha \to \alpha$. By definition of $g$ and the Tarski-Vaught test, for every $\beta < \alpha$ and $b \in M_\beta^n$, if there is a $a \in M$ such that $M \models \psi(a, b)$ then there is a $a' \in M_{g(\beta)}$ such that for some $\gamma \leq g(\beta)$, $M_\gamma \models \psi(a', b)$, hence $M_\alpha \preceq M$.

Now $g$ is an increasing function, so for any $\alpha_0$ the ordinal
$$\alpha = \sup_{n < \omega} g^{\circ n}(\alpha_0)$$
is closd under $g$ and is larger than $\alpha_0$. Taking $\alpha_0$ arbitrarily large we see that $\alpha$ is arbitrarily large, but $\alpha \in S$, so $S$ is unbounded.

The last paragraph is obvious.
\end{proof}

\begin{theorem}[reflection]
  \index{reflection principle}
  The class of ordinals $\alpha$ such that $V_\alpha \prec V$ is a club.
\end{theorem}
\begin{proof}
The $V_\alpha$, $\alpha \in \Ord$, meet the hypotheses of the above theorem (this is obvious if ZFC is consistent, and otherwise it follows by the principle of explosion).
\end{proof}
So it is impossible to distinguish $V$! Anything that we can say about it was already true for a club class of submodels.

\section{The constructible sets}
We now consider those sets which are constructible from the ordinals.

\begin{definition}
Let $L_0 = 0$, $L_{\alpha + 1} = \Def^{L_\alpha}$, and $L_\gamma = \bigcup_{\alpha < \gamma} L_\alpha$ for $\gamma$ a limit ordinal. Let $L = \bigcup_{\alpha \in \Ord} L_\alpha$. If $x \in L$, we say that $x$ is a \dfn{constructible set}.
\end{definition}
By induction, one can easily check that $L_\alpha$ is a transitive set and $L_\alpha \subset L_\beta$ for $\alpha < \beta$. Moreover, we have $L_\alpha \subseteq V_\alpha$, and every finite subset of $L_\alpha$ lies in $L_{\alpha + 1}$. Moreover, for $\alpha \leq \omega$, we have $L_\alpha = V_\alpha$. But, on the other hand, if $\alpha \geq \omega$ then $
\card L_\alpha = \aleph_0(\card \alpha) = \card \alpha$ so $\card L_\alpha = \card \alpha$, which usually (but not always) implies $L_\alpha \neq V_\alpha$.

\begin{theorem}
$L$ is a model of ZF.
\end{theorem}
\begin{proof}
Since $L$ is a transitive class, extensionality and foundation hold. Clearly $\emptyset, \omega \in L$ so empty set and infinity hold. Pairing, union, replacement, and power set follow from the definition of definability. The only nontrivial axiom is replacement. Suppose that for every $A, w \in L$ and every $x \in A$ there is a unique $y \in L$ such that $\phi^L(x, y, A, w)$. Let $\alpha$ be the sup, taken over the ranks (in $L$) of all $y \in L$ such there is an $x \in A$ with $\phi^L(x, y, A, w)$.

Let $Y = L_{\alpha + 1}$; then $Y \in L$ and for every $y \in V$ such that there is a $x \in A$ such that $\phi^L(x, y, A, w)$, $y \in Y$. Therefore $y \in L$, so $L$ is a model of replacement.
\end{proof}

\begin{axiom}[$V = L$]
  \index{$V=L$}
    Every set is constructible.
\end{axiom}
In other words, $V_\alpha = L_\alpha$ for every $\alpha \in \Ord$.

Since the notion of ``definability" is absolute, and $\Ord$ is absolute to $L$, it follows that $L_\alpha$ is absolute to $L$. In particular, $L$ proves that for every $x \in V$, $x \in L$, so $L \models (V=L)$.
\begin{theorem}
If ZF is consistent, then $V = L$ is consistent.
\end{theorem}
\begin{proof}
If $V$ exists, then so does $L$, and $L$ is a model of $V = L$.
\end{proof}

In fact, $L$ is in some sense the minimal inner model of ZF.
\begin{lemma}
Let $M$ be a transitive proper class, and suppose $M$ is a model of ZF. Then $\Ord$ is absolute to $M$, and $\Ord$ is a subclass of $M$.
\end{lemma}
\begin{proof}
The rank of a set $x$ is defined by recursion on $x$. Therefore rank is absolute to $M$. If $\alpha \in \Ord$, $M$ is not contained in $V_\alpha$ since $M$ is a proper class. So there is a $x \in M \setminus V_\alpha$ whose rank $\beta$ has $\beta \geq \alpha$; therefore $\beta \in M$, and since $M$ is transitive, $\beta \in M$. So $\Ord^M = \Ord$.
\end{proof}

\begin{theorem}
Let $M$ be a transitive proper class, and suppose $M$ is a model of ZF. Then $L$ is absolute to $M$, and $L$ is a subclass of $M$.
\end{theorem}
\begin{proof}
Since $\Ord$ is absolute to $M$, so is $L_\alpha$. Therefore $L^M = L$.
\end{proof}

We now show that $L$ satisfies an especially strong form of the axiom of choice.
\begin{axiom}[global choice]
  \index{axiom of global choice}
  There is a well-ordering of $V$.
\end{axiom}
In fact, the axiom of global choice is equivalent to the existence of a class bijection $V \to \Ord$. Clearly such a bijection gives a well-ordering, and conversely, if there is a well-ordering $<$ of $V$, then every ordinal must embed into $(V, \in)$, which is only possible if $(V, \in) \cong \Ord$ by definition of $\Ord$. The axiom of global choice implies the axiom of choice, since it restricts to a well-ordering of any set.

\begin{theorem}
$L$ is a model of global choice, hence of ZFC. Stronger, its well-ordering is definable.
\end{theorem}
\begin{proof}
We will define a well-ordering $<_\alpha$ of $L_\alpha$ as follows. Let $<_0 = 0$ and let the well-ordering of a limit ordinal be the limit of the well-orderings. Let $<_\alpha^n$ denote the lexicographic ordering on $L_\alpha^n$. Fix a definable enumeration $E_n$ of the set of definable $n$-ary relations on $R$.

For $x \in L_{\alpha + 1}$ let $n_x$ denote the least $n$ such that there is a $s \in L_\alpha^n$ and a definable $(n+1)$-ary relation $R$ on $L_\alpha$ such that $x = \{y \in L_\alpha: (s, y) \in R\}$. Let $s_x$ be the $<_\alpha^{n_x}$th least witness to the definition of $n_x$. Let $m$ be the least index of $R$ in $E_{n+1}$.

For $x, y \in L_{\alpha + 1}$ define $x <_\alpha y$ iff $x, y \in L_\alpha$ and $x <_\alpha y$, $x \in L_\alpha$ and $y \notin L_\alpha$, or $x, y \notin L_\alpha$ and either $n_x < n_y$, or if they are equal then $s_x <_\alpha^{n_x} s_y$, or if they are also equal then $m_x < m_y$.

If $n_x = n_y$, $s_x = s_y$, and $m_x = m_y$, then $x$ and $y$ are defined by the same relation so $x = y$ by extensionality. By induction, $<_\alpha$ is a well-ordering of $L_\alpha$.

Now write $L = L_0 \cup (L_1 \setminus L_0) \cup (L_2 \setminus L_1) \cup \cdots$ and order the $\alpha$th entry in the above disjoint union decomposition using $<_\alpha$.
\end{proof}

\begin{corollary}
ZF with the axiom of global choice, and hence ZFC, is consistent if ZF is.
\end{corollary}

\begin{corollary}
If ZF is consistent, then it is consistent that $\RR$ has a definable well-ordering.
\end{corollary}
However, we cannot prove that such a well-ordering is actually a well-ordering of $\RR$.

\begin{theorem}
    If $V = L$, then for every $\kappa$, $2^\kappa = \kappa^+$.
\end{theorem}
We prove this in the chapter on one measurable cardinal; this follows from Solovay's theorem. The proof is exactly the same except with $U$ removed.





\chapter{Forcing}
The idea behind forcing is that given a model $M$ of ZFC, we want to adjoin a ``generic" set $x \notin M$ so that the smallest model $M[x]$ containing $M$ and $x$ has a certain property $p$. In this chapter we'll go through this procedure with $p = (2^{\aleph_0} = \aleph_2)$.

\section{Generic filters}
Let $M$ be a transitive model of ZFC. (This might not exist, since the existence of transitive models is strictly stronger than the existence of models, which implies that ZFC is consistent by soundness, but we can remove the transitivity assumption later.) It will be frequently be useful to assume that $M$ is countable, which is always possible up to elementary equivalence by the Lowenheim-Skolem theorem.

Let $\PP \in M$ be a poset, and let $1_\PP \in \PP$ be the maximum of $\PP$.

\begin{definition}
    An element $x \in \PP$ is called a $\PP$-\dfn{condition}, and a subset $D \subseteq \PP$ is \dfn{dense} if for every $p \in \PP$ there is a $d \in D$ with $d \leq p$.
\end{definition}
    Proper dense subsets exist, because any downward slice of $\PP$ is dense.

\begin{definition}
    If $x, y \in \PP$ are such that there is $z \in \PP$ with $z \leq x$ and $z \leq y$, then $x$ and $y$ are \dfn{compatible conditions} and we write $x \parallel y$. Otherwise, we write $x \perp y$.

    If $X \subseteq \PP$ is such that every pair in $X$ is incompatible then we say that $X$ is an \dfn{antichain}.
\end{definition}
\begin{definition}
    Let $G \subseteq \PP$ be a filter. We say that $G$ is an $M$-\dfn{generic filter} if for every $D \in M$ which is dense in $\PP$, $D \cap G$ is nonempty.
\end{definition}
    Since we think of $x, y \in \PP$ as conditions, the ordering $\leq$ is an ordering by logical strength. Indeed, $1_\PP$ is the maximal and hence weakest condition (since $G$ is a filter, $1_\PP \in G$ -- $G$ is nonempty since $\PP$ is dense). If $x \leq y$ then $x$ is a weaker condition than $y$ (since $y \in G$ implies $x \in G$). Since $G$ is a filter, any pair in $G$ is compatible.

    Notice that since $\PP$ is in general infinite, its power set $\pset \PP$ is not absolute. That is, if $N$ is a model of ZFC, $(\pset \PP)^M$ might disagree with $(\pset \PP)^N$. That is why it makes sense to define genericity with respect to a particular model $M$.
\begin{lemma}
    For every dense set $D \subseteq \PP$ there is a maximal antichain in $D$.
\end{lemma}
\begin{proof}
    Zorn's lemma.
\end{proof}

\begin{theorem}[Rasiowa-Sikorski]
    \index{Rasiowa-Sikorski theorem}
    If $M$ is countable, then there is an $M$-generic filter in $\PP$.
\end{theorem}
This turns out to be the same as the Baire category theorem (or, equivalently, the axiom of dependent choice). This motivates the terminology ``dense" and ``generic."
\begin{proof}
    Since $M$ is countable, the set $\mathcal D$ of dense sets in $\PP$ is countable, say $\mathcal D = \{D_n: n \in \omega\}$. Let $p_1 \in D_1$. We can inductively choose $p_{n+1} \leq p_n$ such that $p_{n+1} \in D_{n+1}$ since $D_{n+1}$ is dense. Now let $G$ be the smallest filter containing the sequence of $p_n$. Clearly $G$ meets every $D_n$.
\end{proof}

\begin{definition}
    $\PP$ is \dfn{splitting} for every $x \in \PP$, there are $y, z \leq x$ with $y \perp z$.
\end{definition}
    That is, for every condition, there are two incompatible ``possible futures."

\begin{lemma}
    If $\PP$ is splitting and $F \subseteq \PP$ is a filter, then the ideal $F^c$ is dense.
\end{lemma}
\begin{proof}
    If $x \in F$ then there are $y, z \leq x$ which are incompatible. But every pair in $F$ is compatible, so one of them lies in $F^c$. Therefore $F^c$ is dense.
\end{proof}
\begin{theorem}
    If $\PP$ is splitting and $G \subseteq \PP$ is a $M$-generic filter then $G \notin M$.
\end{theorem}
\begin{proof}
    Suppose not. Then by the lemma, $G^c$ is dense, but since $G \in M$, $G^c \in M$. Since $G$ is $M$-generic, $G$ meets every dense set, so $G \cap G^c$ is nonempty, which is absurd.
\end{proof}
\begin{example}
    Let $\PP$ be the infinite binary tree, which is splitting. Then a filter is a branch, and a generic set is an infinite path. So the generic sets are exactly the infinite binary sequences $2^\omega$.
\end{example}

\section{Constructing the generic extension}
Recall that $M$ is a transitive model of ZF, and $\PP \in M$ is a poset with a maximum $1_\PP \in \PP$ and an filter $G \subseteq \PP$. We will eventually assume that $\PP$ is splitting and $G$ is $M$-generic, so that $G \notin M$, but not yet.

We now construct a class $\Name(\PP)$ from $\PP$, which admits an injection $V \to \Name$, by transfinite recursion. First, $\Name_0(\PP) = \emptyset$. If $\Name_\alpha(\PP)$ is defined then
$$\Name_{\alpha + 1}(\PP) = \pset(\Name_\alpha(\PP) \times \PP).$$
Finally we take unions at limit stages, and let $\Name(\PP) = \bigcup_\lambda \Name_\lambda(\PP)$.
\begin{definition}
    A $\PP$-\dfn{name} is an element of $\Name(\PP)$. The \emph{rank} of a $\PP$-name $\tau$ is the least $\lambda$ such that $\tau \in \Name_\lambda(\PP)$.

    We define the $G$-\dfn{interpretation} $\tau^G$ of a $\PP$-name $\tau$ by transfinite recursion. Namely, $\tau^G$ is the set of all interpretations $\sigma^G$ such that the $\PP$-name $\sigma$ has $(\sigma, p) \in \tau$ with $p \in G$.
\end{definition}
We think of the set of interpretations
$$X = \{\tau^G: \tau \in \Name_\lambda(\PP)\}$$
as the ``expansion of $V_\lambda$ by $G$", so ``$X = V_\lambda[G]$".
\begin{example}
    We always have
    $$\Name_2(\PP) = \pset(\{\emptyset\} \times \PP) = \{(\emptyset, p): p \in \PP\}.$$
    So $\Name_2(\PP)$ can be identified with $\pset \PP$. Therefore the interpretation set
    $$X = \{\tau^G: \tau \in \Name_2(\PP)\}$$
    has each element
    $$\tau^G = \{\emptyset: (\emptyset, p) \in \tau, p \in G\}$$
    which means $\tau^G = \{\emptyset\}$ iff there is a $p \in G$ (and $\tau^G = \emptyset$ otherwise). So $X = V_2$. In fact this is true for any $\Name_n$ and any $G$.
\end{example}

Now we can define the extension.
\begin{definition}
The \dfn{generic extension}
$$M[G] = \{\tau^G: \tau \in \Name^M(\PP)\}.$$
\end{definition}

\begin{theorem}
    $M \subseteq M[G]$ and $G \in M[G]$. Moreover, $M[G]$ is transitive.
\end{theorem}
Notice that while $\Name^M(\PP) \subseteq M$, the interpretations of the $\PP$-names may not live in $M$. For example, suppose that $G$ actually is $M$-generic and $\PP$ is splitting. Then $G \notin M$, even though $G \in M[G]$. So we truly have ``adjoined $G$ to $M[G]$."
\begin{proof}
We need to give an injection $M \to M[G]$. So we define the $\PP$-names
$$\hat x = \{(\hat y, 1_\PP): y \in x\}$$
by transfinite recursion. The map $x \mapsto \hat x$ is then an injection $M \to \Name^M(\PP)$.
Finally, we define
$$\dot G = \{(\hat p, p): p \in \PP\}.$$
Since these are $\PP$-names, we consider their interpretations. First,
$$\hat x^G = \{y: (\hat y, 1_\PP) \in \hat x\}$$
by transfinite induction. So we have $M \subseteq M[G]$. Moreover,
$$\dot G^G = \{\sigma^G: (\sigma, p) \in \dot G, p \in G\} = \{\hat p^G: (\hat p, p), p \in \PP\} = \{p: p \in G\} = G.$$
So indeed, $M \cup \{G\} \subseteq M[G]$.

Now we show that $M[G]$ is transitive. Let $x \in M[G]$ and $y \in x$. Then there is a $\sigma \in \Name^M(\PP)$ such that $\sigma^G = x$. So $y = \tau^G$ for some $\tau \in \Name^M(\PP)$. Then
$$\tau \in \sigma \in M.$$
Since $M$ is transitive, $\tau \in M$. Therefore $y \in M[G]$, so $M[G]$ is transitive.
\end{proof}

\begin{theorem}
    $M[G]$ is a model of extensionality, empty set, infinity, pairing, foundation, and union. If $M$ is a model of choice then so is $M[G]$.
\end{theorem}
    So $M[G]$ doesn't quite model ZF, but will happen when we assume that $G$ is actually $M$-generic.
\begin{proof}
    Extensionality and foundation follow because $M[G]$ is transitive; empty set and infinity follow because $M \subseteq M[G]$.

    For pairing, assume $x = \sigma^G$ and $y = \tau^G$. Then $\{x, y\}^{M[G]} = \{(\sigma, 1_\PP), (\tau, 1_\PP)\}^G$ so we're good. Union is similar, and same with choice -- just lift infinite products from $M[G]$ to $\Name^M(\PP)$.
\end{proof}

\section{Forcing semantics}
We now define a relation $\Vdash$. The idea is that if $p \in \PP$ is a condition, then $p \Vdash \varphi(\sigma_1, \dots, \sigma_n)$ iff $M[G] \models \varphi(\sigma_1^G, \dots, \sigma_n^G)$.

\begin{definition}
    Let $M$ be a transitive model of ZF and let $\PP \in M$ be a poset. For $p \in \PP$ and $\varphi$ a LST-formula (possibly with variables), write $p \Vdash \varphi$ to mean:
\begin{enumerate}
    \item If $\varphi = (\tau_1 = \tau_2)$, then for each $(\sigma_1, q_1) \in \tau_1$, let $D_{\sigma_1}^{q_1}$ be the set of all $r \in \PP$ such that if $r \leq q_1$, then there is a $(\sigma_2, q_2) \in \tau_2$ such that $r \leq q_2$ and such that $r \Vdash (\sigma_1 = \sigma_2)$, and let $D_{\sigma_2}^{q_2}$ be defined similarly. Then $D_{\sigma_i}^{q_i}$ are dense below $p$ for $i \in \{1, 2\}$.
    \item If $\varphi = (\tau_1 \in \tau_2)$, then let $D$ be the set of all $q \in \PP$ such that there is $(\tau, r) \in \tau_2$ such that $q \leq r$ and $q \Vdash (\tau = \tau_1)$. Then $D$ is dense below $p$.
    \item If $\varphi = (\exists x \psi(x))$ then let $D$ be the set of all $q \in \PP$ such that there is $\tau$ with $q \Vdash \psi(\tau)$. Then $D$ is dense below $p$.
    \item If $\varphi = \psi \vee \chi$, then $p \Vdash \psi$ and $p \Vdash \chi$.
    \item If $\varphi = \neg\psi$, then $p \not \Vdash \psi$.
\end{enumerate}
    If $p \Vdash \varphi$, then we say that $p$ \dfn{forces} $\varphi$.
\end{definition}
Notice that $D_{\sigma_1}^{q_1}$ encodes $\tau_1 \subseteq \tau_2$. Similarly for the other dense sets we defined.

Also notice that even though $\Vdash$ is a relation in $V$, $\Vdash$ is definable from $M$, since all we have referred to are $\PP$-names, which already lie in $M$.

We now show that $\Vdash$ behaves ``modally" as we desire. First, a very tedious induction that we skip shows the following.
\begin{lemma}
    Let $\Vdash$ be as above. Then:
\begin{enumerate}
    \item If $p \Vdash \varphi$ and $q \leq p$ then $q \Vdash \varphi$.
    \item If the set of $q \in \PP$ such that $q \Vdash \varphi$ is dense below $p$ then $p \Vdash \varphi$.
    \item If $p \not \Vdash \varphi$ then there is a $p' \leq p$ such that $p' \Vdash \neg \varphi$.
\end{enumerate}
\end{lemma}
    Interpreting $\Vdash$ modally, we think of $p \Vdash \varphi$ to mean that $p$ implies that $\varphi$ will be eventually true, and $q \leq p$ to mean that $q$ is a possible future state reachable from $p$. Then, the first condition means that if it is known that $\varphi$ will eventually be true, then it will continue to be known that $\varphi$ will be eventually be true. The second condition means that if in every possible future, we will learn that $\varphi$ will be eventually true, then $\varphi$ is already known to be eventually true in every possible future. The final condition means that if $\varphi$ is not known to be true then there is a possible future where it is false.

\begin{theorem}
    Suppose $M$ is a transitive model of ZF, $\PP \in M$ is a poset, $G \subseteq \PP$ is an $M$-generic filter, and $n \in \omega$. For each LST-formula $\varphi$ with $n$ free variables and each $\sigma_1, \dots, \sigma_n \in \Name^M(\PP)$,
    $$M[G] \models \varphi(\sigma_1^G, \dots, \sigma_n^G)$$
    if and only if there is a condition $p \in G$ such that
    $$p \Vdash \varphi(\sigma_1, \dots, \sigma_n).$$
\end{theorem}
\begin{corollary}
    Let $M$ and $G$ be as above. Then $M[G]$ is a model of ZF, and if $M$ is a model of ZFC then so is $M[G]$.
\end{corollary}
\begin{proof}
    We just need to check comprehension and replacement, and we'll just do comprehension since replacement is similar. For convenience we suppress parameters. Assume $\sigma^G \in M[G]$ and $\varphi$ is an LST-formula with a free variable. We need to show that the set
    $$A = \{x \in \sigma^G: \varphi(x)\} \in M[G].$$
    This can be rewritten as, for some $\rho \in \Name^M(\PP)$,
    $$A = \{\rho^G \in \sigma^G: \exists p \in G(p \Vdash \rho \in \sigma \wedge \varphi(\rho))\}.$$
    Now let
    $$\tau = \{(\rho, p) \in M \times \PP: p \Vdash \varphi(\rho)\}.$$
    Then obviously $\tau^G = A$ and $\tau \in V$. Since $M$ is a model of comprehension and $\Vdash$ is definable in $M$, it follows that $\tau \in \Name^M(\PP)$. So $A \in M[G]$.
\end{proof}

\section{Cardinal collapse}
    We set up some machinery for later.
\begin{definition}
    Let $C$ be an uncountable set of finite sets. Then $C$ is a \dfn{$\Delta$-system} if there is an $R \in C$ such that for every distinct pair $x, y \in C$, $x \cap y = R$.
\end{definition}
\begin{lemma}
    If $C$ is an uncountable set of finite sets, then there is a $\Delta$-system $C' \subseteq C$.
\end{lemma}
\begin{proof}
    By the infinite pigeonhole principle, we can find $n \in \omega$ such that uncountably many sets in $C$ have cardinality $n$, and we replace $C$ with the set of all sets in $C$ with cardinality $n$. Obviously if $n = 1$ then $R = \emptyset$ and we're done.

    Assume that the lemma is true for every $k < n$. If there is an $c \in C$ contained in uncountably many sets, then add $c$ to $C'$ and apply the inductive hypothesis. Otherwise, every $c \in C$ only appears in countably many sets. Now do the usual Baire category theorem trick.
\end{proof}
\begin{definition}
    A poset $\PP$ has the \dfn{countable chain condition} (or is \dfn{ccc}) if every antichain in $\PP$ is countable.
\end{definition}

\begin{theorem}[possible values]
    \index{possible values argument}
    Let $M$ be a transitive model of ZFC and $\PP \in M$ be a poset with an $M$-generic filter $G$. Let $X, Y \in M$ and let $f \in M[G]$ be a function $X \to Y$.

    If $M$ is a model of ``$\PP$ is ccc," then there is a function $F \in M$ such that for each $x \in X$, $f(x) \in F(x)$ and $M$ is a model of ``$F(x)$ is countable".
\end{theorem}
    Let $x \in X$. Notice that we do not assume $f \in M$, so $M$ does not know the value $f(x)$, even though $f(x) \in M$. We think of $F(x)$ as the ``possible values of $f$ that $M$ allows", so $M$ does have some control over its generic extension $M[G]$, ruling out but all but countably many possibilities.
\begin{proof}
    Let $\dot f \in \Name^M(\PP)$ be such that $\dot f^G = f$. Then there is a $p \in \PP$ which forces ``there exists a function $\dot f: \hat X \to \hat Y$".

    Since $M$ thinks $\PP$ is ccc and thinks AC is true, for each $x \in X$ we can use Zorn's lemma to find a maximal set $A(x)$ of incompatible conditions $q \leq p$ such that for each $q \in A(x)$ there is a $y \in Y$ such that $q$ forces ``$\dot f(\hat x) = \hat y$," and $A(x)$ is countable.

    Each of the $q \in A(x)$ forces $f(x)$ to admit a certain value, say $y_q$. So we define
    $$F(x) = \{y \in Y: \exists q \leq p(y = q_y)\}.$$
    By maximality of $A(x)$, $F(x)$ must hit every possible value of $y$ in every possible forcing extension.
\end{proof}

\begin{definition}
    Let $M$ be a transitive model of ZFC. A poset $\PP \in M$ \dfn{preserves cardinals} if for every $M$-generic filter $G \subseteq \PP$ and each ordinal $\lambda$, $M$ is a model of ``$\lambda \in \Card$" if and only if $M[G]$ is a model of ``$\lambda \in \Card$".
\end{definition}
\begin{theorem}
    Let $M$ be a transitive model of ZFC, $\PP \in M$ a poset, and $M$ is a model of ``$\PP$ is ccc". Then $\PP$ preserves cardinals.
\end{theorem}
\begin{proof}
    Taking limits, it suffices to check that $\PP$ preserves \emph{regular cardinals}; namely $M$ is a model of ``$\lambda$ is a regular cardinal" if and only if $M[G]$ is a model of ``$\lambda$ is a regular cardinal."

    Let $\lambda \geq \aleph_2^M$ be an $M$-regular cardinal, and suppose that $\lambda$ is not $M[G]$-regular. Then there is a function $f \in M[G]$, $\overline \lambda \to \lambda$ cofinal, fo some $\overline \lambda < \lambda$. By the possible values argument, there is a function $F \in M$, $\overline \lambda \to 2^\lambda$, such that $f(\alpha) \in F(\alpha)$ and
    $$\card^M F(\alpha) \leq \aleph_0^M < \lambda.$$
    Define a map $\overline \lambda \to \lambda$ by $g(\alpha) = \sup F(\alpha)$. Then $M$ is a model of ``$g$ is cofinal", even though $\lambda$ is regular in $M$, a contradiction.
\end{proof}




\section{Forcing $2^{\aleph_0} = \aleph_2$}
Our goal is to prove the following theorem.
\begin{theorem}[Cohen]
There is a model $M[G]$ of ZFC such that $M[G] \models \neg$ CH.
\end{theorem}

We have two problems forcing $2^{\aleph_0} = \aleph_2$. First, we need to find an $M$ and a $\PP$ such that for each $M$-generic filter $G \subseteq \PP$, $M[G]$ has $\aleph_2$ more real numbers than $M$ (and by constructing $M$ from $L$, we can arrange for $M[G]$ to have $\aleph_2 + \aleph_1 = \aleph_2$ many reals). Second, we need to show that $\aleph_2^M = \aleph_2^{M[G]}$.

\begin{definition}
    For $x \in V$,
    $\Add(x)$ is the set of partial functions $x \to 2$, ordered by reverse inclusion.
\end{definition}
So $p \leq q$ iff $dom(p) \supseteq dom(q)$, and $1_{\Add(x)}$ is the empty function.

\begin{lemma}
    For each $\kappa \in \Card$, $\Add(\kappa \times \omega)$ is ccc.
\end{lemma}
\begin{proof}
    If not, let $\{p_\alpha\}$ be an antichain of length $\aleph_1$, and let $C = \{dom(p_\alpha)\}$. Then $\card C = \aleph_1$, yet $C$ consists only of finite sets, so there is a $\Delta$-system in $C$ with root $R$. Let $B = \{p_\alpha: dom(p_\alpha) \in R\}$. Then each $p_\alpha \in B$ determines a total function $R \to \{0, 1\}$ and $R$ is finite, so $\pset R$ is finite, yet there are supposed to be uncountably many $p_\alpha$.
\end{proof}

\begin{theorem}
    Let $M$ be a countable, transitive model of ZFC + CH, and let
    $$\PP = (\Add(\aleph_2 \times \omega))^M.$$
    Let $G$ be an $M$-generic filter. Then $M[G]$ is a model of ZFC + ``$2^{\aleph_0} = \aleph_2$".
\end{theorem}
    Notice that $\PP$ is splitting. Moreover, such a $G$ exists by the Baire category theorem, since $M$ is transitive, so $G \notin M$.
\begin{proof}
    By thinking of $G$ as a ``measure" on $\aleph_2 \times \omega$, we get a function $\hat G: \aleph_2 \times \omega \to 2$. For each $\alpha \in \aleph_2$, we define
    $$G_\alpha = \{n \in \omega: \hat G(\alpha, n) = 0\}.$$
    Given $\alpha, \beta \in \aleph_2$, the set
    $$D = \{q \in \PP: \exists n \in \omega(q(\alpha, n) \neq q(\beta, n))\}$$
    is dense, so $D$ meets $G$. Therefore the map $\alpha \mapsto G_\alpha$ is injective. The $G_\alpha$ interpret to give new elements of $\pset \omega = 2^{\aleph_0}$ in $M[G]$. So $\card M[G] = \aleph_2^M$, and it remains to show $\aleph_2^M = \aleph_2^{M[G]}$. This happens because $\PP$ is ccc in $M$.
\end{proof}



\chapter{Small large cardinals}
\begin{definition}
A \dfn{regular cardinal} is an infinite cardinal $\kappa$ such that for any set $\Lambda$ of cardinals such that $\card \Lambda < \kappa$ and for every $\lambda \in \Lambda$, $\lambda < \kappa$,
$$\sum_{\lambda \in \Lambda} \lambda < \kappa.$$
A \dfn{singular cardinal} is an infinite cardinal which is irregular.
\end{definition}

Informally, a large cardinal is a regular cardinal which is a witness to some existential axiom $\Phi$ such that $\Phi$ proves that ZFC is consistent. So the following definition is the smallest example of a large cardinal.
\begin{definition}
A \dfn{worldly cardinal} is a regular cardinal $\kappa$ such that $V_\kappa$, the set of all sets of rank $<\kappa$, is a model of ZFC.
\end{definition}

\section{Regular cardinals}
For $f$ a function and $X \subseteq \dom f$ a set, let
$$f"X = \{f(x): x \in X\}$$
be the image of $X$ under $f$.

\begin{definition}
Let $\alpha$ be an ordinal. We say that $X \subseteq \alpha$ is \dfn{cofinal} if for every $\beta < \alpha$ there is a $\gamma \in X$ such that $\beta < \gamma < \alpha$.
If $f: \beta \to \alpha$ is a function and $f"\beta$ is cofinal, we say that $f$ itself is \dfn{cofinal}.
\end{definition}
\begin{definition}
Let $\alpha$ be an ordinal. The \dfn{cofinality} of $\alpha$, $\cof \alpha$, is the least $\beta < \alpha$ such that there is a cofinal function $\beta \to \alpha$.
\end{definition}

We recall that if $\Lambda$ is a set of infinite cardinals, or $\card \Lambda \geq \aleph_0$, we have
$$\sum_{\lambda \in \Lambda} \lambda = \sup_{\lambda \in \Lambda} \lambda,$$
by triviality of cardinal arithmetic. In particular, an infinite cardinal $\kappa$ is regular iff $\cof \kappa = \kappa$.
As a consequence, since $\cof \cof = \cof$, $\cof \alpha$ is a regular cardinal for any $\alpha$.
\begin{example}
$\aleph_0$ is a regular cardinal, since any finite sum of finite numbers is finite. On the other hand, $\aleph_\omega$ is a singular cardinal, since
$$\aleph_\omega = \sup_{n < \omega} \aleph_n.$$
In fact, $\cof \aleph_\omega = \omega$. This actually motivated the axiom schema of replacement (the ``F" in ZFC), since it seems like singular cardinals should exist, yet this is not provable in Zermelo theory alone.
\end{example}

\begin{lemma}
Let $\alpha$ be an ordinal. Then $\cof \alpha$ is the least cardinal $\kappa$ such that there is a partition of $\alpha$ into $\kappa$ many sets of cardinality strictly less than $\card \alpha$.
\end{lemma}
\begin{proof}
Replacing $\alpha$ with $\card \alpha$, we may assume that $\alpha$ is a cardinal. Then notice that the existence of a partition of $\alpha$, $(S_\beta: \beta < \alpha)$ is the same thing as asserting that
$$\sum_{\beta < \alpha} \card S_\beta = \alpha.$$
Thus we may apply the above definition of regularity.
\end{proof}


\section{Inaccessible cardinals}
One way we could ``access" a cardinal from those below it is by summation; this is impossible for regular cardinals. The other way we might hope to is by applying the power set operation.

\begin{definition}
An \dfn{inaccessible cardinal} is a regular cardinal $\kappa$ such that if $\gamma < \kappa$, then $2^\gamma < \kappa$.
\end{definition}

Recall that $V_\kappa$ is the set of all sets of rank $< \kappa$; i.e., those sets of cardinality $< \kappa$ whose elements are sets of rank $< \kappa$.
\begin{example}
$V_\omega$ is the set of all hereditarily finite sets and is roughly equivalent to Peano arithmetic. $V_{\omega + k}$ is the set of all sets that appear in $k + 1$th order arithmetic. $V_{\omega + \omega}$ contains all sets that appear in ``ordinary mathematics." (I think $V_{\omega + \omega}$ is already a model of Borel determinacy since it contains the $\omega$th power set of $\omega$, but I'm not sure.)
\end{example}

We now show that inaccessible cardinals can prove the consistency of ZFC, and hence are much stronger than it. In fact, inaccessibles dwarf even worldly cardinals.
\begin{definition}
A \dfn{Grothendieck universe} $U$ is a transitive uncountable set which is closed under pairing and power set, and closed under unions of size $\gamma$, where $\gamma$ is the cardinality of an element of $U$.
\end{definition}
\begin{lemma}
Every Grothendieck universe $U$ gives a model of ZFC of the form $(U, \in)$, and contains $V_\omega$.
\end{lemma}
\begin{proof}
Obviously $U$ satisfies extensionality, foundation, pairing, power set, and union. Since $U$ is transitive it satisfies empty set and choice.
Since $U$ is transitive and satisfies pairing, it contains products and hence contains all functions $x \to y$ where $x, y \in U$.
Therefore $U$ satisfies replacement.

Since $U$ satisfies empty set and pairing, $1 \in U$ and hence $n \in U$ for any $n$. Therefore $\omega \subseteq U$, yet $U$ is uncountable and transitive, so
contains all subsets of a lesser cardinality than itself. In particular, $\omega \in U$, so $U$ satisfies infinity.
Since $U$ is transitive, it follows that $U$ contains $V_\omega$.
\end{proof}
\begin{corollary}
If there is a Grothendieck universe, then ZFC is consistent. In particular, ZFC cannot prove that Grothendieck universes exist.
\end{corollary}
\begin{proof}
Completeness and incompleteness.
\end{proof}
\begin{example}
If we did not assume Grothendieck universes were uncountable then $V_{-1} = \emptyset$ and $V_\omega$ would be Grothendieck universes. In fact, $V_\omega$ is a model of ZFC minus the axiom of infinity.
\end{example}
By Maddy's principle of whimsical identity, i.e. the philosophical (not mathematical!) axiom that no ordinal can be uniquely determined by any \dfn{whimsical identities}, i.e. properties not forced on it by its definition, this implies that there is a ordinal $\alpha$ such that $V_\alpha$ is a Grothendieck universe.
Therefore the principle of whimsical identity implies that there is an inaccessible cardinal, by the following lemma.
\begin{lemma}
If $U$ is a Grothendieck universe, then $\sup_{x \in U} \card x$ is an inaccessible cardinal. Conversely, if $\kappa$ is inaccessible, then $V_\kappa$ is a Grothendieck universe.
\end{lemma}
\begin{proof}
If $U$ is a Grothendieck universe, let
$$\kappa = \sup_{x \in U} \card x.$$
Since $U$ is uncountable, there is an $x \in U \setminus V_\omega$, so $x$ must not be hereditarily finite, and since $U$ is transitive, this implies that $\kappa \geq \omega$.
Since $U$ is closed under power set, this implies that $\kappa > \omega$, and that if $\gamma < \kappa$, $2^\gamma < 2^\kappa$.

We claim that $\kappa$ is regular and hence inaccessible. Let $\Gamma$ be a set of cardinals $< \kappa$ such that $\card \Gamma < \kappa$. By definition of $\kappa$, we can replace any $\gamma \in \Gamma$ with a $\gamma'$ which is the cardinality of a set in $U$,
to get a set $\Gamma'$ of cardinals $< \kappa$ such that $\card \Gamma' < \kappa$, every $\gamma' \in \Gamma'$ is the cardinality of a set in $U$, and $\sup \Gamma \leq \sup \Gamma'$. We must show $\sup \Gamma' < \kappa$.

In fact, if $f: \Gamma' \to U$ sends each cardinal in $\Gamma'$ to its respective set, then
$$\sup \Gamma' = \card \bigcup_{\gamma' \in \Gamma'} f(\gamma') \in U.$$
Therefore $\sup \Gamma' < \kappa$.

For the converse, let $\kappa$ be inaccessible; then $V_\kappa$ is obviously transitive and uncountable, and closed under pairing and power set. It is closed under small union since $\kappa$ is regular.
\end{proof}
\begin{corollary}
The following are equivalent:
\begin{enumerate}
\item Every set is contained in a Grothendieck universe.
\item There is a proper class of inaccessible cardinals.
\end{enumerate}
\end{corollary}
\begin{proof}
Immediate from the above lemma.
\end{proof}
\begin{axiom}[universe]
\index{axiom of universes}
Every set is contained in a Grothendieck universe.
\end{axiom}
Note that this axiom is implied by the principle of whimsical identity.

\section{Clubs}
Here we record facts about clubs that we will need in the sequel.

Recall that ordinals carry the order topology, so a limit (in the topological sense) is the same as a limit (in the set-theoretic sense), and so a subset of an ordinal is said to be closed if it is closed under taking limits.
\begin{definition}
A \dfn{club} in an ordinal $\alpha$ is a set $\subseteq \alpha$ which is closed and unbounded, i.e. closed and cofinal.
\end{definition}
\begin{lemma}
Let $\kappa$ be an uncountable regular cardinal. Let $c_0,c_1$ be clubs in $\kappa$. Then $c_0 \cap c_1$ is a club.
\end{lemma}
\begin{proof}
Let $\alpha < \kappa$; to show that $c_0 \cap c_1$ is cofinal, we will find an element of $c_0 \cap c_1$ which is larger than $\alpha$.
Let $\beta_n$ be defined inductively to be larger than $\alpha, \beta_0, \dots, \beta_{n-1}$, and the least element of $c_j$ with these properties where $n = j$ mod $2$.
Then $\beta = \lim_{n \to \omega} \beta_n$ is a limit of $c_0 \cap c_1$, which is closed and hence $\ni \beta$, and $\alpha < \beta < \kappa$.
\end{proof}
\begin{definition}
A \dfn{stationary set} in an ordinal $\alpha$ is a set $S \subseteq \alpha$ such that for every club $C$ in $\alpha$, $S \cap C$ is nonempty.
\end{definition}
\begin{lemma}
The intersection of a stationary set and a club in an uncountable regular cardinal is stationary.
\end{lemma}
\begin{proof}
Let $S$ be stationary and $C$ a club; then if $D$ is a club, $S \cap C \cap D = S \cap (C \cap D)$, and $C \cap D$ is a club.
\end{proof}
So one can think of a club set as a very large subset of $\alpha$, and its elements are ``generic" elements of $\alpha$. Stationary sets, then, are those that are ``open" in $\alpha$.
\begin{lemma}
Let $S = S_0 \cup S_1$ be a partition of a stationary set $S$. Then one of the $S_0,S_1$ is stationary.
\end{lemma}
\begin{proof}
Suppose neither $S_0$ nor $S_1$ is stationary, so there are clubs $c_0,c_1$ they do not meet.
If $\beta \in S_0$ then $\beta \notin c_0 \cap c_1$; similarly for $S_1$. So $S$ is not stationary.
\end{proof}

The above notions are mainly of interest for uncountable regular cardinals.
\begin{example}
Suppose that $\cof \kappa = \aleph_0$. Then the above notions are trivial.

To see this, let $f: \aleph_0 \to \kappa$ be cofinal.
(If $\kappa = \aleph_0$ we can take $f = \id$, and if $\kappa = \aleph_\omega$ we can take $f(n) = \aleph_n$.)
Then the image $C$ of $f$ is a discrete subset of $\kappa$, so $C$ is a club. In particular, by translating $C$ by $1$ we obtain two disjoint clubs in $\kappa$, and in fact we can partition a subset $A$ of $\kappa$ such that $A^c$ is bounded into $\aleph_0$ many disjoint clubs.

Let $S \subseteq \kappa$; then $S$ is stationary iff $\card S^c < \kappa$, which happens iff $S^c$ is bounded.
Clearly if $S^c$ is bounded, then $S$ meets every club.
For the converse, $S$ must meet every club in the set $A$, but this is only possible if $S^c$ is bounded in $A$.
\end{example}
Of course, we will mainly be interested in clubs in large cardinals, which are by definition uncountable and regular.

\begin{definition}
A \dfn{Mahlo cardinal} is an inaccessible cardinal $\kappa$ such that
$$\{\lambda < \kappa: \lambda \text{ is an inaccessible cardinal}\}$$
is a stationary set in $\kappa$.
\end{definition}
So Mahlo cardinals dwarf inaccessibles in the same sense that inaccessibles dwarf worldlies and worldlies dwarf ZFC:
if there is just one Mahlo cardinal $\kappa$ in $V$, then $V_\kappa$ is a model of ZFC plus the universe axiom, so $V$ thinks that the universe axiom is consistent, as is the consistency of the consistency of the consistency ... of the consistency of the universe axiom.

If we believe whimsical identity, then we may conclude Grothendieck's universe axiom; in particular, that $\Ord$ is Mahlo and hence a ``dense set" of ordinals are inaccessible cardinals!
In particular, by reflection, one should believe that there is a Mahlo cardinal, and then conclude by whimsical identity again that there are lots of Mahlo cardinals.
Actually, assuming that there is just one measurable cardinal, then there will be lots of Mahlo cardinals, as we will prove later.

\begin{definition}
Let $\nu$ be a regular uncountable cardinal. The \dfn{club filter} of $\nu$, $C_\nu$, is the filter on $\nu$ generated by all clubs in $\nu$.
\end{definition}
\begin{lemma}
For any regular uncountable cardinal $\nu$, $C_\nu$ is a $\nu$-complete filter.
\end{lemma}
\begin{proof}
Let $\{c_\alpha\}_{\alpha < \lambda}$ be a $\lambda$-sequence of clubs where $\lambda < \nu$. We must show that $\bigcap_\alpha c_\alpha$ is a club.
We prove this by induction on $\lambda$. It is clear if $\lambda$ is a successor, since then we can apply the fact that a finite intersection of clubs is a club, so we assume that $\lambda$ is a limit ordinal.

We replace $c_\alpha$ with $\bigcap_{\gamma \leq \alpha} c_\gamma$ and hence assume that we have a decreasing chain of clubs. We are allowed to do this by induction.
Then we let $\beta_\alpha$ be the least element of $c_\alpha$ which is greater than $\beta_\gamma$ if $\gamma < \alpha$, and let $\beta = \lim_{\alpha \to \lambda} \beta_\alpha$, so $\beta$ is a limit of all the $c_\alpha$ which can be made arbitrarily large by choosing $\beta_0$ arbitrarily large.
\end{proof}
The idea of the above proof is just the usual trick of passing to subsequences, but done $\lambda$ many times.

\begin{lemma}
The club filter $C_\kappa$ of an uncountable regular cardinal $\kappa$ is a normal filter.
\end{lemma}
\begin{proof}
Let $(c_\alpha: \alpha < \kappa)$ be a $\kappa$-sequence of clubs in $\kappa$.
We must show their diagonal intersection $c = \Delta_\alpha c_\alpha$ is also a club.
We replace them with the sequence $\bigcap_{\beta < \alpha} c_\beta$, which is a $\kappa$-sequence of clubs such that
$$\Delta_\alpha \bigcap_{\beta < \alpha} c_\beta = c,$$
but has the advantage that it is a decreasing sequence.
A diagonal intersection is an intersection, so $c$ is closed.

It remains to show that $c$ is cofinal. For any $\zeta < \kappa$ we can find a $\kappa$-sequence $(\gamma_\alpha: \alpha < \kappa)$ where $\gamma_\alpha \in c_\alpha$ and $\gamma_\alpha > \zeta$.
Since the $c_\alpha$ form a decreasing sequence of closed sets, the limit of the $\gamma_\alpha$ lies in the diagonal intersection.
\end{proof}

\begin{lemma}[Fodor]
\index{Fodor's lemma}
Let $\kappa$ be an uncountable regular cardinal, $S \subseteq \kappa$ stationary, and $f: S \to \kappa$ such that $f(\alpha) < \alpha$ if $\alpha > 0$.
Then there is a $\gamma < \kappa$ and stationary $S_0 \subseteq \kappa$ such that $f(\alpha) = \gamma$ for any $\alpha \in S_0$.
\end{lemma}
\begin{proof}
We can remove $0$ from $S$ without affecting the statement of the theorem, so assume $0 \notin S$, and assume the lemma fails.
Then for every $\alpha < \kappa$ there is a club $c_\alpha$ such that $c_\alpha \cap f^{-1}(\alpha)$ is empty.
Let $c = \Delta_\alpha c_\alpha$ be the diagonal intersection of all such clubs.
Since the club filter $C_\kappa$ is a normal filter, $c$ is a club and so there is an $\alpha \in S \cap c$.
Therefore $\alpha \in c_\beta$ for every $\beta < \alpha$, and for such $\beta$, $\alpha \notin f^{-1}(\beta)$.
Therefore $f(\alpha) \geq \alpha$, a contradiction.
\end{proof}

\begin{theorem}[Solovay]
\index{Solovay's stationary theorem}
\label{solovay stationary theorem}
Let $\kappa$ be an uncountable regular cardinal, $S \subseteq \kappa$ stationary. Then there is a partition of $S$ into $\kappa$ many stationary sets.
\end{theorem}
\begin{proof}
Let $T$ be the set of all limit ordinals $\alpha \in S$ such that $\cof \alpha = \omega$ or $S \cap \alpha$ is not stationary in $\alpha$.
Then $T$ is stationary. In fact, if $c \subseteq \kappa$ is a club, then its accumulation set $c'$ is also a club, so $c' \cap S$ is nonempty.
Let $\alpha = \min c' \cap S$; then either $\alpha \in T$ or $\cof \alpha$ is uncountable.
Suppose that $\cof \alpha$ is uncountable; we will show that $\alpha \in T$ regardless. Since $\alpha \in c'$, $\alpha$ is a limit of $c$,
so $c \cap \alpha$ is cofinal and hence a club in $\alpha$. Therefore $(c \cap \alpha)' = c' \cap \alpha$ is also a club in $\alpha$.
If $\alpha \notin T$, then $S \cap \alpha$ is stationary in $\alpha$, so it meets $c' \cap \alpha$, so $c' \cap S$ has an element under $\alpha$, a contradiction.
Therefore $\alpha \in T$, so $T$ is stationary.

Let $\alpha \in T$. We claim there is a cardinal $\lambda$ and a function $f_\alpha: \lambda \to \alpha$ which is cofinal, such that $f_\alpha"\lambda \cap T$ is empty.
If $\cof \alpha = \omega$, then let $f_\alpha: \omega \to \alpha$ be cofinal, and if $f_\alpha(n)$ is ever a limit ordinal, replace it with $f_\alpha(n) + 1$. Since $T$ only consists of limit ordinals, $f_\alpha"\omega \cap T$ is empty.
If instead $S \cap \alpha$ is not stationary in $\alpha$, then neither is $T \cap \alpha$, so there is a club $c \subseteq \alpha$ such that $c \cap T$ is empty. Then let $f_\alpha: \card \alpha \to \alpha$ be cofinal and map onto $c$.
This verifies the claim.

Now let
$$T_\eta^\xi = \{\alpha \in T: \xi \in \dom f_\alpha, ~f_\alpha(\xi) \geq \eta\}.$$
\begin{lemma}
There is a $\xi < \kappa$ such that for every $\eta < \kappa$, $T_\eta^\xi$ is stationary in $\kappa$.
\end{lemma}
\begin{proof}
Suppose not; then for every $\xi$ there is an $\eta(\xi)$ and a club $c_\xi$ such that $T_{\eta(\xi)}^\xi \cap c_\xi$ is empty.
Let $c = \Delta_\xi c_\xi$. Then $c$ is a club.

Let $E = \{\alpha < \kappa: \eta"\alpha \subseteq \alpha\}$.
Then $E$ is a club.
Indeed, $\alpha$ is a limit of $E$, then for any $\beta$ approximating $\alpha$, $\eta"\beta \subseteq \beta$; taking the union over all such $\beta$, we see that $\eta"\alpha \subseteq \alpha$, so $E$ is closed.
To see that $E$ is cofinal, suppose not; so there is an $\alpha$ such that for every $\beta > \alpha$, $\eta"\beta \not \subseteq \beta$.
Let $\gamma_0 = \alpha + 1$, and for every $n$, choose $\gamma_{n+1} > \gamma_n$ so that $\eta"\gamma_n \subseteq \gamma_{n+1}$.
Then let $\gamma = \lim_n \gamma_n$. Since $\cof \kappa > \omega$ we have $\alpha < \gamma < \kappa$, but then $\gamma \in E$, a contradiction.
Therefore $E$ is a club, and therefore so is $D = c \cap E$.

Since $D$ is a club and $T$ is stationary, there are $\alpha < \gamma$ such that $\alpha, \gamma \in D \cap T$.
In particular, if $\xi < \alpha$, $\alpha \in c_\xi$. Thus $\alpha \notin T^\xi_{\eta(\xi)}$.
Now let $\xi \in \gamma \cap \dom f_\alpha$; since $\alpha \notin T^\xi_{\eta(\xi)}$, $f_\alpha(\xi) < \eta(\xi)$.
Since $\gamma \in E$, and $\xi < \gamma$, $\eta(\xi) < \gamma$, so $\sup f_\alpha"\gamma \leq \gamma < \alpha$.
Since $f_\alpha$ is cofinal in $\alpha$, $\gamma \in \dom f_\alpha$.

Now $f_\alpha$ is continuous and $\gamma \in T$ is a limit ordinal, $f_\alpha(\gamma) = \sup f_\alpha"\gamma \leq \gamma$.
But as $f_\alpha$ is increasing, $f_\alpha(\gamma) \geq \gamma$. So $f_\alpha(\gamma) = \gamma$.
Therefore $f_\alpha"(\dom \alpha) \cap T$ is nonempty, a contradiction.
\end{proof}
Let $f: T \to \kappa$ satisfy $f(\alpha) = f_\alpha(\xi)$ if $\xi \in \dom f_\alpha$, and otherwise $f(\alpha) = 0$.
By definition, $f(\alpha) < \alpha$, and for every $\eta < \kappa$,
$$T_\eta^\xi = \{\alpha \in T: f(\alpha) \geq \eta\},$$
but $T_\eta^\xi$ is stationary, hence nonempty, so $f$ is cofinal.
Therefore, by Fodor's lemma, for every $\eta < \kappa$ there is a $\gamma \geq \eta$ such that
$$\{\alpha \in T_\eta^\xi: f(\alpha) = \gamma\}$$
is stationary.

We claim that there is a strictly increasing sequence $(\gamma_\eta: \eta < \kappa)$ such that
$$S_\eta = \{\alpha \in T_{\gamma_\eta + 1}^\xi: f(\alpha) = \gamma_{\eta + 1}\}$$
is stationary. This will complete the proof of Solovay's theorem, as the $S_\eta$ are disjoint,
and if $x \in S \setminus \bigcup_\eta S_\eta$ then we can simply add $x$ to $S_0$.
To see the claim, let $\gamma_\lambda$ be the least $\gamma$ that has not yet been used for $\lambda$ a limit ordinal or $0$,
and for any $\gamma_\eta$, let $\gamma_{\eta + 1}$ be the least possible choice of $\gamma$
such that $\{\alpha \in T_{\gamma_\eta + 1}^\xi: f(\alpha) = \gamma\}$ is stationary.
\end{proof}

\section{Ramsey theory}
Most large cardinal axioms can be viewed as asserting the truth of a certain combinatorial sentence, which also happens to imply a deeper statement about the nature of the universe as a whole.
This is well-illustrated by the notion of a weakly compact cardinal, which is one that ostensibly generalizes Ramsey theory, but actually guarantees the existence of inaccessibles.

Recall that if $X$ is a set and $\delta$ is a cardinal, then we may view a partition (i.e. coloring) of $X$ into $\delta$ many sets (resp. many colors) as a function $X \to \delta$.
Recall also that if $\lambda$ is a cardinal,
$$[X]^\lambda = \{x \subseteq X: \card x = \lambda\}.$$
This is mainly of interest when $\lambda < \omega$.
\begin{definition}
Let $f: [X]^n \to \delta$ be a partition. We say that $H \subseteq X$ is a \dfn{homogeneous set} for $f$ if $f|[H]^n$ is constant.
\end{definition}
\begin{definition}
Let $\kappa,\lambda,\delta$ be cardinals, $n < \omega$. We write $\kappa \to (\lambda)_\delta^n$ if for every partition $[\kappa]^n \to \delta$ there is a homogeneous set $H \subseteq \kappa$ such that $\card H = \lambda$.
\end{definition}
\begin{example}
By the Ramsey friendship theorem, any partition $[6]^2 \to 2$ has a homogeneous subset of size $3$. In other words, $6 \to (3)_2^2$.
More generally, we can interpret any partition law of the form $\kappa \to (\lambda)_2^2$ can be interpreted as
``For any party with $\kappa$ individuals, either $\lambda$ many are friends or $\lambda$ many are enemies."
\end{example}
\begin{theorem}[Ramsey theorem]
For every $n, m < \omega$, $\omega \to (\omega)^n_m$.
\end{theorem}
\begin{proof}
We prove this by induction on $n$. If $n = 1$, then we are considering a partition $\omega \to m$, and the result is just the infinite pigeonhole principle.

Now consider a partition $[\omega]^{n+1} \to m$. Define $f_a: [\omega \setminus \{a\}]^n \to m$, $f_a(x) = f(x \cup \{a\})$.
By induction, every $f_a$ has an infinite homogeneous subset. Let $H_0 = \omega$ and let $a_0$ be arbitrary.
Define $H_{i+1}$ to be an infinite $f_{a_i}$-homogeneous subset of $H_i$ and $a_{i+1} \in H_{i+1}$ such that $a_{i+1} \neq a_j$ for any $j \leq i$.
This exists because $H_{i+1}$ is infinite.

Then $(a_i: i \geq k)$ is homogeneous for $f_{a_k}$. Let $g(i) = f_{a_i}(a_i)$. Then $g \in m^\omega$, so has an infinite homogeneous set by the infinite pigeonhole principle.
\end{proof}
\begin{corollary}[finite Ramsey theorem]
For every $n, m, T < \omega$, there is a $k < \omega$ such that $k \to (T)^n_m$.
\end{corollary}
\begin{proof}
If not, then there are $n,m, T$ such that for every $k$, $k \not \to (T)^n_m$.
Consider the language $(c_0, c_1, \dots, c_{m-1}, d_i : i < \omega, f)$ where $f$ is an $n$-ary function symbol.
Consider the theory $T$ generated by the axioms
\begin{enumerate}
\item $\forall x(f(x, \dots, x) = c_0)$.
\item $\forall x_1 \cdots \forall x_n((f(x_1, \dots, x_n) = c_0) \vee (f(x_1, \dots, x_n) = c_1) \vee \cdots \vee (f(x_1, \dots, x_n) = c_m))$.
\item The schema that for all $i\neq j$, $d_i \neq d_j$.
\item The schema that for every permutation $\sigma: n \to n$,
$$\forall x_1 \cdots \forall x_n(f(x_1, \dots, x_n) = f(x_{\sigma(1)}, \dots, x_{\sigma(n)})).$$
\item The schema that for every set $X \subset \omega$ such that $T \leq \card X < \aleph_0$,
$$\bigvee_{\substack{i \neq j\\i' \neq j'\\i,j,i',j' \in X}} f(d_i, d_j) \neq f(d_{i'}, d_{j'}).$$
\end{enumerate}
Consider the finite subset $T_k \subset T$ which restricts the schema to range over $X$ such that $T \leq \card X \leq k$.
We have a model $M_k \models T_k$, where the universe is $\omega \cup \{c_0, \dots, c_{m-1}\}$, $f$ is a coloring function, and the interpretation of the $d_i$ is that $i \mapsto d_i$.
So any finite subset of $T$ is consistent, so by compactness, there is a model $M \models T$.
This contradicts the fact that $\omega \to (\omega)^n_m$.
\end{proof}
\begin{definition}
A \dfn{weakly compact cardinal} is an uncountable cardinal $\kappa$ such that $\kappa \to (\kappa)^2_2$.
\end{definition}
Without the uncountability hypothesis we would have $\omega$ weakly compact. So by whimsical identity, weakly compact cardinals exist.
However, we will show they are quite big soon enough.

The reason why we restrict $n < \omega$, rather than allow $n$ to be any infinite cardinal, is the following lemma.
\begin{theorem}[Erdős-Rado]
\index{Erdős-Rado theorem}
For any infinite cardinals $\kappa,\lambda$, $\kappa \not\to (\lambda)^\lambda_2$.
\end{theorem}
\begin{proof}
Let $<$ be a well-ordering of $[\kappa]^\lambda$. When we talk about minimality it will be with reference to this well-ordering.
Consider the partition $f: [\kappa]^\lambda \to 2$ given by $f(A) = 1$ if $A$ is minimal among all infinite subsets of $A$, or $0$ otherwise.
Suppose there is a homogeneous $H \subseteq \lambda$ of size $\lambda$, so $H \in [\kappa]^\lambda$.

If $H$ isn't minimal among subsets of $H$, let $A \subset H$ be minimal among subsets of $H$. Then $f(A) = 1$, and since $H$ is homogeneous it follows that $f(H) = 1$, a contradiction.
Therefore $H$ is minimal among subsets of $H$. Let $A_0$ be an infinite, coinfinite subset of $H$.

For every $i$, let $A_{i+1}$ be a coinfinite subset of $H$ which contains $A_i$ and infinitely many elements of $H \setminus A_i$.
Thus we have a chain $A_i \subset A_{i+1}$, and since $f(A_i) = 1$ by homogeneity of $H$, $A_i$ is minimal among its subsets, so $A_i > A_{i+1}$.
Therefore $([\kappa]^\lambda, <)$ is not well-founded, a contradiction.
\end{proof}
So it really is important that $n$ be finite! Otherwise we can talk about sets whose cardinality is the same as their own subsets.
Note that the above theorem used the axiom of choice in a very strong way.
\begin{definition}
A \dfn{partition cardinal} is a cardinal $\kappa$ such that $\kappa \to (\kappa)^\kappa_2$.
\end{definition}
So partition cardinals are a form of large cardinal that contradicts the axiom of choice.

We now show that power sets are not weakly compact. This is basically just the proof of the Bolzano-Weierstrass theorem again.
\begin{theorem}
For any infinite cardinal $\kappa$, $2^\kappa \not\to (\kappa^+)_2^2$.
\end{theorem}
\begin{proof}
Let $(x_\alpha: \alpha < 2^\kappa)$ enumerate the binary sequences $\kappa \to 2$, and let $<$ denote the lexicographic ordering on $(x_\alpha: \alpha < 2^\kappa)$.
Consider the partition $[2^\kappa]^2 \to 2$ where $\alpha<\beta$ are friends iff $x_\alpha < x_\beta$.
Suppose that $H$ is a set of all friends or all strangers, and $\card H = \kappa^+$.
Then $H$ indexes a monotone sequence $(x_\alpha: \alpha \in H)$; assume it is increasing by throwing away a constant subsequence if necessary (there can be no decreasing subsequence since lexicographic orderings are well-founded).
Thus all members of $H$ are friends.

Inductively define $(\alpha_\xi, \beta_\xi: \xi < \kappa)$ by
saying that $\beta_0$ is the first ordinal $\beta$ such that $x_\beta(\alpha) = 1$ for some $\alpha$, and $\alpha_0$ is that ordinal. Then let
$$\beta_\xi = \min\{\beta: (\beta > \sup_{\xi'<\xi}\beta_{\xi'}) \wedge (\exists \alpha \in H((\alpha > 1 + \sup_{\xi'<\xi} \alpha_{\xi'}) \wedge (x_\beta(\alpha) = 1)))\},$$
and $\alpha_\xi$ is the minimal witness that appears in the definition of $\beta_\xi$.
We take suprema at limit stages, which is possible since $\kappa^+$ is a successor cardinal, therefore regular, so any map $\xi \to \kappa^+$ cannot be cofinal.

In the above construction, the $\alpha_\xi$ are all in $H$, and they form an increasing sequence in $H$.
But every successor stage fixes one more bit of $x_{\beta_\xi}$, and there are only $\kappa$ many bits.
So the construction halts at $\xi = \kappa$, yet $\card H = \kappa^+$ is regular, so the $\alpha_\xi$ cannot form a cofinal sequence in $H$.

But if $\beta \in H$ and $x_\beta$ is not minimal (in which case there is nothing to do), there is a $\alpha < \kappa$ such that $x_\beta(\alpha) = 1$.
This implies that there is a $\xi < \kappa$ such that $x_\beta < x_{\alpha_\xi}$, and since $H$ indexes an increasing sequence this means $\beta < \alpha_\xi$.
So $(\alpha_\xi: \xi < \kappa)$ is cofinal in $H$, a contradiction.
\end{proof}

\begin{corollary}
$(2^\kappa) \not \to (2^\kappa)^2_2$.
\end{corollary}

\section{Weakly compact cardinals}
Recall that $\kappa$ is said to be a weakly compact cardinal if it is uncountable and satisfies the partition law $\kappa \to (\kappa)_2^2$.
In other words, at every party with $\kappa$ guests, either there are $\kappa$ mutual friends or $\kappa$ mutual strangers.

We first show that weakly compact cardinals are large cardinals.
\begin{lemma}
Weakly compact cardinals are inaccessible.
\end{lemma}
\begin{proof}
Suppose $\kappa$ is weakly compact. Since $2^\lambda \not \to (\lambda^+)^2_2$, if $\lambda < \kappa$, $2^\lambda \not \to (\kappa)^2_2$.
This implies $2^\lambda < \kappa$.

So it suffices to show that $\kappa$ is regular. If not, let $(A_\xi: \xi < \lambda)$ be a partition of $\kappa$, where $\card A_\xi = \delta_\xi$, $\lambda, \delta_\xi < \kappa$.
For $\alpha,\beta < \kappa$, say that $\alpha,\beta$ are friends if there is a $\xi < \lambda$ such that $\alpha,\beta \in A_\xi$.
Then there is an $H$ such that $\card H = \kappa$ and all elements of $H$ are friends, or all elements of $H$ are strangers.
If they are strangers, then choose a function $f: \kappa \to \lambda$ which sends $\alpha$ to the $\xi$ such that $\alpha \in A_\xi$; then $f$ is injective, so $\lambda < \kappa \leq \lambda$, a contradiction.
So they are all friends, and there is a $\xi$ such that $H \subseteq A_\xi$. But then $\delta_\xi < \kappa \leq \delta_\xi$, a contradiction.
\end{proof}

\begin{theorem}
The following are equivalent for an infinite cardinal $\kappa$:
\begin{enumerate}
\item $\kappa$ is weakly compact or $\kappa = \omega$.
\item $\kappa \to (\kappa)^2_2$.
\item For every linear order of length $\kappa$, either there is an ascending sequence of length $\kappa$ or a descending sequence of length $\kappa$.
\end{enumerate}
\end{theorem}
\begin{proof}
The equivalence between the first two criteria is clear.

We now show $2 \Leftrightarrow 3$. Suppose $\kappa \to (\kappa)^2_2$ and $(x_\alpha: \alpha < \kappa)$ enumerates a linear order. Say that $\alpha$ and $\beta$ are friends if $x_\alpha < x_\beta$.
Then there is a set $H \subseteq \kappa $ of cardinality $\kappa$ consisting of friends or strangers. If $H$ consists of friends we're done.
Otherwise, since $\alpha \mapsto x_\alpha$ is surjective, it cannot be constant, so throw away a constant subsequence of $(x_\alpha: \alpha \in H)$ to get a decreasing subsequence.

Conversely, suppose that for every linear order of cardinality $\kappa$, either there is an ascending subsequence of length $\kappa$ or a descending subsequence of length $\kappa$.
Given a partition of $[\kappa]^2$ into friends and strangers, let $\alpha \preceq \beta$ iff $\alpha\leq\beta$ are friends. Otherwise let $\beta \preceq \alpha$.
So we obtain a linear order of length $\kappa$, so it has a monotone sequence of length $\kappa$. If it is increasing we have friends; otherwise, we have strangers.
Therefore $2 \Leftrightarrow 3$.
\end{proof}

We obtain a very concrete corollary from the above analysis.
\begin{corollary}[Bolzano-Weierstrass theorem]
\index{Bolzano-Weierstrass theorem}
$[0, 1]$ is compact.
\end{corollary}
\begin{proof}
It suffices to show that every sequence of real numbers has a monotone subsequence; then if that sequence is bounded, its monotone subsequence will be Cauchy, and we will be done. But this follows by applying the above theorem to the image of $\omega$ under the sequence.
\end{proof}

Another characterization of weakly compact cardinals is in terms of trees.
\begin{definition}
Let $\kappa$ be a cardinal.
An \dfn{Aronszajn tree} of cardinality $\kappa$ is a tree $T$ with $\kappa$ nodes, such that all branches of $T$ have length less than $\kappa$ and all levels of $T$ have width less than $\kappa$.
\end{definition}
So Koenig's lemma says that there are no countable Aronszajn trees. We now use this statement to generalize Ramsey's theorem.

\begin{theorem}
\label{tree characterization of weakly compact}
Let $\kappa$ be an inaccessible cardinal or $\kappa = \omega$.
If there are no Aronszajn trees of cardinality $\kappa$, then $\kappa$ satisfies, for every $m < \omega$ and $\lambda < \kappa$, the partition law $\kappa \to (\kappa)^m_\lambda$.
\end{theorem}
Note that this is just the infinite Ramsey theorem in disguise. We will not need the fact that $\kappa$ is uncountable, so this gives another proof of the infinite Ramsey theorem for $\kappa = \omega$.

To prove the above theorem, we first construct a tree $(\kappa, \prec)$ for any partition $f: [\kappa]^{n+1} \to \eta$ where $\eta$ is infinite.
Put $\beta \prec \alpha$ if $\beta < \alpha$, and for every sequence $\gamma_0 \prec \cdots \prec \gamma_{n-1} \prec \beta$, we have $f(\gamma_0, \dots, \gamma_{n-1}, \beta) = f(\gamma_0, \dots, \gamma_{n-1}, \alpha)$.
\begin{lemma}
For any partition, $(\kappa, \prec)$ is a tree.
\end{lemma}
\begin{proof}
The relation $\preceq$ is antisymmetric and well-founded since $\leq$ is.

We now check that it is transitive by induction on $\alpha$.
Suppose that $\prec$ is transitive in the $\prec$-predecessors of $\alpha$, and $\delta \prec \gamma \prec \alpha$.
Suppose $\gamma_0 \prec \cdots \prec \gamma_{n-1} \prec \delta$. Then by induction, $\gamma_{n-1} \prec \gamma$. So
$$f(\gamma_0, \dots, \gamma_{n-1}, \delta) = f(\gamma_0, \dots, \gamma_{n-1}, \gamma) = f(\gamma_0, \dots, \gamma_{n-1}, \alpha).$$
Therefore $\delta \prec \alpha$.

Finally we must check that if $\delta \prec \beta$ and $\gamma \prec \beta$ then $\delta,\gamma$ are $\prec$-comparable.
Suppose $\delta < \gamma$ and if $\zeta \prec \beta$ and $\zeta < \delta$, then $\zeta$ is comparable any predecssor of $\beta$.
Suppose $\gamma_0 \prec \cdots \prec \gamma_{n-1} \prec \delta$. Then by induction, $\gamma_{n-1} \prec \gamma$. So
$$f(\gamma_0, \dots, \gamma_{n-1}, \delta) = f(\gamma_0, \dots, \gamma_{n-1}, \beta) = f(\gamma_0, \dots, \gamma_{n-1}, \gamma).$$
Therefore $\delta$ and $\gamma$ are comparable.
\end{proof}
\begin{lemma}
Fix a partition of $[\kappa]^{n+1}$.
Let $\lambda$ be a limit ordinal and suppose that $\alpha,\beta \in \kappa$ are both of $\prec$-level $\lambda$.
Then the sets of predecessors of $\alpha,\beta$ are not equal.
\end{lemma}
\begin{proof}
Suppose they are.
Then if $\alpha_0 \prec \cdots \prec \alpha_{n-1} \prec \alpha$, we also have $\alpha_{n-1} \prec \beta$.
Since $\lambda$ is a limit ordinal, pick an $\alpha_n \prec \alpha$ such that $\alpha_{n-1} \prec \alpha_n \prec \beta$. Then
$$f(\alpha_0, \dots, \alpha_{n-1}, \alpha_n) = f(\alpha_0, \dots, \alpha_{n-1}, \alpha_n) = f(\alpha_0, \dots, \alpha_{n-1}, \beta).$$
Therefore $\alpha,\beta$ are comparable and yet at the same level and not equal.
\end{proof}
\begin{lemma}
Fix a partition $[\kappa]^{n+1} \to \eta$.
For every $\delta < \kappa$, the $\delta$-level of $(\kappa, \prec)$ has cardinality $\leq \eta^{\card \delta}$.
\end{lemma}
\begin{proof}
By induction on $\delta$. If $\delta < \omega$ and level $\delta$ has cardinality $\leq \eta$, then level $\delta + 1$ has cardinality $\leq \eta^2 = \eta$.
Take suprema at limit stages.
If $\delta$ is infinite and its level has cardinality $\leq \eta^{\card \delta}$, then $\card [\delta]^{n-1} = \card \delta$, then level $\delta + 1$ has cardinality $\leq (\eta^{\card \delta})^2 = \eta^{\card \delta}$ as well.
\end{proof}

\begin{proof}[Proof of Theorem \ref{tree characterization of weakly compact}]
By induction on $m$. The case $m = 1$ says that one cannot partition $\kappa$ into $\lambda$ subsets of size $< \kappa$, which is true since $\kappa$ is regular.
Now fix a partition $f: [\kappa]^{m-1} \to \lambda$ and consider the induced tree $(\kappa, \prec)$.
For every level $\delta < \kappa$, its cardinality is at most $\lambda^{\card \delta} < \kappa$, since $\kappa$ is inaccessible or $\kappa = \omega$.
Since $(\kappa, \prec)$ is not an Aronszajn tree, it has a branch $X$ of length $\kappa$.
Since $\kappa \to (\kappa)_\lambda^{m-1}$ and $\card X = \kappa$, we consider $f|[X]^{m-1}$ and find a homogeneous subset of $X$ of cardinality $\kappa$.
\end{proof}

\begin{lemma}
Let $(T, \prec)$ be a tree. Then there is an extension of $\prec$ to an ordering $<$ such that $(T, <)$ is a chain, and for every $x \in T$, the successors of $x$ form a segment of $(T, <)$.
\end{lemma}
\begin{proof}
Suppose that $(T_0, <)$ is such that $T_0 \subset T$ and $(T_0, <)$ extends $(T_0, \prec)$ in the way prescribed by the lemma, and $T_0$ is closed under $\prec$-predecessors as a subset of $T$.
Choose a $\prec$-minimal element $x \in T \setminus T_0$ and consider a $\subset$-maximal linear subset $X$ of $T$ whose minimal element is $x$, which exists by Zorn's lemma.
Define $<$ on $X$ by $z < y$ iff $z \prec y$.
Then $(T_0 \cup X, <)$ extends $(T_0 \cup X, \prec)$, in the way prescribed by the lemma, and $T_0 \cup X$ is closed under $\prec$-predecessors as a subset of $T$.
Now run Zorn's lemma on the set of all $(T_0, <)$ with this property.
\end{proof}

\begin{theorem}
The following are equivalent for an infinite cardinal $\kappa$:
\begin{enumerate}
\item $\kappa$ is a weakly compact cardinal, or $\kappa = \omega$.
\item $\kappa$ is a regular cardinal, there are no Aronszajn trees of cardinality $\kappa$, and for every $\lambda < \kappa$, $2^\lambda < \kappa$.
\item For every $\lambda < \kappa$ and $m < \omega$, $\kappa \to (\kappa)_\lambda^m$.
\end{enumerate}
\end{theorem}
\begin{proof}
Clearly $3 \implies 1$ and we just showed $2 \implies 3$. Now to show $1 \implies 2$.

Let $(T, \prec)$ be a tree with $\kappa$ nodes such that every level has cardinality $< \kappa$.
We must prove that a Koenig lemma holds for $T$: $T$ has a branch of length $\kappa$.
We can extend $\prec$ to an ordering $<$ such that $(T, <)$ is a chain, and for every $x \in T$, the successors of $x$ form a segment of $(T, <)$.
Since $\kappa$ is weakly compact or $\kappa = \omega$, there is a $\kappa$-sequence $C$ in $T$. We may assume it is increasing and well-ordered.

Let $T^x = \{z \in T: x \prec z\}$ and let $C^y = \{z \in C: y < z\}$.
Let $D = \{x \in T: \exists y\in C(C^y \subseteq T^z)\}$.
Suppose that $x,x' \in D$. Then there are $y,y' \in C$ such that $C^y \subseteq T^x$ and $C^{y'} \subseteq T^{x'}$.
Assume $y \preceq y'$; then $C^y \subseteq T^x \cap T^{x'}$.
Since $C^y$ is nonempty, because $C^y$ is a tail in the $\kappa$-sequence $C$, that means that $x,x'$ have a common successor in $T$, and hence $x,x'$ are comparable.
Therefore $D$ is a chain.

We now claim that $\card D \geq \kappa$; then if we extend $D$ to a branch, which is always possible by Zorn's lemma, $D$ will be a witness that $T$ is not Aronszajn.
In fact, it suffices to show that for every level $\alpha$, $D$ has an element of level $\alpha$.
Since each level has cardinality $<\kappa$, and $C$, hence $C^y$ by regularity of $\kappa$, has cardinality $\kappa$, $C^y$ has elements of arbitrarily high level.

To see that $D$ has an element of level $\alpha$, let $E$ be the set of all segments $T^x \cap C$ such that $x$ is of level $\alpha$.
Every element of $C$ of sufficiently high level is ine one such segment, so $\bigcup E$ is cofinal in $C$ and hence of cardinality $\kappa$.
But $\card E < \kappa$, since there are $<\kappa$ nodes at level $\alpha$, so since $\kappa$ is regular there must be an $x$ at level $\alpha$ such that $\card(T^x \cap C) = \kappa$.
But then $T^x \cap C$ is cofinal in $C$, so $x \in D$, as required.
\end{proof}

\section{The weak compactness theorem}
Another useful characterization of weakly compact cardinals, which justifies their name, is in terms of the compactness theorem.
\begin{definition}
Let $\kappa,\lambda$ be cardinals. The language $\mathcal L_{\kappa,\lambda}$ consists of
\begin{enumerate}
\item $\kappa$ variables,
\item $\kappa$ relation, function, and constant symbols,
\item Finitary logical connectives and quantifiers,
\item Infinitary connectives $\bigvee_{\xi < \alpha} \varphi_\xi$ and $\bigwedge_{\xi < \alpha} \varphi_\xi$ for every $\alpha < \kappa$, and0
\item Infinitary quantifiers $\exists_{\xi < \alpha} v_\xi$ and $\forall_{\xi < \alpha} v_\xi$ for every $\alpha < \lambda$.
\end{enumerate}
\end{definition}
Here we interpret $\bigvee_{\xi < \alpha} \varphi_\xi = \varphi_0 \vee \varphi_1 \vee \cdots \vee \varphi_\omega \vee \varphi_{\omega + 1}\vee\cdots$
and $\exists_{\xi < \alpha} v_\xi = \exists v_0 \exists v_1 \cdots \exists v_\omega \vee v_{\omega+1} \vee\cdots$.

\begin{definition}
We say that $\mathcal L_{\kappa,\lambda}$ \dfn{satisfies the weak compactness theorem} if, whenever $\Sigma$ is a set of sentences of $\mathcal L_{\kappa,\lambda}$
such that $\card \Sigma \leq \kappa$ and every $\Sigma_0 \subset \Sigma$ such that $\card \Sigma_0 < \kappa$ has a model, then $\Sigma$ has a model.
\end{definition}

\begin{lemma}
If $\kappa$ is a weakly compact cardinal or $\kappa = \omega$ then $\mathcal L_{\kappa,\kappa}$ satisfies the weak compactness theorem.
\end{lemma}
This is basically trivial by what we have done so far.
In fact, the usual proof of the compactness theorem goes through; we use the fact that $\kappa$ has no Aronszajn trees to compensate for the lack of Koenig's lemma.

\begin{lemma}
If $\kappa$ is an inaccessible cardinal and $\mathcal L_{\kappa,\omega}$ satisfies the weak compactness theorem, then $\kappa$ is weakly compact.
\end{lemma}
\begin{proof}
Let $(T, <)$ be a tree of cardinality $\kappa$ and suppose every level of $T$ is of width $< \kappa$. We claim that $T$ satisfies a Koenig lemma.
Consider the $\mathcal L_{\kappa,\omega}$ language with a unary predicate $B$ and constant symbols $c_x$ for every $x \in T$.
Let $\Sigma$ be the set of sentences consisting of $\neg B(c_x) \vee \neg B(c_y)$ if $x,y$ are incomparable, and $\bigvee_{x \in U_\alpha} B(c_x)$ where $U_\alpha$ is the $\alpha$th level of $T$.
So $\Sigma$ says that $\{x \in T: B(c_x)\}$ is a branch of length $\kappa$ through $T$.
If $\Sigma$ has a model $(M, <, X)$ and we interpret the constant symbols as intended, then $\Sigma$ also has a model of the form $(T, <, Y)$ where $Y$ is a branch of length $\kappa$ through $T$.
So it suffices to show that $\Sigma$ has a model.

Let $\Sigma_0 \subset \Sigma$, $\card \Sigma_0 < \kappa$. Since $T$ is inaccessible we can take a sufficiently long initial segment of $T$ and a branch through that initial segment to obtain a model of $\Sigma_0$.
Therefore by the weak compactness theorem, $\Sigma$ has a model.
\end{proof}

Thus we give our final characterization of weak compactness.
\begin{definition}
We say that $\kappa$ is a \dfn{whimsically inacccessible cardinal} if $\kappa$ is inaccessible or $\kappa = \omega$.
\end{definition}
\begin{theorem}
The following are equivalent for an infinite cardinal $\kappa$:
\begin{enumerate}
\item $\kappa$ is weakly compact or $\kappa = \omega$.
\item For every $\lambda < \kappa$ and $n < \omega$, $\kappa \to (\kappa)_\lambda^n$.
\item Every linear order of cardinality $\kappa$ has a monotone $\kappa$-sequence.
\item $\kappa$ is whimsically inaccessible, and there are no Aronszajn trees of size $\kappa$.
\item $\kappa$ is whimsically inaccessible, and $\mathcal L_{\kappa,\kappa}$ satisfies the weak compactness theorem.
\item $\kappa$ is whimsically inaccessible, and $\mathcal L_{\kappa,\omega}$ satisfies the weak compactness theorem.
\end{enumerate}
\end{theorem}






\chapter{Measurable cardinals}
Recall that given a set $x$, an ultrafilter $U$ on $x$ is the set of subsets of $x$ of $\mu$-measure $1$, where $\mu$ is some $2$-valued, finitely additive measure whose algebra of definition is $2^x$.
Filters are similar but we do not assume that $\mu$ is defined for every subset of $x$; thus, if $F$ is a filter, then $x \in F$, $F$ is closed under intersection, $F$ is upwards closed, and $\emptyset \notin F$.
\begin{definition}
Let $\kappa$ be a cardinal and $F$ a filter. We say that $F$ is a \dfn{$\kappa$-complete filter} if $F$ is closed under $\lambda$-intersection for any cardinal $\lambda < \kappa$.
If $F$ is $\aleph_1$-complete, then we say that $F$ is a \dfn{countably complete filter}.
\end{definition}
Thus every filter is $\aleph_0$-complete, and a filter is countably complete if it is closed under countable intersections.
\begin{example}
Let $\kappa$ be an infinite cardinal. Then cofinite filter on $\kappa$ is not countably complete. In fact, $\{\{n\}^c: n < \omega\}$ is a countable set whose intersection is empty, yet whose elements are cofinite subsets of $\kappa \supseteq \omega$.
\end{example}
In ``ordinary mathematics," the only ultrafilters we have to work with are the trivial ultrafilters, and the extensions of cofinite filters that we get by using Zorn's lemma (or the boolean prime ideal theorem). Therefore if $\kappa$ is an uncountable cardinal, we are going to have to look very hard to find a nontrivial $\kappa$-complete ultrafilter.
\begin{definition}
A \dfn{measurable cardinal} is an uncountable cardinal $\kappa$ such that there is a nontrivial $\kappa$-complete ultrafilter on $2^\kappa$.
\end{definition}

\section{Properties of measurable cardinals}
\begin{lemma}
If $\kappa$ is a measurable cardinal then $\kappa$ is regular.
\end{lemma}
\begin{proof}
Let $\Lambda$ be a set of cardinals $ < \kappa$ such that $\card \Lambda < \kappa$, and let $U$ be a witness to the measurability of $\kappa$.
Since $U$ is nontrivial, finite sets are $U$-small, and since $U$ is $\kappa$-complete this implies that for every cardinal $\lambda < \kappa$, $\lambda$ is $U$-small.
By $\kappa$-completeness again, $\sum_{\lambda \in \Lambda} \lambda$ is $U$-small, yet $\kappa$ is $U$-large.
\end{proof}
This gives an amusing measure-theoretic consequence.
\begin{theorem}
For every measurable cardinal $\kappa$ there is a surjective probability measure on $2^\kappa$.
\end{theorem}
\begin{proof}
Let $E$ be the set of all limit ordinals $<\kappa$, and $0$. Then $F = \{\{\lambda + n: n < \omega\}: \lambda \in E\}$ is a partition of $\kappa$ into countable sets, and since $\kappa$ is a regular cardinal, it follows that $\card F = \kappa$.
So let $\alpha_\gamma$ be the $\gamma$th ordinal in $E$; then $\gamma$ ranges over all ordinals $< \kappa$. Let $A_n = \{\alpha_\lambda + n: \lambda < \kappa\}$. Then $\card A_n = \kappa$, so $\{A_n: n < \omega\}$ is a partition of $\kappa$ into countably many sets of cardinality $\kappa$.

Let $U$ be a witness to the measurability of $\kappa$ and let $\mu_n$ be the two-valued measure induced by $U$ on $A_n$. Now let
$$\eta(x) = \sum_{n=1}^\infty 2^{-n}\mu_n(A_n \cap x).$$
Clearly $\eta$ is a probability measure, and since any real in $[0, 1]$ can be written as a series of dyadics (by writing it in base $2$) it is surjective.
\end{proof}

\begin{theorem}
Every measurable cardinal is inaccessible, and hence worldly.
\end{theorem}
\begin{proof}
Let $\kappa$ be a measurable cardinal; then $\kappa$ is regular. Let $\gamma < \kappa$ and assume $\kappa \leq 2^\gamma$.
Let $U$ witness that $\kappa$ is measurable; since we have an injective function $\kappa \to 2^\gamma$ we have an ultrafilter $V$ on $2^\gamma$. We will find a function $f: \gamma \to 2$ such that $\{f\} \in V$, a contradiction.

If $\beta \leq \gamma$ let $A(f, \beta) = \{g \in 2^\gamma: \forall \alpha < \beta(f(\alpha) = g(\alpha))\}$.
This partitions into
$$A(f, \beta) = \{g \in A(f, \beta): g(\beta) = 0\} \cup \{g \in A(f, \beta): g(\beta) = 1\} = A(f, \beta, 0) \cup A(f, \beta, 1)\}.$$
Moreover $A(f, 0) = 2^\gamma$, so $A(f, 0) \in U$. Thus $A(f, \beta, i) \in U$ for some $i$. Let $f(0) = i$.

Assume $A(f, \beta) \in U$ for all $\beta \leq \alpha$ and $\alpha$ is a successor. Then let $f(\alpha) = i$ and continue the induction. Something similar happens at limits, but we have to use
$$A(f, \alpha) = \bigcap_{\beta < \alpha} A(f, \beta).$$

When we finish the induction, $U \ni A(f, \gamma) = \{f\}$, which is impossible.
\end{proof}

\section{Combinatorics of ultrafilters}
In this section we establish a sufficient condition for measurable cardinals to exist, which explains why we were not able to find any ultrafilters other than the obvious ones.
We then consider special types of ultrafilters on measurable cardinals.
\begin{definition}
Let $U$ be a nontrivial ultrafilter. The \dfn{completeness} of $U$, $\comp U$, is the smallest cardinal $\kappa$ such that $U$ is not $\kappa$-complete.
\end{definition}
So $U$ is not closed under intersections of size $\comp U$.
\begin{lemma}
Let $U$ be a nontrivial ultrafilter on $\kappa$. Then $\comp U \leq \kappa$.
\end{lemma}
\begin{proof}
Since $U$ is nontrivial, it contains all cofinite subsets of $\kappa$, in particular those of the form $\{\alpha\}^c$, $\alpha < \kappa$. But
$$\bigcap_{\alpha < \kappa} \{\alpha\}^c = \emptyset,$$
so $U$ is not closed under intersections of size $\kappa$.
\end{proof}
\begin{lemma}
Let $U$ be a nontrivial ultrafilter on $A$, $f: A \to B$, and
$$V = \{X \subseteq B: f^{-1}(X) \in A\}$$
be the pushforward ultrafilter on $B$. Then $V$ is trivial iff $f^{-1}(\{x\}) \in U$ for some $x \in B$; if $V$ is nontrivial then $\comp U \leq \comp V$.
\end{lemma}
\begin{proof}
The first claim is obvious. For the second, if $V$ is not closed under intersections of size $\kappa$, say $\{X_\alpha\}_{\alpha < \kappa}$ has $X_\alpha \in V$ but $\bigcap_{\alpha < \kappa} X_\alpha \notin V$, then $\bigcap_{\alpha < \kappa} f^{-1}(X_\alpha) \notin U$, so neither is $U$ closed under intersections of size $\kappa$.
\end{proof}
\begin{theorem}
There is a countably complete nontrivial ultrafilter iff there is a measurable cardinal.
\end{theorem}
\begin{proof}
If there is a measurable cardinal then by definition there is a countably complete ultrafilter. So we just have to prove the converse.

Let $U$ be a nontrivial, countably complete ultrafilter on $A$ and let $\kappa = \comp U$. Since $U$ is countably complete, $\kappa$ is uncountable. We claim that $\kappa$ is measurable.

We know that $\kappa \leq \card A$.
Let $\{X_\alpha\}_{\alpha < \kappa}$ be such that $X_\alpha \in U$ and $\bigcap_\alpha X_\alpha \notin U$. Then $B = A \setminus \bigcap_\alpha X_\alpha \in U$.
Let
\begin{align*}
f: A &\to \kappa\\
a &\mapsto \begin{cases}
\text{least $\gamma$ such that $a \notin X_\gamma$, if $a \in B$}\\
0,\text{ else.}
\end{cases}
\end{align*}
Then $f(a) = 0$ provided that $a \notin B$, and the pushforward of $U$ by $f$,
$$V = \{X \subseteq \kappa: f^{-1}(X) \in U\},$$
is an ultrafilter such that $\kappa \leq \comp V \leq \kappa$, hence $\comp V = \kappa$, provided that $V$ is nontrivial.

So it suffices to show that $f^{-1}(\beta) \notin U$ for every $\beta < \kappa$. In fact, $f^{-1}(0) = A \setminus B \notin U$. Otherwise, $f^{-1}(\beta)$ is disjoint from $X_\beta$. But $X_\beta \in U$ and no filter can contain two disjoint sets (or else it would contain $\emptyset$).
\end{proof}

\begin{definition}
A \dfn{normal ultrafilter} is an ultrafilter $U$ on an infinite cardinal $\kappa$ such that for every function $f: \kappa \to \kappa$, if
$$\{\alpha < \kappa: f(\alpha) < \alpha\} \in U,$$
then there is a $\beta < \kappa$ such that $f^{-1}(\beta) \in U$.
\end{definition}
Thus, an ultrafilter is normal provided that for every function which sends most elements downwards, it is actually constant on a large set.
The $\beta$ should be ``small" relative to $\kappa$ in the sense that the interval $(\beta, \kappa) \in U$.
\begin{definition}
Let $\kappa$ be a regular cardinal and let $\{X_\alpha\}_{\alpha < \kappa}$ be a $\kappa$-sequence of subsets of $\kappa$. The \dfn{diagonal intersection} is
$$\Delta_{\alpha < \kappa} X_\alpha = \bigcup_\alpha [0, \alpha] \cap X_\alpha.$$
\end{definition}
Thus an ordinal $\beta$ is in the diagonal intersection if it is in $X_\alpha$, for every $\alpha < \beta$ (rather than $\alpha < \kappa$!) Thus diagonal intersection is less strict than intersection; this also motivates the terminology ``diagonal intersection."
\begin{example}
Let $X_\alpha = (\alpha, \kappa)$. Then $\bigcap_\alpha = \emptyset$ but $\Delta_{\alpha < \kappa} X_\alpha = \kappa$.
\end{example}
\begin{lemma}
Let $\kappa$ be a regular cardinal and $U$ an ultrafilter on $\kappa$. Then $U$ is normal iff $U$ is closed under diagonal intersection.
\end{lemma}
\begin{proof}
Suppose that $U$ is abnormal. So there is an $X \in U$ and a $f: \kappa \to \kappa$ such that $f(\alpha) < \alpha$ for every $\alpha \in X$ and $X_\alpha = f^{-1}(\alpha) \notin U$ for all $\alpha < \kappa$. Let $Y = \Delta_{\alpha < \kappa} X_\alpha^c$.
If $\gamma \in X \cap Y$ then $\gamma \notin X_{f(\gamma)}$ since $f(\gamma) < \gamma$, yet $\gamma \in X_{f(\gamma)}$. So $X \cap Y$ is nonempty, and since $X \in U$ it follows that $Y^c \in U$, so $U$ is not closed under diagonal intersection.

Conversely, suppose that $U$ is not closed under diagonal intersection, so that there are $X_\alpha \in U$ such that $X = \Delta_\alpha X_\alpha \notin U$. Then $X^c \in U$, and for every $\alpha \notin X$ there is a $f(\alpha) < \alpha$ such that $\alpha \notin X_{f(\alpha)}$.
Then $f^{-1}(\beta) \subseteq X^c \setminus X_\beta \notin U$ for any $\beta$ so $U$ is abnormal.
\end{proof}

\begin{definition}
An ultrafilter $U$ on an infinite cardinal $\kappa$ is a \dfn{uniform ultrafilter} if for every $X \in U$, $\card X = \kappa$. It is a \dfn{weakly uniform ultrafilter} if for every $\gamma < \kappa$, $(\gamma, \kappa) \in U$.
\end{definition}

\begin{example}
Every trivial ultrafilter is normal, and neither uniform nor weakly uniform. Suppose that $U = \{X \subseteq \kappa: \alpha \in X\}$, and assume $f(\alpha) = \beta$. Then $f^{-1}(\beta) \in U$, so $U$ is normal.
\end{example}

\begin{definition}
The \dfn{generalized cofinite filter} on an infinite set $X$ is the filter of all $A \in 2^X$ such that $\card A^c < \card X$.
\end{definition}

\begin{lemma}
An ultrafilter on an infinite set $X$ is uniform iff it contains the generalized cofinite filter of $X$.
\end{lemma}
\begin{proof}
Let $F$ be the generalized cofinite filter and $U$ an ultrafilter. If $U$ does not contain $F$, let $A \in F \setminus U$; then $A^c \in U$ but $\card A^c < \card X$, so $U$ is not uniform. Conversely, if $U$ is not uniform, let $A \in U$ be such that $\card A < \card X$. Then $A^c \in F \setminus U$.
\end{proof}

\begin{lemma}
An ultrafilter on a regular cardinal is uniform iff it is weakly uniform.
\end{lemma}
\begin{proof}
Let $U$ be an ultrafilter on $\kappa$. Suppose that $U$ is uniform; if there is a tail $(\gamma, \kappa)$ which is not in $U$, then its complement $\gamma \in U$, but $\card \gamma < \kappa$, a contradiction; so $U$ is weakly uniform.
Conversely, if $U$ is weakly uniform and $X \in U$, suppose that $\card X < \kappa$. Then $X$ is bounded, so let $\gamma = \sup X$; it follows that $[0, \gamma] \in U$, contradicting that $U$ is weakly uniform.
\end{proof}

\begin{lemma}
Let $U$ be a nontrivial ultrafilter on a regular cardinal $\kappa$. Then $U$ is $\kappa$-complete iff $U$ is uniform.
\end{lemma}
\begin{proof}
Suppose that $U$ is $\kappa$-complete; to show uniformity, it suffices to show that $U$ contains the generalized cofinite filter $F$. In fact, if $A \in F$, then $\card A^c < \kappa$, so $A^c$ can be written as a union of fewer than $\kappa$ singletons. Since $U$ is nontrivial this implies that $A^c \notin U$.

Suppose that $U$ is uniform and weakly uniform and let $(X_\alpha)_{\alpha < \kappa} \in U$. If $\bigcap_\alpha U_\alpha \notin U$, then its complement $\bigcup_\alpha U_\alpha^c$ is in $U$, so let $\gamma = \sup \bigcup_\alpha U_\alpha^c$. If $\gamma < \kappa$ then $[0, \gamma] \in U$, a contradiction.
Therefore $\bigcup_\alpha U_\alpha^c$ is cofinal, implying that $\bigcap_\alpha U_\alpha$ is bounded and hence $\card \bigcup_\alpha U_\alpha < \kappa$, contradicting that $U$ is uniform.
\end{proof}

\begin{theorem}[Scott]
\index{Scott's theorem}
If $\kappa$ is a measurable cardinal, then there is a normal uniform ultrafilter on $\kappa$.
\end{theorem}
\begin{proof}
Let $U$ be a $\kappa$-complete nontrivial ultrafilter on $\kappa$.

We will construct a function $f$ such that:
\begin{enumerate}
\item For every $\beta$, $f^{-1}(\beta) \notin U$.
\item If $\{g < f\} \in U$ then there is a $\beta$ such that $g^{-1}(\beta) \in U$.
\end{enumerate}
If this is impossible, let $f_0 = \id$.
Since $U$ is nontrivial, $f_0$ satisfies condition 1 and hence fails condition 2. Let $f_{n+1}$ witness the failure of condition 2 for $f_n$, so that $f_{n+1}$ satisfies condition 1 and we can continue the induction.
Let $X_n = \{f_{n+1} < f_n\}$; by countable completeness of $U$, $X = \bigcap_n X_n \in U$. But if $\alpha \in X$, then $f_0(\alpha) > f_1(\alpha) > \cdots,$ a contradiction.

Now let $V$ be the pushforward of $U$ by $f$. By condition 1, $V$ is a nontrivial $\kappa$-complete ultrafilter. So by the lemma $V$ is uniform.

To see that $V$ is normal, let $g: \kappa \to \kappa$ be a function such that
$$S = \{\alpha < \kappa: g(\alpha) < \alpha\} \in V.$$
Now $f^{-1}(S) = \{\alpha < \kappa: g(f(\alpha)) < f(\alpha)\} \in U$. By condition 2, this implies that there is a $\beta$ such that $g \circ f^{-1}(\beta) \in U$, so $g(\beta) \in V$, as desired.
\end{proof}

\begin{example}
Any nontrivial ultrafilter on $\aleph_0$ is uniform since it is $\aleph_0$-complete by definition. Moreover, every nontrivial ultrafilter $U$ on $\aleph_0$ is abnormal. To see this, let $A \in U$ be coinfinite. For any $n$, let $f(n) = n$ if $n \notin A$, or $f(n)$ be the greatest element of $A^c$ under $n$ if $n \in A$. Since $A$ is infinite and coinfinite, $f$ is constant exactly on finite sets, but pushes down on $A$, a contradiction.

Since $\aleph_0$ is the only example of a nonmeasurable cardinal $\kappa$ with a nontrivial $\kappa$-complete ultrafilter (hence, uniform ultrafilter), all the above theory is a bit trivial if $\kappa$ is not a measurable cardinal.
\end{example}

We are interested in normal, uniform ultrafilters on measurable cardinals anyways, since we need them in order to take iterated ultrapowers of $V$ in the sequel.

\section{Ultrapowers by measurable cardinals}
We now show that the existence of a measurable cardinal is actually much stronger than an inaccessible cardinal: it actually gives elementary embeddings of $V$ into class models.

Given an ultrafilter $U$ on some set $A$ and functions $f,g: A \to B$, let $f \sim_U g$ mean that $\{f = g\} \in U$. Given a binary relation $E$ on $U$, let $f E_U g$ mean that $\{fEg\} \in U$. Then the ultrapower is defined by
$$\Pi_U(B, E) = (B^A/U, E_U).$$
By Los' theorem, the inclusion map $B \to B^A/U$ is an elementary embedding for the relation $E$.
We want to replace $(B^A/U, E)$ with an honest-to-god model of set theory, i.e. show that it is isomorphic to $(M, \in)$ for some transitive set $M$.

\begin{lemma}[Mostowski collapse]
\index{Mostowski collapse lemma}
Let $E$ be a well-founded binary relation on $A$ such that $(A, E)$ satisfies the axiom of extensionality. Then there is a transitive set $M$ such that $(M, \in)$ is isomorphic to $(A, E)$.
\end{lemma}
\begin{proof}
Since $E$ is a well-founded binary relation, there is a minimal element $0$ of $A$. By extensionality, $0$ is uniquely defined. Now define $\pi: E \to V$ by sending $\pi(0) = \emptyset$, and
$$\pi(x) = \{\pi(y): yEa\}.$$
Then $\pi(y) \in \pi(x)$ iff $y E x$ and the map $\pi$ is injective. Let $M = \pi(E)$; by construction $M$ is transitive and $(M, \in) \cong (A, E)$ by $\pi$.
\end{proof}

\begin{definition}
Let $(A, E)$ meet the hypotheses of Mostowski's collapse lemma and let $M$ be the transitive set given by the conclusion of that lemma. We call $M$ the \dfn{transitive collapse} of $(A, E)$.
\end{definition}
So, provided that our model of set theory was a well-founded set to begin with, there is no loss in assuming that it is a transitive model. Later we will use a theorem known as Scott's trick to drop the assumption that $A$ was a set, so we can even take the transitive collapse of a proper class.

\begin{example}
There exist models of ZFC which are not well-founded, and in particular do not have a transitive collapse.
To see this, consider the language $(\in, c_1, c_2, \dots)$ where $c_i$ are new constant symbols.
Let $T$ extend ZFC by declaring that $c_{i+1} \in c_i$ for every $i$. Then by compactness, $T$ has a model if ZFC does, and clearly $T$ is not well-founded.
The trouble here is that even though $M$ satisfies the axiom of foundation, $M$ thinks that the sequence $\vec c = \{c_1, c_2, \dots\}$ does not exist, even though it does in $V$.
However, if we were to take the transitive collapse of $M$, then $\vec c$ would appear in $M$, a contradiction.
\end{example}

Of course, we are mainly interested in models that are well-founded anyways, but the inability to take transitive collapse will be a major hurdle to overcome later when we take iterated ultrapowers.

\begin{lemma}
Let $E$ be a well-founded binary relation on $M$ which satisfies the axiom of extensionality, and let $U$ be a countably complete ultrafilter. Then $\Pi_U(M, E)$ is well-founded.
\end{lemma}
\begin{proof}
Suppose not, so there are functions $\{f_n: n \in \omega\}$ such that $[f_{n+1}] E_U [f_n]$, where $[\cdot]$ sends a function to its equivalence class under $U$. Therefore
$$X_n = \{x \in A: f_{n+1}(x) E f_n(x)\} \in U$$
but $U$ is countably complete, so $\bigcap_n X_n \in U$. Therefore there is a $x \in \bigcap_n X_n$.
So $f_{n+1}(x)Ef_n(x)$, yet $E$ is well-founded, a contradiction.
\end{proof}
Therefore we may take the Mostowski collapse of $\Pi_U(M, E)$.
\begin{definition}
Let $(M, E)$ be a well-founded binary relation which satisfies the axiom of extensionality, and let $U$ be a countably complete ultrafilter. The \dfn{transitive ultrapower} of $(M, E)$ by $U$, $\Ult(M, E, U)$, is the Mostowski collapse of the ultrapower of $\Pi_U(M, E)$.
\end{definition}
The ultrapower map $M \to \Ult(M, E, U)$ is then an elementary embedding.

We now carry out the above construction to remove the assumption that $M$ is a set. We define the equivalence class of a function $f: A \to M$ under some ultrafilter $U$ on $A$ by
$$[f]_U = \{g \in M^A: g \sim_U f \text{ and $g$ has minimal rank}\}.$$
The restriction to sets of minimal rank is known as \dfn{Scott's trick}.
Since the rank of elements of $[f]_U$ is bounded, $[f]_U$ is a set, and we may let $\Pi_U(M, E)$ be the class of all such sets.
Los' theorem goes through, but now it is a schema of theorems rather than a first-order theorem.

However, the Mostowski collapse lemma does not go through unchanged.
\begin{example}
Take $(\Ord, E)$ where $\alpha E \beta$ if $\alpha,\beta$ have the same parity and $\alpha < \beta$, or if $\alpha$ is even and $\beta$ is odd.
This has ordertype $\Ord + \Ord$. So if $M$ was the transitive collapse of $(\Ord, E)$, we would have a nontrivial order-embedding $\Ord \to \Ord$.
But, by transfinite induction, the only order-embedding $\Ord \to \Ord$ is constant, so this is a contradiction.
Therefore $(\Ord, E)$ is a counterexample to the Mostowski collapse lemma.
\end{example}
\begin{definition}
By a \dfn{class model} we mean a pair of classes $(M, E)$ where $E \subseteq M^2$ is a well-founded binary relation which satisfies the axiom of extensionality. A \dfn{setlike class model} is a class model $(M, E)$ such that for every $x \in M$,
$$\{y \in M: y E x\}$$
is a set.
\end{definition}
\begin{lemma}[Mostowski collapse lemma with Scott's trick]
\index{Mostowski collapse lemma!with Scott's trick}
Assume that $(M, E)$ is a setlike class model. Then there is a transitive class $N$ such that $(N, \in)$ is isomorphic to $(M, E)$.
\end{lemma}
The proof is the same as before. The asumption that the class model is setlike guarantees that each stage of the induction is a set.

The above construction allows us to take ultrapowers of the universe $V$.
\begin{definition}
If $j: M \to N$ is an elementary embedding, the \dfn{critical point} of $j$, $\crt j$, is the ordinal $\alpha$ such that $j_U|V_\alpha$ is the identity and $j_U(\alpha) > \alpha$, if one exists.
\end{definition}
Clearly if an elementary embedding has a critical point, then it is not the identity. So to show that we can nontrivial ultrapowers of $V$ it suffices to find a critical point.
\begin{theorem}
\label{critical point is a measurable cardinal}
Let $U$ be a countably complete ultrafilter on a set $A$. Let $\kappa = \comp U$ and let $j_U: V \to \Ult(V, U)$ be the ultrapower map. Then:
\begin{enumerate}
\item $\crt j_U = \kappa$.
\item $V_{\kappa + 1}^{\Ult(V, U)} = V_{\kappa + 1}$.
\item $\Ult(V, U)^\kappa \subseteq \Ult(V, U)$.
\item If $A = \kappa$, then $\Ult(V, U)^{\kappa + 1} \not \subseteq \Ult(V, U)$.
\item If $A = \kappa$, then $U \notin \Ult(V, U)$.
\item If $A = \kappa$, then $V_{\kappa + 2} \not \subseteq \Ult(V, U)$.
\end{enumerate}
\end{theorem}
\begin{lemma}
Under the hypotheses of Theorem \ref{critical point is a measurable cardinal}, $j_U|\kappa$ is the identity.
\end{lemma}
\begin{proof}
By induction. Assume that $\alpha < \kappa$ and if $\gamma < \alpha$ then $j_U(\gamma) = \gamma$. Let $c_\alpha$ be the constant function that returns $\alpha$; we must show $[c_\alpha]_U = \alpha$. If $\gamma < \alpha$ then $\gamma = [c_\gamma]_U \in [c_\alpha]_U$ by assumption. So $\alpha \subseteq [c_\alpha]_U$.

Conversely, assume $[f]_U \in [c_\alpha]_U$. By Los, this implies $\{x \in A: f(x) < \alpha\} \in U$. We claim there is a $\gamma < \alpha$ such that $\{x \in A: f(x) = \gamma\} \in U$.
If not, $\bigcap_\gamma \{x \in A: f(x) \neq \gamma\} \in U$ since $U$ is $\kappa$-complete, so $U$-almost every $x \in A$ is sent to $\geq \alpha$, a contradiction. So $[f]_U = \gamma < \alpha$.
\end{proof}
\begin{proof}[Proof of Theorem \ref{critical point is a measurable cardinal}]
We show that $j_U|V_\kappa$ is the identity by induction on $\alpha$. This is clear for limits, so assume that $j_U|V_\alpha$ is the identity.
Let $x \in V_{\alpha + 1}$; then $x \subseteq V_\alpha$, and any $y \in x$ satisfies $y \in V_\alpha$, hence $j_U(y) = y$ by induction.
So $x \subseteq j_U(x)$.
Conversely, $j_U(x) \subseteq V_{j_U(\alpha)}^{\Ult(V, U)} \subseteq V_\alpha^{\Ult(V, U)}$ by the lemma and induction. By induction, this implies $j_U(x) \subseteq x$. Therefore $j_U|V_{\alpha + 1} = \id$.

Now we show $j_U(\kappa) > \kappa$. Thus we must find an $[f] < [c_\kappa]$ such that if $\alpha < \kappa$, $[c_\alpha] < [f]$.
Since $\kappa = \comp U$, there are $\{X_\alpha: \alpha < \kappa\}$ such that $X_\alpha \in U$ and $\bigcap_\alpha X_\alpha \notin U$. Let
$$f(x) = \begin{cases}
\text{ the least $\gamma$ such that $a \notin X_\gamma$, if one exists}\\
0,\text{ else.}
\end{cases}$$
Then $f^{-1}(0) \cap \kappa \in U$, and $[f] < [c_\kappa]$. If $\alpha < \kappa$, $X_\alpha \in U$, so $\{\gamma < \kappa: f(\gamma) > \alpha\} \in U$. Therefore $[f] > [c_\alpha]$. So $j_U(\kappa) \neq \kappa$.

For the second claim, we have already shown $V_\kappa^{\Ult(V, U)} = V_\kappa$, and since $\Ult(V, U) \subseteq V$, $V_{\kappa + 1}^{\Ult(V, U)} \subseteq V_{\kappa + 1}$.
If $x \in V_{\kappa + 1}$, then $x \subseteq V_\kappa$. Since $j_U$ is an elementary embedding, $j_U(x) \subseteq V_{j_U(\kappa)}^{\Ult(V, U)}$ and if $y \in x$, $j_U(y) \in j_U(x)$.
So $x = j_U(x) \cap V_\kappa$ and hence $V_{\kappa + 1} \subseteq V_{\kappa + 1}^{\Ult(V, U)}$.

For the third claim, let $\{[f_\alpha]\}_{\alpha < \kappa}$ be a $\kappa$-sequence. Let
$$g(a) = \{f_\alpha(a)\}_{\alpha < \kappa}.$$
Then $[g](\alpha) = [f_\alpha]$.

Now assume $A = \kappa$. For the fourth claim, we must show that $j_U|\kappa^+ \notin \Ult(V, U)$. We first show that
$$j_U(\kappa^+) = \sup_{\alpha < \kappa^+} j_U(\alpha).$$
If not, then $\sup_\alpha j_U(\alpha) < j_U(\kappa^+)$. Let $[f]$ denote the left-hand side. Then for $U$-almost every $\alpha$, $f(\alpha) < \kappa^+$.
So there is a $\beta < \kappa^+$ so that the image of $f$ is not contained in $\beta$. So $[f] < [c_\beta]$, which is impossible.

For the fourth and fifth claims, it suffices to show that if $U \in \Ult(V, U)$ then $j_U|\kappa^+ \in \Ult(V, U)$.
Indeed, $j_U|\kappa^+$ is computable from $U$ and the $\kappa$-sequences in $\kappa^+$ (which we already showed are in $\Ult(V, U)$).
\end{proof}

\begin{theorem}[Scott-Keisler]
\index{Scott-Keisler theorem}
Let $\kappa$ be an ordinal. The following are equivalent:
\begin{enumerate}
\item $\kappa$ is a measurable cardinal.
\item There is a transitive class $M$ and an elementary embedding $j: V \to M$ such that $\kappa = \crt j$.
\item There is a transitive class $N$ and an elementary embedding $j: V_{\kappa + 1} \to N$ such that $\kappa = \crt j$.
\end{enumerate}
\end{theorem}
\begin{proof}
$1 \implies 2 \implies 3$ is obvious from the above (here $N = M \cap V_{j(\kappa) + 1}$). For the final direction, let
$$U = \{x \subseteq \kappa: \kappa \in j_U(x)\}.$$
We must show that $U$ is a nontrivial, $\kappa$-complete ultrafilter.
To see that $U$ is nontrivial, if $\alpha < \kappa$, then $\kappa \notin \{\alpha\} = j_U(\{\alpha\})$.
Clearly $U$ is closed upwards and nonempty. To see that $U$ is $\kappa$-complete, note that if $\gamma < \kappa$, then $j_U(\gamma) = \gamma$ so
$$j_U\bigcap_{\alpha < \gamma} X_\alpha = \bigcap_{\alpha < \gamma} j_U(X_\alpha) \ni \kappa.$$
\end{proof}
The fact that
$$U = \{x \subseteq \kappa: \kappa \in j_U(x)\}$$
is important in its own right.







\section{Measurable cardinals are very large}
We now show that measurable cardinals humiliate both Mahlo cardinals and $L$.

\begin{theorem}
Let $\kappa$ be a measurable cardinal. Then $\kappa$ is the $\kappa$th Mahlo cardinal in $V$.
\end{theorem}
\begin{proof}
Let $C$ be a club in $\kappa$ and let $j: V \to M$ be an elementary embedding such that $\crt j = \kappa$.
Then $j(C)$ is a club in $j(\kappa)$, so $j(C)$ is cofinal in $j(\kappa)$. But $j(\kappa) > \kappa$ and $\kappa$ is a limit of $C$ in $\kappa^+$, hence in $j(\kappa)$, so $\kappa \in j(C)$.

Since $\kappa$ is inaccessible,
$$M \models \text{$j(C)$ contains an inaccessible cardinal},$$
so $C$ actually contains an inaccessible cardinal. Therefore $S = \{\lambda < \kappa: \text{$\lambda$ is inaccessible}\}$ is stationary in $\kappa$, so $\kappa$ is Mahlo.

Therefore
$$M \models \text{$\kappa$ is Mahlo},$$
since $V_{\kappa + 1} \subseteq M$. But $j(\kappa)$ is also a Mahlo cardinal, so $\kappa < j(\kappa)$. Now if $\alpha < \kappa$, $j(\alpha) = \alpha$ and so
$$M \models \text{There is a Mahlo cardinal $j(\lambda)$ such that $\alpha < j(\lambda) < j(\kappa)$}.$$
Therefore in $V$, there is a Mahlo cardinal $\lambda$ such that $\alpha < \lambda < \kappa$.

Since $\alpha$ was arbitrary, it follows that
$$\kappa = \sup \{\lambda < \kappa: \text{$\lambda$ is a Mahlo cardinal}\}$$
but $\kappa$ is a regular cardinal, so the set of Mahlo cardinals under $\kappa$ must have cardinality $\kappa$.
\end{proof}
In particular, $\kappa$ is the $\kappa$th inaccessible cardinal in $V$.

We now show that not only do measurable cardinals dwarf inaccessible cardinals, they dwarf G\"odel's constructible universe.
\begin{theorem}
If there is a measurable cardinal then $V \neq L$.
\end{theorem}
\begin{proof}
Assume $V = L$ and let $j: V \to M$ be an elementary embedding, so that $\crt j = \kappa$ is a measurable cardinal.
Since $L$ is the smallest inner model of $V$, $V = L = M$. Moreover $j(\kappa) > \kappa$, but
$$L = M \models \text{$j(\kappa)$ is the least measurable cardinal},$$
a contradiction.
\end{proof}
Maddy, Woodin, Steel, and other philosophers argue that the axiom ``There is a proper class of measurable cardinals" is true because it is falsifiable, and yet nobody can falsify it.
(Of course, since this axiom gives a proper class of models of ZFC, it is not provable). Thus we may conclude that $V \neq L$.
This is disappointing, because $L$ decides a lot of independent sentences, such as the continuum hypothesis.
A major program in set theory is to find a $L$-like model, known as the Ultimate $L$, that decides a lot of sentences and is compatible with suitable large cardinals.

\chapter{One measurable cardinal}
We find the minimal model $L[U]$ which contains $L$ and a measurable cardinal.
This model isn't the Ultimate $L$, because it doesn't even have two measurable cardinals; in fact, two measurable cardinals humiliate $L[U]$ in the same way that one measurable cardinal humiliated $L$, but it's a step in the right direction.
In particular, $L[U]$ will satisfy GCH, have a definable well-ordering of $\RR$, and otherwise decide lots of things that we could decide in $L$ but not from ZFC alone.

\section{Generalizing $L$}
Throughout this section, we both use $A$ to mean a fixed set, and a new constant symbol in our language which refers to that set.
\begin{definition}
Let $A$ be a set. Let $\pset_A(X)$ be the set of all subsets of $X$ which are definable in the language $(\in, A)$ with parameters in $X$.

Now let $L_0[A] = \emptyset$, $L_{\alpha + 1}[A] = \pset_A(L_\alpha[A])$, $L_\lambda[A] = \bigcup_{\alpha < \lambda} L_\alpha[A]$, and $L[A] = \bigcup_{\alpha \in \Ord} L_\alpha[A]$.

For any $x \in L[A]$, let $\rank^{L[A]}$ be the least $\alpha$ such that $x \in L_\alpha[A]$.
\end{definition}
Thus we expand the notion of definable power set to allow us to query whether a set is in $A$. Notice that $L[0] = L$ and in fact $\pset_0$ is the definable powerset.

\begin{example}
If $x$ is a real number such that $x \notin L$ (for example, $x$ is a random real), then $L[x] \supset L$. In fact, $L_n[x] = L_n$, and so $L_\omega[x] = L_\omega$. But the formula $y: y \in x$ defines $x$ as a subset of $\omega$ in the language $(\in, x)$, so $L_{\omega + 1}[x] \neq L_{\omega + 1}.$
\end{example}

We should show that $L[A]$ satisfies ZFC. This isn't as trivial as it seems.
\begin{theorem}[Woodin?]
Suppose that there are infinitely many Woodin cardinals in $V$ and a measurable cardinal above them all.
Then $L[\RR]$ is not a model of ZFC, but rather a model of ZF, the Baire category theorem, and the axiom of determinacy.
\end{theorem}

\begin{lemma}
Let $M = (M, \in)$ be a transitive class. Then:
\begin{enumerate}
\item $M \models $ extensionality and foundation.
\item If $\emptyset, \omega \in M$, then $M \models $ empty set and infinity.
\item $M \models $ pairing, union, and powerset iff $\forall x, y \in M(\{x, y\}, \bigcup x, 2^x \in M)$.
\item $M \models $ replacement iff $\forall F \in \Def M \cap M^M(\forall a \in M(F"a \in M))$ and $\forall F \in \Def M \forall a \in M(F \cap a \in M)$.
\item If there is a definable well-ordering of $M$, then $M \models $ choice.
\end{enumerate}
\end{lemma}
Here $\Def M$ is the subclass of $M$ consisting of all sets definable from parameters in $M$ and $F"a = \{F(b): b \in a\}$. The proof is essentially obvious.

\begin{lemma}
$L[A] \models $ ZF.
\end{lemma}
\begin{proof}
$L[A]$ is a transitive class $\ni \emptyset$ by definition. Since $\omega$ is definable from parameters in $L_\omega[A]$, $\omega \in L_{\omega + 1}[A]$.
Similarly, $\{x, y\},\bigcup x, 2^x$ are all definable from the parameters $x,y$, so they are in $L_{\alpha + 1}[A]$ if $x, y\in L_\alpha[A]$.
If $a \in L_\alpha[A]$ and $F \in \Def L_\alpha[A] \cap L_{\alpha}[A]^{L_\alpha[A]}$ then $F"a$ is definable from the parameters $a,F$, so $F"a \in L_{\alpha + 1}[A]$. Similarly for $F \cap a$.
\end{proof}

\begin{lemma}
Let $\kappa$ be an inaccessible cardinal. Then $L_\kappa[A] \models$ ZF.
\end{lemma}
\begin{proof}
Same as above, with some modifications, as below:

For power set, let $x \in L_\kappa[A]$ and $\alpha < \kappa$ such that $x \in L_\alpha[A]$. Let $f: 2^x \cap L_\kappa[A] \to \kappa$ satisfy $f(y) = \rank^{L[A]} y$.
Since $2^x < \kappa$ and $\kappa$ is regular, $f$ is not cofinal, so $f$ carries $2^x \cap L_\kappa[A]$ into some $\lambda < \kappa$.
The part of replacement concerned with $F \in L_\kappa[A]^{L_\kappa[A]}$ is similar.

For replacement, first let $C \in \Def L_\kappa[A]$ and $a \in L_\kappa[A]$. Since $C \in \Def L_\kappa[A]$ there is a formula $\varphi$ which defines $C \cap a$ in $L_\kappa[A]$ from $a$ and a parameter $p \in L_\kappa[A]$.
By reflection, there is an $\alpha$ such that $L_\alpha[A]$ is an elementary substructure of $L_\kappa[A]$, and then $a,p \in L_\alpha[A]$.
Therefore $C \cap a \in L_{\alpha + 1}[A]$.
\end{proof}

\begin{lemma}
Let $A$ be a set. There is a $\Pi_2$ sentence ``$V = L[A]$" in the language $(\in, A)$ such that for any transitive class $M$, $M \models V = L[A]$ iff $M = L[A]$ or there is a limit ordinal $\lambda$ such that $M = L_\lambda[A]$.
\end{lemma}
\begin{proof}
The sentence in question is
$$\forall x \exists \alpha(x \in L_{\alpha+1}[A]).$$
So we need $L_{\alpha+1}[A]$ to have a $\Sigma_1$ definition.
Let $\text{DEF}(x, M)$ mean that $x$ is the set of all $y \in M$ such that $y$ is definable in $(M, \in, A)$ without parameters, which is a $\Sigma_1$ formula, since satisfication is a $\Sigma_1$ relation.
Here we are using the fact that $V_\omega$ interprets $\PA$ to apply G\"odel coding.

Let $\alpha \in \Ord$. We uniquely define a function $f$, the \dfn{$(L[A], \alpha)$-rank function}, by declaring that:
\begin{enumerate}
\item $\dom f = \alpha + 1$.
\item $f(0) = 0$.
\item $\forall \gamma, \delta \leq \alpha$, if $\delta = \gamma + 1$ then $\text{DEF}(f(\gamma), f(\delta))$, and
\item $\forall \lambda \leq \alpha$, if $\lambda$ is a limit ordinal then $f(\lambda) = \bigcup_{\beta < \lambda} f(\beta)$.
\end{enumerate}
Since DEF was a $\Sigma_1$ formula, so is the definition of $f$.

Let
$$\text{LEV}(y, \alpha) = \alpha \in \Ord \wedge \exists f(\text{$f$ is the $(L[A], \alpha)$-rank function} \wedge f(\alpha) = y).$$
Then $\text{LEV}(y, \alpha)$ is a $\Sigma_1$ definition of $y = L_\alpha[A]$.
Finally, let
$$(x \in L_{\alpha+1}[A]) = \exists y((x \in y) \wedge \text{LEV}(y, \alpha)).$$

To check that this works, first note that if $\lambda$ is a limit ordinal, then $L_\lambda[A] \models \text{DEF}(L_{\alpha+1}[A], L_\alpha[A])$.
It is easy and tedious to check that among transitive sets $M$, $M \models V = L[A]$ iff $M = L_\lambda[A]$ for some $\lambda$.
\end{proof}

\section{Solovay's theorem and condensation}
Henceforth we consider the model $L[U]$, where $U$ is a normal uniform ultrafilter on a measurable cardinal.

\begin{definition}
Let $\kappa$ be a regular cardinal and $U$ an ultrafilter. Then $L[U]$ is a \dfn{$\kappa$-model} if
$$L[U] \models \text{$U$ is a normal uniform ultrafilter on $\kappa$}.$$
\end{definition}
In particular, if $L[U]$ is a $\kappa$-model, then
$$L[U] \models \text{$\kappa$ is a measurable cardinal}$$
since measurable cardinals are the only cardinals with normal uniform ultrafilters.

\begin{lemma}
If $\kappa$ is a measurable cardinal, then there is a $\kappa$-model.
\end{lemma}
\begin{proof}
Let $U$ be a normal uniform ultrafilter on $\kappa$. Then $U \cap L[U] \in L[U]$ since $U$ is a set and $L[U]$ is transitive. Now if $x \in L_\alpha[U]$ then it is equivalent to query $x \in U$ and $x \in U \cap L[U]$. Let $U' = U \cap L[U]$; then $L[U'] = L[U]$.

We show that
$$L[U] \models \text{$U'$ is a normal uniform ultrafilter on $\kappa$};$$
then $L[U] = L[U']$ is a $\kappa$-model.
Since $\kappa \in U'$, $U'$ is nonempty. If $x \in U'$ and $x \subseteq y \subseteq \kappa$, and $y \in L[U]$, then $y \in U$ since $U$ is an ultrafilter.
Similarly $U'$ is $\kappa$-complete and maximal, hence a uniform ultrafilter.

It suffices to show that $U'$ is normal. Let $f \in L[U] \cap \kappa^\kappa$ be such that $\{\alpha < \kappa: f(\alpha) < \alpha\} \in U'$.
Then for $U$-almost every $\alpha$, $f(\alpha) < \alpha$, so there is a $\beta \in \Ord \subseteq L[U]$ such that for $U$-almost every $\alpha$, $f(\alpha) = \beta$, since $U$ is normal.
Therefore $U'$ is normal.
\end{proof}

We now show that $L[U]$ is very $L$-like.
\begin{axiom}[global choice]
\index{axiom of global choice}
There is a well-ordering of $V$.
\end{axiom}
This really needs to be stated in class theory, rather than set theory.
Clearly global choice implies choice.

\begin{lemma}
If $L[U]$ is a $\kappa$-model then $L[U]$ is a model of global choice.
Moreover, the well-ordering is definable if we are allowed an oracle that tells us the well-ordering of $\kappa$.
\end{lemma}
\begin{proof}
We show that there is a well-ordering of $L_\alpha[U]$ for any $\alpha$, which is definable from the well-ordering of $\kappa$.
These well-orderings will all cohere in that if $\alpha < \beta$ then the well-ordering of $L_\alpha[U]$ will extend to the well-ordering of $L_\beta[U]$. So we will be able to take a union over all ordinals.

Clearly $L_0[U]$ is well-ordered and we can take unions at limit stages. Now if $L_\alpha[U]$ is well-ordered, we have a lexicographic well-ordering on $L_\alpha[U]^{<\omega}$ and
$$L_{\alpha+1}[U] = \{x \in L_\alpha[U]: \exists [\varphi] \in V_\omega \exists b \in L_\alpha[U]^{<\omega}(L_\alpha[U] \models \varphi(x, b))$$
so the lexicographic well-orderings on $L_\alpha[U]^{<\omega}$ and on $V_\omega$ give a well-ordering of $L_{\alpha+1}[U]$.
The only trouble is that we might throw in some sets from $U$, but if $x \in U$ then $x \subseteq \kappa$ so $x$ is well-ordered in a canonical way.
\end{proof}
This was basically the same proof as in $L$. We see the problem with $L[\RR]$: if enough subsets of $\RR$ are determined, which happens if there are too many large cardinals, there is no good way to make their well-ordering cohere.

\begin{definition}
Let $M$ be a model of ZFC, and for every $\alpha \in \Ord^M$ let $M_\alpha = \{x \in M: \rank^M x \leq \alpha\}$.
Then we say that $M$ has \dfn{condensation} if for every limit ordinal $\lambda$ in $M$ and elementary substructure $X$ of $M_\lambda$, there is a limit ordinal $\xi$ such that $M_\xi$ is the transitive collapse of $X$.
\end{definition}

\begin{lemma}
Every $\kappa$-model $L[U]$ has condensation.
\end{lemma}
\begin{proof}
Since $X$ is well-founded and satisfies extensionality, it has a transitive collapse $M$. But then $M \models V = L[U]$, giving a $\xi$ such that $M = L_\xi[U]$.
\end{proof}
Again, the proof that $L$ has condensation was basically the same.
The significance is that if a model $M$ has condensation, then it is easy to prove that $M$ satisfies GCH at least on a tail end of $\Card$.
Unfortunately, as $\kappa$ was measurable and hence $> 2^{\aleph_0}$, we will not be able to use condensation to prove that $L[U]$ satisfies the continuum hypothesis; we will need some much more powerful machinery to do that.

\begin{lemma}
Let $L[U]$ be a $\kappa$-model. For every $\alpha \geq \kappa$, $\card L_\alpha[U] = \card \alpha$.
\end{lemma}
\begin{proof}
By transfinite induction starting at $\kappa$, whose base case is a transfinite induction up to $\kappa$.

The base case is that $\card L_\kappa[U] = \kappa$.
We have $\kappa \leq \card L_\kappa[U]$ since $\kappa = \sup \Ord^{L_\kappa[U]}$. Conversely, we run a second transfinite induction up to $\kappa$.
Clearly the claim $\card L_\alpha[U] \leq \kappa$ holds at finite and limit stages (as long as $\alpha < \kappa$) since $\kappa$ is a regular cardinal.
If $\card L_\alpha[U] \leq \kappa$ and $\omega \leq \alpha < \kappa$, then
$$\card L_{\alpha + 1}[U] = \card L_\alpha[U] + \aleph_0 = \card L_\alpha[U].$$
Here the first equality is because everything is definable from the previous stage, we aren't adding any subsets of $\kappa$ yet, and there are $\aleph_0$ many definitions.
This implies the base case. Limit stages are clear.

For the successor case, assume $\card L_\alpha[U] = \card \alpha$.
Then
$$\card L_\alpha[U] \leq \card L_{\alpha+1}[U] \leq \kappa + \aleph_0 + \card L_\alpha[U] = \card L_\alpha[U]$$
since we are only adding subsets of $\kappa$ and stuff definable from the previous stage.
Therefore
$$\card(\alpha+1) = \card \alpha = \card L_\alpha[U] = \card L_{\alpha+1}[U].$$
\end{proof}

\begin{theorem}[Solovay]
\index{Solovay's theorem}
Let $L[U]$ be a $\kappa$-model. Then
$$L[U] \models \text{$\kappa$ is the only measurable cardinal and GCH holds above $\kappa$}.$$
\end{theorem}
\begin{proof}
First we prove GCH. Let $\lambda \geq \kappa$; we must show $2^\lambda \leq \lambda^+$.
We know that $\card L_{\lambda^+} = \lambda^+$. Therefore we show $2^\lambda \cap L[U] \subseteq L_{\lambda^+}[U]$.

Let $A \in 2^\lambda \cap L[U]$ and let $\xi = \max(\rank^{L[U]} A, \kappa, U)$, so $A \in L_\xi[U]$.
By reflection, there is an elementary substructure $H$ of $L_\xi[U]$ such that $A \in H$, $\lambda \subseteq H$, and $\card A \in H$.
Moreover, by taking a minimal such structure we can assume $\card H = \lambda$.
By condensation, there is a $\xi'$ such that $L_{\xi'}[U]$ is the transitive collapse of $H$, and in particular $\xi' < \lambda^+$ and $A \in L_{\xi'}[U]$.
Therefore $A \in L_{\lambda^+}[U]$.

Now let $V = L[U]$, and suppose that $\lambda$ is a measurable cardinal equipped with a normal uniform ultrafilter $U'$, such that $\lambda \neq \kappa$.
Let
$$j: V \to \Ult(V, U')$$
be the elementary embedding given by $U'$.
Then $U' \notin \Ult(V, U')$, yet $U' \in V$; we will prove $\Ult(V, U') = V$, a contradiction.

If $\lambda > \kappa$, then $\Ult(V, U') = L[j(U)]$ since $j$ must preserve the truth of the sentence $V = L[U]$. Since $\crt j > \kappa$, $j(U) = U$, so $L[j(U)] = V$, as desired.

If $\lambda < \kappa$, let $M = \Ult(V, U')$.
\begin{lemma}
One has
$$j(U) = U \cap M.$$
\end{lemma}
\begin{proof}[Proof of lemma]
We claim that $j(U) \subseteq U$; then $j(U) \subseteq U \cap M$, and since $U \cap M \subseteq j(U)$ since $j$ is an elementary embedding, this gives the lemma.

Let $[\cdot]$ denote the ultraprojection $V^\lambda \to M$.
Suppose $[f] \in j(U)$, so that $f \in V^\lambda$; we must show $[f] \in U$. Let $j(U) = [c_U]$.
By definition, for $U'$-almost every $\xi < \lambda$, $f(\xi) \in c_U(\xi) = U$, so we may modify $f$ $U'$-almost nowhere and assume that $f \in U^\lambda$.
Since $\lambda < \kappa$,
$$x = \bigcap_{\xi < \lambda} f(\xi) \in U.$$
Let $y = j(x)$, so $y \in j(U)$.

For every $\xi' < \lambda$ we have
$$\bigcap_{\xi < \lambda} f(\xi) \subseteq f(\xi')$$
so
$$y = \left[c_{\bigcap_{\xi < \lambda} f(\xi)} \right] \subseteq [f].$$

We now claim $y \in U$. To see this, let
$$I = \{\lambda < \xi < \kappa: \text{$\xi$ is inaccessible}\}.$$
Since $\kappa$ is Mahlo, $I$ is cofinal in $\kappa$; since $U$ is uniform, it follows that $I \in U$.
Therefore $x \cap I \in U$, so $x$ is cofinal in $\kappa$, and hence $y$ is cofinal in $j(\kappa) > \kappa$, hence cofinal in $\kappa$.

Therefore $y \in U$, and so $[f] \in U$. Therefore $U' \subseteq U$.
\end{proof}
We already showed that $L[U \cap L[U]] = L[U]$ in a previous lemma. Therefore the above lemma gives
$$M = L[U \cap M] = L[U] = V,$$
our desired contradiction.
\end{proof}

\section{Iterated ultrapowers}
Our goal is to show that if $L[U]$ is the $\kappa$-model of a measurable cardinal $\kappa$ then $L[U]$ acts a lot like $L$. In particular, we are going to prove that $L[U]$ thinks GCH is true.
To do this, we need a lot more machinery.

We will mainly be interested in the case when $M$ is a model of ZFC + ``There is exactly one measurable cardinal," but we need not assume this yet; we don't even need to assume that $M$ has power sets.
We also don't need to assume that the ultrafilter $U$ is actually an element of $M$.

\begin{definition}
Let $M = (M, \in)$ be a transitive model of ZFC minus powerset. Let $U$ be an ultrafilter. We say that $M$ is \dfn{$\kappa$-amenable} to $U$ if for all $x \in M$ such that $\card^M x = \kappa$, $x \cap U \in M$. We say that $M$ is \dfn{fully amenable} to $U$ if $M$ is $\kappa$-amenable to $U$ for all $\kappa$.
\end{definition}
Thus if $U$ is actually in $M$, it follows that $M$ is fully amenable. Here $\kappa$ is a cardinal in $M$.

\begin{definition}
Let $M$ be a transitive model of ZFC minus powerset. An \dfn{$M$-ultrafilter} on $\kappa$ is an ultrafilter $U$ on $\kappa$ such that
$$M \models \text{$U$ is a normal uniform ultrafilter on $\kappa$}$$
and $M$ is $\kappa$-amenable to $U$.
\end{definition}
Under these hypotheses, we can again take the ultrapower $\Pi_U(M)$.
It may be that $\kappa$ is countable (even finite), in which case it may not be true that $\Pi_U(M)$ is well-founded.
However, if $\Pi_U(M)$ is well-founded, we again define its transitive collapse $\Ult(M, U)$ using Mostowski's lemma.

\begin{lemma}
Let $M$ be a transitive model of ZFC minus powerset and $U$ an $M$-ultrafilter on $\kappa$. Assume that $\Pi_U(M)$ is well-founded, and let
$$j: M \to \Ult(M, U) = M'$$
be the ultrapower map. Then:
\begin{enumerate}
\item $j$ is cofinal.
\item $j|V_\kappa^M$ is the identity.
\item $V_{\kappa+1}^M = V_{\kappa+1}^{M'}$.
\item $U \notin M'$.
\item $U' = j(U)$ is a $M'$-ultrafilter.
\item $\card M = \card M'$.
\end{enumerate}
\end{lemma}
\begin{proof}
To see that $j$ is cofinal, let $[f] \in M'$; then if $g(x) \ni f(x)$ for all $x$, we have $[f] \in [g]$.
The proofs that $j$ restricts to the identity, preserves $V_{\kappa+1}$, and $U \notin M'$ are the same as before.
Since $U$ is a $M$-ultrafilter, $j$ an elementary embedding, $U'$ is a $M'$-ultrafilter.

For the cardinality claim, note that in $M$, $\card U < \card M$ (since $M$ thinks itself is a proper class); in particular, for any $\alpha \in \Ord^M$ sufficiently large, we have $\card U < \card M_\alpha$ in $M$.
So in $M$,
$$\card M_{j(\alpha)}' = \card M_\alpha^\kappa/U \leq \card M_\alpha.$$
Since $\card M_{j(\alpha)}' \geq \card M_\alpha$ in $M$, $M$ thinks that $\card M_\alpha = \card M_{j(\alpha)}'$.

We claim that if $M \models \card x = \card y$, then in fact $\card x = \card y$. To see this, note that since $M$ is a transitive class, there is actually an honest-to-god bijection $f: x \to y$ between honest-to-god sets in $M$, and so $f$ remains a bijection in $V$. Therefore, for all $\alpha$, $\card M_\alpha = \card M_{j(\alpha)}'$. Therefore
$$\card M = \sup_\alpha \card M_\alpha = \sup_\alpha \card M_{j(\alpha)}' = \sup_\alpha \card M_\alpha' = \card M',$$
where we used the fact that $j$ is cofinal as a map $\Ord^M \to \Ord^{M'}$.
\end{proof}

We now iterate the above construction. Let $M_0 = M$, and let $M_{n+1} = M_n'$. We will need to be able to pass to the direct limit for limit ordinals.
\begin{lemma}
Suppose that $\lambda$ is a limit ordinal, and $(M_\alpha, U_\alpha, \kappa_\alpha)_{\alpha < \lambda}$ is a $\lambda$-sequence such that $U_\alpha$ is a $M_\alpha$-ultrafilters on $\kappa_\alpha$.
Suppose further that we are given elementary embeddings $j_{\alpha,\beta}: M_\alpha \to M_\beta$ if $\alpha < \beta < \lambda$, where $j_{\alpha,\alpha+1}$ is the ultrapower map for $U_\alpha$, and the diagrams
$$\begin{tikzcd}
M_\alpha \arrow[r,"j_{\alpha,\beta}"] \arrow[dr,"j_{\alpha,\gamma}"] & M_\beta \arrow[d,"j_{\beta,\gamma}"]\\ & M_\gamma
\end{tikzcd}$$
all commute. Suppose further that the direct limit of the elementary directed system $\{j_{\alpha,\beta}: \alpha < \beta < \lambda\}$ is well-founded, and let $(M_\lambda, U_\lambda, \kappa_\lambda)$ be its transitive collapse, and let
$$j_{\alpha,\lambda}: M_\alpha \to M_\lambda$$
be the induced elementary embeddings.

Then the ordinal $\kappa_\lambda$ such that $\kappa_\lambda = j_{\alpha,\lambda}(\kappa_\alpha)$ for every $\alpha$ is well-defined, and $U_\lambda$ is a $M_\lambda$-ultrafilter on $\kappa_\lambda$.
\end{lemma}
\begin{proof}
Since the diagrams commute, $j_{\alpha,\lambda}(\kappa_\alpha)$ does not depend on the choice of $\alpha$. Since $j_{\alpha,\lambda}$ is an elementary embedding and $U_\alpha$ is a $M_\alpha$ ultrafilter on $\kappa_\alpha$, the second claim also holds.
\end{proof}
\begin{definition}
With $(M_\alpha, U_\alpha, \kappa_\alpha, j_{\alpha,\beta})_{\alpha<\beta<\tau}$ as in the previous lemma iterated up to $\tau$, we call $(M_0, U_0, \kappa_0)$ \dfn{initial data for the iterated ultrapower construction} and $(M_\alpha, U_\alpha, \kappa_\alpha, j_{\alpha,\beta})_{\alpha<\beta<\tau}$ the \dfn{iteration} of the initial data.
\end{definition}

Notice that we cannot guarantee, just from the assumptions we make above on initial data, that for every $\alpha$, $M_\alpha$ is still well-founded.
If this is the case, then we cannot take the transitive collapse, and the above induction stops prematurely.

\begin{definition}
Let $(M_0, U_0)$ be initial data for the iterated ultrapower construction and $(M_\alpha, U_\alpha)$ its iteration.
Let $\tau$ be the first ordinal such that $M_\tau$ is not well-founded, or $\tau = \Ord$ if no such ordinal exists.
If $\lambda < \tau$, we say that $(M_0, U_0)$ is \dfn{$\lambda$-iterable}. If $\tau = \Ord$, so $(M_0, U_0)$ is $\lambda$-iterable for all $\lambda$, we say that $(M_0, U_0)$ is \dfn{iterable}.
\end{definition}
In the proof of GCH we will mainly use $\omega_1$-iterable initial data. In fact, we will prove that $\omega_1$-iterable initial data is actually just iterable.

\begin{lemma}
Let $(M_0, U_0, \kappa_0)$ be initial data for the iterated ultrapower construction and $(M_\alpha, U_\alpha, \kappa_\alpha)$ its iteration.
Assume that $\tau$ is the first ordinal such that $M_\tau$ is not well-founded.
For every $\alpha < \beta < \tau$:
\begin{enumerate}
\item $\crt j_{\alpha,\beta} = \kappa_\alpha$ and $j_{\alpha,\beta}(\kappa_\alpha) = \kappa_\beta$.
\item $j_{\alpha,\beta}|V_{\kappa_\alpha}\cap M_\alpha$ is the identity and
$$V_{\kappa_{\alpha + 1}} \cap M_\alpha = V_{\kappa_{\alpha + 1}} \cap M_\beta.$$
\item If $\lambda$ is a limit ordinal then
$$\kappa_\lambda = \sup_{\xi < \lambda} \kappa_\xi.$$
\item If $M_0$ is a set then $M_\alpha$ is a set and $\card M_\alpha = \card M_0 \card \alpha$.
\end{enumerate}
\end{lemma}
\begin{proof}
We already proved 1 and 2.

For 3, let $\xi < \kappa_\lambda$; we will find a $\eta$ such that $\kappa_\eta > \xi$. Since the ultrapower map cannot decrease $\kappa$, this will suffice.
In fact let $\xi',\eta$ satisfy
$$\xi = j_{\eta,\lambda}(\xi').$$
Then $M_\eta \models \kappa_\eta > \xi'$ and since $\crt j_{\eta, \lambda} = \kappa_\eta$ it follows that $\xi' = \xi$.

For 4, cardinality does not increase at successor stages, and at limit stages $\lambda$ we take a union of $\lambda$ many models. So
$$\card M_\alpha \leq \card M_0 \card \alpha.$$
Moreover $\card M_0 \leq \card M_\alpha$ and $\{\kappa_{\alpha'}: \alpha' < \alpha\} \in M_\alpha$ so $\card M_\alpha \leq \card M_0$.
\end{proof}
\begin{lemma}
Let $(M_0, U_0)$ be initial data for the iterated ultrapower construction and $(M_\alpha, U_\alpha)$ its iteration.
If $M_0$ is $\alpha + 1$-iterable and $x \in \pset(\kappa_\alpha) \cap M_\alpha$, then $x \in U_\alpha$ iff $\kappa_\alpha \in j_{\alpha,\alpha+1}(x)$.
\end{lemma}
\begin{proof}
We can collapse this down to the case $\alpha = 0$. Then the claim is that $x \in U$ iff $\kappa \in j_U(x)$, which follows from the proof of the Scott-Keisler theorem.
\end{proof}
\begin{lemma}
Let $(M_0, U_0)$ be initial data for the iterated ultrapower construction and $(M_\alpha, U_\alpha)$ its iteration. Assume $(M_0, U_0)$ is $\lambda$-iterable and $\lambda$ is a limit ordinal. Then $x \in U_\lambda$ iff there is an $\alpha < \lambda$ such that $\{\kappa_\gamma: \alpha < \gamma < \lambda\} \subseteq x$.
\end{lemma}
\begin{proof}
Let $\gamma < \lambda,x'$ be such that
$$x = j_{\gamma,\lambda}(x').$$
Since $\crt j_{\gamma + 1, \lambda} \geq \kappa_\lambda$,
$$x' \in U_\gamma \iff \kappa_\gamma \in j_{\gamma,\gamma + 1}(x') \iff \kappa_\gamma \in x.$$
\end{proof}
\begin{lemma}
\label{seed lemma}
Let $(M_0, U_0)$ be initial data for the iterated ultrapower construction and $(M_\alpha, U_\alpha)$ its iteration.
If $(M_0, U_0)$ is $\alpha$-iterable and $x \in M_\alpha$ there are $\xi_1 < \cdots < \xi_n < \alpha$ such that
$$\kappa_{\xi_1} < \cdots < \kappa_{\xi_n}$$
and a function $f: \kappa_0^n \to M_0$, which exists in $M_0$, such that
$$x = j_{0, \alpha}(f)(\kappa_{\xi_1}, \dots, \kappa_{\xi_n}).$$
\end{lemma}
\begin{proof}
By induction on $\alpha$. Let $x \in M_{\alpha + 1}$. Then there is a $g: \kappa_\alpha \to M_\alpha$ such that
$$x = j_{\alpha, \alpha + 1}(g)(\kappa_\alpha).$$
Now find $g': \kappa_0^n \to M_0$ in $M_0$ and $\xi_i$ such that $g = j_{0,\alpha}(g')(\kappa_{\xi_1}, \dots, \kappa_{\xi_n})$ by induction.
Then let
$$f(\gamma_1, \dots, \gamma_{n+1}) = (g'(\gamma_1, \dots, \gamma_n))(\gamma_{n+1}),$$
which works.
The limit case follows by induction too, and we only need finitely many entries before the limit stage so the index $n$ doesn't run over $\omega$.
\end{proof}
This is useful because it allows us to do combinatorial tricks to find the cardinality of the set of all $x$ in the above lemma which satisfy some condition. For example, consider the following lemma, which allows us to bound above the image of $j_{0,\alpha}$ in several arguments that follow.
\begin{lemma}
Let $(M_0, U_0)$ be initial data for the iterated ultrapower construction and $(M_\alpha, U_\alpha)$ its iteration.
Assume $(M_0, U_0)$ is $\alpha$-iterable.
For any $\gamma \in \Ord^M$,
$$j_{0,\alpha}(\gamma) < (\card (\gamma^{\kappa_0}\cap M) \card \alpha)^+.$$
\end{lemma}
\begin{proof}
If $\gamma \leq \kappa_0$ then $j_{0,\alpha}$ sends $\gamma$ to itself, and
$$\gamma < \card (\gamma^{\kappa_0})^+$$
by cardinal arithmetic.

If $\gamma > \kappa_0$, recall $j_{0,\alpha}(\kappa_0) = \kappa_\alpha$, and $j_{0,\alpha}$ blows up $\gamma$ to be much larger than $\kappa_0$.
Let $\eta \in (\kappa_\alpha, j_{0,\alpha}(\gamma))$. By Lemma \ref{seed lemma} we may write
$$\eta = j_{0,\alpha}(f)(\kappa_{\xi_1}, \dots, \kappa_{\xi_n}).$$
Since $\eta < j_{0,\alpha}(\gamma)$, elementarity implies that $f$ maps into $\gamma$.

The number of possible choices of $\eta$ is the number $\card \gamma^{\kappa_0} \cap M$ of possible choices of $f$ times the number of possible choices of inputs for $f$, $\card \alpha^n = \card \alpha$.
\end{proof}
\begin{lemma}
Let $(M_0, U_0)$ be initial data for the iterated ultrapower construction and $(M_\alpha, U_\alpha)$ its iteration.
Assume $(M_0, U_0)$ is $\nu$-iterable, $\nu$ is a cardinal, and
$$\card \kappa_0^{\kappa_0} \cap M < \nu.$$
Then $\kappa_\nu = \nu$.
\end{lemma}
\begin{proof}
By a previous lemma we have $\nu \leq \kappa_\nu$. On the other hand,
$$\kappa_\nu = j_{0,\nu}(\kappa_0) < (\nu \card(\nu^{\kappa_0} \cap M))^+$$
and by assumption this implies that $\kappa_\nu \leq \nu$.
\end{proof}
\begin{lemma}
Let $(M_0, U_0)$ be initial data for the iterated ultrapower construction and $(M_\alpha, U_\alpha)$ its iteration.
Suppose $\theta \in \Ord^M$, $\alpha < \theta$, $(M_0, U_0)$ is $\alpha$-iterable, $\cof^M\theta >\alpha$, and
$$M \models \text{$\theta$ is a strong limit cardinal, and for every $\theta' < \theta$ there are functions $\kappa_0 \to \theta'$.}$$
Then
$$j_{0,\alpha}(\theta) = \theta.$$
\end{lemma}
\begin{proof}
We have $\theta \leq j_{0,\alpha}(\theta)$; to show the opposite, we show that if $\eta < j_{0,\alpha}(\theta)$, $\eta < \theta$.
By Lemma \ref{seed lemma}, we can write
$$\eta = j_{0,\alpha}(f)(\kappa_{\xi_1}, \dots, \kappa_{\xi_n}).$$
As in previous arguments, we can assume that $f$ maps into $\theta$. Since $\cof^{M_0}(\theta) > \alpha$, there is a $\theta' < \theta$ such that $f$ maps into $\theta'$.
Since $\theta$ is a strong limit cardinal,
$$\kappa_0^{\card \theta'} = 2^{\card \theta'} < \theta.$$
Therefore
$$\eta < j_{0,\alpha}(\theta') < (\card((\theta')^{\kappa_0} \cap M) \card \alpha)^+ \leq \theta.$$
\end{proof}

\section{Suitable initial data for iteration}
Thus we have built up a lot of machinery about how the iterated ultrapower construction moves ordinals. Now we consider when initial data is actually $\Ord$-iterable.
\begin{definition}
A \dfn{weakly countably complete ultrafilter} $U$ is an ultrafilter such that for any $\omega$-sequence $X_n$ in $U$, $\bigcap_n X_n$ is nonempty.
\end{definition}
\begin{definition}
Let $(M, U)$ and $(N, W)$ be initial data. We say that $(N, W)$ is \dfn{suitable initial data} if $N$ is countable, $W$ is an $N$-ultrafilter, and there is an elementary embedding $N \to M$ sending $W$ to $U$.
\end{definition}
\begin{theorem}
Let $(M, U)$ and $(N, W)$ be initial data. If $U$ is weakly countably complete in $V$ and $(N, W)$ is suitable initial data, then $(N, W)$ is $\omega_1 + 1$-iterable.
\end{theorem}
\begin{proof}
Let $W$ be an $N$-ultrafilter on $\nu$, and let $(N_\alpha, W_\alpha, \nu_\alpha, j_{\alpha,\beta})$ be its iteration.
It suffices to show that if $\alpha < \omega_1$, then there is an elementary embedding $N_{\alpha + 1}' \to M$, where $N_{\alpha + 1}'$ is the uncollapsed ultrapower of $N_\alpha$; since $M$ is well-founded, so will be $N_{\alpha + 1}'$ (since it's impossible to embed a non-well-founded relational structure in a well-founded relational structure).
Then we can take the transitive collapse $N_{\alpha + 1}$ and keep the induction going.
Limit stages are no problem because the direct limit of a chain of well-founded relational structures is well-founded.

As it turns out, we can make the diagram
$$\begin{tikzcd}
M \\
N_\alpha \arrow[u,"\sigma_\alpha"] \arrow[r,"j_{\alpha,\alpha+1}"] & N_{\alpha + 1}' \arrow[ul,"\sigma_{\alpha+1}'"]
\end{tikzcd}$$
commute by taking
$$\sigma_{\alpha + 1}'([f]) = \sigma_\alpha(f)(\eta_\alpha)$$
for any
$$\eta_\alpha \in \bigcap_{x \in W_\alpha \cap N_\alpha} \sigma_\alpha(x).$$
Such an $\eta_\alpha$ exists since there are only countably many sets $\sigma_\alpha(x)$, since $N_\alpha$ is countable, and $U$ is a weakly countably complete ultrafilter in $V$, and $W_\alpha$ maps to $U$ in the elementary embedding $\sigma_\alpha$.
\end{proof}
\begin{corollary}
Let $(M, U, \kappa)$ be initial data such that
$$M \models \text{$V_{\kappa + 1}$ exists},$$
and $U \in M$, and let $(M_\alpha, U_\alpha)$ be its iteration. The following are equivalent:
\begin{enumerate}
\item $(M, U)$ is iterable.
\item $(M, U)$ is $\omega_1$-iterable.
\item $(M, U)$ is $\omega_1 + 1$-iterable.
\item There is an $\alpha$ such that $(M_\alpha, U_\alpha)$ is weakly countably complete in $V$.=
\item For all initial data $(N, W)$, if there is an elementary embedding $N \to M$ sending $W$ to $U$, then $(N, W)$ is iterable.
\end{enumerate}
\end{corollary}
\begin{proof}
$1 \implies 2$ and $5 \implies 1$ are clear. $2 \implies 3$ holds because limit stages don't break iterability.

For $3 \implies 4$, we are given that $(M_{\omega_1}, U_{\omega_1})$ is well-founded. Let $X_n$ be an $\omega$-sequence in $U_{\omega_1}$. Then for every $n < \omega$ there is an $\alpha_n$ such that
$$\{\kappa_\gamma: \alpha_n < \gamma < \omega_1\} \subseteq X_n.$$
Let $\beta = \sup_{n < \omega} \alpha_n$; then $\beta < \omega_1$ and $\bigcap_{n < \omega} X_n$ contains $\kappa_{\beta + 1}$.
So take $\alpha = \omega_1$.

For $4 \implies 1$ and $1 \implies 5$, use the previous theorem to show that $(M_\alpha, U_\alpha)$ is iterable and hence we have inclusions $(N, W) \to (M, U) \to (M_\alpha, U_\alpha)$.
\end{proof}
\begin{corollary}
Let $(N, \in)$ be a transitive model of ZFC and Let $(M, U, \kappa)$ be initial data such that
$$M \models \text{$V_{\kappa + 1}$ exists},$$
and $U \in M$. Assume $\omega_1 \subseteq N$, and
$$N \models \text{$U$ is an $M$-ultrafilter on $\kappa$.}$$
Then $(M, U)$ is iterable iff
$$N \models \text{$(M, U)$ is iterable.}$$
\end{corollary}
\begin{proof}
It follows by the above corollary, since $N$ gets $\omega_1$ correct.
\end{proof}
The point of the above results is that $\omega_1$-iterability is the same as $\Ord$-iterability, and this is because we are mainly interested in countably complete ultrafilters (since these already give measurable cardinals and well-foundedness).

\section{Mice}
\begin{definition}[Jensen]
Let $\kappa$ be a regular cardinal.
A \dfn{premouse} at $\kappa$ is a pair $(M, U)$ such that $M = (M, \in)$ is a transitive class such that:
\begin{enumerate}
\item $M \models $ ZFC minus powerset,
\item $M \models V_{\kappa + 1}$ exists,
\item $M \models V = L[U]$, and
\item $M \models U$ is a normal uniform ultrafilter on $\kappa$.
\end{enumerate}
A \dfn{mouse} is an iterable premouse.
\end{definition}
Since $M \models V = L[U]$ and is a transitive class, $M = L[U]$ or $M = L_\lambda[U]$ for some limit ordinal $\lambda > \kappa$.
Therefore we abuse notation $L_\Ord[U] = L[U]$ (so $\Ord$ is an ``uncountable limit ordinal").
\begin{lemma}
Let $\kappa$ be a regular cardinal, $\lambda$ a limit ordinal or $\lambda = \Ord$, and $L[U]$ a $\kappa$-model. Then $(L_\lambda[U], U)$ is a premouse at $\kappa$. If $\lambda$ is uncountable, then $L_\lambda[U]$ is actually a mouse.
\end{lemma}
\begin{proof}
Most of the definition of a premouse follows straight from the definitions. Since $L[U]$ is actually a model of ZFC, in particular it thinks that the power set of $V_\kappa$ exists.

If $\lambda$ is a limit ordinal, then $U \cap L_\lambda[U]$ is countably complete, so weakly countably complete. Therefore $L_\lambda[U]$ is iterable, and hence a mouse.
\end{proof}

Let $\nu$ be an uncountable regular cardinal, so its club filter $C_\nu$ is a nontrivial $\nu$-complete filter.
It is not necessarily true that $C_\nu$ is actually an ultrafilter, so this does not imply that $\nu$ is measurable, but sometimes a model will get this wrong and think that club filters are witnesses to measurability.
This is the case when we iterate a mouse at $\kappa$ more than $\kappa^\kappa$ times.
\begin{lemma}
Suppose $(M, U)$ is a mouse at $\kappa$ and $\nu > (\kappa^\kappa)^M$ is a regular cardinal. Then there is a $\lambda$ such that
$$(M_\nu, U_\nu) = (L_\lambda[C_\nu], C_\nu \cap L_\lambda[C_\nu]).$$
\end{lemma}
\begin{proof}
We have $\kappa_\nu = \nu$ and $M_\nu = L_\lambda[U_\nu]$ by previous results. This is where we use $\nu > (\kappa^\kappa)^M$.

Since $U$ is a normal uniform ultrafilter, so is $U_\nu$, so for every $x \in U_\nu$, $\card x = \kappa_\nu = \nu$.
By a previous lemma, there is an $\alpha_x < \nu$ such that
$$f(x) = \{\kappa_\gamma: \alpha_x < \gamma < \nu\} \subseteq x.$$

We claim that $f(x)$ is a club in $\nu$. Since $\nu$ is a limit ordinal,
$$\nu = \sup_{\gamma < \nu} \kappa_\gamma = \sup f(x),$$
so $f(x)$ is cofinal.
If $\beta$ is a limit point of $f(x)$ in $\nu$, let
$$\mu = \sup \{\gamma: \kappa_\gamma < \beta\}.$$
Since $\beta$ is a limit ordinal, $\kappa_\mu = \beta$. This verifies the claim.

Therefore $x$ contains a club; but $x$ was arbitrary, so $U_\nu \subseteq C_\nu \cap L_\lambda[C_\nu]$.
Since $U_\nu$ is maximal, the claim follows.
\end{proof}
\begin{corollary}
Suppose there is a $\kappa$-model, and $\eta$ is a sufficiently large regular cardinal. Then $(L_\eta[C_\eta], C_\eta \cap L_\eta[C_\eta])$ is an $\eta$-model.
\end{corollary}
\begin{proof}
Let $L[U]$ be a $\kappa$-model. Then $L[U]$ is a mouse, so is $(\eta + 1)$-iterable. By the previous lemma, if $\eta > (\kappa^\kappa)^{L[U]}$, then
$$(L[U]_\eta, U_\eta) = (L[C_\eta], C_\eta \cap L[C_\eta]),$$
Since $L[U]_\eta$ is a mouse and $U_\eta$ is a normal uniform ultrafilter on $\kappa_\eta = \eta$,
$$L[U]_\eta \models \text{$\eta$ is an inaccessible cardinal}$$
so we may replace $L[C_\eta]$ with $L_\eta[C_\eta]$ in the above equality.
\end{proof}
Here we are not assuming that $\kappa$ is measurable.

The reason that mice are of interest is that they are models of set theory with large cardinals that can be sorted into a nice linear ordering by letting them iterate.
This is not true for large cardinals that are even bigger than measurable cardinals, because the mice for their inner models actually sort into trees or other more complicated combinatorial structures.
\begin{definition}
Suppose $(M, U)$ and $(M', U')$ are premice at $\kappa$ and $\kappa'$ respectively. We say that $(M, U) \leq (M', U')$ if:
\begin{enumerate}
\item $\kappa = \kappa'$.
\item $U = U' \cap M$.
\item There is an $\alpha \in \Ord^M$ such that $M = L_\alpha[U']$.
\end{enumerate}
\end{definition}
Notice that the first condition $\kappa = \kappa'$ means that the mice are comparable. The second and third conditions basically say that $(M, U)$ is a submouse of $(M', U')$.
\begin{theorem}[comparison]
\index{mouse comparison theorem}
Let $(M, U, \kappa)$ and $(M', U', \kappa')$ are mice. Let
$$\nu > \card(\kappa^\kappa \cap M) + \card((\kappa')^{\kappa'} \cap M')$$
be a regular cardinal.
Then $(M_\nu, U_\nu)$ and $(M_\nu', U_\nu')$ are comparable mice.
\end{theorem}
\begin{proof}
We have
$$(M_\nu, U_\nu) = (L_\alpha[C_\nu], C_\nu \cap L_\alpha[C_\nu])$$
and similarly for $(M_\nu', U_\nu')$. In particular both are mice at $\nu$.
If $\alpha \leq \alpha'$ then $(M_\nu, U_\nu) \leq (M_\nu', U_\nu')$.
\end{proof}

\section{Applications of mice}
In this section we use mice to prove two very powerful theorems about $L[U]$.
\begin{theorem}
Let $\kappa$ be a measurable cardinal and $L[U]$ a $\kappa$-model. Then $L[U]$ satisfies the generalized continuum hypothesis.
\end{theorem}
\begin{theorem}
Let $\kappa$ be a measurable cardinal and $L[U]$ a $\kappa$-model. Then $L[U]$ thinks $\RR$ has a $\Sigma_3$-definable well-ordering.
\end{theorem}
So $L[U]$ looks a lot like $L$! In fact, $L$ also satisfied GCH and thought that $\RR$ had a $\Delta_2$-definable well-ordering.
For the rest of our adventures with large cardinals, we will mainly be looking for inner models of $V$ which generalize $L[U]$ to satisfy stronger large cardinal axioms than the relatively weak axiom ``There is one measurable cardinal."

If $M$ is a model of global choice, let $<_M$ denote the given well-ordering on $M$.
In particular, this will happen when $M$ is a $\kappa$-model.

\begin{lemma}
Let $\kappa$ be a measurable cardinal, $V = L[U]$, and let $(M^1,U_1),(M^2,U_2)$ be mice at $\kappa$.
Let $\lambda$ be a cardinal such that $\crt U_1 > \lambda$ and $\crt U_2 > \lambda$. Then either $((2^\lambda)^{M_1}, <_{M_1})$ is an initial segment of $((2^\lambda)^{M_2}, <_{M_2})$ or vice versa.
\end{lemma}
\begin{proof}
There is a regular cardinal $\theta$ such that $M_\theta^1$ and $M_\theta^2$ are comparable mice. Without loss of generality, assume that $M_\theta^1 \leq M_\theta^2$.
By assumption on $\lambda$, if $x \subseteq \lambda$ and $x \in M^1 \cap M^2$ then $x$ is a fixed point of the elementary embeddings $j_i: M^i \to M_\theta^i$.
Therefore $<_{M_i}$ is an initial segment of $<_{M_i^\theta}$, and since $M_\theta^1 \leq M_\theta^2$, $<_{M_1^\theta}$ is an initial segment of $<_{M_2^\theta}$.
So the $<_{M_i}$ are initial segments of $<_{M_2^\theta}$ and hence one must be an initial segment of each other.
\end{proof}

\begin{proof}[Proof of GCH]
Assume $V = L[U]$. By Solovay's theorem, $\lambda^+ = 2^\lambda$ when $\lambda \geq \kappa$.

Let $\lambda < \kappa$ and $x \in 2^\lambda$, so by Solovay's theorem, $\lambda$ is not a measurable cardinal. Let $\gamma$ be a limit ordinal such that $x, U \in L_\gamma[U]$, $L_\gamma[U]$ thinks $V_{\kappa+1}$ exists, and $L_\gamma[U]$ models ZFC minus powerset.
By reflection, there is an elementary substructure $(H, U \cap H)$ of $L_\gamma[U]$ such that $x, U \in H$ and $\card H = \lambda$.
Then $(H, U \cap H) \models V = L[U]$, so there is a limit ordinal $\delta$ such that the transitive collapse of $(H, U \cap H)$ is $L_\delta[U']$
for some ultrafilter $U'$.

Since $\kappa$ is measurable, $\delta$ is uncountable, so $M = L_\delta[U']$ is a mouse, and $\card M = \lambda$.
Besides, $x \in M$, and since $\lambda$ is not a measurable cardinal, $\crt U' > \lambda$.
Since $V$ is also a mouse, the lemma implies that $<_M$ is an initial segment of $<_V$.
So
$$\card \{y \in 2^\lambda: y <_V x\} = \card \{y \in 2^\lambda: y <_M x\} \leq \card M = \lambda.$$

Therefore any initial segment of $(2^\lambda, <_V)$ has cardinality $\lambda$. But this implies that $\card 2^\lambda \leq \lambda^+$.
Cantor's theorem gives the other direction.
\end{proof}

\begin{proof}[Proof of $\Sigma_3$-well-ordering]
Let $V = L[U]$. We take $\lambda = \aleph_0$ in the above lemma.
Then $x <_V y$ iff there is ($\exists$) a countable premouse $M$ such that for every ($\forall$) countable ordinal $\alpha$, $M$ is $\alpha$-iterable, and $x <_M y$.
Being $\alpha$-iterable is a $\Pi_2$-condition: for all $\beta < \alpha$ and every $\in$-chain $C$ in $M_\beta$, there is a $n < \omega$ such that $\card C = n$.
Moreover a formula of the form $\exists \forall \Pi_2$ is $\Sigma_3$.
\end{proof}

\chapter{Extender models}
Measurable cardinals are nice, but as we saw above, they have some serious deficiencies.
Putting aside for now the issue that the canonical model $L[U]$ could only hold a single measurable cardinal, let's see what else is ``wrong" with them.

\begin{definition}
A regular cardinal $\kappa$ is an \dfn{$\alpha$-strong cardinal} if there is an elementary embedding $j: V \to M$ such that $\crt j = \kappa$, $j(\kappa) > \alpha$ and $V_\alpha \subseteq M$.
A \dfn{strong cardinal} is one which is $\alpha$-strong for every $\alpha \in \Ord$.
\end{definition}
Therefore $\kappa$ is measurable iff $\kappa$ is $\kappa+1$-strong. But not every measurable cardinal is $\kappa+2$-strong, since if $U$ was the ultrafilter
$$U = \{x \subseteq \kappa: \kappa \in j(x)\}$$
where $j$ witnessed measurability of $\kappa$, then $U \notin V_{\kappa+2}^M$.

\begin{definition}
A regular cardinal $\kappa$ is a \dfn{$\lambda$-supercompact cardinal} if there is an elementary embedding $j: V \to M$ such that $j(\kappa) > \lambda$ and $M^\lambda \subseteq M$.
A \dfn{supercompact cardinal} is one which is $\lambda$-supercompact for every $\lambda \in \Card$.
\end{definition}
Again we see that a measurable cardinal is $\kappa$-supercompact but in general is no better, since if $j$ is a witness to the measurability of $\kappa$, then $j|_{\kappa^+} \notin M$, so that $\kappa$ may not be $\kappa^+$-supercompact.

This is frustrating: we had a clean definition of measurability in terms of combinatorics (or ``measure theory") and one would hope that we could define supercompactness similarly.
But clearly ultrapowers are not enough to define supercompactness, so we need some other formulation.

\section{The diagram chase}
Motivated by the above, we introduce another way to construct elementary embeddings from large cardinals, which will hopefully allow us to come up with a better definition of supercompactness.

\begin{definition}
Let $X$ be a class. The \dfn{Skolem hull} $\Hull X$ of $X$ is the intersection of all classes $Y$ such that $Y$ is an elementary substructure of $V$ and $X \subseteq Y$.
\end{definition}

Let $j: V \to M$ be a nontrivial elementary embedding, $\crt j = \kappa$, and
$$U = \{x \subseteq \kappa: \kappa \in j(x)\}$$
the normal uniform ultrafilter given by $j$. Let $i$ be the ultrapower map of $U$. Then we have an elementary diagram
$$\begin{tikzcd}V \arrow[rr, "j"] \arrow[dr, "i"]&& M\\
&\Ult(V, U)\arrow[ur, "\psi"]\end{tikzcd}$$
where the map $\psi$ satisfies
$$\psi([f]) = j(f)(\kappa).$$
Moreover,
$$\psi"(\Ult(V, U)) = \Hull^M(j"(V) \cup \{\kappa\})$$
where as usual,
$$f"(x) = \{f(y): y \in x\}.$$
So, given an elementary embedding $j$, we factored $j$ into maps $i,\psi$ where $i$ is an ultrapower embedding and $\psi$ embeds the ultrapower given by $i$ as an appropriate Skolem hull.
This is basically just the first isomorphism theorem: any ultrapower must contain $j"V$ and $\{\kappa\}$, but we could also embed this ultrapower in a much bigger structure.

Suppose that we want to run the above process but with the critical point $\kappa$ replaced by a possibly much larger cardinal $\lambda$, which would witness the $\delta$-supercompactness of $\kappa$ for any $\delta < \lambda$.
Indeed, if $\kappa$ is $\delta$-supercompact, we can find a cardinal $\lambda > \delta$ and an elementary embedding $j: V \to M$ such that $\crt j = \kappa$ and $j(\kappa) = \lambda$. Let $H = \Hull^M(j"V \cup \{\lambda\})$ and let $N$ be the transitive collapse of $H$.
Then we have an elementary embedding $\psi: N \to M$ such that $\psi"N = H$, and since for every $\alpha \leq \lambda$, $\alpha \in H$, it must be that $\crt \psi > \lambda$.
Then we can find an elementary embedding $i: V \to N$ which makes the elementary diagram
$$\begin{tikzcd}V \arrow[rr, "j"] \arrow[dr, "i"]&& M\\
&N\arrow[ur, "\psi"]\end{tikzcd}$$
commute: just take $i = j \circ \psi^{-1}$; since $\psi"N \subseteq j"V$, this is well-defined.
If we can show that $i$ is an ultrapower embedding, we have accomplished our goal.
In fact, we will show that $i$ is the direct limit of a $\lambda$-sequence of ultrapower embeddings.

\begin{definition}
Let $j: V \to M$ be an elementary embedding, $\crt j = \kappa$, and $\lambda > \kappa$ regular cardinals. We say that an elementary embedding $i: V \to N$ is the \dfn{$(\kappa, \lambda)$-derived extender} for $j$ if:
\begin{enumerate}
\item $N$ is the transitive collapse of $H = \Hull^M(j"V\cup\{\lambda\})$.
\item The elementary embedding $\psi: N \to M$ such that $\psi^{-1}$ is the canonical isomorphism $H \to N$ makes the elementary diagram
$$\begin{tikzcd}
V \arrow[rr,"j"] \arrow[dr,"i"] && M\\
& N \arrow[ur,"\psi"]\end{tikzcd}$$
commute.
\end{enumerate}
\end{definition}
Fix $\lambda$ and suppose we have an elementary embedding $j: V \to M$.
We want to modify $j$ to an ultrapower embedding $i: V \to N$ so that $\crt i \geq \lambda$, such that $i$ is the $(\kappa, \lambda)$-derived extender of $j$.

To this end, let $N$ be the transitive collapse of a Skolem hull $H$ in $M$ which contains all of $\lambda$ as well as $j"V$.
For every $\alpha < \lambda,$ let
$$\kappa_\alpha = \min_{j(\beta) \geq \alpha} \beta.$$
Then consider the $\kappa_\alpha$-complete ultrafilter
$$E_\alpha = \{x \subseteq \kappa_\alpha: \alpha \in j(x)\}.$$
Indeed, $E_\alpha$ is pushed forward by $j$ to a trivial ultrafilter.
So $\Ult(V, E_\alpha)$ is well-founded and we obtain an elementary diagram:
$$\begin{tikzcd}V \arrow[r,"j"] \arrow[d] & M\\
\Ult(V, E_\alpha) \arrow[ur,"j_\alpha"] \arrow[r, "k_\alpha"]& N \arrow[u,"\psi"]
\end{tikzcd}$$
Here $\psi^{-1}$ is the transitive collapse $H \to N$. Since $\alpha < \lambda$, $j_\alpha"(\Ult(V, E_\alpha)) \subseteq j"(V) \subseteq H = \psi"N$. Therefore $k_\alpha$ exists and the diagram commutes.

We want to show that $N$ is the direct limit of some infinite elementary diagram containing the $\Ult(V, E_\alpha)$.

It will be convenient to introduce some notation.
\begin{definition}
Let $x$ be a set and $\beta$ a cardinal. We define
$$\binom x\beta = \{y \in 2^x: \card y = \beta\}.$$
\end{definition}
If $n < \omega$, $a \subset \lambda$, and $\card a = n$, we put
$$E_a = \{x \in \binom{\kappa_a}{n}: a \in x\}.$$
Here
$$\kappa_a = \sum_{\alpha \in a} \kappa_\alpha.$$
Then we have a natural elementary embedding $\Ult(V, E_a) \to \Ult(V, E_b)$ whenever $a \subseteq b$:
if $[f]_{E_a}$, then
$$\dom f \subseteq \binom{\lambda}{\card a},$$
i.e. $f$ takes in $\card a$ arguments. We can therefore define $f': \binom{\lambda}{\card b} \to V$ by
$$f'(x_1, \dots, x_{\card a}, \dots, x_{\card b}) = f(x_1, \dots, x_{\card a}).$$
Then $[f']_{E_b} \in \Ult(V, E_b)$, so the map $[f]_{E_a} \mapsto [f']_{E_b}$ is the elementary embedding we wanted.
Of course, we have $E_\alpha = E_{\{\alpha\}}$, so the directed system of all $E_a$, $a \in \binom{\lambda}{<\omega}$, includes the $E_\alpha$ we constructed already.
Taking the limit $N'$ of the elementary diagram consisting of all $\Ult(V, E_a)$, and recalling that we had elementary embeddings $\Ult(V, E_a) \to N$, we obtain an elementary diagram
$$\begin{tikzcd}
V \arrow[r,"j"] \arrow[d] & M\\
\Ult(V, E_a) \arrow[d] & N \arrow[u] \\
\Ult(V, E_b) \arrow[r] & N' \arrow[u]
\end{tikzcd}$$
whenever $a \subseteq b$. But $N$ was the transitive collapse of $\Hull^M(j"V \cup \{\lambda\})$, so necessarily must be minimal possible. This implies $N' = N$.
Therefore we introduce the notation $N = \Ult(V, E)$ where
$$\Ult(V, E) = \left\{[f, a]: \dom f = \binom{\kappa_a}{\card a}, ~a \subset \lambda, ~\card a < \aleph_0.\right\}.$$
Here $[f, a] = [g, b]$ if there is a $c \supseteq a \cup b$ such that $[f, a]_{E_c} = [g, b]_{E_c}$.
In particular, there is a canonical embedding $V \to \Ult(V, E)$ given by $x \mapsto [c_x, \emptyset]$.

\section{The abstract definition}
We now give an abstract definition of extenders which do not require us to have been given an elementary embedding a priori.

Whenever we have a directed system $X = (X_i: i \in I)$ and an embedding $X_i \to X_j$, $x \in X_i$, we will write $x_i^j$ to mean the lift of $x$ to $X_j$.
\begin{definition}
Let $\kappa,\lambda$ be cardinals. A \dfn{$(\kappa, \lambda)$-extender} is a $\lambda$-sequence
$$E = (E_a: a \in \binom{\lambda}{<\omega}),$$
such that:
\begin{enumerate}
\item For every $a$, let
$$\kappa_a = \min\left\{\kappa \in \Card: \binom{\kappa}{\card a} \in E_a\right\}.$$
Then $E_a$ is a $\kappa$-complete ultrafilter on $\binom{\kappa_a}{\card a}$.
\item There are inclusion maps $E_a \to E_b$ induced whenever $a \subseteq b$, so that $E$ is a directed system.
\item If there are $a \in \binom{\lambda}{<\omega}$, $\dom f = \binom{\kappa_a}{\card a}$, and $i < \card a$, such that for $E_a$-almost every $x \in \binom{\kappa_a}{\card a}$, $f(x) < x_i$,
then there is a $\beta < a_i$ such that if $b = a \cup \{\beta\}$ then for $E_b$-almost every $x \in \binom{\kappa_b}{\card b}$, $f_a^b(x) = x_a^b$.
\item The direct limit of the elementary diagram $(\Ult(V, E_a): a \in \binom{\lambda}{<\omega})$ is well-founded.
\end{enumerate}
The transitive collapse of the direct limit of $(\Ult(V, E_a): a \in \binom{\lambda}{<\omega})$ is denoted $\Ult(V, E)$, and we let $j_E$ denote the canonical elementary embedding $V \to \Ult(V, E)$.
\end{definition}
From this definition, we end up with a directed system of ultrapowers $\Ult(V, E_a)$.
Note that condition (4), and hence the definition of $\Ult(V, E)$ doesn't a priori make sense; we have to check that the conditions before it actually give us an elementary diagram:
\begin{lemma}
Let $E$ be a $(\kappa, \lambda)$-extender. Then the $\Ult(V, E_a)$, as $a$ ranges over $\binom{\lambda}{<\omega}$, form an elementary diagram.
\end{lemma}
\begin{proof}
Fix $a \subseteq b$, $\theta$ a formula, and $f \in V^{\kappa_a}$; then $\Ult(V, E_a) \models \theta([f])$ iff $E_a$-almost every $x \in \kappa_a$ satisfies $\theta(f(x))$; but this happens iff $E_b$-almost every $y \in \kappa_b$ satisfies $\theta(f_a^b(y))$.
This happens iff $\Ult(V, E_b) \models \theta([f_a^b])$.
\end{proof}
Therefore, by the Mostowski collapse lemma, $\Ult(V, E)$ is well-defined.

Obviously conditions (1), (2), and (4) hold for a derived extender. Meanwhile, condition (3) holds for $a,f,i$ exactly if
$$a \in j\left\{x \in \binom{\kappa_a}{\card a}: f(x) < x_i\right\} = \left\{x \in \binom{j(\kappa_a)}{\card a}: j(f)(x) < x_i\right\},$$
so let $\beta = j(f)(a)$.
Therefore every derived extender is actually an extender. We now check the converse:
\begin{theorem}
Assume that $E$ is a $(\kappa, \lambda)$-extender. Then $E$ is the $(\kappa, \lambda)$-derived extender of $j_E$.
\end{theorem}
\begin{proof}
We must show that $x \in E_a$ iff $a \in j_E(x)$. Actually, it suffices to show $[\id, a] = a$: if this true, then $a \in j_E(x)$ iff $[\id, a] \in j_E(x)$ iff
$$x = \left\{y \in \binom{\kappa_a}{\card a}: y \in c_x(\emptyset)\right\} = E_{a \cup \{\emptyset\}}$$
iff $x \in E_a$.

\begin{lemma}
For every $\alpha < \lambda$,
$$\left[\bigcup, \{\alpha\}\right] = \alpha.$$
\end{lemma}
\begin{proof}
By transfinite induction. Assume this is true for $\beta < \alpha$; we must show that every element of $x = \left[\bigcup, \{\alpha\}\right]$ is an ordinal $<\alpha$.
Assume $[f, \alpha] \in x$ and $a' = a \cup \{\alpha\}$. Then
$$E_{a'} \ni \left\{x \in \binom{\kappa_{a'}}{\card a'}: f_a^{a'}(x) \in x_a^{a'}\right\}.$$
By (3) in the definition, there is a $\beta < \alpha$ such that if $b = a' \cup \{\beta\}$ then
$$E_b \ni \left\{ x \in \binom{\kappa_b}{\card b}: f_a^b(x) = x_\beta^b\right\}.$$
Setting $b' = a \cup \{\beta\}$, by the definition of a directed system,
$$E_{b'} \ni \left\{x \in \binom{\kappa_{b'}}{\card b'}: f_a^{b'}(x) = x_\beta^{b'}\right\}.$$
Therefore $[f, a] = \left[\bigcup, \beta\right]$, so by induction $[f, a] = \beta$.
\end{proof}

We now show that $[\id, a] = a$. Suppose $[f, b] \in [\id, a]$. Then
$$E_{a \cup b} \ni \left\{x \in \binom{\kappa_{a \cup b}}{\card a \cup b}: f_b^{a \cup b}(x) \in x_a^{a \cup b}\right\},$$
and as $x_a^{a \cup b}$ is a finite set, the infinite pigeonhole principle gives an $\alpha \in a$ such that
$$E_{a \cup b} \ni \left\{x \in \binom{\kappa_{a \cup b}}{\card a \cup b}: f_b^{a \cup b}(x) \in \bigcup x_{\{\alpha\}}^{a \cup b}\right\},$$
so $[f, b] = \alpha$.
\end{proof}

Summarizing, given an elementary embedding $j: V \to M$ we can find an extender $E$ such that the elementary diagram
$$\begin{tikzcd}
V \arrow[rr,"j"] \arrow[dr,"j_E"] && M\\
& \Ult(V, E) \arrow[ur,"k_E"]
\end{tikzcd}$$
commutes, where $k_E[a, f] = j(f)(a)$ and $j_E(f)(a) = j(f)(a)$ for any function $f$ with $\dom f = \binom{\kappa_a}{\card a}$, and $a \subset \lambda$ finite.
On any finite subset of $\lambda$, $k_E$ restricts to the identity, and if $j(\gamma) \leq \lambda$ then $j_E = j$ when restricted to $\gamma$.

\begin{lemma}
Let $M$ be a transitive, well-founded model of ZFC. Then $\Aut(M)$, the group of isomorphisms $M \to M$, is trivial.
\end{lemma}
\begin{proof}
Let $j: M \to M$ be an automorphism of $M$. We show that $j|V_\alpha^M$ is the identity for every $\alpha$, by transfinite induction.

Clearly this is true if $\alpha = 0$, and at limit stages. Assume $j|V_\alpha^M = \id$, and let $x \in V_{\alpha+1}^M = \pset^M(V_\alpha^M)$.
Then $y \in x$ iff $j(y) \in j(x)$ since $j$ is elementary, and $j$ does not add any more elements to $x$ since it is surjective.
But $j(y) = y$ since $y \in V_\alpha^M$, so this implies $j(x) = x$.
\end{proof}

\begin{theorem}
Let $j: V \to M$ be an elementary embedding and $E$ its derived extender. Let $\gamma \leq \lambda$, $\lambda$ a limit ordinal such that
$$\lambda \geq \card^M V_\gamma^M.$$
Then $V_\gamma^{\Ult(V, E)} = V_\gamma$ and $k_E|V_\gamma^{\Ult(V, E)}$ is the identity.
In particular, $\crt j$ is $\gamma$-strong.
\end{theorem}
\begin{proof}
Let $\nu = \card^{\Ult(V, E)} V_\gamma^{\Ult(V, E)}$. Then $k_E(\nu) = \card^M V_\gamma^M$, so
$$\nu \leq k_E(\nu) \leq \lambda.$$
If $\nu < k_E(\nu)$ then $\nu < \lambda$, but then $k_E$ is the identity on $\{\nu\} \subset \lambda$, a contradiction.
So $\nu$ is a fixed point of $E$.

Now let $\{X_\xi\}_{\xi < \nu}$ be a surjective $\nu$-sequence of sets in $V_\gamma^{\Ult(V, E)}$, in $\Ult(V, E)$.
Then $\{k_E(X_\xi)\}_{\xi < \nu}$ is a surjective $\nu$-sequence of sets in $V_\gamma^M$, so $k_E$ is surjective.
Since $k_E$ is elementary, it follows that $k_E$ is an isomorphism $V_\gamma^{\Ult(V, E)} \to V_\gamma^M$.

But the automorphism group of a well-founded, transitive model is trivial, so this implies that $k_E$ is a unique isomorphism, i.e. the identity.
\end{proof}

\begin{theorem}
Let $E$ be a $(\kappa, \lambda)$-extender, $j_E"\gamma \in \Ult(V, E)$, and $\lambda^\gamma \in \Ult(V, E)$. Then $\crt j_E$ is $\gamma$-supercompact, as witnessed by $\Ult(V, E)$.
\end{theorem}
We omit the proof; it is extremely ugly.

\begin{definition}
Let $E$ be a $(\kappa, \lambda)$-extender. Then the \dfn{critical point} $\crt E$ is $\kappa$ and the \dfn{length} of $E$ is $\lambda$.
\end{definition}
\begin{definition}
Let $E$ be a $(\kappa, \lambda)$-extender. The \dfn{support} of $E$ is
$$\supp E = \sup_{a \subset \lambda} \kappa_a,$$
the \dfn{strength} of $E$ is the largest $\gamma$ such that $V_\gamma \in \Ult(V, E)$, and the \dfn{closure} of $E$ is the smallest $\card \gamma$ such that $\Ult(V, E)^\gamma$ is not contained in $\Ult(V, E)$.
\end{definition}

We now show that extenders allow us to give a first-order definition of ``supercompact" and ``strong."
This is important because the notion of an elementary embedding of $V$ cannot be made rigorous in ZFC; it is a second-order, or class-theoretic, concept.

\begin{theorem}
Let $\gamma > \kappa$. Then $\kappa$ is $\gamma$-strong iff there is a $\lambda$ and a $(\kappa, \lambda)$-extender $E$ such that $j_E(\kappa) > \gamma$ and $E$ has strength $\geq \gamma$.
\end{theorem}
\begin{proof}
If such an extender exists then it witnesses strength. Otherwise, assume $\kappa$ is $\gamma$-strong, witnessed by $j: V \to M$.
Let $\lambda > \gamma + \card^M V_\gamma$. Then there is a $(\kappa, \lambda)$-extender which satisfies the constraints.
\end{proof}

\begin{theorem}
Let $\delta > \kappa$. Then $\kappa$ is $\delta$-supercompact iff there is a $\lambda$ and a $(\kappa, \lambda)$-extender $E$ such that $j_E(\kappa) > \delta$, and $E$ has closure $> \delta$.
\end{theorem}
\begin{proof}
One direction is trivial, so assume that $\kappa$ is $\delta$-supercompact, and $j: V \to M$ is a witness to the $\delta$-supercompactness of $\kappa$.

Let $\lambda > \delta$ be such that $\lambda = \card V_\lambda$. Since $\mu \mapsto \card V_\mu$ is a normal function, its set of fixed points is cofinal, so $\lambda$ exists.
Taking $\lambda$ to be a regular cardinal, we may assume $\cof \lambda > \delta$.
Let $E$ be the $(\kappa, j(\lambda))$-extender derived from $j$.
Since derived extenders agree,
$$V_{j(\lambda)}^{\Ult(V, E)} = V^M_{j(\lambda)},$$
and in particular,
$$j_E"\lambda \in V^M_{j(\lambda) + 1} \subset \Ult(V, E).$$

It remains to show that $j(\lambda)^\delta$ is contained in $\Ult(V, E)$.
Let $f$ be a sequence in $j(\lambda)^\delta$; since $M$ is a witness to the $\delta$-supercompactness of $\kappa$, $f \in M$.
But since $\cof \lambda > \delta$,
$$\cof^M j(\lambda) = j(\cof \lambda) > j(\delta) \geq \delta.$$
Therefore $f$ cannot be cofinal, so $V_\gamma^M$ for some $\gamma < j(\lambda)$. Moreover, $V_\gamma^M = V_\gamma^{\Ult(V, E)}$, so $f \in \Ult(V, E)$.
\end{proof}

Expanding out the definitions of the above properties of an extender, we see that the above theorem implies that $\gamma$-strength, or $\gamma$-supercompactness, is a $\Sigma_2$ definition.
In particular, strength and supercompactness are $\Pi_3$ definitions.

\begin{corollary}
Suppose that $\kappa$ is $(\kappa + 2)$-strong. Then there is a nontrivial uniform ultrafilter $U$ on $\kappa$ such that $U$-almost every element of $\kappa$ is measurable.
\end{corollary}
\begin{proof}
Let $j: V \to \Ult(V, E)$ witness that $\kappa$ is $(\kappa + 2)$-strong, and let
$$U = \{x \subseteq \kappa: \kappa \in j(x)\}$$
be the derived ultrafilter of $j$. Let $x$ be the set of all measurable cardinals $< \kappa$; since $\kappa$ is measurable and $\kappa < j(\kappa)$, $\kappa \in j(x)$.
Therefore $x \in U$.
\end{proof}

The conga line never ends! Strong cardinals humiliate measurable cardinals just as measurable cardinals humiliate Mahlo cardinals, and if there is just one strong cardinal, then $L[U]$ looks nothing at all like $V$.

\section{Reinhardt cardinals}
Recall that we introduced the notions of strong and supercompact cardinals to assert the existence of elementary embeddings $j: V \to M$ such that $M$ was ``not too much bigger" than $V$. Why don't we just cut to the chase and assume that in fact $V = M$?

\begin{definition}
A \dfn{Reinhardt cardinal} is the critical point of an elementary embedding $V \to V$.
\end{definition}

\begin{definition}
A \dfn{Berkeley cardinal} $\kappa$ is a regular cardinal such that for every transitive set $M \ni \kappa$, there is an elementary embedding $j: M \to M$ such that $\crt j = \kappa$.
\end{definition}

Unfortunately, the above notions are inconsistent with ZFC. Actually, Woodin introduced the notion of a Berkeley cardinal as a joke, something that was obviously inconsistent with not just ZFC, but even ZF. But decades later, it is still an open problem to show that Berkeley cardinals are inconsistent with ZF.

\begin{theorem}
There is no Reinhardt cardinal.
\end{theorem}
\begin{proof}
We will use Theorem \ref{solovay stationary theorem}, a form of the axiom of choice due to Solovay.

Let $j: V \to V$ be an elementary embedding and $\kappa_0 = \crt j$.
Let $\kappa_{n+1} = j(\kappa_n)$, and let
$$\lambda = \sup_{n < \omega} \kappa_n.$$
Therefore
$$j(\lambda) = j(\sup_n \kappa_n) = \sup_n j(\kappa_n) = \lambda.$$
So $j(\lambda^+) = \lambda^+$.

Let $S = \{\alpha < \lambda^+: \cof \alpha = \omega\}$.
If $c$ is a club in $\lambda^+$, then $c \cap \Card$ is also a club since $\Card \cap \lambda^+$ is a club, so assume that $c$ only consists of cardinals.
Then let $(\delta_n: n < \omega)$ be an increasing $\omega$-sequence in $c$. By construction $\delta = \sum_n \delta_n$ satisfies $\delta > \delta_n$ for any $n$,
so $\cof \delta = \omega$. Therefore $\delta \in S \cap c$. So $S$ is stationary.
Moreover,
$$j(S) = \{\alpha < j(\lambda^+): \cof \alpha = j(\omega)\} = S$$
since $\lambda^+,\omega$ are fixed points of $j$.

By the Solovay partitioning theorem, there is a partition $S = \bigcup_\alpha S_\alpha$ of $S$ into stationary sets of cardinality $\lambda^+$.
Then let $(T_\alpha: \alpha < \omega) = j(S_\alpha: \alpha < \omega)$, so $j(S_\alpha) = T_{j(\alpha)}$.
Therefore $j(S_\lambda) = T_\lambda$.
Since $j$ is elementary, the $T_\alpha$ form a partition of $S$ into stationary sets of cardinality $\lambda^+$.

Let $F$ be the set of fixed points of $j$ in $\lambda^+$. Then $F$ is closed under $\omega$-limits, and contains the $\kappa_n$, so is cofinal.
Now $T_{\kappa_0}$ is stationary, so in particular it meets the closure $\overline F$ of $F$, say at $\eta$, and by definition of $T_{\kappa_0}$, $\cof \eta = \omega$.
Therefore $\eta$ is an $\omega$-limit of $F$, so $\eta \in F$.
But then $\eta = j(\eta) \in j(S_\gamma) = T_{j(\gamma)}$ for some $\gamma$.
Therefore $\kappa_0 = j(\gamma)$, so $\gamma < \kappa_0$ is not a fixed point of $j$, but $\kappa_0$ is the least ordinal moved by $j$.
\end{proof}

\section{Supercompact cardinals}
Recall that a cardinal $\kappa$ is supercompact if for every $\delta > \kappa$ there is a $\lambda$ and a $(\kappa, \lambda)$-extender $E$ such that $E$ has closure $> \delta$ and $j_E(\kappa) > \delta$.
We want to show that this is equivalent to some definitions in terms of ultrafilters that will be easier to work with in practice.

\begin{definition}
Let $\kappa \leq \lambda$ be cardinals. Then let
$$\pset_\kappa(\lambda) = \{x \subseteq \lambda: \card x < \kappa\}.$$
\end{definition}
\begin{definition}
Let $U$ be an ultrafilter on $\pset_\kappa(\lambda)$.
We say that $U$ is a \dfn{fine ultrafilter} if for every $\alpha < \lambda$ and $U$-almost every $x \in \pset_\kappa(\lambda)$, $\alpha \in x$.
We say that $U$ is a \dfn{normal ultrafilter} if for every $f: \pset_\kappa(\lambda) \to \lambda$ such that for $U$-almost $x \in \pset_\kappa(\lambda)$, $f(x) \in x$, then $f$ is constant $U$-almost everywhere.
\end{definition}
So a fine ultrafilter does not privilege any element of $U$. Normal ultrafilters are similar to in the case of normal ultrafilters on measurable cardinals, which required that if a function pushed down on a large set, then it was actually constant on a large set.
\begin{lemma}
Let $\kappa \leq \lambda$ be cardinals. Then $\kappa$ is $\lambda$-supercompact iff there is a $\kappa$-complete normal fine ultrafilter on $\pset_\kappa(\lambda)$.
\end{lemma}
\begin{proof}
First assume $\kappa$ is $\lambda$-supercompact, and let $j: V \to M$ witness this, so that $\crt j = \kappa$, $j(\kappa) > \lambda$, and $M^\lambda \subseteq M$. Let
$$U = \{x \in \pset_\kappa(\lambda): j"\lambda \in j(x)\}$$
be the derived ultrafilter. Then $U$ is a $\kappa$-complete ultrafilter on $\pset_\kappa(\lambda)$, as in the proof of the Scott-Keisler theorem.

Since $M^\lambda \subseteq M$ and $j"\lambda = \{j(\alpha): \alpha < \lambda\}$ is a $\lambda$-sequence, $j"\lambda \in M$. Moreover,
$$\card j"\lambda = \lambda < j(\kappa).$$
If $\alpha < \kappa$ then $j(\alpha) \in j(\kappa) \subseteq j(\lambda)$ so $j"\kappa \subseteq j(\lambda)$.
So $j$ sends $\pset_\kappa(\lambda)$ into $\pset_{j(\kappa)}(j(\lambda))$.

To see that $U$ is fine, let $\alpha < \lambda$, so that $U$-almost every $x \in \pset_\kappa(\lambda)$ satisfies $\alpha \in x$ iff
$$j"\lambda \in j(\{x \in \pset_\kappa(\lambda): \alpha \in x\}) = \{x \in \pset_{j(\kappa)}(j(\lambda)): j(\alpha) \in x\},$$
i.e. $j(\alpha) \in j"\lambda$, which is clear.

To see that $U$ is normal, let $f: \pset_\kappa(\lambda) \to \lambda$ satisfy for $U$-almost every $x$, $f(x) \in x$.
Unraveling the definitions, this means that $j(f)(j"\lambda) \in j"\lambda$.
Thus there is an $\alpha < \lambda$ such that $j(f)(j"\lambda) = j(\alpha)$, and so if $x \in U$, $j"\lambda \in j(x)$ implies $f(x) = \alpha$.

For the converse, let $U$ be a $\kappa$-complete normal fine ultrafilter on $\pset_\kappa(\lambda)$ and $j: V \to \Ult(V, U)$ the ultrapower map.
\begin{lemma}
$j"\lambda = [\id]$.
\end{lemma}
\begin{proof}
We first prove $j"\lambda \subseteq [\id]$. If $[c_\alpha] = j(\alpha) \in j"\lambda$, then we must show $[c_\alpha] = [\id]$, which is true exactly when $\alpha \in x$ for $U$-almost every $x$, which happens because $U$ is finte.

Conversely, suppose $f \in [\id]$, i.e. for $U$-almost every $x$, $f(x) \in x$. By normality this means that for some $\alpha < \lambda$ and $U$-almost every $x$, $f(x) = \alpha$. Therefore $[f] \in j(\alpha) \in j"\lambda$.
\end{proof}

To see that $j(\kappa) > \lambda$, note that the map $h$ that sends a well-ordered set to its ordertype, and a set without a designated well-ordering to $\card$, is definable, so preserved by $j$.
Therefore
$$\lambda = h(j"\lambda) = h([\id]).$$
But for every $x \in \pset_\kappa(\lambda)$, $h(\id(x)) = h(x) < \kappa = c_\kappa(x)$
so $h([\id]) < h([c_\kappa]) = j(\kappa)$. Thus $\lambda < j(\kappa)$.
In particular, $\kappa < \lambda < j(\kappa)$ so $\crt j \leq \kappa$, but $U$ was $\kappa$-complete so $\crt j \geq \kappa$.

Finally we must show $\Ult(V, U)^\lambda \subseteq \Ult(V, U)$. Let $(f_\alpha: \alpha < \lambda)$ be given.
For any $g \in \pset_\kappa(\lambda)$ let
$$g(x) = (f_\alpha(x): \alpha < \lambda).$$
Then if $\alpha < \lambda$ and $x \in \pset_\kappa(\lambda)$, $c_\alpha(x) = \alpha$, so $g(x)(c_\alpha(x)) = f_\alpha(x)$.
This means that $[g](j(\alpha)) = f_\alpha$. Since $[g] \in \Ult(V, U)$ and $j"\lambda = [\id] \in \Ult(V, U)$,
$$(f_\alpha: \alpha < \lambda) = ([g](j(\alpha)): j(\alpha) \in j"\lambda) \in \Ult(V, U),$$
as desired.
\end{proof}

Summing up the above, we obtain the following corollary.
\begin{corollary}
Let $\kappa$ be a cardinal. Then the following are equivalent:
\begin{enumerate}
\item $\kappa$ is supercompact.
\item For every $\lambda \geq \kappa$ there is a normal fine ultrafilter on $\pset_\kappa(\lambda)$.
\item For every $\lambda \geq \kappa$ there is an extender $E$ such that $\crt E = \kappa$, $j_E(\kappa) > \lambda$, and $E$ has closure $\geq \lambda$.
\end{enumerate}
\end{corollary}

\begin{theorem}[Magidor]
\index{Magidor's theorem}
Let $\kappa$ be a cardinal. Then $\kappa$ is supercompact if and only if for every $\lambda \geq \kappa$ there is a $\delta < \lambda$ and an elementary embedding $j: V_\delta \to V_\lambda$ such that $j \crt j = \kappa$.
\end{theorem}
\begin{proof}
First suppose $\kappa$ is supercompact. Let $\lambda \geq \kappa$ and $k: V \to M$ be a witness that $\kappa$ is $(\card V_\lambda)$-supercompact, so $M^{\card V_\lambda} \subseteq M$.
Therefore $k|V_\lambda \in M$, and since $\lambda < k(\kappa)$,
$$M \models \text{There are $\eta < k(\kappa)$ and $i: V_{\eta} \to V_{j(\lambda)}$ such that $i(\crt i) = k(\kappa)$.}$$
Here the witnesses are $\eta = \lambda$ and $i = k|V_\lambda$.
Since $k$ is an elementary embedding, there are $\delta < \kappa$ and $j: V_\delta \to V_\lambda$ such that $j \crt j = \kappa$.

For the converse, let $\lambda \geq \kappa$; we need a $\kappa$-complete normal fine ultrafilter on $\pset_\kappa(\lambda)$.
By assumption, there is an elementary embedding $j: V_\delta \to V_\lambda$ for some $\delta < \lambda$, and $j$ extends to an elementary embedding $j: V_{\delta + 10} \to V_{\lambda + 10}$, where $j \crt j = \kappa$.
Let $\eta = \crt j$. By assumption, $\pset(\pset_\eta(\delta)) \in V_{\delta + 10}$, and
$$U' = \{x \subseteq \pset_\eta(\delta): j"\delta \in j(x)\}$$
is a $\eta$-complete normal fine ultrafilter on $\pset_\eta(\delta)$. Here the $+10$ guarantees that we have enough iterates of the power set to carry out the argument above.
Then $U = j(U')$ is a $j(\eta) = \kappa$-complete normal fine ultrafilter on $\pset_{j(\eta)}(j(\delta)) = \pset_\kappa(\lambda)$.
\end{proof}

By a similar argument we have:
\begin{corollary}
Let $\kappa$ be a cardinal. Then $\kappa$ is supercompact if and only if for every $\lambda \geq \kappa$ there is an extender $E$ such that $\supp E \subseteq \kappa$, the strength of $E$ is $\geq \lambda$, and $j_E(\crt E) = \kappa$.
\end{corollary}

\begin{theorem}[Solovay]
\index{Solovay's supercompactness theorem}
Suppose $\lambda > \kappa$ are regular cardinals.
Then there is an $X \subseteq \pset_\kappa(\lambda)$ such that $\sup|X: X \to \lambda$ is injective, and for every $\kappa$-complete normal fine ultrafilter $U$ on $\pset_\kappa(\lambda)$, $X \in U$.
\end{theorem}
\begin{proof}
Let
$$S = \{\alpha < \lambda: \cof \alpha = \omega\}.$$
If $c$ is a club in $\lambda$, then so is $c \cap \Card$, so let $(\delta_n: n < \omega)$ be an increasing sequence of cardinals in $c \cap \Card$, which is possible since $c \cap \Card$ is cofinal.
Then their limit $\delta = \sup_{n < \omega} \delta_n$ is also in $c \cap \Card$ since it is closed, so $\delta \in c \cap S$.
Therefore $S$ is stationary, so by Solovay's stationary theorem, there is a partition of $S$ into stationary sets $(S_\alpha: \alpha < \lambda)$.

If $\beta < \lambda$ and $\omega < \cof \beta < \kappa$, let
$$\sigma_\beta = \{\alpha < \beta: S_\alpha \cap \beta \text{ is stationary in } \beta\}.$$
\begin{lemma}
For every $\beta$ such that $\sigma_\beta$ exists, $\sigma_\beta \in \pset_\kappa(\lambda)$.
\end{lemma}
\begin{proof}
By assumption, we have a collection $(S_\alpha \cap \beta: \alpha \in \sigma_\beta)$ of disjoint stationary sets in $\beta$.

Suppose that $\card \sigma_\beta \geq \kappa$. Let $c$ be a club in $\beta$ whose ordertype is $\cof \beta$, which exists since $\cof \beta > \omega$.
Then the sets $S_\alpha \cap \beta \cap c$ are all disjoint, and there are $\kappa$ sets. Thus they induce a partition of (a subset of) $\cof \beta$ into $\kappa$ many nonempty sets, but $\cof \beta < \kappa$ so this is a contradiction.
\end{proof}
Now let
$$X = \{\sigma_\beta \in \pset_\kappa(\lambda): \sup \sigma_\beta = \beta\}.$$
By construction, $\sup|X$ is injective.

Now fix a normal fine $\kappa$-complete ultrafilter $U$ on $\pset_\kappa(\lambda)$. We must show $X \in U$.
Let $j: V \to \Ult(V, U)$ be the ultrapower embedding, and let
$$(T_\alpha: \alpha < j(\lambda)) = j((S_\alpha: \alpha < \lambda)).$$
Let $\lambda^* = \sup j"\lambda$.

Let
$$\sigma_\beta' = \{\alpha < \beta: T_\alpha \cap \beta \text{ is stationary in } \beta\}.$$
Then
$$j(X) = \{\sigma_\beta' \in \pset_{j(\kappa)}(j(\lambda)): \Ult(V, U) \models \sup \sigma_\beta' = \beta\}.$$
\begin{lemma}
In $\Ult(V, U)$, $j"\lambda = \sigma_{\lambda^*}'.$
\end{lemma}
\begin{proof}
First we prove that if $\alpha < \lambda$ then $j(\alpha) \in \sigma_{\lambda^*}'$.
In other words, we must show that $T_{j(\alpha)} \cap \lambda^* = j(\alpha) \cap \lambda^*$ is stationary in $\lambda^*$.

Let $c \in \Ult(V, U)$ be a club in $\lambda^*$, and let
$$D = \{\alpha < \lambda: j(\alpha) \in c\}.$$
We claim that $D$ is an \dfn{$\omega$-club} (as a subset of $\lambda$ in $V$) in the sense that it is cofinal and closed under $\omega$-limits.
To see this, note that $j"\lambda$ is cofinal in $\lambda$, and any $\omega$-limits in $\lambda$ are sent to $\omega$-limits in $\lambda^*$, so $j"\lambda$ is an $\omega$-club.
But $D = j"\lambda \cap c$, so $D$ is also an $\omega$-club.
Therefore there is a $\xi \in D \cap S_\alpha$, but then
$$j(\xi) \in c \cap S_{j(\alpha)} = T_{j(\alpha)},$$
so $j(\alpha) \in \sigma'_{\lambda^*}$.

Conversely, suppose $\alpha \in \sigma'_{\lambda^*}$, so $\Ult(V, U)$ thinks $T_\alpha \cap \lambda$ is stationary in $\lambda^*$.
We will show $\alpha = j(\alpha')$ for some $\alpha' < \lambda$, so $\alpha \in j"\lambda$.

By the same argument as above, $j"\lambda$ is an $\omega$-club in $\Ult(V, U)$, so there is a $\xi < \lambda$ such that
$$j(\xi) \in T_\alpha \cap j"\lambda.$$
But $(j"(S_{\alpha'} \cap \lambda^*): \alpha' < \lambda)$ is a partition of $j"\lambda$.
So we may choose an $\alpha'$ such that
$$j(\xi) \in j"(S_{\alpha'} \cap \lambda^*).$$
Then $\alpha = j(\alpha')$.
\end{proof}
Therefore $\Ult(V, U) \models \sup \sigma_{\lambda^*}' = \lambda^*$, i.e. $j"\lambda \in j(X)$.
Since
$$U = \{Y \subseteq \pset_\kappa(\lambda): j"\lambda \in j(Y)\}$$
this implies that $X \in U$.
\end{proof}
At first it seems surprising that every normal fine ultrafilter would contain a given set, but consider that on $\omega$, for every cofinite set $X$ and every nontrivial ultrafilter $U$, $X \in U$.

\chapter{The Ultimate L}
Recall that if $\kappa$ is a measurable cardinal witnessed by an ultrafilter $U$, then there is a canonical model $L[U]$ where $\kappa$ is the unique measurable cardinal.
Then $\kappa \in L[U]$, so $\kappa \cap L[U] = \kappa \in U$.
We also proved that $U \cap L[U] \in L[U]$. So $L[U]$ inherits the witness $U$ to the measurability of $\kappa$, namely $U \cap L[U]$.
We want the same thing for supercompact cardinals, so we can get a canonical model of one supercompact cardinal.
When we do this, it will turn out that, pending some conjectures of Woodin, such a model would not just be a canonical model for one supercompact cardinal, but every large cardinal whatsoever,

\begin{definition}
Let $T$ be a theory in the language $(\in)$. An \dfn{inner model} of $T$ is a transitive class $M$ such that $(M, \in)$ is a model of $T$ and $\Ord \subset M$.
\end{definition}
We can also talk about inner models of languages which extend $(\in)$. For example, $V$ and $L$ are inner models of ZFC, $V$ being the maximal such model and $L$ being the minimal such model.
Meanwhile $L[U]$ is an inner model of the theory ZFC + ``There is a measurable cardinal", and is also an inner model of the theory ZFC + ``$U$ is a normal fine ultrafilter" in the language $(\in, U)$.
Notice that since an inner model gets the relation $\in$ correct, if it is a model of foundation then it is actually well-founded.

\begin{definition}
Let $\delta$ be a supercompact cardinal.
A \dfn{weak extender model for the supercompactness} of $\delta$ is an inner model $N$ of ZFC such that for every $\lambda \geq \delta$ there is a $\delta$-complete normal fine ultrafilter $U$ on $\pset_\delta(\lambda)$ such that:
\begin{enumerate}
\item $\pset_\delta(\lambda) \cap N \in U$.
\item $U \cap N \in N$.
\end{enumerate}
\end{definition}
We don't know that $\pset_\delta(\lambda) \cap N = \pset_\delta(\lambda)$ a priori, so $\pset_\delta(\lambda) \cap N \in U$ is quite a strong assumption to make on $U$.

\begin{definition}[Hamkins]
Let $N$ be an inner model and $\kappa$ a cardinal. We say $N$ has the \dfn{$\kappa$-covering property} if for every $\tau \subset N$ such that $\card \tau < \kappa$, there is a set $\tau' \in N$ such that $\card \tau' < \kappa$ and $\tau \subseteq \tau'$.
\end{definition}
If an inner model $N$ has $\kappa$-covering, then $N$ is ``fat below $\kappa$": $N$ contains ``most" sets under $\kappa$.
For example, if $\tau$ is a definable subset of $N$, $N$ is an inner model of ZF, and $N$ has $(\card \tau)^+$-covering, then there is a $\tau' \in N$ such that $\card \tau = \card \tau'$ and $\tau \subseteq \tau'$.
But since $N$ has comprehension and $\tau$ was assumed definable, this implies that actually $\tau \in N$.

We now show that weak extender models for the supercompactness of $\delta$ have the $\delta$-covering property.
Since supercompact cardinals are really big, this in particular implies that weak extender models for supercompactness are really fat up to a possibly irrelevant tail.
\begin{lemma}
Suppose $N$ is an inner model of ZFC such that for all $\lambda \geq \delta$, there is a $\delta$-complete normal fine ultrafilter $U$ on $\pset_\delta(\lambda)$ such that $\pset_\delta(\lambda) \cap N \in U$. Then $N$ has the $\delta$-covering property.
\end{lemma}
\begin{proof}
We first check this when $\tau \subset \Ord$. Let $\lambda > \sup \tau$; then $\tau \subseteq \lambda$.
Since $U$ is fine, if $\alpha < \lambda$,
$$A_\alpha = \{\sigma \in \pset_\delta(\lambda): \alpha \in \sigma\} \in U.$$
Let $A = \bigcap_{\alpha \in \tau} A_\alpha$; since $\card \tau < \delta$, $A \in U$.
Since $\pset_\delta(\lambda) \cap N \in U$ and $U$ is an ultrafilter, $A \cap \pset_\delta(\lambda) \cap N$ is nonempty.
So let $\tau' \in A \cap \pset_\delta(\lambda) \cap N$.
Then $\card \tau' < \delta$ and $A_\alpha \subseteq \tau'$ for any $\alpha < \tau$, hence $\tau \subseteq \tau'$.

Now if $\tau$ is arbitrary, choose a well-ordering of $\tau$ in $N$ and precede as above.
\end{proof}

\section{Averting the dichotomy}
We now give another way, besides the $\delta$-covering property, in which weak extender models are much closer to $V$ than $L$.
\begin{definition}
Let $N$ be an inner model of ZFC and $\delta$ a regular cardinal.
We say that $N$ is \dfn{close to $V$} (in the sense of singular cardinals) above $\delta$ if for every singular cardinal $\gamma > \delta$:
\begin{enumerate}
\item $N \models $ $\gamma$ is a singular cardinal.
\item $(\gamma^+)^N = \gamma^+$.
\end{enumerate}
If $N$ is close to $V$ above $\aleph_0$, we simply say that $N$ is \dfn{close to $V$}.
\end{definition}
For some motivation for the definition, consider the following theorem of Jensen.
\begin{definition}
If $A$ is a set, define its \dfn{sharp} $A^{\#}$ to be the definable real number which codes the set of true sentences about $L[A]$ in the language $(\in, A, \aleph_n : n < \omega)$.
\end{definition}
So $0^{\#}$ cannot be definable, and thus its definition makes no sense, if $V = L$, by Tarski's theorem on the undefinability of truth.
Thus the existence of $0^{\#}$ is a ``switch" that tells us how well $L$ approximates $V$.
Similarly $U^{\#}$ is a ``switch" that tells us how well $L[U]$ approximates $V$.
These switches are very dangerous to have, because of the following theorem.
\begin{theorem}[Jensen's covering theorem]
\index{Jensen's convering theorem}
Exactly one of the following is true:
\begin{enumerate}
\item $0^{\#}$ does not exist, and $L$ is close to $V$.
\item $0^{\#}$ exists, and for every uncountable cardinal $\kappa$,
$$L \models \text{$\kappa$ is an inaccessible cardinal}.$$
\end{enumerate}
Moreover, if there is a measurable cardinal, then $0^{\#}$ exists.
\end{theorem}
Intuitively, if there is a measurable cardinal, then $L$ is so pathetically small that we can define truth in it, and thinks that even $\aleph_1$ is inaccessible.
This is a bad property of $L$: by whimsical identity, $0^{\#}$ exists, so any axiom which says that $V$ looks like $L$ in some way needs to be thrown out.
The proof of Jensen's covering theorem is notoriously difficult, and we omit it.

Notice that if $N$ is close to $V$, then the calamitous behavior that $N$ thinks that every uncountable cardinal is a large cardinal is impossible:
it would then be the case that such cardinals are regular in $N$, and hence are actually regular cardinals, but the class of singular cardinals is proper.

A dichotomy similar to $L$ is not possible for weak extender properties, however.
\begin{theorem}
\label{WEM are close to V}
Let $N$ be a weak extender model for the supercompactness of $\delta$. Then $N$ is close to $V$ above $\delta$.
\end{theorem}
To prove this theorem, we will need the following lemma.
\begin{lemma}
Let $N$ be a weak extender model for the supercompactness of $\delta$ and $\lambda \geq \delta$ is a regular cardinal in $N$.
Then $\cof \lambda = \card \lambda$ in $V$.
\end{lemma}
\begin{proof}
Since $N$ is a weak extender model for the supercompactness of $\delta$ and satisfies Solovay's supercompactness theorem, there is a $\delta$-complete normal fine ultrafilter $U$ and a set $X \in U \cap N$ such that $\pset_\delta(\lambda) \cap N \in U$, $U \cap N \in U$, and $\sup|X$ is injective.

Now let $c \subseteq \lambda$ be a club of ordertype $\cof \lambda$. Let $X_c = \{\sigma \in X: \sup \sigma \in c\}$.
Then $j_U(c)$ is a club and $j_U"c \subseteq j_U(c)$, so $j_U"\lambda \in j_U(X)$ and $\sup \sigma \in j_U(c)$, so $j_U"\lambda \in j_U(X_c)$. Therefore $X_c \in U$.
Since $U$ is fine, $\bigcup X_c = \lambda$.
On the other hand, the $\delta$-covering property applied to each $\sigma \in \pset_\delta(\lambda)$ implies that $\delta \leq \cof \lambda$, so
$$\card X_c \leq \delta \card c = \delta \cof \lambda = \cof \lambda.$$
Therefore
$$\card \lambda \leq \delta \card X_c = \cof \lambda.$$
Since $\card \lambda \geq \cof \lambda$, $\card \lambda = \cof \lambda$.
\end{proof}
\begin{proof}[Proof of Theorem \ref{WEM are close to V}]
Suppose $\gamma$ is regular in $N$. Then
$$\gamma < \cof \gamma = \card \gamma = \gamma,$$
a contradiction.

Now suppose $(\gamma^+)^N < \gamma^+$. Then $(\gamma^+)^N$ is regular in $N$ since it is a successor in $N$. Therefore
$$\cof (\gamma^+)^N = \card (\gamma^+)^N = \gamma.$$
But $\cof(\gamma^+)^N < \gamma$ since $\gamma$ is singular.
\end{proof}

\section{Magidor's weak extender model formulation}
\begin{theorem}
\label{magidor weak extender}
Let $N$ be an inner model of ZFC and $\delta$ a regular cardinal.
Then $N$ is a weak extender model for the supercompactness of $\delta$ iff
for every $\lambda > \delta$ and $a \in V_\lambda$ there are $\delta' < \lambda' < \delta$, $a' \in V_{\lambda'}$, and
$$j: V_{\lambda' + 1} \to V_{\lambda + 1}$$
such that $\crt j = \delta'$ where $j(\delta') = \delta$, $j(a') = a$, $j(N \cap V_{\lambda'}) = N \cap V_\lambda$, and
$$j|(N \cap V_{\lambda' + 1}) \in N.$$
\end{theorem}
In other words, weak extender models $N$ for supercompactness can be witnessed by choosing an elementary embedding $j: V_{\lambda' + 1} \to V_{\lambda + 1}$, where $\crt j < \lambda' < \delta < \lambda$, which restricts to a member of $N$ and preserves $N$ and a given set $a$.
Our goal with this theorem is to show that weak extender models actually contain most elementary embeddings, and so contain arbitrarily large cardinals.
This makes them drastically different than other inner models such as $L[U]$.

To prove the above theorem we will need the following lemma.
\begin{lemma}
Let $N$ be an inner model of ZFC, $\delta$ a regular cardinal,
$$\lambda = \card^N(N \cap V_\lambda)$$
a regular cardinal such that $\lambda > \delta$, and $U$ a witness that $N$ is a weak extender model for the $\lambda$-supercompactness of $\delta$, i.e. $\pset_\delta(\lambda) \cap N \in U$ and $U \cap N \in N$.

Suppose that $j: V \to \Ult(V, U)$ is the ultrapower map. Then
$$j(N \cap V_\delta) \cap V_\lambda = N \cap V_\lambda.$$
\end{lemma}
This lemma says that the ultrapower actually preserves $N \cap V_\delta$, which is really huge.
\begin{proof}
We claim
$$N \cap V_\lambda \subseteq j(N \cap V_\delta) \cap V_\lambda.$$
Now
$$U = \{X \subseteq \pset_\delta(\lambda): j"\lambda \in j(X)\},$$
so since $\pset_\delta(\lambda) \cap N \in U$,
$$j"\lambda \in j(\pset_\delta(\lambda) \cap N) \subseteq j(N).$$

Since $\lambda = \card^N(N \cap V_\lambda)$, there is a bijection $e: \lambda \to N \cap V_\lambda$ such that $e \in N$.
Let $x \in N \cap V_\lambda$, and let $\alpha < \lambda$ be such that $e(\alpha) = x$. Then
$$j(x) = j(e(\alpha)) = j(e)(j(\alpha)).$$
So
$$j"(N \cap V_\lambda) = j(e)"(j"\lambda) \in j(N).$$

Let $T: M \to L$ be the transitive collapse map.
But $N \cap V_\lambda$ is isomorphic to $j"(N \cap V_\lambda)$, yet is a transitive class; thus $N \cap V_\lambda = T(j"(N \cap V_\lambda))$ and in particular,
$$N \cap V_\lambda = T(j(e)"(j"\lambda)).$$
So $N \cap V_\lambda \subseteq j(N)$.
But $j(\delta) > \lambda$, so $N \cap V_\lambda \subseteq j(V_\delta)$ and hence
$$N \cap V_\lambda \subseteq j(N \cap V_\delta) \cap V_\lambda.$$

For the converse, note that since $N \cap V_\lambda = T(j(e)"(j"\lambda))$, it suffices to show that
$$j(N \cap V_\delta) \cap V_\lambda \subseteq T(j(e)"(j"\lambda)).$$

Let $X$ be the set of all $\sigma \in \pset_\delta(\lambda)$ such that:
\begin{enumerate}
\item $e"\sigma$ is an elementary substructure of $N \cap V_\lambda$, and
\item $T(e"\sigma) = N \cap V_\gamma$ where $\gamma$ is the ordertype of $\sigma$.
\end{enumerate}
Then $j(X)$ is the set of all $\sigma \in \pset_{j(\delta)}(j(\lambda))$ such that:
\begin{enumerate}
\item $j(e)"\sigma$ is an elementary substructure of $j(N) \cap V_\lambda$, and
\item $T(j(e)"\sigma) = N \cap V_\gamma$ where $\gamma$ is the ordertype of $\sigma$.
\end{enumerate}
If $X \in U$, then $j"\lambda \in j(X)$, and hence
$$T(j(e)"(j"\lambda)) = j(N) \cap V_\lambda = j(N \cap V_\delta) \cap V_\lambda,$$
which proves the claim. So we must show $X \in U$.

Let $X' = X \cap N$; we claim $X' \in U$. Since $U' = U \cap N$ is a $\kappa$-complete normal fine ultrafilter on $\pset_\delta(\lambda) \cap N$, if $j': N \to \Ult(N, U')$ denotes its ultrapower map, then:
\begin{enumerate}
\item $j'(e)"((j')"\lambda)$ is an elementary substructure of $j'(N) \cap V_{j'(\lambda)}$.
\item $T(j'(e)"((j')"\lambda)) = N \cap V_\lambda$.
\item $\Ult(N, U') \cap V_\lambda = N \cap V_\lambda$.
\end{enumerate}
Therefore $(j')"\lambda \in j'(X')$.
So $X' \in U'$, hence $X' \in U$, hence $X \in U$.
\end{proof}

\begin{proof}[Proof of Theorem \ref{magidor weak extender}]
Suppose the second. Let $\gamma > \delta$ be such that $\gamma = \card V_\gamma$.
By hypothesis applied to $\lambda = \gamma + \omega$ (ordinal addition here) to get $\delta' < \lambda' < \delta$ and $j: V_{\lambda' + 1} \to V_{\lambda + 1}$ with the desired properties. Here $a$ can be whatever.

Since $\gamma$ is definable in $V_{\lambda + 1}$ (since $\lambda = \gamma + \omega$), let $\gamma'$ have the same definition in $V_{\lambda' + 1}$, so $j(\gamma') = \gamma$. So
$$\delta' < \gamma' < \lambda' < \delta < \lambda < \gamma$$
and $j"\gamma' \in V_{\lambda + 1}$.

Let
$$U' = \{X \subseteq \pset_{\delta'}(\lambda'): j"\gamma' \in j(X)\}$$
and $U = j(U')$. Then $U'$ is a $\delta'$-complete normal fine ultrafilter on $\pset_{\delta'}(\lambda')$, so $U$ is a $\delta$-complete normal fine ultrafilter on $\pset_\delta(\lambda)$. In particular $\delta$ is $\lambda$-supercompact.

We must show that $U$ is a witness to the sentence ``$N$ is a weak extender model for the $\lambda$-supercompactness of $\delta$."
\begin{lemma}
$\pset_{\delta'}(\gamma') \cap N \in U'$.
\end{lemma}
\begin{proof}
By hypothesis $j(N \cap V_{\lambda'}) = N \cap V_\lambda$. So if $a' \in V_{\lambda'}$,
$$j(a' \cap N) = j(a') \cap j(N) = j(a') \cap N.$$
Taking $a' = \pset_{\delta'(\gamma')}$ $\pset_{\delta'}(\gamma) \in V_{\lambda'}$, so $j(\pset_{\delta'}(\gamma') \cap N) = \pset_\delta(\gamma) \cap N$ and hence $j"\gamma' \in N$ implies
$$j"\gamma' \in \pset_\delta(\gamma) \cap N = j(\pset_{\delta'}(\gamma') \cap N),$$
so the claim follows by definition of $U'$.
\end{proof}
Taking $j$ of everything in the previous lemma we get $\pset_\delta(\gamma) \cap N \in U$, which was desired.

Moreover, by hypothesis, $j|(N \cap V_{\lambda' + 1}) \in N$. In particular $j(U' \cap N) \in N$. But $j(U' \cap N) = U \cap N$.
So $U$ witnesses that $N$ is a weak extender model for the $\lambda$-supercompactness of $\delta$.

Now suppose that $N$ is a weak extender model for the supercompactness of $\delta$ and $\lambda > \delta$ is a regular cardinal.
Let $\gamma > \lambda$ and suppose $\gamma = \card V_\gamma$.
By the lemma applied to an appropriate regular cardinal $>\gamma$, there is an ultrafilter $U$ and an elementary embedding $j: V \to \Ult(V, U)$ such that $j(\delta) > \lambda$, $\Ult(V, U)$ is closed under $\card V_{\gamma+1}$-sequences, $j(N \cap V_\delta) \cap V_\lambda = N \cap V_\lambda$, and $j"\lambda \in j(N)$.
By assumption, $V_{\lambda+1} = V_{\lambda+1}^{\Ult(V,U)}$. So the elementary embedding $j$ sends $V_{\lambda+1} = V_{\lambda+1}^{\Ult(V,U)}$ to $V_{j(\lambda)+1}^{\Ult(V,U)}$, hence given $a \in V_\lambda$,
$\Ult(V, U)$ thinks there are $\delta' < j(\delta), \lambda' < j(\delta),a'$ and an elementary embedding $k: V_{\lambda'+1} \to V_{j(\lambda)}+1$, $\crt k = \delta'$, $k(\delta') = j(\delta)$, $k(a') = j(a)$, $k(j(N) \cap V_{\lambda' + 1}) \in j(N)$.
The witness here is $k = j|V_{\lambda+1}$. Pulling back along $j: V \to \Ult(V, U)$, we see the claim.
\end{proof}

\section{Woodin's universality theorem}
We now show that weak extender models contain very large elementary embeddings.
\begin{theorem}
Suppose $N$ is a weak extender model for the supercompactness of $\delta$. Let $\gamma \in \Card^N$ and let
$$j: \Hull((\gamma^+)^N) \to M$$
be an elementary embedding such that $\crt j \geq \delta$ and $M \subseteq N$. Then $j \in N$.
\end{theorem}
\begin{proof}
Let $\gamma > \lambda$ be given, where $j \in V_\lambda$ and $\lambda = \card V_\lambda$.
By Theorem \ref{magidor weak extender}, there are $\delta',\gamma',\lambda',j' \in V_{\lambda'}$, and an elementary embedding
$$\pi: V_{\lambda' + 1} \to V_{\lambda + 1}$$
such that $\crt \pi = \delta'$, $\pi(\delta') = \delta$, $\pi(j') = j$, $\pi|N \in N$, and
$$\pi(N \cap V_{\lambda'}) = N \cap V_\lambda.$$
In particular, $j'$ is an elementary embedding $\Hull((\gamma')^+)^N \to M'$ where $M' \subseteq N$ is some class.
If $j' \in N$, then since $\pi|N \in N$, $j = \pi(j') \in N$ as well.
So it suffices to show $j' \in N$.

We are given $\pi|(N \cap V_{\gamma'}) \in N$.
Thus
$$\pi|\Hull(((\gamma')^+)^N) \in \Hull((\gamma^+)^N) \subseteq N.$$

Let $a' \in \Hull(((\gamma')^+)^N)$ and $j'(a') = b' \in M'$. Since $a' \in V_{\lambda' + 1}$,
$$\pi(j'(a')) = \pi(b') = \pi(j')(\pi(a')) = j(\pi(a')).$$
Thus $\pi \circ j' = j \circ \pi$.
But $a',b'$ are in the hull, so
$$j(\pi(a')) = j(\pi|\Hull(((\gamma')^+)^N))(j(a')).$$
All this junk is actually in $N$, so $j' \in N$.
\end{proof}
Thus we have shown that $N$ contains lots of elementary embeddings, and so is nothing like the inner models under it like $L$ or $L[U]$.
In particular, there is no generalization of Scott's theorem to weak extender models for the supercompactness of $\delta$. For example:
\begin{corollary}
Suppose that $N$ is a weak extender model for the supercompactness of $\delta$ and $\kappa > \delta$ is a supercompact cardinal. Then
$$N \models \text{$\kappa$ is supercompact.}$$
\end{corollary}
\begin{proof}
Let $\lambda > \kappa$. Then there is an elementary embedding $j: V \to M$ which witnesses that $\kappa$ is $\lambda$-supercompact and for every $\alpha > \lambda$, the restriction $j|(N \cap V_{\alpha+2})$ is in $N$. Therefore $j|N \in N$. So $N$ thinks that $\kappa$ is $\lambda$-supercompact.
\end{proof}

\section{The HOD dichotomy}
Let $N$ be a weak extender model for the supercompactness of $\delta$.
\begin{definition}
A set is \dfn{ordinal-definable} if it is definable from parameters in $\Ord$.
It is \dfn{hereditarily ordinal-definable} if its transitive closure is ordinal-definable.
The class $\HOD$ consists of all hereditarily ordinal-definable sets.
\end{definition}
It's pretty easy to see that $\HOD$ is an inner model of ZFC; the proof is essentially the same as the proof that $L$ is. Moreover, $L \subseteq \HOD$.
A major open problem is to show that $N \subseteq \HOD$; this would imply that $N$ has all the good properties of $L$ and none of the bad ones. In fact, a conjecture of Woodin says that we can even have $N = \HOD$.

But first we need to show that $\HOD$ is not like $L$ in the sense that there is no analogue of $0^\sharp$.
\begin{definition}
Let $\kappa$ be a regular cardinal. We say that $\kappa$ is a \dfn{strongly measurable cardinal in $\HOD$} if there is a $\lambda < \kappa$ such that $(2^\lambda)^\HOD < \kappa$, and $\HOD$ thinks that there there is no partition of $\{\alpha < \kappa: \cof \alpha = \omega\}$ into $\lambda$ many stationary sets which are stationary in $V$.
\end{definition}
This seems a little silly, because $\HOD$ is a model of ZFC and so by Solovay's stationary theorem, $\HOD$ thinks there is a partition of $\{\alpha < \kappa: \cof \alpha = \omega\}$ into $\lambda$ many sets (even $\kappa$ many sets), but the point is that those sets will not actually be stationary in $V$.

We should check that if a cardinal is measurable in $\HOD$ then it is measurable in $\HOD$. This is not nearly as tautological as it sounds.
\begin{lemma}
Let $\kappa$ be a strongly measurable cardinal in $\HOD$.
Then there is a stationary set $S \subseteq \{\alpha < \kappa: \cof \alpha = \omega\}$ such that $S \in \HOD$ and there is no partition of $S$ into multiple sets which are stationary in $V$.
\end{lemma}
\begin{proof}
Suppose not, and let $\lambda$ be so large that we cannot partition $A = \{\alpha < \kappa: \cof \alpha = \omega\}$ into $\lambda$ many sets which are stationary in $V$.
\begin{lemma}
There is a binary tree $T \in \HOD$ of height $\lambda + 1$ and a sequence $(S_t: t \in T)$ such that:
\begin{enumerate}
\item $S_\Box = A$. Here $\Box$ is the empty sequence.
\item For every $t \in T$, $S_t$ is a stationary subset of $A$ in $V$.
\item For every $t \in T$, $t+0$ and $t+1$ are in $T$. Here $+$ is concatenation.
\item For every $t \in T$, $S_t = S_{t+0}\cup S_{t+1}$, and $S_{t+0} \cap S_{t+1}$ is empty.
\item For every $t \in T$ such that the length $\beta$ of $t$ is a limit ordinal,
$$S_t = \bigcap_{\alpha < \beta} S_{t|\alpha}.$$
\item For every limit ordinal $\beta \leq \lambda$ and every vertex $t: \beta \to 2$, if $t \notin T$ but for every $\alpha < \beta$, $t|\alpha \in T$, then $\bigcap_{\alpha<\beta} S_{t|\alpha}$ is not stationary.
\end{enumerate}
\end{lemma}
Intuitively, the binary tree $(S_t: t \in T)$ partitions $A$ into smaller and smaller sets which are stationary in $V$, splitting each set into two each time.
\begin{proof}
Since $\HOD$ is a model of choice we can use transfinite recursion.
At stage $\alpha+1$, choose a splitting $S_t = S_{t+0} \cup S_{t+1}$ for each $t$ of length $\alpha$, which exists by our contradiction hypothesis.

The hard work is at limit stages. Let $\beta$ be a limit stage. Then
$$A = \bigcup_t \bigcap_{\alpha < \beta} S_{t|\alpha}$$
where $t$ ranges over all $t \in \HOD$ which is a branch of length $\beta$ in $T$.
Since $2^\beta < \kappa$ and the intersection of $<\kappa$ clubs is a club, so there is a branch $t$ such that
$$S_t = \bigcap_{\beta < \alpha} S_{t|\alpha}$$
is stationary. For every such branch, put it in $t$ and keep the induction going.
\end{proof}

Let $t$ be a branch of $T$ of length $\lambda + 1$. Then let $S^\alpha = S_{t|\alpha} \setminus S_{t|\alpha+1}$.
Since $S_{t|\alpha}$ split into two stationary sets, $S_{t|\alpha+1} = S_{(t|\alpha)=0}$ and $S_{(t|\alpha)+1}$, $S^\alpha$ is stationary if $\alpha \leq \lambda$.
So $(S^\alpha:\alpha \leq \lambda)$ is a partition of $A$ into $\lambda$ stationary sets in $V$, a contradiction.
\end{proof}

\begin{lemma}
Let $\kappa$ be a strongly measurable cardinal in $\HOD$. Then $\HOD$ thinks that $\kappa$ is a measurable cardinal.
\end{lemma}
\begin{proof}
Using the previous lemma, let $F$ be the club filter
$$F = \{X \subseteq S: \text{There is a club $c$ in $\kappa$ such that $c \cap S \subseteq X$}.$$
Let $F' = F \cap \HOD$. Then elements of $F'$ are in $\HOD$, yet $F$ and $\HOD$ are ordinal-definable; therefore $F' \in \HOD$.
Since $F$ is a $\kappa$-complete filter on $2^S$ in $V$, $F'$ is a $\kappa$-complete filter on $2^S$ in $\HOD$.

We claim that $\HOD$ thinks that $F'$ is an ultrafilter, which implies that $\HOD$ thinks that $\kappa$ is measurable.
Suppose that $S_0 \in \HOD$ and $S_0 \subseteq S$. Let $S_1 = S \setminus S_0$.
Since $S$ is stationary in $\HOD$, at least one of the $S_j$ is too, say $S_0$.
But then $S_1$ cannot be stationary in $\HOD$, for if it were then $S = S_0 \cup S_1$ would partition $S$ into sets which are stationary in $V$, a contradiction.
So $S_0 \in F'$, so $\HOD$ thinks $F'$ is a ultrafilter.
\end{proof}
This is beginning to look dangerously like the situation with $L$ and $0^\sharp$.
If $\kappa$ is an uncountable cardinal in $V$ and $0^\sharp$ exists, then $L$ thinks that $\kappa$ is an inaccessible cardinal, even if $\kappa$ was actually a singular cardinal. Something similar \emph{apparently} happens for $\HOD$.

\begin{definition}
Let $\kappa$ be a regular cardinal and $\eta > \kappa$ an ordinal. We say that $\kappa$ is an \dfn{$\eta$-extendible cardinal} if there is a $\lambda > \kappa+\eta$ and an elementary embedding
$$j: V_{\kappa+\eta} \to V_\lambda$$
such that $\crt j = \kappa$. An \dfn{extendible cardinal} is one which is $\eta$-extendible for every $\eta$.
\end{definition}
It follows straight from the definitions that every extendible cardinal is supercompact, strong, measurable, etc.
Moreover, by universality, every extendible cardinal $\kappa$ remains extendible in any weak extender model for the supercompactness of any cardinal $\delta$ such that $\kappa \geq \delta$.
Intuitively, every extendible cardinal is so ridiculously huge that fragments of the universe all start to blur together.

\begin{theorem}[$\HOD$ Dichotomy Theorem; Woodin]
Suppose that $\delta$ is an extendible cardinal. Then exactly one of the following is true:
\begin{enumerate}
\item $\HOD$ is a weak extender model for the supercompactness of $\delta$.
\item For every regular cardinal $\kappa > \delta$, $\kappa$ is strongly measurable in $\HOD$.
\end{enumerate}
\end{theorem}
This looks at least superficially like the Jensen covering lemma and the $0^\sharp$ dichotomy.
In fact, if $\HOD$ is a weak extender model for the supercompactness of $\delta$, then by a previous lemma about such models, then $\HOD$ is close to $V$ above $\delta$.
However, Woodin actually thinks that the $\HOD$ Dichotomy Theorem is not actually a dichotomy; i.e. that $\HOD$ is actually a weak extender model for the supercompactness of every extendible cardinal in $V$.
\begin{conjecture}[$\HOD$ Conjecture; Woodin]
For every extendible cardinal $\delta$, $\HOD$ is a weak extender model for the supercompactness of $\delta$.
\end{conjecture}
Even though the $\HOD$ Dichotomy Theorem is a dichotomy, we can use it to prove lots of things about $\HOD$. For example:
\begin{corollary}
If there is an extendible cardinal, then there is a measurable cardinal in $\HOD$.
\end{corollary}
\begin{proof}
Let $\delta$ be an extendible cardinal.
If $\HOD$ is a weak extender model for the supercompactness of $\delta$, then $\delta$ is measurable in $\HOD$ and we can go home.
Otherwise, the $\HOD$ Dichotomy Theorem implies that there is a proper class of cardinals which are strongly measurable in $\HOD$.
\end{proof}
\begin{corollary}
Let $\delta$ be an extendible cardinal. If there is a regular cardinal $> \delta$ which is not strongly measurable in $\HOD$, then $\HOD$ is close to $V$ above $\delta$.
\end{corollary}

To prove the $\HOD$ Dichotomy Theorem, we use the following lemma, which says that we can keep reflecting regular cardinals which are not strongly measurable in $\HOD$ higher and higher until we have a proper class of them.
\begin{lemma}
Suppose that $\delta$ is an extendible cardinal and $\gamma_0 > \delta$ is a regular cardinal which is not strongly measurable in $\HOD$.
Then there is a proper class of regular cardinals above $\gamma_0$ which are not strongly measurable in $\HOD$.
\end{lemma}
\begin{proof}
Let $\alpha > \lambda_0 > \gamma_0$ be such that $V_{\lambda_0}$ is a $\Sigma_2$-substructure of $V$.
This exists because we can always restrict to $\Sigma_2$ sentences and use reflection, then blow up to a sufficiently large cardinal and take transitive collapse.
Suppose that $\lambda_0 = \card V_{\lambda_0}$.
Since $\HOD$ has a $\Sigma_2$ definition, we have
$$\HOD^{V_{\lambda_0}} = \HOD \cap V_{\lambda_0}.$$

Since $\delta$ is an extendible cardinal, there is an elementary embedding
$$j: V_{\alpha + 1} \to V_{j(\alpha) + 1}$$
such that $\crt j =\delta$ and $j(\delta) > \alpha$.
For every $\lambda < \gamma_0$, if $\card^\HOD 2^\lambda < \gamma_0$ then there is a partition $(S_\alpha: \alpha < \lambda)$ of $S = \{\alpha < \gamma_0: \cof \alpha = \omega\}$ into stationary sets, such that $(S_\alpha: \alpha < \lambda) \in \HOD$.
By elementarity, this remains true in $V_{\lambda_0}$, hence in $V_{j(\lambda_0)}$.
\end{proof}

\begin{proof}[Proof of the $\HOD$ Dichotomy Theorem]
Let $\gamma_0 > \delta$ be a regular cardinal which is not strongly measurable in $\HOD$.
By the lemma, there is a proper class of regular cardinals which are not strongly measurable in $\HOD$.
We must show that $\HOD$ is a weak extender model for the supercompactnesss of $\delta$.

Let $\lambda > \delta$. It suffices to prove the following lemma:
\begin{lemma}
There is a $\delta$-complete, normal fine ultrafilter $U$ on $\pset_\delta(\lambda)$ such that $\pset_\delta(\lambda) \cap \HOD \in U$ and $U \cap \HOD \in \HOD$.
\end{lemma}
Let $\lambda_0 > \lambda$ be such that $\card V_{\lambda_0} = \lambda_0$.
By a previous lemma, there is a regular cardinal $\gamma_0 > \card^\HOD 2^{\lambda_0}$ which is not strongly measurable in $\HOD$.

Since $\gamma_0$ is not strongly measurable in $\HOD$ and $\card^\HOD 2^{\lambda_0} < \gamma_0$, there is a partition $(S_\alpha: \alpha < \lambda_0) \in \HOD$ of
$$S^{\gamma_0} = \{\alpha < \gamma_0: \cof \alpha = \omega\}$$
such that for every $\alpha < \lambda_0$, $S_\alpha$ is stationary in $V$.

Using reflection, let $\lambda_1 > \gamma_0$ be such that $V_{\lambda_1}$ is a $\Sigma_2$ substructure of $V$ and $\card V_{\lambda_1} = \lambda_1$. Therefore
$$\HOD^{V_{\lambda_1}} = \HOD \cap V_{\lambda_1}.$$
In particular,
$$(S_\alpha: \alpha < \lambda_0) \in \HOD^{V_{\lambda_1}}.$$
Since $\delta$ is an extendible cardinal, there is an elementary embedding
$$j: V_{\lambda_1 + 1} \to V_{j(\lambda_1) + 1}$$
such that $\crt j = \delta$ and $j(\delta) > \lambda_1$.
So we may define
$$(T_\alpha: \alpha < j(\lambda_0)) = j(S_\alpha: \alpha < \lambda_0)$$
which is a partition in $\HOD \cap V_{j(\lambda_1)}$ of
$$S^{j(\gamma_0)} = \{\alpha < j(\gamma_0): \cof \alpha = \omega\}$$
into $j(\lambda_0)$ sets which are stationary in $V$.

We now have the ordinals
$$\delta < \lambda < \lambda_0 < \gamma_0 < \lambda_1 < j(\gamma_0).$$
Moreover, $j(\gamma_0)$ is a regular cardinal and $\crt j < \gamma_0$, so $j(\gamma_0) > \sup j"\gamma_0$.
Let $Z$ be the set of all $\alpha < j(\lambda_0)$ such that $T_{j(\alpha)} \cap \sup j"\gamma_0$ is stationary in $\sup j"\gamma_0$.
\begin{lemma}
$Z = j"\lambda_0$.
\end{lemma}
\begin{proof}
We first show $j"\lambda_0 \subseteq Z$. Let $\alpha < \lambda_0$ and $C \subseteq \sup j"\gamma_0$ a club.
Let $D$ be the cofinal set $\{\alpha < \gamma_0: j(\alpha) \in C\}$. Then $\overline D$ is a club.
Moreover, since $j$ is continuous at points of cofinality $\omega$,
$$\overline D \cap S^{\gamma_0} = D \cap S^{\gamma_0}.$$
But $S_\alpha \subseteq S^{\gamma_0}$ is stationary in $V$, so there is a $\beta \in S_\alpha \cap D = S_\alpha \cap \overline D$.
Thus $j(\beta) \in T_{j(\alpha)} \cap C$.

Conversely, let $\alpha \in Z$. So $T_\alpha \cap \sup j"\gamma_0$ is stationary, hence meets the club
$$C = \overline{j"\gamma_0} \cap \sup j"\gamma_0.$$
Let $\beta \in T_\alpha \cap \sup"\gamma_0 \cap C$. But $C \cap j(S^{\gamma_0}) = j"(S^{\gamma_0})^V$ so $\beta \in j"(S^{\gamma_0})^V$.
Let $\beta' \mapsto \beta$. So $j(\beta') \in T_\alpha$.
The $S_\eta$ are a partition, so let $\alpha'$ be such that $S_{\alpha'} \ni \beta'$. Then $\beta \in T_{j(\alpha')}$. But $\beta \in T_\alpha$, so $\alpha = j(\alpha)'$.
\end{proof}
In particular, $j"\lambda_0 \in \HOD$, since the definition of $Z$ was from parameters in $\HOD$.

Since $\lambda_0 = \card V_{\lambda_0}$,
$$\lambda_0 = \card^\HOD(\HOD \cap V_{\lambda_0}).$$
In particular, $\HOD$ thinks there is a bijection $\lambda_0 \to \HOD \cap V_{\lambda_0}$.
Thus
$$j"(\HOD \cap V_{\lambda_0}) = j(\pi)"(j"\lambda_0).$$
We proved that $j(\pi)$ and $j(\lambda_0)$ are in $\HOD$. Therefore $j"(\HOD \cap V_{\lambda_0}) \in \HOD$.

Now let
$$U = \{X \subseteq \pset_\delta(\lambda): j"\lambda \in j(X)\}.$$
Since $j"(\HOD \cap V_{\lambda_0}) \in \HOD$ and $\lambda < \lambda_0$, $U$ is definable in $\HOD$.
Therefore $U \cap \HOD \in \HOD$. It follows that $\pset_\delta(\lambda) \cap \HOD \in U$.
\end{proof}


\section{$V$ is Ultimate L}
We have all the pieces in play to give the ultimate generalization of $L$, assuming of course that several open problems are solved.

\begin{definition}
A set $A \subseteq \RR$ is said to be \dfn{universally Baire} if for every compact Hausdorff space $X$ and every continuous function $f: X \to \RR$, there is an open set $U \subseteq X$ such that the symmetric difference $U \Delta f^{-1}(A)$ is meager in $X$.
\end{definition}
Thus a univerally Baire set has the property of Baire in every compact Hausdorff space, even those of extremely large cardinality. This is a ridiculously strong property.

\begin{definition}
An inner model $N$ is said to be \dfn{Ultimate $L$} if there is a proper class of Woodin cardinals in $N$ and, for every $\Sigma_2$ sentence $\varphi$ such that $N \models \varphi$, there is a universally Baire set $A$ such that $\HOD^{L(A, \RR)} \models \varphi$.
\end{definition}

The first hint that Ultimate $L$ might be something special is that it decides lots of sentences:
\begin{theorem}[Woodin]
If $N$ is Ultimate $L$ then $N \models$ the continuum hypothesis, and $N = \HOD^N$.
\end{theorem}

But the real significance of Ultimate $L$ is that it cannot be broken by forcing.
\begin{theorem}[Woodin]
Suppose that $\PP$ is a forcing notion for $V = \text{Ultimate } L$. Then $V$ actually is Ultimate $L$, and forcing by $\PP$ is the identity map.
\end{theorem}
If we believe that $V = \text{Ultimate } L$ is actually a true axiom, then this means that forcing fails completely, and the spectre of independence is essentially ``cured," as long as we also believe that sufficiently large cardinals are consistent.

There are two problems with the axiom $V = \text{Ultimate } L$, however.

First, it's not clear that such an axiom is consistent; we should be able to prove its consistency from large cardinal axioms. For $L$ and $L[U]$ we proved consistency by exhibiting a model of $V = L$ or $V = L[U]$. The former always existed, while the latter existed pending large cardinal axioms.

Secondly, we want $V = \text{Ultimate } L$ to have large cardinals. This is why we threw out $L$ and $L[U]$ in the first place!
If we are in the bad case of the $\HOD$ Dichotomy, everything goes to hell, as if $N$ is Ultimate $L$, then $N = \HOD^N$, yet $V \neq \HOD$, so $N$ is nothing like $V$.

\begin{conjecture}[the Ultimate $L$ Conjecture; Woodin]
Suppose that $\delta$ is an extendible cardinal. Then there is an Ultimate $L$ which is a weak extender model for the supercompactness of $\delta$.
\end{conjecture}
This is a $\Sigma_1$ statement in the language of arithmetic, so it must be either true or false. If it is true and there is an extendible cardinal, then we have found our Ultimate $L$ and we can go home.
By the universality theorem, Ultimate $L$ has all large cardinals; it rules out forcing, and decides all statements about $\RR$ and other ``small" sets.
In particular, the continuum hypothesis is very likely true in Platonic reality if the Ultimate $L$ Conjecture is true.
Moreover, the Ultimate $L$ Conjecture implies the $\HOD$ Conjecture, showing that the $\HOD$ Dichotomy Theorem is not actually a dichotomy, and among other amusing consequences, showing that ZF alone can prove that Berkeley cardinals do not exist.
Thus several major open problems in set theory would be solved in one fell swoop.

If the Ultimate $L$ Conjecture is false, then Woodin's program ends in failure.



\newpage
\printindex

\end{document}
