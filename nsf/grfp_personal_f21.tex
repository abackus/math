% --------------- 12 POINT FONT -------------------------------
\documentclass[12pt]{article}
% --------------- 10 POINT FONT FOR CAPTIONS ------------------
%\usepackage[font=footnotesize]{caption}
% --------------- NY TIMES FONT -------------------------------
\usepackage{times}
% --------------- 1 INCH MARGINS ------------------------------
\usepackage[margin=1in]{geometry}
% --------------- LINE SPACING --------------------------------
\usepackage{setspace}
\singlespacing
%\doublespacing
% --------------- SMALL SECTION TITLES ------------------------
\usepackage[tiny,compact]{titlesec}
% --------------- MATH PACKAGES -------------------------------
\usepackage{amsmath,amsthm,amssymb,mathrsfs}
\usepackage{hyperref}

\usepackage[backend=bibtex,maxcitenames=50,maxnames=50]{biblatex}
\addbibresource{grfp_research_f21.bib}
\renewbibmacro{in:}{}
\DeclareFieldFormat{pages}{\footnotesize #1}

\begin{document}
% --------------- TITLE AND NAME ------------------------------
%\begin{center}
%\large{\bf Personal } \\
%\end{center}
% --------------- CONTENT -------------------------------------
% \noindent
% \textbf{Purpose:} I might be able to do good research (see \cite{rudin1987real}).\\
%
% \noindent
% \textbf{Background:} I might be able to do good research.\\
%
% \noindent
% \textbf{Hypothesis:} I might be able to do good research.
% $$\sin x = \cos x$$
%
% \noindent
% \textbf{Research Plan:} I might be able to do good research.\\
% \textbf{Aim 1} You can type your Aim 1 here if you have one. If you don't then you can delete the subsection\\
% \textbf{Aim 2} You can type your Aim 1 here if you have one. If you don't then you can delete the subsection\\
%
% \noindent
% \textbf{Intellectual Merit:} I might be able to do good research.\\
%
%
% \noindent
% \textbf{Broader Impacts:} I might be able to do good research.\\

Prior to my undergraduate years, I had very little exposure to mathematics beyond the usual high school sequence: my high school was fatally underfunded, and enrichment activities took a backseat to just getting everyone to graduate.
At the urging of my mother, who thought it crucial that I saw examples of leaders who, like myself, were Mexican-American, I attended talks by my hometown's resident scientist, the astronaut Jos\'e Hern\'andez; but while Hern\'andez' influence convinced me to be a scientist, I had no plan for my future beyond that vague goal.

As such, I did not know that ``mathematician" was even a career, and I ended up in the field somewhat by accident, after fumbling through programming and physics classes in my first year of college at UC Berkeley and realizing that my true talents lay elsewhere.
At Berkeley, I devoured both my classes and any research or teaching opportunity I could seize.
I graduated with high honors in 2020, having completed six graduate-level courses and a thesis, and then began work on my doctorate at Brown University.

My first research experience was at the San Diego State REU in 2018\footnote{Vaidya, N., et al. \url{http://www.sci.sdsu.edu/math-reu/2018-1.pdf}}.
Working with Dr. Naveen Vaidya's DiMoLab (or Disease Modeling Laboratory), I studied a continuous-time model for the spread of HIV amongst cells in a host.
Our team was able to give a formula for the \emph{basic reproduction number} $\mathcal R_0$ of the model -- that is, the expected number of cells that each infected cell will, in turn, infect, in an otherwise healthy host -- in terms of the parameters.
I was then able to argue that if $\mathcal R_0 < 1$, then the infection will die out, while if $\mathcal R_0 > 1$, then the infection will tend to a steady state; the latter models the course of an HIV infection much more effectively.
Thus I learned how to work with a team of collaborators, some more comfortable with biology or computer science than mathematics.

Thanks to my time at DiMoLab, I became interested in the qualitative behavior of differential equations, and after a reading course with Dr. Charles Pugh on Peixoto's theorem on the structural stability of ODE on closed surfaces, I traveled to the University of Houston for their Summer School in Dynamics.
It was here that I learned the art of conjecture:
Why does this proof need the Stone-Weierstrass theorem?
These theorems seem like they would be useful for PDE -- can we extend them to infinite-dimensional function spaces?
Is this result still true in the nonuniformly hyperbolic setting?
My head spun with such questions.

I returned home from Houston to work on my bachelor's thesis, guided by Dr. Maciej Zworski.
In my thesis and a later preprint\footnote{Backus, A. \emph{The Breit-Wigner series for noncompactly supported potentials on the line.} arXiv:2005.13765.}, now in revision for submission, I study the eigenvalue problem for the Schr\"odinger operator $P = V - \partial_x^2$ on $\mathbb R$, where $V \in L^\infty(\mathbb R)$ decays so fast that we can define the notion of \emph{resonance} of $P$, a sort of generalization of eigenvalue.
It is of practical interest to be able to compute the \emph{Breit-Wigner series} $\sum_j \operatorname{Im} \lambda_j/|\lambda_j|^2$, where the sum is taken over all resonances.
If $V$ has compact support, then the series is known to converge; I conjectured that the converse was true, and proved this if $V$ is holomorphic.
This project was largely carried out at the beginning of the pandemic, while in isolation.
Thus while it introduced me to both modern complex analysis, more importantly, I gained the basic skills one needs for individual work: the ability to find theorems in established literature and the perserverance to keep searching for a solution, even when the previous several attempted arguments have failed.

While at Brown, I have found myself in something of a pickle, but a good pickle: there's just too much interesting math that I want to learn!
Last spring, I took a course in Riemann surfaces taught by Dr. Georgios Daskalopoulos, and during the course I asked if he could advise me for my topics exam the following semester, which would be on geometric measure theory.
I had learned very little geometry as an undergraduate, and hoped to correct this blind spot.
I enjoyed this project immensely -- see my research proposal for more on that.
To supplement my work on the topics exam, I participated in a reading group with students of Zworski on Fourier integral operators, which are a generalization of linear differential operators which arise in the study of the wave equation and that are best understood in terms of symplectic geometry.

My interest in numerics originated in a PDE course taught by Dr. Javier G\'omez-Serrano, in which, on the final exam, we had to prove a theorem from the 90's in spectral geometry.
The best way to do this, as far as I can tell, is to numerically check several nasty Bessel function inequalities; I was able to sketch how one could use a computer to check these inequalities, though my proof was far from rigorous.
G\'omez-Serrano took pity on me and gave me an ``A" in spite of my ignorance.
After this incident, I resolved to learn numerical theory and interval arithmetic.
I have since worked to improve my programming skills, drawing on my experience from the countless hours that I spent as a teenager building up the codebase of the Final Fantasy Wiki\footnote{Backus, A., as ``Catuse167". \url{https://tinyurl.com/intangirpfix}}
and a programming project in college in which we developed a Sage tool for computations in infinite-dimensional Lie algebras\footnote{Backus, A., Connick, P., and Lin, J. \emph{An algorithm for computing root multiplicities in Kac-Moody algebras.} arXiv:1912.04540.}.
I believe that numerically checking the plausibility of a conjecture before attempting a proof will soon become standard practice; furthermore, computer-assisted proof, in which one reduces a conjecture to an inequality of real numbers and then leaves a computer to check the inequality, is an unexploited vein in analysis.

I also recognize that mathematicians, as part of the scientific community, have a moral responsibility to the betterment of our society, which compels them to work on endeavors not entirely insular to mathematics itself.
Von Neumann\footnote{Von Neumann, J. \emph{The Mathematician}, part of \emph{Works of the Mind}, U. Chicago Press (1947), pp180--196.} phrased this observation somewhat more crudely: ``after much `abstract' inbreeding, a mathematical subject is in danger of degeneration".
As discussed above, I already have the ability to communicate well with scientists and engineers as well as the numerical and programming skills needed to make practical computations, rather than just prove theorems.
To augment these two skills, two of my goals for 2022 are to take a reading course in water waves with Dr. Susanna Haziot, and to become familiar with the literature of computational geometry, hoping to tie my work in geometric PDE (see my research proposal) to its more concrete applications.

On a related note, let me note that I had a fairly comfortable childhood (my hometown's failing public school system aside), and it was not until a few years into college that I awokened to the inequities of the mathematical community, when I had to work with a professor who acted quite friendly with me.
Over the following weeks, however, I learned that he treated a few of his female students utterly differently, being utterly dismissive of their concerns about their research and their adviser's conduct.
This culminated him in berating one of his students about the presentation she was going to give that afternoon, far too late for her to ``fix" it.
According to him, the only reason she had been given funding is because her adviser had recommended it, rather than her own merits -- a patent absurdity when one considers the quality of her contributions to our research.

Since that experience, I have become fully committed to service to the scientific community, and especially to underprivileged students.
When I was an undergraduate, I served the community as the curator of the Mathematics Undergraduate Student Association at Berkeley, as well as a teaching assistant to several math and computer science courses.
Both in my capacities as curator and as TA, I frequently lobbied Berkeley to the benefit of the undergraduate community.
For example, after many students reported the conduct of a professor who did not bother giving a syllabus or even vaguely follow the intended curriculum of the course he was teaching, I met with the curriculum chair to discuss possible courses of action.
And after I was hired as a TA to a professor, who suffered a disability that required him to lecture by writing on a transparency on a projector, was not able to lecture effectively because loose live wires were bleeding out of the power supply, I met with higher and higher ranking bureaucrats until I was finally able to secure funding for replacement equipment.

I am most proud of my work as curator in the reconstruction of Math 74, our proof-writing class.
Math 74 existed as a course taught by department faculty that was cut due to the Great Recession.
As a consequence, many math majors, especially those who ended up in mathematics after leaving the computer science major, arrived in real analysis or abstract algebra without much of an idea of how to write proofs.
In 2018, MUSA offered a clone of Math 74 as a experimental under-the-table course; using what we had learned from that experience, I spent much of July 2019 working on bringing the curriculum up to par.
The next semester, the waitlist peaked at 40 students.
I loosely based the course on the model of the Houston Summer School's problem sessions: a member of the teaching staff would teach one day per week, and assign some exercises.
Then, students would collaborate on a more challenging exercise later in the week, which they would then present to the class, developing both rudimentary problem-solving skills as well as mathematical communication skills.
Our version of Math 74, which was just a shadow of what the Berkeley math department could offer, was so popular that the department bowed to pressure and reintroduced the true Math 74 in Spring 2021.

At Brown, I have continued my service by working on the committee for our American Mathematical Society chapter, and by teaching multiple reading courses for undergraduates.
At the AMS chapter, we are planning social activities to help Brown's graduate students reconnect after two years of isolation due to COVID-19.
We are also working on holding a conference for graduate students across New England, to be held this spring.
In this project we have prioritized marketing to women and underrepresented minority students in the region.

With regards to reading courses, my philosophy is that any undergraduate course that does not have to follow a pre-set curriculum should be student-driven.
The students choose the topics; I find literature at their level which discusses these topics.
For example, my current reading course was broadly construed to be about models of traffic flow.
The simplest such equation is the Burgers equation $\partial_t u + u \partial_x u = 0$, where $u = u(t, x) \in [-1, 1]$ represents the severity of the traffic.
Since all three students are interested in numerical analysis, I am leading them through literature on numerically solving the Burgers equation with the goal of having as a final project building a program in MATLAB which simulates a New England highway.
Such a project is particularly suitable for undergraduates, because the basic equation one learns first how to solve in a numerical PDE course is the transport equation $\partial_t u + a \partial_x u = 0$, which at small times can be viewed as a linearization of the Burgers equation about $a = u$ -- so there is lots of elementary numerical literature on the Burgers equation.

Summarizing, as an NSF graduate fellow I would have three goals, all of them backed by years of experience: the betterment of my own research skills, collaboration with not just other mathematicians but scientists and engineers in other adjacent fields, and continued activism for the inclusivity of the mathematical community.
Thank you for considering my application.

% --------------- WORKS CITED (10pt FONT) ---------------------

% \begin{thebibliography}{aa}
%
% \bibitem{amau} Maurer, Andrew B. \emph{\LaTeX \ Template for the National Science Foundation's Graduate Research fellowship Program}.
%
% \end{thebibliography}
\small
\printbibliography
\end{document}

% -------------------------------------------------------------

% -------------------------------------------------------------
