% --------------- 12 POINT FONT -------------------------------
\documentclass[12pt]{article}
% --------------- 10 POINT FONT FOR CAPTIONS ------------------
%\usepackage[font=footnotesize]{caption}
% --------------- NY TIMES FONT -------------------------------
\usepackage{times}
% --------------- 1 INCH MARGINS ------------------------------
\usepackage[margin=1in]{geometry}
% --------------- LINE SPACING --------------------------------
\usepackage{setspace}
\singlespacing
%\doublespacing
% --------------- SMALL SECTION TITLES ------------------------
\usepackage[tiny,compact]{titlesec}
% --------------- MATH PACKAGES -------------------------------
\usepackage{amsmath,amsthm,amssymb,mathrsfs}
\usepackage{hyperref}

\usepackage[backend=bibtex,maxcitenames=50,maxnames=50]{biblatex}
%\addbibresource{grfp_research_f21.bib}
\renewbibmacro{in:}{}
\DeclareFieldFormat{pages}{\footnotesize #1}

\begin{document}
In the 2022-23 academic year, I began a research program to better understand the geometry of the solutions of the \emph{$p$-Laplace equation}, which is a differential equation that depends on a parameter $1 \leq p \leq \infty$, focusing on the special cases $p = 1$ and $p = \infty$.
Solutions of the $p$-Laplacian are functions that assign a number $u(x)$ to each point of some shape $M$, so we can ask about their \emph{level sets}, the set of points where $u(x)$ equals some fixed number.
The level sets are \emph{minimal surfaces} -- they have zero mean curvature, or in other words, if the level sets were rubber sheets, they would be stretched taut.
If $M$ is flat, this was already known, and is essential to the \emph{level set method} of computational geometry, which allows one to computationally study problems in geometry by realizing shapes as level sets of certain functions, and is an important technical tool in image processing and computational biophysics.
However, I established the minimality of level sets without assuming anything about the geometry of $M$.
I also showed that if you put together all of the level sets at once, you obtain a certain kind of shape called a \emph{measured oriented minimal lamination} (MOML), and similarly, if you have an MOML, it can always be obtained from a solution of the $1$-Laplace equation in this way.
To accomplish this, I needed to prove certain auxiliary results of a purely geometric character; related results were already in the literature, but I organized them into a clear and coherent theory.
I established these results in a series of two papers that should appear on the arXiv and be submitted to journals this summer.

My original motivation for considering the $1$-Laplacian came from \emph{Teichm\"uller theory}, which is a branch of geometry which measures ``how similar two shapes are''.
For example, a circle and a slightly deformed oval are very similar, but a circle and a very long, skinny oval are not close at all.
The similarity of two shapes $M, N$ can be understood by understanding the similarity of MOMLs in $M, N$, so it's natural to try to look for an equation which describes MOMLs, and to require that the basic properties of that equation hold even if $M$ is not flat.
There are various results which assert that the $1$-Laplacian can be understood by understanding certain equations related to the $\infty$-Laplacian; I am currently investigating the possibility that this can be accomplished in using an equation called the \emph{$\infty$-Maxwell equation}.

% --------------- WORKS CITED (10pt FONT) ---------------------

% \begin{thebibliography}{aa}
%
% \bibitem{amau} Maurer, Andrew B. \emph{\LaTeX \ Template for the National Science Foundation's Graduate Research fellowship Program}.
%
% \end{thebibliography}
%\small
%\printbibliography
\end{document}

% -------------------------------------------------------------

% -------------------------------------------------------------
