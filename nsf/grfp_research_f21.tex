% --------------- 12 POINT FONT -------------------------------
\documentclass[11pt]{article}
% --------------- 10 POINT FONT FOR CAPTIONS ------------------
\usepackage[font=footnotesize]{caption}
% --------------- NY TIMES FONT -------------------------------
\usepackage{times}
% --------------- 1 INCH MARGINS ------------------------------
\usepackage[margin=1in]{geometry}
% --------------- LINE SPACING --------------------------------
\usepackage{setspace}
\singlespacing
%\doublespacing
% --------------- SMALL SECTION TITLES ------------------------
\usepackage[tiny,compact]{titlesec}
% --------------- MATH PACKAGES -------------------------------
\usepackage{amsmath,amsthm,amssymb,mathrsfs}

\usepackage[backend=bibtex,maxcitenames=50,maxnames=50]{biblatex}
\addbibresource{grfp_research_f21.bib}
\renewbibmacro{in:}{}
\DeclareFieldFormat{pages}{#1}

\begin{document}
% --------------- TITLE AND NAME ------------------------------
\begin{center}
\large{\bf Geodesic laminations by least-gradient maps} \\
\end{center}
% --------------- CONTENT -------------------------------------
% \noindent
% \textbf{Purpose:} I might be able to do good research (see \cite{rudin1987real}).\\
%
% \noindent
% \textbf{Background:} I might be able to do good research.\\
%
% \noindent
% \textbf{Hypothesis:} I might be able to do good research.
% $$\sin x = \cos x$$
%
% \noindent
% \textbf{Research Plan:} I might be able to do good research.\\
% \textbf{Aim 1} You can type your Aim 1 here if you have one. If you don't then you can delete the subsection\\
% \textbf{Aim 2} You can type your Aim 1 here if you have one. If you don't then you can delete the subsection\\
%
% \noindent
% \textbf{Intellectual Merit:} I might be able to do good research.\\
%
%
% \noindent
% \textbf{Broader Impacts:} I might be able to do good research.\\
Hilbert's 19th problem asks whether the problem of minimizing $\int \mathscr L$ subject to boundary data, where $\mathscr L$ is an analytic convex \emph{Lagrangian}, only has analytic solutions. Minimizers of $\int |du|^2$ are exactly the harmonic functions, $\Delta u = 0$, so this problem generalizes the fact that harmonic functions are analytic.
It was answered in the affirmative by de Giorgi \cite{deGiorgi57}.

Another kind of optimization problem asks one to find a hypersurface $\Sigma \subset M$ in a Riemannian manifold $M$, which minimizes the surface area of $\Sigma$, given the boundary $\partial \Sigma$.
Minimal surfaces blah blah blah if $M$ is a surface then minimal surfaces in $M$ are nothing more than geodesics in $M$.

I propose to study \emph{least-gradient functions}, which are minimizers $u$ of $\int |du|$ which have locally bounded variation.
Least-gradient functions need not be continuous, as $|\cdot|
$ is not analytic, but we still have an analogue of Hilbert's 19th problem in dimension $d \leq 7$: the level sets $\partial \{u > y\}$, $y \in \mathbb R$, of a least-gradient function are all analytic minimal surfaces \cite{Giusti77}.

Roughly speaking, the idea of the proof is to construct tangent spaces to $\partial U$, which may in general not even be vector spaces, but are if $\partial U$ is minimal and $d \leq 7$.
One then deduces, using a lemma of de Giorgi, that the unit normal $\mathbf n$ to $\partial U$ is continuous.
Since $\mathbf n$ is continuous, $\partial U$ can be written as the graph of a $C^1$ function $f$ which minimizes $\int \sqrt{1 + |df|^2}$ and hence is analytic by Hilbert's 19th problem.
I have generalized this argument to Riemannian manifolds of dimension $\leq 7$.
In fact, since the claim is local, one can work in open neighborhoods of points on $\partial U$ which are so small that the Riemannian metric appears to be just a slight perturbation of the Euclidean metric when written in suitable coordinates.
As a result, some (but not all) steps of the proof can be reduced to the Euclidean case by a perturbation argument.

The above notion of least-gradient function turns out to be unsuitable on closed manifolds $M$, as the vacuity of the boundary condition means that the only minimizers of $\int |du|$ are constant functions.
In a recent preprint \cite{daskalopoulos2020transverse}, Uhlenbeck and my advisor, Daskalopoulos, study instead \emph{least-gradient maps} $u: M \to S^1$, with $S^1$ the circle, which minimize $\int |du|$ in their homotopy class.
This notion is particularly interesting in the case of $M$ a closed surface, as I expect that my above argument should extend to the case of least-gradient maps and show that every least-gradient map induces a \emph{geodesic lamination} on $M$.
That is, the ``minimal surfaces" are all geodesics and partition a closed subset of $M$.

If $M$ is a closed hyperbolic surface -- that is, a closed surface whose universal cover is the hyperbolic plane -- then geodesic laminations on $M$ have been well-studied...



% --------------- WORKS CITED (10pt FONT) ---------------------

\footnotesize
% \begin{thebibliography}{aa}
%
% \bibitem{amau} Maurer, Andrew B. \emph{\LaTeX \ Template for the National Science Foundation's Graduate Research fellowship Program}.
%
% \end{thebibliography}
\printbibliography
\end{document}

% -------------------------------------------------------------

% -------------------------------------------------------------
