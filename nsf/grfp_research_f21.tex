% --------------- 12 POINT FONT -------------------------------
\documentclass[12pt]{article}
% --------------- 10 POINT FONT FOR CAPTIONS ------------------
%\usepackage[font=footnotesize]{caption}
% --------------- NY TIMES FONT -------------------------------
\usepackage{times}
% --------------- 1 INCH MARGINS ------------------------------
\usepackage[margin=1in]{geometry}
% --------------- LINE SPACING --------------------------------
\usepackage{setspace}
\singlespacing
%\doublespacing
% --------------- SMALL SECTION TITLES ------------------------
\usepackage[tiny,compact]{titlesec}
% --------------- MATH PACKAGES -------------------------------
\usepackage{amsmath,amsthm,amssymb,mathrsfs}
\usepackage{hyperref}

\usepackage[backend=bibtex,maxcitenames=50,maxnames=50]{biblatex}
\addbibresource{grfp_research_f21.bib}
\renewbibmacro{in:}{}
\DeclareFieldFormat{pages}{#1}

\begin{document}
% --------------- TITLE AND NAME ------------------------------
%\begin{center}
%\bf Laminations by minimal surfaces induced by least-gradient functions \\
%\end{center}
% --------------- CONTENT -------------------------------------
% \noindent
% \textbf{Purpose:} I might be able to do good research (see \cite{rudin1987real}).\\
%
% \noindent
% \textbf{Background:} I might be able to do good research.\\
%
% \noindent
% \textbf{Hypothesis:} I might be able to do good research.
% $$\sin x = \cos x$$
%
% \noindent
% \textbf{Research Plan:} I might be able to do good research.\\
% \textbf{Aim 1} You can type your Aim 1 here if you have one. If you don't then you can delete the subsection\\
% \textbf{Aim 2} You can type your Aim 1 here if you have one. If you don't then you can delete the subsection\\
%
% \noindent
% \textbf{Intellectual Merit:} I might be able to do good research.\\
%
%
% \noindent
% \textbf{Broader Impacts:} I might be able to do good research.\\
A \emph{minimal surface} is a submanifold $\Sigma \subset M$ in a manifold $M$ which minimizes the surface area of $\Sigma$ subject to a constraint.
The problem of finding minimal surfaces is fundamental.
In pure mathematics, such geometric optimization problems go back to \emph{Dido's problem} of ancient Carthage.
They are a basic example in elliptic PDE, and were the original impetus for the development of geometric measure theory.
As a sample of the applications, minimal surfaces are used to prove the positive mass theorem in general relativity, to determine the correct shape of artificial hearts\footnote{Bates, P. W., et al. ``Minimal Molecular Surfaces and Their Applications." \emph{J. Comput. Chem.} 29, pp380--391.}, and as models for biomolecules\footnote{Walker, D. K., et al. ``Development and in vitro assessment of a new two-leaflet replacement heart valve designed using computer-generated bubble surfaces." \emph{Med. Biol. Eng. Comput.} 21, pp31--38.}.
Particularly important examples are \emph{geodesics}, or minimal surfaces of dimension $1$.
(For example, a geodesic in the euclidean plane is the same thing as a straight line.)

A natural generalization of minimal surfaces, which I propose to study, are \emph{least-gradient functions}.
These are functions $u$ of bounded variation whose derivatives $du$ minimize $\int |du|$.
If $U$ is a region in $M$, and $u = 1$ on $U$, $u = 0$ elsewhere, then $\int |du|$ is the surface area of the boundary $\partial U$.
Therefore, if $u$ has least gradient, then $\partial U$ is a minimal surface.

Thurston\footnote{Thurston, W. \emph{Minimal stretch maps between hyperbolic surfaces}. arXiv:9801039.} introduced \emph{best-Lipschitz functions}, those which instead minimize $\sup |du|$, with a view to his theory of \emph{geodesic laminations} -- that is, closed sets which can be partitioned into simple closed geodesics.
He conjectured a duality theory relating best-Lipschitz and least-gradient functions, inspired by the max-flow/min-cut duality of computer science.
Uhlenbeck and my advisor, Daskalopoulos, recently initiated a research program\footnote{Daskalopoulos, G. and Uhlenbeck, K. \emph{Transverse measures and best Lipschtiz and least gradient maps}. arXiv:2010.06551.} to develop best-Lipschitz/least-gradient duality.
As an application, they deduce that \emph{if $u$ is a least-gradient function on a closed hyperbolic surface, then the support of $du$ is a geodesic lamination}.
Geodesic laminations are a basic object of study in Teichm\"uller theory, and this theorem gives a powerful tool for constructing them.

Daskalopoulos and Uhlenbeck conjecture that one can prove the above theorem without using best-Lipschitz/least-gradient duality, but rather by studying least-gradient functions directly.
To this end, I propose to develop a regularity theory for least-gradient functions on manifolds.
As we discussed above, in general, functions of least gradient may not even be continuous, so one needs to make precise what is meant for a least-gradient function to be regular.
If $u$ is a least-gradient function, then the level sets $\partial \{u > y\}$, $y \in \mathbb R$, are minimal surfaces, which \emph{a priori} may be ill-behaved.
A regularity theory would amount to showing that such minimal surfaces are, in fact, smooth.
If this is true, then a standard argument implies above the conjecture.

Hilbert's 19th problem posits that the problem of minimizing $\int \mathscr L$, where $\mathscr L$ is analytic and convex in $u$, only has analytic solutions.
It was proven by de Giorgi\footnote{de Giorgi, E. ``Sulla differenziabilità e l'analiticità delle estremali degli integrali multipli regolari". \emph{Memorie della Accademia delle Scienze di Torino. Classe di Scienze Fisiche, Matematicahe e Naturali}. 3, 25--43.}.
So, to show that the level sets $N = \partial \{u > y\}$ are analytic, one just needs to show that $N$ is a $C^1$-differentiable manifold.
If this is true, then $N$ can locally be written as the graph of a $C^1$-differentiable function $f$ which minimizes the \emph{minimal surface functional}, $A = \int \sqrt{1 + |df|^2}$, which is the surface area of $N$.
The function $x \mapsto \sqrt{1 + |x|^2}$ is analytic and convex, so in this case, $N$ is analytic by Hilbert's 19th problem.

The strategy outlined above has been carried out in the euclidean case by the Italian school of de Giorgi, Miranda, et al., in dimensions $\in [2, 7]$; they also constructed a counterexample to regularity in dimension $8$\footnote{Guisti, E. \emph{Minimal surfaces and functions of bounded variation}. Canberra, 1977.}.
The key step of the proof is a lemma of de Giorgi concerning the \emph{excess} $\Lambda_r$ of $N$ in a ball $B$ of radius $r$ centered at a point $x$, which roughly measures how badly $N$ fails to be a hyperplane in $B$.
According to the de Giorgi lemma, once $r$ is chosen small enough, $\Lambda_{r/2}$ is smaller than one would naively expect just from rescaling $\Lambda_r$.
By induction, then, $\Lambda_{r/2^n}$ must be tiny.
Once this control on $\Lambda_{r/2^n}$ is established, one can ``blow up $N$ at a point $x$" by taking $n \to \infty$, causing $N$ to look like a hyperplane.
Then the normal vector $\mathbf n$ to $N$ is continuous in a neighborhood of $x$, since clearly the normal to a hyperplane is continuous.
The continuity of $\mathbf n$ implies that $N$ is $C^1$-differentiable.

However, the de Giorgi lemma is not quite true in the noneuclidean case, for a number of reasons.
One does not have a notion of hyperplane, and the metric may have nonconstant curvature.
One also needs to define ``normal vector" and ``excess" in a coordinate-free manner that still works in the measurable category.
Finally, the proof of the de Giorgi lemma needs to be modified to account for the extra terms one picks up due to the curvature of $M$.
To overcome these obstacles, I have already introduced a notion of vector bundle-valued Radon measure, and found bounds on the curvature terms which I believe are sufficiently strong.
Thus I expect to be able to prove a noneuclidean analogue of the de Giorgi lemma, and hence I should be able to prove the conjecture of Daskalopoulos--Uhlenbeck in a particularly strong form: \emph{in any orientable Riemannian manifold of dimension $\in [2, 7]$, the support of the derivative of a least-gradient function defines a lamination by smooth minimal hypersurfaces}.

The above proposal originated as a topics exam given to me by Daskalopoulos.
Since then, I have become more ambitious and now hope to resolve some of the other problems posed by Daskalopoulos--Uhlenbeck, making good use of my background in PDE to further develop the theory of best-Lipschitz/least-gradient duality.
For example, Daskalopoulos--Uhlenbeck show that \emph{if $v$ is a best-Lipschitz function on a closed hyperbolic manifold $M$, then the set of points $x \in M$ which maximize $|dv_x|$ define a geodesic lamination $\lambda$}.
(In fact, if $u$ is the least-gradient dual to the best-Lipschitz function $v$, then $\lambda$ is the geodesic lamination I propose to study above.)
They conjecture that this result should hold in higher generality, and that it would hold if one showed a suitable regularity result for best-Lipschitz functions.

Evans--Savin showed\footnote{Evans, L. C., and Savin, O. ``$C^{1, \alpha}$ regularity for infinity harmonic functions in two dimensions". \emph{Calculus of Variations} 32. pp325--347.} that \emph{every best-Lipschitz function on $\mathbb R^2$ is $C^{1+\varepsilon}$-differentiable}, by methods analogous to the proof of regularity of minimal surfaces.
They start by assuming that the best-Lipschitz function $u$ has a graph which is ``close to a hyperplane" (this time in an $L^\infty$ rather than $L^1$ sense) and show that one can zoom in infinitely far on $u$ to obtain a function that can be approximately well by linear functions.
Therefore, after a rescaling, $u$ can be approximated by linear functions in a suitably strong sense that we obtain a bound on $||du||_{C^\varepsilon_{loc}}$.

It would be very interesting, and a major step to the generalization of the above theorem of Daskalopoulos--Uhlenbeck, to show that in fact, best-Lipschitz functions on arbitrary surfaces are $C^{1+\varepsilon}$-differentiable.
A synthesis of the methods of Evans--Savin with the means by which I will generalize the de Giorgi lemma has the potential to show $C^{1+\varepsilon}$ regularity of best-Lipschitz functions on surfaces.
More generally, I hope to extend many classical results about PDE to the noneuclidean setting.


% --------------- WORKS CITED (10pt FONT) ---------------------

\footnotesize
% \begin{thebibliography}{aa}
%
% \bibitem{amau} Maurer, Andrew B. \emph{\LaTeX \ Template for the National Science Foundation's Graduate Research fellowship Program}.
%
% \end{thebibliography}
\printbibliography
\end{document}

% -------------------------------------------------------------

% -------------------------------------------------------------
