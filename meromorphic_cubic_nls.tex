
\documentclass[12pt]{report}
\usepackage[utf8]{inputenc}
\usepackage[margin=1in]{geometry}
\usepackage{amsmath,amsthm,amssymb}
\usepackage{mathrsfs}

\usepackage{enumitem}
%\usepackage[shortlabels]{enumerate}
\usepackage{tikz-cd}
\usepackage{mathtools}
\usepackage{amsfonts}
\usepackage{amscd}
\usepackage{makeidx}
\usepackage{enumitem}
\title{meromorphically continuing the resolvent}
\author{Aidan Backus}
\date{December 2019}


\newcommand{\NN}{\mathbb{N}}
\newcommand{\ZZ}{\mathbb{Z}}
\newcommand{\QQ}{\mathbb{Q}}
\newcommand{\RR}{\mathbb{R}}
\newcommand{\CC}{\mathbb{C}}
\newcommand{\PP}{\mathbb{P}}
\newcommand{\DD}{\mathbb{D}}

\newcommand{\Torus}{\mathbb{T}}

\newcommand{\AAA}{\mathcal A}
\newcommand{\BB}{\mathcal B}
\newcommand{\HH}{\mathcal H}

\newcommand{\CVect}{\mathbf{Vect}_\CC}
\newcommand{\Grp}{\mathbf{Grp}}
\newcommand{\Open}{\mathbf{Open}}
\newcommand{\Set}{\mathbf{Set}}

\DeclareMathOperator{\atanh}{atanh}
\DeclareMathOperator{\sech}{sech}

\DeclareMathOperator{\coker}{coker}
\DeclareMathOperator{\rank}{rank}

\newcommand{\dbar}{\overline\partial}

\def\Xint#1{\mathchoice
{\XXint\displaystyle\textstyle{#1}}%
{\XXint\textstyle\scriptstyle{#1}}%
{\XXint\scriptstyle\scriptscriptstyle{#1}}%
{\XXint\scriptscriptstyle\scriptscriptstyle{#1}}%
\!\int}
\def\XXint#1#2#3{{\setbox0=\hbox{$#1{#2#3}{\int}$ }
\vcenter{\hbox{$#2#3$ }}\kern-.6\wd0}}
\def\ddashint{\Xint=}
\def\dashint{\Xint-}

\renewcommand{\Re}{\operatorname{Re}}
\renewcommand{\Im}{\operatorname{Im}}
\newcommand{\dfn}[1]{\emph{#1}\index{#1}}

\usepackage{color}
\usepackage{hyperref}
\hypersetup{
    colorlinks=true, % make the links colored
    linkcolor=blue, % color TOC links in blue
    urlcolor=red, % color URLs in red
    linktoc=all % 'all' will create links for everything in the TOC
    %Ning added hyperlinks to the table of contents 6/17/19
}

\theoremstyle{definition}
\newtheorem{theorem}{Theorem}[chapter]
\newtheorem{lemma}[theorem]{Lemma}
\newtheorem{proposition}[theorem]{Proposition}
\newtheorem{corollary}[theorem]{Corollary}
\newtheorem{axiom}[theorem]{Axiom}
\newtheorem{conjecture}[theorem]{Conjecture}
\newtheorem{definition}[theorem]{Definition}
\newtheorem{remark}[theorem]{Remark}
\newtheorem{example}[theorem]{Example}
\newtheorem{exercise}[theorem]{Exercise}
\newtheorem{problem}[theorem]{Problem}

\makeindex

\begin{document}

\maketitle

\tableofcontents

\chapter{Cubic NLS stuff}
We are interested in the equation
$$i\partial_t u + \partial_x^2u/2 + q\delta_0 u + u|u|^2 = 0,$$
the cubic NLS with a small Dirac perturbation ($0 < |q| \ll 1$).
Up to a rescaling, we can take
$$v_q(x) = \sech(|x| + \atanh q)$$
to be the ground state solution to the cubic NLS with perturbation. One has $||v_q||_2^2 = 2(1 - q)$.

\section{Symp geometry}
Let $M$ be a symplectic manifold, $H$ a smooth function on $M$, which we interpret as the Hamiltonian.
Then $dH$ is a $1$-form on $M$ -- if $(x_j)_j$ is a coordinate system on $M$ then
$$dH = \sum_j \frac{\partial H}{\partial x_j}~dx_j.$$
Since $M$ is a symplectic manifold, we have a fixed nongenerate $2$-form $\omega$ on tangent spaces $T_xM$, namely the symplectic form of $M$. So we have an isomorphism $T^*_xM \to T_xM$, $\varphi \mapsto w$, where for every $v \in T_xM$,
$$(\varphi, v) = \omega(v, w).$$
In particular since $dH$ is a $1$-form it can be identified with a vector field $\Xi_H: M \to TM$ such that
$$(dH, v) = \omega(v, \Xi_H).$$
The vector field $\Xi_H$ is known as the Hamiltonian vector field, and the induced flow $\RR \times M \to M$ is known as the Hamiltonian flow.
The point is that the Hamiltonian flow preserves $H$.
The linearization of $\Xi_H$ is a map $TM \to TM$ known as the Hamiltonian map. Hamiltonian maps are covered in Hormander III, 21.5, and more general considerations are in 21.1.

We consider the symplectic structure on the \emph{real} Hilbert space $H^1(\RR \to \CC)$ determined by
$$\omega(u, v) = \Im \int_{-\infty}^\infty u\overline v.$$
Since $u\overline u = |u|^2$ is real, $\omega$ is a closed form, hence symplectic.
Since $H^1(\RR \to \CC)$ is a symplectic vector space, every point looks like the origin so to specify a tangent vector is the same thing as specifying a point.

We have an embedding
\begin{align*}
  H^1(\RR \to \CC) &\to H^1(\RR \to \CC)^2\\
  u &\mapsto (u, \overline u)
\end{align*}
and any linear operator on $H^1(\RR \to \CC)$ gives a linear operator on the subspace $\Delta$ of $H^1(\RR \to \CC)^2$ given by vectors of the form $(u, \overline u)$.

The cubic NLS with perturbation determines a Hamiltonian flow on $H^1(\RR \to \CC)$ with the Hamiltonian
$$H_q(v) = \frac{1}{4}\int_{-\infty}^\infty |\partial_xv(x)|^2 - |v(x)|^4~dx - \frac{1}{2}q|v(0)|^2.$$
Then
$$\Xi_{H_q}(u) = -i \left(-\frac{1}{2}\partial_x^2u - |u|^2u - q\delta_0 u\right).$$
Let $\mathcal F_q$ be the Hamiltonian map of $\Xi_{H_q}$. Writing $U = \begin{bmatrix}1&-i\\1&i\end{bmatrix}/\sqrt 2$, we let $\mathcal H_q/2 = U\mathcal F_qU^*/i$.
Thus $\mathcal H_q$ can be thought of (up to a unitary change of coordinates) the linearization of the cubic NLS.

Now
$$\mathcal H_q = \begin{bmatrix}-\partial^2 + 1 \\&\partial^2 - 1\end{bmatrix} + v^2\begin{bmatrix}-4&-2\\2&4\end{bmatrix}-2q\delta_0\begin{bmatrix}1\\&-1\end{bmatrix}.$$
In particular
$$\mathcal H_0 = \begin{bmatrix}-\partial^2 + 1 \\&\partial^2 - 1\end{bmatrix} + \sech^2 \begin{bmatrix}-4 & -2\\2 & 4\end{bmatrix}.$$
We should be thinking of $\mathcal H_q$ as a perturbation of $\mathcal H_0$.

\section{The resolvent of $\mathcal H_0$}
We want to compute the resolvent $R_0(\lambda) = (\mathcal H_0 - \lambda)^{-1}$.

One has
$$\mathcal H_0^2 = (\partial - 1)^2 + 12 \sech^4 + (\partial^2 - 1)\sech^2 \begin{bmatrix}4 & 2\\2 & 4\end{bmatrix} + \sech^2 \begin{bmatrix}4 & -2\\-2&4\end{bmatrix}(\partial^2 - 1).$$
This is self-adjoint, so its spectrum is contained in $\RR$. In particular the spectrum of $\mathcal H_0$ is only purely real and purely imaginary.

\section{Hamiltonian observables are Fredholm on tori}
As an example of how to show that a linear operator is Fredholm we show that $V - \Delta$ is Fredholm on $\Torus^n$.
This is ``equivalent" to having discrete spectrum, because if $P$ is Fredholm then the proof ``should" also be valid for $P - \lambda$ which hence should have a finite-dimensional eigenspace so none of the dumb stuff associated to continuous spectrum can happen. I think this is discussed in Evans 6.5 but I haven't checked.
The goal is to show that $\mathcal H_0$ is Fredholm. One can then use analytic Fredholm theory to show that the resolvent $R_0$ has a meromorphic continuation to some Riemann surface.

\begin{theorem}
Let $V \in L^2(\Torus^n \to \CC)$, then $P = V - \Delta$ is Fredholm as an operator
$$P: H^2(\Torus^n) \to L^2(\Torus^n).$$
\end{theorem}
\begin{proof}
We must show that $\ker P$ and $\coker P$ are finite-dimensional.

First for $\ker P$. Let the unit ball $B$ of $\ker P$ have the topology of $L^2$. If $u \in B$, then
$$||u||_{H^2}^2 = \sum_{\xi \in \ZZ^n} (1 + |\xi|^2)|\hat u(\xi)|^2 = \int_{\Torus^n} |(1 + \Delta)u|^2 \leq ||u||_{L^2}^2 + ||Vu||_{L^2}^2.$$
Since $V \in L^2$,
$$||u||_{H^2}^2 \leq (1 + ||V||_{L^2}^2)|u||_{L^2}^2.$$
so $B$ embeds continuously in $H^2$, but $H^2$ embeds compactly in $L^2$, so $B$ is homeomorphic to $B$ with the topology of $H^2$, and is compact.
Since $\ker P$ has a compact unit ball, $\dim \ker P$ is finite by the converse to the Heine-Borel theorem.

For $\coker P$, note that $P^* = V^* - \Delta$ meets the same hypotheses as $P$, so $\ker P^*$ is finite-dimensional by the above argument.
Now $P$ is continuous, since
$$||Pu||_{L^2} \leq ||Vu||_{L^2} + ||\Delta u||_{L^2} \leq ||V||_{L^2}(||u||_{L^2} + ||\Delta u||_{L^2}) \leq ||V||_{L^2} ||u||_{H^2}.$$
So by the closed graph theorem, the graph of $P$ is closed and hence the image of $P$ is closed.
Therefore $\coker P = \ker P^*$ which is finite-dimensional.
\end{proof}

Note that the above compact kernel trick works whenever we have an elliptic estimate, i.e.
$$||u||_{H^2} \lesssim ||u||_{L^2} + ||Pu||_{L^2}.$$
An interesting related estimate on functions
$$\{u:u\in H^{s+1},~Pu \in H^s\} \to H^s$$
is
$$||u||_{H^{s+1}} \lesssim ||u||_{H^s} + ||Pu||_{H^s}.$$
Here $s$ depends on the coefficient of the leading term of $P$.

Also note that operators with a parametrix are Fredholm. In fact if $P$ has a parametrix, $Pu = \delta + f$ where $f = O(h^\infty)$ then $f$ is a compact operator and $\delta$ is the identity, so $P$ is invertible modulo the compact operator $f$.

\section{Inverting $\Box$}
Prof Zworski asked me for which function spaces the operator $\Box = \partial_t^2 - \partial_x^2$ is invertible, acting on $C^\infty(\Torus^1 \times [0, 1])$ (periodic in $x$, compact in $t$).
See the energy estimates in Appendix E5 of Zworski-Dyatlov for a hint. TODO Solve me

\section{The Bose invariant}
Whenever we want to solve a second-order linear ODE, say
$$(a\partial^2 + b\partial + c)\phi = 0,$$
we introduce a quantity known as the Bose invariant. This is already useful for the Schrodinger equation with $a = -1$, $b = 0$, $c = V$. Let
$$h(r) = \exp\int_0^r \frac{b}{2a}$$
and let $A$ be the operator $A = a\partial^2 + b\partial + c$ (so the symbol of $A$ is $a\xi^2 + b\xi + c$) so that
$$a(r)^{-1}h(r)A(r, \partial_r)h(r)^{-1} = \partial_r^2 + I(r)$$
where
$$I = \frac{4ac - 2ab' + 2ba' - b^2}{4a^2}$$
is the \emph{Bose invariant}.
Let $\psi = h\phi$, thus
$$(\partial^2 + I)\psi = 0.$$
Thus the Bose invariant allows us to eliminate the first-order term $b\partial$ and the factor $a$.
Thus this change of variables allows us to henceforth assume $a = 1$, $b = 0$.

To see how Bose invariants are affected by a change of variables we introduce the \emph{Schwarz derivative}
$$\{r, y\} = \left(\frac{r''(y)}{r'(y)}\right)' - \frac{1}{2}\left(\frac{r''(y)}{r'(y)}\right)^2.$$
Then if $I$ is the Bose invariant in $r$ and we change variables to $y$,
$$J(y) = (r'(y))^2I(r(y)) + \frac{1}{2}\{r, y\}$$
is the new Bose invariant in $z$.
The map $I \mapsto J$ is known as a \emph{Liouville transform}.

As an example one considers the hypergeometric equation, given in one variable $z$ by the kernel of the operator
$$z(1-z)\partial_z^2 + (c - (a+b+1)z)\partial_z - ab.$$
Its Bose invariant is
$$I(a, b, c; z) = \frac{(1 - a^2 - b^2 + 2ab)z^2 + 2(-2ab+ac+bc-c)z + 2c - c^2}{4z^2(1-z)^2}.$$

\chapter{$\mathcal H_0$ is Fredholm}
\section{Change of coordinates}
The operator
$$\mathcal H_0 = \begin{bmatrix}-\partial^2 + 1 \\&\partial^2 - 1\end{bmatrix} + \sech^2 \begin{bmatrix}-4 & -2\\2 & 4\end{bmatrix}$$
acts on functions $\RR \to \CC^2$. This is annoying so we use $\tanh$ to change coordinates from $\RR$ to $(0, 1)$ which is bounded.

Let $y = \tanh x$, let $L = \begin{bmatrix}1 & \\ &-1\end{bmatrix}$, then
$$\mathcal{\tilde H}_0 = \left(1 - (1 - y^2)^2 \partial_y^2\right)L + (1 - y^2)\begin{bmatrix}-4 & -2\\2 & 4\end{bmatrix}.$$
This acts on functions $(-1, 1) \to \CC^2$.
One hopes to show that $\mathcal R_0(\lambda) = \mathcal{\tilde H}_0 - \lambda^2$ is Fredholm on some open set, say $\Im \lambda \gg 1$.

We also have
$$\frac{1}{\sech^2(\atanh y + \atanh q)} = \frac{1}{(1 - q^2)(1 - y^2)} + \frac{2qy}{(1 - q^2)(1 - y^2)} + \frac{q^2y^2}{(1 - q^2)(1 - y^2)}.$$
Therefore
$$\mathcal{\tilde H}_q = \left(1 - (1 - y^2)^2 \partial_y^2\right)L + \tilde v_q(y) \begin{bmatrix}-4 & -2\\2 & 4\end{bmatrix} - 2q\delta_0 L$$
where
$$v_q(x) = \left(\frac{1}{(1 - q^2)(1 - y^2)} + \frac{2qy}{(1 - q^2)(1 - y^2)} + \frac{q^2y^2}{(1 - q^2)(1 - y^2)}\right)^{-1}.$$
The operator $2q\delta_0L$ is ``compact" (since its image consists of rescaled delta functions) and can be thought of as having ``norm" $2|q|$ even though it doesn't literally send $L^2$ to itself.
Here I'm thinking of multiplication by $\delta_0$ as having ``norm $1$" on some suitable space of distributions. No idea if this is legal.

The function $v_q - v_0$ is ``small" in $L^\infty$ (hence as an operator on $L^2$). In fact $||v_q - v_0||_{L^\infty} \lesssim |q|$ if $|q|$ is small. To see this note that
$$||\partial \tilde v_0||_{L^\infty} < 1$$
so
\begin{align*}|\sech^2(\atanh y + \atanh q) - \sech^2(\atanh y)| &= |\sech^2(\atanh y + q + O(q^3)) - \sech^2(\atanh y)| \\&< |q + O(q^3)| \lesssim |q|.\end{align*}
Thus one can reasonably think of $\mathcal{\tilde H}_q$ as a perturbation of $\mathcal{\tilde H}_0$ and try to use $\mathcal R_0$ to meromorphically continue $\mathcal R_q$.
In fact, by Neumann series methods, we prove that since
$$\mathcal R_q(\lambda) = (\mathcal{\tilde H}_q - \mathcal{\tilde H}_0 + \mathcal{\tilde H}_0 - \lambda^2)^{-1} = (\mathcal{\tilde H}_0 - \lambda^2 + E(q))^{-1}$$
where
$$E_q = (\tilde v_q - \tilde v_0)\begin{bmatrix}-4 & -2\\2 & 4\end{bmatrix} - 2q\delta_0L$$
and $||E_q|| \lesssim |q|$, so if $|q|$ is so small that
$$||E_q|| < \frac{1}{||\mathcal R_0(\lambda)||},$$
then
$$||\mathcal R_q(\lambda)|| \leq \frac{||\mathcal R_0(\lambda)||}{1 - ||E_q||}.$$
Thus we are really interested in $||\mathcal R_0(\lambda)||$!

\section{Meromorphic continuation of $\mathcal{\tilde H}_0$}
We adopt the convention
$$||(u, v)||_X = ||u||_X + ||v||_X$$
for all spaces $X$.

This can probably be done algebraically using ODE techniques. We know that $\mathcal{\tilde H}_0$ is an ordinary differential opreator of order $2$ so its kernel is at most $2$-dimensoinal.
One can check the cokernel using the adjoint trick or something, but we probably need the Baire category thm for that.
In fact $\mathcal{\tilde H}_0^*$ meets the same hypotheses as $\mathcal{\tilde H}_0$ and is clearly continuous $H^2 \to L^2$ so the usual closed graph thm trick goes through.

\section{Another way of arguing for Fredholm}
In this section consider the free Hamiltonian
$$H_0 = (\Delta - 1)L.$$
Let $H = H_0 + V$ such that:
\begin{enumerate}
  \item $V$ is an exponentially-decaying, bounded, matrix-valued potential (so $V \in L^\infty(\RR \to \CC^{2 \times 2})$).
  \item $H^2$ is self-adjoint, so its spectrum is contained in $\RR \cup i\RR$.
\end{enumerate}
For example we could take
$$V(x) = \sech^2(|x|)A$$
where
$$A = \begin{bmatrix}-4 & -2\\ 2 & 4\end{bmatrix}.$$
We can worry about $q$-perturbations later.

Let $z$ be a spectral parameter.
Then
$$(H_0 - z)(1 - R_0(z)V) = H_0 - z + (H_0 - z)R_0(z)V = H_0 - z + V = H - z.$$
Therefore if $R_V(z) = (H - z)^{-1}$ we have
$$R_V(z) = (1 - R_0(z)V)^{-1}R_0(z).$$
Thus we are interested in when $R_0(z)V$ is invertible.
Let $S$ be the spectrum of $R_0(z)$, so
$$S = (-\infty, -1) \cup (1, \infty).$$

First note that $||V||_{L^2 \to L^2} = ||V||_{L^\infty}$. On the other hand
$$||R_0(z)||_{L^2 \to L^2} = \frac{1}{d(z, S \setminus 0)} \leq \frac{1}{d(z, \RR)} = \frac{1}{|\Im z|}.$$
Thus if $|\Im z|^{-1} < ||V||_{L^\infty}$, $1 - R_0(z)V$ is invertible and hence $R_V(z)$ exists.
Thus in particular we have convergence of the resolvent in the first quadrant $\Im z \gtrsim_V 1$, $\Re z > 0$, as well as in the second quadrant, $\Im z \gtrsim_V 1$, $\Re z < 0$.
There we have the estimate
$$||R_V(z)||_{L^2 \to L^2} \leq \frac{1}{\Im z}.$$

Moreover $R_0(z)V$ is compact. To see this note that $R_0(z)$ is bounded, and so we just have to show that $V$ is compact.
To do this we approximate $V$ by an operator $V_n = \pi_nV$ where $\pi_n$ is a projection onto the space $X_n$ simple functions which are supported on $[-n, n]$ and constant on $[m/n, (m+1)/n]$ for each $|m| < n^2$.
Since $\dim X_n = n^2$, $\rank V_n = n^2$. Moreover
$$||V - V_n||_{L^\infty} \leq ||1 - \pi_n||_{L^2 \to L^2} ||V||_{L^\infty}$$
and $Vu$ can be approximated in $L^2$ by an exponentially decaying, continuous function; for exponentially decaying, continuous $f$, let us show $\pi_nf \to f$ uniformly in $f$ (which then implies $||1 - \pi_n|| \to 0$).
Say $f(x) = O(e^{-\gamma|x|})$, then
$$||\pi_nf - f||_{L^2} \leq ||\pi_nf - f|[-n, n]||_{L^2} + ||f||_{L^2(\RR \setminus [-n, n])}$$
but
$$||f||_{L^2(\RR \setminus [-n, n])} \leq 2||f||_{L^2}e^{-n\gamma}$$
provided $n$ is large enough, by the pigeonhole principle.
Moreover, by Riemann integrability of $|f|^2$, $\pi_nf \to f$ uniformly on compact sets, hence in $L^2$. Therefore
$$||\pi_nf - f||_{L^2} \leq o_n(1)||f||_{L^2} + ||f||_{L^2}e^{-n\gamma}$$
which vanishes. Thus $V_n \to V$ in the operator norm of $L^2 \to L^2$.

Since $R_0(z)V$ is compact, $(1 - R_0(z)V)^{-1}$ is Fredholm even if $\Im z$ is not large.
Thus it admits an analytic continuation to the whole complex plane, except for a discrete set $P$ of poles of finite rank and the set $S$.
Moreover $R_0(z)$ is holomorphic away from $0$, so $R_V(z)$ extends to $\CC$.

It would be nice if we had a bound on $R_V(\lambda)$ for $\Im \lambda < 0$.

We want to compute the monodromy of $R_V(z)$.


\newpage
\printindex

\end{document}
