
\documentclass[12pt]{report}
\usepackage[utf8]{inputenc}
\usepackage[margin=1in]{geometry}
\usepackage{amsmath,amsthm,amssymb}
\usepackage{mathrsfs}

\usepackage{enumitem}
%\usepackage[shortlabels]{enumerate}
\usepackage{tikz-cd}
\usepackage{mathtools}
\usepackage{amsfonts}
\usepackage{amscd}
\usepackage{makeidx}
\usepackage{enumitem}
\title{meromorphically continuing the resolvent}
\author{Aidan Backus}
\date{December 2019}


\newcommand{\NN}{\mathbb{N}}
\newcommand{\ZZ}{\mathbb{Z}}
\newcommand{\QQ}{\mathbb{Q}}
\newcommand{\RR}{\mathbb{R}}
\newcommand{\CC}{\mathbb{C}}
\newcommand{\PP}{\mathbb{P}}
\newcommand{\DD}{\mathbb{D}}

\newcommand{\Torus}{\mathbb{T}}

\newcommand{\AAA}{\mathcal A}
\newcommand{\BB}{\mathcal B}
\newcommand{\HH}{\mathcal H}

\newcommand{\CVect}{\mathbf{Vect}_\CC}
\newcommand{\Grp}{\mathbf{Grp}}
\newcommand{\Open}{\mathbf{Open}}
\newcommand{\Set}{\mathbf{Set}}

\DeclareMathOperator{\atanh}{atanh}
\DeclareMathOperator{\sech}{sech}

\DeclareMathOperator{\coker}{coker}
\DeclareMathOperator{\diag}{diag}
\DeclareMathOperator{\Gal}{Gal}
\DeclareMathOperator{\id}{id}
\DeclareMathOperator{\sgn}{sgn}
\DeclareMathOperator{\rank}{rank}

\newcommand{\dbar}{\overline\partial}

\def\Xint#1{\mathchoice
{\XXint\displaystyle\textstyle{#1}}%
{\XXint\textstyle\scriptstyle{#1}}%
{\XXint\scriptstyle\scriptscriptstyle{#1}}%
{\XXint\scriptscriptstyle\scriptscriptstyle{#1}}%
\!\int}
\def\XXint#1#2#3{{\setbox0=\hbox{$#1{#2#3}{\int}$ }
\vcenter{\hbox{$#2#3$ }}\kern-.6\wd0}}
\def\ddashint{\Xint=}
\def\dashint{\Xint-}

\renewcommand{\Re}{\operatorname{Re}}
\renewcommand{\Im}{\operatorname{Im}}
\newcommand{\dfn}[1]{\emph{#1}\index{#1}}

\usepackage{color}
\usepackage{hyperref}
\hypersetup{
    colorlinks=true, % make the links colored
    linkcolor=blue, % color TOC links in blue
    urlcolor=red, % color URLs in red
    linktoc=all % 'all' will create links for everything in the TOC
    %Ning added hyperlinks to the table of contents 6/17/19
}

\theoremstyle{definition}
\newtheorem{theorem}{Theorem}[chapter]
\newtheorem{lemma}[theorem]{Lemma}
\newtheorem{proposition}[theorem]{Proposition}
\newtheorem{corollary}[theorem]{Corollary}
\newtheorem{axiom}[theorem]{Axiom}
\newtheorem{conjecture}[theorem]{Conjecture}
\newtheorem{definition}[theorem]{Definition}
\newtheorem{remark}[theorem]{Remark}
\newtheorem{example}[theorem]{Example}
\newtheorem{exercise}[theorem]{Exercise}
\newtheorem{problem}[theorem]{Problem}

\makeindex

\begin{document}

\maketitle

\tableofcontents

\chapter{Cubic NLS stuff}
We are interested in the equation
$$i\partial_t u + \partial_x^2u/2 + q\delta_0 u + u|u|^2 = 0,$$
the cubic NLS with a small Dirac perturbation ($0 < |q| \ll 1$).
Up to a rescaling, we can take
$$v_q(x) = \sech(|x| + \atanh q)$$
to be the ground state solution to the cubic NLS with perturbation. One has $||v_q||_2^2 = 2(1 - q)$.

\section{Symp geometry}
Let $M$ be a symplectic manifold, $H$ a smooth function on $M$, which we interpret as the Hamiltonian.
Then $dH$ is a $1$-form on $M$ -- if $(x_j)_j$ is a coordinate system on $M$ then
$$dH = \sum_j \frac{\partial H}{\partial x_j}~dx_j.$$
Since $M$ is a symplectic manifold, we have a fixed nongenerate $2$-form $\omega$ on tangent spaces $T_xM$, namely the symplectic form of $M$. So we have an isomorphism $T^*_xM \to T_xM$, $\varphi \mapsto w$, where for every $v \in T_xM$,
$$(\varphi, v) = \omega(v, w).$$
In particular since $dH$ is a $1$-form it can be identified with a vector field $\Xi_H: M \to TM$ such that
$$(dH, v) = \omega(v, \Xi_H).$$
The vector field $\Xi_H$ is known as the Hamiltonian vector field, and the induced flow $\RR \times M \to M$ is known as the Hamiltonian flow.
The point is that the Hamiltonian flow preserves $H$.
The linearization of $\Xi_H$ is a map $TM \to TM$ known as the Hamiltonian map. Hamiltonian maps are covered in Hormander III, 21.5, and more general considerations are in 21.1.

We consider the symplectic structure on the \emph{real} Hilbert space $H^1(\RR \to \CC)$ determined by
$$\omega(u, v) = \Im \int_{-\infty}^\infty u\overline v.$$
Since $u\overline u = |u|^2$ is real, $\omega$ is a closed form, hence symplectic.
Since $H^1(\RR \to \CC)$ is a symplectic vector space, every point looks like the origin so to specify a tangent vector is the same thing as specifying a point.

We have an embedding
\begin{align*}
  H^1(\RR \to \CC) &\to H^1(\RR \to \CC)^2\\
  u &\mapsto (u, \overline u)
\end{align*}
and any linear operator on $H^1(\RR \to \CC)$ gives a linear operator on the subspace $\Delta$ of $H^1(\RR \to \CC)^2$ given by vectors of the form $(u, \overline u)$.

The cubic NLS with perturbation determines a Hamiltonian flow on $H^1(\RR \to \CC)$ with the Hamiltonian
$$H_q(v) = \frac{1}{4}\int_{-\infty}^\infty |\partial_xv(x)|^2 - |v(x)|^4~dx - \frac{1}{2}q|v(0)|^2.$$
Then
$$\Xi_{H_q}(u) = -i \left(-\frac{1}{2}\partial_x^2u - |u|^2u - q\delta_0 u\right).$$
Let $\mathcal F_q$ be the Hamiltonian map of $\Xi_{H_q}$. Writing $U = \begin{bmatrix}1&-i\\1&i\end{bmatrix}/\sqrt 2$, we let $\mathcal H_q/2 = U\mathcal F_qU^*/i$.
Thus $\mathcal H_q$ can be thought of (up to a unitary change of coordinates) the linearization of the cubic NLS.

Now
$$\mathcal H_q = \begin{bmatrix}-\partial^2 + 1 \\&\partial^2 - 1\end{bmatrix} + v^2\begin{bmatrix}-4&-2\\2&4\end{bmatrix}-2q\delta_0\begin{bmatrix}1\\&-1\end{bmatrix}.$$
In particular
$$\mathcal H_0 = \begin{bmatrix}-\partial^2 + 1 \\&\partial^2 - 1\end{bmatrix} + \sech^2 \begin{bmatrix}-4 & -2\\2 & 4\end{bmatrix}.$$
We should be thinking of $\mathcal H_q$ as a perturbation of $\mathcal H_0$.

\section{The resolvent of $\mathcal H_0$}
We want to compute the resolvent $R_0(\lambda) = (\mathcal H_0 - \lambda)^{-1}$.

One has
$$\mathcal H_0^2 = (\partial - 1)^2 + 12 \sech^4 + (\partial^2 - 1)\sech^2 \begin{bmatrix}4 & 2\\2 & 4\end{bmatrix} + \sech^2 \begin{bmatrix}4 & -2\\-2&4\end{bmatrix}(\partial^2 - 1).$$
This is self-adjoint, so its spectrum is contained in $\RR$. In particular the spectrum of $\mathcal H_0$ is only purely real and purely imaginary.

\section{Hamiltonian observables are Fredholm on tori}
As an example of how to show that a linear operator is Fredholm we show that $V - \Delta$ is Fredholm on $\Torus^n$.
This is ``equivalent" to having discrete spectrum, because if $P$ is Fredholm then the proof ``should" also be valid for $P - \lambda$ which hence should have a finite-dimensional eigenspace so none of the dumb stuff associated to continuous spectrum can happen. I think this is discussed in Evans 6.5 but I haven't checked.
The goal is to show that $\mathcal H_0$ is Fredholm. One can then use analytic Fredholm theory to show that the resolvent $R_0$ has a meromorphic continuation to some Riemann surface.

\begin{theorem}
Let $V \in L^2(\Torus^n \to \CC)$, then $P = V - \Delta$ is Fredholm as an operator
$$P: H^2(\Torus^n) \to L^2(\Torus^n).$$
\end{theorem}
\begin{proof}
We must show that $\ker P$ and $\coker P$ are finite-dimensional.

First for $\ker P$. Let the unit ball $B$ of $\ker P$ have the topology of $L^2$. If $u \in B$, then
$$||u||_{H^2}^2 = \sum_{\xi \in \ZZ^n} (1 + |\xi|^2)|\hat u(\xi)|^2 = \int_{\Torus^n} |(1 + \Delta)u|^2 \leq ||u||_{L^2}^2 + ||Vu||_{L^2}^2.$$
Since $V \in L^2$,
$$||u||_{H^2}^2 \leq (1 + ||V||_{L^2}^2)|u||_{L^2}^2.$$
so $B$ embeds continuously in $H^2$, but $H^2$ embeds compactly in $L^2$, so $B$ is homeomorphic to $B$ with the topology of $H^2$, and is compact.
Since $\ker P$ has a compact unit ball, $\dim \ker P$ is finite by the converse to the Heine-Borel theorem.

For $\coker P$, note that $P^* = V^* - \Delta$ meets the same hypotheses as $P$, so $\ker P^*$ is finite-dimensional by the above argument.
Now $P$ is continuous, since
$$||Pu||_{L^2} \leq ||Vu||_{L^2} + ||\Delta u||_{L^2} \leq ||V||_{L^2}(||u||_{L^2} + ||\Delta u||_{L^2}) \leq ||V||_{L^2} ||u||_{H^2}.$$
So by the closed graph theorem, the graph of $P$ is closed and hence the image of $P$ is closed.
Therefore $\coker P = \ker P^*$ which is finite-dimensional.
\end{proof}

Note that the above compact kernel trick works whenever we have an elliptic estimate, i.e.
$$||u||_{H^2} \lesssim ||u||_{L^2} + ||Pu||_{L^2}.$$
An interesting related estimate on functions
$$\{u:u\in H^{s+1},~Pu \in H^s\} \to H^s$$
is
$$||u||_{H^{s+1}} \lesssim ||u||_{H^s} + ||Pu||_{H^s}.$$
Here $s$ depends on the coefficient of the leading term of $P$.

Also note that operators with a parametrix are Fredholm. In fact if $P$ has a parametrix, $Pu = \delta + f$ where $f = O(h^\infty)$ then $f$ is a compact operator and $\delta$ is the identity, so $P$ is invertible modulo the compact operator $f$.

\section{Inverting $\Box$}
Prof Zworski asked me for which function spaces the operator $\Box = \partial_t^2 - \partial_x^2$ is invertible, acting on $C^\infty(\Torus^1 \times [0, 1])$ (periodic in $x$, compact in $t$).
See the energy estimates in Appendix E5 of Zworski-Dyatlov for a hint. TODO Solve me

\section{The Bose invariant}
Whenever we want to solve a second-order linear ODE, say
$$(a\partial^2 + b\partial + c)\phi = 0,$$
we introduce a quantity known as the Bose invariant. This is already useful for the Schrodinger equation with $a = -1$, $b = 0$, $c = V$. Let
$$h(r) = \exp\int_0^r \frac{b}{2a}$$
and let $A$ be the operator $A = a\partial^2 + b\partial + c$ (so the symbol of $A$ is $a\xi^2 + b\xi + c$) so that
$$a(r)^{-1}h(r)A(r, \partial_r)h(r)^{-1} = \partial_r^2 + I(r)$$
where
$$I = \frac{4ac - 2ab' + 2ba' - b^2}{4a^2}$$
is the \emph{Bose invariant}.
Let $\psi = h\phi$, thus
$$(\partial^2 + I)\psi = 0.$$
Thus the Bose invariant allows us to eliminate the first-order term $b\partial$ and the factor $a$.
Thus this change of variables allows us to henceforth assume $a = 1$, $b = 0$.

To see how Bose invariants are affected by a change of variables we introduce the \emph{Schwarz derivative}
$$\{r, y\} = \left(\frac{r''(y)}{r'(y)}\right)' - \frac{1}{2}\left(\frac{r''(y)}{r'(y)}\right)^2.$$
Then if $I$ is the Bose invariant in $r$ and we change variables to $y$,
$$J(y) = (r'(y))^2I(r(y)) + \frac{1}{2}\{r, y\}$$
is the new Bose invariant in $z$.
The map $I \mapsto J$ is known as a \emph{Liouville transform}.

As an example one considers the hypergeometric equation, given in one variable $z$ by the kernel of the operator
$$z(1-z)\partial_z^2 + (c - (a+b+1)z)\partial_z - ab.$$
Its Bose invariant is
$$I(a, b, c; z) = \frac{(1 - a^2 - b^2 + 2ab)z^2 + 2(-2ab+ac+bc-c)z + 2c - c^2}{4z^2(1-z)^2}.$$

\chapter{$\mathcal H_0$ is Fredholm}
\section{Change of coordinates}
The operator
$$\mathcal H_0 = \begin{bmatrix}-\partial^2 + 1 \\&\partial^2 - 1\end{bmatrix} + \sech^2 \begin{bmatrix}-4 & -2\\2 & 4\end{bmatrix}$$
acts on functions $\RR \to \CC^2$. This is annoying so we use $\tanh$ to change coordinates from $\RR$ to $(0, 1)$ which is bounded.

Let $y = \tanh x$, let $L = \begin{bmatrix}1 & \\ &-1\end{bmatrix}$, then
$$\mathcal{\tilde H}_0 = \left(1 - (1 - y^2)^2 \partial_y^2\right)L + (1 - y^2)\begin{bmatrix}-4 & -2\\2 & 4\end{bmatrix}.$$
This acts on functions $(-1, 1) \to \CC^2$.
One hopes to show that $\mathcal R_0(\lambda) = \mathcal{\tilde H}_0 - \lambda^2$ is Fredholm on some open set, say $\Im \lambda \gg 1$.

We also have
$$\frac{1}{\sech^2(\atanh y + \atanh q)} = \frac{1}{(1 - q^2)(1 - y^2)} + \frac{2qy}{(1 - q^2)(1 - y^2)} + \frac{q^2y^2}{(1 - q^2)(1 - y^2)}.$$
Therefore
$$\mathcal{\tilde H}_q = \left(1 - (1 - y^2)^2 \partial_y^2\right)L + \tilde v_q(y) \begin{bmatrix}-4 & -2\\2 & 4\end{bmatrix} - 2q\delta_0 L$$
where
$$v_q(x) = \left(\frac{1}{(1 - q^2)(1 - y^2)} + \frac{2qy}{(1 - q^2)(1 - y^2)} + \frac{q^2y^2}{(1 - q^2)(1 - y^2)}\right)^{-1}.$$
The operator $2q\delta_0L$ is ``compact" (since its image consists of rescaled delta functions) and can be thought of as having ``norm" $2|q|$ even though it doesn't literally send $L^2$ to itself.
Here I'm thinking of multiplication by $\delta_0$ as having ``norm $1$" on some suitable space of distributions. No idea if this is legal.

The function $v_q - v_0$ is ``small" in $L^\infty$ (hence as an operator on $L^2$). In fact $||v_q - v_0||_{L^\infty} \lesssim |q|$ if $|q|$ is small. To see this note that
$$||\partial \tilde v_0||_{L^\infty} < 1$$
so
\begin{align*}|\sech^2(\atanh y + \atanh q) - \sech^2(\atanh y)| &= |\sech^2(\atanh y + q + O(q^3)) - \sech^2(\atanh y)| \\&< |q + O(q^3)| \lesssim |q|.\end{align*}
Thus one can reasonably think of $\mathcal{\tilde H}_q$ as a perturbation of $\mathcal{\tilde H}_0$ and try to use $\mathcal R_0$ to meromorphically continue $\mathcal R_q$.
In fact, by Neumann series methods, we prove that since
$$\mathcal R_q(\lambda) = (\mathcal{\tilde H}_q - \mathcal{\tilde H}_0 + \mathcal{\tilde H}_0 - \lambda^2)^{-1} = (\mathcal{\tilde H}_0 - \lambda^2 + E(q))^{-1}$$
where
$$E_q = (\tilde v_q - \tilde v_0)\begin{bmatrix}-4 & -2\\2 & 4\end{bmatrix} - 2q\delta_0L$$
and $||E_q|| \lesssim |q|$, so if $|q|$ is so small that
$$||E_q|| < \frac{1}{||\mathcal R_0(\lambda)||},$$
then
$$||\mathcal R_q(\lambda)|| \leq \frac{||\mathcal R_0(\lambda)||}{1 - ||E_q||}.$$
Thus we are really interested in $||\mathcal R_0(\lambda)||$!

\section{Meromorphic continuation of $\mathcal{\tilde H}_0$}
We adopt the convention
$$||(u, v)||_X = ||u||_X + ||v||_X$$
for all spaces $X$.

This can probably be done algebraically using ODE techniques. We know that $\mathcal{\tilde H}_0$ is an ordinary differential opreator of order $2$ so its kernel is at most $2$-dimensoinal.
One can check the cokernel using the adjoint trick or something, but we probably need the Baire category thm for that.
In fact $\mathcal{\tilde H}_0^*$ meets the same hypotheses as $\mathcal{\tilde H}_0$ and is clearly continuous $H^2 \to L^2$ so the usual closed graph thm trick goes through.

\section{Another way of arguing for Fredholm}
In this section consider the free Hamiltonian
$$H_0 = (\Delta - 1)L.$$
Let $H = H_0 + V$ such that:
\begin{enumerate}
  \item $V$ is an exponentially-decaying, bounded, matrix-valued potential (so $V \in L^\infty(\RR \to \CC^{2 \times 2})$).
  \item $H^2$ is self-adjoint, so its spectrum is contained in $\RR \cup i\RR$.
\end{enumerate}
For example we could take
$$V(x) = \sech^2(|x|)A$$
where
$$A = \begin{bmatrix}-4 & -2\\ 2 & 4\end{bmatrix}.$$
We can worry about $q$-perturbations later.

Let $\lambda^2$ be a spectral parameter.
Then
$$(H_0 - \lambda^2)(1 - R_0(\lambda)V) = H_0 - \lambda^2 + (H_0 - \lambda^2)R_0(\lambda)V = H_0 - \lambda + V = H - \lambda.$$
Therefore if $R_V(\lambda) = (H - \lambda^2)^{-1}$ we have
$$R_V(\lambda) = (1 - R_0(\lambda)V)^{-1}R_0(\lambda).$$
Thus we are interested in when $1 - R_0(\lambda)V$ is invertible.
Let $S$ be the spectrum of $H_0$, so
$$S = (-\infty, -1) \cup (1, \infty).$$

First note that $||V||_{L^2 \to L^2} = ||V||_{L^\infty}$. On the other hand
$$||R_0(\lambda)||_{L^2 \to L^2} = \frac{1}{d(\lambda^2, S \setminus 0)} \leq \frac{1}{d(\lambda^2, \RR)} = \frac{1}{2 \Re \lambda \Im \lambda}.$$
Thus if $\Re \lambda \Im \lambda \gtrsim_V 1$, and $\lambda$ is in Quadrant I then $1 - R_0(\lambda)V$ is invertible and hence $R_V(\lambda)$ exists.
Thus in particular we have convergence of the resolvent in Quadrant I, where we have the estimate
$$||R_V(\lambda)||_{L^2 \to L^2} \leq \frac{1}{2 \Re \lambda \Im \lambda}.$$

Moreover $R_0(\lambda)V$ is compact. To see this note that $R_0(\lambda)$ is bounded, and so we just have to show that $V$ is compact.
To do this we approximate $V$ by an operator $V_n = \pi_nV$ where $\pi_n$ is a projection onto the space $X_n$ simple functions which are supported on $[-n, n]$ and constant on $[m/n, (m+1)/n]$ for each $|m| < n^2$.
Since $\dim X_n = n^2$, $\rank V_n = n^2$. Moreover
$$||V - V_n||_{L^\infty} \leq ||1 - \pi_n||_{L^2 \to L^2} ||V||_{L^\infty}$$
and $Vu$ can be approximated in $L^2$ by an exponentially decaying, continuous function; for exponentially decaying, continuous $f$, let us show $\pi_nf \to f$ uniformly in $f$ (which then implies $||1 - \pi_n|| \to 0$).
Say $f(x) = O(e^{-\gamma|x|})$, then
$$||\pi_nf - f||_{L^2} \leq ||\pi_nf - f|[-n, n]||_{L^2} + ||f||_{L^2(\RR \setminus [-n, n])}$$
but
$$||f||_{L^2(\RR \setminus [-n, n])} \leq 2||f||_{L^2}e^{-n\gamma}$$
provided $n$ is large enough, by the pigeonhole principle.
Moreover, by Riemann integrability of $|f|^2$, $\pi_nf \to f$ uniformly on compact sets, hence in $L^2$. Therefore
$$||\pi_nf - f||_{L^2} \leq o_n(1)||f||_{L^2} + ||f||_{L^2}e^{-n\gamma}$$
which vanishes. Thus $V_n \to V$ in the operator norm of $L^2 \to L^2$.

Since $R_0(\lambda)V$ is compact, $1 - R_0(\lambda)V$ is Fredholm even if $\Re \lambda \Im \lambda$ is not large, as long as we're in Quadrant I.
So by analytic Fredholm theory its inverse $(1 - R_0(\lambda))^{-1}$ is meromorphic on Quadrant I.
This implies that $R_V(\lambda)$ is defined for $\lambda$ in Quadrant I.

We want bounds on $R_V(\lambda)$ outside of Quadrant I. This might not be possible but consider Haoren's paper that bounds the exponentially weighted resolvent $VR_V(\lambda)V$ instead.

Froese's paper does the same thing, but uses symmetry in the resolvent $R_0$.
For example, one might try to look into what happens to $R_0(-\lambda)$ or $R_0(\overline\lambda)$ and apply
$$R_V(\lambda) = (1 - R_0(\lambda))^{-1}R_0(\lambda).$$
To do this we need to compute the Green function of $R_0(\lambda)$. For the operator $-\Delta - k^2$, the green function is
$$G_0(x, y, k) = \frac{i}{2k}e^{ik|x-y|}.$$
Thus for the operator $(-\Delta - k^2)L$, the resolvent is $LG_0(k)$.
This allows us to not worry about the $L$.
Take $k^2 = \lambda^2 - 1$, so $k = \sqrt{\lambda^2 - 1}$. Here $\sqrt\cdot$ was chosen to be holomorphic away from the positive real axis (so $k$ is holomorphic in $\lambda$ on the upper half plane).
Then we have
$$R_0(x, y, \lambda) = L\frac{i}{2\sqrt{\lambda^2 - 1}}e^{i\sqrt{\lambda^2 - 1}|x-y|}.$$
Now to play around with branch cuts.


\chapter{The Riemann surface}
\section{Constructing the Riemann surface}
aka Aidan pretends to know algebraic geometry.

Suppose that we want to construct the Riemann surface of a vector-valued holomorphic function $F: Y \to \CC^n$ where $Y \subseteq \CC$ is an open set.
This works as long as we can find a polynomial $\widetilde F: \CC^{n+1} \to \CC$ such that
$$\widetilde F(z, \vec w) = 0$$
exactly if $F(z) = \vec w$. In that case, I think we're justified in taking the system of equations $\widetilde F(z, \vec w) = 0$ to define a subvariety $X$ of $\CC^{n+1}$.
Since this is a system of $n$ equations and $n + 1$ unknowns, ``locally linearizing" or something should imply that if the equations are ``independent" (which should happen if $F$ is nontrivial) that $X$ is an algebraic curve, and if $X$ is nonsingular that means that $X$ is a Riemann surface.
We can view $X$ as the Riemann surface of $F$ by defining its analytic continuation $\overline F: X \to \CC^n$ by $F(z, \vec w) = \vec w$.
This is actually an analytic continuation after we choose an inclusion map $\iota: Y \to X$, which is the same thing as choosing a branch of $F$, since then $\iota(z) = (z, \vec w)$ for some $\vec w$ such that $F(z) = \vec w$, and
$$F(z) = \vec w = \overline F(\iota(z)).$$

Now we make the above rambling rigorous.

If $F$ is a holomorphic map, we let $\nabla F$ denote the total holomorphic derivative of $F$.
I'm pretty sure the following lemma is well-known but I couldn't find a good reference.
\begin{definition}
Let $X,Y$ be complex manifolds.
A \dfn{holomorphic submersion} is a holomorphic map $F: X \to Y$ if $\nabla F$ is surjective on holomorphic tangent spaces.
\end{definition}
The criterion for being a holomorphic submersion says that
$$\rank \nabla F = \dim Y$$
where all dimensions are computed over $\CC$.
\begin{lemma}
Let $U \subseteq \CC^n$ be an open set, $Y$ a complex manifold of (complex) dimension $k$, $k \leq n$, $y \in Y$, and $F: U \to Y$ a holomorphic submersion.
Then $F^{-1}(y)$ is a complex submanifold of $\CC^n$ of dimension $n - k$.
\end{lemma}
\begin{proof}
Since $F$ is a submersion, there is a permutation $\sigma$ of $\{1, \dots, n\}$ such that the columns $\sigma(k+1), \dots, \sigma(n)$ of $\nabla F$ are linearly independent.
Let $G: U \to Y \times \CC^{n - k}$ be the map
$$G(z) = (z_{\sigma(1)}, \dots, z_{\sigma(k)}, F(z)).$$
Then $\nabla G$ has full rank and maps between vector spaces of the same dimension, so $\nabla G$ is an isomorphism.
So by the holomorphic inverse function theorem, $G$ is locally invertible, and $G^{-1}$ is holomorphic.

Locally, let
$$f(z)_j = G^{-1}(z, y)_{\sigma(n-j)},$$
$j \in \{0, \dots, n - k + 1\}$, $z \in \CC^{n - k}$.
Then $f$ is holomorphic and locally the graph of $f$ is $X = F^{-1}(y)$.
That is, for any $x_0 \in X$ we can find a $U \ni x_0$ such that $X \cap U = \{(z, f(z)): z \in U'\}$ for some $U'$.
But then the map $z \mapsto (z, f(z))$, $U \to U'$, is a holomorphic chart, and the transition maps are just the identity on $\CC^{n-k}$, so $X$ is a complex manifold of dimension $n - k$.
\end{proof}
Now if $f: U \to \CC^m$ is a holomorphic map and $F(z, f(z)) = 0$ for all $z$, we can define the \emph{Riemann surface} of $f$ to be the manifold $X = F^{-1}(0)$. Then the map $\pi: X \to \CC^m$, $\pi(z, w) = w$, is an analytic continuation of $f$ after one chooses an embedding $U \to X$.

There's another definition of the Riemann surface of a holomorphic map, namely that of its maximal analytic continuation.
Fix $f: U \to \CC^m$.
Define a category $A(f)$, whose objects are commutative diagrams of holomorphic maps
$$\begin{tikzcd}
X \arrow[r,"\pi"] & \CC^m\\
U \arrow[u,"\iota"] \arrow[ur,"f"]
\end{tikzcd}$$
where $\iota$ is injective and $X$ is a connected Riemann surface.
We call the above diagram, or equivalently the tuple, $(X_1, \pi_1, \iota_1)$, an analytic continuation of $f$.
A morphism $\psi: (X_1, \pi_1, \iota_1) \to (X_2, \pi_2, \iota_2)$ in $C(f)$ is a commutative diagram of holomorphic maps
$$\begin{tikzcd}
X_2 \arrow[r,"\pi_2"] & \CC^m\\
X_1 \arrow[u,"\psi"] \arrow[r,"\pi_1"] & \CC^m \arrow[u]\\
U \arrow[u,"\iota_1"] \arrow[bend left=60,uu,"\iota_2"] \arrow[ur,"f"]
\end{tikzcd}$$
where the map $\CC^m \to \CC^m$ is equality.
Since the diagram
$$\begin{tikzcd}
X_1 \arrow[rr,"\psi"] && X_2\\
&U \arrow[ul,"\iota_1"] \arrow[ur,"\iota_2"]
\end{tikzcd}
$$
commutes it follows that $\psi$ is injective, i.e. $X_1$ is a submanifold of $X_2$.

Clearly $(U, f, \id)$ is the initial object in $A(f)$.
\begin{theorem}[Riemann]
$A(f)$ has final objects.
\end{theorem}
\begin{proof}
By Zorn's lemma. Since all maps are injective, the skeleton of $A(f)$ is a poset $P$, which is clearly nonempty.
Given a chain $C$ in $P$, let $X$ be the union of the chain, $\pi$ and $\iota$ the induced maps. Then $(X, \pi, \iota)$ is clearly an upper bound on $C$.
\end{proof}
The final object is the maximal analytic continuation of $f$. As we will show the Riemann surface of $f$ is its maximal analytic continuation.
To do this we need to introduce yet a third notion of Riemann surface.

\begin{definition}
Let $X$ be a complex manifold.
A \dfn{holomorphic cover} is a surjective holomorphic map $\pi: Y \to X$, $Y$ a complex manifold, such that every $x \in X$ is contained in an open $U \ni x$ such that $\pi^{-1}(U)$ is a disjoint union of isomorphic copies $V$ of $U$, and $\pi: V \to U$ is an isomorphism.
\end{definition}
Let $\pi_1(X)$ be the fundamental group of the complex manifold $X$.
\begin{definition}
Let $X$ be a complex manifold, $\gamma \in \pi_1(X)$, $U \subseteq X$, $f: U \to Y$ a holomorphic map.
We say that $f$ has \dfn{monodromy} along $\gamma$ if for any two analytic continuations $g_1,g_2$ of $f$ along $\gamma$, $g_1 = g_2$.
If $H$ is the group of all $\gamma$ such that $f$ has monodromy along $\gamma$, the group $\pi_1(X)/H$ is called the \dfn{monodromy group} of $f$.
\end{definition}
To see that $H$ is normal note that it is the kernel of the action of $\pi_1(X)$ by ``rotating $f$ about its singularities".

For example if $X = \CC \setminus 0$, $U$ the right-half plane, $\pi_1(X) = \ZZ$, and $z \mapsto 1/z$ has monodromy along any element of $\pi_1(X)$, while $x \mapsto \sqrt x$ only has monodromy along the trivial loop.

We now appeal to the Galois theory of covering spaces.
\begin{definition}
A \dfn{deck transformation} of a holomorphic cover $\pi: Y \to X$ is an isomorphism $\varphi: Y \to Y$ such that $\varphi$ permutes the fibers of $\pi$: $\pi \circ \varphi = \pi$.
The group of deck transformations of $\pi$ is known as the \dfn{Galois group} $\Gal(\pi)$ of $\pi$.
\end{definition}
The fundamental group $\pi_1(X)$ acts on a cover $\pi: Y \to X$ by deck transformations, so we have a morphism of groups
$$\pi_1(X) \to \Gal(\pi).$$
That is, if $\gamma$ is a loop, we may lift $\gamma$ along $\pi$, say to a path $\widetilde \gamma$, which may not be a loop, and if $y \in Y$ we may choose the beginning point of $\widetilde \gamma$ to be $y$ (simply by taking the basepoint of $\pi_1(X)$ to be $\pi(y)$ -- so this only makes sense if $X$ is connected).
Then the endpoint of $\widetilde \gamma$ is the image of $y$ under $\gamma$.

For example, if $X = \CC/\Gamma$ is an elliptic curve (so $\Gamma$ is a lattice), then the projection map $\CC \to X$ is a holomorphic cover, and $\pi_1(X) \cong \ZZ^2$.
To see how $\pi_1(X)$ acts on $\CC$, fix $z \in \CC$; then $z + \Gamma$ is a point on the elliptic curve $X$.
If $\gamma \in \pi_1(X)$ is a loop based at $z + \Gamma$, $\gamma$ lifts to a path $\widetilde \gamma$ through $\CC$ which starts at $z$ and ends at some point $w$ such that $z - w \in \Gamma$. Thus $\gamma z = w$.
Moreover $z - w$ did not depend on $z$, but only on $\gamma$, so $\gamma$ acts on $\CC$ by translation by $z - w$.
In fact, the above construction gives an explicit isomorphism $\ZZ^2 \to \Gamma$.

Let $G$ be a Galois group, say of $Z \to X$. Then if $H$ is a normal subgroup of $G$, we can find a Galois cover $Y \to X$ such that $Z \to Y$ is a Galois cover of $Y$, $\Gal(Y \to X) = G/H$, and $\Gal(Z \to Y) = H$.
Conversely, every Galois group is a quotient of $\pi_1(X)$ as we reasoned above.
This is one form of the fundamental theorem of Galois theory.

\begin{lemma}
Let $X,Y$ be connected Riemann surfaces, $f: X \to \CC^m$ and $g: Y \to \CC^m$ holomorphic, and $\iota: X \to Y$ a holomorphic embedding.
Then $(Y, g, \iota)$ is an analytic continuation of $f$ if and only if there is a holomorphic cover $\pi: Y \to X$ such that $\Gal(\pi)$ is a quotient of the monodromy group $\pi_1(X)/H$ of $f$ and $\iota$ is a section of $\pi$.
\end{lemma}
\begin{proof}
This follows from the definitions once we declare that $\pi^{-1}(x)$ should consist of those elements $y \in Y$ such that there is a branch $j$ of $f$ (i.e. an embedding $j: X \to Y$ such that $g \circ j = f$) such that $j(x) = y$.
Then clearly any such $j$ is a section of $\pi$ (in particular this is true for $\iota$) and $\pi_1(X)/H$ acts on $Y$ by permuting branches (i.e. by deck transformations).
\end{proof}
As a consequence the maximal such connected Riemann surface $Y$ must make the relevant diagrams commute that $Y$ be final in the category of analytic continuations.

\begin{definition}
A variety $X$ is said to be an \dfn{irreducible variety} if there is an irreducible polynomial $g$ such that $X$ is defined by $g = 0$.
\end{definition}

\begin{theorem}
Let $U \subseteq \CC$ be an open set, $f: U \to \CC^m$ a holomorphic function, and $F: \CC^{1 + m} \to \CC^m$ a polynomial such that:
\begin{enumerate}
\item $F$ is a holomorphic submersion.
\item The variety $X$ defined by $F = 0$ is irreducible.
\end{enumerate}
Let $\pi: X \to \CC^m$ be the natural projection. Then $(X, \pi)$ is the maximal analytic continuation of $f$.
\end{theorem}
To prove this I need a black box lemma from modern algebraic geometry.
\begin{lemma}
A variety is irreducible iff it is connected.
\end{lemma}
\begin{proof}
Since $F$ is a holomorphic submersion, $X$ is a complex manifold of dimension $1+m - 1 = m$, hence a Riemann surface.
Since $X$ is nonsingular it is connected, $\pi$ is holomorphic, and obviously $\pi$ is an analytic continuation of $f$.
In particular $X \to Y$, $Y$ the domain of any choice of analytic continuation of $f$ to a discontinuous function inside $\CC$, is a holomorphic cover.
Since $X$ is irreducible it is connected.

Moreover, if $\pi_1(Y)/H$ is the monodromy group of $f$, then clearly $\pi_1(Y)/H$ acts freely on $X$; otherwise $\pi$ would have a nontrivial monodromy group. Therefore $X$ is the maximal holomorphic cover of $Y$.
\end{proof}



\section{Meromorphic continuation of the free resolvent}
Let $D = -i\partial$ be the momentum observable.
Then
$$H_0 - z = \begin{bmatrix}
D^2 + 1 - z\\
&-D^2 - 1 -z
\end{bmatrix}.$$
We are interested in the resolvent $R_0(z) = (H_0 - z)^{-1}$.

Let $R$ be the resolvent of the classical Schrodinger operator,
$$R(\lambda)(D - \lambda^2) = 1.$$
Then
$$R(x, y, \lambda) = \frac{i}{2\lambda} e^{i\lambda|x-y|}.$$
Here we take $\lambda = \sqrt z$ where we are taking the branch of $\sqrt\cdot$ which forces $\Im z \geq 0$; i.e.
$$\sqrt{re^{i\theta}} = e^{i\theta/2}\sqrt r,$$
$\theta \in [0, 2\pi]$ (so that $\theta/2 \in [0, \pi]$). This branch is discontinuous at $\theta = 2\pi$, hence on $\RR_+$.

We need to treat the two resolvents $D^2 + 1 - z$ and $-D^2 - 1 - z$ separately.
For the first resolvent, $D^2 + 1 - z$, we are plugging in $\lambda^2 = z - 1$, and we need $\Im \lambda \geq 0$.
Taking $\widetilde z = z - 1$, we see that we need $\widetilde z \notin \RR_+$, i.e. $z \notin [1, \infty)$.
Similarly, for $-D^2 - 1 - z$, this is defined when $z \notin (-\infty, -1]$. Thus
$$R_0(x, y, z)_{jj} = \frac{i}{2} \exp(i\sqrt{(-1)^{1-j}z -1}|x-y|)((-1)^{1-j}z -1)^{-1/2}$$
and $R_0(x,y,z)_{jk} = 0$ if $j \neq k$.
This is clearly holomorphic in $z$ for $z \notin S$ where
$$S = (-\infty, -1], [1, \infty)$$
is the spectrum of $H_0$.

Now fix $(x, y)$; then we can view $R_0$ as a map $\CC \setminus S \to \CC^2$ by
$$R_0(z)_j = \frac{i}{2} \exp(i\sqrt{(-1)^{1-j}z -1}|x-y|)((-1)^{1-j}z -1)^{-1/2}).$$
The map $w \mapsto \exp(iw|x-y|)/w$ is clearly holomorphic on $\CC \setminus 0$ and has a simple pole at $0$.
In particular no choice of branch is made for it; so to understand the Riemann surface of $R_0$ it suffices to consider the Riemann surface of the function $z \mapsto w$,
$$w_j = \sqrt{(-1)^{1-j}z-1}.$$
That is, $w_j^2 = (-1)^{1-j}z - 1$. Thus the Riemann surface $\Sigma$ is the algebraic variety in $\CC^3 = \{(z, w_1, w_2)\}$ defined by the equations
$$F_j(z, w_1, w_2) = w_j^2 + (-1)^jz + 1 = 0.$$
For a point $(z, w_1, w_2)$ of $\Sigma$, we define
$$R_0(x, y, z, w_1, w_2)_j = \frac{i}{2} \frac{\exp(iw_j|x-y|)}{w_j}.$$

\begin{lemma}
$\Sigma$ is a nonsingular algebraic curve.
\end{lemma}
\begin{proof}
We can locally linearize $F$ as
$$\nabla F(z, w_1, w_2) = \begin{bmatrix}-1 & 2w_1 & 0\\1 & 0 & 2w_2\end{bmatrix}.$$
Then $\rank \nabla F(z, w_1, w_2) = 2$ as long as it is false that $w_1 = w_2 = 0$.
However the set $Z = \{(z, w_1, w_2): w_1 = w_2 = 0\}$ does not meet $\Sigma$ since on $Z \cap \Sigma$, $-z = -1 = z$.
Thus $F$ is a submersion $\CC \setminus Z \to \CC^2$.

Since $\CC^3 \setminus Z$ is an open submanifold of $\CC^3$ and $F$ is a submersion, it follows that $\Sigma$ is a complex manifold of dimension $1$, defined to be the zero set of the polynomial $F$.
Therefore $\Sigma$ is a nonsingular algebraic curve.
\end{proof}
We now construct charts that cover most of $\Sigma$ to describe its geometry. Given signs $\sigma,\tau$, let
$$\varphi_\sigma^\tau(z) = (z, \sqrt{z-1}_\sigma, \sqrt{-z-1}_\tau)$$
where
$$\sqrt{re^{i\theta}}_\pm = e^{i\theta/2}\sqrt r,\quad\theta \in [0, \pm 2\pi),$$
so $\sqrt{re^{i\theta}}_\pm \in \CC_\pm$, $\CC_+$ the upper half-plane and $\CC_-$ the lower half-plane.
We view
$$\varphi_\sigma^\tau: \CC \setminus S \to U_\sigma^\tau \subset \Sigma$$
as a holomorphic chart.
\begin{lemma}
The charts $U_\sigma^\tau$ are disjoint and cover the open dense subset
$$\Sigma' = \{(z, w_1, w_2) \in \Sigma: z \notin S\}$$
of $\Sigma$.
\end{lemma}
\begin{proof}
Clearly any $(z, w_1, w_2) \in \Sigma$ can be approximated by an element of $\Sigma'$ by perturbing $z$ and adjusting the values of $w_1,w_2$ as necessary. Thus $\Sigma'$ is open and dense.

Let $(z, w_1, w_2) \in \Sigma'$. We claim that $w_1 = \sqrt{z-1}_{\sgn \Im w_1}$ and $w_2 = \sqrt{-z-1}_{\sgn \Im w_2}$.
In fact,
$$w_1^2 = z - 1.$$
Thus $w_1$ is a square root of $z - 1$, and $z - 1 \notin [0, \infty)$ implies that $\Im w_1 \neq 0$.
Since $z - 1$ only has two square roots, one in each of $\CC_+$ and $\CC_-$, we can recover $w_1$ by the claimed formula.
A similar argument works for $w_2$.
In particular,
$$(z, w_1, w_2) \in U_{\sgn \Im w_1}^{\sgn \Im w_2}.$$

Finally, to see that the charts are disjoint, note that if $w_j \in \RR$ then $z \in S$, a contradiction. Thus
$$U_\sigma^\tau = \{(z, w_1, w_2) \in S: \sgn \Im w_1 = \sigma,~\sgn \Im w_2 = \tau\}.$$
In particular the $U_\sigma^\tau$ are disjoint.
\end{proof}
From this we can see that $\Sigma$ has four sheets, which intersect at the branch cuts $\Sigma \setminus \Sigma'$.
This seems fishy because last time we agreed that $\Sigma$ should have three sheets but oh well.

By the theory we see that the maximal analytic continuation of $R_0$ is defined on $\Sigma$ by
$$R_0(x, y, z, w)_{jj} = \frac{i}{2w_j}\exp(iw_j|x-y|).$$

\section{Meromorphic continuation of resolvents for compactly supported potentials}
In the previous section we let $\Sigma$ be a certain nonsingular algebraic curve, and meromorphically continued the free resolvent to $\Sigma$.
Now let $V$ be a compactly supported, matrix-valued potential, $\rho$ a cutoff such that $\rho V = V$, and
$$P_V = \diag(D^2 + 1, -D^2 - 1) + V.$$
We must show that the resolvent $R_V(z) = (P_V - z)^{-1}$ has Riemann surface $\Sigma$.
Since $\rho$ is arbitrary it suffices to show this for $\rho R_V(z) \rho$.
Now
$$R_V(z) = (1 - R_0(z)V)^{-1}R_0(z)$$
TODO: follow Zworski's continuation


\section{Meromorphic continuation of resolvents for $\sech^2$ by exponential expansion}
In general meromorphically continuing a noncompactly supported resolvent to $\Sigma$ might not be possible. But one hopes to be able to do it for $\sech^2$ because it admits an expansion in terms of exponentials.

Let $y = e^x$. Then
$$\sech^2 x = 4\frac{y^2}{(y^2 + 1)^2}.$$
Now near infinity we have
$$4\frac{y^2}{(y^2 + 1)^2} = -2\sum_{n=2}^\infty n\cos(n\pi/2)y^{-n}.$$
That means that
$$\sech^2 x = -2\sum_{n=2}^\infty n\cos(n\pi/2)e^{-nx}.$$
Allegedly one can handle each of these terms on a ``strip" of $\Sigma$.



\newpage
\printindex

\end{document}
