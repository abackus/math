
% --------------------------------------------------------------
% This is all preamble stuff that you don't have to worry about.
% Head down to where it says "Start here"
% --------------------------------------------------------------

\documentclass[10pt]{article}

\usepackage[margin=.7in]{geometry}
\usepackage{amsmath,amsthm,amssymb,mathrsfs}
\usepackage{enumitem}
\usepackage{tikz-cd}
\usepackage{mathtools}
\usepackage{amsfonts}
\usepackage{listings}
\usepackage{algorithm2e}
\usepackage{verse,stmaryrd}
\usepackage{fancyvrb}

% Number systems
\newcommand{\NN}{\mathbb{N}}
\newcommand{\ZZ}{\mathbb{Z}}
\newcommand{\QQ}{\mathbb{Q}}
\newcommand{\RR}{\mathbb{R}}
\newcommand{\CC}{\mathbb{C}}
\newcommand{\PP}{\mathbb P}
\newcommand{\FF}{\mathbb F}
\newcommand{\DD}{\mathbb D}
\renewcommand{\epsilon}{\varepsilon}

\newcommand{\Aut}{\operatorname{Aut}}
\newcommand{\coker}{\operatorname{coker}}
\newcommand{\CVect}{\CC\operatorname{-Vect}}
\newcommand{\Cantor}{\mathcal{C}}
\newcommand{\D}{\mathcal{D}}
\newcommand{\card}{\operatorname{card}}
\newcommand{\dbar}{\overline \partial}
\DeclareMathOperator*{\esssup}{ess\,sup}
\newcommand{\GL}{\operatorname{GL}}
\newcommand{\Hom}{\operatorname{Hom}}
\newcommand{\id}{\operatorname{id}}
\newcommand{\Ind}{\operatorname{Ind}}
\newcommand{\Inn}{\operatorname{Inn}}
\newcommand{\interior}{\operatorname{int}}
\newcommand{\lcm}{\operatorname{lcm}}
\newcommand{\mesh}{\operatorname{mesh}}
\newcommand{\LL}{\mathcal L_0}
\newcommand{\Leb}{\mathcal{L}_{\text{loc}}^2}
\newcommand{\Lip}{\operatorname{Lip}}
\newcommand{\ppGL}{\operatorname{PGL}}
\newcommand{\ppic}{\vspace{35mm}}
\newcommand{\ppset}{\mathcal{P}}
\DeclareMathOperator{\proj}{proj}
\DeclareMathOperator*{\Res}{Res}
\newcommand{\Riem}{\mathcal{R}}
\newcommand{\RVect}{\RR\operatorname{-Vect}}
\newcommand{\Sch}{\mathcal{S}}
\newcommand{\SL}{\operatorname{SL}}
\newcommand{\sgn}{\operatorname{sgn}}
\newcommand{\spn}{\operatorname{span}}
\newcommand{\Spec}{\operatorname{Spec}}
\newcommand{\supp}{\operatorname{supp}}
\newcommand{\TT}{\mathcal T}
\DeclareMathOperator{\tr}{tr}

\DeclareMathOperator{\adj}{adj}
\DeclareMathOperator{\curl}{curl}

% Calculus of variations
\DeclareMathOperator{\pp}{\mathbf p}
\DeclareMathOperator{\zz}{\mathbf z}
\DeclareMathOperator{\uu}{\mathbf u}
\DeclareMathOperator{\vv}{\mathbf v}
\DeclareMathOperator{\ww}{\mathbf w}

\DeclareMathOperator{\Olo}{\mathscr O}

% Categories
\newcommand{\Ab}{\mathbf{Ab}}
\newcommand{\Cat}{\mathbf{Cat}}
\newcommand{\Group}{\mathbf{Group}}
\newcommand{\Module}{\mathbf{Module}}
\newcommand{\Set}{\mathbf{Set}}
\DeclareMathOperator{\Fun}{Fun}
\DeclareMathOperator{\Iso}{Iso}

% Complex analysis
\renewcommand{\Re}{\operatorname{Re}}
\renewcommand{\Im}{\operatorname{Im}}

% Logic
\renewcommand{\iff}{\leftrightarrow}
\newcommand{\Henkin}{\operatorname{Henk}}
\newcommand{\PA}{\mathbf{PA}}
\DeclareMathOperator{\proves}{\vdash}

% Group
\DeclareMathOperator{\Gal}{Gal}
\DeclareMathOperator{\Fix}{Fix}
\DeclareMathOperator{\Out}{Out}

% Other symbols
\newcommand{\heart}{\ensuremath\heartsuit}

\DeclareMathOperator{\atanh}{atanh}

% Theorems
\theoremstyle{definition}
\newtheorem*{corollary}{Corollary}
\newtheorem*{falselemma}{Grader's ``Lemma"}
\newtheorem{exer}{Exercise}
\newtheorem{lemma}{Lemma}[exer]
\newtheorem{theorem}[lemma]{Theorem}

\usepackage[backend=bibtex,style=alphabetic,maxcitenames=50,maxnames=50]{biblatex}
\renewbibmacro{in:}{}
\DeclareFieldFormat{pages}{#1}

\begin{document}
\noindent
\large\textbf{Algebraic geometry, HW 1} \hfill \textbf{Aidan Backus} \\
% --------------------------------------------------------------
%                         Start here
% --------------------------------------------------------------\

\begin{exer}[1.10]
Let $\mathscr F = (\mathscr F_i)$ be a direct system of sheaves on $X$.
Show that the direct limit $\varinjlim \mathscr F_i$ of the system in the category of sheaves is the sheafification of $\varinjlim \mathscr F_i(\bullet)$.
\end{exer}

Let $\mathscr G$ be the presheaf $\varinjlim \mathscr F_i(\bullet)$ and
$$\alpha: \mathscr G \to \mathscr G^+$$
its sheafification.
By the universal property of $\mathscr G(U)$, $\varphi_i(U)$ factors through the direct limit map $\psi_i(U): \mathscr F_i(U) \to \mathscr G(U)$.

We now claim that the direct limit maps $\psi_i(U)$ are compatible with restriction.
In fact, if $V \subseteq U$ then the restriction map $\mathscr G(U) \to \mathscr G(V)$ is the direct limit of the directed system of restriction maps $\mathscr F_i(U) \to \mathscr F_i(V)$, and by definition of a direct system of sheaves, the restriction maps $\mathscr F_i(U) \to \mathscr F_i(V)$ are compatible with the maps $\psi_{ij}(U): \mathscr F_i(U) \to \mathscr F_j(U)$.
Therefore $\mathscr G(U) \to \mathscr G(V)$ is also compatible with $\psi_i(U)$.

So we have morphisms of presheaves
$$\psi_i: \mathscr F_i \to \mathscr G$$
which are compatible with the direct system structure of $\mathscr F$.
If we have morphisms of sheaves $\varphi_i: \mathscr F_i \to \mathscr H$ which are compatible with the direct system structure of $\mathscr F$, then we have morphisms of abelian groups
$$\varphi_i(U): \mathscr F_i(U) \to \mathscr H(U)$$
which are compatible with the direct system structure of $\mathscr F(U) = (\mathscr F_i(U))$.
So by the universal property of $\mathscr G(U)$, we can take the limit of the maps $\varphi_i(U)$ to get maps
$$\delta(U): \mathscr G(U) \to \mathscr H(U)$$
with
$$\varphi_i(U) = \delta_i(U) \circ \psi_i(U).$$
Since $\delta(U)$ are the limits of maps which are at once compatible with the direct system structure of $\mathscr F$ and the presheaf structure of $\mathscr F_i$, they in fact define a morphism of presheaves
$$\delta: \mathscr G \to \mathscr H.$$
Applying the universal property of $\alpha$ now gives us a morphism of sheaves $\delta^+: \mathscr G^+ \to \mathscr H$ such that
$$\varphi_i = \delta^+ \circ \alpha \circ \psi_i.$$
Therefore $\mathscr G^+$ is initial among sheaves which the direct system of sheaves $\mathscr F$ maps to, which is the desired universal property.

\begin{exer}[1.14]
Let $\mathscr F$ be a sheaf, and let $s \in \mathscr F(U)$.
Show that $\supp s$ is a closed subset of $U$.
\end{exer}

Let $V$ be the set of $x \in U$ such that $s_x = 0$.
Then for every $x \in V$, there is an open neighborhood $W_x \subseteq U$ such that $s|W_x = 0$ and hence $W_x \subseteq V$.
So $V = \bigcup_{x \in V} W_x$ is open in $U$ and hence its complement, $\supp s$, is closed.

\begin{exer}[2.3a]
Show that $X$ is a reduced scheme iff for every $x \in X$, $\Olo_{X, x}$ has no nilpotent elements.
\end{exer}

Suppose that there is $x \in X$ such that $\Olo_{X, x}$ has a nonzero germ $f$ such that there exists $n \geq 2$ with $f^n = 0$.
Then we can extend $f$ to a representative $f$ defined on an open set $U$ such that $f^n = 0$.
So $U$ witnesses that $X$ is nonreduced.

Conversely, if for every $x \in X$, $\Olo_{X, x}$ has no nilpotent germs, suppose that $f \in \Olo_X(U)$ satisfies $f^n = 0$.
Then the germ $(f_x)^n = 0$ so $f_x = 0$.
Since $f$ always localizes to $0$, $f = 0$, so $X$ is reduced.

\begin{exer}[2.3b]
If $A$ is a ring, we define its reduction $A_{red}$ to be the quotient of $A$ by the nilradical of $A$.
Let $X$ be a scheme and let $\Olo_{X;red}$ be the sheafification of $\Olo_X(\bullet)_{red}$.
Show that $X_{red} = (X, \Olo_{X;red})$ is a scheme and there is a morphism of schemes $X_{red} \to X$ which is a homeomorphism.
\end{exer}

Throughout this problem, if $\mathscr F$ is a presheaf of rings, we define its nilradical $\mathscr N$ to be the presheaf of abelian groups defined by setting $\mathscr N(U)$ to be the nilradical of $\mathscr F(U)$.

By definition, $X_{red}$ is a ringed space.
Moreover, by construction for every open set $U$, we have the reduction map
$$\psi(U): \Olo_X(U) \to \Olo_X(U)_{red}.$$
Then $\psi$ is the quotient map of $\Olo_X$ by its nilradical, so $\psi$ is a morphism of presheaves.
Composing $\psi$ with the sheafification map $\Olo_X(U)_{red} \to \Olo_{X;red}$ defines a morphism of sheaves
$$\varphi: \Olo_X \to \Olo_{X;red}.$$
We define $f: X_{red} \to X$ to be the identity on points and $f^\sharp = \varphi$, so $f = (f, f^\sharp)$ is a morphism of ringed spaces.

We now claim that $X_{red}$ is locally ringed and $f$ is a morphism of locally ringed spaces.
Let $x \in X_{red}$.
Since $\varphi$ is the sheafification of $\psi$, $\psi$ is the quotient map by the nilradical $\mathscr N$ of $\Olo_X$, and sheafification is the identity on stalks,
$$(\Olo_{X;red})_x = \frac{\Olo_{X, x}}{\mathscr N_x} = (\Olo_{X, x})_{red}.$$
Here we used the fact that the direct limit of quotients $\Olo_X(U)/\mathscr N(U)$ is a quotient $\Olo_{X,x}/\mathscr N_x$.
In any case, the reduction of a local ring is a quotient map of local rings, i.e. is a morphism of local rings.
Therefore $(\Olo_{X;red})_x$ is a local ring and
$$\varphi_x: \Olo_{X, x} \to (\Olo_{X;red})_x$$
is a morphism of local rings.
Therefore $X_{red}$ is a locally ringed space and $f$ is a morphism of locally ringed spaces.

Now we claim that $X_{red}$ is a scheme and $f$ is a morphism of schemes.
To check this, we fix an open cover $\mathcal U$ of $X$ by affine schemes.
Then for every $U \in \mathcal U$, say $U = \Spec A$, we obtain
$$f^{-1}(U) = f^{-1}(\Spec A) = \Spec(\varphi(U)(A)) \Spec(A_{red}),$$
so $f^{-1}(U)$ is an affine scheme.
Since $f$ is the identity by points, the pullback of $\mathcal U$ by $f$ is therefore an open cover of $X_{red}$ by affine schemes, so the claim holds.

Finally, since $f$ is the identity on points, $f$ is obviously a homeomorphism.

\begin{exer}[2.3c]
Let $Y$ be a scheme.
Show that $Y_{red}$ is final among reduced schemes over $Y$.
\end{exer}

Let $f: X \to Y$ be a reduced scheme over $Y$.
Then on the level of topological spaces, $f$ factors through the identity map on $Y$ and hence on the map $Y_{red} \to Y$.
We just need to show that the pullback $f^\sharp: \Olo_Y \to \Olo_X$ factors through the reduction map $\Olo_Y \to \Olo_{Y;red}$ in the category of sheaves, but by the universal property of sheafification, that is equivalent to $f^\sharp$ factoring through the reduction map
$$\psi: \Olo_Y \to \Olo_Y(\bullet)_{red}$$
in the category of \emph{pre}sheaves.
But $\psi$ is a quotient map so this just means that $f^\sharp$ vanishes on the nilradical $\mathscr N$ of $\Olo_Y$ and hence drops to a morphism of presheaves
$$\varphi: \frac{\Olo_Y}{\mathscr N} = \Olo_Y(\bullet)_{red} \to \Olo_X.$$
Now let $f^\sharp_{red}: \Olo_{Y;red} \to X$ be the sheafification of $\varphi$.
Then $f_{red} = (f, f^\sharp_{red})$ is a morphism of schemes $X \to Y_{red}$ which witnesses that $Y_{red}$ is final.

\begin{exer}[2.5]
Describe $\Spec \ZZ$ and show that it is final among schemes.
\end{exer}

Points of $\Spec \ZZ$ are either prime numbers or the trivial ideal $\xi$.
If $p$ is prime then we can view $p$ as both a function on $\Spec \ZZ$ and a point of $\Spec \ZZ$; using this duality, $p(p) = 0$ while $p(q)$ is nonzero for any $\Spec \ZZ \setminus p$, thus $\Spec \ZZ \setminus p$ is open and hence $\{p\}$ is a closed point.
On the other hand, if $f$ is a function on $\Spec \ZZ$ such that $f(\xi) = 0$, then $f$ is an integer which is $0$ modulo $0$ and hence $f = 0$ identically; so there are no functions which only vanish on $\xi$ and hence $\xi$ lies in every nonempty open subset of $\Spec \ZZ$.
The sets $\{n: n \neq p\}$ where $p$ is a prime number are thus a basis for the topology of $\Spec \ZZ$.
The open set $U = \{n: n \neq p\}$ has as its ring $\Olo_\ZZ(U)$ the localization of $\ZZ$ to $U$, thus it consists of those fractions without a factor of $p$ in the denominator.
I think we're supposed to imagine $\Spec \ZZ$ as a fiber bundle, with base space the space of prime numbers, and the fiber of the prime number $p$ is the field $\ZZ/p$.

Since $\Spec \ZZ$ is an affine scheme, we have a natural bijection
$$\Hom(X, \Spec \ZZ) \to \Hom(\ZZ, \Olo_X(X))$$
and since $\Olo_X(X)$ is a ring and $\ZZ$ is initial among rings, there is a unique morphism of rings $\ZZ \to \Olo_X(X)$, hence a unique morphism $X \to \Spec \ZZ$.
Therefore $\Spec \ZZ$ is final.

\begin{exer}[3.1]
Show that a morphism $f: X \to Y$ is locally of finite type iff for every affine open subset $V = \Spec B$ of $Y$, $f^{-1}(V)$ can be covered by affine open subsets $U_j = \Spec A_j$ where $A_j$ is a finitely generated $B$-algebra.
\end{exer}

One direction is obvious from the definition of ``locally of finite type" and the fact that $Y$ has an open cover by affine open sets.
To prove the converse, suppose that $f$ is locally of finite type and $V = \Spec B$ is an affine open subset of $Y$.
To save ourselves a lot of words, we let $\mathcal P$ be the set of all rings $P$ such that:
\begin{enumerate}
\item $\Spec P$ is an affine open subset of $Y$.
\item $f^{-1}(\Spec P)$ admits an open cover by spectra of finitely generated $P$-algebras.
\end{enumerate}
We must show that $B \in \mathcal P$.

We first show that $\mathcal P$ is closed under localization.
For every $P \in \mathcal P$ there is a set $\mathcal Q(P)$ of finitely generated $P$-algebras such that $f^{-1}(\Spec P)$ is covered by $\{\Spec Q: Q \in \mathcal Q(P)\}$.
In particular for every $Q$ we can find a morphism of rings $\varphi_{PQ}: P[x_1, \dots, x_n] \to Q$ which is surjective.
Let $S \subset P$ be a multiplicative set.
Elements of $Q[S^{-1}]$ are formal fractions $\varphi_{PQ}(F(x_1, \dots, x_n))/\varphi_{PQ}(s)$ with $s \in S$ and $F(x_1, \dots, x_n) \in P[x_1, \dots, x_n]$ such that for every $t \in S$, $F(x_1, \dots, x_n)t \neq 0$.
But
$$\frac{\varphi_{PQ}(F(x_1, \dots, x_n))}{\varphi_{PQ}(s)} = \varphi_{PQ}[S^{-1}]\left(\frac{F(x_1, \dots, x_n)}{s}\right)$$
so
$$\varphi_{PQ}[S^{-1}]: P[x_1, \dots, x_n, S^{-1}] \to Q[S^{-1}]$$
is surjective.
In particular $Q[S^{-1}]$ is a finitely generated $P[S^{-1}]$-algebra.
Since $f^{-1}(\Spec P[S^{-1}])$ is covered by $\{\Spec Q[S^{-1}]: Q \in \mathcal Q(P)\}$, it follows that $P[S^{-1}] \in \mathcal P$, as desired.

Now suppose that $P$ is a ring with elements $p_1, \dots, p_n \in P$ such that:
\begin{enumerate}
\item $\Spec P$ is an affine open subset of $Y$.
\item $p_1, \dots, p_n$ generate the unit ideal of $P$.
\item For every $i \in I = \{1, \dots, k\}$, $P[{p_i}^{-1}] \in \mathcal P$.
\end{enumerate}
We claim that $P \in \mathcal P$.
In fact for every $i \in I$ there exists an open cover $\mathcal U_i$ of $f^{-1}(\Spec P_i)$ by spectra of finitely generated $P_i$-algebras.
Let $\mathcal U = \bigcup_{i \in I} \mathcal U_i$, so $\mathcal U$ is a set of affine open subsets of $X$.
Since $p_1, \dots, p_n$ generate the unit ideal of $P$, $\{\Spec P_i: i \in I\}$ is an open cover of $\Spec P$, so $\mathcal U$ is an open cover of $f^{-1}(\Spec P)$.
So we just need to show that if $Q$ is a finitely generated $P_i$-algebra then $Q$ is a finitely generated $P$-algebra as well, but we have surjective maps
$$\begin{tikzcd}
P[x_0, \dots, x_n] \arrow[r,"\alpha"] & P_i[x_1, \dots, x_n] \arrow[r,"\beta"] & Q
\end{tikzcd}
$$
where $\alpha$ is defined by $\alpha(x_0) = p_i^{-1}$ and $\beta$ witnesses that $Q$ is finitely generated over $P_i$.
The composite $\beta \circ \alpha$ witnesses that $Q$ is finitely generated over $P$.

Finally, we show $B \in \mathcal P$.
By hypothesis there exists $\mathcal P_1 \subseteq \mathcal P$ such that $\{\Spec P: P \in \mathcal P_1\}$ is an open cover of $Y$.
Since $\Spec B$ is quasicompact, there exists a finite set $\mathcal P_2 \subseteq \mathcal P_1$ such that $\{\Spec P: P \in \mathcal P_2\}$ is an open cover of $\Spec B$.
Now for every $P \in \mathcal P_2$ we can find an open cover $\mathcal U(P)$ of $\Spec B \cap \Spec P$ (finite since $\Spec B \cap \Spec P$ is quasicompact) such that elements of $\mathcal U(P)$ are spectra of $B[{b_i}^{-1}] = P[S^{-1}]$ for some $S$ and $b_i$.
So every element of $\mathcal U(P)$ is the spectrum of an element of $\mathcal P$ since $\mathcal P$ is closed under localization.
Moreover, the covering property means that $b_1, \dots, b_\ell$ generate the unit ideal of $B$, and hence $B \in \mathcal P$.

\begin{exer}[3.6]
Let $X$ be an integral scheme with generic point $\xi$.
Show that $K(X) = \Olo_{X, \xi}$ is a field and for every open affine subset $\Spec A$ of $X$, $K(X)$ is the field of fractions of $A$.
\end{exer}

Since $\xi$ is a generic point, $\xi$ is an element of every nonempty open subset of $X$.
In particular, $\xi \in \Spec A$, so that $K(X) = \Olo_{\Spec A, \xi}$.
Therefore it is no loss to assume that $X$ itself is affine, say $X = \Spec A$, in which case we just need to show that $\Olo_{X, \xi}$ is the field of fractions of $X$.
But in that case, $\Olo_{X, \xi}$ is the localization of $A$ at its zero ideal, which is the field of fractions of $A$ by definition of field of fractions.

\begin{exer}[3.10a]
If $f: X \to Y$ is a morphism and $y \in Y$, show that $X_y$ is homeomorphic to $f^{-1}(y)$.
\end{exer}

Applying the forgetful functor from schemes to spaces we see that $X_y$, as a topological space, is the final space which fits into the commutative diagram
$$\begin{tikzcd}??? \arrow[r] \arrow[d] & \{y\} \arrow[d] \\
X \arrow[r,"f"] & Y\end{tikzcd}$$
We must show that $f^{-1}(y)$ is also final among spaces which fit into this diagram. In fact, $f^{-1}(y)$ maps to $\{y\}$ by $f$ and maps into $X$ by the inclusion map, and the resulting diagram
$$\begin{tikzcd}f^{-1}(y) \arrow[r,"f"] \arrow[d] & \{y\} \arrow[d] \\
X \arrow[r,"f"] & Y\end{tikzcd}$$
is easily seen to commute.
If we are given another commutative diagram
$$\begin{tikzcd} Z \arrow[drr,bend left=30] \arrow[ddr,bend right=20,"g"] \\
& f^{-1}(y) \arrow[r,"f"] \arrow[d] & \{y\} \arrow[d] \\
& X \arrow[r,"f"] & Y\end{tikzcd}$$
set $\tilde g = g|g^{-1}(f^{-1}(y))$. Then, by definition, the diagram
$$\begin{tikzcd} Z \arrow[drr,bend left=30] \arrow[ddr,bend right=20,"g"] \arrow[dr,"\tilde g"] \\
& f^{-1}(y) \arrow[r,"f"] \arrow[d] & \{y\} \arrow[d] \\
& X \arrow[r,"f"] & Y\end{tikzcd}$$
commutes.

\begin{exer}[3.10b]
Let $X = \Spec k[s, t]/(s - t^2)$, $Y = \Spec k[s]$, and $f: X \to Y$ be the opposite of the map $s \mapsto s$:
\begin{enumerate}
\item If $y \in Y$ coresponds to an element of $k^*$, show that $X_y$ consists of two points and has residue field $k$.
\item If $y \in Y$ is the origin, show that $X_y$ is a nonreduced one-point scheme.
\item If $\eta$ is generic in $Y$ show that $X_\eta$ is a one-point scheme whose residue field is a degree-$2$ extension of the residue field of $\eta$.
\end{enumerate}
\end{exer}

We will need that the coproduct of schemes $A \sqcup B$ is their disjoint union on the level of topological spaces and that
$$\Olo_{A \sqcup B}(U) = \Olo_A(A \cap U) \times \Olo_B(B \cap U).$$
The former claim follows by taking the forgetful functor and the latter follows by contravariance of $\Olo_\bullet$.
Since $\Spec$ is contravariant, $\Spec$ turns products into coproducts.

Throughout the below we shamelessly identify the maximal ideals $(s - y)$ and $(s - y, t - z)$ with the closed points $y$ and $(y, z)$.
With this notation we can write $f(y, z) = y$ as long as we're talking about closed points.

Suppose that $y$ corresponds to an element of $k^*$.
Then there are $z_1,z_2$ with ${z_i}^2 = y$.
Let
$$Z = \Spec(k \times k) = \Spec k \sqcup \Spec k = \{z_1, z_2\}.$$
Then $\Olo_{Z,z_i} = \Olo_Z(\{z_i\}) = k$.
We can then define maps $g: Z \to X$, $h: Z \to \Spec k$, where $h(z_i) = y$ and $g(z_i) = (y, z_i)$, since $(y, z_i)$ is a point on $X$.
Write $i: \Spec k = \{y\} \to Y$ for the inclusion map.
The diagram
$$\begin{tikzcd}
Z \arrow[r, "h"] \arrow[d,"g"] & \Spec k \arrow[d,"i"] \\
X \arrow[r,"f"] & Y
\end{tikzcd}
$$
commutes: on the level of points,
$$i(h(z_i)) = i(y) = y = f(y, z_i) = f(g(z_i));$$
and if we are given a local function $u$ on $Y$, its pullback $i^* u$ is just $u|\{y\}$, thus $h^* i^* u(z_i) = u(y)$; meanwhile $f^* u(y, z) = u(y)$ so $g^* f^* u(z_i) = u(y)$.
To see that $Z$ is final, suppose that we are given a commutative diagram
$$\begin{tikzcd} W \arrow[drr,bend left=30,"\alpha"] \arrow[ddr,bend right=20,"\gamma"] \\
& Z \arrow[r,"h"] \arrow[d,"g"] & \Spec k \arrow[d,"i"] \\
& X \arrow[r,"f"] & Y\end{tikzcd}$$
Then for every $w \in W$ such that there is $z \in k$ with $\gamma(w) = (y, z)$, then there exists $i \in \{1, 2\}$ such that $z = z_i$ and so we are entitled to set $\beta(w) = z_i$, and let $\beta^*$ be the pullback of $\gamma^*$ along $g$.
Then the diagram
$$\begin{tikzcd} W \arrow[drr,bend left=30,"\alpha"] \arrow[ddr,bend right=20,"\gamma"] \arrow[dr,"\beta"] \\
& Z \arrow[r,"h"] \arrow[d,"g"] & \Spec k \arrow[d,"i"] \\
& X \arrow[r,"f"] & Y\end{tikzcd}$$
obviously commutes on the level of points, and by definition, $\beta^* \circ g^* = \gamma^*$, so the diagram also commutes on the level of functions.

Now suppose that $y = 0$ is the origin.
Set $Z = \Spec k[t]/(t^2)$, which is clearly a nonreduced scheme with one point $z$, which we also identify with $0$ in $k$.
Let $h: Z \to \Spec k$ be the opposite of the inclusion map $k \to k[t]/(t^2)$ and $g: Z \to X$ be the map $g(z) = (y, z)$.
This is a point on $X$ since $0^2 = 0$.
We now have
$$i(h(z)) = i(y) = y = f(y, z) = f(g(z))$$
so the diagram
$$\begin{tikzcd}
Z \arrow[r, "h"] \arrow[d,"g"] & \Spec k \arrow[d,"i"] \\
X \arrow[r,"f"] & Y
\end{tikzcd}
$$
commutes on the level of points. On the level of functions this diagram is really just
$$
\begin{tikzcd}
\frac{k[t]}{(t^2)} & k \arrow[l, "h^*"] \\
\frac{k[s, t]}{(s - t^2)} \arrow[u, "g^*"] & k[s] \arrow[u, "i^*"] \arrow[l, "f^*"]
\end{tikzcd}
$$
which commutes: no matter what path one takes, the arrows turn $a \in k[s]$ into $a(0)$ and include $a(0)$ into $k[t]/(t^2)$.
To show finality we consider the commutative diagram
$$\begin{tikzcd} W \arrow[drr,bend left=30,"\alpha"] \arrow[ddr,bend right=20,"\gamma"] \\
& Z \arrow[r,"h"] \arrow[d,"g"] & \Spec k \arrow[d,"i"] \\
& X \arrow[r,"f"] & Y\end{tikzcd}$$
and define $\beta$ to be the unique map $W \to Z$ on the level of points.
Now elements of $k[s, t]/(s - t^2)$ can be uniquely written as $k$-linear combinations of $1,s,t$ and elements of $k[t]/(t^2)$ can be uniquely written as $k$-linear combinations of $1,t$.
So, since $\beta^*$ is uniquely determined by $\beta^* 1$, $\gamma^* t$, we can define $\beta^*$ by setting $\beta^* 1 = \gamma^* 1$ and $\beta^* t = \gamma^* t$,
Since $\gamma^*$ makes the relevant commutative diagram commute, so must $\beta^*$.

Finally suppose that $\eta$ is generic in $Y$.
Recall that the residue field of $\eta$ is the function field $K$ of $Y$, which is $k(s)$ since $Y$ is a line.
Let $L = K[\sqrt s]$; recall from Galois theory that the inclusion map $K \to L$ is a field extension of degree $2$.
Let $Z = \Spec L$.
Since $f^{-1}(\eta)$ is the generic point $\xi$ of $X$, which is a single point, we can repeat the exact same argument as in the previous case to see that on the level of points, $Z$ is the fiber of $y$.
But we want to do it on the level of functions.
This amounts to showing that $L$ is initial among rings which fit into a commutative diagram
$$\begin{tikzcd}
k[s] \arrow[r] \arrow[d] & k[s, s^{-1}] \arrow[d] \\
k[\sqrt s] \arrow[r] & ???
\end{tikzcd}$$
But this is pretty much immediate from the definitions, as $L = k[\sqrt s, s^{-1}]$.

\end{document}
