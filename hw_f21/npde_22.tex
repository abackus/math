
% --------------------------------------------------------------
% This is all preamble stuff that you don't have to worry about.
% Head down to where it says "Start here"
% --------------------------------------------------------------

\documentclass[10pt]{article}

\usepackage[margin=.7in]{geometry}
\usepackage{amsmath,amsthm,amssymb,mathrsfs}
\usepackage{enumitem}
\usepackage{tikz-cd}
\usepackage{mathtools}
\usepackage{amsfonts}
\usepackage{listings}
\usepackage{algorithm2e}
\usepackage{verse,stmaryrd}
\usepackage{fancyvrb}

% Number systems
\newcommand{\NN}{\mathbb{N}}
\newcommand{\ZZ}{\mathbb{Z}}
\newcommand{\QQ}{\mathbb{Q}}
\newcommand{\RR}{\mathbb{R}}
\newcommand{\CC}{\mathbb{C}}
\newcommand{\PP}{\mathbb P}
\newcommand{\FF}{\mathbb F}
\newcommand{\DD}{\mathbb D}
\renewcommand{\epsilon}{\varepsilon}

\newcommand{\Aut}{\operatorname{Aut}}
\newcommand{\coker}{\operatorname{coker}}
\newcommand{\CVect}{\CC\operatorname{-Vect}}
\newcommand{\Cantor}{\mathcal{C}}
\newcommand{\D}{\mathcal{D}}
\newcommand{\card}{\operatorname{card}}
\newcommand{\dbar}{\overline \partial}
\DeclareMathOperator*{\esssup}{ess\,sup}
\newcommand{\GL}{\operatorname{GL}}
\newcommand{\Hom}{\operatorname{Hom}}
\newcommand{\id}{\operatorname{id}}
\newcommand{\Ind}{\operatorname{Ind}}
\newcommand{\Inn}{\operatorname{Inn}}
\newcommand{\interior}{\operatorname{int}}
\newcommand{\lcm}{\operatorname{lcm}}
\newcommand{\mesh}{\operatorname{mesh}}
\newcommand{\LL}{\mathcal L_0}
\newcommand{\Leb}{\mathcal{L}_{\text{loc}}^2}
\newcommand{\Lip}{\operatorname{Lip}}
\newcommand{\ppGL}{\operatorname{PGL}}
\newcommand{\ppic}{\vspace{35mm}}
\newcommand{\ppset}{\mathcal{P}}
\DeclareMathOperator{\proj}{proj}
\DeclareMathOperator*{\Res}{Res}
\newcommand{\Riem}{\mathcal{R}}
\newcommand{\RVect}{\RR\operatorname{-Vect}}
\newcommand{\Sch}{\mathcal{S}}
\newcommand{\SL}{\operatorname{SL}}
\newcommand{\sgn}{\operatorname{sgn}}
\newcommand{\spn}{\operatorname{span}}
\newcommand{\Spec}{\operatorname{Spec}}
\newcommand{\supp}{\operatorname{supp}}
\newcommand{\Torus}{\mathbb T}
\DeclareMathOperator{\tr}{tr}

\DeclareMathOperator{\adj}{adj}
\DeclareMathOperator{\curl}{curl}

% Calculus of variations
\DeclareMathOperator{\pp}{\mathbf p}
\DeclareMathOperator{\zz}{\mathbf z}
\DeclareMathOperator{\uu}{\mathbf u}
\DeclareMathOperator{\vv}{\mathbf v}
\DeclareMathOperator{\ww}{\mathbf w}

\DeclareMathOperator{\Olo}{\mathscr O}

% Categories
\newcommand{\Ab}{\mathbf{Ab}}
\newcommand{\Cat}{\mathbf{Cat}}
\newcommand{\Group}{\mathbf{Group}}
\newcommand{\Module}{\mathbf{Module}}
\newcommand{\Set}{\mathbf{Set}}
\DeclareMathOperator{\Fun}{Fun}
\DeclareMathOperator{\Iso}{Iso}

% Complex analysis
\renewcommand{\Re}{\operatorname{Re}}
\renewcommand{\Im}{\operatorname{Im}}

% Logic
\renewcommand{\iff}{\leftrightarrow}
\newcommand{\Henkin}{\operatorname{Henk}}
\newcommand{\PA}{\mathbf{PA}}
\DeclareMathOperator{\proves}{\vdash}

% Group
\DeclareMathOperator{\Gal}{Gal}
\DeclareMathOperator{\Fix}{Fix}
\DeclareMathOperator{\Out}{Out}

% Other symbols
\newcommand{\heart}{\ensuremath\heartsuit}

\DeclareMathOperator{\atanh}{atanh}

% Theorems
\theoremstyle{definition}
\newtheorem*{corollary}{Corollary}
\newtheorem*{falselemma}{Grader's ``Lemma"}
\newtheorem{exer}{Exercise}
\newtheorem{lemma}{Lemma}[exer]
\newtheorem{theorem}[lemma]{Theorem}

\usepackage[backend=bibtex,style=alphabetic,maxcitenames=50,maxnames=50]{biblatex}
\renewbibmacro{in:}{}
\DeclareFieldFormat{pages}{#1}

\begin{document}
\noindent
\large\textbf{Numerical analysis of PDE, HW 2, take 2} \hfill \textbf{Aidan Backus} \\
% --------------------------------------------------------------
%                         Start here
% --------------------------------------------------------------

To solve Problem 4, fix parameters $h, \alpha$ (chosen so that $k = h^\alpha/3$), let $\sigma = h^{\alpha - 2}/3$, let $N$ be an integer approximation to $2\pi/h$, and let $K = 6/h^\alpha$ be the number of iterates needed to compute the solution to the heat equation at time $2$.
Let $v$ be the ``exact solution" obtained from the Crank-Nicholson algorithm with $h = .002$, $\alpha = 2$.

Let $u$ be the approximate solution of the heat equation, with $\alpha$ varying and $h = .01$.
I attached a function \verb+l2_norm+ in the appendix which computes the $L^2$ norm of a function. Computing the $L^2$ norm of the error $u - v$ we get:
\begin{center}\begin{tabular}{c | c | c}
Method & $\alpha$ & $||u - v||_{L^2}$ \\
\hline
Crank-Nicholson & $1$ & $5.3842 \cdot 10^{-4}$ \\
Crank-Nicholson & $1.5$ & $5.3821 \cdot 10^{-4}$ \\
Crank-Nicholson & $2$ & $5.3821 \cdot 10^{-4}$ \\
Forwards Euler & $1$ & $\infty$ \\
Forwards Euler & $1.5$ & $\infty$ \\
Forwards Euler & $2$ & $5.4183 \cdot 10^{-4}$ \\
Backwards Euler & $1$ & $5.5785 \cdot 10^{-4}$ \\
Backwards Euler & $1.5$ & $5.507 \cdot 10^{-4}$ \\
Backwards Euler & $2$ & $5.3464 \cdot 10^{-4}$ \\
Dufort-Frankel & $1$ & $0.1750$ \\
Dufort-Frankel & $1.5$ & $0.0158$ \\
Dufort-Frankel & $2$ & $0.011$
\end{tabular}\end{center}
We see that we don't seem to be able to beat $5 \cdot 10^{-4}$ as an error term. I'm not sure how much of that follows from the fact that $h = .01$, numerical error in the matrix returned by the function \verb+dftmtx+, the fact that our exact solution isn't exact, and how much of it follows from the way that I compute $||u - v||_{L^2}$.

I have attached graphs of the solutions to Dufort-Frankel when $\alpha = 1$ and when $\alpha = 1.5$. I didn't bother for any of the others because they look exactly the same as the exact solution.

My assessment of the different methods is as follows:
\begin{enumerate}
\item Crank-Nicholson and backwards Euler algorithms seem to be good regardless of $\alpha$.
\item Forwards Euler fails completely if $\alpha < 2$, which is not surprising given that it is only stable if $\alpha = 2$. This means forward Euler is probably becomes totally useless as $T \to \infty$ or as $h \to 0$, because of the quadratic growth of the number of iterates $K$, but for small $T$ and $h = .01$, our efficient implementation can handle forward Euler.
\item Dufort-Frankel is really weird. I understand that it is only consistent if $\alpha > 1$, so it makes sense that if $\alpha = 1$ the solution is just garbage. We get a lot of oscillations when $\alpha > 1$, especially when $\alpha = 1.5$. I guess the point is that though the method is consistent, it loses consistency as $\alpha \to 1$. So we would really need to take $h$ smaller than $.01$, or take $\alpha$ larger than $2$, to get a good solution. Another issue is that the initial data is not continuous, so that the first time step destroys some of the properties of the zeroth time step. Since Dufort-Frankel is a two-step method, this is really bad!
\end{enumerate}

Here's my implementations of each method and some comments on how the implementation works.
I have attached the auxiliary functions \verb+circulant+ and \verb+pow+ in an appendix.

For all of the methods except Dufort-Frankel, we use the discrete Fourier transform to rewrite everything in a diagonal matrix $D$.
The point is that since $D$ is diagonal, \verb+D.^K+ actually returns $D^K$.
Using \verb+tic+ and \verb+toc+, I was able to determine that computing \verb+D.^K+ for Crank-Nicholson with $h = .002$ takes $10.491$ seconds on my laptop, as opposed to more clunky ways of computing the exact solution which sounded like they were about to burn out my computer fan.
At the end, we apply \verb+real+ to get rid of the imaginary part that results from using the discrete Fourier transform.
This is harmless because the imaginary part is always equal to $0$ anyways, but it causes MATLAB to throw an error when you try to graph it.

For Dufort-Frankel, one can do a similar trick using tensor products of \verb+dftmtx+, if we view the quantities \verb+prev+ and \verb+curr+ as part of a vector of length $2N$, but I didn't want to go to the hassle of working out the details.
This may be where some of my error comes from -- we would probably accrue somewhat less roundoff error if we just did the diagonalization trick rather than going through this painfully long \verb+for+ loop.
We use backwards Euler to get the first time step, because forwards Euler seems to give too much error already.

\begin{verbatim}
%%% CRANK-NICHOLSON %%%
right_cn_vect = zeros(N, 1);
right_cn_vect(1) = 1 - sigma;
right_cn_vect(2) = sigma / 2;
right_cn_vect(N) = sigma / 2;
right_cn = circulant(right_cn_vect);
left_cn_vect = zeros(N, 1);
left_cn_vect(1) = 1 + sigma;
left_cn_vect(2) = -sigma / 2;
left_cn_vect(N) = -sigma / 2;
left_cn = circulant(left_cn_vect);
time_advance = left_cn \ right_cn;
D = dftmtx(N)' * time_advance * dftmtx(N) / N;
solution_op = dftmtx(N) * pow(D, K) * dftmtx(N)' / N;
u = real(solution_op * f);

%%% FORWARDS EULER %%%
f_euler_vect = zeros(N,1);
f_euler_vect(1) = 1 - 2 * sigma;
f_euler_vect(2) = sigma;
f_euler_vect(N) = sigma;
f_euler = circulant(f_euler_vect);
D = dftmtx(N)' * f_euler * dftmtx(N) / N;
solution_op = dftmtx(N) * pow(D, K) * dftmtx(N)' / N;
u = real(solution_op * f);

%%% BACKWARDS EULER %%%
b_euler_vect = zeros(N,1);
b_euler_vect(1) = 1 + 2 * sigma;
b_euler_vect(2) = -sigma;
b_euler_vect(N) = -sigma;
b_euler = circulant(b_euler_vect);
D = dftmtx(N)' * b_euler * dftmtx(N) / N;
solution_op = dftmtx(N) * pow(D, -K) * dftmtx(N)' / N;
u = real(solution_op * f);

%%% DUFORT-FRANKEL %%%
b_euler_vect = zeros(N,1);
b_euler_vect(1) = 1 + 2 * sigma;
b_euler_vect(2) = -sigma;
b_euler_vect(N) = -sigma;
prev = f;
curr = circulant(b_euler_vect) \ prev;
next = zeros(N,1);
A = (1 - 2 * sigma)/(1 + 2 * sigma);
B = 2 * sigma/(1 + 2 * sigma);
for n=1:K
    for j=1:N
        if j==1
            next(j) = A * prev(1) + B * (curr(N) + curr(2));
        elseif j==N
            next(j) = A * prev(N) + B * (curr(N - 1) + curr(1));
        else
            next(j) = A * prev(j) + B * (curr(j - 1) + curr(j + 1));
        end
    end
    prev = curr;
    curr = next;
end
u = curr;
\end{verbatim}

Here are our auxiliary functions. We already gave code for \verb+circulant+ last time so this time we omit it.
When $h = .01$ we get $N = 628$ and when $h = .002$ we get $N = 3142$.
This justifies our use of these parameters in \verb+l2_norm+.

\begin{verbatim}
% A function that takes in a vector x and returns the circulant matrix
% whose first row is x.
function A = circulant(x)

% A function which takes a diagonal matrix to a high, possibly negative, power.
function A = pow(D, K)
    N = size(D);
    A = zeros(N);
    for i=1:N
        A(i)(i) = D(i)(i)^K
    end
end

% A function that computes the L^2 error of a vector u of length 628 to a
% vector v of length 3142
function m = l2_norm(u, v)
    m = 0;
    for i=1:628
        j = 5 * (i - 1) + 1;
        avg = mean([v(j) v(j + 1) v(j + 2) v(j + 3) v(j + 4)]);
        m = m + (u(i) - avg)^2/628;
    end
    m = sqrt(m);
end
\end{verbatim}



\end{document}
