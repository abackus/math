
% --------------------------------------------------------------
% This is all preamble stuff that you don't have to worry about.
% Head down to where it says "Start here"
% --------------------------------------------------------------

\documentclass[10pt]{article}

\usepackage[margin=.7in]{geometry}
\usepackage{amsmath,amsthm,amssymb,mathrsfs}
\usepackage{enumitem}
\usepackage{tikz-cd}
\usepackage{mathtools}
\usepackage{amsfonts}
\usepackage{listings}
\usepackage{algorithm2e}
\usepackage{verse,stmaryrd}
\usepackage{fancyvrb}

% Number systems
\newcommand{\NN}{\mathbb{N}}
\newcommand{\ZZ}{\mathbb{Z}}
\newcommand{\QQ}{\mathbb{Q}}
\newcommand{\RR}{\mathbb{R}}
\newcommand{\CC}{\mathbb{C}}
\newcommand{\PP}{\mathbb P}
\newcommand{\FF}{\mathbb F}
\newcommand{\DD}{\mathbb D}
\renewcommand{\epsilon}{\varepsilon}

\newcommand{\Aut}{\operatorname{Aut}}
\newcommand{\coker}{\operatorname{coker}}
\newcommand{\CVect}{\CC\operatorname{-Vect}}
\newcommand{\Cantor}{\mathcal{C}}
\newcommand{\D}{\mathcal{D}}
\newcommand{\card}{\operatorname{card}}
\newcommand{\dbar}{\overline \partial}
\DeclareMathOperator*{\esssup}{ess\,sup}
\newcommand{\GL}{\operatorname{GL}}
\newcommand{\Hom}{\operatorname{Hom}}
\newcommand{\id}{\operatorname{id}}
\newcommand{\Ind}{\operatorname{Ind}}
\newcommand{\Inn}{\operatorname{Inn}}
\newcommand{\interior}{\operatorname{int}}
\newcommand{\lcm}{\operatorname{lcm}}
\newcommand{\mesh}{\operatorname{mesh}}
\newcommand{\LL}{\mathcal L_0}
\newcommand{\Leb}{\mathcal{L}_{\text{loc}}^2}
\newcommand{\Lip}{\operatorname{Lip}}
\newcommand{\ppGL}{\operatorname{PGL}}
\newcommand{\ppic}{\vspace{35mm}}
\newcommand{\ppset}{\mathcal{P}}
\DeclareMathOperator{\proj}{proj}
\DeclareMathOperator*{\Res}{Res}
\newcommand{\Riem}{\mathcal{R}}
\newcommand{\RVect}{\RR\operatorname{-Vect}}
\newcommand{\Sch}{\mathcal{S}}
\newcommand{\SL}{\operatorname{SL}}
\newcommand{\sgn}{\operatorname{sgn}}
\newcommand{\spn}{\operatorname{span}}
\newcommand{\Spec}{\operatorname{Spec}}
\newcommand{\supp}{\operatorname{supp}}
\newcommand{\TT}{\mathcal T}
\DeclareMathOperator{\tr}{tr}

\DeclareMathOperator{\adj}{adj}
\DeclareMathOperator{\curl}{curl}

% Calculus of variations
\DeclareMathOperator{\pp}{\mathbf p}
\DeclareMathOperator{\zz}{\mathbf z}
\DeclareMathOperator{\uu}{\mathbf u}
\DeclareMathOperator{\vv}{\mathbf v}
\DeclareMathOperator{\ww}{\mathbf w}

\DeclareMathOperator{\Olo}{\mathscr O}

% Categories
\newcommand{\Ab}{\mathbf{Ab}}
\newcommand{\Cat}{\mathbf{Cat}}
\newcommand{\Group}{\mathbf{Group}}
\newcommand{\Module}{\mathbf{Module}}
\newcommand{\Set}{\mathbf{Set}}
\DeclareMathOperator{\Fun}{Fun}
\DeclareMathOperator{\Iso}{Iso}

% Complex analysis
\renewcommand{\Re}{\operatorname{Re}}
\renewcommand{\Im}{\operatorname{Im}}

% Logic
\renewcommand{\iff}{\leftrightarrow}
\newcommand{\Henkin}{\operatorname{Henk}}
\newcommand{\PA}{\mathbf{PA}}
\DeclareMathOperator{\proves}{\vdash}

% Group
\DeclareMathOperator{\Gal}{Gal}
\DeclareMathOperator{\Fix}{Fix}
\DeclareMathOperator{\Out}{Out}

% Other symbols
\newcommand{\heart}{\ensuremath\heartsuit}

\DeclareMathOperator{\atanh}{atanh}

% Theorems
\theoremstyle{definition}
\newtheorem*{corollary}{Corollary}
\newtheorem*{falselemma}{Grader's ``Lemma"}
\newtheorem{exer}{Exercise}
\newtheorem{lemma}{Lemma}[exer]
\newtheorem{theorem}[lemma]{Theorem}

\usepackage[backend=bibtex,style=alphabetic,maxcitenames=50,maxnames=50]{biblatex}
\renewbibmacro{in:}{}
\DeclareFieldFormat{pages}{#1}

\begin{document}
\noindent
\large\textbf{Algebraic geometry, HW 2} \hfill \textbf{Aidan Backus} \\
% --------------------------------------------------------------
%                         Start here
% --------------------------------------------------------------\

\begin{exer}[4.1]
Show that every finite morphism is proper.
\end{exer}

Let us assume that $X, Y$ are noetherian and $f: X \to Y$ is a finite morphism.
Suppose that $x$ is a $K$-valued point of $X$, so $f \circ x$ is a $K$-valued point of $Y$ and so its image is contained in some open affine $U \subseteq Y$.
So there exist rings $A, B$ such that $U = \Spec B$, $f^{-1}(U) = \Spec A$, and $A$ is a finitely generated $B$-module; and in particular, we can corestrict $x$ to a $K$-valued point of $f^{-1}(U)$ and $f \circ x$ to a $K$-valued point of $U$.

Suppose that we have a valuation ring $R$ in $K$ and a morphism $y: \Spec R \to Y$ making
$$\begin{tikzcd}
\Spec K \arrow[r,"x"] \arrow[d] & X \arrow[d,"f"] \\
\Spec R \arrow[r,"y"] & Y
\end{tikzcd}$$
commute.
Since $R$ is a valuation ring, $R$ is a local ring, and so the image of $y$ is contained in the closure of the image of $f \circ x$.
In particular, the image of $y$ is contained in $U$, so we can corestrict the above diagram to a new commutative diagram:
$$\begin{tikzcd}
\Spec K \arrow[r,"x"] \arrow[d] & \Spec A \arrow[d,"f"] \\
\Spec R \arrow[r,"y"] & \Spec B
\end{tikzcd}$$
Taking the inverse of the functor $\Spec$ we get yet another commutative diagram, now in the category of rings:
$$\begin{tikzcd}
K & A \arrow[l,"x^\sharp"] \\
R \arrow[u] & B \arrow[l, "y^\sharp"] \arrow[u,"f^\sharp"]
\end{tikzcd}$$
We claim that in fact, the image $A'$ of $x^\sharp$ is contained in $R$.
To see this, suppose that $a_1, \dots, a_n$ generate $A$ as a $B$-module.
Then $x^\sharp(a_1), \dots, x^\sharp(a_n)$ generate $A'$ of $A$ as a $B$-module and hence as an $R$-module.
So $A'$ is a finitely generated $R$-module and hence every element of $A'$ satisfies an integral relation over $R$.
Since valuation rings are integrally closed it follows that $A' \subseteq R$.
Thus we obtain a map filling in the diagram
$$\begin{tikzcd}
K & A \arrow[l,"x^\sharp"] \arrow[dl,"\overline x^\sharp"] \\
R \arrow[u] & B \arrow[l, "y^\sharp"] \arrow[u,"f^\sharp"]
\end{tikzcd}$$
and hence, taking $\Spec$ of the whole thing, we can fill in the diagram
$$\begin{tikzcd}
\Spec K \arrow[r,"x"] \arrow[d] & \Spec A \arrow[d,"f"] \\
\Spec R \arrow[r,"y"] \arrow[ur,"\overline x"] & \Spec B
\end{tikzcd}$$
and then coextend this diagram to finally get the long-coveted, hard-earned diagram
$$\begin{tikzcd}
\Spec K \arrow[r,"x"] \arrow[d] & X \arrow[d,"f"] \\
\Spec R \arrow[r,"y"] \arrow[ur,"\overline x"] & Y
\end{tikzcd}$$
It remains to show that $\overline x$ is unique.
This amounts to showing that $f$ is separated, and that amounts to showing that $f$ is locally separated, thus we may assume $Y = \Spec B$.
But $f$ is affine, so then $X = \Spec A$ and thus $f$ is the spectrum of a morphism of rings.
Therefore $f$ is separated.

\end{document}
