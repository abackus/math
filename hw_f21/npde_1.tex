
% --------------------------------------------------------------
% This is all preamble stuff that you don't have to worry about.
% Head down to where it says "Start here"
% --------------------------------------------------------------

\documentclass[10pt]{article}

\usepackage[margin=.7in]{geometry}
\usepackage{amsmath,amsthm,amssymb,mathrsfs}
\usepackage{enumitem}
\usepackage{tikz-cd}
\usepackage{mathtools}
\usepackage{amsfonts}
\usepackage{listings}
\usepackage{algorithm2e}
\usepackage{verse,stmaryrd}
\usepackage{fancyvrb}

% Number systems
\newcommand{\NN}{\mathbb{N}}
\newcommand{\ZZ}{\mathbb{Z}}
\newcommand{\QQ}{\mathbb{Q}}
\newcommand{\RR}{\mathbb{R}}
\newcommand{\CC}{\mathbb{C}}
\newcommand{\PP}{\mathbb P}
\newcommand{\FF}{\mathbb F}
\newcommand{\DD}{\mathbb D}
\renewcommand{\epsilon}{\varepsilon}

\newcommand{\Aut}{\operatorname{Aut}}
\newcommand{\coker}{\operatorname{coker}}
\newcommand{\CVect}{\CC\operatorname{-Vect}}
\newcommand{\Cantor}{\mathcal{C}}
\newcommand{\D}{\mathcal{D}}
\newcommand{\card}{\operatorname{card}}
\newcommand{\dbar}{\overline \partial}
\DeclareMathOperator*{\esssup}{ess\,sup}
\newcommand{\GL}{\operatorname{GL}}
\newcommand{\Hom}{\operatorname{Hom}}
\newcommand{\id}{\operatorname{id}}
\newcommand{\Ind}{\operatorname{Ind}}
\newcommand{\Inn}{\operatorname{Inn}}
\newcommand{\interior}{\operatorname{int}}
\newcommand{\lcm}{\operatorname{lcm}}
\newcommand{\mesh}{\operatorname{mesh}}
\newcommand{\LL}{\mathcal L_0}
\newcommand{\Leb}{\mathcal{L}_{\text{loc}}^2}
\newcommand{\Lip}{\operatorname{Lip}}
\newcommand{\ppGL}{\operatorname{PGL}}
\newcommand{\ppic}{\vspace{35mm}}
\newcommand{\ppset}{\mathcal{P}}
\DeclareMathOperator{\proj}{proj}
\DeclareMathOperator*{\Res}{Res}
\newcommand{\Riem}{\mathcal{R}}
\newcommand{\RVect}{\RR\operatorname{-Vect}}
\newcommand{\Sch}{\mathcal{S}}
\newcommand{\SL}{\operatorname{SL}}
\newcommand{\sgn}{\operatorname{sgn}}
\newcommand{\spn}{\operatorname{span}}
\newcommand{\Spec}{\operatorname{Spec}}
\newcommand{\supp}{\operatorname{supp}}
\newcommand{\TT}{\mathcal T}
\DeclareMathOperator{\tr}{tr}

\DeclareMathOperator{\adj}{adj}
\DeclareMathOperator{\curl}{curl}

% Calculus of variations
\DeclareMathOperator{\pp}{\mathbf p}
\DeclareMathOperator{\zz}{\mathbf z}
\DeclareMathOperator{\uu}{\mathbf u}
\DeclareMathOperator{\vv}{\mathbf v}
\DeclareMathOperator{\ww}{\mathbf w}

\DeclareMathOperator{\Olo}{\mathscr O}

% Categories
\newcommand{\Ab}{\mathbf{Ab}}
\newcommand{\Cat}{\mathbf{Cat}}
\newcommand{\Group}{\mathbf{Group}}
\newcommand{\Module}{\mathbf{Module}}
\newcommand{\Set}{\mathbf{Set}}
\DeclareMathOperator{\Fun}{Fun}
\DeclareMathOperator{\Iso}{Iso}

% Complex analysis
\renewcommand{\Re}{\operatorname{Re}}
\renewcommand{\Im}{\operatorname{Im}}

% Logic
\renewcommand{\iff}{\leftrightarrow}
\newcommand{\Henkin}{\operatorname{Henk}}
\newcommand{\PA}{\mathbf{PA}}
\DeclareMathOperator{\proves}{\vdash}

% Group
\DeclareMathOperator{\Gal}{Gal}
\DeclareMathOperator{\Fix}{Fix}
\DeclareMathOperator{\Out}{Out}

% Other symbols
\newcommand{\heart}{\ensuremath\heartsuit}

\DeclareMathOperator{\atanh}{atanh}

% Theorems
\theoremstyle{definition}
\newtheorem*{corollary}{Corollary}
\newtheorem*{falselemma}{Grader's ``Lemma"}
\newtheorem{exer}{Exercise}
\newtheorem{lemma}{Lemma}[exer]
\newtheorem{theorem}[lemma]{Theorem}

\usepackage[backend=bibtex,style=alphabetic,maxcitenames=50,maxnames=50]{biblatex}
\renewbibmacro{in:}{}
\DeclareFieldFormat{pages}{#1}

\begin{document}
\noindent
\large\textbf{Numerical analysis of PDE, HW 1} \hfill \textbf{Aidan Backus} \\
% --------------------------------------------------------------
%                         Start here
% --------------------------------------------------------------\

\begin{exer}
Consider the method
$$v_j^{n+1} = v_j^n + \frac{k}{h}(v^n_\ell - v^n_{\ell - 1})$$
for the transport equation. In order for $|\hat Q| \leq 1$, should $\ell = j + 1$ or $\ell = j$?
\end{exer}

If $\ell = j + 1$ then $Q = 1 + kD_+$ while if $\ell = j$ then $Q = 1 + kD_-$. So let $Q_\pm = 1 + kD_\pm$.
Taking the Fourier transform,
$$\hat Q_\pm(\omega) = 1 \pm k(e^{\pm i\xi} - 1).$$
We now simplify
\begin{align*}
|\hat Q_\pm(\omega)|^2 &= |1 \pm k(\cos \xi - 1 \pm i\sin \xi)|^2 \\
&= |1 \pm k (\cos \xi - 1)|^2 + k^2 \sin^2 \xi\\
&= 1 + k^2(\cos \xi - 1)^2 + k^2 \sin^2 \xi \pm 2k (\cos \xi - 1)\\
&= 1 + k^2(\cos^2 \xi - 2\cos \xi + 1 + \sin^2 \xi) \pm 2k(\cos \xi - 1)\\
&= 1 + k^2 + k - 2k \cos \xi \pm 2k(\cos \xi - 1).
\end{align*}
Therefore
$$|\hat Q_+(\omega)|^2 = 1 - k + k^2$$
and since we are interested in the case when $k \ll 1$, so in particular $k < 1$, $k^2 < k$ and hence $|\hat Q_+(\omega)|^2 < 1$.
Meanwhile,
$$|\hat Q_-(\omega)|^2 = 1 + 3k + k^2 - 4k \cos \xi$$
so $|\hat Q_-(\omega)|^2 > 1$ if $\cos \xi = -1$ (say).
Therefore we want $\ell = j + 1$.


\begin{exer}
Compute the truncation error $\tau^n$ for the upwind method
$$v^{n + 1} = v^n + kD_+v^n.$$
Can the upwind method be more accurate than the Lax-Wendroff method?
\end{exer}

Assume that $u$ solves the transport equation, so that
$$u(x, t) = f(x + t).$$
Plugging this into the definition of truncation error, we conclude that
$$f(hj + (n + 1)k) - f(hj + nk) = \frac{k}{h}(f(h(j+1) - nk) - f(hj - nk)) + k\tau^n_j$$
and hence
$$\tau^n_j = \frac{f(hj + (n + 1)k - f(hj + nk))}{k} - \frac{f(h(j + 1) - nk) - f(hj - nk)}{h}.$$
To first order, this is
$$\tau^n_j = f'(hj + nk) + O(k^2) - f'(hj + nk) + O(h^2) = O(k^2 + h^2)$$
which is comparable to the error of the Lax-Wendroff method.
So whether the upwind method is more accurate than the Lax-Wendroff method depends on the second derivatives of $f$; in general, it's possible.

\begin{exer}
Consider the method
$$v^{n + 1} = Qv^n$$
where $Q = \sum_{\nu = -r}^s A_\nu(k, h) E^\nu$. Let $N$ be even and show that
\begin{equation}\label{exer 3 concl}
v^n_j = \frac{1}{\sqrt{2\pi}} \sum_{\omega = -N/2}^{N/2} \hat Q(\omega)^n e^{i\omega x_j} \tilde f(\omega)
\end{equation}
where
\begin{equation}\label{exer 3 hyp}
f_j = \frac{1}{\sqrt{2\pi}} \sum_{\omega = -N/2}^{N/2} e^{i\omega x_j} \tilde f(\omega)
\end{equation}
and $\hat Q$ is to be determined.
\end{exer}

We proceed by induction on $n$.
If $n = 1$ then by definition of $Q$ and (\ref{exer 3 hyp}),
$$v^1_j = Qf_j = \sum_{\nu = -r}^s A_\nu(k, h) E^\nu f_j = \frac{1}{\sqrt{2\pi}} \sum_{\omega = -N/2}^{N/2} \sum_{\nu = -r}^s A_\nu(k, h) E^\nu e^{i\omega x_j} \tilde f(\omega).$$
But
$$E^\nu e^{i \omega x_j} = e^{i\nu h\omega} e^{i\omega x_j}$$
so that if we set
$$\hat Q(\omega) = \sum_{\nu = -r}^s A_\nu(k, h) e^{i\nu h \omega},$$
we obtain (\ref{exer 3 concl}) with $n = 1$.

Now suppose that $n \geq 2$.
In particular, by induction, $v^{n - 1}$ satisfies (\ref{exer 3 concl}) with $n$ replaced by $n - 1$.
Then applying the $n = 1$ case with $\tilde f$ replaced by $\hat Q^{n - 1} \tilde f$ we see that $v^n$ satisfies (\ref{exer 3 concl}).

\begin{exer}
Consider the scheme
$$(1 - \theta kD_0) v^{n + 1} = (1 + (1 - \theta)kD_0) v^n.$$
Compute $\hat Q$ and give an explicit formula for $v^n$ in terms of (\ref{exer 3 hyp}).
Show that this scheme is unconditionally stable if $\theta \in [0.5, 1]$.
\end{exer}

Let $A = 1 - \theta kD_0$, so $A$ is the circulant matrix of length $N + 1$ whose first row is
$$A_1 = (1, -\theta \lambda/2, 0, \dots, 0, \theta \lambda/2).$$
Since $A$ is circulant, the eigenvectors of $A$ are given by the Fourier transform
$$\mathcal F(\omega) = \frac{1}{\sqrt{N + 1}} (1, e^{i\xi}, \dots, e^{iN\xi})$$
and the eigenvalues are
$$\mu(\omega) = 1 - i\theta \lambda \sin \xi.$$
Similarly if $B = 1 + (1 - \theta)k D_0$, $B$ is the circulant matrix whose first row is
$$B_1 = (1, (1 - \theta)\lambda/2, 0, \dots, 0, -(1 - \theta)\lambda/2).$$
The eigenvalues are given by
$$\delta(\omega) = 1 + i(1 - \theta)\lambda \sin \xi.$$
Let $\hat A, \hat B$ be the diagonalized matrices.
The scheme is implicitly defined by the equation
$$Av^{n+1} = Bv^n$$
and if we take the Fourier transform of both sides we get
$$\tilde v^{n + 1} = \hat A^{-1} \hat B \tilde v^n.$$
So
$$\hat Q(\omega) = \hat A^{-1}(\omega) \hat B(\omega) = \frac{1 + i(1 - \theta)\lambda \sin \xi}{1 - i\theta \lambda \sin \xi}$$
and hence, if $\theta \in [0, 1]$,
$$|\hat Q(\omega)|^2 = \frac{1 + (1 - \theta)^2\lambda^2 \sin^2 \xi}{1 + \theta^2 \lambda^2 \sin^2 \xi}.$$
If $\theta \geq 1/2$ then $\theta^2 \lambda^2 \sin^2 \xi \geq (1 - \theta)^2 \lambda^2 \sin^2 \xi$ and so the scheme is unconditionally stable.
By the previous question,
$$v^n_j = \frac{1}{\sqrt{2\pi}} \sum_{\omega = -N/2}^{N/2} \left(\frac{1 + i(1 - \theta)\lambda \sin \xi}{1 - i\theta \lambda \sin \xi}\right)^n e^{i\omega x_j} \tilde f(\omega).$$

\begin{exer}
Derive the linear system for the Crank-Nicholson scheme.
\end{exer}

In the notation of the previous exercise with $\theta = 1/2$, the matrix of the Crank-Nicholson system is characterized by
\begin{align*}
A_1 &= (1, -\lambda/4, 0, \dots, 0, \lambda/4)\\
B_1 &= (1, \lambda/4, 0, \dots, 0, -\lambda/4).
\end{align*}
Here $A,B$ are circulant matrices with first row $A_1,B_1$, and the Crank-Nicholson scheme is
$$Av^{n + 1} = Bv^n,$$
thus $Q = A^{-1}B$.

It's not part of the problem but it seemed interesting to me that the Fourier transform of the Crank-Nicholson scheme has a particularly simple form.
In fact, the tangent half-angle formula says that
$$\hat Q(\omega) = \exp\left(2 \atanh \left(\frac{i\lambda}{2} \sin \xi\right)\right).$$

\begin{exer}
Code the upwind scheme, the naive centered scheme, the Lax-Wendroff scheme, the Lax-Frederichs scheme, backwards Euler scheme, and the Crank-Nicholson scheme.
With $\lambda = 1/2$, $h = 2^{-j}$, $j \in \{5, 6, 7\}$, apply these methods to the transport equation with initial data
$$f(x) = \begin{cases} x, & x \in [0, \pi]\\
2\pi - x, & x \in (\pi, 2\pi].
\end{cases}$$
Plot your solutions at time $t = 1$ and give the error in the $L^2_h$ norm with $t = 1$, $h = 2^{-7}$.
Which methods do best?
\end{exer}

This is the truncation error's $L^2_h$-norm with $h = 2^{-7}$ and $t = 1$:

\begin{center}
\begin{tabular}{ l c r }
  Scheme & Error \\
  \hline
  Upwind & $0.0175$ \\
  Naive & $5.16 \cdot 10^8$ \\
  Lax-Frederichs & $0.0347$ \\
  Lax-Wendroff & $0.0101$ \\
  Crank-Nicholson & $0.0106$ \\
  Backwards Euler & $0.0174$
\end{tabular}
\end{center}

The best schemes are Lax-Wendroff and Crank-Nicholson. It makes sense that Lax-Wendroff would be stronger than Lax-Frederichs, since both use vanishing viscosity, but the viscosity coefficient is smaller for Lax-Wendroff.
I'm not sure why Crank-Nicholson is better than backwards Euler; maybe this is just an artifact of how I carried out matrix multiplication.

We used the following auxiliary code:

\begin{verbatim}
% A function that takes in a vector x and returns the circulant matrix
% whose first row is x.

function A = circulant(x)
    A = zeros(length(x), length(x));
    for n = 1:length(x)
        for m = 1:length(x)
            A(n, m) = x(m);
        end
        x = circshift(x, 1);
    end
end

% Modify mod to avoid off-by-one bugs
% Intead of using 0 as a rep, uses the order k of the cyclic group as a rep

function l = obob_mod(n, k)
    l = mod(n, k);
    if l == 0
        l = k;
    end
end
\end{verbatim}

Here were our implementations of the various schemes:

\begin{verbatim}
% An implementation of an upwind scheme for the transport equation
% on a circle.
% f is the initial data, lambda is the ratio
% of spatial to temporal grid sizes, T is the final time.

function v = upwind_scheme(f, lambda, T)
    h = 2 * pi/length(f); % Spatial grid size
    k = h * lambda; % Temporal grid size
    timesteps = round(T / k);

    v = zeros(timesteps, length(f));

    % Initialize v
    for j = 1:length(f)
        v(1, j) = f(j);
    end

    for n = 2:timesteps
        for j = 1:length(f)
            next = obob_mod(j + 1, length(f));
            v(n, j) = v(n - 1, j) + lambda * (v(n - 1, next) - v(n - 1, j));
        end
    end
end

% An implementation of a naive central scheme for the transport equation
% on a circle.
% f is the initial data, lambda is the ratio
% of spatial to temporal grid sizes, T is the final time.

function v = naive_scheme(f, lambda, T)
    h = 2 * pi/length(f); % Spatial grid size
    k = h * lambda; % Temporal grid size
    timesteps = round(T / k);

    v = zeros(timesteps, length(f));

    % Initialize v
    for j = 1:length(f)
        v(1, j) = f(j);
    end

    for n = 2:timesteps
        for j = 1:length(f)
            next = obob_mod(j + 1, length(f));
            last = obob_mod(j - 1, length(f));
            v(n, j) = v(n - 1, j) + lambda * (v(n - 1, next) - v(n - 1, last)) / 2;
        end
    end
end

% An implementation of Lax schemes for the transport equation
% on a circle.
% f is the initial data, lambda is the ratio
% of spatial to temporal grid sizes, T is the final time.
% The viscosity should be "Fred" or "Wend".

function v = lax_schemes(f, lambda, type, T)
    h = 2 * pi/length(f); % Spatial grid size
    k = h * lambda; % Temporal grid size
    timesteps = round(T / k);

    v = zeros(timesteps, length(f));

    % Initialize v
    for j = 1:length(f)
        v(1, j) = f(j);
    end

    for n = 2:timesteps
        for j = 1:length(f)
            next = obob_mod(j + 1, length(f));
            last = obob_mod(j - 1, length(f));
            v(n, j) = v(n - 1, j) + lambda * (v(n - 1, next) - v(n - 1, last)) / 2;
            viscosity = (v(n - 1, next) + v(n - 1, last) - 2 * v(n - 1, j)) / 2;
            if type == "Wend"
                v(n, j) = v(n, j) + lambda^2 * viscosity;
            elseif type == "Fred"
                v(n, j) = v(n, j) + viscosity;
            end
        end
    end
end

% An implementation of an implicit \theta-scheme for the transport equation
% on a circle.
% f is the initial data, theta is the \theta-parameter, lambda is the ratio
% of spatial to temporal grid sizes, T is the final time.
% Need \theta \in [0.5, 1] to get stability.

function v = theta_scheme(f, theta,lambda, T)
    h = 2 * pi/length(f); % Spatial grid size
    k = h * lambda; % Temporal grid size
    timesteps = round(T / k);

    v = zeros(timesteps, length(f));

    % Initialize v
    for j = 1:length(f)
        v(1, j) = f(j);
    end

    % Create the difference operator
    x = zeros(length(f));
    x(1) = 1;
    x(2) = -theta * lambda/2;
    x(length(f)) = theta * lambda/2;
    A = circulant(x);
    x(2) = (1 - theta) * lambda/2;
    x(length(f)) = - (1 - theta) * lambda/2;
    B = circulant(x);
    Q = A\B;

    for n = 2:timesteps
        x = zeros(length(f));
        for j = 1:length(f)
            x(j) = v(n - 1, j);
        end
        x = Q * x;
        for j = 1:length(f)
            v(n, j) = x(j);
        end
    end
end
\end{verbatim}

Finally, we called a scheme using the following code:

\begin{verbatim}
h = 2^(-7);
x = 0:h:2 * pi;
f = zeros(length(x),1);
u = zeros(length(x),1); % Represents u at time 1.

for j = 1:length(x)
    if x(j) <= pi
        f(j) = x(j);
        u(mod(j - 1/h, length(x)) + 1) = x(j);
    else
        f(j) = 2 * pi - x(j);
        u(mod(j - 1/h, length(x)) + 1) = 2 * pi - x(j);
    end
end

v = 0; % SCHEME FUNCTION CALL HERE

uapprox = v(end,:);

err = sqrt(h) * norm(u - transpose(uapprox));

plot(x, u, x, uapprox);
legend({'Analytic solution', 'Numeric solution'});
disp(err);
\end{verbatim}



\end{document}
