
% --------------------------------------------------------------
% This is all preamble stuff that you don't have to worry about.
% Head down to where it says "Start here"
% --------------------------------------------------------------

\documentclass[10pt]{article}

\usepackage[margin=.7in]{geometry}
\usepackage{amsmath,amsthm,amssymb,mathrsfs}
\usepackage{enumitem}
\usepackage{tikz-cd}
\usepackage{mathtools}
\usepackage{amsfonts}
\usepackage{listings}
\usepackage{algorithm2e}
\usepackage{verse,stmaryrd}
\usepackage{fancyvrb}

% Number systems
\newcommand{\NN}{\mathbb{N}}
\newcommand{\ZZ}{\mathbb{Z}}
\newcommand{\QQ}{\mathbb{Q}}
\newcommand{\RR}{\mathbb{R}}
\newcommand{\CC}{\mathbb{C}}
\newcommand{\PP}{\mathbb P}
\newcommand{\FF}{\mathbb F}
\newcommand{\DD}{\mathbb D}
\renewcommand{\epsilon}{\varepsilon}

\newcommand{\Aut}{\operatorname{Aut}}
\newcommand{\coker}{\operatorname{coker}}
\newcommand{\CVect}{\CC\operatorname{-Vect}}
\newcommand{\Cantor}{\mathcal{C}}
\newcommand{\D}{\mathcal{D}}
\newcommand{\card}{\operatorname{card}}
\newcommand{\dbar}{\overline \partial}
\DeclareMathOperator*{\esssup}{ess\,sup}
\newcommand{\GL}{\operatorname{GL}}
\newcommand{\Hom}{\operatorname{Hom}}
\newcommand{\id}{\operatorname{id}}
\newcommand{\Ind}{\operatorname{Ind}}
\newcommand{\Inn}{\operatorname{Inn}}
\newcommand{\interior}{\operatorname{int}}
\newcommand{\lcm}{\operatorname{lcm}}
\newcommand{\mesh}{\operatorname{mesh}}
\newcommand{\LL}{\mathcal L_0}
\newcommand{\Leb}{\mathcal{L}_{\text{loc}}^2}
\newcommand{\Lip}{\operatorname{Lip}}
\newcommand{\ppGL}{\operatorname{PGL}}
\newcommand{\ppic}{\vspace{35mm}}
\newcommand{\ppset}{\mathcal{P}}
\DeclareMathOperator{\proj}{proj}
\DeclareMathOperator*{\Res}{Res}
\newcommand{\Riem}{\mathcal{R}}
\newcommand{\RVect}{\RR\operatorname{-Vect}}
\newcommand{\Sch}{\mathcal{S}}
\newcommand{\SL}{\operatorname{SL}}
\newcommand{\sgn}{\operatorname{sgn}}
\newcommand{\spn}{\operatorname{span}}
\newcommand{\Spec}{\operatorname{Spec}}
\newcommand{\supp}{\operatorname{supp}}
\newcommand{\Torus}{\mathbb T}
\DeclareMathOperator{\tr}{tr}

\DeclareMathOperator{\adj}{adj}
\DeclareMathOperator{\curl}{curl}

% Calculus of variations
\DeclareMathOperator{\pp}{\mathbf p}
\DeclareMathOperator{\zz}{\mathbf z}
\DeclareMathOperator{\uu}{\mathbf u}
\DeclareMathOperator{\vv}{\mathbf v}
\DeclareMathOperator{\ww}{\mathbf w}

\DeclareMathOperator{\Olo}{\mathscr O}

% Categories
\newcommand{\Ab}{\mathbf{Ab}}
\newcommand{\Cat}{\mathbf{Cat}}
\newcommand{\Group}{\mathbf{Group}}
\newcommand{\Module}{\mathbf{Module}}
\newcommand{\Set}{\mathbf{Set}}
\DeclareMathOperator{\Fun}{Fun}
\DeclareMathOperator{\Iso}{Iso}

% Complex analysis
\renewcommand{\Re}{\operatorname{Re}}
\renewcommand{\Im}{\operatorname{Im}}

% Logic
\renewcommand{\iff}{\leftrightarrow}
\newcommand{\Henkin}{\operatorname{Henk}}
\newcommand{\PA}{\mathbf{PA}}
\DeclareMathOperator{\proves}{\vdash}

% Group
\DeclareMathOperator{\Gal}{Gal}
\DeclareMathOperator{\Fix}{Fix}
\DeclareMathOperator{\Out}{Out}

% Other symbols
\newcommand{\heart}{\ensuremath\heartsuit}

\DeclareMathOperator{\atanh}{atanh}

% Theorems
\theoremstyle{definition}
\newtheorem*{corollary}{Corollary}
\newtheorem*{falselemma}{Grader's ``Lemma"}
\newtheorem{exer}{Exercise}
\newtheorem{lemma}{Lemma}[exer]
\newtheorem{theorem}[lemma]{Theorem}

\usepackage[backend=bibtex,style=alphabetic,maxcitenames=50,maxnames=50]{biblatex}
\renewbibmacro{in:}{}
\DeclareFieldFormat{pages}{#1}

\begin{document}
\noindent
\large\textbf{Numerical analysis of PDE, Midterm} \hfill \textbf{Aidan Backus} \\
% --------------------------------------------------------------
%                         Start here
% --------------------------------------------------------------

\begin{exer}
Consider $\partial_t u = (-1)^{m + 1} \partial_x^{(2m)} u$ on $\Torus^1$.
We use the method
$$v^{n+1} = v^n + k(-1)^{m + 1}(D_+ D_-)^m v^n.$$

Compute $\hat Q(\omega)$.

Suppose that $f(x) = \hat f(\omega) e^{i\omega x}$ for some $\omega$.
Write an explicit solution for $v^n$ in terms of $\omega$ and $\hat Q$.

Find a CFL condition so that $|\hat Q| \leq 1$.
\end{exer}

Recall that, with $\xi = h\omega$,
$$\widehat{D_+ D_-}(\omega) = \frac{\cos \xi - 2}{h^2}.$$
Therefore
$$\widehat{(D_+ D_-)^m}(\omega) = \frac{(\cos \xi - 2)^m}{h^{2m}}$$
and so
$$\hat Q(\omega) = 1 + \frac{(-1)^{m + 1}k}{h^{2m}} (\cos \xi - 2)^m.$$

We know that the original PDE is constant-coefficient, so is diagonalized by the Fourier transform. That is, if $f$ is a single wave,
$$v^n_j = e^{i\omega x_j} \hat v^n(\omega)$$
where $\hat v^n(\omega)$ is to be determined. In fact,
$$\hat v^n(\omega) = \hat Q(\omega)^n \hat f(\omega)$$
and hence
$$v^n_j = e^{i\omega x_j} \hat Q(\omega)^n \hat f(\omega).$$

Now we need to get $|\hat Q| \leq 1$.
We know that
$$|\hat Q(\omega)| = \left|1 + \frac{(-1)^{m + 1}k}{h^{2m}}(\cos \xi - 2)^m\right|$$
and it is clear that
$$\sup_\xi \frac{(-1)^{m + 1}k}{h^{2m}}(\cos \xi - 2)^m \leq 0$$
so what we need to show is to find a condition which imposes
$$\inf_\xi \frac{(-1)^{m + 1}k}{h^{2m}}(\cos \xi - 2)^m \geq -2.$$
This is equivalent to imposing
$$\sup_\xi (2 - \cos \xi)^m \leq \frac{2h^{2m}}{k}.$$
The left-hand side is maximized when $\cos \xi = -1$, thus what we need is
$$3^m \leq \frac{2h^{2m}}{k}$$
or equivalently
$$k \leq \left(\frac{2h^2}{3}\right)^m.$$
This is the claimed CFL condition.

\begin{exer}
Consider $\partial_t u = \partial_x^3 u$ and the method
$$v^{n + 1} = v^n + kD_- D_+ D_- v^n.$$
Compute $\hat Q$ and try to find a CFL condition.
\end{exer}

We first compute
$$D_- D_+ D_- e^{i\omega x} = \frac{\cos \xi - 2}{h^2} D_- e^{i\omega x} = \frac{\cos \xi - 2}{h^2e^{i\xi}} e^{i\omega x}$$
where as usual $\xi = h\omega$.
This immediately gives
$$\hat Q(\omega) = 1 + \frac{k \cos \xi - 2k}{h^2 e^{i\xi}}.$$
Unfortunately,
$$\hat Q(\pi/h) = 1 - 3 \frac{k}{h^2e^{i\pi}} = 1 + 3\frac{k}{h^2} > 0,$$
so there is no CFL condition here.
(Intuitively this makes sense: if we write out $|\hat Q(\omega)|^2$ we get a complicated formula in terms of $\sigma = k/h^2$, which is not the correct scaling for a third-order PDE.)

\begin{exer}
Use the method of lines to solve the transport equation, with
$$Q = h^{-1}(-E^2/2 + 2E + a).$$
Try to find $a$ so that
$$|(Q - \partial_x)(e^{i\omega x})| \leq O(\omega^3 h^2).$$
\end{exer}

We claim that $a = -1.5$ works. Indeed, if we set $\xi = h\omega$, then
$$\hat Q(\omega) = h^{-1}(-.5e^{2i\xi} + 2e^{i\xi} - 1.5)$$
On the other hand
$$\widehat{\partial_x}(\omega) = i\omega.$$
So we want to show
$$|i\omega - h^{-1}(-.5e^{2i\xi} + 2e^{i\xi} - 1.5)| \lesssim \xi^2 \omega$$
and if we multiply both sides by $h$ this is equivalent to showing
$$|I| = |i\xi - (-.5e^{2i\xi} + 2e^{i\xi} - 1.5)| \lesssim \xi^3.$$
Now we Taylor expand the left-hand side $I$ to third order to get
\begin{align*}I &= i\xi - .5(1 + 2i\xi - 4\xi^2) + 2(1 + i\xi - \xi^2) - 1.5 + O(\xi^3) \\
&= -1.5 - .5 + 2 + O(\xi^3) = O(\xi^3)
\end{align*}
as desired.

\begin{exer}
Suppose that $P = \Delta a \Delta$ where $\Re a \leq 0$, on $\Torus^2$. Show that $P$ is semibounded.
\end{exer}

Using the self-adjointness of $\Delta$, if we write $a = f + ig$ where $f \leq 0$ and $g$ is real,
\begin{align*}
(Pu, u) + (u, Pu) &= (\Delta a \Delta u, u) + (u, \Delta a \Delta u) = (a \Delta u, \Delta u) + (\Delta u, a \Delta u)\\
&= \int_{\Torus^2} (f + ig) \Delta u \overline{\Delta u} + \Delta u (f - ig) \overline{\Delta u} \\
&= 2\int_{\Torus^2} f \Delta u \overline{\Delta u} = 2(f \Delta u, \Delta u).
\end{align*}
Multiplication by $f$ is a negative-semidefinite operator on $L^2(\Torus^2)$, since $f \leq 0$.
In general, if $H$ is a Hilbert space, $v \in H$, and $T$ is a negative-semidefinite operator on $H$, $(Tv, v) \leq 0$.
Therefore
$$(Pu, u) + (u, Pu) \leq 0 \leq ||u||^2$$
which shows that $P$ is semibounded.

\begin{exer}
Suppose $m = 2$ and consider the problem on $\Torus^1$, $\partial_t u = A \partial_x^6 u$ where $A = \begin{bmatrix}2 & 8 \\  4 & 8\end{bmatrix}$.
Is this problem stable?
\end{exer}

We first diagonalize $A = R^{-1}DR$ where $D$ is a diagonal matrix with entries $5 \pm \sqrt{41}$.
If we set $v = Ru$ and multiply both sides of the PDE by $R$, then commute $R$ with $\partial_t$ and $\partial_x$, we conclude that
$$\partial_t v = D\partial_x^6 v.$$
We further diagonalize the system by taking the Fourier transform to get
$$\partial_t \widehat{v_j}(t, \xi) = ((-1)^j \sqrt{41} - 5)\xi^6 \widehat{v_j}(t, \xi).$$
Solving this ordinary differential equation we get
$$\widehat{v_j}(t, \xi) = e^{t((-1)^j \sqrt{41} - 5)\xi^6} \widehat{v_j}(0, \xi).$$
Taking $|\xi| \to \infty$ we see that the blowup speed becomes arbitrarily fast at high frequencies, if in fact $(-1)^j\sqrt{41} - 5 > 0$.
This happens if $j = 2$.
Therefore $v_2(t, \cdot)$ blows up in $L^2$ as $t \to \infty$ and the blowup speed becomes arbitrarily fast as the frequency of $v_2$ goes to infinity.
Therefore $v$ also blows up, and $u = R^{-1}v$ must also blow up as well.
So the problem is not stable.

\end{document}
