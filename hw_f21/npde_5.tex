
% --------------------------------------------------------------
% This is all preamble stuff that you don't have to worry about.
% Head down to where it says "Start here"
% --------------------------------------------------------------

\documentclass[10pt]{article}

\usepackage[margin=.7in]{geometry}
\usepackage{amsmath,amsthm,amssymb,mathrsfs}
\usepackage{enumitem}
\usepackage{tikz-cd}
\usepackage{mathtools}
\usepackage{amsfonts}
\usepackage{listings}
\usepackage{algorithm2e}
\usepackage{verse,stmaryrd}
\usepackage{fancyvrb}

% Number systems
\newcommand{\NN}{\mathbb{N}}
\newcommand{\ZZ}{\mathbb{Z}}
\newcommand{\QQ}{\mathbb{Q}}
\newcommand{\RR}{\mathbb{R}}
\newcommand{\CC}{\mathbb{C}}
\newcommand{\PP}{\mathbb P}
\newcommand{\FF}{\mathbb F}
\newcommand{\DD}{\mathbb D}
\renewcommand{\epsilon}{\varepsilon}

\newcommand{\Aut}{\operatorname{Aut}}
\newcommand{\coker}{\operatorname{coker}}
\newcommand{\CVect}{\CC\operatorname{-Vect}}
\newcommand{\Cantor}{\mathcal{C}}
\newcommand{\D}{\mathcal{D}}
\newcommand{\card}{\operatorname{card}}
\newcommand{\dbar}{\overline \partial}
\DeclareMathOperator*{\esssup}{ess\,sup}
\newcommand{\GL}{\operatorname{GL}}
\newcommand{\Hom}{\operatorname{Hom}}
\newcommand{\id}{\operatorname{id}}
\newcommand{\Ind}{\operatorname{Ind}}
\newcommand{\Inn}{\operatorname{Inn}}
\newcommand{\interior}{\operatorname{int}}
\newcommand{\lcm}{\operatorname{lcm}}
\newcommand{\mesh}{\operatorname{mesh}}
\newcommand{\LL}{\mathcal L_0}
\newcommand{\Leb}{\mathcal{L}_{\text{loc}}^2}
\newcommand{\Lip}{\operatorname{Lip}}
\newcommand{\ppGL}{\operatorname{PGL}}
\newcommand{\ppic}{\vspace{35mm}}
\newcommand{\ppset}{\mathcal{P}}
\DeclareMathOperator{\proj}{proj}
\DeclareMathOperator*{\Res}{Res}
\newcommand{\Riem}{\mathcal{R}}
\newcommand{\RVect}{\RR\operatorname{-Vect}}
\newcommand{\Sch}{\mathcal{S}}
\newcommand{\SL}{\operatorname{SL}}
\newcommand{\sgn}{\operatorname{sgn}}
\newcommand{\spn}{\operatorname{span}}
\newcommand{\Spec}{\operatorname{Spec}}
\newcommand{\supp}{\operatorname{supp}}
\newcommand{\Torus}{\mathbb T}
\DeclareMathOperator{\tr}{tr}

\DeclareMathOperator{\adj}{adj}
\DeclareMathOperator{\curl}{curl}

% Calculus of variations
\DeclareMathOperator{\pp}{\mathbf p}
\DeclareMathOperator{\zz}{\mathbf z}
\DeclareMathOperator{\uu}{\mathbf u}
\DeclareMathOperator{\vv}{\mathbf v}
\DeclareMathOperator{\ww}{\mathbf w}

\DeclareMathOperator{\Olo}{\mathscr O}

% Categories
\newcommand{\Ab}{\mathbf{Ab}}
\newcommand{\Cat}{\mathbf{Cat}}
\newcommand{\Group}{\mathbf{Group}}
\newcommand{\Module}{\mathbf{Module}}
\newcommand{\Set}{\mathbf{Set}}
\DeclareMathOperator{\Fun}{Fun}
\DeclareMathOperator{\Iso}{Iso}

% Complex analysis
\renewcommand{\Re}{\operatorname{Re}}
\renewcommand{\Im}{\operatorname{Im}}

% Logic
\renewcommand{\iff}{\leftrightarrow}
\newcommand{\Henkin}{\operatorname{Henk}}
\newcommand{\PA}{\mathbf{PA}}
\DeclareMathOperator{\proves}{\vdash}

% Group
\DeclareMathOperator{\Gal}{Gal}
\DeclareMathOperator{\Fix}{Fix}
\DeclareMathOperator{\Out}{Out}

% Other symbols
\newcommand{\heart}{\ensuremath\heartsuit}

\DeclareMathOperator{\atanh}{atanh}

% Theorems
\theoremstyle{definition}
\newtheorem*{corollary}{Corollary}
\newtheorem*{falselemma}{Grader's ``Lemma"}
\newtheorem{exer}{Exercise}
\newtheorem{lemma}{Lemma}[exer]
\newtheorem{theorem}[lemma]{Theorem}

\usepackage[backend=bibtex,style=alphabetic,maxcitenames=50,maxnames=50]{biblatex}
\renewbibmacro{in:}{}
\DeclareFieldFormat{pages}{#1}

\begin{document}
\noindent
\large\textbf{Numerical analysis of PDE, HW 5} \hfill \textbf{Aidan Backus} \\
% --------------------------------------------------------------
%                         Start here
% --------------------------------------------------------------\

\begin{exer}
Prove the Poincar\'e inequality
$$||v||^2 \leq ||D_1^- v||_1^2 + ||D_2^- v||_1^2$$
for any $v \in \mathring P_h$.
\end{exer}

We write $v_{ij} = v(ih, jh)$.
Moreover, $v_{Nj} = 0$, and
$$|v_{mj}| \leq |v_{(m+1)j}| + h|D_1^- v_{(m+1)j}| \leq |v_{(m+1)j}| + h\max_i |D_1^- v_{ij}|,$$
so by induction on $N - m$,
$$\max_i |v_{ij}| \leq Nh \max_i |D_1^- v_{ij}| \leq \max_i |D_1^- v_{ij}|.$$
Thus
$$\max_i |v_{ij}|^2 \leq \max_i |D_1^- v_{ij}|^2 \leq h\sum_i |D_1^- v_{ij}|^2.$$
Therefore
$$\sum_i |v_{ij}|^2 \leq \sum_i |D_1^- v_{ij}|^2$$
and hence
$$\sum_{ij} |v_{ij}|^2 \leq \sum_{ij} |D_1^- v_{ij}|^2.$$
Multiplying both sides by $h^2$, we obtain
$$||v||^2 \leq ||D_1^- v||^2 \leq ||D_1^- v||^2 + ||D_2^- v||^2$$
which was to be shown.

\begin{exer}
Consider the Poisson problem $-\Delta u = f$, with $0$ boundary data.
Show that
$$||D_1^- v||_1^2 + ||D_2^- v||_1^2 \leq ||f||^2.$$
\end{exer}

Integrating by parts and using Cauchy-Schwarz,
$$||D_1^- v||_1^2 + ||D_2^- v||_1^2 = \langle -\Delta v, v\rangle = \langle f, v\rangle \leq ||f|| \cdot ||v||.$$
Integrating the Poincar\'e inequality by parts and using Cauchy-Schwarz,
$$||v||^2 \leq ||D_1^- v||_1^2 + ||D_2^- v||_1^2 = \langle f, v\rangle \leq ||f|| \cdot ||v||$$
which implies that $||v|| \leq ||f||$. Therefore
$$||D_1^- v||_1^2 + ||D_2^- v||_1^2 \leq ||f|| \cdot ||v|| \leq ||f||^2.$$

\begin{exer}
Consider the Newmark method for the wave equation $D_t q^{n+1} = \Delta_h v^{n + 1/2}$, $q^{n + 1/2} = D_t v^{n + 1}$, with initial data $v = u_0$, $q = u_1$, and boundary data $v = q = 0$.
Prove the stability estimate
$$||q^M||^2 + ||D_1^- v^M||_1^2 + ||D_2^- v^M||_1^2 \leq ||u_1||^2 + ||D_1^- u_0||_1^2 + ||D_2^- u_0||_1^2.$$
\end{exer}

It suffices to show that
\begin{equation}\label{WTS 3 1}||q^{n+1}||^2 + ||D_1^- v^{n+1}||_1^2 + ||D_2^- v^{n+1}||_1^2 \leq ||q^n||^2 + ||D_1^- v^n||_1^2 + ||D_2^- v^n||_1^2.
\end{equation}
Indeed, if this is true, then in particular it is true when $n = 0$, so by induction on $n$,
$$||q^{n+1}||^2 + ||D_1^- v^{n+1}||_1^2 + ||D_2^- v^{n+1}||_1^2 \leq ||u_1||^2 + ||D_1^- u_0||_1^2 + ||D_2^- u_0||_1^2$$
and now we can plug in $n = M - 1$.
We moreover know that
$$||D_1^- v^{n+1}||_1^2 + ||D_2^- v^{n+1}||_1^2 = -\langle \Delta v^{n+1}, v^{n+1}\rangle$$
since $v^{n+1} \in \mathring P$.
Plugging this identity into (\ref{WTS 3 1}), we see that it suffices to show that
$$
||q^{n + 1}||^2 - ||q^n||^2 \leq \langle \Delta v^{n+1}, v^{n+1}\rangle - \langle \Delta v^n, v^n\rangle,$$
or equivalently that
\begin{equation}\label{WTS 3 2}
D_t ||q^{n+1}||^2 \leq D_t \langle \Delta v^{n+1}, v^{n+1}\rangle.
\end{equation}
From the discrete product rule the left-hand side of (\ref{WTS 3 2}) is
$$D_t ||q^{n+1}||^2 = \langle q^{n+1}, D_t q^{n+1}\rangle + \langle D_t q^{n+1}, q^n\rangle = 2\langle D_t q^{n+1}, q^{n + 1/2}\rangle,$$
and so, by definition of the Newmark method,
$$D_t ||q^{n+1}||^2 = 2\langle D_t q^{n + 1}, D_t v^{n + 1}\rangle.$$
Applying the discrete product rule and integration by parts to the right-hand side of (\ref{WTS 3 2}),
$$D_t \langle \Delta v^{n+1}, v^{n+1}\rangle = \langle \Delta v^n, D_t v^{n+1}\rangle + \langle D_t \Delta v^{n+1}, v^{n+1}\rangle = 2\langle \Delta v^{n+1/2}, D_t v^{n + 1}\rangle.$$
This is exactly the left-hand side of (\ref{WTS 3 2}).

\begin{exer}
Consider the backwards Euler method for the heat equation
$$D_t v^{n+1} = \Delta v^{n+1}$$
with initial data $v = u_0$ and boundary data $v = 0$.
Show that if $u_0$ is nonnegative then $v^n$ is nonnegative.
\end{exer}

Writing out the backwards Euler method, we have
$$(1 - k\Delta) v^{n + 1} = v^n,$$
so by induction on $n$, it suffices to show that the matrix $M = 1 - k\Delta$ satisfies the hypotheses of Problem 1 on the previous homework.
Here $h^2\Delta_{ii} = -4$, $h^2 \Delta_{ij} = 1$ if $i \sim j$ (where $\sim$ is the adjacency relation), and $\Delta_{ij} = 0$ otherwise.
Thus
$$M_{ij} = \begin{cases}
1 + \frac{4k}{h^2}, & i = j\\
- \frac{k}{h^2}, &i \sim j \\
0, & \text{else.}
\end{cases}$$
Since $\sim$ is symmetric, so is $M$. Since $|\{j: j \sim i\}| \leq 4$, the positivity conditions $M_{ij} \leq 0$ if $i \neq j$, $M_{ii} > 0$, and $M_{ii} \geq - \sum_{j \neq i} M_{ij}$ are clear.
Moreover, the graph for $(0, 1)^2$ is connected, so given $i,j$ we can find $i_1 = i, i_2, \dots, i_\ell = j$ such that $M_{i_ki_{k+1}} \neq 0$, namely we can choose $i_k$ to be a path in the graph from $i$ to $j$.
Finally, $M$ is invertible because it has trivial kernel, which follows from the stability of backwards Euler (which in particular says that $||u|| \leq ||Mu||$ for any $u$).


\end{document}
