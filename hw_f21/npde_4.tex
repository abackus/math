
% --------------------------------------------------------------
% This is all preamble stuff that you don't have to worry about.
% Head down to where it says "Start here"
% --------------------------------------------------------------

\documentclass[10pt]{article}

\usepackage[margin=.7in]{geometry}
\usepackage{amsmath,amsthm,amssymb,mathrsfs}
\usepackage{enumitem}
\usepackage{tikz-cd}
\usepackage{mathtools}
\usepackage{amsfonts}
\usepackage{listings}
\usepackage{algorithm2e}
\usepackage{verse,stmaryrd}
\usepackage{fancyvrb}

% Number systems
\newcommand{\NN}{\mathbb{N}}
\newcommand{\ZZ}{\mathbb{Z}}
\newcommand{\QQ}{\mathbb{Q}}
\newcommand{\RR}{\mathbb{R}}
\newcommand{\CC}{\mathbb{C}}
\newcommand{\PP}{\mathbb P}
\newcommand{\FF}{\mathbb F}
\newcommand{\DD}{\mathbb D}
\renewcommand{\epsilon}{\varepsilon}

\newcommand{\Aut}{\operatorname{Aut}}
\newcommand{\coker}{\operatorname{coker}}
\newcommand{\CVect}{\CC\operatorname{-Vect}}
\newcommand{\Cantor}{\mathcal{C}}
\newcommand{\D}{\mathcal{D}}
\newcommand{\card}{\operatorname{card}}
\newcommand{\dbar}{\overline \partial}
\DeclareMathOperator*{\esssup}{ess\,sup}
\newcommand{\GL}{\operatorname{GL}}
\newcommand{\Hom}{\operatorname{Hom}}
\newcommand{\id}{\operatorname{id}}
\newcommand{\Ind}{\operatorname{Ind}}
\newcommand{\Inn}{\operatorname{Inn}}
\newcommand{\interior}{\operatorname{int}}
\newcommand{\lcm}{\operatorname{lcm}}
\newcommand{\mesh}{\operatorname{mesh}}
\newcommand{\LL}{\mathcal L_0}
\newcommand{\Leb}{\mathcal{L}_{\text{loc}}^2}
\newcommand{\Lip}{\operatorname{Lip}}
\newcommand{\ppGL}{\operatorname{PGL}}
\newcommand{\ppic}{\vspace{35mm}}
\newcommand{\ppset}{\mathcal{P}}
\DeclareMathOperator{\proj}{proj}
\DeclareMathOperator*{\Res}{Res}
\newcommand{\Riem}{\mathcal{R}}
\newcommand{\RVect}{\RR\operatorname{-Vect}}
\newcommand{\Sch}{\mathcal{S}}
\newcommand{\SL}{\operatorname{SL}}
\newcommand{\sgn}{\operatorname{sgn}}
\newcommand{\spn}{\operatorname{span}}
\newcommand{\Spec}{\operatorname{Spec}}
\newcommand{\supp}{\operatorname{supp}}
\newcommand{\Torus}{\mathbb T}
\DeclareMathOperator{\tr}{tr}

\DeclareMathOperator{\adj}{adj}
\DeclareMathOperator{\curl}{curl}

% Calculus of variations
\DeclareMathOperator{\pp}{\mathbf p}
\DeclareMathOperator{\zz}{\mathbf z}
\DeclareMathOperator{\uu}{\mathbf u}
\DeclareMathOperator{\vv}{\mathbf v}
\DeclareMathOperator{\ww}{\mathbf w}

\DeclareMathOperator{\Olo}{\mathscr O}

% Categories
\newcommand{\Ab}{\mathbf{Ab}}
\newcommand{\Cat}{\mathbf{Cat}}
\newcommand{\Group}{\mathbf{Group}}
\newcommand{\Module}{\mathbf{Module}}
\newcommand{\Set}{\mathbf{Set}}
\DeclareMathOperator{\Fun}{Fun}
\DeclareMathOperator{\Iso}{Iso}

% Complex analysis
\renewcommand{\Re}{\operatorname{Re}}
\renewcommand{\Im}{\operatorname{Im}}

% Logic
\renewcommand{\iff}{\leftrightarrow}
\newcommand{\Henkin}{\operatorname{Henk}}
\newcommand{\PA}{\mathbf{PA}}
\DeclareMathOperator{\proves}{\vdash}

% Group
\DeclareMathOperator{\Gal}{Gal}
\DeclareMathOperator{\Fix}{Fix}
\DeclareMathOperator{\Out}{Out}

% Other symbols
\newcommand{\heart}{\ensuremath\heartsuit}

\DeclareMathOperator{\atanh}{atanh}

% Theorems
\theoremstyle{definition}
\newtheorem*{corollary}{Corollary}
\newtheorem*{falselemma}{Grader's ``Lemma"}
\newtheorem{exer}{Exercise}
\newtheorem{lemma}{Lemma}[exer]
\newtheorem{theorem}[lemma]{Theorem}

\usepackage[backend=bibtex,style=alphabetic,maxcitenames=50,maxnames=50]{biblatex}
\renewbibmacro{in:}{}
\DeclareFieldFormat{pages}{#1}

\begin{document}
\noindent
\large\textbf{Numerical analysis of PDE, HW 4} \hfill \textbf{Aidan Backus} \\
% --------------------------------------------------------------
%                         Start here
% --------------------------------------------------------------\

\begin{exer}
Consider a matrix $M$ such that $M$ is symmetric and invertible, $M_{ij} \leq 0$ if $i \neq j$, $M_{ii} > 0$, $M_{ii} \geq -\sum_{i\neq j} M_{ij}$, and for every $i,j$, there exist $i = i_1, \dots, i_\ell = j$ such that $M_{i_k i_{k+1}} \neq 0$.

Show that the matrix for the finite difference method for Poisson's problem satisfies these conditions.

Show that if $M$ satisfies the above conditions and $MU = F$ where $F \geq 0$ entrywise, then $U \geq 0$ entrywise.
\end{exer}

Let $G \subset \NN$ be (an enumeration of) the set of interior vertices of $\Omega$ and let $G^* \subset G$ be the set of those vertices which are not adjacent to the boundary vertices.
Let $M = (M_{ij})_{i,j \in G}$ be the desired matrix and consider the Poisson problem
$$-\Delta u = f, \qquad u|\partial \Omega = g.$$
Let $\sim$ be the edge relation among interior vertices, and let $\partial G_i$ be the set of boundary vertices adjacent to $i$.

If $i \in G$, then the linear equation for $u_i$ is
$$4u_i - \sum_{j \sim i} u_j = f_i + \sum_{j \in \partial G_i} g_j.$$
Thus $u$ solves the equation
$$Mu = h$$
where $h_i = f_i + \sum_{j \in \partial G_i} g_j$, and $M$ is characterized by the facts that:
\begin{enumerate}
\item $M_{ii} = 4$,
\item $M_{ij} = -1$ iff $i \sim j$,
\item and otherwise $M_{ij} = 0$.
\end{enumerate}
Moreover, the fact that $G$ was constructed by a five-point stencil implies that
$$|\{j: j \sim i\}| = 4 - |\partial G_i| \leq 4.$$
The symmetry of $\sim$ implies that $M_{ij} = -1$ iff $M_{ji} = -1$ and therefore $M$ is symmetric. The positivity conditions $M_{ii} > 0$, $M_{ij} \leq 0$, and $M_{ii} \geq -\sum_{i \neq j} M_{ij}$ are now also obvious from the estimate on $|\{j: j \sim i\}|$.

Now let $i,j$ be given. Obviously if $i = j$ we can take $\ell = 2$, and then $M_{i_1i_2} = 4$.
Otherwise, let $i_2, \dots, i_{\ell - 1}$ be the vertices interior to a path from $i$ to $j$.
Then $M_{i_ki_{k + 1}} = -1$ whenever $k < \ell$.

To see that $M$ is invertible, let $v = (v_i)_{i \in G}$ be in the kernel of $M$ (which means that $v$ is harmonic and trace-free), thus
\begin{equation}
\label{MVP}
v_i = \frac{1}{4} \sum_{j \sim i} v_j.
\end{equation}
Since $G$ was the set of vertices of $\Omega \subset \RR^2$, we can embed $G$ in $\ZZ^2$ by identifying $\ZZ^2$ with all the vertices of $\RR^2$.
We extend $v$ to this lattice by setting $v_i = 0$ for every $i \notin G$, and extend $M$ to the graph Laplacian of $\ZZ^2$.
Then (\ref{MVP}) implies that $v$ satisfies the mean-value property at every vertex of $\ZZ^2$.
If $i \in G$, then $v_i$ is the mean of $v$ over vertices distance $1$ to $i$, but by induction, it follows that $v_i$ is the mean of $v$ over vertices distance $d$ to $i$, whenever $d \in \NN$.
But $G$ is a finite graph while $\ZZ^2$ is infinite in every direction, so there exists $d \in \NN$ such that for every vertex $j$ of distance $d$ to $i$, $j \notin G$, and hence $v_j = 0$.
It follows that $v_i = 0$, so the kernel of $M$ vanishes, as we aimed to show.

Finally, let $F$ be entrywise positive and $MU = F$.
If $U$ is not entrywise positive, we can find $\ell$ such that $U_\ell < 0$ is minimal possible.
Since $M_{\ell\ell} \geq -\sum_{i \neq \ell} M_{i\ell}$, it follows that
$$M_{\ell\ell} U_\ell \leq -\sum_{i \neq \ell} M_{i\ell} U_\ell \leq -\sum_{i \neq \ell} M_{i\ell} U_i.$$
But
$$M_{\ell\ell} U_\ell + \sum_{i \neq \ell} M_{i\ell} U_i = \sum_i M_{i\ell} U_i = F_\ell \geq 0,$$
so the inequalities collapse and we get
$$ M_{\ell\ell} U_\ell = -\sum_{i \neq \ell} M_{i\ell} U_i.$$

We claim that in fact, $U_i = U_\ell$ whenever $i \neq \ell$.
If not, then there exists $j$ such that $U_j > U_\ell$, so
$$-\sum_{i \neq \ell} M_{i\ell} U_i < M_{\ell \ell} U_\ell$$
which is a contradiction.

Therefore there exists $\alpha > 0$ such that for every $i$, $U_i = -\alpha$.
So
$$F_i = -\alpha\left(M_{ii} - \sum_{j \neq i} M_{ij}\right) \leq 0$$
which is a contradiction unless $F = 0$. But then $U = 0$ since the kernel of $M$ is trivial, which is a contradiction since $U_\ell < 0$.

\begin{exer}
Show that the Crank-Nicholson method
$$v^{n+1} = v^n + \frac{k}{2} \Delta_h (v^{n + 1} + v^n)$$
is unconditionally stable.
\end{exer}

Recall that $\Delta_h$ is negative-definite, and in particular orthogonally diagonalizable.
Thus the matrix
$$P = 1 - \frac{k\Delta_h}{2}$$
is positive-definite and diagonalizes to the spectrum
$$0 < \lambda_1 \leq \lambda_2 \leq \cdots \leq \lambda_L.$$
If we write $v^n = \sum_{\ell=1}^L v^n_\ell$, where $v^n_\ell$ is an eigenvector of $P$,
\begin{align*}
||v^{n+1}||_h^2 &= ||(1 + P)^{-1} (1 - P) v^n||_h^2 = \sum_{\ell=1}^L ||(1 + P)^{-1} (1 - P) v^n_\ell||_h^2\\
&\leq \frac{1 - \lambda_1}{1 + \lambda_1} \sum_{\ell=1}^L ||v^n_\ell||_h^2 \leq ||v^n||_h^2
\end{align*}
which was to be shown.

\begin{exer}
Implement the five-point stencil for the Poisson problem on $(0, 1)^2$, with exact solution
$$u(x) = \sin x_1 \cos x_1 + x_2^2$$
and $h = 2^{-j}$ where $j \in \{3, \dots, 6\}$. Make a table of errors.
Does it agree with the theory from class?
\end{exer}

The Laplacian of $u$ is
$$f(x) = -\Delta u(x) = 4 \sin x_1 \cos x_2 - 2.$$

So we use the following code (using several auxiliary functions we define after the last problem):
\begin{verbatim}
% Create the data used to solve Poisson's problem
% Trace is top, left, bottom, right
function F = poisson_data(RHS, trace, N)
    F = RHS;
    for i = 1:N
        F(1, i) = F(1, i) + N^2 * trace(1, i);
    end
    for i = 1:N
        F(i, 1) = F(i, 1) + N^2 * trace(2, i);
    end
    for i = 1:N
        F(N, i) = F(N, i) + N^2 * trace(3, i);
    end
    for i = 1:N
        F(i, N) = F(i, N) + N^2 * trace(4, i);
    end
    F = F .* N^(-2);
end

% Generate the Poisson data for Problem 3
function F = pr3_poisson_data(N)
    x = [1/(N+2):1/(N+1):1-1/(N+2)];
    row = 4 .* sin(x) .* cos(x) - 2;
    RHS = zeros(N);
    for i=1:N
        for j=1:N
            RHS(j,i) = row(i);
        end
    end
    trace = zeros([4 N]);
    trig = sin(x) .* cos(x);
    square = (1 - x) .^ 2;
    for i=1:N
        trace(1, i) = trig(i) + 1;
        trace(2, i) = square(i);
        trace(3, i) = trig(i);
        trace(4, i) = sin(1) * cos(1) + square(i);
    end
    F = poisson_data(RHS, trace, N);
end

f = grid2list(pr3_poisson_data(N), N);
u = list2grid(f / graph_laplacian(N), N);
\end{verbatim}

Here is a table of errors:
\begin{center}\begin{tabular}{c | c}
$\log_2 N$ & $||u - v||_{L^\infty}$ \\
\hline
 $3$ & $0.0181$ \\
$4$ & $0.0069$ \\
$5$ & $0.0028$ \\
$6$ & $0.0014$
\end{tabular}\end{center}
I also attached a plot of the case $\log_2 N = 6$.
The other cases look the same so I didn't bother printing them out.

In class we learned that $||u - v||_{L^\infty} \lesssim N^{-2}$, and indeed, doubling $N$ causes the error to decrease by a factor of four... until we hit $N = 2^6$ and we only get half the desired gain. I think what's going on here is that we're running up against the limitations of roundoff errors, which are going to be an issue because on the right-hand side of the linear system we're trying to solve, we have terms from $f$ that are extremely small compared to the terms coming from the trace of $u$.

\begin{exer}
Implement the forwards and backwards Euler methods for the heat equation on $(0, 1)^2$, with initial data
$$u(0, x) = x_1(x_1 - 1)x_2(x_2 - 1)$$
and no boundary data, $h = 1/20$ and $k = 1/20$, $k = 1/(4 \cdot 20^2)$.
Give $||u(t, \cdot)||_h$.
\end{exer}

For both Euler methods, we started with the code

\begin{verbatim}
Laplace = graph_laplacian(N);
x = [0:1/(N+2):1];
x = x([2:N+1]);
u = zeros(N);
for i=1:N
    for j=1:N
        u(i, j) = x(i) * (x(i) - 1) * x(j) * (x(j) - 1);
    end
end
v = grid2list(u, N);
\end{verbatim}

For forwards Euler, we ran the code
\begin{verbatim}
TimeAdvance = eye(N^2) - k * N^2 .* Laplace;
v = v * (TimeAdvance^(1/k));
u = list2grid(v, N);
\end{verbatim}
and for backwards Euler, we ran
\begin{verbatim}
TimeAdvance = inv(eye(N^2) + k * N^2 .* Laplace);
v = v * (TimeAdvance^(1/k));
u = list2grid(v, N);
\end{verbatim}

Here's a table of norms:
\begin{center}\begin{tabular}{c | c | c}
Method & $1/k$ & $||u(1, \cdot)||_{L^2}$ \\
\hline
Forwards Euler & $20$ & $3.314 \cdot 10^{16}$ \\
Forwards Euler & $1600$ & $2.3188 \cdot 10^{-6}$ \\
Backwards Euler & $20$ & $3.1256 \cdot 10^{-5}$ \\
Backwards Euler & $1600$ & $2.5622 \cdot 10^{-6}$ \\
\end{tabular}\end{center}
We also attach plots of the forward Euler solution at $T = 1$.
It's interesting that the singularity caused by instability of forwards Euler concentrates at $1$, and I don't have an explanation for why that should happen.
I didn't bother including the backwards Euler plots because they look exactly same as the stable forwards Euler.
Both the backwards and forwards Euler methods with $1/k = 1600$ take under a second to run, so there's no reason to favor one over the other.

Now we show the auxiliary methods that we use throughout this assignment:

\begin{verbatim}
% Returns true iff two indices refer to adjacent vertices in a NxN grid.
function bool = is_adjacent(i, j, N)
    bool = (i == j);
    if (j < i) % Enforce that i < j
        jtmp = i; i = j; j = jtmp;
    end
    bool = bool || (j == i + N) || ((j == i + 1) && ~(mod(i, N) == 0));
end

% Returns a graph Laplacian.
% Our convention is that this Laplacian is positive-definite!
function Laplace = graph_laplacian(N)
    Laplace = 4 .* eye(N^2);
    for i = 1:N^2
        for j = (i+1):N^2
            if is_adjacent(i, j, N)
                Laplace(i, j) = -1;
                Laplace(j, i) = -1;
            end
        end
    end
end

% Converts between functions on 2-space and lists
function v = grid2list(u, N)
    v = zeros([1 N^2]);
    for i = 1:N
        for j = 1:N
            v((i - 1) * N + j) = u(i, j);
        end
    end
end
function u = list2grid(v, N)
    u = zeros(N);
    for i=1:N
        for j = 1:N
            u(i, j) = v((i-1)*N + j);
        end
    end
end
\end{verbatim}

\end{document}
