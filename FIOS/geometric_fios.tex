\documentclass[reqno,12pt,letterpaper]{amsart}
\RequirePackage{amsmath,amssymb,amsthm,graphicx,mathrsfs,url,slashed}
\RequirePackage[usenames,dvipsnames]{xcolor}
\RequirePackage[colorlinks=true,linkcolor=Red,citecolor=Green]{hyperref}
\RequirePackage{amsxtra}
\usepackage{cancel}
\usepackage{tikz-cd}

\setlength{\textheight}{9.3in} \setlength{\oddsidemargin}{-0.25in}
\setlength{\evensidemargin}{-0.25in} \setlength{\textwidth}{7in}
\setlength{\topmargin}{-0.25in} \setlength{\headheight}{0.18in}
\setlength{\marginparwidth}{1.0in}
\setlength{\abovedisplayskip}{0.2in}
\setlength{\belowdisplayskip}{0.2in}
\setlength{\parskip}{0.05in}
\renewcommand{\baselinestretch}{1.05}

\title[Geometric FIOs]{The geometric POV on Fourier integral operators}
\author{Aidan Backus}
\date{May 2022}

\newcommand{\NN}{\mathbf{N}}
\newcommand{\ZZ}{\mathbf{Z}}
\newcommand{\QQ}{\mathbf{Q}}
\newcommand{\RR}{\mathbf{R}}
\newcommand{\CC}{\mathbf{C}}
\newcommand{\DD}{\mathbf{D}}
\newcommand{\PP}{\mathbf P}
\newcommand{\MM}{\mathbf M}
\newcommand{\II}{\mathbf I}
\newcommand{\Hyp}{\mathbf H}
\newcommand{\Sph}{\mathbf S}
\newcommand{\GL}{\mathbf{GL}}
\newcommand{\Orth}{\mathbf{O}}
\newcommand{\Ball}{\mathbf{B}}

\DeclareMathOperator{\card}{card}
\DeclareMathOperator{\cent}{center}
\DeclareMathOperator{\ch}{ch}
\DeclareMathOperator{\codim}{codim}
\DeclareMathOperator{\diag}{diag}
\DeclareMathOperator{\diam}{diam}
\DeclareMathOperator{\dom}{dom}
\DeclareMathOperator{\Exc}{Exc}
\DeclareMathOperator{\Gal}{Gal}
\DeclareMathOperator{\Hom}{Hom}
\DeclareMathOperator{\Iso}{Iso}
\DeclareMathOperator{\Jac}{Jac}
\DeclareMathOperator{\Lip}{Lip}
\DeclareMathOperator{\Met}{Met}
\DeclareMathOperator{\id}{id}
\DeclareMathOperator{\rad}{rad}
\DeclareMathOperator{\rank}{rank}
\DeclareMathOperator{\Hess}{Hess}
\DeclareMathOperator{\Radon}{Radon}
\DeclareMathOperator*{\Res}{Res}
\DeclareMathOperator{\sgn}{sgn}
\DeclareMathOperator{\singsupp}{sing~supp}
\DeclareMathOperator{\Spec}{Spec}
\DeclareMathOperator{\supp}{supp}
\DeclareMathOperator{\Tan}{Tan}
\newcommand{\tr}{\operatorname{tr}}

\newcommand{\Ric}{\mathrm{Ric}}
\newcommand{\Riem}{\mathrm{Riem}}
\newcommand*\dif{\mathop{}\!\mathrm{d}}
\newcommand{\LapQL}{\Delta^{\mathrm{ql}}}

\newcommand{\dbar}{\overline \partial}

\DeclareMathOperator{\atanh}{atanh}
\DeclareMathOperator{\csch}{csch}
\DeclareMathOperator{\sech}{sech}

\DeclareMathOperator{\Div}{div}
\DeclareMathOperator{\Gram}{Gram}
\DeclareMathOperator{\grad}{grad}
\DeclareMathOperator{\dist}{dist}
\DeclareMathOperator{\Ell}{Ell}
\DeclareMathOperator{\WF}{WF}

\newcommand{\Lagrange}{\mathscr L}
\newcommand{\DirQL}{\mathscr D^{\mathrm{ql}}}
\newcommand{\DirL}{\mathscr D}

\newcommand{\Hilb}{\mathcal H}
\newcommand{\Homology}{\mathrm H}
\newcommand{\normal}{\mathbf n}
\newcommand{\radial}{\mathbf r}
\newcommand{\evect}{\mathbf e}
\newcommand{\vol}{\mathrm{vol}}

\newcommand{\pic}{\vspace{30mm}}
\newcommand{\dfn}[1]{\emph{#1}\index{#1}}

\renewcommand{\Re}{\operatorname{Re}}
\renewcommand{\Im}{\operatorname{Im}}

\newcommand{\loc}{\mathrm{loc}}
\newcommand{\cpt}{\mathrm{cpt}}

\def\Japan#1{\left \langle #1 \right \rangle}

\newtheorem{theorem}{Theorem}[section]
\newtheorem{badtheorem}[theorem]{``Theorem"}
\newtheorem{prop}[theorem]{Proposition}
\newtheorem{lemma}[theorem]{Lemma}
\newtheorem{sublemma}[theorem]{Sublemma}
\newtheorem{claim}[theorem]{Claim}
\newtheorem{proposition}[theorem]{Proposition}
\newtheorem{corollary}[theorem]{Corollary}
\newtheorem{conjecture}[theorem]{Conjecture}
\newtheorem{axiom}[theorem]{Axiom}
\newtheorem{assumption}[theorem]{Assumption}

\theoremstyle{definition}
\newtheorem{definition}[theorem]{Definition}
\newtheorem{remark}[theorem]{Remark}
\newtheorem{example}[theorem]{Example}
\newtheorem{notation}[theorem]{Notation}

\newtheorem{exercise}[theorem]{Discussion topic}
\newtheorem{homework}[theorem]{Homework}
\newtheorem{problem}[theorem]{Problem}

\newtheorem{ack}{Acknowledgements}

\numberwithin{equation}{section}


% Mean
\def\Xint#1{\mathchoice
{\XXint\displaystyle\textstyle{#1}}%
{\XXint\textstyle\scriptstyle{#1}}%
{\XXint\scriptstyle\scriptscriptstyle{#1}}%
{\XXint\scriptscriptstyle\scriptscriptstyle{#1}}%
\!\int}
\def\XXint#1#2#3{{\setbox0=\hbox{$#1{#2#3}{\int}$ }
\vcenter{\hbox{$#2#3$ }}\kern-.6\wd0}}
\def\ddashint{\Xint=}
\def\dashint{\Xint-}

\usepackage[backend=bibtex,style=alphabetic]{biblatex}
\addbibresource{fios.bib}
\renewbibmacro{in:}{}
\DeclareFieldFormat{pages}{#1}


\begin{document}
\begin{abstract}
    These notes are my attempt to understand the geometric perspective on Fourier integral operators.
\end{abstract}

\maketitle

%%%%%%%%%%%%%%%%%%%%%%%%%%%%%%%%%%%%%%%%%%%%%%%%%%%%%%%

% \tableofcontents

\section{Symplectic geometry}
Let $M$ be a manifold which we interpret as the ``classical phase space," and suppose that we have a function 
$$H: M \to \RR$$
which is the \dfn{Hamiltonian}, thus $H(x)$ is the energy of a system in state $x$.
Then according to Hamiltonian mechanics, there should be an $\RR$-action on $M$ which are obtained by integrating the Hamilton equation and describe the evolution of the system.
To get such a vector field, it suffices to associate a vector field $X_H$ to $H$. We make the following physical assumptions:
\begin{enumerate}
\item Since energy is only defined up to a constant, it is natural to require to actually make $X_H$ depend on $\dif H$ rather than $H$ itself. Thus we have an isomorphism $\omega$ which sends $\dif H$ to $X_H$.
\item Since energy is conserved, $\mathcal L_{X_H} H = 0$.
\item The symmetries of spacetime require that $\mathcal L_{X_H} \omega = 0$.
\end{enumerate}
Our first assumption says that $\dif H \mapsto X_H$ should be given by an isomorphism of vector bundles $T'M \to TM$.
Such an isomorphism is encoded by a section $\omega$ of $T'M \otimes T'M$ such that for any tangent vector $v$, $w \mapsto \omega(v, w)$ is an isomorphism.
In that case we have
$$\iota_{X_H} \omega = \dif H.$$
In particular, 
$$0 = \mathcal L_{X_H} H =  (\dif H, X_H) = \omega(X_H, X_H)$$
so we assert that $\omega$ is a $2$-form, i.e. $\omega(v, v) = 0$.
Finally, we assert that 
$$0 = \mathcal L_{X_H} \omega = \dif (\iota_{X_H} \omega) + \iota_{X_H} \dif \omega = \dif^2 H + (\dif \omega, X_H) = (\dif \omega, X_H).$$
Since $H$ was arbitrary we conclude $\omega$ is closed.

\begin{definition}
A \dfn{symplectic $2$-form} is a closed $2$-form $\omega$ such that the map $v \mapsto \iota_v \omega$ is an isomorphism.
The pair $(M, \omega)$ is a \dfn{symplectic manifold}.
A diffeomorphism $\kappa: M \to N$ of symplectic manifolds for which
$$\omega_M = \kappa^* \omega_N$$
is a \dfn{canonical transformation} or a \dfn{symplectomorphism}.
\end{definition}

In fact, for $M$ a manifold, $T'M$ is a symplectic manifold where 
$$\omega = \dif x^\mu \wedge \dif \xi_\mu$$
where $(x^\mu)$ are coordinates on $M$ and $(\xi_\mu)$ are the dual coordinates on each cotangent space.
Thus $\omega$ is a section of the double cotangent bundle of $M$.
On the other hand, the definition of a symplectic $2$-form has no local invariants and implies that locally a symplectic manifold can be written as the cotangent bundle of some other manifold.

\begin{definition}
Let $M$ be a symplectic manifold and suppose that in some coordinate chart, $M$ is symplectomorphic to a cotangent bundle $T'N$.
Let $(x^\mu)$ be coordinates on $N$ and let $(\xi_\mu)$ be the dual coordinates on each cotangent space of $N$.
Then $(x^\mu, \xi_\mu)$ are \dfn{canonical coordinates} on $M$.
\end{definition}

The interpretation here is that $(x^\mu)$ are the position coordinates and $(\xi_\mu)$ are the momentum coordinates.

\begin{definition}
Let $M$ be a symplectic manifold and $H$ a Hamiltonian on $M$. The \dfn{Hamilton field} is the vector field $X_H$ defined by 
$$\iota_{X_H} \omega_M = \dif H.$$
The \dfn{Hamilton flow} is the $\RR$-action on $M$ induced by integrating the Hamilton field.
\end{definition}

\begin{proposition}[Hamilton's equation]
Let $(x^\mu, \xi_\mu)$ be canonical coordinates on a symplectic manifold $M$, and let $H$ be a Hamiltonian.
Then 
$$X_H x = \partial_\xi H, \qquad X_H \xi = -\partial_x H.$$
In particular, any integral curve $(x, \xi)$ of $X_H$ is a minimizer of
$$\int_\gamma \xi_\mu (X_H x)^\mu - H(x, \xi).$$
\end{proposition}


\section{Sheaf cohomology stuff}
Zworski constructs the phase of a Lagrangian submanifold from Poincare lemma...

\section{Semiclassical Strichartz estimates}

\section{Linearizing the nonlinear Schr\"odinger equation}
uwu


\printbibliography

\end{document}
