\documentclass[reqno,12pt,letterpaper]{amsart}
\RequirePackage{amsmath,amssymb,amsthm,graphicx,mathrsfs,url}
\RequirePackage[usenames,dvipsnames]{color}
\RequirePackage[colorlinks=true,linkcolor=Red,citecolor=Green]{hyperref}
\RequirePackage{amsxtra}
\usepackage{tikz-cd}

\setlength{\textheight}{8.50in} \setlength{\oddsidemargin}{0.00in}
\setlength{\evensidemargin}{0.00in} \setlength{\textwidth}{6.08in}
\setlength{\topmargin}{0.00in} \setlength{\headheight}{0.18in}
\setlength{\marginparwidth}{1.0in}
\setlength{\abovedisplayskip}{0.2in}
\setlength{\belowdisplayskip}{0.2in}
\setlength{\parskip}{0.05in}
\renewcommand{\baselinestretch}{1.10}

\title{Harmonic measures}
\author{Aidan Backus}
\date{January 2022}

\newcommand{\NN}{\mathbf{N}}
\newcommand{\ZZ}{\mathbf{Z}}
\newcommand{\QQ}{\mathbf{Q}}
\newcommand{\RR}{\mathbf{R}}
\newcommand{\CC}{\mathbf{C}}
\newcommand{\DD}{\mathbf{D}}
\newcommand{\PP}{\mathbf P}
\newcommand{\MM}{\mathbf M}
\newcommand{\II}{\mathbf I}
\newcommand{\Expect}{\mathbf E}
\newcommand{\Hyp}{\mathbf H}

\DeclareMathOperator{\card}{card}
\DeclareMathOperator{\ch}{ch}
\DeclareMathOperator{\codim}{codim}
\DeclareMathOperator{\diag}{diag}
\DeclareMathOperator{\diam}{diam}
\DeclareMathOperator{\dom}{dom}
\DeclareMathOperator{\Gal}{Gal}
\DeclareMathOperator{\Hom}{Hom}
\DeclareMathOperator{\Jac}{Jac}
\DeclareMathOperator{\Lip}{Lip}
\DeclareMathOperator{\id}{id}
\DeclareMathOperator{\rad}{rad}
\DeclareMathOperator{\rank}{rank}
\DeclareMathOperator{\Radon}{Radon}
\DeclareMathOperator*{\Res}{Res}
\DeclareMathOperator{\sgn}{sgn}
\DeclareMathOperator{\singsupp}{sing~supp}
\DeclareMathOperator{\Spec}{Spec}
\DeclareMathOperator{\supp}{supp}
\DeclareMathOperator{\Tan}{Tan}
\DeclareMathOperator{\vol}{vol}
\newcommand{\tr}{\operatorname{tr}}

\newcommand{\dbar}{\overline \partial}

\DeclareMathOperator{\atanh}{atanh}
\DeclareMathOperator{\csch}{csch}
\DeclareMathOperator{\sech}{sech}

\DeclareMathOperator{\Div}{div}
\DeclareMathOperator{\grad}{grad}
\DeclareMathOperator{\Ell}{Ell}
\DeclareMathOperator{\WF}{WF}

\newcommand{\Hilb}{\mathcal H}

\newcommand{\pic}{\vspace{30mm}}
\newcommand{\dfn}[1]{\emph{#1}\index{#1}}

\renewcommand{\Re}{\operatorname{Re}}
\renewcommand{\Im}{\operatorname{Im}}


\newtheorem{theorem}{Theorem}[section]
\newtheorem{badtheorem}[theorem]{``Theorem"}
\newtheorem{prop}[theorem]{Proposition}
\newtheorem{lemma}[theorem]{Lemma}
\newtheorem{proposition}[theorem]{Proposition}
\newtheorem{corollary}[theorem]{Corollary}
\newtheorem{conjecture}[theorem]{Conjecture}
\newtheorem{axiom}[theorem]{Axiom}

\theoremstyle{definition}
\newtheorem{definition}[theorem]{Definition}
\newtheorem{remark}[theorem]{Remark}
\newtheorem{example}[theorem]{Example}

\newtheorem{exercise}[theorem]{Discussion topic}
\newtheorem{homework}[theorem]{Homework}
\newtheorem{problem}[theorem]{Problem}

\newtheorem{ack}{Acknowledgements}
\newtheorem{notate}{Notation}

\numberwithin{equation}{section}


% Mean
\def\Xint#1{\mathchoice
{\XXint\displaystyle\textstyle{#1}}%
{\XXint\textstyle\scriptstyle{#1}}%
{\XXint\scriptstyle\scriptscriptstyle{#1}}%
{\XXint\scriptscriptstyle\scriptscriptstyle{#1}}%
\!\int}
\def\XXint#1#2#3{{\setbox0=\hbox{$#1{#2#3}{\int}$ }
\vcenter{\hbox{$#2#3$ }}\kern-.6\wd0}}
\def\ddashint{\Xint=}
\def\dashint{\Xint-}

%\usepackage{color}
%\hypersetup{%
%    colorlinks=true, % make the links colored%
%    linkcolor=blue, % color TOC links in blue
%    urlcolor=red, % color URLs in red
%    linktoc=all % 'all' will create links for everything in the TOC
%Ning added hyperlinks to the table of contents 6/17/19
%}

\usepackage[backend=bibtex,style=alphabetic,maxcitenames=50,maxnames=50]{biblatex}
%\addbibresource{linear_phase.bib}
\renewbibmacro{in:}{}
\DeclareFieldFormat{pages}{#1}

\begin{document}

\section{Harmonic measures}
Let $\Omega \subseteq \RR^d$ be an open set with nonempty smooth boundary.
If $u$ is the harmonic function on $\Omega$ with trace $f$, then
$$u(x) = \int_{\partial \Omega} G(x, y) f(y) ~dy$$
for some integral kernel
$$G: \Omega \times \partial \Omega \to \RR.$$

\begin{definition}
The \dfn{harmonic measure} of $\Omega$ at $x \in \Omega$ is the measure $\mu_x$ such that
$$d\mu_x(y) = G(x, y) ~dy.$$
\end{definition}

So we have $u(x) = \int_{\partial \Omega} f ~d\mu_x$.
To study harmonic measures it is convenient to reinterpret $\Delta$ as the ``infinitesimal generator" of Brownian motion.

If $W$ is a Brownian motion then $W_t$ denotes the position of $W$ at time $t \geq 0$, which is a random variable.
The only properties of Brownian motion that we shall need are the following.

\begin{theorem}
Let $W$ be a Brownian motion.
Then (almost surely):
\begin{enumerate}
\item $t \mapsto W_t$ is continuous.
\item (Markov's strong property) For every finite stopping time $\tau$ and deterministic time $t \geq 0$, $W_{\tau + t} - W_\tau$ is independent of $(W_s)_{s \in [0, \tau]}$.
\item (Equidistribution) If $W_0 = x$ and $S_x$ is a small sphere centered at $x$, $W_\tau$ the first time that $W_\tau \in S_x$, then the distribution of $W_x$ is the usual probability measure on the sphere.
\end{enumerate}
\end{theorem}

\begin{definition}
Let $W$ be a Brownian motion such that $W_0 \in \Omega$.
The \dfn{escape time} of $W$ is the first time $\tau$ such that $W_\tau \in \partial \Omega$.
\end{definition}

\begin{lemma}
Let $W$ be a Brownian motion such that $W_0 = x \in \Omega$ and let $\tau$ be its escape time.
Then $\mu_x$ is the distribution of $W_\tau$.
\end{lemma}
\begin{proof}
I follow \S3.1 of M\"orters--Peres loosely.
Given a trace $f$, let
$$u(x) = \Expect(f(W_\tau)|W_0 = x)$$
and let $S_x$ be a small sphere centered on $x$ and $X$ a random point on $S_x$.
Then by Markov's strong property and equidistribution
$$u(x) = \Expect(\Expect(f(W_\tau)|W_0 \in X)) = \Expect(u(X)) = \dashint_{S_x} u(y) ~dy$$
which means that $u(x)$ satisfies the mean-value property.
Also $f(W_0) = f(x)$ clearly, if $x \in \partial \Omega$.
So
\begin{align*}
\Expect(f(W_\tau)|W_0 = x) &= \int_{\partial \Omega} f ~d\mu_x. \qedhere
\end{align*}
\end{proof}

This obviously implies that $\mu_x$ is a probability measure.
Similar results hold for more general stochastic processes, and in fact for the $p$-Laplace equation, $p \in [2, \infty]$.
In case $p = \infty$ this is relevant to my research.

\section{Slit domains}
We identify $\CC = \RR^2$ and let $\Sigma = \{|y| < r\} \setminus I_0$ where $I_0$ is an interval in $\RR$ of length $\leq 1$.
This does not have a smooth boundary but by applying a conformal transformation we can transform it into $\DD \setminus I_0$ and then approximate this new open set by annuli with increasingly eccentric inner circles.
We suppress this approximation in the below lemmata, even though Bourgain--Dyatlov keep it.
So the boundary of $\partial \Sigma$ admits harmonic measures.

\begin{lemma}
If $F: \overline \Sigma \to \CC$ is holomorphic then for every $z \in \Sigma$,
$$\log |F(z)| \leq \int_{\partial \Sigma} \log |F| ~d\mu_z.$$
\end{lemma}
\begin{proof}
Let $u(z) = \int_{\partial \Sigma} \log |F| ~d\mu_z$. Then $u$ is harmonic and has the same trace as the subharmonic function $\log |F|$.
So by the maximum principle, $u$ dominates $\log |F|$.
\end{proof}

The next lemma says that if we start a Brownian motion close, but not too close, to $I_0$, then the Brownian motion will probably hit the slit before it escapes $\Sigma$.

\begin{lemma}
Suppose that $t \in \Sigma \cap \RR$ and let $G = G(t, \cdot)$ be the Radon-Nikod\'ym derivative of $\mu_t$ with respect to the $1$-dimensional measure on $\partial \Sigma$. Then:
\begin{enumerate}
\item If $d(t, I_0) \geq |I_0|/10$ then
$$||G||_{L^p(I_0)} \lesssim_{|I_0|, p} 1.$$
\item If $d(t, I_0) \leq 1$ then for $x \in \RR$,
$$G(x \pm ir) \leq \frac{2}{re^{d(x, I_0)}}.$$
\item If $d(t, I_0) \leq 1$ then
$$\mu_t(I_0) \geq \frac{|I_0|}{8e^{2/r}}.$$
\end{enumerate}
\end{lemma}
\begin{proof}
The proofs are all similar so let's do the last one.
Without loss of generality, $I_0 = [-\ell, 0]$ where $0 < \ell \leq 1$ and $0 < t \leq 1$.
Let $\Sigma' = \{|y| < r\} \setminus (-\infty, 0]$. Then $\Sigma \subset \Sigma'$.
If $W$ is a Brownian motion such that $W_0 = t$ and $W$ escapes $\Sigma'$ via $I_0$ then $W$ also escapes $\Sigma$ via $I_0$.
So
$$\mu_t^\Sigma(I_0) \geq \mu_t^{\Sigma'}(I_0).$$
Also there is a conformal isomorphism
$$\kappa: (\Sigma', t) \to (\Hyp^2, i)$$
such that $[0, a] = [0, \sqrt \ell/2 e^{-\pi/2r}] \subseteq \kappa(I_0)$.
Therefore
$$\mu_t^{\Sigma'}(I_0) = \int_{\kappa(I_0)} P(i, y) ~dy \geq \int_0^a P(i, y) ~dy$$
where $P$ is the Poisson kernel of $\Hyp^2$. One can explicitly compute that this is $\geq \ell/8 e^{-2/r}$.
\end{proof}

\section{Application to fractal uncertainty}
We need to show the following lemma.

\begin{lemma}
Let $\mathcal I$ be a set of nonoverlapping intervals of length $1$, $c_0 > 0$, and for each $I \in \mathcal I$ select $I'' \subset I$ of length $c_0$.
Then for every $r \in (0, 1)$, $\kappa \in (0, e^{-O(r^{-1})}]$, and $f \in L^2(\RR)$, if $\supp f$ is compact, then
$$\sum_{I \in \mathcal I} ||f||_{L^2(I)}^2 \lesssim_{c_0} r^{-1} \left(\sum_{I \in \mathcal I} ||f||_{L^2(I'')}^2 \right)^\kappa \left(\int_{-\infty}^\infty e^{4\pi r|\xi|} |\hat f(\xi)|^2 ~d\xi \right)^{1 - \kappa}.$$
\end{lemma}
\begin{proof}
The function
$$F(z) = \int_{-\infty}^\infty e^{2\pi iz\xi} \hat f(\xi) ~d\xi$$
is entire with $F|\RR = f$, by the Paley-Wiener-Schwartz theorem, and satisfies the growth condition
$$\int_{-\infty}^\infty |F(x \pm ir)|^2 ~dx \leq \int_{-\infty}^\infty e^{4\pi r\xi} |\hat f(\xi)|^2 ~d\xi.$$
Let $I_0$ satisfy $|I_0| = c_0/2$ and have the same center as $I''$ and $\Sigma_I = \{|y| < r\} \setminus I_0$.

If $t \in I \setminus I''$, let $\mu_t$ be its harmonic measure with respect to $\Sigma_I$ and let $\kappa_I = \mu_t(I_0)$.
By (3),
$$\kappa_I \gtrsim c_0 e^{-2/r} \geq \kappa.$$
We also have
\begin{align*}
2 \log |f(t)| &\leq \int_{\partial \Sigma_I} 2 \log |F| ~d\mu_t \\
&= 4\kappa_I \int_{I_0} \frac{\log |f|}{2} \frac{d\mu_t}{\kappa_I} + (1 - \kappa_I) \int_{\partial \Sigma_I \setminus I_0} 2 \log |F| \frac{d\mu_t}{1 - \kappa_I}.
\end{align*}
Notice that $\mu_t/\kappa_I$ is a probability measure on $I_0$ and similarly for the measure on its complement.
So by Jensen's inequality,
\begin{align*}
|f(t)|^2 &\leq \left(\int_{I_0} \sqrt{|f|} \frac{d\mu_t}{\kappa_I}\right)^{4\kappa_I} \left(\int_{\partial \Sigma_I \setminus I_0} |F|^2 \frac{d\mu_t}{1 - \kappa_I}\right)^{1 - \kappa_I}\\
&\lesssim \left(\int_{I_0} \sqrt{|f|} ~d\mu_t\right)^{4\kappa}\left(\int_{I_0} \sqrt{|f|} ~d\mu_t + \int_{\partial \Sigma_I \setminus I_0} |F|^2 ~d\mu_t\right)^{1 - \kappa}.
\end{align*}
By H\"older's inequality with $p = 4/3$ and (1),
$$\int_{I_0} \sqrt{|f|} ~d\mu_t \leq ||f||_{L^2(I_0)}^{1/2}$$
and by (2),
$$\int_{\partial \Sigma_I \setminus I_0} |F|^2 ~d\mu_t \lesssim r^{-1} \int_{|\Im z| = r} e^{-d(\Re z, I)} |F(z)|^2 ~dz.$$
From this and more H\"older we get
$$\sum_{I \in \mathcal I} ||f||_{L^2(I \setminus I'')}^2 \lesssim r^{-1} \left(\sum_{I \in \mathcal I} ||f||_{L^2(I'')}^2\right)^\kappa \left(\int_{|\Im z| = r} |F(z)|^2 ~dz\right)^{1 - \kappa}$$
and
$$\int_{|\Im z| = r} |F(z)|^2 ~dz \leq \int_{-\infty}^\infty e^{4\pi r|\xi|} |\hat f(\xi)|^2 ~d\xi$$
and since $e^{2\pi r|\xi|} \geq 1$ another H\"older gives
\begin{align*}
\sum_{I \in \mathcal I} ||f||_{L^2(I'')}^2 &\leq \left(\sum_{I \in \mathcal I} ||f||_{L^2(I'')}^2\right)^\kappa \left(\int_{-\infty}^\infty e^{4\pi r|\xi|} |\hat f(\xi)|^2 ~d\xi \right)^{1 - \kappa}.
\qedhere \end{align*}
\end{proof}


\end{document}
