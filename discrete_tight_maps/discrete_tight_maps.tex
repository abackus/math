\documentclass[reqno,11pt]{amsart}
\usepackage[letterpaper, margin=1in]{geometry}
\RequirePackage{amsmath,amssymb,amsthm,graphicx,mathrsfs,url,slashed,subcaption}
\RequirePackage[usenames,dvipsnames]{xcolor}
\RequirePackage[colorlinks=true,linkcolor=Red,citecolor=Green]{hyperref}
\RequirePackage{amsxtra}
\usepackage{cancel}
\usepackage{tikz, wrapfig}

% Add the 2020 MSC
\makeatletter
\@namedef{subjclassname@2020}{\textup{2020} Mathematics Subject Classification}
\makeatother

%\usepackage[T1]{fontenc}

% \setlength{\textheight}{9.3in} \setlength{\oddsidemargin}{-0.25in}
% \setlength{\evensidemargin}{-0.25in} \setlength{\textwidth}{7in}
% \setlength{\topmargin}{-0.25in} \setlength{\headheight}{0.18in}
% \setlength{\marginparwidth}{1.0in}
% \setlength{\abovedisplayskip}{0.2in}
% \setlength{\belowdisplayskip}{0.2in}
% \setlength{\parskip}{0.05in}
%\renewcommand{\baselinestretch}{1.05}

\title{Discrete tight maps}
\author{}
\address{}
\email{}
\date{\today}
\keywords{}
\subjclass[2020]{primary: }

\newcommand{\NN}{\mathbf{N}}
\newcommand{\ZZ}{\mathbf{Z}}
\newcommand{\QQ}{\mathbf{Q}}
\newcommand{\RR}{\mathbf{R}}
\newcommand{\CC}{\mathbf{C}}
\newcommand{\DD}{\mathbf{D}}
\newcommand{\PP}{\mathbf P}
\newcommand{\MM}{\mathbf M}
\newcommand{\II}{\mathbf I}
\newcommand{\Hyp}{\mathbf H}
\newcommand{\Sph}{\mathbf S}
\newcommand{\Torus}{\mathbf T}
\newcommand{\Group}{\mathbf G}
\newcommand{\GL}{\mathbf{GL}}
\newcommand{\Orth}{\mathbf{O}}
\newcommand{\SpOrth}{\mathbf{SO}}
\newcommand{\Ball}{\mathbf{B}}

\newcommand*\dif{\mathop{}\!\mathrm{d}}

\DeclareMathOperator{\card}{card}
\DeclareMathOperator{\dist}{dist}
\DeclareMathOperator{\id}{id}
\DeclareMathOperator{\Hom}{Hom}
\DeclareMathOperator{\mesh}{mesh}
\DeclareMathOperator{\coker}{coker}
\DeclareMathOperator{\supp}{supp}
\DeclareMathOperator{\tr}{tr}

\newcommand{\Two}{\mathrm{I\!I}}
\newcommand{\weakto}{\rightharpoonup}

\newcommand{\normal}{\mathbf n}
\newcommand{\vol}{\mathrm{vol}}

\newcommand{\diam}{\mathrm{diam}}
\DeclareMathOperator{\sech}{sech}
\newcommand{\inj}{\mathrm{inj}}
\newcommand{\Lip}{\mathrm{Lip}}
\newcommand{\Riem}{\mathrm{Riem}}

\DeclareMathOperator*{\essinf}{ess\,inf}
\DeclareMathOperator*{\esssup}{ess\,sup}

\newcommand{\dfn}[1]{\emph{#1}\index{#1}}

\renewcommand{\Re}{\operatorname{Re}}
\renewcommand{\Im}{\operatorname{Im}}

\newcommand{\loc}{\mathrm{loc}}
\newcommand{\cpt}{\mathrm{cpt}}

\def\Japan#1{\left \langle #1 \right \rangle}

\newtheorem{theorem}{Theorem}[section]
\newtheorem{badtheorem}[theorem]{``Theorem"}
\newtheorem{prop}[theorem]{Proposition}
\newtheorem{lemma}[theorem]{Lemma}
\newtheorem{sublemma}[theorem]{Sublemma}
\newtheorem{proposition}[theorem]{Proposition}
\newtheorem{corollary}[theorem]{Corollary}
\newtheorem{conjecture}[theorem]{Conjecture}
\newtheorem{axiom}[theorem]{Axiom}
\newtheorem{assumption}[theorem]{Assumption}

\newtheorem{mainthm}{Theorem}
\renewcommand{\themainthm}{\Alph{mainthm}}

\newtheorem{claim}{Claim}[theorem]
\renewcommand{\theclaim}{\thetheorem\Alph{claim}}
% \newtheorem*{claim}{Claim}

\theoremstyle{definition}
\newtheorem{definition}[theorem]{Definition}
\newtheorem{remark}[theorem]{Remark}
\newtheorem{example}[theorem]{Example}
\newtheorem{notation}[theorem]{Notation}

\newtheorem{exercise}[theorem]{Discussion topic}
\newtheorem{homework}[theorem]{Homework}
\newtheorem{problem}[theorem]{Problem}

\makeatletter
\newcommand{\proofpart}[2]{%
  \par
  \addvspace{\medskipamount}%
  \noindent\emph{Part #1: #2.}
}
\makeatother



\numberwithin{equation}{section}


% Mean
\def\Xint#1{\mathchoice
{\XXint\displaystyle\textstyle{#1}}%
{\XXint\textstyle\scriptstyle{#1}}%
{\XXint\scriptstyle\scriptscriptstyle{#1}}%
{\XXint\scriptscriptstyle\scriptscriptstyle{#1}}%
\!\int}
\def\XXint#1#2#3{{\setbox0=\hbox{$#1{#2#3}{\int}$ }
\vcenter{\hbox{$#2#3$ }}\kern-.6\wd0}}
\def\ddashint{\Xint=}
\def\dashint{\Xint-}

\usepackage[backend=bibtex,style=alphabetic,giveninits=true]{biblatex}
\renewcommand*{\bibfont}{\normalfont\footnotesize}
\addbibresource{discrete_tight_maps.bib}
\renewbibmacro{in:}{}
\DeclareFieldFormat{pages}{#1}

\newcommand\todo[1]{\textcolor{red}{TODO: #1}}


\begin{document}
\begin{abstract}
    
\end{abstract}

\maketitle

%%%%%%%%%%%%%%%%%%%%%%%%%%%%%%%%%%%%%%%%%%%%%%%%%%%%%%%
\section{Introduction}
Let $U \subseteq \RR^m$ be an open polytope.

\begin{definition}
For a Lipschitz map $u: U \to \RR^n$, define the \dfn{Lipschitz modulus}
$$Lu(x) := \inf_{r > 0} \Lip(u, B(x, r)).$$
\end{definition}

\begin{definition}[{\cite[\S1.3]{Sheffield12}}]
Let $u, v: U \to \RR^n$ be Lipschitz maps such that $u|_{\partial U} = v|_{\partial U}$.
We say that $u$ is \dfn{tighter} than $v$, written $u \prec v$, if 
$$\sup_{Lu > Lv} Lu < \sup_{Lv > Lu} Lv.$$
If there is no tighter $v$ than $u$, we say that $u$ is \dfn{tight}.
\end{definition}

\begin{theorem}
Every Lipschitz map $f: \partial U \to \RR^n$ has a tight extension $u: U \to \RR^n$.
\end{theorem}

Showing that the finite element approximations converge is hard, because the local Lipschitz constant of the approximation is piecewise constant, and we do not have any a priori estimates on the $W^{s, p}$ norms of the exact solution when $s > 1$.
Therefore we do not expect the approximate local Lipschitz constants to converge in any stronger sense than pointwise convergence almost everywhere.
This difficulty is already implicit in the numerical analysis literature concerning the scalar $\infty$-Laplacian.
Even though finite element approximations to $\infty$-harmonic functions $u$ are available \cite{Loisel_2020}, we are not aware of any estimates on the approximation error for $\dif u$ as in the case of $p$-harmonic functions, $p < \infty$ \cite{Barrett93}.

%%%%%%%%%%%%%%%%%%
\subsection{Acknowlegments}
I would like to thank Georgios Daskalopolous, Brian Freidin, and Nate River for various helpful discussions.

This research was supported by the National Science Foundation's Graduate Research Fellowship Program under Grant No. DGE-2040433.

%%%%%%%%%%%%%%%%%%%
\section{Preliminaries}
\subsection{Notation}
\begin{enumerate}
\item $U$ is an open polytope in $\RR^m$.
\item For a measurable set $E$, $E^\circ$, $\overline E$, and $\vol(E)$ are the interior, closure, and Lebesgue measure of $E$.
\item For a continuous map $u: U \to \RR^n$ and a set $E \subseteq U$, $\Lip(u, E)$ is the Lipschitz constant of $u|_E$.
\item For a precompact measurable set $E$ and $f \in L^1(E)$, $\dashint_E f := \vol(E)^{-1} \int_E f$.
\item For $p \in [1, \infty]$, and a linear map $A$, $|A|_p$ is the $p$th Schatten norm of $A$.
\end{enumerate}

%%%%%%%%%%%%%%%%
\subsection{Measure theory}
\begin{proposition}[one-sided Egorov theorem]\label{one sided Egorov}
Let $X$ be a Polish space, $\mu$ a finite Borel measure on $X$, and $f_n, f$ $\mu$-measurable functions on $X$, such that for $\mu$-almost every $x \in X$,
$$f(x) \leq \liminf_{n \to \infty} f_n(x).$$
Then for every $\varepsilon > 0$ there exists a $\mu$-measurable set $Z_\varepsilon \subseteq X$ such that:
\begin{enumerate}
\item $\mu(Z_\varepsilon) < \varepsilon$.
\item For every $\delta > 0$ there exists $N \in \NN$ such that for every $n \geq N$ and every $x \in X \setminus Z_\varepsilon$,
$$f(x) < f_n(x) + \delta.$$
\end{enumerate}
\end{proposition}
\begin{proof}
Let 
$$E_{n, k} := \bigcup_{m \geq n} \left\{x \in X: f(x) \geq f_m(x) + \frac{1}{k}\right\}.$$
Then for $m \geq n$, $E_{m, k} \subseteq E_{n, k}$; since $f_n \to f$ $\mu$-almost everywhere, it follows that for every $k$, $\bigcap_n E_{n, k}$ is $\mu$-null.
By continuity from above of $\mu$, for every $k$, there exists $n(k)$ such that $\mu(E_{n(k), k}) < \varepsilon/2^k$.
We set $Z_\varepsilon := \bigcup_k E_{n(k), k}$, so $\mu(Z_\varepsilon) < \varepsilon$.
For each $\delta > 0$, if $k \geq \delta^{-1}$, then for
$$x \in X \setminus E_{n(k), k} \subseteq X \setminus Z_\varepsilon,$$
and $n \geq n(k)$, $f(x) < f_n(x) + \delta$.
\end{proof}

\begin{lemma}[lower semicontinuity of vectorial total variation]\label{TV lower semicontinuity}
Let $|\cdot|_M$ be a norm on $n \times m$-matrices, let $u_n, u \in BV(U, \RR^n)$, and suppose that $u_n \to u$ in $L^1$.
Then for any open $V \subseteq U$,
$$\int_V |\dif u|_M(x) \dif x \leq \liminf_{n \to \infty} \int_V |\dif u_n|_M(x) \dif x.$$
\end{lemma}
\begin{proof}
Let $|\cdot|_M'$ be the dual norm to $|\cdot|_M$.
Let
$$\mathscr D := \{X \in C^1_\cpt(U, \RR^{n \times m}): \||X|_M'\|_{C^0} \leq 1\}.$$
Integrating by parts, for any $v \in BV(U, \RR^n)$, 
$$\int_V |\dif v|_M(x) \dif x = \sup_{X \in \mathscr D} \int_V \langle \dif v, X\rangle(x) \dif x = \sup_{X \in \mathscr D} \int_V \langle v, \nabla \cdot X\rangle(x) \dif x.$$
The result follows when we use the $L^1$ convergence to bound
\begin{align*}
\sup_{X \in \mathscr D} \int_V \langle u, \nabla \cdot X\rangle(x) \dif x 
&\leq \liminf_{n \to \infty} \sup_{X \in \mathscr D} \int_V \langle u_n, \nabla \cdot X\rangle(x) \dif x. \qedhere
\end{align*}
\end{proof}

%%%%%%%%%%%%%%%%
\subsection{Lipschitz maps}
Let $u: U \to \RR^n$ be a Lipschitz map.
Its Lipschitz modulus $Lu$ is upper semicontinuous \cite[Lemma 4.2(a)]{Crandall2008}.
Though $Lu$ is not lower semicontinuous, it is ``lower semicontinuous on average"; this was claimed without proof by Sheffield and Smart \cite[\S1.4]{Sheffield12}.

\begin{lemma}[lower semicontinuity on average of $Lu$]\label{partial lower semicontinuity}
Let $x \in U$, let $V \subseteq U$ be an open neighborhood of $x$, and let $\delta > 0$.
Then for any Lipschitz map $u: U \to \RR^n$,
$$\vol(V \cap \{Lu > Lu(x) - \delta\}) > 0.$$
\end{lemma}
\begin{proof}
It suffices to show the contrapositive: if, for almost every $y \in V$, $Lu(y) \leq K$, then $Lu(x) \leq K$ as well.
Furthermore, by shrinking $V$ to be a ball if necessary, we may assume that $V$ is convex.
Let $\Delta_V$ be the diagonal of $V^2$. Then
$$Lu(x) \leq \Lip(u, V) \leq \sup_{(y, z) \in V^2 \setminus \Delta_V} \frac{|u(y) - u(z)|}{|y - z|}.$$
Since the difference quotient $f(y, z) := |u(y) - u(z)|/|y - z|$ is continuous on $V^2 \setminus \Delta_V$, this expression is the $L^\infty$ norm of $f$.
Moreover, $\Delta_V$ is a null subset of $V^2$, so
$$\|f\|_{L^\infty(V^2 \setminus \Delta_V)} = \lim_{p \to \infty} \left[\int_V \int_V \frac{|u(y) - u(z)|^p}{|y - z|^p} \dif y \dif z\right]^{1/p}.$$
We estimate this integral using the mean value and Jensen inequalities:
\begin{align*} 
\int_V \int_V \frac{|u(y) - u(z)|^p}{|y - z|^p} \dif y \dif z 
&\leq \int_V \int_V \frac{1}{|y - z|^p} \left[\int_{[y, z]} Lu(w) \dif w\right]^p \dif y \dif z \\
&\leq \int_V \int_V \frac{1}{|y - z|} \int_{[y, z]} Lu(w)^p \dif w \dif y \dif z. 
\end{align*}
Let $E(y, w)$ be the set of points in $V$ on the line $\ell$ through $y, w$ which are to the right of $w$, where $\ell$ is oriented so that the vector $w - y$ points to the right.
Then $w \in [y, z]$ iff $z \in E(y, w)$, so we can rewrite the bounds of integration using Fubini's theorem as follows:
$$\int_V \int_V \frac{1}{|y - z|} \int_{[y, z]} Lu(w)^p \dif w \dif y \dif z = \int_V \int_V \int_{E(y, w)} Lu(w)^p \frac{\dif z}{|y - z|} \dif y \dif w.$$
The point is that $w$ now ranges over $V$, and we know that for almost every $w \in V$, $Lu(w) \leq K$.
We therefore can estimate this integral, and undo the change in the order of integration, as follows:
\begin{align*} 
\int_V \int_V \int_{E(y, w)} Lu(w)^p \frac{\dif z}{|y - z|} \dif y \dif w
&\leq K^p \int_V \int_V \int_{E(y, w)} \frac{\dif z}{|y - z|} \dif y \dif w \\
&= K^p \int_V \int_V \int_{[y, z]} \frac{\dif w}{|y - z|} \dif y \dif z \\
&= K^p \vol(V)^2.
\end{align*}
Therefore 
\begin{align*}
Lu(x) &\leq \|f\|_{L^\infty(V^2 \setminus \Delta_V)} \leq K \lim_{p \to \infty} \vol(V)^{2/p} = K. \qedhere 
\end{align*}  
\end{proof}

\begin{lemma}
For every Lipschitz map $u: U \to \RR^n$:
\begin{enumerate}
\item For every compact convex set $K \Subset U$,
\begin{equation}\label{Lip is sup of local Lips}
\Lip(u, V) = \max_{x \in K} Lu(x).
\end{equation}
\item For every open convex set $V \subseteq U$,
\begin{equation}\label{mean value estimate}
\Lip(u, V) = \||\dif u|_\infty\|_{L^\infty(V)}.
\end{equation}
\item For every $x \in U$,
\begin{equation}\label{localized mean value estimate}
Lu(x) = \lim_{r \to 0} \||\dif u|_\infty\|_{L^\infty(B(x, r))}.
\end{equation}
\end{enumerate}
\end{lemma}
\begin{proof}
This is essentially the content of \cite[Lemma 4.2]{Crandall2008} when $n = 1$.
The only meaningful difference when $n \geq 2$ is that we have to worry about which matrix norm is applied to $\dif u$, but of course we know the correct norm to use is $|\cdot|_\infty$.
\end{proof}

\begin{lemma}\label{Lip is du}
For every Lipschitz map $u: U \to \RR^n$ and almost every $x \in U$,
$$Lu(x) = |\dif u|_\infty(x).$$
\end{lemma}
\begin{proof}
Let $\varepsilon > 0$.
By Lusin's theorem, there is a closed set $Z_\varepsilon \subset U$ such that $\vol(Z_\varepsilon) < \varepsilon$ and $|\dif u|_\infty$ is continuous on $U \setminus Z_\varepsilon$.
Applying (\ref{localized mean value estimate}) and $x \in U \setminus Z_\varepsilon$,
$$\lim_{r \to 0} \||\dif u|_\infty\|_{L^\infty(B(x, r))} = |\dif u|_\infty(x),$$
hence $Lu(x) = |\dif u|_\infty(x)$.
The claim follows by taking $\varepsilon \to 0$.
\end{proof}

\begin{lemma}[lower semicontinuity of $L$]\label{Lip is lower semicontinuous}
Suppose that $u_k, u: U \to \RR$ are maps with $u_k \to u$ in $L^1$ and $\Lip(u_k, U) \leq K < \infty$.
Then for almost every $x \in U$,
$$Lu(x) \leq \liminf_{k \to \infty} Lu_k(x).$$
\end{lemma}
\begin{proof} 
We first use the Lebesgue differentiation theorem to write, for almost every $x \in U$,
$$Lu(x) = \lim_{r \to 0} \dashint_{B(x, r)} Lu(y) \dif y.$$
By Lemmata \ref{Lip is du} and \ref{TV lower semicontinuity},
\begin{align*}
\dashint_{B(x, r)} Lu(y) \dif y 
&= \dashint_{B(x, r)} |\dif u|_\infty(y) \dif y 
\leq \liminf_{k \to \infty} \dashint_{B(x, r)} |\dif u_k|_\infty(y) \dif y \\
&= \liminf_{k \to \infty} \dashint_{B(x, r)} Lu_k(y) \dif y.
\end{align*}
Because $Lu_k(y) \leq \Lip(u_k, U) \leq K$, we can apply dominated convergence:
$$\liminf_{k \to \infty} \dashint_{B(x, r)} Lu_k(y) \dif y = \dashint_{B(x, r)} \liminf_{k \to \infty} Lu_k(y) \dif y.$$
The result follows when we take $r \to 0$ and apply the Lebesgue differentiation theorem.
\end{proof}

%%%%%%%%%%%%%%%%%%%%%%%%%%%
\section{Application of the finite element method}
\subsection{Construction of eikonal extensions}
\begin{lemma}\label{eikonal extension}
Let $V \subseteq \RR^k$ be a bounded convex open set.
For every Lipschitz map $f: \partial V \to \RR^n$, there exists an extension $u: V \to \RR^n$ of $f$, such that for every $x \in V$,
$$Lu(x) = \Lip(f, \partial V).$$
\end{lemma}

The following attempts to prove Lemma \ref{eikonal extension} might seem like they work, but they don't:
\begin{enumerate}
\item If $V$ is a simplex, and we are allowed to prescribe $f$ on the vertices $X$ of $V$ rather than $\partial V$ (this is the situation we actually care about, anyways), try $u$ to be the linear extension of $f$.
However, we can take
$$V = \{(x, y) \in \RR^2: x > 0, y > 0, x + y < 1\}$$
and $f(0, 0) = 0$, $f(1, 0) = f(0, 1) = 1$.
Then the unique linear extension of $f$ is $u(x, y) = x + y$.
Then $\Lip(f, X) = 1$ but $\Lip(u, V) = \sqrt 2$.
\item If $n = 1$, then try $u$ to be given by Perron's method as
\begin{equation}\label{eikonal equation}
\begin{cases}
|\dif u| = \Lip(f, \partial V) \\
u|_{\partial V} = f.
\end{cases}
\end{equation}
This clearly fails when $n \geq 2$, but even when $n = 1$, I think this fails because (\ref{eikonal equation}) is so overdetermined.
For example, it seems very likely that (\ref{eikonal equation}) has no $C^1$ supersolutions under various hypotheses, which, for example, could compel any $C^1$ extension of $f$ to have a critical point.
Maybe I'll try to write out an explicit counterexample.
\end{enumerate}
We are thus compelled to construct an eikonal extension which is NOT a viscosity solution of (\ref{eikonal equation}) when $n = 1$.

\begin{proof}[Proof of Lemma \ref{eikonal extension}]
If $k = 1$, then $V$ is a line segment and we can take $u$ to be the linear interpolation of $f$.
So we may assume $k \geq 2$.

Let $K := \Lip(f, \partial V)$.
By the Kirszbraun-Valentine theorem, there exists an extension $u_0: V \to \RR^n$ of $f$ such that $\Lip(u_0, V) \leq K$.
Furthermore, define
$$\mathscr U_0 = \mathscr P_0 = \mathscr L_0 = \emptyset.$$

Recall that a \dfn{maximal packing} of an open set $W \subseteq \RR^k$ by balls of radius $r$ is a set $\mathscr P$ of balls contained in $W$ of radius $r$, such that there does not exist a ball $B \subseteq W$ of radius $r$ such that $B$ is disjoint from every ball in $\mathscr P$.
We stress that the balls in the packing $\mathscr P$ are assumed to not intersect $\partial W$.

\begin{claim}[iteration]
Assume we have defined:
\begin{enumerate}
\item Finite sets of points $\mathscr U_0, \dots, \mathscr U_{I - 1}$.
\item Finite set of line segments $\mathscr L_0, \dots, \mathscr L_{I - 1}$.
\item Finite sets of balls $\mathscr P_0, \dots, \mathscr P_{I - 1}$.
\item Lipschitz maps $u_0, \dots, u_{I - 1}$ mapping $V$ into $\RR^n$.
\end{enumerate}
Furthermore, suppose that for every $0 \leq J \leq I - 1$:
\begin{enumerate}
\item $\mathscr P_J = \{B(q, 2^{-J}): q \in \mathscr U_J\}$.
\item $\mathscr L_J = \mathscr L_{J - 1} \cup \{\ell(q): q \in \mathscr U_J\}$ where $\ell(q)$ is a closed line segment in $B(q, 4^{-I})$.
\item If $J \geq 1$, then $\mathscr P_J$ is a maximal packing of $V \setminus \bigcup \mathscr L_{J - 1}$ by balls of radius $2^{-J}$.
\item $u_J|_{\partial V} = f$.
\item $\Lip(u_J, V) \leq K$.
\item For every $\ell \in \mathscr L_J$, $\Lip(u_J, \ell) \geq K$.
\item For every $\ell \in \mathscr L_{J - 1}$, $u_J|_\ell = u_{J - 1}|_\ell$.
\end{enumerate}
Then there exist $\mathscr U_I, \mathscr L_I, \mathscr P_I, u_I$ with the same properties, but $I - 1$ replaced by $I$.
\end{claim}
\begin{proof}[Proof of claim]
Since $\mathscr L_{I - 1}$ is a finite set of closed sets, $\bigcup \mathscr L_{I - 1}$ is itself closed.
Choose $\mathscr U_I \subset V$ such that
$$\mathscr P_I := \{B(q, 2^{-I}): q \in \mathscr U_I\}$$
is a maximal packing of $V \setminus \bigcup \mathscr L_{I - 1}$ by balls of radius $2^{-I}$.
Since $V$ is bounded, $\vol(V) < \infty$, and since $\mathscr P_I$ is disjoint it follows that 
$$\card \mathscr P_I \leq 2^{kI} \vol(V) < \infty$$
so $\mathscr U_I, \mathscr P_I$ are finite.

Since $\partial V \cup \bigcup \mathscr L_{I - 1}$ is closed and disjoint from $\mathscr U_I$, which is finite and hence itself closed, there is an open set $W \supseteq \mathscr U_I$ which is disjoint from a neighborhood of $\partial V \cup \bigcup \mathscr L_{I - 1}$.
Furthermore, we may choose $W = \bigcup_{q \in \mathscr U_I} B(q, r_q)$ for some $r_q < 4^{-I}$.

We now define $u_I|_{(V \setminus W)^\circ} := u_{I - 1}|_{(V \setminus W)^\circ}$.
Thus we immediately have $u_I|_{\partial V} = f$ and, for every $\ell \in \mathscr U_{I - 1}$, $u_I|_\ell = u_{I - 1}|_\ell$.

We now define $\mathscr L_I$ and $u_I|_{\overline W}$.
To this end, let $q \in \mathscr U_I$, and consider the behavior of $u_{I - 1}|_{\overline{B(q, r_q)}}$.
There are two cases:
\begin{enumerate}
\item If there exists $x \in \overline{B(q, r_q)}$ with $Lu_{I - 1}(x) = K$, then we set
$$u_I|_{\overline{B(q, r_q)}} := u_{I - 1}|_{\overline{B(q, r_q)}}.$$ 
\item Otherwise, since $\overline{B(q, r_q)}$ is convex and compact, we must have $\Lip(u_{I - 1}, \overline{B(q, r_q)}) < K$.
Choose $\varphi \in C^\infty_\cpt(B(q, r_q))$ which is rapidly oscillating near $q$.
For each $t > 0$, let
$$v_t := u_{I - 1}|_{\overline{B(q, r_q)}} + t\varphi.$$
Since $w \mapsto \Lip(w, \overline{B(q, r_q)})$ is a seminorm, $\beta(t) := \Lip(v_t, \overline{B(q, r_q)})$ is a continuous function.
Since $\beta(0) < K$ and $\beta(t) \to \infty$ as $\to \infty$, there exists $T > 0$ such that $\beta(T) = K$.
Let $u_I|_{\overline{B(q, r_q)}} := v_T$.
By construction, there exists $x \in \overline{B(q, r_q)}$ such that $Lu_I(x) = K$.
\end{enumerate}
In either case, $\Lip(u_I|_{\overline{B(q, r_q)}}) \leq K$, and we can find $y_n, z_n \to x$ with
$$\frac{|u_I(y_n) - u_I(z_n)|}{|y_n - z_n|} = K.$$
For $n$ large enough, $[y_n, z_n] \subset B(q, 4^{-I})$.
We then set $\ell(q) := [y_n, z_n]$, and observe that
$$\Lip(u_I, \ell(q)) \geq K.$$

After this process is done, we have defined $u_I$ and $\ell(q)$ for every $q \in \mathscr U_I$.
Therefore we have also defined $\mathscr L_I$.
By construction, they satisfy the desired properties.
\end{proof}

\begin{claim}\label{separation of line segments}
For every $\ell, \ell' \in \mathscr L_I$, either $\ell = \ell'$ or
$$\dist(\ell, \ell') \geq 2^{-I} - 4^{-I}.$$
\end{claim}
\begin{proof}[Proof of claim]
Suppose that $\ell \neq \ell'$.
By induction, we may assume that this result holds if $\ell, \ell' \in \mathscr L_{I - 1}$.
So suppose that $\ell \in \mathscr L_I \setminus \mathscr L_{I - 1}$.
Then there exists $q \in \mathscr U_I$ such that $\ell \subset B(q, 4^{-I})$.
We break into cases:
\begin{enumerate}
\item If $\ell' \in \mathscr L_{I - 1}$, then since balls $\mathscr P_I$ are disjoint from $\bigcup \mathscr L_{I - 1}$, $\ell'$ does not intersect $B(q, 2^{-I})$.
\item Otherwise, $\ell' \subset B(q', 4^{-I})$ for some $q' \in \mathscr U_I \setminus \{q\}$. Since $\mathscr P_I$ is disjoint, it follows that $\ell'$ does not intersect $B(q, 2^{-I})$.
\end{enumerate}
In either case, the result follows.
\end{proof}

\begin{claim}\label{almost density of balls}
For every $x \in V$ and every $I \geq 10$,
$$\min\left(\dist(x, \partial V), \min_{J \leq I} \dist(x, \mathscr U_J)\right) \leq 2^{-I + 4}.$$
\end{claim}
\begin{proof}[Proof of claim]
Suppose not, so some ball $B(x, 2^{-I + 4})$ is compactly contained in $V$, and does not contain any point of $\bigcup_{I \leq J} \mathscr U_J$.

Every ball of radius $2^{-I + 1}$ in $B(x, 2^{-I + 4})$ intersects a line segment in $\mathscr L_{I - 1}$.
If this is false of $B(y, 2^{-I + 1})$, then by maximality of the packing $\mathscr P_I$, any point of $V \setminus \bigcup \mathscr L_{I - 1}$ must be within $2^{-I}$ of a ball in $\mathscr P_I$, or $\partial V$, or $\bigcup \mathscr L_{I - 1}$.
But by definition of $y$, no ball in $\mathscr P_I$ intersects $B(y, 2^{-I})$, a set which also does not intersect $\partial V$ or $\mathscr L_{I - 1}$.
This is a contradiction.

In particular, $B(x, 2^{-I + 1})$ intersects a line segment $\ell(q_1) \in \mathscr L_{I - 1}$.
Suppose that $q_1 \in \mathscr U_{J_1}$.
Since $q_1 \notin B(x, 2^{-I + 4})$, but $\ell(q_1) \subset B(q_1, 4^{-{J_1}})$,
$$\dist(\partial B(x, 2^{-I + 4}), B(x, 2^{-I + 1})) < 4^{-J_1}.$$
From this it follows that
$$2^{-2J_1} = 4^{-J_1} > 2^{-I + 4} - 2^{-I + 1} > 2^{-I + 3}.$$
Therefore $J_1 < (I - 3)/2$.

Since $k \geq 2$, there is a ball $B(y, 2^{-I + 1}) \subseteq B(x, 2^{-I + 2})$ which does not intersect $\ell(q_1)$.
Therefore there is a line segment $\ell(q_2) \in \mathscr L_{I - 1}$, $q_2 \in \mathscr U_{J_2}$, which intersects $B(y, 2^{-I + 1})$.
Arguing as above, $J_2 < (I - 3)/2$ as well.

Since $\ell(q_1)$ and $\ell(q_2)$ both intersect $B(x, 2^{-I + 2})$,
$$\dist(\ell(q_1), \ell(q_2)) < 2^{-I + 3}.$$
On the other hand, by Claim \ref{separation of line segments},
$$\dist(\ell(q_1), \ell(q_2)) > 2^{-\frac{I - 3}{2}} - 4^{-\frac{I - 3}{2}},$$
which is a contradiction when $I \geq 10$.
\end{proof}

The bound $\Lip(u_I, V) \leq K$ and the Arzela-Ascoli theorem implies that along a subsequence $u_I \to u$ in $C^0$, where:
\begin{enumerate}
\item $\Lip(u, V) \leq K$.
\item $u|_{\partial V} = f$.
\item For every $I \in \NN$ and $\ell \in \mathscr L_I$, $u|_\ell = u_I|_\ell$.
\end{enumerate}
The last condition implies that if $\ell \in \mathscr L_I$, then $\Lip(u, \ell) \geq K$, so by (\ref{Lip is sup of local Lips}) there exists $x \in \ell$ such that $Lu(x) \geq K$.
But then $\dist(x, \mathscr U_I) < 4^{-I}$.
By Claim \ref{almost density of balls} it follows that for any $y \in V$ and $I \geq 10$, one of the following holds:
\begin{enumerate}
\item $\dist(y, \partial V) \leq 2^{-I + 4}$.
\item There exists $x \in V$ such that $Lu(x) \geq K$ and $|x - y| < 2^{-I + 5}$.
\end{enumerate}
Taking $I \to \infty$ we see that $\{Lu \geq K\}$ is dense in $V$.
But it is also closed in $V$, so $Lu \geq K$ everywhere.
Since in addition $Lu \leq K$, the result follows.
\end{proof}

%%%%%%%%%%%%%%%%%%%%%%%%%%
\subsection{Construction of the combinatorial model}
We introduce some notation for a finite simplicial complex $\tau$.
Let $F_i(\tau)$ be the set of $i$-faces of $\tau$.
The \dfn{mesh size} $\mesh(\tau)$ of $\tau$ is 
$$\mesh(\tau) := \max_{E \in F_1(\tau)} \diam(E).$$
For any face $T \in F_i(\tau)$ with $i \geq 1$, $\diam(T) \leq \mesh(\tau)$.

Now we fix a triangulation $\tau$ of the open polytope $U \subseteq \RR^m$.
For each $h > 0$, define $\tau_h$ to be an iterated barycentric subdivision of $\tau$ such that:
\begin{enumerate}
\item $\mesh(\tau_h) \leq h$.
\item If $h' < h$, then $\tau_{h'}$ is a subdivision of $\tau_h$.
\end{enumerate}

\begin{proposition}[construction of shape maps]\label{construction of shape maps}
Let $X := F_0(\tau_h)$ be the $0$-skeleton.
For each map $u_0: X \to \RR^n$ there exists a map $u: u \to \RR^n$ such that $u|_X = u_0$ and for every $x \in X$,
\begin{equation}\label{Lip constant comes from 1 skeleton}
Lu(x) = \max_{\substack{T \in F_m(\tau_h) \\ x \in T}} \max_{\substack{y, z \in X \\ y, z \in T}} \frac{|u(y) - u(z)|}{|y - z|}.
\end{equation}
\end{proposition}
\begin{proof}
Let $k \geq 1$, and let $T$ be a $k$-face.
Inductively define $u_k|_T$ to be the extension of $u_{k - 1}$ given by Lemma \ref{eikonal extension}; then $u_k$ is Lipschitz on the $k$-skeleton of $\tau_h$, and by induction on $k$, for every $x \in T \setminus \partial T$, $Lu(x) = \Lip(u, V_T)$ where $V_T$ is the set of vertices of $T$.

Let $u := u_m$.
By Lemma \ref{partial lower semicontinuity}, the estimate (\ref{Lip constant comes from 1 skeleton}) holds because it does whenever $x \in T^\circ$ for some $m$-face $T$.
\end{proof}

\begin{definition}
Let $h > 0$.
For every map $u_0: F_0(\tau_h) \to \RR^n$, choose a map $u: U \to \RR^n$ satisfying the conclusion of Proposition \ref{construction of shape maps}.
We call $u$ the $h$-\dfn{shape map} extending $u_0$.
\end{definition}

%%%%%%%%%%%%%%%
\subsection{Approximation of Lipschitz maps}
We define a projection operator $\Pi_h$, by declaring that for every continuous map $v: U \to \RR^n$, $\Pi_h v$ is the unique $h$-shape map such that 
$$\Pi_h v|_{F_0(\tau_h)} = v|_{F_0(\tau_h)}.$$
We can use (\ref{Lip constant comes from 1 skeleton}) to show that for every $x \in U$,
\begin{equation}\label{discrete derivative is vertex lip}
L \Pi_h u(x) = \max_{\substack{T \in F_m(\tau_h) \\ x \in T}} \max_{\substack{y, z \in F_0(\tau_h) \\ y, z \in T}} \frac{|u(y) - u(z)|}{|y - z|}.
\end{equation}
We now use this estimate to show the convergence of $\Pi_h u$ to $u$ whenever $u$ is Lipschitz.

\begin{lemma}\label{uniform convergence}
For every $x \in U$ and every Lipschitz map $u: U \to \RR^n$,
$$|\Pi_h(x) - u(x)| \leq 2h \Lip(u, B(x, h)).$$
\end{lemma}
\begin{proof}
Suppose that $T \in F_m(\tau_h)$ attains the maximum in (\ref{discrete derivative is vertex lip}).
Let $y$ be a vertex of $T$.
Then $\Pi_h u(y) = u(y)$, so 
$$|\Pi_h u(x) - u(x)| \leq |\Pi_h u(y) - u_h(x)| + |u(y) - u(x)|.$$
By (\ref{discrete derivative is vertex lip}) and the fact that $T \subseteq B(x, h)$,
\begin{align*}
|\Pi_h u(y) - u_h(x)| + |u(y) - u(x)| &\leq 2h \Lip(u, T) \leq 2h \Lip(u, B(x, h)). \qedhere
\end{align*}
\end{proof}

\begin{lemma}\label{du converges ae}
For any Lipschitz map $u: U \to \RR^n$, $L\Pi_h u \to Lu$ almost everywhere.
\end{lemma}
\begin{proof}
By Lemma \ref{uniform convergence}, $\Pi_h u \to u$ uniformly.
By (\ref{Lip constant comes from 1 skeleton}), $\Lip(\Pi_h u, U) \leq \Lip(u, U) < \infty$, so by Lemma \ref{Lip is lower semicontinuous},
$$Lu \leq \liminf_{h \to 0} L\Pi_hu$$
almost everywhere.
Thus it suffices to show that for almost every $x \in U$,
\begin{equation}\label{upper semicontinuity of discrete Lips}
\limsup_{h \to 0} L\Pi_h u(x) \leq Lu(x).
\end{equation}
By Lusin's theorem, for every $\varepsilon > 0$ there is a closed set $Z_\varepsilon \subset U$ with $\vol(Z_\varepsilon) < \varepsilon$ such that $Lu$ is continuous away from $Z_\varepsilon$.
It suffices to show (\ref{upper semicontinuity of discrete Lips}) for every $x \in U \setminus Z_\varepsilon$.

By (\ref{discrete derivative is vertex lip}) and (\ref{Lip is sup of local Lips}) for every $h > 0$ there exist $T_h \in F_m(\tau_h)$, $[y_h, z_h] \in F_1(\tau_h)$ and $w_h \in [y_h, z_h]$ such that $x \in T_h$, $[y_h, z_h] \subseteq T_h$, and
$$L \Pi_h u(x) \leq \frac{|u(y_h) - u(z_h)|}{|y_h - z_h|} \leq Lu(w_h).$$
If $h$ is small enough depending on $x$, then $T_h$ is contained in the open set $U \setminus Z_\varepsilon$, so 
\begin{align*}
\limsup_{h \to 0} L \Pi_h u(x) \leq \limsup_{h \to 0} Lu(w_h) &= Lu(x). \qedhere
\end{align*}
\end{proof}

\begin{proposition}\label{discrete tight maps converge}
Let $u, v: U \to \RR^n$ be Lipschitz maps, and let $v_h: U \to \RR^n$ be $h$-shape maps such that:
\begin{enumerate}
\item $u \prec v$.
\item $v_h \to v$ in $L^1$.
\item For some $K < \infty$ and every $h > 0$, $\Lip(v_h, U) \leq K$.
\end{enumerate}
Then there exists $h_* > 0$ such that for $0 < h < h_*$, $\Pi_h u \prec v_h$.
\end{proposition}
\begin{proof}
Let $u_h := \Pi_h u$ and
$$M := \sup_{Lu < Lv} Lv.$$
Since $u \prec v$, there exists $\delta > 0$ such that
$$\sup_{Lv < Lu} Lu(x) < M - \delta.$$

\begin{claim}
If $\delta, h$ are small enough, then 
$$\sup_{Lu_h < Lv_h} Lv_h > M - \delta.$$
\end{claim}
\begin{proof}[Proof of claim]
By definition of $M$, if $\delta$ is taken small enough, then there exists $x \in U$ such that
$$Lu(x) < M - \delta < Lv(x).$$
By Lemma \ref{partial lower semicontinuity}, and the fact that $\{Lu < M - \delta\}$ is open, it follows that the set
$$E := \{y \in U: Lu(y) < M - \delta < Lv(y)\}$$
satisfies $\vol(E) > 0$.
By Lemmata \ref{Lip is lower semicontinuous} and \ref{du converges ae}, we can find $y \in E$ such that
$$Lv(y) \leq \liminf_{h \to 0} Lv_h(y)$$
and $Lu_h(y) \to Lu(y)$.
Therefore for $h$ small enough,
$$Lu_h(y) < M - \delta < Lv_h(y).$$
In particular, $Lv_h(y) > Lu_h(y)$, so 
\begin{align*}
\sup_{Lu_h < Lv_h} Lv_h \geq Lv_h(y) &> M - \delta. \qedhere 
\end{align*}
\end{proof}

\begin{claim}
If $h$ is small enough, then 
$$\sup_{Lv_h < Lu_h} Lu_h \leq M - \delta.$$
\end{claim}
\begin{proof}[Proof of claim]
It suffices to show that for every sufficiently small $h > 0$ and every $\theta > 0$,
\begin{equation}\label{loss of theta}
\sup_{Lv_h < Lu_h - \theta} Lu_h \leq M - \delta.
\end{equation}
Indeed, if $Lv_h(x) < Lu_h(x)$, then there exists $\theta > 0$ such that $Lv_h(x) < Lu_h(x) - \theta$, so if (\ref{loss of theta}) holds, then $Lu_h(x) \leq M - \delta$, as claimed.

In order to prove (\ref{loss of theta}), we fix $\varepsilon > 0$, so by Proposition \ref{one sided Egorov} and Lemmata \ref{Lip is lower semicontinuous} and \ref{du converges ae}, there exists a set $Z_\varepsilon$ such that $\vol(Z_\varepsilon) < \varepsilon$ and $Lu_h \to Lu$ and 
$$Lv \leq \liminf_{h \to 0} Lv_h$$
uniformly on $U \setminus Z_\varepsilon$.
By appending $\partial U$ to $Z_\varepsilon$ and then using continuity from above of Lebesgue measure, we may assume that $Z_\varepsilon$ is open and contains a collar neighborhood of $\partial U$.
In particular, $U \setminus Z_\varepsilon$ is a compact subset of $U$.

We then claim 
\begin{equation}\label{loss of epsilon}
\limsup_{h \to 0} \sup_{\substack{Lv_h(x) < Lu_h(x) - \theta \\ x \notin Z_\varepsilon}} Lu_h(x) > M - \delta.
\end{equation}
To prove (\ref{loss of epsilon}), we use compactness and contradiction.
If (\ref{loss of epsilon}) is false, then for every $h > 0$ along a subsequence, there exist $x_h \in U \setminus Z_\varepsilon$ such that $Lv_h(x_h) < Lu_h(x_h) - \theta$ but $Lu_h(x_h) > M - \delta$.
Since $U \setminus Z_\varepsilon$ is compact, after taking another subsequence, we can find $x \in U \setminus Z_\varepsilon$ with $x_h \to x$.
By the uniform convergence, $Lu(x) \geq M - \delta$ but 
$$Lv(x) \leq \liminf_{h \to 0} Lv_h(x_h) < \liminf_{h \to 0} Lu_h(x_h) - \theta = Lu(x) - \theta,$$
which contradicts the definition of $\delta$.
So (\ref{loss of epsilon}) holds.

To deduce (\ref{loss of theta}) from (\ref{loss of epsilon}), we first let $Z := \bigcap_\varepsilon Z_\varepsilon$ and take $\varepsilon \to 0$ to deduce that for every sufficiently small $h > 0$,
\begin{equation}\label{loss of theta 2}
\sup_{\substack{Lv_h(x) < Lu_h(x) - \theta \\ x \notin Z}} Lu_h(x) \leq M - \delta.
\end{equation}
Let $x \in U$ satisfy $Lv_h(x) < Lu_h(x) - \theta$.
Then for every sufficiently small $\kappa > 0$,
$$Lv_h(x) < Lu_h(x) - \theta - \kappa.$$
Moreover, $V := \{Lv_h < Lu_h(x) - \theta - \kappa\}$ is an open neighborhood of $x$ since $Lv_h$ is upper semicontinuous.
So the set $W := \{Lu_h > Lu_h(x) - \kappa\} \cap V$ has $\vol(W) > 0$ by Lemma \ref{partial lower semicontinuity}.
So $W \setminus Z$ is nonempty and for any $y \in W$,
$$Lv_h(y) < Lu_h(x) - \theta - \kappa < Lu_h(y) - \theta$$
which, combined with (\ref{loss of theta 2}), implies $Lu_h(y) \leq M - \delta$.
Therefore
$$Lu_h(x) < Lu_h(y) + \kappa \leq M - \delta + \kappa.$$
We take $\kappa \to 0$ to deduce (\ref{loss of theta}).
\end{proof}

It is clear that the above claims imply the result.
\end{proof}

%%%%%%%%%%%%%%%%%%%%
\subsection{Existence of tight maps}
We assume that $U \subseteq \RR^m$ is an open convex polytope.
\todo{Since the domain is a length space, probably the convexity assumption can be removed.}
Then $\tau_h$ extends to a triangulation of every polytope in $\partial U$, so it makes sense to ask if a continuous map $f: \partial U \to \RR^n$ is a shape map.

\begin{definition}
An $h$-shape map $u: U \to \RR^n$ is \dfn{$h$-tight} if there does not exist an $h$-shape map $v$ such that $v \prec u$ and $u|_{\partial U} = v|_{\partial U}$.
\end{definition}

\begin{proposition}\label{discrete wellposedness}
Let $h > 0$ and let $f: \partial U \to \RR^n$ be an $h$-shape map.
Then there exists an $h$-tight map $u: U \to \RR^n$ extending $f$.
\end{proposition}
\begin{proof}
Let $G$ be the graph induced by the $1$-skeleton of $\tau_h$, and let $H$ be the subgraph induced by the $1$-skeleton of $\tau_h \cap \partial U$.
We introduce a weight on edges by $\omega(x, y) := |x - y|$.
Let $V_G$ be the set of vertices of $G$.
For an $h$-shape map $u$ and $x \in V_G$,
$$Lu(x) = \max_{y \sim_G x} \frac{|u(x) - u(y)|}{\omega(x, y)}.$$
From this, and the fact that Lipschitz moduli of $h$-shape maps are constant on $m$-faces, it follows that $u \prec v$ iff $u|_{V_G}$ is tighter than $v|_{V_G}$ in the combinatorial sense of Sheffield and Smart \cite[\S1.2]{Sheffield12}.
So there exists an $h$-tight map \cite[Theorem 1.2]{Sheffield12}.\footnote{Strictly speaking, \cite{Sheffield12} deals with unweighted graphs, but this assumption can be removed \cite[Remark 2.2]{Sheffield12}.}
\end{proof}

\begin{lemma}\label{tight implies best lipschitz}
Suppose that $U$ is convex, and let $u, v: U \to \RR^n$ be Lipschitz maps with $u|_{\partial U} = v|_{\partial U}$ and $\Lip(u, U) > \Lip(v, U)$.
Then $v \prec u$.
\end{lemma}
\begin{proof}
By (\ref{Lip is sup of local Lips}), there exists $x \in U$ such that $Lu(x) > \Lip(v, U)$, and then
\begin{align*}
\sup_{Lu > Lv} Lu &> \Lip(v, U) = \sup_U Lv \geq \sup_{Lv > Lu} Lv. \qedhere
\end{align*}
\end{proof}

\begin{theorem}
Let $f: \partial U \to \RR^n$ be a Lipschitz map.
Then there exists a tight map $u: U \to \RR^n$ extending $f$.
Moreover, $\Lip(u, U) = \Lip(f, \partial U)$.
\end{theorem}
\begin{proof}
Let $f_h := \Pi_h f$.
By Proposition \ref{discrete wellposedness} there exist $h$-tight maps $u_h: U \to \RR^n$ extending $f_h$.
By the Kirszbraun-Valentine theorem, there also exist extensions $\tilde w_h: U \to \RR^n$ of $f_h$ with $\Lip(\tilde w_h, U) = \Lip(f_h, \partial U)$.
By (\ref{discrete derivative is vertex lip}), $w_h := \Pi_h \tilde w_h$ satisfies
$$\Lip(w_h, U) = \Lip(f_h, \partial U) \leq \Lip(f, \partial U).$$
Since $f_h$ was already a shape map, the action of $\Pi_h$ on $\tilde w_h$ preserved its restriction to the boundary, so $w_h|_{\partial U} = f_h$.

By Lemma \ref{tight implies best lipschitz}, $\Lip(u_h, U) \leq \Lip(f, \partial U)$.
By the Arzela-Ascoli theorem, after passing to a subsequence, we may assume that $u_h \to u$ in $C^0$ for some Lipschitz map $u: U \to \RR^n$ with $\Lip(u, U) \leq \Lip(f, \partial U)$.
By Lemma \ref{uniform convergence}, $f_h \to f$ in $C^0$, so $u$ is an extension of $f$.

If $u$ is not tight, then there exists a Lipschitz map $v: U \to \RR^n$ with $v \prec u$.
By Proposition \ref{discrete tight maps converge}, there exists $h > 0$ such that $\Pi_h v \prec u_h$.
This is a contradiction, because $u_h$ is tight.
So $u$ is tight.
\end{proof}

%%%%%%%%%%%%%%%%
\section{More stuff to do}
\begin{conjecture}
The same existence result, but with curvature.
\end{conjecture}

\begin{conjecture}
The same existence result holds when the datum is a homotopy class.
The point is that homotopy classes still make sense in the PL category, and we should be able to modify the lexicographic trick of \cite[Theorem 1.2]{Sheffield12} to work for homotopy classes.
\end{conjecture}

\begin{conjecture}
If $u_p$ is a sequence of $p$-harmonic maps, then $u_p \to u$ where $u$ is tight.
In particular, if $u$ is $\infty$-harmonic in the variational sense of \cite{daskalopoulos2022} and is smooth and nonholomorphic, then $u$ is $\infty$-harmonic in the classical sense.
\end{conjecture}

\begin{conjecture}
Tight maps are unique. This cannot be shown by a discretization alone (since $\prec$ is an open condition, so shouldn't be preserved by limits).
\end{conjecture}

\begin{conjecture}
Discrete tight maps have a ``discrete geodesic lamination'' in their nonholomorphic stretch set.
This lamination converges to a (continuum) lamination of $U$.
\end{conjecture}

\begin{conjecture}
If $u$ is tight and $C^2$, then $|\dif u|_\infty$ is a viscosity solution of a degenerate-elliptic PDE.
Therefore $C^2$ tight maps have no critical points, as in the scalar-valued case \cite{Yu2006}.
\end{conjecture}

\begin{conjecture}
Loisel's finite element approximation of the scalar $\infty$-Laplacian \cite{Loisel_2020} converges to a best Lipschitz map which, in general, is not $\infty$-harmonic.
\end{conjecture}

\printbibliography

\end{document}
