\documentclass[reqno,11pt]{amsart}
\usepackage[letterpaper, margin=1in]{geometry}
\RequirePackage{amsmath,amssymb,amsthm,graphicx,mathrsfs,url,slashed,subcaption}
\RequirePackage[usenames,dvipsnames]{xcolor}
\RequirePackage[colorlinks=true,linkcolor=Red,citecolor=Green]{hyperref}
\RequirePackage{amsxtra}
\usepackage{cancel, longtable}
\usepackage{tikz, wrapfig}

% Add the 2020 MSC
\makeatletter
\@namedef{subjclassname@2020}{\textup{2020} Mathematics Subject Classification}
\makeatother

%\usepackage[T1]{fontenc}

% \setlength{\textheight}{9.3in} \setlength{\oddsidemargin}{-0.25in}
% \setlength{\evensidemargin}{-0.25in} \setlength{\textwidth}{7in}
% \setlength{\topmargin}{-0.25in} \setlength{\headheight}{0.18in}
% \setlength{\marginparwidth}{1.0in}
% \setlength{\abovedisplayskip}{0.2in}
% \setlength{\belowdisplayskip}{0.2in}
% \setlength{\parskip}{0.05in}
%\renewcommand{\baselinestretch}{1.05}

\title{Discrete tight maps}
\author{}
\address{}
\email{}
\date{\today}
\keywords{}
\subjclass[2020]{primary: }

\newcommand{\NN}{\mathbf{N}}
\newcommand{\ZZ}{\mathbf{Z}}
\newcommand{\QQ}{\mathbf{Q}}
\newcommand{\RR}{\mathbf{R}}
\newcommand{\CC}{\mathbf{C}}
\newcommand{\DD}{\mathbf{D}}
\newcommand{\PP}{\mathbf P}
\newcommand{\MM}{\mathbf M}
\newcommand{\II}{\mathbf I}
\newcommand{\Hyp}{\mathbf H}
\newcommand{\Sph}{\mathbf S}
\newcommand{\Torus}{\mathbf T}
\newcommand{\Group}{\mathbf G}
\newcommand{\GL}{\mathbf{GL}}
\newcommand{\Orth}{\mathbf{O}}
\newcommand{\SpOrth}{\mathbf{SO}}
\newcommand{\Ball}{\mathbf{B}}

\newcommand*\dif{\mathop{}\!\mathrm{d}}

\DeclareMathOperator{\card}{card}
\DeclareMathOperator{\dist}{dist}
\DeclareMathOperator{\id}{id}
\DeclareMathOperator{\len}{len}
\DeclareMathOperator{\Lex}{Lex}
\DeclareMathOperator{\Hom}{Hom}
\DeclareMathOperator{\mesh}{mesh}
\DeclareMathOperator{\coker}{coker}
\DeclareMathOperator{\supp}{supp}
\DeclareMathOperator{\tr}{tr}

\newcommand{\Two}{\mathrm{I\!I}}
\newcommand{\weakto}{\rightharpoonup}

\newcommand{\normal}{\mathbf n}
\newcommand{\vol}{\mathrm{vol}}

\DeclareMathOperator{\hull}{hull}
\newcommand{\diam}{\mathrm{diam}}
\DeclareMathOperator{\sech}{sech}
\newcommand{\inj}{\mathrm{inj}}
\newcommand{\Lip}{\mathrm{Lip}}
\newcommand{\Riem}{\mathrm{Riem}}

\DeclareMathOperator*{\essinf}{ess\,inf}
\DeclareMathOperator*{\esssup}{ess\,sup}

\newcommand{\dfn}[1]{\emph{#1}\index{#1}}

\renewcommand{\Re}{\operatorname{Re}}
\renewcommand{\Im}{\operatorname{Im}}

\newcommand{\loc}{\mathrm{loc}}
\newcommand{\cpt}{\mathrm{cpt}}

\def\Japan#1{\left \langle #1 \right \rangle}

\newtheorem{theorem}{Theorem}[section]
\newtheorem{badtheorem}[theorem]{``Theorem"}
\newtheorem{prop}[theorem]{Proposition}
\newtheorem{lemma}[theorem]{Lemma}
\newtheorem{sublemma}[theorem]{Sublemma}
\newtheorem{invariant}[theorem]{Invariant}
\newtheorem{proposition}[theorem]{Proposition}
\newtheorem{corollary}[theorem]{Corollary}
\newtheorem{conjecture}[theorem]{Conjecture}
\newtheorem{axiom}[theorem]{Axiom}
\newtheorem{assumption}[theorem]{Assumption}

\newtheorem{mainthm}{Theorem}
\renewcommand{\themainthm}{\Alph{mainthm}}

\newtheorem{claim}{Claim}[theorem]
\renewcommand{\theclaim}{\thetheorem\Alph{claim}}
% \newtheorem*{claim}{Claim}

\theoremstyle{definition}
\newtheorem{definition}[theorem]{Definition}
\newtheorem{remark}[theorem]{Remark}
\newtheorem{example}[theorem]{Example}
\newtheorem{notation}[theorem]{Notation}

\newtheorem{exercise}[theorem]{Discussion topic}
\newtheorem{homework}[theorem]{Homework}
\newtheorem{problem}[theorem]{Problem}

\makeatletter
\newcommand{\proofpart}[2]{%
  \par
  \addvspace{\medskipamount}%
  \noindent\emph{Part #1: #2.}
}
\makeatother



\numberwithin{equation}{section}


% Mean
\def\Xint#1{\mathchoice
{\XXint\displaystyle\textstyle{#1}}%
{\XXint\textstyle\scriptstyle{#1}}%
{\XXint\scriptstyle\scriptscriptstyle{#1}}%
{\XXint\scriptscriptstyle\scriptscriptstyle{#1}}%
\!\int}
\def\XXint#1#2#3{{\setbox0=\hbox{$#1{#2#3}{\int}$ }
\vcenter{\hbox{$#2#3$ }}\kern-.6\wd0}}
\def\ddashint{\Xint=}
\def\dashint{\Xint-}

\usepackage[backend=bibtex,style=alphabetic,giveninits=true]{biblatex}
\renewcommand*{\bibfont}{\normalfont\footnotesize}
\addbibresource{discrete_tight_maps.bib}
\renewbibmacro{in:}{}
\DeclareFieldFormat{pages}{#1}

\newcommand\todo[1]{\textcolor{red}{TODO: #1}}


\begin{document}
\begin{abstract}
    
\end{abstract}

\maketitle

%%%%%%%%%%%%%%%%%%%%%%%%%%%%%%%%%%%%%%%%%%%%%%%%%%%%%%%
\section{Introduction}
Let $U \subseteq \RR^d$ be an open polytope.

\begin{definition}\label{local Lipschitz constant}
For a Lipschitz map $u: U \to \RR^D$, define the \dfn{local Lipschitz constant}
$$Lu(x) := \inf_{r > 0} \Lip(u, B(x, r)).$$
\end{definition}

\begin{definition}[{\cite[\S1.3]{Sheffield12}}]\label{tighter}
Let $u, v: U \to \RR^D$ be Lipschitz maps such that $u|_{\partial U} = v|_{\partial U}$.
We say that $u$ is \dfn{tighter} than $v$, written $u \prec v$, if 
$$\sup_{Lu > Lv} Lu < \sup_{Lv > Lu} Lv.$$
If there is no tighter $v$ than $u$, we say that $u$ is \dfn{tight}.
\end{definition}

\begin{theorem}
Every Lipschitz map $f: \partial U \to \RR^D$ has a tight extension $u: U \to \RR^D$.
\end{theorem}

Showing that the finite element approximations converge is hard, because the local Lipschitz constant of the approximation is piecewise constant, and we do not have any a priori estimates on the $W^{s, p}$ norms of the exact solution when $s > 1$.
Therefore we do not expect the approximate local Lipschitz constants to converge in any stronger sense than pointwise convergence almost everywhere.
This difficulty is already implicit in the numerical analysis literature concerning the scalar $\infty$-Laplacian.
Even though finite element approximations to $\infty$-harmonic functions $u$ are available \cite{Loisel_2020}, we are not aware of any estimates on the approximation error for $\dif u$ as in the case of $p$-harmonic functions, $p < \infty$ \cite{Barrett93}.

%%%%%%%%%%%%%%%%%%
\subsection{Acknowlegments}
I would like to thank Georgios Daskalopolous, Brian Freidin, and Nate River for various helpful discussions.

This research was supported by the National Science Foundation's Graduate Research Fellowship Program under Grant No. DGE-2040433.

%%%%%%%%%%%%%%%%%%%
\section{Preliminaries}
\subsection{Notation}
Here is a brief summary of the notation used throughout this paper.

\begin{longtable}{@{\extracolsep{\fill}}lp{0.75\textwidth}}
\multicolumn{2}{c}{\textbf{General notation}}\\[4pt]
$d$ & dimension of the domain \\
$D$ & dimension of the codomain \\
$|A|_p$ & $p$th Schatten norm of the matrix $A$ ($p \in [1, \infty]$) \\
$\angle (k_1, x, k_2)$ & angle between the vectors $k_1 - x$ and $k_2 - x$ \\
$F_i(\tau)$ & set of $i$-dimensional faces of the simplicial complex $\tau$\\
\\
\multicolumn{2}{c}{\textbf{Sets}}\\[4pt]
$E^\circ$ & interior of the set $E$ \\
$\overline E$ & closure of $E$ \\
$\card(E)$ & cardinality of $E$ \\
$\hull(E)$ & convex hull of $E$ \\
$\vol(E)$ & Lebesgue measure of the Lebesgue measurable set $E$ \\
$\dashint_E u$ & $\vol(E)^{-1} \int_E u$ \\
\\
\multicolumn{2}{c}{\textbf{Lipschitz maps}}\\[4pt]
$\Lip(u, E)$ & Lipschitz constant of $u|_E$\\
$Lu(x)$ & local Lipschitz constant of $u$ at $x$ (Definition \ref{local Lipschitz constant})\\
$u \prec v$ & $u$ is tighter than $v$ (Definition \ref{tighter})\\
$Su(T)$ & combinatorial local Lipschitz constant of $u$ along the $d$-face $T$ (Definition \ref{combinatorial local Lip})\\
$u \prec_C v$ & $u$ is combinatorially tighter than $v$ (Definition \ref{combinatorial tight})\\
$\Pi_h u$ & unique $h$-shape map which agrees with $u$ on vertices (Definition \ref{shape map})
\end{longtable}

%%%%%%%%%%%%%%%%
\subsection{Measure theory}
\begin{proposition}[one-sided Egorov theorem]\label{one sided Egorov}
Let $X$ be a Polish space, $\mu$ a finite Borel measure on $X$, and $f_n, f$ $\mu$-measurable functions on $X$, such that for $\mu$-almost every $x \in X$,
$$f(x) \leq \liminf_{n \to \infty} f_n(x).$$
Then for every $\varepsilon > 0$ there exists a $\mu$-measurable set $Z_\varepsilon \subseteq X$ such that:
\begin{enumerate}
\item $\mu(Z_\varepsilon) < \varepsilon$.
\item For every $\delta > 0$ there exists $N \in \NN$ such that for every $n \geq N$ and every $x \in X \setminus Z_\varepsilon$,
$$f(x) < f_n(x) + \delta.$$
\end{enumerate}
\end{proposition}
\begin{proof}
Let 
$$E_{n, k} := \bigcup_{m \geq n} \left\{x \in X: f(x) \geq f_m(x) + \frac{1}{k}\right\}.$$
Then for $m \geq n$, $E_{m, k} \subseteq E_{n, k}$; since $f_n \to f$ $\mu$-almost everywhere, it follows that for every $k$, $\bigcap_n E_{n, k}$ is $\mu$-null.
By continuity from above of $\mu$, for every $k$, there exists $n(k)$ such that $\mu(E_{n(k), k}) < \varepsilon/2^k$.
We set $Z_\varepsilon := \bigcup_k E_{n(k), k}$, so $\mu(Z_\varepsilon) < \varepsilon$.
For each $\delta > 0$, if $k \geq \delta^{-1}$, then for
$$x \in X \setminus E_{n(k), k} \subseteq X \setminus Z_\varepsilon,$$
and $n \geq n(k)$, $f(x) < f_n(x) + \delta$.
\end{proof}

\begin{lemma}[lower semicontinuity of vectorial total variation]\label{TV lower semicontinuity}
Let $|\cdot|_M$ be a norm on $\Hom(\RR^d, \RR^D)$, let $u_n, u \in BV(U, \RR^D)$, and suppose that $u_n \to u$ in $L^1$.
Then for any open $V \subseteq U$,
$$\int_V |\dif u|_M(x) \dif x \leq \liminf_{n \to \infty} \int_V |\dif u_n|_M(x) \dif x.$$
\end{lemma}
\begin{proof}
Let $|\cdot|_M'$ be the dual norm to $|\cdot|_M$.
Let
$$\mathscr D := \{X \in C^1_\cpt(U, \Hom(\RR^d, \RR^D)): \||X|_M'\|_{C^0} \leq 1\}.$$
Integrating by parts, for any $v \in BV(U, \RR^D)$, 
$$\int_V |\dif v|_M(x) \dif x = \sup_{X \in \mathscr D} \int_V \langle \dif v, X\rangle(x) \dif x = \sup_{X \in \mathscr D} \int_V \langle v, \nabla \cdot X\rangle(x) \dif x.$$
The result follows when we use the $L^1$ convergence to bound
\begin{align*}
\sup_{X \in \mathscr D} \int_V \langle u, \nabla \cdot X\rangle(x) \dif x 
&\leq \liminf_{n \to \infty} \sup_{X \in \mathscr D} \int_V \langle u_n, \nabla \cdot X\rangle(x) \dif x. \qedhere
\end{align*}
\end{proof}

%%%%%%%%%%%%%%%%
\subsection{Lipschitz maps}
Let $u: U \to \RR^D$ be a Lipschitz map.
Its local Lipschitz constant $Lu$ is upper semicontinuous \cite[Lemma 4.2(a)]{Crandall2008}.
Though $Lu$ is not lower semicontinuous, it is ``lower semicontinuous on average"; this was claimed without proof by Sheffield and Smart \cite[\S1.4]{Sheffield12}.

\begin{lemma}[lower semicontinuity on average of $Lu$]\label{partial lower semicontinuity}
Let $x \in U$, let $V \subseteq U$ be an open neighborhood of $x$, and let $\delta > 0$.
Then for any Lipschitz map $u: U \to \RR^D$,
$$\vol(V \cap \{Lu > Lu(x) - \delta\}) > 0.$$
\end{lemma}
\begin{proof}
It suffices to show the contrapositive: if, for almost every $y \in V$, $Lu(y) \leq K$, then $Lu(x) \leq K$ as well.
Furthermore, by shrinking $V$ to be a ball if necessary, we may assume that $V$ is convex.
Let $\Delta_V$ be the diagonal of $V^2$. Then
$$Lu(x) \leq \Lip(u, V) \leq \sup_{(y, z) \in V^2 \setminus \Delta_V} \frac{|u(y) - u(z)|}{|y - z|}.$$
Since the difference quotient $f(y, z) := |u(y) - u(z)|/|y - z|$ is continuous on $V^2 \setminus \Delta_V$, this expression is the $L^\infty$ norm of $f$.
Moreover, $\Delta_V$ is a null subset of $V^2$, so
$$\|f\|_{L^\infty(V^2 \setminus \Delta_V)} = \lim_{p \to \infty} \left[\int_V \int_V \frac{|u(y) - u(z)|^p}{|y - z|^p} \dif y \dif z\right]^{1/p}.$$
We estimate this integral using the mean value and Jensen inequalities:
\begin{align*} 
\int_V \int_V \frac{|u(y) - u(z)|^p}{|y - z|^p} \dif y \dif z 
&\leq \int_V \int_V \frac{1}{|y - z|^p} \left[\int_{[y, z]} Lu(w) \dif w\right]^p \dif y \dif z \\
&\leq \int_V \int_V \frac{1}{|y - z|} \int_{[y, z]} Lu(w)^p \dif w \dif y \dif z. 
\end{align*}
Let $E(y, w)$ be the set of points in $V$ on the line $\ell$ through $y, w$ which are to the right of $w$, where $\ell$ is oriented so that the vector $w - y$ points to the right.
Then $w \in [y, z]$ iff $z \in E(y, w)$, so we can rewrite the bounds of integration using Fubini's theorem as follows:
$$\int_V \int_V \frac{1}{|y - z|} \int_{[y, z]} Lu(w)^p \dif w \dif y \dif z = \int_V \int_V \int_{E(y, w)} Lu(w)^p \frac{\dif z}{|y - z|} \dif y \dif w.$$
The point is that $w$ now ranges over $V$, and we know that for almost every $w \in V$, $Lu(w) \leq K$.
We therefore can estimate this integral, and undo the change in the order of integration, as follows:
\begin{align*} 
\int_V \int_V \int_{E(y, w)} Lu(w)^p \frac{\dif z}{|y - z|} \dif y \dif w
&\leq K^p \int_V \int_V \int_{E(y, w)} \frac{\dif z}{|y - z|} \dif y \dif w \\
&= K^p \int_V \int_V \int_{[y, z]} \frac{\dif w}{|y - z|} \dif y \dif z \\
&= K^p \vol(V)^2.
\end{align*}
Therefore 
\begin{align*}
Lu(x) &\leq \|f\|_{L^\infty(V^2 \setminus \Delta_V)} \leq K \lim_{p \to \infty} \vol(V)^{2/p} = K. \qedhere 
\end{align*}  
\end{proof}

\begin{lemma}
For every Lipschitz map $u: U \to \RR^D$:
\begin{enumerate}
\item For every compact convex set $K \Subset U$,
\begin{equation}\label{Lip is sup of local Lips}
\Lip(u, K) = \max_{x \in K} Lu(x).
\end{equation}
\item For every open convex set $V \subseteq U$,
\begin{equation}\label{mean value estimate}
\Lip(u, V) = \||\dif u|_\infty\|_{L^\infty(V)}.
\end{equation}
\item For every $x \in U$,
\begin{equation}\label{localized mean value estimate}
Lu(x) = \lim_{r \to 0} \||\dif u|_\infty\|_{L^\infty(B(x, r))}.
\end{equation}
\end{enumerate}
\end{lemma}
\begin{proof}
This is essentially the content of \cite[Lemma 4.2]{Crandall2008} when $D = 1$.
The only meaningful difference when $D \geq 2$ is that we have to worry about which matrix norm is applied to $\dif u$, but of course we know the correct norm to use is $|\cdot|_\infty$.
\end{proof}

\begin{lemma}\label{Lip is du}
For every Lipschitz map $u: U \to \RR^D$ and almost every $x \in U$,
$$Lu(x) = |\dif u|_\infty(x).$$
\end{lemma}
\begin{proof}
Let $\varepsilon > 0$.
By Lusin's theorem, there is a closed set $Z_\varepsilon \subset U$ such that $\vol(Z_\varepsilon) < \varepsilon$ and $|\dif u|_\infty$ is continuous on $U \setminus Z_\varepsilon$.
Applying (\ref{localized mean value estimate}) and $x \in U \setminus Z_\varepsilon$,
$$\lim_{r \to 0} \||\dif u|_\infty\|_{L^\infty(B(x, r))} = |\dif u|_\infty(x),$$
hence $Lu(x) = |\dif u|_\infty(x)$.
The claim follows by taking $\varepsilon \to 0$.
\end{proof}

\begin{lemma}[lower semicontinuity of $L$]\label{Lip is lower semicontinuous}
Suppose that $u_k, u: U \to \RR^D$ are maps with $u_k \to u$ in $L^1$ and $\Lip(u_k, U) \leq K < \infty$.
Then for almost every $x \in U$,
$$Lu(x) \leq \liminf_{k \to \infty} Lu_k(x).$$
\end{lemma}
\begin{proof} 
We first use the Lebesgue differentiation theorem to write, for almost every $x \in U$,
$$Lu(x) = \lim_{r \to 0} \dashint_{B(x, r)} Lu(y) \dif y.$$
By Lemmata \ref{Lip is du} and \ref{TV lower semicontinuity},
\begin{align*}
\dashint_{B(x, r)} Lu(y) \dif y 
&= \dashint_{B(x, r)} |\dif u|_\infty(y) \dif y 
\leq \liminf_{k \to \infty} \dashint_{B(x, r)} |\dif u_k|_\infty(y) \dif y \\
&= \liminf_{k \to \infty} \dashint_{B(x, r)} Lu_k(y) \dif y.
\end{align*}
Because $Lu_k(y) \leq \Lip(u_k, U) \leq K$, we can apply dominated convergence:
$$\liminf_{k \to \infty} \dashint_{B(x, r)} Lu_k(y) \dif y = \dashint_{B(x, r)} \liminf_{k \to \infty} Lu_k(y) \dif y.$$
The result follows when we take $r \to 0$ and apply the Lebesgue differentiation theorem.
\end{proof}

\begin{lemma}\label{tight implies best lipschitz}
Suppose that $U$ is convex, and let $u, v: U \to \RR^D$ be Lipschitz maps with $u|_{\partial U} = v|_{\partial U}$ and $\Lip(u, U) > \Lip(v, U)$.
Then $v \prec u$.
\end{lemma}
\begin{proof}
By (\ref{Lip is sup of local Lips}), there exists $x \in U$ such that $Lu(x) > \Lip(v, U)$, and then
\begin{align*}
\sup_{Lu > Lv} Lu &> \Lip(v, U) = \sup_U Lv \geq \sup_{Lv > Lu} Lv. \qedhere
\end{align*}
\end{proof}

%%%%%%%%%%%%%%%%%%%%%
\section{Combinatorially tight maps}
We are going to regularize the problem of tight maps by studying tight maps from simplicial complexes.
This generalizes the approach of Sheffield and Smart \cite[\S2]{Sheffield12}, who studied tight maps from graphs.

\subsection{Lipschitz maps from simplicial complexes}
We first establish some notation.
If $\tau$ is a simplicial complex, we write $F_i(\tau)$ for the set of $i$-dimensional faces of $\tau$.

Throughout this section, we fix:
\begin{enumerate}
\item a finite simplicial complex $\tau$ with oriented edges,
\item a set $X \subseteq F_0(\tau)$ of vertices of $\tau$, which we call the \dfn{boundary}, and 
\item a \dfn{length} $\len(E)$ for each edge $E \in F_1(\tau)$.
\end{enumerate}

The \dfn{dimension} $\dim \tau$ of $\tau$ is the smallest $d$ such that for every $i \geq d + 1$, $F_i(\tau) = \emptyset$.
Let $d := \dim \tau$; we shall always assume that $d \geq 1$.
We assume that for $0 \leq i \leq d - 1$, every $i$-face of $\tau$ is contained in the boundary of some $i + 1$-face.

If $u: F_0(\tau) \to \RR^D$, we define $\dif u: F_1(\tau) \to \RR^D$, by, whenever $E$ is an edge from $x$ to $y$,
$$\dif u(E) := u(y) - u(x).$$

\begin{definition}\label{combinatorial local Lip}
Let $u: F_0(\tau) \to \RR^D$, and let $T \in F_d(\tau)$.
The \dfn{combinatorial local Lipschitz constant} is
$$Su(T) := \max_{\substack{E \in F_1(\tau) \\ E \subseteq T}} \frac{|\dif u(E)|}{\len(E)}.$$
\end{definition}

It is straightforward to check that $\dif$ and $S$ are continuous maps from $(\RR^D)^{F_0(\tau)}$ to $(\RR^D)^{F_1(\tau)}$ and $(\RR^D)^{F_d(\tau)}$ respectively.

\begin{definition}\label{combinatorial tight}
Let $u, v: F_0(\tau) \to \RR^D$.
We say that $u$ is \dfn{combinatorially tighter} than $v$, or write $u \prec_C v$, if $u|_X = v|_X$ and
$$\max_{Su > Sv} Su < \max_{Sv > Su} Sv.$$
If there does not exist $v$ such that $v \prec_C u$, then we say that $u$ is \dfn{combinatorially tight}.
\end{definition}

%%%%%%%%%%%%%%%%%%
\subsection{Existence of combinatorially tight maps}
We want to show that any Dirichlet data extends to a combinatorially tight map.
The proof of this result is similar to the analogous result when $\tau$ is a graph \cite[Theorem 1.2]{Sheffield12}.

To set up the proof, let $N := \card(F_d(\tau))$ and
$$\Omega := \{\mathbf x \in \RR^N: \mathbf x_1 \geq \cdots \geq \mathbf x_N \geq 0.\}$$
We equip $\Omega$ with its lexicographic ordering: $\mathbf x > \mathbf y$ means that for the least $i$ such that $\mathbf x_i \neq \mathbf y_i$, $\mathbf x_i > \mathbf y_i$.

Given $u: F_0(\tau) \to \RR^D$, we define $\Lex(u) \in \Omega$ to be the list of values of $Su(T)$, $T \in F_d(\tau)$, listed in nonincreasing order.
Since $S$ is continuous, $\Lex$ is also continuous. 
The following is immediate from the definitions:

\begin{lemma}\label{combinatorially tight means lexicographic}
Two maps $u, v: F_0(\tau) \to \RR^D$, with $u|_X = v|_X$, satisfy $u \prec_C v$ iff
$$\Lex(u) < \Lex(v).$$
\end{lemma}

\begin{proposition}\label{combinatorially tight existence}
For every $f: X \to \RR^D$ there exists a combinatorially tight map $u: F_0(\tau) \to \RR^D$ such that $u|_X = f$.
\end{proposition}
\begin{proof}
Let $\Omega_0$ be the set of $\mathbf x \in \Omega$, such that for some $u$ with $u|_X = f$, $\mathbf x = \Lex(u)$.
Then $\Omega_0$ is nonempty, and by continuity of $\Lex$, $\Omega_0$ is closed.

Let $\Omega_1$ be the set of $\mathbf x \in \Omega_0$ which minimize $\mathbf x_1$; then $\Omega_1$ is closed as above.
By continuity of $\Lex$ and the finite-dimensionality of the space of maps $F_0(\tau) \to \RR^D$, a minimizer $\mathbf x$ exists, with $\mathbf x_1 < \infty$.
Since $0 \leq \mathbf x_i \leq \mathbf x_1$ for $2 \leq i \leq N$, $\Omega_1$ is bounded.
So $\Omega_1$ is compact and nonempty.

Suppose that $1 \leq j \leq N - 1$ and $\Omega_j$ is compact and nonempty.
We inductively define $\Omega_{j + 1}$ to be the set of $\mathbf x \in \Omega_j$ which minimize $\mathbf x_{j + 1}$.
Then by continuity of $\Lex$ and assumption on $\Omega_j$, $\Omega_{j + 1}$ is compact and nonempty.

Finally, let $\Lex(u) \in \Omega_N$.
Then it is impossible to find $\Lex(v) \in \Omega_0$ such that $\Lex(v) < \Lex(u)$, so by Lemma \ref{combinatorially tight means lexicographic}, $v \prec_C u$, as desired.
\end{proof}


%%%%%%%%%%%%%%%%%%%%%%%%%%%
\section{Eikonal Kirszbraun-Valentine theorem}
In order to pass from a tight map defined on a graph to a continuously defined tight map, it is convenient to know that given a simplex $T$ and a map defined on the vertices of $T$, we can extend to a map defined on all of $T$ with constant local Lipschitz constant.

\begin{theorem}\label{eikonal extension}
Let $V \Subset \RR^d$ be a bounded convex open set, and let $\Gamma \subseteq \partial V$ be closed.
For every Lipschitz map $f: \Gamma \to \RR^D$, there exists an extension $u: V \to \RR^D$ of $f$, such that for every $x \in V$,
$$Lu(x) = \Lip(f, \Gamma).$$
\end{theorem}

Let us motivate the proof of Theorem \ref{eikonal extension}.
First we check that we cannot simply use the linear extension of $f$:

\begin{example}
Let
$$V = \{(x, y) \in \RR^2: x > 0, y > 0, x + y < 1\}$$
and $f(0, 0) = 0$, $f(1, 0) = f(0, 1) = 1$.
Then the unique linear extension of $f$ is $u(x, y) = x + y$.
Then $\Lip(f, X) = 1$ but $\Lip(u, V) = \sqrt 2$.
\end{example}

If we normalize $\Lip(f, \Gamma) = 1$, then Theorem \ref{eikonal extension} is essentially solving the eikonal equation
\begin{equation}\label{eikonal equation}
\begin{cases}
|\dif u|_\infty = 1 \\
u|_\Gamma = f.
\end{cases}
\end{equation}
It is tempting to look for solutions of (\ref{eikonal equation}) using Perron's method.
Of course, if $D \geq 2$, then (\ref{eikonal equation}) is a strongly coupled system, so the usual formulation of Perron's method \cite{Ishii92} fails.
However, this scheme would fail even when $D = 1$, because Perron's method as usually formulated gives a viscosity solution, but viscosity solutions of (\ref{eikonal equation}) do not have to satisfy $u|_\Gamma = f$ in the classical sense \cite[\S7.C]{Crandall92}.

So we need a strategy for constructing \emph{nonviscosity} solutions of (\ref{eikonal equation}).
We know by the Kirszbraun-Valentine theorem that we can construct a subsolution $\tilde u$ of (\ref{eikonal equation}) in the sense that (classically)
$$\begin{cases}
L\tilde u \leq 1 \\
\tilde u|_\Gamma = f.
\end{cases}$$
The idea, as in Perron's method, is to iteratively modify $\tilde u$ in order to increase $L \tilde u$, without ever violating that $\tilde u$ is a subsolution.
We have to carry out this process carefully, however, because our modifications are artificial oscillations, which could experience destructive interference in the limit.
So we make our modifications on spaced-out balls, so that they cannot cancel with each other.

%%%%%%%%%%%%%%%%%%%%%
\subsection{Classical Kirszbraun-Valentine theorem}
We shall need the main lemma in the proof of the Kirszbraun-Valentine theorem elsewhere, so we now state it, and for the reader's convenience recall why it implies the Kirszbraun-Valentine theorem.
We follow \cite[\S3.1]{Kassel17}.

\begin{lemma}\label{KV lemma}
Let $\Gamma \Subset \RR^d$ be a nonempty compact set and $u_0: \Gamma \to \RR^D$ a nonconstant Lipschitz map.
For any $x \in \RR^d \setminus \Gamma$, the function 
\begin{align*} 
\RR^D &\to \RR_+ \\
\xi' &\mapsto C_{\xi'} := \max_{k \in \Gamma} \frac{|\xi' - u_0(k)|}{|x - k|}
\end{align*}
attains a minimum $C_\xi > 0$ at some $\xi \in \RR^D$.

Let $\xi \in \RR^D$ be a minimizer, and let
$$\Gamma' := \left\{k \in \Gamma: \frac{|\xi - u_0(k)|}{|x - k|}\right\}.$$
Then $\xi \in \hull(u_0(\Gamma'))$, and:
\begin{enumerate}
\item either there exist $k_1, k_2 \in \Gamma'$ such that 
$$0 \leq \angle (k_1, x, k_2) < \angle (u_0(k_1), \xi, u_0(k_2)) \leq \pi,$$
\item or there exists a Borel probability measure $\nu$ on $\Gamma'$ such that:
\begin{enumerate}
\item $\xi \in \hull(\supp((u_0)_* \nu))$, and
\item for $\nu$-almost every $k_1, k_2 \in \Gamma'$,
$$\angle (k_1, x, k_2) = \angle (u_0(k_1), \xi, u_0(k_2)).$$
\end{enumerate}
\end{enumerate}
Finally, 
\begin{equation}\label{KV lemma improves Lip}
C_\xi \leq \Lip(u_0, \Gamma).
\end{equation}
\end{lemma}
\begin{proof}
The proof of all assertions except for (\ref{KV lemma improves Lip}) is identical to \cite[Lemma 3.2]{Kassel17}, which was formulated for hyperbolic space.
To establish (\ref{KV lemma improves Lip}), we claim that we can use the other assertions of the lemma to choose $k_1, k_2 \in \Gamma'$ such that 
$$\angle (k_1, x, k_2) \leq \angle (u_0(k_1), \xi, u_0(k_2))$$
and $\angle (u_0(k_1), \xi, u_0(k_2)) > 0$.

If this is not possible, then there is a Borel probability measure $\nu$ such that $\xi \in \hull(\supp((u_0)_* \nu))$ and for $\nu$-almost every $k_1, k_2 \in \Gamma'$ we have
$$\angle (u_0(k_1), \xi, u_0(k_2)) = 0$$
which implies that $u_0(k_1), u_0(k_2), \xi$ are collinear, with $\xi$ lying outside the segment $[u_0(k_1), u_0(k_2)]$.
In particular $K'$ is contained in a line, and $\xi$ is not contained in any line segment $[u_0(k_1'), u_0(k_2')]$ with $k_i' \in \supp \nu$.
This contradicts the fact that $\xi \in \hull(\supp((u_0)_* \nu))$.

With $k_1, k_2 \in \Gamma'$ as above, we may use the definition of $\Gamma'$ and triangle comparison to conclude
\begin{align*}
C_\xi &\leq \frac{|u_0(k_1) - u_0(k_2)|}{|k_1 - k_2|} \leq \Lip(u_0, \Gamma). \qedhere 
\end{align*}
\end{proof}

\begin{proposition}[Kirszbraun-Valentine]
Let $\Gamma \Subset \RR^d$ be a nonempty compact set and $u_0: \Gamma \to \RR^D$ a Lipschitz map.
Then there exists a Lipschitz extension $u: \RR^d \to \RR^D$ of $f$ such that $\Lip(u, \RR^d) = \Lip(u_0, \Gamma)$.
\end{proposition}
\begin{proof}
Let $(x_j)$ be a dense injective sequence in $\RR^d \setminus \Gamma$.
We inductively define $u_{j + 1}(x_i) = u_j(x_i)$ for $i \leq j$.
Let $x_{j + 1}$ be the input to Lemma \ref{KV lemma}, and let $\xi_{j + 1}$ be its output.
Then let $u_{j + 1}(x_{j + 1}) := \xi_{j + 1}$.
We then define $u(x_j) := u_j(x_j)$, so that
$$\Lip(u, \Gamma \cup \{x_j: j \in \NN\}) = \Lip(u_0, K)$$
by induction.
In particular, $u$ extends uniquely to a Lipschitz map defined on all of $\RR^d$ with the same Lipschitz constant.
\end{proof}


%%%%%%%%%%%%%%%%%%%%
\subsection{Modifying the Lipschitz constant on a packing}
We first show that we can modify a Lipschitz map $u$ on a small ball to increase $Lu$ in a controlled manner.
As a consequence, we show that we can modify $Lu$ safely on every ball in a packing.

Throughout the proof of Theorem \ref{eikonal extension}, given a ball $B$ of radius $r$, let $\Phi(B)$ be the concentric ball of radius $r^2$.
We recall some definitions from metric geometry:

\begin{definition}
Let $W \subseteq \RR^d$ be an open set.
\begin{enumerate}
\item An $r$-\dfn{packing} of $W$ is a set of disjoint balls $B \Subset W$ of radius $r$.
\item An $r$-packing $\mathscr P$ of $W$ is \dfn{maximal} if there does not exist an $r$-packing $\mathscr Q$ of $W$ with $\mathscr Q \supset \mathscr P$.
\end{enumerate}
\end{definition}

\begin{lemma}\label{improvement on one ball}
Let $W \subseteq \RR^d$ be a convex open set and $K > 0$.
Let $u: W \to \RR^n$ satisfy $\Lip(u, W) \leq K$.
Then there exists $v: W \to \RR^D$ such that:
\begin{enumerate}
\item $\Lip(v, W) = K$.
\item For some $W' \Subset W$ and every $x \in W \setminus W'$, $v(x) = u(x)$.
\item For every $\varepsilon > 0$, there exists a closed line segment $\ell \Subset W$ such that $\Lip(v, \ell) > K - \varepsilon$.
\end{enumerate}
\end{lemma}
\begin{proof}
We first observe that it suffices to show that there exists $v: W \to \RR^D$ with $v = u$ near $\partial W$, $\Lip(v, W) = K$, and $Lv(x) = K$ for some $x \in W$.
Indeed, if this is true, then there exist $y_n, z_n \in W$ with $y_n, z_n \to x$ and
$$\frac{|v(y_n) - v(z_n)|}{|y_n - z_n|} \to K.$$
If we take $\ell$ to be the line segment $[y_n, z_n]$ for a sufficiently large $n$, the result follows.

So suppose that for every $x \in W$, $Lu(x) < K$.
Choose a convex open set $W' \Subset W$, so $\Lip(u, W') < K$ by (\ref{Lip is sup of local Lips}).
Choose $\varphi \in C^\infty_\cpt(W', \RR^D)$ which is rapidly oscillating on some set $W'' \Subset W'$, and define
$$u_t(x) := u + t\varphi.$$
Since $w \mapsto \Lip(w, W')$ is a seminorm, $\beta(t) := \Lip(u_t, W')$ is a continuous function on $\RR_+$.
Moreover, $\beta(0) < K$ and $\beta(t) \to \infty$ as $t \to \infty$, so there exists $T > 0$ such that $\beta(T) = K$.
We then set $v := u_T$.
By (\ref{Lip is sup of local Lips}) and the fact that $\overline{W'}$ is a convex compact set, there exists $x \in W'$ such that
$$Lv(x) = \beta(T) = K.$$
Moreover, for $y \in W'$ we have $Lv(y) \leq K$ and for $y \in W \setminus W'$ we have $Lv(y) < K$.
So by (\ref{Lip is sup of local Lips}) and convexity of $W$, $\Lip(v, W) = K$.
\end{proof}

\begin{lemma}\label{improvement on packing}
Let $W \subseteq \RR^d$ be an open set, $K > 0$, and $0 < r \leq 1$.
Let $u: W \to \RR^D$ satisfy $\Lip(u, W) \leq K$.
Then for every $r$-packing $\mathscr P$ of $W$, there exists $v: W \to \RR^D$ such that:
\begin{enumerate}
\item For every $x \in W$, $Lv(x) \leq K$.
\item For every $x \in W \setminus \bigcup \{\Phi(B): B \in \mathscr P\}$, $v(y) = u(y)$.
\item For every $\varepsilon > 0$ and every $B \in \mathscr P$, there exists a closed line segment $\ell \Subset \Phi(B)$ such that $\Lip(v, \ell) > K - \varepsilon$.
\end{enumerate}
\end{lemma}
\begin{proof}
We apply Lemma \ref{improvement on one ball} to each ball $\Phi(B)$ such that $B \in \mathscr P$.
Since $r^2 \leq r$ and $\mathscr P$ is a packing, the balls $\Phi(B)$ are all disjoint.
\end{proof}

%%%%%%%%%%%%%%%%%%%%%%
\subsection{Iterating the packing procedure}
We are going to iteratively apply Lemma \ref{improvement on packing} to finer and finer packings; at each scale, we get some improvement to a given Lipschitz map $u_0$.
Later we shall take a limit as the scale goes to $0$; it is important that we choose our packings so carefully that in the limit, our improvements do not cancel each other.

\begin{lemma}\label{iterating the packing}
Let $u_0: V \to \RR^D$ be a map with $\Lip(u_0, V) \leq K$, and $\mathscr L_0 := \emptyset$.
Then there exist
\begin{enumerate}
\item maps $u_i: V \to \RR^D$ with $\Lip(u_i, V) \leq K$,
\item finite sets of line segments $\mathscr L_i \supset \mathscr L_{i - 1}$,
\item and maximal $2^{-i}$-packings $\mathscr P_i$ of $V \setminus \bigcup \mathscr L_{i - 1}$,
\end{enumerate}
such that for every $i \geq 1$,
\begin{enumerate}
\item \label{balls and lines} there is a bijection
$$\ell: \mathscr P_i \to \mathscr L_i \setminus \mathscr L_{i - 1},$$
which sends $B \in \mathscr P_i$ to a line segment $\ell(B) \Subset \Phi(B)$ such that
$$\Lip(u_i, \ell(B)) > K - 2^{-i},$$
\item \label{preservation} and for each $x \in V \setminus \bigcup \{\Phi(B): B \in \mathscr P_i\}$,
$$u_i(x) = u_{i - 1}(x).$$
\end{enumerate}
\end{lemma}
\begin{proof}
We proceed by induction. Assume that we have defined $u_{i - 1}, \mathscr L_{i - 1}, \mathscr P_{i - 1}$ to have the desired properties.
Let $W := V \setminus \bigcup \mathscr L_{i - 1}$.
Then $W$ is an open set since $\mathscr L_{i - 1}$ is a finite set of closed sets, so we may choose a maximal $2^{-i}$-packing $\mathscr P_i$ of $W$.
Let $Z := \bigcup \{\Phi(B): B \in \mathscr P_i\}$.

By Lemma \ref{improvement on packing}, we can find $u_i: W \to \RR^D$ and line segments $\ell(B)$, $B \in \mathscr P_i$, such that $u_i$ agrees with $u_{i - 1}$ away from $Z$ and satisfies $Lu_i \leq K$, but for each $B \in \mathscr P_i$,
$$\Lip(u_i, \ell(B)) > K - 2^{-i}.$$
Since $V \setminus W$ does not intersect $Z$, $u_i = u_{i - 1}$ near $W$, and in particular, extends uniquely to all of $V$ while satisfying (\ref{preservation}).
Let
$$\mathscr L_i := \mathscr L_{i - 1} \cup \{\ell(B): B \in \mathscr P_i\}.$$
Since $V$ is bounded and hence has finite volume, the packing $\mathscr P_i$ is finite.
Since $\mathscr L_{i - 1}$ is finite, so is $\mathscr L_i$, and $\mathscr L_i$ satisfies (\ref{balls and lines}).
\end{proof}

Throughout the rest of this section, we fix a Lipschitz map $u_0: V \to \RR^n$, and let $\mathscr L_i, \mathscr P_i$ be as in Lemma \ref{iterating the packing}.
We are interested in the geometry of the packings $\mathscr P_i$; specifically, we want to show that any point in $V$ can be well-approximated by centers of balls in $\mathscr P_i$.

\begin{lemma}\label{separation of line segments}
For every $\ell, \ell' \in \mathscr L_i$, either $\ell = \ell'$ or
$$\dist(\ell, \ell') > 2^{-i} - 4^{-i}.$$
\end{lemma}
\begin{proof}
Suppose that $\ell \neq \ell'$.
By induction, we may assume that this result holds if $\ell, \ell' \in \mathscr L_{i - 1}$.
So suppose that $\ell \in \mathscr L_i \setminus \mathscr L_{i - 1}$, so by Lemma \ref{iterating the packing}(\ref{balls and lines}) there exists $B \in \mathscr P_i$ such that $\ell \Subset \Phi(B)$.

We claim that $\ell'$ does not intersect $B$.
Indeed, if $\ell' \in \mathscr L_{i - 1}$, then since balls in $\mathscr P_i$ are disjoint from $\bigcup \mathscr L_{i - 1}$, $\ell'$ does not intersect $B$.
Otherwise, by Lemma \ref{iterating the packing}(\ref{balls and lines}), there exists $B' \in \mathscr P_i$ such that $\ell' \subset \Phi(B') \Subset B'$. Since $\mathscr P_i$ is a packing, $\ell'$ does not intersect $B$.
In either case, the result follows, because
\begin{align*} 
\dist(\ell, \ell') &\geq \dist(\ell, V \setminus B) > \dist(\Phi(B), V \setminus B) > 2^{-i} - 4^{-i}. \qedhere
\end{align*}
\end{proof}

\begin{lemma}\label{almost density of balls}
Suppose that $d \geq 2$. Let
$$\mathscr U_i := \{x \in V: B(x, 2^{-i}) \in \mathscr P_i\}.$$
For every $i \geq 10$ and $x \in V$,
$$\min\left(\dist(x, \partial V), \min_{j \leq i} \dist(x, \mathscr U_j)\right) \leq 2^{-i + 4}.$$
\end{lemma}
\begin{proof}
Suppose not, so some ball $B(x, 2^{-i + 4}) \Subset V$ does not contain any point of $\bigcup_{j \leq i} \mathscr U_j$.
Then every ball $B(y, 2^{-i + 1}) \subseteq B(x, 2^{-i + 4})$ intersects a line segment in $\mathscr L_{i - 1}$.
Otherwise, by maximality of the packing $\mathscr P_i$, any point of $V \setminus \bigcup \mathscr L_{i - 1}$ must be within $2^{-i}$ of a ball in $\mathscr P_i$, or $\partial V$, or $\bigcup \mathscr L_{i - 1}$.
But by definition of $y$, no ball in $\mathscr P_i$ intersects $B(y, 2^{-i})$, a set which also does not intersect $\partial V$ or $\mathscr L_{i - 1}$.
This is a contradiction.

Applying the previous paragraph with $y := x$, we can find $j_1 \leq i - 1$ and $q_1 \in \mathscr U_{j_1}$ such that $\ell_1 := \ell(B(q_1, 2^{-j_1}))$ intersects $B(x, 2^{-i + 1})$.
Since $q_1 \notin B(x, 2^{-i + 4})$, but $\ell_1 \subset \Phi(B(q_1, 2^{-j_1})) = B(q_1, 4^{-{j_1}})$,
$$\dist(\partial B(x, 2^{-i + 4}), B(x, 2^{-i + 1})) < 4^{-j_1}.$$
From this it follows that
$$2^{-2j_1} = 4^{-j_1} > 2^{-i + 4} - 2^{-i + 1} > 2^{-i + 3}.$$
Therefore $j_1 < (i - 3)/2$.

Since $d \geq 2$, there is a ball $B(y, 2^{-i + 1}) \subseteq B(x, 2^{-i + 2})$ which does not intersect $\ell_1$.
Therefore there exist $j_2 \leq i - 1$ and $q_2 \in \mathscr U_{j_2}$ such that $\ell_2 := \ell(B(q_2, 2^{-j_2}))$ intersects $B(y, 2^{-i + 1})$.
Arguing as above, we see that $j_2 \leq (i - 3)/2$ as well.

Since $\ell_1$ and $\ell_2$ both intersect $B(x, 2^{-i + 2})$,
$$\dist(\ell_1, \ell_2) < 2^{-i + 3}.$$
On the other hand, by Lemma \ref{separation of line segments},
$$\dist(\ell_1, \ell_2) > 2^{-\frac{i - 3}{2}} - 4^{-\frac{i - 3}{2}},$$
which is a contradiction when $i \geq 10$.
\end{proof}

As we shall see, we can get good estimates on $Lu_i(x)$ provided that $x$ is $\sim 2^{-i}$-close to $\mathscr U_j$ for some $j \sim i$.
Motivated by Lemma \ref{almost density of balls}, we shall show exponential decay of the volume of an exceptional set
\begin{equation}\label{exponential exceptional set}
Z_i := \bigcup \left\{B(x, 2^{-i+4}): x \in \partial V \cup \bigcup_{j \leq i/2} \mathscr U_j\right\}.
\end{equation}

\begin{lemma}\label{decay of the exceptional set}
Suppose that $d \geq 2$.
Then there exists $C > 0$ which depends only on $V$ such that $\vol(Z_i) \leq C/2^i$.
\end{lemma}
\begin{proof}
The Minkowski codimension of $\partial V$ is $1$ with uniform constants at all scales so the volume of a $2^{-i+4}$-neighborhood $T$ of $\partial U$ is $\lesssim 2^{-i}$.
Next, we estimate
$$\card \mathscr U_j \leq \frac{\vol(\bigcup \mathscr U_j)}{\vol(B(0, 2^{-j}))} \lesssim 2^{jd} \vol(V) \lesssim 2^{jd}.$$
Summing in $j$ we obtain 
$$\card\left(\bigcup_{j \leq i/2} \mathscr U_j\right) \leq \sum_{j=1}^{i/2} 2^{jd} = \frac{2^{id/2} - 2^d}{1 - 2^{-d}} \lesssim 2^{id/2}.$$
Thus we estimate
\begin{align*}
\vol(Z_i)
&\leq \vol(T) + \card\left(\bigcup_{j \leq i/2} \mathscr U_j\right) \vol(B(0, 2^{-i})) 
\lesssim 2^{-is} + 2^{id/2} 2^{-id}.
\end{align*}
Since $d \geq 2$, this quantity is $\lesssim 2^{-i}$.
\end{proof}

%%%%%%%%%%%%%%
\subsection{Proof of Theorem \ref{eikonal extension}}
If $d = 1$, then $V$ is an interval and we can choose $u$ to be a linear extension of $f$.
Henceforth we assume $d \geq 2$.

Let $K := \Lip(f, \Gamma)$.
By the Kirszbraun-Valentine theorem, there exists a Lipschitz extension $u_0: V \to \RR^D$ of $f$ with $\Lip(u_0, V) \leq K$.
Let $u_i, \mathscr P_i, \mathscr L_i$ be as in Lemma \ref{iterating the packing}.
Since $\Lip(u_i, V) \leq K$, we can apply the Arzela-Ascoli theorem and pass to a subsequence to find a Lipschitz extension $u: V \to \RR^D$ of $f$ with $\Lip(u, V) \leq K$ and $u_i \to u$ in $C^0$.

It remins to show that $Lu = K$ everywhere.
To this end, let $Z_i$ be as in (\ref{exponential exceptional set}).
Introduce the exceptional set $Z := \bigcap_{i \geq 10} \bigcup_{j \geq i} Z_j$.
By Lemma \ref{decay of the exceptional set},
$$\vol\left(\bigcup_{j \geq i} Z_j\right) \lesssim \sum_{j = i}^\infty 2^{-j} = 2^{-i + 1},$$
so by continuity from above, $Z$ is null.
In particular $V \setminus Z$ is dense.

For every $x \in V \setminus Z$ and $\varepsilon > 0$, there exists $i \geq 10$ such that
$$2^{-i} < \max(\varepsilon^2, 16 \dist(x, \partial V))$$
and $x \notin \bigcup_{j \geq i} Z_j$.
By Lemma \ref{almost density of balls}, there exists $j \geq i/2$ and $q \in \mathscr U_j$ such that $|x - q| < 2^{-i + 4}$.

From Lemma \ref{iterating the packing}(\ref{balls and lines}), there exists a line segment $\ell \in \mathscr L_j$ such that
\begin{align*}
\ell &\Subset B(q, 4^{-j}) \subseteq B(q, 2^{-i}) \\
\Lip(u_j, \ell) &> K - 2^{-j} \geq K - 2^{-i/2} > K - \varepsilon.
\end{align*}
An induction on Lemma \ref{iterating the packing}(\ref{preservation}) implies that for $p \geq j$, $u_p|_\ell = u_j|_\ell$; taking the limit, we see that $u|_\ell = u_j|_\ell$.
Therefore $\Lip(u, \ell) > K - \varepsilon$, so by (\ref{Lip is sup of local Lips}), there exists $y_i \in \ell$ such that $Lu(y_i) > K - \varepsilon$.
But $|y_i - q| < 2^{-i}$, so $|y_i - x| < 2^{-i + 5}$.

We can now take $i \to \infty$, so $y_i \to x$ with $Lu(y_i) = K$.
Since $Lu(x) \leq K$ and $Lu$ is upper semicontinuous, $Lu(x) = K$ as well.
Since $V \setminus Z$ is dense, the same conclusion holds even if $x \in Z$, proving Theorem \ref{eikonal extension}.



%%%%%%%%%%%%%%%%%%%%%%%%%%%
\section{Existence of tight maps}
\begin{theorem}\label{existence of tight maps}
Let $U \subseteq \RR^d$ be an convex open bounded polytope.
Let $f: \partial U \to \RR^D$ be a Lipschitz map.
Then there exists a tight map $u: U \to \RR^D$ extending $f$.
Moreover, $\Lip(u, U) = \Lip(f, \partial U)$.
\end{theorem}

As proposed by Sheffield and Smart \cite[Question 5.1]{Sheffield12}, we are going to prove Theorem \ref{existence of tight maps} by approximating $U$ by a graph.
Even when $D = 1$, lattice approximations to $U$ are difficult to work with, because the $\infty$-Laplacian only cares about the second derivative of $u$ in the direction of $\nabla u$, which does not have to be a lattice direction \cite[\S3]{Oberman13}.

Instead, we use a triangulation.
This turns out to be technically convenient, because for every simplex $T$ with vertex set $T_0$, $\Lip(u, T_0)$ can be obtained by maximizing over edges of $T$.
This is not the case for lattice models, when the difference quotient in the definition of $\Lip(u, T_0)$ is maximized by two diametrically opposed vertices of a hypercube $T$.

%%%%%%%%%%%%%%%%%%%%%
\subsection{Construction of the combinatorial model}
The \dfn{mesh size} $\mesh(\tau)$ of a finite simplicial complex $\tau$ is 
$$\mesh(\tau) := \max_{E \in F_1(\tau)} \len(E).$$

Fix a triangulation $\tau$ of the open polytope $U \subseteq \RR^d$.
For each $h > 0$, define $\tau_h$ to be an iterated barycentric subdivision of $\tau$ such that:
\begin{enumerate}
\item $\mesh(\tau_h) \leq h$.
\item If $h' < h$, then $\tau_{h'}$ is a subdivision of $\tau_h$.
\end{enumerate}
To construct the shape maps, we first use the following version of the Kirszbraun-Valentine theorem:

\begin{lemma}\label{convex KV theorem}
Let $X := F_0(\tau_h)$ be the $0$-skeleton.
For each map $u_0: X \to \RR^D$ there exists a map $u: u \to \RR^D$ such that $u|_X = u_0$ and for every $T \in F_d(\tau_h)$,
\begin{equation}\label{Lip constant bounded by 1 skeleton}
\Lip(u, T) \leq \Lip(u, X \cap T).
\end{equation}
\end{lemma}
\begin{proof}
Choose a dense injective sequence $\{x_j: j \in \NN\}$ in $U$ such that an initial segment $\{x_0, \dots, x_m\} = X$, and for every $j \geq m + 1$, there exists $T_j \in F_d(\tau)$ such that $x_j \in T^\circ_j$.
Given a map $u_j: \{x_0, \dots, x_{m + j}\} \to \RR^D$, let $\Gamma$ be the set of points $x_i$, $i \in \{0, \dots, m + j\}$, such that $x_i \in T_{j + 1}$.
Use Lemma \ref{KV lemma} to find $\xi \in \RR^D$ which minimizes
$$\lambda := \max_{k \in \Gamma} \frac{|\xi - u_j(k)|}{|x_{j + 1} - k|}.$$
In particular, $\lambda \leq \Lip(u_j, \Gamma)$.
We then set $u_{j + 1}(x_{j + 1}) := \xi$ and $u_{j + 1}(x_i) = u_j(x_i)$ for $i \leq j$.
By induction, we conclude that if $x_{i_1}, x_{i_2} \in T_{j + 1}$ and $\max(i_1, i_2) \leq j + 1$, then
\begin{equation}\label{Lip constant bounded by 1 skeleton pre}
\frac{|u_{j + 1}(x_{i_1}) - u_{j + 1}(x_{i_2})|}{|x_{i_1} - x_{i_2}|} \leq \Lip(u_{j + 1}, X \cap T_{j + 1}).
\end{equation}
Now if we let $u(x_j) = u_j(x_j)$, then $u$ is Lipschitz and defined on a dense subset of $U$.
So it extends uniquely to $U$, and by (\ref{Lip constant bounded by 1 skeleton pre}) we have (\ref{Lip constant bounded by 1 skeleton}).
\end{proof}

\begin{proposition}[construction of shape maps]\label{construction of shape maps}
Let $X := F_0(\tau_h)$ be the $0$-skeleton.
For each map $u_0: X \to \RR^D$ there exists a map $u: u \to \RR^D$ such that $u|_X = u_0$ and for every $x \in X$,
\begin{equation}\label{Lip constant comes from 1 skeleton}
Lu(x) = \max_{\substack{T \in F_d(\tau_h) \\ x \in T}} \Lip(u, X \cap T).
\end{equation}
\end{proposition}
\begin{proof}
Let $u_1$ be the output of Lemma \ref{convex KV theorem}.
By Theorem \ref{eikonal extension}, we can modify $u_1$ on each $d$-face $T$ of $\tau$ to obtain a map $u$ such that $Lu(x) = \Lip(u, T \cap X)$ for $x \in T^\circ$, but such that, if $Y$ denotes the $d - 1$-skeleton of $\tau$, then $u|_Y = u_1|_Y$.
In partiular $u$ is continuous across the $d - 1$-faces of $\tau$.
By Lemma \ref{partial lower semicontinuity}, the estimate (\ref{Lip constant comes from 1 skeleton}) holds because it does whenever $x \in T^\circ$ for some $m$-face $T$.
\end{proof}

\begin{definition}\label{shape map}
Let $h > 0$.
For every map $u_0: F_0(\tau_h) \to \RR^D$, choose a map $u: U \to \RR^D$ satisfying the conclusion of Proposition \ref{construction of shape maps}.
We call $u$ the $h$-\dfn{shape map} extending $u_0$.

Define a projection operator $\Pi_h$, by declaring that for every continuous map $v: U \to \RR^D$, $\Pi_h v$ is the unique $h$-shape map such that 
$$\Pi_h v|_{F_0(\tau_h)} = v|_{F_0(\tau_h)}.$$
\end{definition}

%%%%%%%%%%%%%%%
\subsection{Approximation of Lipschitz maps}
Let $u: U \to \RR^D$ be a Lipschitz map.
We are interested in the sense in which its projections $\Pi_h u$ approximate $u$ as $h \to 0$.
In particular, we must show that $\Pi_h$ preserves the tightness relation $\prec$.
Towards this end, we use (\ref{Lip constant comes from 1 skeleton}) to show that for every $x \in U$,
\begin{equation}\label{discrete derivative is vertex lip}
L \Pi_h u(x) = \max_{\substack{T \in F_d(\tau_h) \\ x \in T}} \Lip(u, T \cap F_0(\tau_h))
\end{equation}
We now use this estimate to show the convergence of $\Pi_h u$ to $u$.

\begin{lemma}\label{uniform convergence}
For every $x \in U$ and every Lipschitz map $u: U \to \RR^D$,
$$|\Pi_h(x) - u(x)| \leq 2h \Lip(u, B(x, h)).$$
\end{lemma}
\begin{proof}
Suppose that $T \in F_d(\tau_h)$ attains the maximum in (\ref{discrete derivative is vertex lip}).
Let $y$ be a vertex of $T$.
Then $\Pi_h u(y) = u(y)$, so 
$$|\Pi_h u(x) - u(x)| \leq |\Pi_h u(y) - u_h(x)| + |u(y) - u(x)|.$$
By (\ref{discrete derivative is vertex lip}) and the fact that $T \subseteq B(x, h)$,
\begin{align*}
|\Pi_h u(y) - u_h(x)| + |u(y) - u(x)| &\leq 2h \Lip(u, T) \leq 2h \Lip(u, B(x, h)). \qedhere
\end{align*}
\end{proof}

\begin{lemma}\label{du converges ae}
For any Lipschitz map $u: U \to \RR^D$, $L\Pi_h u \to Lu$ almost everywhere.
\end{lemma}
\begin{proof}
By Lemma \ref{uniform convergence}, $\Pi_h u \to u$ uniformly.
By (\ref{Lip constant comes from 1 skeleton}), $\Lip(\Pi_h u, U) \leq \Lip(u, U) < \infty$, so by Lemma \ref{Lip is lower semicontinuous},
$$Lu \leq \liminf_{h \to 0} L\Pi_hu$$
almost everywhere.
Thus it suffices to show that for almost every $x \in U$,
\begin{equation}\label{upper semicontinuity of discrete Lips}
\limsup_{h \to 0} L\Pi_h u(x) \leq Lu(x).
\end{equation}
By Lusin's theorem, for every $\varepsilon > 0$ there is a closed set $Z_\varepsilon \subset U$ with $\vol(Z_\varepsilon) < \varepsilon$ such that $Lu$ is continuous away from $Z_\varepsilon$.
It suffices to show (\ref{upper semicontinuity of discrete Lips}) for every $x \in U \setminus Z_\varepsilon$.

By (\ref{discrete derivative is vertex lip}) and (\ref{Lip is sup of local Lips}) for every $h > 0$ there exist $T_h \in F_d(\tau_h)$, $[y_h, z_h] \in F_1(\tau_h)$ and $w_h \in [y_h, z_h]$ such that $x \in T_h$, $[y_h, z_h] \subseteq T_h$, and
$$L \Pi_h u(x) \leq \frac{|u(y_h) - u(z_h)|}{|y_h - z_h|} \leq Lu(w_h).$$
If $h$ is small enough depending on $x$, then $T_h$ is contained in the open set $U \setminus Z_\varepsilon$, so 
\begin{align*}
\limsup_{h \to 0} L \Pi_h u(x) \leq \limsup_{h \to 0} Lu(w_h) &= Lu(x). \qedhere
\end{align*}
\end{proof}

\begin{lemma}\label{separation on a set of positive measure}
Suppose that $u, v: U \to \RR^D$ are Lipschitz maps with $u \prec v$.
Then there exists $\delta > 0$ such that
\begin{align}
&\sup_{Lv < Lu} Lu < \sup_{Lu < Lv} Lv - \delta \\
&\vol\left(\left\{y \in U: Lu(y) < \sup_{Lu < Lv} Lv - \delta < Lv(y)\right\}\right) > 0.
\end{align}
\end{lemma}
\begin{proof}
Let
$$M := \sup_{Lu < Lv} Lv.$$
Since $u \prec v$, there exists $\delta > 0$ such that
$$\sup_{Lv < Lu} Lu < M - \delta.$$
After shrinking $\delta$ if necessary, we may assume that there exists $x \in U$ such that
$$Lu(x) < M - \delta < Lv(x).$$
By Lemma \ref{partial lower semicontinuity}, and the fact that $\{Lu < M - \delta\}$ is open, the result follows.
\end{proof}

\begin{proposition}\label{discrete tight maps converge}
Let $u, v: U \to \RR^D$ be Lipschitz maps, and let $v_h: U \to \RR^D$ be $h$-shape maps such that:
\begin{enumerate}
\item $u \prec v$.
\item $v_h \to v$ in $L^1$.
\item For some $K < \infty$ and every $h > 0$, $\Lip(v_h, U) \leq K$.
\end{enumerate}
Then there exists $h_* > 0$ such that for $0 < h < h_*$, $\Pi_h u \prec v_h$.
\end{proposition}
\begin{proof}
Let $u_h := \Pi_h u$ and
$$M := \sup_{Lu < Lv} Lv.$$
By Lemmata \ref{Lip is lower semicontinuous}, \ref{du converges ae}, and \ref{separation on a set of positive measure}, there exist $\delta > 0$ and $y \in U$ such that:
\begin{enumerate}
\item $\sup_{Lu > Lv} Lu < M - \delta$,
\item $Lv(y) \leq \liminf_{h \to 0} Lv_h(y)$,
\item $Lu(y) = \lim_{h \to 0} Lu_h(y)$,
\item and $Lu(y) < M - \delta < Lv(y)$.
\end{enumerate}
In particular, for $h$ small enough, 
$$Lu_h(y) < M - \delta < Lv_h(y).$$
It follows that 
\begin{equation}\label{tight convergence RHS}
\sup_{Lu_h < Lv_h} Lv_h > M - \delta.
\end{equation}
We now need to prove the analogous bound for $\sup_{Lv_h < Lu_h} Lu_h$.

\begin{claim}\label{tight convergence exceptional set}
For every $\varepsilon > 0$ there exists an exceptional set $Z_\varepsilon \subseteq U$ such that:
\begin{enumerate}
\item uniformly on $U \setminus Z_\varepsilon$, $Lu = \lim_{h \to 0} Lu_h$,
\item uniformly on $U \setminus Z_\varepsilon$, $Lv \leq \liminf_{h \to 0} Lv_h$,
\item $\vol(Z_\varepsilon) < \varepsilon$, and
\item $U \setminus Z_\varepsilon$ is compact.
\end{enumerate}
\end{claim}
\begin{proof}[Proof of claim]
A set $Z_\varepsilon'$ satisfying the first three conditions follows from Proposition \ref{one sided Egorov} and Lemmata \ref{Lip is lower semicontinuous} and \ref{du converges ae}.
The first three conditions still hold if we replace $Z_\varepsilon'$ with $Z_\varepsilon' \cup W$ where $W$ is a collar neighborhood of $\partial U$ of sufficiently small volume.
By continuity of above of Lebesgue measure, we may replace $Z_\varepsilon' \cup W$ with a slightly larger set $Z_\varepsilon$ which is open.
In particular, $U \setminus Z_\varepsilon$ is closed and bounded.
\end{proof}

\begin{claim}\label{loss of epsilon}
For every $\theta > 0$ and $\varepsilon > 0$,
$$\limsup_{h \to 0} \sup_{\substack{Lv_h(x) < Lu_h(x) - \theta \\ x \notin Z_\varepsilon}} Lu_h(x) \leq M - \delta,$$
where $Z_\varepsilon$ is as in Claim \ref{tight convergence exceptional set}.
\end{claim}
\begin{proof}[Proof of claim]
If the claim is false, then then for every $h > 0$ along a subsequence, there exist $x_h \in U \setminus Z_\varepsilon$ such that $Lv_h(x_h) < Lu_h(x_h) - \theta$ but $Lu_h(x_h) > M - \delta$.
Since $U \setminus Z_\varepsilon$ is compact, after taking another subsequence, we can find $x \in U \setminus Z_\varepsilon$ with $x_h \to x$.
By the uniform convergence, $Lu(x) \geq M - \delta$ but 
$$Lv(x) \leq \liminf_{h \to 0} Lv_h(x_h) < \liminf_{h \to 0} Lu_h(x_h) - \theta = Lu(x) - \theta.$$
Therefore 
$$\sup_{Lu > Lv} Lu \geq M - \delta,$$
which contradicts the definition of $\delta$.
\end{proof}

\begin{claim}\label{LHS claim}
If $h$ is small enough, then for every $\theta > 0$,
$$\sup_{Lv_h < Lu_h - \theta} Lu_h \leq M - \delta.$$
\end{claim}
\begin{proof}[Proof of claim]
Let $Z_\varepsilon$ be as in Claim \ref{tight convergence exceptional set}.
Introduce the null set $Z := \bigcap_\varepsilon Z_\varepsilon$ and take $\varepsilon \to 0$ in Claim \ref{loss of epsilon} to deduce that for every sufficiently small $h > 0$,
\begin{equation}\label{loss of theta 2}
\sup_{\substack{Lv_h(x) < Lu_h(x) - \theta \\ x \notin Z}} Lu_h(x) \leq M - \delta.
\end{equation}
Let $x \in U$ satisfy $Lv_h(x) < Lu_h(x) - \theta$.
Then for every sufficiently small $\kappa > 0$,
$$Lv_h(x) < Lu_h(x) - \theta - \kappa.$$
Moreover, $V := \{Lv_h < Lu_h(x) - \theta - \kappa\}$ is an open neighborhood of $x$ since $Lv_h$ is upper semicontinuous.

The set $W := V \cap \{Lu_h > Lu_h(x) - \kappa\}$ has $\vol(W) > 0$ by Lemma \ref{partial lower semicontinuity}.
So $W \setminus Z$ is nonempty and for any $y \in W$,
$$Lv_h(y) < Lu_h(x) - \theta - \kappa < Lu_h(y) - \theta$$
which, combined with (\ref{loss of theta 2}), implies $Lu_h(y) \leq M - \delta$.
Therefore
$$Lu_h(x) < Lu_h(y) + \kappa \leq M - \delta + \kappa.$$
Taking $\kappa \to 0$, the result holds.
\end{proof}

Suppose that $h$ is small enough and $Lv_h(x) < Lu_h(x)$.
Then there exists $\theta > 0$ such that $Lv_h(x) < Lu_h(x) - \theta$, so by Claim \ref{LHS claim} and (\ref{tight convergence RHS}),
$$Lu_h(x) \leq M - \delta < \sup_{Lu_h < Lv_h} Lv_h.$$
Taking a supremum over all $x$, we deduce $u_h \prec v_h$.
\end{proof}

%%%%%%%%%%%%%%%%%%%%
\subsection{Proof of \texorpdfstring{Theorem \ref{existence of tight maps}}{the existence theorem}}
Let $X_h := F_0(\tau_h) \cap \partial U$.
We view $\tau_h$ as a simplicial complex with boundary $X_h$ and edge lengths $\len(E) := |x - y|$ if $E \in F_1(\tau_h)$ is an edge from $x$ to $y$.

Since $U$ is a polytope, $\tau_h$ induces a triangulation of $\partial U$.
Therefore it makes sense to ask if a map $g: \partial U \to \RR^D$ is an $h$-shape map.

Let $\overline f_h := f|_{X_h}$, which we extend to an $h$-shape map $f_h: \partial U \to \RR^D$.
By Proposition \ref{combinatorially tight existence} there exists a combinatorially tight map $\overline u_h: F_0(\tau_h) \to \RR^D$ extending $\overline f_h$.
We can extend $\overline u_h$ to an $h$-shape map $u_h: U \to \RR^D$.

Observe that if $v_h: U \to \RR^D$ is an $h$-shape map and $\overline v_h := v_h|_{F_0(\tau)}$, then for any $T \in F_d(\tau_h)$ and $x \in T$, $Lv_h(x) = S\overline v_h(T)$.
Since $\overline u_h$ is combinatorially tight, it follows that there does not exist an $h$-shape map $v_h$ extending $f_h$ such that $v_h \prec u_h$.

By the Kirszbraun-Valentine theorem, there exist extensions $\tilde w_h: U \to \RR^n$ of $f_h$ with $\Lip(\tilde w_h, U) = \Lip(f_h, \partial U)$.
By (\ref{discrete derivative is vertex lip}), $w_h := \Pi_h \tilde w_h$ satisfies
$$\Lip(w_h, U) = \Lip(f_h, \partial U) \leq \Lip(f, \partial U).$$
Since $f_h$ was already an $h$-shape map, the action of $\Pi_h$ on $\tilde w_h$ preserved its restriction to the boundary, so $w_h|_{\partial U} = f_h$.
It is false that $w_h \prec u_h$, so by the contrapositive of Lemma \ref{tight implies best lipschitz},
$$\Lip(u_h, U) \leq \Lip(w_h, U) \leq \Lip(f, \partial U).$$
So by the Arzela-Ascoli theorem, after passing to a subsequence, we may assume that $u_h \to u$ in $C^0$ for some Lipschitz map $u: U \to \RR^n$ with $\Lip(u, U) \leq \Lip(f, \partial U)$.
By Lemma \ref{uniform convergence}, $f_h \to f$ in $C^0$, so $u$ is an extension of $f$.

If $u$ is not tight, then there exists a Lipschitz map $v: U \to \RR^n$ with $v \prec u$.
By Proposition \ref{discrete tight maps converge}, there exists $h > 0$ such that $\Pi_h v \prec u_h$; this is a contradiction, and shows that $u$ is tight.

%%%%%%%%%%%%%%%%%%
\subsection{The scalar-valued case}
\begin{corollary}
A scalar field $u: U \to \RR$ is tight iff $\Delta_\infty u = 0$.
In particular, the tight map extending a boundary condition is unique.
\end{corollary}
\begin{proof}
\todo{A tight map is AML}, hence $\Delta_\infty u = 0$.
Conversely, if $\Delta_\infty u = 0$, use Theorem \ref{existence of tight maps} to construct a tight function $v$ such that $v|_{\partial U} = u|_{\partial U}$.
Then $\Delta_\infty v = 0$, so by \todo{Jensen's maximum principle}, $u = v$.
\end{proof}

%%%%%%%%%%%%%%%%%%
\section{The geodesic lamination stretched by an AML map}
\begin{definition}
Suppose that $U$ is convex.
Let $u: U \to \RR^D$ be a Lipschitz map.
The \dfn{stretch set} of $u$ is the set of all $x \in U$ such that $Lu(x) = \Lip(u, U)$.
\end{definition}

By upper semicontinuity of $Lu$, the stretch set of any Lipschitz map is closed.
However, since $U$ is not compact, it may be empty.

By the \dfn{harmonic map heat flow} of a continuous map $u: U \to \RR^D$ we mean the solution of the harmonic map heat flow with initial data $u$ and boundary data $u|_{\partial U}$.
Since the target is $\RR^D$, the harmonic map heat flow is a completely decoupled system of $D$ heat equations, so it has short-time existence.
Regularizing a definition using parabolic flows is typical in the geometric analysis literature, see for example \cite{Burkhardt-Guim2019}.
I expect that the below definition can be slightly modified to make sense even when there is a Riemannian metric, since we don't actually care about the global behavior of the harmonic map heat flow, but only its behavior near $x$, so that we may assume that the curvatures are not too ``crazy.''

\begin{definition}
Let $u: U \to \RR^D$ be a continuous map, which we extend to a harmonic map heat flow $[0, S) \times U \to \RR^D$.
We say that $u$ is \dfn{principally nonconformal} (in the sense of the harmonic map heat flow) if $D = 1$, or for every $x \in U$ such that $Lu(x) > 0$, there exists $\varepsilon > 0$ such that for every sufficiently small $s > 0$, 
$$\sigma_1(\dif u(s)(x)) \geq \sigma_2(\dif u(s)(x)) + \varepsilon.$$
\end{definition}

If $u$ is $C^1$ and principally nonconformal in the classical sense that for every $x \in U$,
$$\sigma_1(\dif u(x)) > \sigma_2(\dif u(x)),$$
then its harmonic map heat flow $u(s)$ converges in $C^1$ to $u$, implying that $u$ is principally nonconformal in the sense of the harmonic map heat flow.

\begin{theorem}
Suppose that $U$ is convex.
If a principally nonconformal map $u: U \to \RR^D$ is AML, and its stretch set $S$ is nonempty, then $S$ is the support of a geodesic lamination $\lambda_u$.
Furthermore, $u$ maps $\lambda_u$ linearly, rescaling by a factor of $\Lip(u, U)$, onto a geodesic lamination $\Lambda_u$ of $\RR^D$.
\end{theorem}

%%%%%%%%%%%%%%%%%%%%%%%%%%
\subsection{Constructing geodesics from a \texorpdfstring{$\infty$-subharmonic}{infinity-subharmonic} function}
\begin{lemma}[{\cite[Proposition 6.2]{Crandall2008}}]\label{infinity subharmonic gradient flow}
Let $v: U \to \RR$ be $\infty$-subharmonic and $x \in U$.
Then there exist $T \in (0, \infty]$ and a Lipschitz curve $\gamma: [0, T) \to U$ such that:
\begin{enumerate}
\item $\gamma(0) = x$.
\item For almost every $t \in [0, T)$, $|\gamma'(t)| \leq 1$.
\item Either $T = \infty$ or $\gamma(t) \to \partial U$ as $t \to T$.
\item For every $t \in [0, T)$, $Lv(\gamma(t)) \geq Lv(x)$.
\item For every $t \in [0, T)$, $v(\gamma(t)) \geq v(x) + tLv(x)$.
\item $v \circ \gamma$ is convex.
\end{enumerate}
\end{lemma}

\begin{lemma}\label{constructing the stretch geodesic}
Let $u: U \to \RR^D$ be an AML map and $x \in U$.
Then there exists $T \in (0, T]$ and a Lipschitz curve $\gamma: [0, T) \to U$ such that:
\begin{enumerate}
\item $\gamma(0) = x$.
\item For almost every $t \in [0, T)$, $|\gamma'(t)| \leq 1$.
\item Either $T = \infty$ or $\gamma(t) \to \partial U$ as $t \to T$.
\item For every $t \in [0, T)$, $L(|u - u(x)|)(\gamma(t)) \geq Lu(x)$.
\item For every $t \in [0, T)$, $|u(\gamma(t)) - u(x)| \geq tLu(x)$.
\item $t \mapsto |u(\gamma(t)) - u(x)|$ is convex.
\end{enumerate}
\end{lemma}
\begin{proof}
Since $u$ is AML,
$$v(y) := |u(y) - u(x)|$$
is $\infty$-subharmonic \cite[Proposition 7]{Naor2012}.\footnote{Strictly speaking, the reference shows that $v$ satisfies comparison with cones from above, but this is equivalent when the domain is euclidean \cite[\S2]{Crandall2008}.}
Let $\gamma: [0, T) \to U$ be the curve from Lemma \ref{infinity subharmonic gradient flow}.

We estimate 
\begin{align*} 
Lv(x) 
&= \inf_{r > 0} \sup_{y, z \in B(x, r)} \frac{||u(y) - u(x)| - |u(z) - u(x)||}{|y - z|} \\
&\geq \inf_{r > 0} \sup_{y \in B(x, r)} \frac{||u(y) - u(x)| - |u(x) - u(x)|}{|y - x|} \\
&= \inf_{r > 0} \sup_{y \in B(x, r)} \frac{|u(y) - u(x)|}{|y - x|} \\
&= Lu(x)
\end{align*}
\todo{justify the last equation in the preliminaries} which, combined with the definition of $\gamma$, implies 
\begin{align*}
L(|u - u(x)|)(\gamma(t)) = Lv(\gamma(t)) \geq Lv(x) &\geq Lu(x) \\
|u(\gamma(t)) - u(x)| = v(\gamma(t)) \geq v(x) + tLv(x) &= tLu(x). \qedhere
\end{align*}
\end{proof}

\begin{lemma}\label{geodesics still incomplete}
Let $u: U \to \RR^D$ be an AML map, $x \in U$, $K := \Lip(u, U)$, and suppose that $Lu(x) = K$.
Then there exists a geodesic $\gamma$ in $\RR^d$ and a geodesic $\Gamma$ in $\RR^D$, such that $u|_\gamma$ maps $\gamma$ onto $\Gamma$, and which admit parametrizations such that $\gamma(0) = x$ and
$$\Gamma(Kt) = u(\gamma(t)).$$
\end{lemma}
\begin{proof}
Let $\gamma$ be the curve from Lemma \ref{constructing the stretch geodesic}.
Then for $t > 0$,
$$K = Lu(x) \leq \frac{|u(\gamma(t)) - u(x)|}{t} \leq \frac{|u(\gamma(t)) - u(x)|}{|\gamma(t) - x|} \leq K.$$
The inequalities collapse and force $|\gamma(t) - x| = t$ and 
$$Lu(x) = \frac{|u(\gamma(t)) - u(x)|}{|\gamma(t) - x|}.$$
The former implies that $\gamma$ is a geodesic while the latter implies that $u$ is affine along $\gamma$.
\end{proof}

%%%%%%%%%%%%%%%%%
\subsection{Application of principal nonconformality}
\begin{lemma}\label{Lipschitz maximum principle}
Let $u: [0, S) \times U \to \RR^D$ be a harmonic map heat flow.
Then for every $s > 0$,
$$\sup L(u(s)) \leq \Lip(u, U).$$
\end{lemma}
\begin{proof}
For $\xi \in \Sph^{d - 1}$ and $\eta \in \Sph^{D - 1}$, let
$$v_{\xi, \eta}(s, x) = \langle \partial_x u(s, x), \xi \otimes \eta\rangle.$$
Then $\partial_s v_{\xi, \eta} = \Delta v_{\xi, \eta}$, and $L(u(s))(x) = \sup_{\xi, \eta} v_{\xi, \eta}(s, x)$.
Therefore, in the viscosity sense \cite[Lemma 4.2]{Crandall92},
$$\partial_s L(u(s)) \leq \Delta L(u(s)),$$
and the result follows by the parabolic maximum principle.
\end{proof}

\begin{lemma}
Let $u: U \to \RR^D$ be a principally nonconformal AML map, $x \in U$, $K := \Lip(u, U)$, and suppose that $Lu(x) = K$.
Then there exists a unique complete geodesic $\gamma$ in $\RR^d$ and a geodesic $\Gamma$ in $\RR^D$, such that $u|_\gamma$ maps $\gamma$ onto $\Gamma$, and which admit unit-speed parametrizations such that $\gamma(0) = x$ and
$$\Gamma(Kt) = u(\gamma(t)).$$
\end{lemma}
\begin{proof}
Let $u(s)$ be the harmonic map heat flow of $u$, and let $\varepsilon > 0$ be as in the definition of ``principally nonconformal.''
We first observe that if two unit-speed geodesics $\gamma_1, \gamma_2$ satisfy $\gamma_1(0) = \gamma_2(0) = x$ and are mapped to geodesics $\Gamma_1, \Gamma_2$ in $\RR^D$ by stretching by a factor of $K$, then the angle $\theta \in [0, \pi)$ between $\gamma_1, \gamma_2$ at $x$ must be $0$.
Indeed, if $\theta > 0$, then $\gamma_1'(0)$ and $\gamma_2'(0)$ span a $2$-plane $P$ at $x$.

Since $u(s) \to u$ uniformly as $s \to 0$, we can choose $s$ so small that $\|u(s) - u\|_{C^0} < \varepsilon/4$.
Then for every $t > 0$ small enough depending on the choice of $s$,
\begin{align*}
||u(s, \gamma_j(t)) - u(s, x)| - Kt| 
&= ||u(s, \gamma_j(t)) - u(s, x)| - |u(\gamma_j(t)) - u(x)|| \\
&\leq |u(s, \gamma_j(t)) - u(\gamma_j(t)) - u(s, x) + u(x)| \\
&\leq |u(s, \gamma_j(t)) - u(\gamma_j(t))| + |u(s, x) - u(x)| \\
&< \frac{\varepsilon}{2}.
\end{align*}
Taking $t \to 0$, we see that for any unit vector $\omega \in P$,
$$\langle \dif u(s)(x), \omega\rangle \geq K - \frac{\varepsilon}{2}.$$
On the other hand, by Lemma \ref{Lipschitz maximum principle},
$$\sigma_1(\dif u(s)(x)) \leq \Lip(u(s), U) \leq K.$$
Since $P$ is a $2$-plane, we see that
$$\sigma_2(\dif u(s)(x)) \geq K - \frac{\varepsilon}{2} \geq \sigma_1(\dif u(s)(x)) - \frac{\varepsilon}{2},$$
contradicting the definition of $\varepsilon$.

Now suppose that the claim is false.
It is not false because of a lack of uniqueness, by the above argument, so it is not possible to continue the geodesic $\gamma_0$ furnished by Lemma \ref{geodesics still incomplete} past $x$ while preserving the affineness of $u|_{\gamma_0}$.
Let $\gamma$ be the complete geodesic extending $\gamma_0$.

We claim that $v(y) := |u(y) - u(x)|$ satisfies $v(y) < K|y - x|$ for every $y$ not on $\gamma_0$.
In fact, $v(y) \leq K|y - x|$ by definition of $K$, and if there exists $y$ not in $\gamma_0$ with $v(y) < K|y - x|$, then $u$ must be affine along the geodesic segment $[x, y]$, but then $[x, y]$ must be a segment of $\gamma_0$ by our contradiction assumption and the uniqueness argument above, a contradiction.

\todo{Want to show that this violates tightness (rather than AML).
The idea is that we should be able to decrease $K$ on a small ball at $x$ while leaving $K$ fixed elsewhere.
This suggests that tight maps satisfy a DPP, in fact}
\end{proof}

\todo{This shows that the stretch set is a geodesic lamination (up to regularity issues).
Why is the image also a geodesic lamination? What if two nonintersecting curves intersect after the mapping?}


%%%%%%%%%%%%%%%%
\section{More stuff to do}
\begin{conjecture}
The same existence result, but with curvature.
\end{conjecture}

\begin{conjecture}
The same existence result holds when the datum is a homotopy class.
The point is that homotopy classes still make sense in the PL category, and we should be able to modify the lexicographic trick of \cite[Theorem 1.2]{Sheffield12} to work for homotopy classes.
\end{conjecture}

\begin{conjecture}
If $u_p$ is a sequence of $p$-harmonic maps, then $u_p \to u$ where $u$ is tight.
In particular, if $u$ is $\infty$-harmonic in the variational sense of \cite{daskalopoulos2022} and is smooth and nonholomorphic, then $u$ is $\infty$-harmonic in the classical sense.
\end{conjecture}

\begin{conjecture}
Tight maps are unique. This cannot be shown by a discretization alone (since $\prec$ is an open condition, so shouldn't be preserved by limits).
\end{conjecture}

\begin{conjecture}
Discrete tight maps have a ``discrete geodesic lamination'' in their nonholomorphic stretch set.
This lamination converges to a (continuum) lamination of $U$.
\end{conjecture}

\begin{conjecture}
If $u$ is tight and $C^2$, then $|\dif u|_\infty$ is a viscosity solution of a degenerate-elliptic PDE.
Therefore $C^2$ tight maps have no critical points, as in the scalar-valued case \cite{Yu2006}.
\end{conjecture}

\begin{conjecture}
Loisel's finite element approximation of the scalar $\infty$-Laplacian \cite{Loisel_2020} converges to a best Lipschitz map which, in general, is not $\infty$-harmonic.
\end{conjecture}

\printbibliography

\end{document}
