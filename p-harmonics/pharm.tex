\documentclass[reqno,12pt]{amsart}
\usepackage[letterpaper, margin=1in]{geometry}
\RequirePackage{amsmath,amssymb,amsthm,graphicx,mathrsfs,url,slashed,subcaption}
\RequirePackage[usenames,dvipsnames]{xcolor}
\RequirePackage[colorlinks=true,linkcolor=Red,citecolor=Green]{hyperref}
\RequirePackage{amsxtra}
\usepackage{cancel}
\usepackage{tikz-cd}

% \setlength{\textheight}{9.3in} \setlength{\oddsidemargin}{-0.25in}
% \setlength{\evensidemargin}{-0.25in} \setlength{\textwidth}{7in}
% \setlength{\topmargin}{-0.25in} \setlength{\headheight}{0.18in}
% \setlength{\marginparwidth}{1.0in}
% \setlength{\abovedisplayskip}{0.2in}
% \setlength{\belowdisplayskip}{0.2in}
% \setlength{\parskip}{0.05in}
%\renewcommand{\baselinestretch}{1.05}

\title{p-harmonic forms}
\author{Aidan Backus}
\date{July 2022}

\newcommand{\NN}{\mathbf{N}}
\newcommand{\ZZ}{\mathbf{Z}}
\newcommand{\QQ}{\mathbf{Q}}
\newcommand{\RR}{\mathbf{R}}
\newcommand{\CC}{\mathbf{C}}
\newcommand{\DD}{\mathbf{D}}
\newcommand{\PP}{\mathbf P}
\newcommand{\MM}{\mathbf M}
\newcommand{\II}{\mathbf I}
\newcommand{\Hyp}{\mathbf H}
\newcommand{\Sph}{\mathbf S}
\newcommand{\Group}{\mathbf G}
\newcommand{\GL}{\mathbf{GL}}
\newcommand{\Orth}{\mathbf{O}}
\newcommand{\SpOrth}{\mathbf{SO}}
\newcommand{\Ball}{\mathbf{B}}

\DeclareMathOperator*{\Expect}{\mathbf E}

\DeclareMathOperator{\avg}{avg}
\DeclareMathOperator{\card}{card}
\DeclareMathOperator{\codim}{codim}
\DeclareMathOperator{\diag}{diag}
\DeclareMathOperator{\diam}{diam}
\DeclareMathOperator{\Hom}{Hom}
\DeclareMathOperator{\id}{id}
\DeclareMathOperator{\rad}{rad}
\DeclareMathOperator{\rank}{rank}
\DeclareMathOperator{\Rm}{Rm}
\DeclareMathOperator{\Hess}{Hess}
\DeclareMathOperator{\Hol}{Hol}
\DeclareMathOperator{\sgn}{sgn}
\DeclareMathOperator{\supp}{supp}
\newcommand{\tr}{\operatorname{tr}}


\newcommand{\normal}{\mathbf n}

\newcommand{\Mink}{\mathbf m}
\newcommand{\Ric}{\mathrm{Ric}}
\newcommand{\Riem}{\mathrm{Riem}}
\newcommand*\dif{\mathop{}\!\mathrm{d}}
\newcommand*\Dif{\mathop{}\!\mathrm{D}}

\newcommand{\dbar}{\overline \partial}

\newcommand{\LineB}{\mathscr L}
\newcommand{\Two}{\mathrm{I\!I}}

\newcommand{\Lagrange}{\mathscr L}
\newcommand{\DirQL}{\mathscr D^{\mathrm{ql}}}
\newcommand{\DirL}{\mathscr D}

\newcommand{\vol}{\mathrm{vol}}

\newcommand{\dfn}[1]{\emph{#1}\index{#1}}

\renewcommand{\Re}{\operatorname{Re}}
\renewcommand{\Im}{\operatorname{Im}}

\newcommand{\loc}{\mathrm{loc}}
\newcommand{\cpt}{\mathrm{cpt}}

\def\Japan#1{\left \langle #1 \right \rangle}

\newtheorem{theorem}{Theorem}[section]
\newtheorem{badtheorem}[theorem]{``Theorem"}
\newtheorem{prop}[theorem]{Proposition}
\newtheorem{lemma}[theorem]{Lemma}
\newtheorem{sublemma}[theorem]{Sublemma}
\newtheorem{proposition}[theorem]{Proposition}
\newtheorem{corollary}[theorem]{Corollary}
\newtheorem{conjecture}[theorem]{Conjecture}
\newtheorem{axiom}[theorem]{Axiom}
\newtheorem{assumption}[theorem]{Assumption}

\newtheorem{mainthm}{Theorem}
\renewcommand{\themainthm}{\Alph{mainthm}}

\newtheorem{claim}{Claim}[theorem]
\renewcommand{\theclaim}{\thetheorem\Alph{claim}}

\theoremstyle{definition}
\newtheorem{definition}[theorem]{Definition}
\newtheorem{remark}[theorem]{Remark}
\newtheorem{example}[theorem]{Example}
\newtheorem{notation}[theorem]{Notation}

\newtheorem{exercise}[theorem]{Discussion topic}
\newtheorem{homework}[theorem]{Homework}
\newtheorem{problem}[theorem]{Problem}

\makeatletter
\newcommand{\proofpart}[2]{%
  \par
  \addvspace{\medskipamount}%
  \noindent\emph{Part #1: #2.}
}
\makeatother

\newtheorem{ack}{Acknowledgements}

\numberwithin{equation}{section}


% Mean
\def\Xint#1{\mathchoice
{\XXint\displaystyle\textstyle{#1}}%
{\XXint\textstyle\scriptstyle{#1}}%
{\XXint\scriptstyle\scriptscriptstyle{#1}}%
{\XXint\scriptscriptstyle\scriptscriptstyle{#1}}%
\!\int}
\def\XXint#1#2#3{{\setbox0=\hbox{$#1{#2#3}{\int}$ }
\vcenter{\hbox{$#2#3$ }}\kern-.6\wd0}}
\def\ddashint{\Xint=}
\def\dashint{\Xint-}

\usepackage[backend=bibtex,style=numeric]{biblatex}
\renewcommand*{\bibfont}{\normalfont\footnotesize}
\addbibresource{pharm.bib}
\renewbibmacro{in:}{}
\DeclareFieldFormat{pages}{#1}


\begin{document}
\begin{abstract}
\end{abstract}

\maketitle

%%%%%%%%%%%%%%%%%%%%%%%%%%%%%%%%%%%%%%%%%%%%%%%%%%%%%%%

% \tableofcontents

\section{Preliminaries}
\subsection{The $p$-Laplacian}
Let $1 < p < \infty$.

\begin{definition}
The $p$-\dfn{Dirichlet energy} is
$$\int_M \Lagrange_p(\dif u, \dif^* u) = \int_M \star |\dif u|^p + \star |\dif^* u|^p.$$
The Euler-Lagrange operator for the $p$-Dirichlet energy is called the $p$-\dfn{Laplacian}, $\Delta_p$, and an element of the kernel $\mathscr H_{p, \ell}(M)$ of $\Delta_p$ on $\ell$-forms is called $p$-\dfn{harmonic}.
We suppress the $p$ in case $p = 2$.
\end{definition}

\begin{proposition}
The $p$-Laplace equation is
\begin{equation}\label{EL}
\dif^* (|\dif u|^{p - 2} \dif u) + \dif(|\dif^* u|^{p - 2} \dif^* u) = 0.
\end{equation}
The second variation of the $p$-Dirichlet energy in the direction of a form $v$ is bounded from below by the sum of the two partial second variations
\begin{align*}
(p - 2) g^{-1}(\dif u, \dif v)^2 + |\dif u|^2 \cdot |\dif v|^2 &\geq 0, \\
(p - 2) g^{-1}(\dif^* u, \dif^* v)^2 + |\dif^* u|^2 \cdot |\dif^* v|^2 &\geq 0. 
\end{align*}
\end{proposition}
\begin{proof}
Let $v$ be a form and $u_t := u + tv$, thus
\begin{align*}
\frac{\dif}{\dif t} \int_M \Lagrange_p(\dif u_t, \dif^* u_t) &= \int_M \star \frac{\dif}{\dif t} |\dif u + t \dif v|^p + |\dif^* u + t \dif^* v|^p \\
&= p\int_M \star |\dif u + t \dif v|^{p - 2} g^{-1}(\dif u + t \dif v, \dif v) + |\dif^* u + t \dif^* v|^{p - 2} g^{-1}(\dif^* u, t \dif^* v, \dif^* v).
\end{align*}
Setting $t = 0$ we obtain the weak form 
$$\int_M \star g^{-1}(|\dif u|^{p - 2} \dif u, \dif v) + \star g^{-1}(|\dif^* u|^{p - 2} \dif^* u, \dif^* v) = 0$$
of the claimed Euler-Lagrange equation (\ref{EL}).
On the other hand, if we differentiate again in $t$,
\begin{align*}
\frac{\dif^2}{\dif t^2} \int_M \Lagrange_p(\dif u_t, \dif^* u_t) &= p \int_M \star \frac{\dif}{\dif t} \left[|\dif u_t|^{p - 2} g^{-1}(\dif u_t, \dif v) + |\dif^* u_t|^{p - 2} g^{-1}(\dif^* u_t, \dif^* v)\right]\\
&= p(p - 2) \int_M \star \left[|\dif u_t|^{p - 4} g^{-1}(\dif u_t, \dif v)^2 + |\dif^* u_t|^{p - 4} g^{-1}(\dif^* u_t, \dif^* v)^2\right] \\
&\qquad + p \int_M \star \left[|\dif u_t|^{p - 2} |\dif v|^2 + |\dif^* u_t|^{p - 2} |\dif^* v|^2\right].
\end{align*}
This is nonnegative for $t = 0$ provided that 
\begin{align*}
(p - 2) |\dif u|^{p - 4} g^{-1}(\dif u, \dif v)^2 + |\dif u|^{p - 2} |\dif v|^2  &\geq 0, \\
(p - 2) |\dif^* u|^{p - 4} g^{-1}(\dif^* u, \dif^* v) + |\dif^* u|^{p - 2} |\dif^* v|^2 &\geq 0.
\end{align*}
Dividing through by $|\dif u|^{p - 4}$ and $|\dif^* u|^{p - 4}$, we arrive at the claimed partial second variation.
Applying the Cauchy-Schwarz inequality, we see that 
\begin{align*}
(p - 2) g^{-1}(\dif u, \dif v)^2 + |\dif u|^2 \cdot |\dif v|^2 &\geq (p - 2) |\dif u|^2 \cdot |\dif v|^2 + |\dif u|^2 \cdot |\dif v|^2 \\
&= (p - 1) |\dif u|^2 \cdot |\dif v|^2 \geq 0.
\end{align*}
Similarly
$$(p - 2) g^{-1}(\dif^* u, \dif^* v)^2 + |\dif^* u|^2 \cdot |\dif^* v|^2 \geq (p - 1) |\dif^* u|^2 \cdot |\dif^* v|^2 \geq 0.$$
Therefore the partial second variations are nonnegative.
\end{proof}

\begin{example}
For $M \subseteq \RR^3$, we can identify $1$-forms and $2$-forms with vector fields, and then the $p$-Laplacian of $u$ is 
$$\Delta_p u = \nabla \times (|\nabla \times u|^{p - 2} \nabla \times u) - \nabla(|\nabla \cdot u|^{p - 2} \nabla \cdot u).$$
\end{example}

By \cite[\S3]{Scott95}, the Sobolev space $W^{1, p}(M, (T')^{\wedge \ell} M)$ is the space of $p$-differential forms with finite norm 
$$||u||_{W^{1, p}} := \int_M \star(|u|^p + |\dif u|^p + |\dif^* u|^p).$$
The fact that this space satisfies the usual Sobolev estimates follows from the Gaffney inequality \cite[Proposition 4.3]{Scott95}.
Actually, that paper only states this for $M$ closed and the sheaf of untwisted forms.
I guess it's fine though, as the proof relies mainly on the good local behavior of the Riesz transform, rather than anything global about the manifold.
We do need to worry about the possibility that Gaffney's inequality might degenerate in the limit $p \to 1$ though.

\subsection{Hodge theory}
We let
$$\mathscr K_\ell := \ker \dif \cap \ker \dif^*$$
be the sheaf of closed and coclosed $\ell$-forms.
Then for every $p$, we have the inclusion of sheaves
$$\mathscr K_\ell \subseteq \mathscr H_{2, \ell} \subset C^\infty(\cdot, (T')^{\wedge \ell} M) \subset W^{1, p}_\loc(\cdot, (T')^{\wedge \ell} M).$$
The converse inclusion $\mathscr H_{2, \ell} \subseteq \mathscr K_\ell$ need not hold.

\begin{proposition}
For $p > 1$, the $p$-Dirichlet energy is strictly convex modulo $\mathscr K_\ell$.
Therefore the $p$-Laplace equation has a unique solution in $W^{1, p}_\loc(M, (T')^{\wedge \ell} M)/\mathscr K_\ell$ for any Dirichlet or Neumann data.
For $p > d$, there exists $\alpha = \alpha(p, d) > 0$ such that any solution of the $p$-Laplace equation is an element of $C^\alpha_\loc(M, (T')^{\wedge \ell} M)$.
\end{proposition}
\begin{proof}
The convexity follows from the estimate on the second variation of the $p$-Dirichlet energy and also gives the existence and uniqueness.
Morrey's inequality gives $\alpha$.
\end{proof}

\begin{corollary}[de Rham--Hodge theorem]
If $M$ is closed then we have canonical isomorphisms
$$\mathscr H_{p, \ell}(M) = H^\ell(M, \CC).$$
\end{corollary}
\begin{proof}
By the classical de Rham--Hodge theorem,
$$H^\ell(M, \CC) = \mathscr H_{2, \ell}(M) = \mathscr K_\ell(M).$$
But elements of $\mathscr H_{p, \ell}(M)$ are uniquely defined up to addition of an element of $\mathscr K_\ell(M)$.
\end{proof}

\begin{conjecture}
For $p \gg 1$ and $u$ a $p$-harmonic, $u \in C^{1, \alpha}$.
Also if $\dif u, \dif^* u \neq 0$ then $u \in C^\infty$.
Finally, we can bound the ``size'' of the singular manifolds $\{\dif u = 0\}$ and $\{\dif^* u = 0\}$ in terms of cohomological data about $M$.
1\end{conjecture}

The motivation for this conjecture is the regularity of elliptic systems in \cite{Uhlenbeck77} and the fact that if $d = 2$ the cardinality of the singular manifold is given by the Euler characteristic of $M$ \cite{Daskalopolous20}.
In general, maybe the codimension is bounded from below and we can control the cohomology of the singular manifolds in terms of some characteristic classes of $M$.

%%%%%%%%%%%%%%%%%%%%%%%%
\section{Fenchel-Noether duality}
\begin{definition}
Let $u$ be a $p$-harmonic function. We define the \dfn{Noetherian flux} associated to $u$ by 
$$\psi := -\star |\dif u|^{p - 2} \dif u.$$
\end{definition}

From the definition, $\dif \psi = 0$.
It's a straightforward computation to see that $\dif \psi = 0$ is the conservation law for $p$-harmonic functions induced by the Lie algebra of symmetries $u \mapsto u + c$.
On the other hand, $\psi$ is a solution to the Fenchel dual problem to the $p$-Laplacian, namely the maximization of $-\|\psi\|_{L^q}$ where $1/p + 1/q = 1$ \cite[Chapter IV, Proposition 2.2]{Ekeland99}.
By \cite[Lemma 3.2]{Daskalopolous20}, $\psi$ solves the PDE 
$$\dif^*(|\psi|^{q - 2} \psi) = 0.$$
We introduce the \dfn{Noetherian potential}, a $d - 2$-form $\eta$ which solves $\dif \eta = \psi$, hence 
\begin{equation}\label{gauge invariant qLap}
\dif^*(|\dif \eta|^{q - 2} \dif \eta) = 0.
\end{equation}

The equation (\ref{gauge invariant qLap}) is like the $q$-Laplacian, except it is gauge invariant as it does not specify $\dif^* \eta$.

The dumb choice of gauge here is \dfn{Couloumb gauge} $\dif^* \eta = 0$. If it is exists, then Couloumb gauge implies that $\eta$ is a $q$-harmonic $d-2$-form.

\begin{definition}
The \dfn{gauge $\infty$-Laplace equation} is the PDE on $d-2$-forms
\begin{equation}\label{gauge invariant infLap}
  \langle \nabla (\dif \eta), \dif \eta \rangle \wedge \star \dif \eta = 0.
\end{equation}
\end{definition}

Here $\nabla (\dif \eta) \in \Omega^1 \otimes \Omega^{d - 1}$, and since $\dif \eta \in \Omega^{d - 1}$, their contraction is in $\Omega^1$.
Then $\star \dif \eta \in \Omega^1$, so the left-hand side of (\ref{gauge invariant infLap}) is a $2$-form.

The gauge $\infty$-Laplacian is the formal limit of (\ref{gauge invariant qLap}) as $q \to \infty$.
In fact 
\begin{align*}
  \dif^*(|\dif \eta|^{q - 2} \dif \eta)
  &= (-1)^{d - 1} \star^{-1} \dif(|\dif \eta|^{q - 2} \star \dif \eta) \\
  &= (-1)^{d - 1} \star^{-1} (\dif(|\dif \eta|^{q - 2}) \wedge \star \dif \eta + |\dif \eta|^{q - 2} \dif(\star \dif \eta)) \\
  &= (-1)^{d - 1} \star^{-1} (\dif(|\dif \eta|^{q - 2}) \wedge \star \dif \eta) + |\dif \eta|^{q - 2} \dif^* \dif \eta.
\end{align*}
Since $|\dif \eta|^{q - 2}$ is a scalar field, if $q \geq 4$ then
$$\dif(|\dif \eta|^{q - 2}) = \nabla(|\dif \eta|^{q - 2}) = (q - 2) |\dif \eta|^{q - 4} \langle \nabla(\dif \eta), \dif \eta\rangle.$$
We now consider a family of gauge $q$-harmonic $d-2$-forms $\eta_q$, where $q \in [4, \infty)$, normalized so that $|\dif \eta_q|^{q - 2}$ remains bounded. It solves 
$$|\dif \eta_q|^{-2} \langle \nabla(\dif \eta_q), \dif \eta_q\rangle \wedge \star \dif \eta_q + \frac{(-1)^{d - 1}}{q - 2} \star \dif^* \dif \eta_q = 0.$$
In the limit $\eta$ where $q \to \infty$, the second term drops out and we're left with 
$$|\dif \eta|^{-2} \langle \nabla(\dif \eta), \dif \eta\rangle \wedge \star \dif \eta = 0.$$
If we have a bound on the sets $\{|\eta_q| \ll 1\}$ where $\eta_q$ is not even elliptic up to gauge, then we can multiply by $|\dif \eta|^2$ and get (\ref{gauge invariant infLap}).
This is the same formal derivation of the scalar $\infty$-Laplacian that appears in \cite{Lindqvist14}.

Now suppose that $u$ is a $1$-harmonic function and $\eta$ is the Noetherian poential for $u$.
We can't really use variational solutions since (\ref{gauge invariant qLap}) is Euler-Lagrange for the nonstrictly convex functional $\|\dif \eta\|_{L^q}$.
Instead what follows probably needs to be justified using the Maz\'on--Rossi--Segura de L\'eon (MRS) notion of a weak solution. 

\begin{proposition}
The following are equivalent:
\begin{enumerate}
\item A function of least gradient.
\item An MRS solution of the $1$-Laplace equation.
\item An MRS solution $u$ of the $1$-Laplace equation such that $\dif u/|\dif u|$ is Lipschitz.
\end{enumerate}
\end{proposition}
\begin{proof}
By \cite[Theorem 2.5, (ii) implies (i)]{Mazon14}, $2 \implies 1$ (this is basically a one-liner).
Also $3 \implies 2$ is trivial.
By my stuff, $1 \implies 3$.
\end{proof}

This gives an alternate proof of (a stronger version of!) the MRS theorem \cite[Theorem 1.1(1)]{Mazon14} by completely different means. 

\begin{conjecture}
\begin{enumerate}
\item There is a good (game-theoretic?) notion of weak solution for (\ref{gauge invariant infLap}).
\item If $\eta_q \to \eta$ in $W^{1, q}$ and $\eta_q$ solves (\ref{gauge invariant qLap}), then $\eta$ is weakly gauge $\infty$-harmonic.
\item Every weakly gauge $\infty$-harmonic $d-2$-form $\eta$ which is the Noetherian potential of a measured minimal lamination $\lambda$ satisfies 
$$\supp \lambda = \{|\dif \eta| = \|\dif \eta\|_{L^\infty}\}.$$
\end{enumerate}
\end{conjecture}

If this conjecture is true, then so is even more. On $\supp \lambda$,
$$\dif \eta = \star \frac{\dif u}{|\dif u|} = \star \normal_\lambda$$
where $\dif u$ is the Ruelle-Sullivan current of $\lambda$. So $\dif \eta$ is a way of filling in $\normal_\lambda$ on the plaques.
Maybe choosing the gauge corresponds to choosing the plaques in a nice way?

\section{Fenchel-Noether duality, attempt II}
For simplicity, we assume in this section that $M$ is closed.
However, the problems that we are considering will probably be more interesting if we assume that $M$ is not even complete (so that we have both Dirichlet and homological data).
I don't think this materially affects the below computation.

\subsection{The primal problem}
Let $\tilde M \to M$ be the universal cover of $M$, and let $M_{\rm fun} \Subset \tilde M$ be a fundamental domain of $M$.
A map $u: M \to \Sph^1$ can be identified with a function $\tilde u: \tilde M \to \RR$ by requiring that the diagram
$$\begin{tikzcd}\tilde M \arrow[r, "\tilde u"] \arrow[d] & \RR \arrow[d] \\ M \arrow[r, "u"] & \Sph^1\end{tikzcd}$$
commutes.
The homotopy class $[u]$ is determined by a cohomology class $\xi \in H^1(M, \ZZ)$, namely 
$$\gamma^* \tilde u = \tilde u + \int_\gamma \xi$$
for every $\gamma \in \pi_1(M)$.
Note that if $\tilde u_1 - \tilde u_2$ is identically an integer, then they induce the same map $u$, but this will never be a serious issue.

We consider the problem of minimizing $\|\dif u\|_{L^p}^p/p$ subject to $[u] = \xi$, where $\xi \in H^1(M, \RR)$ is given.
To put this in the framework of \cite{Ekeland99}, we replace $u$ with $\tilde u + u_\xi$ where the former is in $W^{1, p}_0(M_{\rm fun})$ and is extended to all of $\tilde M$ by periodicity, and $u_\xi$ solves 
$$\begin{cases}
\dif u_\xi = \tilde \xi, \\
u_\xi(p) = 0
\end{cases}$$
where $p$ is the barycenter of $M_{\rm fun}$ (the choice of basepoint is unimportant here) and $\tilde \xi$ is a $\pi_1(M)$-invariant $1$-form on $\tilde M$ which drops to a representative of $\xi$ on $M$.

Then $[\tilde u + u_\xi] = [\tilde u] + [u_\xi]$, and since $\tilde u$ was construed to be invariant, we have $[\tilde u + u_\xi] = [u_\xi] = \xi$.
We are then interested in minimizing 
$$J(\tilde u) := G(\dif \tilde u)$$
subject to $\tilde u \in W^{1, p}_0(M_{\rm fun})$, where $G$ is the convex function on $L^p(M, \Omega^1)$,
$$G(\varphi) := \frac{1}{p} \int_{M_{\rm fun}} \star |\varphi + \xi|^p.$$

\begin{proposition}
Let $\tilde u$ be a minimizer of $J$.
The primal problem, solved by $u = \tilde u + u_\xi$, is
\begin{equation}\label{primal pLap}
\begin{cases}
\dif^*(|\dif u|^{p - 2} \dif u) = 0, \\
[u] = \xi.
\end{cases}
\end{equation}
\end{proposition}
\begin{proof}
We already proved $[u] = \xi$.
Assume that $\tilde u$ minimizes $J$, so
$$\frac{\dif}{\dif t} J(\tilde u_t)\bigg|_{t = 0} = 0$$
for all variations $(\tilde u_t)$ of $\tilde u$ with $\tilde u_t - \tilde u$ compactly supported.
That is,
$$0 = \frac{\dif}{\dif t} \frac{1}{p} \int_{M_{\rm fun}} \star |\dif \tilde u_t + \xi|^p\bigg|_{t = 0}.$$
We have 
$$\frac{1}{p} \frac{\partial}{\partial t} |\dif \tilde u_t + \xi|^p = |\dif \tilde u_t + \xi|^{p - 2} \left\langle \dif \tilde u_t + \xi, \frac{\partial}{\partial t} (\dif \tilde u_t + \xi)\right\rangle.$$
We then set $u_t := \tilde u_t + \xi$, and get
\begin{align*}
0
&= \int_{M_{\rm fun}} \star |\dif u|^{p - 2} \left\langle \dif u, \dif \frac{\partial u_t}{\partial t}\right\rangle\bigg|_{t = 0} \\
&= \int_{M_{\rm fun}} \star \dif^* (|\dif u|^{p - 2} \dif u) \frac{\partial u_t}{\partial t}\bigg|_{t = 0} + \int_{\partial M_{\rm fun}} |\dif u|^{p - 2} u \frac{\partial u_t}{\partial t}\bigg|_{t=0} \dif S.
\end{align*}
Now $\frac{\partial u_t}{\partial t} = \frac{\partial \tilde u_t}{\partial t}$ which is compactly supported, so the boundary term drops out. If we set
$$\omega := \star \frac{\partial u_t}{\partial t}\bigg|_{t = 0},$$
then $\omega$ is an arbitrary compactly supported volume form and
$$0 = \int_{M_{\rm fun}} \dif^* (|\dif u|^{p - 2} \dif u) \omega$$
which implies that 
\begin{align*}
\dif^* (|\dif u|^{p - 2} \dif u) &= 0. \qedhere
\end{align*}
\end{proof}

\subsection{The dual problem}
The dual space of $W^{1, p}_0(M_{\rm fun})$ is $W^{-1, q}(M_{\rm fun}, \Omega^d)$ where $(p, q)$ are a H\"older pair.
It is very important that we impose a Dirichlet condition to obtain that.
To compute the Legendre transform of $G$, we recall the dual space of $L^p(M, \Omega^1)$ is $L^q(M, \Omega^{d - 1})$.
Then $\hat G$ is the convex function on $L^q(M, \Omega^{d - 1})$,
$$\hat G(F) = \frac{1}{q} \int_M \star |F|^q - \int_M \xi \wedge F.$$
Here, we need to fix the convention that the dual pairing of $L^p(M, \Omega^1)$ and $L^q(M, \Omega^{d - 1})$ is 
$$L^p(M, \Omega^1) \times L^q(M, \Omega^{d - 1}) \ni (\alpha, \beta) \mapsto \int_M \alpha \wedge \beta.$$

\begin{lemma}
If $F$ is a minimizer of $\hat G$, then
\begin{equation}\label{EL of hat G}
|F|^{q - 2} F = (-1)^{d - 1} \star \xi.
\end{equation}
\end{lemma}
\begin{proof}
Let $(F_t)$ be an arbitrary variation and $\alpha := \frac{\partial F_t}{\partial t}|_{t = 0}$. Then 
$$0 = \frac{\dif}{\dif t} \hat G(F_t)\bigg|_{t = 0} = \int_M \star |F|^{q - 2} \langle F, \alpha \rangle - \xi \wedge \alpha.$$
Since $\alpha$ was arbitrary, for every $d-1$-form $\beta$,
$$|F|^{q - 2} \langle F, \beta\rangle = \star^{-1}(\xi \wedge \beta) = \langle \star^{-1} \xi, \beta\rangle.$$
Recalling that, since $\xi$ is a $1$-form, $\star^{-1} \xi = (-1)^{d - 1} \star \xi$, and $\beta$ is arbitrary, the claim follows.
\end{proof}

Now the extremality relation \cite[(III.4.23)]{Ekeland99} says that 
$$\frac{1}{p} \int_M \star |\dif u|^p + \frac{1}{q} \int_M \star |F|^q - \int_M \xi \wedge F = \int_M \dif \tilde u \wedge F$$
or in other words 
\begin{equation}\label{pre extremality relation}
\int_M \star |\dif u|^p + \frac{1}{q} \int_M |F|^q = \int_M \dif u \wedge F.
\end{equation}

\begin{lemma}
One has the extremality relation
\begin{equation}\label{extremality relation}
F = |\dif u|^{p - 2} \star \dif u.
\end{equation}
\end{lemma}
\begin{proof}
By convexity of $G$ and $\hat G$, for each $u$, (\ref{pre extremality relation}) has exactly one solution $F$.
Suppose that (\ref{extremality relation}); we therefore must show (\ref{pre extremality relation}).
In fact,
$$|F|^q = |\dif u|^{(p - 1)q} = |\dif u|$$
where we used the fact that $(p, q)$ are a H\"older pair. Also $\dif u \wedge F = \star |\dif u|^p$.
Therefore (\ref{pre extremality relation}) holds iff $1/p + 1/q = 1$, but this is the definition of a H\"older pair.
\end{proof}

By (\ref{extremality relation}) and a computation identical to \cite[Lemma 3.2]{Daskalopolous20},
\begin{equation}\label{first order qLap}
\dif^* (|F|^{q - 2} F) = 0,
\end{equation}
which can already be read off from (\ref{EL of hat G}) since $\xi \in H^1(M, \ZZ)$ is closed.

We now for simplicity assume that $d = 3$, so that $H^{d - 2}(\tilde M) = H^1(\tilde M) = 0$.
In particular, by the $p$-Laplace equation (\ref{primal pLap}), $\dif F = 0$, so we can view $F$ as the projection of $\dif A$ for some $1$-form $A$ on $\tilde M$.
In general we can view $A$ as a section of some sheaf of some kind of algebraic objects which don't have identity elements.

\begin{proposition}
Let $A$ be as above, corresponding to a solution $u$ of the primal problem.
The dual problem solved by $A$ is 
\begin{equation}\label{dual qLap}
\begin{cases}
\dif^*(|\dif A|^{q - 2} \dif A) = 0, \\
[A] = \eta
\end{cases}
\end{equation}
where $[A]$ means... and $\eta$ is a class in $H^{d - 2}(M, \RR)$ determined by $\xi$.
\end{proposition}
\begin{proof}
The PDE solved by $A$ is given by (\ref{first order qLap}). ... 


Recall that $F \in L^q(M, \Omega^2)$, so $F$ can be viewed as an invariant $2$-form $\tilde F$ on $\tilde M$.
Thus for every $\gamma \in \pi_1(M)$, $\gamma^* \tilde F = \tilde F$, and since $[\dif, \gamma^*] = 0$,
$$\dif(\gamma^* A - A) = 0,$$
hence there exists a function $\eta_\gamma$ on $\tilde M$ such that 
$$\gamma^* A = A + \dif \eta_\gamma.$$
A straightforward computation shows $\eta_{\delta \gamma} = \eta_\gamma + \eta_\delta$ and hence we have a homomorphism 
$$\eta: H_1(M) \to C^\infty(\tilde M).$$
We thus view $\eta$ as an element of $H^{d - 2}(M) \otimes C^\infty(\tilde M, \Omega^{d - 3})$, which we can identify with a class in $H^{d - 2}(M)$ that we denote by $[A]$.
\end{proof}

\subsection{The one-harmonic case}
Everything in this section is formal, but maybe it can be made precise.

Suppose that $u$ is a $1$-harmonic map and so by \cite{Mazon14} is a limit of $p$-harmonic maps $u_p$ as $p \to 1$.
The dual forms $A_q$ solve (\ref{dual qLap}), which we expand out as 
\begin{align*}
0
&= \dif(|\dif A_q|^{q - 2} \dif A_q) \\
&= (-1)^{d(d - 2) + 1} \star \dif(|\dif A_q|^{q - 2} \star \dif A_q) \\
&= (-1)^{d + 1} \star(\dif(|\dif A_q|^{q - 2}) \wedge \star \dif A_q) + |\dif A_q|^{q - 2} \dif^* \dif A_q.
\end{align*}
Now
$$\dif(|\dif A_q|^{q - 2}) = (q - 4) |\dif A_q|^{q - 4} \langle \nabla (\dif A_q), \dif A_q \rangle$$
which means that 
$$0 = |\dif A_q|^{q - 4} \langle \nabla \dif A_q, \dif A_q \rangle \wedge \star \dif A_q + \star \frac{1}{q - 4} |\dif A_q|^{q - 2} \dif^* \dif A_q.$$
If we make the formal assumption that 
$$|\dif A_q|^q \sim 1$$
as $p \to 1$, then 
$$\langle \nabla \dif A, \dif A\rangle \wedge \star \dif A = 0.$$
In other words, in the $d - 1$ directions which are cotangent to $\dif A$, $\dif(|\dif A|^2) = 0$; that is, in the directions cotangent to $\dif A$, $|\dif A|$ is constant.

Let $\lambda$ be the measured minimal lamination induced by $u$.
Then by (\ref{extremality relation}),
$$\dif A = \star \frac{\dif u}{|\dif u|} = \star \normal_\lambda$$
on $\supp \lambda$. 
So
\begin{enumerate}
\item $\dif A$ is the area form on each leaf of $\lambda$,
\item $\dif A$ is Lipschitz and tangentially $C^\infty$ on $\supp \lambda$.
\end{enumerate}
If (\ref{dual qLap}) has a unique solution up to gauge transformations $A \to A + \dif \chi$, it follows that $\dif A$ is a \emph{canonical} choice of extension of $\star \normal_\lambda$ to the plaques.
This addresses an issue with \cite{Mazon14} where they assert an extension but do not give any canonicity conditions.



\printbibliography

\end{document}
