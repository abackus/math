\documentclass[reqno,10pt]{amsart}
\usepackage[letterpaper, margin=1in]{geometry}
\RequirePackage{amsmath,amssymb,amsthm,graphicx,mathrsfs,url,slashed,subcaption}
\RequirePackage[usenames,dvipsnames]{xcolor}
\RequirePackage[colorlinks=true,linkcolor=Red,citecolor=Green]{hyperref}
\RequirePackage{amsxtra}
\usepackage{cancel}
\usepackage{tikz-cd}

% \setlength{\textheight}{9.3in} \setlength{\oddsidemargin}{-0.25in}
% \setlength{\evensidemargin}{-0.25in} \setlength{\textwidth}{7in}
% \setlength{\topmargin}{-0.25in} \setlength{\headheight}{0.18in}
% \setlength{\marginparwidth}{1.0in}
% \setlength{\abovedisplayskip}{0.2in}
% \setlength{\belowdisplayskip}{0.2in}
% \setlength{\parskip}{0.05in}
%\renewcommand{\baselinestretch}{1.05}

\title{p-harmonic forms}
\author{Aidan Backus}
\date{July 2022}

\newcommand{\NN}{\mathbf{N}}
\newcommand{\ZZ}{\mathbf{Z}}
\newcommand{\QQ}{\mathbf{Q}}
\newcommand{\RR}{\mathbf{R}}
\newcommand{\CC}{\mathbf{C}}
\newcommand{\DD}{\mathbf{D}}
\newcommand{\PP}{\mathbf P}
\newcommand{\MM}{\mathbf M}
\newcommand{\II}{\mathbf I}
\newcommand{\Hyp}{\mathbf H}
\newcommand{\Sph}{\mathbf S}
\newcommand{\Group}{\mathbf G}
\newcommand{\GL}{\mathbf{GL}}
\newcommand{\Orth}{\mathbf{O}}
\newcommand{\SpOrth}{\mathbf{SO}}
\newcommand{\Ball}{\mathbf{B}}

\DeclareMathOperator*{\Expect}{\mathbf E}

\DeclareMathOperator{\avg}{avg}
\DeclareMathOperator{\card}{card}
\DeclareMathOperator{\cent}{center}
\DeclareMathOperator{\ch}{ch}
\DeclareMathOperator{\codim}{codim}
\DeclareMathOperator{\Cyl}{Cyl}
\DeclareMathOperator{\diag}{diag}
\DeclareMathOperator{\diam}{diam}
\DeclareMathOperator{\dom}{dom}
\DeclareMathOperator{\Exc}{Exc}
\newcommand{\ext}{\mathrm{ext}}
\DeclareMathOperator{\Gal}{Gal}
\DeclareMathOperator{\Hom}{Hom}
\DeclareMathOperator{\Iso}{Iso}
\DeclareMathOperator{\Jac}{Jac}
\DeclareMathOperator{\Lip}{Lip}
\DeclareMathOperator{\Met}{Met}
\DeclareMathOperator{\id}{id}
\DeclareMathOperator{\rad}{rad}
\DeclareMathOperator{\rank}{rank}
\DeclareMathOperator{\Rm}{Rm}
\DeclareMathOperator{\Hess}{Hess}
\DeclareMathOperator{\Hol}{Hol}
\DeclareMathOperator{\Prop}{Prop}
\DeclareMathOperator{\Radon}{Radon}
\DeclareMathOperator*{\Res}{Res}
\DeclareMathOperator{\sgn}{sgn}
\DeclareMathOperator{\singsupp}{sing~supp}
\DeclareMathOperator{\Spec}{Spec}
\DeclareMathOperator{\supp}{supp}
\DeclareMathOperator{\Tan}{Tan}
\newcommand{\tr}{\operatorname{tr}}

\newcommand{\Mink}{\mathbf m}
\newcommand{\Ric}{\mathrm{Ric}}
\newcommand{\Riem}{\mathrm{Riem}}
\newcommand*\dif{\mathop{}\!\mathrm{d}}
\newcommand*\Dif{\mathop{}\!\mathrm{D}}
\newcommand{\LapQL}{\Delta^{\mathrm{ql}}}

\newcommand{\dbar}{\overline \partial}

\DeclareMathOperator{\atanh}{atanh}
\DeclareMathOperator{\csch}{csch}
\DeclareMathOperator{\sech}{sech}

\DeclareMathOperator{\Div}{div}
\DeclareMathOperator{\Gram}{Gram}
\DeclareMathOperator{\grad}{grad}
\DeclareMathOperator{\dist}{dist}
\DeclareMathOperator{\spn}{span}
\DeclareMathOperator{\Ell}{Ell}
\DeclareMathOperator{\WF}{WF}

\newcommand{\Two}{\mathrm{I\!I}}

\newcommand{\Lagrange}{\mathscr L}
\newcommand{\DirQL}{\mathscr D^{\mathrm{ql}}}
\newcommand{\DirL}{\mathscr D}

\newcommand{\Hilb}{\mathcal H}
\newcommand{\Homology}{\mathrm H}
\newcommand{\normal}{\mathbf n}
\newcommand{\radial}{\mathbf r}
\newcommand{\evect}{\mathbf e}
\newcommand{\vol}{\mathrm{vol}}

\newcommand{\Bmu}{\boldsymbol \mu}
\newcommand{\Bnu}{\boldsymbol \nu}
\newcommand{\Blambda}{\boldsymbol \lambda}

\newcommand{\pic}{\vspace{30mm}}
\newcommand{\dfn}[1]{\emph{#1}\index{#1}}

\renewcommand{\Re}{\operatorname{Re}}
\renewcommand{\Im}{\operatorname{Im}}

\newcommand{\loc}{\mathrm{loc}}
\newcommand{\cpt}{\mathrm{cpt}}

\def\Japan#1{\left \langle #1 \right \rangle}

\newtheorem{theorem}{Theorem}[section]
\newtheorem{badtheorem}[theorem]{``Theorem"}
\newtheorem{prop}[theorem]{Proposition}
\newtheorem{lemma}[theorem]{Lemma}
\newtheorem{sublemma}[theorem]{Sublemma}
\newtheorem{proposition}[theorem]{Proposition}
\newtheorem{corollary}[theorem]{Corollary}
\newtheorem{conjecture}[theorem]{Conjecture}
\newtheorem{axiom}[theorem]{Axiom}
\newtheorem{assumption}[theorem]{Assumption}

\newtheorem{mainthm}{Theorem}
\renewcommand{\themainthm}{\Alph{mainthm}}

\newtheorem{claim}{Claim}[theorem]
\renewcommand{\theclaim}{\thetheorem\Alph{claim}}

\theoremstyle{definition}
\newtheorem{definition}[theorem]{Definition}
\newtheorem{remark}[theorem]{Remark}
\newtheorem{example}[theorem]{Example}
\newtheorem{notation}[theorem]{Notation}

\newtheorem{exercise}[theorem]{Discussion topic}
\newtheorem{homework}[theorem]{Homework}
\newtheorem{problem}[theorem]{Problem}

\makeatletter
\newcommand{\proofpart}[2]{%
  \par
  \addvspace{\medskipamount}%
  \noindent\emph{Part #1: #2.}
}
\makeatother

\newtheorem{ack}{Acknowledgements}

\numberwithin{equation}{section}


% Mean
\def\Xint#1{\mathchoice
{\XXint\displaystyle\textstyle{#1}}%
{\XXint\textstyle\scriptstyle{#1}}%
{\XXint\scriptstyle\scriptscriptstyle{#1}}%
{\XXint\scriptscriptstyle\scriptscriptstyle{#1}}%
\!\int}
\def\XXint#1#2#3{{\setbox0=\hbox{$#1{#2#3}{\int}$ }
\vcenter{\hbox{$#2#3$ }}\kern-.6\wd0}}
\def\ddashint{\Xint=}
\def\dashint{\Xint-}

\usepackage[backend=bibtex,style=numeric]{biblatex}
\renewcommand*{\bibfont}{\normalfont\footnotesize}
\addbibresource{pharm.bib}
\renewbibmacro{in:}{}
\DeclareFieldFormat{pages}{#1}


\begin{document}
\begin{abstract}
\end{abstract}

\maketitle

%%%%%%%%%%%%%%%%%%%%%%%%%%%%%%%%%%%%%%%%%%%%%%%%%%%%%%%

% \tableofcontents

\section{The p-Laplacian}
Let $1 \leq p < \infty$.

\begin{definition}
The $p$-\dfn{Dirichlet energy} is
$$\int_M \Lagrange_p(\dif u, \dif^* u) = \int_M \star |\dif u|^p + \star |\dif^* u|^p.$$
The Euler-Lagrange operator for the $p$-Dirichlet energy is called the $p$-\dfn{Laplacian}, $\Delta_p$, and an element of the kernel $\mathscr H_{p, \ell}(M)$ of $\Delta_p$ on $\ell$-forms is called $p$-\dfn{harmonic}.
We suppress the $p$ in case $p = 2$.
\end{definition}

\begin{proposition}
The $p$-Laplace equation is
\begin{equation}\label{EL}
\dif^* (|\dif u|^{p - 2} \dif u) + \dif(|\dif^* u|^{p - 2} \dif^* u) = 0.
\end{equation}
The second variation of the $p$-Dirichlet energy in the direction of a form $v$ is bounded from below by the sum of the two partial second variations
\begin{align*}
(p - 2) g^{-1}(\dif u, \dif v)^2 + |\dif u|^2 \cdot |\dif v|^2 &\geq 0, \\
(p - 2) g^{-1}(\dif^* u, \dif^* v)^2 + |\dif^* u|^2 \cdot |\dif^* v|^2 &\geq 0. 
\end{align*}
\end{proposition}
\begin{proof}
Let $v$ be a form and $u_t := u + tv$, thus
\begin{align*}
\frac{\dif}{\dif t} \int_M \Lagrange_p(\dif u_t, \dif^* u_t) &= \int_M \star \frac{\dif}{\dif t} |\dif u + t \dif v|^p + |\dif^* u + t \dif^* v|^p \\
&= p\int_M \star |\dif u + t \dif v|^{p - 2} g^{-1}(\dif u + t \dif v, \dif v) + |\dif^* u + t \dif^* v|^{p - 2} g^{-1}(\dif^* u, t \dif^* v, \dif^* v).
\end{align*}
Setting $t = 0$ we obtain the weak form 
$$\int_M \star g^{-1}(|\dif u|^{p - 2} \dif u, \dif v) + \star g^{-1}(|\dif^* u|^{p - 2} \dif^* u, \dif^* v) = 0$$
of the claimed Euler-Lagrange equation (\ref{EL}).
On the other hand, if we differentiate again in $t$,
\begin{align*}
\frac{\dif^2}{\dif t^2} \int_M \Lagrange_p(\dif u_t, \dif^* u_t) &= p \int_M \star \frac{\dif}{\dif t} \left[|\dif u_t|^{p - 2} g^{-1}(\dif u_t, \dif v) + |\dif^* u_t|^{p - 2} g^{-1}(\dif^* u_t, \dif^* v)\right]\\
&= p(p - 2) \int_M \star \left[|\dif u_t|^{p - 4} g^{-1}(\dif u_t, \dif v)^2 + |\dif^* u_t|^{p - 4} g^{-1}(\dif^* u_t, \dif^* v)^2\right] \\
&\qquad + p \int_M \star \left[|\dif u_t|^{p - 2} |\dif v|^2 + |\dif^* u_t|^{p - 2} |\dif^* v|^2\right].
\end{align*}
This is nonnegative for $t = 0$ provided that 
\begin{align*}
(p - 2) |\dif u|^{p - 4} g^{-1}(\dif u, \dif v)^2 + |\dif u|^{p - 2} |\dif v|^2  &\geq 0, \\
(p - 2) |\dif^* u|^{p - 4} g^{-1}(\dif^* u, \dif^* v) + |\dif^* u|^{p - 2} |\dif^* v|^2 &\geq 0.
\end{align*}
Dividing through by $|\dif u|^{p - 4}$ and $|\dif^* u|^{p - 4}$, we arrive at the claimed partial second variation.
Applying the Cauchy-Schwarz inequality, we see that 
\begin{align*}
(p - 2) g^{-1}(\dif u, \dif v)^2 + |\dif u|^2 \cdot |\dif v|^2 &\geq (p - 2) |\dif u|^2 \cdot |\dif v|^2 + |\dif u|^2 \cdot |\dif v|^2 \\
&= (p - 1) |\dif u|^2 \cdot |\dif v|^2 \geq 0.
\end{align*}
Similarly
$$(p - 2) g^{-1}(\dif^* u, \dif^* v)^2 + |\dif^* u|^2 \cdot |\dif^* v|^2 \geq (p - 1) |\dif^* u|^2 \cdot |\dif^* v|^2 \geq 0.$$
Therefore the partial second variations are nonnegative.
\end{proof}

\begin{example}
For $M \subseteq \RR^3$, we can identify $1$-forms and $2$-forms with vector fields, and then the $p$-Laplacian of $u$ is 
$$\Delta_p u = \nabla \times (|\nabla \times u|^{p - 2} \nabla \times u) - \nabla(|\nabla \cdot u|^{p - 2} \nabla \cdot u).$$
\end{example}

By \cite[\S3]{Scott95}, the Sobolev space $W^{1, p}(M, (T')^{\wedge \ell} M)$ is the space of $p$-differential forms with finite norm 
$$||u||_{W^{1, p}} := \int_M \star(|u|^p + |\dif u|^p + |\dif^* u|^p).$$
The fact that this space satisfies the usual Sobolev estimates follows from the Gaffney inequality \cite[Proposition 4.3]{Scott95}.
Actually, that paper only states this for $M$ closed and the sheaf of untwisted forms.
I guess it's fine though, as the proof relies mainly on the good local behavior of the Riesz transform, rather than anything global about the manifold.

We let
$$\mathscr K_\ell := \ker \dif \cap \ker \dif^*$$
be the sheaf of closed and coclosed $\ell$-forms.
Then for every $p$, we have the inclusion of sheaves
$$\mathscr K_\ell \subseteq \mathscr H_{2, \ell} \subset C^\infty(\cdot, (T')^{\wedge \ell} M) \subset W^{1, p}_\loc(\cdot, (T')^{\wedge \ell} M).$$
The converse inclusion $\mathscr H_{2, \ell} \subseteq \mathscr K_\ell$ need not hold.

\begin{proposition}
For $p > 1$, the $p$-Dirichlet energy is strictly convex modulo $\mathscr K_\ell$.
Therefore the $p$-Laplace equation has a unique solution in $W^{1, p}_\loc(M, (T')^{\wedge \ell} M)/\mathscr K_\ell$.
For $p > d$, there exists $\alpha = \alpha(p, d) > 0$ such that any solution of the $p$-Laplace equation is an element of $C^\alpha_\loc(M, (T')^{\wedge \ell} M)$.
\end{proposition}
\begin{proof}
The convexity follows from the estimate on the second variation of the $p$-Dirichlet energy and also gives the existence and uniqueness.
Morrey's inequality gives $\alpha$.
\end{proof}

\begin{corollary}[de Rham--Hodge theorem]
If $M$ is closed then we have canonical isomorphisms for $p > 1$,
$$\mathscr H_{p, \ell}(M) = H^\ell(M, \CC).$$
\end{corollary}
\begin{proof}
By the classical de Rham--Hodge theorem,
$$H^\ell(M, \CC) = \mathscr H_{2, \ell}(M) = \mathscr K_\ell(M).$$
But elements of $\mathscr H_{p, \ell}(M)$ are uniquely defined up to addition of an element of $\mathscr K_\ell(M)$.
\end{proof}

\section{Harmonic duality}
\begin{definition}
Let $u \in \mathscr H_{p, \ell}(M)$ and let $p, q$ be H\"older conjugate.
We say that a $d-\ell$-form $v$ is a $q$-\dfn{harmonic conjugate} to $u$ if it solves the div-curl system
\begin{align*}
\dif \star v &= |\dif u|^{p - 2} \dif u, \\
\dif^* \star v &= |\dif^* u|^{p - 2} \dif^* u.
\end{align*}
\end{definition}

Clearly the $q$-harmonic conjugate of a $p$-harmonic $\ell$-form is well-defined modulo $\mathscr K_\ell(M)$.
In particular, for untwisted global forms on closed manifolds, this notion is completely boring, because then every $p$-harmonic $\ell$-form is conjugate to every $q$-harmonic $d-\ell$-form, by Poincar\'e duality.
We probably have to talk about twisted cohomology to make it not useless.

\begin{proposition}
If $u$ is $p$-harmonic and $v$ is a $q$-harmonic conjugate of $u$, then $v$ is $q$-harmonic.
\end{proposition}
\begin{proof}
Since $p, q$ are H\"older conjugate, $(p - 1)(q - 2) + p - 2 = 0$.
Therefore for some sign $\sigma$,
\begin{align*}
\dif^*(|\dif v|^{q - 2} \dif v) &= (-1)^\sigma \dif^*(|\dif^* \star v|^{q - 2} \dif^* \star v) \\
&= (-1)^\sigma \dif^* (|\dif^* u|^{(p - 1)(q - 2)} |\dif^* u|^{p - 2} \dif^* u) \\
&= (-1)^\sigma (\dif^*)^2 u = 0.
\end{align*}
A similar computation shows $\dif(|\dif^* v|^{q - 2} \dif^* v) = 0$.
It follows that $\Delta_q v = 0$.
\end{proof}

\section{Stuff to do}

Some natural questions to ask here:
\begin{enumerate}
\item Gaffney's inequality: Are the Sobolev estimates we're using actually valid?
\item Elliptic regularity: $u \in C^{1 + \alpha}_\loc$? What about $u \in C^\infty$ away from the critical points of $u$? See Uhlenbeck's ``Regularity for a class'' or Sung-Jin's div-curl appendix 
\item The best-Lipschitz limit: What happens to a sequence of $p$-harmonics $u_p$ which are uniformly $W^{1, \infty}_\loc$?
\item Twisted cohomology: Formulate all of this using a twisted complex, as in Yonah BW's paper, so that harmonic duality isn't a useless theory.
\item The least-gradient limit: What happens to the duals of a sequence of $p$-harmonics, $p \to \infty$, which have been normalized so the duals are uniformly $W^{1, 1}_\loc$?
\end{enumerate}




\printbibliography

\end{document}
