\documentclass[reqno,11pt]{amsart}
\usepackage[letterpaper, margin=1in]{geometry}
\RequirePackage{amsmath,amssymb,amsthm,graphicx,mathrsfs,url,slashed,subcaption}
\RequirePackage[usenames,dvipsnames]{xcolor}
\RequirePackage[colorlinks=true,linkcolor=Red,citecolor=Green]{hyperref}
\RequirePackage{amsxtra}
\usepackage{cancel}
\usepackage{tikz-cd}

% \setlength{\textheight}{9.3in} \setlength{\oddsidemargin}{-0.25in}
% \setlength{\evensidemargin}{-0.25in} \setlength{\textwidth}{7in}
% \setlength{\topmargin}{-0.25in} \setlength{\headheight}{0.18in}
% \setlength{\marginparwidth}{1.0in}
% \setlength{\abovedisplayskip}{0.2in}
% \setlength{\belowdisplayskip}{0.2in}
% \setlength{\parskip}{0.05in}
%\renewcommand{\baselinestretch}{1.05}

\title{Notes on convex duality}
\author{Aidan Backus}
\date{July 2022}

\newcommand{\NN}{\mathbf{N}}
\newcommand{\ZZ}{\mathbf{Z}}
\newcommand{\QQ}{\mathbf{Q}}
\newcommand{\RR}{\mathbf{R}}
\newcommand{\CC}{\mathbf{C}}
\newcommand{\DD}{\mathbf{D}}
\newcommand{\PP}{\mathbf P}
\newcommand{\MM}{\mathbf M}
\newcommand{\II}{\mathbf I}
\newcommand{\Hyp}{\mathbf H}
\newcommand{\Sph}{\mathbf S}
\newcommand{\Group}{\mathbf G}
\newcommand{\GL}{\mathbf{GL}}
\newcommand{\Orth}{\mathbf{O}}
\newcommand{\SpOrth}{\mathbf{SO}}
\newcommand{\Ball}{\mathbf{B}}

\DeclareMathOperator*{\Expect}{\mathbf E}

\DeclareMathOperator{\avg}{avg}
\DeclareMathOperator{\card}{card}
\DeclareMathOperator{\cent}{center}
\DeclareMathOperator{\ch}{ch}
\DeclareMathOperator{\codim}{codim}
\DeclareMathOperator{\Cyl}{Cyl}
\DeclareMathOperator{\diag}{diag}
\DeclareMathOperator{\diam}{diam}
\DeclareMathOperator{\dom}{dom}
\DeclareMathOperator{\Exc}{Exc}
\newcommand{\ext}{\mathrm{ext}}
\DeclareMathOperator{\Gal}{Gal}
\DeclareMathOperator{\Hom}{Hom}
\DeclareMathOperator{\Iso}{Iso}
\DeclareMathOperator{\Jac}{Jac}
\DeclareMathOperator{\Lip}{Lip}
\DeclareMathOperator{\Met}{Met}
\DeclareMathOperator{\id}{id}
\DeclareMathOperator{\rad}{rad}
\DeclareMathOperator{\rank}{rank}
\DeclareMathOperator{\Rm}{Rm}
\DeclareMathOperator{\Hess}{Hess}
\DeclareMathOperator{\Hol}{Hol}
\DeclareMathOperator{\Prop}{Prop}
\DeclareMathOperator{\Radon}{Radon}
\DeclareMathOperator*{\Res}{Res}
\DeclareMathOperator{\sgn}{sgn}
\DeclareMathOperator{\singsupp}{sing~supp}
\DeclareMathOperator{\Spec}{Spec}
\DeclareMathOperator{\supp}{supp}
\DeclareMathOperator{\Tan}{Tan}
\newcommand{\tr}{\operatorname{tr}}

\newcommand{\Mink}{\mathbf m}
\newcommand{\Ric}{\mathrm{Ric}}
\newcommand{\Riem}{\mathrm{Riem}}
\newcommand*\dif{\mathop{}\!\mathrm{d}}
\newcommand*\Dif{\mathop{}\!\mathrm{D}}
\newcommand{\LapQL}{\Delta^{\mathrm{ql}}}

\newcommand{\dbar}{\overline \partial}

\DeclareMathOperator{\atanh}{atanh}
\DeclareMathOperator{\csch}{csch}
\DeclareMathOperator{\sech}{sech}

\DeclareMathOperator{\Div}{div}
\DeclareMathOperator{\Gram}{Gram}
\DeclareMathOperator{\grad}{grad}
\DeclareMathOperator{\dist}{dist}
\DeclareMathOperator{\spn}{span}
\DeclareMathOperator{\Ell}{Ell}
\DeclareMathOperator{\WF}{WF}

\newcommand{\Two}{\mathrm{I\!I}}

\newcommand{\Lagrange}{\mathscr L}
\newcommand{\DirQL}{\mathscr D^{\mathrm{ql}}}
\newcommand{\DirL}{\mathscr D}

\newcommand{\Hilb}{\mathcal H}
\newcommand{\Homology}{\mathrm H}
\newcommand{\normal}{\mathbf n}
\newcommand{\radial}{\mathbf r}
\newcommand{\evect}{\mathbf e}
\newcommand{\vol}{\mathrm{vol}}

\newcommand{\Bmu}{\boldsymbol \mu}
\newcommand{\Bnu}{\boldsymbol \nu}
\newcommand{\Blambda}{\boldsymbol \lambda}

\newcommand{\pic}{\vspace{30mm}}
\newcommand{\dfn}[1]{\emph{#1}\index{#1}}

\renewcommand{\Re}{\operatorname{Re}}
\renewcommand{\Im}{\operatorname{Im}}

\newcommand{\loc}{\mathrm{loc}}
\newcommand{\cpt}{\mathrm{cpt}}

\def\Japan#1{\left \langle #1 \right \rangle}

\newtheorem{theorem}{Theorem}[section]
\newtheorem{badtheorem}[theorem]{``Theorem"}
\newtheorem{prop}[theorem]{Proposition}
\newtheorem{lemma}[theorem]{Lemma}
\newtheorem{sublemma}[theorem]{Sublemma}
\newtheorem{proposition}[theorem]{Proposition}
\newtheorem{corollary}[theorem]{Corollary}
\newtheorem{conjecture}[theorem]{Conjecture}
\newtheorem{axiom}[theorem]{Axiom}
\newtheorem{assumption}[theorem]{Assumption}

\newtheorem{mainthm}{Theorem}
\renewcommand{\themainthm}{\Alph{mainthm}}

\newtheorem{claim}{Claim}[theorem]
\renewcommand{\theclaim}{\thetheorem\Alph{claim}}

\theoremstyle{definition}
\newtheorem{definition}[theorem]{Definition}
\newtheorem{remark}[theorem]{Remark}
\newtheorem{example}[theorem]{Example}
\newtheorem{notation}[theorem]{Notation}

\newtheorem{exercise}[theorem]{Discussion topic}
\newtheorem{homework}[theorem]{Homework}
\newtheorem{problem}[theorem]{Problem}

\makeatletter
\newcommand{\proofpart}[2]{%
  \par
  \addvspace{\medskipamount}%
  \noindent\emph{Part #1: #2.}
}
\makeatother

\newtheorem{ack}{Acknowledgements}

\numberwithin{equation}{section}


% Mean
\def\Xint#1{\mathchoice
{\XXint\displaystyle\textstyle{#1}}%
{\XXint\textstyle\scriptstyle{#1}}%
{\XXint\scriptstyle\scriptscriptstyle{#1}}%
{\XXint\scriptscriptstyle\scriptscriptstyle{#1}}%
\!\int}
\def\XXint#1#2#3{{\setbox0=\hbox{$#1{#2#3}{\int}$ }
\vcenter{\hbox{$#2#3$ }}\kern-.6\wd0}}
\def\ddashint{\Xint=}
\def\dashint{\Xint-}

\usepackage[backend=bibtex,style=numeric]{biblatex}
\renewcommand*{\bibfont}{\normalfont\footnotesize}
\addbibresource{pharm.bib}
\renewbibmacro{in:}{}
\DeclareFieldFormat{pages}{#1}


\begin{document}

\maketitle

%%%%%%%%%%%%%%%%%%%%%%%%%%%%%%%%%%%%%%%%%%%%%%%%%%%%%%%

\tableofcontents

\section{Preliminaries}
\subsection{Locally convex spaces}
\begin{definition}
A Hausdorff real vector space $V$ is said to be \dfn{locally convex} if every neighborhood filter in $V$ is generated by convex sets.
\end{definition}

\begin{definition}
A hyperplane $H$ in a locally convex space $V$ is said to \dfn{separate} two sets $A, B$ if each of the closed half-spaces given by $H$ contains $A, B$.
\end{definition}

\begin{theorem}[Hanh-Banach]
Let $V$ be a locally convex space, and suppose that $A, B$ are disjoint convex nonempty subsets of $V$.
If $A$ is open, or $A$ is compact and $B$ is closed, then there exists a closed hyperplane which separates $A, B$.
\end{theorem}

Using the Hanh-Banach theorem we can show:

\begin{proposition}
The weak topology on $V$ is Hausdorff, and every closed convex subset of $V$ is weakly closed.
\end{proposition}

Thus we can think of locally convex spaces as the most general spaces which satisfy a good duality theory.

%%%%%%%%%%%%%%%%%%%%%%%%%%%

\subsection{Lower semicontinuous functions}
Throughout, let $V$ be a locally convex space.
We'll write $\RR' := \RR \cup \{+\infty\}$.
Recall that a function $f: V \to \RR'$ is \dfn{lower semicontinuous} (or \dfn{lsc}) if for every $x \in V$,
$$\lim_{y \to x} f(y) \geq f(x).$$
A function is lsc iff its epigraph is closed. 
The space of lsc functions is closed under supremum, so the following definition makes sense.

\begin{definition}
The \dfn{lsc regularization} of a function $f: V \to \RR'$ is the supremum of all lsc functions $\leq f$.
\end{definition}

Equivalently, the lsc regularization of $f$ is the function whose epigraph is the closure of the epigraph of $f$.
If $f$ is convex lsc and $V$ is Banach, then $f$ is actually continuous away from the boundary of $\dom f := \{f < +\infty\}$.
Thus we can think of convex lsc functions as functions that are finite continuous convex on an open set, and $+\infty$ on the opposing open set.

\begin{definition}
$\Gamma(V)$ is the space of convex lsc functions. $\Gamma_0(V)$ is $\Gamma(V) \setminus \{+\infty\}$.
\end{definition}

Equivalently, $\Gamma(V)$ is the space of suprema of sets of continuous affine functions.

\begin{definition}
The \dfn{convex lsc regularization} of a function $f: V \to \RR'$ is the supremum of all convex lsc functions $\leq f$.
\end{definition}

\begin{proposition}
Let $f,g: V \to \RR'$. The following are equivalent:
\begin{enumerate}
\item $g$ is the convex lsc regularization of $f$.
\item $g$ is the supremum of all continuous affine functions $\leq f$.
\item The epigraph of $g$ is the closed convex hull of the epigraph of $f$.
\end{enumerate}
\end{proposition}

%%%%%%%%%%%%%
\subsection{Legendre transform}
Let $V, V'$ be two vector spaces and the dot product $V \times V' \to \RR$ a bilinear function whose left and right kernels are both zero.
Then we can equip $V, V'$ with their respective weak topologies, which realize $V, V'$ as locally convex spaces.
For example, we can take $V = L^p$ and $V' = L^q$, $1/p + 1/q = 1$, $p > 1$, and then $V, V'$ are equipped with their weakstar topologies.
However, this gets a lot messier when $p = 1$ because $L^1$ is a nonreflexive space; we'll need to be very careful in that case.

\begin{definition}
Let $A$ be a subset of a locally convex space $V$, and $H$ a closed hyperplane which meets $A$, such that $A$ meets $H$ and $A$ is contained in a closed half-space given by $H$.
Then $H$ is a \dfn{supporting hyperplane} of $A$, and any point in $H \cap A$ is a \dfn{supporting point} of $A$.
\end{definition}

What are the supporting hyperplanes of the epigraph of a convex function $f: V \to \RR'$?
If $\xi \in V'$, then $\{\xi = \alpha\}$ is a closed hyperplane which lies under the graph of $f$ iff $\alpha \geq x \cdot \xi - f(x)$ for every $x \in V$.

\begin{definition}
The \dfn{Legendre transform}, \dfn{convex conjugate}, or \dfn{polar} of a lsc convex function $f: V \to \RR'$ is 
$$f'(\xi) := \sup_{x \in \dom f} x \cdot \xi - f(x).$$
\end{definition}

By definition, a polar is the supremum of continuous affine functions, so it is lsc convex.
The point is that for every $\xi \in V'$, $\{\xi = f'(\xi)\}$ is a supporting hyperplane of the epigraph of $f$, and there are no other.
This really is a conjugation, in the sense that $f'' = f$.
If we defined the Legendre transform of a general function $f: V \to \RR'$ then $f''$ would instead be the convex lsc regularization of $f$.

%%%%%%%%%%%%
\subsection{Convex optimization}
\begin{definition}
By a \dfn{convex optimization problem} we mean the problem of minimizing an lsc convex function $f: V \to \RR'$, where $V$ is a reflexive Banach space and $\dom f$ is closed convex.
\end{definition}

We think of the condition $\{f < +\infty\}$ as expressing a constraint on the minimizer.

\begin{definition}
A lsc convex function $f$ is \dfn{coercive} if
$$\lim_{x \to \infty} f(x) = +\infty.$$
\end{definition}

\begin{proposition}[direct method]
The convex optimization problem associated to a coercive lsc convex function always has a solution.
It has a unique solution if, in addition, $f$ is strictly convex.
\end{proposition}

\begin{theorem}
Let $A: V \to V'$ and $\varphi: V \to \RR'$ satisfy:
\begin{enumerate}
\item $A$ is weakly continuous on the finite-dimensional subspaces of $V$.
\item For any $x, y \in V$, $(Ax - Ay) \cdot (x - y) \geq 0$.
\item $\varphi$ is lsc convex.
\item There exists $x \in \dom \varphi$ such that 
$$\lim_{y \to \infty} \frac{Ay \cdot (y - x) + \varphi(y)}{|y|} = +\infty.$$
\end{enumerate}
Then for every $\xi \in V'$ there exists $x \in V$ such that for every $y \in V$,
$$(Ax - \xi) \cdot y + \varphi(y) \geq (Ax - \xi) \cdot x + \varphi(x).$$
\end{theorem}

This theorem is especially useful when $\varphi$ is something trivial: an indicator function, or identically $0$.
Then this reduces to the Lax-Milgram lemma and slight generalizations thereof.



\printbibliography

\end{document}
