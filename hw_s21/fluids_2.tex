
% --------------------------------------------------------------
% This is all preamble stuff that you don't have to worry about.
% Head down to where it says "Start here"
% --------------------------------------------------------------

\documentclass[10pt]{article}

\usepackage[margin=.7in]{geometry}
\usepackage{amsmath,amsthm,amssymb}
\usepackage{enumitem}
\usepackage{tikz-cd}
\usepackage{mathtools}
\usepackage{amsfonts}
\usepackage{listings}
\usepackage{algorithm2e}
\usepackage{verse,stmaryrd}
\usepackage{fancyvrb}

% Number systems
\newcommand{\NN}{\mathbb{N}}
\newcommand{\ZZ}{\mathbb{Z}}
\newcommand{\QQ}{\mathbb{Q}}
\newcommand{\RR}{\mathbb{R}}
\newcommand{\CC}{\mathbb{C}}
\newcommand{\PP}{\mathbb P}
\newcommand{\FF}{\mathbb F}
\newcommand{\DD}{\mathbb D}
\renewcommand{\epsilon}{\varepsilon}

\newcommand{\Aut}{\operatorname{Aut}}
\newcommand{\coker}{\operatorname{coker}}
\newcommand{\CVect}{\CC\operatorname{-Vect}}
\newcommand{\Cantor}{\mathcal{C}}
\newcommand{\D}{\mathcal{D}}
\newcommand{\card}{\operatorname{card}}
\newcommand{\dbar}{\overline \partial}
\DeclareMathOperator*{\esssup}{ess\,sup}
\newcommand{\GL}{\operatorname{GL}}
\newcommand{\Hom}{\operatorname{Hom}}
\newcommand{\id}{\operatorname{id}}
\newcommand{\Ind}{\operatorname{Ind}}
\newcommand{\Inn}{\operatorname{Inn}}
\newcommand{\interior}{\operatorname{int}}
\newcommand{\lcm}{\operatorname{lcm}}
\newcommand{\mesh}{\operatorname{mesh}}
\newcommand{\LL}{\mathcal L_0}
\newcommand{\Leb}{\mathcal{L}_{\text{loc}}^2}
\newcommand{\Lip}{\operatorname{Lip}}
\newcommand{\ppGL}{\operatorname{PGL}}
\newcommand{\ppic}{\vspace{35mm}}
\newcommand{\ppset}{\mathcal{P}}
\DeclareMathOperator{\proj}{proj}
\DeclareMathOperator*{\Res}{Res}
\newcommand{\Riem}{\mathcal{R}}
\newcommand{\RVect}{\RR\operatorname{-Vect}}
\newcommand{\Sch}{\mathcal{S}}
\newcommand{\SL}{\operatorname{SL}}
\newcommand{\sgn}{\operatorname{sgn}}
\newcommand{\spn}{\operatorname{span}}
\newcommand{\Spec}{\operatorname{Spec}}
\newcommand{\supp}{\operatorname{supp}}
\newcommand{\TT}{\mathcal T}
\DeclareMathOperator{\tr}{tr}

\DeclareMathOperator{\adj}{adj}
\DeclareMathOperator{\curl}{curl}

% Calculus of variations
\DeclareMathOperator{\pp}{\mathbf p}
\DeclareMathOperator{\zz}{\mathbf z}
\DeclareMathOperator{\uu}{\mathbf u}
\DeclareMathOperator{\vv}{\mathbf v}
\DeclareMathOperator{\ww}{\mathbf w}

% Categories
\newcommand{\Ab}{\mathbf{Ab}}
\newcommand{\Cat}{\mathbf{Cat}}
\newcommand{\Group}{\mathbf{Group}}
\newcommand{\Module}{\mathbf{Module}}
\newcommand{\Set}{\mathbf{Set}}
\DeclareMathOperator{\Fun}{Fun}
\DeclareMathOperator{\Iso}{Iso}

% Complex analysis
\renewcommand{\Re}{\operatorname{Re}}
\renewcommand{\Im}{\operatorname{Im}}

% Logic
\renewcommand{\iff}{\leftrightarrow}
\newcommand{\Henkin}{\operatorname{Henk}}
\newcommand{\PA}{\mathbf{PA}}
\DeclareMathOperator{\proves}{\vdash}

% Group
\DeclareMathOperator{\Gal}{Gal}
\DeclareMathOperator{\Fix}{Fix}
\DeclareMathOperator{\Out}{Out}

% Other symbols
\newcommand{\heart}{\ensuremath\heartsuit}

\DeclareMathOperator{\atanh}{atanh}

% Theorems
\theoremstyle{definition}
\newtheorem*{corollary}{Corollary}
\newtheorem*{falselemma}{Grader's ``Lemma"}
\newtheorem{exer}{Exercise}
\newtheorem{lemma}{Lemma}[exer]
\newtheorem{theorem}[lemma]{Theorem}


\usepackage[backend=bibtex,style=alphabetic,maxcitenames=50,maxnames=50]{biblatex}
\renewbibmacro{in:}{}
\DeclareFieldFormat{pages}{#1}

\begin{document}
\noindent
\large\textbf{Fluid dynamics, HW 2} \hfill \textbf{Aidan Backus} \\
% --------------------------------------------------------------
%                         Start here
% --------------------------------------------------------------\

\begin{exer}
Let $\psi$ be the stream function of a steady solution to the Euler equations with $d = 2$.
Show that if $W$ is a smooth function on $\RR$ then
$$u(x) = (-\partial_2 \psi(\tilde x), \partial_1 \psi(\tilde x), W(\psi(\tilde x)))$$
is a steady solution to the Euler equations with $d = 3$.
\end{exer}

Let $\omega = \curl u$. Then, since $\psi$ is a steady stream with $d = 2$, there is a smooth function $F: \RR \to \RR$ such that
$$
\omega(x) = (\partial_2(W \circ \psi)(\tilde x), -\partial_1(W \circ \psi)(\tilde x), F \circ \psi(\tilde x)).$$
Now
\begin{align*}\partial_1\partial_2(W \circ \psi)(\tilde x) &= \partial_2\partial_1(W \circ \psi)(\tilde x)\\
&= W''(\psi(\tilde x)) \partial_1 \psi(\tilde x) \partial_2\psi(\tilde x) + W'(\psi(\tilde x)) \partial_1\partial_2 \psi(\tilde x)
\end{align*}
while
$$\partial_j^2(W \circ \psi)(\tilde x) = W''(\psi(\tilde x))\partial_j\psi(\tilde x) + W'(\psi(\tilde x))\partial_j^2\psi(\tilde x)$$
so
\begin{align*}
u \cdot \nabla \omega(x) &= \begin{bmatrix}u^1\partial_1\partial_2(W \circ \psi)(\tilde x) + u^2\partial_2^2(W \circ \psi)(\tilde x)\\ -u^1\partial_1^2(W \circ \psi)(\tilde x) - u^2 \partial_1\partial_2(W \circ \psi)(\tilde x)\\ u^1\partial_1(F(\psi(\tilde x))) + u^2\partial_2(F(\psi(\tilde x))))\end{bmatrix}\\
&= \begin{bmatrix}
u^1(\partial_1 \psi(\tilde x) \partial_2\psi(\tilde x) + W'(\psi(\tilde x)) \partial_1\partial_2 \psi(\tilde x)) + u^2(W''(\psi(\tilde x))\partial_2\psi(\tilde x) + W'(\psi(\tilde x))\partial_2^2\psi(\tilde x))\\
-u^1(W''(\psi(\tilde x))\partial_1\psi(\tilde x) + W'(\psi(\tilde x))\partial_1^2\psi(\tilde x)) - u^2(\partial_1 \psi(\tilde x) \partial_2\psi(\tilde x) + W'(\psi(\tilde x)) \partial_1\partial_2 \psi(\tilde x))\\
u^1(F'(\psi(\tilde x))\partial_1\psi(\tilde x)) + u^2(F'(\psi(\tilde x))\partial_2\psi(\tilde x))
\end{bmatrix}\\
&= \begin{bmatrix}
-(\partial_2\psi)^2 \partial_1\psi  + \partial_1\psi \partial_2^2\psi W'(\psi) - \partial_2\psi \partial_1\partial_2 \psi W'(\psi) + \partial_1\psi \partial_2\psi W''(\psi)\\
(\partial_1\psi)^2 \partial_2\psi - \partial_1\psi \partial_1^2\psi W'(\psi) + \partial_1\psi \partial_1\partial_2\psi W'(\psi) -\partial_1\psi \partial_2\psi W''(\psi) \\
0
\end{bmatrix}\\
&= W'(\psi) \partial_2\psi\begin{bmatrix}
-\partial_1\partial_2\psi\\
\partial_2^2\psi\\
W'(\psi) \partial_1\psi
\end{bmatrix} - W'(\psi)\partial_1\psi\begin{bmatrix}
-\partial_2^2\psi\\
\partial_1\partial_2\psi\\
W'(\psi) \partial_2\psi
\end{bmatrix}\\
&= \omega \cdot \nabla u(x).
\end{align*}

\begin{exer}
Let $r,\theta,x_3$ denote cylindrical coordinates.
Assume $\partial_\theta u = u^\theta = 0$ and $\omega = \curl u$. Show that
$$\omega = (\partial_3 u^r - \partial_r u^3) e_\theta.$$
Show that $\xi = \omega^\theta/r$ satisfies the transport equation
\begin{equation}
\label{transport equation}
(\partial_t + u^r \partial_r + u^3\partial_3) \xi = 0.
\end{equation}
\end{exer}

We recall that in cylindrical coordinates,
$$\curl F = (r^{-1}\partial_\theta F^3 - \partial_3 F^\theta)e_r + (\partial_3 F^r - \partial_r F^r)e_\theta + r^{-1}(\partial_r(rF^\theta) - \partial_\theta F^r)e_3.$$
Plugging in $F = u$, all these terms vanish except $\partial_3 u^r e_\theta$ and $\partial_r u^3 e_\theta$, which was desired.

Now we appeal to the lemma in Madja and Bertozzi's book which says that if $h$ is a vector field with
$$(\partial_t + u \cdot \nabla)h = h \cdot \nabla u$$
and $f$ is a scalar field which is invariant under trajectories of $u$, then $\xi = \nabla f \cdot h$ solves the transport equation (\ref{transport equation}).
We can plug in $h = \omega$ as long as we can find an $f$ such that $(\partial_t + u \cdot \nabla)f = 0$ and $\partial_\theta f = r^{-1}$.
Since solving the transport equation (\ref{transport equation}) is a local property, we only need the decomposition $\xi = \nabla f \cdot \omega$ to be valid locally, i.e. we only need a local solution to this system of ODE.
We might as well take $f$ to not depend on $x_3$, in which case the constraints simplify to
$$\begin{cases}
\partial_tf + u^r \partial_r f = 0\\
r\partial_\theta f = 1.
\end{cases}$$
Since this system has a local solution by the Picard-Linde\"of theorem, the claim follows.

\begin{exer}
Let $u = \nabla \varphi$ be a potential solution of the Euler equations, $d = 3$. Show that
$$\nabla_x(\partial_t\varphi + |\nabla_x\varphi|^2/2 + p) = 0.$$
Assume that $u$ is a steady solution in a smooth bounded domain $\Omega$ with specified normal velocity $u \cdot n = V \cdot n$ on $\partial \Omega$.
Here $V$ does not depend on time.
Show that
\begin{equation}
\label{Neumann problem}
\begin{cases}
\Delta \varphi = 0\\
\partial_n \varphi = V \cdot n.
\end{cases}\end{equation}
Show that $u$ is the minimizer of the kinetic energy
$$E(v) = \frac{1}{2} \int_\Omega |v(x)|^2 ~dx$$
among all divergence-free vector fields $v$ satisfying $v \cdot n = V \cdot n$.
\end{exer}

We directly compute
\begin{align*}
\nabla(\partial_t\varphi + \nabla \varphi \cdot \nabla \varphi/2 + p) &= \partial_t u + \nabla(u \cdot u)/2 + \nabla p\\
&= \partial_t u + 2(u \cdot \nabla)u/2 + \nabla p = 0
\end{align*}
since $u$ solves the Euler equations with pressure $p$.

That $\varphi$ is a solution of the Neumann problem (\ref{Neumann problem}) follows immediately from the fact that $\Delta \varphi = \nabla \cdot u = 0$ and $\partial_n \varphi = n \cdot \nabla \varphi = u \cdot n = V \cdot n$.

Now let $v$ be a divergence-free vector field satisfying $v \cdot n = V \cdot n$.
We first claim
\begin{equation}
\label{u is v}
\int_U |u(x)|^2 ~dx = \int_U u(x) \cdot v(x) ~dx.
\end{equation}
To see this, we integrate by parts
$$\int_U \nabla \varphi \cdot (u - v) = \int_{\partial U} (u - v) \cdot (\varphi n) - \int_U \varphi\nabla \cdot(u - v) = 0$$
owing to the fact that $u - v$ is divergence-free and $(u - v)\cdot n = (V - V) \cdot n = 0$.
Moreover,
$$2\int_U u \cdot v \leq 2\int_U |u||v| \leq \int_U |u|^2 + |v|^2$$
by the Cauchy-Schwarz and Cauchy inequalities, so by (\ref{u is v}),
$$\int_U |u(x)|^2 ~dx \leq \frac{1}{2}\int_U |u(x)|^2 + |v(x)|^2 ~dx$$
or in other words,
$$\frac{1}{2} \int_U |u(x)|^2 ~dx \leq \frac{1}{2} \int_U |v(x)|^2 ~dx$$
as desired.




\begin{exer}
Let $d \geq 1$, $s \in \RR$, and $\alpha = s - d/2$. Prove that if $f \in H^s(\RR^d)$ and $0 < \alpha < 1$ then $f \in C^\alpha(\RR^d)$.
\end{exer}

The nonfractional version of this proof was taught to me by Maciej Zworski in his class on microlocal analysis at UC Berkeley.
The only new ideas needed for a fractional version are in the proof of the multiplier estimate (\ref{multiplier estimate}).
Since this version is specialized to $p = 2$ and the original was for arbitrary $p > 1$, you might be able to replace the Littlewood-Paley theory with some more $L^2$-specialized ideas, but I didn't think very hard about this.

We first recall the Littlewood-Paley decomposition from harmonic analysis.
There exist smooth, compactly supported, radial functions $\psi_0,\psi_1$ such that $\psi_0$ is supported on $B(0, 1)$, $\psi_1$ is supported on $B(0, 2) \setminus B(0, 1/2)$, and if $\psi_j(x) = \psi_1(2^{-j}x)$, $j \geq 2$, then the $\psi_j$, $j \geq 0$, form a partition of unity on $\RR^d$,
which we refer to as a dyadic partition of unity since $\psi_j$ is supported near the sphere $\partial B(0, 2^{j-1})$.
The existence of a smooth dyadic partition of unity allows us to introduce the Littlewood-Paley decomposition
\begin{equation}
\label{LP decomp}
u = \sum_{j=0}^\infty \psi_j(D)u
\end{equation}
where the Fourier multiplier $(2^{-j}D)^\ell \psi_j(D)$ by definition satisfies
$$(2^{-j}D)^\ell \psi_j(D)u = \frac{1}{(2\pi)^d} \int_{\RR^d} \hat u(\xi) \xi^\ell \psi_j(\xi) e^{ix\xi} ~d\xi$$
whenever $\ell \in \NN^d$ (so $\xi^\ell = \prod_{i \leq d} \xi_i^{\ell_i}$) and $j \in \NN$.
Thus $\psi_j(D)$ is a projection to frequencies near $2^{j-1}$.
Let $\eta_\ell(\xi) = \xi^\ell \psi_1(\xi)$, $|\ell| \leq 1$. Then $\eta_\ell(2^{-j}D) = (2^{-j}D)^\ell \psi_j$.

By the Cauchy-Schwarz inequality,
$$\eta_\ell(hD)u(x) = \frac{1}{(2\pi)^d} \int_{\RR^d} e^{ix\xi} \eta_\ell(h\xi) \hat u(\xi) ~d\xi \lesssim \left(\int_{\RR^d} \frac{|\eta_\ell(h\xi)|^2}{(1 + |\xi|^s)^2} ~d\xi\right)^{1/2} ||u||_{H^s}.$$
Writing $\zeta = h\xi$ we see that
$$\int_{\RR^d} \frac{|\eta_\ell(h\xi)|^2}{(1 + |\xi|^s)^2} ~d\xi = h^{-d} \int_{\RR^d} \frac{|\eta_\ell(\zeta)|^2}{(1 + h^{-s}|\zeta|^s)^2} ~d\zeta.$$
If $\eta_\ell(\xi) \neq 0$ then $|\xi| \sim 1$, so $(1 + h^{-s}|\xi|^s)^2 \sim h^{-2s}$. Therefore
$$\int_{\RR^d} \frac{|\eta_\ell(h\xi)|^2}{(1 + |\xi|^s)^2} ~d\xi \lesssim h^{2\alpha} ||\eta_\ell||_{L^2}^2,$$
and it is easy to see $||\eta_\ell||_{L^2} \lesssim ||\eta_0||_{L^2}$ uniformly in $\ell$, so one has the multiplier estimate
\begin{equation}
\label{multiplier estimate}
||\eta_\ell(hD)||_{H^s \to L^\infty} \lesssim h^\alpha
\end{equation}
uniformly in $0 < h < 1$ and $|\ell| \leq 1$.

Now fix $f \in H^s$.
Applying the Littlewood-Paley decomposition (\ref{LP decomp}) of $f$,
\begin{align*}
||f||_{L^\infty} &\leq ||\psi_0(D)f||_{L^\infty} + \sum_{j=0}^\infty ||\eta_0(2^{-j}D)f||_{L^\infty}\\
&\leq ||\psi_0(D)f||_{L^\infty} + \sum_{j=0}^\infty 2^{-j\alpha} ||f||_{H^s}.
\end{align*}
using the multiplier estimate (\ref{multiplier estimate}).
We recall that if $P \in L^2$ then $P(D)$ is a bounded operator $L^2 \to L^\infty$. Indeed, in that case
\begin{equation}
\label{L2 symbols are bounded}
|P(D)u(x)| = \left|\int_{\RR^d} \hat P(x - y) u(y) ~dy\right| \lesssim ||P||_{L^2} ||u||_{L^2}
\end{equation}
by the Cauchy-Schwarz inequality, thus $||P(D)||_{L^2 \to L^\infty} \lesssim ||P||_{L^2}$. Since $\psi_0(D)$ is Schwartz and $H^s \subseteq L^2$ it follows that $||\psi_0(D)||_{H^s \to L^\infty} \lesssim 1$.
Meanwhile, $\sum_j 2^{-j\alpha} \lesssim (2^\alpha - 1)^{-1} < \infty$ since $0 < \alpha < 1$, whence
\begin{equation}
\label{sup estimate}
||f||_{L^\infty} \lesssim_\alpha ||f||_{H^s}.
\end{equation}

It remains to bound the H\"older seminorm $[f]_\alpha$ of $f$.
We start by estimating $[\psi_0(D)f]_\alpha$. If $|x - y| \leq 1$ then $|x - y| \leq |x - y|^\alpha$, so in that case it suffices to show that $\psi_0(D)f$ is has finite Lipschitz seminorm $[\psi_0(D)f]_1$.
Indeed, by the Cauchy-Schwarz inequality,
$$[\psi_0(D)f]_1 \leq ||\nabla \psi_0(D) f||_{L^\infty} \leq \sup_x \left|\int_{\RR^d} \nabla_x \psi_0(x - y) \hat f(y) ~dy\right| \lesssim ||\nabla \psi_0||_{L^2} ||f||_{L^2} \lesssim ||f||_{H^s}$$
since $\psi_0$ is Schwartz.
In the other case $|x - y| > 1$ we instead have the trivial bound $1 \leq |x - y|^\alpha$ and
$$|\psi_0(D)f(x) - \psi_0(D)f(y)| \leq 2||\psi_0(D)f||_{L^\infty} \lesssim ||\psi_0||_{L^2} ||f||_{L^2} \lesssim ||f||_{H^s} |x - y|$$
owing to the estimate (\ref{L2 symbols are bounded}) applied to $P = \psi_0$.
Summing up, we have proven the very low frequency estimate
\begin{equation}
\label{very low frequency estimate}
[\psi_0(D)f]_\alpha \lesssim ||f||_{H^s}.
\end{equation}

Now we prove the low frequency estimate
\begin{equation}
\label{low frequency estimate}
|\psi_j(D)f(x) - \psi_j(D)f(y)| \lesssim |x - y| 2^{j(1 - \alpha)} ||f||_{H^s},
\end{equation}
valid provided that $1 \leq j \leq -\log_2 |x - y|$. To see (\ref{low frequency estimate}), we simply bound
\begin{align*}
|\psi_j(D)f(x) - \psi_j(D)f(y)| &\lesssim |\eta_0(2^{-j}D)f(x) - \eta_0(2^{-j}D)f(y)| \leq |x - y| \cdot ||\nabla \eta_0(2^{-j}D)f||_{L^\infty}\\
&= |x - y| \sum_{|\ell| = 1} ||D_\ell \eta_0(2^{-j}D)f||_{L^\infty} = |x - y| \sum_{|\ell| = 1} 2^j ||\eta_\ell(2^{-j}D)f||_{L^\infty}\\
&\lesssim |x - y| 2^{j(1 - \alpha)} ||f||_{H^s}
\end{align*}
owing to the multiplier estimate (\ref{multiplier estimate}).

Next we prove the high frequency estimate
\begin{equation}
\label{high frequency estimate}
|\psi_j(D)f(x) - \psi_j(D)f(y)| \lesssim 2^{-j\alpha} ||f||_{H^s}
\end{equation}
valid provided that $j > -\log_2 |x - y|$. Indeed,
$$|\psi_j(D)f(x) - \psi_j(D)f(y)| \leq 2||\psi_j(D)f||_{L^\infty} \lesssim h^\alpha ||f||_{H^s}$$
owing to the multiplier estimate (\ref{multiplier estimate}).

Now we concatenate the low frequency (\ref{low frequency estimate}) and high frequency (\ref{high frequency estimate}) estimates:
\begin{align*}
\sum_{j=1}^\infty |\psi_j(D)f(x) - \psi_j(D)f(y)| &= \left(\sum_{2^{-j} \geq |x - y|} + \sum_{2^{-j} < |x - y|}\right) |\psi_j(D)f(x) - \psi_j(D)f(y)|\\
&\lesssim ||f||_{H^s} \left(\sum_{2^j \leq |x - y|^{-1}} |x - y|2^{j(1 - \alpha)} + \sum_{2^j > |x - y|^{-1}} 2^{-j\alpha}\right)\\
&\leq ||f||_{H^s} \sum_{j=1}^\infty 2^{-j\alpha} \lesssim 2^{\alpha} ||f||_{H^s}.
\end{align*}
If $|x - y| \lesssim 1$ this gives the bound
$$|\psi_j(D)f(x) - \psi_j(D)f(y)| \lesssim |x - y|^{\alpha} ||f||_{H^s},$$
while when $|x - y| \gtrsim 1$, we instead have the bound
$$|\psi_j(D)f(x) - \psi_j(D)f(y)| \leq 2||\psi_j(D)f||_{L^\infty} \lesssim ||\psi_j||_{L^2} ||f||_{L^2} \lesssim |x - y|^{\alpha} ||f||_{H^s}$$
owing to the operator norm estimate (\ref{L2 symbols are bounded}).
Combining this estimate with the $L^\infty$ estimate (\ref{sup estimate}) and the very low frequency estimate (\ref{very low frequency estimate}), we deduce
$$||f||_{C^\alpha} = ||f||_{L^\infty} + [f]_\alpha \leq ||f||_{L^\infty} + \sum_{j=0}^\infty [\psi_j(D)f]_\alpha \lesssim ||f||_{H^s}$$
which was desired.

\begin{exer}
Let $s \leq \min(s_1, s_2)$ and $s_1 + s_2 > s + d/2$. Prove that
$$||fg||_{H^s} \lesssim ||f||_{H^{s_1}} ||g||_{H^{s_2}}.$$
\end{exer}

It will be convenient to write
$$||h||_{H^t} = ||((\cdot))^t h||_{L^2}$$
where $((\cdot))$ is the Japanese bracket, so $((\xi)) \sim 1$ if $|\xi| \lesssim 1$ and $((\xi)) \sim |\xi|$ if $|\xi| \gtrsim 1$.
From these estimates, we deduce
\begin{equation}
\label{Japanese}
((\xi))^s \lesssim_s ((\xi - \eta))^{s - s_1} + ((\eta))^{s_1} + ((\xi - \eta))^{s_2} ((\eta))^{s - s_2}
\end{equation}
provided $s \leq \min(s_1, s_2)$. Indeed, if $|\xi| \lesssim 1$ then the right-hand side is comparable to $((\eta))^s \geq 1$ and the claim follows.
If instead $|\xi| \gtrsim 1$, then $((\xi))^s \sim |\xi|^s$.
If $|\eta| \lesssim 1$ then the right-hand side is comparable to $|\xi|^{s_2} \geq |\xi|^s$ so the claim follows.
If $|\eta| \sim |\xi|$ then $|\xi - \eta| \lesssim 1$ and so the right-hand side is comparable to $|\eta|^{s_2} \gtrsim |\xi|^s$ and so the claim follows.
Finally, if $|\eta|$ dominates $|\xi|$ then the right-hand side is dominated by $((\eta))^s \gtrsim ((\xi))^s$ and the claim follows.

Since $\widehat{fg} = \hat f * \hat g$, the Japanese bracket estimate (\ref{Japanese}) gives
\begin{align*}||fg||_{H^s}^2 &= \int_{\RR^d} |\hat f * \hat g(\xi) ((\xi))^s|^2 ~d\xi\\
&= \int_{\RR^d} \left|\int_{\RR^d} \hat f(\eta) \hat g(\xi - \eta) ((\xi))^s ~d\eta\right|^2 ~d\xi\\
&\lesssim_s \int_{\RR^d} \bigg|\int_{\RR^d} \hat f(\eta) \hat g(\xi - \eta) ((\eta))^{s_1} ((\xi -\eta))^{s - s_1} ~d\eta \\
&\qquad+\int_{\RR^d} \hat f(\eta) \hat g(\xi - \eta) ((\xi - \eta))^{s_2} ((\eta))^{s - s_2} ~d\eta\bigg|^2 ~d\xi.
\end{align*}
By the $L^2$ triangle inequality,
\begin{align*}||fg||_{H^s} &\lesssim_s \bigg(\int_{\RR^d} \bigg|\int_{\RR^d} \hat f(\eta) \hat g(\xi - \eta) ((\eta))^{s_1} ((\xi -\eta))^{s - s_1} ~d\eta \\
&\qquad+\int_{\RR^d} \hat f(\eta) \hat g(\xi - \eta) ((\xi - \eta))^{s_2} ((\eta))^{s - s_2} ~d\eta\bigg|^2\bigg)^{1/2}\\
&\leq \int_{\RR^d} \left(\int_{\RR^d} |\hat f(\eta) \hat g(\xi - \eta) ((\eta))^{s_1} ((\xi - \eta))^{s - s_1}|^2 ~d\eta\right)^{1/2} ~d\xi\\
&\qquad+\int_{\RR^d} \left(\int_{\RR^d} |\hat f(\xi - \eta) \hat g(\eta) ((\eta))^{s_2} ((\xi - \eta))^{s - s_2}|^2 ~d\eta\right)^{1/2} ~d\xi\\
&= \int_{\RR^d} \left(\int_{\RR^d} |\hat f(\eta) \hat g(\xi) ((\eta))^{s_1} ((\xi))^{s - s_1}|^2 ~d\eta\right)^{1/2} ~d\xi\\
&\qquad+\int_{\RR^d} \left(\int_{\RR^d} |\hat f(\xi) \hat g(\eta) ((\eta))^{s_2} ((\xi))^{s - s_2}|^2 ~d\eta\right)^{1/2} ~d\xi\\
&\leq \left(\int_{\RR^d} |\hat f(\eta) ((\eta))^{s_1}|^2 ~d\eta\right)^{1/2} \int_{\RR^d}  |\hat g(\xi) ((\xi))^{s - s_1}| ~d\xi\\
&\qquad+ \left(\int_{\RR^d} |\hat g(\eta) ((\eta))^{s_2}|^2 ~d\eta\right)^{1/2} \int_{\RR^d} |\hat f(\xi) ((\xi))^{s - s_2}| ~d\xi\\
&= ||f||_{H^{s_1}} \int_{\RR^d} |\hat g(\xi) ((\xi))^{s - s_1}| ~d\xi + ||g||_{H^{s_2}} \int_{\RR^d} |\hat f(\xi) ((\xi))^{s - s_2}| ~d\xi.
\end{align*}
Let us show
\begin{equation}
\label{hat g Holder}
\int_{\RR^d} |\hat g(\xi) ((\xi))^{s - s_1}| ~d\xi \lesssim_s ||g||_{H^{s_2}};
\end{equation}
the analogous bound for $f$ will then follows by symmetry.
By the Cauchy-Schwarz inequality,
\begin{align*}
\int_{\RR^d} |\hat g(\xi) ((\xi))^{s - s_1}| ~d\xi &\leq ||((\cdot))^{s_2}\hat g||_{L^2} \left(\int_{\RR^d} ((\xi))^{2(s - s_1 - s_2)} ~d\xi\right)^{1/2}\\
&= ||g||_{H^{s_2}} ||((\cdot))^{s - s_1 - s_2}||_{L^2}.
\end{align*}
Since $s - s_1 - s_2 > d/2$, we conclude $((\cdot))^{s - s_1 - s_2} \in L^2$, which proves (\ref{hat g Holder}). Therefore
$$||fg||_{H^s} \lesssim_s ||f||_{H^{s_1}} ||g||_{H^{s_2}}$$
which was desired.


\end{document}
