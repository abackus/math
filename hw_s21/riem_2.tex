
% --------------------------------------------------------------
% This is all preamble stuff that you don't have to worry about.
% Head down to where it says "Start here"
% --------------------------------------------------------------

\documentclass[10pt]{article}

\usepackage[margin=.7in]{geometry}
\usepackage{amsmath,amsthm,amssymb,mathrsfs}
\usepackage{enumitem}
\usepackage{tikz-cd}
\usepackage{mathtools}
\usepackage{amsfonts}
\usepackage{listings}
\usepackage{algorithm2e}
\usepackage{verse,stmaryrd}
\usepackage{fancyvrb}

% Number systems
\newcommand{\NN}{\mathbb{N}}
\newcommand{\ZZ}{\mathbb{Z}}
\newcommand{\QQ}{\mathbb{Q}}
\newcommand{\RR}{\mathbb{R}}
\newcommand{\CC}{\mathbb{C}}
\newcommand{\PP}{\mathbb P}
\newcommand{\FF}{\mathbb F}
\newcommand{\DD}{\mathbb D}
\renewcommand{\epsilon}{\varepsilon}

\newcommand{\Aut}{\operatorname{Aut}}
\newcommand{\cl}{\operatorname{cl}}
\newcommand{\ch}{\operatorname{ch}}
\newcommand{\Con}{\operatorname{Con}}
\newcommand{\coker}{\operatorname{coker}}
\newcommand{\CVect}{\CC\operatorname{-Vect}}
\newcommand{\Cantor}{\mathcal{C}}
\newcommand{\D}{\mathcal{D}}
\newcommand{\card}{\operatorname{card}}
\newcommand{\dbar}{\overline \partial}
\newcommand{\diam}{\operatorname{diam}}
\newcommand{\dom}{\operatorname{dom}}
\newcommand{\End}{\operatorname{End}}
\DeclareMathOperator*{\esssup}{ess\,sup}
\newcommand{\GL}{\operatorname{GL}}
\newcommand{\Hom}{\operatorname{Hom}}
\newcommand{\id}{\operatorname{id}}
\newcommand{\Ind}{\operatorname{Ind}}
\newcommand{\Inn}{\operatorname{Inn}}
\newcommand{\interior}{\operatorname{int}}
\newcommand{\lcm}{\operatorname{lcm}}
\newcommand{\mesh}{\operatorname{mesh}}
\newcommand{\LL}{\mathcal L_0}
\newcommand{\Leb}{\mathcal{L}_{\text{loc}}^2}
\newcommand{\Lip}{\operatorname{Lip}}
\newcommand{\ppGL}{\operatorname{PGL}}
\newcommand{\ppic}{\vspace{35mm}}
\newcommand{\ppset}{\mathcal{P}}
\DeclareMathOperator{\proj}{proj}
\DeclareMathOperator*{\Res}{Res}
\newcommand{\Riem}{\mathcal{R}}
\newcommand{\RVect}{\RR\operatorname{-Vect}}
\newcommand{\Sch}{\mathcal{S}}
\newcommand{\SL}{\operatorname{SL}}
\newcommand{\sgn}{\operatorname{sgn}}
\newcommand{\spn}{\operatorname{span}}
\newcommand{\Spec}{\operatorname{Spec}}
\newcommand{\supp}{\operatorname{supp}}
\newcommand{\TT}{\mathcal T}
\DeclareMathOperator{\tr}{tr}

% Calculus of variations
\DeclareMathOperator{\pp}{\mathbf p}
\DeclareMathOperator{\zz}{\mathbf z}
\DeclareMathOperator{\uu}{\mathbf u}
\DeclareMathOperator{\vv}{\mathbf v}
\DeclareMathOperator{\ww}{\mathbf w}

% Categories
\newcommand{\Ab}{\mathbf{Ab}}
\newcommand{\Cat}{\mathbf{Cat}}
\newcommand{\Group}{\mathbf{Group}}
\newcommand{\Module}{\mathbf{Module}}
\newcommand{\Set}{\mathbf{Set}}
\DeclareMathOperator{\Fun}{Fun}
\DeclareMathOperator{\Iso}{Iso}

% Complex analysis
\renewcommand{\Re}{\operatorname{Re}}
\renewcommand{\Im}{\operatorname{Im}}

% Logic
\renewcommand{\iff}{\leftrightarrow}
\newcommand{\Henkin}{\operatorname{Henk}}
\newcommand{\PA}{\mathbf{PA}}
\DeclareMathOperator{\proves}{\vdash}

% Group
\DeclareMathOperator{\Gal}{Gal}
\DeclareMathOperator{\Fix}{Fix}
\DeclareMathOperator{\Out}{Out}

\newcommand{\Mero}{\mathscr M}
\newcommand{\Olo}{\mathscr O}

% Other symbols
\newcommand{\heart}{\ensuremath\heartsuit}

\DeclareMathOperator{\atanh}{atanh}

% Theorems
\theoremstyle{definition}
\newtheorem*{corollary}{Corollary}
\newtheorem*{falselemma}{Grader's ``Lemma"}
\newtheorem{exer}{Exercise}
\newtheorem{lemma}{Lemma}[exer]
\newtheorem{theorem}[lemma]{Theorem}


\usepackage[backend=bibtex,style=alphabetic,maxcitenames=50,maxnames=50]{biblatex}
\renewbibmacro{in:}{}
\DeclareFieldFormat{pages}{#1}

\begin{document}
\noindent
\large\textbf{Riemann surfaces, HW 2} \hfill \textbf{Aidan Backus} \\
% --------------------------------------------------------------
%                         Start here
% --------------------------------------------------------------\

\begin{exer}[Forster 6.2]
Let
$$\mathscr F = \Olo^*/\exp \Olo.$$
Show that $\mathscr F$ is a presheaf but not a sheaf.
\end{exer}

To show that $\mathscr F$ is a presheaf we just need to check that for every diagram of inclusion maps of open sets the dual diagram of restriction maps commutes; this holds since it is already true for $\Olo^*$.

To see that $\mathscr F$ is not a sheaf, we first show that if $U$ is a sufficiently small contractible open set and $p \in U$, then
$$\exp: \Olo(U \setminus p) \to \Olo^*(U \setminus p)$$
is not surjective. Indeed, if $U$ is sufficiently small then $U$ can be chosen to be in a chart and thus we may view as as an open subset of $\CC$, and by the classification of contractible open subsets of $\CC$ we see that we may replace $U$ with the disc $D(0, 1)$ and $p$ with $0$.
Then it remains to show that there is a nonzero function on $D(0, 1) \setminus 0$ which does not have a holomorphic logarithm, but this is obvious since this is already holds for $z$.
It follows that $\mathscr F$ is not the trivial presheaf, since there exists $U$ such that $\mathscr F(U)$ is nonzero.

On the other hand, the stalks of $\mathscr F$ are all zero.
Let $x \in X$ and suppose $f$ is a germ at $x$.
Then there is an open set $U$ such that $f$ extends to a residue class of holomorphic functions $U \to \CC$.
After replacing $U$ with a smaller open set if necessary, we may assume that $\pi_1(U) = 0$, so all the representatives of $f$ on $U$ have a holomorphic logarithm.
Therefore $f|U = 0$, so $f_x = 0$.

But we showed in class that if a presheaf $\mathscr G$ is separated (i.e. satisfies Axiom I of a sheaf) and is zero on every stalk, then $\mathscr G = 0$.
This is a contradiction if $\mathscr F$ is separated.

\begin{exer}[Forster 9.2]
Suppose $p: Y \to X$ is a holomorphic map, $p(b) = a$, and $k$ is the multiplicity of $p$ at $b$. Show that if $\omega$ is a holomorphic $1$-form on $X \setminus a$ then $\Res_b(p^*\omega) = k \Res_a \omega$.
\end{exer}

We can find coordinates in which $p(z) = z^k$ and $a = b = 0$, in which case the residue can be obtained by the classical residue theorem, applied to the loop $\gamma(\theta) = \varepsilon e^{i\theta}$ where $\varepsilon > 0$ is small, which has winding number $1$ at $0$. Then $p_*\gamma(\theta) = \varepsilon e^{ik\theta}$ has winding number $k$ at $0$, so by the residue theorem,
$$2\pi i \Res_b(p^*\omega) = \int_\gamma p^*\omega = \int_{p_*\gamma} \omega = k\int_\gamma \omega = 2\pi ik \Res_a \omega.$$

\begin{exer}[Forster 10.1]
Let $\omega$ be a holomorphic $1$-form on $X$.
Suppose that $\varphi$ is a primitive of $\omega$ in some open subset of $X$ and $(Y, p, f, b)$ is a maximal analytic continuation of $\varphi$.
Prove:
\begin{enumerate}
\item $p$ is a covering space.
\item $f$ is a primitive of $p^*\omega$.
\item The covering space $p$ is an abelian Galois cover.
\end{enumerate}
\end{exer}

Since maximality is a universal property, the maximal analytic continuation is unique, so $Y$ is a connected component of the the \'etal\'e space $|\Olo_X|$ and $p$ is the restriction of the natural covering space $|\Olo_X| \to X$ to a connected component of $|\Olo_X|$, hence is also a covering space.
Moreover, since $d$ commutes with $p^*$ and by definition of an analytic continuation covering space,
$$p^*\varphi = (p_*)^{-1}\varphi = f$$
and thus
$$df = dp^*\varphi = p^*d\varphi = p^*\omega$$
so $f$ is a primitive of $p^*\omega$.

Let $\tilde X$ be the universal cover of $X$, so the monodromy action of $\pi_1(X)$ on $\tilde X$ is transitive.
Let $\psi: \pi_1(X) \to \CC$ be the period morphism of $\omega$, and let $K$ be the kernel of $\psi$.
Then, by the Galois correspondence of covering spaces, $\tilde X/K$ is a Galois cover of $X$ with deck group $\pi_1(X)/K$.
Since $\CC$ is abelian, then, it follows that so is $\pi_1(X)/K$, so $\tilde X/K$ is an abelian cover of $X$, say with respect to a map
$$q: \tilde X/K \to \CC.$$
We claim that in fact $\tilde X/K$ is the domain of a maximal analytic continuation of $\varphi$, as witnessed by $q$.

To see this, recall that since $\pi_1(X)/K$ is the deck group of $\tilde X/K$,
$$q^*|K: K \to \pi_1(\tilde X/K)$$
is an isomorphism. Thus if $[\gamma] \in \pi_1(\tilde X/K)$,
$$\int_\gamma q^*\omega = \int_{q_*\gamma} \omega = \psi(q_*[\gamma]) = 0,$$
so $q^*\omega$ is an exact $1$-form, and hence has a primitive $g$.
Then $dq_*g = q_*dg = q_*q^*\omega = \omega$, so $q_*g$ is a primitive of $\omega$.
Since primitives are unique up to a choice of constant, possibly after adding a constant to $g$, we conclude $q_*g = \varphi$.
Therefore $\tilde X/K$ is the domain of an analytic continuation of $\varphi$.

Now we prove maximality of $\tilde X/K$. If $x \in X$, let $q(y_1) = q(y_2) = x$, and suppose that $q_*g_{y_1} = q_*g_{y_2}$.
Then there is a $\gamma: x \to x$ which lifts to a path $q^*\gamma: y_1 \to y_2$.
But $\omega$ has a primitive, so $\psi(\gamma) = 0$, and hence $q^*\gamma$ is a loop.
Therefore $y_1 = y_2$. By Exercise 7.1 in Forster it follows that $\tilde X/K$ is maximal.

So by uniqueness again, $\tilde X/K = Y$ with $f = g$, $q = p$. But $\tilde X/K$ was an abelian Galois cover of $X$, so the same holds for $Y$.

\begin{exer}[Forster 10.2]
Let $X$ be an elliptic curve. Given a morphism $a: \pi_1(X) \to \CC$ show there is a closed $1$-form on $X$ whose period morphism is $a$.
\end{exer}

Since $X$ is homeomorphic to $(\RR/\ZZ)^2$, $\pi_1(X)$ has a basis $(\gamma_1,\gamma_2)$ where $\gamma_j$ is a loop around the $j$th factor of $(\RR/\ZZ)^2$.
We can choose the homeomorphism $X \to (\RR/\ZZ)^2$ so that $\gamma_j$ lifts to a path in the universal cover $\CC$, $0 \to \alpha_j$, where $\alpha_1\ZZ + \alpha_2\ZZ$ is the lattice determined by $X$.
Then $\Hom(\pi_1(X), \CC)$ is isomorphic to $\CC^2$ according to the map $a \mapsto (a(\gamma_1), a(\gamma_2))$.
So it suffices to find, for every $z \in \CC^2$, a closed $1$-form $\omega$ such that
$$\int_{\gamma_i} \omega = z_i.$$
To do this, we let $F: \CC \to \CC$ be the real-linear map $F(w) = z_1w_1 + iz_2w_2$ where $w = w_1\alpha_1 + w_2\alpha_2$. Then if $n \in \alpha_1\ZZ + \alpha_2\ZZ$, $F(w + n) = F(w) + F(n)$.
But the action of $\alpha_1\ZZ + \alpha_2\ZZ$ on $\CC$ is the deck group of the universal cover $\CC \to X$, so the constants $F(n)$ are summands of automorphy of $F$.
Therefore there is a unique closed $1$-form $\omega$ on $X$ with $dF = p^*\omega$.
But then
$$\int_{\gamma_i} \omega = \int_{p^*\gamma_i} p^*\omega = F(p^*\gamma_i(1)) - F(p^*\gamma_i(0)) = F(\alpha_i) - F(0) = F(\alpha_i) = z_i.$$
This was desired.

\begin{exer}[Forster 10.3]
Let $\omega$ be a meromorphic $1$-form on $X$ which has zero residue at every pole. Show that there is a covering space $p: Y \to X$ and meromorphic function $F$ on $Y$ such that $dF = p^*\omega$.
\end{exer}

Let $\psi: \pi_1(X) \to \CC$ be the period morphism of $\omega$.
Since $\omega$ is not holomorphic we must show that $\psi$ is well-defined.
Let $[\gamma] = [\eta] \in \pi_1(X)$. Then $\gamma - \eta$ is nullhomotopic; choose an ``interior" for $\gamma - \eta$ and let $p_1, \dots, p_n$ be the poles of $\omega$ in the interior. Then
$$\int_{\gamma - \eta} \omega = 2\pi i\sum_{j=1}^n \Res_{p_i} \omega = 0$$
by hypothesis. Therefore $\psi([\gamma]) = \psi([\eta])$, so the definition of $\psi([\gamma])$ does not require one to choose a representative $\gamma$.

Let $K$ be the kernel of $\psi$ and $\tilde X$ the universal cover of $X$.
As in Exercise 10.1 in Forster, we let $Y = \tilde X/K$, so that we obtain a Galois cover $p: Y \to X$ such that $p^*$ restricts to an isomorphism $K \to \pi_1(Y)$.
Now fix $y_0 \in Y$, and given $y \in Y$, choose a path $\gamma: y_0 \to y$. Define
$$F(y) = \int_\gamma p^*\omega.$$
To see that $F$ is well-defined, we must show that $F(y)$ does not depend on the choice of $\gamma$.
So let $\eta: y_0 \to y$ be a path; then
$$\int_\eta p^*\omega = F(y) - \int_{\gamma - \eta} p^*\omega$$
and $\gamma - \eta$ is a loop at $y$, so, since $p^*$ is an isomorphism $K \to \pi_1(Y)$, $p_*[\gamma - \eta] \in K$. Therefore
$$\int_{\gamma - \eta} p^*\omega = \int_{p_*[\gamma - \eta]} \omega = \psi([\gamma - \eta]) = 0.$$
So $F$ is well-defined, and since $F$ is clearly a primitive of $p^*\omega$, it follows that $dF = p^*\omega$.

\begin{exer}[Forster 10.4]
Let $\Gamma$ be a lattice. Use the residue theorem to show that there is no meromorphic function $f$ on $\CC/\Gamma$ having a single pole of order $1$.
\end{exer}

Let $a$ be a pole of order $1$ of $f$. In coordinates near $a$,
$$f(z) = \sum_{j=-1}^\infty \alpha_j z^j$$
where $\alpha_{-1} \neq 0$. Then the residue of $f$ at $a$ is $\alpha_{-1}$.
Therefore if $a_1, \dots, a_k$ are the other poles of $f$,
$$\alpha_{-1} + \sum_{j=1}^k \Res_{a_j}f = 0.$$
Therefore $k \geq 1$.

\begin{exer}[Forster 29.2a]
Let $L \to X$ be a line bundle on a compact Riemann surface $X$.
Let $\deg L = \deg (s)$ where $s \in H^0(\Mero_L)$.
Show that $\deg L$ is well-defined.
\end{exer}

In order that $\deg L$ be well-defined, it must be the case that $\deg(s)$ does not depend on $s$.
Let $t$ be another meromorphic section of $L$.
The quotient $f = s/t$ makes sense as an element of $\Mero(X)$.
Indeed, if $(U_1, \dots, U_n)$ is a (finite, since $X$ is compact) open cover of $X$ by linear charts for $L$, with associated transition functions $g_{ij}: U_i \cap U_j \to \CC^*$, then
$$\frac{s_i}{t_i} = \frac{g_{ij}s_j}{g_{ij}t_j} = \frac{s_j}{t_j}.$$
Therefore $f(z)$ does not depend on the choice of trivialization of $L$ at $z$.

If $\gamma$ is a small loop in an open subset $U$ of $X$ such that $f \in \Olo(U)$, then $\int_\gamma f(z) ~dz = 0$ by the argument principle.
Such a $U$ exists since $f$ only has finitely many zeroes and poles.
By the argument principle again, if $Z$ is the number of zeroes and $P$ is the number of poles of $f$,
$$\int_\gamma f(z) ~dz = -\int_{-\gamma} f(z) ~dz = -2\pi i(Z - P)$$
which implies $Z = P$. But $\deg (s)$ was the number of zeroes minus poles of $s$, and similarly for $t$, so $\deg (s) - \deg(t) = Z - P = 0$.

\begin{exer}[Forster 29.2b]
Let $(U_1, U_2)$ be the cover of $\PP^1$ by charts $U_1 = \{z \in \PP^1: |z| < 3/2\}$, $U_2 = \{z \in \PP^1: |z| > 1/2\}$ (say).
Let $L$ be the line bundle on $\PP^1$ defined by a transition function $g_{12}: U_1 \cap U_2 \to \CC^*$.
Show that
$$\deg L = \frac{1}{2\pi i} \int_{|z| = 1} \frac{g_{12}'(z)}{g_{12}(z)} ~dz.$$
\end{exer}

Let $s$ be the meromorphic section of $L$ defined by $s_2 = 1$. Then all zeroes and poles of $s$ are away from $U_2$, and $s_1 = g_{12}s_2 = g_{12}$.
So by the argument principle,
$$\deg L = \deg(s) = \deg(s_1) = \frac{1}{2\pi i} \int_{|z| = 1} \frac{s_1'(z)}{s_1(z)} ~dz = \frac{1}{2\pi i} \int_{|z| = 1} \frac{g_{12}'(z)}{g_{12}(z)} ~dz$$
as desired.

\end{document}
