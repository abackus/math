
% --------------------------------------------------------------
% This is all preamble stuff that you don't have to worry about.
% Head down to where it says "Start here"
% --------------------------------------------------------------

\documentclass[10pt]{article}

\usepackage[margin=.7in]{geometry}
\usepackage{amsmath,amsthm,amssymb}
\usepackage{enumitem}
\usepackage{tikz-cd}
\usepackage{mathtools}
\usepackage{amsfonts}
\usepackage{listings}
\usepackage{algorithm2e}
\usepackage{verse,stmaryrd}
\usepackage{fancyvrb}

% Number systems
\newcommand{\NN}{\mathbb{N}}
\newcommand{\ZZ}{\mathbb{Z}}
\newcommand{\QQ}{\mathbb{Q}}
\newcommand{\RR}{\mathbb{R}}
\newcommand{\CC}{\mathbb{C}}
\newcommand{\PP}{\mathbb P}
\newcommand{\FF}{\mathbb F}
\newcommand{\DD}{\mathbb D}
\renewcommand{\epsilon}{\varepsilon}

\newcommand{\Aut}{\operatorname{Aut}}
\newcommand{\coker}{\operatorname{coker}}
\newcommand{\CVect}{\CC\operatorname{-Vect}}
\newcommand{\Cantor}{\mathcal{C}}
\newcommand{\D}{\mathcal{D}}
\newcommand{\card}{\operatorname{card}}
\newcommand{\diam}{\operatorname{diam}}
\newcommand{\dbar}{\overline \partial}
\DeclareMathOperator*{\esssup}{ess\,sup}
\newcommand{\GL}{\operatorname{GL}}
\newcommand{\Hom}{\operatorname{Hom}}
\newcommand{\id}{\operatorname{id}}
\newcommand{\Ind}{\operatorname{Ind}}
\newcommand{\Inn}{\operatorname{Inn}}
\newcommand{\interior}{\operatorname{int}}
\newcommand{\lcm}{\operatorname{lcm}}
\newcommand{\mesh}{\operatorname{mesh}}
\newcommand{\LL}{\mathcal L_0}
\newcommand{\Leb}{\mathcal{L}_{\text{loc}}^2}
\newcommand{\ppGL}{\operatorname{PGL}}
\newcommand{\ppic}{\vspace{35mm}}
\newcommand{\ppset}{\mathcal{P}}
\DeclareMathOperator{\proj}{proj}
\DeclareMathOperator*{\Res}{Res}
\newcommand{\Riem}{\mathcal{R}}
\newcommand{\RVect}{\RR\operatorname{-Vect}}
\newcommand{\Sch}{\mathcal{S}}
\newcommand{\SL}{\operatorname{SL}}
\newcommand{\sgn}{\operatorname{sgn}}
\newcommand{\spn}{\operatorname{span}}
\newcommand{\Spec}{\operatorname{Spec}}
\newcommand{\supp}{\operatorname{supp}}
\newcommand{\TT}{\mathcal T}
\DeclareMathOperator{\tr}{tr}

\DeclareMathOperator{\adj}{adj}
\DeclareMathOperator{\curl}{curl}

% Calculus of variations
\DeclareMathOperator{\pp}{\mathbf p}
\DeclareMathOperator{\zz}{\mathbf z}
\DeclareMathOperator{\uu}{\mathbf u}
\DeclareMathOperator{\vv}{\mathbf v}
\DeclareMathOperator{\ww}{\mathbf w}

% Categories
\newcommand{\Ab}{\mathbf{Ab}}
\newcommand{\Cat}{\mathbf{Cat}}
\newcommand{\Group}{\mathbf{Group}}
\newcommand{\Module}{\mathbf{Module}}
\newcommand{\Set}{\mathbf{Set}}
\DeclareMathOperator{\Fun}{Fun}
\DeclareMathOperator{\Iso}{Iso}

% Complex analysis
\renewcommand{\Re}{\operatorname{Re}}
\renewcommand{\Im}{\operatorname{Im}}

% Logic
\renewcommand{\iff}{\leftrightarrow}
\newcommand{\Henkin}{\operatorname{Henk}}
\newcommand{\PA}{\mathbf{PA}}
\DeclareMathOperator{\proves}{\vdash}

% Group
\DeclareMathOperator{\Gal}{Gal}
\DeclareMathOperator{\Fix}{Fix}
\DeclareMathOperator{\Out}{Out}

% Mean
\def\Xint#1{\mathchoice
{\XXint\displaystyle\textstyle{#1}}%
{\XXint\textstyle\scriptstyle{#1}}%
{\XXint\scriptstyle\scriptscriptstyle{#1}}%
{\XXint\scriptscriptstyle\scriptscriptstyle{#1}}%
\!\int}
\def\XXint#1#2#3{{\setbox0=\hbox{$#1{#2#3}{\int}$ }
\vcenter{\hbox{$#2#3$ }}\kern-.6\wd0}}
\def\ddashint{\Xint=}
\def\dashint{\Xint-}

% Other symbols
\newcommand{\heart}{\ensuremath\heartsuit}
\newcommand\knuthup{\mathbin{\uparrow}}

\DeclareMathOperator{\atanh}{atanh}

% Theorems
\theoremstyle{definition}
\newtheorem*{corollary}{Corollary}
\newtheorem*{falselemma}{Grader's ``Lemma"}
\newtheorem{exer}{Exercise}
\newtheorem{lemma}{Lemma}[exer]
\newtheorem{theorem}[lemma]{Theorem}


\usepackage[backend=bibtex,style=alphabetic,maxcitenames=50,maxnames=50]{biblatex}
\renewbibmacro{in:}{}
\DeclareFieldFormat{pages}{#1}

\begin{document}
\noindent
\large\textbf{Fluid dynamics, HW 8} \hfill \textbf{Aidan Backus} \\
% --------------------------------------------------------------
%                         Start here
% --------------------------------------------------------------\

\begin{exer}
Let $\varphi(r) = r(1 - \log r)$ if $r \leq 1$ and $\varphi(r) = 1$ if $r > 1$, and
$$||u||_{LL} = \sup_{x \neq y} \frac{|u(x) - u(y)|}{\varphi(x - y)}.$$
Let $b \in L^\infty(\RR^d \times [0, T]) \cap L^\infty([0, T] \to LL)$, $\nabla \cdot b = 0$, and let $X$ denote the flow generated by $b$.
Let $\rho$ be a standard mollifier, and let $b_\varepsilon = b * \rho_\varepsilon$.
Let $X^\varepsilon$ be the flow generated by $b_\varepsilon$.
Since $b_\varepsilon \in C^\infty_\sigma$, $X^\varepsilon$ is measure-preserving.

Set $g_\varepsilon(x, t) = |X_t(x) - X^\varepsilon_t(x)|$. Show that
\begin{equation}
\label{first bound}
g_\varepsilon(x, t) \leq \int_0^t ||b(s)||_{LL}\varphi(g_\varepsilon(x, s)) ~ds + \int_0^t |b(X^\varepsilon_s(x), s) - b_\varepsilon(X^\varepsilon_s(x), s)| ~ds.
\end{equation}
\end{exer}

We first claim that mollification is a log-Lipschitz contraction. In fact,
\begin{align*}
|f_\epsilon(x) - f_\epsilon(y)| &\leq \int_{\RR^d} |f_\epsilon(x - z) - f_\epsilon(y - z|)\rho\left(\frac{z}{\epsilon}\right) ~\frac{dz}{\varepsilon^d}\\
&\leq ||f||_{LL} \varphi(|x - y|) \int_{\RR^d} \rho\left(\frac{z}{\epsilon}\right) ~\frac{dz}{\varepsilon^d}\\
&= ||f||_{LL} \varphi(|x - y|).
\end{align*}
Dividing both sides by $\varphi(x - y)$, we conclude
$$||f_\epsilon||_{LL} \leq ||f||_{LL}$$
which gives the claim.

Now we can bound
\begin{align*}
|X^\epsilon(x, t) - X(x, t)| &\leq \int_0^t |b_\epsilon(X^\epsilon_s(x), s) - b(X_s(x), s)| ~ds\\
&\leq \int_0^t |b_\epsilon(X_s(x), s) - b(X_s(x), s)| ~ds + \int_0^t |b_\epsilon(X^\epsilon_s(x), s) - b_\epsilon(X_s(x), s)| ~ds\\
&\leq \int_0^t ||b_\epsilon(s)||_{LL} \varphi(g_\epsilon(x, s)) ~ds + \int_0^t |b_\epsilon(X^\epsilon_s(x), s) - b_\epsilon(X_s(x), s)| ~ds\\
&\leq \int_0^t ||b(s)||_{LL} \varphi(g_\epsilon(x, s)) ~ds + \int_0^t |b_\epsilon(X^\epsilon_s(x), s) - b_\epsilon(X_s(x), s)| ~ds
\end{align*}
as desired.


\begin{exer}
Prove that
\begin{equation}
\label{mollification LL approximates}
|b(x, t) - b_\varepsilon(x, t)| \lesssim ||b(t)||_{LL} \varphi(\varepsilon)
\end{equation}
and deduce
\begin{align*}
g_\varepsilon(x, t) &\leq \varphi(\epsilon) \int_0^t ||b(s)||_{LL} ~ds + \int_0^t ||b(s)||_{LL} \varphi(g_\epsilon(x, s)) ~ds\\
&\leq LT\varphi(\epsilon) + L\int_0^t \varphi(g_\epsilon(x, s)) ~ds
\end{align*}
where $L = ||b||_{L^\infty([0, T] \to LL)}$.
\end{exer}

Owing to the support property of $\rho$, we have
\begin{align*}
|b(x, t) - b_\epsilon(x, t)| &= \left|b(x, t) - \int_{\RR^d} b(x - y, t) \rho\left(\frac{y}{\epsilon}\right)~\frac{dy}{\epsilon^d}\right|\\
&\leq \int_{B(0,\epsilon)} |b(x, t) - b(x - y, t)| \rho\left(\frac{y}{\epsilon}\right)~\frac{dy}{\epsilon^d}\\
&\leq ||b(t)||_{LL} \int_{B(0,\epsilon)} \varphi(|y|) \rho\left(\frac{y}{\epsilon}\right)~\frac{dy}{\epsilon^d} \leq ||b(t)||_{LL} \varphi(\epsilon)
\end{align*}
which gives the first claim (\ref{mollification LL approximates}).

We plug (\ref{mollification LL approximates}) into (\ref{first bound}) to get
$$g_\epsilon(x, t) \leq \varphi(\epsilon) \int_0^t ||b(s)||_{LL} ~ds + \int_0^t ||b(s)||_{LL} \varphi(g_\epsilon(x, s)) ~ds.$$
This completes the proof of the second claim, once we realize that $t \leq T$.

\begin{exer}
Prove that there is $\beta > 0$ such that for every $0 < \epsilon \ll 1$,
$$g_\epsilon(x, t) \lesssim \epsilon^\beta.$$
\end{exer}

We proceed by Osgood's lemma with $f = g_\epsilon(x)$, $\mu = \varphi$, $c = LT\varphi(\epsilon)$, $\gamma = L$, and $a = 2LT$. That is possible because
$$g_\epsilon(x, t) \leq LT\varphi(\epsilon) + L\int_0^t \varphi(g_\epsilon(x, s)) ~ds \leq 2LT$$
using the fact that $\varphi \leq 1$.
Osgood's lemma says
\begin{equation}
\label{osgood}
\int_{LT\varphi(\epsilon)}^{g_\epsilon(x, t)} \frac{dr}{\varphi(r)} \leq \int_0^t L~ds \leq LT.
\end{equation}
As the antiderivative of $1/r$ is $\log r$, and $r(1 - \log r)$ is just a logarithmic singularity from being comparable to $r$, the left-hand side of (\ref{osgood}) is at least polylogarithmic in the sense that there is $\theta > 0$ such that
$$\log^\theta\left(\frac{g_\epsilon(x, t)}{LT\varphi(\epsilon)}\right) \leq \int_{LT\varphi(\epsilon)}^{g_\epsilon(x, t)} \frac{dr}{\varphi(r)} \leq LT.$$
Exponentiating both sides, we conclude that
$$g_\epsilon(x, t) \leq \exp\left((LT)^{1/\theta}\right) LT\varphi(\epsilon)$$
and $\varphi$ has polynomial growth, whence the claim.


\begin{exer}
Show that for every $f \in C_c(\RR^d)$,
$$\int_{\RR^d} f(X_t(x)) ~dx = \int_{\RR^d} f(x) ~dx.$$
\end{exer}

Let $\beta \in (0, 1)$, $0 < \epsilon \ll 1$.
As $X^\epsilon$ is a measure-preserving one-parameter group, Exercise 3 gives
\begin{align*}
\left|\int_{\RR^d} f(X_t(x)) - f(x) ~dx\right| &\leq \int_{\RR^d} |f(X_t(x)) - f(X_t^\epsilon(x))| ~dx \leq ||f||_{L^1} ||g_\epsilon||_{L^\infty} \lesssim ||f||_{L^1} \epsilon^\beta.
\end{align*}
Now take the limit as $\epsilon \to 0$ of both sides.


\end{document}
