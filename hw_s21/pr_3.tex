
% --------------------------------------------------------------
% This is all preamble stuff that you don't have to worry about.
% Head down to where it says "Start here"
% --------------------------------------------------------------

\documentclass[10pt]{article}

\usepackage[margin=.7in]{geometry}
\usepackage{amsmath,amsthm,amssymb}
\usepackage{enumitem}
\usepackage{tikz-cd}
\usepackage{mathtools}
\usepackage{amsfonts}
\usepackage{listings}
\usepackage{algorithm2e}
\usepackage{verse,stmaryrd}
\usepackage{fancyvrb}

% Number systems
\newcommand{\NN}{\mathbb{N}}
\newcommand{\ZZ}{\mathbb{Z}}
\newcommand{\QQ}{\mathbb{Q}}
\newcommand{\RR}{\mathbb{R}}
\newcommand{\CC}{\mathbb{C}}
\newcommand{\PP}{\mathbb P}
\newcommand{\FF}{\mathbb F}
\newcommand{\DD}{\mathbb D}
\renewcommand{\epsilon}{\varepsilon}

\newcommand{\Aut}{\operatorname{Aut}}
\newcommand{\cl}{\operatorname{cl}}
\newcommand{\ch}{\operatorname{ch}}
\newcommand{\Con}{\operatorname{Con}}
\newcommand{\coker}{\operatorname{coker}}
\newcommand{\CVect}{\CC\operatorname{-Vect}}
\newcommand{\Cantor}{\mathcal{C}}
\newcommand{\D}{\mathcal{D}}
\newcommand{\card}{\operatorname{card}}
\newcommand{\dbar}{\overline \partial}
\newcommand{\diam}{\operatorname{diam}}
\newcommand{\dom}{\operatorname{dom}}
\newcommand{\End}{\operatorname{End}}
\DeclareMathOperator*{\esssup}{ess\,sup}
\newcommand{\GL}{\operatorname{GL}}
\newcommand{\Hom}{\operatorname{Hom}}
\newcommand{\id}{\operatorname{id}}
\newcommand{\Ind}{\operatorname{Ind}}
\newcommand{\Inn}{\operatorname{Inn}}
\newcommand{\interior}{\operatorname{int}}
\newcommand{\lcm}{\operatorname{lcm}}
\newcommand{\mesh}{\operatorname{mesh}}
\newcommand{\LL}{\mathcal L_0}
\newcommand{\Leb}{\mathcal{L}_{\text{loc}}^2}
\newcommand{\Lip}{\operatorname{Lip}}
\newcommand{\ppGL}{\operatorname{PGL}}
\newcommand{\ppic}{\vspace{35mm}}
\newcommand{\ppset}{\mathcal{P}}
\DeclareMathOperator{\proj}{proj}
\DeclareMathOperator*{\Res}{Res}
\newcommand{\Riem}{\mathcal{R}}
\newcommand{\RVect}{\RR\operatorname{-Vect}}
\newcommand{\Sch}{\mathcal{S}}
\newcommand{\SL}{\operatorname{SL}}
\newcommand{\sgn}{\operatorname{sgn}}
\newcommand{\spn}{\operatorname{span}}
\newcommand{\Spec}{\operatorname{Spec}}
\newcommand{\supp}{\operatorname{supp}}
\newcommand{\TT}{\mathcal T}
\DeclareMathOperator{\tr}{tr}

% Calculus of variations
\DeclareMathOperator{\pp}{\mathbf p}
\DeclareMathOperator{\zz}{\mathbf z}
\DeclareMathOperator{\uu}{\mathbf u}
\DeclareMathOperator{\vv}{\mathbf v}
\DeclareMathOperator{\ww}{\mathbf w}

% Categories
\newcommand{\Ab}{\mathbf{Ab}}
\newcommand{\Cat}{\mathbf{Cat}}
\newcommand{\Group}{\mathbf{Group}}
\newcommand{\Module}{\mathbf{Module}}
\newcommand{\Set}{\mathbf{Set}}
\DeclareMathOperator{\Fun}{Fun}
\DeclareMathOperator{\Iso}{Iso}

% Complex analysis
\renewcommand{\Re}{\operatorname{Re}}
\renewcommand{\Im}{\operatorname{Im}}

% Logic
\renewcommand{\iff}{\leftrightarrow}
\newcommand{\Henkin}{\operatorname{Henk}}
\newcommand{\PA}{\mathbf{PA}}
\DeclareMathOperator{\Var}{Var}
\DeclareMathOperator{\proves}{\vdash}

% Group
\DeclareMathOperator{\Gal}{Gal}
\DeclareMathOperator{\Fix}{Fix}
\DeclareMathOperator{\Out}{Out}

% Other symbols
\newcommand{\heart}{\ensuremath\heartsuit}

\DeclareMathOperator{\atanh}{atanh}

% Theorems
\theoremstyle{definition}
\newtheorem*{corollary}{Corollary}
\newtheorem*{falselemma}{Grader's ``Lemma"}
\newtheorem{exer}{Exercise}
\newtheorem{lemma}{Lemma}[exer]
\newtheorem{theorem}[lemma]{Theorem}


\usepackage[backend=bibtex,style=alphabetic,maxcitenames=50,maxnames=50]{biblatex}
\renewbibmacro{in:}{}
\DeclareFieldFormat{pages}{#1}

\begin{document}
\noindent
\large\textbf{Probability II, HW 3} \hfill \textbf{Aidan Backus} \\
% --------------------------------------------------------------
%                         Start here
% --------------------------------------------------------------\

\begin{exer}
At time $0$ an urn contains $b$ black balls and $w$ white balls.
At each time $n \geq 1$ a ball is randomly chosen from the urn and put back into the urn with another ball of the same color.
Let $B_n$ be the total number of black balls chosen by time $n$ and let
$$M_n = \frac{b + B_n}{n + b + w}$$
be the proportion of black balls in the urn after time $n$.
Show that $M$ is a martingale with respect to the filtration $\mathcal F$ generated by $B$.
Compute the distribution of $M_\infty$ with $b = w = 1$.
\end{exer}

Given $\mathcal F_n$, the probability that a black ball is chosen at time $n + 1$ is $M_n$.
But $E(B_{n+1} - B_n)$ is also that probability and $M_n,B_n$ are $\mathcal F_n$-measurable so
$$E(B_{n+1}|\mathcal F_n) - B_n = M_n.$$
Therefore
\begin{align*}
E(M_{n+1}|\mathcal F_n) &= \frac{b + E(B_{n+1}|\mathcal F_n)}{n + 1 + b + w} = \frac{b + B_n + M_n}{n + 1 + b + w}\\
&= \frac{(n + b + w)(b + B_n) + b + B_n}{(n + b + w)(n + 1 + b + w)} = \frac{b + B_n}{n + b + w} \\
&= M_n\end{align*}
so $M$ is a martingale.

To compute the distribution of $M_\infty$, let $G$ be the set of all events in $\mathcal F_\infty$ of the form $B_n = m$ which have nonzero probability.
We put a graph structure on $G$ by drawing an arrow from $B_n = m$ to the events $B_{n+1} = m$ and $B_{n+1} = m + 1$.
By a path to $B_n = m$ we will mean a sequence of arrows from the event $B_0 = 0$ to an event $B_n = m$.
If $\gamma$ is a path we write $A_\gamma$ for the intersection of all events along $\gamma$, and write $\gamma_j = m$ provided that the event $B_j = m$ is on $\gamma$.
Then
\begin{align*}
P(A_\gamma) &= \prod_{j=1}^n P(B_j = \gamma_j|B_{j=1} = \gamma_{j-1})\\
&= \prod_{j=1}^n \begin{cases}
E(M_{j-1}|B_{j-1} = \gamma_{j-1}), &\gamma_j - \gamma_{j-1} = 1\\
E(1 - M_{j-1}|B_{j-1} = \gamma_{j-1}), &\gamma_j - \gamma_{j-1} = 0
\end{cases}\\
&= \prod_{j=1}^n \frac{1}{j+b+w-1}\prod_{\gamma_j - \gamma_{j-1} = 1} j + w - \gamma_{j-1} - 1 \prod_{\gamma_j - \gamma_{j-1} = 0} b + \gamma_{j - 1}.
\end{align*}
Obviously the first product here does not depend on $\gamma$, but only on $n$.

We claim that neither does the product of the second and third products depend on $\gamma$, but only on the final node along $\gamma$, say $B_n = m$.
In that case, there are $n - m$ many $j$ such that $\gamma_j = \gamma_{j-1}$, and such $j$ completely determine $\gamma$.
However, permuting them does not change the product of the second and third products, since it just permutes the factors appearing in them.

Now we specialize to the case $b = w = 1$.
In that case, the fact that the choice of path does not matter immediately implies that $B_n$ is uniformly distributed on $\{0, \dots, n\}$, thus $M_n$ is uniformly distributed on $\{k/(2+n): k \in \{1, \dots, n + 1\}\}$.
Intuitively, it should follow that $M_\infty$ is uniformly distributed on $[0, 1]$, but let us check this.

To see the claim it suffices to show that if $\mu_n$ is the probability measure
$$\mu_n = \frac{1}{n + 1} \sum_{j=1}^{n+1} \delta_{k/(2+n)}$$
where $\delta_q$ is the Dirac measure at $q$, then $\mu_n$ converges to the Lebesgue measure, in the weak topology of measures.
In fact, $\int_0^1 f ~d\mu_n$ is the $n+1$th midpoint Riemann sum of $f$ whenever $f$ is continuous, so
$$\lim_{n \to \infty} \int_0^1 f ~d\mu_n = \int_0^1 f(x) ~dx$$
where the latter integral is Riemann's. But the Riemann and Lebesgue integrals coincide on continuous functions on compact sets so this shows the desired convergence.

\begin{exer}
Let $X$ be a stochastic process where $X_n$ represents your winnings per unit stake on game $n$, the $X_n$ are iid, and $P(X_n = 1) = p$, $P(X_n = -1) = q$ where $1/2 < p < 1$ and $p + q = 1$.
Let $Z_n$ be your capital after game $n$, and let $Z_0 = 1$ be your initial capital.
Let $\mathcal F_n = \sigma(X_1, \dots, X_n)$ be the induced filtration.

Show that if $A$ is a betting strategy (so $A_{n+1} \in (0, Z_n)$ is your stake in game $n + 1$ and $A$ is $\mathcal F$-predictable) and
$$\alpha = p \log p + q \log q + \log 2$$
is the entropy, the process $\log Z_n - n\alpha$ where $Z$ is induced by $A$ is a supermartingale.

Show that there exists a betting strategy $A^*$ for which $\log Z_n^* - n\alpha$, where $Z^*$ is induced by $A^*$, is a martingale.

Find the optimal betting strategy and the maximal interest rate $E(\log Z_N)$, where $N$ is a fixed time.
\end{exer}

To show that $\log Z_n - n\alpha$ is a supermartingale, we must show that
$$E(\log Z_{n+1} - \alpha|\mathcal F_n) \leq \log Z_n.$$
But
$$Z_{n+1} = Z_n + A_{n+1}X_{n+1},$$
and as $\alpha$ is deterministic, $A$ is predictable, and $X_{n+1}$ has distribution $\nu = p\delta_1 - q\delta_{-1}$ (where $\delta_q$ is the Dirac measure at $q$),
$$E(\log Z_{n+1} - \alpha|\mathcal F_n) = p\log(Z_n + A_{n+1}) + q\log(Z_n - A_{n+1}) - \alpha.$$
Thus
$$\log Z_n - E(\log Z_{n+1} - \alpha|\mathcal F_n) = p \log \frac{2pZ_n}{Z_n + A_{n+1}} + q\log \frac{2qZ_n}{Z_n + A_{n+1}}.$$
since $p + q = 1$.
Now set
$$\overline p = \frac{Z_n + A_{n+1}}{2Z_n}, \quad \overline q = \frac{Z_n - A_{n+1}}{2Z_n}.$$
Then $\overline p + \overline q = 1$, and so $\mu = \overline p \delta_1 + \overline q \delta_{-1}$ is a probability measure.
Therefore
$$\log Z_n - E(\log Z_{n+1} - \alpha|\mathcal F_n) = p \log \frac{p}{\overline p} + q \log \frac{q}{\overline q} = H(\nu|\mu) \geq 0$$
with equality iff $\mu = \nu$; here $H(\nu|\mu)$ is the relative entropy of $\nu$ given $\mu$, and relative entropy was shown to always be nonnegative as a byproduct of one of my solutions to the previous problem set.

We conclude that $\log Z_n - n\alpha$ is a supermartingale, and is a martingale iff $\mu = \nu$.
This can be achieved by requiring $p = \overline p$, thus
$$\frac{Z_n + A_{n+1}^*}{2Z_n} = p.$$
Solving for $A_{n+1}^*$ we deduce that
$$A_{n+1}^* = Z_n(p - q)$$
defines an optimal investment strategy $A^*$.
Indeed, $A^*$ is the unique investment strategy for which $\log Z_n - n\alpha$ is a martingale, and $\log Z_n - n\alpha$ will always be a supermartingale, so $A^*$ really is best possible.
Besides, if $Z^*$ is the martingale induced by $A^*$ then $E(\log Z_N^*) = N\alpha$ for any stopping time $N$ by optional stopping, so the maximal expected interest rate is $N\alpha$ if we stop at time $N$.

\begin{exer}
You are shown cards consecutively from a regular deck of $52$ cards. At some point you request to stop the game.
If the next card is red you win one dollar; otherwise you win nothing. You can stop the game at any time.
Show that no matter what strategy you use, your expected return is $50$ cents.
\end{exer}

Let $A_{n,j,f}$ be the event ``The card drawn at time $n$ is the $j$ of $f$" where $j \in \{1, 2, \dots, 10, J,Q,K\}$, $f \in \{\spadesuit, \heartsuit, \diamondsuit, \clubsuit\}$, and $n \geq 1$.
Let $\mathcal F_n = \sigma(\{E_{m,j,f}:m \leq n,j,f\})$ be the $\sigma$-algebra of events which already may have happened by time $n$, so $\mathcal F$ defines a filtration, and the $A_{m,j,f}$, $m \leq n$, along with the empty event form a $\pi$-system in $\mathcal F_n$.
In particular $\mathcal F_0$ is the trivial $\sigma$-algebra.

Introduce the stochastic process $X_n$ that returns $f$ on $A_{n,j,f}$, thus $X_n$ is the face drawn at time $n$.
Let
$$Y_n = P(X_{n+1} \in \{\heartsuit, \diamondsuit\}|\mathcal F_n).$$
We claim $Y$ is a martingale. In fact
$$E(Y_{n+1}|\mathcal F_n) = P(X_{n+2} \in \{\heartsuit, \diamondsuit\}|\mathcal F_n)$$
by the tower law, but
$$P(X_{n+2} \in \{\heartsuit, \diamondsuit\}|\mathcal F_n) = P(X_{n+1} \in \{\heartsuit, \diamondsuit\}|\mathcal F_n)$$
since for every $k > n$, the probability of drawing a red card at time $k$ given $\mathcal F_n$ is just the proportion of red cards to total cards in the deck at time $n$.
Therefore $Y$ is a martingale.

On the other hand, if $\tau$ is a stopping time then
\begin{align*}
P(X_{\tau+1} \in \{\heartsuit, \diamondsuit\}) &= \sum_{j=0}^{52} P(X_{n+1} \in \{\heartsuit, \diamondsuit\} \cap \tau = j)
= \sum_{j=0}^{52} E(1_{\tau = j} E(1_{X_{n+1} \in \{\heartsuit, \diamondsuit\}}|\mathcal F_n))\\
&= \sum_{j=0}^{52} E(1_{\tau = j} Y_n) = EY_\tau = EY_0
\end{align*}
by optional stopping. Clearly $EY_0 = 1/2$.


\end{document}
