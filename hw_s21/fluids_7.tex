
% --------------------------------------------------------------
% This is all preamble stuff that you don't have to worry about.
% Head down to where it says "Start here"
% --------------------------------------------------------------

\documentclass[10pt]{article}

\usepackage[margin=.7in]{geometry}
\usepackage{amsmath,amsthm,amssymb}
\usepackage{enumitem}
\usepackage{tikz-cd}
\usepackage{mathtools}
\usepackage{amsfonts}
\usepackage{listings}
\usepackage{algorithm2e}
\usepackage{verse,stmaryrd}
\usepackage{fancyvrb}

% Number systems
\newcommand{\NN}{\mathbb{N}}
\newcommand{\ZZ}{\mathbb{Z}}
\newcommand{\QQ}{\mathbb{Q}}
\newcommand{\RR}{\mathbb{R}}
\newcommand{\CC}{\mathbb{C}}
\newcommand{\PP}{\mathbb P}
\newcommand{\FF}{\mathbb F}
\newcommand{\DD}{\mathbb D}
\renewcommand{\epsilon}{\varepsilon}

\newcommand{\Aut}{\operatorname{Aut}}
\newcommand{\coker}{\operatorname{coker}}
\newcommand{\CVect}{\CC\operatorname{-Vect}}
\newcommand{\Cantor}{\mathcal{C}}
\newcommand{\D}{\mathcal{D}}
\newcommand{\card}{\operatorname{card}}
\newcommand{\diam}{\operatorname{diam}}
\newcommand{\dbar}{\overline \partial}
\DeclareMathOperator*{\esssup}{ess\,sup}
\newcommand{\GL}{\operatorname{GL}}
\newcommand{\Hom}{\operatorname{Hom}}
\newcommand{\id}{\operatorname{id}}
\newcommand{\Ind}{\operatorname{Ind}}
\newcommand{\Inn}{\operatorname{Inn}}
\newcommand{\interior}{\operatorname{int}}
\newcommand{\lcm}{\operatorname{lcm}}
\newcommand{\mesh}{\operatorname{mesh}}
\newcommand{\LL}{\mathcal L_0}
\newcommand{\Leb}{\mathcal{L}_{\text{loc}}^2}
\newcommand{\ppGL}{\operatorname{PGL}}
\newcommand{\ppic}{\vspace{35mm}}
\newcommand{\ppset}{\mathcal{P}}
\DeclareMathOperator{\proj}{proj}
\DeclareMathOperator*{\Res}{Res}
\newcommand{\Riem}{\mathcal{R}}
\newcommand{\RVect}{\RR\operatorname{-Vect}}
\newcommand{\Sch}{\mathcal{S}}
\newcommand{\SL}{\operatorname{SL}}
\newcommand{\sgn}{\operatorname{sgn}}
\newcommand{\spn}{\operatorname{span}}
\newcommand{\Spec}{\operatorname{Spec}}
\newcommand{\supp}{\operatorname{supp}}
\newcommand{\TT}{\mathcal T}
\DeclareMathOperator{\tr}{tr}

\DeclareMathOperator{\adj}{adj}
\DeclareMathOperator{\curl}{curl}

% Calculus of variations
\DeclareMathOperator{\pp}{\mathbf p}
\DeclareMathOperator{\zz}{\mathbf z}
\DeclareMathOperator{\uu}{\mathbf u}
\DeclareMathOperator{\vv}{\mathbf v}
\DeclareMathOperator{\ww}{\mathbf w}

% Categories
\newcommand{\Ab}{\mathbf{Ab}}
\newcommand{\Cat}{\mathbf{Cat}}
\newcommand{\Group}{\mathbf{Group}}
\newcommand{\Module}{\mathbf{Module}}
\newcommand{\Set}{\mathbf{Set}}
\DeclareMathOperator{\Fun}{Fun}
\DeclareMathOperator{\Iso}{Iso}

% Complex analysis
\renewcommand{\Re}{\operatorname{Re}}
\renewcommand{\Im}{\operatorname{Im}}

% Logic
\renewcommand{\iff}{\leftrightarrow}
\newcommand{\Henkin}{\operatorname{Henk}}
\newcommand{\PA}{\mathbf{PA}}
\DeclareMathOperator{\proves}{\vdash}

% Group
\DeclareMathOperator{\Gal}{Gal}
\DeclareMathOperator{\Fix}{Fix}
\DeclareMathOperator{\Out}{Out}

% Other symbols
\newcommand{\heart}{\ensuremath\heartsuit}

\DeclareMathOperator{\atanh}{atanh}

% Theorems
\theoremstyle{definition}
\newtheorem*{corollary}{Corollary}
\newtheorem*{falselemma}{Grader's ``Lemma"}
\newtheorem{exer}{Exercise}
\newtheorem{lemma}{Lemma}[exer]
\newtheorem{theorem}[lemma]{Theorem}


\usepackage[backend=bibtex,style=alphabetic,maxcitenames=50,maxnames=50]{biblatex}
\renewbibmacro{in:}{}
\DeclareFieldFormat{pages}{#1}

\begin{document}
\noindent
\large\textbf{Fluid dynamics, HW 7} \hfill \textbf{Aidan Backus} \\
% --------------------------------------------------------------
%                         Start here
% --------------------------------------------------------------\

I got a hint from Erik Bergland while working on this problem set.

\begin{exer}
Given $t_0$, define
$$B(a, b)(t) = \int_{t_0}^t e^{\nu \Delta(t' - t_0)} Q(a(t'), b(t')) dt'$$
to be the unique solution of $\partial_t g = \nu \Delta g + Q(a, b)$ with initial data $0$.
Use Lemma 7.4 to show that
$$||B(a, b)(t)||_{K_6(T)} \lesssim ||a||_{K_6(t)} ||b||_{K_6(t)} g(t_0/t)$$
where
$$g(r) = \int_r^1 (1 - s)^{-3/4} s^{-1/2} ~ds.$$
Show that $g(r) \lesssim (1 - r)^{1/4}$.
\end{exer}

According to the proof of Lemma 7.3, if Lemma 7.4 is true, then
$$||B(u, v)(t)||_{L^6} \lesssim \int_{t_0}^t \frac{1}{\nu^{3/4}(t - t')^{3/4}} ||u(t')||_{L^6} ||v(t')||_{L^6} ~dt'.$$
The Cauchy-Schwarz inequality then gives
\begin{align*}
||B(u, v)(t)||_{L^6} &\lesssim ||u||_{K_6(t)} ||v||_{K_6(t)} \int_{t_0}^t \frac{1}{\nu^{3/4}(t - t')^{3/4}} \frac{1}{(\nu t')^{5/3}} ~dt'\\
&= ||u||_{K_6(t)} ||v||_{K_6(t)} \frac{1}{(\nu t)^{5/3}} \int_{t_0}^t \frac{dt'}{\nu^{3/4}(t - t')^{3/4}}.
\end{align*}
Now we multiply both sides by $(\nu t)^{5/3}$ and optimize in time to get
$$||B(u, v)||_{K_6(t)} \leq ||u||_{K_6(t)} ||v||_{K_6(t)} \int_{t_0}^t \frac{dt'}{\nu^{3/4}(t - t')^{3/4}}.$$
Rescaling the endpoints and absorbing the viscosity into a constant gives
$$\int_{t_0}^t \frac{dt'}{\nu^{3/4}(t - t')^{3/4}} \lesssim_\nu g(t_0/t).$$
The desired bound on $g$ is only interesting for $r \in (1/2, 1)$ (say), and in that case, we can take $s^{-1/2} \sim 1$ for every $s \in (r, 1) \subset (1/2, 1)$.
Thus
$$g(r) \sim \int_r^1 \frac{ds}{(1 - x)^{3/4}} = 4(1 - r)^{1/4},$$
as desired.


\begin{exer}
Let
$$X(T) = C([0, T] \to L^3) \cap K_6(T).$$
Suppose that $u,v \in X(T)$ are two solutions to the Navier-Stokes system with initial data $u_0$.
Let $t_0 \in [0, T]$ be the maximal time such that $u|[0, t_0] = v|[0, t_0]$.
Suppose that $t_0 < T$.
Using the previous exercise, show that there is $\varepsilon > 0$ such that $u|[0, t_0 + \varepsilon] = v|[0, t_0 + \varepsilon]$.
Conclude that the Navier-Stokes system has uniqueness in $X(T)$.
\end{exer}

Suppose that $t_0,u,v$ are as in the problem statement.
As $B$ is a symmetric form we have
$$B(u, u) - B(v, v) = B(u - v, u + v).$$
Moreover, uniqueness for the heat equation and the fixed-point formulation
$$u(t) = e^{\nu (t - t_0)\Delta} u(t_0) + B(u, u)(t)$$
of the Navier-Stokes equation gives
$$u - v = B(u, u) - B(v, v).$$
Therefore for every $t \in (t_0, T]$,
\begin{align*}
||u - v||_{K_6(t)} &= ||B(u, u) - B(v, v)||_{K_6(t)} = ||B(u - v, u + v)||_{K_6(t)}\\
&\lesssim ||u - v||_{K_6(t)} ||u + v||_{K_6(t)} \sqrt[4]{1 - \frac{t_0}{t}}\\
&\leq ||u - v||_{K_6(t)}(||u||_{K_6(T)} + ||v||_{K_6(T)}) \sqrt[4]{1 - \frac{t_0}{t}}.
\end{align*}
Our contradiction assumption implies that we can divide both sides by $||u - v||_{K_6(t)} > 0$.
Absorbing $||u||_{K_6(T)} + ||v||_{K_6(T)} < \infty$ into the implied constant, we get a lower bound
$$1 - \frac{t_0}{t} \gtrsim 1$$
which is uniform in $t$. Taking $t \to t_0$ we conclude $0 \gtrsim 1$, which is absurd.
Therefore there must be $\varepsilon > 0$ such that for every $t \in (t_0, t_0 + \varepsilon]$, $||u - v||_{K_6(t)} = 0$.
Since this contradicts maximality of $t_0$, the time interval on which the Navier-Stokes equations are unique in $K_6(T)$ must be clopen in $[0, T]$, and hence equal to $[0, T]$, since the local uniqueness implies that it is not empty.


\end{document}
