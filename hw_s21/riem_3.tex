% --------------------------------------------------------------
% This is all preamble stuff that you don't have to worry about.
% Head down to where it says "Start here"
% --------------------------------------------------------------
\documentclass[10pt]{article}

\usepackage[margin=.7in]{geometry}
\usepackage{amsmath,amsthm,amssymb,mathrsfs}
\usepackage{enumitem}
\usepackage{tikz-cd}
\usepackage{mathtools}
\usepackage{amsfonts}
\usepackage{listings}
\usepackage{algorithm2e}
\usepackage{verse,stmaryrd}
\usepackage{fancyvrb}

% Number systems
\newcommand{\NN}{\mathbb{N}}
\newcommand{\ZZ}{\mathbb{Z}}
\newcommand{\QQ}{\mathbb{Q}}
\newcommand{\RR}{\mathbb{R}}
\newcommand{\CC}{\mathbb{C}}
\newcommand{\PP}{\mathbb P}
\newcommand{\FF}{\mathbb F}
\newcommand{\DD}{\mathbb D}
\renewcommand{\epsilon}{\varepsilon}

\newcommand{\Aut}{\operatorname{Aut}}
\newcommand{\cl}{\operatorname{cl}}
\newcommand{\ch}{\operatorname{ch}}
\newcommand{\Con}{\operatorname{Con}}
\newcommand{\coker}{\operatorname{coker}}
\newcommand{\CVect}{\CC\operatorname{-Vect}}
\newcommand{\Cantor}{\mathcal{C}}
\newcommand{\D}{\mathcal{D}}
\newcommand{\card}{\operatorname{card}}
\newcommand{\dbar}{\overline \partial}
\newcommand{\diam}{\operatorname{diam}}
\newcommand{\dom}{\operatorname{dom}}
\newcommand{\End}{\operatorname{End}}
\DeclareMathOperator*{\esssup}{ess\,sup}
\newcommand{\GL}{\operatorname{GL}}
\newcommand{\Hom}{\operatorname{Hom}}
\newcommand{\id}{\operatorname{id}}
\newcommand{\Ind}{\operatorname{Ind}}
\newcommand{\Inn}{\operatorname{Inn}}
\newcommand{\interior}{\operatorname{int}}
\newcommand{\lcm}{\operatorname{lcm}}
\newcommand{\mesh}{\operatorname{mesh}}
\newcommand{\LL}{\mathcal L_0}
\newcommand{\Leb}{\mathcal{L}_{\text{loc}}^2}
\newcommand{\Lip}{\operatorname{Lip}}
\newcommand{\ppGL}{\operatorname{PGL}}
\newcommand{\ppic}{\vspace{35mm}}
\newcommand{\ppset}{\mathcal{P}}
\DeclareMathOperator{\proj}{proj}
\DeclareMathOperator*{\Res}{Res}
\newcommand{\Riem}{\mathcal{R}}
\newcommand{\RVect}{\RR\operatorname{-Vect}}
\newcommand{\Sch}{\mathcal{S}}
\newcommand{\SL}{\operatorname{SL}}
\newcommand{\sgn}{\operatorname{sgn}}
\newcommand{\spn}{\operatorname{span}}
\newcommand{\Spec}{\operatorname{Spec}}
\newcommand{\supp}{\operatorname{supp}}
\newcommand{\TT}{\mathcal T}
\DeclareMathOperator{\tr}{tr}

% Calculus of variations
\DeclareMathOperator{\pp}{\mathbf p}
\DeclareMathOperator{\zz}{\mathbf z}
\DeclareMathOperator{\uu}{\mathbf u}
\DeclareMathOperator{\vv}{\mathbf v}
\DeclareMathOperator{\ww}{\mathbf w}

% Categories
\newcommand{\Ab}{\mathbf{Ab}}
\newcommand{\Cat}{\mathbf{Cat}}
\newcommand{\Group}{\mathbf{Group}}
\newcommand{\Module}{\mathbf{Module}}
\newcommand{\Set}{\mathbf{Set}}
\DeclareMathOperator{\Fun}{Fun}
\DeclareMathOperator{\Iso}{Iso}

% Complex analysis
\renewcommand{\Re}{\operatorname{Re}}
\renewcommand{\Im}{\operatorname{Im}}
\newcommand{\Div}{\operatorname{Div}}

% Logic
\renewcommand{\iff}{\leftrightarrow}
\newcommand{\Henkin}{\operatorname{Henk}}
\newcommand{\PA}{\mathbf{PA}}
\DeclareMathOperator{\proves}{\vdash}

% Group
\DeclareMathOperator{\Gal}{Gal}
\DeclareMathOperator{\Fix}{Fix}
\DeclareMathOperator{\Out}{Out}

\newcommand{\Mero}{\mathscr M}
\newcommand{\Olo}{\mathscr O}

% Other symbols
\newcommand{\heart}{\ensuremath\heartsuit}

\DeclareMathOperator{\atanh}{atanh}

% Theorems
\theoremstyle{definition}
\newtheorem*{corollary}{Corollary}
\newtheorem*{falselemma}{Grader's ``Lemma"}
\newtheorem{exer}{Exercise}
\newtheorem{lemma}{Lemma}[exer]
\newtheorem{theorem}[lemma]{Theorem}

\usepackage[backend=bibtex,style=alphabetic,maxcitenames=50,maxnames=50]{biblatex}
\renewbibmacro{in:}{}
\DeclareFieldFormat{pages}{#1}

\begin{document}

\noindent
\large\textbf{Riemann surfaces, HW 3} \hfill \textbf{Aidan Backus} \\
% --------------------------------------------------------------
%                         Start here
% --------------------------------------------------------------\

\begin{exer}[Forster 13.2]
Show that $\mathcal U = (\CC, \PP^1 \setminus 0)$ is an acyclic cover of $\PP^1$ with respect to the sheaf $\Omega$ of holomorphic $1$-forms.
Show that $H^1(\PP^1, \Omega) \cong \CC$ and that $dz/z$ drops to a generator of $H^1(\PP^1, \Omega)$.
\end{exer}

We must show $H^1(\CC, \Omega) = H^1(\PP^1 \setminus 0, \Omega) = 0$.
Now $\PP^1 \setminus 0$ is isomorphic to $\CC$ via the map $f(z) = 1/z$, which induces an isomorphism
$$f^*: H^1(\CC, \Omega) \to H^1(\PP^1 \setminus 0, \Omega).$$
So it suffices to show $H^1(\CC, \Omega) = 0$.
But if $\omega$ is a holomorphic $1$-form on $\CC$, then $\omega$ has a primitive; therefore $d$ induces a short exact sequence of sheaves
$$0 \to \CC \to \Olo \to \Omega \to 0$$
which gives a long exact sequence in cohomology
$$\cdots \to 0 = H^1(\CC, \Olo) \to H^1(\CC, \Omega) \to H^2(\CC, \CC) = 0 \to \cdots$$
and hence $H^1(\CC, \Omega) = 0$.
Therefore $\mathcal U$ is an acyclic cover.

Now let $\omega \in Z^1(\mathcal U, \Omega)$ be a cocycle.
Thus $\omega$ is given by a holomorphic $1$-form $\omega_{12} \in H^0(\CC \setminus 0, \Omega)$.
In particular $\omega_{12}$ has a Laurent series
$$\omega_{12} = \sum_{n=-\infty}^\infty c_n z^n ~dz = \sum_{n=-\infty}^{-2} c_n z^n~dz + c_{-1} \frac{dz}{z} + \sum_{n=0}^\infty c_n z^n~dz = \alpha + \beta + c_{-1} \frac{dz}{z}$$
where $\alpha \in H^0(\PP \setminus 0, \Omega)$ and $\beta \in H^0(\CC, \Omega)$ are both closed $1$-forms.
Therefore $\alpha,\beta$ are exact $1$-forms, say $\alpha = df$, $\beta = dg$.
Therefore $\omega_{12} - c_{-1}dz/z$ is a coboundary.
On the other hand, $c_{-1}dz/z$ is not a coboundary, since it is not an exact $1$-form.
This implies the claims.

\begin{exer}[Forster 15.2]
Show that the sequence
$$0 \to \CC^* \to \Olo^* \to \Omega \to 0$$
given by $f \mapsto (d \log)f = df/f$ is exact.
\end{exer}

Exactness is a stalk-local statement so we can check on a small disc.
Clearly the map $\CC^* \to \Olo^*$ is injective and is annihilated by $f \mapsto df/f$.
Conversely if $f \in \Olo^*$ satisfies $df/f = 0$ then $df = 0$ so $f \in \CC^*$.
Moreover, if $\omega \in \Omega$ then locally $\omega$ has a primitive $g$, and we can set $f = e^g$.
Thus $(d \log)f = dg = \omega$, which gives exactness at $\Omega$.

\begin{exer}[Forster 15.3]
Let $\mathscr D$ be the sheaf of meromorphic $1$-forms that have residue $0$ everywhere.
Show that
$$0 \to \CC \to \mathscr M \to \mathscr D \to 0$$
given by $d$ is exact.
\end{exer}

Exactness at $\CC$ follows because the map $\CC \to \mathscr M$ is obviously injective.
Exactness at $\mathscr M$ follows because the kernel of $d$ consists of constant functions.
For exactness at $\mathscr D$, we must show that every meromorphic function has a derivative whose residue is identically $0$, and conversely that every such meromorphic $1$-form is a derivative.
To see the former, if $\omega$ is a residue-free meromorphic $1$-form, then locally
$$\omega = \sum_{n=-N}^{-1} c_n z^n ~dz + \alpha$$
where $\alpha$ is a holomorphic $1$-form. In that case we can set
$$f(z) = \sum_{n=-N+1}^0 c_{n-1} z^n + g(z)$$
where $g$ is a primitive of $\alpha$, and then $df = \omega$.
Conversely, if $\omega$ is the derivative of a holomorphic function, then the integral of $\omega$ around any closed curve is $0$, so $\omega$ is residue-free by the residue theorem.

\begin{exer}[Forster 15.4]
Let $X$ be an elliptic curve. Show that $H^1(X, \CC) \cong \CC^2$ and $dz,d\overline z$ form a basis of $H^1(X, \CC)$.
\end{exer}

We know $\pi_1(X) = \ZZ * \ZZ$ and so $H_1(X, \ZZ) = \ZZ^2$.
Then
$$H^1(X, \CC) \cong \Hom(H_1(X, \ZZ), \CC) = \CC^2$$
by the de Rham isomorphism.
So it suffices to show that $dz,d\overline z$ are linearly independent in $H^1(X, \CC)$.

Suppose $dz + a~d\overline z$ is a coboundary, where $a \in \CC$.
Then there is $f \in C^\infty(X)$ such that $df = dz + a~d\overline z$.
Using $dz = dx +i~dy$ we conclude that $\partial_x f = a + 1$ and $\partial_y f = -i(a - 1)$.
Fix $y$; then $\{(x, y)\}_x$ is a circle.
We know that by the periodicity, there is no solution to $\partial_x f = a + 1$ on a circle unless $a + 1 = 0$, so $a = -1$.
Therefore $\partial_y f = 2i$.
Fix $x$; then $\{(x, y)\}_y$ is a circle, and by the same argument we conclude $2i = 0$, a contradiction.

\begin{exer}[Forster 16.1]
Let $D$ be a divisor on $\PP^1$. Prove that
$$\begin{cases}
\dim H^0(\PP^1, \Olo_D) = \max(0, 1 + \deg D)\\
\dim H^1(\PP^1, \Olo_D) = \max(0, -1 - \deg D).
\end{cases}$$
\end{exer}

By the Riemann-Roch theorem,
$$\dim H^0(\PP^1, \Olo_D) - \dim H^1(\PP^1, \Olo_D) = 1 + \deg D$$
so it suffices to check the claim for $\dim H^0(\PP^1, \Olo_D)$.
If $\deg D < 0$ then $\dim H^0(\PP^1, \Olo_D) = 0$, which gives the claim.

Otherwise, $\deg D \geq 0$, say $D = \sum_x D_x x$. We shall prove
$$\dim H^0(\PP^1, \Olo_D) = 1 + \deg D$$
by induction on $\deg D$. Set $D = \sum_{y \in \PP^1} D_y y$.
After a M\"obius transformation we may assume that $D_\infty = 0$.
Recall that $H^0(\PP^1, \Olo_D) = \Olo_D(\PP^1)$.

If $\deg D = 0$ then set
$$f(x) = \prod_{y \in \PP^1} (x - y)^{-D_y},$$
where the product has finitely many factors since $\PP^1$ is compact, and the empty product is by definition $1$.
Then $f \in \Olo_D(\PP^1)$ so $\dim \Olo_D(\PP^1) \geq 1$.
Conversely if $g \in \Olo_D(\PP^1)$ is nonzero then the divisor of $g$ must be $D$ (for if not, then the divisor of $g$ would have degree $<0$ and so $g = 0$).
But this exactly implies that $g$ is a scalar multiple of $f$, so $\dim \Olo_D(\PP^1) \leq 1$.

If $\deg D = n + 1$, let $D' < D$ have one fewer pole than $D$.
Then $\deg D' = n$ and we obtain a splitting
$$\Olo_D(\PP^1) = \Olo_{D'}(\PP^1) \oplus V.$$
Then $V$ is exactly the space of meromorphic functions $\PP^1 \to \PP^1$ whose divisor is $D$.
If
$$f(x) = \prod_{y \in \PP^1} (x - y)^{-D_y},$$
then $V$ is spanned by $f$. So $\dim V = 1$ and this implies the claim by induction.


\begin{exer}[Forster 16.2]
Let $X$ be an elliptic curve, $x_0 \in X$, and $P$ the divisor which is $1$ at $x_0$ and $0$ elsewhere. Show that
$$\dim H^0(X, \Olo_{nP}) = \begin{cases}0, &n < 0\\
1, &n = 0\\
n, &n \geq 1
\end{cases}.$$
\end{exer}

If $n < 0$ then we immediately have $\dim H^0(X, \Olo_{nP}) = 0$.
Similarly $H^0(X, \Olo_0) = H^0(X, \Olo) = \CC$.
So we may restrict to the case $n \geq 1$.
In that case we use the fact that $\wp'$ generates the function field of $X$, so that $H^0(X, \Olo_{nP})$ is spanned by $\wp', (\wp')^2, \dots, (\wp')^n$.

\begin{exer}[Forster 16.4]
Let $\mathscr D$ be the sheaf of divisors. Show that $\mathscr D$ actually is a sheaf and $H^1(X, \mathscr D) = 0$.

Let $\Mero^* \to \mathscr D$ be the map which sends a nonvanishing meromorphic function to its divisor. Show that
$$0 \to \Olo^* \to \Mero^* \to \mathscr D \to 0$$
is a short exact sequence, so there is an exact sequence
$$0 \to H^0(X, \Olo^*) \to H^0(X, \Mero^*) \to \Div(X) \to H^1(X, \Olo^*) \to H^1(X, \Mero^*) \to 0.$$
\end{exer}

Let $V \subseteq U$ be open sets.
We have a restriction map $\mathscr D(U) \to \mathscr D(V)$, which is well-defined because if $D$ is a divisor on $U$, then for every compact set $K \subseteq V$, $K$ is compact in $U$, and so $D|K$ has finite support; thus the same remains true in $V$.
Obviously this restriction map commutes with inclusions, so $\mathscr D$ is a presheaf.
Since divisors are determined by their values on points, $\mathscr D$ is separated.

Let $U, V$ be open sets and $D, E$ be divisors on $U, V$ respectively such that $D|U \cap V = E|U \cap V$.
Then we can define $C$, a $\ZZ$-valued function on $U \cup V$, by setting $C|U = D$ and $C|V = E$.
If $K \subseteq U \cup V$ is compact, then $K$ can be written as the union of two compact sets $L \subseteq U$ and $M \subseteq V$.
Then $C|L$ and $C|M$ have finite support so $C|K$ has finite support.
Therefore $C$ is a divisor. So $\mathscr D$ is a sheaf.

Now we show that $\mathscr D$ is acyclic.
Let $\mathcal U$ be an open cover; we show that there is a refinement of $\mathcal U$ with $H^1(\mathcal U, \mathscr D) = 0$.
We can tile $X$ by sets $K \in \mathcal K$ such that for each $U \in \mathcal U$ there is a $K \in \mathcal K$, and then set $\psi_K = 1$ to be the indicator function of $K$.
We can then enumerate the functions $\psi_K$, after refining $\mathcal U$ so that it is countable.

Let $D$ be a cocycle with respect to $\mathcal U$.
Then $\psi_j D_{ij}$ extends to a divisor on $X$, and in particular extends to $U_i$.
Thus we may define
$$C_i = \sum_{j=1}^\infty \psi_j D_{ij},$$
so $C_i$ is a divisor on $U_i$.
In particular, $C_i - C_j = D_{ij}$, so $D$ is a coboundary.
Therefore $\mathscr D$ is acyclic.

Exactness at $\Olo^*$ was already proven.
Moreover $f \in \Olo^*$ iff $f$ has no poles or zeroes; that is, $(f) = 0$, so we have exactness at $\Mero^*$.
For exactness at $\mathscr D$, fix $x \in X$ and assume $D$ is the germ of a divisor at $x$.
Then there is an open set $U$ contained in a compact set contained in the domain of an honest divisor $D'$ which restricts to $D$, so $D'|U$ has finite support and the germ of $D'|U$ is still $D$.
Therefore we can extend $D$ to an honest divisor $D = \sum_{x \in U} d_x x$ with finite support on $U$, and define
$$f(y) = \prod_{x \in U} (y - x)^{d_x}.$$
Then $f$ is a nonvanishing meromorphic function on $U$, whose definition makes sense as $D$ has finite support and so all but finitely many of the factors will be $1$, and which satisfies $(f) = D$.

The zigzag lemma and the fact that $H^0(X, \mathscr D) = \Div(X)$ now gives the desired long exact sequence.

\begin{exer}[Forster 29.1]
Let $E \to X$ be a holomorphic vector bundle of rank $n$.

Let $f$ be a nonvanishing holomorphic section of $E$. Given $x \in X$ let $F_x$ be the span of $f(x)$ in $E_x$.
Show that $F$ is a holomorphic subbundle of $E$ of rank $1$.

Let $f$ be a meromorphic section of $E$. Show that there is a unique subbundle $F$ of $E$ of rank $1$ such that $F$ is a meromorphic section of $E$.
\end{exer}

Let $f$ be a nonvanishing holomorphic section of $E$.
Let $\psi: U \to V$ be a linear chart of $E$.
Then $\psi$ induces an isomorphism $\psi: E_U \to V \times \CC^n$ where $V \subseteq \CC$ is an open set, and thus a nonvanishing section of the trivial bundle $V \times \CC^n \to V$, which can be identified with a map $f_\psi: V \to \CC^n$.
Choose $x \in U$ and a basis $e_1, \dots, e_n$ of $\CC^n$ so that $f_\psi(x)$ is in the span of $e_1$.
Then the set of points $y \in U$ such that $f_\psi(y)$ is not in the span of $e_2, \dots, e_n$ is an open set $W$ which contains $x$, since the span of $e_2, \dots, e_n$ is closed.
Therefore after restricting $\psi$ to $W$, we obtain a linear chart in which $F$, the span of $f_\psi$, is a holomorphic subbundle of $E$, as desired.

Now let $f$ be a meromorphic section of $E$.
Let $Z$ be the set of zeroes and poles of $f$; then $Z$ is discrete.
Arguing as above, away from $Z$ we can find $F$ on which $f$ is a nonvanishing holomorphic section.
On the other hand, if $F'$ is also a line bundle for which $f$ is a nonvanishing holomorphic section, then at every $x \notin Z$, $F_x'$ is the span of $f(x)$, but this implies $F_x' = F_x$.
Therefore $F$ exists and is unique away from $Z$.

Close to $Z$, we have a local isomorphism which allows us to assume that $F$ is the trivial holomorphic bundle $(\DD \setminus 0) \times \CC$, which extends to the trivial bundle $\DD \times \CC$.
Therefore we can extend $F$ over $\DD$.

\begin{exer}[Forster 29.2]
Omitted.
\end{exer}

This exercise was already in Homework 2.

\begin{exer}
Show that the moduli space of line bundles on $X$ up to equivalence is canonically isomorphic to $H^1(X, \Olo^*)$.
\end{exer}

Given a line bundle $L$, let us write $[L]$ to mean its equivalence class, and similarly if $g$ is a $1$-cocycle of $\Olo^*$, let us write $[g]$ for its equivalence class.
Then we let $\psi([L])$ for the equivalence class of the cocycle defined by transition maps for $L$.

We first check that $\psi([L])$ is well-defined.
Let $L$ be a line bundle, and let $\mathscr A$ be a linear cover for $L$.
Then we obtain transition maps $g$, which define a cocycle of $\Olo^*$.
Suppose that $[L'] = [L]$. Then, possibly after refining $\mathscr A$, we may assume that $\mathscr A$ is a linear cover for $L'$, with transition maps $g'$.
In that case, $g = g'$, so $\psi([L]) = \psi([L'])$.

Before we show that $\psi$ is injective, we first show that it is a group homomorphism.
Indeed, if $L,L'$ are line bundles with transition cocycles $g,g'$, then the transition cocycle of $L \otimes L'$ is $gg'$.
So $\psi([L \otimes L']) = \psi([L]) \otimes \psi([L'])$, so to show that $\psi$ is injective we just need to show that its kernel consists of the equivalence class of the trivial bundle.

To see that $\psi$ is injective, suppose that $L$ is a line bundle with transition maps $g$ which form a coboundary with respect to the linear cover $(U_i)$.
Since they are a coboundary, there is a \emph{global} trivialization $h$ with $g_{ij} = h_i/h_j$, implying that $[L] = 0$.





\end{document}
