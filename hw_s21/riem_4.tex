% --------------------------------------------------------------
% This is all preamble stuff that you don't have to worry about.
% Head down to where it says "Start here"
% --------------------------------------------------------------
\documentclass[10pt]{article}

\usepackage[margin=.7in]{geometry}
\usepackage{amsmath,amsthm,amssymb,mathrsfs}
\usepackage{enumitem}
\usepackage{tikz-cd}
\usepackage{mathtools}
\usepackage{amsfonts}
\usepackage{listings}
\usepackage{algorithm2e}
\usepackage{verse,stmaryrd}
\usepackage{fancyvrb}

% Number systems
\newcommand{\NN}{\mathbb{N}}
\newcommand{\ZZ}{\mathbb{Z}}
\newcommand{\QQ}{\mathbb{Q}}
\newcommand{\RR}{\mathbb{R}}
\newcommand{\CC}{\mathbb{C}}
\newcommand{\PP}{\mathbb P}
\newcommand{\FF}{\mathbb F}
\newcommand{\DD}{\mathbb D}
\renewcommand{\epsilon}{\varepsilon}

\newcommand{\Aut}{\operatorname{Aut}}
\newcommand{\cl}{\operatorname{cl}}
\newcommand{\ch}{\operatorname{ch}}
\newcommand{\Con}{\operatorname{Con}}
\newcommand{\coker}{\operatorname{coker}}
\newcommand{\CVect}{\CC\operatorname{-Vect}}
\newcommand{\Cantor}{\mathcal{C}}
\newcommand{\D}{\mathcal{D}}
\newcommand{\card}{\operatorname{card}}
\newcommand{\dbar}{\overline \partial}
\newcommand{\diam}{\operatorname{diam}}
\newcommand{\dom}{\operatorname{dom}}
\newcommand{\End}{\operatorname{End}}
\DeclareMathOperator*{\esssup}{ess\,sup}
\newcommand{\GL}{\operatorname{GL}}
\newcommand{\Hom}{\operatorname{Hom}}
\newcommand{\id}{\operatorname{id}}
\newcommand{\Ind}{\operatorname{Ind}}
\newcommand{\Inn}{\operatorname{Inn}}
\newcommand{\interior}{\operatorname{int}}
\newcommand{\lcm}{\operatorname{lcm}}
\newcommand{\mesh}{\operatorname{mesh}}
\newcommand{\LL}{\mathcal L_0}
\newcommand{\Leb}{\mathcal{L}_{\text{loc}}^2}
\newcommand{\Lip}{\operatorname{Lip}}
\newcommand{\ppGL}{\operatorname{PGL}}
\newcommand{\ppic}{\vspace{35mm}}
\newcommand{\ppset}{\mathcal{P}}
\DeclareMathOperator{\proj}{proj}
\DeclareMathOperator*{\Res}{Res}
\newcommand{\Riem}{\mathcal{R}}
\newcommand{\RVect}{\RR\operatorname{-Vect}}
\newcommand{\Sch}{\mathcal{S}}
\newcommand{\SL}{\operatorname{SL}}
\newcommand{\sgn}{\operatorname{sgn}}
\newcommand{\spn}{\operatorname{span}}
\newcommand{\Spec}{\operatorname{Spec}}
\newcommand{\supp}{\operatorname{supp}}
\newcommand{\TT}{\mathcal T}
\DeclareMathOperator{\tr}{tr}

% Calculus of variations
\DeclareMathOperator{\pp}{\mathbf p}
\DeclareMathOperator{\zz}{\mathbf z}
\DeclareMathOperator{\uu}{\mathbf u}
\DeclareMathOperator{\vv}{\mathbf v}
\DeclareMathOperator{\ww}{\mathbf w}

% Categories
\newcommand{\Ab}{\mathbf{Ab}}
\newcommand{\Cat}{\mathbf{Cat}}
\newcommand{\Group}{\mathbf{Group}}
\newcommand{\Module}{\mathbf{Module}}
\newcommand{\Set}{\mathbf{Set}}
\DeclareMathOperator{\Fun}{Fun}
\DeclareMathOperator{\Iso}{Iso}

% Complex analysis
\renewcommand{\Re}{\operatorname{Re}}
\renewcommand{\Im}{\operatorname{Im}}
\newcommand{\Div}{\operatorname{Div}}

% Logic
\renewcommand{\iff}{\leftrightarrow}
\newcommand{\Henkin}{\operatorname{Henk}}
\newcommand{\PA}{\mathbf{PA}}
\DeclareMathOperator{\proves}{\vdash}

% Group
\DeclareMathOperator{\Gal}{Gal}
\DeclareMathOperator{\Fix}{Fix}
\DeclareMathOperator{\Out}{Out}

\newcommand{\Mero}{\mathscr M}
\newcommand{\Olo}{\mathscr O}

% Other symbols
\newcommand{\heart}{\ensuremath\heartsuit}

\DeclareMathOperator{\atanh}{atanh}

% Theorems
\theoremstyle{definition}
\newtheorem*{corollary}{Corollary}
\newtheorem*{falselemma}{Grader's ``Lemma"}
\newtheorem{exer}{Exercise}
\newtheorem{lemma}{Lemma}[exer]
\newtheorem{theorem}[lemma]{Theorem}

\usepackage[backend=bibtex,style=alphabetic,maxcitenames=50,maxnames=50]{biblatex}
\renewbibmacro{in:}{}
\DeclareFieldFormat{pages}{#1}

\begin{document}

\noindent
\large\textbf{Riemann surfaces, HW 4} \hfill \textbf{Aidan Backus} \\
% --------------------------------------------------------------
%                         Start here
% --------------------------------------------------------------\

I talked extensively with Trent and Steven about these problems.

\begin{exer}[Forster 17.4]
Let $D$ be a divisor on a compact Riemann surface $X$ of genus $g$. Show that
\begin{align*}
h^0(X, \Olo_D) &=0, &\deg D \leq -1\\
0 \leq h^0(X, \Olo_D) &\leq 1 + \deg D, &-1 \leq \deg D \leq g - 1\\
1 - g + \deg D \leq h^0(X, \Olo_D) &\leq g, &g - 1 \leq \deg D \leq 2g - 1\\
h^0(X, \Olo_D) &= 1 - g + \deg D, &\deg D \geq 2g - 1.
\end{align*}
\end{exer}

The case $\deg D \leq -1$ is obvious, since $\Olo_D$ is the space of holomorphic functions with zeroes, but since $X$ is compact the only such functions are the zero function.
This implies that if $\deg D \geq 2g - 1$, then $h^0(X, \Olo_{K - D}) = 0$ (since $\deg(K - D) \leq (2g - 2) - (2g - 1) = -1$) and hence the case $\deg D \geq 2g - 1$ immediately follows from the Riemann-Roch theorem
$$h^0(X, \Olo_D) - h^0(X, \Olo_{D - K}) = 1 - g + \deg D$$
with $h^0(X, \Olo_{K - D}) = 0$.

We now claim that if $\deg D \geq -1$ then $h^0(X, \Olo_D) \leq \deg D + 1$.
So let $f \in H^0(X, \Olo_D)$. Subtracting off the principal parts of $f$ at each pole, we get a constant term $[f] \in \CC$.
Modulo $[f]$, $f$ is determined by the coefficients $c_j(x)$ by its principal parts $\sum_{j=-D(x)}^{-1} c_j(x) z^j$ at each pole $x$. Here $z$ is a coordinate centered at $x$.
Thus $H^0(X, \Olo_D)$ is a quotient of a $\deg D + 1$-dimensional vector space, with one basis vector for each of the $D(x)$ coefficients at each $x \in \supp D$ plus a basis vector for the constants, and hence $h^0(X, \Olo_D) \leq \deg D + 1$
This immediately solves the case $-1 \leq \deg D \leq g - 1$.

Now to finish out the proof, we need to treat the case $g - 1 \leq \deg D \leq 2g - 1$.
In that case, $-1 \leq \deg(K - D) \leq g - 1$, so that by the previous case,
$$0 \leq h^0(X, \Olo_{K - D}) \leq 1 + \deg D.$$
Plugging this into the Riemann-Roch theorem solves the case $g - 1 \leq \deg D \leq 2g - 1$.

\begin{exer}[Forster 17.5]
Let $K$ be a canonical divisor on a compact Riemann surface of genus $> 0$ and let $D > K$ be a divisor with degree $\deg D - \deg K = 1$.
Show that $\Olo_K$ is globally generated but $\Olo_D$ is not.
\end{exer}

Let $D = K + p$. Since $\deg D = 2g - 1$, $h^1(X, \Olo_D) = 0$.
It follows from the Riemann-Roch theorem that
$$h^0(X, \Olo_D) = 1 + (2g - 1) - g = g.$$
On the other hand, since $h^0(X, \Olo_K) = g$,
$$h^0(X, \Olo_D) - h^0(X, \Olo_K) = 0,$$
and $g > 0$, there cannot exist $f \in H^0(X, \Olo_D) \setminus H^0(X, \Olo_K)$; but this is exactly what global generation at $p$ means.
Therefore $\Olo_D$ is not globally generated.

To show that $K$ is globally generated, let $p \in X$ and $D = K - p$.
Then
$$h^0(X, \Olo_K) = 1 - g + (2g - 2) + h^1(X, \Olo_K) = 1$$
as $h^1(X, \Olo_K) = 1$ by Serre duality.
Here the argument breaks into two cases, using the fact that $K - D = p$.

First suppose that $X$ is an elliptic curve; then $g = 1$ and there are no functions with just one pole on $X$, so $h^0(X, \Olo_p) = h^0(X, \Olo) = 1$.
Thus
$$h^0(X, \Olo_D) = 1 - g + (2g - 3) + h^0(X, \Olo_{K - D}) = 1 - g + (2g - 3) + 1 = 0.$$
But $h^0(X, \Olo_K) = 1 > 0$, so $K$ is globally generated at $p$.

Now suppose that $X$ is not an elliptic curve. Therefore $g \geq 2$, and by the previous part, $h^0(X, \Olo_p) \leq 2$. (Actually, $h^0(X, \Olo_p) = 1$, but this takes a bit more work, and I'm very lazy.)
Therefore
$$h^0(X, \Olo_D) = 1 - g + (2g - 3) + h^0(X, \Olo_{K - D}) = g - 2.$$
Since $g > 0$,
$$h^0(X, \Olo_K) = g > g - 2 = h^0(X, \Olo_D)$$
and again $K$ is globally generated at $p$.

\begin{exer}[Forster 17.7]
Let $X$ be a curve of genus $2$, and suppose that $H^0(X, \Omega)$ is generated by $\omega_1,\omega_2$.
Let $f = \omega_1/\omega_2$. Show that
$$f: X \to \PP^1$$
is a branched double cover.
\end{exer}

Let $K$ be the canonical divisor of $X$ with no poles; then $\Omega = \Olo_K$.
Furthermore, $K$ is globally generated by a previous exercise, so there does not exist $x \in X$ such that every global section of $\Olo_K$ vanishes at $x$; in particular, if $\omega_1(x) = 0$ then $\omega_2(x) \neq 0$.
On the other hand, $\deg K = 2$ since $X$ is a curve of genus $2$, and since $K$ has no poles, it follows that sections of $\Olo_K$ must have exactly two zeroes counting multiplicity.
Since $X$ is compact, $f$ is a proper morphism, so the preimage of a single point is enough to compute $\deg f$.
Therefore $\deg f = 2$.

By the way, I think this argument shows that if there is a globally generated divisor $D$ of degree $d$ on $X$ such that $h^0(X, \Olo_D) \geq 2$, then any two-dimensional subspace of $h^0(X, \Olo_D)$ determines a degree-$d$ cover of $\PP^1$.
So either covers of $\PP^1$ are abundant or globally generated divisors with at least two sections are rare; I'm not sure which.

\begin{exer}[Forster 19.1]
Let $X$ be a compact Riemann surface. Show that $d\mathscr E^{0,1}(X) = d'd'' \mathscr E(X)$.

Let $\mathscr H$ be the sheaf of harmonic functions on $X$. Show that
$$H^1(X, \mathscr H) \cong \mathscr E^2(X)/d'd'' \mathscr E(X) \cong \CC.$$

Let $\omega$ be a global $2$-form. Show that there is $f \in \mathscr E(X)$ such that $d' d'' f = \omega$ iff $\iint_X \omega = 0$.
\end{exer}

Since
$$\mathscr E^{0,1}(X) = d'' \mathscr E(X) \oplus \overline \Omega(X),$$
we conclude that
$$d\mathscr E^{0,1}(X) = dd'' \mathscr E(X) + d\overline \Omega(X).$$
However, if $\omega \in \overline \Omega(X)$, then we can write $\omega = \sum_j \overline f_j ~d\overline g_j$ where the $f_j,g_j$ are holomorphic.
So
$$d\omega = \sum_{j=1}^J d\overline f_j \wedge d\overline g_j = \sum_{j=1}^J \frac{\partial f_j}{\partial \overline z} ~d\overline z \wedge \frac{\partial g_j}{\partial \overline z} ~d\overline z = 0$$
and hence $d\overline \Omega(X) = 0$. On the other hand, $dd'' = d'd''$. This completes the proof that
$$d\mathscr E^{0,1}(X) = d'd'' \mathscr E(X).$$

We now consider the short exact sequence
$$0 \to \mathscr H \to \mathscr E \to \mathscr E^2 \to 0$$
where $d'd'': \mathscr E \to \mathscr E^2$. To see why this is in fact exact, we first note that $\mathscr H$ is clearly the kernel of $d'd''$, and besides that, in a simply connected open set, $\omega \in \mathscr E^2$ can be written as $\omega = d'd''f$, since in that case the first and second cohomologies vanish.
Since $\mathscr E$ is an acyclic sheaf, this short exact sequence induces a resolution
$$\cdots \to \mathscr E(X) \to \mathscr E^2(X) \to H^1(X, \mathscr H) \to 0$$
of $H^1(X, \mathscr H)$, which gives the isomorphism
$$H^1(X, \mathscr H) \cong \frac{\mathscr E^2(X)}{d'd''\mathscr E(X)}.$$
Furthemore we claim
$$H^2(X, \CC) \cong \frac{\mathscr E^2(X)}{d'd''\mathscr E(X)}.$$
To see this, it suffices to show $d'd'' \mathscr E(X) = d\mathscr E^1(X)$.
But $\mathscr E^1(X) = \mathscr E^{1, 0}(X) \oplus \mathscr E^{0, 1}(X)$.
We have already shown that $d\mathscr E^{0,1}(X) = d'd'' \mathscr E(X)$; the proof for $\mathscr E^{1,0}(X)$ is similar since $d'd'' + d''d' = 0$.
Since $H^2(X, \CC)$ consists only of multiples of the volume form on $X$, $H^2(X, \CC) \cong \CC$.

We now observe that clearly $\iint_X d'd''f = 0$ by Stokes' theorem.
For the converse, suppose $\iint_X \omega = 0$.
Then there is $\eta \in \mathscr E^1(X)$ with $d\eta = \omega$.
Write $\eta = \eta_1 + \eta_2$ where $\eta_1 \in \mathscr E^{1,0}(X)$ and $\eta_2 \in \mathscr E^{0,1}(X)$.
Then $\eta_1 = d'f_1 + \theta_1$ where $f_1 \in \mathscr E(X)$ and $d\theta_1 = 0$, and similarly $\eta_2 = d''f_2 + \theta_2$.
Thus
$$\omega = dd'f_1 + dd''f_2 + d(\theta_1 + \theta_2) = d''d' f_1 + d'd'' f_2.$$
As $d''d' + d'd'' = 0$ we see the claim.

\begin{exer}[Forster 19.2]
Let $X$ be an elliptic curve. For every $f \in \mathscr E(X)$ define the mean value $Mf$ by
$$Mf = \left(\iint_X f ~dz \wedge d\overline z\right)\left(\iint_X dz \wedge d\overline z\right)^{-1}.$$
If $\omega = f~dz + g~d\overline z$ let $M\omega = Mf ~dz + Mg ~d\overline z$.

If $\omega$ is a global $1$-form with $d\omega = 0$, show that $[\omega] = [M\omega]$. Show that the mapping $M$ from the kernel of $d$ to $\text{Harm}^1(X)$ is induces an isomorphism $H^1(X, \CC) \to \text{Harm}^1(X)$, where $H^1(X, \CC)$ is in the sense of de Rham.
\end{exer}

Consider the algebraic surface $X \times X$.
A basis for $H^1(X \times X, \CC)$ is $dz, dw, d\overline z, d\overline w$.
The dual basis of $H_1(X \times X, \CC)$ can be denoted $\gamma_1, \overline \gamma_1, \gamma_2, \overline \gamma_2$; then $\int_{\gamma_1} ~d\overline z = 0$ and similarly.
So if $f$ is a simple function, $\int_\gamma f~d\overline z = \int_{\overline \gamma} f ~dz = 0$; therefore the same happens to general $f$, by dominated convergence.
Furthermore, the cohomology class of $\omega$ is determined by $\int_\gamma \omega$ and $\int_{\overline \gamma} \omega$.
Let $dV = dz \wedge d\overline z$.
By Fubini's theorem,
\begin{align*}
\int_\gamma M\omega &= \int_{\gamma_1} Mf ~dz = \int_{\gamma_1} \iint_X f(w) ~dw \wedge d\overline w ~dz\\
&= \iint_X \int_{\gamma_1} f(w) ~dz ~dw \wedge d\overline w = \int_{\gamma_2} \int_{\overline \gamma_2} \int_{\gamma_1} f(w) ~dz ~dw ~d\overline w\\
&= \int_{\gamma_2} f(w) ~dw \int_{\gamma_1} ~dz \int_{\overline \gamma_1} ~d\overline z = \int_{\gamma_2} f(w) ~dw \iint_X ~dV \\
&= V(X) \int_\gamma \omega.
\end{align*}
Similarly for $\overline \gamma$.

Furthermore, $M\omega$ is harmonic for every $\omega$, since $M\omega$ is a constant $dz$, which is holomorphic, plus a constant $d\overline z$, which is antiholomorphic.
We just showed that if $M\omega = 0$ then $[\omega] = 0$, so $\omega$ is exact.
We conclude that $M$ drops to a map on cohomology
$$M: H^1(X, \CC) \to \text{Harm}^1(X)$$
which is surjective since a form which is constant $dz$ plus constant $d\overline z$ is mapped to itself.
Since we have already shown (via the Hodge theorem) the existence of \emph{some} isomorphism $H^1(X, \CC) \to \text{Harm}^1(X)$, and $H^1(X, \CC)$ is finite-dimensional since $X$ is an elliptic curve and therefore is compact, every surjective linear map between them must also be an isomorphism.
So $M$ is an isomorphism.

\begin{exer}[Forster 21.1]
Let $X$ be a compact Riemann surface, $Y \subseteq X$ an open set such that $X \setminus Y$ has nonempty interior.
Let $D$ be a divisor on $X$. Show that there is a nontrivial meromorphic function $f$ on $X$ such that $(f)|Y = D|Y$.
\end{exer}

If $g = 0$ then $X = \PP^1$ and $Y$ is isomorphic to a bounded open subset of $\CC$.
In that case, if $D|Y = \sum_j c_j y_j$, we can just set
$$f(z) = \sum_j (z - y_j)^{c_j}.$$
Henceforth we assume $g \geq 1$.

We now need the following analytic lemma. Let $C_k(Y, \CC)$ be the $k$th chain group in $Y$ with coefficients in $\CC$, and let $[\alpha, \beta]$ denote any path from $\alpha$ to $\beta$.
\begin{lemma}
Let $f_0, \dots, f_{g - 1}$ be holomorphic functions on the unit disc $\DD$ such that the derivatives $f_j'$ are linearly independent.
Then for every $w \in \CC^g$ there is $\gamma \in C_1(\DD, \CC)$ such that for every $j < g$, $\int_\gamma df_j = w_j$.
\end{lemma}
\begin{proof}
Let $\zeta \in \DD^g$ be a parameter to be decided later.
Define a linear map $S: C_1(X, \CC) \to \CC^g$, by
$$S\rho = \left(\int_\rho df_0, \dots, \int_\rho df_{g - 1}\right),$$
and use the parameter $\zeta$ to define $\rho_0, \dots, \rho_{g - 1} \in C_1(\DD, \CC)$ by $\rho_j = [0, \zeta_j]$.
Then $S$ restricts to a linear map $T: \CC[\rho_0, \dots, \rho_{g - 1}] \to \CC^g$.
Thus in particular
$$T_{jk} = f_j(\zeta_k) - f_j(0).$$
We must show that there is a parameter $\zeta$ such that $\det T$ is nonzero; then for every $w$ we can let $\gamma = T^{-1}(w)$.
Since the $f_j'$ are linearly independent, $F(\zeta) = \det T$ is a nondegenerate holomorphic map $\DD^g \to \CC$, so the set $Z$ of zeroes of $F$ is a complex submanifold of $\DD^g$ of codimension $1$.
In particular, the complement of $Z$ is nonempty, as desired.
\end{proof}

Suppose that $D|Y = \sum_j c_j y_j$, let $B_1 \subseteq X \setminus Y$ be a ball and $b \in B_1$, and consider the $1$-chain $\gamma_0 = \sum_j c_j [b, y_j]$.
Then let $B_2 \subseteq B_1 \setminus b$ be another ball, so that we can write the basis $\omega_0, \dots, \omega_{g - 1}$ of $H^0(X, \Omega)$ as $\omega_j = df_j$.
By the lemma, we can find $\gamma_1 \in C_1(B_2)$ such that for every $j < g$,
$$\int_{\gamma_1} \omega_j = \int_{\gamma_0} \omega_j.$$
In particular, integration against $\gamma = \gamma_0 - \gamma_1$ annihilates $H^0(X, \Omega)$, so $\partial \gamma$ is a principal divisor and $D|Y = \partial \gamma|Y$.
Therefore if $f$ is a solution to $\partial \gamma$, $f$ is also a solution to the problem posed in the theorem.





\end{document}
