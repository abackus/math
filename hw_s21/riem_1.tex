
% --------------------------------------------------------------
% This is all preamble stuff that you don't have to worry about.
% Head down to where it says "Start here"
% --------------------------------------------------------------

\documentclass[10pt]{article}

\usepackage[margin=.7in]{geometry}
\usepackage{amsmath,amsthm,amssymb}
\usepackage{enumitem}
\usepackage{tikz-cd}
\usepackage{mathtools}
\usepackage{amsfonts}
\usepackage{listings}
\usepackage{algorithm2e}
\usepackage{verse,stmaryrd}
\usepackage{fancyvrb}

% Number systems
\newcommand{\NN}{\mathbb{N}}
\newcommand{\ZZ}{\mathbb{Z}}
\newcommand{\QQ}{\mathbb{Q}}
\newcommand{\RR}{\mathbb{R}}
\newcommand{\CC}{\mathbb{C}}
\newcommand{\PP}{\mathbb P}
\newcommand{\FF}{\mathbb F}
\newcommand{\DD}{\mathbb D}
\renewcommand{\epsilon}{\varepsilon}

\newcommand{\Aut}{\operatorname{Aut}}
\newcommand{\cl}{\operatorname{cl}}
\newcommand{\ch}{\operatorname{ch}}
\newcommand{\Con}{\operatorname{Con}}
\newcommand{\coker}{\operatorname{coker}}
\newcommand{\CVect}{\CC\operatorname{-Vect}}
\newcommand{\Cantor}{\mathcal{C}}
\newcommand{\D}{\mathcal{D}}
\newcommand{\card}{\operatorname{card}}
\newcommand{\dbar}{\overline \partial}
\newcommand{\diam}{\operatorname{diam}}
\newcommand{\dom}{\operatorname{dom}}
\newcommand{\End}{\operatorname{End}}
\DeclareMathOperator*{\esssup}{ess\,sup}
\newcommand{\GL}{\operatorname{GL}}
\newcommand{\Hom}{\operatorname{Hom}}
\newcommand{\id}{\operatorname{id}}
\newcommand{\Ind}{\operatorname{Ind}}
\newcommand{\Inn}{\operatorname{Inn}}
\newcommand{\interior}{\operatorname{int}}
\newcommand{\lcm}{\operatorname{lcm}}
\newcommand{\mesh}{\operatorname{mesh}}
\newcommand{\LL}{\mathcal L_0}
\newcommand{\Leb}{\mathcal{L}_{\text{loc}}^2}
\newcommand{\Lip}{\operatorname{Lip}}
\newcommand{\ppGL}{\operatorname{PGL}}
\newcommand{\ppic}{\vspace{35mm}}
\newcommand{\ppset}{\mathcal{P}}
\DeclareMathOperator{\proj}{proj}
\DeclareMathOperator*{\Res}{Res}
\newcommand{\Riem}{\mathcal{R}}
\newcommand{\RVect}{\RR\operatorname{-Vect}}
\newcommand{\Sch}{\mathcal{S}}
\newcommand{\SL}{\operatorname{SL}}
\newcommand{\sgn}{\operatorname{sgn}}
\newcommand{\spn}{\operatorname{span}}
\newcommand{\Spec}{\operatorname{Spec}}
\newcommand{\supp}{\operatorname{supp}}
\newcommand{\TT}{\mathcal T}
\DeclareMathOperator{\tr}{tr}

% Calculus of variations
\DeclareMathOperator{\pp}{\mathbf p}
\DeclareMathOperator{\zz}{\mathbf z}
\DeclareMathOperator{\uu}{\mathbf u}
\DeclareMathOperator{\vv}{\mathbf v}
\DeclareMathOperator{\ww}{\mathbf w}

% Categories
\newcommand{\Ab}{\mathbf{Ab}}
\newcommand{\Cat}{\mathbf{Cat}}
\newcommand{\Group}{\mathbf{Group}}
\newcommand{\Module}{\mathbf{Module}}
\newcommand{\Set}{\mathbf{Set}}
\DeclareMathOperator{\Fun}{Fun}
\DeclareMathOperator{\Iso}{Iso}

% Complex analysis
\renewcommand{\Re}{\operatorname{Re}}
\renewcommand{\Im}{\operatorname{Im}}

% Logic
\renewcommand{\iff}{\leftrightarrow}
\newcommand{\Henkin}{\operatorname{Henk}}
\newcommand{\PA}{\mathbf{PA}}
\DeclareMathOperator{\proves}{\vdash}

% Group
\DeclareMathOperator{\Gal}{Gal}
\DeclareMathOperator{\Fix}{Fix}
\DeclareMathOperator{\Out}{Out}

% Other symbols
\newcommand{\heart}{\ensuremath\heartsuit}

\DeclareMathOperator{\atanh}{atanh}

% Theorems
\theoremstyle{definition}
\newtheorem*{corollary}{Corollary}
\newtheorem*{falselemma}{Grader's ``Lemma"}
\newtheorem{exer}{Exercise}
\newtheorem{lemma}{Lemma}[exer]
\newtheorem{theorem}[lemma]{Theorem}


\usepackage[backend=bibtex,style=alphabetic,maxcitenames=50,maxnames=50]{biblatex}
\renewbibmacro{in:}{}
\DeclareFieldFormat{pages}{#1}

\begin{document}
\noindent
\large\textbf{Riemann surfaces, HW 1} \hfill \textbf{Aidan Backus} \\
% --------------------------------------------------------------
%                         Start here
% --------------------------------------------------------------\

\begin{exer}[Forster 1.5a]
Let $\Gamma,\Gamma'$ be lattices in $\CC$, and $\alpha \in \CC^*$ satisfies $\alpha \Gamma \subseteq \Gamma'$.
Show that $\alpha$ induces a holomorphic map $\CC/\Gamma \to \CC/\Gamma'$ which is biholomorphic iff $\alpha \Gamma = \Gamma'$.
\end{exer}

A holomorphic map $F: \CC/\Gamma \to \CC/\Gamma'$ lifts to a holomorphic map $F^\sharp: \CC \to \CC$ such that the projection maps $p: \CC \to \CC/\Gamma, p': \CC \to \CC/\Gamma'$ make the diagram
$$\begin{tikzcd}
\CC \arrow[r,"F^\sharp"] \arrow[d,"p"] & \CC \arrow[d,"p'"]\\
\CC/\Gamma \arrow[r,"F"] & \CC/\Gamma'
\end{tikzcd}$$
commute.
In this case, $p' \circ F^\sharp$ is constant on the fibers of $p$; that is, if $z_1 \in z_2 + \Gamma$, then $F^\sharp(z_1) \in F^\sharp(z_2) + \Gamma'$.
To see why the lift exists, we recall that $p$ is a universal cover.
The map $F \circ p$ therefore induces a trivial pushforward in homotopy, so, since $p'$ is a covering space, $F \circ p$ lifts to a map $F^\sharp = \widetilde{F \circ p}: \CC \to \CC$.

Conversely, given a holomorphic map $G: \CC \to \CC$ such that $p' \circ G$ is constant on the fibers of $p$ (thus if $z_1 \in z_2 + \Gamma$ then $G(z_1) \in G(z_2) + \Gamma'$) then we can define $G^\flat: \CC/\Gamma \to \CC/\Gamma'$ to be the unique map to make the diagram
$$\begin{tikzcd}
\CC \arrow[r,"G"] \arrow[d,"p"] & \CC \arrow[d,"p'"]\\
\CC/\Gamma \arrow[r,"G^\flat"] & \CC/\Gamma'
\end{tikzcd}$$
commute. Clearly $G^\flat$ is holomorphic since $p, p'$ are locally biholomorphic and $G$ is holomorphic.
To see that $G^\flat$ is well-defined, we note that the only arbitrary choice in the definition of $G^\flat(z)$, $z \in \CC/\Gamma$, is the lift $z^\sharp \in \CC$ of $z$.
But $p'(G(z^\sharp))$ only depends on $p(z^\sharp)$ so this is no choice at all.

Now let $\alpha$ be given, with $\alpha \Gamma \subseteq \Gamma'$.
Suppose $z_1 \in z_2 + \Gamma$. Then $\alpha z_1 \in \alpha z_2 + \alpha \Gamma \subseteq \alpha z_2 + \Gamma'$.
So $\alpha$ drops to a holomorphic map $\alpha^\flat: \CC/\Gamma \to \CC/\Gamma'$.

Now suppose that $\alpha \Gamma = \Gamma'$.
Then $\alpha^{-1} \Gamma' = \Gamma$, so $\alpha^{-1}$ drops to a holomorphic map $(\alpha^{-1})^\flat$.
We get a commutative diagram
$$\begin{tikzcd}
\CC \arrow[r,"\alpha"] \arrow[d,"p"] & \CC \arrow[r,"\alpha^{-1}"] \arrow[d,"p'"] & \CC \arrow[d,"p"]\\
\CC/\Gamma \arrow[r,"\alpha^\flat"] & \CC/\Gamma' \arrow[r,"(\alpha^{-1})^\flat"] & \CC/\Gamma
\end{tikzcd}$$
where the top row composes to the identity $1$. But $1^\flat$ is the identity on $\CC/\Gamma$, since $G \mapsto G^\flat$ is well-defined, and the identity clearly satisfies the definition of $1^\flat$.
The same thing happens when one instead composes $1 = \alpha \circ \alpha^{-1}$.
Therefore $(\alpha^\flat)^{-1} = (\alpha^{-1})^\flat$.
So $\alpha^\flat$ is biholomorphic.

Conversely, suppose that $\alpha \Gamma \neq \Gamma'$.
We claim that $\alpha^\flat$ is not injective.
Since $\Gamma, \Gamma'$ are subgroups of $\CC$, $0 \in \Gamma \cap \Gamma'$.
Let $z_1 = p(0)$, $w = p'(0)$. Then $\alpha^\flat(z_1) = w$.
Since $\alpha \Gamma \subset \Gamma'$, there is a $z_2^\sharp \notin \Gamma$ such that $\alpha z_2^\sharp \in \Gamma'$, and hence $p'(\alpha z_2^\sharp) = w$.
Let $z_2 = p(z_2^\sharp)$; then $z_1 \neq z_2$ since $z_2^\sharp \notin \Gamma$ but $0 \in \Gamma$, but $\alpha^\flat(z_2) = w$, so $\alpha^\flat$ is not injective.

\begin{exer}[Forster 1.5b]
Show every $\CC/\Gamma$ is isomorphic to one of the form
$$X(\tau) = \frac{\CC}{\ZZ + \tau \ZZ}$$
with $\Im \tau > 0$.
\end{exer}

Let $\Gamma$ be given, and suppose that $\Gamma$ is generated by $\beta_1, \beta_2$.
Then $(\beta_1^{-1})\Gamma$ is gneerated by $1$ and $\beta_2/\beta_1$, and by the previous exercise,
$$(\beta_1^{-1})^\flat: \frac{\CC}{\Gamma} \to \frac{\CC}{(\beta_1^{-1})\Gamma}$$
is an isomorphism. Therefore, without loss of generality, we may assume that $\Gamma = \ZZ + \tau\ZZ$, with $\tau = \beta_2/\beta_1$.
Since $\beta_1,\beta_2$ are linearly independent over $\RR$, so are $1,\tau$; therefore $\Im \tau \neq 0$.
If $\Im \tau < 0$ we can replace $\tau$ with $-\tau$ without affecting $\Gamma$, and thus assume $\Im \tau > 0$.

\begin{exer}[Forster 1.5c]
Let $A \in \SL(2, \ZZ)$ be a M\"obius transformation and $\Im \tau > 0$. Let $\tau' = A\tau$. Show that $X(\tau)$ and $X(\tau')$ are isomorphic.
\end{exer}

First, if $A = \begin{bmatrix}a&b\\c&d\end{bmatrix}$ then
$$A\begin{bmatrix}1\\\tau\end{bmatrix} = \begin{bmatrix}a + b\tau\\c+d\tau\end{bmatrix}.$$
By Forster's exercise 1.4, it follows that $\ZZ + \tau\ZZ = (a+b\tau)\ZZ + (c+d\tau)\ZZ$.
This implies $X(\tau) = \CC/((a+b\tau)\ZZ + (c+d\tau)\ZZ)$.
By exercise 1.5b we have an isomorphism
$$((c+d\tau)^{-1})^\flat: \frac{\CC}{(a+b\tau)\ZZ + (c+d\tau)\ZZ} \to \frac{\CC}{\ZZ + ((a+b\tau)/(c+d\tau))\ZZ} = \frac{\CC}{\ZZ + \tau'\ZZ} = X(\tau').$$

\begin{exer}[Forster 5.4]
Let $\Gamma,\Gamma'$ be lattices, $f: \CC/\Gamma \to \CC/\Gamma'$ a nonconstant holomorphic map, and $f(0) = 0$.
Show that there is a unique $\alpha \in \CC^*$, $F(z) = \alpha z$, making the diagram
$$\begin{tikzcd}
\CC \arrow[r,"F"] \arrow[d,"p"] & \CC \arrow[d,"p'"]\\
\CC/\Gamma \arrow[r,"f"] & \CC/\Gamma'\end{tikzcd}$$
commute, and $\alpha \Gamma \subseteq \Gamma'$. Show that $f$ is a covering space whose deck group is $\Gamma'/\alpha \Gamma$.
\end{exer}

In the notation of our solution to Forster 1.5a, we have already shown that $F = f^\sharp$ exists; this only used the facts that $p$ is a universal cover and $p'$ is a covering space.
Our only freedom in the definition of $F$ is to choose $F(0)$, and we set $F(0) = 0$, which is possible since $f(0) = 0$.
Let $z \in \CC/\Gamma$. If $p(z^\sharp) = z$ then there is a neighborhood $U \ni z^\sharp$ such that $\{U + \gamma\}_{\gamma \in \Gamma}$ evenly covers a neighborhood of $z$.
Therefore $F'|U$ is a translate of $F'|U + \gamma$ for each $\gamma \in \Gamma$, so $F'$ is bounded; so by Liouville's theorem $F'' = 0$.
Since $f$ is nonconstant, $F' \neq 0$; therefore there is an $\alpha \in \CC^*$ with $F' = \alpha$.
Now the conditions $F' = \alpha$, $F(0) = 0$ imply $F(z) = \alpha z$.
In order that $F$ drop to a map $\CC/\Gamma \to \CC/\Gamma'$ it must be the case that $F$ maps a fundamental domain of $\Gamma$ to a fundamental domain of $\Gamma'$, which is only possible if $\alpha \Gamma \subseteq \Gamma'$.
Since $F$ was uniquely determined by the hypothesis $F(0) = 0$, $\alpha$ is the only possible constant defining a lift of $f$.

Since $X,Y$ are compact and $f$ is holomorphic, $f$ is proper.
Locally, $f$ looks like $F$, which is a local homeomorphism since $F'$ has no zeroes. Therefore $f$ is a covering space.
Since $F(z) = \alpha z$, $F$ is a group homomorphism; therefore so is $p' \circ F$, and the kernel of $p' \circ F$ is $\alpha^{-1}\Gamma'$.
Since $p' \circ F$ is invariant under translation by $\Gamma$, it drops along the quotient map $p$ to $f$, which is therefore a group homomorphism of kernel $\alpha^{-1}\Gamma'/\Gamma$.
Therefore translations of $\CC/\Gamma$ by elements of $\alpha^{-1}\Gamma'/\Gamma$ leave $f$ invariant, and $f$ drops to an isomorphism
$$f^\flat: \frac{\CC/\Gamma}{\alpha^{-1}\Gamma'/\Gamma} \to \frac{\CC}{\Gamma'}.$$
So if $\psi: \CC/\Gamma \to \CC/\Gamma$ is a deck transformation of $f$, then $\psi$ is also a deck transformation of the covering space $q: \CC/\Gamma \to (\CC/\Gamma)/(\alpha^{-1}\Gamma'/\Gamma)$.
That implies that $\psi$ acts by permuting each coset of $\alpha^{-1}\Gamma'/\Gamma$, which can only be done in a continuous way if $\psi$ is translation by an element of $\alpha^{-1}\Gamma'/\Gamma$.
Therefore $\alpha^{-1}\Gamma'/\Gamma$ is the deck group of $f$.
The groups $\alpha^{-1}\Gamma'/\Gamma$ and $\Gamma'/\alpha\Gamma$ are isomorphic in the obvious way, so the deck group of $f$ is $\Gamma'/\alpha\Gamma$.

\begin{exer}[Forster 5.5]
Let $X = \CC \setminus \{\pm 2\}$, $Y = \CC \setminus \{\pm 1, \pm 2\}$, $p: Y \to X$ the map
$$p(z) = z^3 - 3z.$$
Prove that $p$ is a $3$-sheeted covering space. Compute the deck group of $p$ and show that $p$ is not a Galois covering space.
\end{exer}

We first check that $p$ is well-defined. Indeed, if $p(z) = 2$ then $z = -1$ or $z = 2$, while if $p(z) = -2$ then $z = 1$ or $z = -2$.
So $p$ indeed carries $Y$ into $X$.

To see that $p$ is a covering space, note that $p'(z) = 3z^2 - 3$ is zero exactly at $z = \pm 1$, thus is nonzero on $Y$.
Therefore by the inverse function theorem, $p$ is a local homeomorphism.
Now let $w \in Y$.
Since $p$ is a depressed cubic, the discriminant $\Delta(w)$ of the polynomial $p - w$ is
$$\Delta(w) = -27(w^2 - 4).$$
This discriminant is zero exactly if $w = \pm 2$, but then $w \notin X$.
So $p(z) = w$ has three distinct solutions, so there is a neighborhood $U$ of $w$ so small that $p^{-1}(W)$ consists of the union of three disjoint open sets, around each of the solutions.

If $\psi: Y \to Y$ is a deck transformation of $p$, then $\psi$ is holomorphic, since it is a lift of the identity map on $X$ -- which is holomorphic -- along the holomorphic covering space $p$.
We can coextend $\psi$ to a holomorphic map $Y \to \PP^1$.
Then if $\psi$ has an essential singularity at $z \in \PP^1$, then there are arbitrarily small punctured neighborhoods $U \subseteq Y$ of $z$ such that $V = \psi(U)$ contains all but two points of $\PP^1$.
Since $\psi$ is a local homeomorphism we can choose $U$ so that $\psi|U$ is injective; but then shrinking $U$ must preserve this property, while still hitting all but two points of $\PP^1$, a contradiction, so $\psi$ can have no essential singularities.
Therefore $\psi$ extends to a biholomorphic map $\PP^1 \to \PP^1$, so $\psi$ is rational over $\CC$.

By definition of a deck transformation, for every $w \in X$, $\psi$ must permute the roots of $p - w$.
The zeroes of $p - w$ are $-w, -w \pm \sqrt 3$, so either $\psi(-w) = -w$ or $\psi(-w) = - w\pm \sqrt 3$.
If $\psi(0) = 0$ then there is a $\varepsilon > 0$ such that for every $|w| < \varepsilon$, $\psi(w) = w$, since otherwise $\psi(w) = w \pm \sqrt 3$ for $w$ arbitrarily close to $0$, and so by continuity $\psi(0) = \pm \sqrt 3$, a contradiction.
So $\psi(w) = w$ for every $|w| < \varepsilon$ and hence for every $w \in \PP^1$.
The other option is $\psi(0) = \pm \sqrt 3$; then, reasoning as before, we see that there is a $\varepsilon > 0$ such that if $|w| < \varepsilon$ then $\psi(w) = w \pm \sqrt 3$, and hence this is true for every $w \in \PP^1$.
But nontrivial translations do not preserve the fibers of the nonconstant polynomial $p$, so this is a contradiction.

We conclude that the only deck transformation of $p$ is the identity.
If $p$ was a Galois covering space then the trivial group would act transitively on the fibers of $p$, so $p$ would be injective, which is false.





\end{document}
