
% --------------------------------------------------------------
% This is all preamble stuff that you don't have to worry about.
% Head down to where it says "Start here"
% --------------------------------------------------------------

\documentclass[10pt]{article}

\usepackage[margin=.7in]{geometry}
\usepackage{amsmath,amsthm,amssymb}
\usepackage{enumitem}
\usepackage{tikz-cd}
\usepackage{mathtools}
\usepackage{amsfonts}
\usepackage{listings}
\usepackage{algorithm2e}
\usepackage{verse,stmaryrd}
\usepackage{fancyvrb}

% Number systems
\newcommand{\NN}{\mathbb{N}}
\newcommand{\ZZ}{\mathbb{Z}}
\newcommand{\QQ}{\mathbb{Q}}
\newcommand{\RR}{\mathbb{R}}
\newcommand{\CC}{\mathbb{C}}
\newcommand{\PP}{\mathbb P}
\newcommand{\FF}{\mathbb F}
\newcommand{\DD}{\mathbb D}
\renewcommand{\epsilon}{\varepsilon}

\newcommand{\Aut}{\operatorname{Aut}}
\newcommand{\coker}{\operatorname{coker}}
\newcommand{\CVect}{\CC\operatorname{-Vect}}
\newcommand{\Cantor}{\mathcal{C}}
\newcommand{\D}{\mathcal{D}}
\newcommand{\card}{\operatorname{card}}
\newcommand{\diam}{\operatorname{diam}}
\newcommand{\dbar}{\overline \partial}
\DeclareMathOperator*{\esssup}{ess\,sup}
\newcommand{\GL}{\operatorname{GL}}
\newcommand{\Hom}{\operatorname{Hom}}
\newcommand{\id}{\operatorname{id}}
\newcommand{\Ind}{\operatorname{Ind}}
\newcommand{\Inn}{\operatorname{Inn}}
\newcommand{\interior}{\operatorname{int}}
\newcommand{\lcm}{\operatorname{lcm}}
\newcommand{\mesh}{\operatorname{mesh}}
\newcommand{\LL}{\mathcal L_0}
\newcommand{\Leb}{\mathcal{L}_{\text{loc}}^2}
\newcommand{\ppGL}{\operatorname{PGL}}
\newcommand{\ppic}{\vspace{35mm}}
\newcommand{\ppset}{\mathcal{P}}
\DeclareMathOperator{\proj}{proj}
\DeclareMathOperator*{\Res}{Res}
\newcommand{\Riem}{\mathcal{R}}
\newcommand{\RVect}{\RR\operatorname{-Vect}}
\newcommand{\Sch}{\mathcal{S}}
\newcommand{\SL}{\operatorname{SL}}
\newcommand{\sgn}{\operatorname{sgn}}
\newcommand{\spn}{\operatorname{span}}
\newcommand{\Spec}{\operatorname{Spec}}
\newcommand{\supp}{\operatorname{supp}}
\newcommand{\TT}{\mathcal T}
\DeclareMathOperator{\tr}{tr}

\DeclareMathOperator{\adj}{adj}
\DeclareMathOperator{\curl}{curl}

% Calculus of variations
\DeclareMathOperator{\pp}{\mathbf p}
\DeclareMathOperator{\zz}{\mathbf z}
\DeclareMathOperator{\uu}{\mathbf u}
\DeclareMathOperator{\vv}{\mathbf v}
\DeclareMathOperator{\ww}{\mathbf w}

% Categories
\newcommand{\Ab}{\mathbf{Ab}}
\newcommand{\Cat}{\mathbf{Cat}}
\newcommand{\Group}{\mathbf{Group}}
\newcommand{\Module}{\mathbf{Module}}
\newcommand{\Set}{\mathbf{Set}}
\DeclareMathOperator{\Fun}{Fun}
\DeclareMathOperator{\Iso}{Iso}

% Complex analysis
\renewcommand{\Re}{\operatorname{Re}}
\renewcommand{\Im}{\operatorname{Im}}

% Logic
\renewcommand{\iff}{\leftrightarrow}
\newcommand{\Henkin}{\operatorname{Henk}}
\newcommand{\PA}{\mathbf{PA}}
\DeclareMathOperator{\proves}{\vdash}

% Group
\DeclareMathOperator{\Gal}{Gal}
\DeclareMathOperator{\Fix}{Fix}
\DeclareMathOperator{\Out}{Out}

% Mean
\def\Xint#1{\mathchoice
{\XXint\displaystyle\textstyle{#1}}%
{\XXint\textstyle\scriptstyle{#1}}%
{\XXint\scriptstyle\scriptscriptstyle{#1}}%
{\XXint\scriptscriptstyle\scriptscriptstyle{#1}}%
\!\int}
\def\XXint#1#2#3{{\setbox0=\hbox{$#1{#2#3}{\int}$ }
\vcenter{\hbox{$#2#3$ }}\kern-.6\wd0}}
\def\ddashint{\Xint=}
\def\dashint{\Xint-}

% Other symbols
\newcommand{\heart}{\ensuremath\heartsuit}
\newcommand\knuthup{\mathbin{\uparrow}}

\DeclareMathOperator{\atanh}{atanh}

% Theorems
\theoremstyle{definition}
\newtheorem*{corollary}{Corollary}
\newtheorem*{falselemma}{Grader's ``Lemma"}
\newtheorem{exer}{Exercise}
\newtheorem{lemma}{Lemma}[exer]
\newtheorem{theorem}[lemma]{Theorem}


\usepackage[backend=bibtex,style=alphabetic,maxcitenames=50,maxnames=50]{biblatex}
\renewbibmacro{in:}{}
\DeclareFieldFormat{pages}{#1}

\begin{document}
\noindent
\large\textbf{Fluid dynamics, Final Exam} \hfill \textbf{Aidan Backus} \\
% --------------------------------------------------------------
%                         Start here
% --------------------------------------------------------------\

\begin{exer}
Consider the Navier-Stokes equations on $\RR^3$ with initial data $u_0 \in L^2_\sigma \cap L^3$.
Let $u \in C([0, T] \to L^3) \cap K_6(T)$ be the unique local solution given by Theorem 7.5.

Prove that $u \in L^\infty([0, T] \to L^2)$.

Prove that if $||u_0||_{L^3}$ is small enough then $u$ is a global solution with
$$||u(t)||_{L^3} \lesssim t^{-1/4}.$$
\end{exer}

Recall (proof of Theorem 7.5) that if we set $u^0 = e^{\nu t\Delta} u_0$,
$$u^{n + 1} = e^{\nu t\Delta} u_0 + B(u^n, u^n),$$
then $u^n \to u$ in $K_6(T)$ if $T$ is small enough. We first bound $e^{\nu t\Delta}$. Namely, the convolution kernel is nonnegative with total integral $1$, thus is uniformly boudned in $L^1$.
So by Young's inequality, $u_0 \mapsto (t \mapsto e^{\nu t\Delta} u_0)$ is a contraction $L^2 \to L^\infty([0, T] \to L^2)$.

......


Now suppose that $||u_0||_{L^3} \ll \nu$.
From the proof of Theorem 7.5, uniformly in $T > 0$,
$$||e^{\nu t \Delta}u_0||_{K_6(T)} \lesssim ||u_0||_{L^3} \ll \nu,$$
and the uniformity means that Theorem 7.6 implies that there is a global solution $u \in K_6(T)$ with
$$||u||_{K_6(T)} \lesssim ||e^{\nu t \Delta}u_0||_{K_6(T)}.$$
Suppose that $t \leq 1$; then $u \in C([0, T] \to L^3)$ so $||u(t)||_{L^3}$ is bounded above by some constant.
On the other hand, if $t > 1$, then
$$||u(t)||_{L^6} \leq (\nu t)^{-1/4} \sup_{T > 0} ||u||_{K_6(T)} \lesssim t^{-1/4}.$$



\begin{exer}
Let $\omega_0 \in L^1 \cap L^\infty$ and let $\omega \in L^\infty(\RR \to L^1 \cap L^\infty)$ be the Yudovich solution with initial data $\omega_0$.
Let $X$ denote the flow map. Show that for every $T \in [0, \infty)$ and $p \in [1, \infty)$, $\omega \in C([0, T] \to L^p)$.

Let $D_1,D_2$ be separated bounded sets. Show that if $\omega_0 = 1_{D_1} - 1_{D_2}$ then $\omega(t) = 1_{D_1(t)} - 1_{D_2(t)}$ where $D_j(t) = X_t(D_j)$.

Show that $D_1(t),D_2(t)$ are separated and $d(D_1(t), D_2(t))$ has a lower bound in terms of $d(D_1, D_2)$.

Show that $|D_1(t)| + |D_2(t)|$ is constant in time.
\end{exer}

By Yudovich's theorem, $\omega \in L^\infty([0, T] \to L^1 \cap L^\infty)$.
We want to bound $\omega(t) - \omega(s)$ in $L^p$, and by the measure-preserving property of $X$ it suffices to do so when $s = 0$.
To do this, consider the linear maps $\psi_t$,
$$\psi_t \omega_0(x) = \omega_0(x) - \omega_0(X_{-t}(x)).$$
Then
$$||\psi_t \omega_0||_{L^p} \leq 2||\omega_0||_{L^p}$$
so $||\psi_t||_{L^p \to L^p} \leq 2$.
We will later show that Yudovich solutions are continuous in initial data.
So, since simple functions are dense in $L^p$, $(\psi_t)$ is bounded in operator norm on $L^p$, and $\psi_t$ is linear, we may assume that $\omega$ is a vortex patch, i.e. $\omega_0 = 1_E$ for some Borel set $E$.
Let $\rho$ be the metric on Borel sets, i.e.
$$\rho(E, F) = |E \setminus F| + |F \setminus E|.$$
Then
$$||\psi_t 1_E||_{L^p}^p = \int_{\RR^2} |1_E(x) - 1_{X_t(E)}(x)|^p ~dx = \rho(E, X_t(E)).$$
As $\rho(E, X_t(E)) \to 0$ as $t \to 0$, this gives the claim.

Now, to see that Yudovich solutions are continuous (in $L^p$) in initial data, we use induction: $\omega^0 = \omega_0$ and $u^n$ is continuous with respect to $\omega^{n - 1}$ since $K \in L^1$, and then $\omega^n$ solves
$$\partial_t \omega^n + u^n \cdot \nabla \omega^n = 0$$
where $u^n$ is given. The operator $\partial_t + u^n \cdot \nabla$ is continuous in some sense of pseudodifferential operators, I think, and you should be able to invert it to get that $u^n \mapsto \omega^n$ is continuous.
But unfortunately I'm out of time. :-(



Now suppose that $\omega_0 = 1_{D_1} - 1_{D_2}$.
Then
$$\omega(x, t) = 1_{D_1}(X_t^{-1}(x)) - 1_{D_2}(X_t^{-1}(x)) = 1_{D_1(t)}(x) - 1_{D_2(t)}(x)$$
since $\omega$ is a Yudovich solution.

To get the separation results, we need to construct a suitable foliation of $\RR^2$.
Since $\nabla^\perp \Delta^{-1}$ is a pseudodifferential operator of order $-1$ and $\omega(t) \in L^1 \cap L^\infty$, $u(t)$ is a differentiable vector field.
Thus $\dot X = u(X)$ implies that $X$ is a one-parameter group which acts by $C^1$ diffeomorphisms.

Let $x,y \in \RR^2$ be distinct points and let $x(t),y(t)$ be their trajectories under $X$. We will find a lower bound on $d|x(t), y(t)|$.......

Now let $(x, y)$ be a pair of points that minimizes $|x - y|$ among $x \in D_1$, $y \in D_2$.
By the above, we get a lower bound on $d(D_1(t), D_2(t))$.
In particular they are separated.
Thus if $E(t) = D_1(t) \cup D_2(t)$,
$$|E(t)| = |D_1(t)| + |D_2(t)|.$$
But $E(t)$ is the set of points on which $|\omega(t)| > 0$, which is exactly the set of points on which $|\omega(t)| = 1$.
This is constant, since the $L^1$ norm of $\omega$ is.

\begin{exer}
Consider the forced Navier-Stokes equations
$$\begin{cases}
\partial_t \omega + (u \cdot \nabla)\omega = \nu \Delta \omega + \nabla^\perp \cdot f\\
u = \nabla^\perp \Delta^{-1}\omega\\
f(x) = (\sin x_2, 0)\end{cases}$$
on the torus.

Show that $\overline \omega(x) = -(1/\nu)\cos x_2$ is a steady solution.

Let $\omega$ be a global smooth solution and $w = \omega - \overline \omega$.
Let $\phi = \Delta^{-1}w$ be the stream of $w$.
Show that
$$\partial_t \Delta \phi - \partial_2 \phi \Delta \partial_1 \phi + \partial_1 \phi \Delta \partial_2 \phi + \frac{\sin x_2}{\nu}\partial_1(\Delta \phi+\phi) = \nu \Delta^2\phi.$$

Show that
\begin{equation}
\label{energy balance}
\frac{1}{2} \frac{d}{dt} \int_{T^2} (\Delta \phi(x))^2 - |\nabla \phi(x)|^2 ~dx + \nu \int_{T^2} |\nabla \Delta \phi(x)|^2 - (\Delta \phi(x))^2 ~dx = 0.
\end{equation}

Show that
$$\delta(t) = ||\Delta \phi(t)||_{L^2}^2 - ||\nabla \phi(t)||_{L^2}^2$$
satisfies
$$0 \leq \delta(t) \leq \delta(0) e^{-\nu t}.$$

Show that
$$\lim_{t \to \infty} \omega(t) \overline \omega$$
in $L^2$ exponentially fast.

Show that $\overline \omega$ is the unique steady solution of the forced Navier-Stokes equations.
\end{exer}

We first observe that $\overline \omega$ is an eigenfunction of $\Delta$, namely
$$\Delta \overline \omega + \overline \omega = 0.$$
Thus $\overline u = (\sin x_2, 0)/\nu$ while $\nabla \omega = (0, \sin x_2)/\nu$, and thus the left-hand side of the Navier-Stokes equation is $0$.
The right-hand side is
$$\nu \Delta \overline \omega(x) - \cos x_2 = \cos x_2 - \cos x_2 = 0$$
as well.

Now we rewrite
$$\partial_t \Delta \phi - \partial_2 \phi \Delta \partial_1 \phi + \partial_1 \phi \Delta \partial_2 \phi + \frac{\sin x_2}{\nu}\partial_1(\Delta \phi+\phi) = \nu \Delta^2\phi$$
as
$$\partial_t \omega - \partial_2 \phi \partial_2 \omega + \partial_1 \phi \partial_2 \omega - \partial_1 \phi \partial_2 \overline \omega + \frac{\sin x_2}{\nu} \partial_1(\omega + \phi) = \nu \Delta \omega - \nu \overline \omega$$
by commuting $\Delta$ with partial derivatives.
But $-\nu \overline \omega = \nabla^\perp \cdot f$ and $\nu \overline u_2 = \sin x_2$, so we can simplify this equation further to
$$\partial_t \omega + (\nabla^\perp \Delta^{-1}\omega \cdot \nabla)\omega = \nu \Delta \omega + \nabla^\perp \cdot f.$$
This is exactly the Navier-Stokes equation for $\omega$.

To show (\ref{energy balance}) we first compute
\begin{align*}
\frac{1}{2} \frac{d}{dt} \int_{T^2} (\Delta \phi(x))^2 ~dx &= \int_{T^2} \Delta \phi(x) \partial_t \Delta \phi(x) ~dx\\
&= \int_{T^2} \nu \Delta \phi \Delta^2 \phi + \Delta \phi \partial_2 \phi \Delta \partial_1 \phi - \Delta \phi \partial_1 \phi \Delta \partial_2 \phi - \frac{\sin x_2}{\nu} \partial_1(\Delta \phi + \phi) \Delta \phi.
\end{align*}
Now let us integrate by parts using the fact that $T^2$ is a closed manifold.
In fact
$$\int_{T^2} \nu \Delta \phi \Delta^2 \phi = \int_{T^2} -\nu |\nabla \Delta \phi|^2.$$
so
$$\frac{1}{2} \frac{d}{dt} \int_{T^2} (\Delta \phi(x))^2 ~dx = \int_{T^2}-\nu |\nabla \Delta \phi|^2 + \Delta \phi \partial_2 \phi \Delta \partial_1 \phi - \Delta \phi \partial_1 \phi \Delta \partial_2 \phi - \frac{\sin x_2}{\nu} \partial_1(\Delta \phi + \phi) \Delta \phi.$$
Thus
$$\frac{1}{2} \frac{d}{dt} \int_{T^2} (\Delta \phi(x))^2 ~dx + \nu \int_{T^2} |\nabla \Delta \phi(x)|^2 ~dx = \int_{T^2} \Delta \phi \partial_2 \phi \Delta \partial_1 \phi - \Delta \phi \partial_1 \phi \Delta \partial_2 \phi - \frac{\sin x_2}{\nu} \partial_1(\Delta \phi + \phi) \Delta \phi.$$

Now we compute
\begin{align*}
-\frac{1}{2} \frac{d}{dt} \int_{T^2} |\nabla \phi|^2 &= -\int_{T^2} \partial_1 \phi \partial_1 \partial_t \phi + \partial_2 \phi \partial_2 \partial_t \phi\\
&= \int_{T^2} \phi \partial_1^2 \partial_t \phi + \phi \partial_2^2 \partial_t \phi\\
&= \int_{T^2} \phi \Delta \partial_t \phi\\
&= \int_{T^2} \nu \phi \Delta^2 \phi + \phi \partial_2 \phi \Delta \partial_1 \phi - \phi \partial_1 \phi \Delta \partial_2 \phi - \phi \frac{\sin x_2}{\nu} \partial_1(\Delta \phi + \phi).
\end{align*}
Thus
$$-\frac{1}{2} \frac{d}{dt} \int_{T^2} |\nabla \phi|^2 - \nu \int_{T^2} (\Delta \phi(x))^2 = \int_{T^2} \phi \partial_2 \phi \Delta \partial_1 \phi - \phi \partial_1 \phi \Delta \partial_2 \phi - \phi \frac{\sin x_2}{\nu} \partial_1(\Delta \phi + \phi).$$
Furthermore, integrating by parts,
$$\int_{T^2} \frac{\sin x_2}{\nu}(\Delta \phi + \phi) \partial_1(\Delta \phi + \phi) = -\int_{T^2}\frac{\sin x_2}{\nu}(\Delta \phi + \phi) \partial_1(\Delta \phi + \phi)$$
and thus
$$\int_{T^2} \frac{\sin x_2}{\nu}(\Delta \phi + \phi) \partial_1(\Delta \phi + \phi) = 0.$$
Moreover,
\begin{align*}
 \int_{T^2} (\phi + \Delta \phi) \partial_2 \phi \Delta \partial_1 \phi - (\phi + \Delta \phi) \partial_1 \phi \Delta \partial_2 \phi
  &= \int_{T^2} \partial_2 \phi \nabla(\phi + \Delta \phi) \cdot \nabla \partial_1 \phi \\
  &\qquad+ (u + \Delta u - u - \Delta u)(\nabla \partial_1 \phi \cdot \nabla \partial_2 \phi) \\
  &\qquad- \partial_1 \phi \nabla(\phi + \Delta \phi) \cdot \nabla \partial_2 \phi\\
  &= 0.
\end{align*}
This completes the proof of (\ref{energy balance}).

In the sequel we will need the Fourier decomposition
$$\phi(x, t) = \sum_{k \in \ZZ^2} \phi_k(t) e^{ikx}$$
of $\phi$. Since $w$ is smooth, so is $\phi$, thus $\phi \in L^2(T^2)$.
That implies that this decomposition exists, with $\phi_k \in L^2(T^2)$.
In this decomposition we have
$$\nabla \phi(x, t) = i\sum_{k \in \ZZ^2} \phi_k(t) e^{ikx} k$$
and also
$$\Delta \phi(x, t) = -\sum_{k \in \ZZ^2} \phi_k(t) e^{ikx} |k|^2.$$
Thus we have
$$\delta(t) = \sum_{k \in \ZZ^2} (|k|^4 - |k|^2) |\phi_k(t)|^2$$
and in particular
$$\delta(t) = \sum_{|k| \geq 2} (|k|^4 - |k|^2) |\phi_k(t)|^2$$
since terms of frequency $\leq 1$ vanish.
Since $k \in \ZZ$, it immediately follows that $\delta(t) \geq 0$.

Now let us bound $\delta$. In fact,
\begin{align*}
\delta'(t) &= -2\nu \int_{T^2} |\nabla \Delta \phi(x, t)|^2 - |\Delta \phi(x, t)|^2 ~dx\\
&= -2\nu \sum_{|k| \geq 2} (|k|^6 - |k|^4) |\phi_k(t)|^2\\
&= -2\nu \sum_{|k| \geq 2} |k|^2 (|k|^4 - |k|^2) |\phi_k(t)|^2\\
&\leq -\nu \delta(t)
\end{align*}
so by Gr\"onwall's inequality,
$$\delta(t) \leq e^{-\nu t} \delta(0).$$

Let $P$ be the Littlewood-Paley projector to frequencies $\leq 1$.
Then $P$ commutes with quadratic differential operators with coefficients that have frequency $\leq 1$.
This holds because such operators preserves the Hilbert space of functions of frequency $\leq 1$.

Let us now show exponential decay of $||\omega(t) - \overline \omega||_{L^2}$.
In fact,
$$||\omega(t) - \overline \omega||_{L^2}^2 = ||\Delta \phi(t)||_{L^2}^2 \leq \delta(0) e^{-\nu t} + ||\nabla \phi(t)||_{L^2}^2.$$
Now
$$||\nabla \phi(t)||_{L^2}^2 = \sum_{|k| = 1} |\phi_k(t)|^2 + \sum_{|k| \geq 2} |k|^2 |\phi_k(t)|^2$$
but
$$||\Delta \phi(t)||_{L^2}^2 = \sum_{|k| = 1} |\phi_k(t)|^2 + \sum_{|k| \geq 2} |k|^4 |\phi_k(t)|^2$$
and $4|k|^2 \leq |k|^4$ provided $|k| \geq 2$, thus
$$||\nabla \phi(t)||_{L^2}^2 \leq \frac{||\Delta \phi(t)||_{L^2}^2}{4} + \frac{3}{4} \sum_{|k| = 1} |\phi_k(t)|^2.$$
On the other hand, $P\phi$ solves the Littlewood-Paley projection of the forced Navier-Stokes equation in stream formulation, namely
$$\partial_t P\phi - \nu \Delta P\phi = \partial_2 P\phi\Delta \partial_1 P\phi - \partial_1 P\phi\Delta \partial_2 P\phi - \frac{\sin x_2}{\nu}\partial_1(\Delta P\phi + P\phi),$$
since that is a quadratic differential equation whose coefficients are constant or of frequency $1$.
However, $\Delta Pu = -Pu$ for any $u$, so all the terms cancel out and we are left with
$$\partial_t P\phi + \nu P\phi = 0,$$
so that $P\phi$ decays as fast as $e^{-\nu t}$.
Thus we have
$$\frac{3}{4} \sum_{|k| = 1} |\phi_k(t)|^2 = \frac{3}{4} ||P\phi(t)||_{L^2}^2 \leq \frac{3}{4} ||\phi(0)||_{L^2}^2 e^{-\nu t}.$$
In particular,
$$||\Delta \phi(t)||_{L^2}^2 \leq \delta(0) e^{-\nu t} + \frac{||\Delta \phi(t)||_{L^2}^2}{4} + \frac{3}{4} ||\phi(0)||_{L^2}^2 e^{-\nu t},$$
or in other words
$$||\omega(t) - \overline \omega||_{L^2}^2 \leq \left(\frac{4}{3} \delta(0) + ||\phi(0)||_{L^2}^2\right) e^{-\nu t},$$
which was desired.

Finally, suppose that $\omega$ is a steady solution of the Navier-Stokes equations.
Then
$$\lim_{t \to \infty} ||\omega - \overline \omega||_{L^2} = 0$$
but the expression inside the limit does not depend on $t$, and thus must be $0$. So $\omega = \overline \omega$.


\end{document}
