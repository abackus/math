
% --------------------------------------------------------------
% This is all preamble stuff that you don't have to worry about.
% Head down to where it says "Start here"
% --------------------------------------------------------------

\documentclass[10pt]{article}

\usepackage[margin=.7in]{geometry}
\usepackage{amsmath,amsthm,amssymb}
\usepackage{enumitem}
\usepackage{tikz-cd}
\usepackage{mathtools}
\usepackage{amsfonts}
\usepackage{listings}
\usepackage{algorithm2e}
\usepackage{verse,stmaryrd}
\usepackage{fancyvrb}

% Number systems
\newcommand{\NN}{\mathbb{N}}
\newcommand{\ZZ}{\mathbb{Z}}
\newcommand{\QQ}{\mathbb{Q}}
\newcommand{\RR}{\mathbb{R}}
\newcommand{\CC}{\mathbb{C}}
\newcommand{\PP}{\mathbb P}
\newcommand{\FF}{\mathbb F}
\newcommand{\DD}{\mathbb D}
\renewcommand{\epsilon}{\varepsilon}

\newcommand{\Aut}{\operatorname{Aut}}
\newcommand{\coker}{\operatorname{coker}}
\newcommand{\CVect}{\CC\operatorname{-Vect}}
\newcommand{\Cantor}{\mathcal{C}}
\newcommand{\D}{\mathcal{D}}
\newcommand{\card}{\operatorname{card}}
\newcommand{\dbar}{\overline \partial}
\DeclareMathOperator*{\esssup}{ess\,sup}
\newcommand{\GL}{\operatorname{GL}}
\newcommand{\Hom}{\operatorname{Hom}}
\newcommand{\id}{\operatorname{id}}
\newcommand{\Ind}{\operatorname{Ind}}
\newcommand{\Inn}{\operatorname{Inn}}
\newcommand{\interior}{\operatorname{int}}
\newcommand{\lcm}{\operatorname{lcm}}
\newcommand{\mesh}{\operatorname{mesh}}
\newcommand{\LL}{\mathcal L_0}
\newcommand{\Leb}{\mathcal{L}_{\text{loc}}^2}
\newcommand{\Lip}{\operatorname{Lip}}
\newcommand{\ppGL}{\operatorname{PGL}}
\newcommand{\ppic}{\vspace{35mm}}
\newcommand{\ppset}{\mathcal{P}}
\DeclareMathOperator{\proj}{proj}
\DeclareMathOperator*{\Res}{Res}
\newcommand{\Riem}{\mathcal{R}}
\newcommand{\RVect}{\RR\operatorname{-Vect}}
\newcommand{\Sch}{\mathcal{S}}
\newcommand{\SL}{\operatorname{SL}}
\newcommand{\sgn}{\operatorname{sgn}}
\newcommand{\spn}{\operatorname{span}}
\newcommand{\Spec}{\operatorname{Spec}}
\newcommand{\supp}{\operatorname{supp}}
\newcommand{\TT}{\mathcal T}
\DeclareMathOperator{\tr}{tr}

\DeclareMathOperator{\adj}{adj}
\DeclareMathOperator{\curl}{curl}

% Calculus of variations
\DeclareMathOperator{\pp}{\mathbf p}
\DeclareMathOperator{\zz}{\mathbf z}
\DeclareMathOperator{\uu}{\mathbf u}
\DeclareMathOperator{\vv}{\mathbf v}
\DeclareMathOperator{\ww}{\mathbf w}

% Categories
\newcommand{\Ab}{\mathbf{Ab}}
\newcommand{\Cat}{\mathbf{Cat}}
\newcommand{\Group}{\mathbf{Group}}
\newcommand{\Module}{\mathbf{Module}}
\newcommand{\Set}{\mathbf{Set}}
\DeclareMathOperator{\Fun}{Fun}
\DeclareMathOperator{\Iso}{Iso}

% Complex analysis
\renewcommand{\Re}{\operatorname{Re}}
\renewcommand{\Im}{\operatorname{Im}}

% Logic
\renewcommand{\iff}{\leftrightarrow}
\newcommand{\Henkin}{\operatorname{Henk}}
\newcommand{\PA}{\mathbf{PA}}
\DeclareMathOperator{\proves}{\vdash}

% Group
\DeclareMathOperator{\Gal}{Gal}
\DeclareMathOperator{\Fix}{Fix}
\DeclareMathOperator{\Out}{Out}

% Other symbols
\newcommand{\heart}{\ensuremath\heartsuit}

\DeclareMathOperator{\atanh}{atanh}

% Theorems
\theoremstyle{definition}
\newtheorem*{corollary}{Corollary}
\newtheorem*{falselemma}{Grader's ``Lemma"}
\newtheorem{exer}{Exercise}
\newtheorem{lemma}{Lemma}[exer]
\newtheorem{theorem}[lemma]{Theorem}


\usepackage[backend=bibtex,style=alphabetic,maxcitenames=50,maxnames=50]{biblatex}
\renewbibmacro{in:}{}
\DeclareFieldFormat{pages}{#1}

\begin{document}
\noindent
\large\textbf{Fluid dynamics, HW 3} \hfill \textbf{Aidan Backus} \\
% --------------------------------------------------------------
%                         Start here
% --------------------------------------------------------------\

\begin{exer}
Suppose that $u^\varepsilon$ is the unique solution to
$$\partial_t u^\varepsilon + \PP J_\varepsilon(J_\varepsilon u^\varepsilon \cdot \nabla J_\varepsilon u^\varepsilon) = 0$$
with initial data $J_\varepsilon u_0$.
Suppose that a subsequence $u_n$ of the $u^\varepsilon$ converges to $u \in L^\infty([0, T_0] \to H^r(\RR^d))$ whenever $r < s$.
Show that
\begin{equation}
\label{Euler}
\partial_t u + \PP(u \cdot \nabla u) = 0.
\end{equation}
\end{exer}

Let $\mathcal D'(\RR^d \to X)$ be the space of distributions on $\RR^d$, valued in a Banach space $X$.
This space will be convenient for us to work in, because a priori we only have $u \in C([0, T_0] \to H^s(\RR^d))$ and so it will be difficult to justify that $\partial_t u$ exists in a sufficiently strong sense without already knowing that $u$ solves the Euler equations.

We claim that
\begin{equation}
\label{limiting Euler}
\lim_{n \to \infty} \partial_t u_n + \PP J_n(J_n u_n \cdot \nabla J_n u_n) = \partial_t u + \PP(u \cdot \nabla u)
\end{equation}
in the topology of $\mathcal D'([0, T_0] \to L^2(\RR^d))$.
If that is true, then since the left-hand side of (\ref{limiting Euler}) is almost surely $0$ in time and the right-hand side is continuous in time, (\ref{Euler}) follows.

First, since $u \in C([0, T_0] \to H^s(\RR^d))$, in particular $u \in \mathcal D'(\RR^d \to L^2(\RR^d))$, the space of distributions, so $\partial_t u \in \mathcal D'(\RR^d)$.
We claim that $\partial_t u_n \to \partial_t u$ in the topology of $\mathcal D'(\RR^d \to L^2(\RR^d))$.
If not then after passing to a subsequence we may assume that there is a test function $v$ and an $\eta > 0$ such that for every $n$,
$$\int_{\RR^d} \partial_t (u - u_n)v > \eta,$$
where the integral is meant in the sense of pairing of distributions. Integrating by parts, we find a test function $w = -\partial_tv$ such that
$$\int_{\RR^d} (u - u_n)w > \eta,$$
and hence conclude that $u - u_n$ misses a neighborhood of $0$ in $\mathcal D'(\RR^d \to L^2(\RR^d))$, but convergence in $L^\infty$ implies convergence in $\mathcal D'$, a contradiction.

Since $\PP$ is bounded on $H^s$, in particular it is continuous in the strong operator topology of $H^s$.
Moreover, $J_\varepsilon \to 1$ in the strong operator topology of $H^s$.
Therefore for almost every fixed time $t$,
$$\lim_{n \to \infty} \PP J_n(J_n u_n(t) \cdot \nabla J_n u_n(t)) = \PP\left(u(t) \cdot \lim_{n \to \infty} \nabla J_n u_n(t)\right)$$
in the topology of $C([0, T] \to H^s(\RR^d))$.
Now $\nabla$ is continuous as a map $H^s \to L^2$ since $s > 1 + d/2 \geq 1$, and thus we may commute $\nabla$ with the limit at the price of dropping to a weaker topology.
Thus for almost every fixed time $t$,
$$\lim_{n \to \infty} \PP J_n(J_n u_n(t) \cdot \nabla J_n u_n(t)) = \PP\left(u(t) \cdot \nabla u(t)\right)$$
in the topology of $C([0, T] \to L^2(\RR^d))$.
Since uniform convergence implies convergence in $\mathcal D'$ we deduce (\ref{limiting Euler}) as desired.

\begin{exer}
Let $u_{0j} \in H^s(\RR^d)$ be divergence-free vector fields. Assume $u_j$ is the solution to (\ref{Euler}) with initial data $u_{0j}$.
Show that
$$||u_1 - u_2||_{L^\infty([0, T] \to L^2(\RR^d))} \leq ||u_{01} - u_{02}||_{L^2(\RR^d)} \exp \int_0^T ||\nabla u_1(t)||_{L^\infty(\RR^d)} ~dt.$$
\end{exer}

Let $\tilde u = u_1 - u_2$.
Subtracting (\ref{Euler}) from itself and using the identity
$$u_1 \cdot \nabla \tilde u + \tilde u \cdot \nabla u_2 = u_1 \cdot \nabla u_1 + u_2 \cdot \nabla u_2$$
we deduce
$$\partial_t \tilde u + \PP(u_1 \cdot \nabla \tilde u + \tilde u \cdot \nabla u_2) = 0.$$
Let $(\cdot, \cdot)$ denote the $L^2$ inner product, so that
$$(\partial_t \tilde u, \tilde u) + (\PP(u_1 \cdot \nabla \tilde u), \tilde u) + (\PP(\tilde u \cdot \nabla u_2), \tilde u) = 0.$$
Since $\PP$ is an orthogonal projection, it is self-adjoint, and since $\tilde u$ is divergence-free, we conclude
$$(\partial_t \tilde u, \tilde u) + (u_1 \cdot \nabla \tilde u, \tilde u) + (\tilde u \cdot \nabla u_2, \tilde u) = 0.$$
We compute, since $u_1$ is divergence-free and so commutes with $\nabla \cdot$,
$$(u_1 \cdot \nabla \tilde u, \tilde u) = \int_{\RR^d} (u_1 \cdot \nabla \tilde u)\tilde u
= \frac{1}{2} \int_{\RR^d} u_1 \cdot \nabla \cdot (\tilde u^2) = \frac{1}{2} \int_{\RR^d} \nabla \cdot(u_1 \cdot \tilde u^2).$$
By the divergence theorem, it follows that
$$(\partial_t \tilde u, \tilde u) + (\tilde u \cdot \nabla u_2, \tilde u) = 0.$$
Rewriting and using the Cauchy-Schwarz inequality,
$$\partial_t ||\tilde u(t)||_{L^2}^2 = -2(-\tilde u(t) \cdot \nabla u_2(t), \tilde u(t)) \leq 2||\nabla u_2(t)||_{L^\infty} ||\tilde u(t)||_{L^2}^2.$$
But
$$\frac{\partial_t ||\tilde u(t)||_{L^2}^2}{||\tilde u(t)||_{L^2}} = 2\partial_t ||\tilde u(t)||_{L^2}$$
so this implies the inequality
$$\partial_t ||\tilde u(t)||_{L^2} \leq ||\nabla u_2(t)||_{L^\infty} ||\tilde u(t)||_{L^2}.$$
Integrating,
$$||\tilde u(t)||_{L^2} \leq \int_0^t ||\nabla u_2(s)||_{L^2} ||\tilde u(s)||_{L^2} ~ds.$$
By Gr\"onwall's inequality,
$$||\tilde u||_{L^\infty([0, T]) \to L^2(\RR^d)} \leq ||\tilde u_0||_{L^2} \exp \int_0^T ||\nabla u_2(t)||_{L^2} ~dt.$$
By symmetry,
$$||\tilde u||_{L^\infty([0, T]) \to L^2(\RR^d)} \leq ||\tilde u_0||_{L^2} \exp \int_0^T ||\nabla u_1(t)||_{L^2} ~dt.$$
This was the bound we wanted.


\end{document}
