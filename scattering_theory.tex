\documentclass[12pt]{report}
\usepackage[utf8]{inputenc}
\usepackage[margin=1in]{geometry} 
\usepackage{amsmath,amsthm,amssymb}
\usepackage{mathrsfs}

\usepackage{enumitem}
%\usepackage[shortlabels]{enumerate}
\usepackage{tikz-cd}
\usepackage{mathtools}
\usepackage{amsfonts}
\usepackage{amscd}
\usepackage{makeidx}
\usepackage{enumitem}

\title{Scattering theory}
\author{Aidan Backus}
\date{Fall 2019}

\newcommand{\NN}{\mathbf{N}}
\newcommand{\ZZ}{\mathbf{Z}}
\newcommand{\QQ}{\mathbf{Q}}
\newcommand{\RR}{\mathbf{R}}
\newcommand{\CC}{\mathbf{C}}

\newcommand{\pic}{\vspace{30mm}}
\newcommand{\dfn}[1]{\emph{#1}\index{#1}}

\renewcommand{\Re}{\operatorname{Re}}
\renewcommand{\Im}{\operatorname{Im}}

\newtheorem{theorem}{Theorem}[chapter]
\newtheorem{badtheorem}[theorem]{``Theorem"}
\newtheorem{prop}[theorem]{Proposition}
\newtheorem{lemma}[theorem]{Lemma}
\newtheorem{proposition}[theorem]{Proposition}
\newtheorem{corollary}[theorem]{Corollary}
\newtheorem{conjecture}[theorem]{Conjecture}
\newtheorem{axiom}[theorem]{Axiom}

\theoremstyle{definition}
\newtheorem{definition}[theorem]{Definition}
\newtheorem{remark}[theorem]{Remark}
\newtheorem{example}[theorem]{Example}

\theoremstyle{remark}
\newtheorem{exercise}[theorem]{Discussion topic}
\newtheorem{homework}[theorem]{Homework}
\newtheorem{problem}[theorem]{Problem}

\usepackage{color}
\usepackage{hyperref}
\hypersetup{
    colorlinks=true, % make the links colored
    linkcolor=blue, % color TOC links in blue
    urlcolor=red, % color URLs in red
    linktoc=all % 'all' will create links for everything in the TOC
    %Ning added hyperlinks to the table of contents 6/17/19
}

\makeindex
\begin{document}

\maketitle

\tableofcontents

\newpage

\chapter{Zworski's talk}
Idea: use dynamical systems to study PDE.

Semiclassical microlocal analysis: the phase space is $T^*X$, and we have a semiclassical parameter $h \approx 0$. For an ``observable" (i.e. symbol) $a \in C^\infty(T^*X)$ the quantization is
$$Op_h(a) = a(x, -ih \partial_x).$$
For example, $a(x, \xi) = x_j$ is the position operator, and $a(x, \xi) = \xi_j$ is the momentum operator $-ih\partial_{x_j}$. In fact we have the commutator relation
$$[Op_h(a), Op_h(b)] = -ihOp_h(\{a, b\}) + O(h^2).$$
This resembles the Heisenberg commutation relation except with a quadratic error term.

\section{Anosov flow}
\begin{definition}
	An \dfn{Anosov flow} is a flow on a compact manifold for which the tangent space decomposes $T_p(X) = E_0(p) \oplus E_s(p) \oplus E_u(p)$
	where $E_s$ and $E_u$ are the exponentially stable and unstable bundles of the flow.
\end{definition}
\begin{theorem}[Smale's conjecture]
	For an Anosov flow, the Ruelle zeta function has a meromorphic continuation to $\CC$.
\end{theorem}
Let $\varphi$ be an Anosov flow, $\varphi_t = \exp tX$. We consider the equation
$$(-iX - \lambda)u = 0.$$
We want to study this operator similar to ``scattering theory" but the operator $-iX - \lambda$ is not elliptic. Instead of ``scattering" we study singularities as $\xi \to \infty$. 

Let $M$ be a smooth, oriented, compact Riemannian surface of negative curvature and genus $g$.
\begin{theorem}[Dyatlov-Zworski]
	The order of vanishing of the Ruelle zeta function at $0$ is $2g - 2$.
\end{theorem}


\printindex

\end{document}
