\section{Regularity of minimal hypersurfaces}\label{DeGiorgiSection}
In this section we prove specialize to $\Hyp^d$ and show regularity of hypersurfaces.

\subsection{Excess: First properties}
The quantity which we seek to estimate is the so-called excess of a function of locally bounded variation.
Notions of excess for $\RR^d$ appear in the monographs \cite[\S5.3.1]{federer2014geometric} and \cite[Chapter 6]{Giusti77}.

We introduce the coordinate system $(x, y, z) \in \RR_+ \times \RR^{d - 2} \times \RR$ with the hyperbolic half-space metric
\begin{equation}\label{hyperbolic metric}
g = \frac{dx^2 + dy_1^2 + \cdots + dy_{d - 2}^2 + dz^2}{x^2}.
\end{equation}
This does not agree with the usual convention for the hyperbolic half-space metric, but will be advantageous for our purposes when we represent $C^1$ hypersurfaces as graphs of functions $\omega: \RR_+ \times \RR^{d - 2} \to \RR$; then such a hypersurface can be written $\{z = \omega(x, y)\}$.
We also define the \dfn{hyperbolic origin} $O = (1, 0, 0)$ and write $B_r = B(O, r)$.
Whenever we sum over an index $\mu$, we mean that $\mu$ ranges over
$$\mu \in \{x, y_1, \dots, y_{d - 2}, z\}.$$

\begin{definition}
Let $u \in BV_l(\Hyp^d)$. The \dfn{approximate derivative} to $u$ at scale $n$ is the covector in $T_O' \Hyp^d$ defined by
$$\normal^{(n)}_\mu = \frac{\int_{B_{2^{-n}}} x \partial_\mu u ~\vol}{\int_{B_{2^{-n}}} |du| ~\vol}.$$
\end{definition}

Since $g(x\partial_\mu, x\partial_\nu) = \delta_{\mu\nu}$, $x\partial_\mu u = \normal_\mu$ and so, if $O$ is a Lebesgue point, then $\normal^{(n)} \to \normal_U(O)$ pointwise.
But $\Iso(\Hyp^d)$ acts transitively on $\Hyp^d$ and so the conjugation of $\normal^{(n)}$ to other cotangent spaces by $\Iso(\Hyp^d)$ defines a continuous section of $T'\Hyp^d$ defined on $\partial^* U$ which converges to $\normal_U$ pointwise $|du|\vol$-almost everywhere.
Moreover it is clear that the conjugation of $\normal^{(n)}$ to $T'_P\Hyp^d$ is continuous in $P$.
So to show $\normal$ is continuous, it suffices to show that $(\normal^{(n)})$ is locally uniformly Cauchy, and this goal is accomplished by the excess:

\begin{definition}
The \dfn{excess} of $u \in BV_l(\Hyp^d)$ at scale $n$ is
$$\Exc_n(u) = 2^{n(d - 1)} (\exp(2^{-n}) - |\normal^{(n)}|) \int_{B_{2^{-n}}} |du| ~\vol$$
where $|\normal^{(n)}|$ denotes the length of $\normal^{(n)}$ in $T_O' \Hyp^d$.
\end{definition}

To motivate the somewhat strange term $\exp(2^{-n})$ we observe that since $(dx^\mu)$ is an orthonormal basis of $T_O' \Hyp^d$ we have
$$|\normal^{(n)}|^2 = \sum_\mu \left(\int_{B_{2^{-n}}} \normal_\mu ~d\lambda_n\right)^2$$
where
$$\lambda_n(E) = \frac{\int_E |du|~\vol}{\int_{B_{2^{-n}}} |du| ~\vol}$$
for Borel subsets $E$ of $2^{-n}$, so $\lambda_n$ is a probability measure on $B_{2^{-n}}$.
Since $\lambda_n$ is a probability measure, the Cauchy-Schwarz inequality gives
$$|\normal^{(n)}|^2 \leq \int_{B_{2^{-n}}} \sum_\mu \normal_\mu^2 ~d\lambda_n = \int_{B_{2^{-n}}} x^2 ~d\lambda_n(x) \leq \sup_{(x, y, z) \in B_{2^{-n}}} x^2.$$
Writing $ds = \sqrt g$ for the line element in $\Hyp^d$, we have
$$2^{-n} = \int_1^x ds = \int_1^x \frac{dt}{t} = \log x$$
if $(x, y, z) \in \partial B_{2^{-n}}$ and $x$ is maximal possible. It follows that
\begin{equation}\label{conormal length bound}
|\normal^{(n)}| \leq \exp(2^{-n})
\end{equation}
and so in particular
$$\Exc_n(u) \geq 0.$$
A similar argument shows that $\Exc_n(u) \geq Exc_{n + 1}(u)$ and that if one defined a ``fractional excess'' $\Exc_r(u)$, where $r$ is a real number in $[n, n + 1]$, then $\Exc_r(u) \in [\Exc_{n + 1}(u), \Exc_n(u)]$.

From the law of cosines and (\ref{conormal length bound}),
\begin{align*}
|\normal^{(n)} - \normal^{(n + m)}|^2 &= |\normal^{(n)}|^2 + |\normal^{(n + m)}|^2 - 2 g(\normal^{(n)}, \normal^{(n + m)})\\
&\leq 2(\exp(2^{-2n}) - g(\normal^{(n)}, \normal^{(n + m)})) \\
&= \frac{2}{\int_{B_{2^{-(n + m)}}} |du| ~\vol} \int_{B_{2^{-(n + m)}}} \exp(2^{-2n}) |du| - \sum_\mu x \partial_\mu u \normal_\mu^{(n)} ~\vol.
\end{align*}
Applying (\ref{conormal length bound}) again,
$$\exp(2^{-2n}) |du| - \sum_\mu x \partial_\mu u \normal_\mu^{(n)} \geq 0$$
on $B_{2^{-n}}$, so it follows that
\begin{align*}
|\normal^{(n)} - \normal^{(n + m)}|^2 &\leq \frac{2}{\int_{B_{2^{-(n + m)}}} |du| ~\vol} \int_{B_{2^{-n }}} \exp(2^{-2n}) |du| - \sum_\mu x \partial_\mu u \normal_\mu^{(n)} ~\vol\\
&= \frac{2}{\int_{B_{2^{-(n + m)}}} |du| ~\vol} (\exp(2^{-2n}) - |\normal^{(n)}|^2) \int_{B_{2^{-n}}} |du| ~\vol.
\end{align*}
We can now apply the inequality
$$x^2 - y^2 \leq 4(x - y),$$
which is valid when $x \geq \max(1, y)$ (thus for $x = \exp(2^{-n})$ and $y = |\normal^{(n)}|$)
to deduce\footnote{An analogous estimate on the approximate derivative for a function on $\RR^d$, without the correction term $\exp(2^{-n}) - 1$, appears in \cite[pg661]{Miranda66}.}
\begin{equation}\label{excess bounds Cauchy sequence}
|\normal^{(n)} - \normal^{(n + m)}|^2 \leq \frac{2^{3 + n(1 - d)}}{\int_{B_{2^{-(n + m)}}} |du| ~\vol} \Exc_n(u).
\end{equation}

In addition to the estimate (\ref{excess bounds Cauchy sequence}), an essential property of the excess is rotation invariance, which we now formulate.

\begin{notation}
We define an action
\begin{equation}\label{hyperbolic rotation}
    \Phi: \Orth(\RR^d) \to \Iso(\Hyp^d)
\end{equation}
of the orthogonal group $\Orth(\RR^d)$, as well as the representation
\begin{align*}
\Orth(\RR^d) &\to \GL(BV_l(\RR^d))\\
A^* u(P) &:= u(\Phi(A)(P))
\end{align*}
of $\Orth(\RR^d)$ on $BV_l(\RR^d)$, as follows.
Recall that the hyperbolic metric in the ball model $\DD^d$ is radial, so the natural action of the orthogonal group $\Orth(\RR^d)$ on $\DD^d$ is by isometry, and if we use the Cayley transform to identify the origin of $\DD^d$ with $O$, then we obtain a faithful action (\ref{hyperbolic rotation}) whose orbits are the spheres $\partial B_r$ centered on $O$.
\end{notation}

\begin{lemma}\label{excess rotation invariant}
For every $A \in \Orth(\RR^d)$ and $u \in BV_l(\Hyp^d)$,
$$\Exc_n(A^* u) = \Exc_n(u).$$
\end{lemma}
\begin{proof}
Since $A$ is an isometry, $|(d A^* u)_{\Phi(A)(P)}| = |(du)_P|$, so
\begin{align*}
\int_{B_{2^{-n}}} |dA^*u| ~\vol &= \int_0^{2^{-n}} \int_{\partial B_r} |dA^* u| ~\vol_{\partial B_r} ~dr = \int_0^{2^{-n}} \int_{\partial B_r} |du| ~\vol_{\partial B_r} ~dr\\
&= \int_{B_{2^{-n}}} |du| ~\vol
\end{align*}
since $(A^{-1})^* \vol_{\partial B_r} = \vol_{\partial B_r}$.
Since $A$ is an isometry, $|A_* (x \partial_\mu)| = 1$ and $(A_*(x \partial_\mu))_\mu$ is an orthonormal basis of $T_O \Hyp^d$.
The claim follows.
\end{proof}

\subsection{Excess: Comparison to Plateau energy}
Our next task is to relate the excess of $1_U$ to the extrinsic geometry of $\partial U$\footnote{For an euclidean analogue, see \cite[TODO]{Miranda66}}.
To do this we will use the machinery of Appendices \ref{coarea section} and \ref{crossproducts} and the following lemma, which is an analogue of \cite[Theorem 4.8]{Giusti77} and allows us to represent $\partial U$ as a graph:

\begin{lemma}\label{hopfKilling}
Let $N$ be a $C^r$ hypersurface in $\Hyp^d$ which bounds an open set, $r \geq 1$, and
$$||\normal_N^\sharp - x\partial_z||_{L^\infty(N)} \leq \kappa^2$$
where $\kappa \in [0, 1)$.
Then there exists a scale $n^* \in \ZZ$ and a function
$$\omega \in C^r(\RR_+ \times \RR^{d - 2} \to \RR)$$
such that
\begin{align}
    N \cap B_{2^{-n^*}} &= \{(x, y, z) \in B_{2^{-n^*}}: z = \omega(x, y)\}, \label{N is a graph}\\
    ||d\omega||_{L^\infty(B_{2^{-n^*}})} &\leq \kappa. \label{derivative bounds}
\end{align}
\end{lemma}
\begin{proof}
We write $N = \partial U$, $u = 1_U$.
From the law of cosines and the fact that $\normal^\sharp$ and $x\partial_z$ both have unit length,
$$||\normal^\sharp - x\partial_z||_{L^\infty(N)}^2 = 2(1 - (\normal, x\partial_z)) = 2\left(1 - \frac{x\partial_z u}{|du|}\right).$$
Therefore
$$x\partial_z u \geq \left(1 - \frac{\kappa^2}{2}\right) |du|.$$
Given $\alpha$, a vector of unit length in $\RR_x \times \RR^{d - 2}_y \times (\RR_+)_z$, we can define a unit vector field
$$X_\alpha = \sum_\mu \alpha^\mu x\partial_\mu.$$
Then
$$\alpha^z x\partial_z u \geq \alpha^z \left(1 - \frac{\kappa^2}{2}\right) |du|$$
so from the Cauchy-Schwarz inequality and the fact that
$$1 - \left(1 - \frac{\kappa^2}{2}\right)^2 \leq \kappa^2,$$
we have
$$\left|\sum_{\mu \neq z} \alpha^\mu x\partial_\mu z\right| \leq \kappa\sqrt{1 - (\alpha^z)^2} |du|$$
and hence
$$X_\alpha u \geq \left(\alpha^z - \frac{\alpha^z \kappa^2}{2} - \kappa\sqrt{1 - (\alpha^z)^2}\right)|du| \geq (\alpha^z - \kappa)|du|.$$
So by Proposition \ref{Giusti46}, if $\alpha^z > \kappa$, then every integral curve of $X_\alpha$ passes through $\partial U$ exactly once.
But the integral curve of $X_\alpha$ which passes through $(0, y, z)$ is the euclidean line
$$t \mapsto (t \alpha^x + y_1 + t \alpha^{y_1}, \dots, y_{d - 2} + t \alpha^{y_{d - 2}}, t \alpha^z)$$
so the existence of a scale $n^*$ and a $C^r$ function $\omega$ satisfying (\ref{N is a graph}, \ref{derivative bounds}) follows from the $C^r$ implicit function theorem.
\end{proof}

We now pause to introduce a large amount of notation that we will use throughout Section \ref{DeGiorgiSection}.

\begin{definition}
The \dfn{hyperbolic Plateau energy} of a $1$-form $\psi$ on $(\RR_+)_x \times \RR^{d - 2}_y$ is the $d-1$-form
$$\Lagrange(\psi) = x^{1 - d} \sqrt{1 + |\psi|^2} ~dxd$$
where the metric is euclidean.
\end{definition}

\begin{notation}[hyperbolic space as a line bundle]\label{hyperbolic line bundle}
If $\omega$ is a $C^r$ function $\Omega \to \RR_z$ with graph $N \subseteq \Hyp^d$, where $\Omega \subseteq (\RR_+)_x \times \RR^{d - 2}_y$ is open, we introduce the locally closed $C^r$ embedding
\begin{align*}
    \Psi_N: \Omega &\to \Hyp^d \\
    (x, y) &\mapsto (x, y, \omega(x))
\end{align*}
which identifies $\Omega$ with $N$.
We also introduce the projection
\begin{align*}
    \Pi: \Hyp^d &\to (\RR_+)_x \times \RR^{d - 2}_y\\
    (x, y, z) &\mapsto (x, y)
\end{align*}
of which $\Psi_N$ is a section.
\end{notation}

It follows immediately from Example \ref{graphs in riemannian manifolds} that
\begin{equation}\label{Lagrangian formula}
\Psi_N^* \vol_N = \Lagrange(d\omega)
\end{equation}
if $N$ is the graph of $\omega \in C^1(\Omega)$.

\begin{notation}[tensor fields on euclidean space]
Since $\RR_+ \times \RR^{d - 2}$ has flat Levi-Civita connection, we can identify tensor fields on $\RR_+ \times \RR^{d - 2}$ with smooth maps $\RR_+ \times \RR^{d - 2} \to \Hilb$ for some finite-dimensional Hilbert space $\Hilb$.
Using the Bochner integral (see Appendix \ref{coarea section}), it therefore makes sense to integrate a tensor field to obtain an element of $\Hilb$ (that is, a tensor), and we write
$$A_n T = \dashint_{\Pi(B_{2^{-n}})} T$$
for the Bochner mean of $T$ over $\Pi(B_{2^{-n}})$ with respect to Lebesgue measure.
It also makes sense to identify a tensor with a constant tensor field, which we will do without mention.
\end{notation}

\begin{notation}[miscellany]
We write $\tilde B_r$ for the euclidean ball in $\RR_+ \times \RR^{d - 2}$ centered on $(1, 0, \dots, 0)$ of radius $r > 0$.
We write $x[a,b] = [ax, ab]$ and $x + [a, b] = [a + x, b + x]$.
\end{notation}

\begin{lemma}\label{excess vs plateau energy}
Let $\omega \in C^1(\Omega)$. Then
\begin{equation}
    \Exc_n(1_{\omega(x, y) < z}) \in 2^{n(d - 1)} \exp(2^{-n}) \int_{\Pi(B_{2^{-n}})} \Lagrange(d\omega) - [\exp(-2^{-2n}), 1]\Lagrange(A_n d\omega). \label{excess vs Lagrangian}
\end{equation}
\end{lemma}
\begin{proof}
Let $u(x, y, z) = 1_{\omega(x, y) < z}$, let $N$ be the graph of $\omega$, and let $\normal^{(n)}$ be the approximate derivative of $u$.
It follows from the definitions, the fact that $|du|$ is the Radon-Nikod\'ym derivative $\vol_N/\vol$, and (\ref{Lagrangian formula}), that
$$\normal^{(n)}_\mu = \int_{B_{2^{-n}}} \normal_\mu |du| ~\vol = \int_{B_{2^{-n}} \cap N} \normal_\mu \vol_N = \int_{\Pi(B_{2^{-n}})} (\Psi_N^* \normal)_\mu \Lagrange(d\omega).$$
From Example \ref{graphs in riemannian manifolds},
$$(\Psi^* \normal)_z \Lagrange(d\omega) = x^p ~dxdy$$
where $p = 3/2 - d$. Similarly we have
$$(\Psi^* \normal)_i \Lagrange(d\omega) = x^p \partial_i \omega(x, y) ~dxdy$$
whenever $i \in \{x, y_1, \dots, y_{d - 2}\}$.
Therefore
\begin{align*}
    |\normal^{(n)}|^2 \left(\int_{B_{2^{-n}}} |du| ~\vol\right)^2 &= \left(\int_{\Pi(B_{2^{-n}})} x^p ~dxdy\right)^2 + \sum_i \left(\int_{\Pi(B_{2^{-n}})} x^p \partial_i \omega(x, y) ~dxdy \right)^2 \\
    &\in |\Pi(B_{2^{-n}})|^2 (1 + |A_nd\omega|^2) \left[\inf_{(x, y, z) \in B_{2^{-n}}} x^{2p}, \sup_{(x, y, z) \in B_{2^{-n}}} x^{2p}\right] \\
    &= \left(\int_{\Pi(B_{2^{-n}})} \Lagrange(A_nd\omega)\right)^2 \left[\inf_{(x, y, z) \in B_{2^{-n}}} x, \sup_{(x, y, z) \in B_{2^{-n}}} x\right].
\end{align*}
From the interval arithmetic identity
$$[\alpha x, \alpha x] - [\alpha^{-1}y, \alpha y] = \alpha(x - [\alpha^{-2}, 1]y),$$
valid for $\alpha \geq 1$ (and hence $\alpha = \exp(2^{-n})$),
we conclude (\ref{excess vs Lagrangian}).
\end{proof}

%%%%%%%%%%%%%%%%%%%%%%%%%%%%%%%%%%%%%%%%%%%%%%%%%

\subsection{de Giorgi lemma: Comparison to Dirichlet energy}
The main idea in the proof of regularity is to iterate the de Giorgi lemma, a certain estimate on the excess, which in the euclidean case is given by \cite[TODO]{Miranda66}.
The proof of the de Giorgi lemma proceeds in three steps: first, one approximates the Plateau energy by the Dirichlet energy and applies the mean-value property of harmonic functions; second, one shows that the de Giorgi lemma holds for $C^1$ functions using an estimate similar to Lemma \ref{excess vs plateau energy}; and finally, one shows that the de Giorgi lemma holds in general using an estimate similar to Proposition \ref{mollifier quant} and rotation-invariance of the excess.

\begin{notation}
Let $\DirL(\psi)$ denote the Dirichlet energy of a $1$-form on $(\RR_+)_x \times \RR^{d - 2}_y$, which is the $d-1$-form
$$\DirL(\psi) = \frac{|\psi|^2}{2} ~dxdy,$$
where as usual the norm is euclidean.
Thus $dh$ is a minimizer of $\DirL$ iff $h$ is harmonic.
\end{notation}

To compare the Dirichlet and Plateau energies, let $\psi_1, \psi_2$ be $1$-forms.
We suppress the factor of $dxdy$.
From Taylor's theorem, there exists $\xi \in [|\psi_1|, |\psi_2|]$ such that
\begin{equation}\label{Taylor remainder Dirichlet}
\Lagrange(\psi_1) - \Lagrange(\psi_2) = x^{1 - d}\left(\frac{|\psi_1|^2 - |\psi_2|^2}{2\sqrt{1 + |\psi_2|^2}} - \frac{(|\psi_1|^2 - |\psi_2|^2)^2}{8(1 + \xi^2)^{3/2}}\right).
\end{equation}
The second term of (\ref{Taylor remainder Dirichlet}) is negative and $\sqrt{1 + |\psi_2|^2} \geq 1$, so it follows that
\begin{equation}\label{Taylor lower bound}
\Lagrange(\psi_1) - \Lagrange(\psi_2) \leq x^{1 - d} (\DirL(\psi_1) - \DirL(\psi_2)).
\end{equation}
If in addition $||\psi_2||_{L^\infty} \leq 1$, then
$$4\sqrt{1 + |\psi_2|^2} \leq 8 \leq 8(1 + \xi^2)^{3/2}$$
so we conclude
\begin{equation}\label{Taylor upper bound}
\Lagrange(\psi_1) - \Lagrange(\psi_2) \geq \frac{x^{1 - d}}{\sqrt{1 + |\psi_2|^2}} (\DirL(\psi_1) - \DirL(\psi_2) - (\DirL(\psi_1) - \DirL(\psi_2))^2).
\end{equation}

The utility of the Dirichlet energy is that for every harmonic function $h$, by \cite[Lemma 4.1]{Miranda66} (ultimately a consequence of the mean-value property),
\begin{equation}\label{Miranda41}
\int_{\tilde B_{2^{-(n+1)}}} \DirL(dh) - \DirL(A_n dh) \leq 2^{-(d + 1)} \int_{\tilde B_{2^{-n}}} \DirL(dh) - \DirL(A_n dh).
\end{equation}
Moreover, the mean-value property of $dh$ implies that for every $1$-form $\psi$,
\begin{equation}\label{MVP derivative}
\int_{\tilde B_{2^{-n}}} \DirL(dh - \psi) = \int_{\tilde B_{2^{-n}}} \DirL(dh) - \DirL(\psi).
\end{equation}

We are now ready to complete the first step of the proof of the de Giorgi lemma, analogous to \cite[Teorema 4.3]{Miranda66}.
On first reading, the reader may take $c = 10^{-3}$, $O(c) = 10^{-1}$ and $n^* = 10$; in fact, $c$ will later be chosen to be a dimensional constant.
Roughly speaking, one should think of $\beta$ as comparable to $2^{-(n - n^*)}$, and $\kappa$ as the ``error incurred by mollification".

\begin{lemma}\label{DGL1}
For every $c > 0$ there exists a scale $n^* \in \ZZ$ with the following property:

Let $\omega \in C^1(\Omega)$, suppose that $\kappa, \beta \in (0, 1)$ and $n \geq n^*$ satisfy
\begin{align}
||d\omega||_{L^\infty(\tilde B_{2^{-n}})} &\leq \kappa, \label{DGL1 1}\\
\int_{\tilde B_{2^{-n}}} \Lagrange(d\omega) - \Lagrange(A_n d\omega) &\leq \beta, \label{DGL1 2}\\
\int_{\tilde B_{2^{-n}}} \Lagrange(d\omega) &\leq \eta(\{(x, y, z) \in \Hyp^d: z < \omega(x, y)\}, 2^{-n}) + \beta \kappa. \label{DGL1 3}
\end{align}
Then
$$\int_{\tilde B_{2^{-(n + 1)}}}\Lagrange(d\omega) - \Lagrange(A_{n + 1} d\omega) \leq (1 + O(c)) 2^{-(d + 1)} \beta + O(\beta \sqrt \kappa)$$
where all constants only depend on $d$.
\end{lemma}
\begin{proof}
Choose $n^*$ so large that
\begin{equation}\label{x to c}
1 - c \leq \inf_{(x, y) \in \tilde B_{2^{-n^*}}} x^{1 - d} < \sup_{(x, y) \in \tilde B_{2^{-n^*}}} x^{1 - d} \leq 1 + c
\end{equation}
and suppose that $n \geq n^*$.
Let $h$ be the harmonic function on $\tilde B_{2^{-n}}$ such that
\begin{equation}\label{trace equation}
h = \omega \text{ on } \partial \tilde B_{2^{-n}}.
\end{equation}
By definition of $\eta$ and (\ref{DGL1 3}),
\begin{align*}
\int_{\tilde B_{2^{-(n + 1)}}} \Lagrange(d\omega) - \Lagrange(dh)
&\leq \int_{\tilde B_{2^{-n}}} \Lagrange(d\omega) - \eta(\{(x, y, z) \in \Hyp^d: z < \omega(x, y)\}, 2^{-n}).
\end{align*}
Therefore
\begin{equation}\label{bound on domega - dh}
\int_{\tilde B_{2^{-(n + 1)}}} \Lagrange(d\omega) - \Lagrange(dh) \leq \beta\kappa.
\end{equation}

We now follow the proof of \cite[Lemma 4.2]{Miranda66}.
By (\ref{Taylor lower bound}, \ref{x to c}),
$$\int_{\tilde B_{2^{-(n + 1)}}} \Lagrange(d\omega) - \Lagrange(A_{n + 1}d\omega) \leq (1 + c)\int_{\tilde B_{2^{-(n + 1)}}} \DirL(d\omega) - \DirL(A_{n + 1}d\omega).$$
Since $A_{n + 1}d\omega$ is the mean of $d\omega$, for every $\varepsilon \in (0, 1)$,
\begin{align*}
\int_{\tilde B_{2^{-(n + 1)}}} \DirL(d\omega) - \DirL(A_{n + 1}d\omega)
&\leq \int_{\tilde B_{2^{-(n + 1)}}} \DirL(d\omega - A_nd\omega) \\
&\leq (1 + \varepsilon^{-1}) \int_{\tilde B_{2^{-(n + 1)}}} \DirL(d(\omega - h)) \\
&\qquad +(1 + \varepsilon) \int_{\tilde B_{2^{-(n + 1)}}} \DirL(dh - A_nd\omega) \\
&=: O(\varepsilon^{-1}) I + (1 + \varepsilon) J.
\end{align*}
From the positivity of Dirichlet energy and (\ref{MVP derivative}, \ref{Taylor upper bound}),
\begin{align*}
I &\leq \int_{\tilde B_{2^{-n}}} \DirL(d(\omega - h)) = \int_{\tilde B_{2^{-n}}} \DirL(d\omega) - \DirL(dh) \lesssim \int_{\tilde B_{2^{-n}}} \Lagrange(d\omega) - \Lagrange(dh)
\end{align*}
so by (\ref{bound on domega - dh}),
\begin{equation}\label{bound on I}
I \lesssim \beta\kappa.
\end{equation}
Moreover, by (\ref{MVP derivative}),
$$J = \int_{\tilde B_{2^{-(n + 1)}}} \DirL(dh) - \DirL(A_nd\omega).$$
From (\ref{trace equation}) and Stokes' theorem, there are constants $C_m > 0$ such that
$$A_m d\omega = C_m \int_{\partial \tilde B_{2^{-m}}} \omega ~dS = C_m \int_{\partial \tilde B_{2^{-m}}} h ~dS = A_m dh$$
which along with (\ref{Miranda41}, \ref{MVP derivative}) implies that
$$J \leq 2^{-(d + 1)} \int_{\tilde B_{2^{-n}}} \DirL(dh) - \DirL(A_nd\omega) = 2^{-(d + 1)} \int_{\tilde B_{2^{-n}}} \DirL(dh - A_n d\omega).$$
We further estimate, using (\ref{bound on I}),
\begin{align*}
J &\leq (1 + \varepsilon) 2^{-(d + 1)} \int_{\tilde B_{2^{-n}}} \DirL(d\omega - A_n d\omega) + O(\varepsilon^{-1} I) \\
&:= (1 + \varepsilon) 2^{-(d + 1)} K + O(\varepsilon^{-1} \beta \kappa).
\end{align*}
To estimate $K$ we apply (\ref{MVP derivative}, \ref{x to c}, \ref{Taylor upper bound}) to obtain
\begin{align*}
K &= \int_{\tilde B_{2^{-n}}} \DirL(d\omega) - \DirL(A_n d\omega) \\
&\leq (1 + O(c)) \int_{\tilde B_{2^{-n}}} \Lagrange(d\omega) - \Lagrange(A_n d\omega) + O(1) \int_{\tilde B_{2^{-n}}} (\DirL(d\omega) - \DirL(A_n d\omega))^2.
\end{align*}
From (\ref{DGL1 1}, \ref{DGL1 2}), it follows that
\begin{align*}
K &\leq (1 + O(c))\beta + O(||d\omega||_{L^\infty(\tilde B_{2^{-n^*}})}) \int_{\tilde B_{2^{-n}}} \DirL(d\omega) - \DirL(A_n d\omega)\\
&\leq \beta(1 + O(c + \kappa)).
\end{align*}
If we set $\varepsilon = \sqrt \kappa$ then it follows that
\begin{align*}
\int_{\tilde B_{2^{-(n+1)}}} \DirL(d\omega) - \DirL(A_n d\omega) &\leq (1 + O(c)) 2^{-(d + 1)} \beta + O(\beta \sqrt \kappa). \qedhere
\end{align*}
\end{proof}

%%%%%%%%%%%%%%%%%%%%%%%%%%%%%%%%%%%%%%%%%%%%%%%%%

\subsection{de Giorgi lemma: \texorpdfstring{$C^1$}{C1} and general cases}
The next step is the $C^1$ case, which is analogous to \cite[Teorema 4.4]{Miranda66}.

\begin{lemma}[de Giorgi lemma on $\Hyp^d$, $C^1$ case]\label{DGL2}
For every $c > 0$ there exists a scale $n^* \in \ZZ$ with the following property:

Let $N$ be a $C^1$ hypersurface in $B_{2^{-n}}$, $n \geq n^*$, with unit normal field $\normal^\sharp$, such that $N$ bounds an open set $U$.
If $\kappa \in (0, 1), \alpha \in \RR_+$ are parameters such that
\begin{align}
\Exc_n(U) &\leq \alpha, \label{DGL2 1}\\
|N \cap B_{2^{-n}}| &\leq \eta(U, B_{2^{-n}}) + 2^{n(1 - d)}\alpha \kappa, \label{DGL2 2}\\
||\normal^\sharp - x\partial_z||_{L^\infty(N \cap B_{2^{-n}})} &\leq \kappa^2, \label{DGL2 3}
\end{align}
then
$$\Exc_{n + 1}(U) \leq \frac{1 + O(c)}{2} \alpha + O(\alpha \sqrt \kappa) + O(4^{-n}).$$
\end{lemma}
\begin{proof}
From Lemma \ref{hopfKilling} and (\ref{DGL2 3}), there exists $n_1 \in \ZZ$ and a function $\omega \in C^1(\RR_+ \times \RR^{d - 2})$ satisfying the derivative bound (\ref{DGL1 1}) for any $n \geq n_1$
and such that the graph of $\omega$ over $\Pi(B_{2^{-n_1}})$ is $N \cap A_1$.
Let $\varepsilon > 0$; then we can find a scale $n_2 \geq n_1$ such that if $n \geq n_2$ then
$$\Pi(B_{(1 - \varepsilon) 2^{-(n+1)}}) \subseteq \tilde B_{2^{-(n+1)}} \text{ and } \tilde B_{2^{-n}} \subseteq \Pi(B_{(1 + \varepsilon) 2^{-n}}).$$
Up to a multiplicative loss of $1 + c$, all quantities defined for $(1 - \varepsilon)2^{-(n + 1)}$ can be replaced with $2^{-(n + 1)}$; similarly for $2^{-n}$ and $(1 + \varepsilon) 2^{-n}$ (TODO: Justify this) for $\varepsilon$ chosen small enough depending on $c$.
In order to apply Lemma \ref{DGL1} we apply Lemma \ref{excess vs plateau energy} and (\ref{DGL2 1}) to obtain
$$\int_{\tilde B_{2^{-n}}} \Lagrange(d\omega) - \Lagrange(A_n d\omega) \leq (1 + c)2^{n(1 - d)}\alpha.$$
Similarly from (\ref{Lagrangian formula}, \ref{DGL2 2}), we obtain
$$\int_{\tilde B_{2^{-n}}} \Lagrange(d\omega) \leq \eta(U, B_{2^{-n}}) + (1 + c) 2^{n(1-d)}\alpha \kappa,$$
and so we have met the hypotheses of Lemma \ref{DGL1} with
$$\beta = (1 + c)2^{n(1 - d)}\alpha.$$
Therefore there exists $n_3 \geq n_2$ such that for every $n \geq n_3$,
$$\int_{\tilde B_{2^{-(n+1)}}} \Lagrange(d\omega) - \Lagrange(A_nd\omega) \leq 2^{(n + 1)(1-d)}\left[ \frac{1 + O(c)}{2} \alpha + O(\alpha \sqrt \kappa)\right].$$
To convert this estimate back into a result about the excess we use Lemma \ref{excess vs plateau energy}.
There exists a scale $n^* \geq n_3$ such that if $n \geq n_4$ then $\exp(2^{-(n + 1)}) \leq 1 + c$, so
\begin{align*}
\Exc_{n + 1}(U) &\leq (1 + O(c)) 2^{(n+1)(d-1)} \int_{\tilde B_{2^{-(n+1)}}} \Lagrange(d\omega) - \exp(-2^{-n}) \Lagrange(A_nd\omega)\\
&\leq O(4^{-n}) + (1 + O(c)) 2^{(n+1)(d-1)} \int_{\tilde B_{2^{-(n+1)}}} \Lagrange(d\omega) - \Lagrange(A_nd\omega)\\
&\leq O(4^{-n}) + \frac{1 + O(c)}{2} \alpha + O(\alpha \sqrt \kappa). \qedhere
\end{align*}
\end{proof}

We now prove the de Giorgi lemma.
The proof settles the choice of the dimensional constant $c$.

\begin{proposition}[de Giorgi lemma on $\Hyp^d$, general case]\label{DGL 3}
There exist $\sigma, n^*, C > 0$ such that for every set $U$ of least perimeter in $B_{2^{-n}} \subseteq \Hyp^d$,
$$\Exc_n(U) < \sigma \text{ and } n \geq n^* \implies \Exc_{n+1}(U) \leq \frac{51}{100} \Exc_n(U) + \frac{C}{4^n}.$$
\end{proposition}
\begin{proof}
By Lemma \ref{excess rotation invariant} and the fact that $x\partial_z$ is a unit vector field, it is no loss of generality to assume that
$$\normal^{(n)}_U = |\normal^{(n)}_U| x\partial_z.$$
Under this assumption, if we write $u = 1_U$ then
$$\Exc_n(U) = 2^{n(d - 1)} \exp(2^{-n}) \int_{B_{2^{-n}}} |du| ~\vol - 2^{n(d - 1)} \int_{B_{2^{-n}}} x\partial_z u ~\vol.$$
The injectivity radius of $O$ is infinite and so, if $\sigma < \exp(2^{-n}) \gamma_*$, we obtain from Proposition \ref{mollifier quant}, for every $\kappa > 0$, a $C^1$ hypersurface $N$ which bounds an open set $V$ which, if $n$ is chosen large enough (TODO: Justify taking $t \to 1$, $\varepsilon \to 0$), satisfies
\begin{align*}
|N \cap B_{2^{-n}}| &\leq \eta(V, B_{2^{-n}}) + \kappa \Exc_n(U, B_{2^{-n}}) \\
\Exc_n(V, B_{2^{-n}}) &\in \Exc_n(U, B_{2^{-n}})[1 - \kappa, 1 + \kappa] \\
||\normal^\sharp_N - x\partial_z||_{L^\infty} &\leq \kappa.
\end{align*}
Therefore by Lemma \ref{DGL2}, there exist dimensional constants $C_1,C_2,C > 0$ such that
$$\Exc_{n + 1}(V) \leq \frac{1 + C_1c}{2} \Exc_n(U, B_{2^{-n}}) + C_2 \Exc_n(U, B_{2^{-n}}) \sqrt \kappa + \frac{2C}{4^n}.$$
We now choose $c = C_1/50$, which yields a constant $C_3 > 0$ such that
$$\Exc_n(U, B_{2^{-n}}) \leq \frac{51}{100} \Exc_n(U, B_{2^{-n}}) + C_3 \Exc_n(U, B_{2^{-n}}) \sqrt \kappa + \frac{2C}{4^n}.$$
We can then choose $\kappa$ small enough that
$$C_3 \Exc_n(U, B_{2^{-n}}) \sqrt \kappa \leq \frac{C}{4^n}$$
to complete the proof.
\end{proof}

%%%%%%%%%%%%%%%%%%%%%%%%%%%%%%%%%%%%%%%%%%%%%%%%%

\subsection{Induction on scale}
We now prove Theorem \ref{main lma}.

The first case that we consider is when $U$ is a set of locally finite perimeter for which there exists a scale $n_2$ such that $U$ has least perimeter in $B_{2^{-n_2}}$ and
\begin{equation}\label{induction on scale:base case}
    \Exc_{n_2}(U) < \sigma
\end{equation}
where $\sigma, n_1$ are the constants given by de Giorgi's lemma, Proposition \ref{DGL 3}.
A straightforward induction shows that if $b > a$, $c$ are real numbers and $(x_n) \subset \RR_+$ is a sequence such that
\begin{equation}\label{induction hypothesis}
x_n \leq \frac{x_{n - 1}}{a} + \frac{c}{b^n},
\end{equation}
then
\begin{equation}\label{induction conclusion}
x_n \leq a^{-n}\left[x_0 + \frac{bc}{b - a}\right].
\end{equation}
Applying (\ref{induction hypothesis}, \ref{induction on scale:base case}) and de Giorgi's lemma, we obtain (\ref{induction hypothesis}) with $x_n = \Exc_{n - n^*}(U)$, $n^* = \max(n_1, n_2)$, $a = 100/51$, $b = 4$, and $c = C$.
So (\ref{induction conclusion}) reads 
$$\Exc_n(U) \leq (0.51)^{n - n^*} \left[\sigma + \frac{51}{26}C\right] \lesssim_{n_2} (0.51)^n$$
and hence from (\ref{excess bounds Cauchy sequence}),
$$|\normal^{(n)} - \normal^{(n + m)}|^2 \lesssim_{n_2} \frac{(0.51)^{nd}}{|\partial^* U \cap B_{2^{-(n + m)}}}.$$
So Proposition \ref{doubling dimension} implies that 
$$|\normal^{(n)} - \normal^{(n + 1)}|^2 \lesssim_{n_2} 2^{n(d - 1)} (0.51)^{nd} \lesssim 2^{n(d - 1) - 0.97 nd} = 2^{0.03nd - n}.$$
However, as $d \leq 7$\footnote{One could just as well replace $51/100$ with a function of $d$ here. The only \emph{essential} use of $d \leq 7$ is in Proposition \ref{blowup theorem}.}, we conclude 
$$|\normal^{(n)} - \normal^{(n + 1)}| \lesssim_{n_2} \alpha^n$$
where $\alpha = 2^{-0.35}$, and hence 
$$|\normal^{(n)} - \normal^{(n + m)}| \lesssim_{n_2} \sum_{\ell=n}^\infty \alpha^\ell < \alpha^n.$$
From Proposition \ref{LebDiff}, it follows that 
\begin{equation}\label{normal convergence rate}
    |\normal^{(n)} - \normal| \lesssim_{n_2} \alpha^n.
\end{equation}

Now let $U$ be an arbitrary set of locally finite perimeter, which has least perimeter in an open set $V$.
If $P, Q \in \partial U \cap V$ are close enough, $P \in \partial^* U$, there is a unique geodesic $\gamma_Q$ from $Q$ to $P$.
The transitivity of $\Iso(\Hyp^d)$ means that we may assume that $P = O$.
Integrating the Killing vectors which are tangent to $\gamma$, we obtain an isometry $A_Q \in \Iso(\Hyp^d)$ which maps $O$ to $Q$.
We can then define $\normal^{(n)}_U(Q) \in T_Q' V$ by first computing $\normal^{(n)}_{A_Q^* U} \in T_O' V$, and then applying the linear map $T_O' V \to T_Q' V$ induced by $A_Q$ and the Levi-Civita connection of $\Hyp^d$.
Thus $\normal^{(n)}_U$ is a $|d1_U|\vol$-measurable $1$-form on $\Hyp^d$.
An inspection of the definitions shows that $\normal^{(n)}_U$ is continuous on $\partial^* U$.
We denote by $n_2(Q)$ the scale in (\ref{induction on scale:base case}), or $n_2(Q) = \infty$ if no such scale exists.

We claim that if $Q_k \to O$, then $n_2(Q_k)$ is eventually equal to $n_2(O)$, and is finite.
Let $\Hilb \subseteq T_O\Hyp^d$ denote the tangent space to $\partial^* U$ given by Proposition \ref{blowup theorem}.
We can approximate $\Hilb$ by the tangent rescalings of $\partial^* U$ at $P$, and in turn approximate those by the tangent rescalings of $\partial^* U$ at $Q_k$ if $k$ is large (translated to $T_P \Hyp^d$ by the isometry $A_Q$).
On the other hand it is clear that, if $Z$ is a half-space bounded by $\Hilb$, then $\Exc_n((\exp_P)_* Z) \lesssim 2^{-n}$.
In particular there exists $n_3 \in \ZZ$ such that $\Exc_{n_3}((\exp_P)_* Z) < \sigma/2$.
For $k$ large, it follows that $n_2(Q_k) \leq n_3$.
We omit the details here, as they are extremely similar to \cite[Teorema 4]{Miranda67}.

Since the rate of convergence in (\ref{normal convergence rate}) only depends on $n_2$, it follows that $\normal^{(n)} \to \normal$ locally uniformly on $\overline{\partial^* U}$, so that $\normal$ is continuous on $\overline{\partial^* U}$.
Therefore Proposition \ref{locality of Caccioppoli} implies that $\partial U$ is a $C^1$ minimal hypersurface.
Elliptic theory \cite{morrey2009multiple} now implies that $\partial U$ is as smooth as possible.