\section{Regularity of minimal hypersurfaces}\label{DeGiorgiSection}
In this section we prove specialize to $\Hyp^d$ and show regularity of hypersurfaces, Theorem \ref{main lma}.

We now fix some notation for the remainder of this section.
We introduce the coordinate system
$$(x, y, z) \in \RR_+ \times \RR^{d - 2} \times \RR = \Hyp^d$$
with the hyperbolic half-space metric
\begin{equation}\label{hyperbolic metric}
g = \frac{dx^2 + dy_1^2 + \cdots + dy_{d - 2}^2 + dz^2}{x^2}.
\end{equation}
This does not agree with the usual convention for the hyperbolic half-space metric, but will be advantageous for our purposes when we represent $C^1$ hypersurfaces as graphs of functions $\omega: \RR_+ \times \RR^{d - 2} \to \RR$; then such a hypersurface can be written $\{z = \omega(x, y)\}$.
We also define the \dfn{hyperbolic origin} $O = (1, 0, 0)$ and write $B_r = B(O, r)$.
Whenever we sum over an index $\mu$, we mean that $\mu$ ranges over
$$\mu \in \{x, y_1, \dots, y_{d - 2}, z\}.$$
However, the variable $y$ plays no meaningful r\^ole in the proof of Theorem \ref{main lma} and the reader can specialize to the case $d = 2$ without oversimplifying the proof.
We also write $B_r$ for the hyperbolic of radius $r$, and record that 
\begin{equation}\label{sup in a ball}
\sup_{(x, y, z) \in B_r} x = e^r.
\end{equation}

\begin{definition}
Let $u \in BV_l(\Hyp^d)$. The \dfn{approximate derivative} to $u$ at scale $n$ is the covector in $T_O' \Hyp^d$ defined by
$$\normal^{(n)}_\mu = \frac{\int_{B_{2^{-n}}} x \partial_\mu u ~\vol}{\int_{B_{2^{-n}}} |du| ~\vol}.$$
\end{definition}

Since $g(x\partial_\mu, x\partial_\nu) = \delta_{\mu\nu}$, $x\partial_\mu u = \normal_\mu|du|$.
Therefore, if $O$ is a Lebesgue point and $u = 1_U$, then $\normal^{(n)} \to \normal_U(O)$.

We can use the action of $\Iso(\Hyp^d)$ to define approximations to points $\normal_U(P)$, $P \in \Hyp^d$, as well.
To do this, let $A \in \Iso(\Hyp^d)$ map $O$ to $P$; then 
$$\lim_{n \to \infty} (A^{-1})^* \normal^{(n)}(A^* u) = \normal_U(P)$$
if $P$ is a Lebesgue point of $A^* u$.
Actually, given a small open subset $\Omega$ of $\Hyp^d$, we can find a smooth family of isometries $A_P$ of $\Hyp^d$ parametrized by $P \in \Omega$ with $A_P(O) = P$.
It is clear that $(A^{-1})^* \normal^{(n)}(A^* u)$ depends continuously on $A$, and for almost every $P \in \Omega$, $P$ is a Lebesgue point.
So
$$\lim_{n \to \infty} (A_P^{-1})^* \normal^{(n)}(A_P^* u) = \normal_U(P)$$
is an almost-everywhere approximation of $\normal_U$ by a continuous $1$-form.
So our task is to show that $(\normal^{(n)})$ is a Cauchy sequence whose rate of convergence is unaffected by conjugation by isometries.

%%%%%%%%%%%%%%%%%%%%%%%%%%%%%%%%%%%%%%
\subsection{Representation as graphs}

We now show that $C^r$ hypersurfaces $N$ in $\Hyp^d$ can be represented as graphs of functions on $\RR_+ \times \RR^{d - 2}$ in a quantitative way.
This allows us to reduce the study of $N$ to the study of a weighted Plateau equation on $\RR_+ \times \RR^{d - 2}$.\footnote{One can identify $\RR_+ \times \RR^{d - 2}$ with $\Hyp^{d - 1}$, but we do \emph{not} do so here. Any balls, norms, et cetra, on $\RR_+ \times \RR^{d - 2}$ are meant in the euclidean sense.}
This result is a generalization of \cite[Theorem 4.8]{Giusti77}, and its proof is similar, but one needs to check that the integral curves are straight lines.

\begin{notation}
Let $\tilde B_r$ be the ball in $\RR_+ \times \RR^{d - 2}$ centered on $(1, 0, \dots, 0)$ of radius $r > 0$.
\end{notation}

\begin{lemma}\label{hopfKilling}
Let $N$ be a $C^r$ hypersurface in $\Hyp^d$ which bounds an open set, $r \in (0, 1)$, and
\begin{equation}\label{kappa-correct alignment}
||\normal_N^\sharp - x\partial_z||_{L^\infty(\tilde B_r \times \RR)} \leq \kappa^2
\end{equation}
where $\kappa \in [0, 1)$.
Then there exists a function $\omega \in C^r(\tilde B_r \to \RR)$
such that
\begin{align}
    N \cap (\tilde B_r \times \RR) &= \{(x, y, z) \in \tilde B_r\times \RR: z = \omega(x, y)\}, \label{N is a graph}\\
    ||d\omega||_{L^\infty(\tilde B_r)} &\leq \kappa. \label{derivative bounds}
\end{align}
\end{lemma}
\begin{proof}
We write $N = \partial U$, $u = 1_U$.
From the law of cosines and the fact that $\normal^\sharp$ and $x\partial_z$ both have unit length,
$$||\normal^\sharp - x\partial_z||_{L^\infty(N)}^2 = 2(1 - (\normal, x\partial_z)) = 2\left(1 - \frac{x\partial_z u}{|du|}\right).$$
Therefore
$$x\partial_z u \geq \left(1 - \frac{\kappa^2}{2}\right) |du|.$$
Given $\alpha$, a vector of unit length in $\RR_x \times \RR^{d - 2}_y \times (\RR_+)_z$, we can define a unit vector field
$$X_\alpha = \sum_\mu \alpha^\mu x\partial_\mu.$$
Then
$$\alpha^z x\partial_z u \geq \alpha^z \left(1 - \frac{\kappa^2}{2}\right) |du|$$
so from the Cauchy-Schwarz inequality and the fact that
$$1 - \left(1 - \frac{\kappa^2}{2}\right)^2 \leq \kappa^2,$$
we have
$$\left|\sum_{\mu \neq z} \alpha^\mu x\partial_\mu z\right| \leq \kappa\sqrt{1 - (\alpha^z)^2} |du|$$
and hence
\begin{equation}\label{hypothesis for Giusti47}
X_\alpha u \geq \left(\alpha^z - \frac{\alpha^z \kappa^2}{2} - \kappa\sqrt{1 - (\alpha^z)^2}\right)|du| \geq (\alpha^z - \kappa)|du|.
\end{equation}
Integrating $X_\alpha$ with initial conditions $P = (x_0, y_0, z_0) \in N$, we obtain the integral curve $(x(t), y(t), z(t))$,
\begin{align*}
x(t) &= x_0 e^{\alpha^x t}, \\
y_i(t) &= (y_0)_i + x_0 \frac{\alpha^{y_i}}{\alpha^x}(e^{\alpha^x t} - 1),\\
z(t) &= z_0 + x_0 \frac{\alpha^z}{\alpha^x}(e^{\alpha^x t} - 1)
\end{align*}
which extends to the line $\ell_\alpha = \{P + s\alpha: s \in \RR\}$ for $\alpha$ fixed. 
By \cite[Remark 4.7]{Giusti77} and (\ref{hypothesis for Giusti47}), $N$ meets the cone $\bigcup_{\alpha^z > \kappa} \ell_\alpha$
only at $P$. This implies that $N$ is the graph of a function $\omega$, and 
$$|\omega(x_2, y_2) - \omega(x_1, y_1)|^2 \leq \kappa^2 |x_2 - x_1|^2 + \kappa^2 |y_2 - y_1|^2$$
which gives (\ref{derivative bounds}).
\end{proof}

The hypothesis (\ref{kappa-correct alignment}) is important enough that we make the following definition which asserts the normal vector points in the $z$ direction \emph{on average} close to $O$.\footnote{It makes no sense to ask that the normal vector points in the $z$ direction at $O$, since $\normal^\sharp_U$ is just a measurable section and so may not be well-defined at $O$.}

\begin{definition}
A set $U$ of locally finite perimeter is \dfn{correctly aligned} at scale $n \in \ZZ$ if 
$$(\normal^{(n)}_U)^\sharp = |\normal^{(n)}_U|\partial_z.$$
\end{definition}

Since we frequently will represent use Lemma \ref{hopfKilling} to replace $C^r$ hypersurfaces with graphs, we introduce the following notation.

\begin{notation}[hyperbolic space as a line bundle]\label{hyperbolic line bundle}
    If $\omega$ is a $C^r$ function $\Omega \to \RR_z$ with graph $N \subseteq \Hyp^d$, where $\Omega \subseteq (\RR_+)_x \times \RR^{d - 2}_y$ is open, we introduce the locally closed $C^r$ embedding
    \begin{align*}
        \Psi_N: \Omega &\to \Hyp^d \\
        (x, y) &\mapsto (x, y, \omega(x))
    \end{align*}
    which identifies $\Omega$ with $N$.
    We also introduce the projection
    \begin{align*}
        \Pi: \Hyp^d &\to (\RR_+)_x \times \RR^{d - 2}_y\\
        (x, y, z) &\mapsto (x, y)
    \end{align*}
    of which $\Psi_N$ is a section.
\end{notation}

Our next task is to show that $\omega$ solves a Plateau-type PDE if $N$ is a minimal hypersurface in $\Hyp^d$.
To that end, suppose that we are given a $d$-dimensional oriented real Hilbert space $(\Hilb, a)$, and an oriented basis $(\partial_0, \dots, \partial_{d - 1})$ which may not be orthonormal but which satisfies for every $i$
\begin{equation}\label{0th coordinate orthogonal}
a_{0i} = 0
\end{equation}
where $a_{\mu\nu} = a(\partial_\mu, \partial_\nu)$.
Here and in what follows Greek indices range over $0, \dots, d - 1$ while Latin indices range over $1, \dots, d - 1$\footnote{but $a$ is honestly an inner product, and not a Lorentz form}.

We write $a_{\hat 0 \hat 0}$ for the inner product on the span $\Hilb_0^\perp$ of $\partial_1, \dots, \partial_{d - 1}$ defined by $(a_{\hat 0 \hat 0})_{ij} = a_{ij}$.
We as usual write $a^{\mu\nu}$, $(a^{\hat 0 \hat 0})^{ij}$ for the components of the dual inner product, and $\delta_\mu^\nu$ for the components of the identity matrix.

\begin{definition}
The \dfn{Gramian matrix} of $v_1, \dots, v_{d - 1}$ is
$$\Gram(v_1, \dots, v_{d - 1})_{ij} = a(v_i, v_j).$$
The \dfn{cross product} $v_1 \times \cdots \times v_{d - 1}$ of vectors $v_1, \dots, v_{d - 1} \in \Hilb$ is defined to be the vector $v_0$
such that:
\begin{enumerate}
\item $a(v_0, v_i) = 0$,
\item $((-1)^{d - 1} v_0, v_1, \dots, v_{d - 1})$ is positively oriented, and
\item the length is
$$a(v_0, v_0) = |\det \Gram(v_1, \dots, v_{d - 1})|.$$
\end{enumerate}
\end{definition}

Here the orientation convention is chosen so that if $\partial_0, \dots, \partial_{d - 1}$ is an orthonormal basis, thus $a_{\mu\nu} = \delta_{\mu\nu}$, then the cross product is computed by the formal determinant
\begin{equation}\label{formal determinant}
v_1 \times \cdots \times v_{d - 1} = \begin{vmatrix}\partial_0 & \cdots & \partial_{d - 1} \\
v_1^0 & \cdots & v_1^{d - 1}\\
& \vdots \\
v_{d - 1}^0 & \cdots & v_{d - 1}^{d - 1}\end{vmatrix}
\end{equation}
which of course agrees with the orientation convention of the cross product on $\RR^3$.

\begin{lemma} \label{cross product formula}
Suppose that $\phi_1, \dots, \phi_{d - 1} \in \Hilb$ satisfy
\begin{equation}\label{cross product formula hypothesis}
\phi_i^\mu = \delta_i^\mu + \delta_0^\mu \psi_i
\end{equation}
for some $\psi \in (\Hilb_0^\perp)'$.
Let $h_{ij} = (a_{00})^{-1} a_{ij}$, and let $\normal \in \Hilb'$ be the unit covector which annihilates $\phi_1, \dots, \phi_{d - 1}$ with $((-1)^{d - 1}\normal^\sharp, \phi_1, \dots, \phi_{d - 1})$ positively oriented.
Then
\begin{align}
|\det \Gram(\phi_1, \dots, \phi_{d - 1})| &= (a_{00})^{d - 1} (1 + |\psi|_{h^{-1}}^2) \det h, \label{WeinsteinAronszajn} \\
(\phi_1 \times \cdots \times \phi_{d - 1})^\mu &= (a_{00})^{d/2} \sum_\nu a^{\mu \nu}(\delta_\nu^0 - \delta^i_\nu \psi_i) \sqrt{\det h}, \label{CrossProduct} \\
\normal_\mu &= \sqrt{\frac{g_{00}}{1 + |\psi|_{h^{-1}}^2}} (\delta^0_\mu - \delta^i_\mu \psi_i) \label{conormal crossproduct}.
\end{align}
\end{lemma}
\begin{proof}
In this proof, we use the Einstein convention.
We begin by computing
$$\Gram(\phi_1, \dots, \phi_{d - 1})_{ij} = a_{\mu \nu} \phi_i^\mu \phi_j^\nu = a_{00} \psi_i \psi_j + a_{ij}$$
which are the components of $a_{00}\psi \otimes \psi + a_{\hat 0 \hat 0} \in (\Hilb_0^\perp \otimes \Hilb_0^\perp)'$.
By the Weinstein-Aronszajn theorem \cite{Tao13}, $\det(1 + h^{-1}(\psi \otimes \psi)) = 1 + |\psi|_{h^{-1}}^2$, so
\begin{align*}
\det(a_{00}\psi \otimes \psi + g_{\hat 0 \hat 0})
&= (a_{00})^{d - 1} \det(\psi \otimes \psi + h) = (a_{00})^{d - 1} \det(h^{-1}(\psi \otimes \psi) + 1) \det h \\
&= (a_{00})^{d - 1} (1 + |\psi|_{h^{-1}}^2) \det h.
\end{align*}
Since $h$ is a quadratic form of signature $(+, \cdots, +)$, its determinant is positive and so (\ref{WeinsteinAronszajn}) holds.

We begin the proof of (\ref{CrossProduct}) by checking orthogonality:
\begin{align*}
a_{\mu\nu} a^{\mu \lambda} (\delta^0_\lambda - \delta^i_\lambda \psi_i)(\delta^\nu_0 \psi_i + \delta^\nu_i)
&= \delta^\lambda_\nu (\delta^0_\lambda - \delta^i_\lambda \psi_i)(\delta^\nu_0 \psi_i + \delta_i^\nu)
= \psi_i - \psi_i = 0.
\end{align*}
To check orientation we may assume that $a_{\mu\nu} = \delta_{\mu\nu}$ in which case we just need to check agreement with (\ref{formal determinant}):
$$\begin{vmatrix} \partial_0 && \cdots && \partial_{d - 1} \\
\psi_1 & 1 & 0 & \cdots & 0 \\
\psi_2 & 0 & 1 & \cdots & 0\\
&& \vdots \\
\psi_{d - 1} & 0 & \cdots & 0 & 1
\end{vmatrix} = \partial_0 - \sum_i \psi_i \partial_i.$$
To see that its length is $\det \Gram(\phi_1, \dots, \phi_{d - 1})$ we compute
\begin{align*}
a_{\mu \nu} a^{\mu \lambda}(\delta^0_\lambda - \delta^i_\lambda \psi_i) a^{\nu \kappa}(\delta_\kappa^0 - \delta_\kappa^j \psi_j)
&= (\delta_\mu^0 - \delta_\nu^i \psi_i)(a^{0 \nu} - a^{j \nu} \psi_j)\\
&= a^{00} - a^{0j} \psi_j - a^{0i} \psi_i + a^{ij} \psi_i \psi_j.
\end{align*}
Recalling (\ref{0th coordinate orthogonal}) we can rewrite this as
$$a_{\mu \nu} a^{\mu \lambda}(\delta^0_\lambda - \delta^i_\lambda \psi_i) a^{\nu \kappa}(\delta_\kappa^0 - \delta_\kappa^j \psi_j) = (a_{00})^{-1} + (a_{\hat 0 \hat 0})^{-1}(\psi \otimes \psi).$$
But $a_{00} (a_{\hat 0 \hat 0})^{-1} = h^{-1}$ so we deduce from (\ref{WeinsteinAronszajn})
$$a_{\mu \nu} a^{\mu \lambda}(\delta^0_\lambda - \delta^i_\lambda \psi_i) a^{\nu \kappa}(\delta_\kappa^0 - \delta_\kappa^j \psi_j) = (a_{00})^{-1} (1 + |\psi|_{h^{-1}}^2)
= \frac{|\det \Gram(\phi_1, \dots, \phi_{d - 1})|}{(a_{00})^d \det h}$$
which gives (\ref{CrossProduct}).
From (\ref{WeinsteinAronszajn}, \ref{CrossProduct}), (\ref{conormal crossproduct}) is immediate.
\end{proof}

Let $M$ be an open submanifold of $\Hyp^d = (\RR_+)_x \times \RR^{d - 2}_y \times \RR_z$ with the hyperbolic metric (\ref{hyperbolic metric}).
Let $\omega$ be a $C^1$ function on $M \cap (\RR_+ \times \RR^{d - 2})$, and write
$$N = \{(x, y, z) \in M: z = \omega(x, y)\}$$
for its graph in $M$.
Let $\partial_1, \dots, \partial_{d - 1}$ denote the coordinate vector fields on $\RR_+ \times \RR^{d - 2}$, and $\partial_0$ the coordinate vector field on $\RR_z$.
Then, in the language of Notation \ref{hyperbolic line bundle}, with $\Psi = \Psi_N$, $\psi = d\omega$, $\Hilb = T_{(x, y, \omega(x, y))} M$, $a = g|T_{(x, y, \omega(x, y))} M$, and $\phi_i = \Psi_* \partial_i$, we have (\ref{cross product formula hypothesis}).
In particular, $\phi_1, \dots, \phi_{d - 1}$ span $T_{(x, y, \omega(x, y))}N$.
We also have 
$$h_{ij} = x^2 g_{ij} = x^2 x^{-2} \delta_{ij} = \delta_{ij}.$$
So by Lemma \ref{cross product formula}, the conormal $\normal$ to $N$ at $(x, y, \omega(x, y))$ is given by 
$$(\Psi^* \normal)_\mu = \sqrt{\frac{x}{1 + |d\omega|^2}} (\delta_\mu^0 - \delta_\mu^i \omega_{,i})$$
where $|\cdot|$ is the euclidean length.
If we denote $\slashed g_{ij} = g(\phi_i, \phi_j)$, then
$$\slashed g = \Gram(\phi_1, \dots, \phi_{d - 1})$$
and so the volume form on $N$ is given by 
$$\Psi^* \vol_N = x^{1 - d} \sqrt{1 + |d\omega|^2} ~dx dy.$$
Therefore we introduce:

\begin{definition}
    The \dfn{hyperbolic Plateau energy} of a $1$-form $\psi$ on $(\RR_+)_x \times \RR^{d - 2}_y$ is the $d-1$-form
    $$\Lagrange(\psi) = x^{1 - d} \sqrt{1 + |\psi|^2} ~dxdy$$
    where the metric is euclidean.
\end{definition}
    
We now have
\begin{equation}\label{Lagrangian formula}
    \Psi_N^* \vol_N = \Lagrange(d\omega)
\end{equation}
if $N$ is the graph of $\omega \in C^1(\Omega)$, $\Omega \subseteq \RR_+ \times \RR^{d - 2}$.
If $\normal$ denotes the conormal $1$-form then we have 
\begin{equation}\label{Lagrangian normal z}
(\Psi^*_N \normal)_z \Lagrange(d\omega) = x^p ~dxdy
\end{equation}
where $p = 3/2 - d$, and 
\begin{equation}\label{Lagrangian normal xy}
(\Psi^*_N \normal)_i \Lagrange(d\omega) = x^p \partial_i \omega(x, y) ~dxdy
\end{equation}
whenever $i \in \{x, y_1, \dots, y_{d - 2}\}$.

If $N$ is a minimal hypersurface, then $d\omega$ is a minimizer of $\Lagrange(d\omega)$, and in particular solves the Plateau-type PDE 
\begin{equation}\label{Plateau PDE}
    \Div \frac{\grad \omega}{\sqrt{1 + |\grad \omega|^2}} + \frac{1 - d}{x^d \sqrt{1 + |\grad \omega|^2}} \partial_x \omega = 0,
\end{equation}
though we will seldom use this PDE directly.

%%%%%%%%%%%%%%%%%%%%%%%%%%%%%%%%%%%%%

\subsection{Excess versus Plateau energy}
In this subsection we introduce the \dfn{excess} of a set $U$ of locally finite perimeter, which roughly speaking determines how much the normal vector to $U$ can twist in a neighborhood of $O$.
We then show that the excess controls the modulus of continuity of the normal vector and in turn is controlled by the Plateau energy.
Notions of excess for $\RR^d$ appear in the monographs of Federer \cite[\S5.3.1]{federer2014geometric} and Giusti \cite[Chapter 6]{Giusti77}.

We begin by showing that on average, the errors incurred by the hyperbolic metric are negligible.
In fact, we will always be able to disregard terms of order $4^{-n}$, so our next few lemmata allow us to ignore 

\begin{notation}
We identify $m$-forms on $\RR_+ \times \RR^{d - 2}$ with smooth maps $\RR_+ \times \RR^{d - 2} \to (\RR^{d - 1})^{\wedge m}$.
We identify vectors in $(\RR^{d - 1})^{\wedge m}$ with constant $m$-forms, and for a $m$-form $\psi$ define the constant $m$-form
$$(A_n \psi)_I = \dashint_{\tilde B_{2^{-n}}} \psi_I(x, y) ~dxdy$$
where $I$ ranges over ascending $m$-multiindices.
\end{notation}

\begin{lemma}\label{ball difference is harmless}
The symmetric difference of balls has measure 
$$|\Pi(B_{2^{-n}}) \Delta \tilde B_{2^{-n}}| \lesssim 2^{-nd}.$$
In particular, if $\psi$ is a $m$-form then 
$$(A_n \psi)_I = \dashint_{\tilde B_{2^{-n}}} \psi_I(x, y) ~dxdy + O(2^{-n}).$$
\end{lemma}
\begin{proof}
TODO. It's true if $d = 2$ at least. We need to rewrite a lot of the rest of the proof to use $\tilde B$ instead of $\Pi B$.
\end{proof}

\begin{lemma}\label{average of xp is harmless}
For every $p \in \RR$ one has 
$$A_n(x^p) = 1 + O(4^{-n})$$
where the implied constant depends on $p$.
\end{lemma}
\begin{proof}
TODO: Simplify this by using $\tilde B$
According to (\ref{sup in a ball}), we can find a minimal rectangle $I \Subset \RR^{d - 2}$ such that
$$\Pi(B_{2^{-n}}) \subseteq (\exp(-2^{-n}), \exp(2^{-n})) \times I,$$
so by Fubini's theorem,
\begin{align*}|A_n(x^p - 1)| &= \left|\dashint_{\Pi(B_{2^{-n}})} (x^p - 1) ~dxdy\right| \lesssim \frac{|I|}{|\Pi(B_{2^{-n}})|} \left|\int_{\exp(-2^{-n})}^{\exp(2^{-n})} (x^p - 1)~ dx\right|;
\end{align*}
from minimality we have $|I| \lesssim 2^{n + 1} |\Pi(B_{2^{-n}})|$.
Making the substitution $\tilde x = \log x$ and disposing of the Jacobian as $1/2 \leq e^{\tilde x} \leq 2$, we obtain 
$$|A_n(x^p - 1)| \lesssim \left|\dashint_{-2^{-n}}^{2^{-n}} (e^{p\tilde x} - 1) ~d\tilde x\right|.$$
From Taylor's theorem and the parity of $\tilde x$, 
\begin{align*}
|A_n(x^p - 1)| &\lesssim \left|\dashint_{-2^{-n}}^{2^{-n}} \tilde x + O(\tilde x^2) ~d\tilde x\right| \\
&\lesssim 2^n \int_0^{2^{-n}} \tilde x^2 ~d\tilde x \lesssim 2^n 8^{-n} = 4^{-n}. \qedhere 
\end{align*}
\end{proof}

\begin{lemma}\label{normal has length 1}
Let $\normal^{(n)}$ be the approximate derivative of $1_U$ where $U$ is a set of locally finite perimeter.
$$|\normal^{(n)}| \leq 1 + O(4^{-n}).$$
\end{lemma}
\begin{proof}
Let $\kappa > 0$ be a parameter to be chosen.
We may assume that $U$ is correctly aligned.
By Proposition \ref{mollifier quant} there exists a set $V$ of $C^1$ perimeter such that ... choose $\kappa \ll 4^{-n}$ in the end. TODO
\end{proof}

\begin{definition}
The \dfn{excess} of a set $U$ of locally finite perimeter at scale $n$ is
$$\Exc_n(U) = 2^{n(d - 1)} (1 - |\normal^{(n)}|) \int_{B_{2^{-n}}} |d1_U| ~\vol.$$
\end{definition}

The first basic property of the excess is rotation-invariance, which we now formulate.

\begin{notation}
We define an action
\begin{equation}\label{hyperbolic rotation}
    \Phi: \Orth(\RR^d) \to \Iso(\Hyp^d)
\end{equation}
of the orthogonal group $\Orth(\RR^d)$, as well as the representation
\begin{align*}
\Orth(\RR^d) &\to \GL(BV_l(\RR^d))\\
A^* u(P) &:= u(\Phi(A)(P))
\end{align*}
of $\Orth(\RR^d)$ on $BV_l(\RR^d)$, as follows.
Recall that the hyperbolic metric in the ball model $\DD^d$ is radial, so the natural action of the orthogonal group $\Orth(\RR^d)$ on $\DD^d$ is by isometry, and if we use the Cayley transform to identify the origin of $\DD^d$ with $O$, then we obtain a faithful action (\ref{hyperbolic rotation}) whose orbits are the spheres $\partial B_r$ centered on $O$.
\end{notation}

\begin{lemma}\label{excess rotation invariant}
For every $A \in \Orth(\RR^d)$ and $u \in BV_l(\Hyp^d)$,
$$\Exc_n(A^* u) = \Exc_n(u).$$
\end{lemma}
\begin{proof}
Since $A$ is an isometry, $|(d A^* u)_{\Phi(A)(P)}| = |(du)_P|$, so
\begin{align*}
\int_{B_{2^{-n}}} |dA^*u| ~\vol &= \int_0^{2^{-n}} \int_{\partial B_r} |dA^* u| ~\vol_{\partial B_r} ~dr = \int_0^{2^{-n}} \int_{\partial B_r} |du| ~\vol_{\partial B_r} ~dr\\
&= \int_{B_{2^{-n}}} |du| ~\vol
\end{align*}
since $(A^{-1})^* \vol_{\partial B_r} = \vol_{\partial B_r}$.
Since $A$ is an isometry, $|A_* (x \partial_\mu)| = 1$ and $(A_*(x \partial_\mu))_\mu$ is an orthonormal basis of $T_O \Hyp^d$.
The claim follows.
\end{proof}

We now show that the rate of convergence of the approximate derivative is given by the excess.
An analogous estimate for the euclidean case is due to Miranda \cite[pg661]{Miranda66}; our contribution is to deal with the correction terms that arise from the hyperbolic metric.

\begin{lemma} \label{excess bounds Cauchy sequence}
Let $U$ be a set of locally finite perimeter.
Let $\normal^{(n)}$ be the approximate derivative of $1_U$ and suppose that $n$ is larger than some absolute constant. Then 
$$|\normal^{(n)} - \normal^{(n + m)}|^2 \leq \frac{2^{3 + n(1 - d)}}{\int_{B_{2^{-(n + m)}}} |du| ~\vol} \Exc_n(u) + O(2^{m(d - 1) - 2n}).$$
\end{lemma}
\begin{proof}
From the law of cosines and Lemma \ref{normal has length 1},
\begin{align*}
|\normal^{(n)} - \normal^{(n + m)}|^2 &= |\normal^{(n)}|^2 + |\normal^{(n + m)}|^2 - 2 g(\normal^{(n)}, \normal^{(n + m)})\\
&\leq 2(1 - g(\normal^{(n)}, \normal^{(n + m)}) + O(16^{-n})) \\
&= \frac{2}{\int_{B_{2^{-(n + m)}}} |du| ~\vol} \int_{B_{2^{-(n + m)}}} (1 + O(16^{-n}))|du| - \sum_\mu x \partial_\mu u \normal_\mu^{(n)} ~\vol.
\end{align*}
Applying Lemma \ref{normal has length 1} alongside the Cauchy-Schwarz inequality,
$$\sum_\mu x \partial_\mu u \normal_\mu^{(n)} \leq |du|(1 + O(4^{-n}))$$
so
\begin{align*}
|\normal^{(n)} - \normal^{(n + m)}|^2 &\leq \frac{2}{\int_{B_{2^{-(n + m)}}} |du| ~\vol} \int_{B_{2^{-n}}} (1 + O(4^{-n}))|du| - \sum_\mu x \partial_\mu u \normal_\mu^{(n)} ~\vol\\
&= \frac{2}{\int_{B_{2^{-(n + m)}}} |du| ~\vol} (1 + O(4^{-n}) - |\normal^{(n)}|^2) \int_{B_{2^{-n}}} |du| ~\vol.
\end{align*}
The claim now follows from the inequality $1 - a^2 \leq 1 - 2a$ valid for $a < 2$ (and in particular for $a = |\normal^{(n)}|^2$ if $n$ is large).
\end{proof}

We now show that the excess is controlled by the Plateau energy.
In the euclidean case, this result is not an approximation but an equality (c.f. \cite[pg83]{Giusti77}), and so in that case the proof of the following lemma is trivial.

\begin{lemma}\label{excess vs plateau energy}
Let $\omega \in C^1(\Omega)$ satisfy $||d\omega||_{L^\infty} \lesssim 1$. Then
\begin{align*}
    \Exc_n(1_{\omega(x, y) < z}) = 2^{n(d - 1)} \left[\int_{\Pi(B_{2^{-n}})} \Lagrange(d\omega) - \Lagrange(A_n d\omega)\right] + O(4^{-n}).
\end{align*}
\end{lemma}
\begin{proof}
Let $u(x, y, z) = 1_{\omega(x, y) < z}$, let $N$ be the graph of $\omega$, and let $\normal^{(n)}$ be the approximate derivative of $u$.
It follows from the definitions, the fact that $|du|$ is the Radon-Nikod\'ym derivative $\vol_N/\vol$, and (\ref{Lagrangian formula}), that
$$\normal^{(n)}_\mu \int_{B_{2^{-n}}} |du| ~\vol = \int_{B_{2^{-n}}} \normal_\mu |du| ~\vol = \int_{B_{2^{-n}} \cap N} \normal_\mu \vol_N = \int_{\Pi(B_{2^{-n}})} (\Psi_N^* \normal)_\mu \Lagrange(d\omega).$$
We apply Lemma \ref{ball difference is harmless} to deduce 
$$\normal^{(n)}_\mu \int_{B_{2^{-n}}} |du| ~\vol = \int_{\tilde B_{2^{-n}}} (\Psi_N^* \normal)_\mu \Lagrange(d\omega) + O(2^{-nd}).$$
From the fact that $(dx, dy_1, \dots, dy_{d - 2}, dz)$ is an orthonormal basis of $T_O' \Hyp^d$, it follows that 
$$|\normal^{(n)}|^2 \left(\int_{B_{2^{-n}}} |du| ~\vol\right)^2 = \sum_\mu \left(\int_{\Pi(B_{2^{-n}})} (\Psi_N^* \normal)_\mu \Lagrange(d\omega)\right)^2 + O(2^{(1 - 2d)n})$$
so from (\ref{Lagrangian normal z}, \ref{Lagrangian normal xy}),
\begin{align*}
    |\normal^{(n)}|^2 \left(\int_{B_{2^{-n}}} |du| ~\vol\right)^2 &= \left(\int_{\Pi(B_{2^{-n}})} x^p ~dxdy\right)^2 + \sum_i \left(\int_{\Pi(B_{2^{-n}})} x^p \partial_i \omega(x, y) ~dxdy \right)^2 \\
    &= |\Pi(B_{2^{-n}})|^2 |A_n (x^p)|^2 \left(1 + \frac{|A_n(x^p d\omega)|^2}{|A_n (x^p)|^2}\right) + O(2^{(1 - 2d)n}).
\end{align*}
Using Lemma \ref{average of xp is harmless} and Taylor's theorem,
\begin{align*}
    |A_n(x^p d\omega) - A_n (d\omega)| &\leq ||d\omega||_{L^\infty} |A_n (x^p - 1)| \lesssim 4^{-n}
\end{align*}
and so by Proposition \ref{doubling dimension},
$$|\normal^{(n)}| \int_{B_{2^{-n}}} |du| ~\vol = |\Pi(B_{2^{-n}})| \sqrt{1 + |A_n d\omega|^2} + O(2^{-n(d + 1)}).$$
We rewrite 
$$|\Pi(B_{2^{-n}})| \sqrt{1 + |A_n d\omega|^2} = \int_{\Pi(B_{2^{-n}})} \Lagrange(A_nd\omega) + \int_{\Pi(B_{2^{-n}})} (1 - x^{1 - d})\sqrt{1 + |A_n d\omega|^2} ~dxdy$$
and bound the remainder term using Lemma \ref{average of xp is harmless} and Proposition \ref{doubling dimension}:
$$0 \leq \int_{\Pi(B_{2^{-n}})} (1 - x^{1 - d})\sqrt{1 + |A_n d\omega|^2} ~dxdy \lesssim \int_{\Pi(B_{2^{-n}})} (1 - x^{1-d}) ~dxdy \lesssim O(2^{-n(d + 1)}).$$
The claim now follows from the definition of the excess.
\end{proof}

%%%%%%%%%%%%%%%%%%%%%%%%%%%%%%%%%%
\subsection{Plateau versus Dirichlet}

Here we compare the Plateau energy to the Dirichlet energy, and apply potential theory to get a multiplicative gain when we pass from a scale $n$ to its successor $n + 1$.
Such a strategy was classically employed by Miranda \cite[Teorema 4.3]{Miranda66} to show the regularity of minimal hypersurfaces in $\RR^d$.
To see why one should expect it to work, even in the hyperbolic case, let $\omega \in C^1$, and note that (\ref{derivative bounds}) gives a sufficient condition for $|d\omega|$ to be small in a neighborhood of $x = 1, y = 0$.
But the linearization of Plateau's equation (\ref{Plateau PDE}) at $x = 1, y = 0, |d\omega| = 0$ is the Laplace equation $\Delta \omega = 0$, so $\omega$ is well-approximated by harmonic functions.
The desired gain holds for harmonic functions (\ref{Miranda41}) and so should at least approximately hold for the minimal surfaces.\footnote{The generalization of (\ref{Plateau PDE}) to arbitrary Riemannian manifolds linearizes to an elliptic PDE of second order, and it would be interesting to show that such PDE also satisfy an analogue of (\ref{Miranda41}), but we do not attempt this here.}

We turn to the details.
Let
$$\DirL(\psi) = \frac{|\psi|^2}{2} ~dxdy$$
be the Dirichlet energy of a $1$-form $\psi = d\omega$ on $(\RR_+)_x \times \RR^{d - 2}_y$.
In the below computation we suppress the $dxdy$.
Given $1$-forms $\psi_1, \psi_2$, there exists $\xi \in [|\psi_1|, |\psi_2|]$ such that
\begin{equation}\label{Taylor remainder Dirichlet}
\Lagrange(\psi_1) - \Lagrange(\psi_2) = x^{1 - d}\left(\frac{|\psi_1|^2 - |\psi_2|^2}{2\sqrt{1 + |\psi_2|^2}} - \frac{(|\psi_1|^2 - |\psi_2|^2)^2}{8(1 + \xi^2)^{3/2}}\right).
\end{equation}
The second term of (\ref{Taylor remainder Dirichlet}) is negative and $\sqrt{1 + |\psi_2|^2} \geq 1$, so it follows that
\begin{equation}\label{Taylor lower bound}
\Lagrange(\psi_1) - \Lagrange(\psi_2) \leq x^{1 - d} (\DirL(\psi_1) - \DirL(\psi_2)).
\end{equation}
If in addition $||\psi_2||_{L^\infty} \leq 1$, then
$$4\sqrt{1 + |\psi_2|^2} \leq 8 \leq 8(1 + \xi^2)^{3/2}$$
so we conclude
\begin{equation}\label{Taylor upper bound}
\Lagrange(\psi_1) - \Lagrange(\psi_2) \geq \frac{x^{1 - d}}{\sqrt{1 + |\psi_2|^2}} (\DirL(\psi_1) - \DirL(\psi_2) - (\DirL(\psi_1) - \DirL(\psi_2))^2).
\end{equation}
For every harmonic function $h$, by \cite[Lemma 4.1]{Miranda66},
\begin{equation}\label{Miranda41}
\int_{\tilde B_{2^{-(n+1)}}} \DirL(dh) - \DirL(A_n dh) \leq 2^{-(d + 1)} \int_{\tilde B_{2^{-n}}} \DirL(dh) - \DirL(A_n dh).
\end{equation}
Moreover, the mean-value property of $dh$ implies that for every $1$-form $\psi$,
\begin{equation}\label{MVP derivative}
\int_{\tilde B_{2^{-n}}} \DirL(dh - \psi) = \int_{\tilde B_{2^{-n}}} \DirL(dh) - \DirL(\psi).
\end{equation}

On first reading of the below lemma, the reader may take $c = 10^{-3}$, $O(c) = 10^{-1}$ and $n^* = 10$; in fact, $c$ will later be chosen to be a dimensional constant.
Roughly speaking, one should think of $\beta$ as comparable to $2^{-(n - n^*)}$, and $\kappa$ as the ``error incurred by mollification".

\begin{lemma}\label{DGL1}
For every $c > 0$ there exists a scale $n^* \in \ZZ$ with the following property:

Let $\omega \in C^1(\Omega)$, suppose that $\kappa, \beta \in (0, 1)$ and $n \geq n^*$ satisfy
\begin{align}
||d\omega||_{L^\infty(\tilde B_{2^{-n}})} &\leq \kappa, \label{DGL1 1}\\
\int_{\tilde B_{2^{-n}}} \Lagrange(d\omega) - \Lagrange(A_n d\omega) &\leq \beta, \label{DGL1 2}\\
\int_{\tilde B_{2^{-n}}} \Lagrange(d\omega) &\leq \eta(\{(x, y, z) \in \Hyp^d: z < \omega(x, y)\}, 2^{-n}) + \beta \kappa. \label{DGL1 3}
\end{align}
Then
$$\int_{\tilde B_{2^{-(n + 1)}}}\Lagrange(d\omega) - \Lagrange(A_{n + 1} d\omega) \leq (1 + O(c)) 2^{-(d + 1)} \beta + O(\beta \sqrt \kappa)$$
where all constants only depend on $d$.
\end{lemma}
\begin{proof}
Choose $n^*$ so large that
\begin{equation}\label{x to c}
1 - c \leq \inf_{(x, y) \in \tilde B_{2^{-n^*}}} x^{1 - d} < \sup_{(x, y) \in \tilde B_{2^{-n^*}}} x^{1 - d} \leq 1 + c
\end{equation}
and suppose that $n \geq n^*$.
Let $h$ be the harmonic function on $\tilde B_{2^{-n}}$ such that
\begin{equation}\label{trace equation}
h = \omega \text{ on } \partial \tilde B_{2^{-n}}.
\end{equation}
By definition of $\eta$ and (\ref{DGL1 3}),
\begin{align*}
\int_{\tilde B_{2^{-(n + 1)}}} \Lagrange(d\omega) - \Lagrange(dh)
&\leq \int_{\tilde B_{2^{-n}}} \Lagrange(d\omega) - \eta(\{(x, y, z) \in \Hyp^d: z < \omega(x, y)\}, 2^{-n}).
\end{align*}
Therefore
\begin{equation}\label{bound on domega - dh}
\int_{\tilde B_{2^{-(n + 1)}}} \Lagrange(d\omega) - \Lagrange(dh) \leq \beta\kappa.
\end{equation}

The rest of the proof is essentially identical to \cite[Lemma 4.2]{Miranda66}, but as this is a crucial step, we include it for completeness.
By (\ref{Taylor lower bound}, \ref{x to c}),
$$\int_{\tilde B_{2^{-(n + 1)}}} \Lagrange(d\omega) - \Lagrange(A_{n + 1}d\omega) \leq (1 + c)\int_{\tilde B_{2^{-(n + 1)}}} \DirL(d\omega) - \DirL(A_{n + 1}d\omega).$$
Since $A_{n + 1}d\omega$ is the mean of $d\omega$, for every $\varepsilon \in (0, 1)$,
\begin{align*}
\int_{\tilde B_{2^{-(n + 1)}}} \DirL(d\omega) - \DirL(A_{n + 1}d\omega)
&\leq \int_{\tilde B_{2^{-(n + 1)}}} \DirL(d\omega - A_nd\omega) \\
&\leq (1 + \varepsilon^{-1}) \int_{\tilde B_{2^{-(n + 1)}}} \DirL(d(\omega - h)) \\
&\qquad +(1 + \varepsilon) \int_{\tilde B_{2^{-(n + 1)}}} \DirL(dh - A_nd\omega) \\
&=: O(\varepsilon^{-1}) I + (1 + \varepsilon) J.
\end{align*}
From the positivity of Dirichlet energy and (\ref{MVP derivative}, \ref{Taylor upper bound}),
\begin{align*}
I &\leq \int_{\tilde B_{2^{-n}}} \DirL(d(\omega - h)) = \int_{\tilde B_{2^{-n}}} \DirL(d\omega) - \DirL(dh) \lesssim \int_{\tilde B_{2^{-n}}} \Lagrange(d\omega) - \Lagrange(dh)
\end{align*}
so by (\ref{bound on domega - dh}),
\begin{equation}\label{bound on I}
I \lesssim \beta\kappa.
\end{equation}
Moreover, by (\ref{MVP derivative}),
$$J = \int_{\tilde B_{2^{-(n + 1)}}} \DirL(dh) - \DirL(A_nd\omega).$$
From (\ref{trace equation}) and Stokes' theorem, there are constants $C_m > 0$ such that
$$A_m d\omega = C_m \int_{\partial \tilde B_{2^{-m}}} \omega ~dS = C_m \int_{\partial \tilde B_{2^{-m}}} h ~dS = A_m dh$$
which along with (\ref{Miranda41}, \ref{MVP derivative}) implies that
$$J \leq 2^{-(d + 1)} \int_{\tilde B_{2^{-n}}} \DirL(dh) - \DirL(A_nd\omega) = 2^{-(d + 1)} \int_{\tilde B_{2^{-n}}} \DirL(dh - A_n d\omega).$$
We further estimate, using (\ref{bound on I}),
\begin{align*}
J &\leq (1 + \varepsilon) 2^{-(d + 1)} \int_{\tilde B_{2^{-n}}} \DirL(d\omega - A_n d\omega) + O(\varepsilon^{-1} I) \\
&:= (1 + \varepsilon) 2^{-(d + 1)} K + O(\varepsilon^{-1} \beta \kappa).
\end{align*}
To estimate $K$ we apply (\ref{MVP derivative}, \ref{x to c}, \ref{Taylor upper bound}) to obtain
\begin{align*}
K &= \int_{\tilde B_{2^{-n}}} \DirL(d\omega) - \DirL(A_n d\omega) \\
&\leq (1 + O(c)) \int_{\tilde B_{2^{-n}}} \Lagrange(d\omega) - \Lagrange(A_n d\omega) + O(1) \int_{\tilde B_{2^{-n}}} (\DirL(d\omega) - \DirL(A_n d\omega))^2.
\end{align*}
From (\ref{DGL1 1}, \ref{DGL1 2}), it follows that
\begin{align*}
K &\leq (1 + O(c))\beta + O(||d\omega||_{L^\infty(\tilde B_{2^{-n^*}})}) \int_{\tilde B_{2^{-n}}} \DirL(d\omega) - \DirL(A_n d\omega)\\
&\leq \beta(1 + O(c + \kappa)).
\end{align*}
If we set $\varepsilon = \sqrt \kappa$ then it follows that
\begin{align*}
\int_{\tilde B_{2^{-(n+1)}}} \DirL(d\omega) - \DirL(A_n d\omega) &\leq (1 + O(c)) 2^{-(d + 1)} \beta + O(\beta \sqrt \kappa). \qedhere
\end{align*}
\end{proof}

%%%%%%%%%%%%%%%%%%%%%%%%%%%%%%%%%%%%%%%%%%%%%%%%%

\subsection{de Giorgi lemma}
The classical de Giorgi lemma \cite{deGiorgi61} transfers the multiplicative gains of Lemma \ref{DGL1} to the excess of a set of least perimeter.
Iterating it allows us then to deduce the regularity of the reduced boundary.
For our analogue on $\Hyp^d$, we follow Miranda, and begin by generalizing \cite[Teorema 4.4]{Miranda66}, which shows that the de Giorgi lemma holds for $C^1$ hypersurfaces which are approximately minimal.
The general case will then follow from Proposition \ref{mollifier quant}. 

\begin{lemma}[de Giorgi lemma on $\Hyp^d$, $C^1$ case]\label{DGL2}
For every $c > 0$ there exists a scale $n^* \in \ZZ$ with the following property:

Let $N$ be a $C^1$ hypersurface in $B_{2^{-n}}$, $n \geq n^*$, with unit normal field $\normal^\sharp$, such that $N$ bounds an open set $U$.
If $\kappa \in (0, 1), \alpha \in \RR_+$ are parameters such that
\begin{align}
\Exc_n(U) &\leq \alpha, \label{DGL2 1}\\
|N \cap B_{2^{-n}}| &\leq \eta(U, B_{2^{-n}}) + 2^{n(1 - d)}\alpha \kappa, \label{DGL2 2}\\
||\normal^\sharp - x\partial_z||_{L^\infty(N \cap B_{2^{-n}})} &\leq \kappa^2, \label{DGL2 3}
\end{align}
then
$$\Exc_{n + 1}(U) \leq \frac{1 + O(c)}{2} \alpha + O(\alpha \sqrt \kappa) + O(4^{-n}).$$
\end{lemma}
\begin{proof}
TODO: Rewrite this whole thing
From Lemma \ref{hopfKilling} and (\ref{DGL2 3}), there exists $n_1 \in \ZZ$ and a function $\omega \in C^1(\RR_+ \times \RR^{d - 2})$ satisfying the derivative bound (\ref{DGL1 1}) for any $n \geq n_1$
and such that the graph of $\omega$ over $\Pi(B_{2^{-n_1}})$ is $N \cap A_1$.
Let $\varepsilon > 0$; then we can find a scale $n_2 \geq n_1$ such that if $n \geq n_2$ then
$$\Pi(B_{(1 - \varepsilon) 2^{-(n+1)}}) \subseteq \tilde B_{2^{-(n+1)}} \text{ and } \tilde B_{2^{-n}} \subseteq \Pi(B_{(1 + \varepsilon) 2^{-n}}).$$
Up to a multiplicative loss of $1 + c$, all quantities defined for $(1 - \varepsilon)2^{-(n + 1)}$ can be replaced with $2^{-(n + 1)}$; similarly for $2^{-n}$ and $(1 + \varepsilon) 2^{-n}$ (TODO: Justify this) for $\varepsilon$ chosen small enough depending on $c$.
In order to apply Lemma \ref{DGL1} we apply Lemma \ref{excess vs plateau energy} and (\ref{DGL2 1}) to obtain
$$\int_{\tilde B_{2^{-n}}} \Lagrange(d\omega) - \Lagrange(A_n d\omega) \leq (1 + c)2^{n(1 - d)}\alpha.$$
Similarly from (\ref{Lagrangian formula}, \ref{DGL2 2}), we obtain
$$\int_{\tilde B_{2^{-n}}} \Lagrange(d\omega) \leq \eta(U, B_{2^{-n}}) + (1 + c) 2^{n(1-d)}\alpha \kappa,$$
and so we have met the hypotheses of Lemma \ref{DGL1} with
$$\beta = (1 + c)2^{n(1 - d)}\alpha.$$
Therefore there exists $n_3 \geq n_2$ such that for every $n \geq n_3$,
$$\int_{\tilde B_{2^{-(n+1)}}} \Lagrange(d\omega) - \Lagrange(A_nd\omega) \leq 2^{(n + 1)(1-d)}\left[ \frac{1 + O(c)}{2} \alpha + O(\alpha \sqrt \kappa)\right].$$
To convert this estimate back into a result about the excess we use Lemma \ref{excess vs plateau energy}.
There exists a scale $n^* \geq n_3$ such that if $n \geq n_4$ then $\exp(2^{-(n + 1)}) \leq 1 + c$, so
\begin{align*}
\Exc_{n + 1}(U) &\leq (1 + O(c)) 2^{(n+1)(d-1)} \int_{\tilde B_{2^{-(n+1)}}} \Lagrange(d\omega) - \exp(-2^{-n}) \Lagrange(A_nd\omega)\\
&\leq O(4^{-n}) + (1 + O(c)) 2^{(n+1)(d-1)} \int_{\tilde B_{2^{-(n+1)}}} \Lagrange(d\omega) - \Lagrange(A_nd\omega)\\
&\leq O(4^{-n}) + \frac{1 + O(c)}{2} \alpha + O(\alpha \sqrt \kappa). \qedhere
\end{align*}
\end{proof}

We now prove the de Giorgi lemma.
The proof settles the choice of the dimensional constant $c$.

\begin{proposition}[de Giorgi lemma on $\Hyp^d$, general case]\label{DGL 3}
There exist $\sigma, n^*, C > 0$ such that for every set $U$ of least perimeter in $B_{2^{-n}} \subseteq \Hyp^d$,
$$\Exc_n(U) < \sigma \text{ and } n \geq n^* \implies \Exc_{n+1}(U) \leq \frac{51}{100} \Exc_n(U) + \frac{C}{4^n}.$$
\end{proposition}
\begin{proof}
By Lemma \ref{excess rotation invariant} and the fact that $x\partial_z$ is a unit vector field, it is no loss of generality to assume that
$$\normal^{(n)}_U = |\normal^{(n)}_U| x\partial_z.$$
Under this assumption, if we write $u = 1_U$ then
$$\Exc_n(U) = 2^{n(d - 1)} \exp(2^{-n}) \int_{B_{2^{-n}}} |du| ~\vol - 2^{n(d - 1)} \int_{B_{2^{-n}}} x\partial_z u ~\vol.$$
The injectivity radius of $O$ is infinite and so, if $\sigma < \exp(2^{-n}) \gamma_*$, we obtain from Proposition \ref{mollifier quant}, for every $\kappa > 0$, a $C^1$ hypersurface $N$ which bounds an open set $V$ which, if $n$ is chosen large enough (TODO: Justify taking $t \to 1$, $\varepsilon \to 0$), satisfies
\begin{align*}
|N \cap B_{2^{-n}}| &\leq \eta(V, B_{2^{-n}}) + \kappa \Exc_n(U, B_{2^{-n}}) \\
\Exc_n(V, B_{2^{-n}}) &\in \Exc_n(U, B_{2^{-n}})[1 - \kappa, 1 + \kappa] \\
||\normal^\sharp_N - x\partial_z||_{L^\infty} &\leq \kappa.
\end{align*}
Therefore by Lemma \ref{DGL2}, there exist dimensional constants $C_1,C_2,C > 0$ such that
$$\Exc_{n + 1}(V) \leq \frac{1 + C_1c}{2} \Exc_n(U, B_{2^{-n}}) + C_2 \Exc_n(U, B_{2^{-n}}) \sqrt \kappa + \frac{2C}{4^n}.$$
We now choose $c = C_1/50$, which yields a constant $C_3 > 0$ such that
$$\Exc_n(U, B_{2^{-n}}) \leq \frac{51}{100} \Exc_n(U, B_{2^{-n}}) + C_3 \Exc_n(U, B_{2^{-n}}) \sqrt \kappa + \frac{2C}{4^n}.$$
We can then choose $\kappa$ small enough that
$$C_3 \Exc_n(U, B_{2^{-n}}) \sqrt \kappa \leq \frac{C}{4^n}$$
to complete the proof.
\end{proof}

%%%%%%%%%%%%%%%%%%%%%%%%%%%%%%%%%%%%%%%%%%%%%%%%%

\subsection{Induction on scale}
We now prove Theorem \ref{main lma}.

The first case that we consider is when $U$ is a set of locally finite perimeter for which there exists a scale $n_2$ such that $U$ has least perimeter in $B_{2^{-n_2}}$ and
\begin{equation}\label{induction on scale:base case}
    \Exc_{n_2}(U) < \sigma
\end{equation}
where $\sigma, n_1$ are the constants given by de Giorgi's lemma, Proposition \ref{DGL 3}.
A straightforward induction shows that if $b > a$, $c$ are real numbers and $(x_n) \subset \RR_+$ is a sequence such that
\begin{equation}\label{induction hypothesis}
x_n \leq \frac{x_{n - 1}}{a} + \frac{c}{b^n},
\end{equation}
then
\begin{equation}\label{induction conclusion}
x_n \leq a^{-n}\left[x_0 + \frac{bc}{b - a}\right].
\end{equation}
Applying (\ref{induction hypothesis}, \ref{induction on scale:base case}) and de Giorgi's lemma, we obtain (\ref{induction hypothesis}) with $x_n = \Exc_{n - n^*}(U)$, $n^* = \max(n_1, n_2)$, $a = 100/51$, $b = 4$, and $c = C$.
So (\ref{induction conclusion}) reads 
$$\Exc_n(U) \leq (0.51)^{n - n^*} \left[\sigma + \frac{51}{26}C\right] \lesssim_{n_2} (0.51)^n$$
and hence from Lemma \ref{excess bounds Cauchy sequence},
$$|\normal^{(n)} - \normal^{(n + m)}|^2 \lesssim_{n_2} \frac{(0.51)^{nd}}{|\partial^* U \cap B_{2^{-(n + m)}}}.$$
So Proposition \ref{doubling dimension} implies that 
$$|\normal^{(n)} - \normal^{(n + 1)}|^2 \lesssim_{n_2} 2^{n(d - 1)} (0.51)^{nd} \lesssim 2^{n(d - 1) - 0.97 nd} = 2^{0.03nd - n}.$$
However, as $d \leq 7$\footnote{One could just as well replace $51/100$ with a function of $d$ here. The only \emph{essential} use of $d \leq 7$ is in Proposition \ref{blowup theorem}.}, we conclude 
$$|\normal^{(n)} - \normal^{(n + 1)}| \lesssim_{n_2} \alpha^n$$
where $\alpha = 2^{-0.35}$, and hence 
$$|\normal^{(n)} - \normal^{(n + m)}| \lesssim_{n_2} \sum_{\ell=n}^\infty \alpha^\ell < \alpha^n.$$
From Proposition \ref{LebDiff}, it follows that 
\begin{equation}\label{normal convergence rate}
    |\normal^{(n)} - \normal| \lesssim_{n_2} \alpha^n.
\end{equation}

Now let $U$ be an arbitrary set of locally finite perimeter, which has least perimeter in an open set $V$.
If $P, Q \in \partial U \cap V$ are close enough, $P \in \partial^* U$, there is a unique geodesic $\gamma_Q$ from $Q$ to $P$.
The transitivity of $\Iso(\Hyp^d)$ means that we may assume that $P = O$.
Integrating the Killing vectors which are tangent to $\gamma$, we obtain an isometry $A_Q \in \Iso(\Hyp^d)$ which maps $O$ to $Q$.
We can then define $\normal^{(n)}_U(Q) \in T_Q' V$ by first computing $\normal^{(n)}_{A_Q^* U} \in T_O' V$, and then applying the linear map $T_O' V \to T_Q' V$ induced by $A_Q$ and the Levi-Civita connection of $\Hyp^d$.
Thus $\normal^{(n)}_U$ is a $|d1_U|\vol$-measurable $1$-form on $\Hyp^d$.
An inspection of the definitions shows that $\normal^{(n)}_U$ is continuous on $\partial^* U$.
We denote by $n_2(Q)$ the scale in (\ref{induction on scale:base case}), or $n_2(Q) = \infty$ if no such scale exists.

We claim that if $Q_k \to O$, then $n_2(Q_k)$ is eventually equal to $n_2(O)$, and is finite.
Let $\Hilb \subseteq T_O\Hyp^d$ denote the tangent space to $\partial^* U$ given by Proposition \ref{blowup theorem}.
We can approximate $\Hilb$ by the tangent rescalings of $\partial^* U$ at $P$, and in turn approximate those by the tangent rescalings of $\partial^* U$ at $Q_k$ if $k$ is large (translated to $T_P \Hyp^d$ by the isometry $A_Q$).
On the other hand it is clear that, if $Z$ is a half-space bounded by $\Hilb$, then $\Exc_n((\exp_P)_* Z) \lesssim 2^{-n}$.
In particular there exists $n_3 \in \ZZ$ such that $\Exc_{n_3}((\exp_P)_* Z) < \sigma/2$.
For $k$ large, it follows that $n_2(Q_k) \leq n_3$.
We omit the details here, as they are extremely similar to \cite[Teorema 4]{Miranda67}.

Since the rate of convergence in (\ref{normal convergence rate}) only depends on $n_2$, it follows that $\normal^{(n)} \to \normal$ locally uniformly on $\overline{\partial^* U}$, so that $\normal$ is continuous on $\overline{\partial^* U}$.
Therefore Proposition \ref{locality of Caccioppoli} implies that $\partial U$ is a $C^1$ minimal hypersurface.
Elliptic theory \cite{morrey2009multiple} now implies that $\partial U$ is smooth, and analytic if $M$ is.