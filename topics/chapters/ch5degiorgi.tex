\section{Regularity of minimal hypersurfaces}\label{DeGiorgiSection}
In this section we prove specialize to $\Hyp^d$ and show regularity of hypersurfaces.

\subsection{Excess: first properties}
The quantity which we seek to estimate is the so-called excess of a function of locally bounded variation.
Notions of excess for $\RR^d$ appear in the monographs \cite[\S5.3.1]{federer2014geometric} and \cite[Chapter 6]{Giusti77}.

We introduce the coordinate system $(x, y, z) \in \RR_+ \times \RR^{d - 2} \times \RR$ with the hyperbolic half-space metric 
\begin{equation}\label{hyperbolic metric}
g = \frac{dx^2 + dy_1^2 + \cdots + dy_{d - 2}^2 + dz^2}{x^2}.
\end{equation}
This does not agree with the usual convention for the hyperbolic half-space metric, but will be advantageous for our purposes when we represent $C^1$ hypersurfaces as graphs of functions $\omega: \RR_+ \times \RR^{d - 2} \to \RR$; then such a hypersurface can be written $\{z = \omega(x, y)\}$.
We also define the \dfn{hyperbolic origin} $O = (1, 0, 0)$ and write $B_r = B(O, r)$.
Whenever we sum over an index $\mu$, we mean that $\mu$ ranges over 
$$\mu \in \{x, y_1, \dots, y_{d - 2}, z\}.$$

\begin{definition}
Let $u \in BV_l(\Hyp^d)$. The \dfn{approximate derivative} to $u$ at scale $n$ is the covector in $T_O' \Hyp^d$ defined by
$$\normal^{(n)}_\mu = \frac{\int_{B_{2^{-n}}} x \partial_\mu u ~\vol}{\int_{B_{2^{-n}}} |du| ~\vol}.$$
\end{definition}

Since $g(x\partial_\mu, x\partial_\nu) = \delta_{\mu\nu}$, $x\partial_\mu u = \normal_\mu$ and so, if $O$ is a Lebesgue point, then $\normal^{(n)} \to \normal_U(O)$ pointwise.
But $\Iso(\Hyp^d)$ acts transitively on $\Hyp^d$ and so the conjugation of $\normal^{(n)}$ to other cotangent spaces by $\Iso(\Hyp^d)$ defines a continuous section of $T'\Hyp^d$ defined on $\partial^* U$ which converges to $\normal_U$ pointwise $|du|\vol$-almost everywhere.
Moreover it is clear that the conjugation of $\normal^{(n)}$ to $T'_P\Hyp^d$ is continuous in $P$.
So to show $\normal$ is continuous, it suffices to show that $(\normal^{(n)})$ is locally uniformly Cauchy, and this goal is accomplished by the excess:

\begin{definition}
The \dfn{excess} of $u \in BV_l(\Hyp^d)$ at scale $n$ is 
$$\Exc_n(u) = 2^{n(d - 1)} (\exp(2^{-n}) - |\normal^{(n)}|) \int_{B_{2^{-n}}} |du| ~\vol$$
where $|\normal^{(n)}|$ denotes the length of $\normal^{(n)}$ in $T_O' \Hyp^d$.
\end{definition}

To motivate the somewhat strange term $\exp(2^{-n})$ we observe that since $(dx^\mu)$ is an orthonormal basis of $T_O' \Hyp^d$ we have 
$$|\normal^{(n)}|^2 = \sum_\mu \left(\int_{B_{2^{-n}}} \normal_\mu ~d\lambda_n\right)^2$$
where
$$\lambda_n(E) = \frac{\int_E |du|~\vol}{\int_{B_{2^{-n}}} |du| ~\vol}$$
for Borel subsets $E$ of $2^{-n}$, so $\lambda_n$ is a probability measure on $B_{2^{-n}}$.
Since $\lambda_n$ is a probability measure, the Cauchy-Schwarz inequality gives 
$$|\normal^{(n)}|^2 \leq \int_{B_{2^{-n}}} \sum_\mu \normal_\mu^2 ~d\lambda_n = \int_{B_{2^{-n}}} x^2 ~d\lambda_n(x) \leq \sup_{(x, y, z) \in B_{2^{-n}}} x^2.$$
Writing $ds = \sqrt g$ for the line element in $\Hyp^d$, we have
$$2^{-n} = \int_1^x ds = \int_1^x \frac{dt}{t} = \log x$$
if $(x, y, z) \in \partial B_{2^{-n}}$ and $x$ is maximal possible. It follows that
\begin{equation}\label{conormal length bound}
|\normal^{(n)}| \leq \exp(2^{-n})
\end{equation}
and so in particular
$$\Exc_n(u) \geq 0.$$

From the law of cosines and (\ref{conormal length bound}),
\begin{align*}
|\normal^{(n)} - \normal^{(n + m)}|^2 &= |\normal^{(n)}|^2 + |\normal^{(n + m)}|^2 - 2 g(\normal^{(n)}, \normal^{(n + m)})\\
&\leq 2(\exp(2^{-2n}) - g(\normal^{(n)}, \normal^{(n + m)})) \\
&= \frac{2}{\int_{B_{2^{-(n + m)}}} |du| ~\vol} \int_{B_{2^{-(n + m)}}} \exp(2^{-2n}) |du| - \sum_\mu x \partial_\mu u \normal_\mu^{(n)} ~\vol.
\end{align*}
Applying (\ref{conormal length bound}) again,
$$\exp(2^{-2n}) |du| - \sum_\mu x \partial_\mu u \normal_\mu^{(n)} \geq 0$$
on $B_{2^{-n}}$, so it follows that 
\begin{align*}
|\normal^{(n)} - \normal^{(n + m)}|^2 &\leq \frac{2}{\int_{B_{2^{-(n + m)}}} |du| ~\vol} \int_{B_{2^{-n }}} \exp(2^{-2n}) |du| - \sum_\mu x \partial_\mu u \normal_\mu^{(n)} ~\vol\\
&= \frac{2}{\int_{B_{2^{-(n + m)}}} |du| ~\vol} (\exp(2^{-2n}) - |\normal^{(n)}|^2) \int_{B_{2^{-n}}} |du| ~\vol.
\end{align*}
We can now apply the inequality 
$$x^2 - y^2 \leq 4(x - y),$$
which is valid when $x \geq \max(1, y)$ (thus for $x = \exp(2^{-n})$ and $y = |\normal^{(n)}|$)
to deduce\footnote{An analogous estimate on the approximate derivative for a function on $\RR^d$, without the correction term $\exp(2^{-n}) - 1$, appears in \cite[pg661]{Miranda66}.}
\begin{equation}\label{excess bounds Cauchy sequence}
|\normal^{(n)} - \normal^{(n + m)}|^2 \leq \frac{2^{3 + n(1 - d)}}{\int_{B_{2^{-(n + m)}}} |du| ~\vol} \Exc_n(u).
\end{equation} 

In addition to the estimate (\ref{excess bounds Cauchy sequence}), an essential property of the excess is rotation invariance, which we now formulate.

\begin{notation}
We define an action
\begin{equation}\label{hyperbolic rotation}
    \Phi: \Orth(\RR^d) \to \Iso(\Hyp^d)
\end{equation}
of the orthogonal group $\Orth(\RR^d)$, as well as the representation
$$A^* u(P) = u(\Phi(A)(P))$$
of $\Orth(\RR^d)$ on $BV_l(\RR^d)$, as follows.
Recall that the hyperbolic metric in the ball model $\DD^d$ is radial, so the natural action of the orthogonal group $\Orth(\RR^d)$ on $\DD^d$ is by isometry, and if we use the Cayley transform to identify the origin of $\DD^d$ with $O$, then we obtain a faithful action (\ref{hyperbolic rotation}) whose orbits are the spheres $\partial B_r$ centered on $O$.
\end{notation}

\begin{lemma}\label{excess rotation invariant}
For every $A \in \Orth(\RR^d)$ and $u \in BV_l(\Hyp^d)$,
$$\Exc_n(A^* u) = \Exc_n(u).$$
\end{lemma}
\begin{proof}
Since $A$ is an isometry, $|(d A^* u)_{\Phi(A)(P)}| = |(du)_P|$, so
\begin{align*}
\int_{B_{2^{-n}}} |dA^*u| ~\vol &= \int_0^{2^{-n}} \int_{\partial B_r} |dA^* u| ~\vol_{\partial B_r} ~dr = \int_0^{2^{-n}} \int_{\partial B_r} |du| ~\vol_{\partial B_r} ~dr\\
&= \int_{B_{2^{-n}}} |du| ~\vol
\end{align*}
since $(A^{-1})^* \vol_{\partial B_r} = \vol_{\partial B_r}$.
Since $A$ is an isometry, $|A_* (x \partial_\mu)| = 1$ and $(A_*(x \partial_\mu))_\mu$ is an orthonormal basis of $T_O \Hyp^d$.
The claim follows.
\end{proof}

\subsection{Excess and Plateau energy}
Our next task is to formulate and prove Lemma \ref{excess vs plateau energy}, which relates the excess of $1_U$ to the extrinsic geometry of $\partial U$.
To do this we will use the machinery of Appendices \ref{coarea section} and \ref{crossproducts} and the following lemma, which is an analogue of \cite[Theorem 4.8]{Giusti77} and allows us to represent $\partial U$ as a graph:

\begin{lemma}\label{hopfKilling}
Let $N$ be a $C^r$ hypersurface in $\Hyp^d$ which bounds an open set, $r \geq 1$, and 
$$||\normal_N^\sharp - x\partial_z||_{L^\infty(N)} \leq \kappa^2$$
where $\kappa \in [0, 1)$.
Then there exists a scale $n^* \in \ZZ$ and a function
$$\omega \in C^r(\RR_+ \times \RR^{d - 2} \to \RR)$$
such that 
\begin{align}
    N \cap B_{2^{-n^*}} &= \{(x, y, z) \in B_{2^{-n^*}}: z = \omega(x, y)\}, \label{N is a graph}\\
    ||d\omega||_{L^\infty(B_{2^{-n^*}})} &\leq \kappa. \label{derivative bounds}
\end{align}
\end{lemma}
\begin{proof}
We write $N = \partial U$, $u = 1_U$.
From the law of cosines and the fact that $\normal^\sharp$ and $x\partial_z$ both have unit length,
$$||\normal^\sharp - x\partial_z||_{L^\infty(N)}^2 = 2(1 - (\normal, x\partial_z)) = 2\left(1 - \frac{x\partial_z u}{|du|}\right).$$
Therefore 
$$x\partial_z u \geq \left(1 - \frac{\kappa^2}{2}\right) |du|.$$
Given $\alpha$, a vector of unit length in $\RR_x \times \RR^{d - 2}_y \times (\RR_+)_z$, we can define a unit vector field
$$X_\alpha = \sum_\mu \alpha^\mu x\partial_\mu.$$
Then 
$$\alpha^z x\partial_z u \geq \alpha^z \left(1 - \frac{\kappa^2}{2}\right) |du|$$
so from the Cauchy-Schwarz inequality and the fact that 
$$1 - \left(1 - \frac{\kappa^2}{2}\right)^2 \leq \kappa^2,$$
we have 
$$\left|\sum_{\mu \neq z} \alpha^\mu x\partial_\mu z\right| \leq \kappa\sqrt{1 - (\alpha^z)^2} |du|$$
and hence 
$$X_\alpha u \geq \left(\alpha^z - \frac{\alpha^z \kappa^2}{2} - \kappa\sqrt{1 - (\alpha^z)^2}\right)|du| \geq (\alpha^z - \kappa)|du|.$$
So by Proposition \ref{Giusti46}, if $\alpha^z > \kappa$, then every integral curve of $X_\alpha$ passes through $\partial U$ exactly once.
But the integral curve of $X_\alpha$ which passes through $(0, y, z)$ is the euclidean line 
$$t \mapsto (t \alpha^x + y_1 + t \alpha^{y_1}, \dots, y_{d - 2} + t \alpha^{y_{d - 2}}, t \alpha^z)$$
so the existence of a scale $n^*$ and a $C^r$ function $\omega$ satisfying (\ref{N is a graph}, \ref{derivative bounds}) follows from the $C^r$ implicit function theorem. 
\end{proof}

We now pause to introduce a large amount of notation that we will use throughout Section \ref{DeGiorgiSection}.

\begin{definition}
The \dfn{hyperbolic Plateau energy} of a $1$-form $\psi$ on $(\RR_+)_x \times \RR^{d - 2}_y$ is the $d-1$-form
$$\Lagrange(\psi) = x^{1 - d} \sqrt{1 + |\psi|^2} ~\vol$$
where the metric and volume form are both euclidean.
\end{definition}

\begin{notation}[hyperbolic space as a line bundle]\label{hyperbolic line bundle}
If $\omega$ is a $C^r$ function $\Omega \to \RR_z$ with graph $N \subseteq \Hyp^d$, where $\Omega \subseteq (\RR_+)_x \times \RR^{d - 2}_y$ is open, we introduce the closed $C^r$ embedding
\begin{align*}
    \Psi_N: \Omega &\to \Hyp^d \\
    (x, y) &\mapsto (x, y, \omega(x)).
\end{align*}
It is a $C^r$ diffeomorphism $\Omega \to N$.
We also introduce the projection 
\begin{align*}
    \Pi: \Hyp^d &\to (\RR_+)_x \times \RR^{d - 2}_y\\
    (x, y, z) &\mapsto (x, y)
\end{align*}
of which $\Psi_N$ is a section.
\end{notation}

It follows immediately from Example \ref{graphs in riemannian manifolds} that 
\begin{equation}\label{Lagrangian formula}
\Psi_N^* \vol_N = \Lagrange(d\omega)
\end{equation}
if $N$ is the graph of $\omega \in C^1(\Omega)$.

\begin{notation}[tensor fields on euclidean space]
Since $\RR_+ \times \RR^{d - 2}$ has flat Levi-Civita connection, we can identify tensor fields on $\RR_+ \times \RR^{d - 2}$ with smooth maps $\RR_+ \times \RR^{d - 2} \to \Hilb$ for some finite-dimensional Hilbert space $\Hilb$.
Using the Bochner integral (see Appendix \ref{coarea section}), it therefore makes sense to integrate a tensor field to obtain an element of $\Hilb$ (that is, a tensor), and we write
$$A_n T = \dashint_{\Pi(B_{2^{-n}})} T$$
for the Bochner mean of $T$ over $\Pi(B_{2^{-n}})$ with respect to Lebesgue measure.
It also makes sense to identify a tensor with a constant tensor field, which we will do without mention.
\end{notation}

\begin{notation}[miscellany]
We write $\tilde B_r$ for the euclidean ball in $\RR_+ \times \RR^{d - 2}$ centered on $(1, 0, \dots, 0)$ of radius $r > 0$.
We write $x[a,b] = [ax, ab]$ and $x + [a, b] = [a + x, b + x]$.
\end{notation}

\begin{lemma}\label{excess vs plateau energy}
Let $\omega \in C^1(\Omega)$. Then
\begin{equation}
    \Exc_n(1_{\omega(x, y) < z}) \in 2^{n(d - 1)} \left(\int_{\Pi(B_{2^{-n}})} \Lagrange(d\omega) - \Lagrange(A_n d\omega)\right) [\exp(-2^{-n}), \exp(2^{-n})]. \label{excess vs Lagrangian}
\end{equation}
\end{lemma}
\begin{proof}
Let $u(x, y, z) = 1_{\omega(x, y) < z}$, let $N$ be the graph of $\omega$, and let $\normal^{(n)}$ be the approximate derivative of $u$.
It follows from the definitions, the fact that $|du|$ is the Radon-Nikod\'ym derivative $\vol_N/\vol$, and (\ref{Lagrangian formula}), that 
$$\normal^{(n)}_\mu = \int_{B_{2^{-n}}} \normal_\mu |du| ~\vol = \int_{B_{2^{-n}} \cap N} \normal_\mu \vol_N = \int_{\Pi(B_{2^{-n}})} (\Psi_N^* \normal)_\mu \Lagrange(d\omega).$$
From Example \ref{graphs in riemannian manifolds}, 
$$(\Psi^* \normal)_z \Lagrange(d\omega) = x^p ~dxdy$$
where $p = 3/2 - d$. Similarly we have 
$$(\Psi^* \normal)_i \Lagrange(d\omega) = x^p \partial_i \omega(x, y) ~dxdy$$
whenever $i \in \{x, y_1, \dots, y_{d - 2}\}$.
Therefore 
\begin{align*}
    |\normal^{(n)}|^2 \int_{B_{2^{-n}}} |du| ~\vol &= \left(\int_{\Pi(B_{2^{-n}})} x^p ~dxdy\right)^2 + \sum_i \left(\int_{\Pi(B_{2^{-n}})} x^p \partial_i \omega(x, y) ~dxdy \right)^2 \\
    &\in |\Pi(B_{2^{-n}})|^2 (1 + |A_nd\omega|^2) \left[\inf_{(x, y, z) \in B_{2^{-n}}} x^{p/2}, \sup_{(x, y, z) \in B_{2^{-n}}} x^{p/2}\right] \\
    &= \left(\int_{\Pi(B_{2^{-n}})} \Lagrange(A_nd\omega)\right)^2 \left[\inf_{(x, y, z) \in B_{2^{-n}}} x^q, \sup_{(x, y, z) \in B_{2^{-n}}} x^q\right] 
\end{align*}
where $4q = 2d - 1$. 
\end{proof}

%%%%%%%%%%%%%%%%%%%%%%%%%%%%%%%%%%%%%%%%%%%%%%%%%

\subsection{Removing the \texorpdfstring{$C^1$}{C1} assumption}\label{proof of DGL}
We are now ready to prove Proposition \ref{de Giorgi lemma}, and also settle the choice of $c > 0$.
By local homogeneity, if we can find $\sigma = \sigma(P)$, then $\sigma$ is locally constant and hence constant, so we may restrict to any particular $P \in M$.
Let $U$ be a set of locally finite perimeter, $u = 1_U$, and $\Psi' = (K_\mu')$ an orthonormal Killing frame.
Let $v_r = a^{-1}\int_{B_\rho} K_\mu' u/|K_\mu'| ~\vol \evect^\mu$, $a = |\int_{B_\rho} K_\mu' u/|K_\mu'| ~\vol \evect^\mu|$, so $v \in \RR^d$ has unit length and we can find an orthogonal matrix $O$ with $Ov = \evect^0$.
We then define $\Psi = (K_\mu)$ where $K_\mu = O^\nu_\mu K_\nu'$, so that 
$$\int_{B_\rho} K_\mu u ~\vol = O^\nu_\mu \int_{B_\rho} K_\nu' u ~\vol = aO^\nu_\mu v_\nu = a\delta_\mu^0.$$
Suppose that $R$ is small enough that $1/2 \leq g(\partial_\mu', \partial_\mu') \leq 2$ on $B_R$.

Let $\gamma^* > 0$ be the constant from Proposition \ref{mollifier quant} with $\delta = \rho$, and suppose that $\sigma \leq \gamma^*$ and $R$ is less than the injectivity radius of $M$.
Then for almost every $t \in (0, \rho)$, there exists a set $V \subseteq B_t$ with $C^1$ perimeter and satisfying (\ref{mollifier quant1}, \ref{mollifier quant2}, \ref{mollifier quant3}, \ref{mollifier quant4}) with $X = \partial_0'$.
It follows that $V$ satisfies (\ref{DGLC1 normal points up}, \ref{DGLC1 almost minimal}) with $B_\rho$ replaced by $B_t$ and $\beta \geq \rho^{d - 1} \gamma$, $\kappa = \varepsilon^{1/2}$.
From (\ref{mollifier quant2}, \ref{mollifier quant3}),
$$|\Lambda(U, t) - \Lambda(V, t)| \lesssim \varepsilon \rho^{d - 1} \gamma.$$
Therefore, (\ref{DGLC1 small excess}) also holds for $V, B_t$ with
$$\beta = \rho^{d - 1} \gamma(1 + O(\varepsilon)).$$
Let $\rho^* > 0$ be the constant from Lemma \ref{DGLC1} and suppose that $R < \rho^*$.
Then
$$\Lambda(V, \rho/2) \leq \frac{1 + O(c + c\varepsilon^{1/4})}{2^{d + 1}} \beta,$$
so putting everything together, we find an absolute constant $b > 0$ such that if $t < \rho$ is large enough depending on $\varepsilon$,
$$\Lambda(U, \rho/2) \leq \frac{1 + b(c + c\varepsilon^{1/4} + \varepsilon)}{2^{d + 1}} \Lambda(U, \rho/2).$$
We may now choose $c, \varepsilon$ so small depending on $b$ that 
$$\Lambda(U, \rho/2) \leq 2^{-d} \Lambda(U, \rho)$$
which was to be shown.