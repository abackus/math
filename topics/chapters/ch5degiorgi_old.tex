\section{de Giorgi lemma}



\begin{lemma}\label{volume form is Lagrangian}
Let $N$ be a $C^1$ hypersurface in $M$, such that there exists a Killing foliation $\chi$ associated to a triple $(H, \partial_0, P)$ and a $C^1$ function $\omega$ whose graph is $N$, c.f. (\ref{N is a graph}).
Let $\vol_N$ be the volume form induced on $N$ by $g$, and let $\varphi(x) = (x, \omega(x))$ be the induced diffeomorphism $H \to N$.
Then:
\begin{enumerate}
\item The area Lagrangian $\Lagrange$ associated to $(H, X, P)$ satisfies
\begin{equation}\label{volume form pullback}
\varphi^* \vol_N = \Lagrange(d\omega).
\end{equation}
\item Let $\partial_1, \dots, \partial_{d - 1}$ be (the translates by $\partial_0$ of) a quasieuclidean frame on $H$. Then
$$\varphi^*(u_{,\mu} ~\vol) = \lambda^d (\delta_\mu^0 - \delta^i_\mu \omega_{,i}) ~\vol_h.$$
\end{enumerate}
\end{lemma}
\begin{proof}
Since $\partial_0$ is a Killing field,
\begin{equation}\label{preservation of scalar field}
0 = \delta^\lambda_0 g_{\mu \nu,\lambda} + \delta^\lambda_{0,\mu} g_{\nu\lambda} + \delta^\lambda_{0,\nu} g_{\mu \lambda} = g_{\mu\nu,0},
\end{equation}
so the scalar field $g_{\mu\nu}$ is preserved by $\varphi^*$.
Moreover, since $\partial_0$ is assumed normal to $H$, $g_{0i} = 0$.
If we write $\slashed g$ for the metric induced by $g$ on $N$, then
$$\varphi^* \vol_N = \varphi^* \sqrt{\det \slashed g} ~dx = \sqrt{\varphi^* \det \slashed g} ~dx.$$
From Lemma \ref{cross product formula} with $\psi = d\omega$, we obtain
\begin{align*}
\varphi^* \vol_N &= \lambda\sqrt{1 + |d\omega|_{h^{-1}}^2} \sqrt{\det h} ~dx = \Lagrange(d\omega)
\end{align*}
which proves (\ref{volume form pullback}).
Using (\ref{volume form pullback}, \ref{preservation of scalar field}) and the fact that $N$ is $C^1$, we fix a small open set $S$ and compute
\begin{align*}
\int_S u_{,\mu} ~\vol &= \int_{S \cap N} \normal_\mu ~\vol_N \\
&= \int_{\varphi^{-1}(S \cap N)} \sqrt{\frac{g_{00}}{1 + |\psi|_{h^{-1}}^2}} (\delta_\mu^0 - \delta_\mu^i \omega_{,i})\Lagrange(d\omega) \\
&= \int_{\varphi^{-1}(S \cap N)} \lambda^d (\delta_\mu^0 - \delta_\mu^i \omega_{,i}) \vol_h \qedhere.
\end{align*}
\end{proof}

\begin{notation}
Write $A(\rho)f = f(h, B_\rho, \Psi)$ for a $1$-form $f$ on $H$, where $B_\rho = B_h(P, \rho)$.
\end{notation}

\begin{lemma}\label{excess vs MVP}
In the situation of Lemma \ref{volume form is Lagrangian}, with $\Psi$ the trivialization of $T^* H$ induced by $\partial_1, \dots, \partial_{d - 1}$, $\Lambda_\Psi(U, r)$ is an element of the closed interval
$$\left[(1 - O(c))\int_{B_r} \Lagrange(dw) - \sqrt{1 + |A(\rho)d\omega|^2}\vol_h, (1 + O(c)) \int_{B_r} \Lagrange(dw) - \sqrt{1 + |A(\rho)d\omega|^2}\vol_h\right]$$
where the implied constant is absolute.
\end{lemma}
\begin{proof}
TODO: Prove it! Also we haven't actually defined $c$ yet so this is putting the cart before the horse, but that's easy to fix
\end{proof}

%%%%%%%%%%%%%%%%%%%%%%%%%%%%%%%%%%%%%%%%%%%%%

\subsection{Deforming the mean-value property}
Our next task is to show that the area Lagrangian (\ref{area Lagrangian}) satisfies an approximate mean-value property, which gives an improvement on \cite[Teorema 4.3]{Miranda66}.
TODO: Rewrite this to account for $A(\rho)$.

\begin{notation}
Let $(H, h, P)$ be a pointed Riemannian manifold of dimension $d - 1$.
We define a metric $\tilde h$ with vanishing curvature by considering normal coordinates $x_1, \dots, x_{d - 1}$ with respect to $h$ and setting $\tilde h_{ij} = \delta_{ij}$, thus
$$h_{ij} = \tilde h_{ij} + O(x^2)$$
where the implied constant depends continuously on $h$ and its derivatives at $P$.
We then define the $\tilde h$-Dirichlet energy
$$\DirL(\psi) = \frac{\tilde h^{ij} \psi_i \psi_j}{2}\vol_{\tilde h}.$$
Let $B_r$ denote a ball centered on $P$ in the euclidean space $(H, \tilde h)$, not in $M$.
Let $A(r)f$ denote the $\tilde h$-mean of $f$ over $B_r$.
Note that we can take means of $1$-forms as well as functions, since $\tilde h$ is flat.
\end{notation}

\begin{lemma}\label{Taylor lemma}
Let $(H, h, P)$ be a pointed Riemannian manifold of dimension $d - 1$, let $\lambda$ be a scalar field on $H$, let $\Lagrange$ satisfy (\ref{area Lagrangian}), and suppose that there exists $c \in (0, 1)$ such that for every $1$-form $\psi$,
\begin{align}
1 - c \leq |\lambda|, g_{00} \leq 1 + c, \label{estimate on Killing weight} \\
1 - c \leq \frac{|\psi|^2_{h^{-1}}}{|\psi|^2_{\tilde h^{-1}}} \leq 1 + c, \label{estimate on normal coordinates}\\
1 - c \leq \sqrt{\frac{\det h}{\det \tilde h}} \leq 1 + c, \label{estimate on normal volume form}
\end{align}
If $p,q$ are $1$-form on $H$ such that $|q|^2_{h^{-1}} \lesssim 1$, then $\Lagrange(p) - \Lagrange(q)$ lies in the closed interval
$$\left[\frac{1 - O(c)}{\Lagrange(q)}(\DirL(p) - \DirL(q)) - O(\DirL(p) - \DirL(q))^2, (1 + O(c))(\DirL(p) - \DirL(q))\right]$$
where all implied constants are absolute.
\end{lemma}
\begin{proof}
Let $\DirL_h(Y) = \sqrt{1 + h_{ij} Y^i Y^j}/2 ~\vol_h$ be the Dirichlet energy with respect to $h$.
Applying (\ref{estimate on normal coordinates}, \ref{estimate on normal volume form}),
\begin{equation}\label{estimate on normal coordinates 2}
1 - O(c) \leq \frac{\DirL(p) - \DirL(q)}{\DirL_h(p) - \DirL_h(q)} \leq 1 + O(c).
\end{equation}
By Taylor's theorem, there exists $\xi \geq 0$ between $|p|$ and $|q|$ such that
\begin{align*}
\lambda^{-1}(\Lagrange(p) - \Lagrange(q)) &= \sqrt{1 + |p|^2} - \sqrt{1 + |q|^2}\\
&= \frac{|p|^2 - |q|^2}{2 \sqrt{1 + |q|^2}} - \frac{(|p|^2 - |q|^2)^2}{8(1 + \xi^2)^{3/2}} \\
&= \frac{\DirL_h(p) - \DirL_h(q)}{\sqrt{1 + |q|^2}} - \frac{(\DirL_h(p) - \DirL_h(q))^2}{2(1 + \xi^2)^{3/2}}.
\end{align*}
Since $|q|^2 \geq 0$ and the second term is nonpositive, it follows from (\ref{estimate on Killing weight}, \ref{estimate on normal coordinates 2}) that
$$\Lagrange(p) - \Lagrange(q) \leq (1 + c)(\DirL_h(p) - \DirL_h(q)) \leq (1 + O(c))(\DirL(p) - \DirL(q)).$$
If $|q|^2_h$ is small enough,
$$2\Lagrange(q) \leq 8 \leq 8(1 + \xi^2)^{3/2},$$
whence
\begin{align*}
\frac{(1 + c)^{-1}(\DirL(p) - \DirL(q))}{\Lagrange(q)} - \frac{2(1 + c)^{-2}(\DirL(p) - \DirL(q))^2}{\Lagrange(q)} &\leq \Lagrange(p) - \Lagrange(q).
\end{align*}
Another application of (\ref{estimate on normal coordinates 2}) completes the proof.
\end{proof}

\begin{lemma}\label{Dirichlet}
For every $c \in (0, 1)$ there exists $\rho^* = \rho^*(c, g, P) > 0$ which depends continuously on $P$ and with the following property.

Let $H$ be a hypersurface which is normal to a Killing field, and for which $\chi,\Lagrange$ are given by Definition \ref{Killing setup}.
Let $\omega: H \to \RR$ be $C^1$, and suppose that $U \subseteq M$ is an open set such that
$$\partial U = \{\chi(x, \omega(x)): x \in H\}.$$
Then for every $\rho \in (0, \rho^*)$ and $\alpha, \kappa, \beta \in (0, 1)$ such that
\begin{align}
||d\omega||_{L^\infty(B_\rho)} &\leq \kappa, \label{Dirichlet small derivative}\\
\int_{B_\rho} \Lagrange(d\omega) - \Lagrange(A(\rho)d\omega) &\leq \beta, \label{Dirichlet close to mean} \\
\int_{B_\rho} \Lagrange(d\omega) &\leq \eta(U, \rho) + \beta \kappa, \label{Dirichlet close to minimal}
\end{align}
it follows that
\begin{equation}\label{Dirichlet gain}
\int_{B_{\alpha\rho}} \Lagrange(d\omega) - \Lagrange(A(\alpha\rho)d\omega) \leq (1 + O(c))\alpha^{d + 1}\beta + O(\beta \kappa^{1/2})
\end{equation}
where all implied constants are absolute.
\end{lemma}
The role of these parameters is as follows: we will take $\beta,\kappa \to 0$ in a proof by compactness and contradiction, and when we apply Proposition \ref{mollifier proposition}, we will take $\alpha \to 1/2$. One should think of $\beta$ as much smaller than $\kappa$, which in turn is much smaller than $c,\alpha,\rho^*$.
Indeed, $c$ will later be chosen absolutely, so $\rho^*$ only depends on $P,g$.
Once we have fixed $\rho^*$, we will iteratively halve the scale $\rho$ so that we can capitalize on (\ref{Dirichlet gain}).
\begin{proof}
Let $h,\lambda$ be as in Definition \ref{Killing setup}.
Then $\lambda(P) = 1$ and $\lambda$ depends only on $g,P$.
So we can use (\ref{Dirichlet small derivative}) and the fact that $\kappa < 1$ to select $\rho^*$ independently of $\omega$, so that such that
if $\rho < \rho^*$, then $\lambda$ is controlled by (\ref{estimate on Killing weight}),
$h$ is controlled by (\ref{estimate on normal coordinates}, \ref{estimate on normal volume form}), and
$$||d\omega||_{L^\infty(B_\rho)} \cdot |B_\rho| \lesssim 1.$$
Thus we will be able to apply Lemma \ref{Taylor lemma}.
Since $h$ is of the form $h(v, w) = g(X, X)^{-1} g(v, w)$ for a Killing field $X$, and the space of Killing fields defined near $P$ is finite-dimensional, $h$ and its derivatives are all controlled by $g,P$ and so $\rho^*$ is determined by $c,P,g$.

Let $u$ be the $\tilde h$-harmonic function on $B_\rho$ such that $u|\partial B_\rho = \omega|\partial B_\rho$.
By definition of $\eta(U, \rho)$ and (\ref{Dirichlet close to minimal}),
\begin{equation}\label{Dirichlet close to harmonic}
\int_{B_\rho} \Lagrange(d\omega) - \Lagrange(\nabla u) \leq \int_{B_\rho} \Lagrange(d\omega) - \eta(U, \rho) \leq \beta\kappa.
\end{equation}
We now proceed analogously to the proof of \cite[Lemma 4.2]{Miranda66}. By Lemma \ref{Taylor lemma},
\begin{align*}
\int_{B_{\alpha\rho}} \Lagrange(d\omega) - \Lagrange(A(\alpha\rho)d\omega) &\leq (1 + c)\int_{B_{\alpha\rho}} \DirL(d\omega) - \DirL(A(\alpha\rho)d\omega) \\
&= \frac{1 + c}{2} \int_{B_{\alpha\rho}} |d\omega|^2 - |A(\alpha\rho)d\omega|^2.
\end{align*}
Since $A(\alpha\rho)d\omega$ is the mean of $d\omega$, we have for every $\varepsilon > 0$
\begin{align*}
\int_{B_{\alpha\rho}} |d\omega|^2 - |A(\alpha\rho)d\omega|^2 &\leq \int_{B_{\alpha\rho}} (d\omega - A(\rho)d\omega)^2 ~dx \\
&\leq (1 + \varepsilon^{-1})\int_{B_\rho} |d\omega - \nabla u|^2 ~dx\\
&\qquad + (1 + \varepsilon) \int_{B_{\alpha\rho}} |\nabla u - A(\rho)d\omega|^2 ~dx\\
&=: O(\varepsilon^{-1})I + (1 + \varepsilon)J.
\end{align*}
To estimate $I$, we use the mean-value property, Lemma \ref{Taylor lemma}, and (\ref{Dirichlet close to harmonic}):
\begin{align*}
I &= \int_{B_\rho} |d\omega - \nabla u|^2 = \int_{B_\rho} |d\omega|^2 - |\nabla u|^2 \lesssim \int_{B_\rho} \Lagrange(d\omega) - \Lagrange(\nabla u) \leq \beta \kappa.
\end{align*}
To estimate $J$, we apply \cite[Lemma 4.1]{Miranda66}:
\begin{align*}
J &= \int_{B_\rho} |\nabla u|^2 - |A(\rho)d\omega|^2 = \int_{B_\rho} |\nabla u|^2 - |A(\rho)\nabla u|^2 \\
&\leq \alpha^{d + 1} \int_{B_\rho} |\nabla u|^2 - |A(\rho)\nabla u|^2 \leq \alpha^{d + 1} \int_{B_\rho} |d\omega|^2 - |A(\rho)d\omega|^2 \\
&\leq \alpha^{d + 1} \int_{B_\rho} |d\omega - A(\rho)d\omega|^2 ~dx \\
&\leq (1 + \varepsilon)\alpha^{d + 1} \int_{B_\rho} |d\omega - A(\rho)d\omega|^2  + O(\varepsilon^{-1})\int_{B_\rho} |d\omega - \nabla u|^2\\
&=: (1 + \varepsilon)\alpha^{d + 1}K + O(\varepsilon^{-1})I.
\end{align*}
We already estimated $I \lesssim \beta \kappa$, and now we estimate $K$: by Lemma \ref{Taylor lemma} and (\ref{Dirichlet small derivative}, \ref{Dirichlet close to mean}),
\begin{align*}
K &= \int_{B_\rho} |d\omega|^2 - |A(\rho)d\omega|^2\\
&\leq 2\int_{B_\rho} \DirL(d\omega) - \DirL(A(\rho)d\omega)\\
&\leq 2(1 + O(c))\int_{B_\rho} \Lagrange(d\omega) - \Lagrange(A(\rho)d\omega) + O(1) \int_{B_\rho} (\DirL(d\omega) - \DirL(A(\rho)d\omega))^2\\
&\leq 2(1 + O(c))\beta + O(1) ||d\omega||_{L^\infty(B_\rho)} \int_{B_\rho} \DirL(d\omega) - \DirL(A(\rho)d\omega) \\
&\leq 2\beta(1 + O(c + \kappa)).
\end{align*}
If we set $\varepsilon = \kappa^{1/2}$ then we conclude
\begin{align*}
\int_{B_{\alpha\rho}} \Lagrange(d\omega) - \Lagrange(A(\alpha\rho)d\omega)
&\leq (1 + O(c + \varepsilon)) \alpha^{d + 1}\beta + O(\beta \kappa \varepsilon^{-1})\\
&\leq (1 + O(c))\alpha^{d + 1} \beta + O(\beta \kappa^{1/2}). \qedhere
\end{align*}
\end{proof}

\subsection{\texorpdfstring{$C^1$}{C1} case}
With the above setup, we now are ready to show the $C^1$ case of de Giorgi's lemma \cite[Teorema 4.4]{Miranda66}.

\begin{lemma}\label{DGLC1}
Suppose that $M$ is a locally homogeneous Riemannian manifold, $U$ is a set of least perimeter, $\normal = \normal_U$, $\beta, \kappa, c > 0$, and $P \in \partial^* U$.
Let $(H, X)$ be the tangent hypersurface and Killing field induced by $(M, P)$ (c.f. Lemma \ref{hopfKilling}), and let $\Psi$ be the trivialization of $T^* H$ induced by a quasieuclidean frame on $(H, P)$ (c.f. Lemma \ref{extend to quasieuclid}).
If $\rho < \rho^*$ where $\rho^* = \rho^*(c, P, g)$ depends continuously on $P$, and
\begin{align}
||\normal^\sharp - X||_{L^\infty(B_\rho)} \leq \kappa^2, \label{DGLC1 normal points up} \\
|\partial U \cap B(P, \rho)| \leq \eta(U, B(P, \rho)) + \beta \kappa, \label{DGLC1 almost minimal} \\
\Lambda_\Psi(U, \rho) \leq \beta, \label{DGLC1 small excess}
\end{align}
then 
\begin{equation}
\Lambda_\Psi(U, \rho/2) \leq \frac{1 + O(c)}{2^{d + 1}} \beta + O(\beta \kappa^{1/2}) \label{DGLC1 conclusion}
\end{equation}
where all implied constants are absolute.
\end{lemma}
\begin{proof}
Using Lemma \ref{hopfKilling}, there exists a Killing foliation $\chi$ associated to $(H, X)$, an open submanifold $M' \subseteq M$, and a $C^1$ function
$$\omega: H \cap M' \to \RR$$
such that $N = \partial U \cap M'$ equals $\{(x, \omega(x)): x \in H \cap M'\}$, $\omega(x) = 0$, and 
$$||d\omega||_{L^\infty(B(P, \rho))} \lesssim ||(\normal, X) - (X, X)||_{L^\infty(B(P, \rho))}^{1/2}.$$
The space of Killing fields is finite-dimensional and so has a compact $L^\infty$-unit ball; therefore we can choose $\rho^*$ independently of $U$ so that $||X||_{L^\infty(B(P, \rho^*))} \leq 2$, in which case (\ref{DGLC1 normal points up}) gives
\begin{equation}\label{DGLC1 small derivative}
||d\omega||_{L^\infty(B(P, \rho))} \leq A\kappa
\end{equation}
for some absolute constant $A > 0$. If we assume that $\kappa < 1/A$, then it follows that 
$$||\omega||_{L^\infty(B(P, \rho))} < \rho ||d\omega||_{L^\infty(B(P, \rho))} \leq \rho.$$
Therefore $N \subseteq S_\rho$ and using the comparability of norms (TODO: Fill in the annoying details here) we may replace $B(P, \rho)$ with $S_\rho$.
From Lemma \ref{volume form is Lagrangian} and (\ref{DGLC1 almost minimal}),
$$\int_{B_\rho} \Lagrange(d\omega) \leq \eta(U, B(P, \rho)) + \beta \kappa$$
and Lemma \ref{excess vs MVP}, (\ref{DGLC1 small excess}) gives 
$$\int_{B_\rho} \Lagrange(d\omega) - \sqrt{1 + |A_\Psi(\rho)d\omega|^2} \leq (1 + O(c)) \beta$$
where the constant is absolute.
We may therefore apply Lemma \ref{Dirichlet} and (\ref{DGLC1 small derivative}) to obtain
$$\int_{B_{\rho/2}} \Lagrange(d\omega) - \sqrt{1 + |A_\Psi(\rho)d\omega|^2} \leq \frac{1 + O(c)}{2^{d + 1}} \beta + O(\beta \kappa^{1/2}) \beta$$
and another application of Lemmata \ref{volume form is Lagrangian}, \ref{excess vs MVP} now gives (\ref{DGLC1 conclusion}).
\end{proof}
