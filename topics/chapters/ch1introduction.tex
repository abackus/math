\section{Introduction}
Throughout this paper, let $M$ be an oriented Riemannian manifold of metric $g$ and dimension $d$.
TODO: Adjust these definitions to account for universal cover.

\begin{definition}\label{main definitions}
A function $u$ of locally bounded variation has \dfn{least gradient} if for every compactly supported function $v$ of bounded variation, such that $\supp v \subseteq U \Subset M$,
$$\int_U |du| ~\vol \leq \int_U |du + dv| ~\vol.$$
A set $U$ of locally finite perimeter has \dfn{least perimeter} if $1_U$ has least gradient.
\end{definition}

\begin{definition}
A \dfn{minimal lamination} in $M$ is a partition of a closed subset of $M$ into smooth hypersurfaces with zero mean curvature.
The minimal lamination $\lambda$ is \dfn{analytic} if $(M, g)$ is analytic and each of the hypersurfaces in $\lambda$ is analytic.
\end{definition}

Our first theorem is an analogue of the maximum principle for least gradient functions on euclidean space \cite[Proposition 3.4]{górny2017planar}.

\begin{theorem}[maximum principle]\label{main thm}
Suppose that $2 \leq d \leq 7$ and suppose that $M = \Hyp^d/\Gamma$ is hyperbolic. 
Let $u: M \to \RR$ be a function of least gradient, and $A_y = \partial \{u > y\}$.
Then $(A_y)_{y \in \RR}$ is a minimal lamination in $M$, which is analytic if $(M, g)$ is.
\end{theorem}

If $d = 2$, then Theorem \ref{main thm} can be extended to manifolds with boundary, just as in \cite[Corollary 3.5]{górny2017planar}:

\begin{theorem}[maximum principle up to the boundary]\label{main crly}
Let $\overline \Sigma$ be a convex hyperbolic surface with boundary and suppose that $u: \Sigma \to \RR$ is a function of least gradient defined on the interior $\Sigma$ of $\overline \Sigma$.
Then, if $A_y = \partial \{u > y\}$, $(A_y)_{y \in \RR}$ extends to a geodesic lamination of $\overline \Sigma$.
\end{theorem}

%%%%%%%%%%%%%%%%%%%%%%%%%%%%%%%%%%%%%%%%%%%%%%%%%%

\subsection{Reduction to regularity of minimal hypersurfaces}

Theorems \ref{main thm} and \ref{main crly} are a consequence of the following regularity theorem for sets of least perimeter, which is our main theorem:

\begin{theorem}[regularity of minimal hypersurfaces]\label{main lma}
Suppose that $2 \leq d \leq 7$.
Then every set of least perimeter in $\Hyp^d$ is bounded by a smooth minimal hypersurface $N$.
Furthermore, $N$ is analytic if $(\Hyp^d, g)$ is.
\end{theorem}

Analogous results to Theorem \ref{main lma} include regularity of minimal currents weighted by an elliptic integrand, proven by Federer \cite[\S5.3]{federer2014geometric}, and regularity of sets of least perimeter on $\RR^d$ developed by Miranda \cite{Miranda64} \cite{Miranda66} \cite{Miranda67}.
Our strategy closely mirrors Miranda's, and in this paper we emphasize the steps of the proof where significant modifications of Miranda's arguments are necessary.

\begin{proof}[Proof of Theorems \ref{main thm} and \ref{main crly}]
We first pass to the universal cover $\Hyp^d$.
By Corollary \ref{level sets are minimal}, for every function $u$ of least gradient, the superlevel sets of $u$ have least perimeter.
Let
\begin{equation}\label{lamination union}
A = \bigcup_y \partial \{u > y\},
\end{equation}
let $B$ be the interior of $\{du = 0\}$, and let $x \in M$.
Then $x \in B$ iff $u = u(x)$ near $x$, but that happens iff for every $y < u(x)$, $x$ is interior to $\{u > y\}$ and for every $y \geq u(x)$, $x$ is exterior to $\{u > y\}$.
This happens iff for every $y \in \RR$, $x$ is either interior or exterior to $\{u > y\}$, thus $x \notin \partial \{u > y\}$, which happens iff $x \notin A$.
Thus $\{A, B\}$ is a partition of $M$, so $A$ is closed.
Moreover, the sets $\{u > y\}$ are totally ordered by $\subseteq$, so the sets $\partial \{u > y\}$ are disjoint.
They are also hypersurfaces with the desired amount of regularity, by Theorem \ref{main lma}.
This proves Theorem \ref{main thm}.

Given Theorem \ref{main thm} and the monotonicity formula, Proposition \ref{Monotonicity Formula}, the proof of Theorem \ref{main crly} is identical to that of \cite[Corollary 3.5]{górny2017planar}.
\end{proof}

%%%%%%%%%%%%%%%%%%%%%%%%%%%%%%%%%%%%%%%%%%%%%%%%%%

\subsection{Application to \texorpdfstring{$\infty$}{infinity}-harmonic functions}
Our interest in hyperbolic manifolds is motivated by a recent preprint of Daskalopoulos--Uhlenbeck \cite{daskalopoulos2020transverse}, which in turn was inspired by Thurston's $L^\infty$-Teichm\"uller theory \cite{thurston1998minimal}.

\begin{definition}
A function $u: M \to \RR$ is \dfn{best-Lipschitz} if for every compactly supported function $v$ of bounded variation, such that $\supp v \subseteq U \Subset M$,
$$||du||_{L^\infty} \leq ||d(u + v)||_{L^\infty}.$$
If $p \in (1, \infty)$, then $u$ is $p$-\dfn{harmonic} if for every such $v$,
$$||du||_{L^p} \leq ||d(u + v)||_{L^p}.$$
\end{definition}

\begin{definition}
If a best-Lipschitz function $u$ is the weak limit in $L^r$ for $r > d$ of $p$-harmonic functions as $p \to \infty$, we call $u$ \dfn{$\infty$-harmonic}.
In that case we define the \dfn{set of maximum stretch}
$$\lambda_u = \{x \in M: L(x) = \sup L\}$$
where $L(x)$ denotes the local Lipschitz constant of $u$ at $x$.
\end{definition}

\begin{theorem}\label{infinity harmonic laminations}
Suppose that $M$ is a closed hyperbolic surface and $u$ is an $\infty$-harmonic function. Then the set of maximum stretch $\lambda_u$ is a geodesic lamination in $M$.
\end{theorem}
\begin{proof}
See \cite[\S5]{daskalopoulos2020transverse}.
\end{proof}

Daskalopoulos--Uhlenbeck prove Theorem \ref{infinity harmonic laminations} by viewing $u$ as a viscosity solution to the $\infty$-Laplace equation 
\begin{equation}\label{infinity laplace}
    \Hess u(\grad u, \grad u) = 0.
\end{equation}
However, the theory of viscosity solutions of (\ref{infinity laplace}) \cite{Aronsson1984} \cite{Evans08} is still nascent, and Daskalopoulos--Uhlenbeck ask for a proof of Theorem \ref{infinity harmonic laminations} that bypasses (\ref{infinity laplace}) altogether.
We give a partial resolution of this problem, as follows.

Let $M$ be a surface and $1/p + 1/q = 1$.
One can show that given $p$-harmonic functions $u_p$, we can find a best-Lipschitz function $u_\infty$ which is the weak limit of a subsequence of $(u_p)$ as $p \to \infty$ in $L^r$ for $r$ large.
Moreover, the antiderivative of $|du_p|^{p - 2} * du_p$ is $q$-harmonic.
We introduce the rescaling $k_p \in \RR_+$ defined by 
$$||k_p du_p||_{L^p}^p = k_p$$
and define a closed $1$-form $dv_q$ by
$$dv_q = k_p^{p - 1} |du_p|^{p - 2} * du_p.$$

\begin{proposition}\label{existence of dual sections}
Let $u$ be an $\infty$-harmonic function on a closed hyperbolic surface $M$, and let $dv_q$ be as above. Then there exists an affine bundle $E \to M$, sections $v_q$ of $E$ which are antiderivatives of $dv_q$, and a section $v$ of $E$, such that:
\begin{enumerate}
\item $v_q \to v$ in $L^\infty$ on a subsequence as $q \to 1$,
\item $dv_q \to dv$ vaguely on a subsequence,
\item $v$ has least gradient,
\item and $\supp dv$ is contained in the set of maximum stretch $\lambda_u$.
\end{enumerate}
\end{proposition}
\begin{proof}
See \cite[\S6]{daskalopoulos2020transverse}.
\end{proof}

We now show a weaker version of Theorem \ref{infinity harmonic laminations} whose proof does not require viscosity solution theory for (\ref{infinity laplace}):

\begin{corollary}
Let $u$ be an $\infty$-harmonic function on a closed hyperbolic surface $M$.
Then the dual section $v$ of Proposition \ref{existence of dual sections} induces a geodesic lamination $\lambda \subseteq \lambda_u$.
\end{corollary}
\begin{proof}
This is immediate from Theorem \ref{main thm}.
\end{proof}

We note that it is not clear that $\lambda = \lambda_u$ in the above corollary, essentially because it is not clear that the map $du \mapsto dv$ is injective (so $dv$ could be $0$ on $\lambda_u$).

Daskalopoulos--Uhlenbeck also ask for a partial converse to the above results and the fact that $dv$ endows $\lambda_u$ with the structure of an oriented measured lamination, which we now prove. For the definition of the Ruelle-Sullivan $1$-current of an oriented, transversely measured geodesic lamination, see \cite[\S8]{daskalopoulos2020transverse} or the original paper of Ruelle--Sullivan \cite{Ruelle75}.

\begin{corollary}\label{ruelle sullivan antiderivative}
Let $\lambda$ be an oriented, transversely measured geodesic lamination on a closed hyperbolic surface $M$, and let $dv$ be the Ruelle-Sullivan $1$-current induced by $\lambda$.
Then there exists a function $\tilde v: \Hyp^2 \to \RR$ of least gradient whose derivative $d\tilde v$ drops to $dv$.
\end{corollary}
\begin{proof}
As observed by Daskalopoulos--Uhlenbeck \cite[\S9]{daskalopoulos2020transverse}, if we lift $dv$ to a $1$-current $d\tilde v$ on $\Hyp^2$, then $d\tilde v$ is exact and any antiderivative $\tilde v$ of $d\tilde v$ has superlevel sets $\{\tilde v \geq y\}$ which are bounded by geodesics.
The claim now follows from Proposition \ref{minimal bounding implies least gradient}.
\end{proof}

%%%%%%%%%%%%%%%%%%%%%%%%%%%%%%%%%%%%%%%%%%%%%%%

\subsection{Application to computational geometry}
Recent work by Loisel \cite{Loisel20} shows that the barrier method minimizes the $p$-Dirichlet energy $||\nabla u||_{L^p(\Omega)}$
subject to boundary data on euclidean space, with $O(n^{1/2} \log n)$ time complexity uniformly in $p \in [1, \infty]$, where $n$ is the cardinality of the given triangulation $T$ of $\Omega$.
One can adapt this method to the hyperbolic setting by quadrature... TODO how to compute $|\nabla u|_{L^p}$ in this setting??
Applying this method with $p = 1$, we can construct functions of least gradient.

We shall now show how Theorem \ref{main thm} associates a minimal lamination $\lambda_\alpha$ to each cohomology class $\alpha \in H^1(M, \RR)$, and how we can use Loisel's algorithm to compute $\lambda_\alpha$.
Let $\Gamma = \pi_1(M)$, so $M = \Hyp^d/\Gamma$, and fix a fundamental polytope $\Omega$ of $\Gamma$.
By the Hurcewiz theorem, $(\RR, \alpha)$ is a representation of $\Gamma$.
We consider the space $E$ of $\alpha$-equivariant functions $f: \partial \Omega \to \RR$, thus
\begin{equation}\label{boundary data for Loisel}
f(\gamma x) = f(x) + \alpha(\gamma)
\end{equation}
for every $\gamma \in \Gamma$, which are constant on each face.
The relation (\ref{boundary data for Loisel}) is an undetermined boundary condition for functions in $E$, and so we consider the subspace $E'$ of functions which in addition are zero on a maximal set of faces such that we do not determine $f|\partial \Omega = 0$, thus any function $E' \subseteq E$ has a completely determined trace $t$ on $\partial \Omega$.
By Corollary \ref{compactness}, there exists a function $u \in E'$ which has least gradient on $\Omega$.
By Theorem \ref{main thm}, we obtain a minimal lamination of $\Omega$ and hence of $M$. 

Following \cite[\S4]{Loisel20}, we extend $t$ to $\Omega$ and approximate it by piecewise-linear elements on $\Omega$.
One can then use Loisel's algorithm to minimize $||\nabla v + \nabla t||_{L^1(\Omega)}$ in $W^{1, 1}_0(\Omega)$ and put $u = v - t$ to obtain a minimal lamination of $\Omega$.

TODO: Do some numerical experiments, show what minimal laminations in a fundamental polytope in $\Hyp^3$ look like


%%%%%%%%%%%%%%%%%%%%%%%%%%%%%%%%%%%%%%%%%%%%%%%

\subsection{Some open problems}
We believe that Theorem \ref{main lma} could be extended to a wider class of Riemannian manifolds:

\begin{conjecture}\label{main conj}
Suppose that $2 \leq d \leq 7$ and $M$ is a simply connected Riemannian $d$-fold. Then Theorem \ref{main lma} holds with $\Hyp^d$ replaced by $M$.
\end{conjecture}

This conjecture implies Theorems \ref{main thm}, \ref{main crly} for arbitrary $M$ by passing to the universal cover.
Since our proof strongly uses the symmetries of $\Hyp^d$, it would be reasonable to first try to prove Conjecture \ref{main conj} for locally homogeneous manifolds, and especially those locally homogeneous manifolds $M$ such that for every $P \in M$ there is a natural action of $\Orth(\RR^d)$ on $M$ by isometry that fixes $P$.
Another line of attack would be to show that our methods are perturbative, so that if $(M, g)$ is a simply connected Riemannian manifold such that $g$ is ``close to constant negative curvature'' in some sense, then Conjecture \ref{main conj} holds for $(M, g)$.

It seems highly unlikely that Theorem \ref{main lma} can be extended to $\Hyp^8$, due to the existence of Simons cones in $\RR^8$ \cite[Theorem A]{BOMBIERI1969}.
Simons cones are not of least perimeter in $\Hyp^8$, but one could potentially perturb them to obtain a set of least perimeter whose boundary is singular:

\begin{problem}
    Construct a set $U$ of least perimeter in $\Hyp^8$ such that $\partial U$ has a singularity of codimension $8$.
\end{problem}

We would also like to know that the best-Lipschitz/least-gradient duality gives a complete proof of Theorem \ref{infinity harmonic laminations}:

\begin{conjecture}
Let $u$ be an $\infty$-harmonic function with dual least-gradient section $v$.
Then $\supp dv = \lambda_u$.
\end{conjecture}

%%%%%%%%%%%%%%%%%%%%%%%%%%%%%%%%%%%%%%%%%%%%%%%

\subsection{Outline of the paper}
We begin with the preliminaries.
In \S\ref{RiemMeasureThy}, we record facts that we will use about sets of locally finite perimeter on Riemannian manifolds.
We introduce the notion of a \dfn{bundle-valued Radon measure}, which is a generalized section of a vector bundle that, upon trivialization, becomes a vector-valued Radon measure.
With the basic theory of bundle-valued Radon measures it is easy to show that the reduced boundary of a set of least perimeter is independent of the choice of metric.
This allows us in \S\ref{LeastGradientFunctions} to develop the theory of functions of least gradient on Riemannian manifolds. A generalization of Miranda's theorem \cite[Teorema 3]{Miranda67} on the stability of functions of least gradient, and generalizations of its standard consequences, easily fall out.
We then focus on sets of least perimeter: we prove a monotonicity formula, compute the dimension of the reduced boundary as $d - 1$, and prove the existence of tangent spaces to the reduced boundary for $d \leq 7$.

We are then ready to prove Theorem \ref{main lma}.
In \S\ref{MollifierSection}, we show Proposition \ref{mollifier quant}, which says that a set of least perimeter can be approximated by $C^1$ hypersurfaces that are approximately minimal.
This result is then used in \S\ref{DeGiorgiSection} to prove Theorem \ref{main lma}. We do this by proving an analogue of the de Giorgi regularity lemma \cite[Teorema 5.7]{Miranda66} for $\Hyp^d$. The de Giorgi lemma is the only point of the proof where we use the constant negative curvature of $M$.
We rely on the transitivity of $\Iso(\Hyp^d)$ as well as the fact that one can represent hypersurfaces in $\Hyp^d$ on euclidean space in a way which approximately preserves the surface area and normal vector field.

We include an appendix in which we prove a coarea formula, Proposition \ref{Coarea2}, that we will frequently use and could not find a reference for.

%%%%%%%%%%%%%%%%%%%%%%%%%%%%%%%%%%%%%%%%%%%%%%%%

\subsection{Acknowledgements}
I would like to thank Georgios Daskalopoulos for suggesting this project and for many helpful discussions.
I would also like to thank Joshua Lin for suggesting the proof of Lemma \ref{cross product formula}.
