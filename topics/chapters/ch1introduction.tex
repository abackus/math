\section{Introduction}
Throughout this paper, let $M$ be an oriented Riemannian manifold of metric $g$ and dimension $d$.
TODO: Adjust these definitions to account for universal cover.

\begin{definition}\label{main definitions}
A function $u$ of locally bounded variation has \dfn{least gradient} if for every compactly supported function $v$ of bounded variation, such that $\supp v \subseteq U \Subset M$,
$$\int_U |du| ~\vol \leq \int_U |du + dv| ~\vol.$$
A set $U$ of locally finite perimeter class has \dfn{least perimeter} if $1_U$ has least gradient.
\end{definition}

\begin{definition}
A \dfn{minimal lamination} in $M$ is a partition of a closed subset of $M$ into smooth hypersurfaces with zero mean curvature.
The minimal lamination $\lambda$ is \dfn{analytic} if $(M, g)$ is analytic and each of the hypersurfaces in $\lambda$ is analytic.
\end{definition}

Our first theorem is an analogue of the maximum principle for least gradient functions on euclidean space \cite[Proposition 3.4]{górny2017planar}.

\begin{theorem}[maximum principle]\label{main thm}
Suppose that $2 \leq d \leq 7$ and suppose that $M = \Hyp^d/\Gamma$ is hyperbolic. 
Let $u: M \to \RR$ be a function of least gradient, and $A_y = \partial \{u > y\}$.
Then $(A_y)_{y \in \RR}$ is a minimal lamination in $M$, which is analytic if $(M, g)$ is.
\end{theorem}

If $d = 2$, then Theorem \ref{main thm} can be extended to manifolds with boundary, just as in \cite[Corollary 3.5]{górny2017planar}:

\begin{theorem}[maximum principle up to the boundary]\label{main crly}
Let $\overline \Sigma$ be a convex hyperbolic surface with boundary and suppose that $u: \Sigma \to \RR$ is a function of least gradient defined on the interior $\Sigma$ of $\overline \Sigma$.
Then, if $A_y = \partial \{u > y\}$, $(A_y)_{y \in \RR}$ extends to a geodesic lamination of $\overline \Sigma$.
\end{theorem}

Theorems \ref{main thm} and \ref{main crly} can be easily shown to follow from standard results and the following regularity theorem for sets of least perimeter, which is our main theorem:

\begin{theorem}[regularity of minimal hypersurfaces]\label{main lma}
Suppose that $2 \leq d \leq 7$.
Then every set of least perimeter in $\Hyp^d$ is bounded by a smooth minimal hypersurface $N$.
Furthermore, $N$ is analytic if $(\Hyp^d, g)$ is.
\end{theorem}

Analogous results to Theorem \ref{main lma} include regularity of minimal currents weighted by an elliptic integrand, proven by Federer \cite[\S5.3]{federer2014geometric}, and regularity of sets of least perimeter on $\RR^d$ developed by Miranda \cite{Miranda64} \cite{Miranda66} \cite{Miranda67}.
Our strategy closely mirrors Miranda's, and in this paper we emphasize the steps of the proof where significant modifications of Miranda's arguments are necessary.

%%%%%%%%%%%%%%%%%%%%%%%%%%%%%%%%%%%%%%%%%%%%%%%%%%

\subsection{$\infty$-harmonic/least-gradient duality}
Our interest in hyperbolic manifolds is motivated by a recent preprint of Daskalopoulos--Uhlenbeck \cite{daskalopoulos2020transverse}, which in turn was inspired by Thurston's $L^\infty$-Teichm\"uller theory \cite{thurston1998minimal}.
The innovation of Daskalopoulos--Uhlenbeck is the \dfn{$\infty$-harmonic/least-gradient duality} that we now formulate.

\begin{definition}
A function $u: M \to \RR$ is \dfn{best-Lipschitz} if for every compactly supported function $v$ of bounded variation, such that $\supp v \subseteq U \Subset M$,
$$||du||_{L^\infty} \leq ||d(u + v)||_{L^\infty}.$$
If $p \in (1, \infty)$, then $u$ is $p$-\dfn{harmonic} if for every such $v$,
$$||du||_{L^p} \leq ||d(u + v)||_{L^p}.$$
If a best-Lipschitz function $u$ is the weak limit in $L^r$ for $r > d$ of $p$-harmonic functions as $p \to \infty$, we call $u$ $\infty$-\dfn{harmonic}.
\end{definition}

One can show \cite[Theorem 2.4]{daskalopoulos2020transverse} that given $p$-harmonic functions $u_p$, we can find a best-Lipschitz function $u_\infty$ which is the weak limit of a subsequence of $(u_p)$ as $p \to \infty$ in $L^r$ for $r$ large.
On the other hand, if $q$ is the H\"older dual of $p$ and $d = 2$, then the antiderivative $v_q$ of $|du_p|^{p - 2} * du_p$ is $q$-harmonic.
Moreover, if $M$ is a surface, then as $q \to 1$, a subsequence of $(v_q)$ converges in $L^\infty_l$ to a function $v_1 \in L^\infty_l \cap BV_l$ such that $dv_q \to dv_1$ vaguely and $v_1$ has least gradient \cite[Theorem 6.10]{daskalopoulos2020transverse}.

\begin{definition}
Let $u$ be $\infty$-harmonic on a surface $M$. The \dfn{least-gradient dual} of $u$, denoted $u'$, is the function $v_1$ given by the above construction.
\end{definition}

$\infty$-harmonic functions on hyperbolic manifolds are closely associated to geodesic laminations:

\begin{theorem}\label{infinity harmonic laminations}
Suppose that $M$ is a closed hyperbolic surface and $u$ is an $\infty$-harmonic function whose local Lipschitz constant at $x$ is denoted $L(x)$. Then
$$\lambda_u = \{x \in M: L(x) = \sup L\}$$
is a geodesic lamination in $M$.
\end{theorem}

Daskalopoulos--Uhlenbeck prove Theorem \ref{infinity harmonic laminations} by viewing $u$ as a viscosity solution to the $\infty$-Laplace equation 
\begin{equation}\label{infinity laplace}
    \Hess u(\grad u, \grad u) = 0,
\end{equation}
c.f. \cite[\S5]{daskalopoulos2020transverse}.
However, the theory of viscosity solutions of (\ref{infinity laplace}) is still nascent, c.f. \cite{Aronsson1984} \cite{Evans08}, and Daskalopoulos--Uhlenbeck ask \cite[\S9]{daskalopoulos2020transverse} for a proof of Theorem \ref{infinity harmonic laminations} that bypasses (\ref{infinity laplace}) altogether.
We accomplish this by applying Theorem \ref{main thm} to the least-gradient dual $u'$.

TODO: Say something about hyperbolic threefolds fibered over $S^1$

%%%%%%%%%%%%%%%%%%%%%%%%%%%%%%%%%%%%%%%%%%%%%%%

\subsection{Outline of the paper}
We begin with the preliminaries.
In \S\ref{RiemMeasureThy}, we record facts that we will use about sets of locally finite perimeter on Riemannian manifolds.
We introduce the notion of a \dfn{bundle-valued Radon measure}, which is a generalized section of a vector bundle that, upon trivialization, becomes a vector-valued Radon measure.
TODO: Get rid of this step, just turn everything into currents
With the basic theory of bundle-valued Radon measures it is easy to show that the reduced boundary of a set of least perimeter is independent of the choice of metric.
This allows us in \S\ref{LeastGradientFunctions} to develop the theory of functions of least gradient on Riemannian manifolds. A generalization of Miranda's theorem \cite[Teorema 3]{Miranda67} on the stability of functions of least gradient, and generalizations of its standard consequences, easily fall out.
We then focus on sets of least perimeter: we prove a monotonicity formula, compute the dimension of the reduced boundary as $d - 1$, and prove the existence of tangent spaces to the reduced boundary for $d \leq 7$.

We are then ready to prove Theorem \ref{main lma}.
In \S\ref{MollifierSection}, we show Proposition \ref{mollifier quant}, which says that a set of least perimeter can be approximated by $C^1$ hypersurfaces that are approximately minimal.
This result is then used in \S\ref{DeGiorgiSection} to prove Theorem \ref{main lma}. We do this by proving an analogue of the de Giorgi regularity lemma \cite[Teorema 5.7]{Miranda66} for $\Hyp^d$. The de Giorgi lemma is the only point of the proof where we use the constant negative curvature of $M$.
We rely on the transitivity of $\Iso(\Hyp^d)$ as well as the fact that one can represent hypersurfaces in $\Hyp^d$ on euclidean space in a way which approximately preserves the surface area and normal vector field.

In \S\ref{proof of main thm}, we prove Theorems \ref{main thm}, \ref{main crly}, and \ref{infinity harmonic laminations} by applying Theorem \ref{main lma} and the theory of \S\ref{LeastGradientFunctions}.

We include two appendices in which we prove a coarea formula and carry out some computations involving higher cross products in $\Hyp^d$.

%%%%%%%%%%%%%%%%%%%%%%%%%%%%%%%%%%%%%%%%%%%%%%%

\subsection{Some open problems}
We believe that Theorem \ref{main lma} could be extended to a wider class of Riemannian manifolds:

\begin{conjecture}\label{main conj}
Suppose that $2 \leq d \leq 7$ and $M$ is a simply connected Riemannian manifold. Then Theorem \ref{main lma} holds with $\Hyp^d$ replaced by $M$.
\end{conjecture}

This conjecture implies Theorems \ref{main thm}, \ref{main crly} for arbitrary $M$ by passing to the universal cover.
Since our proof strongly uses the symmetries of $\Hyp^d$, it would be reasonable to first try to prove Conjecture \ref{main conj} for locally homogeneous manifolds, and especially those locally homogeneous manifolds $M$ such that for every $P \in M$ there is a natural action of $\Orth(\RR^d)$ on $M$ by isometry that fixes $P$.
Another line of attack would be to show that our methods are perturbative, so that if $(M, g)$ is a simply connected Riemannian manifold such that $g$ is ``close to constant negative curvature'' in some sense, then Conjecture \ref{main conj} holds for $(M, g)$.

It seems highly unlikely that Theorem \ref{main lma} can be extended to $\Hyp^8$, due to the existence of Simons cones in $\RR^8$ (c.f. Theorem \ref{minimal cones in R8}).
Simons cones are not of least perimeter in $\Hyp^8$, but one could potentially perturb them to obtain a set of least perimeter whose boundary is singular:

\begin{problem}
    Construct a set $U$ of least perimeter in $\Hyp^8$ such that $\partial U$ has a singularity of codimension $8$.
\end{problem}

Since least-gradient functions can be viewed as weak solutions of the $1$-Laplace equation (c.f. (\ref{EulerLagrange})), one should be able to modify the methods of \cite{Loisel20} to numerically construct minimal laminations:

\begin{problem}\label{numerical conj}
    Let $M$ be a closed hyperbolic manifold.
    Construct and implement a numerical algorithm which takes a homotopy class $[u] \in [M, S^1]$ as data and returns a minimal lamination of $M$ whose associated least-gradient map $u$ has homotopy class $[u]$.
\end{problem}

Resolution of Problem \ref{numerical conj} would facilitate numerical experiments of geometric analysts interested in the study of minimal laminations. 

%%%%%%%%%%%%%%%%%%%%%%%%%%%%%%%%%%%%%%%%%%%%%%%%

\subsection{Acknowledgements}
I would like to thank Georgios Daskalopoulos for suggesting this project and for many helpful discussions.
I would also like to thank Joshua Lin for suggesting the proof of Lemma \ref{cross product formula}.
