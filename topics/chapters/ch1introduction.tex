\section{Introduction}
Let $M$ be an oriented Riemannian manifold of metric $g$ and dimension $d$.

\begin{definition}\label{main definitions}
A function $u$ of locally bounded variation has \dfn{least gradient} if for every compactly supported function $v$ of bounded variation, such that $\supp v \subseteq U \Subset M$,
$$\int_U |du| ~\vol \leq \int_U |du + dv| ~\vol.$$
A set $U$ of locally finite perimeter has \dfn{least perimeter} if $1_U$ has least gradient.
\end{definition}

\begin{definition}
A \dfn{minimal lamination} in $M$ is a partition of a closed subset of $M$ into smooth hypersurfaces with zero mean curvature.
The minimal lamination $\lambda$ is \dfn{analytic} if $(M, g)$ is analytic and each of the hypersurfaces in $\lambda$ is analytic.
\end{definition}

This paper is dedicated to the proof of the following generalization of the maximum principle for least gradient functions on euclidean space \cite[Proposition 3.4]{górny2017planar}.

\begin{theorem}[maximum principle]\label{main thm}
Suppose that $2 \leq d \leq 7$ and
\begin{enumerate}
\item either $g$ has constant sectional curvature $\leq 0$,
\item or $g$ has constant curvature and $d = 2$.
\end{enumerate}
Let $u: M \to \RR$ be a function of least gradient, and $A_y = \partial \{u > y\}$.
Then $(A_y)_{y \in \RR}$ is a minimal lamination, which is analytic if $(M, g)$ is.
\end{theorem}

We are primarily concerned with the case that $M$ is a closed hyperbolic manifold, which is the setting of several erstwhile open problems stated in \cite[\S9]{daskalopoulos2020transverse} which Theorem \ref{main thm} can be used to solve. See \S\ref{BestDuality}.

If $d = 2$, then Theorem \ref{main thm} can be extended to manifolds with boundary, just as in \cite[Corollary 3.5]{górny2017planar}:

\begin{theorem}[maximum principle up to the boundary]\label{main crly}
Let $\overline \Sigma$ be a convex surface with boundary and suppose that $u: \Sigma \to \RR$ is a function of least gradient defined on the interior $\Sigma$ of $\overline \Sigma$.
Then, if $A_y = \partial \{u > y\}$, $(A_y)_{y \in \RR}$ extends to a geodesic lamination of $\overline \Sigma$.
\end{theorem}

Theorems \ref{main thm} and \ref{main crly} can be easily shown to follow from standard results and the following regularity theorem for sets of least perimeter, which is the main theorem of the present paper:

\begin{theorem}[regularity of minimal hypersurfaces]\label{main lma}
Suppose that $2 \leq d \leq 7$ and
\begin{enumerate}
\item either $g$ has constant sectional curvature $\leq 0$,
\item or $g$ has constant curvature and $d = 2$.
\end{enumerate}
Then every set of least perimeter is bounded by a smooth minimal hypersurface $N$.
Furthermore, $N$ is analytic if $(M, g)$ is.
\end{theorem}

A proof of an analogous result to Theorem \ref{main lma} for currents paired against an elliptic integrand is given by \cite[\S5.3]{federer2014geometric}; our proof uses a similar strategy but is rather ``hands-on" in that it avoids the use of homological integration, instead using the theory of functions of bounded variation as developed by Miranda in \cite{Miranda64} \cite{Miranda66} \cite{Miranda67}, and facts about minimal hypersurfaces in euclidean space.

Though the hypotheses of Theorem \ref{main lma} look odd, we only use the assumption of nonpositive curvature at a critical point to deduce that a certain elliptic operator which we construct is actually the euclidean Laplace operator when written in correctly chosen coordinates.
In general, such an operator would be a perturbation of the euclidean Laplacian and so a suitable analysis of perturbations of elliptic operators should extend all of our results to the case of constant sectional curvature, so it is reasonable to conjecture the following:

\begin{conjecture}\label{main conj}
Suppose that $2 \leq d \leq 7$ and $g$ has constant sectional curvature.
Then the level sets of a function of least gradient are a minimal lamination.
\end{conjecture}

The assumption of constant sectional curvature should be somewhat harder to remove, as we rely on the existence of certain Killing fields, and at one point use the Killing-Hopf theorem to reduce a general result to the euclidean case.
However, if one could show Conjecture \ref{main conj}, a proof for any Riemannian manifold of appropriate dimension should not be out of reach.

%%%%%%%%%%%%%%%%%%%%%%%%%%%%%%%%%%%%%%%%%%%%%%%

\subsection{Outline of the paper}
We begin with the preliminaries.
In \S\ref{RiemMeasureThy}, we record facts that we will use about sets of locally finite perimeter on Riemannian manifolds.
We introduce the notion of a \dfn{bundle-valued Radon measure}, which is a generalized section of a vector bundle that, upon trivialization, becomes a vector-valued Radon measure.
With the basic theory of bundle-valued Radon measures it is easy to show that the reduced boundary of a set of least perimeter is independent of the choice of metric.
We also show that a coarea formula holds in this setting, which we use in \S\ref{LeastGradientFunctions} to develop the theory of functions of least gradient on Riemannian manifolds. A generalization of Miranda's theorem \cite[Teorema 3]{Miranda67} on the stability of functions of least gradient, and generalizations of its standard consequences, easily fall out from the coarea formula.
We then focus on sets of least perimeter: we prove a monotonicity formula, compute the dimension of the reduced boundary as $d - 1$, and prove the existence of tangent spaces to the reduced boundary.

We are then ready to prove Theorem \ref{main lma}.
In \S\ref{MollifierSection}, we show Proposition \ref{mollifier proposition}, which says that a set of least perimeter can be approximated by $C^1$ hypersurfaces that are approximately minimal.
This result is then used in \S\ref{DeGiorgiSection}, to prove a generalization, Proposition \ref{de Giorgi}, of the de Giorgi regularity lemma \cite[Teorema 5.7]{Miranda66} for sets of least perimeter.
Most of the difficulty in this step, and indeed in this present paper, comes from proving the $C^1$ case of de Giorgi's lemma, as Proposition \ref{mollifier proposition} can be used to reduce Proposition \ref{de Giorgi} to this case.

In the euclidean case, the $C^1$ de Giorgi lemma is proven by comparing the surface area of the graph of a $C^1$ function $\omega$ to its Dirichlet energy $\int |d\omega|^2~\vol$, and then apply the mean-value property and the fact that $[\Delta, \partial_j] = 0$.
This does not quite work in general, as the commutator of the Laplace-Beltrami operator with a coordinate vector field will in general be nonzero, and the mean-value theorem only approximately holds for general Laplace-Beltrami operators.
Our main innovation here is to show that for a suitable space form, one has enough Killing fields to carry out a similar argument, in which the Dirichlet energy is actually the Lagrangian for the euclidean Laplacian.

In \S\ref{proof of main thm}, we use a standard inductive argument to show that Proposition \ref{de Giorgi} implies Theorem \ref{main lma}.
We then use Theorem \ref{main lma} and the general theory developed in \S\ref{LeastGradientFunctions} to prove Theorems \ref{main thm} and \ref{main crly}.

In \S\ref{BestDuality}, we use Theorem \ref{main thm} to use certain conjectures of Daskalopoulos and Uhlenbeck \cite{daskalopoulos2020transverse} motivated by Thurston's asymmetric metric \cite{thurston1998minimal}.

%%%%%%%%%%%%%%%%%%%%%%%%%%%%%%%%%%%%%%%%%%%%%%%%

\subsection{Acknowledgements}
I would like to thank Georgios Daskalopoulos for suggesting this project and for many helpful discussions.
I would also like to thank Joshua Lin for helpful discussions and in particular for suggesting the proof of Lemma \ref{cross product formula}.
