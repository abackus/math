\section{Introduction}
Throughout this paper, let $M$ be an oriented Riemannian manifold of metric $g$ and dimension $d$.
TODO: Adjust these definitions to account for universal cover.

\begin{definition}\label{main definitions}
A function $u$ of locally bounded variation has \dfn{least gradient} if for every compactly supported function $v$ of bounded variation, such that $\supp v \subseteq U \Subset M$,
$$\int_U |du| ~\vol \leq \int_U |du + dv| ~\vol.$$
A set $U$ of locally finite perimeter has \dfn{least perimeter} if $1_U$ has least gradient.
\end{definition}

\begin{definition}
A \dfn{minimal lamination} in $M$ is a partition of a closed subset of $M$ into smooth hypersurfaces with zero mean curvature.
The minimal lamination $\lambda$ is \dfn{analytic} if $(M, g)$ is analytic and each of the hypersurfaces in $\lambda$ is analytic.
\end{definition}

Our first theorem is an analogue of the maximum principle for least gradient functions on euclidean space \cite[Proposition 3.4]{górny2017planar}.

\begin{theorem}[maximum principle]\label{main thm}
Suppose that $2 \leq d \leq 7$ and suppose that $M = \Hyp^d/\Gamma$ is hyperbolic.
Let $u: M \to \RR$ be a function of least gradient, and $A_y = \partial \{u > y\}$.
Then $(A_y)_{y \in \RR}$ is a minimal lamination in $M$, which is analytic if $(M, g)$ is.
\end{theorem}

%%%%%%%%%%%%%%%%%%%%%%%%%%%%%%%%%%%%%%%%%%%%%%%%%%

\subsection{Reduction to regularity of minimal hypersurfaces}

Theorem \ref{main thm} is a consequence of the following regularity theorem for sets of least perimeter, which is our main theorem:

\begin{theorem}[regularity of minimal hypersurfaces]\label{main lma}
Suppose that $2 \leq d \leq 7$.
Then every set of least perimeter in $\Hyp^d$ is bounded by a smooth minimal hypersurface $N$.
Furthermore, $N$ is analytic if $(\Hyp^d, g)$ is.
\end{theorem}

Analogous results to Theorem \ref{main lma} include regularity of minimal currents weighted by an elliptic integrand, proven by Federer \cite[\S5.3]{federer2014geometric}, and regularity of sets of least perimeter on $\RR^d$ developed by Miranda \cite{Miranda64} \cite{Miranda66} \cite{Miranda67}.
Our strategy closely mirrors Miranda's, and in this paper we emphasize the steps of the proof where significant modifications of Miranda's arguments are necessary.

\begin{proof}[Proof of Theorem \ref{main thm}]
We first pass to the universal cover $\Hyp^d$.
By Corollary \ref{level sets are minimal}, for every function $u$ of least gradient, the superlevel sets of $u$ have least perimeter.
Let
\begin{equation}\label{lamination union}
A = \bigcup_y \partial \{u > y\},
\end{equation}
let $B$ be the interior of $\{du = 0\}$, and let $x \in M$.
Then $x \in B$ iff $u = u(x)$ near $x$, but that happens iff for every $y < u(x)$, $x$ is interior to $\{u > y\}$ and for every $y \geq u(x)$, $x$ is exterior to $\{u > y\}$.
This happens iff for every $y \in \RR$, $x$ is either interior or exterior to $\{u > y\}$, thus $x \notin \partial \{u > y\}$, which happens iff $x \notin A$.
Thus $\{A, B\}$ is a partition of $M$, so $A$ is closed.
Moreover, the sets $\{u > y\}$ are totally ordered by $\subseteq$, so the sets $\partial \{u > y\}$ are disjoint.
They are also hypersurfaces with the desired amount of regularity, by Theorem \ref{main lma}.
\end{proof}

%%%%%%%%%%%%%%%%%%%%%%%%%%%%%%%%%%%%%%%%%%%%%%%

\subsection{Some open problems}
We believe that Theorem \ref{main lma} could be extended to a wider class of Riemannian manifolds:

\begin{conjecture}\label{main conj}
Suppose that $2 \leq d \leq 7$ and $M$ is a simply connected Riemannian $d$-fold. Then Theorem \ref{main lma} holds with $\Hyp^d$ replaced by $M$.
\end{conjecture}

This conjecture implies Theorems \ref{main thm}, \ref{main crly} for arbitrary $M$ by passing to the universal cover.
Since our proof strongly uses the symmetries of $\Hyp^d$, it would be reasonable to first try to prove Conjecture \ref{main conj} for locally homogeneous manifolds, and especially those locally homogeneous manifolds $M$ such that for every $P \in M$ there is a natural action of $\Orth(\RR^d)$ on $M$ by isometry that fixes $P$.
Another line of attack would be to show that our methods are perturbative, so that if $(M, g)$ is a simply connected Riemannian manifold such that $g$ is ``close to constant negative curvature'' in some sense, then Conjecture \ref{main conj} holds for $(M, g)$.

It seems highly unlikely that Theorem \ref{main lma} can be extended to $\Hyp^8$, due to the existence of Simons cones in $\RR^8$ \cite[Theorem A]{BOMBIERI1969}.
Simons cones are not of least perimeter in $\Hyp^8$, but one could potentially perturb them to obtain a set of least perimeter whose boundary is singular:

\begin{problem}
    Construct a set $U$ of least perimeter in $\Hyp^8$ such that $\partial U$ has a singularity of codimension $8$.
\end{problem}

We would also like to know that the best-Lipschitz/least-gradient duality gives a complete proof of Theorem \ref{infinity harmonic laminations}:

\begin{conjecture}
Let $u$ be an $\infty$-harmonic function with dual least-gradient section $v$.
Then $\supp dv = \lambda_u$.
\end{conjecture}

%%%%%%%%%%%%%%%%%%%%%%%%%%%%%%%%%%%%%%%%%%%%%%%

\subsection{Outline of the paper}
We begin with the preliminaries in \S\ref{LeastGradientFunctions} by developing the theory of functions of least gradient on Riemannian manifolds. A generalization of Miranda's theorem \cite[Teorema 3]{Miranda67} on the stability of functions of least gradient, and generalizations of its standard consequences, easily fall out.
We then focus on sets of least perimeter: we prove a monotonicity formula, compute the dimension of the reduced boundary as $d - 1$, and prove the existence of tangent spaces to the reduced boundary for $d \leq 7$.

We are then ready to prove Theorem \ref{main lma}.
In \S\ref{MollifierSection}, we show Proposition \ref{mollifier quant}, which says that a set of least perimeter can be approximated by $C^1$ hypersurfaces that are approximately minimal.
This result is then used in \S\ref{DeGiorgiSection} to prove Theorem \ref{main lma}. We do this by proving an analogue of the de Giorgi regularity lemma \cite[Teorema 5.7]{Miranda66} for $\Hyp^d$. The de Giorgi lemma is the only point of the proof where we use the constant negative curvature of $M$.
We rely on the transitivity of $\Iso(\Hyp^d)$ as well as the fact that one can represent hypersurfaces in $\Hyp^d$ on euclidean space in a way which approximately preserves the surface area and normal vector field.

In \S\ref{proof of main thm} we discuss several applications of Theorem \ref{main thm}.

%%%%%%%%%%%%%%%%%%%%%%%%%%%%%%%%%%%%%%%%%%%%%%%%

\subsection{Acknowledgements}
I would like to thank Georgios Daskalopoulos for suggesting this project and for many helpful discussions.
I would also like to thank Joshua Lin for suggesting the proof of Lemma \ref{cross product formula}.
