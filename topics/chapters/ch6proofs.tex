\section{Applications}\label{proof of main thm}
\subsection{Well-posedness of the surface Dirichlet problem}
TODO 

If $d = 2$, then Theorem \ref{main thm} can be extended to manifolds with boundary, just as in \cite[Corollary 3.5]{górny2017planar}:

\begin{proposition}[maximum principle up to the boundary]\label{main crly}
Let $\overline \Sigma$ be a convex hyperbolic surface with boundary and suppose that $u: \Sigma \to \RR$ is a function of least gradient defined on the interior $\Sigma$ of $\overline \Sigma$.
Then, if $A_y = \partial \{u > y\}$, $(A_y)_{y \in \RR}$ extends to a geodesic lamination of $\overline \Sigma$.
\end{proposition}
\begin{proof}
From Proposition \ref{doubling dimension}, each connected component of $\partial \{u > y\}$ has surface area $\geq cr^{d - 1}$ in a ball $B_r$, but the total surface area of $\partial \{u > y\} \cap B_r$ is $\leq Cr^{d - 1}$, so the decomposition of $\partial \{u > y\}$ into minimal hypersurfaces is a union of at most $C/c < \infty$ sets.
Given the above facts, the reduction to Theorem \ref{main thm} is identical to that of \cite[Corollary 3.5]{górny2017planar}.
\end{proof}

%%%%%%%%%%%%%%%%%%%%%%%%%%%%%%

\subsection{Duality with \texorpdfstring{$\infty$}{infinity}-harmonic functions}
Our interest in hyperbolic manifolds is motivated by a recent preprint of Daskalopoulos--Uhlenbeck \cite{daskalopoulos2020transverse}, which in turn was inspired by Thurston's $L^\infty$-Teichm\"uller theory \cite{thurston1998minimal}. 

\begin{definition}
If a function $u$ is the weak limit in $L^r$ for $r > d$ of $p$-harmonic functions as $p \to \infty$, we call $u$ \dfn{$\infty$-harmonic}.
For an $\infty$-harmonic function $u$ we define the \dfn{maximum-stretch locus}
$$\lambda_u := \{x \in M: L(x) = \sup L\}$$
where $L(x)$ denotes the local Lipschitz constant of $u$ at $x$.
\end{definition}

In \cite[\S5]{daskalopoulos2020transverse}, Daskalopoulos--Uhlenbeck prove the following theorem by considering the viscosity solution theory of $\infty$-Laplace equation
\begin{equation}\label{infinity laplace}
    \Hess u(\grad u, \grad u) = 0.
\end{equation}

\begin{theorem}\label{infinity harmonic laminations}
Suppose that $M$ is a closed hyperbolic surface and $u$ is an $\infty$-harmonic function. Then the maximum-stretch locus $\lambda_u$ is a geodesic lamination in $M$.
\end{theorem}

However, the theory of viscosity solutions of (\ref{infinity laplace}) is still nascent, and Daskalopoulos--Uhlenbeck ask for a proof of Theorem \ref{infinity harmonic laminations} that bypasses (\ref{infinity laplace}) altogether, c.f. \cite[Problem 9.5]{daskalopoulos2020transverse}.
We give a partial resolution of this problem by proving \cite[Theorem-Conjecture 9.6]{daskalopoulos2020transverse}:

\begin{corollary}\label{maximum stretch contains lamination}
Let $u$ be an $\infty$-harmonic function on a closed hyperbolic surface $M$.
Then the maximum-stretch locus $\lambda_u$ contains a geodesic lamination.
\end{corollary}
\begin{proof}
By \cite[\S6]{daskalopoulos2020transverse}\footnote{The proof that such a section exists does not use Theorem \ref{infinity harmonic laminations}.}, there exists an affine bundle $E \to M$ and a section $v$ of least gradient\footnote{by which we mean that if one lifts $v$ to the universal cover $\Hyp^d$ then $v$ can be trivialized to a function of least gradient} of $E$ such that $\supp \dif v \subseteq \lambda_u$.
By Theorem \ref{main thm}, $\supp \dif v$ is a geodesic lamination.
\end{proof}

It is not clear that $\supp \dif v = \lambda_u$ in the above corollary, essentially because it is not clear that the map $\dif u \mapsto \dif v$ is injective (so $\dif v$ could be $0$ somewhere on $\lambda_u$).
We believe that this should be true, and state this as Conjecture \ref{two laminations agree}.

Daskalopoulos--Uhlenbeck also ask for a partial converse to the fact that $\dif v$ endows $\lambda_u$ with the structure of an oriented measured lamination.
We now prove this, resolving \cite[Problem 9.7]{daskalopoulos2020transverse}.
For the definition of the Ruelle-Sullivan $1$-current of an oriented, transversely measured geodesic lamination, see \cite[\S8]{daskalopoulos2020transverse} or the original paper of Ruelle--Sullivan \cite{Ruelle75}.

\begin{proposition}\label{minimal bounding implies least gradient}
Let $u \in BV_\loc(M)$, where $M$ is a hyperbolic manifold, and suppose that the level sets $\partial \{u > y\}$ define a minimal lamination.
Then $u$ is a function of least gradient.
\end{proposition}
\begin{proof}
Fix $U \Subset M$ and let $v$ be a competitor in $U$, thus $v \in BV(U)$ and $u - v$ is trace-free.
Then for $T: BV(U) \to L^1(U)$ the trace map, $\{Tu > y\} = \{Tv > y\}$ by (\ref{convergence of trace}).
It follows that $\partial^* \{v > y\}$ is a competitor to $\partial \{u > y\}$ in $U$ and hence, since $\partial \{u > y\}$ is minimal,
$$|\partial^* \{u > y\} \cap U| \leq |\partial^* \{v > y\} \cap U|.$$
That $u$ has least gradient now follows from the coarea formula, Proposition \ref{coarea formula}.
\end{proof}

\begin{corollary}\label{ruelle sullivan antiderivative}
Let $\lambda$ be an oriented, transversely measured geodesic lamination on a closed hyperbolic surface $M$, and let $\dif v$ be the Ruelle-Sullivan $1$-current induced by $\lambda$.
Then $\dif v$ is the derivative of a section of least gradient $v: M \to E$, for some affine bundle $E \to M$.
\end{corollary}
\begin{proof}
As observed in \cite[\S9]{daskalopoulos2020transverse}, if we lift $\dif v$ to a $1$-current $\dif \tilde v$ on $\Hyp^2$, then $\dif \tilde v$ is exact and any antiderivative $\tilde v$ of $\dif \tilde v$ has superlevel sets $\{\tilde v \geq y\}$ which are bounded by geodesics.
Moreover we can choose $\tilde v$ to be $\pi_1(M)$-equivariant.
The claim now follows from Proposition \ref{minimal bounding implies least gradient} and the realization of $\pi_1(M)$-equivariant functions on $\Hyp^2$ as sections of an affine bundle, c.f. \cite[\S4]{daskalopoulos2020transverse}.
\end{proof}

%%%%%%%%%%%%%%%%%%%%%%%%%%%%%%%%%%%%%%%%%%%%%%%

\subsection{Computational geometry}
Recent work by Loisel \cite{Loisel20} shows that the barrier method minimizes the $p$-Dirichlet energy $||\nabla u||_{L^p(\Omega)}$
subject to boundary data on euclidean space, with $O(n^{1/2} \log n)$ time complexity uniformly in $p \in [1, \infty]$, where $n$ is the cardinality of the given triangulation $T$ of $\Omega$.
One can adapt this method to the hyperbolic setting by quadrature... TODO how to compute $|\nabla u|_{L^p}$ in this setting??
Applying this method with $p = 1$, we can construct functions of least gradient.

We shall now show how Theorem \ref{main thm} associates a minimal lamination $\lambda_\alpha$ to each cohomology class $\alpha \in H^1(M, \RR)$, and how we can use Loisel's algorithm to compute $\lambda_\alpha$.
Let $\Gamma = \pi_1(M)$, so $M = \Hyp^d/\Gamma$, and fix a fundamental polytope $\Omega$ of $\Gamma$.
By the Hurcewiz theorem, $(\RR, \alpha)$ is a representation of $\Gamma$.
We consider the space $E$ of $\alpha$-equivariant functions $f: \partial \Omega \to \RR$, thus
\begin{equation}\label{boundary data for Loisel}
f(\gamma x) = f(x) + \alpha(\gamma)
\end{equation}
for every $\gamma \in \Gamma$, which are constant on each face.
The relation (\ref{boundary data for Loisel}) is an undetermined boundary condition for functions in $E$, and so we consider the subspace $E'$ of functions which in addition are zero on a maximal set of faces such that we do not determine $f|\partial \Omega = 0$, thus any function $E' \subseteq E$ has a completely determined trace $t$ on $\partial \Omega$.
By Corollary \ref{compactness}, there exists a function $u \in E'$ which has least gradient on $\Omega$.
By Theorem \ref{main thm}, we obtain a minimal lamination of $\Omega$ and hence of $M$.

Following \cite[\S4]{Loisel20}, we extend $t$ to $\Omega$ and approximate it by piecewise-linear elements on $\Omega$.
One can then use Loisel's algorithm to minimize $||\nabla v + \nabla t||_{L^1(\Omega)}$ in $W^{1, 1}_0(\Omega)$ and put $u = v - t$ to obtain a minimal lamination of $\Omega$.

TODO: Do some numerical experiments, show what minimal laminations in a fundamental polytope in $\Hyp^3$ look like
