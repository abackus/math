\section{Construction of laminations} \label{proof of main thm}

\subsection{The least-gradient case}
Our next task is to prove Theorem \ref{main thm} and Theorem \ref{main crly}.
We first can pass to the universal cover $\Hyp^d$.
By Corollary \ref{level sets are minimal}, for every function $u$ of least gradient, the superlevel sets of $u$ have least perimeter.
Let
\begin{equation}\label{lamination union}
A = \bigcup_y \partial \{u > y\},
\end{equation} $B$ the interior of $\{du = 0\}$, and $x \in M$.
Then $x \in B$ iff $u = u(x)$ near $x$, but that happens iff for every $y < u(x)$, $x$ is interior to $\{u > y\}$ and for every $y \geq u(x)$, $x$ is exterior to $\{u > y\}$.
This happens iff for every $y \in \RR$, $x$ is either interior or exterior to $\{u > y\}$, thus $x \notin \partial \{u > y\}$, which happens iff $x \notin A$.
Thus $\{A, B\}$ is a partition of $M$, so $A$ is closed.
Moreover, the sets $\{u > y\}$ are totally ordered by $\subseteq$, so the sets $\partial \{u > y\}$ are disjoint.
They are also hypersurfaces which are as smooth as possible, by Theorem \ref{main lma}.
This proves Theorem \ref{main thm}.

Given Theorem \ref{main thm} and Proposition \ref{Monotonicity Formula}, the proof of Theorem \ref{main crly} is identical to that of \cite[Proposition 3.4]{górny2017planar}.

\subsection{The $\infty$-harmonic case}
Now we prove Theorem \ref{infinity harmonic laminations}.
Let $u$ be an $\infty$-harmonic function on a closed hyperbolic surface $M = \Hyp^2/\Gamma$.
Let $\Omega$ be a fundamental domain of $\Gamma$.
We lift $u$ to an $\infty$-harmonic function $\tilde u$ on $M$ such that $\tilde u \in W^{1, \infty}(\Omega)$; after rescaling we may assume that
\begin{equation}\label{normalized infinity harmonic}
    ||d\tilde u||_{L^\infty(\Omega)} = 1.
\end{equation}
We can then find $p$-harmonic functions $u_p$ on $\Hyp^2$ such that $u_p \to \tilde u$ in $W^{1, \infty}(\Omega)$; the dual $1$-forms
\begin{equation}\label{conjugate harmonic eqn}
    dv_q = |du_p|^{p - 2} * du_p
\end{equation}
defined for $1/p + 1/q = 1$, converge vaguely as $q \to 1$ to a closed $1$-form $dv$, whose antiderivative $v$ is the limit of the antiderivatives $v_q$ in $L^\infty(\Omega)$; see \cite[\S4]{daskalopoulos2020transverse}.
By \cite[\S6]{daskalopoulos2020transverse}, $v$ has least gradient and $\supp dv \subseteq \lambda_u$.
By Theorem \ref{main thm}, $\supp dv$ is a geodesic lamination in $\lambda_u$.

To obtain the converse inclusion, let $x \in \lambda_u$. TODO: I guess hyperbolicity implies that $v$ must not be continuous here, using (\ref{normalized infinity harmonic}) somewhow