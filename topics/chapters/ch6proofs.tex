\section{Proofs of main theorems}\label{proof of main thm}
\subsection{Regularity of minimal hypersurfaces}
We now prove Theorem \ref{main lma} by iterating Proposition \ref{de Giorgi lemma}.
Let $U$ be a set of least perimeter in $M$, $d \leq 7$, let $u = 1_U$, and let $\Psi = (\partial_0, \dots, \partial_{d - 1})$ be a Killing frame at $P \in \partial U$.
Let $\normal$ be the conormal $1$-form to $U$.
There exists $Q \in \partial^* U$ arbitrarily close to $P$, and since $d \leq 7$, Proposition \ref{blowup theorem} furnishes a tangent space to $\partial^* U$ at $Q$, which the tangent rescalings $u_t$ of $u$ tend to in total variation on sets with no singularities as $t \to 0$.
In particular for $t$ small, if $u_t = 1_{U_t}$, then for every $\rho > 0$,
\begin{equation}\label{tangent rescaling de Giorgi}
    \Lambda_{T_PM}(U_t, \rho) \leq \sigma (\rho/4)^{d - 1}
\end{equation}
and if $t$ is small enough then on $\{v: g(v, v) < 2t^{1/2}\}$ the exponential map is an approximate isometry in the sense that (\ref{tangent rescaling de Giorgi}) implies 
$$\Lambda_M (U, B(Q, \rho)) < \sigma (\rho/2)^{d - 1}.$$
Here $\Lambda_{T_PM}$ is taken with respect to $\Psi' = (\partial_0|P, \dots, \partial_{d - 1}|P)$ (which is an orthonormal basis of $T_PM$) and $\Lambda_M$ is taken with respect to $\Psi$.
Taking $Q$ so close to $P$ that $B(P, \rho/2) \subseteq B(Q, \rho)$, we conclude
\begin{equation}\label{base case de Giorgi}
    \Lambda_M (U, B(P, \rho/2)) < \sigma (\rho/2)^{d - 1}.
\end{equation}

Let $\normal^{(m)}(P) = \int_{B(P,2^{-m})} (du)_\Psi ~\vol / \int_{B(P,2^{-m})} |du| ~\vol$, which is continuous by continuity of measure.
Reasoning identically to the proof of \cite[(6.11)]{Miranda66}, we have 
$$|\normal^{(n)}(P) - \normal^{(n + m)}(P)|^2 \lesssim \frac{\Lambda(U, B(P, 2^{-n}))}{|\partial^* U \cap B_{2^{-(n + m)}}|}$$
for every $n$, where the constant is absolute.
If $n_0$ is large enough that $2^{n_0 + 1} < \rho$ then an induction with base case (\ref{base case de Giorgi}) and inductive case Proposition \ref{de Giorgi lemma} imply that if $\rho < R$,
$$|\normal^{(n)}(P) - \normal^{(n + m)}(P)|^2 \lesssim \frac{2^{(n_0-n)d}}{|\partial^* U \cap B_{2^{-(n + m)}}|}.$$
Since $P \in \partial U$, Proposition \ref{doubling dimension} implies that $|\partial^* U \cap B_{2^{-(n + m)}}| \gtrsim 2^{(n + m)(d - 1)}$
and so, if $m \leq n$,
$$|\normal^{(n)}(P) - \normal^{(n + m)}(P)|^2 \lesssim 2^{-n} R^d.$$
But $R$ can be chosen locally uniformly, so it follows that $(\normal^{(n)})$ is locally uniformly Cauchy.
By Proposition \ref{LebDiff}, it converges to $\normal$ almost everywhere, and so $\normal$ is continuous and hence by Proposition \ref{locality of Caccioppoli}, $\partial U$ is a $C^1$ minimal hypersurface.
Elliptic theory \cite{morrey2009multiple} now implies that $\partial U$ is as smooth as possible.

\subsection{Existence of laminations}
Our next task is to prove Theorem \ref{main thm}.
By Corollary \ref{level sets are minimal}, for every function $u$ of least gradient, the superlevel sets of $u$ have least perimeter.
Let
\begin{equation}\label{lamination union}
A = \bigcup_y \partial \{u > y\},
\end{equation} $B$ the interior of $\{du = 0\}$, and $x \in M$.
Then $x \in B$ iff $u = u(x)$ near $x$, but that happens iff for every $y < u(x)$, $x$ is interior to $\{u > y\}$ and for every $y \geq u(x)$, $x$ is exterior to $\{u > y\}$.
This happens iff for every $y \in \RR$, $x$ is either interior or exterior to $\{u > y\}$, thus $x \notin \partial \{u > y\}$, which happens iff $x \notin A$.
Thus $\{A, B\}$ is a partition of $M$, so $A$ is closed.
Moreover, the sets $\{u > y\}$ are totally ordered by $\subseteq$, so the sets $\partial \{u > y\}$ are disjoint.
They are also hypersurfaces which are as smooth as possible, by Theorem \ref{main lma}.

\subsection{Convex surfaces with boundary}
Given Theorem \ref{main thm} and Proposition \ref{Monotonicity Formula}, the proof of Theorem \ref{main crly} is identical to that of \cite[Proposition 3.4]{górny2017planar}.

% Suppose that $M = \Sigma \subset \overline \Sigma$, and that Theorem \ref{main crly} is false for $u$.
% That is, we cannot extend the geodesic lamination given by Theorem \ref{main thm} to a lamination of $\overline \Sigma$.
% Therefore there exist disjoint geodesics $\gamma_1$ and $\gamma_2$ which intersect on $\partial \Sigma$ and bound superlevel sets $\{u > y_i\}$ of $u$.

% Suppose that $\gamma_1$ and $\gamma_2$ intersect at $x_0$, and $\gamma_i$ passes through $x_i$ on the way to $x_0$, so that $x_0, x_1, x_2$ bound an open, nondegenerate geodesic triangle $\Delta \subset \overline \Sigma$. This makes sense, because $\overline \Sigma$ is convex.
% By Proposition \ref{Monotonicity Formula}, the proof of \cite[Remark 37.9]{simon1983GMT} shows that there exist only finitely many connected components of $A$ in $\Delta$.
% So, after replacing $\gamma_2$ with a geodesic closer to $\gamma_1$ as necessary, we may assume that either $A$ does not intersect $\Delta$, or $A$ contains $\Delta$.
% By replacing $A$ with its complement if necessary, we may assume that $A$ does not meet $\Delta$.

% However, $v = 1_{u^{-1}((y_1, y_2))}$ is a function of least gradient, and $v = 1$ on $\Delta$ but $v = 0$ on the opposite sides of $\gamma_i$.
% So if we replace $v$ with $w = v - 1_\Delta$, $w$ has the same trace as $v$, but since $\Delta$ is a nondegenerate triangle,
% $$\int_U |dw| ~\vol = |\partial(\{u > y\} \setminus \Delta) \cap U| < |\partial \{u > y\} \cap U| = \int_U |dv| ~\vol$$
% whenever $U$ is a precompact neighborhood of $\overline \Delta$ in $\overline \Sigma$.
% Therefore $v$ does not have least gradient, which is a contradiction.
