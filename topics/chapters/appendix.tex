\appendix \section{Bochner integration and coarea} \label{coarea section}
Let $F$ be a separable Fr\'echet space over $\CC$, and $(\Omega, P)$ a measure space.
We can then define the \dfn{Bochner integral} of a $P$-measurable function $f: \Omega \to F$, which we write as $\int_\Omega f ~dP \in F$.
See \cite{Rieffel70}, \cite{MO47721}, and \cite[Chapter V]{yosida2012functional}.
We recall a few facts that we will need:

\begin{theorem}[Pettis]
Let $f: \Omega \to F$ be any function.
Then $f$ is $P$-measurable iff for every $X \in F'$, $\omega \mapsto \langle f(\omega), X\rangle$ is $P$-measurable.
In this case,
$$\left\langle \int_\Omega f ~dP, X\right\rangle = \int_\Omega \langle f(\omega), X\rangle ~dP(\omega).$$
If $F = \CC^r$, then the Bochner integral is just the componentwise Lebesgue integral.
\end{theorem}

Throughout this section we consider the superlevel sets $E_y = \{u > y\}$ of a function $u \in BV_l(M)$, and the resulting $T'M$-valued Radon measures
$$\omega(y) = d1_{E_y} ~\vol.$$
Let $\mu = |du| ~\vol$.

We first observe that for every $X \in \mathcal D(M, TM)$,
$$\langle \omega(y), X\rangle = -\int_{E_y} \mathcal L_X\vol$$
is measurable in $y$, since $E_y$ is monotone in $y$.
So by Pettis' theorem, $\omega$ is measurable in $y$ with respect to the weak topology of measures.

\begin{lemma}[coarea formula for measures]\label{Coarea1}
One has
$$du ~\vol = \int_{-\infty}^\infty \omega(y) ~dy.$$
\end{lemma}
\begin{proof}
Fix $X \in C_c(M, TM)$. We may assume that $u \geq 0$, and we must show
\begin{equation}
\label{gradient is integral of fibers}
\int_M (du, X) ~\vol = \int_{-\infty}^\infty \langle \omega(y), X\rangle ~dy.
\end{equation}
Since $u \geq 0$,
\begin{align*}
\int_M (du, X) ~\vol &= -\int_M u~\mathcal L_X\vol = -\int_M \int_0^{u(x)} dy ~\mathcal L_X\vol\\
&= -\int_0^\infty \int_{E_y} ~\mathcal L_X\vol ~dy = \int_0^\infty \langle \omega(y), X\rangle ~dy.
\end{align*}
by Fubini's theorem.
If $y < 0$ then $1_{E_y} = 1$ so $\omega(y) = 0$, so we conclude (\ref{gradient is integral of fibers}).
\end{proof}

Let $p: L \to M$ be the trivial line bundle with its induced metric $h$, and let $\eta$ be the volume form induced by $h$.
If $W$ is a vector field on $L$, we will write $W_1$ for the projection of $W$ onto $M$ and $W_2$ for its projection onto $\CC$.
Then Cartan's magic formula implies that if $W_2$ is constant, then
\begin{equation}
\label{Lie derivative computation}
\mathcal L_W\eta = \mathcal L_W\vol \wedge dy.
\end{equation}

\begin{lemma}\label{coarea converse}
Suppose that $W \in \mathcal D(L, TL)$ depends on a parameter $n \in \NN$, such that $W_2 = 0$ and for every $y \in \RR$, $X = W_1(\cdot, y)$ is a maximizing sequence for $\langle \omega(y), X\rangle$ subject to $X \prec U$.
Then
$$\int_{-\infty}^\infty \langle \omega(y), W(y)\rangle ~dy \leq \mu(U).$$
\end{lemma}
\begin{proof}
Let
$$E = \{(x, y) \in L: x \in E_y\}$$
be the undergraph of $u$.
By Fubini's theorem and (\ref{Lie derivative computation}),
\begin{align*}
\int_{-\infty}^\infty \langle \omega(y), W(y)\rangle ~dy &= -\int_{-\infty}^\infty \int_{E_y} \mathcal L_W\vol ~dy = -\iint_E ~\mathcal L_W\eta = \int_M (d1_E, W) ~\vol.
\end{align*}

Let $(u_m)$ be a mollification of $u$, so that $u_m \to u$ in the weak topology of distributions.
Then if $\chi$ is a cutoff, $\langle u_m, \chi\rangle \to \langle u, \chi\rangle$; taking a sequence of $\chi$ which increase to the indicator function of a compact set $K$, we conclude that $u_m \to u$ in $L^1(K)$, and hence $u_m \to u$ in $L^1_l$.

Let $E^{(m)}$ be the undergraph of $u_m$, and $E^{(m)}_y = \{u_m > y\}$.
Then for every test function $v$,
\begin{align*}
\langle 1_{E^{(m)}} - 1_E, v\rangle &= \int_{E^{(m)} \Delta E} v ~\vol \leq |(E^{(m)} \Delta E) \cap (\supp v \times \RR)| \cdot ||v||_{L^\infty}\\
&\leq ||v||_{L^\infty} \int_{\supp v} |u_m - u| ~\vol \to 0
\end{align*}
so $1_{E^{(m)}} \to 1_E$ in the weak topology of distributions.
Therefore
$$\lim_{m \to \infty} \int_M (d1_{E^{(m)}}, W) ~\vol = \int_M (d1_E, W) ~\vol.$$

Since $u_m$ is smooth, its graph $F_m = \partial E^{(m)}$ is a smooth manifold.
Let $\nu_m$ be the upwards unit normal field of $F_m$ and let $\vol_m$ be the volume form on $F_m$ induced by $\eta$.
Then
$$\langle d 1_{E^{(m)}}, W\rangle = -\iint_{E^{(m)}} \mathcal L_W\eta = -\int_{F_m} h(\nu_n, W) ~\vol_m.$$
Let $q_m = p|F_m$ and $Y_m$ be the vector field $(Y_m)_1 = -d u_m$, $(Y_m)_2 = 1$.
Since $F_m$ is a graph, $q_m: F_m \to M$ is a diffeomorphism, $(q_m)_*\nu_m = Y_m/|Y_m|$, and
$$(q_m)_* \vol_m = |Y_m| ~\vol.$$
Therefore
$$\int_{F_m} h(\nu_n, W) ~\vol_m = \int_M h(Y_m, W) ~\vol = \int_M g(d u_m, W_1) ~\vol = \int_M (d u_m, W_1) ~\vol.$$
Thus
\begin{align*}
|\langle d 1_E, W\rangle| &= \lim_{m \to \infty} |\langle d u_m, W_1\rangle| \leq \mu(U). \qedhere
\end{align*}
\end{proof}

\begin{proposition}[coarea formula for $BV_l$ functions]\label{Coarea2}
Let $u \in BV(M)$, let $E_y = \{u > y\}$, and let $\omega(y) = d1_{E_y} ~\vol$.
Then, if $\mu$ is the total variation of $du ~\vol$,
$$\mu = \int_{-\infty}^\infty |\omega(y)| ~dy.$$
\end{proposition}
\begin{proof}
By Lemma \ref{Coarea1} and the triangle inequality,
$$\mu \leq \int_{-\infty}^\infty |\omega(y)| ~dy.$$
So we just need to prove the converse.

Let $U \Subset M$.
Suppose that for every $y \in \RR$, $X = X^{(n)}_y$ is a maximizing sequence for $\langle \omega(y), X\rangle$ subject to $X \prec U$.
Since $\omega$ with respect to the weak topology of measures, for every $n$, $X^{(n)}_y(x)$ can be chosen to be measurable in $(x, y)$; indeed, we can take $X^{(n)}_y$ to be a smooth approximation to the Radon measure $d 1_{E_y}/|d 1_{E_y}|_{TV}$ in the weak topology of distributions, which is a product of the measurable functions $\omega$ and $y \mapsto 1/|\omega(y)|$.

By an approximation argument, we can find $W^{(n)} \in C_c(L, TL)$ such that $W^{(n)}_2 = 0$ and for every $y \in \RR$, $X = W^{(n)}_1(\cdot, y)$ is a maximizing sequence for $\langle d 1_{E_y}, X\rangle$ subject to $X \prec U$.
Let us now suppress the $n$ and write $W(y) = W^{(n)}(\cdot, y)$.

By Lemma \ref{coarea converse}, since $W$ has compact support, the integrand $\langle d 1_{E_y}, W(y)\rangle$ is uniformly bounded in $y$.
Therefore, by Fatou's lemma,
\begin{align*}
\int_{-\infty}^\infty |\omega(y)| ~dy &= \int_{-\infty}^\infty \lim_{n \to \infty} \langle \omega(y), W(y)\rangle ~dy \leq \liminf_{n \to \infty} \int_{-\infty}^\infty \langle \omega(y), W(y)\rangle ~dy \\
&\leq \mu(U). \qedhere
\end{align*}
\end{proof}

%%%%%%%%%%%%%%%%%%%%%%%%%%%%%%%

\section{Higher cross-products} \label{crossproducts}
Suppose that we are given a $d$-dimensional oriented real Hilbert space $(\Hilb, g)$, and an oriented basis $(\partial_0, \dots, \partial_{d - 1})$ which may not be orthonormal but which satisfies for every $i$
\begin{equation}\label{0th coordinate orthogonal}
g_{0i} = 0
\end{equation}
where $g_{\mu\nu} = g(\partial_\mu, \partial_\nu)$.

\begin{notation}[Einstein summation]\label{EinsteinNotation}
In what follows we use Einstein summation.
In particular, Latin indices range over $1, \dots, d - 1$ and index coordinate directions on $H$ and Greek indices range over $0, \dots, d - 1$.
\end{notation}

We write $g_{\hat 0 \hat 0}$ for the inner product on the span $\Hilb_0^\perp$ of $\partial_1, \dots, \partial_{d - 1}$ defined by $(g_{\hat 0 \hat 0})_{ij} = g_{ij}$.
We as usual write $g^{\mu\nu}$, $(g^{\hat 0 \hat 0})^{ij}$ for the components of the dual inner product, and $\delta_\mu^\nu$ for the components of the identity matrix.

\begin{definition}
The \dfn{Gramian matrix} of $v_1, \dots, v_{d - 1}$ is
$$\Gram(v_1, \dots, v_{d - 1})_{ij} = g_{\mu \nu} v_i^\mu v_j^\nu.$$
The \dfn{cross product} $v_1 \times \cdots \times v_{d - 1}$ of vectors $v_1, \dots, v_{d - 1} \in \Hilb$ is defined to be the vector $v_0$
such that:
\begin{enumerate}
\item $g(v_0, v_i) = 0$,
\item $((-1)^{d - 1} v_0, v_1, \dots, v_{d - 1})$ is positively oriented, and
\item the length is
$$|v_0|_g^2 = |\det \Gram(v_1, \dots, v_{d - 1})|.$$
\end{enumerate}
\end{definition}

Here the orientation convention is chosen so that if $\partial_0, \dots, \partial_{d - 1}$ is an orthonormal basis, thus $g_{\mu\nu} = \delta_{\mu\nu}$, then the cross product is computed by the formal determinant
\begin{equation}\label{formal determinant}
v_1 \times \cdots \times v_{d - 1} = \begin{vmatrix}\partial_0 & \cdots & \partial_{d - 1} \\
v_1^0 & \cdots & v_1^{d - 1}\\
& \vdots \\
v_{d - 1}^0 & \cdots & v_{d - 1}^{d - 1}\end{vmatrix}
\end{equation}
which of course agrees with the orientation convention of the cross product on $\RR^3$.

\begin{lemma} \label{cross product formula}
Suppose that $\phi_1, \dots, \phi_{d - 1} \in \Hilb$ satisfy
\begin{equation}\label{cross product formula hypothesis}
\phi_i^\mu = \delta_i^\mu + \delta_0^\mu \psi_i
\end{equation}
for some $\psi \in (\Hilb_0^\perp)'$.
Let $h_{ij} = (g_{00})^{-1} g_{ij}$, and let $\normal \in \Hilb'$ be the unit covector which annihilates $\phi_1, \dots, \phi_{d - 1}$ with $((-1)^{d - 1}\normal^\sharp, \phi_1, \dots, \phi_{d - 1})$ positively oriented.
Then
\begin{align}
|\det \Gram(\phi_1, \dots, \phi_{d - 1})| &= (g_{00})^{d - 1} (1 + |\psi|_{h^{-1}}^2) \det h, \label{WeinsteinAronszajn} \\
(\phi_1 \times \cdots \times \phi_{d - 1})^\mu &= (g_{00})^{d/2} g^{\mu \nu}(\delta_\nu^0 - \delta^i_\nu \psi_i) \sqrt{\det h}, \label{CrossProduct} \\
\normal_\mu &= \sqrt{\frac{g_{00}}{1 + |\psi|_{h^{-1}}^2}} (\delta^0_\mu - \delta^i_\mu \psi_i) \label{conormal crossproduct}.
\end{align}
\end{lemma}
\begin{proof}
We begin by computing
$$\Gram(\phi_1, \dots, \phi_{d - 1})_{ij} = g_{\mu \nu} \phi_i^\mu \phi_j^\nu = g_{00} \psi_i \psi_j + g_{ij}$$
which are the components of $g_{00}\psi \otimes \psi + g_{\hat 0 \hat 0} \in (\Hilb_0^\perp \otimes \Hilb_0^\perp)'$.
By the Weinstein-Aronszajn theorem \cite{Tao13}, $\det(1 + h^{-1}(\psi \otimes \psi)) = 1 + |\psi|_{h^{-1}}^2$, so
\begin{align*}
\det(g_{00}\psi \otimes \psi + g_{\hat 0 \hat 0})
&= (g_{00})^{d - 1} \det(\psi \otimes \psi + h) = (g_{00})^{d - 1} \det(h^{-1}(\psi \otimes \psi) + 1) \det h \\
&= (g_{00})^{d - 1} (1 + |\psi|_{h^{-1}}^2) \det h.
\end{align*}
Since $h$ is a quadratic form of signature $(+, \cdots, +)$, its determinant is positive and so (\ref{WeinsteinAronszajn}) holds.

We begin the proof of (\ref{CrossProduct}) by checking orthogonality:
\begin{align*}
g_{\mu\nu} g^{\mu \lambda} (\delta^0_\lambda - \delta^i_\lambda \psi_i)(\delta^\nu_0 \psi_i + \delta^\nu_i)
&= \delta^\lambda_\nu (\delta^0_\lambda - \delta^i_\lambda \psi_i)(\delta^\nu_0 \psi_i + \delta_i^\nu)
= \psi_i - \psi_i = 0.
\end{align*}
To check orientation we may assume that $g_{\mu\nu} = \delta_{\mu\nu}$ in which case we just need to check agreement with (\ref{formal determinant}):
$$\begin{vmatrix} \partial_0 && \cdots && \partial_{d - 1} \\
\psi_1 & 1 & 0 & \cdots & 0 \\
\psi_2 & 0 & 1 & \cdots & 0\\
&& \vdots \\
\psi_{d - 1} & 0 & \cdots & 0 & 1
\end{vmatrix} = \partial_0 - \sum_i \psi_i \partial_i.$$
To see that its length is $\det \Gram(\phi_1, \dots, \phi_{d - 1})$ we compute
\begin{align*}
g_{\mu \nu} g^{\mu \lambda}(\delta^0_\lambda - \delta^i_\lambda \psi_i) g^{\nu \kappa}(\delta_\kappa^0 - \delta_\kappa^j \psi_j)
&= (\delta_\mu^0 - \delta_\nu^i \psi_i)(g^{0 \nu} - g^{j \nu} \psi_j)\\
&= g^{00} - g^{0j} \psi_j - g^{0i} \psi_i + g^{ij} \psi_i \psi_j.
\end{align*}
Recalling (\ref{0th coordinate orthogonal}) we can rewrite this as
$$g_{\mu \nu} g^{\mu \lambda}(\delta^0_\lambda - \delta^i_\lambda \psi_i) g^{\nu \kappa}(\delta_\kappa^0 - \delta_\kappa^j \psi_j) = (g_{00})^{-1} + (g_{\hat 0 \hat 0})^{-1}(\psi \otimes \psi).$$
But $g_{00} (g_{\hat 0 \hat 0})^{-1} = h^{-1}$ so we deduce from (\ref{WeinsteinAronszajn})
$$g_{\mu \nu} g^{\mu \lambda}(\delta^0_\lambda - \delta^i_\lambda \psi_i) g^{\nu \kappa}(\delta_\kappa^0 - \delta_\kappa^j \psi_j) = (g_{00})^{-1} (1 + |\psi|_{h^{-1}}^2)
= \frac{|\det \Gram(\phi_1, \dots, \phi_{d - 1})|}{(g_{00})^d \det h}$$
which gives (\ref{CrossProduct}).
From (\ref{WeinsteinAronszajn}, \ref{CrossProduct}), (\ref{conormal crossproduct}) is immediate.
\end{proof}

\begin{example}\label{graphs in riemannian manifolds}
Let $M$ be an open submanifold of $\Hyp^d = (\RR_+)_x \times \RR^{d - 2}_y \times \RR_z$ with the hyperbolic metric (\ref{hyperbolic metric}).
Let $\omega$ be a $C^1$ function on $M \cap (\RR_+ \times \RR^{d - 2})$, and write
$$N = \{(x, y, z) \in M: z = \omega(x, y)\}$$
for its graph in $M$.
Let $\partial_1, \dots, \partial_{d - 1}$ denote the coordinate vector fields on $\RR_+ \times \RR^{d - 2}$, and $\partial_0$ the coordinate vector field on $\RR_z$.
Then, in the language of Notation \ref{hyperbolic line bundle}, with $\Psi = \Psi_N$, $\psi = d\omega$, $\Hilb = T_{(x, y, \omega(x, y))} M$, and $\phi_i = \Psi_* \partial_i$, we have (\ref{cross product formula hypothesis}).
In particular, $\phi_1, \dots, \phi_{d - 1}$ span $T_{(x, y, \omega(x, y))}N$.
We also have 
$$h_{ij} = x^2 g_{ij} = x^2 x^{-2} \delta_{ij} = \delta_{ij}.$$
So by Lemma \ref{cross product formula}, the conormal $\normal$ to $N$ at $(x, y, \omega(x, y))$ is given by 
$$(\Psi^* \normal)_\mu = \sqrt{\frac{x}{1 + |d\omega|^2}} (\delta_\mu^0 - \delta_\mu^i \omega_{,i})$$
where $|\cdot|$ is the euclidean length.
If we denote $\slashed g_{ij} = g(\phi_i, \phi_j)$, then
$$\slashed g = \Gram(\phi_1, \dots, \phi_{d - 1})$$
and so the volume form on $N$ is given by 
$$\Psi^* \vol_N = x^{1 - d} \sqrt{1 + |d\omega|^2} ~dx dy.$$
\end{example}