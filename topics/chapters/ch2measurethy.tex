\section{Monotonicity formula}\label{Monotonicity}
We now state a monotonicity formula for functions of least gradient on Riemannian manifolds.
This will allow us to explicitly bound the surface area of a set of least perimeter, and in turn obtain several estimates that will be used in the proof of Proposition \ref{mollifier quant}.

Throughout this section, we fix normal coordinates $(x^\mu)$ centered on a point $P \in M$.
We then define the closed $d-1$-form
\begin{equation}\label{d1 form}
\psi := \dif x^2 \wedge \dif x^3 \wedge \cdots \wedge \dif x^d.
\end{equation}
It has length
\begin{equation}\label{norm of d1 form}
|\psi| = \sqrt{g^{11}} = e^{O(|x|^2)}
\end{equation}
by \cite[Lemma 3.4]{schoen1994lectures}. We write $B_r := B(P, r)$.
We also note that the Taylor expansion of the volume form \cite{gray1974volume} gives for $0 < r_1, r_2 < R$ with $R$ small that there exists $A \in \RR$ such that
\begin{equation}\label{taylor expand the sphere form}
\frac{\vol_{\partial B_{r_2}}}{\vol_{\partial B_{r_1}}} = e^{A(r_2^2 - r_1^2)} \frac{r_2^{d - 1}}{r_1^{d - 1}}.
\end{equation}

The monotonicity formula that we shall need follows from the following fact which can be isolated from the proof of \cite[Lemma 5.8]{Giusti77}.

\begin{lemma}\label{monotonicity lemma}
Let $u \in C^1(B_R)$, $0 < r_1 < r_2 < R$, and let
$$E(r) = \int_{B_r} *|\dif u| - \eta(u, r),$$
so that $E(R) = 0$ iff $u$ has least gradient. Then there exists $A \in \RR$ such that for $R > 0$ small,
\begin{equation}\label{monotonicity lemma eqn}0 \leq \int_{B_{r_2} \setminus B_{r_1}} *r^{1 - d}\frac{(\partial_ru)^2}{|\dif u|} \lesssim \int_{r_1}^{r_2} \partial_r \left[e^{Ar^2} r^{1-d}\int_{B_r} *|\dif u|\right] + (d - 1) r^{-d} E(r) \dif r.\end{equation}
\end{lemma}
\begin{proof}
We fix $r^* \in [r_1, r_2]$ and introduce a competitor $v(r, \theta) = u(r^*, \theta)$.
Let $N = \partial B_{r^*}$, $U = B_{r^*}$.

From the definition of $\eta$,
\begin{equation}\label{consequence of least gradient monotone}
    \eta(u, r^*) \leq \int_U *|\dif v| = \int_0^{r^*} \int_{\partial B_r} *_r |\dif v| \dif r.
\end{equation}
From (\ref{taylor expand the sphere form}) we get the existence of $\tilde A \geq 0$ such that for every $r < r^*$,
$$\frac{\vol_{\partial B_r}}{\vol_{\partial B_{r^*}}} \leq e^{\tilde A (r^*)^2} \frac{r^{d - 1}}{(r^*)^{d - 1}},$$
and hence since $\partial_r v = 0$ we conclude
$$\int_{\partial B_r} *_r |\dif v| \leq e^{\tilde A(r^*)^2} \frac{\tilde r^{d - 1}}{(r^*)^{d - 1}} \int_N *_N |\dif v|.$$
Applying (\ref{consequence of least gradient monotone}) and Fubini's theorem,
\begin{align*}
\eta(u, r^*) &\leq  e^{\tilde A(r^*)^2} \int_0^{r^*} \frac{r^{d - 1}}{(r^*)^{d - 1}} \dif r \cdot \int_N *_N |\dif v| = \frac{r^* e^{\tilde A(r^*)^2}}{d} \int_N *_N |\dif v|\\
&\leq \frac{r^* e^{\tilde A(r^*)^2}}{d - 1} \int_N *_N |\dif v|.
\end{align*}
By Gauss' lemma, $\dif v$ is the orthogonal projection of $\dif u$ onto $T'N$, and its orthocomplement is $\partial_r u$. Therefore by Taylor's theorem,
$$\int_N *_N |\dif v| \leq \int_N *_N |\dif u| \sqrt{1 - \frac{(\partial_r u)^2}{|\dif u|^2}} \leq \int_N *_N \left[|\dif u| - \frac{(\partial_r u)^2}{2 |\dif u|}\right]$$
and hence from the definition of $E$,
$$\frac{(r^*)^{1 - d}}{d - 1} \int_N *_N \frac{(\partial_r u)^2}{2|\dif u|} \leq \frac{(r^*)^{1 - d}}{d - 1} \int_N *_N |\dif u| - \frac{e^{-\tilde A(r^*)^2}}{(r^*)^d}\left[\int_U *|\dif u| - E(r^*)\right].$$
We now select $A$ so that for every $r$ small,
$$-e^{-\tilde Ar^2} \leq -(1 - 2A(d - 1)r^2),$$
so that, independently of $r^*$,
$$\frac{(r^*)^{1 - d}}{d - 1} \int_N *_N |\dif u| - \frac{e^{-\tilde A(r^*)^2}}{(r^*)^d} \int_U *|\dif u| \leq e^{-A(r^*)^2} \partial_r\left(e^{Ar^2} r^{1 - d} \int_{B_r} *|\dif u|\right)|_{r = r^*}.$$

% By definition of $\partial_\Theta$,
% $$|\dif u|^2 = \left(\frac{1 - r^2}{2} \partial_r u\right)^2 + |\partial_\Theta u|^2,$$
% which can be rewritten using Taylor's theorem applied to $\sqrt\cdot$, and the fact that $|\partial_r| = 2/(1 - r^2)$, as
% $$|\partial_\Theta u| = |\dif u|\sqrt{1 - \frac{(1 - r^2)^2}{4|\dif u|} (\partial_r u)^2} \leq |\dif u| - \frac{(\partial_r u)^2}{32 |\dif u|}.$$
% Therefore by definition of $v$,
% \begin{align*}
% \int_N *_N |\dif v| = \int_N *_N |\partial_\Theta u| \leq \int_N *_N\left[|\dif u| - \frac{(\partial_ru)^2}{32 |\dif u|}\right].
% \end{align*}
% We combine this estimate with (\ref{consequence of least gradient monotone}) and (\ref{monotone fubini}) to deduce
% $$\int_N *_N \frac{(\partial_ru)^2}{|\dif u|} \lesssim \int_N *_N |\dif u| - \left[\int_0^{r^*} \frac{\tilde r^{d - 1}}{(r^*)^{d - 1}} \dif \tilde r\right]^{-1} \int_{\{r < r^*\}} *|\dif u|.$$
% Ingrating in $\dif \tilde r$ and applying the formula for differentiation of a moving region we deduce
% \begin{align*}
% \int_N *_N \frac{(\partial_ru)^2}{|\dif u|}
% &\lesssim \int_{\{r = r^*\}} *_{r^*} |\dif u| - \frac{d}{r^*} \int_{\{r < r^*\}} *|\dif u| \\
% &\leq \int_{\{r = r^*\}} *_{r^*} |\dif u| - \frac{d - 1}{r^*} \int_{\{r < r^*\}} *|\dif u| \\
% &\leq (r^*)^{d - 1} F'(r^*).
% \end{align*}
% This is the integral form of (\ref{monotonicity lemma eqn}).
\end{proof}

\begin{proposition}[monotonicity formula]\label{Monotonicity Formula}
Suppose that $g$ has negative Ricci curvature and $u$ is a function of least gradient in $B_R$ where $R$ is small. Then for $0 < r < R$,
$$\partial_r r^{1 - d} \int_{B_r} *|\dif u| \geq 0.$$
\end{proposition}
\begin{proof}
We may approximate $u$ by $C^1$ functions, and then apply Lemma \ref{monotonicity lemma}.
\end{proof}

We now give several consequences of the monotonicity formula that we will use in the sequel.
Analogous results can be found in various, sometimes unnumbered, inequalities in \cite[Chapter 5]{Giusti77}.

\begin{lemma}\label{least perimeter minimal size}
For a set $U$ of least perimeter, if $P \in \partial^* U$ and $d \leq 7$, one has
$$\lim_{r \to 0} r^{1 - d} |\partial^* U \cap B(P, r)| = \frac{|\Sph^{d - 2}|}{d - 1}.$$
\end{lemma}
\begin{proof}
Choose a sequence of $r \to 0$; then there is a subsequence along which the limit in Proposition \ref{blowup theorem} exists.
Thus for $r$ in the subsequence, $r^{1 - d} |\partial^* U \cap B(P, r)|$ is approximated as $r \to 0$ by $r^{1 - d}|\partial C \cap B'(0, r)|$ where $0 \in T_PM$, $C$ is a half-space in $T_PM$, and $B'$ denotes an euclidean ball in $T_PM$.
But $|\partial C \cap B'(0, r)|$ is the measure of the unit ball in $\RR^{d - 1}$, which is $|\Sph^{d - 2}|/(d - 1)$.
\end{proof}

\begin{proposition}[surface area estimate]\label{doubling dimension}
Let $U$ be a set of least perimeter in a ball $B_r$, and suppose that $d \leq 7$ and $g$ has negative Ricci curvature.
If the center of $B_r$ is contained in $\partial^* U$, then
$$\frac{|\Sph^{d - 2}|}{d - 1} r^{d - 1} \leq |\partial^*U \cap B_r| \leq |\Sph^{d - 1}|(1 + O(r^2))r^{d - 1}.$$
\end{proposition}
\begin{proof}
The upper bound on $|\partial^* U \cap B_r|$ is obtained by using (\ref{a priori estimate 2}) and the fact that the surface area of $\partial B_r$ is $|\Sph^{d - 1}|(1 + O(r^2))r^{d - 1}$ (see e.g. \cite{gray1974volume}).
The lower bound is obtained from Proposition \ref{Monotonicity Formula}, which implies that
$$\lim_{\rho \to 0} \rho^{1 - d} |\partial^* U \cap B_\rho| \leq |\partial^* U \cap B_r|.$$
The left-hand side is given by Lemma \ref{least perimeter minimal size}.
\end{proof}

\begin{lemma}\label{bounding the G}
Let $u$ be a function of least gradient in $B_R$ and $0 < r_1 < r_2 < R$. Then there exists $A > 0$ such that for $R$ small,
\begin{align*}
&\left|r_2^{1 - d} \int_{B_{r_2}} \dif u \wedge \psi - r_1^{1 - d} \int_{B_{r_1}} \dif u \wedge \psi \right| \\
&\qquad \lesssim A(r_2^2 - r_1^2)||u||_{L^\infty} \\
&\qquad + \sqrt{\left(1 + (d - 1) \log \frac{r_2}{r_1}\right) r_2^{1 - d} \int_{B_{r_2}} *|\dif u| \cdot \int_{r_1}^{r_2} \partial_r \left[r^{1 - d}\int_{B_r} *|\dif u|\right] \dif r}.
\end{align*}
As the curvature of $g$ tends to $0$ uniformly near $P$, the implied constant remains constant but $A \to 0$.
\end{lemma}
\begin{proof}
As in the arguments of \cite[Chapter 5]{Giusti77} we may assume $u \in C^1$.
Let $(x^\mu)$ be normal coordinates, thus
$$\int_{B_r} \dif u \wedge \psi = \int_{\partial B_r} *_r u(1 + O(r^2))(\normal_{\partial B_r}, \partial_{x^1}).$$
In polar coordinates $(\theta^i)$, $x^1 = r \cos \theta^1$ and hence
$$(\normal_{\partial B_r}, \partial_{x^1}) = (1 + O(r^2)) \cos \theta^1.$$
Writing $u_r(\theta) = u(r, \theta)$, we have
$$*_r u = u_r r^{d - 1}(1 + O(r^2)) \prod_{i=1}^{d - 1} \sin^{d - i - 1}(\theta^i) \dif \theta^i$$
which means that
$$r^{1 - d} \int_{B_r} \dif u \wedge \psi = e^{f(r^2)} \int_{\Sph^{d - 1}} u_r \cos \theta^1 \prod_{i=1}^{d - 1} \sin^{d - i - 1}(\theta^i) \dif \theta^i$$
for some family of functions $f = f(r)$ parametrized by $\theta$, with $f(0) = 0$ and such that $f'(0) \to 0$ as the curvature of $g$ tends to $0$.
We therefore have
$$
\left|r_2^{1 - d} \int_{B_{r_2}} \dif u \wedge \psi - r_1^{1 - d} \int_{B_{r_1}} \dif u \wedge \psi \right| \leq \int_{\Sph^{d - 1}} *_{\Sph^{d - 1}} |e^{f(r_2^2)} u_{r_2} - e^{f(r_1^2)} u_{r_1}|.$$
By \cite[Lemma 5.3]{Giusti77} this quantity is
\begin{align*}
\leq \int_{r_1}^{r_2} \int_{\Sph^{d - 1}} *_{\Sph^{d - 1}} |\partial_r e^{f(r^2)}u| \dif r &= \int_{B_{r_2} \setminus B_{r_1}} *(\det g)^{-1/2}r^{1 - d}|\partial_r (e^{f(r^2)}u)|\\
&\leq \int_{B_{r_2} \setminus B_{r_1}} *(\det g)^{-1/2}r^{1 - d}e^{f(r^2)})(|\partial_ru| + 2r|f'(r^2)u|)\\
&\leq 2\int_{B_{r_2} \setminus B_{r_1}} r^{1 - d}|\partial_ru| + 4\int_{B_{r_2} \setminus B_{r_1}} r^{2-d} |f'(r^2)u|\\
&=: 2I + 4J,
\end{align*}
where we assumed that $R$ was so small that $(\det g)^{-1/2} e^{f(r^2)} \leq 2$.

From the Cauchy-Schwarz inequality we can bound
$$I^2 \leq \left(\int_{B_{r_2} \setminus B_{r_1}} *r^{1 - d}|\dif u|\right) \cdot \left(\int_{B_{r_2} \setminus B_{r_1}} *r^{1 - d} \frac{(\partial_r u)^2}{|\dif u|}\right) =: I_1I_2.$$
Since $u$ has least gradient, a straightforward modification of \cite[Lemma 5.11]{Giusti77} gives
$$I_1 \leq \left(1 + (d - 1) \log \frac{r_2}{r_1}\right) r_2^{1 - d} \int_{B_{r_2}} *|\dif u|.$$
From Lemma \ref{monotonicity lemma},
$$I_2 \lesssim \int_{r_1}^{r_2} \partial_r \left[r^{1 - d}\int_{B_r} *|\dif u|\right] \dif r.$$
Finally we bound for $R$ small enough that $\sqrt{\det g} \leq 2$,
$$J \leq 2 ||f'||_{L^\infty} \cdot ||u||_{L^\infty} \int_{r_1}^{r_2} r^{d - 1}r^{2 - d} \dif r \lesssim ||u||_{L^\infty} (r_2^2 - r_1^2)$$
where the implied constant $\to 0$ as the curvature of $g$ tends to $0$.
\end{proof}

\begin{proposition}[monotonicity formula with $\psi$]\label{sharp monotonicity}
Let $U$ be a set of least perimeter, $u = 1_U$, and assume that $d \leq 7$ and $g$ has negative Ricci curvature.
If $P \in \partial^* U$ then
\begin{align*}
\left|r_2^{1 - d} \int_{B_{r_2}} \dif u \wedge \psi - r_1^{1 - d} \int_{B_{r_1}} \dif u \wedge \psi \right| \lesssim \left(3 + (d - 1) \log \frac{r_2}{r_1}\right) \sqrt{e^{O(r_2^2)}|\partial^* U \cap B_{r_2}| - |\partial^* U \cap B_{r_1}|}.
\end{align*}
\end{proposition}
\begin{proof}
Let us bound the right-hand side in Lemma \ref{bounding the G}.
We have from Proposition \ref{doubling dimension} that
$$\sqrt{\left(1 + (d - 1) \log \frac{r_2}{r_1}\right) r_2^{1 - d} \int_{B_{r_2}} *|\dif u|} \lesssim 2 + (d - 1) \log \frac{r_2}{r_1}.$$
Moreover by Proposition \ref{doubling dimension} and Proposition \ref{Monotonicity Formula},
\begin{align*}
A(r_2^2 - r_1^2) &\leq Ar_2^2 |\partial^* U \cap B_{r_2}| \\
&\leq e^{O(r_2^2)}|\partial^* U \cap B_{r_2}| - |\partial^* U \cap B_{r_1}| \\
&\lesssim \sqrt{e^{O(r_2^2)}|\partial^* U \cap B_{r_2}| - |\partial^* U \cap B_{r_1}|}. \qedhere
\end{align*}
\end{proof}

Finally we give an inequality that can be isolated from the proof of \cite[Theorem 7.3]{Giusti77} and that, while not very interesting by itself, is useful when applied with Proposition \ref{sharp monotonicity}.

\begin{proposition}\label{scalar curvature monotonicity}
Let $U$ be a set of least perimeter in a ball $B_R = B(P, R)$ with $P \in \partial^* U$ and $R$ small, $u = 1_U$, and suppose that $d \leq 7$.
Then for every $r_1, r_2 \in (0, R)$,
$$r_2^{1 - d}\int_{B_{r_2}} \dif u \wedge \psi \leq r_1^{1 - d}|\partial^*U \cap B_{r_1}| + O(R^2).$$
\end{proposition}
\begin{proof}
Since $\psi$ is a positive function times $\partial_{x^1}$, by Gauss' lemma the set $V = \{x \in \partial B_{r_2}: x^1 > 0\}$ is the set on which
$$g^{\mu\nu} (\normal_{B_{r_2}})_\mu (*\psi)_\nu > 0,$$
at least for $R$ small.
In particular $\psi$ has the same orientation as $\partial B_{r_2}$ on $V$. Combining this fact with $\dif \psi = 0$ we obtain
$$\int_{B_{r_2}} \dif u \wedge \psi = \int_{U \cap \partial B_{r_2}} \psi \leq \int_{U \cap V} \psi \leq \int_V \psi.$$
Moreover, $V$ is contractible for $R$ small, so $\psi$ is exact, thus we can choose a $d-2$-form $\varphi$ on a neighborhood of $V$ with $\dif \varphi = \psi$.
Moreover, viewed as a $d-1$-chain in $M$, the $d-2$-chain $\partial V$ is $\{x \in \partial B_{r_2}: x^1 = 0\}$, so if we set $D = \{x \in B_{r_2}: x^1 = 0\}$, then $\partial D = \partial V$ and so
$$\int_V \psi = \int_{\partial V} \varphi = \int_D \psi \leq |D| \cdot ||\psi||_{L^\infty}.$$
Estimating $|D|$ using \cite{gray1974volume} and using (\ref{norm of d1 form}), we obtain
$$r_2^{1 - d} \int_{B_{r_2}} \dif u \wedge \psi \leq \frac{|\Sph^{d - 2}|}{d - 1} + O(R^2)$$
and the claim now follows from Proposition \ref{doubling dimension}.
\end{proof}
