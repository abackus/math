
\section{Applications to best-Lipschitz/least-gradient duality}\label{BestDuality}

\subsection{etc}

In this section we ... \cite{thurston1998minimal} \cite{daskalopoulos2020transverse}... \ref{proof of main thm}

\begin{theorem} \label{small theorem}
Some conjecture is true.
\end{theorem}

% \subsection{Quasieuclidean frames}


% \begin{definition}
%     Let $P \in M$.
%     By a \dfn{quasieuclidean frame} (or $g$-quasieuclidean frame, if we must emphasize the metric) on $(M, P)$ we mean a tuple of smooth vector fields $(\partial_0, \dots, \partial_{d - 1})$ on $M$ such that:
%     \begin{enumerate}
%     \item (Euclidean at $P$) $(\partial_0, \dots, \partial_{d - 1})$ forms an orthonormal basis of $T_PM$.
%     \item (Commutator-free) $\mathcal L_\mu \partial_\nu = 0$.
%     \item (Divergence-free) $\mathcal L_\mu \vol = 0$.
%     \item (Almost orthonormal) If $\mu \neq \nu$ then $g_{\mu\nu} \leq 1/(4d)$, while $1/2 \leq g_{\mu\mu} \leq 2$.
%     \end{enumerate}
%     \end{definition}

% Before proving Proposition \ref{mollifier proposition}, we check that it is not vacuous in that one can always find a suitable coordinate system, possibly after rescaling the metric:

% \begin{lemma}\label{extend to quasieuclid}
% Quasieuclidean frames have the following properties:
% \begin{enumerate}
% \item For every $P \in M$ there exists a quasieuclidean frame near $P$, which can be taken to include any particular vector field $X$ which is nonzero near $P$ with $\mathcal L_X\vol = 0$.
% \item If there is a quasieuclidean frame on $(A, P)$ and $\tilde g(v, w) = g(\lambda v, \lambda w)$ is a rescaling of $g$, $\lambda > 0$, then there still exists a $\tilde g$-quasieuclidean frame on $(A, P)$.
% \item If $U$ is a set of $g$-least perimeter then there exists $\lambda > 0$, $\tilde g(v, w) = g(\lambda v, \lambda w)$, such that $U$ has $\tilde g$-least perimeter and there exists a $\tilde g$-quasieuclidean frame on $(B_{\tilde g}(P, 1), P)$.
% \item If $(\partial_0, \dots, \partial_{d - 1})$ is quasieuclidean, then it defines a basis of each tangent space.
% \end{enumerate}
% \end{lemma}
% \begin{proof}
% To construct a quasieuclidean frame we begin by choosing coordinates $x_0, \dots, x_{d - 1}$ for which
% $$\vol = dx_0 \wedge \cdots \wedge dx_{d - 1}.$$
% Then we set $\partial_\mu = \partial_{x_\mu}$ so
% $$\mathcal L_\mu\vol = \sum_\mu dx_0 \wedge \cdots \wedge d\frac{dx_\mu}{dx_\mu} \wedge \cdots \wedge dx_{d - 1} = 0.$$
% Clearly this construction can be done with $x_0$ chosen to be a primitive of a given divergence-free vector field.
% It is also clear that $\mathcal L_\mu \partial_\nu = 0$ since this is true in the euclidean case.
% We can then apply the Gram-Schmidt algorithm to modify this basis to obtain $g_{\mu\nu} = \delta_{\mu\nu}$ at $P$, and the vector fields produced by this algorithm are a linear combination of those input, and therefore must also be free of divergence and commutators.
% The almost orthonormality follows on a small open subset of $P$ by continuity.
% We can also repeat this construction, but with the Gram-Schmidt condition replaced by $g_{\mu\nu} = \lambda^{-2} \delta_{\mu\nu}$, to get a $\tilde g$-quasieuclidean frame.

% Now we can select $\lambda$ so small that we obtain a $\tilde g$-quasieuclidean frame on $(B_{\tilde g}(P, 1), P)$.
% Then if $u = 1_U$ has $g$-least gradient, $p = (d - 1)/2$, and $B$ is a small open set,
% $$\int_B |du| ~\vol_{\tilde g} = \lambda^p \int_B |du| ~\vol_g \leq \lambda^p \int_B |du + dv| ~\vol_g = \int_B |du + dv| ~\vol_{\tilde g}$$
% whenever $v \in BV_c(B)$.
% It follows that $u$ has $\tilde g$-least gradient.

% Finally, if $(\partial_0, \dots, \partial_{d - 1})$ is quasieuclidean, but we have
% $$0 = c_0 \partial_0, \dots, c_{d - 1} \partial_{d - 1},$$
% where there exists $i$ such that $c_i \neq 0$, we set $g_{ij} = g(\partial_i, \partial_j)$, and estimate
% \begin{align*}
% 0 &= \sum_{ij} c_ic_j g_{ij} \gtrsim \sum_i g_{ii} - \sum_{i \neq j} |g_{ij}| \geq \frac{d}{2} - \frac{d^2}{4d} = \frac{d}{4}
% \end{align*}
% which is a contradiction.
% \end{proof}