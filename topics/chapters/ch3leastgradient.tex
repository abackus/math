\section{Functions of least gradient}\label{LeastGradientFunctions}
\subsection{Riemannian measure theory}
Let us fix a Riemannian manifold $M$ of dimension $d$ and recall some basic measure-theoretic facts on $M$.
See \cite[Chapter 1]{Giusti77} for a review of the analogous facts over $\RR^d$, and see \cite{simon1983GMT} for the definition of an $\ell$-current.

We write $\int_U \omega \wedge \psi$ for the pairing of an $\ell$-current $\omega$ with a compactly supported $\ell$-form $\psi$ in an open set $U$.
We identify the distributional derivative of a function $u$ with the $d-1$-current
$$\int_U \dif u \wedge \psi = -\int_U u \wedge \dif \psi.$$
A function $u$ is in $BV(U)$ iff its derivative $\dif u$ has finite total variation
$$\int_U *|\dif u| := \sup_{\substack{||\psi||_{L^\infty} \leq 1\\\supp \psi \Subset V}} \int_U \dif u \wedge \psi.$$
Whether a current has locally finite total variation is independent of the Riemannian metric and so $BV_\loc(M)$ is also independently defined.

Now let $u \in BV_\loc(M)$.
Then by \cite[Theorem 4.14]{simon1983GMT}, there exists a $*|\dif u|$-measurable section $f$ of the cosphere bundle $S'M$ such that for every compactly supported $d-1$-form $\psi$,
\begin{equation}\label{RNy formula}
\int_M \dif u \wedge \psi = \int_M f|\dif u| \wedge \psi.
\end{equation}

For a vector field $X$, we write $*(Xu) := \dif u \wedge *(X^\flat)$.
The section $f$ of (\ref{RNy formula}) is given pointwise $*|\dif u|$-almost everywhere, in any local coordinates $(y_1, \dots, y_d)$, by
\begin{equation}\label{Lebesgue point definition}
    f(x) = \sum_{i = 1}^d \left[\lim_{r \to 0} \frac{\int_{B(x, r)} *\partial_{y_i} u}{\int_{B(x, r)} *|\dif u|}\right] ~\dif y^i,
\end{equation}
according to the Besicovitch differentiation theorem; here we view $(dy^i)$ as a basis of $T'_xM$.
However, whether the limit $f(x)$ in (\ref{Lebesgue point definition}) exists, or indeed its value as a point of $S'_xM$, do not depend on the Riemannian metric or the choice of coordinates.

\begin{definition}
The points $x$ for which the limit (\ref{Lebesgue point definition}) exists and satisfies $|f(x)| = 1$ are called the \dfn{Lebesgue points} of $\dif u$.
\end{definition}

The Lebesgue points are particularly important when $U$ is an indicator function:

\begin{definition}
Let $U$ be a set of locally finite perimeter, and let $u = 1_U$. Then:
\begin{enumerate}
\item The \dfn{measure-theoretic boundary} $\partial U$ is the set of points whose Lebesgue density with respect to $M$ is $\in (0, 1)$.
\item The set of Lebesgue points of $\dif u$ is the \dfn{reduced boundary} $\partial^* U$.
\item The $*|\dif u|$-measurable $1$-form $f$ defined by (\ref{Lebesgue point definition}) is the \dfn{conormal $1$-form} $\normal_U$ to $\partial U$.
\end{enumerate}
\end{definition}

Our definition of reduced boundary and conormal $1$-form follows \cite[Definition 3.3]{Giusti77} and is due to \cite{deGiorgi55}.
See \cite{Battista_2021} for an equivalent definition of reduced boundary on Riemannian manifolds, and see \cite[Chapter 6]{Pugh02} for the definition of Lebesgue density.

\begin{proposition}\label{locality of Caccioppoli}
    Let $U$ be a set of locally finite perimeter with conormal $1$-form $\normal$.
    Then:
    \begin{enumerate}
    \item $\partial^* U$ is either empty or $d-1$-dimensional in the Hausdorff sense, and is $d-1$-rectifiable.
    \item $\partial^* U$ is a dense subset of $\partial U$.
    \item If $\normal$ extends to a continuous $1$-form on $\partial U$, then $\partial^* U = \partial U$ is a $C^1$ hypersurface.
    \item If $\partial^* U = \partial U$ is a smooth hypersurface, then $\normal$ is the conormal $1$-form on $\partial U$ as defined in differential topology, and $*|\dif 1_U|$ is the induced volume form on $\partial U$.
\end{enumerate}
\end{proposition}
\begin{proof}
Most of the assertions of this proposition are diffeomorphism-invariant, so we may assume that $M = \RR^d$ and appeal to \cite[Chapters 2-4]{Giusti77}.
The proof that $*|\dif 1_U|$ is the induced volume form is identical to \cite[Example 1.4]{Giusti77}.
\end{proof}

\begin{definition}
Let $M$ be a Riemannian manifold, let $U$ be a set of locally finite perimeter, and let $E$ be a Borel set.
The \dfn{perimeter} of $U$ in $E$ is
$$|E \cap \partial^* U| := \int_E *|\dif 1_U|.$$
\end{definition}

\begin{proposition}[coarea formula]\label{Coarea2}
Let $M$ be a Riemannian manifold and $u \in BV_\loc(M)$. Then for every open set $E$,
\begin{equation}\label{coarea formula}
\int_E *|\dif u| = \int_{-\infty}^\infty |E \cap \partial \{u > y\}| \dif y.
\end{equation}
\end{proposition}
\begin{proof}
We follow \cite[Theorem 1.23]{Giusti77}, which first proves (\ref{coarea formula}) for $u \in C^\infty(\RR^d)$ using piecewise linear functions.
Such functions are not available for our purposes; instead we note that if $u \in C^\infty(\RR^d)$ and $u$ has no critical points then (\ref{coarea formula}) follows from Fubini's theorem, the fact that $|E \cap \partial \{u > y\}|$ is the surface area of $E \cap \{u = y\}$ (by Proposition \ref{locality of Caccioppoli}), and the change-of-variables formula.
However the left-hand side of (\ref{coarea formula}) is unaffected by critical points of $u$, and the right-hand side of (\ref{coarea formula}) is unaffected by critical values of $u$ by Sard's theorem.
So (\ref{coarea formula}) holds for $u \in C^\infty(\RR^d)$.

The rest of the proof is identical to \cite[Theorem 1.23]{Giusti77}, so we omit the details.
Taking a sequence in $C^\infty(M)$ that converges to $u$ in $L^1_\loc(M)$\footnote{Recall that $C^\infty(M)$ is not dense in $BV_\loc(M)$.}, and applying Fatou's lemma and the semicontinuity of total variation, we conclude the $\geq$ direction of (\ref{coarea formula}).
Moreover, Stokes' theorem gives that for every $d-1$-form $\psi$ such that $||\psi||_{L^\infty} \leq 1$ and $\supp \psi \Subset E$,
$$\int_E u \wedge \dif \psi = \int_{-\infty}^\infty \int_E |\psi| * |\dif 1_{\partial \{u > y\}}| \dif y \leq \int_{-\infty}^\infty |E \cap \partial \{u > y\}| \dif y.$$
Taking the supremum over $\psi$ we obtain the direction $\leq$ in (\ref{coarea formula}).
\end{proof}

%%%%%%%%%%%%%%%%%%%%
\subsection{Miranda's trace and stability theorems}
We now assert Miranda's trace theorem for $BV$ functions and stability theorem for least-gradient functions.
We also recall some a priori estimates for least-gradient functions.

\begin{proposition}[Miranda trace theorem]\label{traces}
Let $U \Subset M$ be an open set with Lipschitz boundary.
For every $u \in BV_\loc(U)$ there exists $v \in L^1_\loc(\partial U)$ such that for every $d-1$-form $\psi$,
\begin{equation}\label{Miranda IBP}
\int_U \dif u \wedge \psi + \int_U u \wedge \dif \psi = \int_{\partial U} v\psi.
\end{equation}
Moreover, for almost every $x \in \partial U$,
\begin{equation}\label{convergence of trace}
\int_{U \cap B(x, \varepsilon)} *|v(x) - u| \ll \varepsilon^d.
\end{equation}
\end{proposition}
\begin{proof}
The assertion (\ref{Miranda IBP}) is diffeomorphism-invariant and so follows from \cite[Teorema 1]{Miranda67}, and (\ref{convergence of trace}) also follows from that result if we are willing to drop a constant factor.
% We can find a smooth vector field $Y$ which points inwards along $N$, with flow maps $\Phi_t$ and a time interval $I \ni 0$, such that for every $t \in I$ and $y \in N$, $|Y(\Phi_t(y))| = 1$, and such that the integral curves $\{\Phi_t(y): t \in I\}$ with $y \in N$ cross $N$ exactly once.
% For the rest of the proof we follow \cite[Lemma 2.4]{Giusti77} and omit the details.
%
% Let $U_t$ be the open set bounded by $N_t = \Phi_t(N)$.
% Integrating by parts, it follows that for almost every $t \in I$ and every test field $X$,
% \begin{equation}\label{almost Miranda IBP}
% \int_{U_t} (du, X) \vol + \int_{U_t} u ~\mathcal L_X\vol = \int_{N_t} (\normal_{N_t}, uX) \vol_{N_t}.
% \end{equation}
% Writing $v_t = \Phi_t^* u$, we see from the mean value theorem applied along each integral curve of $Y$ that for almost every $s, t \in I$ with $s < t$ and every Borel set $A \subseteq N$,
% \begin{equation}\label{Cauchy trace}
% \int_A |v_t - v_s| \vol_N \leq \int_s^t \int_{\Phi_r(A)} |du| \vol_{\Phi_r(N)} ~dr.
% \end{equation}
% The right-hand side of (\ref{Cauchy trace}) tends to $0$ as $t \to 0$, so $(v_t)$ converges in $L^1_\loc(N)$ to a function $v$, which by (\ref{almost Miranda IBP}) must satisfy (\ref{Miranda IBP}).
%
% To obtain (\ref{convergence of trace}) we observe that for $\varepsilon$ small and almost every $x \in N$,
% \begin{align*}
% \int_{U \cap B(x, \varepsilon)} |v(x) - u| \vol &\leq \int_{N \cap B(x, 2\varepsilon)} \int_0^{2\varepsilon} |v(x) - \Phi_t^* u| ~dt \vol\\
% &\leq 2\varepsilon \int_{N \cap B(x, 2\varepsilon)} |v(x) - v| \vol + \int_{N \cap B(x, 2\varepsilon)} \int_0^{2\varepsilon} |v - \Phi_t^* u| ~dt \vol.
% \end{align*}
% The first term is $\ll \varepsilon^d$ for almost every $x$ by the Lebesgue differentiation theorem, and by applying (\ref{Cauchy trace}) and \cite[Lemma 2.3]{Giusti77} in flow-box coordinates with respect to $Y$m, one can show that the second term is also $\ll \varepsilon^d$.
\end{proof}

To state our a priori estimates we define
$$\eta(u, U) := \inf_{v \in BV_\cpt(U)} \int_U *|\dif(u + v)|$$
for $u \in BV_\loc(M)$ and $U \Subset M$, so that $u$ has least gradient iff $\eta(u, U) = \int_U *|\dif u|$ for every $U$.

Suppose that $u, v \in BV_\loc(M)$ and $U \Subset M$ is bounded by a Lipschitz hypersurface $N$. Armed with the trace theorem, it is straightforward to generalize \cite[Lemma 5.6]{Giusti77}, thus
\begin{equation}
|\eta(u, U) - \eta(v, U)| \leq ||u - v||_{L^1(N)}. \label{a priori estimate 1}
\end{equation}
In case $v = 0$, we note that by (\ref{convergence of trace}), the trace map is a contraction in $L^\infty$ norm, thus we have the a priori estimate
\begin{equation}
\eta(u, U) \leq ||u||_{L^1(N)} \leq |N| \cdot ||u||_{L^\infty(M)}. \label{a priori estimate 2}
\end{equation}

\begin{definition}
A sequence $(u_n)$ in $BV_\loc(M)$ has \dfn{approximately least gradient} if for every open $U \Subset M$,
$$\limsup_{n \to \infty} \int_U *|\dif u_n| \leq \liminf_{n \to \infty} \eta(u_n, U).$$
\end{definition}

\begin{proposition}[Miranda stability theorem]\label{Miranda convergence}
If a sequence of functions $(u_n)$ has approximately least gradient and $u_n \to u$ in $L^1_\loc(M)$, then $u$ has least gradient, and for every open set $U \Subset M$ with Lipschitz boundary such that $\int_{\partial U} *|\dif u| = 0$, one has
\begin{equation}\label{convergence in total variation}
\lim_{n \to \infty} \int_U *|\dif u_n| = \int_U *|\dif u|.
\end{equation}
\end{proposition}
\begin{proof}
The proof is similar to Teorema 3 and Osservazione 3 in \cite{Miranda67}; we just note the necessary modifications.
Suitable generalizations of Teorema 2 and Osservazione 2 follow from Proposition \ref{traces}.
One needs to add a term of size $o(1)$ to the right-hand side of the inequalities (2.8), (2.9), (2.13), and (2.14); however, in the limit, this term vanishes and so the conclusions (2.15) and (2.16) are unaffected.
\end{proof}

\begin{corollary}\label{compactness}
Every sequence $(u_n)$ of approximately least gradient converges in $L^1_\loc$ and almost everywhere along a subsequence to a function of least gradient $u$ such that for every open set $U \Subset M$ of Lipschitz boundary such that $\int_{\partial U} *|\dif u| = 0$, one has (\ref{convergence in total variation}).
\end{corollary}

\begin{proposition}\label{level sets are minimal}
For every function $u$ of least gradient, the superlevel sets $\{u > y\}$ have least perimeter.
If we instead have a sequence $(u_n)$ of approximately least gradient, then $(\{u_n > y\})$ has approximately least perimeter.
\end{proposition}
\begin{proof}
In the proof of \cite[Theorem 1]{BOMBIERI1969}, replace the coarea formula \cite[Theorem 1.6]{Miranda66} with Proposition \ref{Coarea2} and replace \cite[Teorema 3]{Miranda67} with Proposition \ref{Miranda convergence}.
\end{proof}

%%%%%%%%%%%%%%%%%%%%%%%%%%%%%%%%%%%%%%%%%%%%

\subsection{Blowup of the reduced boundary}
Now let us study the blowup of $M$ at a point $p$ on the reduced boundary of a set $U$ of least perimeter, giving a generalization of the conjunction of \cite[Theorem 9.3]{Giusti77} and \cite[Theorem 6.2.2]{Simons68}.

\begin{definition}
For a function $u$ on $M$, $P \in M$ we define the \dfn{tangent rescaling} of $u$ at $P$ to be the net of functions $u_t: T_PM \to \RR$, given by
$$u_t(v) = u\left(\exp_P(tv)\right)$$
\end{definition}

\begin{proposition}\label{blowup theorem}
Suppose that $U$ is an open set with least perimeter in $B(P, r)$, $P \in \partial^* U$, and $u = 1_U$.
Furthermore, suppose that $d \leq 7$.
Then the tangent rescaling of $u$ converges as $t \to 0$ along a subsequence (that we also denote $t \to 0$) in $L^1_\loc$ and almost everywhere, to the indicator function $v$ of a half-space $C \subset T_PM$ such that $0 \in \partial C$.
Moreover, for every open set $V \Subset T_PM$ of Lipschitz boundary such that $\int_{\partial V} *|\dif v| = 0$ we have the convergence of total variation
$$\lim_{t \to 0} \int_V *|\dif u_t| = \int_V *|\dif v|.$$
\end{proposition}
\begin{proof}
We claim that the tangent rescaling $(u_t)$ has approximately least gradient in every precompact open subset of $T_PM$ (where we give $T_PM$ its euclidean metric). If this true, then by Corollary \ref{compactness}, there exists a set $C$ of least perimeter in $T_PM$, such that the tangent rescaling converges to $v := 1_C$ in the desired sense.
But $T_PM$ is isometric to $\RR^d$, $d \leq 7$, so by the Bernstein--Fleming theorem \cite[Theorem 17.3]{Giusti77}\footnote{for an easier proof of the $d = 2$ case see \cite[\S5]{Fleming62}}, $\partial C$ is a hyperplane.
The fact that $0 \in \partial C$ follows from the fact that $P \in \partial^* U$.

To prove the claim, write $|\cdot|'$, $*'$ for the notions taken in the tangent space with its euclidean geometry, and write $U_t$ for the set indicated by $u_t$.
If $V$ is a precompact open subset of $T_PM$, $V_t = \{v \in T_PM: tv \in V\}$, then we have the scale-invariance
\begin{equation}\label{almost blowup scale invariance}
|\partial^* U_t \cap V|' = t^{1 - d}|\partial^* U_1 \cap V_{1/t}|'.
\end{equation}
From (\ref{almost blowup scale invariance}) and the Taylor expansion of $g$ in normal coordinates \cite[Lemma 3.4]{schoen1994lectures},
$$t^{d - 1} |\partial^* U_t \cap V|' = |\partial^* U_1 \cap V_{1/t}|' \leq (1 + O(t^2)) |\partial^* U \cap \exp_P(V_{1/t})|.$$
For every $w \in BV_\cpt(V)$, the least-gradient nature of $u$ gives
$$|\partial^* U \cap \exp_P(V_{1/t})| \leq \int_{V_{1/t}} *'|\dif(u + (\exp_P)_* w_{1/t})| \leq (1 + O(t^2))\int_{V_{1/t}} *'|\dif(u_1 + w_{1/t})|.$$
Therefore, after applying (\ref{almost blowup scale invariance}) and the Taylor expansion again,
$$|\partial^* U_t \cap V|' \leq (t^{1 - d} + O(t^{3 - d})) \int_{V_{1/t}} *' |\dif (u_1 + w_{1/t})| = (1 + O(t^2)) \int_V *' |\dif (u_t + w)|.$$
Since $V,w$ were arbitrary, we conclude that $(u_t)$ has approximately least gradient.
\end{proof}

%%%%%%%%%%%%%%%%%%%%%%%%%%%%%%%%%%%%%%%%%%%%%%%%%%

\subsection{Hyperbolic coordinates}
Before beginning the regularity proof, we describe the coordinate system on $\Hyp^d$ that we shall use in the remainder of this paper. We first inroduce the coordinates
$$(x, y, z) \in \RR_+ \times \RR^{d - 2} \times \RR = \Hyp^d$$
with the Poincar\'e half-space metric
\begin{equation}\label{hyperbolic metric}
g = \frac{\dif x^2 + \dif y_1^2 + \cdots + \dif y_{d - 2}^2 + \dif z^2}{x^2}.
\end{equation}
This does not agree with the usual convention for the Poincar\'e half-space, but will be advantageous when we represent $C^1$ hypersurfaces as graphs of functions $\omega: \RR_+ \times \RR^{d - 2} \to \RR$; then such a hypersurface can be written $\{z = \omega(x, y)\}$.
The variable $y$ plays no meaningful r\^ole in the proof of Theorem \ref{main lma} and the reader can specialize to the case $d = 2$ without oversimplifying the proof.
We also define the \dfn{hyperbolic origin} $O = (1, 0, 0)$.
We record that, for $B_r = B(O, r)$,
\begin{equation}\label{sup in a ball}
\sup_{(x, y, z) \in B_r} x = e^r.
\end{equation}

% The above coordinate system is useful for studying $C^1$ hypersurfaces, as we already noted, but for reduction to the $C^1$ case, most of our estimates will be in polar coordinates using the Poincar\'e ball metric
% \begin{equation}\label{radial metric}
% g = 4\frac{\dif r^2 + r^2 \sum_{i=1}^{d-1} (\prod_{j=1}^{i-1} \sin^2 \theta_j) \dif\theta_i^2}{(1 - r^2)^2},
% \end{equation}
% (where the empty product is $1$)
% for
% $$(r, \theta) \in (0, 1) \times \Sph^{d - 1} = \Hyp^d \setminus \{O\}$$
% and where the hyperbolic origin $O$ satisfies $r = 0$.
% As in the half-space case, the variable $\theta$ has only a minor r\^ole.
% The metric (\ref{radial metric}) is the rescaling of the euclidean metric in polar coordinates by the conformal factor $4(1 - r^2)^{-2}$ and so
% \begin{equation}\label{volume in polar coordinates}
% \mathrm{vol} = 2^d \frac{r^{d - 1}}{(1 - r^2)^d} \dif r \cdot \prod_{i=1}^{d - 1} \sin^{d - (1 + i)} \theta_i \dif \theta_i^2.
% \end{equation}
% Therefore, if
% $$\Phi_\rho: \Sph^{d - 1} \to \Hyp^d$$
% is the closed embedding which maps $\Sph^{d - 1}$ to $\{r = \rho\}$, then we have
% \begin{equation}\label{volume of a sphere in polar coordinates}
% \frac{\Phi_\rho^* \mathrm{vol}}{\mathrm{vol}_{\Sph^{d - 1}}} = 2^d \frac{\rho^{d - 1}}{(1 - \rho^2)^d}.
% \end{equation}
% We denote by $*_\rho$ the Hodge star on the hypersurface $\{r = \rho\}$.

%%%%%%%%%%%%%%%%%%%%%%%%%%%%%%%%%%%%%%%%%%%%%%%%%%%%%%%%

\subsection{Monotonicity formula and codimension}
We now state a monotonicity formula for functions of least gradient on Riemannian manifolds $(M, g)$, and as a consequence we show that any set of least perimeter is bounded by a set of codimension $1$ in a suitable measure-theoretic sense.

Due to the following lemma, it will be convenient to assume that $g$ has negative Ricci curvature; however, the results of this section can be modified to drop this assumption, as in \cite[Theorem 7.11]{MarquesXX}.

\begin{lemma}\label{divergence estimate}
Let $g$ have negative Ricci curvature, let $P \in M$, let $(x^\mu)$ be normal coordinates centered on $P$, and let $r$ be the distance to $P$. Then $X = r\grad r$ satisfies
$$\Div X \geq \partial_\mu X^\mu$$
in a neighborhood of $P$.
\end{lemma}
\begin{proof}
Let $(R_{\mu\nu})$ be the Ricci tensor at $P$; then one has the Taylor expansion \cite[Lemma 3.4]{schoen1994lectures}
$$\Div X = \partial_\mu X^\mu + r\partial_r(\log \sqrt{\det g}) = \partial_\mu X^\mu + r\partial_r\left(\frac{-R_{\mu\nu}}{6}x^\mu x^\nu + O(|x|^3)\right).$$
Let $\lambda > 0$ be the conorm of $(R_{\mu\nu})$. The second term can be estimated as
$$r\partial_r\left(\frac{-R_{\mu\nu}}{6}x^\mu x^\nu + O(|x|^3)\right) \geq \frac{\lambda}{4}r^2 \geq 0$$
for $r$ small enough.
\end{proof}

\begin{proposition}[monotonicity formula]\label{Monotonicity Formula}
Suppose that $g$ has negative Ricci curvature and $u$ is a function of least gradient in $B_\rho$ where $\rho$ is small. Then for $0 < r < \rho$,
$$\partial_r r^{1 - d} \int_{B_r} *|\dif u| \geq 0.$$
\end{proposition}
\begin{proof}
We shall use the arguments of Marques \cite[Theorem 7.11]{MarquesXX}.
Since $u$ has least gradient, $u$ is locally bounded and so for $\rho$ small we may assume that $||u||_{L^\infty} < M/2$ for some $M > 0$.
Let $(u_n)$ be a sequence in $C^\infty$ converging to $u$ in $L^1$ such that $\int_{B_\rho} *|\dif u_n| \to \int_{B_\rho} *|\dif u|$.
In particular $(u_n)$ has approximately least gradient.
From the coarea formula, Proposition \ref{coarea formula}, we have for $n$ large
$$\partial_r r^{1 - d} \int_{B_r} *|\dif u_n| = \int_{-M}^M \partial_r r^{1 - d} |\{u_n = y\} \cap B_r| \dif y.$$
Let $y$ be a regular value of $u_n$, so that $N := \{u_n = y\}$ is $C^\infty$.
Let $\Pi: TM \to TN$ be the orthogonal projection, let $X = r\grad r$, and let $\delta_r N$ be the first variation of the area of $N$ in the direction $(1 - \Pi)X$ in $B_r$, then by \cite[Lemma 2.4, pg12]{MarquesXX},
$$\Div_N X = \Div_N(\Pi X) - \frac{\delta_r N}{|N \cap B_r|}.$$
A computation in normal coordinates using Lemma \ref{divergence estimate} and \cite[Proposition 7.4]{MarquesXX} gives
$$\Div_N(\Pi X) \geq \frac{\delta_r N}{|N \cap B_r|} + d - 1.$$
From \cite[pg16]{MarquesXX} we have for
$$\varepsilon_n(y) := \inf_{r > 0} \frac{\delta_r \{u_n = y\}}{(d - 1)|\{u_n = y\} \cap B_r|}$$
that
$$\partial_r |\{u_n = y\} \cap B_r| \geq \frac{d - 1}{r} (1 + \varepsilon_n(y)) |\{u_n = y\} \cap B_r|$$
and hence
$$\partial_r r^{(1 - d)(1 + \varepsilon_n(y))} |\{u_n = y\} \cap B_r| \geq 0.$$
But $\{u_n = y\}$ has approximately least perimeter for almost every $y$ by Proposition \ref{level sets are minimal}, so $\varepsilon_n \to 0$ almost everywhere.
The claim now follows by taking $n \to \infty$.
\end{proof}

We now give an explicit estimate on the surface area of a minimal perimeter, following \cite[pg74]{Giusti77}.

\begin{proposition}\label{doubling dimension}
Let $U$ be a set of least perimeter in a ball $B_r$, and suppose that $d \leq 7$ and $g$ has negative Ricci curvature.
If the center of $B_r$ is contained in $\partial^* U$, then
$$\frac{|\Sph^{d - 2}|}{d - 1} r^{d - 1} \leq |\partial^*U \cap B_r| \leq |\Sph^{d - 1}|(1 + O(r^2))r^{d - 1}.$$
\end{proposition}
\begin{proof}
The upper bound on $|\partial^* U \cap B_r|$ is obtained by using (\ref{a priori estimate 2}) and the fact that the surface area of $\partial B_r$ is $|\Sph^{d - 1}|(1 + O(r^2))r^{d - 1}$ (see e.g. \cite{gray1974volume}).
The lower bound is obtained from Proposition \ref{Monotonicity Formula}, which implies that
$$A := \lim_{\rho \to 0} \rho^{1 - d} |\partial^* U \cap B_\rho| \leq |\partial^* U \cap B_r|.$$
The tangent rescaling $(u_t)$ of $1_U$ converges to $1_C$, with $C$ a half-space, by Proposition \ref{blowup theorem}.
We take $\rho \to 0$ along the (full-measure) subset of $\RR$ for which we have $\int_{B_\rho} *|\dif u_t| \to \int_{B_\rho} *|\dif 1_C|$,
and use the fact that since $C$ is a half-space, $\rho^{1-d} |\partial C \cap B_\rho|$ is the volume of the unit ball in $\RR^{d - 1}$, which is $|\Sph^{d - 2}|/(d - 1)$.
\end{proof}

We conclude this section with a few technical estimates that we will use in the proof of Proposition \ref{mollifier quant}, that follow from the above results.

\begin{lemma}\label{closed 1-form for normal}
Let $(x^\mu)$ be normal coordinates centered on $P \in M$.
Then there exists a closed $d-1$-form $\psi$ defined near $P$, which is a scalar field times $*\partial_{x^1}$, such that
\begin{equation}\label{norm of good d1 form}||\psi||_{L^\infty} = 1 + O(|x|^2).\end{equation}
\end{lemma}
\begin{proof}
Let $\psi$ be the $d-1$-form
$$\psi = *(\det g)^{-1/2} \partial_{x^1}$$
where $\det g$ is taken in the coordinates $(x^\mu)$. Then the norm is as desired by \cite[Lemma 3.4]{schoen1994lectures}, and
\begin{align*}
\dif \psi &= *\Div((\det g)^{-1/2} \partial_{x^1}) = *(\det g)^{-1/2} \partial_{x^1}((\det g)^0) = 0. \qedhere
\end{align*}
\end{proof}

\begin{lemma}\label{scalar curvature monotonicity}
Let $U$ be a set of least perimeter in a ball $B_R = B(P, R)$ with $P \in \partial^* U$ and $R$ small, suppose that $d \leq 7$ and $g$ has negative Ricci curvature, let $(x^\mu)$ be normal coordinates centered on $P$, and let $\psi$ be given by Lemma \ref{closed 1-form for normal}.
Then for every $r_1, r_2 \in (0, R)$,
$$r_2^{1 - d}\int_{B_{r_2}} \dif u \wedge \psi \leq r_1^{1 - d}|\partial^*U \cap B_{r_1}| + O(R^2).$$
\end{lemma}
\begin{proof}
By Lemma \ref{closed 1-form for normal}, $\psi$ is closed.
Since $\psi$ is a positive function times $\partial_{x^1}$, by Gauss' lemma the set $V = \{x \in \partial B_{r_2}: x^1 > 0\}$ is the set on which $*\psi > 0$, at least for $R$ small.
The nonnegativity of $u\psi$ on $V$ and the fact that $\psi$ is closed imply that
$$\int_{B_{r_2}} \dif u \wedge \psi = \int_{U \cap \partial B_{r_2}} \psi \leq \int_{U \cap V} \psi \leq \int_V \psi.$$
Moreover, $V$ is contractible for $R$ small, so $\psi$ is exact, thus we can choose a $d-2$-form $\varphi$ on a neighborhood of $V$ with $\dif \varphi = \psi$.
Moreover, viewed as a $d-1$-chain in $M$, the $d-2$-chain $\partial V$ is $\{x \in \partial B_{r_2}: x^1 = 0\}$, so if we set $D = \{x \in B_{r_2}: x^1 = 0\}$, then $\partial D = \partial V$ and so
$$\int_V \psi = \int_{\partial V} \varphi = \int_D \psi \leq |D| \cdot ||\psi||_{L^\infty}.$$
Estimating $|D|$ using \cite{gray1974volume} and using (\ref{norm of good d1 form}), we obtain
$$r_2^{1 - d} \int_{B_{r_2}} \dif u \wedge \psi \leq \frac{|\Sph^{d - 2}|}{d - 1} + O(R^2)$$
and the claim now follows from Proposition \ref{doubling dimension}.
\end{proof}

\begin{lemma}\label{bounding the G}
Let $U$ be a set of least perimeter in a ball $B_r$, suppose that $d \leq 7$ and $g$ has negative Ricci curvature, and let $X$ be a divergence-free vector field with $||X||_{L^\infty} \leq 1$.
If the center of $B_r$ is contained in $\partial^* U$, and $u = 1_U$, then for $r > 0$ small,
$$\partial_r r^{1 - d} \int_{B_r} *Xu \lesssim \partial_r r^{1 - d} |\partial^* U \cap B_r|.$$
\end{lemma}
\begin{proof}
TODO
\end{proof}


% We fix $0 < r_1 < r_2 < 1/2$ and denote $U := \{r_1 < r < r_2\}$.
% We also define, for $u$ a function of least gradient on $\Hyp^d$, the monotone quantity
% \begin{equation}\label{monotone quantity}
%     F(\rho) = \rho^{1 - d} \int_{\{r < \rho\}} *|\dif u|.
% \end{equation}
% We begin with the following analogue of \cite[pg68]{Giusti77}, which shows that in the $C^1$ case, $F$ really is monotone:

% \begin{lemma}\label{monotonicity lemma}
% Let $u \in C^1(\{r < 3/4\})$ be a function of least gradient. Then we have
% \begin{equation}\label{monotonicity lemma eqn}0 \leq \int_U *r^{1 - d}\frac{(\partial_ru)^2}{|\dif u|} \lesssim F(r_2) - F(r_1).\end{equation}
% \end{lemma}
% \begin{proof}
% We fix $r^* \in [r_1, r_2]$ and introduce a competitor $v(r, \Theta) = u(r^*, \Theta)$. We define $\partial_\Theta w(\rho, \Theta)$ for $w \in C^1$ to be the orthogonal projection of $\dif w(\rho, \Theta)$ onto $T'\{r = \rho\}$.
% Let $N = \partial B_{r^*}$.

% Since $u$ has least gradient,
% \begin{equation}\label{consequence of least gradient monotone}
%     \int_{\{r < r^*\}} *|\dif u| \leq \int_{\{r < r^*\}} *|\dif v| = \int_0^{r^*} \int_{\{r = r^*\}} *_{r^*} |\dif v| \dif \tilde r.
% \end{equation}
% Applying (\ref{volume in polar coordinates}) and the fact that $\partial_r v = 0$, we see that for any $\tilde r < r^*$,
% $$\int_{\{r = \tilde r\}} *_{\tilde r} |\dif v| \leq \frac{\tilde r^{d - 1}}{(r^*)^{d - 1}} \int_N *_N |\dif v|.$$
% So by Fubini's theorem,
% \begin{equation}\label{monotone fubini}
%     \int_0^{r^*} \int_{\{r = \tilde r\}} *_{\tilde r} |\dif v| \dif \tilde r \leq \int_0^{r^*} \frac{\tilde r^{d - 1}}{(r^*)^{d - 1}} \dif \tilde r \cdot \int_N *_N |\dif v|.
% \end{equation}
% By definition of $\partial_\Theta$,
% $$|\dif u|^2 = \left(\frac{1 - r^2}{2} \partial_r u\right)^2 + |\partial_\Theta u|^2,$$
% which can be rewritten using Taylor's theorem applied to $\sqrt\cdot$, and the fact that $|\partial_r| = 2/(1 - r^2)$, as
% $$|\partial_\Theta u| = |\dif u|\sqrt{1 - \frac{(1 - r^2)^2}{4|\dif u|} (\partial_r u)^2} \leq |\dif u| - \frac{(\partial_r u)^2}{32 |\dif u|}.$$
% Therefore by definition of $v$,
% \begin{align*}
% \int_N *_N |\dif v| = \int_N *_N |\partial_\Theta u| \leq \int_N *_N\left[|\dif u| - \frac{(\partial_ru)^2}{32 |\dif u|}\right].
% \end{align*}
% We combine this estimate with (\ref{consequence of least gradient monotone}) and (\ref{monotone fubini}) to deduce
% $$\int_N *_N \frac{(\partial_ru)^2}{|\dif u|} \lesssim \int_N *_N |\dif u| - \left[\int_0^{r^*} \frac{\tilde r^{d - 1}}{(r^*)^{d - 1}} \dif \tilde r\right]^{-1} \int_{\{r < r^*\}} *|\dif u|.$$
% Ingrating in $\dif \tilde r$ and applying the formula for differentiation of a moving region we deduce
% \begin{align*}
% \int_N *_N \frac{(\partial_ru)^2}{|\dif u|}
% &\lesssim \int_{\{r = r^*\}} *_{r^*} |\dif u| - \frac{d}{r^*} \int_{\{r < r^*\}} *|\dif u| \\
% &\leq \int_{\{r = r^*\}} *_{r^*} |\dif u| - \frac{d - 1}{r^*} \int_{\{r < r^*\}} *|\dif u| \\
% &\leq (r^*)^{d - 1} F'(r^*).
% \end{align*}
% This is the integral form of (\ref{monotonicity lemma eqn}).
% \end{proof}

% We now prove the monotonicity formula, following the exposition of \cite[pg71]{Giusti77}.

% \begin{proposition}[monotonicity formula]\label{Monotonicity Formula}
% For every function $u \in BV(\{u < 3/4\})$ of least gradient, $F$ is nondecreasing, and if we set
% $$G(\rho) = \rho^{1 - d} \int_{\{\rho < r\}} *x\partial_zu,$$
% and in addition assume $u \in L^\infty(\{u < 3/4\})$, then
% \begin{equation}\label{StrongMonotonicity}
% G(r_2) - G(r_1) \lesssim_{||u||_{L^\infty}} r_2 + \sqrt{\left(1 + (d - 1)\log \frac{r_2}{r_1}\right)(F(r_2) - F(r_1))}.
% \end{equation}
% \end{proposition}
% \begin{proof}
% By an approximation argument analogous to \cite[pg68]{Giusti77} we may assume that $u \in C^1(\{u < 3/4\})$.
% Then by Lemma \ref{monotonicity lemma}, $F$ is nondecreasing and it suffices to show that
% \begin{equation}\label{monotonicity wts}
% G(r_2) - G(r_1) \lesssim r_2 + \sqrt{\left(1 + (d - 1)\log \frac{r_2}{r_1}\right) \cdot \int_U *r^{1 - d}\frac{(\partial_ru)^2}{|\dif u|}}.
% \end{equation}
% Here and in the sequel all constants are allowed to depend on $||u||_{L^\infty}$.

% To estimate the left-hand side of (\ref{monotonicity wts}) we define
% $$\alpha = g\left(x \partial_z, \frac{\partial_r}{|\partial_r|}\right).$$
% Since $\partial_r/|\partial_r|$ is normal to every sphere $\{r < \rho\}$, and $x\partial_z$ is divergence-free, the divergence theorem implies that
% $$G(\rho) = 2^d \int_{\{r = \rho\}} *_\rho r^{1 - d} u\alpha (1 - r^2)^{-d}.$$
% Let $\dif \theta$ denote the usual measure on $\Sph^{d - 1}$, and let
% $$v = (1 - r^2)^{-d} \alpha u.$$
% Then the change of coordinates $\Phi_\rho$ with Jacobian (\ref{volume of a sphere in polar coordinates}) gives
% $$G(r_2) - G(r_1) = 2^d \int_{\Sph^{d - 1}} v(r_2, \theta) - v(r_1, \theta) \dif \theta = \int_{\Sph^{d - 1}} \int_{r_1}^{r_2} \partial_r v(r, \theta) \dif r \dif \theta.$$
% From the Leibniz rule,
% $$\partial_r v = (1 - r^2)^{-d} \alpha \partial_r u + u\partial_r(\alpha(1 - r^2)^{-d}).$$
% By the Cauchy-Schwarz inequality, $||\alpha||_{L^\infty} \leq 1$, so the second term is $O(||u||_{L^\infty})$. Thus for any fixed $\theta$,
% $$\int_{r_1}^{r_2} \partial_r v(r, \theta) \dif r = O(||u||_{L^\infty})(r_2 - r_1) + \int_{r_1}^{r_2} (1 - r^2)^{-d} \alpha \partial_r u(r, \theta) \dif r.$$
% Changing coordinates back we obtain
% $$G(r_2) - G(r_1) = O(r_2) + \int_U *r^{1 - d} \alpha \partial_r u.$$
% Since $||\alpha||_{L^\infty} \leq 1$, H\"older's inequality now gives\footnote{Compare and contrast \cite[Lemma 5.3]{Giusti77}, which is not available to us because $r^{1 - d} \partial_r$ is not divergence-free.}
% $$G(r_2) - G(r_1) \leq O(r_2) + \int_U *r^{1 - d} |\partial_r u|.$$
% By the Cauchy-Schwarz inequality,
% \begin{align*}
% \left(\int_U *r^{1 - d} |\partial_r u| \right)^2 &\leq \left(\int_U *r^{1 - d}\frac{(\partial_ru)^2}{|\dif u|}\right)\left(\int_U *r^{1 - d}|\dif u|\right).
% \end{align*}
% By \cite[Lemma 5.11]{Giusti77} (where we replace Giusti's (5.7) with the fact that $F' \geq 0$),
% $$\int_U *r^{1 - d}|\dif u| \leq \left[1 + (d - 1) \log \frac{r_2}{r_1}\right] r_2^{1 - d} \int_{\{r < r_2\}} *|\dif u|.$$
% Moreover, by the coarea formula, Proposition \ref{Coarea2},
% \begin{equation}\label{monotonicity coarea}
% \int_{\{r < r_2\}} *|\dif u| = \int_{-||u||_{L^\infty}}^{||u||_{L^\infty}} |\partial^* \{u > y\} \cap \{r < r_2\}| \dif y.
% \end{equation}
% By Corollary \ref{level sets are minimal}, the perimeters taken in (\ref{monotonicity coarea}) are minimal, thus by (\ref{a priori estimate 2}) are $\leq |\partial \{r < r_2\}| \lesssim r_2^{d - 1}$.
% This gives the desired estimate (\ref{monotonicity wts}).
% \end{proof}
