\section{Functions of least gradient}\label{LeastGradientFunctions}
\subsection{Riemannian measure theory}
Let us fix a Riemannian manifold $M$ and recall some basic measure-theoretic facts on $M$.
See \cite[Chapter 1]{Giusti77} for a review of the analogous facts over $\RR^d$.

We write $\int_U \omega \wedge \psi$ for the pairing of a $\ell$-current $\omega$ with a compactly supported $\ell$-form $\psi$.
We identify the distributional derivative of a function $u$ with the $d-1$-current
$$\int_U du \wedge \psi = -\int_U u \wedge d\psi.$$
A function $u$ is in $BV_\loc(M)$ iff its derivative $du$ has locally finite total variation
$$\int_U |du|\vol := \sup_{\substack{||\psi||_{L^\infty} \leq 1\\\supp \psi \Subset V}} \int_U du \wedge \psi,$$
thus $\int_U |\omega|\vol$ is finite for every $U \Subset M$.
Whether a current has locally finite total variation is independent of the Riemannian metric and so $BV_\loc$ is also independently defined.

Now let $u \in BV_\loc(M)$.
Then by \cite[Theorem 4.14]{simon1983GMT}, there exists a $(|du|\vol)$-measurable section $f$ of the cosphere bundle $S'M$ such that for every compactly supported $d-1$-form $\psi$,
\begin{equation}\label{RNy formula}
\int_M du \wedge \psi = \int_M *(f \wedge \psi) |du|\vol.
\end{equation}

For a vector field $X$, we write $Xu\vol := du \wedge *(X^\flat)$.
The section $f$ of (\ref{RNy formula}) is given pointwise $|du|$-almost everywhere, in any local coordinates $(y_1, \dots, y_d)$, by
\begin{equation}\label{Lebesgue point definition}
    f(x) = \lim_{r \to 0} \sum_i \frac{\int_{B(x, r)} \partial_{y_i} u\vol}{\int_{B(x, r)} |du|\vol} ~dy^i,
\end{equation}
according to the Besicovitch differentiation theorem; here we view $(dy^i)$ as a basis of $S'_xM$.
However, whether the limit $f(x)$ in (\ref{Lebesgue point definition}) exists, or indeed its value as a point of $S'_xM$, do not depend on the Riemannian metric.

\begin{definition}
The points $x$ for which the limit (\ref{Lebesgue point definition}) exists and satisfies $|f(x)| = 1$ are called the \dfn{Lebesgue points} of $du$.
\end{definition}

The Lebesgue points are particularly important when $U$ is an indicator function:

\begin{definition}
Let $U$ be a set of locally finite perimeter, and let $u = 1_U$. Then:
\begin{enumerate}
\item The \dfn{measure-theoretic boundary} $\partial U$ is the set of points whose Lebesgue density with respect to $M$ is $\in (0, 1)$.
\item The set of Lebesgue points of $du$ is the \dfn{reduced boundary} $\partial^* U$.
\item The $(|du|\vol)$-measurable $1$-form $f$ defined by (\ref{Lebesgue point definition}) is the \dfn{conormal $1$-form} $\normal_U$ to $\partial U$.
\end{enumerate}
\end{definition}

Our definition of reduced boundary and conormal $1$-form follows \cite[Definition 3.3]{Giusti77} and is due to \cite{deGiorgi55}.
See \cite{Battista_2021} for an equivalent definition of reduced boundary on Riemannian manifolds, and see \cite[Chapter 6]{Pugh02} for the definition of Lebesgue density.

\begin{proposition}\label{locality of Caccioppoli}
    Let $U$ be a set of locally finite perimeter with conormal $1$-form $\normal$.
    Then:
    \begin{enumerate}
    \item $\partial^* U$ is either empty or $d-1$-dimensional in the Hausdorff sense, and is $d-1$-rectifiable.
    \item $\partial^* U$ is a dense subset of $\partial U$.
    \item If $\normal$ extends to a continuous $1$-form on $\partial U$, then $\partial^* U = \partial U$ is a $C^1$ hypersurface.
    \item If $\partial^* U = \partial U$ is a smooth hypersurface, then $\normal$ is the conormal $1$-form on $\partial U$ as defined in differential topology, and $|d1_U|\vol$ is the induced volume form on $\partial U$.
\end{enumerate}
\end{proposition}
\begin{proof}
Most of the assertions of this proposition are diffeomorphism-invariant, so we may assume that $M = \RR^d$ and appeal to \cite[Chapters 2-4]{Giusti77}.
The proof that $|d1_U|\vol$ is the induced volume form is identical to \cite[Example 1.4]{Giusti77}.
\end{proof}

\begin{definition}
Let $M$ be a Riemannian manifold, let $U$ be a set of locally finite perimeter, and let $E$ be a Borel set.
The \dfn{perimeter} of $U$ in $E$ is
$$|E \cap \partial^* U| := \int_E |d1_U|\vol.$$
\end{definition}

\begin{proposition}[coarea formula]\label{Coarea2}
Let $M$ be a Riemannian manifold and $u \in BV_\loc(M)$. Then for every open set $E$,
\begin{equation}\label{coarea formula}
\int_E |du|\vol = \int_{-\infty}^\infty |E \cap \partial \{u > y\}| ~dy.
\end{equation}
\end{proposition}
\begin{proof}
We follow \cite[Theorem 1.23]{Giusti77}, which proves (\ref{coarea formula}) for $u \in C^\infty(\RR^d)$ using piecewise linear functions.
Such functions are not available for our purposes; instead we note that if $u \in C^\infty(\RR^d)$ and $u$ has no critical points then (\ref{coarea formula}) follows from Fubini's theorem, the fact that $|E \cap \partial \{u > y\}|$ is the surface area of $E \cap \{u = y\}$ (by Proposition \ref{locality of Caccioppoli}), and the change-of-variables formula.
However the left-hand side of (\ref{coarea formula}) is unaffected by critical points of $u$, and the right-hand side of (\ref{coarea formula}) is unaffected by critical values of $u$ by Sard's theorem.
So (\ref{coarea formula}) holds for $u \in C^\infty(\RR^d)$.

The rest of the proof is identical to \cite[Theorem 1.23]{Giusti77}, so we omit the details.
Taking a sequence in $C^\infty(M)$ that converges to $u \in BV_\loc(M)$, and applying Fatou's lemma and the semicontinuity of total variation, we conclude the $\geq$ direction of (\ref{coarea formula}).
Moreover, Stokes' theorem gives that for every $d-1$-form $\psi$ such that $||\psi||_{L^\infty} \leq 1$ and $\supp \psi \Subset E$,
$$\int_E u \wedge d\psi = \int_{-\infty}^\infty \int_E |\psi||d1_{\partial \{u > y\}}|*1 ~dy \leq \int_{-\infty}^\infty |E \cap \partial \{u > y\}| ~dy.$$
Taking the supremum over $\psi$ we obtain the direction $\leq$ in (\ref{coarea formula}).
\end{proof}

%%%%%%%%%%%%%%%%%%%%
\subsection{Miranda's trace and stability theorems}
We now assert Miranda's trace theorem for $BV$ functions and stability theorem for least-gradient functions.
We also recall some a priori estimates for least-gradient functions.

\begin{proposition}[Miranda trace theorem]\label{traces}
Let $U \Subset M$ be an open set with Lipschitz boundary.
For every $u \in BV_\loc(U)$ there exists $v \in L^1_\loc(\partial U)$ such that for every $d-1$-form $\psi$,
\begin{equation}\label{Miranda IBP}
\int_U du \wedge \psi + \int_U u \wedge d\psi = \int_{\partial U} v\psi.
\end{equation}
Moreover, for almost every $x \in \partial U$,
\begin{equation}\label{convergence of trace}
\int_{U \cap B(x, \varepsilon)} |v(x) - u| \vol \ll \varepsilon^d.
\end{equation}
\end{proposition}
\begin{proof}
The assertion (\ref{Miranda IBP}) is diffeomorphism-invariant and so follows from \cite[Teorema 1]{Miranda67}, and (\ref{convergence of trace}) also follows from that result if we are willing to drop a constant factor.
% We can find a smooth vector field $Y$ which points inwards along $N$, with flow maps $\Phi_t$ and a time interval $I \ni 0$, such that for every $t \in I$ and $y \in N$, $|Y(\Phi_t(y))| = 1$, and such that the integral curves $\{\Phi_t(y): t \in I\}$ with $y \in N$ cross $N$ exactly once.
% For the rest of the proof we follow \cite[Lemma 2.4]{Giusti77} and omit the details.
%
% Let $U_t$ be the open set bounded by $N_t = \Phi_t(N)$.
% Integrating by parts, it follows that for almost every $t \in I$ and every test field $X$,
% \begin{equation}\label{almost Miranda IBP}
% \int_{U_t} (du, X) \vol + \int_{U_t} u ~\mathcal L_X\vol = \int_{N_t} (\normal_{N_t}, uX) \vol_{N_t}.
% \end{equation}
% Writing $v_t = \Phi_t^* u$, we see from the mean value theorem applied along each integral curve of $Y$ that for almost every $s, t \in I$ with $s < t$ and every Borel set $A \subseteq N$,
% \begin{equation}\label{Cauchy trace}
% \int_A |v_t - v_s| \vol_N \leq \int_s^t \int_{\Phi_r(A)} |du| \vol_{\Phi_r(N)} ~dr.
% \end{equation}
% The right-hand side of (\ref{Cauchy trace}) tends to $0$ as $t \to 0$, so $(v_t)$ converges in $L^1_\loc(N)$ to a function $v$, which by (\ref{almost Miranda IBP}) must satisfy (\ref{Miranda IBP}).
%
% To obtain (\ref{convergence of trace}) we observe that for $\varepsilon$ small and almost every $x \in N$,
% \begin{align*}
% \int_{U \cap B(x, \varepsilon)} |v(x) - u| \vol &\leq \int_{N \cap B(x, 2\varepsilon)} \int_0^{2\varepsilon} |v(x) - \Phi_t^* u| ~dt \vol\\
% &\leq 2\varepsilon \int_{N \cap B(x, 2\varepsilon)} |v(x) - v| \vol + \int_{N \cap B(x, 2\varepsilon)} \int_0^{2\varepsilon} |v - \Phi_t^* u| ~dt \vol.
% \end{align*}
% The first term is $\ll \varepsilon^d$ for almost every $x$ by the Lebesgue differentiation theorem, and by applying (\ref{Cauchy trace}) and \cite[Lemma 2.3]{Giusti77} in flow-box coordinates with respect to $Y$m, one can show that the second term is also $\ll \varepsilon^d$.
\end{proof}

To state our a priori estimates we define
$$\eta(u, U) := \inf_{v \in BV_\cpt(U)} \int_U |d(u+v)| \vol$$
for $u \in BV_\loc(M)$ and $U \Subset M$, so that $u$ has least gradient iff $\eta(u, U) = \int_U |du| \vol$ for every $U$.

Suppose that $u, v \in BV_\loc(M)$ and $U \Subset M$ is bounded by a Lipschitz hypersurface $N$. Armed with the trace theorem, it is straightforward to generalize \cite[Lemma 5.6]{Giusti77}, thus
\begin{equation}
|\eta(u, U) - \eta(v, U)| \leq ||u - v||_{L^1(N)}. \label{a priori estimate 1}
\end{equation}
In case $v = 0$, we note that by (\ref{convergence of trace}), the trace map is a contraction in $L^\infty$ norm, thus we have the a priori estimate
\begin{equation}
\eta(u, U) \leq ||u||_{L^1(N)} \leq |N| \cdot ||u||_{L^\infty(M)}. \label{a priori estimate 2}
\end{equation}

\begin{definition}
A sequence $(u_n)$ in $BV_\loc(M)$ has \dfn{approximately least gradient} if for every open $U \Subset M$,
$$\limsup_{n \to \infty} \int_U |du_n| \vol \leq \liminf_{n \to \infty} \eta(u_n, U).$$
\end{definition}

\begin{proposition}[Miranda stability theorem]\label{Miranda convergence}
If a sequence of functions $(u_n)$ has approximately least gradient and $u_n \to u$ in $L^1_\loc(M)$, then $u$ has least gradient, and for every open set $U \Subset M$ with Lipschitz boundary such that $\int_{\partial U} |du|\vol = 0$, one has
\begin{equation}\label{convergence in total variation}
\lim_{n \to \infty} \int_U |du_n|\vol = \int_U |du| \vol.
\end{equation}
\end{proposition}
\begin{proof}
The proof is similar to Teorema 3 and Osservazione 3 in \cite{Miranda67}; we just note the necessary modifications.
Suitable generalizations of Teorema 2 and Osservazione 2 follow from Proposition \ref{traces}.
One needs to add a term of size $o(1)$ to the right-hand side of the inequalities (2.8), (2.9), (2.13), and (2.14); however, in the limit, this term vanishes and so the conclusions (2.15) and (2.16) are unaffected.
\end{proof}

\begin{corollary}\label{compactness}
Every sequence $(u_n)$ of approximately least gradient converges in $L^1_\loc$ and almost everywhere along a subsequence to a function of least gradient $u$ such that for every open set $U \Subset M$ of Lipschitz boundary such that $\int_{\partial U} |du|\vol = 0$, one has (\ref{convergence in total variation}).
\end{corollary}

\begin{proposition}\label{level sets are minimal}
For every function $u$ of least gradient, the superlevel sets $\{u > y\}$ have least perimeter.
\end{proposition}
\begin{proof}
In the proof of \cite[Theorem 1]{BOMBIERI1969}, replace the coarea formula \cite[Theorem 1.6]{Miranda66} with Proposition \ref{Coarea2} and replace \cite[Teorema 3]{Miranda67} with Proposition \ref{Miranda convergence}.
\end{proof}

%%%%%%%%%%%%%%%%%%%%%%%%%%%%%%%%%%%%%%%%%%%%%%%%%%%%%%%%%%%%%

\subsection{Dimension of reduced boundary}
Our next purpose is to show that the reduced boundary of a set of least perimeter has dimension $d - 1$ in the following sense.
For the proof we fill in a sketch of Mooney \cite[Theorem 12]{Mooney11}.

\begin{proposition}\label{doubling dimension}
Let $U$ be a set of least perimeter. Then for every $P \in \partial^* U$, $B_r = B(P, r)$ satisfies, locally uniformly in $P$,
$$|\partial^* U \cap B_r| \sim r^{d - 1}.$$
\end{proposition}
\begin{proof}
Let $\mu =|d1_U| \vol$. From (\ref{a priori estimate 2}) we have $\mu(B_r) \lesssim r^{d - 1}$.
Conversely, suppose that $P$ lies in the support $\partial^* U$ of $\mu$, let $\nu = |d1_U|\vol$, and observe that if we set $q = d/(d - 1)$, then by Sobolev embedding\footnote{Sobolev and Poincar\'e inequalities hold for $BV$ functions on $\RR^d$ \cite[\S5.6.1]{evans1991measure},
and since locally a Riemannian volume form is a perturbation of the euclidean volume form, the inequalities that we use here also hold.}
$$\nu(B_r)^{1/q} = ||1_{E \cap B_r}||_{L^q} \lesssim \int_M |d1_{E \cap B_r}| \vol = |\partial^*(U \cap B_r)|.$$
From the Leibniz inequality $\partial^*(U \cap B_r) \subseteq (\partial^* U \cap B_r) \cup (U \cap \partial B_r)$, we have
$$\nu(B_r)^{1/q} \leq |\partial^* U \cap B_r| + |U \cap \partial B_r| \leq 2|U \cap \partial B_r| = 2\frac{d}{dr} \nu(B_r).$$
We make the change of variables $u(r) = \nu(B_r)^{1/q}$, so that we obtain
$$u(r) \leq 2 \frac{d}{dr} u(r)^q = 2qu(r)^{q - 1}u'(r).$$
Moreover, $u(0) = 0$ and $u^{2 - q} \leq 2qu'$.
Since $q > 1$, a separation of variables yields
$$2qr \leq \int_0^{u(r)} \tilde u^{q - 2}~d\tilde u = \frac{u(r)^{q - 1}}{q - 1}.$$
Thus $\nu(B_r) \gtrsim r^{q/(q - 1)} = r^d$.
Since $\partial^* (M \setminus U) = \partial^* U$, if $\overline{\nu} = |d1_{M \setminus U}|\vol$, then we similarly have $\overline{\nu}(B_r) \gtrsim r^d$.

We now fix $r > 0$ and let $[u] = |B_r \cap U|/|B_r|$ be the mean of $u$ on $B_r$. Then by the Poincar\'e inequality,
\begin{align*}
\mu(B_r) &= \int_{B_r} |du|\vol \geq r^{-1} ||u - [u]||_{L^1(B_r)} \\
&= \frac{1}{r} \left[\int_{B_r \cap U} 1 -[u] \vol + \int_{B_r \setminus U} [u] \right] \\
&= \frac{|B_r||B_r \cap U| - |B_r \cap U|^2 + |B_r \cap U||B_r \setminus U|}{r|B_r|} \\
&= 2\frac{|B_r \cap U||B_r \setminus U|}{r|B_r|}\\
&\gtrsim r^{-d-1} \nu(B_r) \overline{\nu}(B_r) \gtrsim r^{2d-d-1} = r^{d - 1}. \qedhere
\end{align*}
\end{proof}

%%%%%%%%%%%%%%%%%%%%%%%%%%%%%%%%%%%%%%%%%%%%

\subsection{Blowup of the reduced boundary}
Now let us study the blowup of $M$ at a point $p$ on the reduced boundary of a set $U$ of least perimeter, giving a generalization of the conjunction of \cite[Theorem 9.3]{Giusti77} and \cite[Theorem 6.2.2]{Simons68}.

\begin{definition}
For a function $u$ on $M$, $P \in M$ we define the \dfn{tangent rescaling} of $u$ at $P$ to be the net of functions
$$u_t(v) = u\left(\exp_P(tv)\right)$$
on $T_PM$, as $t \to 0$.
\end{definition}

\begin{proposition}\label{blowup theorem}
Suppose that $U$ is an open set with least perimeter in $B(P, r)$, $P \in \partial^* U$, and $u = 1_U$.
Furthermore, suppose that $d \leq 7$.
Then the tangent rescaling of $u$ converges along a subsequence, in $L^1_\loc$ and almost everywhere, to the indicator function $u$ of a half-space $C \subset T_PM$ such that $0 \in \partial C$.
The convergence in addition satisfies (\ref{convergence in total variation}) for every open set $U \Subset T_PM$ of Lipschitz boundary such that $u$ has no jump discontinuity along $\partial U$.
\end{proposition}
\begin{proof}
We claim that the tangent rescaling $(u_t)$ has approximately least gradient in every precompact open subset of $T_PM$ (where we give $T_PM$ its euclidean metric). If this true, then by Corollary \ref{compactness}, there exists a set $C$ of least perimeter in $T_PM$, such that the tangent rescaling converges to $1_C$.
But $T_PM$ is isometric to $\RR^d$, $d \leq 7$, so by the Bernstein--Fleming theorem \cite[Theorem 17.3]{Giusti77} \cite[\S5]{Fleming62}, $\partial C$ is a hyperplane.
The fact that $0 \in \partial C$ follows from the fact that $P \in \partial^* U$.

To prove the claim, write $|\cdot|'$, $\vol'$ for the notions taken in the tangent space with its euclidean geometry, and write $U_t$ for the set indicated by $u_t$.
If $V$ is a precompact open subset of $T_PM$, $V_t = \{v \in T_PM: tv \in V\}$, then we have the scale-invariance
\begin{equation}\label{almost blowup scale invariance}
|\partial^* U_t \cap V|' = t^{1 - d}|\partial^* U_1 \cap V_{1/t}|'.
\end{equation}
From \cite[Lemma 3.4]{schoen1994lectures} and (\ref{almost blowup scale invariance}),
$$t^{d - 1} |\partial^* U_t \cap V|' = |\partial^* U_1 \cap V_{1/t}|' \leq (1 + O(t^2)) |\partial^* U \cap \exp_P(V_{1/t})|.$$
For every $w \in BV_\cpt(V)$, the least-gradient nature of $u$ gives
$$|\partial^* U \cap \exp_P(V_{1/t})| \leq \int_{V_{1/t}} |d(u + (\exp_P)_* w_{1/t})| \vol \leq (1 + O(t^2))\int_{V_{1/t}} |d(u_1 + w_{1/t})| \vol'.$$
Therefore, after applying (\ref{almost blowup scale invariance}) again,
$$|\partial^* U_t \cap V|' \leq (t^{1 - d} + O(t^{3 - d})) \int_{V_{1/t}} |d(u_1 + w_{1/t})| \vol' = (1 + O(t^2)) \int_V |d(u_t + w)| \vol'.$$
Since $V,w$ were arbitrary, we conclude that $(u_t)$ has approximately least gradient.
\end{proof}
