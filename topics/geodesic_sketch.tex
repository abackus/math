\documentclass[reqno,12pt,letterpaper]{amsart}
\RequirePackage{amsmath,amssymb,amsthm,graphicx,mathrsfs,url}
\RequirePackage[usenames,dvipsnames]{color}
\RequirePackage[colorlinks=true,linkcolor=Red,citecolor=Green]{hyperref}
\RequirePackage{amsxtra}
\usepackage{cancel}
\usepackage{tikz-cd}

\setlength{\textheight}{8.50in} \setlength{\oddsidemargin}{0.00in}
\setlength{\evensidemargin}{0.00in} \setlength{\textwidth}{6.08in}
\setlength{\topmargin}{0.00in} \setlength{\headheight}{0.18in}
\setlength{\marginparwidth}{1.0in}
\setlength{\abovedisplayskip}{0.2in}
\setlength{\belowdisplayskip}{0.2in}
\setlength{\parskip}{0.05in}
\renewcommand{\baselinestretch}{1.10}

\title[Geodesic laminations by minimal currents]{Geodesic laminations by minimal currents}
\author{Aidan Backus}
\date{July 2021}

\newcommand{\NN}{\mathbf{N}}
\newcommand{\ZZ}{\mathbf{Z}}
\newcommand{\QQ}{\mathbf{Q}}
\newcommand{\RR}{\mathbf{R}}
\newcommand{\CC}{\mathbf{C}}
\newcommand{\DD}{\mathbf{D}}
\newcommand{\PP}{\mathbf P}
\newcommand{\MM}{\mathbf M}
\newcommand{\II}{\mathbf I}
\newcommand{\Hyp}{\mathbf H}

\DeclareMathOperator{\card}{card}
\DeclareMathOperator{\cent}{center}
\DeclareMathOperator{\ch}{ch}
\DeclareMathOperator{\codim}{codim}
\DeclareMathOperator{\diag}{diag}
\DeclareMathOperator{\diam}{diam}
\DeclareMathOperator{\dom}{dom}
\DeclareMathOperator{\Gal}{Gal}
\DeclareMathOperator{\Hom}{Hom}
\DeclareMathOperator{\Jac}{Jac}
\DeclareMathOperator{\Lip}{Lip}
\DeclareMathOperator{\Met}{Met}
\DeclareMathOperator{\id}{id}
\DeclareMathOperator{\rad}{rad}
\DeclareMathOperator{\rank}{rank}
\DeclareMathOperator{\Radon}{Radon}
\DeclareMathOperator*{\Res}{Res}
\DeclareMathOperator{\sgn}{sgn}
\DeclareMathOperator{\singsupp}{sing~supp}
\DeclareMathOperator{\Spec}{Spec}
\DeclareMathOperator{\supp}{supp}
\DeclareMathOperator{\Tan}{Tan}
\newcommand{\tr}{\operatorname{tr}}

\newcommand{\Ric}{\mathrm{Ric}}
\newcommand{\Riem}{\mathrm{Riem}}

\newcommand{\dbar}{\overline \partial}

\DeclareMathOperator{\atanh}{atanh}
\DeclareMathOperator{\csch}{csch}
\DeclareMathOperator{\sech}{sech}

\DeclareMathOperator{\Div}{div}
\DeclareMathOperator{\grad}{grad}
\DeclareMathOperator{\Ell}{Ell}
\DeclareMathOperator{\WF}{WF}

\newcommand{\Hilb}{\mathcal H}
\newcommand{\normal}{\mathbf n}
\newcommand{\vol}{\mathrm{vol}}

\newcommand{\pic}{\vspace{30mm}}
\newcommand{\dfn}[1]{\emph{#1}\index{#1}}

\renewcommand{\Re}{\operatorname{Re}}
\renewcommand{\Im}{\operatorname{Im}}


\newtheorem{theorem}{Theorem}[section]
\newtheorem{badtheorem}[theorem]{``Theorem"}
\newtheorem{prop}[theorem]{Proposition}
\newtheorem{lemma}[theorem]{Lemma}
\newtheorem{claim}[theorem]{Claim}
\newtheorem{proposition}[theorem]{Proposition}
\newtheorem{corollary}[theorem]{Corollary}
\newtheorem{conjecture}[theorem]{Conjecture}
\newtheorem{axiom}[theorem]{Axiom}
\newtheorem{assumption}[theorem]{Assumption}

\theoremstyle{definition}
\newtheorem{definition}[theorem]{Definition}
\newtheorem{remark}[theorem]{Remark}
\newtheorem{example}[theorem]{Example}
\newtheorem{notation}[theorem]{Notation}

\newtheorem{exercise}[theorem]{Discussion topic}
\newtheorem{homework}[theorem]{Homework}
\newtheorem{problem}[theorem]{Problem}

\newtheorem{ack}{Acknowledgements}

\numberwithin{equation}{section}


% Mean
\def\Xint#1{\mathchoice
{\XXint\displaystyle\textstyle{#1}}%
{\XXint\textstyle\scriptstyle{#1}}%
{\XXint\scriptstyle\scriptscriptstyle{#1}}%
{\XXint\scriptscriptstyle\scriptscriptstyle{#1}}%
\!\int}
\def\XXint#1#2#3{{\setbox0=\hbox{$#1{#2#3}{\int}$ }
\vcenter{\hbox{$#2#3$ }}\kern-.6\wd0}}
\def\ddashint{\Xint=}
\def\dashint{\Xint-}

%\usepackage{color}
%\hypersetup{%
%    colorlinks=true, % make the links colored%
%    linkcolor=blue, % color TOC links in blue
%    urlcolor=red, % color URLs in red
%    linktoc=all % 'all' will create links for everything in the TOC
%Ning added hyperlinks to the table of contents 6/17/19
%}

% style=alphabetic
\usepackage[backend=bibtex,maxcitenames=50,maxnames=50]{biblatex}
\addbibresource{topics.bib}
\renewbibmacro{in:}{}
\DeclareFieldFormat{pages}{#1}

\begin{document}

\maketitle

%%%%%%%%%%%%%%%%%%%%%%%%%%%%%%%%%%%%%%%%%%%%%%%%%%%%%%%

\section{Measure theory}
Fix a smooth manifold $M$.

\begin{definition}
Let $F$ be a normed complex vector bundle of finite rank over $M$.
A \dfn{$F$-valued Radon measure} is a continuous linear functional on $C_c(M, F')$.
The \dfn{sphere bundle} $SF$ is $\{(x, v) \in F: |v| = 1\}$, which we also identify with $(F \setminus \{0\})/\RR^+$.
\end{definition}

\begin{lemma}[Riesz-Markov representation theorem]
Let $\omega$ be an $F$-vaulued Radon measure and $\mu$ the total variation of $\omega$.
Then there exists a $\mu$-measurable section $f$ of $SF$ such that for every $X \in C_c(M, F')$,
$$\langle \omega, X\rangle = \int_M (f, X) ~d\mu.$$
Furthermore, $f$ does not depend on the norm of $F$, and is defined $\mu$-almost everywhere.
\end{lemma}
\begin{proof}
Just use classical Riesz-Markov and check the transition functions.
\end{proof}

\begin{lemma}[Lebesgue differentiation theorem]
Let $\mu$ be a Radon measure on $M$, $f \in L^1_{loc}(M, SF, \mu)$, and
$$f(x) = \lim_{(g, B, \varphi)} \varphi^{-1}\left(\frac{1}{\mu(B)} \int_B f_\varphi ~d\mu\right)$$
where $(g, B, \varphi)$ ranges over the directed system of Riemannian metrics $g$, $g$-balls $B$, and trivializations $\varphi$ over $B$ of $F$.
Then $f(x)$ exists $\mu$-almost everywhere.
\end{lemma}
\begin{proof}
Restrict to those metrics $g$ whose Besicovitch number is $\leq N$ on balls of radius $1/N$, and show that this is true for those metrics using the Hardy-Littlewood maximal inequality.
Then take $N \to \infty$; since $\NN$ is countable, the exceptional set is still null.
\end{proof}

The point is that the choice of trivialization and the choice of metric don't affect $f$.

For our next few results, choose a metric $g$.

\begin{lemma}[trace theorem]
Let $U$ be an open set, $N = \partial U$ Lipschitz.
Then for every $u \in BV_{loc}(M)$ there exists $v \in L^1_{loc}(N)$ such that for every $X \in C_c(M, TM)$,
$$\int_U (du, X) ~\vol + \int_U u ~\mathcal L_X\vol = \int_N vg(X, \normal) ~\vol_N.$$
\end{lemma}
\begin{proof}
Diffeomorphism invariance, or copy the euclidean proof verbatim.
\end{proof}

\begin{definition}
Let $U$ be a Caccioppoli set, $\omega = d1_U ~\vol$, and write $\omega = \normal \mu$ where $\mu$ is the total variation of $\omega$, since $\omega$ is a $T'M$-valued Radon measure.
The \dfn{reduced boundary} $\partial^* U$ is the set of points $x$ with $|\normal(x)| = 1$.
The \dfn{conormal $1$-form} to $\partial^* U$ is $\normal$.
\end{definition}

From the Lebesgue differentiation theorem it is obvious that $\normal$ and $\partial^* U$ are diffeomorphism-invariant.

\begin{lemma}
If $U$ is Caccioppoli then $\partial^* U$ is empty or $d-1$-dimensional, is $d-1$-rectifiable, and is dense in $\partial U$.
If $\normal$ extends to a continuous $1$-form on $\partial U$, then $\partial U = \partial^* U$ is $C^1$.
\end{lemma}
\begin{proof}
Diffeomorphism invariance.
\end{proof}

\begin{lemma}[coarea formula]
Let $u \in BV_{loc}(M)$, $E_y = \{u > y\}$, $\omega(y) = d1_{E_y} ~\vol$, $\mu$ the total variation of $du ~\vol$.
Then
$$\mu = \int_{-\infty}^\infty |\omega(y)| ~dy.$$
Here the integral is meant in the Bochner sense on the space of $T'M$-valued Radon measures.
\end{lemma}
\begin{proof}
Modify the proof of the euclidean case slightly.
\end{proof}

\section{Functions of least gradient}
\begin{definition}
A function $u \in BV_l$ has \dfn{least gradient} if for every $v \in BV_c$,
$$\int_U |du| ~\vol \leq \int_U |du + dv| ~\vol$$
provided $U$ contains the support of $v$.

A set $A$ has \dfn{least perimeter} if $1_A$ has least gradient.
\end{definition}

Let
$$\eta(u, U) = \inf_v \int_U |d(u + v)| ~\vol.$$
We have the following a priori estimate:

\begin{lemma}
Let $u,v \in BV(M)$ and suppose $\partial U$ is Lipschitz. Then
$$|\eta(u, U) - \eta(v, U)| \leq \int_{\partial U} |u - v| ~\vol_{\partial U}.$$
\end{lemma}
\begin{proof}
Immediate from the classical proof plus the fact that the coarea and trace theorems are true.
\end{proof}

\begin{lemma}
If $U$ has least perimeter and its conormal $1$-form is continuous, then $\partial U$ is analytic.
(If the metric is smooth and not analytic then $\partial U$ is just smooth.)
\end{lemma}
\begin{proof}
By the implicit function theorem, if the conormal is continuous, then locally $\partial U$ is the graph of a $C^1$ minimizer $f$ of $\int \sqrt{1 + |df|^2}$.
By Hilbert's 19th problem, $f$ is analytic.
If $d = 2$ the proof is even easier: if the conormal is continuous, then $\partial U$ is a geodesic, so it's analytic.
\end{proof}

\begin{definition}
A sequence $(u_n)$ has \dfn{approximately least gradient} if
$$\limsup_{n \to \infty} \int_U |du_n| ~\vol \leq \liminf_{n \to \infty} \eta(u_n, U)$$
where the rate of convergence is uniform in $U$(?).
\end{definition}

\begin{definition}
Let $(u_n)$ be a sequence in $BV_l(M)$ which converges in $L^1_l$ to $u$.
We say that a Lipschitz hypersurface $N$ \dfn{has no singularities} of $(u_n)$ if:
\begin{enumerate}
\item \label{cond1Mir} $\sup_n \int_N |du_n| ~\vol = 0$.
\item \label{cond2Mir} $(u_n)$ is bounded in $L^1(N, \vol_N)$.
\item \label{cond3Mir} $\int_N |du| ~\vol = 0$.
\item \label{cond4Mir} $u_n \to u$ in $L^1(N, \vol_N)$.
\end{enumerate}
We say that $N$ \dfn{has no singularities} of $u \in BV_l(M)$ if $N$ has no singularities of the sequence $u_n = u$.
By Condition $k$ we mean the $k$th bullet in the above list.
\end{definition}

\begin{lemma}
Let $(u_n)$ be a sequence in $BV_l(M)$ which converges in $L^1_l(U)$. Then:
\begin{enumerate}
\item \label{probabilistic balls} For every $x \in M$ and $R > 0$ such that $B(x, R) \Subset M$ and almost every $r \in (0, R]$, $\partial B(x, r)$ has no singularities of $(u_n)$.
\item \label{probabilistic hypersurfaces} For every $U \Subset M$ there exists $U \subseteq V \Subset M$ such that $\partial V$ has no singularities of $(u_n)$.
\end{enumerate}
\end{lemma}
\begin{proof}
Probabilistic method. Miranda and Giusti both thought this was so obvious that they stated it without proof.
\end{proof}

\begin{theorem}[Miranda stability theorem]
If $(u_n)$ has approximately least gradient and converges in $L^1_{loc}$ to $u$, then $u$ has least gradient and $\int_U du_n \to \int_U du$ for any open set $U$ such that $\partial U$ has no singularities of $(u_n)$.
\end{theorem}
\begin{proof}
Though this is technically a stronger statement of the classical Miranda stability theorem its proof is the same.
\end{proof}

\begin{corollary}
If $u$ has least gradient then $\{u > y\}$ has least perimeter.
\end{corollary}
\begin{proof}
Follows from an argument in Bombieri's paper since we have the coarea and Miranda stability theorems.
\end{proof}

\begin{corollary}
Let $(u_n)$ have approximately least gradient and be indicator functions.
Then a subsequence of $(u_n)$ converges ae (for the volume form), in $L^1_{loc}$, and in total variation on sets with no singularities to the indicator function of a set of least perimeter.
\end{corollary}
\begin{proof}
The forgetful map $BV \to L^1$ is compact, so we get a subsequence in $L^1_{loc}$ from that.
Then convergence in $L^1$ implies converges ae along a subsequence.
Miranda stability theorem implies that it's least perimeter.
\end{proof}

\begin{proposition}[monotonicity formula]
Let $E$ be a Caccioppoli set in $\RR^d$ such that for some $h \geq 0$ and $R > 0$ and $r \in (0, R)$,
$$|\partial^* E \cap B_r| \leq (1 + h) \eta(E, r).$$
Then for every $0 < \rho < r < R$ and $\alpha = \log(r/\rho)$,
\begin{align*}
&|r^{1-d} |\partial^* E \cap B_r| - \rho^{1 - d} |\partial^* E \cap B_\rho||^2 \\
&\qquad \lesssim_d (1 + h) (1 + \alpha + h\alpha^2)(r^{1 - d} |\partial^* E \cap B_r| - \rho^{1 - d} |\partial^* E \cap B_r| + O_d(h\alpha)).
\end{align*}
\end{proposition}
\begin{proof}
Follows from a theorem of Giusti.
\end{proof}

Since this was true for Caccioppoli sets in $\RR^d$, in a small coordinate patch it's true for sets of least perimeter on $M$, where the implied constants depend on the curvature and the patch.
For example we can bound $h$ in terms of the radius of the patch and the norm of the Ricci tensor, provided that we're in normal coordinates (just Taylor expand the volume form).
Note that for minimal surfaces on manifolds we already have the exponential monotonicity formula
$$\partial_r(e^{Ar^2}r^{1-d} |\partial^* E \cap B_r|) \geq 0$$
but this seems unsuitable because we can write this out as
$$e^{Ar^2}\partial_r (r^{1-d}|\partial^* E \cap B_r|) \gtrsim -r^{2-d} |\partial^* E \cap B_r|.$$
With the heuristic $|\partial^* E \cap B_r| \sim r^{d - 1}$ (justified by our next estimate) we conclude that if $\rho < r$,
$$r^{1-d}|\partial^* E \cap B_r| - \rho^{1 - d}|\partial^* E \cap B_\rho| \gtrsim \rho-r.$$
This might be useful some times (but I would have to prove it!! It's only known for smooth functions I think).

But in our application we usually have $\rho \sim r$, in which case the above formula reads
$$r^{1-d}|\partial^* E \cap B_r| - \rho^{1 - d}|\partial^* E \cap B_\rho| \gtrsim -r.$$
Under these assumptions the logarithmic monotonicity formula reads
$$r^{1-d}|\partial^* E \cap B_r| - \rho^{1 - d}|\partial^* E \cap B_\rho| \gtrsim -h$$
and we can take $h \sim r^2$ using the formula for the volume form in normal coordinates.
Thus we have
$$r^{1-d}|\partial^* E \cap B_r| - \rho^{1 - d}|\partial^* E \cap B_\rho| \gtrsim -r^2.$$
So the log formula seems a lot stronger, provided that we can assume that $\rho \sim r$.
This is not a trivial assumption and I think that there's a step in my proof of the de Giorgi lemma where I need to better justify it.

\begin{proposition}[density formula]
Let $E$ be a Caccioppoli set (in whatever Riemannian manifold), $p \in \partial^* E$, and $0 < r_0 \ll 1$.
If for every $r \in (0, r_0)$,
$$|\partial^* E \cap B(p, r)| \leq 2\eta(E, B(p, r)),$$
then
$|E \cap B_r|$, $|E^c \cap B_r|$, and $|\partial^* E \cap B_r|$ are bounded from below by $Cr^{d - 1}$.
\end{proposition}
\begin{proof}
This takes some work, since the hypothesis and conclusion are preserved by measure-preserving isomorphisms, not diffeomorphisms, and because the hypothesis is somewhat weaker than the classical density formula (which, aside from being in euclidean space where we have nice inequalities, is for minimal surfaces only).
You can prove it using Sobolev inequalities, a priori estimate, and Gr\"onwall inequality (where $1 - r$ represents ``time").
\end{proof}

\begin{proposition}[blowing up the manifold at a point]
Fix $p$ and $0 < \zeta \ll 1$ depending on $p,g$.
Suppose that $U$ is an open set with least perimeter in $B(p, \zeta)$ and $p \in \partial^* U$.
Let $A = \exp_p^* U$, $A_t = \{v \in T_pM: tv \in A\}$, and
$$u_t = 1_{A_t}: T_pM \to \RR.$$
Then $(u_t)$ has approximately least gradient on $T_pM$, and in fact for every $V \Subset T_pM$ with Lipschitz boundary,
$$\int_V |du_t| \leq (1 + ct^2) \eta_{T_pM}(A_t, V) \lesssim |\partial V|.$$
So after passing to a subsequence, $u_t \to u_0$ in $L^1_{loc}$, almost everywhere, and in total variation on sets with no singularities, where $u_0$ is the indicator function of a half-space $C$ such that $0 \in \partial C$.
We have an asymptotic formula for $c$ in terms of the Ricci tensor of $g$.
\end{proposition}
\begin{proof}
Since $T_pM$ is euclidean, $d-1$-dimensional Hausdorff measure scales nicely on it.
This can be used, along with our a priori estimate, to prove the blowup estimate on $u_t$.
Miranda and compactness give our subsequence, and $C$ must have least perimeter in the euclidean sense.
Therefore the euclidean version of this theorem gives that $\partial C$ is smooth.
But $C$ is a blowup, so if $\partial C$ has a tangent space it must be equal to it.
Therefore $\partial C$ is a hyperplane.
\end{proof}

\section{de Giorgi lemma}
In this section we fix coordinates on $M$ and so can define
$$\Lambda(U, V) = \int_V |d1_U| ~\vol - \left|\int_V d1_U ~\vol\right|$$
whenever $U$ is a Caccioppoli set and $V$ has Lipschitz boundary.
We also assume that $g$ is bounded in some sufficiently strong topology ($C^3$ should be overkill) on $M$.

\begin{proposition}[de Giorgi lemma]
There exist $\sigma, r > 0$ such that for every Caccioppoli set $U$ of least perimeter and every ball $V$ of radius $\rho \in (0, r)$ such that $\Lambda(U, V) < \sigma \rho^{d - 1}$,
$$\Lambda(U, V/2) < 2^{-d} \Lambda(U, V).$$
\end{proposition}

To prove this proposition we start with a well-known special case:

\begin{proposition}
Let $(L_n)$ be a sequence of open subsets of $\RR^d$ with $C^1$ boundary, $\rho > 0$, and $(\beta_n) \subset \RR^d$.
Supppose that
$$\Lambda(L_n, B(0, \rho)) \leq \beta_n,$$
$$\lim_{n \to \infty} \inf_{\partial L_j \cap B(0, \rho)} \normal_{L_j}(x)_d = 1,$$
and
$$|\partial L_n \cap B(0, \rho)| \leq \eta(L_n, B(0, \rho)) + o(\beta_n),$$
then
$$\limsup_{n \to \infty} \Lambda(L_n, B(0, \rho/2)) \leq 2^{-(d+1)} \beta_n.$$
\end{proposition}
\begin{proof}
Proven by Miranda using properties of the Laplace operator.
\end{proof}

Now to begin the proof of the de Giorgi lemma we argue by contradiction.
So for the rest of the section we assume that it's false.

\begin{lemma}[contradiction assumption]
There exist $(U_n)$ and $V_n = B_g(x_n, \rho_n)$ such that $U_n$ has least perimeter in $V_n$ with respect to $g$, $\rho_n \to 0$,
$$\beta_n = \frac{\Lambda_g(U_n, V_n)}{\rho_n^{d - 1}} = 0$$
and
$$\Lambda_g(U_n, V_n/2) \geq 2^{-d} \Lambda_g(U_n, V_n).$$
\end{lemma}
\begin{proof}
Obvious from our assumption.
\end{proof}

In our next lemma, and the rest of the section, we take $\Lambda$, $\eta$, etc with respect to the euclidean metric on $\RR^d$.

\begin{lemma}
There exists $c > 0$ and a sequence of Caccioppoli subsets $E_n$ of $\RR^d$ and $(\rho_n) \in \ell^1$ such that for every $A \subseteq B(0, 1)$,
$$|\partial^* E_n \cap A| \leq (1 + c\rho_n^2)^2 \eta(E_n, A)$$
and $(1 + c\rho_n^2)^2 \leq 2$. Also $\int_{B(0, 1)} d1_{E_n}$ is a scalar multiple of $v_d$ and
$$\Lambda(E_n, B(0, 1/2)) \geq 2^{-d} (1 + c\rho_n^2) \Lambda(E_n, B(0, 1)).$$
Moreover
$$\gamma_n = \Lambda(E_n, B(0, 1))$$
is in $\ell^1$.
\end{lemma}
\begin{proof}
Follows from writing out our previous lemma in coordinates, which incurs a small error to go from $g$ to the euclidean metric, and then spending some of the symmetries of euclidean space.
\end{proof}

We define the convolution kernel, previously introduced by Giusti,
$$\chi_\varepsilon(x) = C\varepsilon^{-d} \left(1 - \frac{|x|}{\varepsilon}\right) \vee 0.$$
Note that $\chi_\varepsilon \to \delta_0$ in the weak topology of measures.
This kernel can't give us anything better than $C^1$ since it has a $|\cdot|$ and a $\vee$ but it does have
$$1 - \delta - \frac{|x - \xi|}{\varepsilon} \lesssim \varepsilon^d \chi_\varepsilon(x - y) \lesssim 1 + \delta - \frac{|x - \xi|}{\varepsilon}$$
for every $x,\xi \in B(0, 1)$, $y \in B(\xi, \delta\varepsilon)$, and $0 < \delta \ll 1$.
This estimate is extremely convenient in proving the next lemma, and we don't need anything more than $C^1$ to prove that the conormal is continuous, so it's fine.
We define
$$f^{(\varepsilon)} = f * \chi_\varepsilon.$$

\begin{lemma}
Let $\varphi_{n,\ell} = (1_{E_n} ~\vol)^{(\gamma_n^\ell)}$.
Then $\varphi_{n,\ell}$ is $C^1$, and there exist $\lambda_{n,\ell} > 0$, $I_n \subseteq B(0, 1)$ such that for every $\ell$, $\lambda_{n,\ell} \to 0$ and $I_{n,\ell} \to B(0, 1)$. Moreover if $x \in I_n$ then
$$\partial_d \varphi_{n, \ell} > (1 - \lambda_{n, \ell})|d\varphi_{n, \ell}|.$$
\end{lemma}
\begin{proof}
Let $\delta > 0$ to be chosen.
Let $\varepsilon = \gamma_n^\ell$ and write $(|d1_{E_n} - \partial_d 1_{E_n}) ~\vol = (f_1 + f_2) ~\vol$ where $f_1 = f1_{B(0, \varepsilon(1-2\delta))}$.
Then we need to estimate $f_i^{(\varepsilon)}$.
It's easy to show
$$f_2^{(\varepsilon)}(x) \lesssim \delta \gamma_n^{1 - d} (|d1_{E_n}| ~\vol)^{(\varepsilon)}(x).$$
This is because $f_2$ is supported outside $B(0, (1-2\delta)\varepsilon)$ and $\chi_\varepsilon$ is supported inside $B(0, \varepsilon)$.

For $f_1^{(\varepsilon)}$ we note that $f_1$ is supported on $X = B(0, \varepsilon) \cap \partial^* E_n$ so we find a maximal set of disjoint balls $V$ of radius $\delta \varepsilon$. Then the $2$-dilates of those balls cover $X$.
We then estimate $(f1_{2V})^{(\varepsilon)}$ and sum over all balls.
Using the monotonicity and density formulae and a lot of analysis we see that if $\delta \geq \gamma_n^d$ then
$$f_1^{(\varepsilon)}(x) \lesssim |d1_{E_n}(x)|^{(\varepsilon)}$$
I think. Currently my proof of this fact has an error but I think it's just a typo.
Amyways this suggests that if we let $\delta = \gamma_n^d$ and $n$ is so large that $\delta$ is small enough we're done.
The point is that we chose $\delta$ to not be too small so we could exploit the gain from the log-monotonicity formula.
\end{proof}

\begin{lemma}
There exist $(L_n)$ with $C^1$ boundary such that $d1_{L_n} - d1_{E_n} \to 0$ in the weak topology of measures and for almost every $r \in (0, 1)$,
$$|\partial L_n \cap B(0, r)| \leq |\partial E_n \cap B(0, r)| + o(\gamma_n),$$
$$||1_{L_n} - 1_{E_n}||_{L^1(\partial B(0, r))} \ll \gamma_n,$$
$$\lim_{n \to \infty} \inf_{\partial L_n \cap B(0, r)} (\normal_{E_n})_d = 1,$$
$$|\partial^* E_n \cap B(0, r)| \leq |\partial L_n \cap B(0, r)| + o(\rho_n + \gamma_n).$$
\end{lemma}
\begin{proof}
Since $\varphi_{n,\ell}$ is $C^1$ and the previous lemma shows that it has no critical points, its level sets are $C^1$ hypersurfaces.
Using the mean value theorem we can optimize to find the best such hypersurface which has these properties.
\end{proof}

Here's where I'm stuck. I need to modify the above arguments to somehow get rid of the stupid $\rho_n$ in the above lemma to get
$$|\partial^* E_n \cap B(0, r)| \leq |\partial L_n \cap B(0, r)| + o(\gamma_n).$$
If that happens, the above lemma plus our contradiction assumption contradicts an above lemma of Miranda.

Once that happens, we're basically done, because we can estimate $\normal$ and show that it is continuous.


\end{document}
