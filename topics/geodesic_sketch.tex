\documentclass[reqno,12pt,letterpaper]{amsart}
\RequirePackage{amsmath,amssymb,amsthm,graphicx,mathrsfs,url}
\RequirePackage[usenames,dvipsnames]{color}
\RequirePackage[colorlinks=true,linkcolor=Red,citecolor=Green]{hyperref}
\RequirePackage{amsxtra}
\usepackage{cancel}
\usepackage{tikz-cd}

\setlength{\textheight}{8.50in} \setlength{\oddsidemargin}{0.00in}
\setlength{\evensidemargin}{0.00in} \setlength{\textwidth}{6.08in}
\setlength{\topmargin}{0.00in} \setlength{\headheight}{0.18in}
\setlength{\marginparwidth}{1.0in}
\setlength{\abovedisplayskip}{0.2in}
\setlength{\belowdisplayskip}{0.2in}
\setlength{\parskip}{0.05in}
\renewcommand{\baselinestretch}{1.10}

\title[Geodesic laminations by minimal currents]{Geodesic laminations by minimal currents}
\author{Aidan Backus}
\date{July 2021}

\newcommand{\NN}{\mathbf{N}}
\newcommand{\ZZ}{\mathbf{Z}}
\newcommand{\QQ}{\mathbf{Q}}
\newcommand{\RR}{\mathbf{R}}
\newcommand{\CC}{\mathbf{C}}
\newcommand{\DD}{\mathbf{D}}
\newcommand{\PP}{\mathbf P}
\newcommand{\MM}{\mathbf M}
\newcommand{\II}{\mathbf I}
\newcommand{\Hyp}{\mathbf H}

\DeclareMathOperator{\card}{card}
\DeclareMathOperator{\cent}{center}
\DeclareMathOperator{\ch}{ch}
\DeclareMathOperator{\codim}{codim}
\DeclareMathOperator{\diag}{diag}
\DeclareMathOperator{\diam}{diam}
\DeclareMathOperator{\dom}{dom}
\DeclareMathOperator{\Gal}{Gal}
\DeclareMathOperator{\Hom}{Hom}
\DeclareMathOperator{\Jac}{Jac}
\DeclareMathOperator{\Lip}{Lip}
\DeclareMathOperator{\Met}{Met}
\DeclareMathOperator{\id}{id}
\DeclareMathOperator{\rad}{rad}
\DeclareMathOperator{\rank}{rank}
\DeclareMathOperator{\Radon}{Radon}
\DeclareMathOperator*{\Res}{Res}
\DeclareMathOperator{\sgn}{sgn}
\DeclareMathOperator{\singsupp}{sing~supp}
\DeclareMathOperator{\Spec}{Spec}
\DeclareMathOperator{\supp}{supp}
\DeclareMathOperator{\Tan}{Tan}
\newcommand{\tr}{\operatorname{tr}}

\newcommand{\Ric}{\mathrm{Ric}}
\newcommand{\Riem}{\mathrm{Riem}}

\newcommand{\dbar}{\overline \partial}

\DeclareMathOperator{\atanh}{atanh}
\DeclareMathOperator{\csch}{csch}
\DeclareMathOperator{\sech}{sech}

\DeclareMathOperator{\Div}{div}
\DeclareMathOperator{\grad}{grad}
\DeclareMathOperator{\Ell}{Ell}
\DeclareMathOperator{\WF}{WF}

\newcommand{\Hilb}{\mathcal H}
\newcommand{\normal}{\mathbf n}
\newcommand{\vol}{\mathrm{vol}}

\newcommand{\pic}{\vspace{30mm}}
\newcommand{\dfn}[1]{\emph{#1}\index{#1}}

\renewcommand{\Re}{\operatorname{Re}}
\renewcommand{\Im}{\operatorname{Im}}


\newtheorem{theorem}{Theorem}[section]
\newtheorem{badtheorem}[theorem]{``Theorem"}
\newtheorem{prop}[theorem]{Proposition}
\newtheorem{lemma}[theorem]{Lemma}
\newtheorem{claim}[theorem]{Claim}
\newtheorem{proposition}[theorem]{Proposition}
\newtheorem{corollary}[theorem]{Corollary}
\newtheorem{conjecture}[theorem]{Conjecture}
\newtheorem{axiom}[theorem]{Axiom}
\newtheorem{assumption}[theorem]{Assumption}

\theoremstyle{definition}
\newtheorem{definition}[theorem]{Definition}
\newtheorem{remark}[theorem]{Remark}
\newtheorem{example}[theorem]{Example}
\newtheorem{notation}[theorem]{Notation}

\newtheorem{exercise}[theorem]{Discussion topic}
\newtheorem{homework}[theorem]{Homework}
\newtheorem{problem}[theorem]{Problem}

\newtheorem{ack}{Acknowledgements}

\numberwithin{equation}{section}


% Mean
\def\Xint#1{\mathchoice
{\XXint\displaystyle\textstyle{#1}}%
{\XXint\textstyle\scriptstyle{#1}}%
{\XXint\scriptstyle\scriptscriptstyle{#1}}%
{\XXint\scriptscriptstyle\scriptscriptstyle{#1}}%
\!\int}
\def\XXint#1#2#3{{\setbox0=\hbox{$#1{#2#3}{\int}$ }
\vcenter{\hbox{$#2#3$ }}\kern-.6\wd0}}
\def\ddashint{\Xint=}
\def\dashint{\Xint-}

%\usepackage{color}
%\hypersetup{%
%    colorlinks=true, % make the links colored%
%    linkcolor=blue, % color TOC links in blue
%    urlcolor=red, % color URLs in red
%    linktoc=all % 'all' will create links for everything in the TOC
%Ning added hyperlinks to the table of contents 6/17/19
%}

% style=alphabetic
\usepackage[backend=bibtex,maxcitenames=50,maxnames=50]{biblatex}
\addbibresource{topics.bib}
\renewbibmacro{in:}{}
\DeclareFieldFormat{pages}{#1}

\begin{document}

\maketitle

%%%%%%%%%%%%%%%%%%%%%%%%%%%%%%%%%%%%%%%%%%%%%%%%%%%%%%%

\section{Measure theory}
Fix a smooth manifold $M$.

\begin{definition}
Let $F$ be a normed complex vector bundle of finite rank over $M$.
A \dfn{$F$-valued Radon measure} is a continuous linear functional on $C_c(M, F')$.
The \dfn{sphere bundle} $SF$ is $\{(x, v) \in F: |v| = 1\}$, which we also identify with $(F \setminus \{0\})/\RR^+$.
\end{definition}

\begin{lemma}[Riesz-Markov representation theorem]
Let $\omega$ be an $F$-vaulued Radon measure and $\mu$ the total variation of $\omega$.
Then there exists a $\mu$-measurable section $f$ of $SF$ such that for every $X \in C_c(M, F')$,
$$\langle \omega, X\rangle = \int_M (f, X) ~d\mu.$$
Furthermore, $f$ does not depend on the norm of $F$, and is defined $\mu$-almost everywhere.
\end{lemma}
\begin{proof}
Just use classical Riesz-Markov and check the transition functions.
\end{proof}

\begin{lemma}[Lebesgue differentiation theorem]
Let $\mu$ be a Radon measure on $M$, $f \in L^1_{loc}(M, SF, \mu)$, and
$$f(x) = \lim_{(g, B, \varphi)} \varphi^{-1}\left(\frac{1}{\mu(B)} \int_B f_\varphi ~d\mu\right)$$
where $(g, B, \varphi)$ ranges over the directed system of Riemannian metrics $g$, $g$-balls $B$, and trivializations $\varphi$ over $B$ of $F$.
Then $f(x)$ exists $\mu$-almost everywhere.
\end{lemma}
\begin{proof}
Restrict to those metrics $g$ whose Besicovitch number is $\leq N$ on balls of radius $1/N$, and show that this is true for those metrics using the Hardy-Littlewood maximal inequality.
Then take $N \to \infty$; since $\NN$ is countable, the exceptional set is still null.
\end{proof}

The point is that the choice of trivialization and the choice of metric don't affect $f$.

For our next few results, choose a metric $g$.

\begin{lemma}[trace theorem]
Let $U$ be an open set, $N = \partial U$ Lipschitz.
Then for every $u \in BV_{loc}(M)$ there exists $v \in L^1_{loc}(N)$ such that for every $X \in C_c(M, TM)$,
$$\int_U (du, X) ~\vol + \int_U u ~\mathcal L_X\vol = \int_N vg(X, \normal) ~\vol_N.$$
\end{lemma}
\begin{proof}
Diffeomorphism invariance, or copy the euclidean proof verbatim.
\end{proof}

\begin{definition}
Let $U$ be a Caccioppoli set, $\omega = d1_U ~\vol$, and write $\omega = \normal \mu$ where $\mu$ is the total variation of $\omega$, since $\omega$ is a $T'M$-valued Radon measure.
The \dfn{reduced boundary} $\partial^* U$ is the set of points $x$ with $|\normal(x)| = 1$.
The \dfn{conormal $1$-form} to $\partial^* U$ is $\normal$.
\end{definition}

From the Lebesgue differentiation theorem it is obvious that $\normal$ and $\partial^* U$ are diffeomorphism-invariant.

\begin{lemma}
If $U$ is Caccioppoli then $\partial^* U$ is empty or $d-1$-dimensional, is $d-1$-rectifiable, and is dense in $\partial U$.
If $\normal$ extends to a continuous $1$-form on $\partial U$, then $\partial U = \partial^* U$ is $C^1$.
\end{lemma}
\begin{proof}
Diffeomorphism invariance.
\end{proof}

\begin{lemma}[coarea formula]
Let $u \in BV_{loc}(M)$, $E_y = \{u > y\}$, $\omega(y) = d1_{E_y} ~\vol$, $\mu$ the total variation of $du ~\vol$.
Then
$$\mu = \int_{-\infty}^\infty |\omega(y)| ~dy.$$
Here the integral is meant in the Bochner sense on the space of $T'M$-valued Radon measures.
\end{lemma}
\begin{proof}
Modify the proof of the euclidean case slightly.
\end{proof}


\end{document}
