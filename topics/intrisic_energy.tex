\documentclass[reqno,12pt,letterpaper]{amsart}
\RequirePackage{amsmath,amssymb,amsthm,graphicx,mathrsfs,url,mathtools,slashed}
\RequirePackage[usenames,dvipsnames]{color}
\RequirePackage[colorlinks=true,linkcolor=Red,citecolor=Green]{hyperref}
\RequirePackage{amsxtra}
\usepackage{cancel}
\usepackage{tikz-cd}

\setlength{\textheight}{9in} \setlength{\oddsidemargin}{-0.25in}
\setlength{\evensidemargin}{-0.25in} \setlength{\textwidth}{7in}
\setlength{\topmargin}{-0.25in} \setlength{\headheight}{0.18in}
\setlength{\marginparwidth}{1.0in}
\setlength{\abovedisplayskip}{0.2in}
\setlength{\belowdisplayskip}{0.2in}
\setlength{\parskip}{0.05in}
\renewcommand{\baselinestretch}{1.05}

\title[Geodesic laminations by minimal currents]{Geodesic laminations by minimal currents}
\author{Aidan Backus}
\date{July 2021}

\newcommand{\NN}{\mathbf{N}}
\newcommand{\ZZ}{\mathbf{Z}}
\newcommand{\QQ}{\mathbf{Q}}
\newcommand{\RR}{\mathbf{R}}
\newcommand{\CC}{\mathbf{C}}
\newcommand{\DD}{\mathbf{D}}
\newcommand{\PP}{\mathbf P}
\newcommand{\MM}{\mathbf M}
\newcommand{\II}{\mathbf I}
\newcommand{\Hyp}{\mathbf H}
\newcommand{\SL}{\mathrm{SL}}
\newcommand{\evect}{\mathbf e}

\newcommand{\Dir}{\mathscr D}
\newcommand{\Lagrange}{\mathscr L}

\DeclareMathOperator{\card}{card}
\DeclareMathOperator{\cent}{center}
\DeclareMathOperator{\ch}{ch}
\DeclareMathOperator{\codim}{codim}
\DeclareMathOperator{\diag}{diag}
\DeclareMathOperator{\diam}{diam}
\DeclareMathOperator{\dom}{dom}
\DeclareMathOperator{\Gal}{Gal}
\DeclareMathOperator{\Hom}{Hom}
\DeclareMathOperator{\Jac}{Jac}
\DeclareMathOperator{\Lip}{Lip}
\DeclareMathOperator{\Met}{Met}
\DeclareMathOperator{\id}{id}
\DeclareMathOperator{\rad}{rad}
\DeclareMathOperator{\rank}{rank}
\DeclareMathOperator{\Radon}{Radon}
\DeclareMathOperator*{\Res}{Res}
\DeclareMathOperator{\sgn}{sgn}
\DeclareMathOperator{\singsupp}{sing~supp}
\DeclareMathOperator{\Spec}{Spec}
\DeclareMathOperator{\supp}{supp}
\DeclareMathOperator{\Tan}{Tan}
\newcommand{\tr}{\operatorname{tr}}

\newcommand{\Ric}{\mathrm{Ric}}
\newcommand{\Riem}{\mathrm{Riem}}

\newcommand{\dbar}{\overline \partial}

\DeclareMathOperator{\atanh}{atanh}
\DeclareMathOperator{\arcosh}{arcosh}
\DeclareMathOperator{\csch}{csch}
\DeclareMathOperator{\sech}{sech}

\DeclareMathOperator{\Div}{div}
\DeclareMathOperator{\grad}{grad}
\DeclareMathOperator{\Ell}{Ell}
\DeclareMathOperator{\WF}{WF}

\newcommand{\Hilb}{\mathcal H}
\newcommand{\Homology}{\mathrm H_{\mathrm{dR}}}
\newcommand{\normal}{\mathbf n}
\newcommand{\vol}{\mathrm{vol}}

\newcommand{\pic}{\vspace{30mm}}
\newcommand{\dfn}[1]{\emph{#1}\index{#1}}

\renewcommand{\Re}{\operatorname{Re}}
\renewcommand{\Im}{\operatorname{Im}}


\newtheorem{theorem}{Theorem}[section]
\newtheorem{badtheorem}[theorem]{``Theorem"}
\newtheorem{prop}[theorem]{Proposition}
\newtheorem{lemma}[theorem]{Lemma}
\newtheorem{claim}[theorem]{Claim}
\newtheorem{proposition}[theorem]{Proposition}
\newtheorem{corollary}[theorem]{Corollary}
\newtheorem{conjecture}[theorem]{Conjecture}
\newtheorem{axiom}[theorem]{Axiom}
\newtheorem{assumption}[theorem]{Assumption}

\theoremstyle{definition}
\newtheorem{definition}[theorem]{Definition}
\newtheorem{remark}[theorem]{Remark}
\newtheorem{example}[theorem]{Example}
\newtheorem{notation}[theorem]{Notation}

\newtheorem{exercise}[theorem]{Discussion topic}
\newtheorem{homework}[theorem]{Homework}
\newtheorem{problem}[theorem]{Problem}

\newtheorem{ack}{Acknowledgements}

\numberwithin{equation}{section}


% Mean
\def\Xint#1{\mathchoice
{\XXint\displaystyle\textstyle{#1}}%
{\XXint\textstyle\scriptstyle{#1}}%
{\XXint\scriptstyle\scriptscriptstyle{#1}}%
{\XXint\scriptscriptstyle\scriptscriptstyle{#1}}%
\!\int}
\def\XXint#1#2#3{{\setbox0=\hbox{$#1{#2#3}{\int}$ }
\vcenter{\hbox{$#2#3$ }}\kern-.6\wd0}}
\def\ddashint{\Xint=}
\def\dashint{\Xint-}

%\usepackage{color}
%\hypersetup{%
%    colorlinks=true, % make the links colored%
%    linkcolor=blue, % color TOC links in blue
%    urlcolor=red, % color URLs in red
%    linktoc=all % 'all' will create links for everything in the TOC
%Ning added hyperlinks to the table of contents 6/17/19
%}

% style=alphabetic
\usepackage[backend=bibtex,maxcitenames=50,maxnames=50]{biblatex}
\addbibresource{topics.bib}
\renewbibmacro{in:}{}
\DeclareFieldFormat{pages}{#1}

\begin{document}
%%%%%%%%%%%%%%%%%%%%%%%%%%%%%%%%%%%%%%%%%%%%%%%%%%%%%%%

% \tableofcontents

\section{Intrinsic Dirichlet energy}
Let $N$ be a $C^1$ submanifold of the Riemannian manifold $M$, and let $P \in N$.
We have a smooth hypersurface
$$H = \exp_P(T_PM)$$
and we consider the normal bundle $T^\perp H$.
According to the tubular neighborhood theorem, the exponential map
\begin{align*}
T^\perp H &\to M \\
(x, v) &\mapsto \exp_x(v)
\end{align*}
is a local diffeomorphism along $H$, say into an open set $U \subseteq M$ which contains an open neighborhood (which we henceforth also denote $H$) of $P$ and is foliated by normal translates of $H$.
In particular, the unit vector field $X$ which is normal to each leaf of the foliation of $U$ induces geodesics $\gamma_x$, $x \in H$, defined by
$$\begin{cases}
\gamma_x(0) = x\\
\gamma_x'(0) = X_x
\end{cases}$$
and hence natural coordinates
$$H \times I \ni (x, y) \mapsto \gamma_x(y) \in U.$$
The key features of these coordinates are:
\begin{enumerate}
\item No arbitrary choices were made in their construction (except possibly the size of $H$ and $I$).
\item $H = \{y = 0\}$.
\item By the implicit function theorem (possibly after shrinking $H$ and $I$) there exists a $C^1$ function $f: H \to I$ such that $N = \{y = f(x)\}$.
\item The function $f$ has a double zero at $P$ (essentially because $H$ was constructed to be tangent to $N$).
\item The metric takes the form
\begin{equation}\label{orthometric}
g = \slashed g(x, y) ~dx^2 + dy^2.
\end{equation}
\end{enumerate}

The surface area of $N$ defines the Lagrangian
$$\Lagrange (x, y, p) = \sqrt{|p|_{\slashed g(x, y)}^2 + 1} \sqrt{\det \slashed g(x, 0)} ~|dx|$$
where $\sqrt{\det \slashed g(x, y)} ~|dx|$ is the area element on $H$.
Indeed,
\begin{align*}
\Lagrange(x, f(x), \nabla f(x))
&= \sqrt{|\nabla f(x)|_{\slashed g(x, f(x))}^2 + 1}  \sqrt{\det \slashed g(x, 0)} ~|dx|\\
&= \sqrt{|(1, \nabla f(x))|_{g(x, f(x))}^2 \cdot \det \slashed g(x, 0)} ~|dx|
\end{align*}
is the area element on $N$. Here we used the formula (\ref{orthometric}).
Taylor expanding in $p$, we get
$$\Lagrange (x, y, p) \approx \left(1 + \frac{|p|_{\slashed g(x, y)}^2}{2} \right) \sqrt{\det \slashed g(x, 0)} ~|dx|.$$
We introduce the semilinearized Lagrangian
$$\Dir (x, y, p) = |p|_{\slashed g(x, y)}^2 \sqrt{\det \slashed g(x, 0)} ~|dx|.$$

\begin{definition}
The \dfn{Dirichlet energy} of $N$ is
$$\Dir N = \Dir(x, f(x), \nabla f(x)).$$
\end{definition}
The idea is that, because $N$ is tangent to $H$, the part of the Dirichlet energy that arises from the ``slope" of $N$ at $H$ is ignored by this definition.
Thus, the only terms in $\nabla f$ that honestly contribute to $\Dir N$ are those that are caused by the ``twisting" of $N$.

Now we compute the Euler-Lagrange operator of $\Dir$.
We use Einstein notation, with Latin symbols ranging over $1, \dots, d - 1$ and Greek symbols ranging over $0, \dots, d - 1$, with the $0$th index referring to the $y$-direction.
The Euler-Lagrange equation is
$$\Dir_{,0}|_{(y, p) = (f, \nabla f)} = \partial^k \left(\frac{\partial \Dir}{\partial p_k}|_{(y, p) = (f, \nabla f)}\right).$$
Also
$$\Dir_{,0} = \slashed g_{ij,0} p^i p^j \sqrt{\slashed g}$$
while
$$\frac{\partial \Dir}{\partial p_k} = 2\slashed g_{jk} p^j \sqrt{\slashed g}$$
so if we denote $\tilde f(x) = (x, f(x))$,
$$\frac{\partial \Dir}{\partial p_k}|_{(y, p) = (f, \nabla f)} = 2\slashed g_{jk}(\tilde f) f^{,j} \sqrt{\slashed g}$$
and hence
\begin{align*}
\partial^k \frac{\partial \Dir}{\partial p_k}|_{(y, p) = (f, \nabla f)}
&= 2\partial^k(\slashed g_{jk}(\tilde f)) f^{,j} \sqrt{\slashed g} + 2\slashed g_{jk}(\tilde f) f^{,jk} \sqrt{\slashed g} + 2\slashed g_{jk}(\tilde f) f^{,j} \partial^k \sqrt{\slashed g}\\
&= 2(\slashed g_{jk}^{,\mu} (\tilde f) \tilde f_\mu^{,k} f^{,j} + \slashed g_{jk}(\tilde f) f^{,jk} + \slashed g_{jk}(\tilde f) f^{,j} \partial^k) \sqrt{\slashed g}.
\end{align*}
Thus the Euler-Lagrange equation is
$$\slashed g_{ij}(\tilde f) f^{,ij} = \frac{\slashed g_{ij,0}(\tilde f)}{2} f^{,i} f^{,j} - \slashed g_{ij}^{,\mu} (\tilde f) \tilde f_\mu^{,i} f^{,j} - \slashed g_{ij}(\tilde f) f^{,j} \partial^i (\log \sqrt{\slashed g}).$$
Breaking up the sum over $\mu$ we get
\begin{align*}
\slashed g_{ij}^{,\mu} (\tilde f) \tilde f_\mu^{,i} f^{,j}
&= \slashed g_{ij}^{,k} (\tilde f) x_k^{,i} f^{,j} + \slashed g_{ij}^{,0} (\tilde f) f^{,i} f^{,j} \\
&= \slashed g_{ij}^{,i} (\tilde f) f^{,j} + \slashed g_{ij}^{,0} (\tilde f) f^{,i} f^{,j}
\end{align*}
whence the Euler-Lagrange equation simplifies to
$$\slashed g_{ij}(\tilde f) f^{,ij} + \frac{\slashed g_{ij, 0} (\tilde f)}{2} f^{,i} f^{,j} + \slashed g^{,i}_{ij}(\tilde f) f^{,j} + \slashed g_{ij}(\tilde f) (\log \sqrt{\slashed g})^{,i} f^{,j} = 0$$
which is a quasilinear elliptic PDE for the function $f$.
We will proceed by comparing this to the Laplace-Beltrami equation
$$\Delta_{\slashed g} f = \slashed g_{ij} f^{,ij} + \slashed g_{ij}^{,i} f^{,j} + \slashed g_{ij} (\log \sqrt{\slashed g})^{,i} f^{,j} = 0.$$
This estimate is good when $||f||_{C^1}$ is small, and is useful because if $\Delta_{\slashed g} f = 0$ and $f$ has a double zero at the center of the ball $B$ of radius $ < \rho_*$ and dimension $d - 1$, then
$$\int_B |df|^2 \sqrt{\slashed g} \leq \frac{1}{2^{d + 1}} \int_{(2 + \varepsilon_*)B} |df|^2 \sqrt{\slashed g},$$
where $\varepsilon_* \in (0, 1)$ is given and
$$\rho_* = \rho_*(\varepsilon_*, \slashed g) > 0$$
does not depend on $f$.
In particular, $\slashed g$ is determined by $H$, which in turn is determined by its exponential pullback in $T_PM$.
But such an exponential pullback is a point in the projective space $T_P^\mathrm{proj} M$, and the projective tangent bundle $T^\mathrm{proj} M$ is fiberwise compact.
Therefore we have
$$\rho_* = \rho_*(\varepsilon_*, g, \Omega) > 0$$
where $\Omega$ is a precompact open set containing $P$.

\printbibliography


\end{document}
