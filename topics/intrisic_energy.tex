\documentclass[reqno,12pt,letterpaper]{amsart}
\RequirePackage{amsmath,amssymb,amsthm,graphicx,mathrsfs,url,mathtools}
\RequirePackage[usenames,dvipsnames]{color}
\RequirePackage[colorlinks=true,linkcolor=Red,citecolor=Green]{hyperref}
\RequirePackage{amsxtra}
\usepackage{cancel}
\usepackage{tikz-cd}

\setlength{\textheight}{9in} \setlength{\oddsidemargin}{-0.25in}
\setlength{\evensidemargin}{-0.25in} \setlength{\textwidth}{7in}
\setlength{\topmargin}{-0.25in} \setlength{\headheight}{0.18in}
\setlength{\marginparwidth}{1.0in}
\setlength{\abovedisplayskip}{0.2in}
\setlength{\belowdisplayskip}{0.2in}
\setlength{\parskip}{0.05in}
\renewcommand{\baselinestretch}{1.05}

\title[Geodesic laminations by minimal currents]{Geodesic laminations by minimal currents}
\author{Aidan Backus}
\date{July 2021}

\newcommand{\NN}{\mathbf{N}}
\newcommand{\ZZ}{\mathbf{Z}}
\newcommand{\QQ}{\mathbf{Q}}
\newcommand{\RR}{\mathbf{R}}
\newcommand{\CC}{\mathbf{C}}
\newcommand{\DD}{\mathbf{D}}
\newcommand{\PP}{\mathbf P}
\newcommand{\MM}{\mathbf M}
\newcommand{\II}{\mathbf I}
\newcommand{\Hyp}{\mathbf H}
\newcommand{\SL}{\mathrm{SL}}
\newcommand{\evect}{\mathbf e}

\newcommand{\Dir}{\mathscr D}
\newcommand{\Lagrange}{\mathscr L}

\DeclareMathOperator{\card}{card}
\DeclareMathOperator{\cent}{center}
\DeclareMathOperator{\ch}{ch}
\DeclareMathOperator{\codim}{codim}
\DeclareMathOperator{\diag}{diag}
\DeclareMathOperator{\diam}{diam}
\DeclareMathOperator{\dom}{dom}
\DeclareMathOperator{\Gal}{Gal}
\DeclareMathOperator{\Hom}{Hom}
\DeclareMathOperator{\Jac}{Jac}
\DeclareMathOperator{\Lip}{Lip}
\DeclareMathOperator{\Met}{Met}
\DeclareMathOperator{\id}{id}
\DeclareMathOperator{\rad}{rad}
\DeclareMathOperator{\rank}{rank}
\DeclareMathOperator{\Radon}{Radon}
\DeclareMathOperator*{\Res}{Res}
\DeclareMathOperator{\sgn}{sgn}
\DeclareMathOperator{\singsupp}{sing~supp}
\DeclareMathOperator{\Spec}{Spec}
\DeclareMathOperator{\supp}{supp}
\DeclareMathOperator{\Tan}{Tan}
\newcommand{\tr}{\operatorname{tr}}

\newcommand{\Ric}{\mathrm{Ric}}
\newcommand{\Riem}{\mathrm{Riem}}

\newcommand{\dbar}{\overline \partial}

\DeclareMathOperator{\atanh}{atanh}
\DeclareMathOperator{\arcosh}{arcosh}
\DeclareMathOperator{\csch}{csch}
\DeclareMathOperator{\sech}{sech}

\DeclareMathOperator{\Div}{div}
\DeclareMathOperator{\grad}{grad}
\DeclareMathOperator{\Ell}{Ell}
\DeclareMathOperator{\WF}{WF}

\newcommand{\Hilb}{\mathcal H}
\newcommand{\Homology}{\mathrm H_{\mathrm{dR}}}
\newcommand{\normal}{\mathbf n}
\newcommand{\vol}{\mathrm{vol}}

\newcommand{\pic}{\vspace{30mm}}
\newcommand{\dfn}[1]{\emph{#1}\index{#1}}

\renewcommand{\Re}{\operatorname{Re}}
\renewcommand{\Im}{\operatorname{Im}}


\newtheorem{theorem}{Theorem}[section]
\newtheorem{badtheorem}[theorem]{``Theorem"}
\newtheorem{prop}[theorem]{Proposition}
\newtheorem{lemma}[theorem]{Lemma}
\newtheorem{claim}[theorem]{Claim}
\newtheorem{proposition}[theorem]{Proposition}
\newtheorem{corollary}[theorem]{Corollary}
\newtheorem{conjecture}[theorem]{Conjecture}
\newtheorem{axiom}[theorem]{Axiom}
\newtheorem{assumption}[theorem]{Assumption}

\theoremstyle{definition}
\newtheorem{definition}[theorem]{Definition}
\newtheorem{remark}[theorem]{Remark}
\newtheorem{example}[theorem]{Example}
\newtheorem{notation}[theorem]{Notation}

\newtheorem{exercise}[theorem]{Discussion topic}
\newtheorem{homework}[theorem]{Homework}
\newtheorem{problem}[theorem]{Problem}

\newtheorem{ack}{Acknowledgements}

\numberwithin{equation}{section}


% Mean
\def\Xint#1{\mathchoice
{\XXint\displaystyle\textstyle{#1}}%
{\XXint\textstyle\scriptstyle{#1}}%
{\XXint\scriptstyle\scriptscriptstyle{#1}}%
{\XXint\scriptscriptstyle\scriptscriptstyle{#1}}%
\!\int}
\def\XXint#1#2#3{{\setbox0=\hbox{$#1{#2#3}{\int}$ }
\vcenter{\hbox{$#2#3$ }}\kern-.6\wd0}}
\def\ddashint{\Xint=}
\def\dashint{\Xint-}

%\usepackage{color}
%\hypersetup{%
%    colorlinks=true, % make the links colored%
%    linkcolor=blue, % color TOC links in blue
%    urlcolor=red, % color URLs in red
%    linktoc=all % 'all' will create links for everything in the TOC
%Ning added hyperlinks to the table of contents 6/17/19
%}

% style=alphabetic
\usepackage[backend=bibtex,maxcitenames=50,maxnames=50]{biblatex}
\addbibresource{topics.bib}
\renewbibmacro{in:}{}
\DeclareFieldFormat{pages}{#1}

\begin{document}
%%%%%%%%%%%%%%%%%%%%%%%%%%%%%%%%%%%%%%%%%%%%%%%%%%%%%%%

% \tableofcontents

\section{Intrinsic Dirichlet energy}
Let $N$ be a $C^1$ submanifold of the Riemannian manifold $M$.
Then we have an orthogonal splitting
$$T_PM = T_PN \oplus T_PN^\perp.$$
Now $T_PN^\perp$ is isomorphic as a Hilbert space to $\RR$, and up to a choice of sign, $T_PN^\perp$ is canonically identified with $\RR$.
Pushing this splitting forward along the exponential map, we get a favored local product structure $(x, y)$ where $\{y = 0\}$ is a smooth hypersurface $H_P = \exp_P(T_PN)$.
By the implicit function theorem, $N$ is cut out by a function $\{y = f(x)\}$, with
$$f: T_PN \to \RR$$
(using the identification of $T_PN^\perp$ with $\RR$).
As long as we have a fixed orientation on $M$, we made no arbitrary choices in this construction.

In the coordinates $(x, y)$, the metric has the block form
$$g = \begin{bmatrix}1 & g_{12} \\ g_{21} & g_{22}\end{bmatrix} \in \begin{bmatrix} \RR & T_PN' \\ T_PN' & T_PN' \otimes T_PN'\end{bmatrix}.$$
This implies that the area element of $N$ is
\begin{align*}
|ds| &= \sqrt{dx^2 + 2g_{12}(x, y) ~dxdy + g_{22}(x, y) ~dy^2} \\
&= \sqrt{1 + 2(g_{12}(x, f(x)), \nabla f) + (g_{22}(x, f(x)), \nabla f(x) \otimes \nabla f(x))} ~|dx|
\end{align*}
where $dx$ is the euclidean area form on $T_PN$.
More generally, we define a Lagrangian density
$$\Lagrange(z, p, x) = \sqrt{1 + 2(g_{12}(x, z), p) + (g_{22}(x, z), p \otimes p)} ~|dx|$$
for $x, p \in T_PN$ and $z \in \RR$.
Taylor expanding $\Lagrange(f, p, x)$ about $p = 0$ we get
$$\Lagrange(z, p, x) \approx (1 + ((g_{12}(x, z), p) + \frac{1}{2}(g_{22}(x, z), p \otimes p))) ~|dx|.$$
We therefore let
$$\Dir(z, p, x) = (2(g_{12}(x, z), p) + (g_{22}(x, z), p \otimes p)) ~|dx|.$$

\begin{definition}
The \dfn{intrinsic Dirichlet energy} of $N$ at $P$ is the Lagrangian density
$$\Dir N(x) = \Dir(f(x), \nabla f(x), x).$$
If $z,p = f, \nabla f$ is a local minimizer of $\Dir$, we call $N$ a \dfn{harmonic hypersurface} in a neighborhood of $P$.
\end{definition}

The idea here is that, while we could view $N$ as a graph in uncountably many ways, the Dirichlet energy of such a graph consists of both the intrinsic Dirichlet energy, which by definition is intrinsic to the embedding $N \xhookrightarrow{} M$ and a term accounting for the ``slope" of $N$, which is extrinsic to the embedding in that it also depends on the choice of a local product structure on $M$.
By imposing that the slope of the graph be $0$ at $P$, we remove the extrinsic content from the definition of the Dirichlet energy.

Our next task is to deduce a mean-value property for the intrinsic Dirichlet energy.

\printbibliography


\end{document}
