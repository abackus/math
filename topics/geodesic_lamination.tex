\documentclass[reqno,12pt,letterpaper]{amsart}
\RequirePackage{amsmath,amssymb,amsthm,graphicx,mathrsfs,url}
\RequirePackage[usenames,dvipsnames]{color}
\RequirePackage[colorlinks=true,linkcolor=Red,citecolor=Green]{hyperref}
\RequirePackage{amsxtra}
\usepackage{cancel}
\usepackage{tikz-cd}

\setlength{\textheight}{8.50in} \setlength{\oddsidemargin}{0.00in}
\setlength{\evensidemargin}{0.00in} \setlength{\textwidth}{6.08in}
\setlength{\topmargin}{0.00in} \setlength{\headheight}{0.18in}
\setlength{\marginparwidth}{1.0in}
\setlength{\abovedisplayskip}{0.2in}
\setlength{\belowdisplayskip}{0.2in}
\setlength{\parskip}{0.05in}
\renewcommand{\baselinestretch}{1.10}

\title[Geodesic laminations by minimal currents]{Geodesic laminations by minimal currents}
\author{Aidan Backus}
\date{July 2021}

\newcommand{\NN}{\mathbf{N}}
\newcommand{\ZZ}{\mathbf{Z}}
\newcommand{\QQ}{\mathbf{Q}}
\newcommand{\RR}{\mathbf{R}}
\newcommand{\CC}{\mathbf{C}}
\newcommand{\DD}{\mathbf{D}}
\newcommand{\PP}{\mathbf P}
\newcommand{\MM}{\mathbf M}
\newcommand{\II}{\mathbf I}
\newcommand{\Hyp}{\mathbf H}

\DeclareMathOperator{\card}{card}
\DeclareMathOperator{\cent}{center}
\DeclareMathOperator{\ch}{ch}
\DeclareMathOperator{\codim}{codim}
\DeclareMathOperator{\diag}{diag}
\DeclareMathOperator{\diam}{diam}
\DeclareMathOperator{\dom}{dom}
\DeclareMathOperator{\Gal}{Gal}
\DeclareMathOperator{\Hom}{Hom}
\DeclareMathOperator{\Jac}{Jac}
\DeclareMathOperator{\Lip}{Lip}
\DeclareMathOperator{\Met}{Met}
\DeclareMathOperator{\id}{id}
\DeclareMathOperator{\rad}{rad}
\DeclareMathOperator{\rank}{rank}
\DeclareMathOperator{\Radon}{Radon}
\DeclareMathOperator*{\Res}{Res}
\DeclareMathOperator{\sgn}{sgn}
\DeclareMathOperator{\singsupp}{sing~supp}
\DeclareMathOperator{\Spec}{Spec}
\DeclareMathOperator{\supp}{supp}
\DeclareMathOperator{\Tan}{Tan}
\newcommand{\tr}{\operatorname{tr}}

\newcommand{\Ric}{\mathrm{Ric}}
\newcommand{\Riem}{\mathrm{Riem}}

\newcommand{\dbar}{\overline \partial}

\DeclareMathOperator{\atanh}{atanh}
\DeclareMathOperator{\csch}{csch}
\DeclareMathOperator{\sech}{sech}

\DeclareMathOperator{\Div}{div}
\DeclareMathOperator{\grad}{grad}
\DeclareMathOperator{\Ell}{Ell}
\DeclareMathOperator{\WF}{WF}

\newcommand{\Hilb}{\mathcal H}
\newcommand{\Homology}{\mathrm H_{\mathrm{dR}}}
\newcommand{\normal}{\mathbf n}
\newcommand{\vol}{\mathrm{vol}}

\newcommand{\pic}{\vspace{30mm}}
\newcommand{\dfn}[1]{\emph{#1}\index{#1}}

\renewcommand{\Re}{\operatorname{Re}}
\renewcommand{\Im}{\operatorname{Im}}


\newtheorem{theorem}{Theorem}[section]
\newtheorem{badtheorem}[theorem]{``Theorem"}
\newtheorem{prop}[theorem]{Proposition}
\newtheorem{lemma}[theorem]{Lemma}
\newtheorem{claim}[theorem]{Claim}
\newtheorem{proposition}[theorem]{Proposition}
\newtheorem{corollary}[theorem]{Corollary}
\newtheorem{conjecture}[theorem]{Conjecture}
\newtheorem{axiom}[theorem]{Axiom}
\newtheorem{assumption}[theorem]{Assumption}

\theoremstyle{definition}
\newtheorem{definition}[theorem]{Definition}
\newtheorem{remark}[theorem]{Remark}
\newtheorem{example}[theorem]{Example}
\newtheorem{notation}[theorem]{Notation}

\newtheorem{exercise}[theorem]{Discussion topic}
\newtheorem{homework}[theorem]{Homework}
\newtheorem{problem}[theorem]{Problem}

\newtheorem{ack}{Acknowledgements}

\numberwithin{equation}{section}


% Mean
\def\Xint#1{\mathchoice
{\XXint\displaystyle\textstyle{#1}}%
{\XXint\textstyle\scriptstyle{#1}}%
{\XXint\scriptstyle\scriptscriptstyle{#1}}%
{\XXint\scriptscriptstyle\scriptscriptstyle{#1}}%
\!\int}
\def\XXint#1#2#3{{\setbox0=\hbox{$#1{#2#3}{\int}$ }
\vcenter{\hbox{$#2#3$ }}\kern-.6\wd0}}
\def\ddashint{\Xint=}
\def\dashint{\Xint-}

%\usepackage{color}
%\hypersetup{%
%    colorlinks=true, % make the links colored%
%    linkcolor=blue, % color TOC links in blue
%    urlcolor=red, % color URLs in red
%    linktoc=all % 'all' will create links for everything in the TOC
%Ning added hyperlinks to the table of contents 6/17/19
%}

% style=alphabetic
\usepackage[backend=bibtex,maxcitenames=50,maxnames=50]{biblatex}
\addbibresource{topics.bib}
\renewbibmacro{in:}{}
\DeclareFieldFormat{pages}{#1}

\begin{document}
\begin{abstract}
Topics exam, Fall 2021.
\end{abstract}

\maketitle

%%%%%%%%%%%%%%%%%%%%%%%%%%%%%%%%%%%%%%%%%%%%%%%%%%%%%%%

% \tableofcontents

\section{Introduction}
Let $M$ be an oriented Riemannian manifold of metric $g$ and dimension $d \geq 2$.

\begin{theorem}\label{main thm}
Let $u: M \to \RR$ be a function of least gradient, $d \leq 7$, and $A_y = \partial \{u > y\}$.
Then $(A_y)_{y \in \RR}$ is a lamination of $u$ by smooth minimal hypersurfaces, which are analytic if $g$ is.
\end{theorem}

If $M = \RR^d$, then Theorem \ref{main thm} is essentially a standard result; see \cite[Proposition 3.4]{górny2017planar} for an exposition in that case.
A proof of an analogous result for currents is given by \cite[\S5.3]{federer2014geometric}; our proof uses a similar strategy but is rather ``hands-on" in that it avoids the use of homological integration theory.

In the case of a surface, Theorem \ref{main thm} can be stated in a somewhat stronger form, just as in \cite[Corollary 3.5]{górny2017planar}.

\begin{theorem}\label{main crly}
Let $\overline \Sigma$ be a convex surface with boundary and suppose that $u: \Sigma \to \RR$ is a function of least gradient defined on the interior $\Sigma$ of $\overline \Sigma$.
Then, if $A_y = \partial \{u > y\}$, $(A_y)_{y \in \RR}$ extends to a geodesic lamination of $\overline \Sigma$.
\end{theorem}




%%%%%%%%%%%%%%%%%%%%%%%%%%%%%%%%%%%%%%%%%%%%%%%

\subsection{Outline of the paper}
We begin with the preliminaries in Section \ref{RiemMeasureThy}, which records basic results on functions of approximately least gradient, including a generalization of Miranda's theorem \cite[Teorema 3]{Miranda67} on the stability of functions of least gradient.
We also show that the reduced boundary of a Caccioppoli set is metric-independent.

We are then ready to prove Theorems \ref{main thm} and \ref{main crly}.
In Section \ref{inequalities} we prove a monotonicity formula and an isoperimetric inequality, which show that the blowup of a set of least perimeter is as smooth as possible, and in Section \ref{DGL section}, we prove a de Giorgi-type lemma.
Then in Section \ref{proof of main thm} we combine the results of the previous two sections to prove Theorems \ref{main thm} and \ref{main crly}.

In Appendix \ref{coarea section} we deduce a coarea formula that is used throughout the paper.

%%%%%%%%%%%%%%%%%%%%%%%%%%%%%%%%%%%%%%%%%%%%%%%%

\subsection{Acknowledgements}
I would like to thank Georgios Daskalopoulos for suggesting this project and for many helpful discussions.

%%%%%%%%%%%%%%%%%%%%%%%%%%%%%%%%%%%%%%%%%%%%%%%%%%%%%%%%%%%%%%%%%%%%%%%%%%%%%%%%%%%%%%%%%

\section{Riemannian measure theory}
\subsection{Notation and conventions}
\begin{notation}
If $F$ is a presheaf of function spaces, we write $u \in F_l(U)$ to mean that for every $V \Subset U$, $u \in F(V)$.
We write $u \in F_c(U)$ to mean that $u \in F(U)$ and $\supp u \Subset U$.
\end{notation}

\begin{notation}[volume forms]
We reserve the letter $d$ for dimension or exterior differentiation, and so to avoid awkwardness such as $\int |du| dV$ we write $\vol$ for the Riemannian volume form.
If $N$ is a closed submanifold we write $\vol_N$ to indicate the pullback of $\vol$ along the inclusion map $M \to N$.
\end{notation}

\begin{notation}[vector bundles]
Let $E$ be a vector bundle, which we will always assume is normed, with dual $E'$.
If $u,v$ are sections of $E',E$ respectively, we write $(u, v)$ for their fiberwise pairing, which is a function $M \to \RR$.
We write $\langle u, v\rangle$ or $\int_M (u, v) ~\vol$, for their $L^2$-duality pairing, which is a real number.
If $u$ is a section of $E$, we write $u \prec U$ to mean that $||u||_{L^\infty} \lesssim 1$ and $\supp u \Subset U$.
\end{notation}

\begin{definition}
Let $u$ be a smooth section.
If, if $g$ is analytic then $u$ is analytic, then $u$ is \dfn{as smooth as possible}.
\end{definition}

\subsection{The Bochner integral}
Let $F$ be a separable Fr\'echet space over $\CC$, and $(\Omega, P)$ a measure space.
We can then define the \dfn{Bochner integral} of a $P$-measurable function $f: \Omega \to F$, which we write as $\int_\Omega f ~dP \in F$.
See \cite{Rieffel70}, \cite{MO47721}, and \cite[Chapter V]{yosida2012functional}.
We recall a few facts that we will need:

\begin{theorem}[Pettis]
Let $f: \Omega \to F$ be any function.
Then $f$ is $P$-measurable iff for every $X \in F'$, $\omega \mapsto \langle f(\omega), X\rangle$ is $P$-measurable.
In this case,
$$\left\langle \int_\Omega f ~dP, X\right\rangle = \int_\Omega \langle f(\omega), X\rangle ~dP(\omega).$$
If $F = \CC^r$, then the Bochner integral is just the componentwise Lebesgue integral.
\end{theorem}

\subsection{Bundle-valued Radon measures}
Let $F \to M$ be a normed vector bundle of rank $r$.
We equip the space $C(K, F)$ of continuous sections of $F$ on a compact set $K$ with its supremum norm.
The Banach spaces $C(K, F)$ form an inverse system with respect to restriction and therefore define the topological vector space
$$C_0(M, F) = \varprojlim C(K, F).$$

According to the Riesz-Markov theorem, the space of $\CC^r$-valued Radon measures on $M$ is canonically identified with $C_0(M, (\CC^r)')'$, thus we define:

\begin{definition}
The topological dual space $\mathcal R(M, F) = C_0(M, F')'$ is the space of \dfn{$F$-valued Radon measures} on $M$.
\end{definition}

If we write $\mathcal R(U, F)$ to mean $\mathcal R(U, F|U)$, we equip $\mathcal R(U, F)$ with the weak topology of measures.
Then $\mathcal R(U, F)$ is a separable Fr\'echet space, with seminorms $|\langle \cdot, f_j\rangle|$ where $(f_j)$ is a countable basis for a dense subspace of $C_0(U, F)$.
Thus $\mathcal R(\cdot, F)$ is a sheaf of separable Fr\'echet spaces.

\begin{definition}
If $\omega \in \mathcal R(M, F)$ we define the \dfn{total variation} $|\omega|$ to be the positive Radon measure such that on every open set $U$,
$$|\omega|(U) = \sup_{X \prec U} \langle \omega, X\rangle,$$
where $X$ ranges over $C_0(M, F')$.
\end{definition}

Since $\RR_+$ acts on $F$ by scalar multiplication, we can define the \dfn{sphere bundle} $SF = (F \setminus 0)/\RR_+$.
Since $F$ has a norm, $SF$ is naturally identified with the bundle of elements of $F$ with length $1$.

\begin{proposition}[Riesz-Markov representation]\label{HanhJordan}
Let $\omega$ be a $F$-valued Radon measure and let $\mu = |\omega|$.
Then there exists a $\mu$-measurable section $f$ of $SF$ such that for every section $X \in C_0(M, F', \mu)$,
\begin{equation}\label{RNy formula}
\langle \omega, X\rangle = \int_M (f, X) ~d\mu.
\end{equation}
Furthermore, $f$ is unique up to a $\mu$-null set, and does not depend on the norm of $F$.
\end{proposition}
\begin{proof}
Fix an open cover $(U_i)$ of $M$ by charts which trivialize $F$, so that $U_i$ is precompact in $M$.
Let $(g_{ij})$ be the transition functions and $(g_{ij}')$ the induced transition functions for the dual bundle $F'$.
Then can view $\omega_i = \omega|U_i$ as an element of $C(U_i, (\CC^r)')'$, by the precompactness of $U_i$.
Hence by the Riesz-Markov theorem \cite[Theorem 4.14]{simon1983GMT}, there exists a $\mu$-measurable section $f_i$ of $SF$ for which (\ref{RNy formula}) holds for $\omega_i$, provided that $X \in C(U_i, (\CC^r)')$.

We now show that the $f_i$ are restrictions of a global section $f$, thus we must show $f_j = g_{ij} \circ f_i$ on $\CC^r$.
To this end, fix $X \in C_0(M, F)$ which is supported in $U_i \cap U_j$ and write $X_i \in C(U_i, (\CC^r)')$ for the trivialization of $X$ with respect to $U_i$.
Then $X_j = g_{ij}' \circ X_i$, and
\begin{align*}
\int_E (f_i, X_i) ~d\mu &= \langle \omega_i, X_i\rangle = \langle \omega_j, X_j\rangle = \int_E (f_j, X_j) ~d\mu = \int_E (f_j, g_{ij}' \circ X_i) ~d\mu.
\end{align*}

By Urysohn's lemma, $C_c(U_i \cap U_j, (\CC^r)', \mu)$ separates points in $L^1(U_i \cap U_j, \CC^r, \mu)$.
Therefore, since $X$ was arbitrary, $f_i = g_{ji} \circ f_j$; thus we obtain a unique global section $f$ of $SF$.

Finally, if we change the norm of $F$, replacing $|\cdot|$ with $|\cdot|'$, then we obtain a smooth function $h: F \to \RR_+$ so that if $v \in F_x$, then $|v|' = h(x, v)|v|$.
The change of norm gives us a new section $f'$ such that $f' = f/h(\cdot, f'(\cdot))$.
Thus $f'$ defines the same section of $SF$ as $f$.
\end{proof}

At this stage we have only defined $f$ as a $\mu$-equivalence class of sections of $SF$, so we now use the Lebesgue differentiation theorem to choose the ``correct" representative.
We state the differentiation theorem in a somewhat strange way, to ensure that the representative chosen is metric-independent.

\begin{definition}
A \dfn{Besicovitch cover} $\mathcal U$ of a metric space $X$ is a set of open balls, so that every $x \in X$ is the center of an element of $\mathcal U$.
The \dfn{Besicovitch number} $N \in \NN$ of $X$ is the best constant such that for every $x \in U$ and Besicovitch cover $\mathcal U$ of $B(x, 1/N)$, there exist $\mathcal U_1, \dots \mathcal U_N \subset \mathcal U$ such that $\bigcup_{n=1}^N \mathcal U_n$ is an open cover of $B(x, 1/N)$ and $\mathcal U_n$ is disjoint.
\end{definition}

It follows from the theory of \cite[\S2.8]{federer2014geometric} that for every Riemannian metric $g$, the Besicovitch number of $(M, g)$ is finite; \cite{Shi91} motivates why we restrict to small balls $B(x, 1/N)$.

For each $x \in M$, let $\mathcal A(x)$ denote the set of all pairs $(g, B, \varphi)$ where:
\begin{enumerate}
\item $g$ is a Riemannian metric on $M$,
\item $B$ is an open ball centered at $x$ with respect to $g$, and
\item $\varphi$ is a trivialization of $F$ over $B$.
\end{enumerate}
Then $\mathcal A(x)$ is a directed system, where the order is given by reverse inclusion of balls.
Given $(g, B, \varphi) \in \mathcal A(x)$, we obtain a $\mu$-measurable function $f_\varphi: B \to \CC^r$ obtained by trivializing the section $f$.
We define the average
$$f(g, B, \varphi) = \varphi^{-1}\left(\frac{1}{\mu(B)} \int_B f_\varphi ~d\mu\right),$$
which is a point in $F_x$.

\begin{proposition}[Lebesgue differentiation theorem]
Let $\mu$ be a Radon measure on $M$, let $f \in L^1_l(M, SF, \mu)$, and let
$$f(x) = \lim_{(g, B, \varphi) \in \mathcal A(x)} f(g, B, \varphi).$$
Then the limit defining $f(x)$ converges for $\mu$-almost every $x \in M$ to a point in the sphere $SF_x$.
\end{proposition}
If $f(x)$ exists and is $\in SF_x$, we call $x$ a \dfn{Lebesgue point} of the section $f$.
\begin{proof}
This is obvious if $f$ has a representative in $C_c(M, SF)$; besides, by a partition of unity argument, we may assume that $\mu$ has compact support.
We can then select $(f_n)$ in $C_c(M, SF)$ converging in $L^1(M, SF, \mu)$ and almost everywhere to $f$.
Setting $h_n = |f_n - f|$, we can define the average
$$h_n(g, B) = \frac{1}{\mu(B)} \int_B h_n ~d\mu,$$
which converges to $0$ in $L^1(M, \mu)$.

Fix $N \in \NN$ and let $\mathcal B_N$ be the set of Riemannian metrics with Besicovitch number $\leq N$.
This makes sense if we restrict to a neighborhood of the compact support of $\mu$.
For each metric $g \in \mathcal B_N$, we have the Hardy-Littlewood inequality \cite[Lemma 4.1.1a]{Ledrappier85}
\begin{equation}\label{HardyLittlewood}
||\sup_{r \in (0, 1/N)} h_n(g, B_g(\cdot, r))||_{L^{1, \infty}(M, \mu)} \leq N ||h_n||_{L^1(M, \mu)}.
\end{equation}
By (\ref{HardyLittlewood}) and the convergence $h_n \to 0$ in $L^1$,
$$\lim_{n \to \infty} ||\sup_{0 \in (0, 1/N)} h_n(g, B_g(\cdot, r))||_{L^{1, \infty}(M, \mu)} = 0$$
uniformly in $g \in \mathcal B_N$.
Therefore we may pass to a subsequence along which, for $\mu$-almost every $x$,
$$\lim_{n \to \infty} \sup_{(g, r) \in \mathcal B_N \times (0, 1/N)} h_n(g, B_g(x, r)) = 0.$$
By the triangle inequality, if
$$\mathcal A_N(x) = \{(g, B, \varphi) \in \mathcal A(x): g \in \mathcal B_N\},$$
then (after passing to a subsequence again)
$$\lim_{n \to \infty} \sup_{(g, B, \varphi) \in \mathcal A_N(x)} |f_n(g, B_g(x, r), \varphi) - f(g, B_g(x, r), \varphi)| = 0.$$
But $f_n(g, B, \varphi) \to f(x)$ everywhere, $f(x) \in SF_x$, and $SF_x$ is closed, so there exists a $\mu$-null set $Z_N$ such that outside of $Z_N$,
$$\lim_{(g, B, \varphi) \in \mathcal A_N(x)} f(g, B, \varphi) \in SF_x.$$
Taking $Z = \bigcup_{N \in \NN} Z_N$, we see that $Z$ is $\mu$-null, which was to be shown.
\end{proof}

\subsection{Differentiation and boundary}
In this section we fix a Riemannian metric.

\begin{definition}
A function in $L^1(M)$ has \dfn{bounded variation} if its distributional derivative is a $T'M$-valued Radon measure of finite total variation.
We write $BV$ for the presheaf of functions of bounded variation.
A \dfn{Caccioppoli set} is an open set whose indicator function has locally bounded variation.
\end{definition}

If $u$ is a function of bounded variation we write $du ~\vol$ for its derivative.

Sequences $(u_n)$ in $BV_l(M)$ with $u_n \to u$ in $L^1_l(M)$ satisfy the lower semicontinuity property
\begin{equation}
\label{RieszMarkovDistr}
\int_M |du| ~\vol \leq \liminf_{n \to \infty} \int_M |du_n| ~\vol.
\end{equation}
which follows by testing against smooth functions, and the forgetful map
\begin{equation}\label{Forget}
BV_l(M) \to L^1_l(M)
\end{equation}
is compact. We refer to \cite[Chapter 1]{Giusti77} for a review of the space $BV_l(M)$.
Our next result can be deduced by applying a partition of unity argument and then copying the proof of \cite[Teorema 1]{Miranda67} verbatim:

\begin{proposition}[trace theorem]\label{traces}
Let $U$ be an open set such that $N = \partial U$ is a Lipschitz hypersurface.
For every $u \in BV_l(M)$ there exists a trace $v \in L^1_l(N)$ such that for every $X \in C_c(M, TM)$,
\begin{equation}\label{Miranda IBP}
\int_U (du, X) ~\vol + \int_U u ~\mathcal L_X\vol = \int_N vg(X, \nu) ~\vol_N.
\end{equation}
Moreover, $v$ is determined by the germ of $u$ at $\partial U$.
If $u$ is an indicator function then so is $v$.
\end{proposition}

Let $U$ be a Caccioppoli set.
The notion of reduced boundary to $U$ was first introduced in \cite{deGiorgi55}; see \cite{Battista_2021} for an equivalent definition.
To construct it, let $\omega = d1_U ~\vol$, which by Proposition \ref{HanhJordan} can be expressed as $\omega = \normal \mu$, where $\normal$ is a section of $ST'M$ which is independent of $g$.

\begin{definition}
Let $U,\normal$ be as above.
The \dfn{reduced boundary} $\partial^* U$ of a Caccioppoli set $U$ is the set of Lebesgue points of $\normal$.
The \dfn{conormal $1$-form} to $\partial^* U$ is $\normal$.
The \dfn{tangent bundle} $T\partial^* U$ to $\partial^* U$ is the kernel bundle of $\normal$.
\end{definition}

The tangent bundle is well-defined and gives a measurable vector bundle of rank $d-1$ over $\partial^* U$, because $\normal$ is nonzero almost everywhere, and so has constant rank $1$.
We prefer to work with $\normal$ than $\normal^\sharp$, to obtain metric-independence.
In fact, metric-independence and well-known facts about the euclidean case \cite[Chapters 2-4]{Giusti77} \cite{deGiorgi55} imply:

\begin{proposition}\label{locality of Caccioppoli}
Let $U$ be a Caccioppoli set.
Then:
\begin{enumerate}
\item $\partial^* U$ is either empty or $d-1$-dimensional in the Hausdorff sense, and is rectifiable with respect to $d-1$-dimensional Hausdorff measure.
\item $\partial^* U$ is dense in the measure-theoretic boundary $\partial U$.
\item If $\normal$ extends to a continuous $1$-form on $\partial U$, then $\partial^* U = \partial U$ is a $C^1$ hypersurface.
\end{enumerate}
\end{proposition}

\begin{notation}
We write $|\partial^* U|$ for $\int_M |d1_U| ~\vol$.
This does not collide with the notation $|U|$ for the volume of $U$, since $U$ has Hausdorff dimension $d$.
\end{notation}

\subsection{The coarea formula} \label{coarea section}
Let $M = (M, g)$ be a Riemannian manifold.
Throughout this section we consider the superlevel sets $E_y = \{u > y\}$ of a function $u \in BV_l(M)$, and the resulting $T'M$-valued Radon measures
$$\omega(y) = d1_{E_y} ~\vol.$$
Let $\mu = |du| ~\vol$.

We first observe that for every $X \in \mathcal D(M, TM)$,
$$\langle \omega(y), X\rangle = -\int_{E_y} \mathcal L_X\vol$$
is measurable in $y$, since $E_y$ is monotone in $y$.
So by Pettis' theorem, $\omega$ is measurable in $y$ with respect to the weak topology of measures.

\begin{lemma}[coarea formula for measures]\label{Coarea1}
One has
$$du ~\vol = \int_{-\infty}^\infty \omega(y) ~dy.$$
\end{lemma}
\begin{proof}
Fix $X \in C_c(M, TM)$. We may assume that $u \geq 0$, and we must show
\begin{equation}
\label{gradient is integral of fibers}
\int_M (du, X) ~\vol = \int_{-\infty}^\infty \langle \omega(y), X\rangle ~dy.
\end{equation}
Since $u \geq 0$,
\begin{align*}
\int_M (du, X) ~\vol &= -\int_M u~\mathcal L_X\vol = -\int_M \int_0^{u(x)} dy ~\mathcal L_X\vol\\
&= -\int_0^\infty \int_{E_y} ~\mathcal L_X\vol ~dy = \int_0^\infty \langle \omega(y), X\rangle ~dy.
\end{align*}
by Fubini's theorem.
If $y < 0$ then $1_{E_y} = 1$ so $\omega(y) = 0$, so we conclude (\ref{gradient is integral of fibers}).
\end{proof}

Let $p: L \to M$ be the trivial line bundle with its induced metric $h$, and let $\eta$ be the volume form induced by $h$.
If $W$ is a vector field on $L$, we will write $W_1$ for the projection of $W$ onto $M$ and $W_2$ for its projection onto $\CC$.
Then Cartan's magic formula implies that if $W_2$ is constant, then
\begin{equation}
\label{Lie derivative computation}
\mathcal L_W\eta = \mathcal L_W\vol \wedge dy.
\end{equation}

\begin{lemma}\label{coarea converse}
Suppose that $W \in \mathcal D(L, TL)$ depends on a parameter $n \in \NN$, such that $W_2 = 0$ and for every $y \in \RR$, $X = W_1(\cdot, y)$ is a maximizing sequence for $\langle \omega(y), X\rangle$ subject to $X \prec U$.
Then
$$\int_{-\infty}^\infty \langle \omega(y), W(y)\rangle ~dy \leq \mu(U).$$
\end{lemma}
\begin{proof}
Let
$$E = \{(x, y) \in L: x \in E_y\}$$
be the undergraph of $u$.
By Fubini's theorem and (\ref{Lie derivative computation}),
\begin{align*}
\int_{-\infty}^\infty \langle \omega(y), W(y)\rangle ~dy &= -\int_{-\infty}^\infty \int_{E_y} \mathcal L_W\vol ~dy = -\iint_E ~\mathcal L_W\eta = \int_M (d1_E, W) ~\vol.
\end{align*}

Let $(u_m)$ be a mollification of $u$, so that $u_m \to u$ in the weak topology of distributions.
Then if $\chi$ is a cutoff, $\langle u_m, \chi\rangle \to \langle u, \chi\rangle$; taking a sequence of $\chi$ which increase to the indicator function of a compact set $K$, we conclude that $u_m \to u$ in $L^1(K)$, and hence $u_m \to u$ in $L^1_l$.

Let $E^{(m)}$ be the undergraph of $u_m$, and $E^{(m)}_y = \{u_m > y\}$.
Then for every test function $v$,
\begin{align*}
\langle 1_{E^{(m)}} - 1_E, v\rangle &= \int_{E^{(m)} \Delta E} v ~\vol \leq |(E^{(m)} \Delta E) \cap (\supp v \times \RR)| \cdot ||v||_{L^\infty}\\
&\leq ||v||_{L^\infty} \int_{\supp v} |u_m - u| ~\vol \to 0
\end{align*}
so $1_{E^{(m)}} \to 1_E$ in the weak topology of distributions.
Therefore
$$\lim_{m \to \infty} \int_M (d1_{E^{(m)}}, W) ~\vol = \int_M (d1_E, W) ~\vol.$$

Since $u_m$ is smooth, its graph $F_m = \partial E^{(m)}$ is a smooth manifold.
Let $\nu_m$ be the upwards unit normal field of $F_m$ and let $\vol_m$ be the volume form on $F_m$ induced by $\eta$.
Then
$$\langle d 1_{E^{(m)}}, W\rangle = -\iint_{E^{(m)}} \mathcal L_W\eta = -\int_{F_m} h(\nu_n, W) ~\vol_m.$$
Let $q_m = p|F_m$ and $Y_m$ be the vector field $(Y_m)_1 = -d u_m$, $(Y_m)_2 = 1$.
Since $F_m$ is a graph, $q_m: F_m \to M$ is a diffeomorphism, $(q_m)_*\nu_m = Y_m/|Y_m|$, and
$$(q_m)_* \vol_m = |Y_m| ~\vol.$$
Therefore
$$\int_{F_m} h(\nu_n, W) ~\vol_m = \int_M h(Y_m, W) ~\vol = \int_M g(d u_m, W_1) ~\vol = \int_M (d u_m, W_1) ~\vol.$$
Thus
\begin{align*}
|\langle d 1_E, W\rangle| &= \lim_{m \to \infty} |\langle d u_m, W_1\rangle| \leq \mu(U). \qedhere
\end{align*}
\end{proof}

\begin{proposition}[coarea formula for $BV_l$ functions]\label{Coarea2}
Let $u \in BV(M)$, let $E_y = \{u > y\}$, and let $\omega(y) = d1_{E_y} ~\vol$.
Then, if $\mu$ is the total variation of $du ~\vol$,
$$\mu = \int_{-\infty}^\infty |\omega(y)| ~dy.$$
\end{proposition}
\begin{proof}
By Lemma \ref{Coarea1} and the triangle inequality,
$$\mu \leq \int_{-\infty}^\infty |\omega(y)| ~dy.$$
So we just need to prove the converse.

Let $U \Subset M$.
Suppose that for every $y \in \RR$, $X = X^{(n)}_y$ is a maximizing sequence for $\langle \omega(y), X\rangle$ subject to $X \prec U$.
Since $\omega$ with respect to the weak topology of measures, for every $n$, $X^{(n)}_y(x)$ can be chosen to be measurable in $(x, y)$; indeed, we can take $X^{(n)}_y$ to be a smooth approximation to the Radon measure $d 1_{E_y}/|d 1_{E_y}|_{TV}$ in the weak topology of distributions, which is a product of the measurable functions $\omega$ and $y \mapsto 1/|\omega(y)|$.

By an approximation argument, we can find $W^{(n)} \in C_c(L, TL)$ such that $W^{(n)}_2 = 0$ and for every $y \in \RR$, $X = W^{(n)}_1(\cdot, y)$ is a maximizing sequence for $\langle d 1_{E_y}, X\rangle$ subject to $X \prec U$.
Let us now suppress the $n$ and write $W(y) = W^{(n)}(\cdot, y)$.

By Lemma \ref{coarea converse}, since $W$ has compact support, the integrand $\langle d 1_{E_y}, W(y)\rangle$ is uniformly bounded in $y$.
Therefore, by Fatou's lemma,
\begin{align*}
\int_{-\infty}^\infty |\omega(y)| ~dy &= \int_{-\infty}^\infty \lim_{n \to \infty} \langle \omega(y), W(y)\rangle ~dy \leq \liminf_{n \to \infty} \int_{-\infty}^\infty \langle \omega(y), W(y)\rangle ~dy \\
&\leq \mu(U). \qedhere
\end{align*}
\end{proof}

%%%%%%%%%%%%%%%%%%%%%%%%%%%%%%%%%%%%%%%%%%%%%%%%%%%%%%%%%%%

\subsection{Change of metric}\label{change of metric}
Let us investigate how the volume form depends on the choice of metric.
If $\vol$ is the volume form of $g$, $\vol'$ the flat volume form induced by exponential normal coordinates $x$ centered at $p$, we have
the Taylor expansion \cite[p59]{chow2006hamilton}
\begin{equation}\label{Taylor expansion of determinant}
\frac{\vol}{\vol'} = 1 - \frac{1}{3} \Ric_p(x, x) - \frac{1}{6} \nabla \Ric_p(x, x, x) + a_4x^4 + \cdots
\end{equation}
where $a_j$, $j \geq 4$, are functions of the Riemann tensor and its covariant derivatives.

Let $g_1,g_2$ be metrics defined near a point $p$ and let $\beta = (e_1, \dots, e_d)$ be an orthonormal basis for $T_pM$ with respect to $(g_1)_p$ and $(g_2)_p$.
If $\vol_i$, resp. $\Ric^{(i)}$, is the volume form, resp. Ricci tensor, associated to $g_i$, in the geodesic normal coordinate $x$ associated to $\beta$ in a small ball $B(p, \rho)$, then by (\ref{Taylor expansion of determinant}),
\begin{equation}\label{change of volume form}
\frac{\vol_2}{\vol_1} = \frac{1 - \Ric_p^{(2)}(x, x)/6 + O(x^3)}{1 - \Ric_p^{(1)}(x, x)/6 + O(x^3)} = 1 + O(\rho^2).
\end{equation}

Let $\zeta_0$ be the injectivity radius of
$$\exp_p: T_pM \to M.$$
Let $a_j$ be the Taylor coefficients in (\ref{Taylor expansion of determinant}) and let $k \geq 2$ be the least index $j \geq 1$ such that $a_j \neq 0$.
Then let $c = 2|a_k|$.
Since on $B(p, 1)$, $c|x|^k \leq c|x|^2$ and so
$$1 - \frac{c}{2}|x|^2 - O(x^{k+1}) \leq \frac{\vol}{\vol'} \leq 1 + \frac{c}{2}|x|^2 + O(x^{k+1}).$$
We select $\zeta < \zeta_0$ so small that on $B(p, \zeta) \cap B(p, 1)$, the term $O(x^{k+1})$ is at most $c/2$, so that
\begin{equation}\label{definition of c-zeta}
0.5 < 1 - c|x|^2 \leq \frac{\vol}{\vol'} \leq 1 + c|x|^2 < 2.
\end{equation}

By (\ref{Taylor expansion of determinant}), in the generic case $k = 2$, $c = |\Ric_p|/3$, which motivates the following definition:

\begin{definition}
Let $\zeta,c$ be as in (\ref{definition of c-zeta}).
We call $\zeta$ the \dfn{strong injectivity radius} of $p$ and $c$ the \dfn{Ricci-Taylor error}.
\end{definition}

Write
$$\eta'(u, U) = \inf_{v \in BV_c(U)} \int_U |d(u + v)| ~\vol'$$
for the analogue of $\eta$ with respect to the flat metric induced by normal coordinates.

\begin{lemma}\label{flattening of eta}
If $U \subseteq B(p, \zeta) \cap B(p, r)$ with $r < 1$, $U \Subset M$, then for every $u \in BV_l(M)$,
\begin{equation}\label{flattening of eta equation}
(1 - cr^2) \eta'(u, U) \leq \eta(u, U) \leq (1 + cr^2) \eta'(u, U).
\end{equation}
\end{lemma}
\begin{proof}
Let $x$ be a normal coordinate centered on $p$.
Choose $(u_n)$ in $BV_l(M)$ which is a minimizing sequence for $\int_U |du_n| ~\vol'$ subject to the trace condition $u_n|\partial U = u|\partial U$.
Thus $\int_U |du_n| ~\vol' \to \eta'(u, U)$ and by (\ref{definition of c-zeta}),
$$(1 - cr^2) \int_U |du_n| ~\vol' \leq \int_U |du_n| (1 - c|x|^2) ~\vol' \leq \int_U |du_n| ~\vol = \eta(u, U).$$
Taking $n \to \infty$ we get $(1 - cr^2) \eta'(u, U) \leq \eta(u, U)$.
The other estimate in (\ref{flattening of eta equation}) is similar.
\end{proof}

\section{Functions of least gradient}\label{RiemMeasureThy}
\begin{definition}\label{main definitions}
A function $u \in BV_l(M)$ has \dfn{least gradient} if for every $v \in BV_c(M)$ and $\supp v \subseteq U \Subset M$,
$$\int_U |du| ~\vol \leq \int_U |du + dv| ~\vol.$$
A Caccioppoli set $U$ has \dfn{least perimeter} if $1_U$ has least gradient.
\end{definition}

Functions of least gradient can be viewed as the correct notion of weak solution to the Dirichlet problem for the Euler-Lagrange equation
\begin{equation}\label{EulerLagrange}
\Div \frac{\grad u}{|\grad u|} = 0.
\end{equation}

Our next few results follow from Proposition \ref{traces} and the proofs of \cite[Teorema 2]{Miranda67} and \cite[Lemma 5.6]{Giusti77}.

\begin{proposition}[gluing]\label{gluing}
Let $N$ be a Lipschitz hypersurface which separates $M$ into $U_1,U_2$.
If $u_j \in BV(U_j)$ and $u \in L^1_l(M)$ is the function such that $u|U_j = u_j$, then $u \in BV(M)$.
Moreover,
\begin{equation}
\label{glued BV norm}
\int_N |du| ~\vol = \int_N |u_1 - u_2| ~\vol_N.
\end{equation}
\end{proposition}

\begin{notation}
If $u \in BV(M)$ and $U \Subset M$, we write
$$\eta(u, U) = \inf_{v \prec U} \int_U |d(u+v)| ~\vol$$
so that $u$ has least gradient iff $\eta(u, U) = \int_U |du| ~\vol$ for every $U$.
\end{notation}

\begin{lemma}[a priori estimates]\label{estimates on good set}
Let $u, v \in BV(M)$, let $U \Subset M$ have a Lipschitz boundary $N$. Then
\begin{equation}\label{a priori estimate 1}
|\eta(u, U) - \eta(v, U)| \leq \int_N |u - v| ~\vol_N.
\end{equation}
In particular
\begin{equation}\label{a priori estimate 2}
\eta(u, U) \leq \int_N |u| ~\vol_N.
\end{equation}
\end{lemma}

We will need the following theorem \cite[Theorem 6.2.2]{Simons68} \cite[Theorem A]{BOMBIERI1969}, which in particular shows that the assumption $d \leq 7$ in Theorem \ref{main thm} is sharp.
Recall that a \dfn{minimal cone} $C$ is a cone of least perimeter with vertex at the origin.

\begin{theorem}\label{minimal cones in R8}
The following are equivalent:
\begin{enumerate}
\item $d \leq 7$.
\item The boundary of every minimal cone in $\RR^d$ is $C^1$.
\item The boundary of every minimal cone in $\RR^d$ is a hyperplane.
\end{enumerate}
\end{theorem}

\subsection{The Miranda stability theorem}\label{MirandaStability}
The exponential pullback $\exp_p^* u$ of a function $u$ of least gradient defined near $p \in M$ need not have least gradient.
However, in a small ball $B$ around $p$, we will be able to show that $\eta(u, B) \approx |du|_{TV}(B)$ in a sense to be made precise later.
This observation motivates the following definition.

\begin{definition}
A sequence $(u_n)$ of functions in $BV(M)$ has \dfn{approximately least gradient} if
$$\limsup_{n \to \infty} \int_U |du_n| ~\vol \leq \liminf_{n \to \infty} \eta(u_n, U)$$
uniformly as $U$ ranges over open sets $\Subset M$.
\end{definition}

To study sequences of approximately least gradient, we need a semicontinuity theorem for the total variation, which in the euclidean case was shown by Miranda \cite[Teorema 3]{Miranda67}.

\begin{definition}
Let $(u_n)$ be a sequence in $BV_l(M)$ which converges in $L^1_l$ to $u$.
We say that a Lipschitz hypersurface $N$ \dfn{has no singularities} of $(u_n)$ if:
\begin{enumerate}
\item \label{cond1Mir} $\sup_n \int_N |du_n| ~\vol = 0$.
\item \label{cond2Mir} $(u_n)$ is bounded in $L^1(N, \vol_N)$.
\item \label{cond3Mir} $\int_N |du| ~\vol = 0$.
\item \label{cond4Mir} $u_n \to u$ in $L^1(N, \vol_N)$.
\end{enumerate}
We say that $N$ \dfn{has no singularities} of $u \in BV_l(M)$ if $N$ has no singularities of the sequence $u_n = u$.
By Condition $k$ we mean the $k$th bullet in the above list.
\end{definition}

\begin{lemma}\label{probabilistic method}
Let $(u_n)$ be a sequence in $BV_l(M)$ which converges in $L^1_l(U)$. Then:
\begin{enumerate}
\item \label{probabilistic balls} For every $x \in M$ and $R > 0$ such that $B(x, R) \Subset M$ and almost every $r \in (0, R]$, $\partial B(x, r)$ has no singularities of $(u_n)$.
\item \label{probabilistic hypersurfaces} For every $U \Subset M$ there exists $U \subseteq V \Subset M$ such that $\partial V$ has no singularities of $(u_n)$.
\end{enumerate}
\end{lemma}
\begin{proof}
We first prove (\ref{probabilistic balls}).
Let $r$ be drawn from $[R/2, R]$ uniformly at random; we claim that almost surely, $\partial B(x, r)$ has no singularities of $(u_n)$.
Let
$$A = \{s > 0: \int_{\partial B(x, s)} |du| ~\vol > 0\}.$$
Then
$$\sum_{s \in A} \int_{\partial B(x, s)} |du| ~\vol \leq \int_{\partial B(x, R)} |du| ~\vol < \infty$$
since $|du|$ is a Radon measure and $B(x, R) \Subset M$.
Therefore $A$ is countable,
%Let $A_n = \{|du_n|_{TV}(N) > 0\}$ and let $A_\infty = \{|du|_{TV}(N) > 0\}$.
%Then for every $n \in \NN \cup \{\infty\}$, writing $u_\infty = u$,
%$$\sum_{s \in A_n} |du_n|_{TV}(\partial B(x, s)) \leq |du_n|_{TV}(B(x, R)) < \infty$$
%since $|du_n|_{TV}$ is a Radon measure and $B(x, R) \Subset M$.
%Since each of the summands is nonzero by definition of $A_n$, it follows that $A_n$ is countable, and in particular null.
%Therefore Conditions \ref{cond1Mir} and \ref{cond3Mir} hold almost surely.
so Condition \ref{cond3Mir} holds almost surely.
We omit the proof that the other conditions hold almost surely as it is similar.

To prove (\ref{probabilistic hypersurfaces}), let $U \Subset W \Subset M$, and for every $x \in \partial U$, let $R_x \in (0, d(x, \partial W))$.
Then, by (\ref{probabilistic balls}), for every $x \in \partial U$, there exists $r_x \in (0, R_x)$ such that $\partial B(x, r_x)$ has no singularities of $(u_n)$.
Let $\mathcal U$ be the open cover of $\overline U$ given by the balls $B(x, r_x)$, as well as $U$ itself.
Since $\overline U$ is compact, there exists a finite subcover $\mathcal U_0$ of $\mathcal U$.
Let $V$ be the union of the sets in $\mathcal U_0$.
Then $\partial V$ is the boundary of a union of finitely many balls $B(x, r_x)$ whose boundaries have have no singularities, and therefore has no singularities.
\end{proof}

We recall that $BV_l(M)$ is not separable, so it will be useful to have a somewhat weaker topology on $BV_l(M)$, as follows:

\begin{definition}
A sequence of functions $(u_n)$ in $BV_l(M)$ converges \dfn{in total variation on sets with no singularities} to $u \in BV_l(M)$ if $u_n \to u$ in $L^1_l(M)$ and for every set $A \Subset M$ such that $\partial A$ has no singularities,
\begin{equation}\label{convergence in TV}
\lim_{n \to \infty} \int_A |du_n| ~\vol = \int_A |du| ~\vol.
\end{equation}
\end{definition}

\begin{proposition}[Miranda stability theorem]\label{Miranda convergence}
If a sequence of functions $(u_n)$ has approximately least gradient and converges in $L^1_l$, then its limit $u$ has least gradient, and $u_n \to u$ in total variation on sets with no singularities.
\end{proposition}
\begin{proof}
By Lemma \ref{probabilistic method} for every $U$ open $\Subset M$ we can find $U \subseteq V \Subset M$ such that $V$ is open and $\partial V$ has no singularities.

We first prove that $u \in BV_l(M)$.
Let $v_n = (1 - 1_U)u_n$, so $v_n \in BV_l(M)$ by Proposition \ref{gluing}.
Since $(u_n)$ has approximately least gradient, if $n$ is large enough then
$$\int_{\overline V} |du_n| ~\vol \leq \eta(u_n, \overline V) + 1 \leq \int_{\overline V} |du_n| ~\vol + 1.$$
So by Proposition \ref{gluing} and Condition \ref{cond1Mir},
$$\int_U |du_n| ~\vol \leq \int_{\overline V} |du_n| ~\vol + 1 = \int_{\partial V} |u_n| ~\vol_{\partial V} + 1.$$
Thus, by (\ref{RieszMarkovDistr}) and Condition \ref{cond2Mir},
$$\int_U |du| ~\vol \leq \limsup_{n \to \infty} \int_U |du_n| ~\vol \leq \limsup_{n \to \infty} \int_{\partial V} |u_n| ~\vol_{\partial V} + 1 < \infty.$$
Therefore $u \in BV_l(M)$.

Let $v$ be a perturbation of $u$, thus $v \in BV_l(M)$ and $u = v$ on $M \setminus U$.
Such a perturbation exists, since $u \in BV_l(M)$.
Let
$$v_n(x) = \begin{cases}
v(x), &x \in V\\
u_n(x), &x \notin V
\end{cases}.$$
By Proposition \ref{gluing} and Condition \ref{cond3Mir}, $v_n \in BV_l$ and
\begin{equation}\label{gluing vn}
\int_{\overline V} |dv_n| ~\vol = \int_V |dv| ~\vol + \int_{\partial V} |u - u_n| ~\vol_{\partial V}.
\end{equation}
Let $\varepsilon > 0$. Then for $n$ large enough, since $(u_n)$ has approximately least gradient and $u_n - v_n$ is trace-free,
$$\int_V |du_n| ~\vol \leq \int_{\overline V} |du_n| ~\vol \leq \eta(u_n, \overline V) + \varepsilon \leq \int_{\overline V} |dv_n| ~\vol + \varepsilon.$$
By (\ref{gluing vn}), it follows that
$$\int_V |du_n| ~\vol \leq \int_V |dv| ~\vol + \int_{\partial V} |u - u_n| ~\vol_{\partial V} + \varepsilon.$$
By Condition \ref{cond4Mir} and (\ref{RieszMarkovDistr}),
$$\int_V |du| ~\vol \leq \int_V |dv| ~\vol + \varepsilon.$$
Since $\varepsilon > 0$ and $u = v$ on $M \setminus U$, it follows that $\int_U |du| ~\vol \leq \int_U |dv| ~\vol + \varepsilon$.
Taking $\varepsilon \to 0$, it follows that $u$ has least gradient.

Finally we prove (\ref{convergence in TV}).
Let $(u_{n_\ell})_\ell$ be a subsequence of $(u_n)$ such that $\int_A |du_{n_\ell}| ~\vol$ is Cauchy in $\ell$.
By Lemma \ref{estimates on good set} and Condition \ref{cond4Mir},
$$\lim_{n \to \infty} |\eta(u, A) - \eta(u_n, A)| \leq \lim_{n \to \infty} \int_{\partial A} |u - u_n| ~\vol_{\partial A} = 0.$$
Since $(u_n)$ has approximately least gradient, for every $\varepsilon > 0$ and $n$ large enough, $\int_A |du_n| ~\vol \leq \eta(u_n, A) + \varepsilon$.
Thus by (\ref{RieszMarkovDistr}) and the fact that $(\int_A |du_{n_\ell}| ~\vol)_\ell$ is Cauchy,
\begin{align*}
\int_A |du| ~\vol &\leq \lim_{\ell \to \infty} \int_A |du_{n_\ell}| ~\vol \leq \lim_{\ell \to \infty} \eta(u_{n_\ell}, A) + \varepsilon\\
&= \eta(u, A) + \varepsilon = \int_A |du| ~\vol + \varepsilon
\end{align*}
where the last equality follows because $u$ has least gradient.
Since $\varepsilon$ and the subsequence $(u_{n_\ell})$ were arbitrary, the inequalities collapse to give (\ref{convergence in TV}).
\end{proof}

\begin{corollary}\label{level sets are minimal}
For every $u$ of least gradient, the superlevel sets $\{u > t\}$ have least perimeter.
\end{corollary}
\begin{proof}
In the proof of \cite[Theorem 1]{BOMBIERI1969}, replace \cite[Theorem 1.6]{Miranda66} with Proposition \ref{Coarea2} and replace \cite[Theorem 3]{Miranda67} with Proposition \ref{Miranda convergence}.
\end{proof}

\begin{corollary}\label{compactness}
Let $(u_n)$ be a sequence of indicator functions of approximately least gradient.
Then there is a subsequence of $(u_n)$ which converges almost everywhere and in total variation on sets with no singularities to the indicator function of a set of least perimeter.
\end{corollary}
\begin{proof}
If $n$ is large enough, then by Proposition \ref{traces}, for every $U \Subset M$,
$$\int_U |du_n| ~\vol \leq \eta(u_n, U) + 1 \leq |\partial U| + 1$$
which gives a uniform bound in $BV_l$.
Since the forgetful map (\ref{Forget}) is compact, a subsequence of $(u_n)$ converges to a function $u$ in $L^1_l$.
By Proposition \ref{Miranda convergence}, $u$ has least gradient and (\ref{convergence in TV}) holds.
By taking a further subsequence we can guarantee the convergence pointwise almost everywhere.
The convergence almost everywhere implies that there is a Caccioppoli set $U$ such that $u = 1_U$, which necessarily has least perimeter.
\end{proof}

%%%%%%%%%%%%%%%%%%%%%%%%%%%%%%%%%%%%%

\subsection{Monotonicity formulae}\label{inequalities}
The monotonicity formula
\begin{equation}\label{classic monotonicity formula}
\partial_r e^{Ar^2}r^{1 - d} |\partial^* U \cap B(p, r)| \geq 0
\end{equation}
for smooth minimal hypersurfaces on Riemannian manifolds incurs a multiplicative loss of $e^{Ar^2}$ \cite[\S7]{MarquesXX}.
We will mainly be interested in the difference $r^{1-d} |\partial^* U \cap B(p, r)| - \rho^{1-d} |\partial^* U \cap B(p, \rho)|$.
Writing out (\ref{classic monotonicity formula}),
$$e^{Ar^2}\partial_r(r^{1 - d} |\partial^* U \cap B(p, r)|) \gtrsim -r^{2 - d} |\partial^* E \cap B(p, r)|.$$
By the cusp estimate Proposition \ref{uniform density estimate} that we will prove shortly, it follows that
$$r^{1-d} |\partial^* U \cap B(p, r)| - \rho^{1-d} |\partial^* U \cap B(p, \rho)| \gtrsim \rho - r.$$
On the other hand, we will shortly show a monotonicity formula of the form
$$r^{1-d} |\partial^* U \cap B(p, r)| - \rho^{1-d} |\partial^* U \cap B(p, \rho)| \gtrsim -h \log \frac{r}{\rho}$$
where $h$ is a parameter that, in our application, will be comparable to $r^2$.
Moreover, in our application, we will have $\rho$ comparable to $r$, which gives us a gain of factor $r$ over the classical monotonicity formula.

\begin{lemma}\label{approximate monotonicity}
Let $E$ be a Caccioppoli set in $\RR^d$ equipped with its euclidean metric such that for some $h \geq 0$ and $R > 0$ and every $0 < r < R$,
$$|\partial^* E \cap B_r| \leq (1 + h)\eta(f, r).$$
Then for every $0 < \rho < r < R$, with $\alpha = \log(r/\rho)$,
\begin{align*}
&\left|r^{1 - d} |\partial^* E \cap B_r| - \rho^{1 - d} |\partial^* E \cap B_\rho|\right|^2 \\
&\qquad \lesssim_d (1 + h)(1 + \alpha + h\alpha^2) \left(r^{1 - d} |\partial^* E \cap B_r| - \rho^{1 - d} |\partial^* E \cap B_\rho| + d\omega_d h \alpha\right).
\end{align*}
\end{lemma}
\begin{proof}
Let $f = 1_E$.
We recall from Proposition \ref{estimates on good set} that $\eta(f, s) \leq |\partial B_s|$, so that we can define
\begin{align*}
\psi(s) &= \int_{B_s} |df|~\vol - \eta(f, s)\\
&\leq h\eta(f, s) \leq h|\partial B_s| \leq d\omega_d hs^{d - 1}.
\end{align*}
From \cite[Proposition 5.12]{Giusti77},
$$\left|r^{1 - d} \int_{B_r} df ~\vol - \rho^{1 - d} \int_{B_\rho} df ~\vol\right|^2 \leq PQ,$$
where
\begin{align*}
P &= 2r^{1 - d}(1 + (d - 1)\alpha) \int_{B_r} |df| + 2(d - 1)^2 \int_\rho^r \log \frac{s}{\rho} \psi(s) \frac{ds}{s^d},\\
Q &= r^{1 - d} \int_{B_r} |df| - \rho^{1 - d} \int_{B_\rho} |df| + (d - 1)\int_\rho^r \psi(s) \frac{ds}{s^d}.
\end{align*}
Evidently $\log(s/\rho) \leq \alpha$ so we estimate
$$\frac{1}{\alpha} \int_\rho^r \log \frac{s}{\rho} \psi(s) \frac{ds}{s^d} \leq \int_\rho^r \psi(s) \frac{ds}{s^d} \leq d\omega_d h \int_\rho^r \frac{ds}{s} = d\omega_d h\alpha.$$
Thus
$$\int_{B_r} |df|~\vol \leq (1 + h)\eta(f, r) \leq (1 + h)|\partial B_r| \leq d\omega_d(1 + h) r^{d - 1}.$$
These estimates immediately give
\begin{align*}
P &\lesssim_d (1 + h)(1 + \alpha + h\alpha^2),\\
Q &\leq r^{1 - d} \int_{B_r} |df|~\vol - \rho^{1 - d} \int_{B_\rho} |df|~\vol + d\omega_d h\alpha. \qedhere
\end{align*}
\end{proof}

\begin{lemma}\label{approximate monotonicity 2}
Let $E$ be a Caccioppoli set in $\RR^d$ such that for some $0 \leq h < 1$ and $R > 0$ and every $0 < r < R$,
$$|\partial^* E \cap B_r| \leq (1 + h)\eta(f, r).$$
Then for every $0 < r < R$,
$$\partial_r \left(r^{1 - d} |\partial^* E \cap B_r| + d\omega_d h \log r\right) \geq 0.$$
\end{lemma}
\begin{proof}
Dividing by both sides of Lemma \ref{approximate monotonicity} by $(1 + h)(1 + \alpha + h\alpha^2) \geq 0$,
$$r^{1 - d} |\partial^* E \cap B_r| - \rho^{1 - d} |\partial^* E \cap B_\rho| + d\omega_d h \alpha \geq 0.$$
Writing out $\alpha = \log r - \log \rho$ we now see the claim.
\end{proof}

\subsection{Ruling out cusps}
We now show that a hypersurface which is approximately minimal cannot have ``cusp"-like singularities.

\begin{proposition}\label{uniform density estimate}
Suppose that $E$ is a Caccioppoli set in $M$, $p \in \partial^* E$, and $r_0 > 0$ is smaller than the strong injectivity radius of $p$.
If for every $r \in (0, r_0)$,
\begin{equation}\label{density estimate hypothesis}
|\partial^* E \cap B(p, r)| \leq 2\eta(1_E, B(p, r)),
\end{equation}
then for every such $r$,
\begin{align*}|E \cap B(p, r)|, ~|(M \setminus E) \cap B(p, r)| &\gtrsim r^d\\
|\partial^* E \cap B(p, r)| &\gtrsim r^{d - 1}.\end{align*}
\end{proposition}
\begin{proof}
Let $q = d/(d-1)$ and write $B_r = B(p, r)$, $f_r = 1_{E \cap B_r}$, $F(r) = |E \cap B_r|$, so $F'(r) = |E \cap \partial B_r|$.
Sobolev and Poincar\'e inequalities hold for $BV$ functions on $\RR^d$ \cite[\S5.6.1]{evans1991measure}.
By (\ref{definition of c-zeta}) and Sobolev embedding,
\begin{align*}
F(r)^{1/q} &= ||f_r||_{L^q(M)} \lesssim \int_M |df_r| ~\vol = |\partial^*(E \cap B_r)|\\
&\leq |\partial^* E \cap B_r| + |E \cap \partial B_r|.
\end{align*}
Therefore, by (\ref{density estimate hypothesis}),
$$|E \cap B_r|^{1/q} \leq 2\eta(1_E, B_r) + F'(r).$$
The trace of $1_E$ on $\partial B_r$ is equal to the trace of a function which is supported in an arbitrarily small neighborhood of $E \cap \partial B_r$, so
$$\eta(1_E, B_r) \leq |E \cap \partial B_r| = F'(r)$$
and hence
$$F(r)^{1/q} \lesssim F'(r).$$
TODO: Bound this with reverse Gr\"onwall

$F(r) = |E \cap B_r$
Since $|E \cap \partial B(p, r)| = \partial_r F(r)$,
$$F(r)^{1/q} \lesssim_d F'(r) \leq F'(r).$$
This gives the estimate on $|E \cap B(p, r)|$.
Since $M \setminus E$ shares its perimeter with $E$, the same argument with $M \setminus E$ replacing $E$ gives the analogous bound on $|(M \setminus E) \cap B(p, r)|$.

Let $f = 1_{E \cap B(p, r)}$ and let
$$[f] = \frac{1}{|B(p, r)|} \int_{B(p, r)} f ~\vol'.$$
By (\ref{definition of c-zeta}),
$$|\partial^* (E \cap B(p, r))| \geq 0.5 |\partial^* (E \cap B(p, r))|' = 0.5 \int_{B(p, r)} |df| ~\vol'.$$
By the $BV$ Poincar\'e inequality \cite[\S5.6.1]{evans1991measure}, it follows that
\begin{align*}
|\partial^* (E \cap B(p, r))| &\gtrsim_d ||f - [f]||_{L^1(B(p, r), \vol')} \\
&= ||1 - [f]||_{L^1(B(p, r) \cap E, \vol')} + ||[f]||_{L^1(B(p, r) \setminus E, \vol')}.
\end{align*}
From (\ref{definition of c-zeta}),
$$|E \cap B(p, r)| \leq 2\omega_{d - 1} r^{d - 1} [f] \leq 4|E \cap B(p, r)|,$$
so (TODO: This is wrong!)
\begin{align*}
|\partial^* (E \cap B(p, r))| &\gtrsim_d ||1 - [f]||_{L^1(B(p, r) \cap E, \vol')} + ||[f]||_{L^1(B(p, r) \setminus E, \vol')}\\
&\geq 0.5 (1 - [f]) |B(p, r) \cap E| + 0.5 [f] |B(p, r) \setminus E|\\
&\geq 0.5 r^{1-d}|E \cap B(p, r)||E \setminus B(p, r)| \gtrsim_d r^{1 - d}. \qedhere
\end{align*}
\end{proof}

\subsection{Blowup of the reduced boundary}
Now let us study the blowup of $M$ at a point $p$ on the reduced boundary of a set $U$ of least perimeter.
In this regard, we will be interested in objects that depend on a small parameter $t > 0$, but may depend on a choice of subsequence of $t_n \to 0$.
Thus we will suppress all subindices and implicitly pass to subsequences whenever referring to a limit $t \to 0$.\footnote{This can be made rigorous as follows. Fix a nonprincipal ultrafilter $\mathbf p$ on $2^{\aleph_0}$, and \emph{define} any limit as $t \to 0$ to be a $\mathbf p$-limit. See \cite{Tao07} for details. TODO: Is this actually useful? When I do even take subsequences?}

\begin{proposition}\label{blowup theorem}
Fix $p \in M$ with $\zeta \in (0, 1)$ chosen to be at most the strong injectivity radius of $p$.
Suppose that $U$ is an open set with least perimeter in $B(p, \zeta)$ and $p \in \partial^* U$.
Let $A = {\exp_p}^* U$, $A_t = \{v \in T_pM: tv \in A\}$, and $u_t = 1_{A_t}$.
Then:
\begin{enumerate}
\item The sequence $(u_t)$ has approximately least gradient with respect to the flat metric on $T_pM$ and in fact satisfies the estimate
\begin{equation}\label{approximately least gradient target}|
\partial^* A_t \cap V|' \leq (1 + ct^2)\eta'(A_t, V) \leq \eta'(U_t, V) + ct^2|\partial^* V|'
\end{equation}
for every Caccioppoli set $V \Subset T_pM$.
\item There exists an indicator function $u_0 \in BV_l(B'(r_0))$ of least gradient such that $u_t \to u_0$ in $L^1_l$, almost everywhere, and in total variation on sets with no singularities.
\item Let $C = \{u_0 = 1\}$. Then $0 \in \partial C$ and, if $d \leq 7$, then $\partial C$ is a hyerplane.
\end{enumerate}
\end{proposition}
\begin{proof}
TODO: Update this proof to make it a hyperplane and also not suck.

Suppose that (\ref{approximately least gradient target}) holds.
Then $(u_t)$ clearly is a net of indicator functions of least gradient, so the existence of $u_0$ and $C$ is immediate from Corollary \ref{compactness}.
The analyticity of $\partial C$ then follows from \cite[Corollary 9.5]{Giusti77}, since $T_pM$ is flat, $d \leq 7$, and $C$ has least perimeter in $T_pM \cong \RR^d$.

Therefore we just have to prove the estimate (\ref{approximately least gradient target}), and that $0 \in \partial C$.

Let $B'_r$ denote the ball (with respect to the flat metric) of radius $r > 0$ centered at $0$ in $T_pM$; then ${\exp_p}_* B'_r = B(p, r)$ if $r < \zeta$.
Moreover, $A_t$ enjoys the scaling invariance
\begin{equation}\label{scaling of psi}
|\partial^* A_t \cap B'_r|' = t^{1 - d} |\partial^* A \cap B'_r|'.
\end{equation}
To prove (\ref{scaling of psi}), let $\mathscr H^{d - 1}$ denote $(d-1)$-dimensional Hausdorff measure with respect to the euclidean metric.
Then by \cite[Theorem 4.4]{Giusti77},
$$|\partial^* A \cap E|' = \mathscr H^{d - 1}(\partial^* A \cap E),$$
and $\mathscr H^{d - 1}$ scales as desired.

Let $V \Subset B'_\zeta$ be an arbitrary open Caccioppoli set.
We write $V_t = \{x \in \RR^d: tx \in V\}$, so that $V_{1/t} \subseteq B'_{t\zeta}$ and so by (\ref{definition of c-zeta}), $\vol/\vol' \approx 1 \pm ct^2$ on $V_t$.

\begin{claim}\label{blowup claim 1}
$(U_t)$ satisfies the estimate (\ref{approximately least gradient target}).
\end{claim}

We must show that for every $w \in BV_c(V)$\footnote{The definition of bounded variation only depends on the equivalence class of a measure with respect to mutual absolute continuity, and in particular does not depend on a choice of volume form.},
\begin{equation}\label{approximately least gradient target 2}
|\partial^* A_t \cap V|' \leq (1 + ct^2) \int_V |du_t + dw| ~\vol'.
\end{equation}
Indeed, if (\ref{approximately least gradient target 2}) holds, then we can take the infimum over all $w$ to get
$$|\partial^*AU_t \cap V|' \leq (1 + ct^2) \eta'(u_t, V)$$
and then complete the proof using Lemma \ref{estimates on good set} to bound $ct^2 \eta'(u_t, V) \leq ct^2 |\partial^* V|$.

We now prove (\ref{approximately least gradient target 2}).
Using (\ref{scaling of psi}, \ref{definition of c-zeta}),
$$t^{d - 1} | \partial^* A_t \cap V|' = |\partial^* A \cap V_{1/t}|' \leq (1 + ct^2) |\partial^* A \cap V_{1/t}|.$$
For a function $a$ we let $a_t$ denote the rescaling $a_t(x) = a(tx)$.
Since $U$ has least perimeter,
$$|\partial^* A \cap V_{1/t}| \leq \int_{V_{1/t}} |d(u + w_{1/t})| ~\vol = \int_{V_{1/t}} |d(u + w_{1/t})| \frac{\vol}{\vol'} ~\vol'.$$
Thus
$$|\partial^* A \cap V_{1/t}| \leq (1 + ct^2) \int_{V_{1/t}} |d(u + w_{1/t})| ~\vol'$$
and so
$$|\partial^* A_t \cap V|' \leq (t^{1 - d} + ct^{3 - d}) \int_{V_{1/t}} |d(u + w_{1/t})| ~\vol'.$$
Rescaling using (\ref{scaling of psi}), we obtain (\ref{approximately least gradient target 2}) and hence Claim \ref{blowup claim 1}.

\begin{claim}
$0 \in \partial C$.
\end{claim}

Owing to (\ref{approximately least gradient target}, \ref{definition of c-zeta}), if $r,t$ are small enough then
$$|\partial^* A_t \cap B'_r| \leq 1.5|\partial^* A_t \cap B'_r|' \leq 1.5(1 + ct^2) \eta'(u_t, B'_r) \leq 2 \eta(u_t, B'_r)$$
and hence by Proposition \ref{uniform density estimate},
$$|\partial C \cap B'(r)| \gtrsim r^{d - 1}.$$
Since $\partial C$ is a smooth $(d-1)$-dimensional hypersurface which is closed as a subset of $T_pM$, this is only possible if $0 \in \partial C$.
\end{proof}

One can show that just to show the existence of a (possibly singular) tangent cone, one only needs that $U$ is a Caccioppoli set.
This just requires a slight modification of our argument to show that $(u_t)$ is bounded in $BV_l$, as in the proof of \cite[Theorem 9.3]{Giusti77}.
However, we will never need this fact.

%%%%%%%%%%%%%%%%%%%%%%%%%%%%%%%%

\section{The minimal surface equation}
TODO: Exposit

\subsection{Tubular trivializations}
Up to this point we have carefully carried out constructions which are diffeomorphism invariant; this led to easy proofs of any relevant fact that could be reduced to its euclidean case.
At this point in the proof, however, we break diffeomorphism invariance.
We have three reasons to do this:
\begin{enumerate}
\item We need to analyze the minimal surface equation, and to do this we ust write a hypersurface as a graph. Thus we need to put a local product structure on $M$.
\item We need a well-defined notion of hyperplane, at least locally, on $M$, so that we can compare minimal surfaces to hyperplanes; this is the essence of our bootstrapping argument.
\item In the course of the above analysis, we will obtain certain linear elliptic operators $P$ which depend on the curvature tensor $\Riem$, and so we want to choose coordinates in which $\Riem$ is so close to vanishing that $P$ can be well-approximated by the classical Laplacian.
\end{enumerate}
So, we now need to choose a suitable atlas on $M$ and suitable local trivializations of $TM$.
We call these coordinates ``tubular trivializations".

Let $\DD^{d - 1}$ denote the unit ball in $\RR^{d - 1}$ and let $e_j$ be the $j$th coordinate vector in $\RR^d$.

\begin{definition}
A tuple $V = (V, y, z, \psi)$ is called a \dfn{tubular trivialization} of $TM$ based at $(x, v) \in STM$ if $V$ is an open neighborhood of $x$ with the following properties.
Let $\gamma: J \to M$ be an injective geodesic such that $\gamma(0) = x$ and $\gamma'(0) = v$.
\begin{enumerate}
\item (Coordinates) We have a diffeomorphism
$$(y, z): V \to \DD^{d - 1} \times J.$$
\item (Centering) $\{y = 0\}$ is the image of $\gamma$ and $\{z = 0\}$ contains $x$.
\item (Orthogonality) The exponential pullback of $\{z = 0\}$ is the orthocomplement of $v$ in $T_x V$.
\item (Trivial tangent bundle) We have a trivialization
$$(y, z, \psi): TV \to \DD^{d - 1} \times J \times \RR^d$$
such that $\gamma'(z) \mapsto e_1$.
\item (Isometry) $\psi$ is an isometry on $T_xV$.
\item (Approximately euclidean) The metric $g$ takes the form $g = I + O(|y|^2 + |z|^2)$ when written out in the above coordinate frame.
\end{enumerate}
\end{definition}

\begin{lemma}
For every $(x, v) \in STM$ there exists a tubular trivialization of $TM$ based at $(x, v)$.
\end{lemma}
\begin{proof}
Let $\gamma: J \to M$ be an injective geodesic such that $\gamma(0) = x$ and $\gamma'(0) = v$.
By definition of normal coordinates, we can choose normal coordinates $(y, z): V \to \RR^{d - 1} \times J$ based at $x$ so that $\gamma$ is the image of $\{y = 0\}$ and $z(x) = 0$.
By restricting the codomain of $z$ we can then impose that $(y, z)$ is a diffeomorphism $V \to J \times \DD^{d - 1}$.
The isometry, orthogonality, and approximately euclidean properties follow from the fact that the normal coordinates are based at $x$.
Finally, we can smoothly perturb the coordinate basis vectors in $T_xM$ to coordinate basis vectors in $T_{x'}M$, $x' \in V$, so that $\gamma'(z) \mapsto e_1$.
\end{proof}

If $F$ is any Borel subset of $V$ and $\omega$ is a $1$-form-valued Radon measure on $F$ such that $\int_F |\omega| < \infty$, we can make sense of the integral $\int_F \omega$ as a covector based at $x$.
Indeed, $\psi$ induces coordinate vector fields $X_j = \psi^* e_j$ on $V$, and we can define
$$\int_F \omega = \langle \omega, X_j\rangle ~d\xi^j$$
where $\xi^j$ is dual to $X_j$.
The reader should note carefully that while we do have the triangle inequality
\begin{equation}\label{excess triangle inequality}
\left|\int_F \omega\right| \leq \int_F |\omega|,
\end{equation}
the quantity $\int_F \omega$ depends strongly on the choice of trivialization.

\subsection{The surface area Lagrangian}
Let $U$ be a set of locally finite perimeter, and suppose that $N = \partial^* U$ is $C^1$.
If $I$ is chosen small enough, the implicit function theorem and the fact that $\normal_U(x)^\sharp = X_1$ implies that there exists a $C^1$ function $f: \DD^{d - 1} \to J$ with $N = \{z = f(y)\}$, hence $U = \{z < f(y)\}$, and $df(0) = 0$.
Thus $\Lambda(U, x, \rho) = 0$ is equivalent to $df = 0$ on a suitable neighborhood of $0$, and we have a diffeomorphism $\tilde f: \DD^{d - 1} \to N$.
After extending $\normal_U^\sharp$ to a unit vector field $X$ on $V$ by $X(y, z) = \normal^\sharp(y, 0)$, let $\mathscr L = \tilde f^* \iota_X \vol$, so that $\int_{\DD^{d - 1}} \mathscr L$ is the surface area of $N$.
To simplify $\mathscr L$, we note that along $N$,
$$X = \frac{*\bigwedge_{i < d} \partial_i \tilde f}{|*\bigwedge_{i < d} \partial_i \tilde f|}.$$

\begin{lemma}
Let $(Y_i)$ be the orthonormal frame on $V = \DD^{d - 1} \times J$, with $Y_i = \nabla y_i$ if $i < d$ and $Y_d = \nabla z$.
Then
$$*\bigwedge_{i<d} \partial^i \tilde f = Y_d - \sum_{i<d} \partial^i fY_i.$$
\end{lemma}
\begin{proof}
Since $Y$ is an orthonormal frame, the Hodge star can be computed from the formal determinant
\begin{align*}
*\bigwedge_{i < d} \partial^i \tilde f &=
\begin{vmatrix} \partial^1 \tilde f^1 & \cdots & \partial^1 \tilde f^d \\
& \vdots \\
\partial^{d - 1} \tilde f^1 & \cdots & \partial^{d - 1} \tilde f^d \\
Y_1 & \cdots & Y_d
\end{vmatrix} = \begin{vmatrix} 1 & 0 & \cdots & 0 & \partial^1 f \\
0 & 1 & \cdots & 0 & \partial^2 f \\
& & \vdots \\
0 & 0 & \cdots & 1 & \partial^{d - 1} f \\
Y_1 & Y_2 & \cdots & Y_{d - 1} & Y_d
\end{vmatrix}
\end{align*}
which can be simplified by induction. If $d = 2$ then we get
$$*\bigwedge_{i < d} \partial^i \tilde f = Y_2 - \partial^1 f Y_1.$$
Otherwise, we can write
\begin{align*}
*\bigwedge_{i < d} \partial^i \tilde f &= \begin{vmatrix}  1 & 0 & \cdots & 0 & \partial_2 f \\
0 & 1 & \cdots & 0 & \partial^3 f \\
& & \vdots \\
0 & 0 & \cdots & 1 & \partial^{d - 1} f \\
Y_2 & Y_3 & \cdots & Y_{d - 1} & Y_d
\end{vmatrix} + (-1)^{d - 1} \partial^1 f
\begin{vmatrix}
0 & 1 & \cdots & 0 & 0 \\
& & \vdots \\
0 & 0 & \cdots & 1 & 0 \\
0 & 0 & \cdots & 0 & 1 \\
Y_1 & Y_2 & \cdots & Y_{d - 2} & Y_{d - 1}
\end{vmatrix}\\
& q= Y_d - \sum_{1 < i < d} \partial_if Y_i + (-1)^{d - 1} (-1)^d \partial_1 f Y_1 \\
&= Y_d - \partial_if Y_i. \qedhere
\end{align*}
\end{proof}

From the above computation we obtain
$$X = \frac{Y_d - \partial^i f Y_i}{\sqrt{1 + |\nabla f|_{\tilde g}^2}},$$
where $\tilde g_{ij} = g(Y^i, Y^j)$.
Thus
\begin{align*}
\mathscr L &= \sqrt{\frac{\det \tilde g}{1 + |\nabla f|_{\tilde g}^2}} \tilde f^* \left(\iota_{Y_d}(dz \wedge dy) - \partial^i f \iota_{Y_i}(dz \wedge dy) \right)\\
&= \sqrt{\frac{\det \tilde g}{1 + |\nabla f|_{\tilde g}^2}} \tilde f^* \left(dy + (-1)^i \partial^i f ~dz \wedge dy^1 \wedge \cdots \wedge \widehat{dy^i} \wedge \cdots dy^{d - 1}\right).
\end{align*}
Since $dz = \partial_i f(y) ~dy^i$, we conclude
\begin{align*}
\mathscr L &= \sqrt{\frac{\det \tilde g}{1 + |\nabla f|_{\tilde g}^2}} (1 + |\nabla f|_{\tilde g}^2) ~dy\\
&= \sqrt{\det(\tilde g))(1 + |\nabla f|_{\tilde g}^2)} ~dy.
\end{align*}
TODO: Check this computation, this is important.

Since $\mathscr L$ is the Lagrangian for the minimal surface equation on $V$, we can now read off some properties of sets of least perimeter from $\mathscr L$.

\begin{proposition}[de Giorgi-Nash-Moser theory]\label{regularity of reduced boundary}
Let $U$ be a set of least perimeter with conormal $1$-form $\normal$.
If $\normal$ is continuous, then $\partial U$ is as smooth as possible.
\end{proposition}
\begin{proof}
According to Proposition \ref{locality of Caccioppoli}, $\partial U$ is $C^1$, so we can cover $\partial U$ by tubular trivializations of $TM$.
In each such trivialization, $\partial U$ is the graph of a $C^1$ function which minimizes $\int_{\DD^{d - 1}} \mathscr L$.
Since $\mathscr L$ is uniformly convex and as smooth as possible, de Giorgi-Nash-Moser theory implies that $f$ is as smooth as possible \cite{morrey2009multiple}.
\end{proof}

\subsection{Linearized elliptic equations}


%%%%%%%%%%%%%%%%%%%%%%%%%%%%%%%

\section{de Giorgi's lemma}
In this section we prove a bootstrapping lemma... history...
it is necessary to break diffeomorphism-invariance so that we have a well-defined notion of coordinate hyperplane, and so we will need to construct a suitable atlas on a neighborhood of $\partial^* U$ before we can even state the de Giorgi lemma... TODO: Move this exposition to the previous section...

\begin{definition}
Let $U$ be a set of locally finite perimeter, let $x \in \partial^* U$, and choose a tubular trivialization $V$ of $TM$ based at $(x, \normal_U(x)^\sharp)$ such that $|\partial^* U \cap V| < \infty$.
If $\rho > 0$ and $B(x, \rho) \subseteq V$, we define the \dfn{excess} of $U$ at $\rho$ with respect to $V$ to be
$$\Lambda(U, x, \rho) = \rho^{1 - d}\int_{B(x, \rho)} |d1_U| ~\vol - \rho^{1 - d}\left|\int_{B(x, \rho)} d1_U ~\vol\right|.$$
\end{definition}

By (\ref{excess triangle inequality}), $\Lambda(U, x, \rho)$ is nonnegative.
Intuitively, $\Lambda(U, x, \rho)$ measures how badly $\partial^*U \cap B(x, \rho)$ fails to be the hypersurface $\{z = 0\} \cap \{|y| < \rho\}$, which pulls back by the exponential map to a disc in a hyperplane.
Indeed, if $U = \{z < 0\}$ then the coordinate vector field $X_1 = \psi^* e_1$ is always perpendicular to $\partial^* U$ and hence $X_1 = \normal_U^\sharp$. This implies $|\int_{B(x, \rho)} d1_U ~\vol| = \int_{B(x, \rho)} |d1_U| ~\vol$.
Conversely, if $U \neq \{z < 0\}$, then since $\normal_U(x)^\sharp = X_1$, $\normal_U$ cannot be constant and so must, when integrated against $X$, experience some cancellation; this forces $\Lambda(U, x, \rho) > 0$ for $\rho$ small enough.

\begin{proposition}\label{bootstrap minimal surfaces}
Let $U$ be a set of least perimeter, let $x \in \partial^* U$, and choose a tubular trivialization of $TM$ based at $(x, \normal_U(x)^\sharp)$.
Then for every $\varepsilon > 0$ there exist $\sigma, \rho_0 > 0$, such that for every $\rho \in (0, \rho_0)$ such that $\Lambda(U, x, \rho) < \sigma$,
$$\Lambda\left(U, x, \frac{\rho}{2 + \varepsilon}\right) < \frac{\Lambda(U, x, \rho)}{2}.$$
\end{proposition}

\subsection{Estimates on elliptic equations}
In this subsection, we fix $\ell \geq 1$, not necessarily equal to $d$, and prove the following proposition.

\begin{proposition}\label{bootstrap elliptic}
Let $\varepsilon > 0$, let $P$ be a uniformly elliptic operator with smooth coefficients on $B_{2(1 + \varepsilon)}\rho$, a ball centered on the origin in $\RR^\ell$, such that $P$ admits a divergence form
$$P = -\Div A \nabla + b \cdot \nabla + c$$
such that $A(0) = 1$, $A$ is symmetric and has a critical point at $0$, and $b(0) = c(0) = 0$.
If $\rho > 0$ is small enough depending on $P, \varepsilon$, $Pu = 0$ on $B_{2(1 + \varepsilon)}\rho$, and $0$ is a critical point of $u$, then
$$||du||_{L^2(B_\rho)}^2 \leq 2^{-\ell-2}||du||_{L^2(B_{2(1 + \varepsilon)\rho})}^2.$$
\end{proposition}

Before proving Proposition \ref{bootstrap elliptic} we note how it will be used in our application:

\begin{corollary}\label{bootstrap Laplace-Beltrami}
Let $N$ be a $\ell$-dimensional Riemannian manifold.
Let $\varepsilon > 0$, and suppose that $x \in N$, and let $\rho > 0$ be small enough depending on $x, \varepsilon$.
For every harmonic function $u$ with a critical point at $x$,
$$||du||_{L^2(B(x, \rho))}^2 \leq 2^{-\ell-2} ||du||_{L^2(B(x, 2(1 + \varepsilon)\rho))}.$$
\end{corollary}
\begin{proof}
By working in normal coordinates centered on $x$, we may assume that $x = 0$ in $N = \RR^\ell$ and $g^{ij} = \delta^{ij} + O(|x|^2)$ is symmetric.
Since the Laplace-Beltrami operator is defined by
$$\Delta = \frac{1}{\sqrt{\det g}} \partial_i \sqrt{\det g} g^{ij} \partial_j,$$
if we set $P = -\Delta$, then $A^{ij} = -g^{ij}$, $c = 0$, and
$$b^j(0) = \partial_i \log \sqrt{|\det g(0)|} g^{ij}(0) = 0,$$
so the claim follows from Proposition \ref{bootstrap elliptic}.
\end{proof}

We begin the proof of Proposition \ref{bootstrap elliptic} by first demonstrating it in a special case: the Laplacian on $\RR^\ell$.

\begin{lemma}[estimates on the Laplace operator]\label{bootstrap Laplace}
Suppose that $\Delta u = 0$ on $B_{2\rho}$.
Then
$$||du||_{L^2(B_\rho)}^2 \leq \frac{||du||_{L^2(B_{2\rho})}^2}{2^{d + 2}} + |du(0)|^2 \cdot |B_{2\rho} \setminus B_\rho|.$$
\end{lemma}
\begin{proof}
Since the commutator $[\Delta, d] = 0$, $du$ is harmonic and thus satisfies the mean value property.
Now the claim follows from \cite[Lemma 4.1]{Miranda66}.
\end{proof}

To prove an analogous result for $P$, we apply Taylor's theorem to the hypotheses on $P$ to deduce that
\begin{equation}\label{Taylor coefficients are small}
|A(x) - 1| + |b(x)| + |c(x)| \lesssim \rho.
\end{equation}
This suggests that we should approximate $P$ by the Laplace operator.
To accomplish this, we introduce the trace operator
\begin{align*}
T: H^s(B_{2(1 + \varepsilon)\rho}) &\to H^{s-.5}(\partial B_{2(1 + \varepsilon/2)\rho})
\end{align*}
whenever $s > .5$.
Scaling considerations show that
$$||T||_{H^s(B_{2(1 + \varepsilon)\rho}) \to H^{s - .5}(\partial B_{2(1 + \varepsilon/2)\rho})} \sim \rho^{-1/2}.$$
Let
$$G_P, G_{-\Delta}: H^{s - .5}(\partial B_{2(1 + \varepsilon/2)\rho}) \to H^{s + 1.5}(B_{2(1 + \varepsilon/2)\rho})$$
be the operators which solve the Dirichlet problems for $P,-\Delta$.
Scaling considerations show that $G_P, G_{-\Delta} \sim \rho^{1/2}$ in $H^{s - .5} \to H^{s + 1.5}$.

If $Pu = 0$, then $G_PTu = u|B_{2(1 + \varepsilon/2)\rho}$, while $G_{-\Delta}Tu$ is a harmonic function on $B_{2(1 + \varepsilon/2)\rho}$ which approximates $u$ in the following sense.

\begin{lemma}[harmonic approximation]\label{approx harmonic}
If $Pu = 0$, then for every $s, t \in \RR$,
$$||G_{-\Delta}Tu - u||_{H^t(B_{2(1 + \varepsilon/2)})} \lesssim_{s, t} \rho ||u||_{H^s(B_\rho)}.$$
\end{lemma}
\begin{proof}
We prove this by comparing the Green's functions of $-\Delta$ and $P$. Let
$$P^* = -\Div A \nabla - b \cdot \nabla + (c - \Div b)$$
be the formal adjoint to $P$, so that
\begin{equation}\label{formal adjoint equation}
\int_{B_{2(1 + \varepsilon/2)\rho}} uP^*v - vPu ~\vol = \int_{\partial B_{2(1 + \varepsilon/2)\rho}} uv b \cdot A\normal + v\partial_{A\normal} u - u\partial_{A\normal} v ~\vol_{\partial B_{2(1 + \varepsilon/2)\rho}}
\end{equation}
where $\normal$ is normal to $\partial B_{2(1 + \varepsilon/2)\rho}$.
If we define $F_P$ to solve the Dirichlet problem
\begin{equation}\label{defining the Green function}
\begin{cases}P^* F_P(\cdot, y) = \delta_y,\\
F_P|\RR^\ell \times \partial B_{2(1 + \varepsilon/2)\rho} = 0,
\end{cases}
\end{equation}
then (\ref{formal adjoint equation}) simplifies to
$$u(x) = \int_{\partial B_{2(1 + \varepsilon/2)\rho}} F_P(x, y)u(y) b(y) \cdot \normal - u(y) (\partial_{A \normal} F(x, \cdot))(y) ~\vol_{\partial B_{2(1 + \varepsilon/2)\rho}}(y).$$
In particular,
$$G_P(x, y) = F_P(x, y)b(y) \cdot \normal - (\partial_{A \normal} F_P(x, \cdot))(y).$$
The solution to the Dirichlet problem (\ref{defining the Green function}) is
$$F_P(x, y) = K_P(y - x) - K_P\left(2(1 + \varepsilon/2)\rho \frac{y}{|y|} - |y| \frac{x}{2(1 + \varepsilon/2)\rho}\right)$$
where $K_P$ is the newtonian kernel associated to $P$.
Thus, if $L = G_P - G_{-\Delta}$,
\begin{align*}
L(x, y) &= (b \cdot \normal(y) - \partial_{(A - 1)\normal})F_P(x, y) + \partial_\normal (F_P - F_{-\Delta})(x, y)
\end{align*}
where all normal derivatives are taken in $y$.

In order to use (\ref{Taylor coefficients are small}), we estimate, with $A_\rho = B_{2(1 + \varepsilon/2)\rho} \times \partial B_{2(1 + \varepsilon/2)\rho}$ and $w = (G_P - G_{-\Delta})v$,
\begin{align*}
||w||_{H^s(B_{2(1 + \varepsilon/2)\rho})}^2 &\lesssim_s ||(1 + |\Delta|)^{s/2} \int_{\partial B_{2(1 + \varepsilon/2)\rho}} L(\cdot, y) v(y) ~dy||_{L^2(B_{2(1 + \varepsilon/2)\rho})}^2\\
&\lesssim \rho^2 \left|\int_{B_{2(1 + \varepsilon/2)\rho}} (1 + |\Delta|)^{(s+1)/2} \int_{\partial B_{2(1 + \varepsilon/2)\rho}} F_{-\Delta}(x, y) v(y) ~dy ~dx\right|^2 \\
&\qquad + \rho^2 \iint_{A_\rho} |(1 + |\Delta|)^{(s+1)/2} (F_P(x, y) - F_{-\Delta}(x, y)) v(y)|^2 ~dy ~dx \\
&\qquad + \iint_{A_\rho} |(1 + |\Delta|)^{s/2} (F_P(x, y) - F_{-\Delta}(x, y)) v(y)|^2 ~dy ~dx \\
&\lesssim \rho^2 ||G_{-\Delta}v(x)||_{H^{s+1}(B_{2(1 + \varepsilon/2)\rho})}^2\\
&\qquad + \iint_{A_\rho} |(1 + |\Delta|)^{(s+1)/2} (F_P(x, y) - F_{-\Delta}(x, y)) v(y)|^2 ~dy ~dx\\
&\lesssim: \rho ||v||_{H^{s + 2}(\partial B_{2(1 + \varepsilon/2)\rho})}^2 + I.
\end{align*}
To control $I$, we apply Taylor's theorem to the map $Q \mapsto Q^{-1}$ centered on $-\Delta$ to conclude that
\begin{equation}\label{linearized inversion}
P^{-1} = -\Delta^{-1} - \Delta^{-1}(P + \Delta)\Delta^{-1} + \cdots.
\end{equation}
By (\ref{Taylor coefficients are small}),
$$||P + \Delta||_{H^r(B_{2(1 + \varepsilon/2)\rho}) \to H^{r+2}(B_{2(1 + \varepsilon/2)\rho})} \lesssim \rho,$$
so by (\ref{linearized inversion}),
$$||(K_P - K_{-\Delta})(\cdot, y)||_{H^{s + 1}(B_{2(1 + \varepsilon/2)\rho})} \lesssim_s \rho ||\Delta^{-2} \delta_y||_{H^{s + 1}(B_{2(1 + \varepsilon/2)\rho})} \lesssim \rho ||\delta_0||_{H^{s-3}}.$$
Moreover, $\delta_0 \in H^{s - 3}$ provided that
\begin{equation}\label{delta regularity}
s < 3 - \frac{\ell}{2}.
\end{equation}

Assuming (\ref{delta regularity}),
\begin{align*}
I &\lesssim ||v||_{L^\infty(\partial B_{2(1 + \varepsilon/2)\rho})}^2 \iint_{A_\rho} |(1 + |\Delta|)^{(s+1)/2} F_P(x, y) - F_{-\Delta}(x, y)|^2 ~dy ~dx\\
&\lesssim \rho^{\ell - 1} ||v||_{L^\infty(\partial B_{2(1 + \varepsilon/2)\rho})}^2 \sup_{y \in \partial B_{2(1 + \varepsilon/2)\rho}} ||F_P(\cdot, y) - F_{-\Delta}(\cdot, y)||_{H^{s + 1}(B_{2(1 + \varepsilon/2)\rho})}^2\\
&\lesssim \rho^{\ell - 1} ||v||_{L^\infty(\partial B_{2(1 + \varepsilon/2)\rho})}^2 \sup_{y \in \partial B_{2(1 + \varepsilon/2)\rho}} ||K_P(\cdot, y) - K_{-\Delta}(\cdot, y)||_{H^{s + 1}(B_{2(1 + \varepsilon/2)\rho})}^2\\
&\lesssim \rho^{\ell + 1} ||v||_{L^\infty(\partial B_{2(1 + \varepsilon/2)\rho})}^2.
\end{align*}
By Sobolev embedding, there exists $s' > 0$ so large that
$$||w||_{L^\infty(\partial B_{2(1 + \varepsilon/2)\rho})}^2 \lesssim \rho^{1-\ell} ||v||_{H^{s'}(\partial B_{2(1 + \varepsilon/2)\rho})}^2.$$
Now if we assume that $s' > s + 2$, it follows that
$$||w||_{H^s(B_{2(1 + \varepsilon/2)\rho})} \lesssim \rho^{1/2} ||v||_{H^{s'}(\partial B_{2(1 + \varepsilon/2)\rho})}.$$
By elliptic regularity applied to $-\Delta$, we may replace $s$ with an arbitrarily large exponent and this inequality will still hold, so we can drop the assumption (\ref{delta regularity}).
We can also take $s' > 0.5$ and write $v = Tu$, where $Pu = 0$ (so $G_PTu = u|\partial B_{2(1 + \varepsilon/2)\rho}$) so that
$$||w||_{H^s(B_{2(1 + \varepsilon/2)\rho})} \lesssim \rho ||u||_{H^{s' + .5}(B_{2(1 + \varepsilon)\rho})}.$$
Now we can apply elliptic regularity to $P$ and replace $s'$ with an arbitrarily small exponent.
\end{proof}

Now let $u$ meet the hypotheses of Proposition \ref{boostrap elliptic}.
After applying a translation, we may assume that $\int_{B_\rho} u~\vol = 0$.
Then by elliptic regularity and the Poincar\'e inequality, for every $s > 0$,
\begin{equation}\label{poincare error}
||u||_{H^s(B_\rho)}^2 \lesssim_s ||u||_{L^2(B_\rho)}^2 \lesssim \rho^2 ||du||_{L^2(B_\rho)}^2.
\end{equation}
Write $u = v + w$ where $v = G_{-\Delta}Tu$, so that by Lemma \ref{approx harmonic} and (\ref{poincare error}),
$$||dw||_{L^2(B_{2(1 + \varepsilon/2)\rho})}^2 \lesssim \rho^2 ||u||_{H^1(B_{2(1 + 3\varepsilon/4)\rho})}^2 \lesssim \rho^4 ||du||_{L^2(B_{2(1 + 3\varepsilon/4)\rho})}^2.$$
Moreover, $\Delta v = 0$.
By Sobolev embedding and (\ref{poincare error}), there exists $s > 0$ such that
$$|dv(0)|^2 = |dw(0)|^2 \lesssim ||w||_{C^1(B_\rho)}^2 \lesssim \rho^2||u||_{H^s(B_\rho)}^2 \lesssim \rho^4||du||_{L^2(B_{2(1 + 3\varepsilon/4)\rho})}^2.$$
So by Lemma \ref{bootstrap Laplace},
\begin{align*}
||du||_{L^2(B_\rho)}^2 &\leq ||dv||_{L^2(B_\rho)}^2 + ||dw||_{L^2(B_\rho)}^2 \\
&\leq 2^{-\ell-2} ||dv||_{L^2(B_{2\rho})}^2 + |dv(0)|^2 + O(\rho^4) ||du||_{L^2(B_{2(1 + 3\varepsilon/4)\rho})}^2\\
&\leq 2^{-\ell-2} ||dv||_{L^2(B_{2\rho})}^2 + O(\rho^4) ||du||_{L^2(B_{2(1 + 3\varepsilon/4)\rho})}^2.
\end{align*}
A similar argument shows
\begin{align*}
||dv||_{L^2(B_{2\rho})}^2 &\leq ||du||_{L^2(B_{2\rho})}^2 + O(\rho^4) ||du||_{L^2(B_{2(1 + 3\varepsilon/4)\rho})}^2,
\end{align*}
so
$$||du||_{L^2(B_\rho)}^2 \lesssim 2^{-\ell-2}||du||_{L^2(B_{2\rho})}^2 + O(\rho^4) ||du||_{L^2(B_{2(1 + 3\varepsilon/4)\rho})}^2.$$
To control the error term, we need a topological lemma.

\begin{lemma}\label{Poincare lemma}
Let $A$ be the image under $d$ of $\ker P$, where $P$ is viewed as having domain $H^2(B_{2(1 + 3\varepsilon/4)\rho})$.
Then for every $1$-form $\omega \in A$,
$$||\omega||_{L^2(B_{(1 + 3\varepsilon/4)})} \lesssim \rho^{1/2} ||T\omega||_{L^2(\partial B_{1 + 3\varepsilon/4})}.$$
\end{lemma}
\begin{proof}
Let
\begin{align*}
R_k &= \partial_i [a^{ij}, \partial_k] \partial_j + [b^i, \partial_k] \partial_i + [c, \partial_k]\\
&= \partial_i a^{ij}_{,k} \partial_j + b^i_{,k} \partial_i + c_{,k}.
\end{align*}
Then $R_k$ degenerates to a first-order operator at the origin by (\ref{Taylor coefficients are small}).
Setting $Ru = R_ku ~dx^k$, and choosing a solution operator $d^{-1}$, which is possible since
$$\Homology^1(B_{2(1 + 3\varepsilon/4)\rho}) = 0,$$
we see that if $\rho$ is small enough, then the pseudodifferential operator
$$Q = dPd^{-1} + Rd^{-1}$$
is uniformly elliptic of second order.
Since $Q$ annihilates $A$, the ellipticity of $Q$ implies the claim.
\end{proof}

Lemma \ref{Poincare lemma} holds for $\omega = du$, so
\begin{align*}
||du||_{L^2(B_{2(1 + 3\varepsilon/4)\rho})} &\lesssim \rho^{1/2} ||Tdu||_{L^2(\partial B_{2(1 + 3\varepsilon/4)\rho})}
 \leq \rho^{1/2} ||Tdu||_{L^2(\partial(B_{2(1 + \varepsilon)\rho} \setminus B_{2(1 + 3\varepsilon/4)\rho}))} \\
&\lesssim ||u||_{L^2(B_{2(1 + \varepsilon)\rho} \setminus B_{2(1 + 3\varepsilon/4)\rho})}.
\end{align*}
Thus we have shown that
$$||du||_{L^2(B_\rho)}^2 \leq 2^{-\ell-2}||du||_{L^2(B_{2\rho})}^2 + O(\rho^4) ||du||_{L^2(B_{2(1 + \varepsilon)\rho} \setminus B_{2\rho})}^2$$
which is sufficient if $\rho$ is chosen so small that $O(\rho^4) < 2^{-\ell-2}$.
This completes the proof of Proposition \ref{bootstrap elliptic}.

\subsection{Estimates on $C^1$ hypersurfaces}
\begin{proposition}
\label{bootstrap C1}
Let $(L_n)$ be a sequence of Caccioppoli sets in $M$ with $C^1$ boundary, $\varepsilon > 0$, and $(\beta_n) \subseteq \RR^+$ such that
$$\Lambda(L_n, B(x_n, \rho_n)) \leq \beta_n.$$
If
$$\lim_{n \to \infty} \inf_{\partial L_n \cap B(x_n, \rho_n)} (\normal_{L_n})_d(x_n) = 1,$$
then
$$\limsup_{n \to \infty} \Lambda(L_n, B(x_n, \rho_n/2)) \leq \frac{\beta_n}{2^{d + 1}}.$$
\end{proposition}

\subsection{Regularizing minimal surfaces}
Throughout this subsection we suppose that de Giorgi's lemma is false, and work towards constructing a counterexample to Proposition \ref{bootstrap C1}.
From our contradiction assumption, we have:

\begin{assumption} \label{contradict DGL}
There exists a sequence of sets $(E_n)$ of least perimeter in $M$ and $\rho_n \to 0$ such that, if
$$\gamma_n = \Lambda(E_n, B(x_n, \rho_n)),$$
then
\begin{equation}\label{contradict DGL bootstrap}
\gamma_n \ll \rho_n^{d - 1},
\end{equation}
and for every $n \in \NN$,
\begin{equation}\label{contradict DGL eqn}
\Lambda(E_n, B(x_n, \rho_n/2)) \geq \frac{\gamma_n}{2^d}.
\end{equation}
\end{assumption}

To violate Proposition \ref{bootstrap C1}, we need to mollify the singular hypersurfaces $\partial E_n$ in a neighborhood of $x_n$.
Due to our uniform bound on $g$ and TODO: injectivity radius estimates... pass to a subsequence... ROTATE, to assume
$$\int_{B(x_n, \rho_n)} d1_{E_n}~\vol = (0, 0, \dots, 0, c_d)$$
TODO we may assume that $\rho_n$ is so small that we can find polar coordinates
$$\varphi_n: B(x_n, \rho_n) \to \RR_+ \times S^{d - 1}$$
which satisfy Gauss' lemma, and we write
$$F_n(r, \Theta) ~dr ~d\Theta = (\varphi_n)_* \vol.$$
We introduce the convolution measure
$$d\mu_{n, \varepsilon} = C_{n, \varepsilon, d} \left(1 - \frac{r}{\varepsilon}\right)1_{r < \varepsilon} F_n(r, \theta) ~dr ~d\Theta,$$
where $C_{n, \varepsilon, d}$ is chosen so $\mu_{n, \varepsilon}(\RR_+ \times S^{d - 1}) = 1$.
Let $\psi_{n, \ell} = 1_{E_n} * d\mu_{n, \gamma_n^\ell \rho_n}$.

\begin{lemma}\label{mollify props}
Let $0 < \delta \lesssim 1$ and $x, y, \xi \in \{r < \varepsilon\}$. Then:
\begin{enumerate}
\item \label{conv in weak top} The measure $\mu_{n, \varepsilon}$ converges to a point mass at $\{0\}$ in the weak topology of measures as $\varepsilon \to 0$.
\item \label{scale of normalizer} $C_{n, \varepsilon, d} \sim_d \varepsilon^{-d}$ as $\varepsilon \to 0$.
\item \label{approximate y interior} If $d(y, \xi) < \delta\varepsilon$, then
$$\varepsilon^{-d}\left(1 - \delta - \frac{d(x, \xi)}{\varepsilon}\right) \lesssim_d \frac{d\mu_{n, \varepsilon}}{d((\varphi_n)_* \vol)}(x - y) \lesssim_d \varepsilon^{-d}\left(1 + \delta - \frac{d(x, \xi)}{\varepsilon}\right).$$
\item \label{approximate y boundary} If $y \in \{\varepsilon(1 - 2\delta) < r < \varepsilon\}$, then
$$\frac{d\mu_{n, \varepsilon}}{d((\varphi_n)_*\vol)}(x - y) \lesssim_d \delta \varepsilon^{-d}.$$
\item \label{mollified is W2} $\psi_{n, \ell} \in W^{2, \infty} \subseteq C^1$.
\end{enumerate}
\end{lemma}
\begin{proof}
(\ref{conv in weak top}) follows because $\mu_{n, \varepsilon}$ is a Radon probability measure whose support shrinks down to $\{0\}$.
(\ref{scale of normalizer}, \ref{approximate y interior}, \ref{approximate y boundary}) follow because $g$ is bounded, and hence is comparable (uniformly in $n$) to the euclidean metric; in the euclidean metric these topologies are clear.
Finally, (\ref{mollified is W2}) can be proven by the estimate
$$|d\psi_{n, \ell}(y_1) - d\psi_{n, \ell}(y_2)| \lesssim_{n, \ell} \mu_{n, \gamma_n^\ell \rho_n}((E_n - y_1) \Delta (E_n - y_2)),$$
the fact that $d\mu_{n, \varepsilon}/d((\varphi_n)_*\vol) \in W^{1, \infty}$, and Sobolev embedding.
\end{proof}

\begin{lemma}
There exist $\lambda_{n, \ell} > 0$ and open sets $I_{n, \ell} \subseteq \{r < \rho_n\}$ such that for every $\ell$,
\begin{enumerate}
\item \label{density of In} $\lim_{n \to \infty} \frac{|I_n|}{|\{r < \rho_n\}|} = 1$,
\item $\lim_{n \to \infty} \lambda_{n, \ell} = 0$,
\item and for every $r \in I_{n, \ell}$,
$$\partial_d \psi_{n, \ell}(r, \Theta) > (1 - \lambda_{n, \ell})|d\psi_{n, \ell}(r, \Theta)|.$$
\end{enumerate}
\end{lemma}
\begin{proof}
Let $\varepsilon = \gamma_n^\ell \rho_n$ and $\sigma = \gamma_n^{0.5/(d - 1)} \rho_n$.
By (\ref{contradict DGL bootstrap}), $\gamma_n \to 0$, so after removing finitely many $n$ we may assume that $\varepsilon < \sigma$.
We let
$$I = I_{n, \ell} = \{r < \rho_n - 2\sigma\} \cap \psi_{n, \ell}^{-1}(J),$$
$$J = (d^2 \gamma_n^2, 1 - d^2 \gamma_n^2).$$
It follows from the definitions that $\sigma \to 0$ and $J \to (0, 1)$.
Then, since $\psi_{n, \ell}$ is $C^1$, $\mu_{n, \ell}$ is an approximation to the identity, and $1_{E_n}$ is two-valued, $\psi_{n, \ell}^{-1}(J)$ must approximate $E_n$.
This proves (\ref{density of In}).

Let $f = |d1_{E_n}| - \partial_d 1_{E_n}$ and fix $x \in I$.
We want to estimate $f * d\mu_{n, \varepsilon}(x)$, so we decompose $f = f_1 + f_2$ with $f_1 = f1_{\{r < \varepsilon(1 - 2\delta)\}}$, where $\delta \in (0, 0.5)$ is a parameter to be determined that may depend on $n, \ell$.

\begin{claim}[control close to the boundary]
If $n$ is large enough, then
$$f_2 * d\mu_{n, \varepsilon}(x) \lesssim_d \delta \gamma_n^{1 - d} (|d1_{E_n}|~\vol * d\mu_{n, \varepsilon})(x).$$
\end{claim}
\begin{proof}[Proof of claim]
Let $h = f_2 * d\mu_{n, \varepsilon}$ and $Z = \{\varepsilon(1 - 2\delta) < r < \varepsilon\}$.
By Lemma \ref{mollify props},
$$h(x) \leq 2 \int_Z |d1_{E_n}|(y) ~d\mu_{n, \varepsilon}(x - y) \lesssim_d \frac{\delta}{\varepsilon^d} |\partial^* E_n \cap B(x_n, \varepsilon)|.$$
Since $E_n$ has least perimeter,
$$|\partial^* E_n \cap B(x_n, \varepsilon)| \leq |E_n \cap \partial B(x_n, \varepsilon)| \lesssim_d \varepsilon^{d - 1}$$
and hence
$$h(x) \lesssim_d \frac{\delta}{\varepsilon}.$$
We now bound
$$\psi_{n, \ell}(x) = \int_{E_n \setminus B(x, d(x, E))} \lesssim \varepsilon^{-d}...$$
\end{proof}


\end{proof}





%%%%%%%%%%%%%%%%%%%%%%%%%%%%%

\section{de Giorgi's lemma}\label{DGL section}
We now prove a variant of the de Giorgi lemma. Recall that $\Lambda(U, V)$ is defined to be the excess of $U$ in $V$, which is well-defined inside some coordinate chart.

\begin{theorem}[de Giorgi's lemma]\label{DGL}
Suppose that $M$ is an open submanifold of $\RR^d$ and $g$ is bounded in $C^\infty(M)$.
Then for every $\alpha \in (0, 1)$ there exist $\sigma, r > 0$ such that for every Caccioppoli set $U$ of least perimeter and every ball $V$ of radius $\rho \in (0, r)$ such that $\Lambda(U, V) < \sigma \rho^d$,
$$\Lambda(U, \alpha V_n) < \alpha^d \Lambda(U, V_n).$$
\end{theorem}

Note that in the classical de Giorgi lemma, one instead hypothesizes $\Lambda(U, V) < \sigma \rho^{d - 1}$.
This loss will not matter, however, because our blowup construction will allow us to take $\Lambda(U, V)$ arbitrarily small.

We begin by recalling the first part of the proof of the de Giorgi lemma in the case of $(M, g) = \RR^d$ (in which case we can take $r = \infty$).
The hypothesis $\Lambda(U, V) < \sigma \rho^{d - 1}$ implies that $U$ is approximately a hyperplane.
If $U$ is the epigraph of a $C^1$ function $f$, then it follows that $|df|$ is small and $f$ solves the minimal surface equation.

Let $\mathscr L$ be defined by (\ref{Lagrangian density}). Then $f$ minimizes $\mathscr L$.
But the linearization of $\mathscr L$ about $|df| = 0$ is Dirichlet's Lagrangian
$$\mathscr I = (1 + \frac{|df|^2}{2}) ~\vol,$$
and since we are assuming $|df| \ll 1$, it follows that $||\Delta f|| \ll 1$ in some sense.
But there is an analogous statement to the de Giorgi lemma for harmonic functions \cite[Lemma 4.1]{Miranda66}, which we have some hope of applying here.
These considerations yield the following theorem \cite[Teorema 4.4]{Miranda66}:

\begin{theorem}\label{DGL C1}
Let $(L_n)$ be a sequence of Caccioppoli sets in $\RR^d$, $\rho > 0$, and $(\beta_n) \subset \RR^d$.
If $\Lambda(L_n, B(0, \rho)) \leq \beta_n$, $\partial L_n$ is a $C^1$ hypersurface in $B(0, \rho)$,
$$\lim_{n \to \infty} \inf_{\partial L_j \cap B(0, \rho)} (\normal_{L_j})_d(x) = 1,$$
and
$$|\partial L_n \cap B(0, \rho)| - \eta(L_n, B(0, \rho)) \ll \beta_n,$$
then for every $\alpha \in (0, 1)$,
$$\limsup_{n \to \infty} \Lambda(L_n, B(0, \alpha \rho)) \leq \alpha^{d + 1} \beta_n.$$
\end{theorem}

The above result essentially proves the de Giorgi lemma in case $\partial U$ is a $C^1$ hypersurface.
In fact, we will use this result to prove Theorem \ref{DGL} by a regularization argument.

\subsection{Some easy reductions}
We begin by reducing to a special case, in which, among other things, $M$ is diffeomorphic to $\RR^d$.
Let $v_1, \dots, v_d$ be the standard basis of $\RR^d$.

\begin{notation}
We write $\Lambda_g$ and similar to emphasize that we are using the volume form induced by $g$.
If we do not use a subscript, we will refer to the euclidean volume form on $\RR^d$.
\end{notation}

Following the methods of \cite[\S4-5]{Miranda66}, it will be convenient to reason by contradiction, hence we assume:

\begin{assumption} \label{DGL contradictor}
There exists a sequence of Caccioppoli sets $(U_n)$ in $M$, and a sequence of balls $V_n = B_g(x_n, \rho_n)$ such that $(U_n)$ has least perimeter with respect to $g$, $\rho_n \to 0$,
$$\lim_{n \to \infty} \beta_n := \lim_{n \to \infty} \frac{\Lambda_g(U_n, V_n)}{{\rho_n}^{d - 1}} = 0,$$
and for every $n \in \NN$,
\begin{equation}\label{contradiction DGL}
\Lambda_g(U_n, \alpha V_n) \geq \alpha^d \Lambda_g(U_n, V_n).
\end{equation}
\end{assumption}

\begin{lemma}
There exists $c > 0$ and a sequence $(E_n)$ of Caccioppoli sets in $\RR^d$ which has approximately least perimeter in $B(0, 1)$ in the particularly strong sense that there exists $(\rho_n) \in \ell^1$ which witnesses that for every $A \subseteq B(0, 1)$,
\begin{equation}\label{DGL ALP}
|\partial^* E_n \cap A| \leq (1 + c{\rho_n}^2)^2 \eta(E_n, A)
\end{equation}
and which satisfies
\begin{equation}\label{DGL bound by 2}
(1 + c{\rho_n}^2)^2 \leq 2.
\end{equation}
In addition,
\begin{equation}\label{DGL conormal axis}
\int_{B(0, 1)} d1_{E_n} ~\vol \text{ is a scalar multiple of } v_d
\end{equation}
and
\begin{equation}\label{scaled contradiction DGL}
\Lambda(E_n, B(0, \alpha)) \geq \alpha^d(1 + c{\rho_n}^2) \Lambda(E_n, B(0, 1))
\end{equation}
Finally, the sequences of excesses
\begin{equation}\label{scaled summable DGL}
\gamma_n = \Lambda(E_n, B(0, 1))
\end{equation}
is summable.
\end{lemma}
\begin{proof}
Let $(U_n),(\beta_n),(\rho_n)$ be as in Assumption \ref{DGL contradictor}.
After taking a subsequence we may in addition assume that $(\beta_n),(\rho_n) \in \ell^1$.
Since $g$ is bounded, there exists a \emph{uniform} strong injectivity radius $\zeta > 0$ on $M$.
After rescaling $g$, we can assume that $\zeta > 1$, and after discarding finitely many $n$ we may assume that for every $n$, $\rho_n < 1$.
Similarly, there is a uniform Ricci-Taylor error $c > 0$ on $M$, and we will always assume (\ref{DGL bound by 2})
which can be guaranteed by discarding finitely many $n$.

In particular $B_{T_{x_n}M}(0, \rho_n)$ is the exponential pullback $\exp_{x_n}^*(V_n)$.
For every $n$, we can find an isometric isomorphism
$$\varphi_n: T_{x_n}M \to \RR^d$$
such that $\int_{B_{T_{x_n}M}(0, \rho_n)} d1_{\exp_{x_n}^* U_n} ~\vol$ lies in the span of $\varphi_n^{-1}(v_d)$.
Since balls in $\RR^d$ with its flat metric, centered on $0$, are thus identified with balls in $(M, g)$, we now drop the subscript denoting the metric.

We now let
$$F_n = (\varphi_n)_* \circ \exp_{x_n}^* U_n \subseteq \RR^d,$$
so that $\int_{B'(0, \rho_n)} d1_{F_n} ~\vol$ lies in the span of $v_d$ and $F_n$ has least perimeter in $B(0, \rho_n)$ with respect to $(\varphi_n)_* \circ \exp_{x_n}^* g$.
If
$$E_n = \{x \in \RR^d: \rho_n x \in F_n\},$$ then (\ref{DGL conormal axis}) holds
and $E_n$ has least perimeter in $B(0, 1)$ with respect to
$$g = \rho_n^{-0.5} (\varphi_n)_* \circ \exp_{x_n}^* g.$$

We now estimate
\begin{align*}
|\partial^* E_n \cap A| &= {\rho_n}^{-1} |\partial^* F_n \cap \rho_n A|\\
&\leq \frac{1 + c{\rho_n}^2}{\rho_n} |\partial^* F_n \cap \rho_n A|_{g_n}\\
&\leq \frac{(1 + c{\rho_n}^2)^2}{\rho_n} \eta(F_n, \rho_n A)\\
&= (1 + c{\rho_n}^2) \eta(F_n, A).
\end{align*}
Therefore $E_n$ has approximately least perimeter in $\RR^d$, in the sense of (\ref{DGL ALP}).
A similar computation applied to (\ref{contradiction DGL}) shows that (\ref{scaled contradiction DGL}).
We similarly bound $||(\gamma_n)||_{\ell^1} \leq 2 ||(\beta_n)||_{\ell^1}$.
\end{proof}

Throughout the rest of the proof, we just work on $\RR^d$ and discard $(M, g)$.
We will allow ourselves to pass to a subsequence in $n$, and leave reiindexing implict.

\subsection{Regularizing the boundary}
The sequence $(E_n)$ that we have just constructed would contradict Theorem \ref{DGL C1} if $\partial E_n$ was smooth.
However, this is not true, so we now regularize $(E_n)$ using a modification of the arguments of \cite[Chapter 7]{Giusti77}.
Thus we introduce the convolution kernel
$$\chi_\varepsilon(x) = C \varepsilon^{-d}\left(1 - \frac{|x|}{\varepsilon}\right) \vee 0.$$
Here $C > 0$ is the constant that enforces $||\chi_\varepsilon||_{L^1} = 1$.
The advantange of this kernel is that for every $\delta > 0$ small enough and every ball $V = B(\xi, \delta\varepsilon)$, $y \in V$ satisfies
\begin{equation}\label{kernel on balls}
\frac{C}{\varepsilon^d}\left(1 - \delta - \frac{|x - \xi|}{\varepsilon}\right) \leq \chi_\varepsilon(x - y) \leq \frac{C}{\varepsilon^d}\left(1 + \delta - \frac{|x - \xi|}{\varepsilon}\right).
\end{equation}

\begin{notation}
If $f$ is a Radon measure, we write $f^{(\varepsilon)} = f * \chi_\varepsilon$.
\end{notation}

We define $\varphi_{n, \ell} = (1_{E_n} ~\vol)^{({\gamma_n}^\ell)}$.
Then, by \cite[Lemma 7.1]{Giusti77}, $\varphi_{n, \ell}$ is a $C^1$ function.
Since $\chi_\varepsilon$ clearly converges to the Dirac measure $\delta_0$ in the weak topology of measures as $\varepsilon \to 0$, $\lim_\ell \varphi_{n, \ell} = 1_{E_n} ~\vol$ in the weak topology of measures.

\begin{lemma}
There exist $\lambda_{n,\ell} > 0$ and open sets $I_n \subseteq B(0, 1)$ such that for every $\ell$, $\lim_n \lambda_{n,\ell} = 0$ and $I_{n, \ell}$ grows to $B(0, 1)$ as $n \to \infty$, such that for every $x \in I_n$,
$$\partial_d \varphi_{n,\ell} > (1 - \lambda_{n, \ell}) |d\varphi_{n,\ell}|.$$
\end{lemma}
\begin{proof}
We set $\varepsilon = {\gamma_n}^\ell$; after removing finitely many $n$, we may assume $\gamma_n \in (0, 1)$ and hence if $\sigma = {\gamma_n}^{0.5/(d - 1)}$, $\varepsilon < \sigma$.
Let $f = |d1_{E_n}| - \partial_d 1_{E_n}$ and select $x \in I$ where
$$I = I_{n, \ell} = B(0, 1 - 2\sigma) \cap \varphi_{n, \ell}^*(d^2 \gamma_n^2, 1 - d^2 \gamma_n^2).$$
Taking $n \to \infty$ and using $(\gamma_n) \in \ell^1$ we see that $I_{n, \ell}$ grows to $B(0, 1)$.

It is our purpose to estimate $f^{(\varepsilon)}(x)$, and to this end we write $f = f_1 + f_2$ where $f_1 = f1_{B(0, \varepsilon(1 - 2\delta))}$, where $\delta \in (0, 0.5)$ is a quantity which is allowed to depend on $n$.

We begin by constructing an open cover of the domain $B(0, \varepsilon)$ of $\chi_\varepsilon$.
More precisely, we use a greedy algorithm to find a maximal finite set $\mathcal V$ of disjoint balls of radius $\delta \varepsilon$ centered on $\partial^* E \cap B_{\varepsilon(1 - 2\delta)}$, and let $U = B(0, \varepsilon) \setminus \overline{B(0, \varepsilon(1 - 2\delta))}$.
Then $\mathcal U = \{2V: V \in \mathcal V\} \cup \{U\}$ is an open cover of $B(0, 1)$.
We will bound the integral defining $f^{(\varepsilon)}$ in each of the open sets in $\mathcal U$ separately.

\begin{claim}[control close to the boundary]\label{molly close}
If $n$ is large enough, then
$${f_2}^{(\varepsilon)}(x) \lesssim_d \delta {\gamma_n}^{1 - d} |d1_E|^{(\varepsilon)}(x).$$
\end{claim}
\begin{proof}[Proof of claim]
Since $x \in I$, for every $y \in U$.
$$\chi_\varepsilon(x - y) \lesssim_d \frac{\delta}{\varepsilon^d}.$$
This gives
$${f_2}^{(\varepsilon)}(x) \leq 2\int_U \chi_\varepsilon(x - y) |d1_E|(y) ~\vol(y) \lesssim_d \frac{\delta}{\varepsilon^d} |\partial^* E \cap B(0, \varepsilon)|.$$
Since $(E_n)$ has approximately least perimeter, if $n$ is large enough then
$$|\partial^* E_n \cap B(0, \varepsilon)| \leq 2|\partial B(0, \varepsilon)| \lesssim_d \varepsilon^{d - 1}$$
and so we conclude
$${f_2}^{(\varepsilon)}(x) \leq 2\int_U \chi_\varepsilon(x - y) ~\vol(y) \lesssim_d \frac{\delta}{\varepsilon}.$$
This estimate is comparable to \cite[(7.8)]{Giusti77}, and now (after throwing away a constant factor) we can copy the argument following that estimate verbatim.
\end{proof}

Since $\mathcal U \setminus \{U\}$ is an open cover of $B(0, \varepsilon) \setminus U$,
$${f_1}^{(\varepsilon)}(x) \leq \sum_{V \in \mathcal V} (f1_{2V})^{(\varepsilon)}(x).$$
So now we bound each of the summands $(f1_{2V})^{(\varepsilon)}$.

\begin{claim}[control in dilated balls]\label{dilated balls claim}
Suppose that $\delta \geq {\gamma_n}^d$.
Then there exist $K_n > 0$, $K_n \to 0$, such that for every $V \in \mathcal V$,
$$(f1_{2V})^{(\varepsilon)}(x) \leq K_n \frac{\delta^{d - 1}}{\varepsilon}\left(1 + 2\delta + \frac{|x - \xi|}{\varepsilon}\right).$$
\end{claim}
\begin{proof}[Proof of claim]
By (\ref{kernel on balls}), if $\xi$ is the center of $V$ then
\begin{equation}\label{dilated balls 0}
(f1_{2V})^{(\varepsilon)}(x) \lesssim_d \varepsilon^{-d} \left(1 + 2\delta + \frac{|x - \xi|}{\varepsilon}\right) \int_{2V} f ~\vol.
\end{equation}
By (\ref{DGL ALP}), for every $h > 0$, if $n$ is large enough then $|\partial^* E \cap V| \leq (1 + h)\eta(E, V)$, so we can add and subtract $\partial_d 1_{E_n}$ from the integrand and apply Lemma \ref{approximate monotonicity 2} with $2\delta \varepsilon < \sigma$ to see that $\int_{2V} f ~\vol$ is
\begin{equation}\label{dilated balls 1}
\lesssim_d (\delta \varepsilon)^{d - 1} \left(\sigma^{1 - d} \int_{B(\xi, \sigma)} (f + \partial_d 1_{E_n}) ~\vol - (\delta\varepsilon)^{1 - d} \int_V \partial_d 1_{E_n} ~\vol + h\log\frac{\sigma}{2 \delta\varepsilon}\right).
\end{equation}
By (\ref{DGL conormal axis}) and the fact that $\sigma^{1 - d} = {\gamma_n}^{-0.5}$,
\begin{equation}\label{DGL conormal axis consequence}
\sigma^{1 - d}\int_{B(\xi, \sigma)} f ~\vol \leq {\gamma_n}^{-0.5} \int_{B(0, 1)} f ~\vol = {\gamma_n}^{-0.5} \Lambda(E_n, B(0, 1)) = \sqrt{\gamma_n}.
\end{equation}
Meanwhile, by Lemma \ref{approximate monotonicity},
\begin{align*}
&\sigma^{1 - d} \int_{B(\xi, \sigma)} \partial_d1_{E_n} ~\vol - (2\delta\varepsilon)^{1 - d} \int_{2V} \partial_d1_{E_n} ~\vol\\
& \qquad \leq \left|\sigma^{1 - d} \int_{B(\xi, \sigma)} d1_{E_n} ~\vol - (2\delta\varepsilon)^{1 - d} \int_{2V} d1_{E_n} ~\vol\right|\\
& \qquad \lesssim_d \sqrt{\sigma^{1 - d} |\partial^* E_n \cap B(\xi, \sigma)| - (2\delta\varepsilon)^{1 - d} |\partial^* E_n \cap 2V| + h \log \frac{\sigma}{2\delta\varepsilon}}.
\end{align*}
If $h$ is small enough this estimate and (\ref{DGL conormal axis consequence}) allow us to simplify (\ref{dilated balls 1}) to see that $\int_{2V} f ~\vol$ is
\begin{equation}\label{dilated balls 2}
\lesssim_d (\delta \varepsilon)^{d - 1} \left(\sqrt{\gamma_n} + \sqrt{h \log \frac{\sigma}{2\delta\varepsilon}} + \sqrt{\sigma^{1 - d} |\partial^* E_n \cap B(\xi, \sigma)| - (2\delta\varepsilon)^{1 - d} |\partial^* E_n \cap 2V|}\right).
\end{equation}
We simplify the third square root as
\begin{align*}
&\sigma^{1 - d} \int_{B(\xi, \sigma)} |d1_{E_n}| ~\vol - (2\delta\varepsilon)^{1 - d} \int_{2V} |d1_{E_n}| ~\vol\\
&\qquad = \left(\sigma^{1 - d} \int_{B(\xi, \sigma)} f ~\vol\right) + \left(\sigma^{1 - d} \int_{B(\xi, \sigma)} \partial_d 1_{E_n} ~\vol\right) - \left((2\delta\varepsilon)^{1 - d} \int_{2V} |d1_{E_n}| ~\vol\right)\\
&\qquad=: P + Q - R.
\end{align*}
By (\ref{DGL conormal axis consequence}), $P \leq \sqrt{\gamma_n}$.
If $\normal$ denotes the unit normal to $\partial B(\xi, \sigma)$ then
$$Q = \sigma^{1 - d} \int_{\partial B(x, \sigma) \cap E_n} \normal_d ~\vol \leq \sigma^{1 - d} \int_{\partial B(x, \sigma)} ~\vol = \omega_{d - 1}.$$
On the other hand, $\xi \in \partial^* E_n$, so if we write $N$ for the disk bounded by the equator in $2V$, then
$$|2V \cap \partial^* E_n| \geq |2V \cap N| = \omega_{d - 1}(2\delta\varepsilon)^{d - 1}$$
or in other words $R \geq \omega_{d - 1}$.
Summing up, $P + Q - R \leq \sqrt{\gamma_n}$ and hence we can simplify (\ref{dilated balls 2}) to
$$\int_{2V} f ~\vol \lesssim_d (\delta \varepsilon)^{d - 1} \left(\sqrt{\gamma_n} + \sqrt{h \log \frac{\sigma}{2\delta\varepsilon}}\right).$$
Moreover, $\sigma \lesssim 2\delta\varepsilon$ since $\delta \geq \gamma^d$, so $\log(\sigma/2\delta\varepsilon) \lesssim 1$.
As $n \to \infty$, $h \to 0$, so combining this estimate with (\ref{dilated balls 0}) completes the proof.
\end{proof}

We now sum up (\ref{dilated balls claim}) over $\mathcal V$ to obtain a bound on all of $B(0, \varepsilon) \setminus U$.

\begin{claim}[control far from the boundary]\label{molly far}
One has
$${f_1}^{(\varepsilon)}(x) \lesssim_d |d1_{E_n}(x)|^{(\varepsilon)}.$$
\end{claim}
\begin{proof}[Proof of claim]
The support property of $\chi_\varepsilon$ and the disjointness of $\mathcal V$ imply
$$|d1_{E_n}(x)|^{(\varepsilon)} \geq \sum_{V \in \mathcal V} \int_V \chi_\varepsilon(x - y) |d1_{E_n}(y)| ~\vol(y)$$
so by (\ref{kernel on balls}), Proposition \ref{uniform density estimate}, and Claim \ref{dilated balls claim},
\begin{align*}
|d1_{E_n}(x)|^{(\varepsilon)} &\gtrsim_d \sum_{V \in \mathcal V} \varepsilon^{-d} \left(1 - \delta - \frac{|x - \cent V|}{\varepsilon}\right) |\partial^* E_n \cap V|\\
& \gtrsim_d \sum_{V \in \mathcal V} \frac{\delta^{d - 1}}{\varepsilon} \geq \frac{{f_1}^{(\varepsilon)}(x)}{K_n}. \qedhere
\end{align*}
\end{proof}

If $n$ is large enough then we may take $\delta = {\gamma_n}^d$. Thus, by Claims \ref{molly close} and \ref{molly far},
$$f^{(\varepsilon)}(x) \lesssim_d ({\gamma_n}^d + K_n)|d1_{E_n}|^{(\varepsilon)}(x).$$
Expanding out the definition of $f$ and letting $\lambda_{n,\ell}$ be a constant multiple of ${\gamma_n}^d + K_n$ (where the constant depends on $\ell$) will complete the proof.
\end{proof}

Now we observe that if $x \in I_{n, \ell}$,
$$\partial_d \varphi_{n,\ell}(x) \geq (1 - \lambda_{n, \ell}) |d1_{E_n}|^{(\varepsilon)}(x) > 0.$$
Since $\varphi_{n, \ell}$ is $C^1$ it follows that the level sets of $\varphi_{n, \ell}$ are $C^1$ hypersurfaces in $\RR^d$.
We can now mimic the proof of \cite[(7.22--7.23)]{Giusti77} with \cite[(5.14)]{Giusti77} replaced by the estimate
\begin{equation}\label{Giusti514}
|\partial^* E_n \cap B(0, r)| \leq 2|\partial B(0, r)| \lesssim r^{d - 1}
\end{equation}
(valid for $n$ large, by (\ref{DGL ALP})) to conclude that there exist Caccioppoli sets $L_n$ with $C^1$ boundary such that $d1_{L_n} - d1_{E_n} \to 0$ in the weak topology of measures and for almost every $r \in (0, 1)$,
\begin{equation}\label{Giusti722}
|\partial L_n \cap B(0, r)| - |\partial E_n \cap B(0, r)| \ll \gamma_n,
\end{equation}
\begin{equation}\label{Giusti723}
||1_{L_n} - 1_{E_n}||_{L^1(\partial B(0, r))} \ll \gamma_n,
\end{equation}
\begin{equation}\label{Giusti717}
\lim_{n \to \infty} \inf_{\partial L_n \cap B(0, r)} (\normal_{E_n})_d = 1.
\end{equation}

\subsection{Deriving a contradiction}
The complementary bound to (\ref{Giusti722}), while true in the setting of \cite[Chapter 7]{Giusti77}, is false here because $E_n$ may fail to have least perimeter.
Instead, we prove a slightly weaker but still sufficient bound:

\begin{lemma}
For almost every $r \in (0, 1)$,
$$|\partial^* E_n \cap B(0, r)| - |\partial L_n \cap B(0, r)| \ll \rho_n + \gamma_n.$$
\end{lemma}
\begin{proof}
We first show that $|\partial L_n \cap B(0, 1)|$ is uniformly bounded in $n$.
Since $d1_{L_n} - d1_{E_n} \to 0$ in the weak topology of measures,
$$|\partial L_n \cap B(0, 1)| \lesssim |\partial^* E_n \cap B(0, 1)|$$
if $n$ is large enough.
By (\ref{Giusti514}), then,
\begin{equation}\label{bounds on Ln}
|\partial L_n \cap B(0, 1)| \lesssim 1
\end{equation}
as desired.

By (\ref{DGL ALP}), Lemma \ref{estimates on good set}, and (\ref{Giusti723}),
\begin{align*}
|\partial^* E_n \cap B(0, r)| &\leq (1 + c{\rho_n}^2)^2 \eta(E_n, B(0, r)) \\
&\leq (1 + c{\rho_n}^2)^2(\eta(L_n, B(0, r)) + ||1_{E_n} - 1_{L_n}||_{L^1(\partial B(0, r))})\\
&\leq (1 + c{\rho_n}^2)^2 \eta(L_n, B(0, r)) + o(\gamma_n)\\
&\leq |\partial L_n \cap B(0, r)| + O({\rho_n}^2) |\partial L_n \cap B(0, r)| + o(\gamma_n).
\end{align*}
By (\ref{bounds on Ln}), ${\rho_n}^2 |\partial L_n \cap B(0, r)| = o(\rho_n)$ so this suffices.
\end{proof}

Combining the previous lemma with (\ref{Giusti722}), we obtain
\begin{equation}\label{Giusti714}
\left||\partial L_n \cap B(0, r)| - |\partial^* E_n \cap B(0, r)|\right| \ll \rho_n + \gamma_n
\end{equation}
and combining (\ref{Giusti714}) with the coarea formula gives
\begin{equation}\label{Giusti713}
\left|\int_{B(0, r)} d1_{L_n} ~\vol - \int_{B(0, r)} d1_{E_n} ~\vol\right| \ll \rho_n + \gamma_n.
\end{equation}
Combining (\ref{Giusti714}) with (\ref{DGL ALP}) implies that $(L_n)$ has approximately least perimeter in the strong sense that
\begin{equation}\label{Giusti712}
|\partial L_n \cap B(0, r)| - \eta(L_n, B(0, r)) \ll \rho_n + \gamma_n.
\end{equation}
By (\ref{scaled summable DGL}, \ref{Giusti717}, \ref{Giusti712}),
\begin{equation}\label{bound the excess}
|\Lambda(L_n, B(0, r)) - \gamma_n| \ll \rho_n + \gamma_n.
\end{equation}

For almost every $r \in (0, 1)$ and every $s \in (0, r)$ we can make the following argument: by (\ref{bound the excess}) and Theorem \ref{DGL C1} with $\beta_n = \rho_n + \gamma_n$,
$$\limsup_{n \to \infty} \frac{\Lambda(L_n, B(0, \alpha s))}{\rho_n + \gamma_n} \leq \frac{\alpha^{d + 1}}{s^{d + 1}}.$$
By (\ref{Giusti714}, \ref{Giusti713}), then,
$$\limsup_{n \to \infty} \frac{\Lambda(E_n, B(0, \alpha s))}{\rho_n + \gamma_n} \leq \frac{\alpha^{d + 1}}{s^{d + 1}}.$$
Taking $r, s \to 1$ and applying (\ref{scaled summable DGL}),
$$\limsup_{n \to \infty} \frac{\Lambda(E_n, B(0, \alpha))}{\rho_n + \Lambda(E_n, B(0, 1))} \leq \alpha^{d + 1},$$
so for every $h > 0$ we can find $n$ so large that
$$\Lambda(E_n, B(0, \alpha)) \leq \alpha^{d + 1}(1 + h)(\rho_n + \Lambda(E_n, B(0, 1))).$$
But, if $h$ is small enough depending on $\alpha$, this contradicts (\ref{scaled contradiction DGL}) and the fact that $\rho_n \to 0$.
Therefore Assumption \ref{DGL contradictor} is untenable, and so Proposition \ref{DGL} holds.

\section{Proofs of main theorems}\label{proof of main thm}
We are finally ready to prove Theorems \ref{main thm} and \ref{main crly}.

Throughout this section, let $u$ be a function of least gradient.
By Corollary \ref{level sets are minimal}, the superlevel sets of $u$ have least perimeter.

\subsection{Regularity of minimal hypersurfaces}
Let $U$ be a superlevel set of $u$.
We want to show that $N = \partial U$ is as smooth as possible, and to this end, we might as well assume that $u = 1_U$.

We first may cover $M$ by normal coordinate charts $A_x$ centered on $x \in M$, selected so that $A_x$ is precompact in $M$.
In particular, if we fix a particular $x$, then the averaged $1$-forms $\int_V du ~\vol$ are well-defined for $V \subseteq A_x$, so the excesses $\Lambda(U, V)$ are well-defined.
We write $\Lambda_x$ for the excess as computed in $A_x$, and write $N^* = \partial^* U$.
Since $A_x$ is a precompact chart, we can select $\sigma_x$ to be the constant that appears in Proposition \ref{DGL}.

\begin{lemma}
For every $x \in N^*$ and sufficiently small $\rho > 0$ such that $B(x, \rho) \subseteq A_x$, there is an open set $x \in A_x' \subseteq A_x$ and $\delta > 0$ such that for every $t \in (0, \delta)$, $s \in (0, \rho)$, and $y \in A_x'$ such that $B(y, s) \subseteq A_x$,
\begin{equation}\label{basecase}\Lambda_x(U_t, B(y, s)) < \sigma_x s^d.\end{equation}
\end{lemma}
\begin{proof}
By Proposition \ref{blowup theorem}, there exists a half-space $C_x$ obtained as the blowup of $\exp_x^* U$\footnote{This is the only point in the argument when we use the fact that $d \leq 7$!}.
Since $C_x$ is a half-space and the coordinate chart $A_x$ is normal, $\Lambda_x((\exp_x)_* C_x, V) = 0$ whenever $V \subseteq A_x$.
In particular, if $V$ has no singularities with respect to the blowup sequence $(u_t)$,
$$\lim_{t \to 0} \Lambda_x(U_t, V) = \Lambda_x((\exp_x)_* C_x, V) = 0.$$
Here $U_t = (\exp_x)_* \{u_t = 1\}$ is the blowup of $U$.
So for every $\rho$ such that $B(x, \rho) \subseteq V$, there exists $\delta > 0$ such that if $t < \delta$,
$$\Lambda_x(U_t, B(x, \rho)) < \frac{\sigma_x}{2} \rho^d.$$
By continuity of measure, it follows that for every $y \in A_x$ close enough to $x$, and $0 < s < \rho$ such that $B(y, s) \subseteq A_x$, the claim holds.
\end{proof}

Fix $x \in N^*$, $t < \delta$, and let
$$\normal_s(y) = \frac{\int_{B(y, s)} du_t ~\vol}{\int_{B(y, s)} |du_t| ~\vol},$$
whenever $y \in A_x'$ and $B(y, s) \subseteq A_x$.
Then $\normal_s$ is continuous, since
$$\left|\int_{B(y_1, s)} du_t ~\vol - \int_{B(y_2, s)} du_t ~\vol\right| \leq \int_{B(y_1, s) \Delta B(y_2, s)} |du_t| ~\vol$$
where $\Delta$ denotes symmetric difference; the right-hand side vanishes as $y_2 \to y_1$ by continuity of measure.
We now show that $(\normal_s)$ is uniformly Cauchy as $s \to 0$.

\begin{lemma}
One has
$$|\normal_s(y) - \normal_r(y)| \lesssim_x \sqrt s$$
uniformly in $y \in A_x'$, whenever $0 < r < s < \rho$ and $B(y, s) \subseteq A_x$.
\end{lemma}
\begin{proof}
After throwing away a constant factor we may assume that $s = \rho/2^k$ for some $k$, and $r = \beta s$ for some $\beta \in (0, 1)$.
Since $|\normal_s(y)|,|\normal_r(y)| \leq 1$,
$$|\normal_s(y) - \normal_r(y)| \lesssim \sqrt{1 - (\normal_s(y), \normal_r(y))}.$$
Then (TODO: This seems dubious, check it)
\begin{align*}
(1 - (\normal_s(y), \normal_r(y)))  &\leq \frac{1}{|N^* \cap B(y, r)|} \int_{B(y, \beta \rho/2^k)} |du_t| - \left(\normal_s(y), \frac{du_t}{|du_t|}\right) |du_t| ~\vol\\
&\leq \frac{1}{|N^* \cap B(y, r)|} \int_{B(y, \rho/2^k)} |du_t| - \frac{(\normal_s(y), du_t)}{|du_t|} |du_t| ~\vol \\
&\leq \frac{1}{|N^* \cap B(y, r)|} \int_{B(y, \rho/2^k)} |du_t| - |\normal_s(y)|^2 |du_t| ~\vol\\
&\leq \frac{2}{|N^* \cap B(y, r)|} (1 - |\normal_s(y)|)\int_{B(y, \rho/2^k)} |du_t| ~\vol\\
&= 2\frac{\Lambda_x(U, B(y, s))}{|N^* \cap B(y, r)|}.
\end{align*}
By inducting on Proposition \ref{DGL} in $k$, the base case (\ref{basecase}), and Proposition \ref{uniform density estimate},
\begin{align*}
\frac{\Lambda_x(U, B(y, s))}{|N^* \cap B(y, r)|} < 2^{-kd} \frac{\Lambda_x(U, B(y, \rho))}{r^{d - 1}} \lesssim s
\end{align*}
and therefore
\begin{align*}
|\normal_s(y) - \normal_r(y)| &\lesssim \sqrt{\frac{\Lambda_x(U, B(y, s))}{|N^* \cap B(y, r)|}} \lesssim \sqrt s. \qedhere
\end{align*}
\end{proof}

By the above lemma, $(\normal_s)$ is uniformly Cauchy on $A_x'$ as $s \to 0$.
But from the definition of $\normal(y)$, and the fact that $U_t$ is just a rescaling of $U$, if $(\normal_s(y))$ converges to anything as $s \to 0$, it must converge to $\normal$.
Since $\normal_s$ is continuous, it follows that $\normal$ extends to a continuous $1$-form on $A_x' \cap N$.
By Proposition \ref{locality of Caccioppoli}, $(A_x')_{x \in N^*}$ defines an open cover of $N$, so $\normal$ extends to a continuous $1$-form on $N$, and hence by Proposition \ref{regularity of reduced boundary}, $N$ is as smooth as possible.

\subsection{Existence of laminations}
Now we consider the general case when $u$ is a function of least gradient.
Let
\begin{equation}\label{lamination union}
A = \bigcup_y \partial \{u > y\},
\end{equation} $B$ the interior of $\{du = 0\}$, and $x \in M$.
Then $x \in B$ iff $u = u(x)$ near $x$, but that happens iff for every $y < u(x)$, $x$ is interior to $\{u > y\}$ and for every $y \geq u(x)$, $x$ is exterior to $\{u > y\}$.
This happens iff for every $y \in \RR$, $x$ is either interior or exterior to $\{u > y\}$, thus $x \notin \partial \{u > y\}$, which happens iff $x \notin A$.
Thus $\{A, B\}$ is a partition of $M$, so $A$ is closed.
Moreover, the sets $\{u > y\}$ are totally ordered by $\subseteq$, so the sets $\partial \{u > y\}$ are disjoint.
They are also hypersurfaces which are as smooth as possible, by the previous section.
This proves Theorem \ref{main thm}.

\subsection{Convex surfaces with boundary}
The proof of Theorem \ref{main crly} is essentially identical to that of \cite[Proposition 3.4]{górny2017planar}; we give the details here for completeness.

Suppose that $M = \Sigma \subset \overline \Sigma$, and that Theorem \ref{main crly} is false for $u$.
That is, we cannot extend the geodesic lamination that we constructed above to a lamination of $\overline \Sigma$.
Therefore there exist disjoint geodesics $\gamma_1$ and $\gamma_2$ which intersect on $\partial \Sigma$ and bound superlevel sets $\{u > y_i\}$ of $u$.

Suppose that $\gamma_1$ and $\gamma_2$ intersect at $x_0$, and $\gamma_i$ passes through $x_i$ on the way to $x_0$, so that $x_0, x_1, x_2$ bound an open, nondegenerate geodesic triangle $\Delta \subset \overline \Sigma$. This makes sense, because $\overline \Sigma$ is convex.
Since we have access to the monotonicity formula (\ref{classic monotonicity formula}), the proof of \cite[Remark 37.9]{simon1983GMT} shows that there exist only finitely many connected components of $A$ in $\Delta$.
So, after replacing $\gamma_2$ with a geodesic closer to $\gamma_1$ as necessary, we may assume that either $A$ does not intersect $\Delta$, or $A$ contains $\Delta$.
By replacing $A$ with its complement if necessary, we may assume that $A$ does not meet $\Delta$.

However, $v = 1_{u^{-1}((y_1, y_2))}$ is a function of least gradient, and $v = 1$ on $\Delta$ but $v = 0$ on the opposite sides of $\gamma_i$.
So if we replace $v$ with $w = v - 1_\Delta$, $w$ has the same trace as $v$, but since $\Delta$ is a nondegenerate triangle,
$$\int_U |dw| ~\vol = |\partial(\{u > y\} \setminus \Delta) \cap U| < |\partial \{u > y\} \cap U| = \int_U |dv| ~\vol$$
whenever $U$ is a precompact neighborhood of $\overline \Delta$ in $\overline \Sigma$.
Therefore $v$ does not have least gradient, which is a contradiction.


%%%%%%%%%%%%%%%%%%%%%%%%%%%%%%%%%%%%%%%%%%%%%%%%%%%%%%%%%%%%%%%%%%%%



\printbibliography


\end{document}
