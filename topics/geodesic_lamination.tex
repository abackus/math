\documentclass[reqno,12pt,letterpaper]{amsart}
\RequirePackage{amsmath,amssymb,amsthm,graphicx,mathrsfs,url,slashed}
\RequirePackage[usenames,dvipsnames]{color}
\RequirePackage[colorlinks=true,linkcolor=Red,citecolor=Green]{hyperref}
\RequirePackage{amsxtra}
\usepackage{cancel}
\usepackage{tikz-cd}

\setlength{\textheight}{9in} \setlength{\oddsidemargin}{-0.25in}
\setlength{\evensidemargin}{-0.25in} \setlength{\textwidth}{7in}
\setlength{\topmargin}{-0.25in} \setlength{\headheight}{0.18in}
\setlength{\marginparwidth}{1.0in}
\setlength{\abovedisplayskip}{0.2in}
\setlength{\belowdisplayskip}{0.2in}
\setlength{\parskip}{0.05in}
\renewcommand{\baselinestretch}{1.05}

\title[Least gradient maximum principle]{The maximum principle for functions of Riemannian least gradient}
\author{Aidan Backus}
\date{July 2021}

\newcommand{\NN}{\mathbf{N}}
\newcommand{\ZZ}{\mathbf{Z}}
\newcommand{\QQ}{\mathbf{Q}}
\newcommand{\RR}{\mathbf{R}}
\newcommand{\CC}{\mathbf{C}}
\newcommand{\DD}{\mathbf{D}}
\newcommand{\PP}{\mathbf P}
\newcommand{\MM}{\mathbf M}
\newcommand{\II}{\mathbf I}
\newcommand{\Hyp}{\mathbf H}
\newcommand{\Sph}{\mathbf S}

\DeclareMathOperator{\card}{card}
\DeclareMathOperator{\cent}{center}
\DeclareMathOperator{\ch}{ch}
\DeclareMathOperator{\codim}{codim}
\DeclareMathOperator{\diag}{diag}
\DeclareMathOperator{\diam}{diam}
\DeclareMathOperator{\dom}{dom}
\DeclareMathOperator{\Gal}{Gal}
\DeclareMathOperator{\Hom}{Hom}
\DeclareMathOperator{\Jac}{Jac}
\DeclareMathOperator{\Lip}{Lip}
\DeclareMathOperator{\Met}{Met}
\DeclareMathOperator{\id}{id}
\DeclareMathOperator{\rad}{rad}
\DeclareMathOperator{\rank}{rank}
\DeclareMathOperator{\Hess}{Hess}
\DeclareMathOperator{\Radon}{Radon}
\DeclareMathOperator*{\Res}{Res}
\DeclareMathOperator{\sgn}{sgn}
\DeclareMathOperator{\singsupp}{sing~supp}
\DeclareMathOperator{\Spec}{Spec}
\DeclareMathOperator{\supp}{supp}
\DeclareMathOperator{\Tan}{Tan}
\newcommand{\tr}{\operatorname{tr}}

\newcommand{\Ric}{\mathrm{Ric}}
\newcommand{\Riem}{\mathrm{Riem}}
\newcommand{\LapQL}{\Delta^{\mathrm{ql}}}

\newcommand{\dbar}{\overline \partial}

\DeclareMathOperator{\atanh}{atanh}
\DeclareMathOperator{\csch}{csch}
\DeclareMathOperator{\sech}{sech}

\DeclareMathOperator{\Div}{div}
\DeclareMathOperator{\grad}{grad}
\DeclareMathOperator{\Ell}{Ell}
\DeclareMathOperator{\WF}{WF}

\newcommand{\Lagrange}{\mathscr L}
\newcommand{\DirQL}{\mathscr D^{\mathrm{ql}}}
\newcommand{\DirL}{\mathscr D^{\mathrm{lin}}}

\newcommand{\Hilb}{\mathcal H}
\newcommand{\Homology}{\mathrm H}
\newcommand{\normal}{\mathbf n}
\newcommand{\vol}{\mathrm{vol}}

\newcommand{\pic}{\vspace{30mm}}
\newcommand{\dfn}[1]{\emph{#1}\index{#1}}

\renewcommand{\Re}{\operatorname{Re}}
\renewcommand{\Im}{\operatorname{Im}}

\def\Japan#1{\left \langle #1 \right \rangle}

\newtheorem{theorem}{Theorem}[section]
\newtheorem{badtheorem}[theorem]{``Theorem"}
\newtheorem{prop}[theorem]{Proposition}
\newtheorem{lemma}[theorem]{Lemma}
\newtheorem{claim}[theorem]{Claim}
\newtheorem{proposition}[theorem]{Proposition}
\newtheorem{corollary}[theorem]{Corollary}
\newtheorem{conjecture}[theorem]{Conjecture}
\newtheorem{axiom}[theorem]{Axiom}
\newtheorem{assumption}[theorem]{Assumption}

\theoremstyle{definition}
\newtheorem{definition}[theorem]{Definition}
\newtheorem{remark}[theorem]{Remark}
\newtheorem{example}[theorem]{Example}
\newtheorem{notation}[theorem]{Notation}

\newtheorem{exercise}[theorem]{Discussion topic}
\newtheorem{homework}[theorem]{Homework}
\newtheorem{problem}[theorem]{Problem}

\newtheorem{ack}{Acknowledgements}

\numberwithin{equation}{section}


% Mean
\def\Xint#1{\mathchoice
{\XXint\displaystyle\textstyle{#1}}%
{\XXint\textstyle\scriptstyle{#1}}%
{\XXint\scriptstyle\scriptscriptstyle{#1}}%
{\XXint\scriptscriptstyle\scriptscriptstyle{#1}}%
\!\int}
\def\XXint#1#2#3{{\setbox0=\hbox{$#1{#2#3}{\int}$ }
\vcenter{\hbox{$#2#3$ }}\kern-.6\wd0}}
\def\ddashint{\Xint=}
\def\dashint{\Xint-}

%\usepackage{color}
%\hypersetup{%
%    colorlinks=true, % make the links colored%
%    linkcolor=blue, % color TOC links in blue
%    urlcolor=red, % color URLs in red
%    linktoc=all % 'all' will create links for everything in the TOC
%Ning added hyperlinks to the table of contents 6/17/19
%}

% style=alphabetic
\usepackage[backend=bibtex,maxcitenames=50,maxnames=50]{biblatex}
\addbibresource{topics.bib}
\renewbibmacro{in:}{}
\DeclareFieldFormat{pages}{#1}

\begin{document}
\begin{abstract}
Topics exam, Fall 2021.
\end{abstract}

\maketitle

%%%%%%%%%%%%%%%%%%%%%%%%%%%%%%%%%%%%%%%%%%%%%%%%%%%%%%%

% \tableofcontents

\section{Introduction}
Let $M$ be an oriented Riemannian manifold of metric $g$ and dimension $d$.

\begin{definition}\label{main definitions}
A function $u$ of locally bounded variation has \dfn{least gradient} if for every compactly supported function $v$ of bounded variation, such that $\supp v \subseteq U \Subset M$,
$$\int_U |du| ~\vol \leq \int_U |du + dv| ~\vol.$$
A set $U$ of locally finite perimeter has \dfn{least perimeter} if $1_U$ has least gradient.
\end{definition}

\begin{definition}
A \dfn{minimal lamination} in $M$ is a partition of a closed subset of $M$ into smooth hypersurfaces with zero mean curvature.
The minimal lamination $\lambda$ is \dfn{analytic} if $(M, g)$ is analytic and each of the hypersurfaces in $\lambda$ is analytic.
\end{definition}

This paper is dedicated to the proof of the following generalization of the maximum principle for least gradient functions on euclidean space \cite[Proposition 3.4]{górny2017planar}.

\begin{theorem}[maximum principle]\label{main thm}
Suppose that $2 \leq d \leq 7$ and
\begin{enumerate}
\item either $g$ has constant sectional curvature $\leq 0$,
\item or $g$ has constant curvature and $d = 2$.
\end{enumerate}
Let $u: M \to \RR$ be a function of least gradient, and $A_y = \partial \{u > y\}$.
Then $(A_y)_{y \in \RR}$ is a minimal lamination, which is analytic if $(M, g)$ is.
\end{theorem}

We are primarily concerned with the case that $M$ is a closed hyperbolic manifold, which is the setting of several erstwhile open problems stated in \cite[\S9]{daskalopoulos2020transverse} which Theorem \ref{main thm} can be used to solve. See \S\ref{duality}.

If $d = 2$, then Theorem \ref{main thm} can be extended to manifolds with boundary, just as in \cite[Corollary 3.5]{górny2017planar}:

\begin{theorem}[maximum principle up to the boundary]\label{main crly}
Let $\overline \Sigma$ be a convex surface with boundary and suppose that $u: \Sigma \to \RR$ is a function of least gradient defined on the interior $\Sigma$ of $\overline \Sigma$.
Then, if $A_y = \partial \{u > y\}$, $(A_y)_{y \in \RR}$ extends to a geodesic lamination of $\overline \Sigma$.
\end{theorem}

Theorems \ref{main thm} and \ref{main crly} can be easily shown to follow from standard results and the following regularity theorem for sets of least perimeter, which is the main theorem of the present paper:

\begin{theorem}[regularity of minimal hypersurfaces]\label{main lma}
Suppose that $2 \leq d \leq 7$ and
\begin{enumerate}
\item either $g$ has constant sectional curvature $\leq 0$,
\item or $g$ has constant curvature and $d = 2$.
\end{enumerate}
Then every set of least perimeter is bounded by a smooth minimal hypersurface $N$.
Furthermore, $N$ is analytic if $(M, g)$ is.
\end{theorem}

A proof of an analogous result to Theorem \ref{main lma} for currents paired against an elliptic integrand is given by \cite[\S5.3]{federer2014geometric}; our proof uses a similar strategy but is rather ``hands-on" in that it avoids the use of homological integration, instead using the theory of functions of bounded variation as developed by Miranda in \cite{Miranda64} \cite{Miranda66} \cite{Miranda67}, and facts about minimal hypersurfaces in euclidean space.

Though the hypotheses of Theorem \ref{main lma} look odd, we only use the assumption of nonpositive curvature at a critical point to deduce that a certain elliptic operator which we construct is actually the euclidean Laplace operator when written in correctly chosen coordinates.
In general, such an operator would be a perturbation of the euclidean Laplacian and so a suitable analysis of perturbations of elliptic operators should extend all of our results to the case of constant sectional curvature, so it is reasonable to conjecture the following:

\begin{conjecture}\label{main conj}
Suppose that $2 \leq d \leq 7$ and $g$ has constant sectional curvature.
Then the level sets of a function of least gradient are a minimal lamination.
\end{conjecture}

The assumption of constant sectional curvature should be somewhat harder to remove, as we rely on the existence of certain Killing fields, and at one point use the Killing-Hopf theorem to reduce a general result to the euclidean case.
However, if one could show Conjecture \ref{main conj}, a proof for any Riemannian manifold of appropriate dimension should not be out of reach.

%%%%%%%%%%%%%%%%%%%%%%%%%%%%%%%%%%%%%%%%%%%%%%%

\subsection{Outline of the paper}
We begin with the preliminaries.
In \S\ref{RiemMeasureThy}, we record facts that we will use about sets of locally finite perimeter on Riemannian manifolds.
We introduce the notion of a \dfn{bundle-valued Radon measure}, which is a generalized section of a vector bundle that, upon trivialization, becomes a vector-valued Radon measure.
With the basic theory of bundle-valued Radon measures it is easy to show that the reduced boundary of a set of least perimeter is independent of the choice of metric.
We also show that a coarea formula holds in this setting, which we use in \S\ref{LeastGradientFunctions} to develop the theory of functions of least gradient on Riemannian manifolds. A generalization of Miranda's theorem \cite[Teorema 3]{Miranda67} on the stability of functions of least gradient, and generalizations of its standard consequences, easily fall out from the coarea formula.
We then focus on sets of least perimeter: we prove a monotonicity formula, compute the dimension of the reduced boundary as $d - 1$, and prove the existence of tangent spaces to the reduced boundary.

We are then ready to prove Theorem \ref{main lma}.
In \S\ref{MollifierSection}, we show Proposition \ref{mollifier proposition}, which says that a set of least perimeter can be approximated by $C^1$ hypersurfaces that are approximately minimal.
This result is then used in \S\ref{DeGiorgiSection}, to prove a generalization, Proposition \ref{de Giorgi}, of the de Giorgi regularity lemma \cite[Teorema 5.7]{Miranda66} for sets of least perimeter.
Most of the difficulty in this step, and indeed in this present paper, comes from proving the $C^1$ case of de Giorgi's lemma, as Proposition \ref{mollifier proposition} can be used to reduce Proposition \ref{de Giorgi} to this case.

In the euclidean case, the $C^1$ de Giorgi lemma is proven by comparing the surface area of the graph of a $C^1$ function $\omega$ to its Dirichlet energy $\int |d\omega|^2~\vol$, and then apply the mean-value property and the fact that $[\Delta, \partial_j] = 0$.
This does not quite work in general, as the commutator of the Laplace-Beltrami operator with a coordinate vector field will in general be nonzero, and the mean-value theorem only approximately holds for general Laplace-Beltrami operators.
Our main innovation here is to show that for a suitable space form, one has enough Killing fields to carry out a similar argument, in which the Dirichlet energy is actually the Lagrangian for the euclidean Laplacian.

In \S\ref{proof of main thm}, we use a standard inductive argument to show that Proposition \ref{de Giorgi} implies Theorem \ref{main lma}.
We then use Theorem \ref{main lma} and the general theory developed in \S\ref{LeastGradientFunctions} to prove Theorems \ref{main thm} and \ref{main crly}.

In \S\ref{duality}, we use Theorem \ref{main thm} to use certain conjectures of Daskalopoulos and Uhlenbeck \cite{daskalopoulos2020transverse}.

%%%%%%%%%%%%%%%%%%%%%%%%%%%%%%%%%%%%%%%%%%%%%%%%

\subsection{Acknowledgements}
I would like to thank Georgios Daskalopoulos for suggesting this project and for many helpful discussions.

%%%%%%%%%%%%%%%%%%%%%%%%%%%%%%%%%%%%%%%%%%%%%%%%%%%%%%%%%%%%%%%%%%%%%%%%%%%%%%%%%%%%%%%%%

\section{Riemannian measure theory}\label{RiemMeasureThy}
\subsection{Notation and conventions}
\begin{notation}[presheaves]
If $F$ is a presheaf of function spaces, we write $u \in F_l(U)$ to mean that for every $V \Subset U$, $u \in F(V)$.
We write $u \in F_c(U)$ to mean that $u \in F(U)$ and $\supp u \Subset U$.
\end{notation}

\begin{notation}[volume forms]
We reserve the letter $d$ for dimension or exterior differentiation, and so to avoid awkwardness such as $\int |du| dV$ we write $\vol$ for the Riemannian volume form.
If $N$ is a closed submanifold we write $\vol_N$ to indicate the pullback of $\vol$ along the inclusion map $M \to N$.
\end{notation}

\begin{notation}[vector bundles]
Let $E$ be a vector bundle, which we will always assume is normed, with dual $E'$.
If $u,v$ are sections of $E',E$ respectively, we write $(u, v)$ for their fiberwise pairing, which is a function $M \to \RR$.
We write $\langle u, v\rangle$ or $\int_M (u, v) ~\vol$, for their $L^2$-duality pairing, which is a real number.
If $u$ is a section of $E$, we write $u \prec U$ to mean that $||u||_{L^\infty} \lesssim 1$ and $\supp u \Subset U$.
\end{notation}

\begin{notation}[Einstein summation]\label{EinsteinNotation}
We will have occasion to use the Einstein summation convention on a manifold which, in the smooth category, can be expressed as a product $H \times I$ where $H$ is $d-1$-dimensional and $I$ is $1$-dimensional.
Latin indices range over $1, \dots, d - 1$ and index coordinate directions on $H$; Greek indices range over $0, \dots, d - 1$, and the $0$th coordinate is the coordinate on $I$.
However, we stress that the metric on $H \times I$ is Riemannian rather than Lorentzian.
\end{notation}

\begin{definition}
Let $u$ be a smooth section.
Then $u$ is \dfn{as smooth as possible} if either $u$ is analytic or $g$ is not analytic.
\end{definition}

%%%%%%%%%%%%%%%%%%%%%%%%%%%%%%%%%%%%%%%%%%%%%%%%%%%%%%%%%%%%

\subsection{The Bochner integral}
Let $F$ be a separable Fr\'echet space over $\CC$, and $(\Omega, P)$ a measure space.
We can then define the \dfn{Bochner integral} of a $P$-measurable function $f: \Omega \to F$, which we write as $\int_\Omega f ~dP \in F$.
See \cite{Rieffel70}, \cite{MO47721}, and \cite[Chapter V]{yosida2012functional}.
We recall a few facts that we will need:

\begin{theorem}[Pettis]
Let $f: \Omega \to F$ be any function.
Then $f$ is $P$-measurable iff for every $X \in F'$, $\omega \mapsto \langle f(\omega), X\rangle$ is $P$-measurable.
In this case,
$$\left\langle \int_\Omega f ~dP, X\right\rangle = \int_\Omega \langle f(\omega), X\rangle ~dP(\omega).$$
If $F = \CC^r$, then the Bochner integral is just the componentwise Lebesgue integral.
\end{theorem}

%%%%%%%%%%%%%%%%%%%%%%%%%%%%%%%%%%%%%%%%%%%%%%%%%%%%%%%%%%%%

\subsection{Bundle-valued Radon measures}
Let $F \to M$ be a normed vector bundle of rank $r$.
We equip the space $C(K, F)$ of continuous sections of $F$ on a compact set $K$ with its supremum norm.
The Banach spaces $C(K, F)$ form an inverse system with respect to restriction and therefore define the topological vector space
$$C_0(M, F) = \varprojlim C(K, F).$$

According to the Riesz-Markov theorem, the space of $\CC^r$-valued Radon measures on $M$ is canonically identified with $C_0(M, (\CC^r)')'$, thus we define:

\begin{definition}
The topological dual space $\mathcal R(M, F) = C_0(M, F')'$ is the space of \dfn{$F$-valued Radon measures} on $M$.
\end{definition}

If we write $\mathcal R(U, F)$ to mean $\mathcal R(U, F|U)$, we equip $\mathcal R(U, F)$ with the weak topology of measures.
Then $\mathcal R(U, F)$ is a separable Fr\'echet space, with seminorms $|\langle \cdot, f_j\rangle|$ where $(f_j)$ is a countable basis for a dense subspace of $C_0(U, F)$.
Thus $\mathcal R(\cdot, F)$ is a sheaf of separable Fr\'echet spaces.

\begin{definition}
If $\omega \in \mathcal R(M, F)$ we define the \dfn{total variation} $|\omega|$ to be the positive Radon measure such that on every open set $U$,
$$|\omega|(U) = \sup_{X \prec U} \langle \omega, X\rangle,$$
where $X$ ranges over $C_0(M, F')$.
\end{definition}

Since $\RR_+$ acts on $F$ by scalar multiplication, we can define the \dfn{sphere bundle} $SF = (F \setminus 0)/\RR_+$.
Since $F$ has a norm, $SF$ is naturally identified with the bundle of elements of $F$ with length $1$.

\begin{proposition}[Riesz-Markov representation]\label{HanhJordan}
Let $\omega$ be a $F$-valued Radon measure and let $\mu = |\omega|$.
Then there exists a $\mu$-measurable section $f$ of $SF$ such that for every section $X \in C_0(M, F', \mu)$,
\begin{equation}\label{RNy formula}
\langle \omega, X\rangle = \int_M (f, X) ~d\mu.
\end{equation}
Furthermore, $f$ is unique up to a $\mu$-null set, and does not depend on the norm of $F$.
\end{proposition}
\begin{proof}
Fix an open cover $(U_i)$ of $M$ by charts which trivialize $F$, so that $U_i$ is precompact in $M$.
Let $(g_{ij})$ be the transition functions and $(g_{ij}')$ the induced transition functions for the dual bundle $F'$.
Then can view $\omega_i = \omega|U_i$ as an element of $C(U_i, (\CC^r)')'$, by the precompactness of $U_i$.
Hence by the Riesz-Markov theorem \cite[Theorem 4.14]{simon1983GMT}, there exists a $\mu$-measurable section $f_i$ of $SF$ for which (\ref{RNy formula}) holds for $\omega_i$, provided that $X \in C(U_i, (\CC^r)')$.

We now show that the $f_i$ are restrictions of a global section $f$, thus we must show $f_j = g_{ij} \circ f_i$ on $\CC^r$.
To this end, fix $X \in C_0(M, F)$ which is supported in $U_i \cap U_j$ and write $X_i \in C(U_i, (\CC^r)')$ for the trivialization of $X$ with respect to $U_i$.
Then $X_j = g_{ij}' \circ X_i$, and
\begin{align*}
\int_E (f_i, X_i) ~d\mu &= \langle \omega_i, X_i\rangle = \langle \omega_j, X_j\rangle = \int_E (f_j, X_j) ~d\mu = \int_E (f_j, g_{ij}' \circ X_i) ~d\mu.
\end{align*}

By Urysohn's lemma, $C_c(U_i \cap U_j, (\CC^r)', \mu)$ separates points in $L^1(U_i \cap U_j, \CC^r, \mu)$.
Therefore, since $X$ was arbitrary, $f_i = g_{ji} \circ f_j$; thus we obtain a unique global section $f$ of $SF$.

Finally, if we change the norm of $F$, replacing $|\cdot|$ with $|\cdot|'$, then we obtain a smooth function $h: F \to \RR_+$ so that if $v \in F_x$, then $|v|' = h(x, v)|v|$.
The change of norm gives us a new section $f'$ such that $f' = f/h(\cdot, f'(\cdot))$.
Thus $f'$ defines the same section of $SF$ as $f$.
\end{proof}

At this stage we have only defined $f$ as a $\mu$-equivalence class of sections of $SF$, so we now use the Lebesgue differentiation theorem to choose the ``correct" representative.
We state the differentiation theorem in a somewhat strange way, to ensure that the representative chosen is metric-independent.

\begin{definition}
A \dfn{Besicovitch cover} $\mathcal U$ of a metric space $X$ is a set of open balls, so that every $x \in X$ is the center of an element of $\mathcal U$.
The \dfn{Besicovitch number} $N \in \NN$ of $X$ is the best constant such that for every $x \in U$ and Besicovitch cover $\mathcal U$ of $B(x, 1/N)$, there exist $\mathcal U_1, \dots \mathcal U_N \subset \mathcal U$ such that $\bigcup_{n=1}^N \mathcal U_n$ is an open cover of $B(x, 1/N)$ and $\mathcal U_n$ is disjoint.
\end{definition}

It follows from the theory of \cite[\S2.8]{federer2014geometric} that for every Riemannian metric $g$, the Besicovitch number of $(M, g)$ is finite; \cite{Shi91} motivates why we restrict to small balls $B(x, 1/N)$.

For each $x \in M$, let $\mathcal A(x)$ denote the set of all pairs $(g, B, \varphi)$ where:
\begin{enumerate}
\item $g$ is a Riemannian metric on $M$,
\item $B$ is an open ball centered at $x$ with respect to $g$, and
\item $\varphi$ is a trivialization of $F$ over $B$.
\end{enumerate}
Then $\mathcal A(x)$ is a directed system, where the order is given by reverse inclusion of balls.
Given $(g, B, \varphi) \in \mathcal A(x)$, we obtain a $\mu$-measurable function $f_\varphi: B \to \CC^r$ obtained by trivializing the section $f$.
We define the average
$$f(g, B, \varphi) = \varphi^{-1}\left(\frac{1}{\mu(B)} \int_B f_\varphi ~d\mu\right),$$
which is a point in $F_x$.

\begin{proposition}[Lebesgue differentiation theorem]\label{LebDiff}
Let $\mu$ be a Radon measure on $M$, let $f \in L^1_l(M, SF, \mu)$, and let
$$f(x) = \lim_{(g, B, \varphi) \in \mathcal A(x)} f(g, B, \varphi).$$
Then the limit defining $f(x)$ converges for $\mu$-almost every $x \in M$ to a point in the sphere $SF_x$.
\end{proposition}
If $f(x)$ exists and is $\in SF_x$, we call $x$ a \dfn{Lebesgue point} of the section $f$.
\begin{proof}
This is obvious if $f$ has a representative in $C_c(M, SF)$; besides, by a partition of unity argument, we may assume that $\mu$ has compact support.
We can then select $(f_n)$ in $C_c(M, SF)$ converging in $L^1(M, SF, \mu)$ and almost everywhere to $f$.
Setting $h_n = |f_n - f|$, we can define the average
$$h_n(g, B) = \frac{1}{\mu(B)} \int_B h_n ~d\mu,$$
which converges to $0$ in $L^1(M, \mu)$.

Fix $N \in \NN$ and let $\mathcal B_N$ be the set of Riemannian metrics with Besicovitch number $\leq N$.
This makes sense if we restrict to a neighborhood of the compact support of $\mu$.
For each metric $g \in \mathcal B_N$, we have the Hardy-Littlewood inequality \cite[Lemma 4.1.1a]{Ledrappier85}
\begin{equation}\label{HardyLittlewood}
||\sup_{r \in (0, 1/N)} h_n(g, B_g(\cdot, r))||_{L^{1, \infty}(M, \mu)} \leq N ||h_n||_{L^1(M, \mu)}.
\end{equation}
By (\ref{HardyLittlewood}) and the convergence $h_n \to 0$ in $L^1$,
$$\lim_{n \to \infty} ||\sup_{0 \in (0, 1/N)} h_n(g, B_g(\cdot, r))||_{L^{1, \infty}(M, \mu)} = 0$$
uniformly in $g \in \mathcal B_N$.
Therefore we may pass to a subsequence along which, for $\mu$-almost every $x$,
$$\lim_{n \to \infty} \sup_{(g, r) \in \mathcal B_N \times (0, 1/N)} h_n(g, B_g(x, r)) = 0.$$
By the triangle inequality, if
$$\mathcal A_N(x) = \{(g, B, \varphi) \in \mathcal A(x): g \in \mathcal B_N\},$$
then (after passing to a subsequence again)
$$\lim_{n \to \infty} \sup_{(g, B, \varphi) \in \mathcal A_N(x)} |f_n(g, B_g(x, r), \varphi) - f(g, B_g(x, r), \varphi)| = 0.$$
But $f_n(g, B, \varphi) \to f(x)$ everywhere, $f(x) \in SF_x$, and $SF_x$ is closed, so there exists a $\mu$-null set $Z_N$ such that outside of $Z_N$,
$$\lim_{(g, B, \varphi) \in \mathcal A_N(x)} f(g, B, \varphi) \in SF_x.$$
Taking $Z = \bigcup_{N \in \NN} Z_N$, we see that $Z$ is $\mu$-null, which was to be shown.
\end{proof}

\subsection{Differentiation and boundary}
In this section we fix a Riemannian metric.

\begin{definition}
A function in $L^1(M)$ has \dfn{bounded variation} if its distributional derivative is a $T'M$-valued Radon measure of finite total variation.
We write $BV$ for the presheaf of functions of bounded variation.
An open set has \dfn{finite perimeter} if its indicator function has bounded variation.
\end{definition}

\begin{notation}
If $u$ is a function of locally bounded variation we write $du ~\vol$ for its derivative.
\end{notation}

Sequences $(u_n)$ in $BV_l(M)$ with $u_n \to u$ in $L^1_l(M)$ satisfy the lower semicontinuity property
\begin{equation}
\label{RieszMarkovDistr}
\int_M |du| ~\vol \leq \liminf_{n \to \infty} \int_M |du_n| ~\vol.
\end{equation}
which follows by testing against smooth functions, and the forgetful map
\begin{equation}\label{Forget}
BV_l(M) \to L^1_l(M)
\end{equation}
is compact. We refer to \cite[Chapter 1]{Giusti77} for a review of the space $BV_l(M)$.
Our next result can be deduced by applying a partition of unity argument and then copying the proof of \cite[Teorema 1]{Miranda67} verbatim:

\begin{proposition}[trace theorem]\label{traces}
Let $U$ be an open set such that $N = \partial U$ is a Lipschitz hypersurface.
For every $u \in BV_l(M)$ there exists a trace $v \in L^1_l(N)$ such that for every $X \in C_c(M, TM)$,
\begin{equation}\label{Miranda IBP}
\int_U (du, X) ~\vol + \int_U u ~\mathcal L_X\vol = \int_N vg(X, \normal_N) ~\vol_N.
\end{equation}
Moreover, $v$ is determined by the germ of $u$ at $\partial U$.
If $u$ is an indicator function then so is $v$.
\end{proposition}

Let $U$ be a set of locally finite perimeter.
The notion of reduced boundary to $U$ was first introduced in \cite{deGiorgi55}; see \cite{Battista_2021} for an equivalent definition.
To construct it, let $\omega = d1_U ~\vol$, which by Proposition \ref{HanhJordan} can be expressed as $\omega = \normal \mu$, where $\normal$ is a section of $ST'M$ which is independent of $g$.

\begin{definition}
Let $U,\normal$ be as above.
The \dfn{reduced boundary} $\partial^* U$ of a set $U$ of locally finite perimeter is the set of Lebesgue points of $\normal$.
The \dfn{conormal $1$-form} to $\partial^* U$ is $\normal$.
The \dfn{tangent bundle} $T\partial^* U$ to $\partial^* U$ is the kernel bundle of $\normal$.
\end{definition}

The tangent bundle is well-defined and gives a measurable vector bundle of rank $d-1$ over $\partial^* U$, because $\normal$ is nonzero almost everywhere, and so has constant rank $1$.
We prefer to work with $\normal$ than $\normal^\sharp$, to obtain metric-independence.
In fact, metric-independence and well-known facts about the euclidean case \cite[Chapters 2-4]{Giusti77} \cite{deGiorgi55} imply:

\begin{proposition}\label{locality of Caccioppoli}
Let $U$ be a set of locally finite perimeter.
Then:
\begin{enumerate}
\item $\partial^* U$ is either empty or $d-1$-dimensional in the Hausdorff sense, and is rectifiable with respect to $d-1$-dimensional Hausdorff measure.
\item $\partial^* U$ is dense in the measure-theoretic boundary $\partial U$.
\item If $\normal$ extends to a continuous $1$-form on $\partial U$, then $\partial^* U = \partial U$ is a $C^1$ hypersurface.
\end{enumerate}
\end{proposition}

\begin{notation}
We write $|\partial^* U|$ for $\int_M |d1_U| ~\vol$.
This does not collide with the notation $|U|$ for the volume of $U$, since $U$ has Hausdorff dimension $d$.
\end{notation}

%%%%%%%%%%%%%%%%%%%%%%%%%%%%%%%%%%%%%%%%%%%%%%%%%%%%%%

\subsection{The coarea formula} \label{coarea section}
Throughout this section we consider the superlevel sets $E_y = \{u > y\}$ of a function $u \in BV_l(M)$, and the resulting $T'M$-valued Radon measures
$$\omega(y) = d1_{E_y} ~\vol.$$
Let $\mu = |du| ~\vol$.

We first observe that for every $X \in \mathcal D(M, TM)$,
$$\langle \omega(y), X\rangle = -\int_{E_y} \mathcal L_X\vol$$
is measurable in $y$, since $E_y$ is monotone in $y$.
So by Pettis' theorem, $\omega$ is measurable in $y$ with respect to the weak topology of measures.

\begin{lemma}[coarea formula for measures]\label{Coarea1}
One has
$$du ~\vol = \int_{-\infty}^\infty \omega(y) ~dy.$$
\end{lemma}
\begin{proof}
Fix $X \in C_c(M, TM)$. We may assume that $u \geq 0$, and we must show
\begin{equation}
\label{gradient is integral of fibers}
\int_M (du, X) ~\vol = \int_{-\infty}^\infty \langle \omega(y), X\rangle ~dy.
\end{equation}
Since $u \geq 0$,
\begin{align*}
\int_M (du, X) ~\vol &= -\int_M u~\mathcal L_X\vol = -\int_M \int_0^{u(x)} dy ~\mathcal L_X\vol\\
&= -\int_0^\infty \int_{E_y} ~\mathcal L_X\vol ~dy = \int_0^\infty \langle \omega(y), X\rangle ~dy.
\end{align*}
by Fubini's theorem.
If $y < 0$ then $1_{E_y} = 1$ so $\omega(y) = 0$, so we conclude (\ref{gradient is integral of fibers}).
\end{proof}

Let $p: L \to M$ be the trivial line bundle with its induced metric $h$, and let $\eta$ be the volume form induced by $h$.
If $W$ is a vector field on $L$, we will write $W_1$ for the projection of $W$ onto $M$ and $W_2$ for its projection onto $\CC$.
Then Cartan's magic formula implies that if $W_2$ is constant, then
\begin{equation}
\label{Lie derivative computation}
\mathcal L_W\eta = \mathcal L_W\vol \wedge dy.
\end{equation}

\begin{lemma}\label{coarea converse}
Suppose that $W \in \mathcal D(L, TL)$ depends on a parameter $n \in \NN$, such that $W_2 = 0$ and for every $y \in \RR$, $X = W_1(\cdot, y)$ is a maximizing sequence for $\langle \omega(y), X\rangle$ subject to $X \prec U$.
Then
$$\int_{-\infty}^\infty \langle \omega(y), W(y)\rangle ~dy \leq \mu(U).$$
\end{lemma}
\begin{proof}
Let
$$E = \{(x, y) \in L: x \in E_y\}$$
be the undergraph of $u$.
By Fubini's theorem and (\ref{Lie derivative computation}),
\begin{align*}
\int_{-\infty}^\infty \langle \omega(y), W(y)\rangle ~dy &= -\int_{-\infty}^\infty \int_{E_y} \mathcal L_W\vol ~dy = -\iint_E ~\mathcal L_W\eta = \int_M (d1_E, W) ~\vol.
\end{align*}

Let $(u_m)$ be a mollification of $u$, so that $u_m \to u$ in the weak topology of distributions.
Then if $\chi$ is a cutoff, $\langle u_m, \chi\rangle \to \langle u, \chi\rangle$; taking a sequence of $\chi$ which increase to the indicator function of a compact set $K$, we conclude that $u_m \to u$ in $L^1(K)$, and hence $u_m \to u$ in $L^1_l$.

Let $E^{(m)}$ be the undergraph of $u_m$, and $E^{(m)}_y = \{u_m > y\}$.
Then for every test function $v$,
\begin{align*}
\langle 1_{E^{(m)}} - 1_E, v\rangle &= \int_{E^{(m)} \Delta E} v ~\vol \leq |(E^{(m)} \Delta E) \cap (\supp v \times \RR)| \cdot ||v||_{L^\infty}\\
&\leq ||v||_{L^\infty} \int_{\supp v} |u_m - u| ~\vol \to 0
\end{align*}
so $1_{E^{(m)}} \to 1_E$ in the weak topology of distributions.
Therefore
$$\lim_{m \to \infty} \int_M (d1_{E^{(m)}}, W) ~\vol = \int_M (d1_E, W) ~\vol.$$

Since $u_m$ is smooth, its graph $F_m = \partial E^{(m)}$ is a smooth manifold.
Let $\nu_m$ be the upwards unit normal field of $F_m$ and let $\vol_m$ be the volume form on $F_m$ induced by $\eta$.
Then
$$\langle d 1_{E^{(m)}}, W\rangle = -\iint_{E^{(m)}} \mathcal L_W\eta = -\int_{F_m} h(\nu_n, W) ~\vol_m.$$
Let $q_m = p|F_m$ and $Y_m$ be the vector field $(Y_m)_1 = -d u_m$, $(Y_m)_2 = 1$.
Since $F_m$ is a graph, $q_m: F_m \to M$ is a diffeomorphism, $(q_m)_*\nu_m = Y_m/|Y_m|$, and
$$(q_m)_* \vol_m = |Y_m| ~\vol.$$
Therefore
$$\int_{F_m} h(\nu_n, W) ~\vol_m = \int_M h(Y_m, W) ~\vol = \int_M g(d u_m, W_1) ~\vol = \int_M (d u_m, W_1) ~\vol.$$
Thus
\begin{align*}
|\langle d 1_E, W\rangle| &= \lim_{m \to \infty} |\langle d u_m, W_1\rangle| \leq \mu(U). \qedhere
\end{align*}
\end{proof}

\begin{proposition}[coarea formula for $BV_l$ functions]\label{Coarea2}
Let $u \in BV(M)$, let $E_y = \{u > y\}$, and let $\omega(y) = d1_{E_y} ~\vol$.
Then, if $\mu$ is the total variation of $du ~\vol$,
$$\mu = \int_{-\infty}^\infty |\omega(y)| ~dy.$$
\end{proposition}
\begin{proof}
By Lemma \ref{Coarea1} and the triangle inequality,
$$\mu \leq \int_{-\infty}^\infty |\omega(y)| ~dy.$$
So we just need to prove the converse.

Let $U \Subset M$.
Suppose that for every $y \in \RR$, $X = X^{(n)}_y$ is a maximizing sequence for $\langle \omega(y), X\rangle$ subject to $X \prec U$.
Since $\omega$ with respect to the weak topology of measures, for every $n$, $X^{(n)}_y(x)$ can be chosen to be measurable in $(x, y)$; indeed, we can take $X^{(n)}_y$ to be a smooth approximation to the Radon measure $d 1_{E_y}/|d 1_{E_y}|_{TV}$ in the weak topology of distributions, which is a product of the measurable functions $\omega$ and $y \mapsto 1/|\omega(y)|$.

By an approximation argument, we can find $W^{(n)} \in C_c(L, TL)$ such that $W^{(n)}_2 = 0$ and for every $y \in \RR$, $X = W^{(n)}_1(\cdot, y)$ is a maximizing sequence for $\langle d 1_{E_y}, X\rangle$ subject to $X \prec U$.
Let us now suppress the $n$ and write $W(y) = W^{(n)}(\cdot, y)$.

By Lemma \ref{coarea converse}, since $W$ has compact support, the integrand $\langle d 1_{E_y}, W(y)\rangle$ is uniformly bounded in $y$.
Therefore, by Fatou's lemma,
\begin{align*}
\int_{-\infty}^\infty |\omega(y)| ~dy &= \int_{-\infty}^\infty \lim_{n \to \infty} \langle \omega(y), W(y)\rangle ~dy \leq \liminf_{n \to \infty} \int_{-\infty}^\infty \langle \omega(y), W(y)\rangle ~dy \\
&\leq \mu(U). \qedhere
\end{align*}
\end{proof}

%%%%%%%%%%%%%%%%%%%%%%%%%%%%%%%%%%%%%%%%%%%%%%%%%%%%%%%

\section{Functions of least gradient}\label{LeastGradientFunctions}
The purpose of this paper is to study weak solutions to the PDE
\begin{equation}\label{EulerLagrange}
\Div \frac{\grad u}{|\grad u|} = 0.
\end{equation}
One can view (\ref{EulerLagrange}) as the formal limit as the $p$-Laplace equation as $p \to 1$; this perspective is taken in \cite[\S4]{daskalopoulos2020transverse}.
However, we will instead (\ref{EulerLagrange}) as the formal Euler-Lagrange equation induced by the variational problem of minimizing $|du|~\vol$, so that weak solutions to (\ref{EulerLagrange}) are functions of least gradient.

\begin{notation}
If $u \in BV(M)$ and $U \Subset M$, we write
$$\eta(u, U) = \inf_{v \prec U} \int_U |d(u+v)| ~\vol$$
so that $u$ has least gradient iff $\eta(u, U) = \int_U |du| ~\vol$ for every $U$.
\end{notation}

\begin{definition}
A \dfn{minimal cone} in $\RR^d$ is a cone of least perimeter with vertex at the origin of $\RR^d$.
\end{definition}

\begin{theorem}\label{minimal cones in R8}
The following are equivalent:
\begin{enumerate}
\item $d \leq 7$.
\item The boundary of every minimal cone in $\RR^d$ is $C^1$.
\item The boundary of every minimal cone in $\RR^d$ is a hyperplane.
\end{enumerate}
\end{theorem}
\begin{proof}
Immediate from \cite[Theorem 6.2.2]{Simons68} and \cite[Theorem A]{BOMBIERI1969}.
\end{proof}

Therefore we cannot sharpen Theorem \ref{main thm} to include the case $d \geq 8$.

\subsection{The Miranda stability theorem}\label{MirandaStability}
The exponential pullback $\exp_p^* u$ of a function $u$ of least gradient defined near $p \in M$ need not have least gradient.
However, in a small ball $B$ around $p$, we will be able to show that $\eta(u, B) \approx |du|_{TV}(B)$ in a sense to be made precise later.
This observation motivates the following definition.

\begin{definition}
A sequence $(u_n)$ of functions in $BV(M)$ has \dfn{approximately least gradient} if
$$\limsup_{n \to \infty} \int_U |du_n| ~\vol \leq \liminf_{n \to \infty} \eta(u_n, U)$$
uniformly as $U$ ranges over open sets $\Subset M$.
\end{definition}

To study sequences of approximately least gradient, we need a semicontinuity theorem for the total variation, which in the euclidean case was shown by Miranda \cite[Teorema 3]{Miranda67}.

\begin{definition}
Let $(u_n)$ be a sequence in $BV_l(M)$ which converges in $L^1_l$ to $u$.
We say that a Lipschitz hypersurface $N$ \dfn{has no singularities} of $(u_n)$ if:
\begin{enumerate}
\item \label{cond1Mir} $\sup_n \int_N |du_n| ~\vol = 0$.
\item \label{cond2Mir} $(u_n)$ is bounded in $L^1(N, \vol_N)$.
\item \label{cond3Mir} $\int_N |du| ~\vol = 0$.
\item \label{cond4Mir} $u_n \to u$ in $L^1(N, \vol_N)$.
\end{enumerate}
We say that $N$ \dfn{has no singularities} of $u \in BV_l(M)$ if $N$ has no singularities of the sequence $u_n = u$.
By Condition $k$ we mean the $k$th bullet in the above list.
\end{definition}

\begin{lemma}\label{probabilistic method}
Let $(u_n)$ be a sequence in $BV_l(M)$ which converges in $L^1_l(U)$. Then:
\begin{enumerate}
\item \label{probabilistic balls} For every $x \in M$ and $R > 0$ such that $B(x, R) \Subset M$ and almost every $r \in (0, R]$, $\partial B(x, r)$ has no singularities of $(u_n)$.
\item \label{probabilistic hypersurfaces} For every $U \Subset M$ there exists $U \subseteq V \Subset M$ such that $\partial V$ has no singularities of $(u_n)$.
\end{enumerate}
\end{lemma}
\begin{proof}
We first prove (\ref{probabilistic balls}).
Let $r$ be drawn from $[R/2, R]$ uniformly at random; we claim that almost surely, $\partial B(x, r)$ has no singularities of $(u_n)$.
Let
$$A = \{s > 0: \int_{\partial B(x, s)} |du| ~\vol > 0\}.$$
Then
$$\sum_{s \in A} \int_{\partial B(x, s)} |du| ~\vol \leq \int_{\partial B(x, R)} |du| ~\vol < \infty$$
since $|du|$ is a Radon measure and $B(x, R) \Subset M$.
Therefore $A$ is countable,
%Let $A_n = \{|du_n|_{TV}(N) > 0\}$ and let $A_\infty = \{|du|_{TV}(N) > 0\}$.
%Then for every $n \in \NN \cup \{\infty\}$, writing $u_\infty = u$,
%$$\sum_{s \in A_n} |du_n|_{TV}(\partial B(x, s)) \leq |du_n|_{TV}(B(x, R)) < \infty$$
%since $|du_n|_{TV}$ is a Radon measure and $B(x, R) \Subset M$.
%Since each of the summands is nonzero by definition of $A_n$, it follows that $A_n$ is countable, and in particular null.
%Therefore Conditions \ref{cond1Mir} and \ref{cond3Mir} hold almost surely.
so Condition \ref{cond3Mir} holds almost surely.
We omit the proof that the other conditions hold almost surely as it is similar.

To prove (\ref{probabilistic hypersurfaces}), let $U \Subset W \Subset M$, and for every $x \in \partial U$, let $R_x \in (0, d(x, \partial W))$.
Then, by (\ref{probabilistic balls}), for every $x \in \partial U$, there exists $r_x \in (0, R_x)$ such that $\partial B(x, r_x)$ has no singularities of $(u_n)$.
Let $\mathcal U$ be the open cover of $\overline U$ given by the balls $B(x, r_x)$, as well as $U$ itself.
Since $\overline U$ is compact, there exists a finite subcover $\mathcal U_0$ of $\mathcal U$.
Let $V$ be the union of the sets in $\mathcal U_0$.
Then $\partial V$ is the boundary of a union of finitely many balls $B(x, r_x)$ whose boundaries have have no singularities, and therefore has no singularities.
\end{proof}

We recall that $BV_l(M)$ is not separable, so it will be useful to have a somewhat weaker topology on $BV_l(M)$, as follows:

\begin{definition}
A sequence of functions $(u_n)$ in $BV_l(M)$ converges \dfn{in total variation on sets with no singularities} to $u \in BV_l(M)$ if $u_n \to u$ in $L^1_l(M)$ and for every set $A \Subset M$ such that $\partial A$ has no singularities,
\begin{equation}\label{convergence in TV}
\lim_{n \to \infty} \int_A |du_n| ~\vol = \int_A |du| ~\vol.
\end{equation}
\end{definition}

We omit the proofs of our next few results, as they follow easily from Proposition \ref{traces} and the proofs of \cite[Teorema 2]{Miranda67} and \cite[Lemma 5.6]{Giusti77}.

\begin{proposition}[gluing]\label{gluing}
Let $N$ be a Lipschitz hypersurface which separates $M$ into $U_1,U_2$.
If $u_j \in BV(U_j)$ and $u \in L^1_l(M)$ is the function such that $u|U_j = u_j$, then $u \in BV(M)$.
Moreover,
\begin{equation}
\label{glued BV norm}
\int_N |du| ~\vol = \int_N |u_1 - u_2| ~\vol_N.
\end{equation}
\end{proposition}

\begin{lemma}\label{estimates on good set}
Let $u, v \in BV(M)$, let $U \Subset M$ have a Lipschitz boundary $N$. Then
\begin{equation}\label{a priori estimate 1}
|\eta(u, U) - \eta(v, U)| \leq \int_N |u - v| ~\vol_N.
\end{equation}
In particular
\begin{equation}\label{a priori estimate 2}
\eta(u, U) \leq \int_N |u| ~\vol_N.
\end{equation}
\end{lemma}

\begin{proposition}[Miranda stability theorem]\label{Miranda convergence}
If a sequence of functions $(u_n)$ has approximately least gradient and converges in $L^1_l$, then its limit $u$ has least gradient, and $u_n \to u$ in total variation on sets with no singularities.
\end{proposition}
\begin{proof}
By Lemma \ref{probabilistic method} for every $U$ open $\Subset M$ we can find $U \subseteq V \Subset M$ such that $V$ is open and $\partial V$ has no singularities.
Now the proof is identical to that of \cite[Teorema 3]{Miranda67}, using Proposition \ref{gluing} and Lemma \ref{estimates on good set} whenever the proof of \cite[Teorema 3]{Miranda67} calls for their euclidean counterparts.
\end{proof}

\begin{corollary}\label{level sets are minimal}
For every $u$ of least gradient, the superlevel sets $\{u > t\}$ have least perimeter.
\end{corollary}
\begin{proof}
In the proof of \cite[Theorem 1]{BOMBIERI1969}, replace \cite[Theorem 1.6]{Miranda66} with Proposition \ref{Coarea2} and replace \cite[Theorem 3]{Miranda67} with Proposition \ref{Miranda convergence}.
\end{proof}

\begin{corollary}\label{compactness}
Let $(u_n)$ be a sequence of indicator functions of approximately least gradient.
Then there is a subsequence of $(u_n)$ which converges almost everywhere and in total variation on sets with no singularities to the indicator function of a set of least perimeter.
\end{corollary}
\begin{proof}
If $n$ is large enough, then by Proposition \ref{traces}, for every $U \Subset M$,
$$\int_U |du_n| ~\vol \leq \eta(u_n, U) + 1 \leq |\partial U| + 1$$
which gives a uniform bound in $BV_l$.
Since the forgetful map (\ref{Forget}) is compact, a subsequence of $(u_n)$ converges to a function $u$ in $L^1_l$.
By Proposition \ref{Miranda convergence}, $u$ has least gradient and (\ref{convergence in TV}) holds.
By taking a further subsequence we can guarantee the convergence pointwise almost everywhere.
The convergence almost everywhere implies that there is a set $U$ of locally finite perimeter such that $u = 1_U$, which necessarily has least perimeter.
\end{proof}

%%%%%%%%%%%%%%%%%%%%%%%%%%%%%%%%%%%%%

\subsection{Monotonicity}\label{inequalities}
Fix $N$, a smooth minimal hypersurface in $M$, and $P \in N$. Then one has
\begin{equation}\label{classic monotonicity}
\frac{d}{dr} e^{Ar^2}r^{1 - d} |N \cap B_r| \geq 0
\end{equation}
for some $A \in \RR$, where as usual $B_r = B(P, r)$ \cite[\S7]{MarquesXX}.
This is not quite good enough for our purposes because we need sharper lower bound than $0$ and because we cannot assume that $N$ is smooth.
However, sharper monotonicity formulae are available on euclidean space \cite[Proposition 5.12]{Giusti77}, and so in this section, we provide a highly streamlined version of the argument of \cite[Chapter 5]{Giusti77}, suitably modified to account for the presence of a nonzero Ricci tensor, which is the source of the constant $A$ in (\ref{classic monotonicity}).
We begin by considering some general facts about polar coordinates:

\begin{lemma}\label{rescale the sphere form}
Let $s < r$ and view $\vol_{\partial B_r}$ as a volume form on $\Sph^{d - 1}$.
Then as $r \to 0$,
$$\frac{\vol_{\partial B_r}}{\vol_{\partial B_s}}(\Theta) = \frac{r^{d - 1}}{s^{d - 1}} \frac{1 - r^2 \Ric_P(\Theta, \Theta)/6 + O(r^3)}{1 - s^2 \Ric_P(\Theta, \Theta) + O(s^3)}.$$
\end{lemma}
\begin{proof}
Writing $dx$ for the usual euclidean volume form, expanding $\vol$ in polar coordinates gives
$$\vol_{\partial B_r} = \iota_{\partial r} \vol = (1 - \Ric_P(r\Theta, r\Theta)/6 + O(r^3)) ~\iota_{\partial r} dx.$$
But $\iota_{\partial r} dx = r^{d - 1} \vol_{\Sph^{d - 1}}(\Theta)$ whence
\begin{align*}
\frac{\vol_{\partial B_r}}{\vol_{\partial B_s}}(\Theta) &= \frac{r^{d - 1}}{s^{d - 1}} \frac{1 - r^2 \Ric_P(\Theta, \Theta)/6 + O(r^3)}{1 - s^2 \Ric_P(\Theta, \Theta)/6 +
O(s^3)} \frac{\vol_{\Sph^{d - 1}}(\Theta)}{\vol_{\Sph^{d - 1}}(\Theta)}. \qedhere
\end{align*}
\end{proof}

\begin{lemma}\label{GaussLeibniz}
If $r$ is so small that Gauss' polar coordinates lemma applies to $B_r$, then
$$\frac{d}{dr} \int_{B_r} |du| ~\vol = \int_{\partial B_r} |du| ~\vol_{\partial B_r}.$$
\end{lemma}
\begin{proof}
Let $\mathbf v_{B_r}$ be the velocity of $B_r$.
By the Leibniz integral theorem,
$$\frac{d}{dr} \int_{B_r} |du| ~\vol = \int_{\partial B_r} g(\mathbf v_{B_r}, \normal_{B_r}) |du| ~\vol_{\partial B_r},$$
but by Gauss' lemma, $\mathbf v_{B_r} = \partial_r = \normal_{B_r}$ and $g(\partial_r, \partial_r) = 1$.
\end{proof}

\begin{lemma}\label{quasiradial}
Let $\partial_0$ be a smooth unit-length vector field near $P$. Then for every $\Theta \in \Sph^{d - 1}$,
\begin{equation}\label{quasiradial claim}
|1 - |g(\partial_0, \partial_r)(r, \Theta)|| \lesssim r.
\end{equation}
\end{lemma}
\begin{proof}
We can find smooth functions $f_0, \dots, f_{d - 1}$ such that on $\{r > 0\}$,
$$\partial_0 = f_0 \partial_r + f_1 \partial_{\theta_1} + \cdots + f_{d - 1} \partial_{\theta_{d - 1}}$$
where $\Theta = (\theta_1, \dots, \theta_{d - 1})$.
By Gauss' lemma, $g(\partial_0, \partial_r) = f_0$, but nontrivial linear combinations of the vector fields $\partial_{\theta_i}$ cannot be continuouly extend to $P$.
Therefore in order for $\partial_0$ to be continuous, $\sum_i f_i \partial_{\theta_i} \to 0$ as $r \to 0$, whence $f_0 \to 1$ since $\partial_0$ has unit length.
Since $r \mapsto g(\partial_0, \partial_r)(r, \Theta)$ is smooth and tends to $1$ it must convergence at least at a linear speed.
\end{proof}

\begin{proposition}\label{Monotonicity Formula}
For every $P \in M$ there exists a continuous function $f$ such that $f(r) = O(r^2)$ as $r \to 0$, and such that for every function $u \geq 0$ of least gradient, every $0 < r_1 < r_2$ small enough, and every smooth unit-length vector field $\partial_0$,
\begin{align*}
0
&\leq \left(r_2^{1 - d} \int_{B_{r_2}} \partial_0u ~\vol - r_1^{1 - d} \int_{B_{r_1}} \partial_0u ~\vol \right)^2\\
&\leq 2\left[r_2^{1 - d}\left(1 + \log \frac{r_2}{r_1}\right) \int_{B_{r_2}} |du| ~\vol\right] \left[e^{f(r_2)} r_2^{1 - d}\int_{B_{r_2}} |du| ~\vol - e^{f(r_1)} r_1^{1 - d} \int_{B_{r_1}} |du| ~\vol \right].
\end{align*}
In particular, for every $r > 0$ small enough,
\begin{equation}\label{TrueMonotonicity}
\frac{d}{dr} e^{f(r)} r^{1 - d} \int_{B_r} |du| ~\vol \geq 0.
\end{equation}
Finally, $f$ can be taken uniform on compact subsets of $M$.
\end{proposition}
\begin{proof}
To set up the proof, let $A = B_{r_2} \setminus B_{r_1}$ and let $F(r) = r^{1 - d} \int_{B_r} |du| ~\vol$.
By an approximation argument we may assume that $u$ is $C^1$, and by rescaling $g$ suitably we may assume that Gauss' polar coordinates lemma applies on $B_1$.
We omit the details of these reductions.

We start the proper proof by integrating by parts:
\begin{align*}
\int_{B_r} \partial_0u ~\vol &= \int_{\partial B_r} ug(\partial_0, \partial_r) ~\vol_{\partial B_r} \\
&= \int_{\partial B_1} u(r, \Theta) g(\partial_0, \partial_r)(r, \Theta) \frac{\vol_{\partial B_r}(\Theta)}{\vol_{\partial B_1}(\Theta)} ~\vol_{\partial B_1}(\Theta)
\end{align*}
and hence we may estimate
\begin{align*}
|\int_{B_{r_2}} &r_2^{1 - d} \partial_0u ~\vol -\int_{B_{r_1}} r_1^{1 - d} \partial_0u ~\vol|\\
&\leq \int_{\partial B_1} \left|r_2^{1 - d}u(r_2, \Theta) g(\partial_0, \partial_r)(r_2, \Theta) \frac{\vol_{\partial B_{r_2}}(\Theta)}{\vol_{\partial B_1}(\Theta)}
- r_1^{1 - d}u(r_1, \Theta) g(\partial_0, \partial_r)(r_1, \Theta) \frac{\vol_{\partial B_{r_1}}(\Theta)}{\vol_{\partial B_1}(\Theta)}\right|
~\vol_{\partial B_1}(\Theta) \\
&\leq 1.5 \int_{\partial B_1} |u(r_2, \Theta) - u(r_1, \Theta)| + |u(r_1, \Theta)|
\left|1 - \frac{g(\partial_0, \partial_r)(r_2, \Theta) \vol_{\partial B_{r_2}}(\Theta)}{g(\partial_0, \partial_r)(r_1, \Theta) \vol_{\partial B_{r_1}}(\Theta)}\right| ~\vol_{\partial B_1}(\Theta)
\end{align*}
for $r_2$ small enough, where the estimate
$$\left|g(\partial_0, \partial_r)(r_1, \Theta) \frac{\vol_{\partial B_{r_1}}(\Theta)}{\vol_{\partial B_1}(\Theta)}\right| \leq 1.5 r_2^{1 - d}$$
follows from Lemmata \ref{rescale the sphere form} and \ref{quasiradial}, if $r_2$ is chosen so small that the error factor from Lemma \ref{quasiradial} is at most $1.5$.

To control the error term, we further bound
\begin{align*}
\int_{\partial B_1} |u(r_1, \Theta)|
&\left|1 - \frac{g(\partial_0, \partial_r)(r_2, \Theta) \vol_{\partial B_{r_2}}(\Theta)}{g(\partial_0, \partial_r)(r_1, \Theta) \vol_{\partial B_{r_1}}(\Theta)}\right| ~\vol_{\partial B_1}(\Theta) \\
&\leq ||u(r_1, \cdot)||_{L^\infty} |\partial B_1| \sup_{\Theta \in \Sph^{d - 1}} \left|1 - \frac{g(\partial_0, \partial_r)(r_2, \Theta) \vol_{\partial B_{r_2}}(\Theta)}{g(\partial_0, \partial_r)(r_1, \Theta) \vol_{\partial B_{r_1}}(\Theta)}\right|
\end{align*}
and using Lemmata \ref{rescale the sphere form} and \ref{quasiradial}, one can show that
$$\sup_{\Theta \in \Sph^{d - 1}} \left|1 - \frac{g(\partial_0, \partial_r)(r_2, \Theta) \vol_{\partial B_{r_2}}(\Theta)}{g(\partial_0, \partial_r)(r_1, \Theta) \vol_{\partial B_{r_1}}(\Theta)}\right| \lesssim r_2$$
so we finally have
$$
|\int_{B_{r_2}} r_2^{1 - d}u ~\vol - \int_{B_{r_1}} r_1^{1 - d}u ~\vol|
\leq O(r_2) + 1.5 \int_{\partial B_1} |u(r_2, \Theta) - u(r_1, \Theta)| ~\vol_{\partial B_1}.$$
Using a similar argument to the above, one can show that
$$1.5 \int_{\partial B_1} |u(r_2, \Theta) - u(r_1, \Theta)| ~\vol_{\partial B_1} \leq O(r_2) + 2 \int_{\partial A} r^{1 - d} ug(\partial_r, \normal_{\partial A}) ~\vol_{\partial A}.$$

Since $u \geq 0$,
$$\int_A \partial_r  r^{1 - d}u ~\vol = (1 - d)\int_A r^{-d}u ~\vol \leq 0$$
whence integration by parts gives
$$\int_{\partial A} r^{1 - d} u g(\partial_r, \normal_{\partial A}) ~\vol_{\partial A} \leq \int_A r^{1 - d} \partial_r u ~\vol \leq \int_A r^{1 - d} |\partial_r u| ~\vol.$$
By the Cauchy-Schwarz inequality,
\begin{equation}\label{monotonicity CauchySchwarz}
\left(\int_A r^{1 - d} |\partial_r u| ~\vol \right)^2 \leq \left(\int_A r^{1 - d}|du|~\vol\right)\left(\int_A r^{1 - d}\frac{(\partial_ru)^2}{|du|} ~\vol\right) =: IJ.
\end{equation}
To estimate $J$ we fix $r^* \in [r_1, r_2]$ and introduce a competitor $v(r, \Theta) = u(r^*, \Theta)$. Then $\partial_r v = 0$, so
$$\int_{B_{r^*}} |du| ~\vol \leq \int_{B_{r^*}} |dv| ~\vol = \int_{B_{r^*}} |\partial_\Theta v| ~\vol = \int_0^{r^*} \int_{\partial B_r} |\partial_\Theta v(r, \Theta)| ~\vol_{\partial B_r}(\Theta) ~dr.$$
By Lemma \ref{rescale the sphere form},
$$\int_{\partial B_r} |\partial_\Theta v(r, \Theta)| ~\vol_{\partial B_r}(\Theta) \leq \frac{r^{d - 1}}{(r^*)^{d - 1}}(1 + O((r^*)^2)) \int_{\partial B_{r^*}} |\partial_\Theta v(r^*, \Theta)| ~\vol_{\partial B_{r^*}}(\Theta).$$
But $\partial_\Theta v = \partial_\Theta u$, so for every $r \in [r_1, r_2]$,
\begin{align*}
\int_{B_r} |du| ~\vol &\leq \frac{r + O(r^3)}{d - 1} \int_{\partial B_r} |\partial_\Theta u| ~\vol_{\partial B_r}\\
&= \frac{r + O(r^3)}{d - 1} \int_{\partial B_r} |du| \sqrt{1 - \frac{(\partial_r u)^2}{|du|^2}} ~\vol_{\partial B_r} \\
&\leq \frac{r + O(r^3)}{d - 1} \int_{\partial B_r} |du| - \frac{1}{2} \frac{(\partial_r u)^2}{|du|} ~\vol_{\partial B_r}.
\end{align*}
Therefore
\begin{align*}
J &= \int_{r_1}^{r_2} r^{1 - d} \int_{\partial B_r} \frac{(\partial_r u)^2}{|du|} ~\vol \\
&\leq 2 \int_{r_1}^{r_2} r^{1 - d} \left[\int_{\partial B_r} |du| ~\vol_{\partial B_r} + \frac{d - 1}{r + O(r^3)} \int_{B_r} |du| ~\vol\right] ~dr.
\end{align*}
By Lemma \ref{GaussLeibniz},
\begin{align*}
F'(r) &= r^{1 - d}\left[\frac{d - 1}{r} \int_{B_r} |du| ~\vol + \int_{\partial B_r} |du| g(\partial_r, \partial_r) ~\vol_{\partial B_r}\right]\\
&= r^{1 - d}\left[\frac{d - 1}{r + O(r^3)} \int_{B_r} |du| ~\vol + \int_{\partial B_r} |du| ~\vol_{\partial B_r} + O(r^3) \int_{B_r}|du| ~\vol\right].
\end{align*}
Moreover, we can use integration by parts to estimate
$$\int_{r_1}^{r_2} r^2 F(r) ~dr = r_2^2 F(r_2) - r_1^2 F(r_1) + O(r_2^3)(F(r_2) - F(r_1)).$$
Using the Taylor expansion of $e^x$ we can choose $f(r) = O(r^2)$ so that by the fundamental theorem of calculus,
\begin{align*}
J &\leq 2\int_{r_1}^{r_2} F'(r) + O(r^2)F(r) ~\vol ~dr = 2(e^{f(r_2)} F(r_2) - e^{f(r_1)} F(r_1)).
\end{align*}

We now estimate $I$ using Lemma \ref{GaussLeibniz} and integration by parts, namely
\begin{align*}
I &= \int_{r_1}^{r_2} r^{1 - d} \frac{d}{dr} \int_{B_r} |du| ~\vol ~dr \\
&= F(r_2) - F(r_1) - \int_{r_1}^{r_2} \int_{B_r} |du| ~\vol ~\frac{d}{dr} r^{1 - d} ~dr\\
&\leq F(r_2) + (d - 1) \int_{r_1}^{r_2} r^{-d} \int_{B_r} |du| ~\vol ~dr
\end{align*}
since clearly $F(r_1) \geq 0$.
The monotonicity formula (\ref{TrueMonotonicity}) follows for $N = \partial^* U$ from (\ref{monotonicity CauchySchwarz}), the estimate on $J$, and the fact that $I \geq 0$.
So
$$r^{-d} \int_{B_r} |du| ~\vol = r^{-1} F(r) \leq r^{-1} F(r_2).$$
This gives us the desired estimate
\begin{align*}
I &\leq F(r_2)\left[1 + \int_{r_1}^{r_2} \frac{dr}{r}\right] = F(r_2)\left[1 + \log\frac{r_2}{r_1}\right].
\end{align*}
Finally, the local uniformity of $f$ follows easily from the proof thus far and the fact that $g$ and its Ricci tensor are locally bounded.
\end{proof}


%%%%%%%%%%%%%%%%%%%%%%%%%%%%%%%%%%%%%%%%%%%%%%%%%%%%%%%%%%%%%

\subsection{Dimension of the reduced boundary}
Our next purpose is to show that the reduced boundary of a set of least perimeter has dimension $d - 1$ in the following sense:

\begin{definition}
Let $\mu$ be a Radon measure on a metric space $X$.
The \dfn{local dimension} $\dim_P \mu$ of $\mu$ at $P \in X$ is
$$\dim_P \mu = \lim_{r \to 0} \frac{\log \mu(B(P, r))}{\log r}.$$
\end{definition}

Equivalently, $\dim_P \mu = \ell$ iff $\mu(B(P, r)) \sim r^\ell$ as $r \to 0$.

\begin{proposition}\label{doubling dimension}
Let $U$ be a set of least perimeter and $P \in \partial^* U$. Then
$$\dim_P |d1_U|~\vol = d - 1.$$
\end{proposition}
\begin{proof}
Since $u = 1_U$ has least gradient, in particular its gradient in $1_{\overline U}$ is less than that of $1_{B_r}$, so if we set $\mu =|d1_U| ~\vol$, then
$$\mu(B_r) \leq |U \cap \partial B_r| \lesssim r^{d - 1}.$$
Conversely, let $\nu = 1_U ~\vol$, and observe that if we set $q = d/(d - 1)$, then by Sobolev embedding\footnote{Sobolev and Poincar\'e inequalities hold for $BV$ functions on $\RR^d$ \cite[\S5.6.1]{evans1991measure},
and since locally a Riemannian volume form is a perturbation of the euclidean volume form, the inequalities that we use here also hold.}
$$\nu(B_r)^{1/q} = ||1_{E \cap B_r}||_{L^q} \lesssim \int_M |d1_{E \cap B_r}| ~\vol = |\partial^*(U \cap B_r)|.$$
The reduced boundary satisfies the ``Leibniz inequality"
$$\partial^*(U \cap B_r) \subseteq (\partial^* U \cap B_r) \cup (U \cap \partial B_r)$$
so
$$\nu(B_r)^{1/q} \leq |\partial^* U \cap B_r| + |U \cap \partial B_r| \leq 2|U \cap \partial B_r| = 2\frac{d}{dr} \nu(B_r).$$
We make the change of variables $u(r) = \nu(B_r)^{1/q}$, so that we obtain
$$u(r) \leq 2 \frac{d}{dr} u(r)^q = 2qu(r)^{q - 1}u'(r)$$
and the initial condition $u(0) = 0$. Thus $u^{2 - q} \leq 2qu'$.
Since $q > 1$, a separation of variables yields
$$2qr \leq \int_0^{u(r)} \tilde u^{q - 2}~d\tilde u = \frac{u(r)^{q - 1}}{q - 1}.$$
Thus
$$\nu(B_r) \gtrsim r^{q/(q - 1)} = r^d.$$
Since $\partial^* (M \setminus U) = \partial^* U$, if $\overline{\nu} = |d1_{M \setminus U}| ~\vol$, then we similarly have $\overline{\nu}(B_r) \gtrsim r^d$.

We now fix $r > 0$ and let $[u] = |B_r \cap U|/|B_r|$ be the mean of $u$ on $B_r$. Then by the Poincar\'e inequality,
\begin{align*}
\mu(B_r) &= \int_{B_r} |du|~\vol \geq r^{-1} ||u - [u]||_{L^1(B_r)} \\
&= \frac{1}{r} \left[\int_{B_r \cap U} 1 -[u] ~\vol + \int_{B_r \setminus U} [u] \right] \\
&= \frac{|B_r||B_r \cap U| - |B_r \cap U|^2 + |B_r \cap U||B_r \setminus U|}{r|B_r|} \\
&= 2\frac{|B_r \cap U||B_r \setminus U|}{r|B_r|}\\
&\gtrsim r^{-d-1} \nu(B_r) \overline{\nu}(B_r) \gtrsim r^{2d-d-1} = r^{d - 1}. \qedhere
\end{align*}
\end{proof}

\subsection{Blowup of the reduced boundary}
Now let us study the blowup of $M$ at a point $p$ on the reduced boundary of a set $U$ of least perimeter, giving a generalization of the conjunction of \cite[Theorem 9.3]{Giusti77} and \cite[Theorem 6.2.2]{Simons68}.

\begin{definition}
For a function $u$ on $M$, $P \in M$ we define the \dfn{tangent rescaling} of $u$ at $P$ to be the net of functions
$$u_t(v) = u\left(\exp_P(tv)\right)$$
on $T_PM$, as $t \to 0$.
\end{definition}

We always view $T_PM$ as carrying the euclidean metric induced by $g$, which defines the notion of approximately least gradient used in the following lemma.

\begin{lemma}\label{almost blowup theorem}
Suppose that $U$ is a set of least perimeter near $P$, $P \in \partial^* U$, and $u = 1_U$.
Then the tangent rescaling $(u_t)$ of $u$ has approximately least gradient in every ball $B_{T_PM}(0, r)$.
\end{lemma}
\begin{proof}
We write $|\cdot|'$, $\vol'$ for the notions taken in the tangent space with its euclidean geometry.
We also write $U_t$ for the set indicated by $u_t$.
If $V$ is a precompact open subset of $T_PM$, $V_t = \{v \in T_PM: tv \in V\}$, then
\begin{equation}\label{almost blowup volume form}
\frac{\exp_P^* \vol}{\vol'} = 1 - \frac{(\Ric_P)_{ij}v^iv^j}{6} + \cdots = 1 + O(t^2)
\end{equation}
on $V_t$ and we have the scale-invariance
\begin{equation}\label{almost blowup scale invariance}
|\partial^* U_t \cap V|' = t^{1 - d}|\partial^* U_1 \cap V_{1/t}|'.
\end{equation}

From (\ref{almost blowup scale invariance}, \ref{almost blowup volume form}),
$$t^{d - 1} |\partial^* U_t \cap V|' = |\partial^* U_1 \cap V_{1/t}|' \leq (1 + O(t^2)) |\partial^* U \cap \exp_P(V_{1/t})|.$$
For every $w \in BV_c(V)$, the least-gradient nature and (\ref{almost blowup volume form}) of $u$ gives
$$|\partial^* U \cap \exp_P(V_{1/t})| \leq \int_{V_{1/t}} |d(u + (\exp_P)_* w_{1/t})| ~\vol \leq (1 + O(t^2))\int_{V_{1/t}} |d(u_1 + w_{1/t})| ~\vol'.$$
Therefore by (\ref{almost blowup scale invariance}, \ref{almost blowup volume form}),
$$|\partial^* U_t \cap V|' \leq (t^{1 - d} + O(t^{3 - d})) \int_{V_{1/t}} |d(u_1 + w_{1/t})| \vol' = (1 + O(t^2)) \int_V |d(u_t + w)| ~\vol'.$$
Since $V,w$ were arbitrary, we conclude that $(u_t)$ has approximately least gradient.
\end{proof}

\begin{proposition}\label{blowup theorem}
Suppose that $U$ is an open set with least perimeter in $B(P, r)$, $P \in \partial^* U$, and $u = 1_U$.
Furthermore, suppose that $d \leq 7$.
Then the tangent rescaling of $u$ converges along a subsequence, in $L^1_l$ and in total variation on sets with no singularities, to the indicator function of a hyperplane $T_P \partial^* U$ in $T_PM$ such that $0 \in T_P \partial^* U$.
\end{proposition}
\begin{proof}
By Lemma \ref{almost blowup theorem} and Corollary \ref{compactness}, there exists a set $C$ of least perimeter in $T_PM$, such that the tangent rescaling converges to $1_C$.
But $T_PM$ is isometric to $\RR^d$, $d \leq 7$, so by the Bernstein--Fleming theorem \cite[Theorem 17.3]{Giusti77} \cite[\S5]{Fleming62}, $\partial C$ is a hyperplane.
The fact that $0 \in \partial C$ follows from the fact that $P \in \partial^* U$.
\end{proof}

\begin{definition}
The hyperplane $T_P \partial^* U$ given by Proposition \ref{blowup theorem} is a \dfn{tangent space} to $\partial^* U$ at $P$.
\end{definition}

We stress that, until we show that that $\partial^* U$ is a smooth hypersurface, it is far from obvious from that the tangent hyperplane is unique.
Moreover, if $d \geq 8$, Theorem \ref{minimal cones in R8} precludes that we can even show \emph{existence} of a tangent hyperplane.

%%%%%%%%%%%%%%%%%%%%%%%%%%%%%%%%%%%%%%%%%%%%%%%%%%%%%%

\section{Mollification of sets of least perimeter}\label{MollifierSection}
In this section our goal is to show that given a set $U$ of least perimeter, we can find a $C^1$ hypersurface $N$ which approximates $\partial^* U$ arbitrarily well, in the following precise sense.

\begin{definition}
By a \dfn{quasieuclidean frame} at $P$ we mean a tuple of vector fields $(\partial_0, \dots, \partial_{d - 1})$ of unit length defined near $P$ which induce an orthonormal basis on $T_PM$.
\end{definition}

\begin{proposition}\label{mollifier proposition}
Let $(U_n)$ be a sequence of sets of least perimeter with indicator $u_n$, let $(\partial_0, \dots, \partial_{d - 1})$ be a quasieuclidean frame at $P$, let
$$\gamma_n = \int_{B_1} |du_n| - \partial_0u_n ~\vol,$$
where we write $B_t = B(P, t)$, and suppose that $(\gamma_n) \in \ell^1$.

Then for almost every $t \in (0, 1)$ there exist open sets $V_n$ with $C^1$ boundary and indicator $v_n$, such that
$$|\partial V_n \cap B_t| \leq \eta(V_n, B_t) + o(\gamma_n),$$
$$\int_{B_t} |du_n| - |dv_n| ~\vol \ll \gamma_n,$$
$$\left|\int_{B_t} \partial_j(u_n - v_n)~\vol\right| \ll \gamma_n,$$
and $\normal_{L_n}$ converges to $1$ uniformly near $P$.
\end{proposition}

Our proof shall be similar to the euclidean proof \cite[Lemma 5.5]{Miranda66}, but with some care taken to control the error terms arising from the nonvanishing of the Ricci tensor of $g$.
Once said control is established, the proof is essentially the same.

%%%%%%%%%%%%%%%%%%%%%%%%%%%%%%%%%%%%%%%%%%%%%%%%%%%%%%%

\subsection{The Lipschitz mollifier}
Fix a point $P \in M$ and a radius $R > 0$, so that $B_R = B(P, R)$ satisfies Gauss' polar coordinates lemma.
We begin by rescaling the metric $g$, so that $R \geq 1$.
We then introduce the polar coordinates
$$(r, \Theta): B_1 \setminus P \to (0, 1) \times \Sph^{d - 1}.$$
In these coordinates, we will frequently have occasion to estimate $|\partial B_r|$; it is given by
\begin{align}\label{area of sphere form}
|\partial B_r| &= |\Sph^{d - 1}|r^{d - 1} + \kappa r^{d + 1} + O(r^{d + 3})\\
\label{normalized scalar curvature}
\kappa &:= -\frac{R(P)|\Sph^{d - 1}|}{6d},
\end{align}
where $R$ is the scalar curvature field of $g$.
We will frequently use without mention that $\kappa$ is uniformly bounded on compact sets.

By assumption, $1$ is smaller than the injectivity radius of $P$, so we can identify $B_1$ with a ball in the tangent space $T_PM$.
Thus we define the addition and subtraction of two points in $M$.
If $f \in C_c(B_1, TM^{\wedge d})$ is a compactly supported volume form, $u \in L^1_l(B_1)$, we may define the convolution $f * u$ by extending $f$ to a volume form on $M$ with compact support in $B_1$, so $f(x - y)$ is well-defined even if $x - y \notin B_1$.

\begin{definition}
We will use the \dfn{Lipschitz mollifier}
$$\chi_\varepsilon(r, \Theta) = \frac{C_\varepsilon}{\varepsilon^d}\left(1 - \frac{r}{\varepsilon}\right)1_{r < \varepsilon}\vol(r, \Theta).$$
Here $C_\varepsilon > 0$ is chosen so that $\int_{B_\varepsilon} \chi_\varepsilon = 1$.
For a function $u \in L^1_l$, we define its \dfn{mollification} $u_\varepsilon = \chi_\varepsilon * u$.
\end{definition}

This mollifier was essentially introduced by \cite[Lemma 7.1]{Giusti77} for use with sets of least perimeter.
The point is that, for the purposes of our later arguments, we will only need $u_\varepsilon \in C^1$, and so the fact that $\partial_r \chi_\varepsilon$ is constant almost everywhere will be both convenient and harmless.

\begin{lemma}\label{approximation of mollifier}
Let $\chi_\varepsilon^{\mathrm{euc}}$ be the euclidean Lipschitz mollifier
$$\chi_\varepsilon^{\mathrm{euc}}(r, \Theta) = \frac{C^{\mathrm{euc}}}{\varepsilon^d}\left(1 - \frac{r}{\varepsilon}\right)1_{r < \varepsilon}r^{d - 1}~ \vol_{\Sph^{d - 1}}(\Theta)dr$$
where $C^{\mathrm{euc}}$ normalizes the mollifier to have integral $1$ (and in particular does not depend on $\varepsilon$).
The Radon-Nikod\'ym derivative $f_\varepsilon = \chi_\varepsilon/\chi_\varepsilon^{\mathrm{euc}}$ tends to $1$ on $B_\varepsilon$ as $\varepsilon \to 0$.
\end{lemma}
\begin{proof}
One easily has
$$|\log f_\varepsilon| = |\log(C_\varepsilon\sqrt{\det g}) - \log C^{\mathrm{euc}} | \lesssim \left|\frac{1}{C_\varepsilon \sqrt{\det g}} - \frac{1}{C^{\mathrm{euc}}}\right|$$
on $B_\varepsilon$.
In polar coordinates, $\sqrt{\det g} = 1 - O(r^2)$,
so $1/\sqrt{\det g} \sim 1$ and
\begin{align*}
\frac{1}{C_\varepsilon} = \int_{B_\varepsilon} \varepsilon^{-d}(1 - r/\varepsilon)(1 + O(\varepsilon^2)) ~dr &\sim \frac{1}{C^{\mathrm{euc}}}. \qedhere
\end{align*}
\end{proof}

It follows from Lemma \ref{approximation of mollifier} that $C_\varepsilon \sim 1$ as $\varepsilon \to 0$.
Moreover, if $d(Q, y) < \delta$, then
\begin{equation}\label{approximation of mollifier 2}
\frac{\chi_\varepsilon(x - y)}{\vol(x - y)} \sim_{\delta,\varepsilon} \frac{1}{\varepsilon^d}\left(1 - \frac{d(x, Q)}{\varepsilon}\right),
\end{equation}
with implied constant $C_{\delta,\varepsilon}$ such that $C_{\delta,\varepsilon}/C_\varepsilon \to 1$ as $\delta,\varepsilon \to 0$.
This follows from Lemma \ref{approximation of mollifier} because it is true if $g$ is the euclidean metric.
We also have the estimate
$$|du_\varepsilon(x) - du_\varepsilon(x + v)| \leq ||d(\chi_\varepsilon/\vol)||_{L^\infty} \int_{B_\varepsilon} |u(x - y) - u(x + v - y)| ~\vol(y),$$
so if $u$ is an indicator function, then $d(u_\varepsilon) = (du)_\varepsilon$ is a continuous $1$-form.

Having established the basic properties of $\chi_\varepsilon$, we now prove the generalization of \cite[Lemma 7.1]{Giusti77} that we will need.

\begin{lemma}\label{Giusti71}
For every indicator function $u = 1_U$, there exists $f$ such that $f(\rho, \varepsilon) \to 0$ as $\rho,\varepsilon \to 0$ and if $f(\rho, \varepsilon) < u_\varepsilon(x) < 1 - f(\rho, \varepsilon)$,
then
\begin{equation}\label{Giusti71 claim}
d(x, \partial U) < \varepsilon(1 - \rho).
\end{equation}
\end{lemma}
\begin{proof}
Suppose that (\ref{Giusti71 claim}) fails for some $x \in M$. Then either $d(x, U) \geq \varepsilon(1 - \rho)$ or $d(x, M \setminus U) \geq \varepsilon(1 - \rho)$.
Setting $\omega(y) = \chi_\varepsilon(x - y)$, $\omega$ is a volume form such that
$$u_\varepsilon(x) = \int_U \omega.$$
By Lemma \ref{approximation of mollifier}, we can write $C_\varepsilon = C^{\mathrm{euc}} + \delta$ for some $\delta = \delta(\varepsilon) > 0$ such that $\delta \to 0$ as $\varepsilon \to 0$.
Moreover, $C^{\mathrm{euc}} = d(d + 1)|\Sph^{d - 1}|$ \cite[Lemma 7.1]{Giusti77}, thus
\begin{equation}\label{estimating the constant}
C_\varepsilon = \frac{d(d + 1)}{|\Sph^{d - 1}|} + \delta.
\end{equation}
This allows us to estimate $\omega$ on the annulus $A = B(x, \varepsilon) \setminus B(x, \varepsilon(1 - \rho))$.
In fact, by (\ref{area of sphere form}, \ref{estimating the constant}) and the fundamental theorem of calculus,
\begin{align*}
\int_A \omega &\leq \frac{C_\varepsilon}{\varepsilon^d}\int_{\varepsilon(1 - \rho)}^\varepsilon |\partial B_r| \left(1 - \frac{r}{\varepsilon}\right) ~dr\\
&= \frac{C_\varepsilon}{\varepsilon^d} \int_{\varepsilon(1 - \rho)}^\varepsilon (|\Sph^{d - 1}|r^{d - 1} + \kappa(x) r^{d + 1} + O(r^{d + 3}))(1 + r\varepsilon^{-1}) ~dr \\
&= C_\varepsilon|\Sph^{d - 1}|\left[\frac{1}{d(d + 1)} - \frac{(1 - \rho)^d}{d(d + 1)} + O(\varepsilon^2)\right] \\
&= \frac{C_\varepsilon + O(\delta)}{C_\varepsilon}(1 - (1 - \rho)^d) + O(\varepsilon^2) \\
&:= f(\varepsilon, \rho)
\end{align*}
which $\to 0$ as $\varepsilon, \rho \to 0$.

If $d(x, U) \geq \varepsilon(1 - \rho)$, then $u\omega$ is supported inside $A$, thus $u_\varepsilon(x) \leq f(\varepsilon, \rho)$.
Conversely, if $d(x, M \setminus U) \geq \varepsilon(1 - \rho)$, then $(1 - u)\omega$ is supported inside $A$, thus $1 - u_\varepsilon(x) \leq f(\varepsilon, \rho)$.
\end{proof}

%%%%%%%%%%%%%%%%%%%%%%%%%%%%%%%%%%%%%%%%%%%%%%%%%%%%%%

\subsection{Estimates on the conormal}
We need an analogue of \cite[Theorem 7.3]{Giusti77}.
Before we prove it, we prove the following estimate.

\begin{lemma}\label{scalar curvature monotonicity}
If $r_1 < r_2$, $U$ is a set of least perimeter, $u$ is an indicator function, and $Q \in \partial^* U$,
$$r_2^{1 - d}\int_{B(Q, r_2)} \partial_0u ~\vol - r_1^{1 - d}|\partial^* U \cap B(Q, r_1)| \leq (r_2^2 - r_1^2)\kappa(Q) + O(r_2^4).$$
\end{lemma}
\begin{proof}
Since $U$ is a set of least perimeter, in particular its reduced boundary in $B(Q, r_1)$ has less area than $\partial B(Q, r_1)$.
By the fundamental theorem of calculus and the Cauchy-Schwarz inequality,
$$\int_{B(Q, r_2)} \partial_0u ~\vol = \int_{\partial B(Q, r_2) \cap U} g(\normal_{\partial B(Q, r_2)}, \partial_0) ~\vol \leq \int_{\partial B(Q, r_2) \cap U} \vol = |\partial B(Q, r_2)|.$$
Applying (\ref{area of sphere form}),
\begin{align*}
r_2^{1 - d}|\partial B(Q, r_2)| - r_1^{1 - d}|\partial^* U \cap B(Q, r_1)|
&\leq r_2^{1 - d}|\partial B(Q, r_2)| - r_1^{1 - d}|\partial B(Q, r_1)| \\
&\leq |\Sph^{d - 1}| + r_2^2\kappa(Q) + O(r_2^4) - |\Sph^{d - 1}| - r_1^2 \kappa(Q) + O(r_1^4)\\
&\leq (r_2^2 - r_1^2)\kappa(Q) + O(r_2^4).\qedhere
\end{align*}
\end{proof}

\begin{proposition}\label{main mollifier lemma}
Let $\gamma, p > 0$, let $U$ be a set of least perimeter, $u = 1_U$, and suppose that
\begin{equation}\label{hypothesis on main mollifier lemma}
\int_{B_1} (|du| - \partial_0 u) ~\vol \leq \gamma.
\end{equation}
Let $\varepsilon = \gamma^p$, $\sigma = \gamma^{1/(2(d - 1))}$, and $\varphi = u_\varepsilon$. Then
\begin{equation}\label{claim on main mollifier lemma}
\inf_{\substack{r < \sigma\\\varphi \in (f(\gamma), 1 - f(\gamma))}} \frac{\partial_0 \varphi}{|d\varphi|} > 1 - O(\gamma^{O(1)})
\end{equation}
where $f(\gamma) \to 0$ as $\gamma \to 0$.
Moreover, for every $y \in (f(\gamma), 1 - f(\gamma))$, the level set $\partial \{\varphi > y\} \cap \{r < \sigma\}$ is a $C^1$ hypersurface.
\end{proposition}

The proof of Proposition \ref{main mollifier lemma} is long, but quite similar to the proof of \cite[Theorem 7.3]{Giusti77}.
We break it up into several steps covering the rest of the section.
Here is the main idea. We define a small parameter $\delta$, and observe that $|du| ~\vol$ is a Radon measure of doubling dimension $d - 1$.
We thus cover the domain with open sets of volume $\sim \delta^d\varepsilon^d$ where $\delta > 0$ is a small parameter that determines the cardinality of the cover.
Then $|du| - \partial_0u ~\vol$ assigns a slightly shrunken version of each such set a measure $\sim \delta^{d - 1} \varepsilon^{d - 1}$, while $|du|~\vol$ assigns the set in the cover a comparable measure.
Thus $|du| - \partial_0u ~\vol$ is controlled by $|du|~\vol$.

\subsubsection{Covering the domain of integration}
Let $\delta = \gamma^d > 0$ and select disjoint balls $V_1, \dots, V_N$, centered on $Q_n$, in $\partial^* U \cap B_{\varepsilon(1 - 2\delta)}$ of radius $\delta\varepsilon$ so that the dilates $2V_n$ cover $\partial^* U \cap B_{\varepsilon(1 - 2\delta)}$.
It is easy to show that such a cover exists, because $\overline{\partial^* U \cap B_{\varepsilon(1 - 2\delta)}}$ is compact if $\gamma$ is small enough, so for such a $\gamma$ we can greedily select $Q_n \in \overline{\partial^* U \cap B_{\varepsilon(1 - 2\delta)}}$ to maximize $\min(d(Q_1, Q_n), \dots, d(Q_{n - 1}, Q_n))$.
We set $V_0 = B_\varepsilon \setminus B_{\varepsilon(1 - 2\delta)}$.

Since $du$ is supported in $\bigcup_n 2V_n$,
$$|d\varphi|(x) - \partial_0\varphi(x) = \int_{B_\varepsilon} \chi_\varepsilon(x - \cdot)(|du| - \partial_0 u) \leq \sum_{n=0}^N \int_{2V_n} \chi_\varepsilon(x - \cdot)(|du| - \partial_0 u).$$
So, we shall show
\begin{equation}\label{bound on balls}
\int_{2V_n} \chi_\varepsilon(x - \cdot)(|du| - \partial_0u) \lesssim_{g, p, P} \gamma^{O(1)} \int_{V_n} \chi_\varepsilon(x - \cdot)|du|
\end{equation}
for $n \geq 1$ and
\begin{equation}\label{bound on balls 2}
\int_{V_0} \chi_\varepsilon(x - \cdot)(|du| - \partial_0u) \lesssim_{g, p, P} \gamma^{O(1)} \int_{B_\varepsilon} \chi_\varepsilon(x - \cdot)|du|,
\end{equation}
provided that $x = (r, \Theta)$ satisfies $r < \sigma$ and $\varphi \in (o(\gamma), 1 - o(\gamma))$.
If (\ref{bound on balls}) is true, then since the balls $V_n$ are disjoint, we can sum over $n$ to obtain
\begin{equation}\label{claim on main mollifier lemma 2}|d\varphi|(x) - \partial_0\varphi(x) \lesssim_{g, p, P} \gamma^{O(1)} \int_{B_\varepsilon} \chi_\varepsilon(x - \cdot)|du| \leq \gamma^{O(1)} |d\varphi|(x),
\end{equation}
which implies (\ref{claim on main mollifier lemma}).
In the proof of (\ref{bound on balls}), all constants will be allowed to depend on $p,g,P$, so we suppress the explicit dependence.

TODO Make a picture of the covering

\subsubsection{Estimating on the outer annulus}
We now prove (\ref{bound on balls 2}).
From (\ref{approximation of mollifier 2}) it easily follows that for $y \in V_0$, $\chi_\varepsilon(x - y)/\vol(x - y) \lesssim \frac{\delta}{\varepsilon^d}$,
whence, by minimality of $\partial^* U$,
\begin{align*}
\int_{V_0} \chi_\varepsilon(x - \cdot)(|du| - \partial_0u) &\lesssim \frac{\delta}{\varepsilon^d} \int_{B_\varepsilon} |du| ~\vol \lesssim \frac{\delta}{\varepsilon^d} |\partial B_\varepsilon| \lesssim \frac{\delta}{\varepsilon}.
\end{align*}
By Lemma \ref{Giusti71}, since $\varphi \in (f(\gamma), 1 - f(\gamma))$, $d(x, \partial U) < \varepsilon(1 - \gamma)$, so in particular we can find $Q \in \partial^* U$ such that $d(x, Q) < \varepsilon(1 - \gamma)$.
If $d(y, Q) < \gamma\varepsilon/2$, then
$$d(x, y) \leq \varepsilon - \gamma\varepsilon + \frac{\gamma\varepsilon}{2} \leq \varepsilon - \frac{\gamma\varepsilon}{2},$$
so by (\ref{approximation of mollifier 2}), $\chi_\varepsilon(x - y)/\vol(x - y) \gtrsim \frac{\gamma}{\varepsilon^d}$
for every $y \in B(Q, \gamma\varepsilon/2)$.
In particular, since $\delta = \gamma^d$, minimality of $\partial^* U$ gives
\begin{align*}
\int_{V_0} \chi_\varepsilon(x - \cdot)(|du| - \partial_0u) &\lesssim \frac{\delta}{\gamma^{d - 1}} \frac{\gamma^{d - 1}}{\varepsilon}\\
&\lesssim \gamma |\partial B(Q, \gamma\varepsilon/2)| \int_{B(Q, \gamma\varepsilon/2)} \chi_\varepsilon(x - \cdot) \\
&\lesssim \gamma \int_{B(Q, \gamma\varepsilon/2)} \chi_\varepsilon(x - \cdot) |du|\\
&\lesssim \gamma \int_{B_\varepsilon} \chi_\varepsilon(x - \cdot) |du|
\end{align*}
which implies (\ref{bound on balls 2}).

\subsubsection{Estimating on inner balls}
We complete the proof of Proposition \ref{main mollifier lemma} by showing (\ref{bound on balls}) for each fixed $n \geq 1$.
By (\ref{approximation of mollifier 2}), if $\gamma$ is small enough depending on $g$\footnote{One might worry that we frequently rescale $g$ in this paper.
However, in all of our rescalings, the curvature tensor remains in some bounded set, so these rescalings will never send $\gamma$ to $0$.}, then for every $y \in 2V_n$,
\begin{align*}
\int_{2V_n} \chi_\varepsilon(x - \cdot)(|du| - \partial_0u) &\lesssim F_n(x) \int_{2V_n} |du| - \partial_0u ~\vol, \\
F_n(x) &:= \frac{1}{\varepsilon^d}\left(1 - \frac{d(x, Q_n)}{\varepsilon}\right).
\end{align*}
If $\gamma$ is chosen small enough, then $\sigma > 2\delta\varepsilon$ and so if we set $W_n = B(Q_n, \sigma)$ and apply Proposition \ref{Monotonicity Formula},
\begin{align*}
\int_{2V_n}|du| - \partial_0u ~\vol &\leq
B \left[\sigma^{1 - d}\int_{W_n} |du| - \partial_0u ~\vol + \sigma^{1 - d}\int_{W_n} \partial_0u ~\vol - (2\delta\varepsilon)^{1 - d}\int_{2V_n} \partial_0u ~\vol \right],\\
B &:= e^{O(1)(\sigma^2 - 4\delta^2\varepsilon^2)}(2\delta\varepsilon)^{d - 1}.
\end{align*}
From Taylor's theorem and the fact that $\sigma > 2\delta\varepsilon$,
\begin{align*}
B \lesssim \delta^{d - 1} \varepsilon^{d - 1} + \sigma^2 \delta^{d - 1} \varepsilon^{d - 1} \lesssim \delta^{d - 1} \varepsilon^{d - 1}
\end{align*}
if $\gamma$ is small.
By (\ref{hypothesis on main mollifier lemma}),
$$\sigma^{1 - d}\int_{W_n} |du| - \partial_0u ~\vol \leq \gamma^{\frac{1 - d}{2(d - 1)} + 1} = \gamma^{1/2}.$$
By Proposition \ref{Monotonicity Formula},
\begin{align*}
\sigma^{1 - d}\int_{W_n} \partial_0u ~\vol - (2\delta\varepsilon)^{1 - d}\int_{2V_n} \partial_0u ~\vol &\lesssim (1 + \alpha)\sqrt{\sigma^{1 - d} \int_{W_n} |du| ~\vol - (2\delta\varepsilon)^{1 - d} \int_{2V_n} |du| ~\vol},\\
&\alpha := (d - 1)\log \frac{\sigma}{2\delta\varepsilon}.
\end{align*}
By Lemma \ref{scalar curvature monotonicity},
$$\sigma^{1 - d} \int_{W_n} \partial_0 u ~\vol - (2\delta\varepsilon)^{1 - d} \int_{2V_n} |du| ~\vol \leq (\sigma^2 -4\delta^2\varepsilon^2) \kappa(Q_n) + O(\sigma^4) \lesssim \sigma^2,$$
so by (\ref{hypothesis on main mollifier lemma}),
\begin{align*}
\sigma^{1 - d} \int_{W_n} |du| ~\vol - (2\delta\varepsilon)^{1 - d} \int_{2V_n} |du| ~\vol &= \sigma^{1 - d} \int_{W_n} |du| - \partial_0u ~\vol \\
&\qquad + \sigma^{1 - d} \int_{W_n} \partial_0u ~\vol - (2\delta\varepsilon)^{d - 1} \int_{W_n} |du| ~\vol \\
&\leq \gamma + O(\sigma^2) \lesssim \sigma^2.
\end{align*}
It follows from the definitions that
$$(1 + \alpha)\sigma \lesssim -\gamma^{1/2(d - 1)} \log \gamma \lesssim \gamma^{1/3(d - 1)}.$$
Summing up everything in this step of the proof thus far,
\begin{equation}\label{big bound 1}
\int_{2V_n} \chi_\varepsilon(x - \cdot)(|du| - \partial_0u) ~\vol \lesssim \delta^{d - 1} \varepsilon^{d - 1} F_n(x) \gamma^{1/3(d - 1)}.
\end{equation}
Since $U$ has least perimeter, Proposition \ref{doubling dimension} implies that
$$\delta^{d - 1} \varepsilon^{d - 1} \lesssim \int_{V_n} |du| ~\vol,$$
so by (\ref{approximation of mollifier 2}, \ref{big bound 1}),
$$\int_{2V_n} \chi_\varepsilon(x - \cdot)(|du| - \partial_0u) ~\vol \lesssim \gamma^{1/3(d - 1)} \int_{V_n} \chi_\varepsilon(x - \cdot)|du|$$
which implies (\ref{bound on balls}).

\subsubsection{Regularity of level sets}
We now fix
$$x \in \partial^* U \cap \{r < \sigma\} \cap \{\varphi \in (f(\gamma), 1 - f(\gamma)),$$
and choose $\gamma$ so small that (\ref{claim on main mollifier lemma 2}) simplifies to $\partial_0 \varphi(x) \gtrsim |d\varphi(x)|$.
Thus, in particular,
$$\partial_0 \varphi(x) \gtrsim \int_{B_\varepsilon} \chi_\varepsilon(x - \cdot) |du| > 0.$$
We already showed that $d\varphi$ is continuous, so the level sets of $\varphi$ must be $C^1$.
This completes the proof of Proposition \ref{Giusti71}.

%%%%%%%%%%%%%%%%%%%%%%%%%%%%%%%%%%%%%%%%%%%%%%%%%%%%%

\subsection{Proof of Proposition \ref{mollifier proposition}}
Select $t \in (0, 1)$ uniformly at random.
Let $w_n = (u_n)_{\gamma_n^4}$, let $f$ be as in Proposition \ref{main mollifier lemma}, and let $a_n = f(\gamma_n)$, $b_n = 1 - f(\gamma_n)$.
By Proposition \ref{Coarea2},
$$\int_{B_t} |dv_n| ~\vol = \int_0^1 |\partial^* \{w_n > y\} \cap B_t| ~dy \geq \int_{a_n}^{b_n} |\partial^* \{w_n > y\} \cap B_t| ~dy.$$
By the mean value theorem, there exists $y_n \in (a_n, b_n)$ such that
\begin{equation}\label{MVT mollifier}
|\partial^* \{w_n > y_n\} \cap B_t| \leq \frac{1}{b_n - a_n} \int_{B_t} |dw_n| ~\vol.
\end{equation}
If set $V_n = \{w_n > y_n\}$, $v_n = 1_{V_n}$, then $V_n$ has $C^1$ boundary in $B_t$ by definition of $a_n, b_n$.
It remains to show that $v_n$ has the desired properties.

\subsubsection{Estimates on surface area}
Let $M = (\gamma_n)_{\ell^1} < \infty$ and
$$\mu(E) = \sum_{n=1}^\infty \gamma_n \int_E |du_n| ~\vol;$$
then
$$\mu(B_1) \leq M\sup_{n \in \NN} \int_{B_1} |du_n| ~\vol \leq M |\partial B_1| < \infty$$
since $U_n$ has least perimeter in $B_1$.
So $\mu$ is a finite Radon measure on $B_1$, and hence almost surely,
\begin{equation}\label{partial approximation to surface area}
\limsup_{n \to \infty} \gamma_n^{-3} \int_{B_{t + \gamma_n^4} \setminus B_t} |du_n| \leq \limsup_{n \to \infty} \gamma_n^{-4} \mu(B_{t + \gamma_n^4} \setminus B_t) < \infty.
\end{equation}
By \cite[Lemma 7.2]{Giusti77} and Lemma \ref{approximation of mollifier}, one has
$$\limsup_{n \to \infty} \int_{B_t} |du_n| - |dw_n| ~\vol \leq \limsup_{n \to \infty} \int_{B_{t + \gamma_n^4} \setminus B_t} |du_n| ~\vol,$$
so from the fact that $\gamma_n \to 0$, and (\ref{partial approximation to surface area}),
$$\limsup_{n \to \infty} \gamma_n^{-2} \int_{B_t} |du_n| - |dw_n| ~\vol \leq 0.$$
From (\ref{MVT mollifier}) and the fact that $a_n \to 0$, $b_n \to 1$, we conclude that almost surely,
\begin{equation}\label{approximation of surface area}
|\partial V_n \cap B_t| \leq |\partial^* U_n \cap B_t| + o(\gamma_n).
\end{equation}

\subsubsection{Estimates on spheres}
Let
$$f_n(t) = \gamma_n^{-4} \int_{B_t} |u_n - w_n| ~\vol.$$
By \cite[Lemma 7.2]{Giusti77}, Lemma \ref{approximation of mollifier}, and the fact that $U_j$ has least perimeter in $B_1$,
$$\limsup_{n \to \infty} f_n(t) \leq \limsup_{n \to \infty} \int_{B_1} |du_n| ~\vol \leq |\partial B_1|.$$
Moreover, $f_n$ is monotone.
This implies that almost surely, $f_n'(t)$ is uniformly bounded in $n$.
But
$$f_n'(t) = \gamma_n^{-4} \int_{\partial B_t} |u_n - w_n| ~\vol_{\partial B_t},$$
so
$$\int_{\partial B_t} |u_n - w_n| ~\vol_{\partial B_t} \ll \gamma_n^3.$$
We now set $z_n = \min(y_n, 1 - y_n)$ and estimate
\begin{align*}
\int_{\partial B_t} |u_n - v_n| ~\vol_{\partial B_t} &= |\partial B_t \cap U_n \Delta V_n| \\
&= |\partial B_t \cap V_n \setminus U_n| + |\partial B_t \cap U_n \setminus V_n| \\
&\leq \frac{y_n}{z_n} |\partial B_t \cap V_n \setminus U_n| + \frac{1 - y_n}{z_n} |\partial B_t \cap U_n \setminus V_n|.
\end{align*}
From definition of $V_n$, $w_n - u_n > y_n$ on $V_n \setminus U_n$ and $u_n - w_n > 1 - y_n$ on $U_n \setminus V_n$, so
\begin{align*}
\int_{\partial B_t} |u_n - v_n| ~\vol_{\partial B_t} &\leq z_n^{-1} \int_{\partial B_t \cap U_n \setminus V_n} |u_n - w_n| ~\vol_{\partial B_t} + z_n^{-1}\int_{\partial B_t \cap V_n \setminus U_n} |u_n - w_n| ~\vol_{\partial B_t} \\
&\leq z_n^{-1} \int_{\partial B_t} |u_n - w_n| ~\vol_{\partial B_t}.
\end{align*}
But
$$z_n^{-1} \leq \max(y_n^{-1}, (1 - y_n)^{-1}) \leq \max(a_n^{-1}, b_n^{-1})$$

$$\lim_{n \to \infty} \gamma_n^{-1} \int_{\partial B_t} |u_n - v_n| ~\vol_{\partial B_t} = 0.$$

\subsubsection{etc}




%%%%%%%%%%%%%%%%%%%%%%%%%%%%%%%%%%%%%%%%%%%%%%%%%%%%%%

\section{de Giorgi lemma}\label{DeGiorgiSection}

Let $N = \partial U$ be a $C^1$ hypersurface in $M$ and let $P \in N$.

\subsection{Killing fields on space forms}
In euclidean space, one can view a $C^1$ hypersurface as a graph using the usual coordinates, possibly rotated.
We would like to view $C^1$ hypersurfaces in $M$ as graphs, and to do so in a way which is favorable for our later work, we will need assume that the coordinate system of $M$ is foliated by hypersurfaces which are all isometric to each other.

\begin{definition}
Let $H$ be a hypersurface in some open subset of $M$ which is normal to a Killing field $X$.
We define a scalar field $\lambda$ on $H$, the \dfn{lapse} of $(H, X)$, and a metric $h$ on $H$, the \dfn{leaf metric} of $(H, X)$, by $\lambda = g(X, X)^{(d - 1)/2}$ and $h(v, w) = g(X, X)^{-1} g(v, w)$.
\end{definition}

\begin{lemma}\label{hopfKilling}
Assume that $M$ is a space form with curvature $\kappa$, and either $d = 2$ or $\kappa \leq 0$.
Then for every pointed $C^1$ hypersurface $(N, P)$ in $M$ there exists a hypersurface $H \ni P$ in some neighborhood of $P$ and a Killing field $X$ such that:
\begin{enumerate}
\item $H$ is tangent to $N$ at $P$, and $H$ is normal to $X$.
\item $g(X(P), X(P)) = 1$.
\item The Riemann curvature of the leaf metric $h$ of $(H, X)$ is $0$.
\end{enumerate}
\end{lemma}
\begin{proof}
By rescaling $g$ we may assume that $\kappa \in \{ -1, 0, 1\}$.
Then by the Hopf-Killing theorem we may assume that either $M = \RR^d$, $M = \Hyp^d$, or $M = \Sph^2$.
The case $M = \RR^d$ is obvious.

If $M = \Hyp^d$, we equip $M$ with upper half-plane coordinates $(x, y) \in \RR^{d - 1} \times \RR_+$, chosen so that $P = (0, 0, \dots, 0, 1)$ and $N$ is tangent to $H = \{x_1 = 0\}$ at $P$.
Then we can take $X = \partial_{x_1}$, since the upper half-plane coordinates are conformal (so $X$ is normal to $H$), and since $g(X, X) = y^{-2}$, so that $g(X(P), X(P)) = 1$ and
$$h = y^2y^{-2}(dx_2^2 + \cdots + dx_{d - 1}^2 + dy^2) = dx_2^2 + \cdots + dx_{d-1}^2 + dy^2$$
is the euclidean metric on $H$.

If $M = \Sph^2$, we equip almost all of $M$ with spherical coordinates $(\theta, \varphi) \in (0, \pi) \times (-\pi, \pi)$, chosen so that $P = (\pi/2, 0)$ and $N$ is tangent to $H = \{\theta = \pi/2\}$ at $P$.
Then we can take $X = \partial_\theta$, since $g(X, \partial_\varphi) = 0$ and
$$h = 1^{-1} \sin^2(\pi/2) ~d\varphi^2 = d\varphi^2$$
is the euclidean metric on $H$.
\end{proof}

Assume that $M$ is as in Lemma \ref{hopfKilling}, and let $x$ be a coordinate system on $H$ so that $h = dx^2$ and $P = 0$.
We write $(x, y) = e^{yX}x$, which defines a coordinate system near $H$ such that the hypersurfaces $\{y = \text{const}\}$ are all isometric by the action of $X$.
We abuse notation and call the domain of this coordinate system $M$.
By the implict function theorem, we can find $\omega: H \to \RR$ whose graph $\{y = \omega(x)\}$ is $H$, and we have a $C^1$ diffeomorphism $\varphi: H \to N$ by $\varphi(x) = (x, \omega(x))$.

\begin{definition}
The \dfn{area Lagrangian} is the volume form on $H$ defined for $p$ tangent to $H$ by
$$\Lagrange(p) = \lambda \sqrt{1 + |p|_h^2} \vol_h,$$
where $\lambda,h$ are the lapse and leaf metric on $H$.
\end{definition}

Before we prove our next lemma, we pause to remind the reader of our conventions for Einstein summation, Notation \ref{EinsteinNotation}.

\begin{lemma}
The natural area form on $N$ is given by
$$\varphi^* \vol_N = \Lagrange(\nabla \omega).$$
\end{lemma}
\begin{proof}
If we write $\slashed g$ for the metric on $N$, then
$$\varphi^* \vol_N = \varphi^* \sqrt{\det \slashed g} = \sqrt{\varphi^* \det \slashed g}.$$
To compute $\det \slashed g$ we put $\phi_i^\mu = (\varphi_* \partial_i)^\mu$, thus
$$\varphi^* \slashed g_{ij}(x) = g_{\mu\nu}(\varphi(x))\phi^\mu_i(x) \phi^\nu_j(x).$$
Since $X$ is a Killing field and $\varphi(x) = e^{\omega(x)X}x$,
$$g(\varphi(x)) = g(x) + \int_0^{\omega(x)} \mathcal L_X g(e^{tX}x) ~dt = g(x).$$
Moreover, $\phi_i^0 = \omega_{,i}$ while $\phi_i^j = \delta_i^j$, and $g_{0i}(x) = 0$ since $X$ is normal to $H$.
So
$$\varphi^* \slashed g_{ij} = g_{\mu\nu} \varphi_i^\mu \varphi_j^\nu = g_{00} \omega_{,i} \omega_{,j} + g_{ij} = (g_{00}d\omega \otimes d\omega + g_{\hat 0 \hat 0})_{ij}.$$
Now by the Weinstein-Aronszajn theorem,
\begin{align*}
\det(g_{00}d\omega \otimes d\omega + g_{\hat 0 \hat 0})
&= \lambda^2 \det(d\omega \otimes d\omega + h)\\
&= \lambda^2 (1 + h^{-1}(d \omega \otimes d\omega)) \det h \\
&= \lambda^2 (1 + |\nabla \omega|_h^2) \det h. \qedhere
\end{align*}
\end{proof}

We write
$$\DirL(p) = \frac{|p|_h^2}{2}\vol_h$$
for the Dirichlet energy of $p$ with respect to $h$.
To control the lapse, we assume that there exists a constant $0 < c < 1$ such that
$$(1 + c)^{-1} \leq |\lambda|, |\lambda|^{-1} \leq 1 + c.$$
This assumption essentially has no content, since $\omega$ was only defined in a small ball around $P$ anyways.
We regard $c$ as an error term that we will hold on until the end, when we show that that we can pay for it by decreasing the speed of convergence of the approximation to the conormal.

We also abuse notation and write $f\vol_h$ as just $f$, as the only metric on $H$ that we will use from now on is the leaf metric $h$.

\begin{lemma}\label{Taylor lemma}
If $|q|_h \lesssim 1$ then $\Lagrange(p) - \Lagrange(q)$ lies in the closed interval
$$\left[\frac{(1 + c)^{-1}}{\Lagrange(q)}(\DirL(p) - \DirL(q)) - \frac{2(1 + c)^{-2}}{\Lagrange(q)}(\DirL(p) - \DirL(q))^2, (1 + c)(\DirL(p) - \DirL(q))\right].$$
\end{lemma}
\begin{proof}
By Taylor's theorem, there exists $\xi \geq 0$ between $|p|$ and $|q|$ such that
\begin{align*}
\lambda^{-1}(\Lagrange(p) - \Lagrange(q)) &= \sqrt{1 + |p|^2} - \sqrt{1 + |q|^2}\\
&= \frac{1}{2 \sqrt{1 + |q|^2}(|p|^2 - |q|^2)} - \frac{1}{8(1 + \xi^2)^{3/2}}(|p|^2 - |q|^2)^2 \\
&= \frac{\DirL(p) - \DirL(q)}{\sqrt{1 + |q|^2}} - \frac{(\DirL(p) - \DirL(q))^2}{2(1 + \xi^2)^{3/2}}.
\end{align*}
Since $|q|^2 \geq 0$ and the second term is nonpositive, it follows that
$$\Lagrange(p) - \Lagrange(q) \leq (1 + c)(\DirL(p) - \DirL(q)).$$
and, since $|q|^2 \leq 15$,
$$2\Lagrange(q) \leq 8 \leq 8(1 + \xi^2)^{3/2},$$
whence
\begin{align*}
\frac{(1 + c)^{-1}\DirL(p) - \DirL(q)}{\Lagrange(q)} - \frac{2(1 + c)^{-2}(\DirL(p) - \DirL(q))^2}{\Lagrange(q)} &\leq \Lagrange(p) - \Lagrange(q).\qedhere
\end{align*}
\end{proof}

\subsection{Applying the mean-value property}
Assume that we are in the situation of Lemma \ref{hopfKilling} with leaf metric $h = dx^2$.
Let $B_r$ denote a ball in the euclidean space $H$, not in $M$.
Let $A(r)f$ denote the mean of $f$ over $B_r$.
Fix $\rho > 0$, and parameters $\beta, \kappa > 0$.
In this situation, we have the following generalization of \cite[Teorema 4.3]{Miranda66}, with essentially the same proof:

\begin{lemma}
Assume that $||\nabla \omega||_{L^\infty(B_\rho)} \leq \kappa$,
$$\int_{B_\rho} \Lagrange(\nabla \omega) - \Lagrange(A(\rho)\nabla \omega) \leq \beta,$$
and $N$ is approximately minimal in the sense that
$$\int_{B_\rho} \Lagrange(\nabla \omega) \leq \eta(U, \rho) + \beta \kappa.$$
Then
$$\int_{B_{\rho/2}} \Lagrange(\nabla \omega) - \Lagrange(A(\rho/2)\nabla \omega) \leq \frac{(1 + c)^2}{2^{d + 1}}\beta + O(\beta \kappa^{1/2}).$$
\end{lemma}
\begin{proof}
Let $u$ be the harmonic function on $B_\rho$ with trace $\omega$.
By definition of $\eta(U, \rho)$,
$$\int_{B_\rho} \Lagrange(\nabla \omega) - \Lagrange(\nabla u) \leq \int_{B_\rho} \Lagrange(\nabla \omega) - \eta(N, \rho) \leq \beta\kappa.$$
By Lemma \ref{Taylor lemma},
\begin{align*}
\int_{B_{\rho/2}} \Lagrange(\nabla \omega) - \Lagrange(A(\rho/2)\nabla \omega) &\leq (1 + c)\int_{B_{\rho/2}} \DirL(\nabla \omega) - \DirL(A(\rho/2)\nabla \omega) \\
&= \frac{1 + c}{2} \int_{B_{\rho/2}} |\nabla \omega|^2 - |A(\rho/2)\nabla\omega|^2.
\end{align*}
Since $A(\rho/2)\nabla \omega$ is the mean of $\nabla \omega$, we have for every $\varepsilon > 0$
\begin{align*}
\int_{B_{\rho/2}} |\nabla \omega|^2 - |A(\rho/2)\nabla \omega|^2 &\leq \int_{B_{\rho/2}} (\nabla \omega - A(\rho)\nabla \omega)^2 ~dx \\
&\leq (1 + \varepsilon^{-1})\int_{B_\rho} |\nabla \omega - \nabla u|^2 ~dx\\
&\qquad + (1 + \varepsilon) \int_{B_{\rho/2}} |\nabla u - A(\rho)\nabla \omega|^2 ~dx\\
&=: O(\varepsilon^{-1})I + (1 + \varepsilon)J.
\end{align*}
To estimate $I$, we use the mean-value property to show
\begin{align*}
I &= \int_{B_\rho} |\nabla \omega - \nabla u|^2 = \int_{B_\rho} |\nabla \omega|^2 - |\nabla u|^2 \\
&\lesssim \int_{B_\rho} \int_{B_\rho} \Lagrange(\nabla \omega) - \Lagrange(\nabla u) \leq \beta \kappa.
\end{align*}
To estimate $J$, we bound
\begin{align*}
J &= \int_{B_\rho} |\nabla u|^2 - |A(\rho)\nabla \omega|^2 = \int_{B_\rho} |\nabla u|^2 - |A(\rho)\nabla u|^2 \\
&\leq \frac{1}{2^{d + 1}} \int_{B_\rho} |\nabla u|^2 - |A(\rho)\nabla u|^2 \leq \frac{1}{2^{d + 1}} \int_{B_\rho} |\nabla \omega|^2 - |A(\rho)\nabla \omega|^2 \\
&\leq \frac{1}{2^{d + 1}} \int_{B_\rho} |\nabla \omega - A(\rho)\nabla \omega|^2 ~dx \\
&\leq \frac{1 + \varepsilon}{2^{d + 1}} \int_{B_\rho} |\nabla \omega - A(\rho)\nabla \omega|^2  + O(\varepsilon^{-1})\int_{B_\rho} |\nabla \omega - \nabla u|^2\\
&=: \frac{1 + \varepsilon}{2^{d + 1}}K + O(\varepsilon^{-1})I.
\end{align*}
We already estimated $I \lesssim \beta \kappa$, and now we estimate $K$: by Lemma \ref{Taylor lemma},
\begin{align*}
K &= \int_{B_\rho} |\nabla \omega|^2 - |A(\rho)\nabla \omega|^2\\
&\leq 2\int_{B_\rho} \DirL(\nabla \omega) - \DirL(A(\rho)\nabla \omega)\\
&\leq 2(1 + c)\int_{B_\rho} \Lagrange(\nabla \omega) - \Lagrange(A(\rho)\nabla\omega) + O(1) \int_{B_\rho} (\DirL(\nabla \omega) - \DirL(A(\rho)\nabla \omega))^2\\
&\leq 2(1 + c)\beta + O(1) ||\nabla \omega||_{L^\infty(B_\rho)} \int_{B_\rho} \DirL(\nabla \omega) - \DirL(A(\rho)\nabla \omega) \\
&\leq 2(1 + c + O(\kappa))\beta.
\end{align*}
If we set $\varepsilon = \kappa^{1/2}$ then we conclude
\begin{align*}
\int_{B_{\rho/2}} \Lagrange(\nabla \omega) - \Lagrange(A(\rho/2)\nabla \omega)
&\leq \frac{(1 + c)^2(1 + \varepsilon)^2}{2^{d + 1}}\beta + O(\beta \kappa \varepsilon^{-1})\\
&\leq \frac{(1 + c)^2}{2^{d + 1}}\beta + O(\beta \kappa^{1/2}). \qedhere
\end{align*}
\end{proof}

\subsection{de Giorgi's lemma, $C^1$ case}
With the above setup, we now are ready to show the $C^1$ case of de Giorgi's lemma \cite[Teorema 4.4]{Miranda66}.
We again let $B_r$ denote a ball in $M$, not in $H$.

\begin{definition}
The (unnormalized) \dfn{excess} of a set $U$ of locally finite perimeter with respect to a quasieuclidean frame $(\partial_0, \dots, \partial_{d - 1})$ at $P$ is defined by
$$\Lambda_U(\rho) = \int_{B_\rho} |du| ~\vol - \sqrt{\sum_{j=0}^{d - 1} \int_{B_\rho} \partial_ju ~\vol}.$$
\end{definition}

\begin{lemma}
Let $(U_n)$ be a sequence of open sets with $C^1$ boundary, and fix a quasieuclidean frame $(\partial_0, \dots, \partial_{d - 1})$ at $P$.
If
$$\Lambda(U_n, \rho) \leq \beta_n,$$
$$\lim_{n \to \infty} ||g(\normal_{U_n}, \partial_0) - 1||_{L^\infty(B_\rho)} = 0,$$
$$\eta(U_n, \rho) \ll \beta_n,$$
then
$$\limsup_{n \to \infty} \frac{\Lambda(U_n, \rho/2)}{\beta_n} \leq \frac{(1 + c)^2}{2^{d + 1}}.$$
\end{lemma}

\subsection{Mollification}
The same results as in the previous section hold without $N$ a $C^1$ hypersurface.
In particular:

\begin{proposition}\label{de Giorgi}
For every $c > 0$ there exists $\sigma > 0$ and $r > 0$ such that for every set $U$ of least perimeter, if
$\rho < r$ and
$$\Lambda(U, \rho) < \sigma \rho^{d - 1},$$
then
$$\Lambda(U, \rho/2) < \frac{(1 + c)^2}{2^d} \Lambda(U, \rho).$$
\end{proposition}

We now settle the choice of $c = c(d)$: we select $c$ so that that
$$\frac{(1 + c)^2}{2^d} < \lambda^{-d}$$
for some $\lambda \in (0, 1)$ such that
$$\frac{d - 1}{d} < \log_2 \lambda < 1.$$
By induction on Proposition \ref{de Giorgi}, there exists $\tau \in (0, 1)$ such that
\begin{equation}\label{inductive de Giorgi}
\Lambda(U, 2^{-n}) \lesssim 2^{-n(d - \tau)}.
\end{equation}

\subsection{Continuity of the conormal}
Let
$$\normal_s(P) = \frac{\int_{B_s(P)} du}{\int_{B_s(P)} |du|}.$$
Then $\normal_s \to \normal$ almost everywhere and $\normal_s$ is obviously continuous.
We want to get uniform convergence, thus we use the following lemma:

\begin{lemma}
Suppose that $0 < s < t < \rho$ are dyadic,
$$\Lambda(U, \rho) < \sigma \rho^{d - 1},$$
and $P \in \partial U$.
Then there exists $\delta > 0$ such that
$$|\normal_s(P) - \normal_t(P)| \lesssim t^\delta.$$
\end{lemma}
\begin{proof}
Write $s = 2^{-m}$, $t = 2^{-n}$ (so $m > n$), and set $v_k = \normal_{2^{-k}}(P)$, $M_k = \int_{B_{2^{-k}}} |du|$.
Then
$$|v_n - v_m|^2 \lesssim 1 - (v_n, v_m).$$
Gistui--Miranda show that
$$(1 - (v_n, v_m))M_m \leq 2M_n(1 - |v_n|) = 2\Lambda(U, t).$$
By (\ref{inductive de Giorgi}),
$$(1 - (v_n, v_m))M_m \lesssim 2^{-n(d - \tau)}$$
and hence, since $M_m \gtrsim 2^{-n(d - 1)}$, we conclude
\begin{align*}
|v_n - v_m|^2 &\lesssim 2^{-n(1 - \tau)} = t^{1 - \tau}. \qedhere
\end{align*}
\end{proof}

It follows from the above and the main theorem of Miranda66 that if
$$\Lambda(U, \rho) < \sigma \rho^{d - 1},$$
then $\normal$ is continuous in a neighborhood of $P$, and therefore $N$ is analytic.



%%%%%%%%%%%%%%%%%%%%%%%%%%%%%%%%%%%%%%%%%%%%%%%%%%%%%

\section{Proofs of main theorems}\label{proof of main thm}
We are finally ready to prove Theorems \ref{main thm} and \ref{main crly}.

Throughout this section, let $u$ be a function of least gradient.
By Corollary \ref{level sets are minimal}, the superlevel sets of $u$ have least perimeter.

\subsection{Regularity of minimal hypersurfaces}
Let $U$ be a superlevel set of $u$.
We want to show that $N = \partial U$ is as smooth as possible, and to this end, we might as well assume that $u = 1_U$.

We first may cover $M$ by normal coordinate charts $A_x$ centered on $x \in M$, selected so that $A_x$ is precompact in $M$.
In particular, if we fix a particular $x$, then the averaged $1$-forms $\int_V du ~\vol$ are well-defined for $V \subseteq A_x$, so the excesses $\Lambda(U, V)$ are well-defined.
We write $\Lambda_x$ for the excess as computed in $A_x$, and write $N^* = \partial^* U$.
Since $A_x$ is a precompact chart, we can select $\sigma_x$ to be the constant that appears in Proposition \ref{DGL}.

\begin{lemma}
For every $x \in N^*$ and sufficiently small $\rho > 0$ such that $B(x, \rho) \subseteq A_x$, there is an open set $x \in A_x' \subseteq A_x$ and $\delta > 0$ such that for every $t \in (0, \delta)$, $s \in (0, \rho)$, and $y \in A_x'$ such that $B(y, s) \subseteq A_x$,
\begin{equation}\label{basecase}\Lambda_x(U_t, B(y, s)) < \sigma_x s^d.\end{equation}
\end{lemma}
\begin{proof}
By Proposition \ref{blowup theorem}, there exists a half-space $C_x$ obtained as the blowup of $\exp_x^* U$\footnote{This is the only point in the argument when we use the fact that $d \leq 7$!}.
Since $C_x$ is a half-space and the coordinate chart $A_x$ is normal, $\Lambda_x((\exp_x)_* C_x, V) = 0$ whenever $V \subseteq A_x$.
In particular, if $V$ has no singularities with respect to the blowup sequence $(u_t)$,
$$\lim_{t \to 0} \Lambda_x(U_t, V) = \Lambda_x((\exp_x)_* C_x, V) = 0.$$
Here $U_t = (\exp_x)_* \{u_t = 1\}$ is the blowup of $U$.
So for every $\rho$ such that $B(x, \rho) \subseteq V$, there exists $\delta > 0$ such that if $t < \delta$,
$$\Lambda_x(U_t, B(x, \rho)) < \frac{\sigma_x}{2} \rho^d.$$
By continuity of measure, it follows that for every $y \in A_x$ close enough to $x$, and $0 < s < \rho$ such that $B(y, s) \subseteq A_x$, the claim holds.
\end{proof}

Fix $x \in N^*$, $t < \delta$, and let
$$\normal_s(y) = \frac{\int_{B(y, s)} du_t ~\vol}{\int_{B(y, s)} |du_t| ~\vol},$$
whenever $y \in A_x'$ and $B(y, s) \subseteq A_x$.
Then $\normal_s$ is continuous, since
$$\left|\int_{B(y_1, s)} du_t ~\vol - \int_{B(y_2, s)} du_t ~\vol\right| \leq \int_{B(y_1, s) \Delta B(y_2, s)} |du_t| ~\vol$$
where $\Delta$ denotes symmetric difference; the right-hand side vanishes as $y_2 \to y_1$ by continuity of measure.
We now show that $(\normal_s)$ is uniformly Cauchy as $s \to 0$.

\begin{lemma}
One has
$$|\normal_s(y) - \normal_r(y)| \lesssim_x \sqrt s$$
uniformly in $y \in A_x'$, whenever $0 < r < s < \rho$ and $B(y, s) \subseteq A_x$.
\end{lemma}
\begin{proof}
After throwing away a constant factor we may assume that $s = \rho/2^k$ for some $k$, and $r = \beta s$ for some $\beta \in (0, 1)$.
Since $|\normal_s(y)|,|\normal_r(y)| \leq 1$,
$$|\normal_s(y) - \normal_r(y)| \lesssim \sqrt{1 - (\normal_s(y), \normal_r(y))}.$$
Then (TODO: This seems dubious, check it)
\begin{align*}
(1 - (\normal_s(y), \normal_r(y)))  &\leq \frac{1}{|N^* \cap B(y, r)|} \int_{B(y, \beta \rho/2^k)} |du_t| - \left(\normal_s(y), \frac{du_t}{|du_t|}\right) |du_t| ~\vol\\
&\leq \frac{1}{|N^* \cap B(y, r)|} \int_{B(y, \rho/2^k)} |du_t| - \frac{(\normal_s(y), du_t)}{|du_t|} |du_t| ~\vol \\
&\leq \frac{1}{|N^* \cap B(y, r)|} \int_{B(y, \rho/2^k)} |du_t| - |\normal_s(y)|^2 |du_t| ~\vol\\
&\leq \frac{2}{|N^* \cap B(y, r)|} (1 - |\normal_s(y)|)\int_{B(y, \rho/2^k)} |du_t| ~\vol\\
&= 2\frac{\Lambda_x(U, B(y, s))}{|N^* \cap B(y, r)|}.
\end{align*}
By inducting on Proposition \ref{DGL} in $k$, the base case (\ref{basecase}), and Proposition \ref{doubling dimension},
\begin{align*}
\frac{\Lambda_x(U, B(y, s))}{|N^* \cap B(y, r)|} < 2^{-kd} \frac{\Lambda_x(U, B(y, \rho))}{r^{d - 1}} \lesssim s
\end{align*}
and therefore
\begin{align*}
|\normal_s(y) - \normal_r(y)| &\lesssim \sqrt{\frac{\Lambda_x(U, B(y, s))}{|N^* \cap B(y, r)|}} \lesssim \sqrt s. \qedhere
\end{align*}
\end{proof}

By the above lemma, $(\normal_s)$ is uniformly Cauchy on $A_x'$ as $s \to 0$.
But from the definition of $\normal(y)$, and the fact that $U_t$ is just a rescaling of $U$, if $(\normal_s(y))$ converges to anything as $s \to 0$, it must converge to $\normal$.
Since $\normal_s$ is continuous, it follows that $\normal$ extends to a continuous $1$-form on $A_x' \cap N$.
By Proposition \ref{locality of Caccioppoli}, $(A_x')_{x \in N^*}$ defines an open cover of $N$, so $\normal$ extends to a continuous $1$-form on $N$, and hence by Proposition \ref{regularity of reduced boundary}, $N$ is as smooth as possible.

\subsection{Existence of laminations}
Now we consider the general case when $u$ is a function of least gradient.
Let
\begin{equation}\label{lamination union}
A = \bigcup_y \partial \{u > y\},
\end{equation} $B$ the interior of $\{du = 0\}$, and $x \in M$.
Then $x \in B$ iff $u = u(x)$ near $x$, but that happens iff for every $y < u(x)$, $x$ is interior to $\{u > y\}$ and for every $y \geq u(x)$, $x$ is exterior to $\{u > y\}$.
This happens iff for every $y \in \RR$, $x$ is either interior or exterior to $\{u > y\}$, thus $x \notin \partial \{u > y\}$, which happens iff $x \notin A$.
Thus $\{A, B\}$ is a partition of $M$, so $A$ is closed.
Moreover, the sets $\{u > y\}$ are totally ordered by $\subseteq$, so the sets $\partial \{u > y\}$ are disjoint.
They are also hypersurfaces which are as smooth as possible, by the previous section.
This proves Theorem \ref{main thm}.

\subsection{Convex surfaces with boundary}
The proof of Theorem \ref{main crly} is essentially identical to that of \cite[Proposition 3.4]{górny2017planar}; we give the details here for completeness.

Suppose that $M = \Sigma \subset \overline \Sigma$, and that Theorem \ref{main crly} is false for $u$.
That is, we cannot extend the geodesic lamination that we constructed above to a lamination of $\overline \Sigma$.
Therefore there exist disjoint geodesics $\gamma_1$ and $\gamma_2$ which intersect on $\partial \Sigma$ and bound superlevel sets $\{u > y_i\}$ of $u$.

Suppose that $\gamma_1$ and $\gamma_2$ intersect at $x_0$, and $\gamma_i$ passes through $x_i$ on the way to $x_0$, so that $x_0, x_1, x_2$ bound an open, nondegenerate geodesic triangle $\Delta \subset \overline \Sigma$. This makes sense, because $\overline \Sigma$ is convex.
By Proposition \ref{Monotonicity Formula}, the proof of \cite[Remark 37.9]{simon1983GMT} shows that there exist only finitely many connected components of $A$ in $\Delta$.
So, after replacing $\gamma_2$ with a geodesic closer to $\gamma_1$ as necessary, we may assume that either $A$ does not intersect $\Delta$, or $A$ contains $\Delta$.
By replacing $A$ with its complement if necessary, we may assume that $A$ does not meet $\Delta$.

However, $v = 1_{u^{-1}((y_1, y_2))}$ is a function of least gradient, and $v = 1$ on $\Delta$ but $v = 0$ on the opposite sides of $\gamma_i$.
So if we replace $v$ with $w = v - 1_\Delta$, $w$ has the same trace as $v$, but since $\Delta$ is a nondegenerate triangle,
$$\int_U |dw| ~\vol = |\partial(\{u > y\} \setminus \Delta) \cap U| < |\partial \{u > y\} \cap U| = \int_U |dv| ~\vol$$
whenever $U$ is a precompact neighborhood of $\overline \Delta$ in $\overline \Sigma$.
Therefore $v$ does not have least gradient, which is a contradiction.

%%%%%%%%%%%%%%%%%%%%%%%%%%%%%%%%%%%%%%%%%%%%%%%%%%%%%%%%%%%%%%%%%%%%

\section{Applications to best-Lipschitz/least-gradient duality}
\label{duality}


%%%%%%%%%%%%%%%%%%%%%%%%%%%%%%%%%%%%%%%%%%%%%%%%%%%%%%%%%%%%%%%%%%%%

\printbibliography


\end{document}
