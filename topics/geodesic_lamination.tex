\documentclass[reqno,12pt,letterpaper]{amsart}
\RequirePackage{amsmath,amssymb,amsthm,graphicx,mathrsfs,url,slashed}
\RequirePackage[usenames,dvipsnames]{color}
\RequirePackage[colorlinks=true,linkcolor=Red,citecolor=Green]{hyperref}
\RequirePackage{amsxtra}
\usepackage{cancel}
\usepackage{tikz-cd}

\setlength{\textheight}{9in} \setlength{\oddsidemargin}{-0.25in}
\setlength{\evensidemargin}{-0.25in} \setlength{\textwidth}{7in}
\setlength{\topmargin}{-0.25in} \setlength{\headheight}{0.18in}
\setlength{\marginparwidth}{1.0in}
\setlength{\abovedisplayskip}{0.2in}
\setlength{\belowdisplayskip}{0.2in}
\setlength{\parskip}{0.05in}
\renewcommand{\baselinestretch}{1.05}

\title[Geodesic laminations by minimal currents]{Geodesic laminations by minimal currents}
\author{Aidan Backus}
\date{July 2021}

\newcommand{\NN}{\mathbf{N}}
\newcommand{\ZZ}{\mathbf{Z}}
\newcommand{\QQ}{\mathbf{Q}}
\newcommand{\RR}{\mathbf{R}}
\newcommand{\CC}{\mathbf{C}}
\newcommand{\DD}{\mathbf{D}}
\newcommand{\PP}{\mathbf P}
\newcommand{\MM}{\mathbf M}
\newcommand{\II}{\mathbf I}
\newcommand{\Hyp}{\mathbf H}
\newcommand{\Sph}{\mathbf S}

\DeclareMathOperator{\card}{card}
\DeclareMathOperator{\cent}{center}
\DeclareMathOperator{\ch}{ch}
\DeclareMathOperator{\codim}{codim}
\DeclareMathOperator{\diag}{diag}
\DeclareMathOperator{\diam}{diam}
\DeclareMathOperator{\dom}{dom}
\DeclareMathOperator{\Gal}{Gal}
\DeclareMathOperator{\Hom}{Hom}
\DeclareMathOperator{\Jac}{Jac}
\DeclareMathOperator{\Lip}{Lip}
\DeclareMathOperator{\Met}{Met}
\DeclareMathOperator{\id}{id}
\DeclareMathOperator{\rad}{rad}
\DeclareMathOperator{\rank}{rank}
\DeclareMathOperator{\Hess}{Hess}
\DeclareMathOperator{\Radon}{Radon}
\DeclareMathOperator*{\Res}{Res}
\DeclareMathOperator{\sgn}{sgn}
\DeclareMathOperator{\singsupp}{sing~supp}
\DeclareMathOperator{\Spec}{Spec}
\DeclareMathOperator{\supp}{supp}
\DeclareMathOperator{\Tan}{Tan}
\newcommand{\tr}{\operatorname{tr}}

\newcommand{\Ric}{\mathrm{Ric}}
\newcommand{\Riem}{\mathrm{Riem}}
\newcommand{\LapQL}{\Delta^{\mathrm{ql}}}

\newcommand{\dbar}{\overline \partial}

\DeclareMathOperator{\atanh}{atanh}
\DeclareMathOperator{\csch}{csch}
\DeclareMathOperator{\sech}{sech}

\DeclareMathOperator{\Div}{div}
\DeclareMathOperator{\grad}{grad}
\DeclareMathOperator{\Ell}{Ell}
\DeclareMathOperator{\WF}{WF}

\newcommand{\Lagrange}{\mathscr L}
\newcommand{\DirQL}{\mathscr D^{\mathrm{ql}}}
\newcommand{\DirL}{\mathscr D^{\mathrm{lin}}}

\newcommand{\Hilb}{\mathcal H}
\newcommand{\Homology}{\mathrm H}
\newcommand{\normal}{\mathbf n}
\newcommand{\vol}{\mathrm{vol}}

\newcommand{\pic}{\vspace{30mm}}
\newcommand{\dfn}[1]{\emph{#1}\index{#1}}

\renewcommand{\Re}{\operatorname{Re}}
\renewcommand{\Im}{\operatorname{Im}}

\def\Japan#1{\left \langle #1 \right \rangle}

\newtheorem{theorem}{Theorem}[section]
\newtheorem{badtheorem}[theorem]{``Theorem"}
\newtheorem{prop}[theorem]{Proposition}
\newtheorem{lemma}[theorem]{Lemma}
\newtheorem{claim}[theorem]{Claim}
\newtheorem{proposition}[theorem]{Proposition}
\newtheorem{corollary}[theorem]{Corollary}
\newtheorem{conjecture}[theorem]{Conjecture}
\newtheorem{axiom}[theorem]{Axiom}
\newtheorem{assumption}[theorem]{Assumption}

\theoremstyle{definition}
\newtheorem{definition}[theorem]{Definition}
\newtheorem{remark}[theorem]{Remark}
\newtheorem{example}[theorem]{Example}
\newtheorem{notation}[theorem]{Notation}

\newtheorem{exercise}[theorem]{Discussion topic}
\newtheorem{homework}[theorem]{Homework}
\newtheorem{problem}[theorem]{Problem}

\newtheorem{ack}{Acknowledgements}

\numberwithin{equation}{section}


% Mean
\def\Xint#1{\mathchoice
{\XXint\displaystyle\textstyle{#1}}%
{\XXint\textstyle\scriptstyle{#1}}%
{\XXint\scriptstyle\scriptscriptstyle{#1}}%
{\XXint\scriptscriptstyle\scriptscriptstyle{#1}}%
\!\int}
\def\XXint#1#2#3{{\setbox0=\hbox{$#1{#2#3}{\int}$ }
\vcenter{\hbox{$#2#3$ }}\kern-.6\wd0}}
\def\ddashint{\Xint=}
\def\dashint{\Xint-}

%\usepackage{color}
%\hypersetup{%
%    colorlinks=true, % make the links colored%
%    linkcolor=blue, % color TOC links in blue
%    urlcolor=red, % color URLs in red
%    linktoc=all % 'all' will create links for everything in the TOC
%Ning added hyperlinks to the table of contents 6/17/19
%}

% style=alphabetic
\usepackage[backend=bibtex,maxcitenames=50,maxnames=50]{biblatex}
\addbibresource{topics.bib}
\renewbibmacro{in:}{}
\DeclareFieldFormat{pages}{#1}

\begin{document}
\begin{abstract}
Topics exam, Fall 2021.
\end{abstract}

\maketitle

%%%%%%%%%%%%%%%%%%%%%%%%%%%%%%%%%%%%%%%%%%%%%%%%%%%%%%%

% \tableofcontents

\section{Introduction}
Let $M$ be an oriented Riemannian manifold of metric $g$ and dimension $d \geq 2$.

\begin{theorem}\label{main thm}
Let $u: M \to \RR$ be a function of least gradient, $d \leq 7$, and $A_y = \partial \{u > y\}$.
Then $(A_y)_{y \in \RR}$ is a lamination of $u$ by smooth minimal hypersurfaces, which are analytic if $g$ is.
\end{theorem}

If $M = \RR^d$, then Theorem \ref{main thm} is essentially a standard result; see \cite[Proposition 3.4]{górny2017planar} for an exposition in that case.
A proof of an analogous result for currents is given by \cite[\S5.3]{federer2014geometric}; our proof uses a similar strategy but is rather ``hands-on" in that it avoids the use of homological integration theory.

In the case of a surface, Theorem \ref{main thm} can be stated in a somewhat stronger form, just as in \cite[Corollary 3.5]{górny2017planar}.

\begin{theorem}\label{main crly}
Let $\overline \Sigma$ be a convex surface with boundary and suppose that $u: \Sigma \to \RR$ is a function of least gradient defined on the interior $\Sigma$ of $\overline \Sigma$.
Then, if $A_y = \partial \{u > y\}$, $(A_y)_{y \in \RR}$ extends to a geodesic lamination of $\overline \Sigma$.
\end{theorem}

%%%%%%%%%%%%%%%%%%%%%%%%%%%%%%%%%%%%%%%%%%%%%%%

\subsection{Outline of the paper}
We begin with the preliminaries in Section \ref{RiemMeasureThy}, which records basic results on functions of approximately least gradient, including a generalization of Miranda's theorem \cite[Teorema 3]{Miranda67} on the stability of functions of least gradient.
We also show that the reduced boundary of a Caccioppoli set is metric-independent.

We are then ready to prove Theorems \ref{main thm} and \ref{main crly}.
In Section \ref{inequalities} we prove a monotonicity formula and an isoperimetric inequality, which show that the blowup of a set of least perimeter is as smooth as possible, and in Section \ref{DGL section}, we prove a de Giorgi-type lemma.
Then in Section \ref{proof of main thm} we combine the results of the previous two sections to prove Theorems \ref{main thm} and \ref{main crly}.

In Appendix \ref{coarea section} we deduce a coarea formula that is used throughout the paper.

%%%%%%%%%%%%%%%%%%%%%%%%%%%%%%%%%%%%%%%%%%%%%%%%

\subsection{Acknowledgements}
I would like to thank Georgios Daskalopoulos for suggesting this project and for many helpful discussions.

%%%%%%%%%%%%%%%%%%%%%%%%%%%%%%%%%%%%%%%%%%%%%%%%%%%%%%%%%%%%%%%%%%%%%%%%%%%%%%%%%%%%%%%%%

\section{Riemannian measure theory}
\subsection{Notation and conventions}
\begin{notation}
If $F$ is a presheaf of function spaces, we write $u \in F_l(U)$ to mean that for every $V \Subset U$, $u \in F(V)$.
We write $u \in F_c(U)$ to mean that $u \in F(U)$ and $\supp u \Subset U$.
\end{notation}

\begin{notation}[volume forms]
We reserve the letter $d$ for dimension or exterior differentiation, and so to avoid awkwardness such as $\int |du| dV$ we write $\vol$ for the Riemannian volume form.
If $N$ is a closed submanifold we write $\vol_N$ to indicate the pullback of $\vol$ along the inclusion map $M \to N$.
\end{notation}

\begin{notation}[vector bundles]
Let $E$ be a vector bundle, which we will always assume is normed, with dual $E'$.
If $u,v$ are sections of $E',E$ respectively, we write $(u, v)$ for their fiberwise pairing, which is a function $M \to \RR$.
We write $\langle u, v\rangle$ or $\int_M (u, v) ~\vol$, for their $L^2$-duality pairing, which is a real number.
If $u$ is a section of $E$, we write $u \prec U$ to mean that $||u||_{L^\infty} \lesssim 1$ and $\supp u \Subset U$.
\end{notation}

\begin{definition}
Let $u$ be a smooth section.
If, if $g$ is analytic then $u$ is analytic, then $u$ is \dfn{as smooth as possible}.
\end{definition}

\subsection{The Bochner integral}
Let $F$ be a separable Fr\'echet space over $\CC$, and $(\Omega, P)$ a measure space.
We can then define the \dfn{Bochner integral} of a $P$-measurable function $f: \Omega \to F$, which we write as $\int_\Omega f ~dP \in F$.
See \cite{Rieffel70}, \cite{MO47721}, and \cite[Chapter V]{yosida2012functional}.
We recall a few facts that we will need:

\begin{theorem}[Pettis]
Let $f: \Omega \to F$ be any function.
Then $f$ is $P$-measurable iff for every $X \in F'$, $\omega \mapsto \langle f(\omega), X\rangle$ is $P$-measurable.
In this case,
$$\left\langle \int_\Omega f ~dP, X\right\rangle = \int_\Omega \langle f(\omega), X\rangle ~dP(\omega).$$
If $F = \CC^r$, then the Bochner integral is just the componentwise Lebesgue integral.
\end{theorem}

\subsection{Bundle-valued Radon measures}
Let $F \to M$ be a normed vector bundle of rank $r$.
We equip the space $C(K, F)$ of continuous sections of $F$ on a compact set $K$ with its supremum norm.
The Banach spaces $C(K, F)$ form an inverse system with respect to restriction and therefore define the topological vector space
$$C_0(M, F) = \varprojlim C(K, F).$$

According to the Riesz-Markov theorem, the space of $\CC^r$-valued Radon measures on $M$ is canonically identified with $C_0(M, (\CC^r)')'$, thus we define:

\begin{definition}
The topological dual space $\mathcal R(M, F) = C_0(M, F')'$ is the space of \dfn{$F$-valued Radon measures} on $M$.
\end{definition}

If we write $\mathcal R(U, F)$ to mean $\mathcal R(U, F|U)$, we equip $\mathcal R(U, F)$ with the weak topology of measures.
Then $\mathcal R(U, F)$ is a separable Fr\'echet space, with seminorms $|\langle \cdot, f_j\rangle|$ where $(f_j)$ is a countable basis for a dense subspace of $C_0(U, F)$.
Thus $\mathcal R(\cdot, F)$ is a sheaf of separable Fr\'echet spaces.

\begin{definition}
If $\omega \in \mathcal R(M, F)$ we define the \dfn{total variation} $|\omega|$ to be the positive Radon measure such that on every open set $U$,
$$|\omega|(U) = \sup_{X \prec U} \langle \omega, X\rangle,$$
where $X$ ranges over $C_0(M, F')$.
\end{definition}

Since $\RR_+$ acts on $F$ by scalar multiplication, we can define the \dfn{sphere bundle} $SF = (F \setminus 0)/\RR_+$.
Since $F$ has a norm, $SF$ is naturally identified with the bundle of elements of $F$ with length $1$.

\begin{proposition}[Riesz-Markov representation]\label{HanhJordan}
Let $\omega$ be a $F$-valued Radon measure and let $\mu = |\omega|$.
Then there exists a $\mu$-measurable section $f$ of $SF$ such that for every section $X \in C_0(M, F', \mu)$,
\begin{equation}\label{RNy formula}
\langle \omega, X\rangle = \int_M (f, X) ~d\mu.
\end{equation}
Furthermore, $f$ is unique up to a $\mu$-null set, and does not depend on the norm of $F$.
\end{proposition}
\begin{proof}
Fix an open cover $(U_i)$ of $M$ by charts which trivialize $F$, so that $U_i$ is precompact in $M$.
Let $(g_{ij})$ be the transition functions and $(g_{ij}')$ the induced transition functions for the dual bundle $F'$.
Then can view $\omega_i = \omega|U_i$ as an element of $C(U_i, (\CC^r)')'$, by the precompactness of $U_i$.
Hence by the Riesz-Markov theorem \cite[Theorem 4.14]{simon1983GMT}, there exists a $\mu$-measurable section $f_i$ of $SF$ for which (\ref{RNy formula}) holds for $\omega_i$, provided that $X \in C(U_i, (\CC^r)')$.

We now show that the $f_i$ are restrictions of a global section $f$, thus we must show $f_j = g_{ij} \circ f_i$ on $\CC^r$.
To this end, fix $X \in C_0(M, F)$ which is supported in $U_i \cap U_j$ and write $X_i \in C(U_i, (\CC^r)')$ for the trivialization of $X$ with respect to $U_i$.
Then $X_j = g_{ij}' \circ X_i$, and
\begin{align*}
\int_E (f_i, X_i) ~d\mu &= \langle \omega_i, X_i\rangle = \langle \omega_j, X_j\rangle = \int_E (f_j, X_j) ~d\mu = \int_E (f_j, g_{ij}' \circ X_i) ~d\mu.
\end{align*}

By Urysohn's lemma, $C_c(U_i \cap U_j, (\CC^r)', \mu)$ separates points in $L^1(U_i \cap U_j, \CC^r, \mu)$.
Therefore, since $X$ was arbitrary, $f_i = g_{ji} \circ f_j$; thus we obtain a unique global section $f$ of $SF$.

Finally, if we change the norm of $F$, replacing $|\cdot|$ with $|\cdot|'$, then we obtain a smooth function $h: F \to \RR_+$ so that if $v \in F_x$, then $|v|' = h(x, v)|v|$.
The change of norm gives us a new section $f'$ such that $f' = f/h(\cdot, f'(\cdot))$.
Thus $f'$ defines the same section of $SF$ as $f$.
\end{proof}

At this stage we have only defined $f$ as a $\mu$-equivalence class of sections of $SF$, so we now use the Lebesgue differentiation theorem to choose the ``correct" representative.
We state the differentiation theorem in a somewhat strange way, to ensure that the representative chosen is metric-independent.

\begin{definition}
A \dfn{Besicovitch cover} $\mathcal U$ of a metric space $X$ is a set of open balls, so that every $x \in X$ is the center of an element of $\mathcal U$.
The \dfn{Besicovitch number} $N \in \NN$ of $X$ is the best constant such that for every $x \in U$ and Besicovitch cover $\mathcal U$ of $B(x, 1/N)$, there exist $\mathcal U_1, \dots \mathcal U_N \subset \mathcal U$ such that $\bigcup_{n=1}^N \mathcal U_n$ is an open cover of $B(x, 1/N)$ and $\mathcal U_n$ is disjoint.
\end{definition}

It follows from the theory of \cite[\S2.8]{federer2014geometric} that for every Riemannian metric $g$, the Besicovitch number of $(M, g)$ is finite; \cite{Shi91} motivates why we restrict to small balls $B(x, 1/N)$.

For each $x \in M$, let $\mathcal A(x)$ denote the set of all pairs $(g, B, \varphi)$ where:
\begin{enumerate}
\item $g$ is a Riemannian metric on $M$,
\item $B$ is an open ball centered at $x$ with respect to $g$, and
\item $\varphi$ is a trivialization of $F$ over $B$.
\end{enumerate}
Then $\mathcal A(x)$ is a directed system, where the order is given by reverse inclusion of balls.
Given $(g, B, \varphi) \in \mathcal A(x)$, we obtain a $\mu$-measurable function $f_\varphi: B \to \CC^r$ obtained by trivializing the section $f$.
We define the average
$$f(g, B, \varphi) = \varphi^{-1}\left(\frac{1}{\mu(B)} \int_B f_\varphi ~d\mu\right),$$
which is a point in $F_x$.

\begin{proposition}[Lebesgue differentiation theorem]
Let $\mu$ be a Radon measure on $M$, let $f \in L^1_l(M, SF, \mu)$, and let
$$f(x) = \lim_{(g, B, \varphi) \in \mathcal A(x)} f(g, B, \varphi).$$
Then the limit defining $f(x)$ converges for $\mu$-almost every $x \in M$ to a point in the sphere $SF_x$.
\end{proposition}
If $f(x)$ exists and is $\in SF_x$, we call $x$ a \dfn{Lebesgue point} of the section $f$.
\begin{proof}
This is obvious if $f$ has a representative in $C_c(M, SF)$; besides, by a partition of unity argument, we may assume that $\mu$ has compact support.
We can then select $(f_n)$ in $C_c(M, SF)$ converging in $L^1(M, SF, \mu)$ and almost everywhere to $f$.
Setting $h_n = |f_n - f|$, we can define the average
$$h_n(g, B) = \frac{1}{\mu(B)} \int_B h_n ~d\mu,$$
which converges to $0$ in $L^1(M, \mu)$.

Fix $N \in \NN$ and let $\mathcal B_N$ be the set of Riemannian metrics with Besicovitch number $\leq N$.
This makes sense if we restrict to a neighborhood of the compact support of $\mu$.
For each metric $g \in \mathcal B_N$, we have the Hardy-Littlewood inequality \cite[Lemma 4.1.1a]{Ledrappier85}
\begin{equation}\label{HardyLittlewood}
||\sup_{r \in (0, 1/N)} h_n(g, B_g(\cdot, r))||_{L^{1, \infty}(M, \mu)} \leq N ||h_n||_{L^1(M, \mu)}.
\end{equation}
By (\ref{HardyLittlewood}) and the convergence $h_n \to 0$ in $L^1$,
$$\lim_{n \to \infty} ||\sup_{0 \in (0, 1/N)} h_n(g, B_g(\cdot, r))||_{L^{1, \infty}(M, \mu)} = 0$$
uniformly in $g \in \mathcal B_N$.
Therefore we may pass to a subsequence along which, for $\mu$-almost every $x$,
$$\lim_{n \to \infty} \sup_{(g, r) \in \mathcal B_N \times (0, 1/N)} h_n(g, B_g(x, r)) = 0.$$
By the triangle inequality, if
$$\mathcal A_N(x) = \{(g, B, \varphi) \in \mathcal A(x): g \in \mathcal B_N\},$$
then (after passing to a subsequence again)
$$\lim_{n \to \infty} \sup_{(g, B, \varphi) \in \mathcal A_N(x)} |f_n(g, B_g(x, r), \varphi) - f(g, B_g(x, r), \varphi)| = 0.$$
But $f_n(g, B, \varphi) \to f(x)$ everywhere, $f(x) \in SF_x$, and $SF_x$ is closed, so there exists a $\mu$-null set $Z_N$ such that outside of $Z_N$,
$$\lim_{(g, B, \varphi) \in \mathcal A_N(x)} f(g, B, \varphi) \in SF_x.$$
Taking $Z = \bigcup_{N \in \NN} Z_N$, we see that $Z$ is $\mu$-null, which was to be shown.
\end{proof}

\subsection{Differentiation and boundary}
In this section we fix a Riemannian metric.

\begin{definition}
A function in $L^1(M)$ has \dfn{bounded variation} if its distributional derivative is a $T'M$-valued Radon measure of finite total variation.
We write $BV$ for the presheaf of functions of bounded variation.
A \dfn{Caccioppoli set} is an open set whose indicator function has locally bounded variation.
\end{definition}

If $u$ is a function of bounded variation we write $du ~\vol$ for its derivative.

Sequences $(u_n)$ in $BV_l(M)$ with $u_n \to u$ in $L^1_l(M)$ satisfy the lower semicontinuity property
\begin{equation}
\label{RieszMarkovDistr}
\int_M |du| ~\vol \leq \liminf_{n \to \infty} \int_M |du_n| ~\vol.
\end{equation}
which follows by testing against smooth functions, and the forgetful map
\begin{equation}\label{Forget}
BV_l(M) \to L^1_l(M)
\end{equation}
is compact. We refer to \cite[Chapter 1]{Giusti77} for a review of the space $BV_l(M)$.
Our next result can be deduced by applying a partition of unity argument and then copying the proof of \cite[Teorema 1]{Miranda67} verbatim:

\begin{proposition}[trace theorem]\label{traces}
Let $U$ be an open set such that $N = \partial U$ is a Lipschitz hypersurface.
For every $u \in BV_l(M)$ there exists a trace $v \in L^1_l(N)$ such that for every $X \in C_c(M, TM)$,
\begin{equation}\label{Miranda IBP}
\int_U (du, X) ~\vol + \int_U u ~\mathcal L_X\vol = \int_N vg(X, \nu) ~\vol_N.
\end{equation}
Moreover, $v$ is determined by the germ of $u$ at $\partial U$.
If $u$ is an indicator function then so is $v$.
\end{proposition}

Let $U$ be a Caccioppoli set.
The notion of reduced boundary to $U$ was first introduced in \cite{deGiorgi55}; see \cite{Battista_2021} for an equivalent definition.
To construct it, let $\omega = d1_U ~\vol$, which by Proposition \ref{HanhJordan} can be expressed as $\omega = \normal \mu$, where $\normal$ is a section of $ST'M$ which is independent of $g$.

\begin{definition}
Let $U,\normal$ be as above.
The \dfn{reduced boundary} $\partial^* U$ of a Caccioppoli set $U$ is the set of Lebesgue points of $\normal$.
The \dfn{conormal $1$-form} to $\partial^* U$ is $\normal$.
The \dfn{tangent bundle} $T\partial^* U$ to $\partial^* U$ is the kernel bundle of $\normal$.
\end{definition}

The tangent bundle is well-defined and gives a measurable vector bundle of rank $d-1$ over $\partial^* U$, because $\normal$ is nonzero almost everywhere, and so has constant rank $1$.
We prefer to work with $\normal$ than $\normal^\sharp$, to obtain metric-independence.
In fact, metric-independence and well-known facts about the euclidean case \cite[Chapters 2-4]{Giusti77} \cite{deGiorgi55} imply:

\begin{proposition}\label{locality of Caccioppoli}
Let $U$ be a Caccioppoli set.
Then:
\begin{enumerate}
\item $\partial^* U$ is either empty or $d-1$-dimensional in the Hausdorff sense, and is rectifiable with respect to $d-1$-dimensional Hausdorff measure.
\item $\partial^* U$ is dense in the measure-theoretic boundary $\partial U$.
\item If $\normal$ extends to a continuous $1$-form on $\partial U$, then $\partial^* U = \partial U$ is a $C^1$ hypersurface.
\end{enumerate}
\end{proposition}

\begin{notation}
We write $|\partial^* U|$ for $\int_M |d1_U| ~\vol$.
This does not collide with the notation $|U|$ for the volume of $U$, since $U$ has Hausdorff dimension $d$.
\end{notation}

\subsection{The coarea formula} \label{coarea section}
Let $M = (M, g)$ be a Riemannian manifold.
Throughout this section we consider the superlevel sets $E_y = \{u > y\}$ of a function $u \in BV_l(M)$, and the resulting $T'M$-valued Radon measures
$$\omega(y) = d1_{E_y} ~\vol.$$
Let $\mu = |du| ~\vol$.

We first observe that for every $X \in \mathcal D(M, TM)$,
$$\langle \omega(y), X\rangle = -\int_{E_y} \mathcal L_X\vol$$
is measurable in $y$, since $E_y$ is monotone in $y$.
So by Pettis' theorem, $\omega$ is measurable in $y$ with respect to the weak topology of measures.

\begin{lemma}[coarea formula for measures]\label{Coarea1}
One has
$$du ~\vol = \int_{-\infty}^\infty \omega(y) ~dy.$$
\end{lemma}
\begin{proof}
Fix $X \in C_c(M, TM)$. We may assume that $u \geq 0$, and we must show
\begin{equation}
\label{gradient is integral of fibers}
\int_M (du, X) ~\vol = \int_{-\infty}^\infty \langle \omega(y), X\rangle ~dy.
\end{equation}
Since $u \geq 0$,
\begin{align*}
\int_M (du, X) ~\vol &= -\int_M u~\mathcal L_X\vol = -\int_M \int_0^{u(x)} dy ~\mathcal L_X\vol\\
&= -\int_0^\infty \int_{E_y} ~\mathcal L_X\vol ~dy = \int_0^\infty \langle \omega(y), X\rangle ~dy.
\end{align*}
by Fubini's theorem.
If $y < 0$ then $1_{E_y} = 1$ so $\omega(y) = 0$, so we conclude (\ref{gradient is integral of fibers}).
\end{proof}

Let $p: L \to M$ be the trivial line bundle with its induced metric $h$, and let $\eta$ be the volume form induced by $h$.
If $W$ is a vector field on $L$, we will write $W_1$ for the projection of $W$ onto $M$ and $W_2$ for its projection onto $\CC$.
Then Cartan's magic formula implies that if $W_2$ is constant, then
\begin{equation}
\label{Lie derivative computation}
\mathcal L_W\eta = \mathcal L_W\vol \wedge dy.
\end{equation}

\begin{lemma}\label{coarea converse}
Suppose that $W \in \mathcal D(L, TL)$ depends on a parameter $n \in \NN$, such that $W_2 = 0$ and for every $y \in \RR$, $X = W_1(\cdot, y)$ is a maximizing sequence for $\langle \omega(y), X\rangle$ subject to $X \prec U$.
Then
$$\int_{-\infty}^\infty \langle \omega(y), W(y)\rangle ~dy \leq \mu(U).$$
\end{lemma}
\begin{proof}
Let
$$E = \{(x, y) \in L: x \in E_y\}$$
be the undergraph of $u$.
By Fubini's theorem and (\ref{Lie derivative computation}),
\begin{align*}
\int_{-\infty}^\infty \langle \omega(y), W(y)\rangle ~dy &= -\int_{-\infty}^\infty \int_{E_y} \mathcal L_W\vol ~dy = -\iint_E ~\mathcal L_W\eta = \int_M (d1_E, W) ~\vol.
\end{align*}

Let $(u_m)$ be a mollification of $u$, so that $u_m \to u$ in the weak topology of distributions.
Then if $\chi$ is a cutoff, $\langle u_m, \chi\rangle \to \langle u, \chi\rangle$; taking a sequence of $\chi$ which increase to the indicator function of a compact set $K$, we conclude that $u_m \to u$ in $L^1(K)$, and hence $u_m \to u$ in $L^1_l$.

Let $E^{(m)}$ be the undergraph of $u_m$, and $E^{(m)}_y = \{u_m > y\}$.
Then for every test function $v$,
\begin{align*}
\langle 1_{E^{(m)}} - 1_E, v\rangle &= \int_{E^{(m)} \Delta E} v ~\vol \leq |(E^{(m)} \Delta E) \cap (\supp v \times \RR)| \cdot ||v||_{L^\infty}\\
&\leq ||v||_{L^\infty} \int_{\supp v} |u_m - u| ~\vol \to 0
\end{align*}
so $1_{E^{(m)}} \to 1_E$ in the weak topology of distributions.
Therefore
$$\lim_{m \to \infty} \int_M (d1_{E^{(m)}}, W) ~\vol = \int_M (d1_E, W) ~\vol.$$

Since $u_m$ is smooth, its graph $F_m = \partial E^{(m)}$ is a smooth manifold.
Let $\nu_m$ be the upwards unit normal field of $F_m$ and let $\vol_m$ be the volume form on $F_m$ induced by $\eta$.
Then
$$\langle d 1_{E^{(m)}}, W\rangle = -\iint_{E^{(m)}} \mathcal L_W\eta = -\int_{F_m} h(\nu_n, W) ~\vol_m.$$
Let $q_m = p|F_m$ and $Y_m$ be the vector field $(Y_m)_1 = -d u_m$, $(Y_m)_2 = 1$.
Since $F_m$ is a graph, $q_m: F_m \to M$ is a diffeomorphism, $(q_m)_*\nu_m = Y_m/|Y_m|$, and
$$(q_m)_* \vol_m = |Y_m| ~\vol.$$
Therefore
$$\int_{F_m} h(\nu_n, W) ~\vol_m = \int_M h(Y_m, W) ~\vol = \int_M g(d u_m, W_1) ~\vol = \int_M (d u_m, W_1) ~\vol.$$
Thus
\begin{align*}
|\langle d 1_E, W\rangle| &= \lim_{m \to \infty} |\langle d u_m, W_1\rangle| \leq \mu(U). \qedhere
\end{align*}
\end{proof}

\begin{proposition}[coarea formula for $BV_l$ functions]\label{Coarea2}
Let $u \in BV(M)$, let $E_y = \{u > y\}$, and let $\omega(y) = d1_{E_y} ~\vol$.
Then, if $\mu$ is the total variation of $du ~\vol$,
$$\mu = \int_{-\infty}^\infty |\omega(y)| ~dy.$$
\end{proposition}
\begin{proof}
By Lemma \ref{Coarea1} and the triangle inequality,
$$\mu \leq \int_{-\infty}^\infty |\omega(y)| ~dy.$$
So we just need to prove the converse.

Let $U \Subset M$.
Suppose that for every $y \in \RR$, $X = X^{(n)}_y$ is a maximizing sequence for $\langle \omega(y), X\rangle$ subject to $X \prec U$.
Since $\omega$ with respect to the weak topology of measures, for every $n$, $X^{(n)}_y(x)$ can be chosen to be measurable in $(x, y)$; indeed, we can take $X^{(n)}_y$ to be a smooth approximation to the Radon measure $d 1_{E_y}/|d 1_{E_y}|_{TV}$ in the weak topology of distributions, which is a product of the measurable functions $\omega$ and $y \mapsto 1/|\omega(y)|$.

By an approximation argument, we can find $W^{(n)} \in C_c(L, TL)$ such that $W^{(n)}_2 = 0$ and for every $y \in \RR$, $X = W^{(n)}_1(\cdot, y)$ is a maximizing sequence for $\langle d 1_{E_y}, X\rangle$ subject to $X \prec U$.
Let us now suppress the $n$ and write $W(y) = W^{(n)}(\cdot, y)$.

By Lemma \ref{coarea converse}, since $W$ has compact support, the integrand $\langle d 1_{E_y}, W(y)\rangle$ is uniformly bounded in $y$.
Therefore, by Fatou's lemma,
\begin{align*}
\int_{-\infty}^\infty |\omega(y)| ~dy &= \int_{-\infty}^\infty \lim_{n \to \infty} \langle \omega(y), W(y)\rangle ~dy \leq \liminf_{n \to \infty} \int_{-\infty}^\infty \langle \omega(y), W(y)\rangle ~dy \\
&\leq \mu(U). \qedhere
\end{align*}
\end{proof}

%%%%%%%%%%%%%%%%%%%%%%%%%%%%%%%%%%%%%%%%%%%%%%%%%%%%%%%%%%%

\subsection{Change of metric}\label{change of metric}
Let us investigate how the volume form depends on the choice of metric.
If $\vol$ is the volume form of $g$, $\vol'$ the flat volume form induced by exponential normal coordinates $x$ centered at $p$, we have
the Taylor expansion \cite[p59]{chow2006hamilton}
\begin{equation}\label{Taylor expansion of determinant}
\frac{\vol}{\vol'} = 1 - \frac{1}{3} \Ric_p(x, x) - \frac{1}{6} \nabla \Ric_p(x, x, x) + a_4x^4 + \cdots
\end{equation}
where $a_j$, $j \geq 4$, are functions of the Riemann tensor and its covariant derivatives.

Let $g_1,g_2$ be metrics defined near a point $p$ and let $\beta = (e_1, \dots, e_d)$ be an orthonormal basis for $T_pM$ with respect to $(g_1)_p$ and $(g_2)_p$.
If $\vol_i$, resp. $\Ric^{(i)}$, is the volume form, resp. Ricci tensor, associated to $g_i$, in the geodesic normal coordinate $x$ associated to $\beta$ in a small ball $B(p, \rho)$, then by (\ref{Taylor expansion of determinant}),
\begin{equation}\label{change of volume form}
\frac{\vol_2}{\vol_1} = \frac{1 - \Ric_p^{(2)}(x, x)/6 + O(x^3)}{1 - \Ric_p^{(1)}(x, x)/6 + O(x^3)} = 1 + O(\rho^2).
\end{equation}

Let $\zeta_0$ be the injectivity radius of
$$\exp_p: T_pM \to M.$$
Let $a_j$ be the Taylor coefficients in (\ref{Taylor expansion of determinant}) and let $k \geq 2$ be the least index $j \geq 1$ such that $a_j \neq 0$.
Then let $c = 2|a_k|$.
Since on $B(p, 1)$, $c|x|^k \leq c|x|^2$ and so
$$1 - \frac{c}{2}|x|^2 - O(x^{k+1}) \leq \frac{\vol}{\vol'} \leq 1 + \frac{c}{2}|x|^2 + O(x^{k+1}).$$
We select $\zeta < \zeta_0$ so small that on $B(p, \zeta) \cap B(p, 1)$, the term $O(x^{k+1})$ is at most $c/2$, so that
\begin{equation}\label{definition of c-zeta}
0.5 < 1 - c|x|^2 \leq \frac{\vol}{\vol'} \leq 1 + c|x|^2 < 2.
\end{equation}

By (\ref{Taylor expansion of determinant}), in the generic case $k = 2$, $c = |\Ric_p|/3$, which motivates the following definition:

\begin{definition}
Let $\zeta,c$ be as in (\ref{definition of c-zeta}).
We call $\zeta$ the \dfn{strong injectivity radius} of $p$ and $c$ the \dfn{Ricci-Taylor error}.
\end{definition}

Write
$$\eta'(u, U) = \inf_{v \in BV_c(U)} \int_U |d(u + v)| ~\vol'$$
for the analogue of $\eta$ with respect to the flat metric induced by normal coordinates.

\begin{lemma}\label{flattening of eta}
If $U \subseteq B(p, \zeta) \cap B(p, r)$ with $r < 1$, $U \Subset M$, then for every $u \in BV_l(M)$,
\begin{equation}\label{flattening of eta equation}
(1 - cr^2) \eta'(u, U) \leq \eta(u, U) \leq (1 + cr^2) \eta'(u, U).
\end{equation}
\end{lemma}
\begin{proof}
Let $x$ be a normal coordinate centered on $p$.
Choose $(u_n)$ in $BV_l(M)$ which is a minimizing sequence for $\int_U |du_n| ~\vol'$ subject to the trace condition $u_n|\partial U = u|\partial U$.
Thus $\int_U |du_n| ~\vol' \to \eta'(u, U)$ and by (\ref{definition of c-zeta}),
$$(1 - cr^2) \int_U |du_n| ~\vol' \leq \int_U |du_n| (1 - c|x|^2) ~\vol' \leq \int_U |du_n| ~\vol = \eta(u, U).$$
Taking $n \to \infty$ we get $(1 - cr^2) \eta'(u, U) \leq \eta(u, U)$.
The other estimate in (\ref{flattening of eta equation}) is similar.
\end{proof}

\section{Functions of least gradient}\label{RiemMeasureThy}
\begin{definition}\label{main definitions}
A function $u \in BV_l(M)$ has \dfn{least gradient} if for every $v \in BV_c(M)$ and $\supp v \subseteq U \Subset M$,
$$\int_U |du| ~\vol \leq \int_U |du + dv| ~\vol.$$
A Caccioppoli set $U$ has \dfn{least perimeter} if $1_U$ has least gradient.
\end{definition}

Functions of least gradient can be viewed as the correct notion of weak solution to the Dirichlet problem for the Euler-Lagrange equation
\begin{equation}\label{EulerLagrange}
\Div \frac{\grad u}{|\grad u|} = 0.
\end{equation}

Our next few results follow from Proposition \ref{traces} and the proofs of \cite[Teorema 2]{Miranda67} and \cite[Lemma 5.6]{Giusti77}.

\begin{proposition}[gluing]\label{gluing}
Let $N$ be a Lipschitz hypersurface which separates $M$ into $U_1,U_2$.
If $u_j \in BV(U_j)$ and $u \in L^1_l(M)$ is the function such that $u|U_j = u_j$, then $u \in BV(M)$.
Moreover,
\begin{equation}
\label{glued BV norm}
\int_N |du| ~\vol = \int_N |u_1 - u_2| ~\vol_N.
\end{equation}
\end{proposition}

\begin{notation}
If $u \in BV(M)$ and $U \Subset M$, we write
$$\eta(u, U) = \inf_{v \prec U} \int_U |d(u+v)| ~\vol$$
so that $u$ has least gradient iff $\eta(u, U) = \int_U |du| ~\vol$ for every $U$.
\end{notation}

\begin{lemma}[a priori estimates]\label{estimates on good set}
Let $u, v \in BV(M)$, let $U \Subset M$ have a Lipschitz boundary $N$. Then
\begin{equation}\label{a priori estimate 1}
|\eta(u, U) - \eta(v, U)| \leq \int_N |u - v| ~\vol_N.
\end{equation}
In particular
\begin{equation}\label{a priori estimate 2}
\eta(u, U) \leq \int_N |u| ~\vol_N.
\end{equation}
\end{lemma}

We will need the following theorem \cite[Theorem 6.2.2]{Simons68} \cite[Theorem A]{BOMBIERI1969}, which in particular shows that the assumption $d \leq 7$ in Theorem \ref{main thm} is sharp.
Recall that a \dfn{minimal cone} $C$ is a cone of least perimeter with vertex at the origin.

\begin{theorem}\label{minimal cones in R8}
The following are equivalent:
\begin{enumerate}
\item $d \leq 7$.
\item The boundary of every minimal cone in $\RR^d$ is $C^1$.
\item The boundary of every minimal cone in $\RR^d$ is a hyperplane.
\end{enumerate}
\end{theorem}

\subsection{The Miranda stability theorem}\label{MirandaStability}
The exponential pullback $\exp_p^* u$ of a function $u$ of least gradient defined near $p \in M$ need not have least gradient.
However, in a small ball $B$ around $p$, we will be able to show that $\eta(u, B) \approx |du|_{TV}(B)$ in a sense to be made precise later.
This observation motivates the following definition.

\begin{definition}
A sequence $(u_n)$ of functions in $BV(M)$ has \dfn{approximately least gradient} if
$$\limsup_{n \to \infty} \int_U |du_n| ~\vol \leq \liminf_{n \to \infty} \eta(u_n, U)$$
uniformly as $U$ ranges over open sets $\Subset M$.
\end{definition}

To study sequences of approximately least gradient, we need a semicontinuity theorem for the total variation, which in the euclidean case was shown by Miranda \cite[Teorema 3]{Miranda67}.

\begin{definition}
Let $(u_n)$ be a sequence in $BV_l(M)$ which converges in $L^1_l$ to $u$.
We say that a Lipschitz hypersurface $N$ \dfn{has no singularities} of $(u_n)$ if:
\begin{enumerate}
\item \label{cond1Mir} $\sup_n \int_N |du_n| ~\vol = 0$.
\item \label{cond2Mir} $(u_n)$ is bounded in $L^1(N, \vol_N)$.
\item \label{cond3Mir} $\int_N |du| ~\vol = 0$.
\item \label{cond4Mir} $u_n \to u$ in $L^1(N, \vol_N)$.
\end{enumerate}
We say that $N$ \dfn{has no singularities} of $u \in BV_l(M)$ if $N$ has no singularities of the sequence $u_n = u$.
By Condition $k$ we mean the $k$th bullet in the above list.
\end{definition}

\begin{lemma}\label{probabilistic method}
Let $(u_n)$ be a sequence in $BV_l(M)$ which converges in $L^1_l(U)$. Then:
\begin{enumerate}
\item \label{probabilistic balls} For every $x \in M$ and $R > 0$ such that $B(x, R) \Subset M$ and almost every $r \in (0, R]$, $\partial B(x, r)$ has no singularities of $(u_n)$.
\item \label{probabilistic hypersurfaces} For every $U \Subset M$ there exists $U \subseteq V \Subset M$ such that $\partial V$ has no singularities of $(u_n)$.
\end{enumerate}
\end{lemma}
\begin{proof}
We first prove (\ref{probabilistic balls}).
Let $r$ be drawn from $[R/2, R]$ uniformly at random; we claim that almost surely, $\partial B(x, r)$ has no singularities of $(u_n)$.
Let
$$A = \{s > 0: \int_{\partial B(x, s)} |du| ~\vol > 0\}.$$
Then
$$\sum_{s \in A} \int_{\partial B(x, s)} |du| ~\vol \leq \int_{\partial B(x, R)} |du| ~\vol < \infty$$
since $|du|$ is a Radon measure and $B(x, R) \Subset M$.
Therefore $A$ is countable,
%Let $A_n = \{|du_n|_{TV}(N) > 0\}$ and let $A_\infty = \{|du|_{TV}(N) > 0\}$.
%Then for every $n \in \NN \cup \{\infty\}$, writing $u_\infty = u$,
%$$\sum_{s \in A_n} |du_n|_{TV}(\partial B(x, s)) \leq |du_n|_{TV}(B(x, R)) < \infty$$
%since $|du_n|_{TV}$ is a Radon measure and $B(x, R) \Subset M$.
%Since each of the summands is nonzero by definition of $A_n$, it follows that $A_n$ is countable, and in particular null.
%Therefore Conditions \ref{cond1Mir} and \ref{cond3Mir} hold almost surely.
so Condition \ref{cond3Mir} holds almost surely.
We omit the proof that the other conditions hold almost surely as it is similar.

To prove (\ref{probabilistic hypersurfaces}), let $U \Subset W \Subset M$, and for every $x \in \partial U$, let $R_x \in (0, d(x, \partial W))$.
Then, by (\ref{probabilistic balls}), for every $x \in \partial U$, there exists $r_x \in (0, R_x)$ such that $\partial B(x, r_x)$ has no singularities of $(u_n)$.
Let $\mathcal U$ be the open cover of $\overline U$ given by the balls $B(x, r_x)$, as well as $U$ itself.
Since $\overline U$ is compact, there exists a finite subcover $\mathcal U_0$ of $\mathcal U$.
Let $V$ be the union of the sets in $\mathcal U_0$.
Then $\partial V$ is the boundary of a union of finitely many balls $B(x, r_x)$ whose boundaries have have no singularities, and therefore has no singularities.
\end{proof}

We recall that $BV_l(M)$ is not separable, so it will be useful to have a somewhat weaker topology on $BV_l(M)$, as follows:

\begin{definition}
A sequence of functions $(u_n)$ in $BV_l(M)$ converges \dfn{in total variation on sets with no singularities} to $u \in BV_l(M)$ if $u_n \to u$ in $L^1_l(M)$ and for every set $A \Subset M$ such that $\partial A$ has no singularities,
\begin{equation}\label{convergence in TV}
\lim_{n \to \infty} \int_A |du_n| ~\vol = \int_A |du| ~\vol.
\end{equation}
\end{definition}

\begin{proposition}[Miranda stability theorem]\label{Miranda convergence}
If a sequence of functions $(u_n)$ has approximately least gradient and converges in $L^1_l$, then its limit $u$ has least gradient, and $u_n \to u$ in total variation on sets with no singularities.
\end{proposition}
\begin{proof}
By Lemma \ref{probabilistic method} for every $U$ open $\Subset M$ we can find $U \subseteq V \Subset M$ such that $V$ is open and $\partial V$ has no singularities.

We first prove that $u \in BV_l(M)$.
Let $v_n = (1 - 1_U)u_n$, so $v_n \in BV_l(M)$ by Proposition \ref{gluing}.
Since $(u_n)$ has approximately least gradient, if $n$ is large enough then
$$\int_{\overline V} |du_n| ~\vol \leq \eta(u_n, \overline V) + 1 \leq \int_{\overline V} |du_n| ~\vol + 1.$$
So by Proposition \ref{gluing} and Condition \ref{cond1Mir},
$$\int_U |du_n| ~\vol \leq \int_{\overline V} |du_n| ~\vol + 1 = \int_{\partial V} |u_n| ~\vol_{\partial V} + 1.$$
Thus, by (\ref{RieszMarkovDistr}) and Condition \ref{cond2Mir},
$$\int_U |du| ~\vol \leq \limsup_{n \to \infty} \int_U |du_n| ~\vol \leq \limsup_{n \to \infty} \int_{\partial V} |u_n| ~\vol_{\partial V} + 1 < \infty.$$
Therefore $u \in BV_l(M)$.

Let $v$ be a perturbation of $u$, thus $v \in BV_l(M)$ and $u = v$ on $M \setminus U$.
Such a perturbation exists, since $u \in BV_l(M)$.
Let
$$v_n(x) = \begin{cases}
v(x), &x \in V\\
u_n(x), &x \notin V
\end{cases}.$$
By Proposition \ref{gluing} and Condition \ref{cond3Mir}, $v_n \in BV_l$ and
\begin{equation}\label{gluing vn}
\int_{\overline V} |dv_n| ~\vol = \int_V |dv| ~\vol + \int_{\partial V} |u - u_n| ~\vol_{\partial V}.
\end{equation}
Let $\varepsilon > 0$. Then for $n$ large enough, since $(u_n)$ has approximately least gradient and $u_n - v_n$ is trace-free,
$$\int_V |du_n| ~\vol \leq \int_{\overline V} |du_n| ~\vol \leq \eta(u_n, \overline V) + \varepsilon \leq \int_{\overline V} |dv_n| ~\vol + \varepsilon.$$
By (\ref{gluing vn}), it follows that
$$\int_V |du_n| ~\vol \leq \int_V |dv| ~\vol + \int_{\partial V} |u - u_n| ~\vol_{\partial V} + \varepsilon.$$
By Condition \ref{cond4Mir} and (\ref{RieszMarkovDistr}),
$$\int_V |du| ~\vol \leq \int_V |dv| ~\vol + \varepsilon.$$
Since $\varepsilon > 0$ and $u = v$ on $M \setminus U$, it follows that $\int_U |du| ~\vol \leq \int_U |dv| ~\vol + \varepsilon$.
Taking $\varepsilon \to 0$, it follows that $u$ has least gradient.

Finally we prove (\ref{convergence in TV}).
Let $(u_{n_\ell})_\ell$ be a subsequence of $(u_n)$ such that $\int_A |du_{n_\ell}| ~\vol$ is Cauchy in $\ell$.
By Lemma \ref{estimates on good set} and Condition \ref{cond4Mir},
$$\lim_{n \to \infty} |\eta(u, A) - \eta(u_n, A)| \leq \lim_{n \to \infty} \int_{\partial A} |u - u_n| ~\vol_{\partial A} = 0.$$
Since $(u_n)$ has approximately least gradient, for every $\varepsilon > 0$ and $n$ large enough, $\int_A |du_n| ~\vol \leq \eta(u_n, A) + \varepsilon$.
Thus by (\ref{RieszMarkovDistr}) and the fact that $(\int_A |du_{n_\ell}| ~\vol)_\ell$ is Cauchy,
\begin{align*}
\int_A |du| ~\vol &\leq \lim_{\ell \to \infty} \int_A |du_{n_\ell}| ~\vol \leq \lim_{\ell \to \infty} \eta(u_{n_\ell}, A) + \varepsilon\\
&= \eta(u, A) + \varepsilon = \int_A |du| ~\vol + \varepsilon
\end{align*}
where the last equality follows because $u$ has least gradient.
Since $\varepsilon$ and the subsequence $(u_{n_\ell})$ were arbitrary, the inequalities collapse to give (\ref{convergence in TV}).
\end{proof}

\begin{corollary}\label{level sets are minimal}
For every $u$ of least gradient, the superlevel sets $\{u > t\}$ have least perimeter.
\end{corollary}
\begin{proof}
In the proof of \cite[Theorem 1]{BOMBIERI1969}, replace \cite[Theorem 1.6]{Miranda66} with Proposition \ref{Coarea2} and replace \cite[Theorem 3]{Miranda67} with Proposition \ref{Miranda convergence}.
\end{proof}

\begin{corollary}\label{compactness}
Let $(u_n)$ be a sequence of indicator functions of approximately least gradient.
Then there is a subsequence of $(u_n)$ which converges almost everywhere and in total variation on sets with no singularities to the indicator function of a set of least perimeter.
\end{corollary}
\begin{proof}
If $n$ is large enough, then by Proposition \ref{traces}, for every $U \Subset M$,
$$\int_U |du_n| ~\vol \leq \eta(u_n, U) + 1 \leq |\partial U| + 1$$
which gives a uniform bound in $BV_l$.
Since the forgetful map (\ref{Forget}) is compact, a subsequence of $(u_n)$ converges to a function $u$ in $L^1_l$.
By Proposition \ref{Miranda convergence}, $u$ has least gradient and (\ref{convergence in TV}) holds.
By taking a further subsequence we can guarantee the convergence pointwise almost everywhere.
The convergence almost everywhere implies that there is a Caccioppoli set $U$ such that $u = 1_U$, which necessarily has least perimeter.
\end{proof}

%%%%%%%%%%%%%%%%%%%%%%%%%%%%%%%%%%%%%

\subsection{Monotonicity formulae}\label{inequalities}
The monotonicity formula
\begin{equation}\label{classic monotonicity formula}
\partial_r e^{Ar^2}r^{1 - d} |\partial^* U \cap B(p, r)| \geq 0
\end{equation}
for smooth minimal hypersurfaces on Riemannian manifolds incurs a multiplicative loss of $e^{Ar^2}$ \cite[\S7]{MarquesXX}.
We will mainly be interested in the difference $r^{1-d} |\partial^* U \cap B(p, r)| - \rho^{1-d} |\partial^* U \cap B(p, \rho)|$.
Writing out (\ref{classic monotonicity formula}),
$$e^{Ar^2}\partial_r(r^{1 - d} |\partial^* U \cap B(p, r)|) \gtrsim -r^{2 - d} |\partial^* E \cap B(p, r)|.$$
By the cusp estimate Proposition \ref{uniform density estimate} that we will prove shortly, it follows that
$$r^{1-d} |\partial^* U \cap B(p, r)| - \rho^{1-d} |\partial^* U \cap B(p, \rho)| \gtrsim \rho - r.$$
On the other hand, we will shortly show a monotonicity formula of the form
$$r^{1-d} |\partial^* U \cap B(p, r)| - \rho^{1-d} |\partial^* U \cap B(p, \rho)| \gtrsim -h \log \frac{r}{\rho}$$
where $h$ is a parameter that, in our application, will be comparable to $r^2$.
Moreover, in our application, we will have $\rho$ comparable to $r$, which gives us a gain of factor $r$ over the classical monotonicity formula.

\begin{lemma}\label{approximate monotonicity}
Let $E$ be a Caccioppoli set in $\RR^d$ equipped with its euclidean metric such that for some $h \geq 0$ and $R > 0$ and every $0 < r < R$,
$$|\partial^* E \cap B_r| \leq (1 + h)\eta(f, r).$$
Then for every $0 < \rho < r < R$, with $\alpha = \log(r/\rho)$,
\begin{align*}
&\left|r^{1 - d} |\partial^* E \cap B_r| - \rho^{1 - d} |\partial^* E \cap B_\rho|\right|^2 \\
&\qquad \lesssim_d (1 + h)(1 + \alpha + h\alpha^2) \left(r^{1 - d} |\partial^* E \cap B_r| - \rho^{1 - d} |\partial^* E \cap B_\rho| + d\omega_d h \alpha\right).
\end{align*}
\end{lemma}
\begin{proof}
Let $f = 1_E$.
We recall from Proposition \ref{estimates on good set} that $\eta(f, s) \leq |\partial B_s|$, so that we can define
\begin{align*}
\psi(s) &= \int_{B_s} |df|~\vol - \eta(f, s)\\
&\leq h\eta(f, s) \leq h|\partial B_s| \leq d\omega_d hs^{d - 1}.
\end{align*}
From \cite[Proposition 5.12]{Giusti77},
$$\left|r^{1 - d} \int_{B_r} df ~\vol - \rho^{1 - d} \int_{B_\rho} df ~\vol\right|^2 \leq PQ,$$
where
\begin{align*}
P &= 2r^{1 - d}(1 + (d - 1)\alpha) \int_{B_r} |df| + 2(d - 1)^2 \int_\rho^r \log \frac{s}{\rho} \psi(s) \frac{ds}{s^d},\\
Q &= r^{1 - d} \int_{B_r} |df| - \rho^{1 - d} \int_{B_\rho} |df| + (d - 1)\int_\rho^r \psi(s) \frac{ds}{s^d}.
\end{align*}
Evidently $\log(s/\rho) \leq \alpha$ so we estimate
$$\frac{1}{\alpha} \int_\rho^r \log \frac{s}{\rho} \psi(s) \frac{ds}{s^d} \leq \int_\rho^r \psi(s) \frac{ds}{s^d} \leq d\omega_d h \int_\rho^r \frac{ds}{s} = d\omega_d h\alpha.$$
Thus
$$\int_{B_r} |df|~\vol \leq (1 + h)\eta(f, r) \leq (1 + h)|\partial B_r| \leq d\omega_d(1 + h) r^{d - 1}.$$
These estimates immediately give
\begin{align*}
P &\lesssim_d (1 + h)(1 + \alpha + h\alpha^2),\\
Q &\leq r^{1 - d} \int_{B_r} |df|~\vol - \rho^{1 - d} \int_{B_\rho} |df|~\vol + d\omega_d h\alpha. \qedhere
\end{align*}
\end{proof}

\begin{lemma}\label{approximate monotonicity 2}
Let $E$ be a Caccioppoli set in $\RR^d$ such that for some $0 \leq h < 1$ and $R > 0$ and every $0 < r < R$,
$$|\partial^* E \cap B_r| \leq (1 + h)\eta(f, r).$$
Then for every $0 < r < R$,
$$\partial_r \left(r^{1 - d} |\partial^* E \cap B_r| + d\omega_d h \log r\right) \geq 0.$$
\end{lemma}
\begin{proof}
Dividing by both sides of Lemma \ref{approximate monotonicity} by $(1 + h)(1 + \alpha + h\alpha^2) \geq 0$,
$$r^{1 - d} |\partial^* E \cap B_r| - \rho^{1 - d} |\partial^* E \cap B_\rho| + d\omega_d h \alpha \geq 0.$$
Writing out $\alpha = \log r - \log \rho$ we now see the claim.
\end{proof}

\subsection{Ruling out cusps}
We now show that a hypersurface which is approximately minimal cannot have ``cusp"-like singularities.

\begin{proposition}\label{uniform density estimate}
Suppose that $E$ is a Caccioppoli set in $M$, $p \in \partial^* E$, and $r_0 > 0$ is smaller than the strong injectivity radius of $p$.
If for every $r \in (0, r_0)$,
\begin{equation}\label{density estimate hypothesis}
|\partial^* E \cap B(p, r)| \leq 2\eta(1_E, B(p, r)),
\end{equation}
then for every such $r$,
\begin{align*}|E \cap B(p, r)|, ~|(M \setminus E) \cap B(p, r)| &\gtrsim r^d\\
|\partial^* E \cap B(p, r)| &\gtrsim r^{d - 1}.\end{align*}
\end{proposition}
\begin{proof}
Let $q = d/(d-1)$ and write $B_r = B(p, r)$, $f_r = 1_{E \cap B_r}$, $F(r) = |E \cap B_r|$, so $F'(r) = |E \cap \partial B_r|$.
Sobolev and Poincar\'e inequalities hold for $BV$ functions on $\RR^d$ \cite[\S5.6.1]{evans1991measure}.
By (\ref{definition of c-zeta}) and Sobolev embedding,
\begin{align*}
F(r)^{1/q} &= ||f_r||_{L^q(M)} \lesssim \int_M |df_r| ~\vol = |\partial^*(E \cap B_r)|\\
&\leq |\partial^* E \cap B_r| + |E \cap \partial B_r|.
\end{align*}
Therefore, by (\ref{density estimate hypothesis}),
$$|E \cap B_r|^{1/q} \leq 2\eta(1_E, B_r) + F'(r).$$
The trace of $1_E$ on $\partial B_r$ is equal to the trace of a function which is supported in an arbitrarily small neighborhood of $E \cap \partial B_r$, so
$$\eta(1_E, B_r) \leq |E \cap \partial B_r| = F'(r)$$
and hence
$$F(r)^{1/q} \lesssim F'(r).$$
TODO: Bound this with reverse Gr\"onwall

$F(r) = |E \cap B_r$
Since $|E \cap \partial B(p, r)| = \partial_r F(r)$,
$$F(r)^{1/q} \lesssim_d F'(r) \leq F'(r).$$
This gives the estimate on $|E \cap B(p, r)|$.
Since $M \setminus E$ shares its perimeter with $E$, the same argument with $M \setminus E$ replacing $E$ gives the analogous bound on $|(M \setminus E) \cap B(p, r)|$.

Let $f = 1_{E \cap B(p, r)}$ and let
$$[f] = \frac{1}{|B(p, r)|} \int_{B(p, r)} f ~\vol'.$$
By (\ref{definition of c-zeta}),
$$|\partial^* (E \cap B(p, r))| \geq 0.5 |\partial^* (E \cap B(p, r))|' = 0.5 \int_{B(p, r)} |df| ~\vol'.$$
By the $BV$ Poincar\'e inequality \cite[\S5.6.1]{evans1991measure}, it follows that
\begin{align*}
|\partial^* (E \cap B(p, r))| &\gtrsim_d ||f - [f]||_{L^1(B(p, r), \vol')} \\
&= ||1 - [f]||_{L^1(B(p, r) \cap E, \vol')} + ||[f]||_{L^1(B(p, r) \setminus E, \vol')}.
\end{align*}
From (\ref{definition of c-zeta}),
$$|E \cap B(p, r)| \leq 2\omega_{d - 1} r^{d - 1} [f] \leq 4|E \cap B(p, r)|,$$
so (TODO: This is wrong!)
\begin{align*}
|\partial^* (E \cap B(p, r))| &\gtrsim_d ||1 - [f]||_{L^1(B(p, r) \cap E, \vol')} + ||[f]||_{L^1(B(p, r) \setminus E, \vol')}\\
&\geq 0.5 (1 - [f]) |B(p, r) \cap E| + 0.5 [f] |B(p, r) \setminus E|\\
&\geq 0.5 r^{1-d}|E \cap B(p, r)||E \setminus B(p, r)| \gtrsim_d r^{1 - d}. \qedhere
\end{align*}
\end{proof}

\subsection{Blowup of the reduced boundary}
Now let us study the blowup of $M$ at a point $p$ on the reduced boundary of a set $U$ of least perimeter.
In this regard, we will be interested in objects that depend on a small parameter $t > 0$, but may depend on a choice of subsequence of $t_n \to 0$.
Thus we will suppress all subindices and implicitly pass to subsequences whenever referring to a limit $t \to 0$.\footnote{This can be made rigorous as follows. Fix a nonprincipal ultrafilter $\mathbf p$ on $2^{\aleph_0}$, and \emph{define} any limit as $t \to 0$ to be a $\mathbf p$-limit. See \cite{Tao07} for details. TODO: Is this actually useful? When I do even take subsequences?}

\begin{proposition}\label{blowup theorem}
Fix $p \in M$ with $\zeta \in (0, 1)$ chosen to be at most the strong injectivity radius of $p$.
Suppose that $U$ is an open set with least perimeter in $B(p, \zeta)$ and $p \in \partial^* U$.
Let $A = {\exp_p}^* U$, $A_t = \{v \in T_pM: tv \in A\}$, and $u_t = 1_{A_t}$.
Then:
\begin{enumerate}
\item The sequence $(u_t)$ has approximately least gradient with respect to the flat metric on $T_pM$ and in fact satisfies the estimate
\begin{equation}\label{approximately least gradient target}|
\partial^* A_t \cap V|' \leq (1 + ct^2)\eta'(A_t, V) \leq \eta'(U_t, V) + ct^2|\partial^* V|'
\end{equation}
for every Caccioppoli set $V \Subset T_pM$.
\item There exists an indicator function $u_0 \in BV_l(B'(r_0))$ of least gradient such that $u_t \to u_0$ in $L^1_l$, almost everywhere, and in total variation on sets with no singularities.
\item Let $C = \{u_0 = 1\}$. Then $0 \in \partial C$ and, if $d \leq 7$, then $\partial C$ is a hyerplane.
\end{enumerate}
\end{proposition}
\begin{proof}
TODO: Update this proof to make it a hyperplane and also not suck.

Suppose that (\ref{approximately least gradient target}) holds.
Then $(u_t)$ clearly is a net of indicator functions of least gradient, so the existence of $u_0$ and $C$ is immediate from Corollary \ref{compactness}.
The analyticity of $\partial C$ then follows from \cite[Corollary 9.5]{Giusti77}, since $T_pM$ is flat, $d \leq 7$, and $C$ has least perimeter in $T_pM \cong \RR^d$.

Therefore we just have to prove the estimate (\ref{approximately least gradient target}), and that $0 \in \partial C$.

Let $B'_r$ denote the ball (with respect to the flat metric) of radius $r > 0$ centered at $0$ in $T_pM$; then ${\exp_p}_* B'_r = B(p, r)$ if $r < \zeta$.
Moreover, $A_t$ enjoys the scaling invariance
\begin{equation}\label{scaling of psi}
|\partial^* A_t \cap B'_r|' = t^{1 - d} |\partial^* A \cap B'_r|'.
\end{equation}
To prove (\ref{scaling of psi}), let $\mathscr H^{d - 1}$ denote $(d-1)$-dimensional Hausdorff measure with respect to the euclidean metric.
Then by \cite[Theorem 4.4]{Giusti77},
$$|\partial^* A \cap E|' = \mathscr H^{d - 1}(\partial^* A \cap E),$$
and $\mathscr H^{d - 1}$ scales as desired.

Let $V \Subset B'_\zeta$ be an arbitrary open Caccioppoli set.
We write $V_t = \{x \in \RR^d: tx \in V\}$, so that $V_{1/t} \subseteq B'_{t\zeta}$ and so by (\ref{definition of c-zeta}), $\vol/\vol' \approx 1 \pm ct^2$ on $V_t$.

\begin{claim}\label{blowup claim 1}
$(U_t)$ satisfies the estimate (\ref{approximately least gradient target}).
\end{claim}

We must show that for every $w \in BV_c(V)$\footnote{The definition of bounded variation only depends on the equivalence class of a measure with respect to mutual absolute continuity, and in particular does not depend on a choice of volume form.},
\begin{equation}\label{approximately least gradient target 2}
|\partial^* A_t \cap V|' \leq (1 + ct^2) \int_V |du_t + dw| ~\vol'.
\end{equation}
Indeed, if (\ref{approximately least gradient target 2}) holds, then we can take the infimum over all $w$ to get
$$|\partial^*AU_t \cap V|' \leq (1 + ct^2) \eta'(u_t, V)$$
and then complete the proof using Lemma \ref{estimates on good set} to bound $ct^2 \eta'(u_t, V) \leq ct^2 |\partial^* V|$.

We now prove (\ref{approximately least gradient target 2}).
Using (\ref{scaling of psi}, \ref{definition of c-zeta}),
$$t^{d - 1} | \partial^* A_t \cap V|' = |\partial^* A \cap V_{1/t}|' \leq (1 + ct^2) |\partial^* A \cap V_{1/t}|.$$
For a function $a$ we let $a_t$ denote the rescaling $a_t(x) = a(tx)$.
Since $U$ has least perimeter,
$$|\partial^* A \cap V_{1/t}| \leq \int_{V_{1/t}} |d(u + w_{1/t})| ~\vol = \int_{V_{1/t}} |d(u + w_{1/t})| \frac{\vol}{\vol'} ~\vol'.$$
Thus
$$|\partial^* A \cap V_{1/t}| \leq (1 + ct^2) \int_{V_{1/t}} |d(u + w_{1/t})| ~\vol'$$
and so
$$|\partial^* A_t \cap V|' \leq (t^{1 - d} + ct^{3 - d}) \int_{V_{1/t}} |d(u + w_{1/t})| ~\vol'.$$
Rescaling using (\ref{scaling of psi}), we obtain (\ref{approximately least gradient target 2}) and hence Claim \ref{blowup claim 1}.

\begin{claim}
$0 \in \partial C$.
\end{claim}

Owing to (\ref{approximately least gradient target}, \ref{definition of c-zeta}), if $r,t$ are small enough then
$$|\partial^* A_t \cap B'_r| \leq 1.5|\partial^* A_t \cap B'_r|' \leq 1.5(1 + ct^2) \eta'(u_t, B'_r) \leq 2 \eta(u_t, B'_r)$$
and hence by Proposition \ref{uniform density estimate},
$$|\partial C \cap B'(r)| \gtrsim r^{d - 1}.$$
Since $\partial C$ is a smooth $(d-1)$-dimensional hypersurface which is closed as a subset of $T_pM$, this is only possible if $0 \in \partial C$.
\end{proof}

One can show that just to show the existence of a (possibly singular) tangent cone, one only needs that $U$ is a Caccioppoli set.
This just requires a slight modification of our argument to show that $(u_t)$ is bounded in $BV_l$, as in the proof of \cite[Theorem 9.3]{Giusti77}.
However, we will never need this fact.

%%%%%%%%%%%%%%%%%%%%%%%%%%%%%%%%

\section{Deformations of the mean-value property}
In this subsection, we fix $\ell \geq 1$, not necessarily equal to $d$, and prove the following proposition.

\begin{proposition}\label{bootstrap elliptic}
Let $\varepsilon > 0$, let $P$ be a uniformly elliptic operator with smooth coefficients on $B_{2(1 + \varepsilon)}\rho$, a ball centered on the origin in $\RR^\ell$, such that $P$ admits a divergence form
$$P = -\Div A \nabla + b \cdot \nabla + c$$
such that $A(0) = 1$, $A$ is symmetric and has a critical point at $0$, and $b(0) = c(0) = 0$.
Let $\omega$ be a volume form on $\RR^\ell$ whose Radon-Nikod\'ym derivative has a zero at $0$.
If $\rho > 0$ is small enough depending on $P, \varepsilon, \omega$, $Pu = 0$ on $B_{2(1 + \varepsilon)}\rho$, and $0$ is a critical point of $u$, then
$$||du||_{L^2(B_\rho)}^2 \leq 2^{-\ell-2}||du||_{L^2(B_{2(1 + \varepsilon)\rho})}^2.$$
\end{proposition}

Before proving Proposition \ref{bootstrap elliptic} we note how it will be used in our application, as well as record the special case from which the general case will follow.

\begin{proposition}\label{bootstrap Laplace-Beltrami}
Let $(N, \slashed g)$ be a $\ell$-dimensional Riemannian manifold and let $\Omega \Subset N$.
Then for every $\varepsilon > 0$ there exists
$$\rho_* = \rho_*(\varepsilon, \Omega, \slashed g) > 0$$
withe following property.
For every $x \in \Omega$, every $\rho < \rho_*$, and every function $u$ such that $\Delta_{\slashed g} u = 0$ and $\nabla u(x) = 0$,
$$||du||_{L^2(B(x, \rho))}^2 \leq 2^{-\ell-2} ||du||_{L^2(B(x, 2(1 + \varepsilon)\rho))}^2.$$
Moreover, if the Riemann curvature tensor of $\slashed g$ vanishes, then $\rho_* = \min(1, \rho_{**})$ where $\rho_{**}$ is the infimum of all injectivity radii on $\Omega$.
The proof in the case that the Riemann curvature tensor vanishes does not require Proposition \ref{bootstrap elliptic}.
\end{proposition}
\begin{proof}
If the Riemann curvature tensor vanishes, then we can replace $B(x, \rho_{**})$ with the ball of radius $\rho_{**}$ in $\RR^\ell$ with its euclidean metric.
So the claim is obvious from \cite[Lemma 4.1]{Miranda66}.

Otherwise, we may apply normal coordinates to assume that $N = \RR^\ell$, $x = 0$, and $\slashed g^{ij} = \delta^{ij} + O(|x|^2)$.
Since the Laplace-Beltrami operator is defined by
$$\Delta_{\slashed g} = \frac{1}{\sqrt{\det g}} \partial_i \sqrt{\det g} g^{ij} \partial_j,$$
if we set $P = -\Delta$, then $A^{ij} = -g^{ij}$, $c = 0$, and
$$b^j(0) = \partial_i \log \sqrt{|\det g(0)|} g^{ij}(0) = 0,$$
so the claim follows from Proposition \ref{bootstrap elliptic} with $\omega = \slashed \vol$.
\end{proof}

\begin{definition}
We call $\rho_*$ as in Proposition \ref{bootstrap Laplace-Beltrami} the \dfn{averaging scale} of $\slashed g$ on $\Omega$ with \dfn{tolerance} $\varepsilon$.
\end{definition}

We have shown that Proposition \ref{bootstrap elliptic} implies the noneuclidean case in Proposition \ref{bootstrap Laplace-Beltrami},
and we have shown the euclidean case of Proposition \ref{bootstrap Laplace-Beltrami} without appealing to Proposition \ref{bootstrap elliptic}.
The rest of this section is dedicated to showing that the euclidean case of Proposition \ref{bootstrap Laplace-Beltrami} implies Proposition \ref{bootstrap elliptic},
which will in particular imply that the noneuclidean case of Proposition \ref{bootstrap Laplace-Beltrami} holds.

We begin by applying Taylor's theorem to the hypotheses on $P$ to deduce that
\begin{equation}\label{Taylor coefficients are small}
|A(x) - 1| + |b(x)| + |c(x)| \lesssim |x|,
\end{equation}
at least for $\rho$ small depending on $A, b, c$.
This suggests that we should approximate $P$ by the (euclidean) Laplace operator.
To accomplish this, we introduce the trace operator
\begin{align*}
T: H^s(B_{2(1 + \varepsilon)\rho}) &\to H^{s-.5}(\partial B_{2(1 + \varepsilon/2)\rho})
\end{align*}
whenever $s > .5$.
Scaling considerations show that
$$||T||_{H^s(B_{2(1 + \varepsilon)\rho}) \to H^{s - .5}(\partial B_{2(1 + \varepsilon/2)\rho})} \sim \rho^{-1/2}.$$
Let
$$G_P, G_{-\Delta}: H^{s - .5}(\partial B_{2(1 + \varepsilon/2)\rho}) \to H^{s + 1.5}(B_{2(1 + \varepsilon/2)\rho})$$
be the operators which solve the Dirichlet problems for $P,-\Delta$.
Then $||G_P||, ||G_{-\Delta}|| \sim \rho^{1/2}$ in $H^{s - .5} \to H^{s + 1.5}$.

If $Pu = 0$, then $G_PTu = u|B_{2(1 + \varepsilon/2)\rho}$, while $G_{-\Delta}Tu$ is a harmonic function on $B_{2(1 + \varepsilon/2)\rho}$ which approximates $u$ in the following sense.

\begin{lemma}[harmonic approximation]\label{approx harmonic}
If $Pu = 0$, then for every $s, t \in \RR$,
$$||G_{-\Delta}Tu - u||_{H^t(B_{2(1 + \varepsilon/2)})} \lesssim_{s, t} \rho ||u||_{H^s(B_\rho)}.$$
\end{lemma}
\begin{proof}
We prove this by comparing the Green's functions of $-\Delta$ and $P$. Let
$$P^* = -\Div A \nabla - b \cdot \nabla + (c - \Div b)$$
be the formal adjoint to $P$, so that
\begin{equation}\label{formal adjoint equation}
\int_{B_{2(1 + \varepsilon/2)\rho}} uP^*v - vPu ~\vol = \int_{\partial B_{2(1 + \varepsilon/2)\rho}} uv b \cdot A\normal + v\partial_{A\normal} u - u\partial_{A\normal} v ~\vol_{\partial B_{2(1 + \varepsilon/2)\rho}}
\end{equation}
where $\normal$ is normal to $\partial B_{2(1 + \varepsilon/2)\rho}$.
If we define $F_P$ to solve the Dirichlet problem
\begin{equation}\label{defining the Green function}
\begin{cases}P^* F_P(\cdot, y) = \delta_y,\\
F_P|\RR^\ell \times \partial B_{2(1 + \varepsilon/2)\rho} = 0,
\end{cases}
\end{equation}
then (\ref{formal adjoint equation}) simplifies to
$$u(x) = \int_{\partial B_{2(1 + \varepsilon/2)\rho}} F_P(x, y)u(y) b(y) \cdot \normal - u(y) (\partial_{A \normal} F(x, \cdot))(y) ~\vol_{\partial B_{2(1 + \varepsilon/2)\rho}}(y).$$
In particular,
$$G_P(x, y) = F_P(x, y)b(y) \cdot \normal - (\partial_{A \normal} F_P(x, \cdot))(y).$$
The solution to the Dirichlet problem (\ref{defining the Green function}) is
$$F_P(x, y) = K_P(y - x) - K_P\left(2(1 + \varepsilon/2)\rho \frac{y}{|y|} - |y| \frac{x}{2(1 + \varepsilon/2)\rho}\right)$$
where $K_P$ is the newtonian kernel associated to $P$.
Thus, if $L = G_P - G_{-\Delta}$,
\begin{align*}
L(x, y) &= (b \cdot \normal(y) - \partial_{(A - 1)\normal})F_P(x, y) + \partial_\normal (F_P - F_{-\Delta})(x, y)
\end{align*}
where all normal derivatives are taken in $y$.

Let $\Japan{\Delta}^{s/2}$ be the pseudodifferential operator with symbol $\Japan{\xi}^s$.
Then it follows from the definitions that
$$||w||_{H^s} = ||\Japan{\Delta}^{s/2} w||_{L^2}.$$
In order to use (\ref{Taylor coefficients are small}), we estimate, with $A_\rho = B_{2(1 + \varepsilon/2)\rho} \times \partial B_{2(1 + \varepsilon/2)\rho}$ and $w = (G_P - G_{-\Delta})v$,
\begin{align*}
||w||_{H^s(B_{2(1 + \varepsilon/2)\rho})}^2 &\lesssim_s ||\Japan{\Delta}^{s/2} \int_{\partial B_{2(1 + \varepsilon/2)\rho}} L(\cdot, y) v(y) ~dy||_{L^2(B_{2(1 + \varepsilon/2)\rho})}^2\\
&\lesssim \rho^2 \left|\int_{B_{2(1 + \varepsilon/2)\rho}} \Japan{\Delta_x}^{(s+1)/2} \int_{\partial B_{2(1 + \varepsilon/2)\rho}} F_{-\Delta}(x, y) v(y) ~dy ~dx\right|^2 \\
&\qquad + \rho^2 \iint_{A_\rho} |\Japan{\Delta_x}^{(s+1)/2} (F_P(x, y) - F_{-\Delta}(x, y)) v(y)|^2 ~dy ~dx \\
&\qquad + \iint_{A_\rho} |\Japan{\Delta_x}^{s/2} (F_P(x, y) - F_{-\Delta}(x, y)) v(y)|^2 ~dy ~dx \\
&\lesssim \rho^2 ||G_{-\Delta}v(x)||_{H^{s+1}(B_{2(1 + \varepsilon/2)\rho})}^2\\
&\qquad + \iint_{A_\rho} |\Japan{\Delta_x}^{(s+1)/2} (F_P(x, y) - F_{-\Delta}(x, y)) v(y)|^2 ~dy ~dx\\
&\lesssim: \rho ||v||_{H^{s + 2}(\partial B_{2(1 + \varepsilon/2)\rho})}^2 + I.
\end{align*}
To control $I$, we apply Taylor's theorem to the map $Q \mapsto Q^{-1}$ centered on $-\Delta$ to conclude that
\begin{equation}\label{linearized inversion}
P^{-1} = -\Delta^{-1} - \Delta^{-1}(P + \Delta)\Delta^{-1} + \cdots.
\end{equation}
By (\ref{Taylor coefficients are small}),
$$||P + \Delta||_{H^r(B_{2(1 + \varepsilon/2)\rho}) \to H^{r+2}(B_{2(1 + \varepsilon/2)\rho})} \lesssim \rho,$$
so by (\ref{linearized inversion}),
$$||(K_P - K_{-\Delta})(\cdot, y)||_{H^{s + 1}(B_{2(1 + \varepsilon/2)\rho})} \lesssim_s \rho ||\Delta^{-2} \delta_y||_{H^{s + 1}(B_{2(1 + \varepsilon/2)\rho})} \lesssim \rho ||\delta_0||_{H^{s-3}}.$$
Moreover, $\delta_0 \in H^{s - 3}$ provided that
\begin{equation}\label{delta regularity}
s < 3 - \frac{\ell}{2}.
\end{equation}

Assuming (\ref{delta regularity}),
\begin{align*}
I &\lesssim ||v||_{L^\infty(\partial B_{2(1 + \varepsilon/2)\rho})}^2 \iint_{A_\rho} |\Japan{\Delta_x}^{(s+1)/2}(F_P - F_{-\Delta})(x, y)|^2 ~dy ~dx\\
&\lesssim \rho^{\ell - 1} ||v||_{L^\infty(\partial B_{2(1 + \varepsilon/2)\rho})}^2 \sup_{y \in \partial B_{2(1 + \varepsilon/2)\rho}} ||F_P(\cdot, y) - F_{-\Delta}(\cdot, y)||_{H^{s + 1}(B_{2(1 + \varepsilon/2)\rho})}^2\\
&\lesssim \rho^{\ell - 1} ||v||_{L^\infty(\partial B_{2(1 + \varepsilon/2)\rho})}^2 \sup_{y \in \partial B_{2(1 + \varepsilon/2)\rho}} ||K_P(\cdot, y) - K_{-\Delta}(\cdot, y)||_{H^{s + 1}(B_{2(1 + \varepsilon/2)\rho})}^2\\
&\lesssim \rho^{\ell + 1} ||v||_{L^\infty(\partial B_{2(1 + \varepsilon/2)\rho})}^2.
\end{align*}
By Sobolev embedding, there exists $s' > 0$ so large that
$$||w||_{L^\infty(\partial B_{2(1 + \varepsilon/2)\rho})}^2 \lesssim \rho^{1-\ell} ||v||_{H^{s'}(\partial B_{2(1 + \varepsilon/2)\rho})}^2.$$
Now if we assume that $s' > s + 2$, it follows that
$$||w||_{H^s(B_{2(1 + \varepsilon/2)\rho})} \lesssim \rho^{1/2} ||v||_{H^{s'}(\partial B_{2(1 + \varepsilon/2)\rho})}.$$
By elliptic regularity applied to $-\Delta$, we may replace $s$ with an arbitrarily large exponent and this inequality will still hold, so we can drop the assumption (\ref{delta regularity}).
We can also take $s' > 0.5$ and write $v = Tu$, where $Pu = 0$ (so $G_PTu = u|\partial B_{2(1 + \varepsilon/2)\rho}$) so that
$$||w||_{H^s(B_{2(1 + \varepsilon/2)\rho})} \lesssim \rho ||u||_{H^{s' + .5}(B_{2(1 + \varepsilon)\rho})}.$$
Now we can apply elliptic regularity to $P$ and replace $s'$ with an arbitrarily small exponent.
\end{proof}

We now need to control the error that arises in the above approximation lemma.

\begin{lemma}\label{Poincare lemma}
Suppose that $u \in H^2(B_{2(1 + 3\varepsilon/4)\rho})$, $Pu = 0$, and $\int_{B_{2(1 + 3\varepsilon/4)\rho}} u = 0$.
Then
\begin{equation}\label{Dirichlet problem for Q}
||du||_{L^2(B_{(1 + 3\varepsilon/4)}, T^* \RR^\ell)} \lesssim \rho^{1/2} ||Tdu||_{H^1(\partial B_{(1 + 3\varepsilon/4)\rho}, T^* \RR^\ell)}.
\end{equation}
\end{lemma}
\begin{proof}
We begin by constructing a suitable exterior integration operator.
So let $H^s_{\mathrm{cl}}(S, T^* \RR^\ell)$ be the space of $H^s$ closed $1$-forms on an open set $S$ of finite measure.
If in addition $\Homology^1(S, \CC) = 0$, then there exists a unique integration operator
$$d^{-1}: H^s_{\mathrm{cl}}(S, T^* \RR^\ell) \to H^{s + 1}(S)$$
which sends every $1$-form with uniformly bounded derivatives to a function of mean zero.
By an approximation argument, then, every $1$-form in the image of $d^{-1}$ must have mean zero.
Since $d^{-1}$ is an elliptic parametrix, it is a pseudodifferential operator of order $-1$.

Let $Q = dPd^{-1}$, which is well-defined as an operator
$$Q: H^{s + 2}_{\mathrm{cl}}(B_{2(1 + 3\varepsilon/4)\rho}, T^* \RR^\ell) \to H^s(B_{2(1 + 3\varepsilon/4)\rho})$$
because $B_{2(1 + 3\varepsilon/4)\rho}$ is contractible.
Then $Q$ is a pseudodifferential operator of second order, and since the principal symbol $\sigma$ of a pseudodifferential operator is invariant under conjugation by an pseudodifferential operator of lower order, $\sigma_Q = \sigma_P$.
Therefore $Q$ is uniformly elliptic, so we can solve the Dirichlet problem for $Q$, say with a solution operator
$$G_Q: T(H^1_{\mathrm{cl}}(B_{2(1 + 3\varepsilon/4)\rho}, T^* \RR^\ell)) \to L^2(B_{2(1 + 3\varepsilon/4)\rho}).$$
Since $H^1_{\mathrm{cl}}(B_{2(1 + 3\varepsilon/4)\rho}, T^* \RR^\ell)$ is the kernel of
$$d: H^1(B_{2(1 + 3\varepsilon/4)\rho}, T^* \RR^\ell) \to L^2(B_{2(1 + 3\varepsilon/4)\rho}, T^* \RR^\ell \wedge T^* \RR^\ell),$$
it is a Hilbert space and therefore, since $G_Q$ was defined with no arbitrary choices made, $G_Q$ must be bounded.
Scaling considerations then show that $||G_Q|| \lesssim \rho^{1/2}$.

Since $u$ has mean $0$, $d^{-1}du = 0$, so
$$Qdu = dPd^{-1}du = dPu = 0$$
since $Pu$ is constant.
Therefore $du = G_QTdu$, so the estimate on $||G_Q||$ readily implies (\ref{Dirichlet problem for Q}).
\end{proof}

TODO: The rest of this proof is wrong!!

Let $u, \omega$ meet the hypotheses of Proposition \ref{bootstrap elliptic} and let $\lambda = 2(1 + 3\varepsilon/4)$.
We can write $\omega = *(1 + f)$ where $f(x) = O(|x|^2)$ and $*$ is the euclidean Hodge star.
Thus for every $v \in L^1$,
\begin{equation}\label{estimates on the volume form}
(1 - O(\rho^2)) ||v||_{L^1} \leq \int_{B_{\lambda \rho}} |v|\omega \leq (1 + O(\rho^2)) ||v||_{L^1}.
\end{equation}
After applying a translation, we may assume that $\int_{B_{\lambda \rho}} u = 0$.
Then by elliptic regularity, the Poincar\'e inequality, and (\ref{estimates on the volume form}), for every $s > 0$,
\begin{equation}\label{poincare error}
||u||_{H^s(B_{\lambda \rho}, \omega)}^2 \lesssim_s ||u||_{L^2(B_{\lambda \rho}, \omega)}^2 \lesssim \rho^2 ||du||_{L^2(B_{\lambda \rho}, \omega)}^2 \lesssim (\rho^2 + \rho^4) ||du||_{L^2(B_{\lambda \rho})}^2 \lesssim \rho^2 ||du||_{L^2(B_{\lambda \rho})}^2.
\end{equation}
Write $u = v + w$ where $v = G_{-\Delta}Tu$, so that by Lemma \ref{approx harmonic} and (\ref{poincare error}),
$$||dw||_{L^2(B_{\lambda \rho}, \omega)}^2 \lesssim \rho^2 ||u||_{H^1(B_{\lambda \rho}, \omega)}^2 \lesssim \rho^4 ||du||_{L^2(B_{\lambda \rho})}^2.$$
Moreover, $\Delta v = 0$ and $w$ is trace-free.
Since $|du(0)| = 0$,
$$||du||_{B_r} \lesssim |\Hess u(0)|r^2$$
for $r$ small.
Moreover, $\Delta dv = d\Delta v = 0$, so $dv$ satisfies a mean-value property and hence, by the Cauchy-Schwarz inequality,
\begin{align*}
|dv(0)|^2 &\lesssim \rho^{2(1 - \ell)} \left(\int_{\partial B_{\lambda \rho}} |dv|\right)^2 = \rho^{2(1 - \ell)} \int_{\partial B_{\lambda \rho}} |du|^2 \cdot \int_{\partial B_{\lambda \rho}} 1
\end{align*}
which implies
$$|dv(0)|^2 \lesssim |\Hess u(0)|^2 \rho^2.$$
So by the euclidean case of Proposition \ref{bootstrap Laplace-Beltrami},
\begin{align*}
||du||_{L^2(B_\rho)}^2 &\leq ||dv||_{L^2(B_\rho)}^2 + ||dw||_{L^2(B_\rho)}^2 \\
&\leq 2^{-\ell-2} ||dv||_{L^2(B_{2\rho})}^2 + |dv(0)|^2 + O(\rho^4) ||du||_{L^2(B_{2(1 + 3\varepsilon/4)\rho})}^2\\
&\leq 2^{-\ell-2} ||dv||_{L^2(B_{2\rho})}^2 + O(\rho^2) |\Hess u(0)|^2 + O(\rho^4) ||du||_{L^2(B_{2(1 + 3\varepsilon/4)\rho})}^2.
\end{align*}
A similar argument shows
\begin{align*}
||dv||_{L^2(B_{2\rho})}^2 &\leq ||du||_{L^2(B_{2\rho})}^2 + O(\rho^2) |\Hess u(0)|^2 + O(\rho^4) ||du||_{L^2(B_{2(1 + 3\varepsilon/4)\rho})}^2,
\end{align*}
so
$$||du||_{L^2(B_\rho)}^2 \leq 2^{-\ell-2}||du||_{L^2(B_{2\rho})}^2 + O(\rho^2) |\Hess u(0)|^2 + O(\rho^4) ||du||_{L^2(B_{2(1 + 3\varepsilon/4)\rho})}^2.$$
By Lemma \ref{Poincare lemma},
\begin{align*}
||du||_{L^2(B_{2(1 + 3\varepsilon/4)\rho})} &\lesssim \rho^{1/2} ||Tdu||_{L^2(\partial B_{2(1 + 3\varepsilon/4)\rho})}
 \leq \rho^{1/2} ||Tdu||_{L^2(\partial(B_{2(1 + \varepsilon)\rho} \setminus B_{2(1 + 3\varepsilon/4)\rho}))} \\
&\lesssim ||u||_{L^2(B_{2(1 + \varepsilon)\rho} \setminus B_{2(1 + 3\varepsilon/4)\rho})}.
\end{align*}
Thus we have shown that
$$||du||_{L^2(B_\rho)}^2 \leq 2^{-\ell-2}||du||_{L^2(B_{2\rho})}^2 + O(\rho^4) ||du||_{L^2(B_{2(1 + \varepsilon)\rho} \setminus B_{2\rho})}^2$$
which is sufficient if $\rho$ is chosen so small that $O(\rho^4) < 2^{-\ell-2}$.
This completes the proof of Proposition \ref{bootstrap elliptic}.

%%%%%%%%%%%%%%%%%%%%%%%%%%%%%%%%%%%%%%%%%%%%%%%%%%%%
%
% \section{Quasilinear and linear hypersurface Laplacians}
% Let $N$ be a hypersurface in $\RR^d$.
% By the implicit function theorem, we can locally write $N$ as the graph of a function $f: \RR^{d - 1} \to \RR$.
% Then the Dirichlet energy $|df|^2 ~\vol_{d - 1}$, or more accurately $(1 + |df|^2/2) ~\vol_{d - 1}$, can be viewed as an approximation to the area element of $N$, which is the key observation used in the proof of \cite[Teorema 4.4]{Miranda66}, and ultimately in the theory of regularity of minimal surfaces.
% However, the effectiveness of this approximation is not rotation-invariant; if we rotate $\RR^d$ so that $N$ is nearly horizontal, then the approximation is much more accurate than if $N$ is nearly vertical.
%
% Thus, if $N$ is more generally a hypersurface in $M$, it is more natural to define the \emph{intrinsic} Dirichlet energy of an inclusion map $N \to M$ by viewing $N$ as a graph in some coordinate frame that is defined in terms of $N$ itself, so that $N$ is nearly horizontal in some sense.
% There are a handful of ways one could presumably do this, but we will construct a coordinate frame with this property, such that the metric on $M$ takes a particularly simple form.
% This will be very convenient for us, as it will cause several terms in the resulting Euler-Lagrange equations to vanish.
%
% To begin the construction, let $N$ be a $C^1$ hypersurface in the Riemannian manifold $M$, and let $P \in N$.
% We have a smooth hypersurface
% $$H = \exp_P(T_PM)$$
% and we consider the normal bundle $T^\perp H$.
% According to the tubular neighborhood theorem, the exponential map
% \begin{align*}
% T^\perp H &\to M \\
% (x, v) &\mapsto \exp_x(v)
% \end{align*}
% is a local diffeomorphism along $H$, say into an open set $U \subseteq M$ which contains an open neighborhood (which we henceforth also denote $H$) of $P$ and is foliated by normal translates of $H$.
% In particular, the unit vector field $X$ which is normal to each leaf of the foliation of $U$ induces geodesics $\gamma_x$, $x \in H$, defined by
% $$\begin{cases}
% \gamma_x(0) = x\\
% \gamma_x'(0) = X_x
% \end{cases}$$
% and hence coordinates
% $$H \times I \ni (x, y) \mapsto \gamma_x(y) \in U.$$
%
% \begin{lemma}
% The coordinates $(x, y): H \times I \to U$ satisfy:
% \begin{enumerate}
% \item (Natural) No arbitrary choices were made in the above construction (except possibly the size of $H$ and $I$).
% \item (Normalized) $H = \{y = 0\}$.
% \item (Graph structure) There exists a $C^1$ function $f: H \to I$ such that $N = \{y = f(x)\}$.
% \item (Approximately horizontal) The function $f$ has a double zero at $P$.
% \item (Synchronous gauge) The metric takes the form
% \begin{equation}\label{orthometric}
% g = \slashed g(x, y) ~dx^2 + dy^2.
% \end{equation}
% \end{enumerate}
% \end{lemma}
% \begin{proof}
% Naturality and normalization follow from the definitions. The graph structure follows from the implicit function theorem and the fact that $N$ is tangent to $H$.
% The fact that $N$ is tangent to $H$ also implies that $f(P) = \nabla f(P) = 0$.
% Finally the fact that $X_x$ is the unit normal field to each leaf of the foliation implies that $g_{xy} = 0$ and $g_{yy} = 1$, so the only nontrivial terms of $g$ are $g_{xx} = \slashed g$.
% \end{proof}
%
% Since $x$ is a coordinate on $H$, let
% $$\sqrt{\slashed g(x)} ~|dx| = \sqrt{\det \slashed g(x, 0)} ~|dx|$$
% denote the natural area element on $H$.
% In the following we let $0, 1, \dots, d-1$ index the coordinate directions, with $0$ the $y$-direction and $1, \dots, d-1$ the $x$-directions.
% We let $\mu, \nu, \dots$ range over $0, 1, \dots, d-1$ and $i,j, \dots$ range over $1, \dots, d - 1$.
% We write $(1, p)^\mu$ to mean $p^\mu$ if $\mu \neq 0$ and $1$ if $\mu = 0$.
%
% \begin{definition}
% The \dfn{area Lagrangian} is the area element on $H$ defined by
% $$\Lagrange = \sqrt{g(x, y)_{\mu\nu} (1, p)^\mu (1, p)^\nu} \sqrt{\slashed g(x)} ~|dx|,$$
% for a given $y \in I$ and $p \in \RR^{d - 1}$.
% \end{definition}
%
% Identifying $H$ with $N$ using the diffeomorphism $x \mapsto (x, f(x))$, $\Lagrange(x, f(x), \nabla f(x))$ becomes the natural area element on $N$.
% Moreover, because of our choice of coordinates, it takes a particularly simple form:
% $$\Lagrange = \sqrt{1 + \slashed g(x, y)_{ij} p^i p^j} \sqrt{\slashed g(x, 0)} ~|dx|.$$
% We now Taylor expand about $p = 0$ to get
% $$\Lagrange \approx \left(1 + \frac{\slashed g(x, y)_{ij} p^i p^j}{2}\right) \sqrt{\slashed g(x)} ~|dx|.$$
% (We will make this approximation more precise later.)
% The second term in the Taylor expansion is important enough that we give it a special name.
% To do so we introduce the notation
% $$\tilde f(x) = (x, f(x)),$$
% write $\slashed g = \slashed g(x, 0)$ and $\slashed g(\tilde f) = \slashed g(\tilde f(x))$.
%
% \begin{definition}
% The \dfn{quasilinear Dirichlet energy} is the area element on $H$ given by
% $$\DirQL = \slashed g(\tilde y)_{ij} p^i p^j \sqrt{\slashed g} ~|dx|$$
% for a given $y \in I$ and $p \in \RR^{d - 1}$.
% Its Euler-Lagrange operator is the \dfn{quasilinear Laplacian} $\LapQL$.
% \end{definition}
%
% Compare $\DirQL$ to the classical Dirichlet energy
% $$\DirL = \slashed g_{ij} p^i p^j \sqrt{\slashed g} ~|dx|$$
% whose Euler-Lagrange operator is the Laplacian
% $$\Delta_{\slashed g} = \slashed g_{ij} \partial^i \partial^j + \slashed g_{ij}^{,i} \partial^j + \slashed g_{ij} (\log \sqrt{\slashed g})^{,i} \partial^j.$$
% Our next lemma is a perturbation of this fact:
%
% \begin{lemma}
% The quasilinear Laplacian satisfies
% $$\LapQL f = \slashed g(\tilde f)_{ij} f^{,ij} + \frac{\slashed g^{,0}(\tilde f)_{ij}}{2} f^{,i} f^{,j} + \slashed g^{,i}(\tilde f)_{ij} f^{,j} + \slashed g(\tilde f)_{ij} (\log \sqrt{\slashed g})^{,i} f^{,j}.$$
% \end{lemma}
% \begin{proof}
% TODO: Proof has lots of typos, pls fix if we actually need this lemma
%
% The Euler-Lagrange equation is
% $$(\DirL)^{,0}|_{(y, p) = (f, \nabla f)} = \partial^k \left(\frac{\partial \DirL}{\partial p_k}|_{(y, p) = (f, \nabla f)}\right).$$
% Also
% $$(\DirL)^{,0} = \slashed g_{ij,0} p^i p^j \sqrt{\slashed g}$$
% while
% $$\frac{\partial \DirL}{\partial p_k} = 2\slashed g_{jk} p^j \sqrt{\slashed g}$$
% so if we denote $\tilde f(x) = (x, f(x))$,
% $$\frac{\partial \DirL}{\partial p_k}|_{(y, p) = (f, \nabla f)} = 2\slashed g_{jk}(\tilde f) f^{,j} \sqrt{\slashed g}$$
% and hence
% \begin{align*}
% \partial^k \frac{\partial \DirL}{\partial p_k}|_{(y, p) = (f, \nabla f)}
% &= 2\partial^k(\slashed g_{jk}(\tilde f)) f^{,j} \sqrt{\slashed g} + 2\slashed g_{jk}(\tilde f) f^{,jk} \sqrt{\slashed g} + 2\slashed g_{jk}(\tilde f) f^{,j} \partial^k \sqrt{\slashed g}\\
% &= 2(\slashed g_{jk}^{,\mu} (\tilde f) \tilde f_\mu^{,k} f^{,j} + \slashed g_{jk}(\tilde f) f^{,jk} + \slashed g_{jk}(\tilde f) f^{,j} \partial^k) \sqrt{\slashed g}.
% \end{align*}
% Thus the Euler-Lagrange equation is
% $$\slashed g_{ij}(\tilde f) f^{,ij} = \frac{\slashed g^{,0}(\tilde f)_{ij}}{2} f^{,i} f^{,j} - \slashed g_{ij}^{,\mu} (\tilde f) \tilde f_\mu^{,i} f^{,j} - \slashed g_{ij}(\tilde f) f^{,j} \partial^i (\log \sqrt{\slashed g}).$$
% Breaking up the sum over $\mu$ we get
% \begin{align*}
% \slashed g_{ij}^{,\mu} (\tilde f) \tilde f_\mu^{,i} f^{,j}
% &= \slashed g_{ij}^{,k} (\tilde f) x_k^{,i} f^{,j} + \slashed g_{ij}^{,0} (\tilde f) f^{,i} f^{,j} \\
% &= \slashed g_{ij}^{,i} (\tilde f) f^{,j} + \slashed g_{ij}^{,0} (\tilde f) f^{,i} f^{,j}
% \end{align*}
% whence the Euler-Lagrange equation simplifies to
% \begin{align*}
% \slashed g_{ij}(\tilde f) f^{,ij} + \frac{\slashed g_{ij, 0} (\tilde f)}{2} f^{,i} f^{,j} + \slashed g^{,i}_{ij}(\tilde f) f^{,j} + \slashed g_{ij}(\tilde f) (\log \sqrt{\slashed g})^{,i} f^{,j} &= 0. \qedhere\end{align*}
% \end{proof}
%
% Now we discuss in what sense the quasilinear and classical Laplacians approximate the minimal surface operator:
%
% \begin{lemma}
% For every $y,z,p,q$ such that $|q|_{\slashed g(\tilde z)}^2 \leq 15$ and $|z| \leq C|y|$ for some fixed $C > 0$,
% $$\frac{\DirQL(y, p) - \DirQL(z, q)}{2 \sqrt{1 + |q|^2_{\slashed g(\tilde z)}}} - \frac{(\DirQL(y, p) - \DirQL(z, q))^2}{2\Lagrange(z, q)} \leq \Lagrange(y, p) - \Lagrange(z, q) \leq \frac{\DirQL(y, p) - \DirQL(z, q)}{2}.$$
% Moreover,
% $$\DirQL(y, p) - \DirQL(z, q) = \DirL(y, p) - \DirL(z, q) + \slashed g_{ij}^{,0}((y - z) p^i p^j + z(p^i p^j - q^i q^j))\sqrt{\slashed g} + \kappa$$
% where the error term $\kappa$ satisfies
% $$|\kappa| \lesssim_{C, \slashed g} |y|^2(|p|^2 + |q|^2).$$
% \end{lemma}
% \begin{proof}
% By Taylor's theorem, there exists $\xi \geq 0$ between $|p|_{\slashed g(\tilde y)}$ and $|q|_{\slashed g(\tilde z)}$ such that
% \begin{align*}
% \Lagrange(y, p) - \Lagrange(z, q) &= (\sqrt{1 + |p|^2_{\slashed g(\tilde y)}} - \sqrt{1 + |q|^2_{\slashed g(\tilde z)}}) \sqrt{\slashed g} \\
% &= \frac{\sqrt{\slashed g}}{2 \sqrt{1 + |q|^2_{\slashed g(\tilde z)}}}(|p|^2_{\slashed g(\tilde y)} - |q|^2_{\slashed g(\tilde z)}) - \frac{\sqrt{\slashed g}}{8(1 + \xi^2)^{3/2}}(|p|^2_{\slashed g(\tilde y)} - |q|^2_{\slashed g(\tilde z)})^2 \\
% &= \frac{\DirQL(y, p) - \DirQL(z, q)}{2 \sqrt{1 + |q|^2_{\slashed g(\tilde z)}}} - \frac{(\DirQL(y, p) - \DirQL(z, q))^2}{8(1 + \xi^2)^{3/2}\sqrt{\slashed g}}.
% \end{align*}
% Since $|q|_{\slashed g(\tilde z)}^2 \geq 0$ and the second term is nonpositive, it follows that
% $$\Lagrange(y, p) - \Lagrange(z, q) \leq \frac{\DirQL(y, p) - \DirQL(z, q)}{2}.$$
% and, since $|q|_{\slashed g(\tilde z)}^2 \leq 15$,
% $$2\Lagrange(z, q) \leq 8\sqrt{\slashed g} \leq 8\sqrt{\slashed g}(1 + \xi^2)^{3/2},$$
% whence
% $$\frac{\DirQL(y, p) - \DirQL(z, q)}{2 \sqrt{1 + |q|^2_{\slashed g(\tilde z)}}} - \frac{(\DirQL(y, p) - \DirQL(z, q))^2}{2\Lagrange(z, q)} \leq \Lagrange(y, p) - \Lagrange(z, q).$$
% We can further expand $\DirQL$ in terms of $\DirL$ by using Taylor's theorem to find $\eta, \zeta$ such that $|\eta| < y^2$ and $|\zeta| < z^2$ and
% \begin{align*}
% \DirQL(y, p) - \DirQL(z, q) &= \DirL(y, p) - \DirL(z, q) + (\slashed g(\tilde y)_{ij} p^i p^j - \slashed g(\tilde z)_{ij} q^i q^j - \slashed g_{ij}(p^ip^j - q^iq^j))\sqrt{\slashed g}\\
% &= \DirL(y, p) - \DirL(z, q) + (\slashed g_{ij}^{,0}(yp^i p^j - zq^i q^j) + \slashed g_{ij}^{,00}(\eta p^ip^j - \zeta q^iq^j)) \sqrt{\slashed g}\\
% &= \DirL(y, p) - \DirL(z, q) + \slashed g_{ij}^{,0}((y - z) p^i p^j + z(p^i p^j - q^i q^j))\sqrt{\slashed g} + \slashed g_{ij}^{,00}(\eta p^ip^j - \zeta q^iq^j)\sqrt{\slashed g}.
% \end{align*}
% Since $|\eta| + |\zeta| \lesssim y^2$ the claim follows.
% \end{proof}
%
% The operator $\Delta_{\slashed g}$ satisfies a mean value property:
%
% \begin{lemma}
% There exists a dimensional constant $C$ such that on $(H, \slashed g)$, a function $u$ is harmonic iff for every $(Q, r) \in H \times \RR_+$,
% $$u(Q) - \int_{\partial B(Q, r)} = C (\slashed R_{ij}(Q) u^{,ij}(Q) + 1.5\slashed R_{,i}(Q) u^{,i}(Q))r^4 + O(r^6).$$
% If $\slashed g$ is Einstein then the $r^4$ term vanishes.
% \end{lemma}
% \begin{proof}
% https://cuhkmath.wordpress.com/2015/08/14/mean-value-theorems-for-harmonic-functions-on-riemannian-manifolds/
% \end{proof}
%
% \begin{lemma}
% Let $N_n$ be a sequence of $C^1$ hypersurfaces in $M$ which are tangent to $H$ at $P$ and, in the coordinates induced by $H$, let $\omega_n: B_r \to I$ have graph $N_n$.
% Let $\Delta_{\slashed g} u_n = 0$ have trace $\omega_n|\partial B_r$.
% Let $Af(r)$ be the average of $f$ on $B_r$.
% Suppose that
% \begin{enumerate}
% \item $||\nabla \omega_n||_{L^\infty} \ll \beta_n$.
% \item
% $$\int_{B_\rho} \Lagrange(\omega_n, \nabla \omega_n) - \Lagrange(\omega_n, A\nabla \omega_n(\rho)) \leq \beta_n.$$
% \item
% $$\int_{B_{\alpha \rho}} \Lagrange(\omega_n, \nabla \omega_n) - \Lagrange(u_n, \nabla u_n) \ll \beta_j.$$
% \item $||\nabla \omega_n||_{L^\infty} \lesssim 1$.
% \end{enumerate}
% Then
% $$\limsup_{n \to \infty} \frac{1}{\beta_n}\left[O(\rho^{2 + d}) + \int_{B_{\rho/2}} \Lagrange(\omega_n, \nabla \omega_n) - \Lagrange(\omega_n, A\nabla \omega_n(\rho/2))\right] \leq 2^{1 - d}.$$
% \end{lemma}
% \begin{proof}
% Let
% $$2I_1 = \int_{B_{\rho/2}} \Lagrange(\omega_n, \nabla \omega_n) - \Lagrange(\omega_n, A\nabla \omega_n(\rho/2)),$$
% then
% \begin{align*}
% I_1 &\leq \int_{B_{\rho/2}} \DirQL(\omega_n, \nabla \omega_n) - \DirQL(\omega_n, A\nabla \omega_n(\rho/2))\\
% &= \int_{B_{\rho/2}} \slashed g(\tilde \omega_n)_{ij}(\omega_n^{,i} \omega_n^{,j} - A\omega_n^{,i}(\rho/2) A\omega_n^{,j}(\rho/2)) \sqrt{\slashed g}\\
% &= \int_{B_{\rho/2}} \slashed g(\tilde \omega_n)_{ij}((\omega_n^{,i} - A\omega_n^{,i}(\rho/2))(\omega_n^{,j} - A\omega_n^{,j}(\rho/2)))\sqrt{\slashed g}
% + 2 A\omega_n^{,j}\int_{B_{\rho/2}} \slashed g(\tilde \omega_j)_{ij}(\omega_n^{,i} - A\omega_n^{,i}(\rho/2)) \sqrt{\slashed g}\\
% &= I_2 + I_3.
% \end{align*}
%
% Using the Taylor expansion $|\tilde \omega_n| \lesssim |x|$ where $|x|$ denotes the distance from $x$ to $P$, Riemann's asymptotic formula reads
% \begin{equation}\label{Riemann asymptotics}
% |\slashed g(\tilde \omega_n)_{ij} - \delta_{ij}| \lesssim |x|^2.
% \end{equation}
% Because of this estimate, it will frequently be convenient to use the estimate
% \begin{equation}\label{integrated asymptotics}
% \int_{B_R} |x|^2 \lesssim R^{d + 1}
% \end{equation}
% which follows from the polar integration
% $$\int_{B_R} |x|^2 = \int_0^R r^2 |\partial B_r| ~dr \lesssim \int_0^R r^d ~dr = \frac{R^{d + 1}}{d + 1}.$$
%
% \begin{claim}
% One has
% $$I_2 \leq O(\rho^{3 + d}) + \sum_i ||\omega_n^{,i} - A\omega_n^{,i}||_{B_{\rho/2}}^2.$$
% \end{claim}
% \begin{proof}[Proof of claim]
% Using (\ref{Riemann asymptotics}),
% $$I_2 = \sum_i \left[||\omega_n^{,i} - A\omega_n^{,i}(\rho/2)||_{L^2(B_{\rho/2})}^2 - \int_{B_{\rho/2}} O(|x|^2) |\omega_n^{,i} - A\omega_n^{,i}(\rho/2)|^2 \sqrt{\slashed g} \right].$$
% The distance $||f - g||_{L^2}$, ranging over constants $g$, is minimized when $g$ is the mean of $f$.
% Therefore
% \begin{align*}
% \sum_i ||\omega_n^{,i} - A\omega_n^{,i}(\rho/2)||_{L^2(B_{\rho/2})}^2 &\leq \sum_i ||\omega_n^{,i} - A\omega_n^{,i}(\rho)||_{L^2(B_{\rho/2})}^2.
% \end{align*}
% In short, we have
% \begin{align*}
% I_2 &\leq \int_{B_{\rho/2}} \slashed g(\tilde \omega_n)_{ij}((\omega_n^{,i} - A\omega_n^{,i}(\rho))(\omega_n^{,j} - A\omega_n^{,j}(\rho)))\sqrt{\slashed g} + ||\nabla \omega_n||_{L^\infty}^2 \int_{B_{\rho/2}} O(|x|^2).
% \end{align*}
% But $||\nabla \omega_n||_{L^\infty}^2 \lesssim \rho^2$ so the claim follows from (\ref{integrated asymptotics}).
% \end{proof}
%
% \begin{claim}
% $I_3 \lesssim \rho^{2 + d}$.
% \end{claim}
% \begin{proof}[Proof of claim]
% Clearly
% $$\int_{B_{\rho/2}} \omega_n^{,j} - A\omega_n^{,j} \sqrt{\slashed g} = 0,$$
% so by (\ref{Riemann asymptotics}) and (\ref{integrated asymptotics}),
% \begin{align*}
% \left|\int_{B_{\rho/2}} \slashed g(\tilde \omega_n)_{ij}(\omega_n^{,i} - A\omega_n^{,i}(\rho/2)) \sqrt{\slashed g}\right| &\leq \int_{B_{\rho/2}} O(|x|^2)|\omega_n^{,i} - A\omega_n^{,i}(\rho/2)| \lesssim \rho^{d + 1}.
% \end{align*}
% We then apply the estimate
% \begin{align*}
% |A\omega_n^{,j}| &\leq ||\omega_n^{,j}||_{L^\infty} \lesssim \rho. \qedhere
% \end{align*}
% \end{proof}
%
% From the above two claims we conclude that
% $$I_1 \leq O(\rho^{2 + d}) + \sum_i ||\omega_n^{,i} - A\omega_n^{,i}||_{B_{\rho/2}}^2.$$
%
%
% \end{proof}

\section{de Giorgi lemma}
Let $N = \partial U$ be a $C^1$ hypersurface in $M$ and let $P \in N$.

\begin{assumption} \label{Assume Killing Field}
There exists a local Killing field $X$ and a local hypersurface $H$ such that:
\begin{enumerate}
\item For every $v \in T_PN$, $g(X(P), v) = 0$.
\item $|X(P)| = 1$.
\item $X$ is normal to $H$.
\end{enumerate}
\end{assumption}

\begin{example}
Suppose that $M$ is a surface of constant curvature.
Then we may satisfy Assumption \ref{Assume Killing Field}, as follows.
After passing to the universal cover and rescaling the Ricci scalar, we may assume that either $M = \RR^2$, $M = \Hyp^2$ in the upper half-plane model, or $M = \Sph^2$.

The assumption is obviously true if $M = \RR^2$.
If $M = \Hyp^2$, we may assume that $(P, v) = ((0, 1), (1, 0))$, in which case $X = \partial_1$ and $H = \{x = 0\}$.
If $M = \Sph^2$, we may assume that $P$ is a point on the equator and $v$ points north, in which case $X$ generates rotation about the $z$-axis and $H$ is the meridian through $P$.
\end{example}

\subsection{Taylor expansion of the area form}
We write $(x, y) = e^{yX}x$, which defines a coordinate system near $H$ such that the hypersurfaces $\{y = \text{const}\}$ are all isometric by the action of $X$.
We abuse notation and call the domain of this coordinate system $M$.
Then we can find a function $\omega: H \to \RR$ whose graph $\{y = \omega(x)\}$ is $H$, and we have a diffeomorphism $\varphi: H \to N$ by $\varphi(x) = (x, \omega(x))$.

\begin{definition}
The \dfn{lapse} is the scalar field on $H$, $\lambda = {g_{00}}^{(d - 1)/2}$, and the \dfn{leaf metric} is the metric on $H$, $h = g_{\hat 0 \hat 0}/g_{00}$.
\end{definition}

\begin{definition}
The \dfn{area Lagrangian} is the volume form on $H$ defined for $p$ tangent to $H$ by
$$\Lagrange(p) = \lambda \sqrt{1 + |p|_h^2} \vol_h,$$
where $\lambda,h$ are the lapse and leaf metric on $H$.
\end{definition}

\begin{lemma}
The natural area form on $N$ is given by
$$\vol_N = \varphi_* \Lagrange(\nabla \omega).$$
\end{lemma}
\begin{proof}
If we write $\slashed g$ for the metric on $N$, then
$$\varphi^* \vol_N = \varphi^* \sqrt{\det \slashed g} = \sqrt{\varphi^* \det \slashed g}.$$
To compute $\det \slashed g$ we put $\phi_i^\mu = (\varphi_* \partial_i)^\mu$, thus
$$\varphi^* \slashed g_{ij}(x) = g_{\mu\nu}(\varphi(x))\phi^\mu_i(x) \phi^\nu_j(x).$$
Since $X$ is a Killing field and $\varphi(x) = e^{\omega(x)X}x$,
$$g(\varphi(x)) = g(x) + \int_0^{\omega(x)} \mathcal L_X g(e^{tX}x) ~dt = g(x).$$
Moreover, $\phi_i^0 = \omega_{,i}$ while $\phi_i^j = \delta_i^j$, and $g_{0i}(x) = 0$ since $X$ is normal to $H$.
So
$$\varphi^* \slashed g_{ij} = g_{\mu\nu} \varphi_i^\mu \varphi_j^\nu = g_{00} \omega_{,i} \omega_{,j} + g_{ij} = (g_{00}d\omega \otimes d\omega + g_{\hat 0 \hat 0})_{ij}.$$
Now by the Weinstein-Aronszajn theorem,
\begin{align*}
\det(g_{00}d\omega \otimes d\omega + g_{\hat 0 \hat 0})
&= \lambda^2 \det(d\omega \otimes d\omega + h)\\
&= \lambda^2 (1 + h^{-1}(d \omega \otimes d\omega)) \det h \\
&= \lambda^2 (1 + |\nabla \omega|_h^2) \det h. \qedhere
\end{align*}
\end{proof}

We write
$$\DirL(p) = \frac{|p|_h^2}{2}\vol_h$$
for the Dirichlet energy of $p$ with respect to $h$.
We assume that there exists a constant $0 < c < 1$ such that
$$(1 + c)^{-1} \leq |\lambda|, |\lambda|^{-1} \leq 1 + c.$$
This assumption essentially has no content, since $\omega$ was only defined in a small ball around $P$ anyways.
We regard $c$ as an error term that we will hold on until the end, when we show that that we can pay for it by decreasing the speed of convergence of the approximation to the conormal.

We also abuse notation and write $f\vol_h$ as just $f$, as the only metric on $H$ that we will use from now on is the leaf metric $h$.

\begin{lemma}\label{Taylor lemma}
If $|q|_h \lesssim 1$ then $\Lagrange(p) - \Lagrange(q)$ lies in the closed interval
$$\left[\frac{(1 + c)^{-1}}{\Lagrange(q)}(\DirL(p) - \DirL(q)) - \frac{2(1 + c)^{-2}}{\Lagrange(q)}(\DirL(p) - \DirL(q))^2, (1 + c)(\DirL(p) - \DirL(q))\right].$$
\end{lemma}
\begin{proof}
By Taylor's theorem, there exists $\xi \geq 0$ between $|p|$ and $|q|$ such that
\begin{align*}
\lambda^{-1}(\Lagrange(p) - \Lagrange(q)) &= \sqrt{1 + |p|^2} - \sqrt{1 + |q|^2}\\
&= \frac{1}{2 \sqrt{1 + |q|^2}(|p|^2 - |q|^2)} - \frac{1}{8(1 + \xi^2)^{3/2}}(|p|^2 - |q|^2)^2 \\
&= \frac{\DirL(p) - \DirL(q)}{\sqrt{1 + |q|^2}} - \frac{(\DirL(p) - \DirL(q))^2}{2(1 + \xi^2)^{3/2}}.
\end{align*}
Since $|q|^2 \geq 0$ and the second term is nonpositive, it follows that
$$\Lagrange(p) - \Lagrange(q) \leq (1 + c)(\DirL(p) - \DirL(q)).$$
and, since $|q|^2 \leq 15$,
$$2\Lagrange(q) \leq 8 \leq 8(1 + \xi^2)^{3/2},$$
whence
\begin{align*}
\frac{(1 + c)^{-1}\DirL(p) - \DirL(q)}{\Lagrange(q)} - \frac{2(1 + c)^{-2}(\DirL(p) - \DirL(q))^2}{\Lagrange(q)} &\leq \Lagrange(p) - \Lagrange(q).\qedhere
\end{align*}
\end{proof}

\subsection{de Giorgi's lemma for $C^1$ functions}
Throughout this section whenever we refer to some intrinsic geometric property of $H$, we mean with respect to $h$.

\begin{assumption}\label{Assume Dimensions}
$d = 2$.
\end{assumption}

Therefore $H$ is a curve, which we parametrize with an arc length parameter $x$ centered on $P$.
Thus $h = 1$ and $\Delta_h = \partial^2$.
Let $B_r$ denote the interval centered on $P$ of radius $r$.
Let $A(r)f$ denote the mean of $f$ over $B_r$.
Fix $\rho > 0$, and parameters $\beta, \kappa > 0$.

\begin{lemma}
Assume that $||\nabla \omega||_{L^\infty(B_\rho)} \leq \kappa$,
$$\int_{B_\rho} \Lagrange(\nabla \omega) - \Lagrange(A(\rho)\nabla \omega) \leq \beta,$$
and $\omega$ approximately solves the minimal surface equation in the sense that
$$\int_{B_\rho} \Lagrange(\nabla \omega) \leq \eta(U, \rho) + \beta \kappa.$$
Then
$$\int_{B_{\rho/2}} \Lagrange(\nabla \omega) - \Lagrange(A(\rho/2)\nabla \omega) \leq \frac{(1 + c)^2}{2^{d + 1}}\beta + O(\beta \kappa^{1/2}).$$
\end{lemma}
\begin{proof}
Let $u$ be the harmonic function on $B_\rho$ with trace $\omega$.
By definition of $\eta(U, \rho)$,
$$\int_{B_\rho} \Lagrange(\nabla \omega) - \Lagrange(\nabla u) \leq \int_{B_\rho} \Lagrange(\nabla \omega) - \eta(N, \rho) \leq \beta\kappa.$$
By Lemma \ref{Taylor lemma},
\begin{align*}
\int_{B_{\rho/2}} \Lagrange(\nabla \omega) - \Lagrange(A(\rho/2)\nabla \omega) &\leq (1 + c)\int_{B_{\rho/2}} \DirL(\nabla \omega) - \DirL(A(\rho/2)\nabla \omega) \\
&= \frac{1 + c}{2} \int_{B_{\rho/2}} |\nabla \omega|^2 - |A(\rho/2)\nabla\omega|^2.
\end{align*}
Since $A(\rho/2)\nabla \omega$ is the mean of $\nabla \omega$, we have for every $\varepsilon > 0$
\begin{align*}
\int_{B_{\rho/2}} |\nabla \omega|^2 - |A(\rho/2)\nabla \omega|^2 &\leq \int_{B_{\rho/2}} (\nabla \omega - A(\rho)\nabla \omega)^2 ~dx \\
&\leq (1 + \varepsilon^{-1})\int_{B_\rho} |\nabla \omega - \nabla u|^2 ~dx\\
&\qquad + (1 + \varepsilon) \int_{B_{\rho/2}} |\nabla u - A(\rho)\nabla \omega|^2 ~dx\\
&=: O(\varepsilon^{-1})I + (1 + \varepsilon)J.
\end{align*}
To estimate $I$, we use the mean-value property to show
\begin{align*}
I &= \int_{B_\rho} |\nabla \omega - \nabla u|^2 = \int_{B_\rho} |\nabla \omega|^2 - |\nabla u|^2 \\
&\lesssim \int_{B_\rho} \int_{B_\rho} \Lagrange(\nabla \omega) - \Lagrange(\nabla u) \leq \beta \kappa.
\end{align*}
To estimate $J$, we bound
\begin{align*}
J &= \int_{B_\rho} |\nabla u|^2 - |A(\rho)\nabla \omega|^2 = \int_{B_\rho} |\nabla u|^2 - |A(\rho)\nabla u|^2 \\
&\leq \frac{1}{2^{d + 1}} \int_{B_\rho} |\nabla u|^2 - |A(\rho)\nabla u|^2 \leq \frac{1}{2^{d + 1}} \int_{B_\rho} |\nabla \omega|^2 - |A(\rho)\nabla \omega|^2 \\
&\leq \frac{1}{2^{d + 1}} \int_{B_\rho} |\nabla \omega - A(\rho)\nabla \omega|^2 ~dx \\
&\leq \frac{1 + \varepsilon}{2^{d + 1}} \int_{B_\rho} |\nabla \omega - A(\rho)\nabla \omega|^2  + O(\varepsilon^{-1})\int_{B_\rho} |\nabla \omega - \nabla u|^2\\
&=: \frac{1 + \varepsilon}{2^{d + 1}}K + O(\varepsilon^{-1})I.
\end{align*}
We already estimated $I \lesssim \beta \kappa$, and now we estimate $K$: by Lemma \ref{Taylor lemma},
\begin{align*}
K &= \int_{B_\rho} |\nabla \omega|^2 - |A(\rho)\nabla \omega|^2\\
&\leq 2\int_{B_\rho} \DirL(\nabla \omega) - \DirL(A(\rho)\nabla \omega)\\
&\leq 2(1 + c)\int_{B_\rho} \Lagrange(\nabla \omega) - \Lagrange(A(\rho)\nabla\omega) + O(1) \int_{B_\rho} (\DirL(\nabla \omega) - \DirL(A(\rho)\nabla \omega))^2\\
&\leq 2(1 + c)\beta + O(1) ||\nabla \omega||_{L^\infty(B_\rho)} \int_{B_\rho} \DirL(\nabla \omega) - \DirL(A(\rho)\nabla \omega) \\
&\leq 2(1 + c + O(\kappa))\beta.
\end{align*}
If we set $\varepsilon = \kappa^{1/2}$ then we conclude
\begin{align*}
\int_{B_{\rho/2}} \Lagrange(\nabla \omega) - \Lagrange(A(\rho/2)\nabla \omega)
&\leq \frac{(1 + c)^2(1 + \varepsilon)^2}{2^{d + 1}}\beta + O(\beta \kappa \varepsilon^{-1})\\
&\leq \frac{(1 + c)^2}{2^{d + 1}}\beta + O(\beta \kappa^{1/2}). \qedhere
\end{align*}
\end{proof}

Let
$$\Lambda(U, \rho) = \int_{B_\rho} |du| - \left|\int_{B_\rho} du \right|$$
where $u = 1_U$.

\begin{lemma}
Suppose that $\Lambda(U, \rho) \leq \beta$, $\normal_0 \geq 1 - \kappa$, and $\eta(U, \rho) \leq \beta \kappa$.
Then
$$\Lambda(U, \rho/2) \leq \frac{(1 + c)^2}{2^{d + 1}} \beta + O(\beta \kappa^{1/2}).$$
\end{lemma}

\subsection{Mollification}
The same results as in the previous section hold without $N$ a $C^1$ hypersurface.
In particular:

\begin{proposition}\label{de Giorgi}
For every $c > 0$ there exists $\sigma > 0$ and $r > 0$ such that for every set $U$ of least perimeter, if
$\rho < r$ and
$$\Lambda(U, \rho) < \sigma \rho^{d - 1},$$
then
$$\Lambda(U, \rho/2) < \frac{(1 + c)^2}{2^d} \Lambda(U, \rho).$$
\end{proposition}

We now settle the choice of $c = c(d)$: we select $c$ so that that
$$\frac{(1 + c)^2}{2^d} < \lambda^{-d}$$
for some $\lambda \in (0, 1)$ such that
$$\frac{d - 1}{d} < \log_2 \lambda < 1.$$
By induction on Proposition \ref{de Giorgi}, there exists $\tau \in (0, 1)$ such that
\begin{equation}\label{inductive de Giorgi}
\Lambda(U, 2^{-n}) \lesssim 2^{-n(d - \tau)}.
\end{equation}

\subsection{Continuity of the conormal}
Let
$$\normal_s(P) = \frac{\int_{B_s(P)} du}{\int_{B_s(P)} |du|}.$$
Then $\normal_s \to \normal$ almost everywhere and $\normal_s$ is obviously continuous.
We want to get uniform convergence, thus we use the following lemma:

\begin{lemma}
Suppose that $0 < s < t < \rho$ are dyadic,
$$\Lambda(U, \rho) < \sigma \rho^{d - 1},$$
and $P \in \partial U$.
Then there exists $\delta > 0$ such that
$$|\normal_s(P) - \normal_t(P)| \lesssim t^\delta.$$
\end{lemma}
\begin{proof}
Write $s = 2^{-m}$, $t = 2^{-n}$ (so $m > n$), and set $v_k = \normal_{2^{-k}}(P)$, $M_k = \int_{B_{2^{-k}}} |du|$.
Then
$$|v_n - v_m|^2 \lesssim 1 - (v_n, v_m).$$
Gistui--Miranda show that
$$(1 - (v_n, v_m))M_m \leq 2M_n(1 - |v_n|) = 2\Lambda(U, t).$$
By (\ref{inductive de Giorgi}),
$$(1 - (v_n, v_m))M_m \lesssim 2^{-n(d - \tau)}$$
and hence, since $M_m \gtrsim 2^{-n(d - 1)}$, we conclude
\begin{align*}
|v_n - v_m|^2 &\lesssim 2^{-n(1 - \tau)} = t^{1 - \tau}. \qedhere
\end{align*}
\end{proof}

It follows from the above and the main theorem of Miranda66 that if
$$\Lambda(U, \rho) < \sigma \rho^{d - 1},$$
then $\normal$ is continuous in a neighborhood of $P$, and therefore $N$ is analytic.

\subsection{Removing the assumptions}
The gain of Assumption \ref{Assume Dimensions} is essentially that it forces $H$ to be euclidean, whence the mean-value property of $\Delta$ is especially well-behaved:

\begin{enumerate}
\item The mean-value property is actually true, not just asymptotically true.
\item The mean of the vector field $\nabla \omega$ is honestly well-defined, since the holonomy bundle of $H$ vanishes.
\item The commutator $[\Delta, \partial_j]$ vanishes if we parametrize $H$ by arc length. Therefore derivatives satisfy the mean-value property.
\end{enumerate}

My idea to remove the above assumption is to use Assumption \ref{Assume Killing Field} to observe that the geometry of $M$ is highly constrained.
Moreover, the extrinsic geometry of $H$, is constrained by the fact that $H$ is normal to a Killing field.
Therefore by Gauss-Codazzi, the intrinsic geometry of $(H, g_{\hat 0 \hat 0})$ is highly constrained as well.
Since $h = g_{\hat 0 \hat 0}/g_{00}$ and we already treat $g_{00} - 1$ as negligible, it should follow that the intrinsic geometry of $(H, h)$ should also be highly constrained.
In particular $\Delta$ should be well-behaved in some sense which gives us that $[\Delta, \partial_j]$ is approximately zero, the mean-value theorem is true to high order, and instead of working with $A\nabla \omega$, we can just work with the individual scalar fields $A\partial_j \omega$ (since $[\Delta, \partial_j] = 0$ anyways).

Assumption \ref{Assume Killing Field} is used to construct $H$, which is the key idea in the proof.
Hopefully, if we remove Assumption \ref{Assume Dimensions}, we can still show that Assumption \ref{Assume Killing Field} follows from the assumption that $M$ has constant curvature.
Now, by adding a bootstrap assumption to the de Giorgi lemma, perhaps we can show that the conclusion of the de Giorgi lemma is stable under perturbations of the derivative of the Riemann tensor of $M$.
Thus we can weaken Assumption \ref{Assume Killing Field} to the assumption that $|R_{ijkl;m}|$ is small.
But $|R_{ijkl;m}| \ll 1$ can be locally imposed by rescaling the metric and so it followws that Assumption \ref{Assume Killing Field} can be removed.


%%%%%%%%%%%%%%%%%%%%%%%%%%%%%%%%%%%%%%%%%%%%%%%%%%%%%

\section{Proofs of main theorems}\label{proof of main thm}
We are finally ready to prove Theorems \ref{main thm} and \ref{main crly}.

Throughout this section, let $u$ be a function of least gradient.
By Corollary \ref{level sets are minimal}, the superlevel sets of $u$ have least perimeter.

\subsection{Regularity of minimal hypersurfaces}
Let $U$ be a superlevel set of $u$.
We want to show that $N = \partial U$ is as smooth as possible, and to this end, we might as well assume that $u = 1_U$.

We first may cover $M$ by normal coordinate charts $A_x$ centered on $x \in M$, selected so that $A_x$ is precompact in $M$.
In particular, if we fix a particular $x$, then the averaged $1$-forms $\int_V du ~\vol$ are well-defined for $V \subseteq A_x$, so the excesses $\Lambda(U, V)$ are well-defined.
We write $\Lambda_x$ for the excess as computed in $A_x$, and write $N^* = \partial^* U$.
Since $A_x$ is a precompact chart, we can select $\sigma_x$ to be the constant that appears in Proposition \ref{DGL}.

\begin{lemma}
For every $x \in N^*$ and sufficiently small $\rho > 0$ such that $B(x, \rho) \subseteq A_x$, there is an open set $x \in A_x' \subseteq A_x$ and $\delta > 0$ such that for every $t \in (0, \delta)$, $s \in (0, \rho)$, and $y \in A_x'$ such that $B(y, s) \subseteq A_x$,
\begin{equation}\label{basecase}\Lambda_x(U_t, B(y, s)) < \sigma_x s^d.\end{equation}
\end{lemma}
\begin{proof}
By Proposition \ref{blowup theorem}, there exists a half-space $C_x$ obtained as the blowup of $\exp_x^* U$\footnote{This is the only point in the argument when we use the fact that $d \leq 7$!}.
Since $C_x$ is a half-space and the coordinate chart $A_x$ is normal, $\Lambda_x((\exp_x)_* C_x, V) = 0$ whenever $V \subseteq A_x$.
In particular, if $V$ has no singularities with respect to the blowup sequence $(u_t)$,
$$\lim_{t \to 0} \Lambda_x(U_t, V) = \Lambda_x((\exp_x)_* C_x, V) = 0.$$
Here $U_t = (\exp_x)_* \{u_t = 1\}$ is the blowup of $U$.
So for every $\rho$ such that $B(x, \rho) \subseteq V$, there exists $\delta > 0$ such that if $t < \delta$,
$$\Lambda_x(U_t, B(x, \rho)) < \frac{\sigma_x}{2} \rho^d.$$
By continuity of measure, it follows that for every $y \in A_x$ close enough to $x$, and $0 < s < \rho$ such that $B(y, s) \subseteq A_x$, the claim holds.
\end{proof}

Fix $x \in N^*$, $t < \delta$, and let
$$\normal_s(y) = \frac{\int_{B(y, s)} du_t ~\vol}{\int_{B(y, s)} |du_t| ~\vol},$$
whenever $y \in A_x'$ and $B(y, s) \subseteq A_x$.
Then $\normal_s$ is continuous, since
$$\left|\int_{B(y_1, s)} du_t ~\vol - \int_{B(y_2, s)} du_t ~\vol\right| \leq \int_{B(y_1, s) \Delta B(y_2, s)} |du_t| ~\vol$$
where $\Delta$ denotes symmetric difference; the right-hand side vanishes as $y_2 \to y_1$ by continuity of measure.
We now show that $(\normal_s)$ is uniformly Cauchy as $s \to 0$.

\begin{lemma}
One has
$$|\normal_s(y) - \normal_r(y)| \lesssim_x \sqrt s$$
uniformly in $y \in A_x'$, whenever $0 < r < s < \rho$ and $B(y, s) \subseteq A_x$.
\end{lemma}
\begin{proof}
After throwing away a constant factor we may assume that $s = \rho/2^k$ for some $k$, and $r = \beta s$ for some $\beta \in (0, 1)$.
Since $|\normal_s(y)|,|\normal_r(y)| \leq 1$,
$$|\normal_s(y) - \normal_r(y)| \lesssim \sqrt{1 - (\normal_s(y), \normal_r(y))}.$$
Then (TODO: This seems dubious, check it)
\begin{align*}
(1 - (\normal_s(y), \normal_r(y)))  &\leq \frac{1}{|N^* \cap B(y, r)|} \int_{B(y, \beta \rho/2^k)} |du_t| - \left(\normal_s(y), \frac{du_t}{|du_t|}\right) |du_t| ~\vol\\
&\leq \frac{1}{|N^* \cap B(y, r)|} \int_{B(y, \rho/2^k)} |du_t| - \frac{(\normal_s(y), du_t)}{|du_t|} |du_t| ~\vol \\
&\leq \frac{1}{|N^* \cap B(y, r)|} \int_{B(y, \rho/2^k)} |du_t| - |\normal_s(y)|^2 |du_t| ~\vol\\
&\leq \frac{2}{|N^* \cap B(y, r)|} (1 - |\normal_s(y)|)\int_{B(y, \rho/2^k)} |du_t| ~\vol\\
&= 2\frac{\Lambda_x(U, B(y, s))}{|N^* \cap B(y, r)|}.
\end{align*}
By inducting on Proposition \ref{DGL} in $k$, the base case (\ref{basecase}), and Proposition \ref{uniform density estimate},
\begin{align*}
\frac{\Lambda_x(U, B(y, s))}{|N^* \cap B(y, r)|} < 2^{-kd} \frac{\Lambda_x(U, B(y, \rho))}{r^{d - 1}} \lesssim s
\end{align*}
and therefore
\begin{align*}
|\normal_s(y) - \normal_r(y)| &\lesssim \sqrt{\frac{\Lambda_x(U, B(y, s))}{|N^* \cap B(y, r)|}} \lesssim \sqrt s. \qedhere
\end{align*}
\end{proof}

By the above lemma, $(\normal_s)$ is uniformly Cauchy on $A_x'$ as $s \to 0$.
But from the definition of $\normal(y)$, and the fact that $U_t$ is just a rescaling of $U$, if $(\normal_s(y))$ converges to anything as $s \to 0$, it must converge to $\normal$.
Since $\normal_s$ is continuous, it follows that $\normal$ extends to a continuous $1$-form on $A_x' \cap N$.
By Proposition \ref{locality of Caccioppoli}, $(A_x')_{x \in N^*}$ defines an open cover of $N$, so $\normal$ extends to a continuous $1$-form on $N$, and hence by Proposition \ref{regularity of reduced boundary}, $N$ is as smooth as possible.

\subsection{Existence of laminations}
Now we consider the general case when $u$ is a function of least gradient.
Let
\begin{equation}\label{lamination union}
A = \bigcup_y \partial \{u > y\},
\end{equation} $B$ the interior of $\{du = 0\}$, and $x \in M$.
Then $x \in B$ iff $u = u(x)$ near $x$, but that happens iff for every $y < u(x)$, $x$ is interior to $\{u > y\}$ and for every $y \geq u(x)$, $x$ is exterior to $\{u > y\}$.
This happens iff for every $y \in \RR$, $x$ is either interior or exterior to $\{u > y\}$, thus $x \notin \partial \{u > y\}$, which happens iff $x \notin A$.
Thus $\{A, B\}$ is a partition of $M$, so $A$ is closed.
Moreover, the sets $\{u > y\}$ are totally ordered by $\subseteq$, so the sets $\partial \{u > y\}$ are disjoint.
They are also hypersurfaces which are as smooth as possible, by the previous section.
This proves Theorem \ref{main thm}.

\subsection{Convex surfaces with boundary}
The proof of Theorem \ref{main crly} is essentially identical to that of \cite[Proposition 3.4]{górny2017planar}; we give the details here for completeness.

Suppose that $M = \Sigma \subset \overline \Sigma$, and that Theorem \ref{main crly} is false for $u$.
That is, we cannot extend the geodesic lamination that we constructed above to a lamination of $\overline \Sigma$.
Therefore there exist disjoint geodesics $\gamma_1$ and $\gamma_2$ which intersect on $\partial \Sigma$ and bound superlevel sets $\{u > y_i\}$ of $u$.

Suppose that $\gamma_1$ and $\gamma_2$ intersect at $x_0$, and $\gamma_i$ passes through $x_i$ on the way to $x_0$, so that $x_0, x_1, x_2$ bound an open, nondegenerate geodesic triangle $\Delta \subset \overline \Sigma$. This makes sense, because $\overline \Sigma$ is convex.
Since we have access to the monotonicity formula (\ref{classic monotonicity formula}), the proof of \cite[Remark 37.9]{simon1983GMT} shows that there exist only finitely many connected components of $A$ in $\Delta$.
So, after replacing $\gamma_2$ with a geodesic closer to $\gamma_1$ as necessary, we may assume that either $A$ does not intersect $\Delta$, or $A$ contains $\Delta$.
By replacing $A$ with its complement if necessary, we may assume that $A$ does not meet $\Delta$.

However, $v = 1_{u^{-1}((y_1, y_2))}$ is a function of least gradient, and $v = 1$ on $\Delta$ but $v = 0$ on the opposite sides of $\gamma_i$.
So if we replace $v$ with $w = v - 1_\Delta$, $w$ has the same trace as $v$, but since $\Delta$ is a nondegenerate triangle,
$$\int_U |dw| ~\vol = |\partial(\{u > y\} \setminus \Delta) \cap U| < |\partial \{u > y\} \cap U| = \int_U |dv| ~\vol$$
whenever $U$ is a precompact neighborhood of $\overline \Delta$ in $\overline \Sigma$.
Therefore $v$ does not have least gradient, which is a contradiction.


%%%%%%%%%%%%%%%%%%%%%%%%%%%%%%%%%%%%%%%%%%%%%%%%%%%%%%%%%%%%%%%%%%%%



\printbibliography


\end{document}
