\documentclass[reqno,12pt,letterpaper]{amsart}
\RequirePackage{amsmath,amssymb,amsthm,graphicx,mathrsfs,url,slashed}
\RequirePackage[usenames,dvipsnames]{xcolor}
\RequirePackage[colorlinks=true,linkcolor=Red,citecolor=Green]{hyperref}
\RequirePackage{amsxtra}
\usepackage{cancel}
\usepackage{tikz-cd}

\setlength{\textheight}{9.3in} \setlength{\oddsidemargin}{-0.25in}
\setlength{\evensidemargin}{-0.25in} \setlength{\textwidth}{7in}
\setlength{\topmargin}{-0.25in} \setlength{\headheight}{0.18in}
\setlength{\marginparwidth}{1.0in}
\setlength{\abovedisplayskip}{0.2in}
\setlength{\belowdisplayskip}{0.2in}
\setlength{\parskip}{0.05in}
\renewcommand{\baselinestretch}{1.05}

\title[Hyperbolic least-gradient maximum principle]{The least-gradient maximum principle on hyperbolic manifolds}
\author{Aidan Backus}
\date{May 2022}

\newcommand{\NN}{\mathbf{N}}
\newcommand{\ZZ}{\mathbf{Z}}
\newcommand{\QQ}{\mathbf{Q}}
\newcommand{\RR}{\mathbf{R}}
\newcommand{\CC}{\mathbf{C}}
\newcommand{\DD}{\mathbf{D}}
\newcommand{\PP}{\mathbf P}
\newcommand{\MM}{\mathbf M}
\newcommand{\II}{\mathbf I}
\newcommand{\Hyp}{\mathbf H}
\newcommand{\Sph}{\mathbf S}
\newcommand{\GL}{\mathbf{GL}}
\newcommand{\Orth}{\mathbf{O}}

\DeclareMathOperator{\card}{card}
\DeclareMathOperator{\cent}{center}
\DeclareMathOperator{\ch}{ch}
\DeclareMathOperator{\codim}{codim}
\DeclareMathOperator{\diag}{diag}
\DeclareMathOperator{\diam}{diam}
\DeclareMathOperator{\dom}{dom}
\DeclareMathOperator{\Exc}{Exc}
\DeclareMathOperator{\Gal}{Gal}
\DeclareMathOperator{\Hom}{Hom}
\DeclareMathOperator{\Iso}{Iso}
\DeclareMathOperator{\Jac}{Jac}
\DeclareMathOperator{\Lip}{Lip}
\DeclareMathOperator{\Met}{Met}
\DeclareMathOperator{\id}{id}
\DeclareMathOperator{\rad}{rad}
\DeclareMathOperator{\rank}{rank}
\DeclareMathOperator{\Hess}{Hess}
\DeclareMathOperator{\Radon}{Radon}
\DeclareMathOperator*{\Res}{Res}
\DeclareMathOperator{\sgn}{sgn}
\DeclareMathOperator{\singsupp}{sing~supp}
\DeclareMathOperator{\Spec}{Spec}
\DeclareMathOperator{\supp}{supp}
\DeclareMathOperator{\Tan}{Tan}
\newcommand{\tr}{\operatorname{tr}}

\newcommand{\Ric}{\mathrm{Ric}}
\newcommand{\Riem}{\mathrm{Riem}}
\newcommand{\LapQL}{\Delta^{\mathrm{ql}}}

\newcommand{\dbar}{\overline \partial}

\DeclareMathOperator{\atanh}{atanh}
\DeclareMathOperator{\csch}{csch}
\DeclareMathOperator{\sech}{sech}

\DeclareMathOperator{\Div}{div}
\DeclareMathOperator{\Gram}{Gram}
\DeclareMathOperator{\grad}{grad}
\DeclareMathOperator{\Ell}{Ell}
\DeclareMathOperator{\WF}{WF}

\newcommand{\Lagrange}{\mathscr L}
\newcommand{\DirQL}{\mathscr D^{\mathrm{ql}}}
\newcommand{\DirL}{\mathscr D}

\newcommand{\Hilb}{\mathcal H}
\newcommand{\Homology}{\mathrm H}
\newcommand{\normal}{\mathbf n}
\newcommand{\evect}{\mathbf e}
\newcommand{\vol}{\mathrm{vol}}

\newcommand{\pic}{\vspace{30mm}}
\newcommand{\dfn}[1]{\emph{#1}\index{#1}}

\renewcommand{\Re}{\operatorname{Re}}
\renewcommand{\Im}{\operatorname{Im}}

\newcommand{\loc}{\mathrm{loc}}
\newcommand{\cpt}{\mathrm{cpt}}

\def\Japan#1{\left \langle #1 \right \rangle}

\newtheorem{theorem}{Theorem}[section]
\newtheorem{badtheorem}[theorem]{``Theorem"}
\newtheorem{prop}[theorem]{Proposition}
\newtheorem{lemma}[theorem]{Lemma}
\newtheorem{claim}[theorem]{Claim}
\newtheorem{proposition}[theorem]{Proposition}
\newtheorem{corollary}[theorem]{Corollary}
\newtheorem{conjecture}[theorem]{Conjecture}
\newtheorem{axiom}[theorem]{Axiom}
\newtheorem{assumption}[theorem]{Assumption}

\theoremstyle{definition}
\newtheorem{definition}[theorem]{Definition}
\newtheorem{remark}[theorem]{Remark}
\newtheorem{example}[theorem]{Example}
\newtheorem{notation}[theorem]{Notation}

\newtheorem{exercise}[theorem]{Discussion topic}
\newtheorem{homework}[theorem]{Homework}
\newtheorem{problem}[theorem]{Problem}

\newtheorem{ack}{Acknowledgements}

\numberwithin{equation}{section}


% Mean
\def\Xint#1{\mathchoice
{\XXint\displaystyle\textstyle{#1}}%
{\XXint\textstyle\scriptstyle{#1}}%
{\XXint\scriptstyle\scriptscriptstyle{#1}}%
{\XXint\scriptscriptstyle\scriptscriptstyle{#1}}%
\!\int}
\def\XXint#1#2#3{{\setbox0=\hbox{$#1{#2#3}{\int}$ }
\vcenter{\hbox{$#2#3$ }}\kern-.6\wd0}}
\def\ddashint{\Xint=}
\def\dashint{\Xint-}

\usepackage[backend=bibtex,style=alphabetic]{biblatex}
\addbibresource{topics.bib}
\renewbibmacro{in:}{}
\DeclareFieldFormat{pages}{#1}


\begin{document}
\begin{abstract}
The least-gradient maximum principle, essentially due to Miranda and de Giorgi in the 1960s, shows that least-gradient functions define a minimal lamination of the support of their derivative.
We extend this result to least-gradient functions on hyperbolic manifolds and propose means for a proof on general Riemannian manifold.
\end{abstract}

\maketitle

%%%%%%%%%%%%%%%%%%%%%%%%%%%%%%%%%%%%%%%%%%%%%%%%%%%%%%%

% \tableofcontents

\let\clearpage\relax
\section{Introduction}
Let $M$ be an oriented Riemannian manifold of metric $g$ and dimension $d$.

\begin{definition}\label{main definitions}
A function $u$ of locally bounded variation has \dfn{least gradient} if for every compactly supported function $v$ of bounded variation, such that $\supp v \subseteq U \Subset M$,
$$\int_U |du| ~\vol \leq \int_U |du + dv| ~\vol.$$
A set $U$ of locally finite perimeter has \dfn{least perimeter} if $1_U$ has least gradient.
\end{definition}

\begin{definition}
A \dfn{minimal lamination} in $M$ is a partition of a closed subset of $M$ into smooth hypersurfaces with zero mean curvature.
The minimal lamination $\lambda$ is \dfn{analytic} if $(M, g)$ is analytic and each of the hypersurfaces in $\lambda$ is analytic.
\end{definition}

This paper is dedicated to the proof of the following generalization of the maximum principle for least gradient functions on euclidean space \cite[Proposition 3.4]{górny2017planar}.

\begin{theorem}[maximum principle]\label{main thm}
Suppose that $2 \leq d \leq 7$ and
\begin{enumerate}
\item either $g$ has constant sectional curvature $\leq 0$,
\item or $g$ has constant curvature and $d = 2$.
\end{enumerate}
Let $u: M \to \RR$ be a function of least gradient, and $A_y = \partial \{u > y\}$.
Then $(A_y)_{y \in \RR}$ is a minimal lamination, which is analytic if $(M, g)$ is.
\end{theorem}

We are primarily concerned with the case that $M$ is a closed hyperbolic manifold, which is the setting of several erstwhile open problems stated in \cite[\S9]{daskalopoulos2020transverse} which Theorem \ref{main thm} can be used to solve. See \S\ref{BestDuality}.

If $d = 2$, then Theorem \ref{main thm} can be extended to manifolds with boundary, just as in \cite[Corollary 3.5]{górny2017planar}:

\begin{theorem}[maximum principle up to the boundary]\label{main crly}
Let $\overline \Sigma$ be a convex surface with boundary and suppose that $u: \Sigma \to \RR$ is a function of least gradient defined on the interior $\Sigma$ of $\overline \Sigma$.
Then, if $A_y = \partial \{u > y\}$, $(A_y)_{y \in \RR}$ extends to a geodesic lamination of $\overline \Sigma$.
\end{theorem}

Theorems \ref{main thm} and \ref{main crly} can be easily shown to follow from standard results and the following regularity theorem for sets of least perimeter, which is the main theorem of the present paper:

\begin{theorem}[regularity of minimal hypersurfaces]\label{main lma}
Suppose that $2 \leq d \leq 7$ and
\begin{enumerate}
\item either $g$ has constant sectional curvature $\leq 0$,
\item or $g$ has constant curvature and $d = 2$.
\end{enumerate}
Then every set of least perimeter is bounded by a smooth minimal hypersurface $N$.
Furthermore, $N$ is analytic if $(M, g)$ is.
\end{theorem}

A proof of an analogous result to Theorem \ref{main lma} for currents paired against an elliptic integrand is given by \cite[\S5.3]{federer2014geometric}; our proof uses a similar strategy but is rather ``hands-on" in that it avoids the use of homological integration, instead using the theory of functions of bounded variation as developed by Miranda in \cite{Miranda64} \cite{Miranda66} \cite{Miranda67}, and facts about minimal hypersurfaces in euclidean space.

Though the hypotheses of Theorem \ref{main lma} look odd, we only use the assumption of nonpositive curvature at a critical point to deduce that a certain elliptic operator which we construct is actually the euclidean Laplace operator when written in correctly chosen coordinates.
In general, such an operator would be a perturbation of the euclidean Laplacian and so a suitable analysis of perturbations of elliptic operators should extend all of our results to the case of constant sectional curvature, so it is reasonable to conjecture the following:

\begin{conjecture}\label{main conj}
Suppose that $2 \leq d \leq 7$ and $g$ has constant sectional curvature.
Then the level sets of a function of least gradient are a minimal lamination.
\end{conjecture}

The assumption of constant sectional curvature should be somewhat harder to remove, as we rely on the existence of certain Killing fields, and at one point use the Killing-Hopf theorem to reduce a general result to the euclidean case.
However, if one could show Conjecture \ref{main conj}, a proof for any Riemannian manifold of appropriate dimension should not be out of reach.

%%%%%%%%%%%%%%%%%%%%%%%%%%%%%%%%%%%%%%%%%%%%%%%

\subsection{Outline of the paper}
We begin with the preliminaries.
In \S\ref{RiemMeasureThy}, we record facts that we will use about sets of locally finite perimeter on Riemannian manifolds.
We introduce the notion of a \dfn{bundle-valued Radon measure}, which is a generalized section of a vector bundle that, upon trivialization, becomes a vector-valued Radon measure.
With the basic theory of bundle-valued Radon measures it is easy to show that the reduced boundary of a set of least perimeter is independent of the choice of metric.
We also show that a coarea formula holds in this setting, which we use in \S\ref{LeastGradientFunctions} to develop the theory of functions of least gradient on Riemannian manifolds. A generalization of Miranda's theorem \cite[Teorema 3]{Miranda67} on the stability of functions of least gradient, and generalizations of its standard consequences, easily fall out from the coarea formula.
We then focus on sets of least perimeter: we prove a monotonicity formula, compute the dimension of the reduced boundary as $d - 1$, and prove the existence of tangent spaces to the reduced boundary.

We are then ready to prove Theorem \ref{main lma}.
In \S\ref{MollifierSection}, we show Proposition \ref{mollifier proposition}, which says that a set of least perimeter can be approximated by $C^1$ hypersurfaces that are approximately minimal.
This result is then used in \S\ref{DeGiorgiSection}, to prove a generalization, Proposition \ref{de Giorgi}, of the de Giorgi regularity lemma \cite[Teorema 5.7]{Miranda66} for sets of least perimeter.
Most of the difficulty in this step, and indeed in this present paper, comes from proving the $C^1$ case of de Giorgi's lemma, as Proposition \ref{mollifier proposition} can be used to reduce Proposition \ref{de Giorgi} to this case.

In the euclidean case, the $C^1$ de Giorgi lemma is proven by comparing the surface area of the graph of a $C^1$ function $\omega$ to its Dirichlet energy $\int |d\omega|^2~\vol$, and then apply the mean-value property and the fact that $[\Delta, \partial_j] = 0$.
This does not quite work in general, as the commutator of the Laplace-Beltrami operator with a coordinate vector field will in general be nonzero, and the mean-value theorem only approximately holds for general Laplace-Beltrami operators.
Our main innovation here is to show that for a suitable space form, one has enough Killing fields to carry out a similar argument, in which the Dirichlet energy is actually the Lagrangian for the euclidean Laplacian.

In \S\ref{proof of main thm}, we use a standard inductive argument to show that Proposition \ref{de Giorgi} implies Theorem \ref{main lma}.
We then use Theorem \ref{main lma} and the general theory developed in \S\ref{LeastGradientFunctions} to prove Theorems \ref{main thm} and \ref{main crly}.

In \S\ref{BestDuality}, we use Theorem \ref{main thm} to use certain conjectures of Daskalopoulos and Uhlenbeck \cite{daskalopoulos2020transverse} motivated by Thurston's asymmetric metric \cite{thurston1998minimal}.

%%%%%%%%%%%%%%%%%%%%%%%%%%%%%%%%%%%%%%%%%%%%%%%%

\subsection{Acknowledgements}
I would like to thank Georgios Daskalopoulos for suggesting this project and for many helpful discussions.
I would also like to thank Joshua Lin for helpful discussions and in particular for suggesting the proof of Lemma \ref{cross product formula}.

\section{Riemannian measure theory}\label{RiemMeasureThy}
\subsection{Notation and conventions}
\begin{notation}[presheaves]
If $F$ is a presheaf of function spaces, we write $u \in F_l(U)$ to mean that for every $V \Subset U$, $u \in F(V)$.
We write $u \in F_c(U)$ to mean that $u \in F(U)$ and $\supp u \Subset U$.
\end{notation}

\begin{notation}[volume forms]
We reserve the letter $d$ for dimension or exterior differentiation, and so to avoid awkwardness such as $\int |du| dV$ we write $\vol$ for the Riemannian volume form.
If $N$ is a closed submanifold we write $\vol_N$ to indicate the pullback of $\vol$ along the inclusion map $M \to N$.
\end{notation}

\begin{notation}[vector bundles]
Let $E$ be a vector bundle, which we will always assume is normed, with dual $E'$.
If $u,v$ are sections of $E',E$ respectively, we write $(u, v)$ for their fiberwise pairing, which is a function $M \to \RR$.
We write $\langle u, v\rangle$ or $\int_M (u, v) ~\vol$, for their $L^2$-duality pairing, which is a real number.
If $u$ is a section of $E$, we write $u \prec U$ to mean that $||u||_{L^\infty} \lesssim 1$ and $\supp u \Subset U$.
\end{notation}

\begin{definition}
Let $u$ be a smooth section.
Then $u$ is \dfn{as smooth as possible} if either $u$ is analytic or $g$ is not analytic.
\end{definition}

%%%%%%%%%%%%%%%%%%%%%%%%%%%%%%%%%%%%%%%%%%%%%%%%%%%%%%%%%%%%

\subsection{Bundle-valued Radon measures}
Let $F \to M$ be a normed vector bundle of rank $r$.
We equip the space $C(K, F)$ of continuous sections of $F$ on a compact set $K$ with its supremum norm.
The Banach spaces $C(K, F)$ form an inverse system with respect to restriction and therefore define the topological vector space
$$C_0(M, F) = \varprojlim C(K, F).$$

According to the Riesz-Markov theorem, the space of $\CC^r$-valued Radon measures on $M$ is canonically identified with $C_0(M, (\CC^r)')'$, thus we define:

\begin{definition}
The topological dual space $\mathcal R(M, F) = C_0(M, F')'$ is the space of \dfn{$F$-valued Radon measures} on $M$.
\end{definition}

If we write $\mathcal R(U, F)$ to mean $\mathcal R(U, F|U)$, we equip $\mathcal R(U, F)$ with the weak topology of measures.
Then $\mathcal R(U, F)$ is a separable Fr\'echet space, with seminorms $|\langle \cdot, f_j\rangle|$ where $(f_j)$ is a countable basis for a dense subspace of $C_0(U, F)$.
Thus $\mathcal R(\cdot, F)$ is a sheaf of separable Fr\'echet spaces.

\begin{definition}
If $\omega \in \mathcal R(M, F)$ we define the \dfn{total variation} $|\omega|$ to be the positive Radon measure such that on every open set $U$,
$$|\omega|(U) = \sup_{X \prec U} \langle \omega, X\rangle,$$
where $X$ ranges over $C_0(M, F')$.
\end{definition}

Since $\RR_+$ acts on $F$ by scalar multiplication, we can define the \dfn{sphere bundle} $SF = (F \setminus 0)/\RR_+$.
Since $F$ has a norm, $SF$ is naturally identified with the bundle of elements of $F$ with length $1$.

\begin{proposition}[Riesz-Markov representation]\label{HanhJordan}
Let $\omega$ be a $F$-valued Radon measure and let $\mu = |\omega|$.
Then there exists a $\mu$-measurable section $f$ of $SF$ such that for every section $X \in C_0(M, F', \mu)$,
\begin{equation}\label{RNy formula}
\langle \omega, X\rangle = \int_M (f, X) ~d\mu.
\end{equation}
Furthermore, $f$ is unique up to a $\mu$-null set, and does not depend on the norm of $F$.
\end{proposition}
\begin{proof}
Fix an open cover $(U_i)$ of $M$ by charts which trivialize $F$, so that $U_i$ is precompact in $M$.
Let $(g_{ij})$ be the transition functions and $(g_{ij}')$ the induced transition functions for the dual bundle $F'$.
Then can view $\omega_i = \omega|U_i$ as an element of $C(U_i, (\CC^r)')'$, by the precompactness of $U_i$.
Hence by the Riesz-Markov theorem \cite[Theorem 4.14]{simon1983GMT}, there exists a $\mu$-measurable section $f_i$ of $SF$ for which (\ref{RNy formula}) holds for $\omega_i$, provided that $X \in C(U_i, (\CC^r)')$.

We now show that the $f_i$ are restrictions of a global section $f$, thus we must show $f_j = g_{ij} \circ f_i$ on $\CC^r$.
To this end, fix $X \in C_0(M, F)$ which is supported in $U_i \cap U_j$ and write $X_i \in C(U_i, (\CC^r)')$ for the trivialization of $X$ with respect to $U_i$.
Then $X_j = g_{ij}' \circ X_i$, and
\begin{align*}
\int_E (f_i, X_i) ~d\mu &= \langle \omega_i, X_i\rangle = \langle \omega_j, X_j\rangle = \int_E (f_j, X_j) ~d\mu = \int_E (f_j, g_{ij}' \circ X_i) ~d\mu.
\end{align*}

By Urysohn's lemma, $C_c(U_i \cap U_j, (\CC^r)', \mu)$ separates points in $L^1(U_i \cap U_j, \CC^r, \mu)$.
Therefore, since $X$ was arbitrary, $f_i = g_{ji} \circ f_j$; thus we obtain a unique global section $f$ of $SF$.

Finally, if we change the norm of $F$, replacing $|\cdot|$ with $|\cdot|'$, then we obtain a smooth function $h: F \to \RR_+$ so that if $v \in F_x$, then $|v|' = h(x, v)|v|$.
The change of norm gives us a new section $f'$ such that $f' = f/h(\cdot, f'(\cdot))$.
Thus $f'$ defines the same section of $SF$ as $f$.
\end{proof}

At this stage we have only defined $f$ as a $\mu$-equivalence class of sections of $SF$, so we now use the Lebesgue differentiation theorem to choose the ``correct" representative.
We state the differentiation theorem in a somewhat strange way, to ensure that the representative chosen is metric-independent.

\begin{definition}
A \dfn{Besicovitch cover} $\mathcal U$ of a metric space $X$ is a set of open balls, so that every $x \in X$ is the center of an element of $\mathcal U$.
The \dfn{Besicovitch number} $N \in \NN$ of $X$ is the best constant such that for every $x \in U$ and Besicovitch cover $\mathcal U$ of $B(x, 1/N)$, there exist $\mathcal U_1, \dots \mathcal U_N \subset \mathcal U$ such that $\bigcup_{n=1}^N \mathcal U_n$ is an open cover of $B(x, 1/N)$ and $\mathcal U_n$ is disjoint.
\end{definition}

It follows from the theory of \cite[\S2.8]{federer2014geometric} that for every Riemannian metric $g$, the Besicovitch number of $(M, g)$ is finite; \cite{Shi91} motivates why we restrict to small balls $B(x, 1/N)$.

For each $x \in M$, let $\mathcal A(x)$ denote the set of all pairs $(g, B, \varphi)$ where:
\begin{enumerate}
\item $g$ is a Riemannian metric on $M$,
\item $B$ is an open ball centered at $x$ with respect to $g$, and
\item $\varphi$ is a trivialization of $F$ over $B$.
\end{enumerate}
Then $\mathcal A(x)$ is a directed system, where the order is given by reverse inclusion of balls.
Given $(g, B, \varphi) \in \mathcal A(x)$, we obtain a $\mu$-measurable function $f_\varphi: B \to \CC^r$ obtained by trivializing the section $f$.
We define the average
\begin{equation}\label{average in a vector bundle}
f(g, B, \varphi) = \varphi^{-1}\left(\frac{1}{\mu(B)} \int_B f_\varphi ~d\mu\right),
\end{equation}
which is a point in $F_x$.

\begin{proposition}[Lebesgue differentiation theorem]\label{LebDiff}
Let $\mu$ be a Radon measure on $M$, let $f \in L^1_l(M, SF, \mu)$, and let
$$f(x) = \lim_{(g, B, \varphi) \in \mathcal A(x)} f(g, B, \varphi).$$
Then the limit defining $f(x)$ converges for $\mu$-almost every $x \in M$ to a point in the sphere $SF_x$.
\end{proposition}
If $f(x)$ exists and is $\in SF_x$, we call $x$ a \dfn{Lebesgue point} of the section $f$.
\begin{proof}
This is obvious if $f$ has a representative in $C_c(M, SF)$; besides, by a partition of unity argument, we may assume that $\mu$ has compact support.
We can then select $(f_n)$ in $C_c(M, SF)$ converging in $L^1(M, SF, \mu)$ and almost everywhere to $f$.
Setting $h_n = |f_n - f|$, we can define the average
$$h_n(g, B) = \frac{1}{\mu(B)} \int_B h_n ~d\mu,$$
which converges to $0$ in $L^1(M, \mu)$.

Fix $N \in \NN$ and let $\mathcal B_N$ be the set of Riemannian metrics with Besicovitch number $\leq N$.
This makes sense if we restrict to a neighborhood of the compact support of $\mu$.
For each metric $g \in \mathcal B_N$, we have the Hardy-Littlewood inequality \cite[Lemma 4.1.1a]{Ledrappier85}
\begin{equation}\label{HardyLittlewood}
||\sup_{r \in (0, 1/N)} h_n(g, B_g(\cdot, r))||_{L^{1, \infty}(M, \mu)} \leq N ||h_n||_{L^1(M, \mu)}.
\end{equation}
By (\ref{HardyLittlewood}) and the convergence $h_n \to 0$ in $L^1$,
$$\lim_{n \to \infty} ||\sup_{0 \in (0, 1/N)} h_n(g, B_g(\cdot, r))||_{L^{1, \infty}(M, \mu)} = 0$$
uniformly in $g \in \mathcal B_N$.
Therefore we may pass to a subsequence along which, for $\mu$-almost every $x$,
$$\lim_{n \to \infty} \sup_{(g, r) \in \mathcal B_N \times (0, 1/N)} h_n(g, B_g(x, r)) = 0.$$
By the triangle inequality, if
$$\mathcal A_N(x) = \{(g, B, \varphi) \in \mathcal A(x): g \in \mathcal B_N\},$$
then (after passing to a subsequence again)
$$\lim_{n \to \infty} \sup_{(g, B, \varphi) \in \mathcal A_N(x)} |f_n(g, B_g(x, r), \varphi) - f(g, B_g(x, r), \varphi)| = 0.$$
But $f_n(g, B, \varphi) \to f(x)$ everywhere, $f(x) \in SF_x$, and $SF_x$ is closed, so there exists a $\mu$-null set $Z_N$ such that outside of $Z_N$,
$$\lim_{(g, B, \varphi) \in \mathcal A_N(x)} f(g, B, \varphi) \in SF_x.$$
Taking $Z = \bigcup_{N \in \NN} Z_N$, we see that $Z$ is $\mu$-null, which was to be shown.
\end{proof}

\subsection{Differentiation and boundary}
In this section we fix a Riemannian metric.

\begin{definition}
A function in $L^1(M)$ has \dfn{bounded variation} if its distributional derivative is a $T'M$-valued Radon measure of finite total variation.
We write $BV$ for the presheaf of functions of bounded variation.
An open set has \dfn{finite perimeter} if its indicator function has bounded variation.
\end{definition}

\begin{notation}
If $u$ is a function of locally bounded variation we write $du ~\vol$ for its derivative.
\end{notation}

Sequences $(u_n)$ in $BV_l(M)$ with $u_n \to u$ in $L^1_l(M)$ satisfy the lower semicontinuity property
\begin{equation}
\label{RieszMarkovDistr}
\int_M |du| ~\vol \leq \liminf_{n \to \infty} \int_M |du_n| ~\vol.
\end{equation}
which follows by testing against smooth functions, and the forgetful map
\begin{equation}\label{Forget}
BV_l(M) \to L^1_l(M)
\end{equation}
is compact. We refer to \cite[Chapter 1]{Giusti77} for a review of the space $BV_l(M)$.
Our next result can be deduced by applying a partition of unity argument and then copying the proof of \cite[Teorema 1]{Miranda67} verbatim:

\begin{proposition}[trace theorem]\label{traces}
Let $U$ be an open set such that $N = \partial U$ is a Lipschitz hypersurface.
For every $u \in BV_l(M)$ there exists a trace $v \in L^1_l(N)$ such that for every $X \in C_c(M, TM)$,
\begin{equation}\label{Miranda IBP}
\int_U (du, X) ~\vol + \int_U u ~\mathcal L_X\vol = \int_N vg(X, \normal_N) ~\vol_N.
\end{equation}
Moreover, $v$ is determined by the germ of $u$ at $\partial U$.
If $u$ is an indicator function then so is $v$.
\end{proposition}

Let $U$ be a set of locally finite perimeter.
The notion of reduced boundary to $U$ was first introduced in \cite{deGiorgi55}; see \cite{Battista_2021} for an equivalent definition.
To construct it, let $\omega = d1_U ~\vol$, which by Proposition \ref{HanhJordan} can be expressed as $\omega = \normal \mu$, where $\normal$ is a section of $ST'M$ which is independent of $g$.

\begin{definition}
Let $U,\normal$ be as above.
The \dfn{reduced boundary} $\partial^* U$ of a set $U$ of locally finite perimeter is the set of Lebesgue points of $\normal$.
The \dfn{conormal $1$-form} to $\partial^* U$ is $\normal$.
\end{definition}

\begin{proposition}\label{metric-independence theorem}
Let $U$ be a set of locally finite perimeter and let $\normal$ be its conormal $1$-form.
Then $\normal$ and $\partial^* U$ do not depend on the metric.
\end{proposition}
\begin{proof}
Immediate from the fact that Propositions \ref{HanhJordan} and \ref{LebDiff} did not assume that $M$ was a Riemannian manifold.
\end{proof}

\begin{proposition}\label{locality of Caccioppoli}
Let $U$ be a set of locally finite perimeter with conormal $1$-form $\normal$.
Then:
\begin{enumerate}
\item $\partial^* U$ is either empty or $d-1$-dimensional in the Hausdorff sense, and is rectifiable with respect to $d-1$-dimensional Hausdorff measure.
\item $\partial^* U$ is dense in the measure-theoretic boundary $\partial U$.
\item If $\normal$ extends to a continuous $1$-form on $\partial U$, then $\partial^* U = \partial U$ is a $C^1$ hypersurface.
\end{enumerate}
\end{proposition}
\begin{proof}
Immediate from Proposition \ref{metric-independence theorem} and well-known facts about the euclidean case \cite[Chapters 2-4]{Giusti77} \cite{deGiorgi55}.
\end{proof}

\begin{notation}
We write $|\partial^* U|$ for $\int_M |d1_U| ~\vol$.
This does not collide with the notation $|U|$ for the volume of $U$, since $U$ has Hausdorff dimension $d$.
\end{notation}

\begin{proposition}\label{Giusti46}
    Let $U \subseteq M$ be a set of $C^1$ boundary, let $X$ be a vector field such that $(\normal_U, X) > 0$, and let $T > 0$.
    If $x: [-T, T] \to M$ is an integral curve of $X$ such that $x(0) \in \partial U$, then for every $t \in (0, T)$, $x(t) \in U$ and $x(-t) \in U$.
\end{proposition}
\begin{proof}
    The claimed assertions are all local, so we may assume that $M$ is diffeomorphic to an open subset of $\RR^d$.
    By Proposition \ref{metric-independence theorem}, we may without loss of generality assume that $M$ is actually isometric to an open subset of $\RR^d$ with its usual metric, $X$ is a coordinate field on $\RR^d$, and $x$ is a line segment.
    By taking a compact exhaustion of $M$ and applying the fact that $\partial U$ is $C^1$, we may also assume that there exists $\delta > 0$ such that $(\normal_U, X) \geq \delta$ almost everywhere.
    This reduces the claimed assertions to \cite[Lemma 4.6, Remark 4.7]{Giusti77}.
\end{proof}
\section{Functions of least gradient}\label{LeastGradientFunctions}
\subsection{Riemannian measure theory}
Let us recall some measure-theoretic facts.
See \cite[Chapter 1]{Giusti77} for analogous results over $\RR^d$, and see \cite{simon1983GMT} for the definition of a de Rham current.
We write $\int_U \omega \wedge \psi$ for the pairing of an $\ell$-current $\omega$ with a compactly supported $\ell$-form $\psi$ in an open set $U$.
We identify the distributional derivative of a function $u$ with the $d-1$-current
$$\int_U \dif u \wedge \psi = -\int_U u \wedge \dif \psi.$$
A function $u$ is in $BV(U)$ iff its derivative $\dif u$ has finite total variation
\begin{equation}\label{total variation}
\int_U *|\dif u| := \sup_{\substack{||\psi||_{L^\infty} \leq 1\\\supp \psi \Subset V}} \int_U \dif u \wedge \psi.
\end{equation}
Whether a current has locally finite total variation is independent of the Riemannian metric and so $BV_\loc(M)$ is also independently defined.

Now let $u \in BV_\loc(M)$.
Then by \cite[Theorem 4.14]{simon1983GMT}, there exists a $*|\dif u|$-measurable section $f$ of the cosphere bundle $S'M$ such that for every compactly supported $d-1$-form $\psi$,
\begin{equation}\label{RNy formula}
\int_M \dif u \wedge \psi = \int_M f|\dif u| \wedge \psi.
\end{equation}

For a vector field $X$, we write $*(Xu) := \dif u \wedge *(X^\flat)$.
The section $f$ of (\ref{RNy formula}) is given pointwise $*|\dif u|$-almost everywhere, in any local coordinates $(y_1, \dots, y_d)$, by
\begin{equation}\label{Lebesgue point definition}
    f(x) = \sum_{i = 1}^d \left[\lim_{r \to 0} \frac{\int_{B(x, r)} *\partial_{y_i} u}{\int_{B(x, r)} *|\dif u|}\right] ~\dif y^i,
\end{equation}
according to the Besicovitch differentiation theorem; here we view $(dy^i)$ as a basis of $T'_xM$.
However, whether the limit $f(x)$ in (\ref{Lebesgue point definition}) exists, or indeed its value as a point of $S'_xM$, do not depend on the Riemannian metric or the choice of coordinates.

\begin{definition}
The points $x$ for which the limit (\ref{Lebesgue point definition}) exists and satisfies $|f(x)| = 1$ are called the \dfn{Lebesgue points} of $\dif u$.
\end{definition}

The Lebesgue points are particularly important when $U$ is an indicator function:

\begin{definition}
Let $U$ be a set of locally finite perimeter, and let $u = 1_U$. Then:
\begin{enumerate}
\item The \dfn{measure-theoretic boundary} $\partial U$ is the set of points whose Lebesgue density with respect to $M$ is $\in (0, 1)$.
\item The set of Lebesgue points of $\dif u$ is the \dfn{reduced boundary} $\partial^* U$.
\item The $*|\dif u|$-measurable $1$-form $f$ defined by (\ref{Lebesgue point definition}) is the \dfn{conormal $1$-form} $\normal_U$ to $\partial U$.
\end{enumerate}
\end{definition}

Our definition of reduced boundary and conormal $1$-form follows \cite[Definition 3.3]{Giusti77} and is due to \cite{deGiorgi55}.
See \cite{Battista_2021} for an equivalent definition of reduced boundary on Riemannian manifolds, and see \cite[Chapter 6]{Pugh02} for the definition of Lebesgue density.

\begin{proposition}\label{locality of Caccioppoli}
    Let $U$ be a set of locally finite perimeter with conormal $1$-form $\normal$.
    Then:
    \begin{enumerate}
    \item $\partial^* U$ is either empty or $d-1$-dimensional in the Hausdorff sense, and is $d-1$-rectifiable.
    \item $\partial^* U$ is a dense subset of $\partial U$.
    \item If $\normal$ extends to a continuous $1$-form on $\partial U$, then $\partial^* U = \partial U$ is a $C^1$ hypersurface.
    \item If $\partial^* U = \partial U$ is a smooth hypersurface, then $\normal$ is the conormal $1$-form on $\partial U$ as defined in differential topology, and $*|\dif 1_U|$ is the induced volume form on $\partial U$.
\end{enumerate}
\end{proposition}
\begin{proof}
Most of the assertions of this proposition are diffeomorphism-invariant, so we may assume that $M = \RR^d$ and appeal to \cite[Chapters 2-4]{Giusti77}.
The proof that $*|\dif 1_U|$ is the induced volume form is identical to \cite[Example 1.4]{Giusti77}.
\end{proof}

\begin{definition}
Let $M$ be a Riemannian manifold, let $U$ be a set of locally finite perimeter, and let $E$ be a Borel set.
The \dfn{perimeter} of $U$ in $E$ is
$$|E \cap \partial^* U| := \int_E *|\dif 1_U|.$$
\end{definition}

\begin{proposition}[coarea formula]\label{Coarea2}
Let $M$ be a Riemannian manifold and $u \in BV_\loc(M)$. Then for every open set $E$,
\begin{equation}\label{coarea formula}
\int_E *|\dif u| = \int_{-\infty}^\infty |E \cap \partial \{u > y\}| \dif y.
\end{equation}
\end{proposition}
\begin{proof}
We follow \cite[Theorem 1.23]{Giusti77}, which first proves (\ref{coarea formula}) for $u \in C^\infty(\RR^d)$ using piecewise linear functions.
Such functions are not available for our purposes; instead we note that if $u \in C^\infty(\RR^d)$ and $u$ has no critical points then (\ref{coarea formula}) follows from Fubini's theorem, the fact that $|E \cap \partial \{u > y\}|$ is the surface area of $E \cap \{u = y\}$ (by Proposition \ref{locality of Caccioppoli}), and the change-of-variables formula.
However the left-hand side of (\ref{coarea formula}) is unaffected by critical points of $u$, and the right-hand side of (\ref{coarea formula}) is unaffected by critical values of $u$ by Sard's theorem.
So (\ref{coarea formula}) holds for $u \in C^\infty(\RR^d)$.

The rest of the proof is identical to \cite[Theorem 1.23]{Giusti77}, so we omit the details.
Taking a sequence in $C^\infty(M)$ that converges to $u$ in $L^1_\loc(M)$\footnote{Recall that $C^\infty(M)$ is not dense in $BV_\loc(M)$.}, and applying Fatou's lemma and the semicontinuity of total variation, we conclude the $\geq$ direction of (\ref{coarea formula}).
Moreover, Stokes' theorem gives that for every $d-1$-form $\psi$ such that $||\psi||_{L^\infty} \leq 1$ and $\supp \psi \Subset E$,
$$\int_E u \wedge \dif \psi = \int_{-\infty}^\infty \int_E |\psi| * |\dif 1_{\partial \{u > y\}}| \dif y \leq \int_{-\infty}^\infty |E \cap \partial \{u > y\}| \dif y.$$
Taking the supremum over $\psi$ we obtain the direction $\leq$ in (\ref{coarea formula}).
\end{proof}

%%%%%%%%%%%%%%%%%%%%
\subsection{Miranda's trace and stability theorems}
We now assert Miranda's trace theorem for $BV$ functions and stability theorem for least-gradient functions.
We also recall some a priori estimates for least-gradient functions.

\begin{proposition}[Miranda trace theorem]\label{traces}
Let $U \Subset M$ be an open set with Lipschitz boundary.
For every $u \in BV_\loc(U)$ there exists $v \in L^1_\loc(\partial U)$ such that for every $d-1$-form $\psi$,
\begin{equation}\label{Miranda IBP}
\int_U \dif u \wedge \psi + \int_U u \wedge \dif \psi = \int_{\partial U} v\psi.
\end{equation}
Moreover, for almost every $x \in \partial U$,
\begin{equation}\label{convergence of trace}
\int_{U \cap B(x, \varepsilon)} *|v(x) - u| \ll \varepsilon^d.
\end{equation}
\end{proposition}
\begin{proof}
The assertion (\ref{Miranda IBP}) is diffeomorphism-invariant and so follows from \cite[Teorema 1]{Miranda67}, and (\ref{convergence of trace}) also follows from that result if we are willing to drop a constant factor.
\end{proof}

To state our a priori estimates we define
$$\eta(u, U) := \inf_{v \in BV_\cpt(U)} \int_U *|\dif(u + v)|$$
for $u \in BV_\loc(M)$ and $U \Subset M$, so that $u$ has least gradient iff $\eta(u, U) = \int_U *|\dif u|$ for every $U$.

Suppose that $u, v \in BV_\loc(M)$ and $U \Subset M$ is bounded by a Lipschitz hypersurface $N$. Armed with the trace theorem, it is straightforward to generalize \cite[Lemma 5.6]{Giusti77}, thus
\begin{equation}
|\eta(u, U) - \eta(v, U)| \leq ||u - v||_{L^1(N)}. \label{a priori estimate 1}
\end{equation}
In case $v = 0$, we note that by (\ref{convergence of trace}), the trace map is a contraction in $L^\infty$ norm, thus we have the a priori estimate
\begin{equation}
\eta(u, U) \leq ||u||_{L^1(N)} \leq |N| \cdot ||u||_{L^\infty(M)}. \label{a priori estimate 2}
\end{equation}

\begin{definition}
A sequence $(u_n)$ in $BV_\loc(M)$ has \dfn{approximately least gradient} if for every open $U \Subset M$,
$$\limsup_{n \to \infty} \int_U *|\dif u_n| \leq \liminf_{n \to \infty} \eta(u_n, U).$$
\end{definition}

\begin{proposition}[Miranda stability theorem]\label{Miranda convergence}
If a sequence of functions $(u_n)$ has approximately least gradient and $u_n \to u$ in $L^1_\loc(M)$, then $u$ has least gradient, and for every open set $U \Subset M$ with Lipschitz boundary such that $\int_{\partial U} *|\dif u| = 0$, one has
\begin{equation}\label{convergence in total variation}
\lim_{n \to \infty} \int_U *|\dif u_n| = \int_U *|\dif u|.
\end{equation}
\end{proposition}
\begin{proof}
The proof is similar to Teorema 3 and Osservazione 3 in \cite{Miranda67}; we just note the necessary modifications.
Suitable generalizations of Teorema 2 and Osservazione 2 follow from Proposition \ref{traces}.
One needs to add a term of size $o(1)$ to the right-hand side of the inequalities (2.8), (2.9), (2.13), and (2.14); however, in the limit, this term vanishes and so the conclusions (2.15) and (2.16) are unaffected.
\end{proof}

\begin{corollary}\label{compactness}
Every sequence $(u_n)$ of approximately least gradient converges in $L^1_\loc$ and almost everywhere along a subsequence to a function of least gradient $u$ such that for every open set $U \Subset M$ of Lipschitz boundary such that $\int_{\partial U} *|\dif u| = 0$, one has (\ref{convergence in total variation}).
\end{corollary}
\begin{proof}
It follows from Proposition \ref{Miranda convergence} and the fact that the natural map $BV_\loc \to L^1_\loc$ is compact.
\end{proof}

\begin{proposition}\label{level sets are minimal}
For every function $u$ of least gradient, the superlevel sets $\{u > y\}$ have least perimeter.
If we instead have a sequence $(u_n)$ of approximately least gradient, then $(\{u_n > y\})$ has approximately least perimeter.
\end{proposition}
\begin{proof}
In the proof of \cite[Theorem 1]{BOMBIERI1969}, replace the coarea formula \cite[Theorem 1.6]{Miranda66} with Proposition \ref{Coarea2} and replace \cite[Teorema 3]{Miranda67} with Proposition \ref{Miranda convergence}.
\end{proof}

%%%%%%%%%%%%%%%%%%%%%%%%%%%%%%%%%%%%%%%%%%%%

\subsection{Blowup of the reduced boundary}
Now let us study the blowup of $M$ at a point $p$ on the reduced boundary of a set $U$ of least perimeter, giving a generalization of the conjunction of \cite[Theorem 9.3]{Giusti77} and \cite[Theorem 6.2.2]{Simons68}.

\begin{definition}
For a function $u$ on $M$, $P \in M$ we define the \dfn{tangent rescaling} of $u$ at $P$ to be the net of functions $u_t: T_PM \to \RR$, given by
$$u_t(v) = u\left(\exp_P(tv)\right)$$
\end{definition}

\begin{proposition}\label{blowup theorem}
Suppose that $U$ is an open set with least perimeter in $B(P, r)$, $P \in \partial^* U$, and $u = 1_U$.
Furthermore, suppose that $d \leq 7$.
Then the tangent rescaling of $u$ converges as $t \to 0$ along a subsequence (that we also denote $t \to 0$) in $L^1_\loc$ and almost everywhere, to the indicator function $v$ of a half-space $C \subset T_PM$ such that $0 \in \partial C$.
Moreover, for every open set $V \Subset T_PM$ of Lipschitz boundary such that $\int_{\partial V} *|\dif v| = 0$ we have the convergence of total variation
$$\lim_{t \to 0} \int_V *|\dif u_t| = \int_V *|\dif v|.$$
\end{proposition}
\begin{proof}
We claim that the tangent rescaling $(u_t)$ has approximately least gradient in every precompact open subset of $T_PM$ (where we give $T_PM$ its euclidean metric). If this true, then by Corollary \ref{compactness}, there exists a set $C$ of least perimeter in $T_PM$, such that the tangent rescaling converges to $v := 1_C$ in the desired sense.
But $T_PM$ is isometric to $\RR^d$, $d \leq 7$, so by the Bernstein--Fleming theorem \cite[Theorem 17.3]{Giusti77}\footnote{for an easier proof of the $d = 2$ case see \cite[\S5]{Fleming62}}, $\partial C$ is a hyperplane.
The fact that $0 \in \partial C$ follows from the fact that $P \in \partial^* U$.

To prove the claim, write $|\cdot|'$, $*'$ for the notions taken in the tangent space with its euclidean geometry, and write $U_t$ for the set indicated by $u_t$.
If $V$ is a precompact open subset of $T_PM$, $V_t = \{v \in T_PM: tv \in V\}$, then we have the scale-invariance
\begin{equation}\label{almost blowup scale invariance}
|\partial^* U_t \cap V|' = t^{1 - d}|\partial^* U_1 \cap V_{1/t}|'.
\end{equation}
From (\ref{almost blowup scale invariance}) and the Taylor expansion of $g$ in normal coordinates \cite[Lemma 3.4]{schoen1994lectures},
$$t^{d - 1} |\partial^* U_t \cap V|' = |\partial^* U_1 \cap V_{1/t}|' \leq (1 + O(t^2)) |\partial^* U \cap \exp_P(V_{1/t})|.$$
For every $w \in BV_\cpt(V)$, the least-gradient nature of $u$ gives
$$|\partial^* U \cap \exp_P(V_{1/t})| \leq \int_{V_{1/t}} *'|\dif(u + (\exp_P)_* w_{1/t})| \leq (1 + O(t^2))\int_{V_{1/t}} *'|\dif(u_1 + w_{1/t})|.$$
Therefore, after applying (\ref{almost blowup scale invariance}) and the Taylor expansion again,
$$|\partial^* U_t \cap V|' \leq (t^{1 - d} + O(t^{3 - d})) \int_{V_{1/t}} *' |\dif (u_1 + w_{1/t})| = (1 + O(t^2)) \int_V *' |\dif (u_t + w)|.$$
Since $V,w$ were arbitrary, we conclude that $(u_t)$ has approximately least gradient.
\end{proof}

\section{Mollification of sets of least perimeter}\label{MollifierSection}
In this section our goal is to show that given a set $U$ of least perimeter, we can find a $C^1$ hypersurface $N$ which approximates $\partial^* U$ arbitrarily well.

\begin{proposition}\label{mollifier quant}
    For every $P \in M$ and every $\varepsilon > 0$, there exists $\gamma^* > 0$ such that $P \mapsto \gamma^*$ is continuous and such that for every $\delta > 0$ such that $\exp_P: \{v: g(v, v) < \delta\} \to M$ is injective, all vector fields $X, Y$ with $1/2 \leq g(X, X), g(Y, Y) \leq 2$ and $g(X, X) = g(Y, Y) = 1$ at $P$, and every set $U$ of least perimeter, $u = 1_U$, such that 
    \begin{equation}\label{bootstrap the mollifier}
    \gamma = \delta^{1 - d}\int_{B(P, \delta)} |du| - Xu ~\vol < \gamma^*:
    \end{equation}
    
    For every almost every $t \in (0, \delta)$, there exists a set $V$ of $C^1$ perimeter in $B(P, t)$, $v = 1_V$, such that 
    \begin{align}
    |V \cap B(P, t)| &\leq \eta(V, B(P, t)) + \varepsilon \delta^{d - 1} \gamma, \label{mollifier quant1}\\
    \left|\int_{B(P, t)} |du| - |dv| ~\vol\right| &\leq \varepsilon \delta^{d - 1} \gamma, \label{mollifier quant2}\\
    \left|\int_{B(P, t)} Y(u - v) ~\vol\right| &\leq \varepsilon \delta^{d - 1} \gamma, \label{mollifier quant3}\\
    (\normal_V, X) &\geq (1 - \varepsilon) g(X, X) \text{ on } B_{t-\varepsilon}. \label{mollifier quant4}
    \end{align}
\end{proposition}

The euclidean analogue of this proposition is \cite[Lemma 5.5]{Miranda66}.
For an exposition of the euclidean proof, we recommend \cite[Chapter 7]{Giusti77}.
We use the same integral kernel as \cite[Chapter 7]{Giusti77}, and much of our argument is the same, but we must take care to control the error terms arising from $\Ric_g$.

We proceed using compactness and contradiction. More precisely, suppose that the following lemma is true:

\begin{lemma}\label{mollifier proposition}
Let $(U_n)$ be a sequence of sets with indicator functions $u_n$, let $X, Y$ be vector fields with $1/2 \leq g(X, X), g(Y, Y) \leq 2$ and $g(X, X) = g(Y, Y) = 1$ at $P$, and let
$$\gamma_n = \int_{B(P, 1)} |du_n| - Xu_n ~\vol.$$
Suppose that in addition:
\begin{enumerate}
\item $(u_n)$ has least gradient.
\item $(\gamma_n) \in \ell^1$.
\item The exponential map $\exp_P$ is injective on $\{v \in T_PM: g(v, v) < 1\}$.
\end{enumerate}

Then for almost every $t \in (0, 1)$ there exist open sets $V_n$ with $C^1$ boundary and indicator $v_n$, such that for every $j \in \{0, \dots, d - 1\}$,
\begin{align}
|\partial V_n \cap B_t| &\leq \eta(V_n, B_t) + o(\gamma_n), \label{mollifier prop1}\\
\left|\int_{B_t} |du_n| - |dv_n| ~\vol\right| &\ll \gamma_n, \label{mollifier prop2}\\
\left|\int_{B_t} Y(u_n - v_n)~\vol\right| &\lesssim \gamma_n^2, \label{mollifier prop3}\\
\lim_{n \to \infty} g(\normal_{L_n}, X) &= g(X, X), \label{mollifier prop4}
\end{align}
and the limit in (\ref{mollifier prop4}) is locally uniform.
\end{lemma}

By rescaling $g$ appropriately, we may assume that $\delta \geq 1$.
Then if the claim is not true, there exist $P \in M, \varepsilon > 0, X, Y$, and $\gamma_n \to 0$, such that there is a set $U_n$ of least perimeter satisfying (\ref{bootstrap the mollifier}) with $U = U_n$, $\gamma = \gamma_n$, and an exceptional set $T_n \subseteq (0, 1)$ of positive measure such that for every $t \in T_n$ and every set $V$ of $C^1$ perimeter in $B(P, t)$, one of (\ref{mollifier quant1}, \ref{mollifier quant2}, \ref{mollifier quant3}, \ref{mollifier quant4}) is false.
After passing to a subsequence we may assume that $(\gamma_n) \in \ell^1$.
If we take $V = V_n$ where $V_n$ is given by Lemma \ref{mollifier proposition}, the failure of the conjunction of (\ref{mollifier quant1}, \ref{mollifier quant2}, \ref{mollifier quant3}, \ref{mollifier quant4}) contradicts the conjunction of (\ref{mollifier prop1}, \ref{mollifier prop2}, \ref{mollifier prop3}, \ref{mollifier prop4}), which completes the proof of Proposition \ref{mollifier quant}.
So it suffices to show Lemma \ref{mollifier proposition}.

%%%%%%%%%%%%%%%%%%%%%%%%%%%%%%%%%%%%%%%%%%%%%%%%%%%%%%%

\subsection{Polar coordinates}
Suppose that we are in the situation of Lemma \ref{mollifier proposition}, namely we are given a point $P \in M$ such that the exponential map is a diffeomorphism
\begin{equation}\label{exp map mollify}
\exp_P: \{v \in T_PM: g(v, v) < 1\} \to B_1.
\end{equation}
Then we may introduce polar coordinates
\begin{equation}\label{polar coordinates mollify}
(r, \Theta): B_1 \setminus \{P\} \to (0, 1) \times \Sph^{d - 1},
\end{equation}
where $\Theta = (\theta_1, \dots, \theta_{d - 1})$,
for which $g(\partial_r, \partial_{\theta_i}) = 0$ and $g(\partial_r, \partial_r) = 1$.
We also define
$$\kappa = -\frac{R(P)|\Sph^{d - 1}|}{6d}$$
where $R$ is the scalar curvature field of $g$.
With this notation we have
\begin{equation}\label{area of sphere form}
|\partial B_\rho| = |\Sph^{d - 1}|\rho^{d - 1} + \kappa \rho^{d + 1} + O(\rho^{d + 3}),
\end{equation}
and $\kappa$ and the above implied constant depend continuously on $P$; see \cite{gray1974volume} for a proof.
We use this fact to estimate the normal vector of a set of least perimeter, as follows.

\begin{lemma}\label{scalar curvature monotonicity}
If $r_1 < r_2$, $U$ is a set of least perimeter, $u = 1_U$, and $Q \in \partial^* U$,
$$r_2^{1 - d}\int_{B(Q, r_2)} Xu ~\vol - r_1^{1 - d}|\partial^* U \cap B(Q, r_1)| \leq (r_2^2 - r_1^2)\kappa(Q) + O(r_2^4).$$
\end{lemma}
\begin{proof}
Since $U$ is a set of least perimeter, in particular its reduced boundary in $B(Q, r_1)$ has less area than $\partial B(Q, r_1)$.
By the fundamental theorem of calculus and the Cauchy-Schwarz inequality,
$$\int_{B(Q, r_2)} Xu ~\vol = \int_{\partial B(Q, r_2) \cap U} g(\normal_{\partial B(Q, r_2)}, X) ~\vol \leq \int_{\partial B(Q, r_2) \cap U} \vol = |\partial B(Q, r_2)|.$$
Applying (\ref{area of sphere form}),
\begin{align*}
r_2^{1 - d}|\partial B(Q, r_2)| - r_1^{1 - d}|\partial^* U \cap B(Q, r_1)|
&\leq r_2^{1 - d}|\partial B(Q, r_2)| - r_1^{1 - d}|\partial B(Q, r_1)| \\
&\leq |\Sph^{d - 1}| + r_2^2\kappa(Q) + O(r_2^4) - |\Sph^{d - 1}| - r_1^2 \kappa(Q) + O(r_1^4)\\
&\leq (r_2^2 - r_1^2)\kappa(Q) + O(r_2^4).\qedhere
\end{align*}
\end{proof}

We now construct the integral kernel that we will use in the proof of Lemma \ref{mollifier proposition}.
Using the identification (\ref{exp map mollify}), we can define the difference of two points in $B_1$ and so can define the convolution $\omega * u$ of a function $u \in L^1_l(B_1)$ with any volume form $\omega$ with compact support in $B_1$.

\begin{definition}
Define the volume form
$$\chi_\varepsilon = \frac{r^{d - 1}}{\varepsilon^d}\left(1 - \frac{r}{\varepsilon}\right)1_{r < \varepsilon} ~dr \wedge d\mu(\Theta)$$
where $\mu$ is the usual measure on $\Sph^{d - 1}$, but normalized so that $\mu(\Sph^{d - 1}) = d(d + 1)$.
For a function $u \in L^1_l$, we define $u_\varepsilon = \chi_\varepsilon * u$.
\end{definition}

The integration kernel $\chi_\varepsilon$ was introduced in \cite[Chapter 7]{Giusti77} in order to mollify sets of least perimeter on euclidean space.
The point is that, for the purposes of our later arguments, we will only need $u_\varepsilon \in C^1$, and so we can use a Lipschitz kernel which satisfies certain convenient estimates \cite[Lemmata 7.1--7.2]{Giusti77}.
Our purpose is now to prove an analogue of such estimates in our setting, but before we do so, we record the basic properties of $\chi_\varepsilon$:

\begin{lemma}\label{mollifier props}
$\chi_\varepsilon$ has the following properties:
\begin{enumerate}
\item $\chi_\varepsilon$ is supported on $B_\varepsilon$.
\item The Radon-Nikod\'ym derivative $F = \chi_\varepsilon/\vol$ is Lipschitz continuous and has the Taylor expansion
\begin{equation}\label{RN mollify}
F(r, \Theta) = \frac{d(d + 1)}{\varepsilon^d} \left(1 - \frac{r}{\varepsilon}\right) 1_{r < \varepsilon}(1 - \kappa r^2 + O(r^4)).
\end{equation}
\item If $u$ is an indicator function then $(du_\varepsilon) = (du)_\varepsilon$ is a continuous $1$-form.
\item $\chi_\varepsilon$ converges in the weakstar topology to the Dirac measure at $Q$.
\item On $B_\varepsilon \setminus \{P\}$, $\partial_{\theta_i} F = 0$ and $\partial_r F = O(r)$.
\item For every $Q \in B_1$, $\delta > 0$, $x \in B_1$, and $y \in B(Q, \delta)$,
\begin{equation}\label{approximation of mollifier 2}
F(x - y) \sim \frac{1}{\varepsilon^d}\left(1 - \frac{d(x, Q)}{\varepsilon}\right)
\end{equation}
where the implied constant remains bounded as $\varepsilon,\delta \to 0$.
\end{enumerate}
\end{lemma}
\begin{proof}
These assertions follow from the definitions and (\ref{area of sphere form}).
\end{proof}

\begin{lemma}\label{Giusti71}
There exists $c > 0$ such that for every indicator function $u = 1_U$, if
$$c\varepsilon^2 + c\rho^2 \leq u_\varepsilon(x) \leq 1 - c\varepsilon^2 - c\rho^2,$$
then
\begin{equation}\label{Giusti71 claim}
d(x, \partial U) < \varepsilon(1 - \rho).
\end{equation}
\end{lemma}
\begin{proof}
Suppose that (\ref{Giusti71 claim}) fails for some $x \in M$. Then either $d(x, U) \geq \varepsilon(1 - \rho)$ or $d(x, M \setminus U) \geq \varepsilon(1 - \rho)$.
We estimate $\omega(y) = u(y)\chi_\varepsilon(x - y)$ on the annulus $A = B(x, \varepsilon) \setminus B(x, \varepsilon(1 - \rho))$ using (\ref{RN mollify}) and Taylor's theorem, as follows:
\begin{align*}
\int_A \omega &= \int_A F ~\vol \\
&= \frac{d(d + 1)}{\varepsilon^d} \int_{\varepsilon(1 - \rho)}^\varepsilon r^{d - 1}\left(1 - \frac{r}{\varepsilon}\right)(1 + O(r^2)) ~dr \\
&= \frac{d(d + 1)}{\varepsilon^d} \int_{\varepsilon(1 - \rho)}^\varepsilon r^{d - 1} - \frac{r^d}{\varepsilon} + O(r^{d + 1}) ~dr\\
&= \frac{d(d + 1)}{\varepsilon^d} \left[\frac{\varepsilon^d}{d}(1 - (1 - \rho)^d) - \frac{\varepsilon^d}{d + 1}(1 - (1 - \rho)^{d + 1}) + O(\varepsilon^{d + 2})\right]\\
&= 1 - (d + 1)(1 - d\rho + O(\rho^2)) + d(1 - (d + 1)\rho + O(\rho^2)) + O(\varepsilon^2) \\
&= O(\rho^2 + \varepsilon^2).
\end{align*}
If $d(x, U) \geq \varepsilon(1 - \rho)$, then $u\omega$ is supported inside $A$, thus $u_\varepsilon(x) \gtrsim \rho^2 + \varepsilon^2$.
Conversely, if $d(x, M \setminus U) \geq \varepsilon(1 - \rho)$, then $(1 - u)\omega$ is supported inside $A$, thus $1 - u_\varepsilon(x) \lesssim \rho^2 + \varepsilon^2$.
\end{proof}

\begin{lemma}\label{Giusti72}
Let $u \in BV(B_1)$ and $\tau, \varepsilon > 0$. If $\tau + \varepsilon \leq 1$, then
$$\int_{B_\tau} |u_\varepsilon - u| ~\vol \lesssim \varepsilon \int_{B_{\tau + \varepsilon}} |du| ~\vol$$
$$\int_{B_\tau} |du_\varepsilon| - |du| ~\vol \lesssim \int_{B_{\tau + \varepsilon} \setminus B_\tau} |du| ~\vol.$$
\end{lemma}
\begin{proof}
If we suppose that
$$c^{-1} \vol^{\mathrm{euc}} \leq \vol \leq c\vol^{\mathrm{euc}}$$
where $\vol^{\mathrm{euc}}$ is the pushforward by $\exp_P$ of the euclidean volume form on $T_PM$ (as given by (\ref{exp map mollify})), then the assertions are immediate from \cite[Lemma 7.2]{Giusti77} with implied constant $c^2$.
Such a constant $c > 0$ must exist, since $\sqrt{\det g}$ is bounded on the compact set $B_1$.
\end{proof}

%%%%%%%%%%%%%%%%%%%%%%%%%%%%%%%%%%%%%%%%%%%%%%%%%%%%%%

\subsection{\texorpdfstring{$C^1$}{C1} level sets}
The main step in the proof of Lemma \ref{mollifier proposition} is to generalize \cite[Theorem 7.3, Remark 7.4]{Giusti77}:

\begin{lemma}\label{main mollifier lemma}
Let $\gamma, p > 0$, let $U$ be a set of least perimeter, $u = 1_U$, and suppose that
\begin{equation}\label{hypothesis on main mollifier lemma}
\int_{B_1} (|du| - Xu) ~\vol \leq \gamma.
\end{equation}
Let $\varepsilon = \gamma^p$, $\sigma = \gamma^{1/(2(d - 1))}$, and $\varphi = u_\varepsilon$. Then there exists $c > 0$ such that
\begin{equation}\label{claim on main mollifier lemma}
\inf_{\substack{r < \sigma\\\varphi \in (c\gamma^2, 1 - c\gamma^2)}} \frac{X \varphi}{|d\varphi|} > 1 - O(\gamma^{O(1)})
\end{equation}
and for every $y \in (c\gamma^2, 1 - c\gamma^2)$ the level set $\partial \{\varphi > y\} \cap \{r < \sigma\}$ is a $C^1$ hypersurface.
\end{lemma}

The proof of Lemma \ref{main mollifier lemma} is analogous to the proof of \cite[Theorem 7.3]{Giusti77}, using the results of \S\ref{LeastGradientFunctions} as a substitute for \cite[Chapter 5]{Giusti77}.
Here is the main idea: $|du| ~\vol$ is a Radon measure of Ahlfors-David dimension $d - 1$ (c.f. Proposition \ref{doubling dimension}), so we may cover the domain of integration with open sets of volume $\sim \delta^d\varepsilon^d$ where $\delta > 0$ is a small parameter that determines the cardinality of the cover.
Then $|du| - Xu ~\vol$ assigns a slightly shrunken version of each such set a measure $\sim \delta^{d - 1} \varepsilon^{d - 1}$, while $|du|~\vol$ assigns the set in the cover a comparable measure, by the monotonicity formula, Proposition \ref{Monotonicity Formula}.
Thus $|du| - Xu ~\vol$ is controlled by $|du|~\vol$.

\begin{proof}[Proof of Lemma \ref{main mollifier lemma}, assuming (\ref{bound on balls}, \ref{bound on balls 2})]
Let $\delta = \gamma^d > 0$ and select disjoint balls $V_1, \dots, V_N$, centered on $Q_n$, in $\partial^* U \cap B_{\varepsilon(1 - 2\delta)}$ of radius $\delta\varepsilon$ so that the dilates $2V_n$ cover $\partial^* U \cap B_{\varepsilon(1 - 2\delta)}$.
It is easy to show that such a cover exists, because $\overline{\partial^* U \cap B_{\varepsilon(1 - 2\delta)}}$ is compact if $\gamma$ is small enough, so for such a $\gamma$ we can greedily select $Q_n \in \overline{\partial^* U \cap B_{\varepsilon(1 - 2\delta)}}$ to maximize $\min(d(Q_1, Q_n), \dots, d(Q_{n - 1}, Q_n))$.
We set $V_0 = B_\varepsilon \setminus B_{\varepsilon(1 - 2\delta)}$.

\begin{figure}[ht]
\caption{The sets $V_0, \dots, V_n$ (in dark grey) are an annulus and several small balls of radius $\delta$, which approximately cover the boundary of the set $U$ (in light grey).}
\includegraphics[width=0.4\textwidth]{covering lemma}
\end{figure}

Since $du$ is supported in $\bigcup_n 2V_n$,
$$|d\varphi|(x) - X\varphi(x) = \int_{B_\varepsilon} \chi_\varepsilon(x - \cdot)(|du| - Xu) \leq \sum_{n=0}^N \int_{2V_n} \chi_\varepsilon(x - \cdot)(|du| - Xu).$$
So, we shall show
\begin{equation}\label{bound on balls}
\int_{2V_n} \chi_\varepsilon(x - \cdot)(|du| - Xu) \lesssim_{g, p, P} \gamma^{O(1)} \int_{V_n} \chi_\varepsilon(x - \cdot)|du|
\end{equation}
for $n \geq 1$ and
\begin{equation}\label{bound on balls 2}
\int_{V_0} \chi_\varepsilon(x - \cdot)(|du| - Xu) \lesssim_{g, p, P} \gamma^{O(1)} \int_{B_\varepsilon} \chi_\varepsilon(x - \cdot)|du|,
\end{equation}
provided that $x = (r, \Theta)$ satisfies $r < \sigma$ and $\varphi \in (o(\gamma), 1 - o(\gamma))$.
If (\ref{bound on balls}, \ref{bound on balls 2}) are true, then since the balls $V_n$ are disjoint, we can sum over $n$ to obtain
\begin{equation}\label{claim on main mollifier lemma 2}|d\varphi|(x) - X\varphi(x) \lesssim_{g, p, P} \gamma^{O(1)} \int_{B_\varepsilon} \chi_\varepsilon(x - \cdot)|du| \leq \gamma^{O(1)} |d\varphi|(x),
\end{equation}
which implies (\ref{claim on main mollifier lemma}). Thus we may fix
$$x \in \partial^* U \cap \{r < \sigma\} \cap \{\varphi \in (f(\gamma), 1 - f(\gamma)),$$
and choose $\gamma$ so small that (\ref{claim on main mollifier lemma 2}) simplifies to $X\varphi(x) \gtrsim |d\varphi(x)|$.
Thus, in particular,
$$X\varphi(x) \gtrsim \int_{B_\varepsilon} \chi_\varepsilon(x - \cdot) |du| > 0.$$
By Lemma \ref{mollifier props}, $d\varphi$ is continuous, so the level sets of $\varphi$ must be $C^1$.
\end{proof}

\begin{proof}[Proof of (\ref{bound on balls})]
By (\ref{approximation of mollifier 2}), if $\gamma$ is small enough depending on $g$\footnote{One might worry that we frequently rescale $g$ in this paper.
However, in all of our rescalings, the curvature tensor remains in some bounded set, so these rescalings will never send $\gamma$ to $0$.}, then for every $y \in 2V_n$,
\begin{align*}
\int_{2V_n} \chi_\varepsilon(x - \cdot)(|du| - Xu) &\lesssim F_n(x) \int_{2V_n} |du| - Xu ~\vol, \\
F_n(x) &:= \frac{1}{\varepsilon^d}\left(1 - \frac{d(x, Q_n)}{\varepsilon}\right).
\end{align*}
If $\gamma$ is chosen small enough, then $\sigma > 2\delta\varepsilon$ and so if we set $W_n = B(Q_n, \sigma)$ and apply Proposition \ref{Monotonicity Formula},
\begin{align*}
\int_{2V_n}|du| - Xu ~\vol &\leq
B \left[\sigma^{1 - d}\int_{W_n} |du| - Xu ~\vol + \sigma^{1 - d}\int_{W_n} Xu ~\vol - (2\delta\varepsilon)^{1 - d}\int_{2V_n} Xu ~\vol \right],\\
B &:= e^{O(1)(\sigma^2 - 4\delta^2\varepsilon^2)}(2\delta\varepsilon)^{d - 1}.
\end{align*}
From Taylor's theorem and the fact that $\sigma > 2\delta\varepsilon$,
\begin{align*}
B \lesssim \delta^{d - 1} \varepsilon^{d - 1} + \sigma^2 \delta^{d - 1} \varepsilon^{d - 1} \lesssim \delta^{d - 1} \varepsilon^{d - 1}
\end{align*}
if $\gamma$ is small.
By (\ref{hypothesis on main mollifier lemma}),
$$\sigma^{1 - d}\int_{W_n} |du| - Xu ~\vol \leq \gamma^{\frac{1 - d}{2(d - 1)} + 1} = \gamma^{1/2}.$$
By Proposition \ref{Monotonicity Formula},
\begin{align*}
\sigma^{1 - d}\int_{W_n} Xu ~\vol - (2\delta\varepsilon)^{1 - d}\int_{2V_n} Xu ~\vol &\lesssim (1 + \alpha)\sqrt{\sigma^{1 - d} \int_{W_n} |du| ~\vol - (2\delta\varepsilon)^{1 - d} \int_{2V_n} |du| ~\vol},\\
&\alpha := (d - 1)\log \frac{\sigma}{2\delta\varepsilon}.
\end{align*}
By Lemma \ref{scalar curvature monotonicity},
$$\sigma^{1 - d} \int_{W_n} X u ~\vol - (2\delta\varepsilon)^{1 - d} \int_{2V_n} |du| ~\vol \leq (\sigma^2 -4\delta^2\varepsilon^2) \kappa(Q_n) + O(\sigma^4) \lesssim \sigma^2,$$
so by (\ref{hypothesis on main mollifier lemma}),
\begin{align*}
\sigma^{1 - d} \int_{W_n} |du| ~\vol - (2\delta\varepsilon)^{1 - d} \int_{2V_n} |du| ~\vol &= \sigma^{1 - d} \int_{W_n} |du| - Xu ~\vol \\
&\qquad + \sigma^{1 - d} \int_{W_n} Xu ~\vol - (2\delta\varepsilon)^{d - 1} \int_{W_n} |du| ~\vol \\
&\leq \gamma + O(\sigma^2) \lesssim \sigma^2.
\end{align*}
It follows from the definitions that
$$(1 + \alpha)\sigma \lesssim -\gamma^{1/2(d - 1)} \log \gamma \lesssim \gamma^{1/3(d - 1)}.$$
Summing up everything in this step of the proof thus far,
\begin{equation}\label{big bound 1}
\int_{2V_n} \chi_\varepsilon(x - \cdot)(|du| - Xu) ~\vol \lesssim \delta^{d - 1} \varepsilon^{d - 1} F_n(x) \gamma^{1/3(d - 1)}.
\end{equation}
Since $U$ has least perimeter, Proposition \ref{doubling dimension} implies that
$$\delta^{d - 1} \varepsilon^{d - 1} \lesssim \int_{V_n} |du| ~\vol,$$
so by (\ref{approximation of mollifier 2}, \ref{big bound 1}),
\begin{align*}
\int_{2V_n} \chi_\varepsilon(x - \cdot)(|du| - Xu) ~\vol
&\lesssim \gamma^{1/3(d - 1)} \int_{V_n} \chi_\varepsilon(x - \cdot)|du|.
\qedhere \end{align*}
\end{proof}

\begin{proof}[Proof of (\ref{bound on balls 2})]
From (\ref{approximation of mollifier 2}) it easily follows that for $y \in V_0$, $\chi_\varepsilon(x - y)/\vol(x - y) \lesssim \frac{\delta}{\varepsilon^d}$,
whence, by minimality of $\partial^* U$,
\begin{align*}
\int_{V_0} \chi_\varepsilon(x - \cdot)(|du| - Xu) &\lesssim \frac{\delta}{\varepsilon^d} \int_{B_\varepsilon} |du| ~\vol \lesssim \frac{\delta}{\varepsilon^d} |\partial B_\varepsilon| \lesssim \frac{\delta}{\varepsilon}.
\end{align*}
By Lemma \ref{Giusti71}, there exists $c > 0$ such that if $\varphi \in (c\gamma^2, 1 - c\gamma^2)$, then $d(x, \partial U) < \varepsilon(1 - \gamma)$, so in particular we can find $Q \in \partial^* U$ such that $d(x, Q) < \varepsilon(1 - \gamma)$.
If $d(y, Q) < \gamma\varepsilon/2$, then
$$d(x, y) \leq \varepsilon - \gamma\varepsilon + \frac{\gamma\varepsilon}{2} \leq \varepsilon - \frac{\gamma\varepsilon}{2},$$
so by (\ref{approximation of mollifier 2}), $\chi_\varepsilon(x - y)/\vol(x - y) \gtrsim \frac{\gamma}{\varepsilon^d}$
for every $y \in B(Q, \gamma\varepsilon/2)$.
In particular, since $\delta = \gamma^d$, minimality of $\partial^* U$ gives
\begin{align*}
\int_{V_0} \chi_\varepsilon(x - \cdot)(|du| - Xu) &\lesssim \frac{\delta}{\gamma^{d - 1}} \frac{\gamma^{d - 1}}{\varepsilon}\\
&\lesssim \gamma |\partial B(Q, \gamma\varepsilon/2)| \int_{B(Q, \gamma\varepsilon/2)} \chi_\varepsilon(x - \cdot) \\
&\lesssim \gamma \int_{B(Q, \gamma\varepsilon/2)} \chi_\varepsilon(x - \cdot) |du|\\
&\lesssim \gamma \int_{B_\varepsilon} \chi_\varepsilon(x - \cdot) |du|. \qedhere
\end{align*}
\end{proof}

%%%%%%%%%%%%%%%%%%%%%%%%%%%%%%%%%%%%%%%%%%%%%%%%%%%%%

\subsection{The qualitative case}
We now complete the proof of Lemma \ref{mollifier proposition}.
Select $t \in (0, 1)$ uniformly at random.
Let $w_n = (u_n)_{\gamma_n^4}$, let $c$ be the constant given by Lemma \ref{main mollifier lemma}, and let $a_n = c\gamma_n$, $b_n = 1 - c\gamma_n$.
0By Proposition \ref{Coarea2},
$$\int_{B_t} |dw_n| ~\vol = \int_0^1 |\partial^* \{w_n > y\} \cap B_t| ~dy \geq \int_{a_n}^{b_n} |\partial^* \{w_n > y\} \cap B_t| ~dy.$$
By the mean value theorem, there exists $y_n \in (a_n, b_n)$ such that
\begin{equation}\label{MVT mollifier}
|\partial^* \{w_n > y_n\} \cap B_t| \leq \frac{1}{b_n - a_n} \int_{B_t} |dw_n| ~\vol.
\end{equation}
If set $V_n = \{w_n > y_n\}$, $v_n = 1_{V_n}$, then $V_n$ has $C^1$ boundary in $B_t$ by definition of $a_n, b_n$, and from (\ref{claim on main mollifier lemma}), the locally uniform convergence (\ref{mollifier prop4}) holds.

So it remains to show that (\ref{mollifier prop1}--\ref{mollifier prop3}) hold almost surely.
Towards this end we will later show that
\begin{align}
|\partial V_n \cap B_t| &\leq |\partial^* U_n \cap B_t| + o(\gamma_n), \label{approximation of surface area} \\
\int_{\partial B_t} |u_n - v_n| ~\vol_{\partial B_t} &\lesssim \gamma_n^2. \label{approximation of volume}
\end{align}
If (\ref{approximation of surface area}, \ref{approximation of volume}) are true,
then by (\ref{a priori estimate 3}, \ref{approximation of volume}),
$$|\partial^* U_n \cap B_t| \leq |\partial V_n \cap B_t| + o(\gamma_n)$$
so by (\ref{approximation of surface area}), (\ref{mollifier prop2}) holds.
From an integration by parts using the estimate $1/2 \leq g(Y, Y) \leq 2$, and (\ref{approximation of volume}),
\begin{align*}
\left|\int_\Omega Y(u_j - v_j) ~\vol\right|
&\leq \left|\int_{\partial \Omega} (u_j - v_j)g(Y, \normal) ~\vol_{\partial \Omega}\right| \\
&\lesssim \int_{\partial \Omega} |u_j - v_j| ~\vol_{\partial \Omega} \lesssim \gamma_n^2
\end{align*}
which gives (\ref{mollifier prop3}).
By the fact that $U_n$ has least perimeter, and (\ref{a priori estimate 1}, \ref{mollifier prop2}),
\begin{align*}
|\partial V_n \cap B_t| &\leq |\partial U_n \cap B_t| + o(\gamma_n) \\
&= \eta(U_n, B_t) + o(\gamma_n)\\
&\leq \eta(V_n, B_t) + \int_{\partial B_t} |u_n - v_n| ~\vol_{\partial B_t} + o(\gamma_n),
\end{align*}
so by (\ref{approximation of volume}), (\ref{mollifier prop1}) holds.

%%%%%%%%%%%%%%%%%%%%%%%%%%%%%%%%%%%%%%%%%%%%%%%%%%%%%%%%%%%%%%%%

\begin{proof}[Proof of (\ref{approximation of surface area})]
By Lemma \ref{Giusti72}, one has
\begin{equation}\label{Giusti 720a}
\limsup_{n \to \infty} \int_{B_t} |du_n| - |dw_n| ~\vol \leq \limsup_{n \to \infty} \int_{B_{t + \gamma_n^4} \setminus B_t} |du_n| ~\vol.
\end{equation}
If we define $\mu = \sum_n \gamma_n|du_n| ~\vol$, then $\mu(B_1) < \infty$, since $(\gamma_n) \in \ell^1$ and $|\partial^* U_n \cap B_1|$ is uniformly bounded (c.f. Proposition \ref{doubling dimension}).
The hypotheses of \cite[(7.20)]{Giusti77} are (\ref{Giusti 720a}) and the fact that $\mu$ is a finite Borel measure on $B_1$, and the conclusion is that almost surely,
\begin{equation}\label{Giusti 720b}
\limsup_{n \to \infty} \gamma_n^{-2} \int_{B_t} |dw_n| - |du_n| ~\vol \leq 0.
\end{equation}
From (\ref{MVT mollifier}, \ref{Giusti 720b}), one has (\ref{approximation of surface area}).
\end{proof}

\begin{proof}[Proof of (\ref{approximation of volume})]
Let
$$f_n(t) = \gamma_n^{-4} \int_{B_t} |u_n - w_n| ~\vol.$$
By Lemma \ref{Giusti72} and the fact that $U_j$ has least perimeter in $B_1$,
$$\limsup_{n \to \infty} f_n(t) \leq \limsup_{n \to \infty} \int_{B_1} |du_n| ~\vol \leq |\partial B_1|.$$
Moreover, $f_n$ is monotone.
This implies that almost surely, $f_n'(t)$ is uniformly bounded in $n$.
But
$$f_n'(t) = \gamma_n^{-4} \int_{\partial B_t} |u_n - w_n| ~\vol_{\partial B_t},$$
so
\begin{equation}\label{mollify cubic gamma}
\int_{\partial B_t} |u_n - w_n| ~\vol_{\partial B_t} \lesssim \gamma_n^4.
\end{equation}
We now set $z_n = \min(y_n, 1 - y_n)$ and estimate
\begin{align*}
\int_{\partial B_t} |u_n - v_n| ~\vol_{\partial B_t} &= |\partial B_t \cap U_n \Delta V_n| \\
&= |\partial B_t \cap V_n \setminus U_n| + |\partial B_t \cap U_n \setminus V_n| \\
&\leq \frac{y_n}{z_n} |\partial B_t \cap V_n \setminus U_n| + \frac{1 - y_n}{z_n} |\partial B_t \cap U_n \setminus V_n|.
\end{align*}
From definition of $V_n$, $w_n - u_n > y_n$ on $V_n \setminus U_n$ and $u_n - w_n > 1 - y_n$ on $U_n \setminus V_n$, so
\begin{align*}
\int_{\partial B_t} |u_n - v_n| ~\vol_{\partial B_t} &\leq z_n^{-1} \int_{\partial B_t \cap U_n \setminus V_n} |u_n - w_n| ~\vol_{\partial B_t} + z_n^{-1}\int_{\partial B_t \cap V_n \setminus U_n} |u_n - w_n| ~\vol_{\partial B_t} \\
&\leq z_n^{-1} \int_{\partial B_t} |u_n - w_n| ~\vol_{\partial B_t}.
\end{align*}
But
$$z_n^{-1} \leq \max(y_n^{-1}, (1 - y_n)^{-1}) \leq \max(a_n^{-1}, b_n^{-1}) \lesssim \gamma_n^{-2}$$
whence
$$\int_{\partial B_t} |u_n - v_n| ~\vol_{\partial B_t} \lesssim \gamma_n^{-2} \int_{\partial B_t} |u_n - w_n| ~\vol_{\partial B_t},$$
so by (\ref{mollify cubic gamma}), (\ref{approximation of volume}) holds.
\end{proof}

\section{Regularity of minimal hypersurfaces}\label{DeGiorgiSection}
In this section we prove specialize to $\Hyp^d$ and show regularity of hypersurfaces.

\subsection{Excess: First properties}
The quantity which we seek to estimate is the so-called excess of a function of locally bounded variation.
Notions of excess for $\RR^d$ appear in the monographs \cite[\S5.3.1]{federer2014geometric} and \cite[Chapter 6]{Giusti77}.

We introduce the coordinate system $(x, y, z) \in \RR_+ \times \RR^{d - 2} \times \RR$ with the hyperbolic half-space metric
\begin{equation}\label{hyperbolic metric}
g = \frac{dx^2 + dy_1^2 + \cdots + dy_{d - 2}^2 + dz^2}{x^2}.
\end{equation}
This does not agree with the usual convention for the hyperbolic half-space metric, but will be advantageous for our purposes when we represent $C^1$ hypersurfaces as graphs of functions $\omega: \RR_+ \times \RR^{d - 2} \to \RR$; then such a hypersurface can be written $\{z = \omega(x, y)\}$.
We also define the \dfn{hyperbolic origin} $O = (1, 0, 0)$ and write $B_r = B(O, r)$.
Whenever we sum over an index $\mu$, we mean that $\mu$ ranges over
$$\mu \in \{x, y_1, \dots, y_{d - 2}, z\}.$$

\begin{definition}
Let $u \in BV_l(\Hyp^d)$. The \dfn{approximate derivative} to $u$ at scale $n$ is the covector in $T_O' \Hyp^d$ defined by
$$\normal^{(n)}_\mu = \frac{\int_{B_{2^{-n}}} x \partial_\mu u ~\vol}{\int_{B_{2^{-n}}} |du| ~\vol}.$$
\end{definition}

Since $g(x\partial_\mu, x\partial_\nu) = \delta_{\mu\nu}$, $x\partial_\mu u = \normal_\mu$ and so, if $O$ is a Lebesgue point, then $\normal^{(n)} \to \normal_U(O)$ pointwise.
But $\Iso(\Hyp^d)$ acts transitively on $\Hyp^d$ and so the conjugation of $\normal^{(n)}$ to other cotangent spaces by $\Iso(\Hyp^d)$ defines a continuous section of $T'\Hyp^d$ defined on $\partial^* U$ which converges to $\normal_U$ pointwise $|du|\vol$-almost everywhere.
Moreover it is clear that the conjugation of $\normal^{(n)}$ to $T'_P\Hyp^d$ is continuous in $P$.
So to show $\normal$ is continuous, it suffices to show that $(\normal^{(n)})$ is locally uniformly Cauchy, and this goal is accomplished by the excess:

\begin{definition}
The \dfn{excess} of $u \in BV_l(\Hyp^d)$ at scale $n$ is
$$\Exc_n(u) = 2^{n(d - 1)} (\exp(2^{-n}) - |\normal^{(n)}|) \int_{B_{2^{-n}}} |du| ~\vol$$
where $|\normal^{(n)}|$ denotes the length of $\normal^{(n)}$ in $T_O' \Hyp^d$.
\end{definition}

To motivate the somewhat strange term $\exp(2^{-n})$ we observe that since $(dx^\mu)$ is an orthonormal basis of $T_O' \Hyp^d$ we have
$$|\normal^{(n)}|^2 = \sum_\mu \left(\int_{B_{2^{-n}}} \normal_\mu ~d\lambda_n\right)^2$$
where
$$\lambda_n(E) = \frac{\int_E |du|~\vol}{\int_{B_{2^{-n}}} |du| ~\vol}$$
for Borel subsets $E$ of $2^{-n}$, so $\lambda_n$ is a probability measure on $B_{2^{-n}}$.
Since $\lambda_n$ is a probability measure, the Cauchy-Schwarz inequality gives
$$|\normal^{(n)}|^2 \leq \int_{B_{2^{-n}}} \sum_\mu \normal_\mu^2 ~d\lambda_n = \int_{B_{2^{-n}}} x^2 ~d\lambda_n(x) \leq \sup_{(x, y, z) \in B_{2^{-n}}} x^2.$$
Writing $ds = \sqrt g$ for the line element in $\Hyp^d$, we have
$$2^{-n} = \int_1^x ds = \int_1^x \frac{dt}{t} = \log x$$
if $(x, y, z) \in \partial B_{2^{-n}}$ and $x$ is maximal possible. It follows that
\begin{equation}\label{conormal length bound}
|\normal^{(n)}| \leq \exp(2^{-n})
\end{equation}
and so in particular
$$\Exc_n(u) \geq 0.$$

From the law of cosines and (\ref{conormal length bound}),
\begin{align*}
|\normal^{(n)} - \normal^{(n + m)}|^2 &= |\normal^{(n)}|^2 + |\normal^{(n + m)}|^2 - 2 g(\normal^{(n)}, \normal^{(n + m)})\\
&\leq 2(\exp(2^{-2n}) - g(\normal^{(n)}, \normal^{(n + m)})) \\
&= \frac{2}{\int_{B_{2^{-(n + m)}}} |du| ~\vol} \int_{B_{2^{-(n + m)}}} \exp(2^{-2n}) |du| - \sum_\mu x \partial_\mu u \normal_\mu^{(n)} ~\vol.
\end{align*}
Applying (\ref{conormal length bound}) again,
$$\exp(2^{-2n}) |du| - \sum_\mu x \partial_\mu u \normal_\mu^{(n)} \geq 0$$
on $B_{2^{-n}}$, so it follows that
\begin{align*}
|\normal^{(n)} - \normal^{(n + m)}|^2 &\leq \frac{2}{\int_{B_{2^{-(n + m)}}} |du| ~\vol} \int_{B_{2^{-n }}} \exp(2^{-2n}) |du| - \sum_\mu x \partial_\mu u \normal_\mu^{(n)} ~\vol\\
&= \frac{2}{\int_{B_{2^{-(n + m)}}} |du| ~\vol} (\exp(2^{-2n}) - |\normal^{(n)}|^2) \int_{B_{2^{-n}}} |du| ~\vol.
\end{align*}
We can now apply the inequality
$$x^2 - y^2 \leq 4(x - y),$$
which is valid when $x \geq \max(1, y)$ (thus for $x = \exp(2^{-n})$ and $y = |\normal^{(n)}|$)
to deduce\footnote{An analogous estimate on the approximate derivative for a function on $\RR^d$, without the correction term $\exp(2^{-n}) - 1$, appears in \cite[pg661]{Miranda66}.}
\begin{equation}\label{excess bounds Cauchy sequence}
|\normal^{(n)} - \normal^{(n + m)}|^2 \leq \frac{2^{3 + n(1 - d)}}{\int_{B_{2^{-(n + m)}}} |du| ~\vol} \Exc_n(u).
\end{equation}

In addition to the estimate (\ref{excess bounds Cauchy sequence}), an essential property of the excess is rotation invariance, which we now formulate.

\begin{notation}
We define an action
\begin{equation}\label{hyperbolic rotation}
    \Phi: \Orth(\RR^d) \to \Iso(\Hyp^d)
\end{equation}
of the orthogonal group $\Orth(\RR^d)$, as well as the representation
\begin{align*}
\Orth(\RR^d) &\to \GL(BV_l(\RR^d))\\
A^* u(P) &:= u(\Phi(A)(P))
\end{align*}
of $\Orth(\RR^d)$ on $BV_l(\RR^d)$, as follows.
Recall that the hyperbolic metric in the ball model $\DD^d$ is radial, so the natural action of the orthogonal group $\Orth(\RR^d)$ on $\DD^d$ is by isometry, and if we use the Cayley transform to identify the origin of $\DD^d$ with $O$, then we obtain a faithful action (\ref{hyperbolic rotation}) whose orbits are the spheres $\partial B_r$ centered on $O$.
\end{notation}

\begin{lemma}\label{excess rotation invariant}
For every $A \in \Orth(\RR^d)$ and $u \in BV_l(\Hyp^d)$,
$$\Exc_n(A^* u) = \Exc_n(u).$$
\end{lemma}
\begin{proof}
Since $A$ is an isometry, $|(d A^* u)_{\Phi(A)(P)}| = |(du)_P|$, so
\begin{align*}
\int_{B_{2^{-n}}} |dA^*u| ~\vol &= \int_0^{2^{-n}} \int_{\partial B_r} |dA^* u| ~\vol_{\partial B_r} ~dr = \int_0^{2^{-n}} \int_{\partial B_r} |du| ~\vol_{\partial B_r} ~dr\\
&= \int_{B_{2^{-n}}} |du| ~\vol
\end{align*}
since $(A^{-1})^* \vol_{\partial B_r} = \vol_{\partial B_r}$.
Since $A$ is an isometry, $|A_* (x \partial_\mu)| = 1$ and $(A_*(x \partial_\mu))_\mu$ is an orthonormal basis of $T_O \Hyp^d$.
The claim follows.
\end{proof}

\subsection{Excess: Comparison to Plateau energy}
Our next task is to relate the excess of $1_U$ to the extrinsic geometry of $\partial U$\footnote{For an euclidean analogue, see \cite[TODO]{Miranda66}}.
To do this we will use the machinery of Appendices \ref{coarea section} and \ref{crossproducts} and the following lemma, which is an analogue of \cite[Theorem 4.8]{Giusti77} and allows us to represent $\partial U$ as a graph:

\begin{lemma}\label{hopfKilling}
Let $N$ be a $C^r$ hypersurface in $\Hyp^d$ which bounds an open set, $r \geq 1$, and
$$||\normal_N^\sharp - x\partial_z||_{L^\infty(N)} \leq \kappa^2$$
where $\kappa \in [0, 1)$.
Then there exists a scale $n^* \in \ZZ$ and a function
$$\omega \in C^r(\RR_+ \times \RR^{d - 2} \to \RR)$$
such that
\begin{align}
    N \cap B_{2^{-n^*}} &= \{(x, y, z) \in B_{2^{-n^*}}: z = \omega(x, y)\}, \label{N is a graph}\\
    ||d\omega||_{L^\infty(B_{2^{-n^*}})} &\leq \kappa. \label{derivative bounds}
\end{align}
\end{lemma}
\begin{proof}
We write $N = \partial U$, $u = 1_U$.
From the law of cosines and the fact that $\normal^\sharp$ and $x\partial_z$ both have unit length,
$$||\normal^\sharp - x\partial_z||_{L^\infty(N)}^2 = 2(1 - (\normal, x\partial_z)) = 2\left(1 - \frac{x\partial_z u}{|du|}\right).$$
Therefore
$$x\partial_z u \geq \left(1 - \frac{\kappa^2}{2}\right) |du|.$$
Given $\alpha$, a vector of unit length in $\RR_x \times \RR^{d - 2}_y \times (\RR_+)_z$, we can define a unit vector field
$$X_\alpha = \sum_\mu \alpha^\mu x\partial_\mu.$$
Then
$$\alpha^z x\partial_z u \geq \alpha^z \left(1 - \frac{\kappa^2}{2}\right) |du|$$
so from the Cauchy-Schwarz inequality and the fact that
$$1 - \left(1 - \frac{\kappa^2}{2}\right)^2 \leq \kappa^2,$$
we have
$$\left|\sum_{\mu \neq z} \alpha^\mu x\partial_\mu z\right| \leq \kappa\sqrt{1 - (\alpha^z)^2} |du|$$
and hence
$$X_\alpha u \geq \left(\alpha^z - \frac{\alpha^z \kappa^2}{2} - \kappa\sqrt{1 - (\alpha^z)^2}\right)|du| \geq (\alpha^z - \kappa)|du|.$$
So by Proposition \ref{Giusti46}, if $\alpha^z > \kappa$, then every integral curve of $X_\alpha$ passes through $\partial U$ exactly once.
But the integral curve of $X_\alpha$ which passes through $(0, y, z)$ is the euclidean line
$$t \mapsto (t \alpha^x + y_1 + t \alpha^{y_1}, \dots, y_{d - 2} + t \alpha^{y_{d - 2}}, t \alpha^z)$$
so the existence of a scale $n^*$ and a $C^r$ function $\omega$ satisfying (\ref{N is a graph}, \ref{derivative bounds}) follows from the $C^r$ implicit function theorem.
\end{proof}

We now pause to introduce a large amount of notation that we will use throughout Section \ref{DeGiorgiSection}.

\begin{definition}
The \dfn{hyperbolic Plateau energy} of a $1$-form $\psi$ on $(\RR_+)_x \times \RR^{d - 2}_y$ is the $d-1$-form
$$\Lagrange(\psi) = x^{1 - d} \sqrt{1 + |\psi|^2} ~dxd$$
where the metric is euclidean.
\end{definition}

\begin{notation}[hyperbolic space as a line bundle]\label{hyperbolic line bundle}
If $\omega$ is a $C^r$ function $\Omega \to \RR_z$ with graph $N \subseteq \Hyp^d$, where $\Omega \subseteq (\RR_+)_x \times \RR^{d - 2}_y$ is open, we introduce the locally closed $C^r$ embedding
\begin{align*}
    \Psi_N: \Omega &\to \Hyp^d \\
    (x, y) &\mapsto (x, y, \omega(x))
\end{align*}
which identifies $\Omega$ with $N$.
We also introduce the projection
\begin{align*}
    \Pi: \Hyp^d &\to (\RR_+)_x \times \RR^{d - 2}_y\\
    (x, y, z) &\mapsto (x, y)
\end{align*}
of which $\Psi_N$ is a section.
\end{notation}

It follows immediately from Example \ref{graphs in riemannian manifolds} that
\begin{equation}\label{Lagrangian formula}
\Psi_N^* \vol_N = \Lagrange(d\omega)
\end{equation}
if $N$ is the graph of $\omega \in C^1(\Omega)$.

\begin{notation}[tensor fields on euclidean space]
Since $\RR_+ \times \RR^{d - 2}$ has flat Levi-Civita connection, we can identify tensor fields on $\RR_+ \times \RR^{d - 2}$ with smooth maps $\RR_+ \times \RR^{d - 2} \to \Hilb$ for some finite-dimensional Hilbert space $\Hilb$.
Using the Bochner integral (see Appendix \ref{coarea section}), it therefore makes sense to integrate a tensor field to obtain an element of $\Hilb$ (that is, a tensor), and we write
$$A_n T = \dashint_{\Pi(B_{2^{-n}})} T$$
for the Bochner mean of $T$ over $\Pi(B_{2^{-n}})$ with respect to Lebesgue measure.
It also makes sense to identify a tensor with a constant tensor field, which we will do without mention.
\end{notation}

\begin{notation}[miscellany]
We write $\tilde B_r$ for the euclidean ball in $\RR_+ \times \RR^{d - 2}$ centered on $(1, 0, \dots, 0)$ of radius $r > 0$.
We write $x[a,b] = [ax, ab]$ and $x + [a, b] = [a + x, b + x]$.
\end{notation}

\begin{lemma}\label{excess vs plateau energy}
Let $\omega \in C^1(\Omega)$. Then
\begin{equation}
    \Exc_n(1_{\omega(x, y) < z}) \in 2^{n(d - 1)} \exp(2^{-n}) \int_{\Pi(B_{2^{-n}})} \Lagrange(d\omega) - [\exp(-2^{-2n}), 1]\Lagrange(A_n d\omega). \label{excess vs Lagrangian}
\end{equation}
\end{lemma}
\begin{proof}
Let $u(x, y, z) = 1_{\omega(x, y) < z}$, let $N$ be the graph of $\omega$, and let $\normal^{(n)}$ be the approximate derivative of $u$.
It follows from the definitions, the fact that $|du|$ is the Radon-Nikod\'ym derivative $\vol_N/\vol$, and (\ref{Lagrangian formula}), that
$$\normal^{(n)}_\mu = \int_{B_{2^{-n}}} \normal_\mu |du| ~\vol = \int_{B_{2^{-n}} \cap N} \normal_\mu \vol_N = \int_{\Pi(B_{2^{-n}})} (\Psi_N^* \normal)_\mu \Lagrange(d\omega).$$
From Example \ref{graphs in riemannian manifolds},
$$(\Psi^* \normal)_z \Lagrange(d\omega) = x^p ~dxdy$$
where $p = 3/2 - d$. Similarly we have
$$(\Psi^* \normal)_i \Lagrange(d\omega) = x^p \partial_i \omega(x, y) ~dxdy$$
whenever $i \in \{x, y_1, \dots, y_{d - 2}\}$.
Therefore
\begin{align*}
    |\normal^{(n)}|^2 \left(\int_{B_{2^{-n}}} |du| ~\vol\right)^2 &= \left(\int_{\Pi(B_{2^{-n}})} x^p ~dxdy\right)^2 + \sum_i \left(\int_{\Pi(B_{2^{-n}})} x^p \partial_i \omega(x, y) ~dxdy \right)^2 \\
    &\in |\Pi(B_{2^{-n}})|^2 (1 + |A_nd\omega|^2) \left[\inf_{(x, y, z) \in B_{2^{-n}}} x^{2p}, \sup_{(x, y, z) \in B_{2^{-n}}} x^{2p}\right] \\
    &= \left(\int_{\Pi(B_{2^{-n}})} \Lagrange(A_nd\omega)\right)^2 \left[\inf_{(x, y, z) \in B_{2^{-n}}} x, \sup_{(x, y, z) \in B_{2^{-n}}} x\right].
\end{align*}
From the interval arithmetic identity
$$[\alpha x, \alpha x] - [\alpha^{-1}y, \alpha y] = \alpha(x - [\alpha^{-2}, 1]y),$$
valid for $\alpha \geq 1$ (and hence $\alpha = \exp(2^{-n})$),
we conclude (\ref{excess vs Lagrangian}).
\end{proof}

%%%%%%%%%%%%%%%%%%%%%%%%%%%%%%%%%%%%%%%%%%%%%%%%%

\subsection{de Giorgi lemma: Comparison to Dirichlet energy}
The main idea in the proof of regularity is to iterate the de Giorgi lemma, a certain estimate on the excess, which in the euclidean case is given by \cite[TODO]{Miranda66}.
The proof of the de Giorgi lemma proceeds in three steps: first, one approximates the Plateau energy by the Dirichlet energy and applies the mean-value property of harmonic functions; second, one shows that the de Giorgi lemma holds for $C^1$ functions using an estimate similar to Lemma \ref{excess vs plateau energy}; and finally, one shows that the de Giorgi lemma holds in general using an estimate similar to Proposition \ref{mollifier quant} and rotation-invariance of the excess.

\begin{notation}
Let $\DirL(\psi)$ denote the Dirichlet energy of a $1$-form on $(\RR_+)_x \times \RR^{d - 2}_y$, which is the $d-1$-form
$$\DirL(\psi) = \frac{|\psi|^2}{2} ~dxdy,$$
where as usual the norm is euclidean.
Thus $dh$ is a minimizer of $\DirL$ iff $h$ is harmonic.
\end{notation}

To compare the Dirichlet and Plateau energies, let $\psi_1, \psi_2$ be $1$-forms.
We suppress the factor of $dxdy$.
From Taylor's theorem, there exists $\xi \in [|\psi_1|, |\psi_2|]$ such that
\begin{equation}\label{Taylor remainder Dirichlet}
\Lagrange(\psi_1) - \Lagrange(\psi_2) = x^{1 - d}\left(\frac{|\psi_1|^2 - |\psi_2|^2}{2\sqrt{1 + |\psi_2|^2}} - \frac{(|\psi_1|^2 - |\psi_2|^2)^2}{8(1 + \xi^2)^{3/2}}\right).
\end{equation}
The second term of (\ref{Taylor remainder Dirichlet}) is negative and $\sqrt{1 + |\psi_2|^2} \geq 1$, so it follows that
\begin{equation}\label{Taylor lower bound}
\Lagrange(\psi_1) - \Lagrange(\psi_2) \leq x^{1 - d} (\DirL(\psi_1) - \DirL(\psi_2)).
\end{equation}
If in addition $||\psi_2||_{L^\infty} \leq 1$, then
$$4\sqrt{1 + |\psi_2|^2} \leq 8 \leq 8(1 + \xi^2)^{3/2}$$
so we conclude
\begin{equation}\label{Taylor upper bound}
\Lagrange(\psi_1) - \Lagrange(\psi_2) \geq \frac{x^{1 - d}}{\sqrt{1 + |\psi_2|^2}} (\DirL(\psi_1) - \DirL(\psi_2) - (\DirL(\psi_1) - \DirL(\psi_2))^2).
\end{equation}

The utility of the Dirichlet energy is that for every harmonic function $h$, by \cite[Lemma 4.1]{Miranda66} (ultimately a consequence of the mean-value property),
\begin{equation}\label{Miranda41}
\int_{\tilde B_{2^{-(n+1)}}} \DirL(dh) - \DirL(A_n dh) \leq 2^{-(d + 1)} \int_{\tilde B_{2^{-n}}} \DirL(dh) - \DirL(A_n dh).
\end{equation}
Moreover, the mean-value property of $dh$ implies that for every $1$-form $\psi$,
\begin{equation}\label{MVP derivative}
\int_{\tilde B_{2^{-n}}} \DirL(dh - \psi) = \int_{\tilde B_{2^{-n}}} \DirL(dh) - \DirL(\psi).
\end{equation}

We are now ready to complete the first step of the proof of the de Giorgi lemma, analogous to \cite[Teorema 4.3]{Miranda66}.
On first reading, the reader may take $c = 10^{-3}$, $O(c) = 10^{-1}$ and $n^* = 10$; in fact, $c$ will later be chosen to be a dimensional constant.
Roughly speaking, one should think of $\beta$ as comparable to $2^{-(n - n^*)}$, and $\kappa$ as the ``error incurred by mollification".

\begin{lemma}\label{DGL1}
For every $c > 0$ there exists a scale $n^* \in \ZZ$ with the following property:

Let $\omega \in C^1(\Omega)$, suppose that $\kappa, \beta \in (0, 1)$ and $n \geq n^*$ satisfy
\begin{align}
||d\omega||_{L^\infty(\tilde B_{2^{-n}})} &\leq \kappa, \label{DGL1 1}\\
\int_{\tilde B_{2^{-n}}} \Lagrange(d\omega) - \Lagrange(A_n d\omega) &\leq \beta, \label{DGL1 2}\\
\int_{\tilde B_{2^{-n}}} \Lagrange(d\omega) &\leq \eta(\{(x, y, z) \in \Hyp^d: z < \omega(x, y)\}, 2^{-n}) + \beta \kappa. \label{DGL1 3}
\end{align}
Then
$$\int_{\tilde B_{2^{-(n + 1)}}}\Lagrange(d\omega) - \Lagrange(A_{n + 1} d\omega) \leq (1 + O(c)) 2^{-(d + 1)} \beta + O(\beta \sqrt \kappa)$$
where all constants only depend on $d$.
\end{lemma}
\begin{proof}
Choose $n^*$ so large that
\begin{equation}\label{x to c}
1 - c \leq \inf_{(x, y) \in \tilde B_{2^{-n^*}}} x^{1 - d} < \sup_{(x, y) \in \tilde B_{2^{-n^*}}} x^{1 - d} \leq 1 + c
\end{equation}
and suppose that $n \geq n^*$.
Let $h$ be the harmonic function on $\tilde B_{2^{-n}}$ such that
\begin{equation}\label{trace equation}
h = \omega \text{ on } \partial \tilde B_{2^{-n}}.
\end{equation}
By definition of $\eta$ and (\ref{DGL1 3}),
\begin{align*}
\int_{\tilde B_{2^{-(n + 1)}}} \Lagrange(d\omega) - \Lagrange(dh)
&\leq \int_{\tilde B_{2^{-n}}} \Lagrange(d\omega) - \eta(\{(x, y, z) \in \Hyp^d: z < \omega(x, y)\}, 2^{-n}).
\end{align*}
Therefore
\begin{equation}\label{bound on domega - dh}
\int_{\tilde B_{2^{-(n + 1)}}} \Lagrange(d\omega) - \Lagrange(dh) \leq \beta\kappa.
\end{equation}

We now follow the proof of \cite[Lemma 4.2]{Miranda66}.
By (\ref{Taylor lower bound}, \ref{x to c}),
$$\int_{\tilde B_{2^{-(n + 1)}}} \Lagrange(d\omega) - \Lagrange(A_{n + 1}d\omega) \leq (1 + c)\int_{\tilde B_{2^{-(n + 1)}}} \DirL(d\omega) - \DirL(A_{n + 1}d\omega).$$
Since $A_{n + 1}d\omega$ is the mean of $d\omega$, for every $\varepsilon \in (0, 1)$,
\begin{align*}
\int_{\tilde B_{2^{-(n + 1)}}} \DirL(d\omega) - \DirL(A_{n + 1}d\omega)
&\leq \int_{\tilde B_{2^{-(n + 1)}}} \DirL(d\omega - A_nd\omega) \\
&\leq (1 + \varepsilon^{-1}) \int_{\tilde B_{2^{-(n + 1)}}} \DirL(d(\omega - h)) \\
&\qquad +(1 + \varepsilon) \int_{\tilde B_{2^{-(n + 1)}}} \DirL(dh - A_nd\omega) \\
&=: O(\varepsilon^{-1}) I + (1 + \varepsilon) J.
\end{align*}
From the positivity of Dirichlet energy and (\ref{MVP derivative}, \ref{Taylor upper bound}),
\begin{align*}
I &\leq \int_{\tilde B_{2^{-n}}} \DirL(d(\omega - h)) = \int_{\tilde B_{2^{-n}}} \DirL(d\omega) - \DirL(dh) \lesssim \int_{\tilde B_{2^{-n}}} \Lagrange(d\omega) - \Lagrange(dh)
\end{align*}
so by (\ref{bound on domega - dh}),
\begin{equation}\label{bound on I}
I \lesssim \beta\kappa.
\end{equation}
Moreover, by (\ref{MVP derivative}),
$$J = \int_{\tilde B_{2^{-(n + 1)}}} \DirL(dh) - \DirL(A_nd\omega).$$
From (\ref{trace equation}) and Stokes' theorem, there are constants $C_m > 0$ such that
$$A_m d\omega = C_m \int_{\partial \tilde B_{2^{-m}}} \omega ~dS = C_m \int_{\partial \tilde B_{2^{-m}}} h ~dS = A_m dh$$
which along with (\ref{Miranda41}, \ref{MVP derivative}) implies that
$$J \leq 2^{-(d + 1)} \int_{\tilde B_{2^{-n}}} \DirL(dh) - \DirL(A_nd\omega) = 2^{-(d + 1)} \int_{\tilde B_{2^{-n}}} \DirL(dh - A_n d\omega).$$
We further estimate, using (\ref{bound on I}),
\begin{align*}
J &\leq (1 + \varepsilon) 2^{-(d + 1)} \int_{\tilde B_{2^{-n}}} \DirL(d\omega - A_n d\omega) + O(\varepsilon^{-1} I) \\
&:= (1 + \varepsilon) 2^{-(d + 1)} K + O(\varepsilon^{-1} \beta \kappa).
\end{align*}
To estimate $K$ we apply (\ref{MVP derivative}, \ref{x to c}, \ref{Taylor upper bound}) to obtain
\begin{align*}
K &= \int_{\tilde B_{2^{-n}}} \DirL(d\omega) - \DirL(A_n d\omega) \\
&\leq (1 + O(c)) \int_{\tilde B_{2^{-n}}} \Lagrange(d\omega) - \Lagrange(A_n d\omega) + O(1) \int_{\tilde B_{2^{-n}}} (\DirL(d\omega) - \DirL(A_n d\omega))^2.
\end{align*}
From (\ref{DGL1 1}, \ref{DGL1 2}), it follows that
\begin{align*}
K &\leq (1 + O(c))\beta + O(||d\omega||_{L^\infty(\tilde B_{2^{-n^*}})}) \int_{\tilde B_{2^{-n}}} \DirL(d\omega) - \DirL(A_n d\omega)\\
&\leq \beta(1 + O(c + \kappa)).
\end{align*}
If we set $\varepsilon = \sqrt \kappa$ then it follows that
\begin{align*}
\int_{\tilde B_{2^{-(n+1)}}} \DirL(d\omega) - \DirL(A_n d\omega) &\leq (1 + O(c)) 2^{-(d + 1)} \beta + O(\beta \sqrt \kappa). \qedhere
\end{align*}
\end{proof}

%%%%%%%%%%%%%%%%%%%%%%%%%%%%%%%%%%%%%%%%%%%%%%%%%

\subsection{de Giorgi lemma: \texorpdfstring{$C^1$}{C1} and general cases}
The next step is the $C^1$ case, which is analogous to \cite[Teorema 4.4]{Miranda66}.

\begin{lemma}[de Giorgi lemma on $\Hyp^d$, $C^1$ case]\label{DGL2}
For every $c > 0$ there exists a scale $n^* \in \ZZ$ with the following property:

Let $N$ be a $C^1$ hypersurface in $B_{2^{-n}}$, $n \geq n^*$, with unit normal field $\normal^\sharp$, such that $N$ bounds an open set $U$.
If $\kappa \in (0, 1), \alpha \in \RR_+$ are parameters such that
\begin{align}
\Exc_n(U) &\leq \alpha, \label{DGL2 1}\\
|N \cap B_{2^{-n}}| &\leq \eta(U, B_{2^{-n}}) + 2^{n(1 - d)}\alpha \kappa, \label{DGL2 2}\\
||\normal^\sharp - x\partial_z||_{L^\infty(N \cap B_{2^{-n}})} &\leq \kappa^2, \label{DGL2 3}
\end{align}
then
$$\Exc_{n + 1}(U) \leq \frac{1 + O(c)}{2} \alpha + O(\alpha \sqrt \kappa) + O(4^{-n}).$$
\end{lemma}
\begin{proof}
From Lemma \ref{hopfKilling} and (\ref{DGL2 3}), there exists $n_1 \in \ZZ$ and a function $\omega \in C^1(\RR_+ \times \RR^{d - 2})$ satisfying the derivative bound (\ref{DGL1 1}) for any $n \geq n_1$
and such that the graph of $\omega$ over $\Pi(B_{2^{-n_1}})$ is $N \cap A_1$.
Let $\varepsilon > 0$; then we can find a scale $n_2 \geq n_1$ such that if $n \geq n_2$ then
$$\Pi(B_{(1 - \varepsilon) 2^{-(n+1)}}) \subseteq \tilde B_{2^{-(n+1)}} \text{ and } \tilde B_{2^{-n}} \subseteq \Pi(B_{(1 + \varepsilon) 2^{-n}}).$$
Up to a multiplicative loss of $1 + c$, all quantities defined for $(1 - \varepsilon)2^{-(n + 1)}$ can be replaced with $2^{-(n + 1)}$; similarly for $2^{-n}$ and $(1 + \varepsilon) 2^{-n}$ (TODO: Justify this) for $\varepsilon$ chosen small enough depending on $c$.
In order to apply Lemma \ref{DGL1} we apply Lemma \ref{excess vs plateau energy} and (\ref{DGL2 1}) to obtain
$$\int_{\tilde B_{2^{-n}}} \Lagrange(d\omega) - \Lagrange(A_n d\omega) \leq (1 + c)2^{n(1 - d)}\alpha.$$
Similarly from (\ref{Lagrangian formula}, \ref{DGL2 2}), we obtain
$$\int_{\tilde B_{2^{-n}}} \Lagrange(d\omega) \leq \eta(U, B_{2^{-n}}) + (1 + c) 2^{n(1-d)}\alpha \kappa,$$
and so we have met the hypotheses of Lemma \ref{DGL1} with
$$\beta = (1 + c)2^{n(1 - d)}\alpha.$$
Therefore there exists $n_3 \geq n_2$ such that for every $n \geq n_3$,
$$\int_{\tilde B_{2^{-(n+1)}}} \Lagrange(d\omega) - \Lagrange(A_nd\omega) \leq 2^{(n + 1)(1-d)}\left[ \frac{1 + O(c)}{2} \alpha + O(\alpha \sqrt \kappa)\right].$$
To convert this estimate back into a result about the excess we use Lemma \ref{excess vs plateau energy}.
There exists a scale $n^* \geq n_3$ such that if $n \geq n_4$ then $\exp(2^{-(n + 1)}) \leq 1 + c$, so
\begin{align*}
\Exc_{n + 1}(U) &\leq (1 + O(c)) 2^{(n+1)(d-1)} \int_{\tilde B_{2^{-(n+1)}}} \Lagrange(d\omega) - \exp(-2^{-n}) \Lagrange(A_nd\omega)\\
&\leq O(4^{-n}) + (1 + O(c)) 2^{(n+1)(d-1)} \int_{\tilde B_{2^{-(n+1)}}} \Lagrange(d\omega) - \Lagrange(A_nd\omega)\\
&\leq O(4^{-n}) + \frac{1 + O(c)}{2} \alpha + O(\alpha \sqrt \kappa). \qedhere
\end{align*}
\end{proof}

We now prove the de Giorgi lemma.
The proof settles the choice of the dimensional constant $c$.

\begin{proposition}[de Giorgi lemma on $\Hyp^d$, general case]
There exist $\sigma, n^*, C > 0$ such that for every set $U$ of least perimeter in $B_{2^{-n}} \subseteq \Hyp^d$,
$$\Exc_n(U) < \sigma \text{ and } n \geq n^* \implies \Exc_{n+1}(U) \leq \frac{51}{100} \Exc_n(U) + \frac{C}{4^n}.$$
\end{proposition}
\begin{proof}
By Lemma \ref{excess rotation invariant} and the fact that $x\partial_z$ is a unit vector field, it is no loss of generality to assume that
$$\normal^{(n)}_U = |\normal^{(n)}_U| x\partial_z.$$
Under this assumption, if we write $u = 1_U$ then
$$\Exc_n(U) = 2^{n(d - 1)} \exp(2^{-n}) \int_{B_{2^{-n}}} |du| ~\vol - 2^{n(d - 1)} \int_{B_{2^{-n}}} x\partial_z u ~\vol.$$
The injectivity radius of $O$ is infinite and so, if $\sigma < \exp(2^{-n}) \gamma_*$, we obtain from Proposition \ref{mollifier quant}, for every $\kappa > 0$, a $C^1$ hypersurface $N$ which bounds an open set $V$ which, if $n$ is chosen large enough (TODO: Justify taking $t \to 1$, $\varepsilon \to 0$), satisfies
\begin{align*}
|N \cap B_{2^{-n}}| &\leq \eta(V, B_{2^{-n}}) + \kappa \Exc_n(U, B_{2^{-n}}) \\
\Exc_n(V, B_{2^{-n}}) &\in \Exc_n(U, B_{2^{-n}})[1 - \kappa, 1 + \kappa] \\
||\normal^\sharp_N - x\partial_z||_{L^\infty} &\leq \kappa.
\end{align*}
Therefore by Lemma \ref{DGL2}, there exist dimensional constants $C_1,C_2,C > 0$ such that
$$\Exc_{n + 1}(V) \leq \frac{1 + C_1c}{2} \Exc_n(U, B_{2^{-n}}) + C_2 \Exc_n(U, B_{2^{-n}}) \sqrt \kappa + \frac{2C}{4^n}.$$
We now choose $c = C_1/50$, which yields a constant $C_3 > 0$ such that
$$\Exc_n(U, B_{2^{-n}}) \leq \frac{51}{100} \Exc_n(U, B_{2^{-n}}) + C_3 \Exc_n(U, B_{2^{-n}}) \sqrt \kappa + \frac{2C}{4^n}.$$
We can then choose $\kappa$ small enough that
$$C_3 \Exc_n(U, B_{2^{-n}}) \sqrt \kappa \leq \frac{C}{4^n}$$
to complete the proof.
\end{proof}

%%%%%%%%%%%%%%%%%%%%%%%%%%%%%%%%%%%%%%%%%%%%%%%%%

\subsection{Induction on scale}


%%%%%%%%%%%%%%%%%%%%%%%%%%%%%%%%%%%%%%%%%%%%%%%%%

\subsection{Removing the \texorpdfstring{$C^1$}{C1} assumption}\label{proof of DGL}
We are now ready to prove Proposition \ref{de Giorgi lemma}, and also settle the choice of $c > 0$.
By local homogeneity, if we can find $\sigma = \sigma(P)$, then $\sigma$ is locally constant and hence constant, so we may restrict to any particular $P \in M$.
Let $U$ be a set of locally finite perimeter, $u = 1_U$, and $\Psi' = (K_\mu')$ an orthonormal Killing frame.
Let $v_r = a^{-1}\int_{B_\rho} K_\mu' u/|K_\mu'| ~\vol \evect^\mu$, $a = |\int_{B_\rho} K_\mu' u/|K_\mu'| ~\vol \evect^\mu|$, so $v \in \RR^d$ has unit length and we can find an orthogonal matrix $O$ with $Ov = \evect^0$.
We then define $\Psi = (K_\mu)$ where $K_\mu = O^\nu_\mu K_\nu'$, so that
$$\int_{B_\rho} K_\mu u ~\vol = O^\nu_\mu \int_{B_\rho} K_\nu' u ~\vol = aO^\nu_\mu v_\nu = a\delta_\mu^0.$$
Suppose that $R$ is small enough that $1/2 \leq g(\partial_\mu', \partial_\mu') \leq 2$ on $B_R$.

Let $\gamma^* > 0$ be the constant from Proposition \ref{mollifier quant} with $\delta = \rho$, and suppose that $\sigma \leq \gamma^*$ and $R$ is less than the injectivity radius of $M$.
Then for almost every $t \in (0, \rho)$, there exists a set $V \subseteq B_t$ with $C^1$ perimeter and satisfying (\ref{mollifier quant1}, \ref{mollifier quant2}, \ref{mollifier quant3}, \ref{mollifier quant4}) with $X = \partial_0'$.
It follows that $V$ satisfies (\ref{DGLC1 normal points up}, \ref{DGLC1 almost minimal}) with $B_\rho$ replaced by $B_t$ and $\beta \geq \rho^{d - 1} \gamma$, $\kappa = \varepsilon^{1/2}$.
From (\ref{mollifier quant2}, \ref{mollifier quant3}),
$$|\Lambda(U, t) - \Lambda(V, t)| \lesssim \varepsilon \rho^{d - 1} \gamma.$$
Therefore, (\ref{DGLC1 small excess}) also holds for $V, B_t$ with
$$\beta = \rho^{d - 1} \gamma(1 + O(\varepsilon)).$$
Let $\rho^* > 0$ be the constant from Lemma \ref{DGLC1} and suppose that $R < \rho^*$.
Then
$$\Lambda(V, \rho/2) \leq \frac{1 + O(c + c\varepsilon^{1/4})}{2^{d + 1}} \beta,$$
so putting everything together, we find an absolute constant $b > 0$ such that if $t < \rho$ is large enough depending on $\varepsilon$,
$$\Lambda(U, \rho/2) \leq \frac{1 + b(c + c\varepsilon^{1/4} + \varepsilon)}{2^{d + 1}} \Lambda(U, \rho/2).$$
We may now choose $c, \varepsilon$ so small depending on $b$ that
$$\Lambda(U, \rho/2) \leq 2^{-d} \Lambda(U, \rho)$$
which was to be shown.

\section{Applications to $\infty$-harmonic functions} \label{proof of main thm}

Now we prove Theorem \ref{infinity harmonic laminations}.
Let $u$ be an $\infty$-harmonic function on a closed hyperbolic surface $M = \Hyp^2/\Gamma$, and let $L(x)$ be its local Lipschitz constant at $x$,
$$L(x) = \lim_{r \to 0} \sup_{d(x, y) < r} \frac{|u(x) - u(y)|}{d(x, y)}.$$
Let $\lambda_u(\Omega)$ be the set of maximal stretch of $u$ on a convex set $\Omega$, defined by
$$\lambda_u(\Omega) = \{x \in \Omega: L(x) = \sup_\Omega L\}.$$
Since $L$ is upper-semicontinuous and $M$ is compact, if $\Omega$ is closed then $\lambda_u(\Omega)$ is nonempty and closed.
In particular this holds for $M = \Omega$, since $M$ is the image of the fundamental domain of the Fuchsian group $\Gamma$ in $\Hyp^2$ and therefore is convex.

Let $v: M \to E$ be the harmonic conjugate of $u$; then $v$ is a section of least gradient and hence $\supp dv$ is a geodesic lamination by Theorem \ref{main thm}.
By \cite[Theorem 6.1]{daskalopoulos2020transverse}, $\supp dv \subseteq \lambda_u(M)$.
Now let $x \in \lambda_u(M) \setminus \supp dv$. So there is a convex open set $\Omega$ containing $x$ on which $dv = 0$. TODO: Check the scaling $k_p$ and if it works show that $du = 0$ on $\Omega$.

% \appendix \section{Bochner integration and coarea} \label{coarea section}
% Let $F$ be a separable Fr\'echet space over $\CC$, and $(\Omega, P)$ a measure space.
% We can then define the \dfn{Bochner integral} of a $P$-measurable function $f: \Omega \to F$, which we write as $\int_\Omega f ~dP \in F$.
% See \cite{Rieffel70}, \cite{MO47721}, and \cite[Chapter V]{yosida2012functional}.
% We recall a few facts that we will need:

% \begin{theorem}[Pettis]
% Let $f: \Omega \to F$ be any function.
% Then $f$ is $P$-measurable iff for every $X \in F'$, $\omega \mapsto \langle f(\omega), X\rangle$ is $P$-measurable.
% In this case,
% $$\left\langle \int_\Omega f ~dP, X\right\rangle = \int_\Omega \langle f(\omega), X\rangle ~dP(\omega).$$
% If $F = \CC^r$, then the Bochner integral is just the componentwise Lebesgue integral.
% \end{theorem}

% Throughout this section we consider the superlevel sets $E_y = \{u > y\}$ of a function $u \in BV_l(M)$, and the resulting $T'M$-valued Radon measures
% $$\omega(y) = d1_{E_y} ~\vol.$$
% Let $\mu = |du| ~\vol$.

% We first observe that for every $X \in \mathcal D(M, TM)$,
% $$\langle \omega(y), X\rangle = -\int_{E_y} \mathcal L_X\vol$$
% is measurable in $y$, since $E_y$ is monotone in $y$.
% So by Pettis' theorem, $\omega$ is measurable in $y$ with respect to the weak topology of measures.

% \begin{lemma}[coarea formula for measures]\label{Coarea1}
% One has
% $$du ~\vol = \int_{-\infty}^\infty \omega(y) ~dy.$$
% \end{lemma}
% \begin{proof}
% Fix $X \in C_c(M, TM)$. We may assume that $u \geq 0$, and we must show
% \begin{equation}
% \label{gradient is integral of fibers}
% \int_M (du, X) ~\vol = \int_{-\infty}^\infty \langle \omega(y), X\rangle ~dy.
% \end{equation}
% Since $u \geq 0$,
% \begin{align*}
% \int_M (du, X) ~\vol &= -\int_M u~\mathcal L_X\vol = -\int_M \int_0^{u(x)} dy ~\mathcal L_X\vol\\
% &= -\int_0^\infty \int_{E_y} ~\mathcal L_X\vol ~dy = \int_0^\infty \langle \omega(y), X\rangle ~dy.
% \end{align*}
% by Fubini's theorem.
% If $y < 0$ then $1_{E_y} = 1$ so $\omega(y) = 0$, so we conclude (\ref{gradient is integral of fibers}).
% \end{proof}

% Let $p: L \to M$ be the trivial line bundle with its induced metric $h$, and let $\eta$ be the volume form induced by $h$.
% If $W$ is a vector field on $L$, we will write $W_1$ for the projection of $W$ onto $M$ and $W_2$ for its projection onto $\CC$.
% Then Cartan's magic formula implies that if $W_2$ is constant, then
% \begin{equation}
% \label{Lie derivative computation}
% \mathcal L_W\eta = \mathcal L_W\vol \wedge dy.
% \end{equation}

% \begin{lemma}\label{coarea converse}
% Suppose that $W \in \mathcal D(L, TL)$ depends on a parameter $n \in \NN$, such that $W_2 = 0$ and for every $y \in \RR$, $X = W_1(\cdot, y)$ is a maximizing sequence for $\langle \omega(y), X\rangle$ subject to $X \prec U$.
% Then
% $$\int_{-\infty}^\infty \langle \omega(y), W(y)\rangle ~dy \leq \mu(U).$$
% \end{lemma}
% \begin{proof}
% Let
% $$E = \{(x, y) \in L: x \in E_y\}$$
% be the undergraph of $u$.
% By Fubini's theorem and (\ref{Lie derivative computation}),
% \begin{align*}
% \int_{-\infty}^\infty \langle \omega(y), W(y)\rangle ~dy &= -\int_{-\infty}^\infty \int_{E_y} \mathcal L_W\vol ~dy = -\iint_E ~\mathcal L_W\eta = \int_M (d1_E, W) ~\vol.
% \end{align*}

% Let $(u_m)$ be a mollification of $u$, so that $u_m \to u$ in the weak topology of distributions.
% Then if $\chi$ is a cutoff, $\langle u_m, \chi\rangle \to \langle u, \chi\rangle$; taking a sequence of $\chi$ which increase to the indicator function of a compact set $K$, we conclude that $u_m \to u$ in $L^1(K)$, and hence $u_m \to u$ in $L^1_l$.

% Let $E^{(m)}$ be the undergraph of $u_m$, and $E^{(m)}_y = \{u_m > y\}$.
% Then for every test function $v$,
% \begin{align*}
% \langle 1_{E^{(m)}} - 1_E, v\rangle &= \int_{E^{(m)} \Delta E} v ~\vol \leq |(E^{(m)} \Delta E) \cap (\supp v \times \RR)| \cdot ||v||_{L^\infty}\\
% &\leq ||v||_{L^\infty} \int_{\supp v} |u_m - u| ~\vol \to 0
% \end{align*}
% so $1_{E^{(m)}} \to 1_E$ in the weak topology of distributions.
% Therefore
% $$\lim_{m \to \infty} \int_M (d1_{E^{(m)}}, W) ~\vol = \int_M (d1_E, W) ~\vol.$$

% Since $u_m$ is smooth, its graph $F_m = \partial E^{(m)}$ is a smooth manifold.
% Let $\nu_m$ be the upwards unit normal field of $F_m$ and let $\vol_m$ be the volume form on $F_m$ induced by $\eta$.
% Then
% $$\langle d 1_{E^{(m)}}, W\rangle = -\iint_{E^{(m)}} \mathcal L_W\eta = -\int_{F_m} h(\nu_n, W) ~\vol_m.$$
% Let $q_m = p|F_m$ and $Y_m$ be the vector field $(Y_m)_1 = -d u_m$, $(Y_m)_2 = 1$.
% Since $F_m$ is a graph, $q_m: F_m \to M$ is a diffeomorphism, $(q_m)_*\nu_m = Y_m/|Y_m|$, and
% $$(q_m)_* \vol_m = |Y_m| ~\vol.$$
% Therefore
% $$\int_{F_m} h(\nu_n, W) ~\vol_m = \int_M h(Y_m, W) ~\vol = \int_M g(d u_m, W_1) ~\vol = \int_M (d u_m, W_1) ~\vol.$$
% Thus
% \begin{align*}
% |\langle d 1_E, W\rangle| &= \lim_{m \to \infty} |\langle d u_m, W_1\rangle| \leq \mu(U). \qedhere
% \end{align*}
% \end{proof}


% \begin{proof}
% By Lemma \ref{Coarea1} and the triangle inequality,
% $$\mu \leq \int_{-\infty}^\infty |\omega(y)| ~dy.$$
% So we just need to prove the converse.

% Let $U \Subset M$.
% Suppose that for every $y \in \RR$, $X = X^{(n)}_y$ is a maximizing sequence for $\langle \omega(y), X\rangle$ subject to $X \prec U$.
% Since $\omega$ with respect to the weak topology of measures, for every $n$, $X^{(n)}_y(x)$ can be chosen to be measurable in $(x, y)$; indeed, we can take $X^{(n)}_y$ to be a smooth approximation to the Radon measure $d 1_{E_y}/|d 1_{E_y}|_{TV}$ in the weak topology of distributions, which is a product of the measurable functions $\omega$ and $y \mapsto 1/|\omega(y)|$.

% By an approximation argument, we can find $W^{(n)} \in C_c(L, TL)$ such that $W^{(n)}_2 = 0$ and for every $y \in \RR$, $X = W^{(n)}_1(\cdot, y)$ is a maximizing sequence for $\langle d 1_{E_y}, X\rangle$ subject to $X \prec U$.
% Let us now suppress the $n$ and write $W(y) = W^{(n)}(\cdot, y)$.

% By Lemma \ref{coarea converse}, since $W$ has compact support, the integrand $\langle d 1_{E_y}, W(y)\rangle$ is uniformly bounded in $y$.
% Therefore, by Fatou's lemma,
% \begin{align*}
% \int_{-\infty}^\infty |\omega(y)| ~dy &= \int_{-\infty}^\infty \lim_{n \to \infty} \langle \omega(y), W(y)\rangle ~dy \leq \liminf_{n \to \infty} \int_{-\infty}^\infty \langle \omega(y), W(y)\rangle ~dy \\
% &\leq \mu(U). \qedhere
% \end{align*}
% \end{proof}


% \nocite{*}
\printbibliography

\end{document}
