\documentclass[reqno,12pt,letterpaper]{amsart}
\RequirePackage{amsmath,amssymb,amsthm,graphicx,mathrsfs,url}
\RequirePackage[usenames,dvipsnames]{color}
\RequirePackage[colorlinks=true,linkcolor=Red,citecolor=Green]{hyperref}
\RequirePackage{amsxtra}
\usepackage{tikz-cd}

\setlength{\textheight}{8.50in} \setlength{\oddsidemargin}{0.00in}
\setlength{\evensidemargin}{0.00in} \setlength{\textwidth}{6.08in}
\setlength{\topmargin}{0.00in} \setlength{\headheight}{0.18in}
\setlength{\marginparwidth}{1.0in}
\setlength{\abovedisplayskip}{0.2in}
\setlength{\belowdisplayskip}{0.2in}
\setlength{\parskip}{0.05in}
\renewcommand{\baselinestretch}{1.10}

\title[Topics exam]{Topics exam}
\author{Aidan Backus}
\date{May 2021}

\newcommand{\NN}{\mathbf{N}}
\newcommand{\ZZ}{\mathbf{Z}}
\newcommand{\QQ}{\mathbf{Q}}
\newcommand{\RR}{\mathbf{R}}
\newcommand{\CC}{\mathbf{C}}
\newcommand{\DD}{\mathbf{D}}
\newcommand{\PP}{\mathbf P}
\newcommand{\MM}{\mathbf M}
\newcommand{\II}{\mathbf I}

\DeclareMathOperator{\card}{card}
\DeclareMathOperator{\ch}{ch}
\DeclareMathOperator{\codim}{codim}
\DeclareMathOperator{\diag}{diag}
\DeclareMathOperator{\diam}{diam}
\DeclareMathOperator{\dom}{dom}
\DeclareMathOperator{\Gal}{Gal}
\DeclareMathOperator{\Hom}{Hom}
\DeclareMathOperator{\Jac}{Jac}
\DeclareMathOperator{\Lip}{Lip}
\DeclareMathOperator{\id}{id}
\DeclareMathOperator{\rad}{rad}
\DeclareMathOperator{\rank}{rank}
\DeclareMathOperator{\Radon}{Radon}
\DeclareMathOperator*{\Res}{Res}
\DeclareMathOperator{\sgn}{sgn}
\DeclareMathOperator{\singsupp}{sing~supp}
\DeclareMathOperator{\Spec}{Spec}
\DeclareMathOperator{\supp}{supp}
\DeclareMathOperator{\Tan}{Tan}
\DeclareMathOperator{\vol}{vol}
\newcommand{\tr}{\operatorname{tr}}

\newcommand{\dbar}{\overline \partial}

\DeclareMathOperator{\atanh}{atanh}
\DeclareMathOperator{\csch}{csch}
\DeclareMathOperator{\sech}{sech}

\DeclareMathOperator{\Div}{div}
\DeclareMathOperator{\grad}{grad}
\DeclareMathOperator{\Ell}{Ell}
\DeclareMathOperator{\WF}{WF}

\newcommand{\Hilb}{\mathcal H}

\newcommand{\pic}{\vspace{30mm}}
\newcommand{\dfn}[1]{\emph{#1}\index{#1}}

\renewcommand{\Re}{\operatorname{Re}}
\renewcommand{\Im}{\operatorname{Im}}


\newtheorem{theorem}{Theorem}[section]
\newtheorem{badtheorem}[theorem]{``Theorem"}
\newtheorem{prop}[theorem]{Proposition}
\newtheorem{lemma}[theorem]{Lemma}
\newtheorem{proposition}[theorem]{Proposition}
\newtheorem{corollary}[theorem]{Corollary}
\newtheorem{conjecture}[theorem]{Conjecture}
\newtheorem{axiom}[theorem]{Axiom}

\theoremstyle{definition}
\newtheorem{definition}[theorem]{Definition}
\newtheorem{remark}[theorem]{Remark}
\newtheorem{example}[theorem]{Example}

\newtheorem{exercise}[theorem]{Discussion topic}
\newtheorem{homework}[theorem]{Homework}
\newtheorem{problem}[theorem]{Problem}

\newtheorem{ack}{Acknowledgements}
\newtheorem{notate}{Notation}

\numberwithin{equation}{section}


% Mean
\def\Xint#1{\mathchoice
{\XXint\displaystyle\textstyle{#1}}%
{\XXint\textstyle\scriptstyle{#1}}%
{\XXint\scriptstyle\scriptscriptstyle{#1}}%
{\XXint\scriptscriptstyle\scriptscriptstyle{#1}}%
\!\int}
\def\XXint#1#2#3{{\setbox0=\hbox{$#1{#2#3}{\int}$ }
\vcenter{\hbox{$#2#3$ }}\kern-.6\wd0}}
\def\ddashint{\Xint=}
\def\dashint{\Xint-}

%\usepackage{color}
%\hypersetup{%
%    colorlinks=true, % make the links colored%
%    linkcolor=blue, % color TOC links in blue
%    urlcolor=red, % color URLs in red
%    linktoc=all % 'all' will create links for everything in the TOC
%Ning added hyperlinks to the table of contents 6/17/19
%}

\usepackage[backend=bibtex,style=alphabetic,maxcitenames=50,maxnames=50]{biblatex}
\addbibresource{topics.bib}
\renewbibmacro{in:}{}
\DeclareFieldFormat{pages}{#1}

\begin{document}
\begin{abstract}
Topics exam, Fall 2021.
\end{abstract}

\maketitle

\tableofcontents

\section{Currents and homological integration}
This section follows \cite{urs2005GMT} and \cite{simon1983GMT}.

Let $\Lambda_m\RR^n$ denote the space of $m$-vectors and $\Lambda^m\RR^n$ denote the space of $m$-covectors.
If $\tau \in \Lambda_m \RR^n$, $\tau = \bigwedge_i v_i$, we write
$$|\tau|^2 = \det(\langle v_i, v_j\rangle)_{ij}.$$
If view $\tau$ as a volume form on an $m$-dimensional submanifold of $\RR^n$, then $|\tau|$ is the Radon-Nikodym derivative of $\tau$ with respect to Lebesgue measure.

\begin{definition}
The \dfn{comass} $||\omega||$ of an $m$-covector $\omega$ is
$$||\omega|| = \sup_{|\tau| \leq 1} \langle \tau, \omega\rangle$$
where $\tau$ ranges over simple $m$-vectors. The \dfn{mass} $||\tau||$ of an $m$-vector $\tau$ is the operator norm of $\tau$ with respect to comass.
\end{definition}

Henceforth if $\omega$ is a differential form, we describe the $L^p$ norms of $\omega$ with respect to its comass. That is,
$$||\omega||_{L^p}^p = \int_{\RR^n} \left(\sup_{|\tau| \leq 1} \langle \tau, \omega(x)\rangle\right)^p ~dx$$
where the supremum ranges over simple $m$-vectors, and similarly,
$$||\omega||_{L^\infty} = \sup_{\substack{x \in \RR^n\\|\tau| \leq 1}} \langle \tau, \omega(x)\rangle.$$

Let $\mathcal D^m\RR^n$ be the space of compactly supported, smooth $m$-forms on $\RR^n$, equipped with the topology for which a sequence $\omega_i$ converges iff there is a compact set $K$ such that $\omega_i$ is eventually supported in $K$, and every derivative of the components of $\omega_i$ converges in $L^\infty(K)$.

\begin{definition}
A $m$-\dfn{current} is a continuous linear functional on $\mathcal D^m\RR^n$.
The space of $m$-currents equipped with the topology of weak convergence is denoted $\mathcal D_m\RR^n$.
\end{definition}

Just as we define the derivative of a distribution to make integration by parts valid, we define the boundary (or derivative) of a current to make Stokes' theorem true.
I will abuse notation by writing $\int_T \varphi$ to mean the duality pairing of a current $T$ with a compactly supported form $\varphi$.

\begin{definition}
Let $T \in \mathcal D_m\RR^n$. The \dfn{boundary} of $T$ is the $m-1$-current
$$\int_{\partial T} \varphi = \int_T d\varphi.$$
\end{definition}

It follows that $\partial^2 = 0$, $\supp \partial T \subseteq \supp T$, and if $T_i \to T$ weakly then $\partial T_i \to \partial T$.
In particular, currents form a chain complex.

\begin{definition}
A \dfn{cycle} is a current $T$ with $\partial T = 0$.
A \dfn{boundary} is a current $\partial T$ for some current $T$.
\end{definition}

\begin{definition}
The \dfn{total variation} of a current $T$ is the Borel measure defined on open sets $U$ by
$$||T||(U) = \sup_{\supp \omega \Subset U} \frac{1}{||\omega||_{L^\infty}} \int_T \omega.$$
The \dfn{mass} of $T$ is $M(T) = ||T||(\RR^n)$.
The space $\MM_m\RR^n$ is the space of currents of finite mass.
\end{definition}

This definition is exactly analogous to the definition of a Radon measure from a continuous linear functional on \emph{continuous} functions of compact support, and the proof is the same.

By Fatou's lemma, we have for every open set $U$ and convergent sequence of currents $(T_i)$,
$$\left|\left|\lim_{i \to \infty} T_i\right|\right|(U) \leq \liminf_{i \to \infty} ||T_i||(U).$$
Therefore the space $\MM_m\RR^n$ is a Banach space, since the limit of a sequence of currents of finite mass also has finite mass.
By the dominated convergence theorem, it follows that for every form $\omega$,
$$\left|\int_T \omega \right| \leq ||\omega||_{L^\infty} M(T).$$

\subsection{Currents as Radon measures}
In the calculus of distributions, $\delta'$ is not a measure, even though $\delta$ is.
We generalize this idea to currents, determining which currents can be viewed as measures.

\begin{definition}
Let $\mu$ be a Radon measure on $\RR^n$ and $\tau$ a smooth $m$-vector field in $L^1_{loc}$.
The current $[\mu, \tau]$ is defined by
$$\int_{[\mu, \tau]} \omega = \int_{\RR^n} \langle \tau(x), \omega(x)\rangle ~d\mu(x).$$
Conversely, if $T$ is a current which admits $\mu, \tau$ with $T = [\mu, \tau]$, we say that $T$ is a \dfn{representable current}.
\end{definition}

\begin{lemma}
\label{Riesz Markov Hilbert}
Let $X$ be a locally compact Hausdorff space, $H$ a finite-dimensional Hilbert space, and $L: C_c(X \to H) \to \RR$ a linear map which is locally finite in the sense that for every compact set $K$,
$$\sup_{\substack{\supp f \subseteq K\\|f| \leq 1}} Lf < \infty.$$
Then there is a Radon measure $\mu$ on $X$ and a $\tau \in L^\infty(X \to H)$ such that $|\tau| = 1$ and
$$Lf = \int_X \langle \tau(x), f(x)\rangle ~d\mu(x)$$
for every $f \in C_c(X \to H)$.
\end{lemma}
\begin{proof}
Let
$$\lambda f = \sup_{|\omega| \leq f} L\omega$$
whenever $f \in C_c(X \to [0, \infty))$.
Then $\lambda$ is a positive linear functional, so there exists $\mu$ such that
$$\lambda f = \int_X f~d\mu.$$
(The proof that $\lambda$ is a positive linear functional is identical to a step in the proof of the Riesz representation theorem in Rudin \cite{rudin1987real}.)
Now set $L_jf = L(fe_j)$ where $e_1, \dots, e_n$ form an orthonormal basis for $H$.
Then $L_j$ is an element of the dual of $L^1(\mu)$, so $L_j$ can be represented by $\tau_j \in L^\infty$.
Then set $\tau = \sum_j \tau_j e_j$.
\end{proof}

\begin{theorem}[Riesz-Markov representation theorem]
An $m$-current $T$ is representable iff $T \in \MM_{m,loc}\RR^n$.
If $T$ is representable, then in particular it is representable using the measure $||T||$.
\end{theorem}
\begin{proof}
Suppose that $T = [\mu, \tau]$ is representable. Then for every compact set $K$, we can find a bounded open set $U$ containing $K$ with $\mu(U) \leq 2\mu(K)$, and
$||T||(U) = \sup_{\supp \omega \Subset U} \int_{\RR^n} \langle \tau(x), \omega(x) ~d\mu(x) < \infty$
since $\tau \in L^1(U) \subseteq L^1_{loc}$ and $\mu(U) \leq 2\mu(K) < \infty$.
Therefore $||T||$ is locally finite.
The converse follows from Lemma \ref{Riesz Markov Hilbert}.
\end{proof}

\begin{definition}
Let $T = [||T||, \tau]$ be a representable current.
The tensor field $\tau$ is called the \dfn{orientation} of $T$.
\end{definition}

Now let $j: X \to \RR^n$ be an embedding of an oriented $m$-manifold.
Then $j$ defines a representable $m$-current $[X]$ by
$$\int_{[X]} \omega = \int_X j^*\omega.$$
The measure $||[X]||$ is the restriction of Hausdorff $m$-measure to $j(X)$, and $\tau$ is the pushforward of the orientation of $X$.
Furthermore, $[\partial X] = \partial[X]$ by Stokes' theorem.

In the special case of (oriented) $0$-manifolds and $1$-manifolds -- that is, points and curves -- we set $[a]$ to be the $0$-current
$$\int_{[a]} f = f(a)$$
and $[\gamma]$ to be the $1$-current
$$\int_{[\gamma]} f = \int_\gamma f(x) ~dx.$$
In particular we write $[a, b] = [[a, b]]$ and hence $\partial[a,b] = [b] - [a]$ by the fundamental theorem of calculus.

To discuss compactness, we put a second topology on $\MM_{m,loc}(\RR^n)$.
Namely, if $U \subseteq \RR^n$ is an open ball (say, with rational center and radius), we define a seminorm on $\MM_{m,loc}\RR^n$ by
$$||T||_U = ||T||(U).$$
Then $\MM_{m,loc}\RR^n$ is a Fr\'echet space, by Fatou's lemma.
In particular, a set in $\MM_{m,loc}\RR^n$ is bounded if it is bounded in every seminorm.
By the Banach-Alaoglu theorem for Fr\'echet spaces, we obtain a weak compactness result:

\begin{theorem}[Banach-Alaoglu]
A set of representable currents is weakly compact iff it is weakly closed and bounded.
\end{theorem}

\subsection{Differentiation of Lipschitz maps}
Previously we have only defined currents which represent integration on smooth manifolds.
Now we will do the same for integration on sets of very low regularity.
We will need to define the tangent bundle of such a set, so that we can talk about orientations thereof.
So let us begin by defining some very weak notions of differentiation.

Let $\Lip f$ denote the Lipschitz seminorm
$$\sup_{x \neq y} \frac{d(f(x), f(y))}{d(x, y)}$$
of a function $f: X \to Y$, where $X,Y$ are metric spaces.

\begin{theorem}[McShane-Whitney]
If $J$ is a set, $A \subseteq X$ metric spaces, and $f: A \to \ell^\infty(J)$ is a function, then there is an extension $\overline f: X \to \ell^\infty(J)$ with
$$\Lip \overline f = \Lip f.$$
\end{theorem}
\begin{proof}
By the axiom of choice, we may assume that $J$ is an ordinal $\kappa$.
We use transfinite induction on $\kappa$.

If $\kappa = 1$, set
$$\overline f(x) = \inf_{a \in A} f(a) + d(a, x) \Lip f.$$
Then the local Lipschitz seminorm of $\overline f$ on $X \setminus A$ is clearly $\Lip f$, so $\Lip \overline f = \Lip f$.

If the result holds for $\kappa$, write $\kappa + 1 = \kappa \cup \{\kappa\}$, and let $f_1$ be the extension on $\ell^\infty(\kappa)$.
Let $f_2$ be the extension on $\{\kappa\}$ we get from the $\kappa = 1$ case.
Then set $\overline f = (f_1, f_2)$. Then
$$\Lip \overline f = \max(\Lip f_1, \Lip f_2) = \Lip f.$$

Now take limits at limit stages.
\end{proof}

\begin{corollary}
If $A \subseteq X$ are metric spaces and $f: A \to \RR^d$ is a function, then there is an extension $\overline f: X \to \RR^d$ with
$$\Lip \overline f \leq \sqrt d \Lip f.$$
\end{corollary}
\begin{proof}
View $\RR^d = \ell^\infty(\{1, \dots, d\})$ and apply the previous lemma. We pick up an error factor of $\sqrt d$ going from $\ell^\infty$ to $\ell^2$.
\end{proof}

Recall that the Sobolev space $W^{1,p}(U)$ is the completion of the space of $u \in C^\infty(U)$ such that
$$||u||_{W^{1, p}} = ||u||_{L^p} + ||\nabla u||_{L^p} < \infty.$$
Let us now show that locally Lipschitz functions are differentiable almost everywhere.
We follow Evans \cite{evans2010partial}.

\begin{lemma}
Let $U$ be an open precompact set.
Then $\Lip U \subseteq W^{1, \infty}(U)$.
\end{lemma}
\begin{proof}
Let $u \in C^\infty(U)$.
Define the partial Newton quotient
$$\partial_i^h u(x) = \frac{u(x + he_i) - u(x)}{h}.$$
Then one has
$$||\partial_i^{-h}u||_{L^2} \lesssim_U ||\partial_i^{-h} u||_{L^\infty} \leq \Lip u$$
since $U$ is precompact. So by the Banach-Alaoglu theorem, after passing to a subsequence we can find $v_i \in L^\infty(U) \subseteq L^2(U)$ such that $||v_i||_{L^\infty} \leq \Lip u$ and
$$\lim_{h \to \infty} \partial_i^{-h}u = v_i$$
in weak $L^2(U)$. Furthermore, $L^2(U) \subseteq L^2_{loc}(\RR^n)$ and for every $\psi \in \mathcal D^0(U)$,
$$\int_{\RR^n} u\partial_i \psi = \lim_{h \to 0} \int_{\RR^n} u\partial_i^h \psi = -\lim_{h \to 0} \int_{\RR^n} \partial_i^{-h}u \psi$$
by a discrete analogue of integration by parts. Thus the weak convergence in $L^2_{loc}(\RR^n)$ gives
$$\int_{\RR^n} u\partial_i \psi = -\int_{\RR^n} v_i\psi$$
so $||\partial_i u||_{L^\infty} \leq \Lip u$.
Also $||u||_{L^\infty} \lesssim_U \Lip u$ since $U$ is precompact, so
$$||u||_{W^{1, \infty}} \lesssim_U \Lip u.$$
Taking completions, we see the claim.
\end{proof}

\begin{lemma}
Let $U \subseteq \RR^n$ be an open precompact set.
Let $p \in (n, \infty)$ and $u \in W^{1,p}(U)$.
Then $u$ is differentiable almost everywhere in $U$.
\end{lemma}
\begin{proof}
Let $v$ be a differentiable function.
Let $[v]_\gamma$ be the $\gamma$-H\"older seminorm of $v$, where $\gamma = 1 - n/p$.
By Morrey's inequality,
$$[v]_\gamma \lesssim_{p, n} ||\nabla v||_{L^p(U)}.$$
This remains true even if $U$ is cut off to a ball containing $x,y \in U$, so
$$\frac{|v(y) - v(x)|}{|x - y|^\gamma} \lesssim_{p, n} \left(\dashint_{B(x, 2|x - y|)} |\nabla v(z)|^p ~dz\right)^{1/p}.$$

We are given that $\nabla u \in L^p(U)$ and hence, by the Lebesgue differentiation theorem, almost every $x \in U$ is a $p$-Lebesgue point in the sense that
$$\lim_{r \to 0} \dashint_{B(x, r)} |\nabla u(x) - \nabla u(y)|^p ~dy = 0.$$
Let $x$ be a $p$-Lebesgue point and let
$$v(y) = u(y) - u(x) - \nabla u(x) \cdot (x - y).$$
Then by Taylor's theorem, $v$ is differentiable, so if $r = |x - y|$,
$$|v(y)| \lesssim_{p, n} r^\gamma \left(\int_{B(x, 2r)}|\nabla u(x) - \nabla u(z)|^p ~dz\right)^{1/p} \leq \left(\dashint_{B(x, 2r)} |\nabla u(x) - \nabla u(z)|^p ~dz\right)^{1/p}.$$
Thus $|v(y)| = o(|x - y|)$ which implies that $u$ is differentiable at $x$.
\end{proof}

\begin{theorem}[Rademacher]
Every locally Lipschitz function is differentiable almost everywhere.
\end{theorem}
\begin{proof}
By cutting off the locally Lipschitz function $u$ to a small precompact set $U$, we can assume $\Lip u < \infty$.
Then $u \in W^{1, \infty}(U) \subseteq W^{1, n + 1}(U)$, so $u$ is differentiable almost everywhere.
\end{proof}

If a function $f: \RR^m \to \RR^n$ is locally Lipschitz, then in particular its derivative is defined as a map of tangent spaces almost everywhere.
Thus we may say that $f$ is an immersion if its derivative is injective, and a submersion if its derivative is surjective.

By a parallelogram norm we will mean a norm induced by an inner product.
Indeed, a norm is a parallelogram norm iff it satisfies the parallelogram identity.
Lang refers to such norms as euclidean norms.
We will show that if $f$ is an immersion at $x$, then there is a parallelogram norm $||\cdot||$ such that for ``most" points $y$ near $x$, $|f(y) - f(x)| \sim ||y - x||$ and the implied factor is small.
This would be not possible if $f$ was not an immersion -- otherwise, for example, $f$ could vanish to infinite order, in which case we would have $0 \sim ||y - x|| > 0$, which is nonsense.

A parallelogram norm $||\cdot||$ is the same data as a symmetric $m \times m$ matrix $g$, namely
$$||v||^2 = v \cdot gv.$$

\begin{theorem}[Borel partition]
Suppose $f: \RR^m \to \RR^n$ is Lipschitz, and $B$ is the set of points at which $f$ is an immersion.
Then for every $\lambda > 1$ there is a countable Borel partition $(B_i)_i$ of $B$ and a sequence of parallelogram norms $||\cdot||_i$ on $\RR^m$ such that for every $x,y \in B_i$ and $v \in \RR^m$,
$$\lambda^{-1} ||v||_i \leq |\nabla f(x) \cdot v| \rangle \lambda ||v||_i$$
and
$$\lambda^{-1} ||y - x||_i \leq |f(y) - f(x)| \leq \lambda ||y - x||_i.$$
\end{theorem}
\begin{proof}
The space of parallelogram matrices is separable, so there are countably many parallelogram norms $||\cdot||_j$ such that for every parallelogram norm $||\cdot||$ and $\epsilon > 0$ there is $j \in \NN$ such that for every $v \in \RR^m$,
$$(1 - \epsilon) ||v||_j \leq ||v|| \leq (1 + \epsilon) ||v||_j.$$
Choose $\delta > 0$ so small that
$$\frac{1}{\lambda} + \delta < 1 < \lambda - \delta.$$
Let $B_{jk}$ be the set of all $x \in B$ such that for every $v \in \RR^m$,
$$(\lambda^{-1} + \delta)||v||_j \leq |\nabla f(x) \cdot v| \leq (\lambda - \delta) ||v||_j$$
and if $|v| < 1/k$ then
$$|f(x + v) - f(x) - \nabla f(x) \cdot v| \leq \delta ||v||_j.$$

Fix $x \in B$. We claim that there are $j,k$ such that $x \in B_{jk}$.
Since $f$ is an immersion at $x$, $A = \nabla f(x)$ satisfies that $A^*A$ is an isomorphism. That is,
$$|Av|^2 = Av \cdot Av = v \cdot A^*A v,$$
so $v \mapsto |Av|$ is a parallelogram norm.
Thus we can find $j$ such that $x \in \bigcup_k B_{jk}$.
Then, since $A$ has finite operator norm and conorm, we can find $c_j > 0$ such that $|v| \leq c_j ||v||_j$.
By Taylor's theorem, we can find $k$ such that
$$|f(x + v) - f(x) - \nabla f(x) \cdot v| \leq \frac{\delta}{c_j} |v|$$
if $|v| < 1/k$, which implies $x \in B_{jk}$.

If we shrink $B_{jk}$ so that $\diam B_{jk} < 1/k$ then
$$|f(x + v) - f(x)| \leq |\nabla f(x) \cdot v| + \delta ||v||_j \leq \lambda ||v||_j$$
and similarly for $\lambda^{-1}$, whenever $x, x + v \in B_{jk}$.
Now after shrinking and reindexing the $B_{jk}$ we get our result.
\end{proof}

Let $\mathcal H^m$ denote Hausdorff measure of dimension $m$.

\begin{definition}
Let $f: \RR^m \to \RR^n$ be a locally Lipschitz map, $x \in \RR^m$, and let $L: \RR^m \to \RR^n$ be the derivative of $f$ at $x$.
Define the \dfn{Jacobian determinant},
$$\Jac f(x) = \sqrt{\det(L^*L)}.$$
\end{definition}

The idea here is that $L$ isn't going to be a square matrix in general but $\sqrt{L^*L}$ is.
In fact $\sqrt{\det(L^*L)}$ is the product of the singular values of $L$, which seems like a reasonable generalization of the determinant to nonsquare matrices.

\begin{theorem}[area formula]
Let $f: \RR^m \to \RR^n$ be a Lipschitz map, $m \leq n$.
For every $u \in L^1(\RR^m)$,
$$\int_{\RR^m} u(x) \Jac f(x) ~dx = \int_{\RR^n} \sum_{f(x) = y} u(x) ~d\mathcal H^m(y).$$
\end{theorem}
\begin{proof}
By dominated convergence, it suffices to check when $u = 1_A$ where $A$ is a Borel set of finite measure.
Moreover, since $\Jac f$ is defined almost everywhere, it is no loss to assume that $\Jac f$ is defined on $A$.
By splitting up $A$ we can break into the cases that $f$ is an immersion on $A$, and $\Jac f = 0$ on $A$.

If $f$ is an immersion on $A$, let $\lambda > 1$.
By the Borel partition theorem and the fact that $f$ is locally injective (since $f$ is an immersion), there is a Borel partition $(A_i)_i$ of $A$ equipped with parallelogram norms $||\cdot||_i$ such that $f|A_i$ is injective, and (after integrating the inequalities in the Borel partition theorem in $\mathcal H^m$),
$$\lambda^{-m} \mathcal H^m_{||\cdot||_i}(A_i) \leq \mathcal H^m(f(A_i)) \leq \lambda^m \mathcal H^m_{||\cdot||_i}(A_i),$$
and for every $x \in A_i,v \in \RR^m$,
$$\lambda^{-1} ||v||_i \leq |\nabla f(x) \cdot v| \leq \lambda ||v||_i.$$
Taking $L \mapsto \sqrt{\det(L^*L)}$ of this inequality we get
$$\mathcal H^m(f(A_i)) \leq \lambda^m \mathcal H^m_{||\cdot||_i}(A_i) = \lambda^m \sqrt{\det(g_i^*g_i)} \mathcal H^m(A_i) \leq \lambda^{2m} \int_{A_i} \Jac f(x) ~dx$$
where $g_i$ is the symmetric matrix associated to $||\cdot||_i$.
Summing over $i$ using the fact $f|A_i$ is injective,
$$\int_{\RR^n} \card(f^{-1}(y) \cap A) ~d\mathcal H^m(y) \leq \lambda^{2m} \int_A \Jac f(x) ~dx.$$
One can similarly bound
$$\int_{\RR^n} \card(f^{-1}(y) \cap A) ~d\mathcal H^m(y) \geq \lambda^{-2m} \int_A \Jac f(x) ~dx.$$
Taking $\lambda \to 1$ we conclude the claim.

If $\Jac f = 0$ on $A$, let $\epsilon > 0$ and let $F: \RR^m \to \RR^{n + m}$ satisfy
$$F(x) = (f(x), \epsilon x).$$
Then
$$\Jac F(x) \leq \epsilon(\Lip f + \epsilon)^{m - 1}$$
and $F$ is an immersion, so
$$\mathcal H^m(f(A)) \leq \mathcal H^m(F(A)) = \int_A \Jac F(x) ~dx \lesssim \epsilon(\Lip f + \epsilon)^{m - 1}.$$
As $\epsilon > 0$ was arbitrary we conclude that $\mathcal H^m(f(A)) = 0$.
\end{proof}

I think the area formula is easies to visualize when $f$ is the extension of a $k$-sheeted Lipschitz covering space $f: F \to E$ where $E \subseteq \RR^n$ is a Borel set of Hausdorff dimension $m$, $u = 1_F$, and if $x \notin F$ then $f(x) \notin E$. Then the area formula reads
$$\int_F \Jac f(x) ~dx = \int_{\RR^n} \sum_{i=1}^k 1_F(x_i(y)) ~d\mathcal H^m(y)$$
where $x_1, \dots, x_k$ are the fibers of the covering space $f$.
But $x_i(y) \in F$ iff $y \in E$, thus
$$\int_{\RR^n} \sum_{i=1}^k 1_F(x_i(y)) ~d\mathcal H^m(y) = k\int_{\RR^n} 1_E ~d\mathcal H^m = k\mathcal H^m(E)$$
and thus
$$\int_F \Jac f(x) ~dx = k\mathcal H^m(E).$$
In particular if $\Jac f = 1$ then this just says that copying a set $k$ times creates a set with $k$ times the area.

\subsection{The approximate tangent bundle}
Let us now define a very weak notion of regularity which is still enough to give a set a tangent bundle.

\begin{lemma}
\label{Lipschitz is nearly C1}
For every $f \in \Lip(U)$ and $\epsilon > 0$, there is $g \in C^1(U)$ such that $|\{f \neq g\}| < \epsilon$.
\end{lemma}
\begin{proof}
By the McShane-Whitney theorem it is no loss to assume $U = \RR^n$.
By Rademacher's theorem, $\nabla f$ is a Borel function which is finite almost everywhere.
By Lusin's theorem, there is a closed set $B \subseteq \RR^m$ such that $|B^c| < \epsilon$ and $\nabla f|B$ is continuous.
Now let
$$r_\delta(x) = \sup_{0 < |y - x| < \delta} \frac{|f(y) - f(x) - \nabla f(x) \cdot (y - x)}{|y - x|}.$$
Then since $\nabla f|B$ is continuous, Taylor's theorem implies that $r_\delta \to 0$ pointwise on $B$.
In particular, by Egorov's theorem, after shrinking $B$ by a set of measure $< \epsilon$, we may assume that $r_\delta \to 0$ in $L^\infty_{loc}(B)$.
Thus we may set $g = f|B$.
Since $B$ is closed and $r_\delta$ is bounded on $B$, $g$ extends to all of $\RR^n$ by the $C^1$ Urysohn lemma.
\end{proof}

\begin{definition}
A set $E \subseteq \RR^d$ is $m$-\dfn{rectifiable} if there are Lipschitz maps $f_i: \RR^m \to \RR^d$, such that
$$\mathcal H^m\left(E \setminus \bigcup_i f_i(\RR^m)\right) = 0.$$
\end{definition}

This definition looks hard to test in general, but we can give a Rademacher-like characterization: a set is $m$-rectifiable if it can be written as the union of differentiable manifolds.

\begin{theorem}
\label{Rademacher}
A set $E \subseteq \RR^d$ is $m$-rectifiable iff there are $m$-dimensional continuously differentiable submanifolds $M_k \subseteq \RR^d$ such that
$$\mathcal H^m\left(E \setminus \bigcup_k M_k\right) = 0.$$
\end{theorem}
\begin{proof}
If this property holds, then $E$ is $m$-rectifiable since we can write $M_k$ locally as the differentiable (and hence Lipschitz) image of $\RR^m$ for each $k$.

Conversely, assume that $E$ is $m$-rectifiable. By Lemma \ref{Lipschitz is nearly C1}, it is no loss to assume that the maps $f_i: \RR^m \to \RR^d$ which rectify $E$ are $C^1$.
Let $U_i$ be the set on which $f_i$ is an immersion.
By the area formula, $\mathcal H^m(f_i(\RR^m \setminus U_i)) = 0$. So $\mathcal H^m(E \setminus \bigcup_i f_i(U_i)) = 0$ since the $f_i$ are Lipschitz and so push forward null sets to null sets.
Since $f_i(U_i)$ is immersed, it is locally diffeomorphic to $U_i$.
As the rank of $f_i$ is lower semicontinuous, $U_i$ is an open submanifold of $\RR^m$.
Therefore $U_i$ is locally a submanifold of $\RR^n$.
\end{proof}

We now introduce the notion of a tangent bundle to a rectifiable set. We first need the notion of a measurable vector bundle.

Let $\mu$ be a measure on a measurable space $E$, $F$ a measurable space, and $p: F \to E$ a $\mu$-measurable map.
If for $\mu$-almost every $x \in E$, the fiber $F_x = p^{-1}(x)$ has the structure of a vector space of dimension $k$, and $E \times \RR^k$ is locally measurably isomorphic to $F$, then we call $p$ a $\mu$-measurable vector bundle over $E$.

Let
$$T_{x,r}(y) = \frac{y - x}{r}.$$
Then $T_{x,r}$ maps $x + B_r$ to $B_1$, so $T_{x,r}(E)$ can be viewed as ``zooming in on $x \in E$" if $r$ is small.

\begin{definition}
Let $E \subseteq \RR^n$ be a Borel set of Hausdorff dimension $m$, and $x \in \RR^n$.
We call a $m$-dimensional vector space $\Tan(E, x) \subseteq \RR^n$ the \dfn{approximate tangent space} of $E$ at $x$ if
\begin{equation}
\label{approx tangent space}
\lim_{r \to 0} \mathcal H^m|T_{x,r}(E) = \mathcal H^m|\Tan(E, x)
\end{equation}
in the weak topology of Radon measures.
If $\Tan(E, x)$ is well-defined for $\mathcal H^m$-almost every $x \in E$, we define the \dfn{approximate tangent bundle} $\Tan E$ to be the $\mathcal H^m$-measurable vector bundle on $E$ induced by the approximate tangent spaces.
\end{definition}

To see why this generalizes the classical notion of tangent bundle, assume that $E$ is a smooth manifold of dimension $m$.
Then if $r > 0$ is small, $T_{x,r}(E)$ moves $E$ so that $x = 0$ and blows up $E$ so much so that $E$ looks like a $m$-dimensional hyperplane near $x$.
Moreover, we only care about behavior near $x$, since the weak topology of Radon measures only tests against functions with compact support.
Thus we have (\ref{approx tangent space}). By Theorem \ref{Rademacher}, $\mathcal H^m$-almost every point of a rectifiable set $E$ has an approximate tangent space, so in particular $E$ has an approximate tangent bundle.

Now let us generalize the notion of orientation to rectifiable sets.
This is just a straightforward generalization of the notion of orientation form of a smooth manifold.

\begin{definition}
Let $E$ be a $m$-rectifiable set.
An \dfn{orientation} of $E$ is a Borel function $\tau: E \to \Lambda_m \RR^n$ where for $\mathcal H^m$-almost every $x \in E$, $|\tau| = 1$, and
$$\tau(x) = \bigwedge_j e_j(x)$$
where $(e_j(x))_j$ spans $\Tan(E, x)$.
If $E = (E, \tau)$ is an oriented $m$-rectifiable set, then we define an integration of differential forms
$$\int_E \omega = \int_E \langle \tau(x), \omega(x)\rangle ~d\mathcal H^m(x).$$
\end{definition}

We view orientations of $E$ as sections of the top tensor bundle $(\Tan E)^{\wedge m}$.

\subsection{Homological integration theory}
Since currents form a chain complex, we should expect them to push forward, rather than pull back.

\begin{definition}
Let $T \in \mathcal D_m\RR^n$, and let $f: \RR^n \to \RR^p$ be a smooth map such that $f|\supp T$ is proper.
For every $\omega \in \mathcal D^m\RR^n$, let $\zeta$ be a cutoff to a neighborhood of the compact set $\supp T \cap \supp f^*\omega$, and define the \dfn{pushforward current} $f_*T$ by
$$\int_{f_*T} \omega = \int_T \zeta f^*\omega.$$
\end{definition}

Since $\zeta$ is a cutoff, this definition is independent of $\zeta$.
One has $\partial f_* = f_* \partial$, and $\supp f_* T \subseteq f(\supp T)$, as well as $(g \circ f)_* = g_* f_*$.

If $T = [||T||, \tau]$ is a representable current which can be pushed forward along $f$, then
$$\int_{f_*T} \omega = \int_{\RR^n} \langle (df_x)_* \tau(x), f^*\omega(x) \rangle ~d||T||(x)$$
and so $f_*T = [f_*||T||, f_* \tau]$ is a representable current.

Now suppose that $T$ is a current. We define
$$[a] \times T = i(a)_*T$$
where $i(a)(x) = (a, x)$. Since $i(a)_*$ commutes with $\partial$ we get $\partial([a] \times T) = [a] \times \partial T$.
Now if $(t, x)$ are coordinates on $\RR^{1 + n}$, we write $\omega = dt \wedge \omega' + \overline \omega$ where $dt \wedge \omega'$ contains all terms with $dt$, and define
$$\int_{[a, b] \times T} \omega = \int_a^b ([t] \times T)(\omega') ~dt.$$
It follows that
$$\partial([a, b] \times T) = [b] \times T - [a] \times T - [a, b] \times \partial T$$
(the Leibniz formula) and
$$||[a, b] \times T|| = ||[a, b]|| \times ||T||.$$

\begin{theorem}[homotopy formula]
Let $f,g \in C^\infty(\RR^n \to \RR^p)$, and let $h: f \to g$ be a smooth homotopy.
Let $T$ be an $m$-current such that $h|([0, 1] \times \supp T)$ is proper. Then
$$(g_* - f_*)T = \partial h_*([0, 1] \times T) + h_*([0, 1] \times \partial T).$$
\end{theorem}
\begin{proof}
One has
\begin{align*}
\partial h_*([0, 1] \times T) &= h_*\partial([0, 1] \times T)\\
&= h_*([1] \times T - [0] \times T - [0, 1] \times \partial T)\\
&= g_*T - f_*T - h_*([0, 1] \times \partial T)
\end{align*}
as desired.
\end{proof}

Suppose that $\mu$ is a distribution and a Radon measure. Then $\mu'$ is not a Radon measure.
On the other hand, the boundary of a \emph{current} may still be a Radon measure.
This is a useful enough property that we give it a horrible name.

\begin{definition}
Let $T$ be an $m$-current. Set
$$N(T) = M(T) + M(\partial T).$$
If $N(T) < \infty$, we call $T$ a \dfn{normal current}.
The space $N_m\RR^n$ is the space of normal currents.
\end{definition}

The point of normal currents is that we can push them forward along \emph{Lipschitz} maps.

\begin{lemma}
\label{affine homotopy mass}
Let $h: f \to g$ be an affine homotopy. Suppose that $h|\supp T$ is a proper map. Then
$$M(h_*[0, 1] \times T) \leq ||f - g||_{L^\infty(\supp T)} \sup_{x \in \supp T} (|df_x| + |dg_x|)^n M(T).$$
\end{lemma}
\begin{proof}
Let $\tau$ be the orientation of $T$; then
\begin{align*}
\int_{h_*([0, 1] \times T)} \omega &= \int_0^1 \int_U \langle \omega|h(t, x), (dh_{(t, x)})_*(e_1 \wedge \tau(x)) ~d||T||(x) ~dt\\
&= \int_0^1 \int_U \langle \omega|h(t, x), (g(x) - f(x)) \wedge (tdg_x + (1 - t)df_x)_* \tau(x)\rangle ~d||T||(x) ~dt
\end{align*}
and as $|\tau| = 1$ the claim follows.
\end{proof}

\begin{lemma}
Let $T$ be a locally normal current on $U$, $f: U \to V$ a Lipschitz map with $f|\supp T$ proper.
Set $f_\sigma = f * \eta_\sigma$ where $\eta$ is a standard mollifier. Then if we define a current $f_*T$ by
$$\int_{f_* T} \omega = \lim_{\sigma \to 0} \int_{(f_\sigma)_* T} \omega,$$
then $f_*T$ is a well-defined locally normal current and $\supp f_*T \subseteq f(\supp T)$.
In particular, if $T$ is a normal current, then so is $f_*T$.
\end{lemma}
\begin{proof}
By the homotopy formula,
$$(f_\sigma)_*T - (f_\tau)_* T = h_*([0, 1] \times T)d + h_*([0, 1] \times \partial T)$$
where $h: f_\sigma \to f_\tau$ is an affine homotopy. So by Lemma \ref{affine homotopy mass},
$$||(f_\sigma)_*T - (f_\tau)_*T|| \lesssim ||f_\sigma - f_\tau||_{L^\infty(f^{-1}(\supp \omega) \cap \supp T)} [f]^n$$
where $[f]$ is the Lipschitz seminorm of $f$.
So $(f_\sigma)_\sigma$ is a Cauchy net and the claim follows.
\end{proof}

Since normal currents push forward along Lipschitz maps, we can define a homology theory.
Namely, for every rectifiable set $X$, we let $N_m(X)$ be the space of $m$-currents with support in $X$.
Then $N(X)$ is a chain complex.

\begin{definition}
Let $X$ be a rectifiable set. The homology at $m$ of $N(X)$, denoted $H_m(X)$, is the $m$th \dfn{de Rham homology} of $X$.
\end{definition}
I'm not sure this is standard terminology but it's dual to de Rham cohomology with compact support so I'm not sure what else to call it.

If $A \subseteq X$ is closed, we define $N_m(X, A) = N_m(X)/N_m(A)$, so we get relative homology groups $H_m(X, A)$.
We get a commutative diagram
$$\begin{tikzcd}
U \cap V \arrow[r,"i_U"] \arrow[d,"i_V"] & U \arrow[d,"j_U"] \\
V \arrow[r,"j_V"] & X
\end{tikzcd}$$
of inclusion maps, whenever $\{U, V\}$ is an open cover of $X$.
Then the sequence
$$0 \to N_m(\overline U \cap \overline V) \to N_m(\overline U) \oplus N_m(\overline V) \to N_m(X) \to 0$$
defined by $(i_U)_* \oplus (i_V)_*$ and $(j_U)_* - (j_V)_*$ is short exact.
The only thing nontrivial about this statement is the surjectivity of $(j_U)_* - (j_V)_*$; indeed, if $T \in N_m(X)$ then we can cut off $T$ to $U,V$, and this gives the claim.
So, by the zigzag lemma, the Mayer-Vietoris sequence in relative homology is long exact, which gives the exactness and excision axioms.
On the other hand, by the homotopy formula, we can check the homotopy axiom; the dimension and additivity axioms are obvious.
In summary:

\begin{theorem}
Let $X$ be a rectifiable set.
The homology of the chain complex of normal currents on $X$ is an Eilenberg-Steenrod homology theory on $X$.
\end{theorem}

\subsection{Integral currents}
Recall that if $X$ is an oriented $m$-submanifold of $\RR^n$ then we obtain a representable $m$-current
$$\int_{[X]} \omega = \int_X \omega.$$
Let us now generalize this in two directions.
First, we will only assume that $X$ is rectifiable, and second, we will allow parts of $X$ to be integrated over multiple times.

\begin{definition}
A current $T \in \mathcal D_m$ is \dfn{locally integer rectifiable} if we can write
$$\int_T \omega = \int_E \theta\omega$$
where $E$ is an oriented $m$-rectifiable set, and the \dfn{multiplicity} $\theta$ is integer-valued.
In this case we write
$$T = [E, \theta]$$
and $T \in \mathcal I_{m,loc}(\RR^n)$.
If in addition $T$ has finite mass we write $T \in \mathcal I_m(\RR^n)$ and call $T$ an \dfn{integer rectifiable current}.
\end{definition}

The idea is that a locally integer rectifiable current is a sum of locally finitely many oriented rectifiable sets.

Thus the mass of the locally integer rectifiable current $T = [E, \theta]$ is
$$||T||(U) = ||\theta||_{L^1(U \cap E, \mathcal H^m)}.$$
We want an analogoue of normal currents in this setting.

\begin{definition}
A \dfn{locally integral current} $T$ is a locally integer rectifiable current such that $\partial T$ is locally integer rectifiable.
An \dfn{integral current} is a locally integral current $T$ such that $T$ and $\partial T$ are integer rectifiable.
We let $\II_{m,loc}$ be the space of locally integral currents and $\II_m$ be the space of integral currents.
\end{definition}

The key fact about integral currents is that the space of integral currents on a compact set is locally compact:

\begin{theorem}
Let $T_i \in \II_m(K)$ satisfy $N(T_i) \lesssim 1$, where $K$ is a compact set.
Then there is a subsequence which weakly converges to an element of $\II_m(K)$.
\end{theorem}
\begin{proof}
Omitted; see the exposition of White \cite{white1989new}.
\end{proof}

\begin{definition}
Let $R$ be a compactly supported integral $m-1$-cycle. The \dfn{Plateau problem} with boundary data $R$ is the problem of finding a compactly supported integral $m$-current $T$ such that $\partial T = R$ which minimizes $M(T)$ among compactly supported integral $m$-currents with boundary $R$.
\end{definition}

One obvious obstruction to solving the Plateau problem is homological -- namely, there could be a cycle $R$ which is not a boundary, in which case the Plateau problem with boundary data $R$ clearly has no solution.

\begin{definition}
Let $M$ be a differentiable submanifold with boundary of $\RR^n$.
The \dfn{integral de Rham homology} $H_\bullet(M; \ZZ)$ of $M$ is the homology of the chain complex of integral currents on $M$.
\end{definition}

Then $H_\bullet(\cdot; \ZZ)$ is a homology theory over $\ZZ$.
I think it's kind of badly behaved for nonorientable rectifiable sets though so I'm not sure if it's an Eilenberg-Steenrod theory.

Let us show that for compact manifolds, the \emph{only} restriction to solving the Plateau problem is homological.

\begin{theorem}
Let $M$ be a compact differentiable submanifold with boundary of $\RR^n$.
Suppose that $R$ is an integral $m - 1$-boundary on $M$, $m \geq 1$.
Then the Plateau problem on $M$ with boundary data $R$ has a solution.
\end{theorem}
\begin{proof}
Let $\mathcal T$ be the set of integral $m$-currents $T$ such that $\partial T = R$; then $\mathcal T$ is nonempty since $R$ is a boundary.
Let $(T_i)$ be a sequence in $\mathcal T$ such that
$$\lim_{i \to \infty} M(T_i) = \inf_{T \in \mathcal T} M(T).$$
Then
$$\sup_i N(T_i) = \sup_i M(T_i) + M(R) < \infty.$$
Since $M$ is compact, the compactness theorem implies that a subsequence converges weakly to some integral current $T$.
Then $\partial T = R$ and $M(T) = \inf_{T' \in \mathcal T} M(T')$.
\end{proof}

\begin{lemma}
One has $H_m(\RR^n; \ZZ) = 0$ if $m \geq 0$.
\end{lemma}
\begin{proof}
Let $R$ be an integral $m$-cycle. Let $h(t, x) = tx$ and let $T = h_\sharp([0, 1] \times R)$.
Then $h$ is a homotopy between $0$ and the identity, so, since $R$ is a cycle,
$$\partial T = R - 0_\sharp R = R.$$
Furthermore, we can write $R = [E, \theta]$ where $E$ is an oriented rectifiable set of dimension $m$ and $\theta$ is integer-valued.
Then, if $\tau$ is the orientation of $E$,
\begin{align*}
\int_T \omega &= \int_{h_\sharp([0, 1] \times R)} \omega = \int_0^1 \int_R h^*\omega \theta \\
&= \int_0^1 \int_E \langle \tau(h(x, t)), \omega(x)\rangle \theta(x) ~d\mathcal H^m(x) ~dt.
\end{align*}
Thus $T = [[0, 1] \times E, \theta]$, where we extend $\theta$ to $[0, 1] \times E$ by $\theta(t, x) = \theta(x)$.
\end{proof}

\begin{corollary}
Let $R$ be a compactly supported integral $m-1$-cycle on $\RR^n$, $0 \leq m \leq n$.
Then the Plateau problem on $\RR^n$ with boundary data $R$ has a solution.
\end{corollary}
\begin{proof}
Since $\RR^n$ has no integral de Rham homology, $R$ is a boundary.
Let $K$ be a compact convex set which contains the support of $R$ and let $p: \RR^n \to K$ retract to the nearest point.
Then $\Lip p = 1$, so $M(p_\sharp T) \leq M(T)$ for every $T$.
If $\partial T = R$ then $\partial(p_\sharp T) = p_\sharp \partial T = p_\sharp R = R$.
So we may replace $\RR^n$ with the compact manifold with boundary $K$.
\end{proof}

\begin{theorem}
Let $M$ be a compact differentiable submanifold of $\RR^n$.
Then every integral de Rham homology class $[T] \in H_m(M; \ZZ)$ has a representative $T$ which minimizes $M(T)$ among cycles in $[T]$.
\end{theorem}
\begin{proof}
Let $(T_i)$ be a minimizing sequence in $[T]$.
Then $N(T_i) = M(T_i)$ since the $T_i$ are cycles, and since $M$ is compact it follows that there is $T$ such that after taking subsequences, $\lim_i T_i = T$ weakly.
So $M(T)$ is minimal.
\end{proof}

\section{Functions of least gradient}
\subsection{Vector bundle-valued Radon measures}
Let $M$ be a manifold and let $U \subseteq M$ be an open set.
Let $E$ be a finite-dimensional normed vector bundle (briefly, a \dfn{normed bundle}) on $M$, with dual $E'$.
We will write $(u, v)$ for the pairing of $u \in E'$ and $v \in E$.

We will abuse notation and, for every finite-dimensional normed vector space $E$, identify $E$ with the trivial bundle $M \times E$.

We will write $A \Subset B$ to mean that $B$ is locally compact and $\overline A$ is a proper compact subset of $B$.

We write $\mathcal D(U, E)$ for the space of smooth sections of $E$ with compact support in $U$ (briefly, \dfn{test sections}).
The topological dual space
$$\mathcal D'(U, E) = \mathcal D(U, E')'$$
is known as the space of \dfn{generalized sections} of $E$; it consists of linear functionals $u$ such that for every $E'$-trivial $V \Subset U$ and every $r \in \NN$ large enough depending on $u$, the restriction of $u$ to $C^r_c(U, E')$ is bounded in the topology of $C^r_c(U, E')$.
A sequence $(u_n)$ in $\mathcal D'(U, E)$ converges in the \dfn{weak topology of distributions} to $u \in \mathcal D'(U, E)$ if for every $v \in \mathcal D(U, E')$,
$$\lim_{n \to \infty} \langle u_n, v\rangle = \langle u, v\rangle.$$
We will call test sections of the tangent bundle \dfn{test fields} and generalized sections of the cotangent bundle \dfn{generalized $1$-forms}. We will identify continuous $1$-forms $\omega$ with generalized $1$-forms by
\begin{equation}
\label{continuous fields are generalized}
\langle \omega, X\rangle = \int_M (\omega, X) ~dV.
\end{equation}

By \dfn{Radon measure}, we will always mean $E$-valued Radon measure.
What this means is that $u \in \mathcal D'(M, E)$ is an $E$-valued Radon measure iff for every $U \Subset M$ and $v \in \mathcal D(U, E)$,
\begin{equation}
\label{total variation bound}
\langle u, v\rangle \lesssim_U ||v||_{L^\infty}.
\end{equation}
If $u$ is a Radon measure on $M$, we write $|u|_{TV}$ to mean the total variation measure of $u$; thus, $|u|_{TV}(U)$ is the best constant that can appear in (\ref{total variation bound}).
If $u$ is a Radon measure, then we can extend the action of $u$ by density to the space $C_c(M, E')$ of all continuous sections of $E'$ of compact support.
In case $E'$ is trivial, this agrees with the classical definition of $E$-valued Radon measure, by the Riesz-Markov theorem.

Let $\mathcal R(M, E)$ be the space of $E$-valued Radon measures on $M$, equipped with the \dfn{weak topology of measures}.
In this topology, a sequence of Radon measures $(u_n)$ converges to $u$ iff for every $v \in C_c(M, E')$,
$$\lim_{n \to \infty} \langle u_n, v\rangle = \langle u, v\rangle.$$
If the span of a countable set $\mathcal V$ is dense in $C_c(M, E')$, then the seminorms $u \mapsto \langle u, v\rangle$, $v \in \mathcal V$, determine the weak topology of measures.
Thus $\mathcal R(M, E)$ is a Fr\'echet space.

Given a Radon measure $u$ and an open set $U$, we can define $u1_U$ by
$$\langle u1_U, v\rangle = \sup_{\chi \in \mathcal D(U, [0, 1])} \langle u, \chi v\rangle.$$
Then $u1_U$ is a Radon measure.
This agrees with the classical definition of cutting off a Radon measure to an open set by the monotone convergence theorem.

Given a measure space $\Omega$, a function $u: \Omega \to \mathcal R(M, E)$ is measurable iff for every $v \in \mathcal D(M, E')$, $\omega \mapsto \langle u(\omega), v\rangle$ is measurable, by Pettis' theorem.
In that case we can define the Bochner-Lebesgue integral, and by Bochner's theorem it satisfies
$$\left\langle\int_\Omega u(\omega) ~d\omega, v\right\rangle = \int_\Omega \langle u(\omega), v\rangle ~d\omega.$$

\subsection{Functions of bounded variation}
Let $(M, g)$ be an oriented Riemannian manifold and $U \subseteq M$ open.
Let $dV$ be the volume element on $M$, let $dA$ be the area element induced by $dV$ on any Lipschitz hypersurface in $M$, and let $\mathcal L_X\eta$ denote the Lie derivative of a tensor field $\eta$ along a vector field $X$.

If $u \in C^1(M)$ and $\partial U$ is Lipschitz, one has the integration by parts formula
\begin{equation}
\label{Riemannian IBP}
\int_U (du, X) ~dV + \int_U u ~\mathcal L_XdV = \int_{\partial U} ug(X, \nu) ~dA
\end{equation}
where $X$ is a test field, $\nu$ is the outwards normal field on $\partial U$, $\mathcal L_X$ is the Lie derivative along $X$.
One also has the divergence theorem
\begin{equation}
\label{divergence theorem}
\mathcal L_XdV = \Div X ~dV.
\end{equation}

If in fact $U = M$, then the right-hand side of (\ref{Riemannian IBP}) vanishes, so that one can define the \dfn{generalized derivative} $du \in \mathcal D'(M, T'M)$ of a generalized function $u \in \mathcal D'(M)$ by
\begin{equation}
\label{distributional derivative dfn}
\langle du, X\rangle = -\int_U u ~\mathcal L_XdV
\end{equation}
for every test field $X \in \mathcal D(M, TM)$. By (\ref{Riemannian IBP}, \ref{continuous fields are generalized}), this generalizes the notion of gradient.

If $u_n \to u$ in the weak topology of distributions, then the same holds for $du_n$.
This implies that if the $u_n$ are continuous functions and $u_n \to u$ locally uniformly, then $du_n$ converges in the weak topology of distributions.

\begin{definition}
Let $E$ be a normed bundle on $M$.
A generalized section $u \in \mathcal D'(M, E)$ defined on $U$ has \dfn{locally bounded variation} if $du$ is a $T'M \otimes E$-valued Radon measure on $U$.
The section $u$ has \dfn{bounded variation} if the seminorm $||u||_{BV} = |du|_{TV}(M)$ is finite.

The presheaf of sections of $E$ with locally bounded variation (resp. bounded variation, bounded variation and compact support) is denoted $BV_l(\cdot, E)$ (resp. $BV(\cdot, E)$, $BV_c(\cdot, E)$).
\end{definition}

Given $u_n \in BV_l$, one has
\begin{equation}
\label{RieszMarkovDistr}
|du|_{TV} \leq \limsup_{n \to \infty} |du_n|_{TV}
\end{equation}
whenever $u_n \to u$ in the weak topology of distributions.

\begin{theorem}[integration by parts + trace theorem]\label{Miranda IBP}
Suppose that $\partial U$ is a Lipschitz hypersurface.
Then for every $u \in BV_l(U)$ such that $|du|_{TV}(U) < \infty$ there is a function $v \in L^1(\partial U)$ such that for every test field $X \in \mathcal D(M, TM)$,
\begin{equation}
\label{Miranda IBP formula}
\langle 1_Udu, X\rangle + \int_U u ~\mathcal L_XdV = \int_{\partial U} vg(X, \nu) ~dA.
\end{equation}
\end{theorem}
\begin{proof}
Let $\varphi$ be a smooth flow on $M$ such that for every $x \in \partial U$ and $t \in [0, 1]$, $\varphi_t(x) \in U$.
Let $N_t$ be the set obtained by flowing $\partial U$ along $\varphi$ for time $t$.
Then, as long as $\varphi$ is chosen to flow slowly enough (so as to avoid self-intersections), $N_t$ is a Lipschitz hypersurface for every $t \in [0, 1]$.
Moreover, $N_t$ partitions $U$ into a closed outer region $\{\varphi_s(x): x \in \partial U, s \in [0, t]\}$ and an open inner region $U_t$.

There exist smooth functions $u_n$ on $U$ such that $u_n \to u$ locally uniformly.
Applying (\ref{Riemannian IBP}) to $u_n$ on $U_t$ and using the locally uniform convergence, we have
\begin{equation}
\label{interior Miranda IBP}
\langle 1_{U_t}du, X\rangle + \int_{U_t} u ~\mathcal L_XdV = \int_{N_t} g(uX, \nu) ~dA.
\end{equation}
By continuity from below, the left-hand side of (\ref{interior Miranda IBP}) converges to the left-hand side of (\ref{Miranda IBP formula}) as $t \to 0$.

If we set $v_t(x) = u(\varphi_t(x))$, then $v_t(x)$ is defined for $x \in \partial U$ and $t \in [0, 1]$.
Moreover, for every $0 < t_1 < t_2$,
\begin{equation}
\label{mean-value inequality}
\int_{\partial U} |v_{t_1} - v_{t_2}| ~dA \leq |du|_{TV}(U_{t_2} \setminus \overline{U_{t_1}}).
\end{equation}
Indeed, (\ref{mean-value inequality}) holds if $u$ is smooth by the mean-value theorem, and for the general case we can find $u_n$ smooth such that $u_n \to u$ in $L^2_l$.
Expanding out both sides of (\ref{mean-value inequality}) using the divergence theorem (\ref{divergence theorem}),
\begin{equation}
\label{approx mean-value inequality}
\int_{\partial U} |u_n(\varphi_{t_1}(x)) - u_n(\varphi_{t_2}(x))| ~dA(x) \leq -\inf_{||X||_{L^\infty} \leq 1} \int_M u_n\cdot \Div X~dV
\end{equation}
where $X$ ranges over test fields on $U_{t_2} \setminus \overline{U_{t_1}}$.
Both sides of (\ref{approx mean-value inequality}) are preserved as $n \to \infty$, due to the convergence in $L^2_l$.
Therefore (\ref{mean-value inequality}) holds.

Since $|du|_{TV}(U) < \infty$, measure continuity and (\ref{mean-value inequality}) implies that
$$\lim_{t_1, t_2 \to 0} \int_{\partial U} |v_{t_1} - v_{t_2}| ~dA = 0.$$
Thus $(v_t)$ is a Cauchy net in $L^1(\partial U)$ as $t \to 0$, say $v_t \to v$.
As $g(X, \nu) \in L^\infty$ and $(L^1)^* = L^\infty$, $g(v_tX, \nu) \to g(vX, \nu)$ in $L^1(\partial U)$.
Therefore the right-hand side of (\ref{interior Miranda IBP}) satisfies
$$\lim_{t \to 0} \int_{N_t} g(uX, \nu) ~dA = \lim_{t \to 0} \int_{\partial U} g(v_tX, \nu) ~dA \int_U g(vX, \nu) ~dA$$
which completes the proof.
\end{proof}

\begin{definition}
If $u,v,U$ are as in Theorem \ref{Miranda IBP}, and $N = \partial U$, then $v$ is called the \dfn{trace} of $u$ along $N$.
\end{definition}

\begin{corollary}[gluing $BV$ functions along a hypersurface]\label{gluing}
Let $N$ be a Lipschitz hypersurface which separates $M$ into $U_1,U_2$.
If $u_j \in BV(U_j)$ and $u \in L^1_l(M)$ is the function such that $u|U_j = u_j$, then $u \in BV(M)$.
Moreover, if $v_j$ is the trace of $u_j$ along $N$, then
\begin{equation}
\label{glued BV norm}
|du|_{TV}(N) = ||v_1 - v_2||_{L^1(N)}.
\end{equation}
\end{corollary}
\begin{proof}
For every test field $X$, Theorem \ref{Miranda IBP} gives
$$\langle du_1, X\rangle + \langle du_2, X\rangle + \int_M u~\mathcal L_XdV = \int_N (v_1 - v_2) g(\nu, X) ~dA,$$
where $\nu$ is the unit normal field on $N$ pointing towards $U_1$.
Therefore
$$|du|_{TV}(M) \leq |du_1|_{TV}(U_1) + |du_2|_{TV}(U_2) + ||v_1 - v_2||_{L^1(N)} < \infty.$$
Moreover, if we take $(X_n)$ to be an approximation to $\nu$, then $X = X_n$ is a maximizing sequence for $g(\nu, X)$ subject to $||X_n||_{L^\infty} \leq 1$.
Moreover, $\langle du_j, X_n\rangle \to 0$ as $n \to \infty$, since the measure of $\supp X_n$ shrinks down to $0$, and
$$\langle du_j, X_1\rangle \leq ||u_j||_{BV(U_j)} < \infty$$
(so we may apply measure continuity). Therefore, taking $n \to \infty$ we conclude (\ref{glued BV norm}).
\end{proof}

%Let us now extend the above results to maps between manifolds.
%Thus let $N$ be another Riemannian manifold.
%The definition of a distribution, even a vector-valued distribution, is local; therefore one can consider the space $\mathcal D'(M \to N)$ of distributions on $M$ valued in $N$.

%If $u: M \to N$ is a smooth map, then its derivative $du$ is a section of the normed bundle
%$$\Hom(T'M, u^*(TN)) = T'M \otimes u^*(TN).$$
%In particular, pairing $du$ with a test field $X$ gives a section $(du, X)$ of the pullback bundle $u^*(TN)$.
%Since the definition of differentiation is local, we can define the generalized derivative of $u \in \mathcal D'(M \to N)$ to be the generalized section $du$ of $T'M \otimes u^*(TN)$ such that for every test field in a trivial chart, the integration by parts formula (\ref{distributional derivative dfn}) holds.

\subsection{Coarea formulae}
Throughout this section we consider the superlevel sets $E_y = \{u \geq y\}$ of a function $u \in BV_l(M)$.

We first observe that for every $X \in \mathcal D(M, TM)$,
$$\langle d1_{E_y}, X\rangle = -\int_{E_y} \mathcal L_XdV$$
is measurable in $y$, since $E_y$ is monotone in $y$.
So by Pettis' theorem, $d1_{E_y}$ is measurable in $y$ with respect to the weak topology of measures.

\begin{proposition}[coarea formula for measures]\label{Coarea1}
If $u \in BV_l(M)$ and $E_y = \{u \geq y\}$, then
$$du = \int_{-\infty}^\infty d1_{E_y} ~dy.$$
\end{proposition}
\begin{proof}
Fix a test field $X \in \mathcal D(M, TM)$.
Breaking up $u$ into positive and negative parts, and exploiting the identity
$$\int_{-\infty}^\infty d1_{E_y} ~dy = \int_{-\infty}^0 d1_{E_y} ~dy + \int_0^\infty d1_{E_y} ~dy,$$
it is no loss to assume that $u \geq 0$.

We must show
\begin{equation}
\label{gradient is integral of fibers}
\langle du, X\rangle = \int_{-\infty}^\infty \langle d1_{E_y}, X\rangle ~dy.
\end{equation}
Since $u \geq 0$,
\begin{align*}
\langle du, X\rangle &= -\int_M u~\mathcal L_XdV = -\int_M \int_0^{u(x)} dy ~\mathcal L_XdV\\
&= -\int_0^\infty \int_{E_y} ~\mathcal L_XdV ~dy = \int_0^\infty \langle d1_{E_y}, X\rangle ~dy.
\end{align*}
by Fubini's theorem.
If $y < 0$ then $1_{E_y} = 1$ so $d1_{E_y} = 0$, so we conclude (\ref{gradient is integral of fibers}).
\end{proof}

Throughout the rest of this section we will be interested in the trivial line bundle
$$p: L \to M.$$
Let $\eta = dV \wedge dy$ be the natural volume form on $L$, induced by the natural metric $h$.
If $W$ is a vector field on $L$, we will write $W_1 = p_* W$ and $W_2$ for the projection of $W$ onto $\RR$.
Then Cartan's magic formula implies that if $W_2$ is constant, then
\begin{equation}
\label{Lie derivative computation}
\mathcal L_W\eta = \mathcal L_WdV \wedge dy.
\end{equation}

\begin{proposition}[coarea formula for $BV_l$ functions]\label{Coarea2}
If $u \in BV_l(M)$ and $E_y = \{u \geq y\}$ then
$$|du|_{TV} = \int_{-\infty}^\infty |d1_{E_y}|_{TV} ~dy.$$
\end{proposition}
\begin{proof}
By Proposition \ref{Coarea1} and the triangle inequality, one has
$$|du|_{TV} \leq \int_{-\infty}^\infty |d1_{E_y}|_{TV} ~dy.$$
So we just need to prove the converse.

Let $U \Subset M$.
Suppose that
\begin{quote}
For every $y \in \RR$, $X = X^{(n)}_y$ is a maximizing sequence for $\langle d1_{E_y}, X\rangle$ subject to $\supp X \subset U$ and $||X||_{L^\infty} \leq 1$.
\end{quote}
Since $d 1_{E_y}$ is measurable in $y$ with respect to the weak topology of measures, for every $n$, $X^{(n)}_y(x)$ can be chosen to be measurable in $(x, y)$; indeed, we can take $X^{(n)}_y$ to be a smooth approximation to the Radon measure $d 1_{E_y}/|d 1_{E_y}|_{TV}$ in the weak topology of distributions, which is a product of the measurable functions $y \mapsto d 1_{E_y}$ and $y \mapsto |d 1_{E_y}|_{TV}^{-1}$.

By an approximation argument, we can find test fields $W^{(n)}$ on $L$ such that $W^{(n)}_2 = 0$ and
\begin{quote}
For every $y \in \RR$, $X = W^{(n)}_1(\cdot, y)$ is a maximizing sequence for $\langle d 1_{E_y}, X\rangle$ subject to $\supp X \subset U$ and $||X||_{L^\infty} \leq 1$.
\end{quote}
Let us now suppress the $n$ and write $W(y) = W^{(n)}(\cdot, y)$.

Suppose that
\begin{equation}
\label{coarea converse}
\int_{-\infty}^\infty \langle d 1_{E_y}, W(y)\rangle ~dy \leq |d u|_{TV}(U)
\end{equation}
holds.
Then, since $W$ has compact support, the integrand $\langle d 1_{E_y}, W(y)\rangle$ is uniformly bounded in $y$.
Therefore, by Fatou's lemma,
\begin{align*}
\int_{-\infty}^\infty |d 1_{E_y}| ~dy &= \int_{-\infty}^\infty \lim_{n \to \infty} \langle d 1_{E_y}, W(y)\rangle ~dy \leq \liminf_{n \to \infty} \int_{-\infty}^\infty \langle d 1_{E_y}, W(y)\rangle ~dy \\
&\leq |d u|_{TV}(U),
\end{align*}
which completes the proof.

It remains to prove (\ref{coarea converse}).
Let
$$E = \{(x, y) \in L: x \in E_y\}$$
be the undergraph of $u$.
By Fubini's theorem and (\ref{Lie derivative computation}),
\begin{align*}
\int_{-\infty}^\infty \langle d 1_{E_y}, W(y)\rangle ~dy &= -\int_{-\infty}^\infty \int_{E_y} \mathcal L_WdV ~dy = -\iint_E ~\mathcal L_W\eta = \langle d 1_E, W\rangle.
\end{align*}

Let $(u_m)$ be a mollification of $u$, so that $u_m \to u$ in the weak topology of distributions.
Then if $\chi$ is a cutoff, $\langle u_m, \chi\rangle \to \langle u, \chi\rangle$; taking a sequence of $\chi$ which increase to the indicator function of a compact set $K$, we conclude that $u_m \to u$ in $L^1(K)$, and hence $u_m \to u$ in $L^1_l$.

Let $E^{(m)}$ be the undergraph of $u_m$, and $E^{(m)}_y = \{u_m \geq y\}$.
Then for every test function $v$,
\begin{align*}
\langle 1_{E^{(m)}} - 1_E, v\rangle &= \int_{E^{(m)} \Delta E} v ~dV \leq |(E^{(m)} \Delta E) \cap (\supp v \times \RR)| \cdot ||v||_{L^\infty}\\
&\leq ||v||_{L^\infty} \int_{\supp v} |u_m - u| ~dV \to 0
\end{align*}
so $1_{E^{(m)}} \to 1_E$ in the weak topology of distributions.
Therefore
$$\lim_{m \to \infty} \langle d 1_E^{(m)}, W\rangle = \langle d 1_E, W\rangle.$$

Since $u_m$ is smooth, its graph $F_m = \partial E^{(m)}$ is a smooth manifold.
Let $\nu_m$ be the upwards unit normal field of $F_m$ and let $dV_m$ be the volume form on $F_m$ induced by $\eta$.
Then
$$\langle d 1_{E^{(m)}}, W\rangle = -\iint_{E^{(m)}} \mathcal L_W\eta = -\int_{F_m} h(\nu_n, W) ~dV_m.$$
Let $q_m = p|F_m$ and $Y_m$ be the vector field $(Y_m)_1 = -d u_m$, $(Y_m)_2 = 1$.
Since $F_m$ is a graph, $q_m: F_m \to M$ is a diffeomorphism, $(q_m)_*\nu_m = Y_m/|Y_m|$, and $(q_m)_*dV_m = |Y_m|~dV$. Therefore
$$\int_{F_m} h(\nu_n, W) ~dV_m = \int_M h(Y_m, W) ~dV = \int_M g(d u_m, W_1) ~dV = \langle d u_m, W_1\rangle.$$
Thus
$$|\langle d 1_E, W\rangle| = \lim_{m \to \infty} |\langle d u_m, W_1\rangle| \leq |d u|_{TV}(U),$$
which completes the proof of (\ref{coarea converse}).
\end{proof}

Let us see how Proposition \ref{Coarea2} implies the classical coarea formula
\begin{equation}
\label{CoareaClassic}
\int_M f |du| ~dV = \int_{-\infty}^\infty \int_{\{u = y\}} f ~dA ~dy
\end{equation}
for locally Lipschitz functions $u: M \to \RR$.
Note that (\ref{CoareaClassic}) makes no sense if $u$ is merely a continuous function of locally bounded variation, as the Devil's staircase is a counterexample.

\begin{lemma}\label{Rademacher manifold}
If $\partial U$ is a Lipschitz hypersurface then $|d1_U|_{TV} = |d1_{\overline U}|_{TV}$ is the surface measure on $\partial U$ induced by $dV$.
\end{lemma}
\begin{proof}
Given a test field $X$,
$$\langle d1_U, X\rangle = -\int_U \mathcal L_XdV = -\int_{\partial U} g(X, \nu) ~dA.$$
Thus if $V$ is open in $\partial U$,
$$|d1_U|_{TV}(V) = \sup_{||X||_{L^\infty} \leq 1} \int_V g(X, \nu) ~dA = \int_V |\nu|_g^2 ~|dA| = \int_V ~|dA|$$
as desired.
Moreover, as $\partial U$ is locally the Lipschitz image of a hyperplane, $\partial U$ is null, so $1_U = 1_{\overline U}$ almost everywhere and hence in the sense of generalized sections.
\end{proof}

\begin{theorem}[coarea formula]
If $u: M \to \RR$ is a locally Lipschitz function and $f \in L^1_l(M)$ then (\ref{CoareaClassic}) holds.
\end{theorem}
\begin{proof}
By Rademacher's theorem, $du \in L^\infty_l(M, T'M)$, and $|du|$ is the Radon-Nikod\'ym derivative of $|du|_{TV}$.
In particular, $f|du| \in L^1_l$.
Moreover, the set $\{du = 0\}$ is null and so does not affect the left-hand side of (\ref{CoareaClassic}); after discarding this null set, $\{u = y\}$ becomes a Lipschitz hypersurface.
Therefore Lemma \ref{Rademacher manifold} and Proposition \ref{Coarea2} immediately imply (\ref{CoareaClassic}).
\end{proof}

\begin{corollary}[integration in polar coordinates]\label{polar coords}
If $f \in L^1_l(B(x, R))$ then
\begin{equation}\label{polar coords formula}
\int_{B(x, R)} f ~dV = \int_0^R \int_{\partial B(x, r)} f ~dA ~dr.
\end{equation}
\end{corollary}
\begin{proof}
Let $u(y) = d(x, y)$. Then the right-hand side of (\ref{polar coords formula}) is exactly the right-hand side of the classical coarea formula (\ref{CoareaClassic}).
Moreover, $du|\partial B(x, r)$ is the unit conormal $1$-form to $\partial B(x, r)$, so $|du| = 1$.
\end{proof}

I think that, in particular, one can deduce coarea formulae for BV or Lipschitz maps between Riemannian manifolds from the above.
But I haven't tried to do so.

\subsection{Functions of least gradient}
\begin{definition}
Let $u \in BV_l(M)$. We say that $u$ has \dfn{least gradient} if for every $v \in BV_c(M)$ and $U \supseteq \supp v$,
$$|d u|_{TV}(U) \leq |d u + d v|_{TV}(U).$$
\end{definition}

\begin{lemma}\label{least gradient splitting}
If $u$ has least gradient and $u_1 = \max(u - \lambda_0, 0)$, $u_2 = \min(u, \lambda_0)$, then $u_1,u_2$ have least gradient.
\end{lemma}
\begin{proof}
Let $\varphi_\lambda$ be the indicator function of $\{u \geq \lambda\}$, and similarly $\varphi_\lambda^{(j)}$ the indicator function of $\{u_j \geq \lambda\}$.
By construction, $u_1,u_2 \in BV_l$.
Moreover, viewing $d \varphi_\bullet^{(j)}$ as a function
$$d \varphi_\bullet^{(j)}: \RR \to \mathcal R(M, TM)$$
we see that $\supp d \varphi_\bullet^{(1)} \subseteq (-\infty, 0]$ and $\supp d \varphi_\bullet^{(2)} \subseteq [\lambda_0, \infty)$.
Therefore $d \varphi_{\bullet + \lambda_0}^{(1)}$ and $d \varphi_\bullet^{(2)}$ have disjoint supports, and so
$$d \varphi_{\bullet + \lambda_0}^{(1)} + d \varphi_\bullet^{(2)} = d \varphi_\bullet$$
almost everywhere.
So
\begin{align*}
|d u|_{TV}
&=\int_{-\infty}^\infty |d \varphi_\lambda|_{TV} ~d\lambda\\
&= \int_{-\infty}^\infty |d \varphi_{\lambda+\lambda_0}^{(1)}|_{TV} + |d \varphi_\lambda^{(2)}|_{TV} ~d\lambda \\
&= \int_{-\infty}^\infty |d \varphi_\lambda^{(1)}|_{TV} ~d\lambda + \int_{-\infty}^\infty |d \varphi_\lambda^{(2)}|_{TV} ~d\lambda\\
&= |d u_1|_{TV} + |d u_2|_{TV}
\end{align*}
by the coarea formula. Since $u$ has least gradient, for every $v \in BV_c$ and $U \subseteq \supp v$,
\begin{align*}|d u_1|_{TV}(U) + |d u_2|_{TV}(U) &= |d u|_{TV}(U) \leq |d(u+v)|_{TV}(U) \\
&\leq |d(u_j + v)|_{TV}(U) + |d u_{j'}|_{TV}(U)
\end{align*}
whenever $(j,j')$ is a permutation of $(1,2)$.
Subtracting $|d u_{j'}|_{TV}(V)$ from both sides we conclude that if $j \in \{1, 2\}$,
$$|d u_j|_{TV}(U) \leq |d(u_j + v)|_{TV}(U)$$
so that $u_j$ has least gradient.
\end{proof}

Let us now show that the space of functions of least gradient is closed \cite[Teorema 3]{Miranda67}.

\begin{definition}
Let $(u_n)$ be a sequence in $BV_l(M)$. We say that a Lipschitz hypersurface $N$ is \dfn{good} for $(u_n)$ if:
\begin{enumerate}
\item \label{cond1Mir} $\sup_n |du_n|_{TV}(N) = 0$.
\item \label{cond2Mir} $\limsup_n ||u_n||_{L^1(N)} < \infty$.
\item \label{cond3Mir} If $u_n \to u$ in $L^1_l(M)$ and $u \in BV_l(M)$, then $|du|_{TV}(N) = 0$.
\item \label{cond4Mir} If $u_n \to u$ in $L^1_l(M)$ and $u \in BV_l(M)$, then $u_n \to u$ in $L^1(N)$.
\end{enumerate}
By Condition $k$ we mean the $k$th bullet in the above list.
\end{definition}

\begin{lemma}\label{probabilistic method}
Let $(u_n)$ be a sequence in $BV_l(M)$ which converges in $L^1_l(U)$. Then for every $x \in M$ and $R > 0$ such that $B(x, R) \Subset M$, there exists $r \in (0, R]$ such that $\partial B(x, r)$ is good.
\end{lemma}
\begin{proof}
Let $I = [\varepsilon, R]$ for some (unimportant) $\varepsilon \in (0, R)$.
Let $r$ be drawn from $I$ uniformly at random.
We will show that almost surely, $B(x, r)$ is a good hypersurface.

We first show that Conditions \ref{cond1Mir} and \ref{cond3Mir} hold almost surely.
Let $A_n = \{|du_n|_{TV}(N) > 0\}$ and let $A_\infty = \{|du|_{TV}(N) > 0\}$.
Then for every $n \in \NN \cup \{\infty\}$, writing $u_\infty = u$,
$$\sum_{s \in A_n} |du_n|_{TV}(\partial B(x, s)) \leq |du_n|_{TV}(B(x, R)) < \infty$$
since $|du_n|_{TV}$ is a Radon measure and $B(x, R) \Subset M$.
Since each of the summands is nonzero by definition of $A_n$, it follows that $A_n$ is countable, and in particular null.

We now show that Condition \ref{cond2Mir} holds almost surely.
This follows from an integration in polar coordinates, Corollary \ref{polar coords}. In fact,
$$E \limsup_{n \to \infty} ||u_n||_{L^1(\partial B(x, r))} = |I|^{-1} \limsup_{n \to \infty} ||u_n||_{L^1(B(x, R))} < \infty,$$
since $(u_n)$ is bounded in $L^1_l$ and $B(x, R) \Subset M$.
Therefore
$$\limsup_n ||u_n||_{L^1(\partial B(x, r))} < \infty$$
almost surely.

Finally we show that Condition \ref{cond4Mir} holds almost surely.
By another integration in polar coordinates,
\begin{align*}\limsup_{n \to \infty} E ||u_n - u||_{L^1(\partial B(x, r))} &\leq E \limsup_{n \to \infty} ||u_n - u||_{L^1(\partial B(x, r))}\\
&= \lim_{n \to \infty} \frac{||u_n - u||_{L^1(B(x, R))}}{|I|} = 0,
\end{align*}
since $B(x, R) \Subset M$.
So $||u_n - u||_{L^1(\partial B(x, r))} \to 0$ almost surely.
\end{proof}

\begin{theorem}[Miranda's convergence theorem]\label{Miranda convergence}
The space of functions of least gradient is closed in $L^1_l(M)$.
\end{theorem}
\begin{proof}
Let $(u_n)$ be a sequence of functions of least gradient, and suppose $u_n \to u$ in $L^1_l(M)$.
In particular, $(u_n)$ is bounded in $L^1_l$.

Fix $U \Subset M$. Let us find $U \subseteq V \Subset M$ such that $\partial V$ is a good hypersurface.
To find such a $V$, we first choose $U \Subset W \Subset M$, and for every $x \in \partial U$, let $R_x \in (0, d(x, \partial W))$.
Then, by Lemma \ref{probabilistic method}, for every $x \in \partial U$, there exists $r_x \in (0, R_x)$ such that $\partial B(x, r_x)$ is a good hypersurface.
Let $\mathcal U$ be the open cover of $\overline U$ given by the balls $B(x, r_x)$, as well as $U$ itself.
Since $\overline U$ is compact, there exists a finite subcover $\mathcal U_0$ of $\mathcal U$.
Let $V$ be the union of the sets in $\mathcal U_0$.
Then $\partial V$ is a union of finitely many good spheres $\partial B(x, r_x)$, and therefore is good.

Let $v_n = (1 - 1_V)u_n$.
By the gluing Corollary \ref{gluing}, the fact that $u_n$ has least gradient, and Condition \ref{cond1Mir},
$$|du_n|_{TV}(U) \leq |du_n|_{TV}(\overline V) \leq |dv_n|_{TV}(\overline V) = ||u_n||_{L^1(\partial V)}.$$
Thus, by (\ref{RieszMarkovDistr}) and Condition \ref{cond2Mir},
$$|du|_{TV}(U) \leq \limsup_{n \to \infty} |du_n|_{TV}(U) \leq \limsup_{n \to \infty} ||u_n||_{L^1(\partial V)} < \infty.$$
Therefore $u \in BV_l(M)$.

Since we have just shown $u \in BV_l(M)$, it remains to show that if $v \in BV_l(M)$ and $u = v$ on $U$, then
$$|du|_{TV}(U) \leq |dv|_{TV}(U).$$
Again using the gluing corollary, if we set
$$v_n(x) = \begin{cases}
v(x), &x \in V\\
u_n(x), &x \notin V
\end{cases}$$
(where we use the support property of $v$ to extend $v$ by $0$ to all of $M$), then, by Condition \ref{cond3Mir},
$$|dv_n|_{TV}(\overline V) = |dv|_{TV}(V) + \int_{\partial V} |u - u_n| ~dA.$$
Since $u_n$ has least gradient, Condition \ref{cond1Mir} implies
$$|du_n|_{TV}(V) = |du_n|_{TV}(\overline V) \leq |dv_n|_{TV}(\overline V) = |dv|_{TV}(V) + \int_{\partial V} |u - u_n| ~dA.$$
By Condition \ref{cond4Mir} and (\ref{RieszMarkovDistr}), it follows that
$$|du|_{TV}(U) \leq |du|_{TV}(V) \leq |dv|_{TV}(V) = |dv|_{TV}(U)$$
as desired.
\end{proof}

\begin{definition}
A \dfn{Caccioppoli set} is a Borel set $E \subseteq M$ such that $1_E \in BV_l(M)$.
\end{definition}

For example, if $E$ is $[0, 1/2] \cup [3/4, 7/8] \cup [15/16, 31/32] \cup \cdots$ then $|1_E'|_{TV}$ consists of Dirac masses at $1 - 2^{-j}$ for each $j$, and so is not Radon near $1$, so $E$ is not Caccioppoli.
The idea is that $|d 1_E|_{TV}$ is a positive Radon measure on the boundary of $E$, which serves as a sort of ``perimeter measure" for $E$.

\begin{definition}
Let $E$ be a Caccioppoli set. We say that $\partial E$ has \dfn{least area}, or $E$ has \dfn{least perimeter}, if $1_E$ has least gradient.
\end{definition}

Let us give a sufficient condition for a set to have least perimeter \cite[Theorem 1]{BOMBIERI1969}.

\begin{theorem}
For every function $u$ of least gradient, every superlevel set of $u$ has least perimeter.
\end{theorem}
\begin{proof}
Let $\varphi_\lambda$ be the indicator function of $\{u \geq \lambda\}$ and let
$$\varphi_{\lambda,\varepsilon} = \min\left(1, \frac{\max(u - \lambda, 0)}{\varepsilon}\right).$$
Then, since $\max(u - \lambda, 0)$ has least gradient, so does $\min(\max(u - \lambda, 0), \varepsilon)$, and so does $\varphi_{\lambda_\varepsilon}$; here we have repeatedly applied Lemma \ref{least gradient splitting}.

First suppose that $\{u = \lambda\}$ is null. One has
$$\varphi_{\lambda,\varepsilon} = \begin{cases}
0, &u \leq \lambda\\
\frac{u - \lambda}{\varepsilon}, &\lambda < u \leq \lambda + \varepsilon\\
1, & u > \lambda + \varepsilon,
\end{cases}$$
and in particular $\varphi_{\lambda,\varepsilon}$ maps to $[0, 1]$. Thus for every $V \Subset U$,
$$\int_V |\varphi_{\lambda,\varepsilon} - \varphi_\lambda| = \int_{V \cap \{\lambda \leq u < \lambda + \varepsilon\}} \left|\frac{u - \lambda}{\varepsilon} - 1\right| \leq |V \cap \{\lambda \leq u < \lambda + \varepsilon\}|.$$
Since $|V| < \infty$ and $|\{u = \lambda\}| = 0$, the right-hand side goes to $0$; since $V$ was arbitrary, $\varphi_{\lambda,\varepsilon} \to \varphi_\lambda$ in $L^1_l$.
So by Miranda's convergence theorem, Theorem \ref{Miranda convergence}, $\varphi_\lambda$ has least gradient (in particular, $\{u \geq \lambda\}$ is Caccioppoli).

Now consider the case of general $\lambda$.
There are only countably many $\lambda'$ with $|\{u = \lambda'\}| > 0$, so we can find a sequence of $\lambda_m < \lambda$ which increase to $\lambda$ such that $\{u = \lambda_m\}$ is null.
Thus, by the previous case, $\varphi_{\lambda_m}$ has least gradient.
For every $V \Subset U$,
$$\int_V |\varphi_{\lambda_m} - \varphi_\lambda| = |\{\lambda_m \leq u < \lambda\}|$$
which goes to $0$ as $m \to \infty$. Therefore by Miranda's convergence theorem, $\varphi_\lambda$ has least gradient.
\end{proof}

\section{The reduced boundary}
\subsection{The direction of a generalized $1$-form}
Let $S_0M$ be the fiber bundle of all $(x, v) \in TM$ such that $|v| \in \{0, 1\}$.
Thus $S_0M$ consists of the sphere bundle of $M$ as well as the zero section of the tangent bundle.

If $\mu$ is a Radon measure, $\tau$ is a $\mu$-measurable section of $S_0M$, and $X$ is a test field, then $|g(\tau, X)| \leq |X|$, so that $g(\tau, X) \in L^1(\mu)$.

\begin{theorem}[Hanh-Jordan]
Let $\omega$ be a $T'M$-valued Radon measure and let $\mu = |\omega|_{TV}$. Then there exists a $\mu$-measurable section $\tau$ of $S_0M$ such that for every test field $X$,
\begin{equation}
\label{Hanh Jordan for 1-forms}
\langle \omega, X\rangle = \int_M g(\tau, X) ~d\mu
\end{equation}
and $\supp \tau \subseteq \supp \omega$.
\end{theorem}
\begin{proof}
Suppose we are given a countable partition $\mathcal U$ of $\mu$-almost all of $M$ into open sets $\Subset M$.
For each $U \in \mathcal U$,
\begin{quote}
Let $\tau^U = \tau^U_n$ be a maximizing sequence for $\langle \omega, \tau^U\rangle$ subject to $||\tau^U||_{L^\infty(\mu)} \leq 1$ and $\supp \tau^U \subseteq \supp 1_U\omega$.
\end{quote}
Then
$$||\tau^U_n||_{L^2(\mu)} \leq \mu(U) \cdot ||\tau^U_n||_{L^\infty(\mu)}^2 \leq |\omega|_{TV}(U) < \infty$$
so by the Banach-Alaoglu theorem, we may assume that there is a $\tau^U \in L^2(\mu)$ such that $\tau^U_n \to \tau^U$ in the weak topology of $L^2(\mu)$ and $\supp \tau^U \subseteq \supp \omega$.
Since $X \in L^2(\mu)$, this implies that $\tau^U$ is a maximizer of $\langle \omega, \tau^U\rangle$.

Let us now check that $\tau^U$ is a section of $S_0M$.
Clearly $||\tau^U||_{L^\infty} \leq 1$ and $x|(\supp 1_U\omega)^c = 0$.
Now suppose that there is a Borel set $E \subseteq \supp 1_U\omega$ such that $|\tau^U|E| < 1$ and $|E^U| > 0$.
Then we can increase $\langle \omega, \tau^U\rangle$ by replacing $\tau^U|E$ with $\tau^U|E/|\tau^U_E|$, contradicting that $\tau^U$ is a maximizer.
So $\tau^U$ is a section of $S_0M$.

Let $\tau = \sum_U 1_U \tau^U$.
Then $\tau$ is a section of $S_0M$ with $\supp \tau \subseteq \supp \omega$, since the $\tau^U$ all vanished outside of $\supp \omega$, and on $\supp \omega \cap U$, $\tau = \tau^U$.
The equation (\ref{Hanh Jordan for 1-forms}) follows from the definition of total variation measure and the fact that $\tau^U$ is a maximizer for every $U$.
\end{proof}

If the above argument works the way I want it to, then
$$\tau(x) = \lim_{\delta \to 0} \frac{\omega(B(x, \delta))}{|\omega|_{TV}(B(x, \delta))}.$$

\begin{definition}
If (\ref{Hanh Jordan for 1-forms}) holds, we call $\tau$ the \dfn{direction} of $\omega$.
\end{definition}

\begin{example}
The hypothesis that $\omega$ is a Radon measure seems to be sharp.
Consider $\omega = \delta_0 - \delta_0'$. Then
$$\langle \omega, X\rangle = X(0) - X'(0)$$
so we can't decide whether $\tau$ points to the right or to the left.
\end{example}

\subsection{The reduced boundary}
If $u = 1_U$ and $N = \partial U$ is Lipschitz, then $du$ is a section of the conormal bundle of $N$.
To see this, let $X$ be a test field which is tangent to $\partial U$.
Then
$$\langle du, X\rangle = -\int_U \mathcal L_XdV = -\int_{\partial U} g(X, \nu) ~dA = 0$$
as desired.
Therefore the direction of $du$ is nothing more than the inward unit normal field of $N$.
That is, we may generalize the definition of normal field, as follows.

\begin{definition}
Let $U$ be an open Caccioppoli set and $\nu$ the direction of $d1_U$.
The support $\partial^* U$ of $\nu$ is known as the \dfn{reduced boundary} of $U$.
We call $\nu$ the \dfn{Gauss map} or \dfn{normal field} of $\partial^* U$.
We call the span $N\partial^* U$ of $\nu$ the \dfn{normal bundle} to $\partial^* U$, and its orthocomplement the \dfn{tangent bundle} $T\partial^* U$.
\end{definition}

Note that this is not quite the reduced boundary in the sense of the Wikipedia page on Caccioppoli sets -- that page defines $\partial^* U$ to be the set on which $\nu$ has length $1$, while I think my $\partial^* U$ is the closure of that set.
In any case, de Giorgi's theorem implies that the difference between the two sets is $d-1$-null and so isn't terribly important.

\begin{theorem}[de Giorgi]
Let $N$ be the reduced boundary of an open Caccioppoli set $U$.
Then $N$ is rectifiable of codimension $1$ and $TN$ is the approximate tangent bundle of $N$.
Moreover, $|d1_U|_{TV}$ is the codimension-$1$ Hausdorff measure on $N$.
\end{theorem}

I need to prove this in general but for now I'll just take it as a black box.
It should follow for Riemannian manifolds from Chapter 4 of de Guisti's book.

\begin{proposition}
If $U$ has least perimeter and $\partial^* U$ is smooth then $\partial^* U$ is the disjoint union of minimal hypersurfaces.
\end{proposition}
\begin{proof}
Suppose that $N = \partial^* U$ is not the disjoint union of minimal hypersurfaces and $\mu$ is the codimension-$1$ Hausdorff measure on $N$.
So there is a contractible open set $B$ in $M$, which we may identify with the unit ball in $\RR^d$, and a hypersurface $N'$ with codimension-$1$ Hausdorff measure $\mu'$, such that $N \setminus B = N' \setminus B$, $N \cap B$ is a hypersurface in $B$, and $\mu'(N' \cap B) < \mu(N \cap B)$.
After a change of coordinates, we may assume that $N' \cap B$ is the plane $\{x_d = 0\} \cap B$ and that $U \cap B = \{x_d < f(x_1, \dots, x_{d-1})\} \cap B$ for some function $f$.

Let
$$U' = (U \setminus B) \cup (\{x_d < 0\} \cap B).$$
Then $|d1_{U'}|_{TV} = \mu'$, so $\partial^* U' = N'$.
Then by minimality, $\mu(N \cap B) \leq \mu'(N' \cap B) < \mu(N \cap B)$, a contradiction.
\end{proof}

Since the only minimal curves are geodesics, we have:

\begin{corollary}
If $M$ is a surface and $U$ has least perimeter, then $\partial^* U$ is not smooth or $\partial^* U$ is the disjoint union of geodesics.
\end{corollary}

Note: the above is NOT what we want! It presupposes that $\partial^* U$ is smooth.

\subsection{The weak maximum principle}
By ``$M$ has low dimension" I mean that there are no singular tangent cones to $M$.

\begin{conjecture}
If $d \leq 7$ then $M$ has low dimension.
\end{conjecture}

I think that the correct formulation of the weak maximal principle is:

\begin{conjecture}[weak maximum principle -- G\'orny Prop 3.4]
Assume that $M$ has low dimension and $U$ has least perimeter.
Suppose that $d = 2$ or $M,g$ are both analytic.
Then $\partial U$ is the locally finite union of smooth minimal hypersurfaces.
\end{conjecture}

\begin{corollary}[G\'orny Prop 3.5]
Assume that the weak maximum principle is true, $d = 2$, and $M$ has low dimension..
If $M$ is geodesically convex, and $U$ has least perimeter, then $\partial^* U$ is the disjoint union of geodesics.
\end{corollary}
\begin{proof}
We know that $\partial U$ is the locally finite union of geodesics.
Suppose that $\overline{xy}$ and $\overline{xz}$ are geodesics in $\partial U$.
Since $M$ is geodesically convex, let $\Delta \subset M$ be the geodesic triangle enclosed by $x,y,z$.
Since $\partial U$ is a locally finite union we may assume that $\partial U$ does not meet the interior of $\Delta$.
Then either $\Delta \subseteq \overline U$, in which case we can reduce the perimeter by removing $\Delta$ from $\overline U$ to replace $\overline{xy} \cup \overline{xz}$ with the shorter segment $\overline{yz}$, or $\Delta \cap \overline U$ has lower codimension, in which we can reduce the perimeter by adding $E$.
\end{proof}

I think that we can remove the assumption that $M$ is geodesically convex in case $u = 1_U$, since having least perimeter is a local property in $\partial U$ and since it seems like manifolds should be locally geodesically convex.
If we can't remove that assumption, then a ``wavy" subset of the plane, or $S^2$, would be a counterexample probably.
Actually we really want to weaken the assumption that $M$ is geodesically convex since $S^2$ is NOT -- two antipodal points lie on the same great circle in two different ways.

\begin{conjecture}[Hilbert's 19th problem]
Assume that $M$ has low dimension and $N \subseteq M$ is a minimal hypersurface.
Then for every $x \in M$ and $r > 0$, $N \cap B(x, r)$ is smooth (as an immersed hypersurface, I think??)
If in addition $M$ and $g$ are analytic, then so is $N \cap B(x, r)$.
\end{conjecture}

This will be the main roadblock.

\begin{theorem}[monotonicity formula, Cod\'a-Marques Thm 7.11]
Let $N$ be a minimal hypersurface in $M$ and $x \in M$.
Let $v(r)$ be the area of $N \cap B(x, r)$ and let $r_0 > 0$ be the injectivity radius of $x$.
Then there exists $A > 0$ such that $e^{Ar^2} v(r)/r^{d - 1}$ is nondecreasing in $r \in (0, r_0)$.
\end{theorem}

This was already proven for Riemannian manifolds by Cod\'a-Marques, so that we don't need to reprove it here.

\begin{lemma}
Let $N$ be a minimal hypersurface in $M$ and $K \subset M$.
Then $N \cap K$ has finitely many components.
\end{lemma}
\begin{proof}
Suppose that there are infinitely many components $N_j$ in arbitrarily small neighborhoods of a point $x$.
Then $N_j$ is minimal, so if $v_j(r)$ is the area of $N_j \cap B(x, r)$ and $r$ is small enough,
$$v_j(r) \geq \frac{r^{d - 1}}{2}.$$
Thus the sequence $(v_j(r))_j$ is not summable, so $N \cap K$ has infinite area and therefore is not rectifiable, a contradiction.
\end{proof}

\begin{conjecture}[local uniqueness]
Suppose that $E \subseteq F$ have least perimeter. If $x \in \partial E \cap \partial F$, then the germs of $\partial E$ and $\partial F$ at $x$ agree.
\end{conjecture}

This follows from the maximum principle for the minimal surface equation, as proven in SWZ (Theorem 2.2).
There's a bit of a chain of dependencies here because we need the maximum principle!
But for the geodesic equation
$$\ddot \gamma^k + {\Gamma^k}_{ij} \dot \gamma^i \dot \gamma^j = 0,$$
I guess it just follows from the fact that the equation is uniformly elliptic.

\begin{proof}[Reduction of WMP to above conjectures]
By the above, $N$ is a locally finite union of smooth manifolds (analytic if $d \geq 3$).
At any point $x$ of self-intersection, the components of $N$ have identical germs.
If $d = 2$ then $N$ solves the geodesic equation.
The analyticity (if $d \geq 3$) or uniqueness for geodesics (if $d \geq 2$) implies that all the components of $N$ passing through $x$ are identical.
So only one component passes through $x$ and $N$ is embedded rather than immersed.
\end{proof}

\subsection{Mean curvature}
On $N$, $\nu$ is defined almost everywhere. At a point $x$ where $\nu$ is defined, consider the geodesic $\gamma_x$ starting at $x$ in the direction $\nu(x)$.
We extend $\nu$ to almost all of a neighborhood of $N$ by setting $\nu(\gamma_x(t)) = \nu(x)$.

We define the covariant derivative of a generalized vector field $X$ by
$$\langle \nabla X, Y\rangle = \langle X^i, {\Gamma^k}_{ij}Y^j\rangle e_k - \langle X^k, Y^j_{,j}\rangle e_k$$
where $Y$ ranges over test fields with support in a coordinate frame $\partial_i$ with Christoffel symbols ${\Gamma^k}_{ij}$.
Thus $\nabla X$ is a generalized section of $\Hom(TM, TM)$.

\begin{definition}
Let $N$ be a reduced boundary with Gauss map $\nu$.
The \dfn{second fundamental form} $\mathbb I$ of $N$ is the generalized section of $\Hom(TN, TN)$ given by $\Pi(\nabla \nu|TN)$ where $\Pi: TM \to TN$ is the orthogonal projection.
The \dfn{mean curvature} $H$ of $N$ is defined by $(d - 1)H = \tr \mathbb I$.
\end{definition}

\begin{example}
Suppose that $U$ is the open Quadrant I in $\RR^2$.
Then for $x > 0$, $\nu(x, 0) = (0, 1)$, and for $y > 0$, $\nu(0, y) = (1, 0)$.
We can extend this function to a function on a neighborhood $V$ of $\partial U$ by putting $\nu(W_1) = (0, 1)$ and $\nu(W_2) = (1, 0)$ where $W_j$ is a neighborhood of the $j$th axis such that $W_1 \cap W_2$ is empty and $V = W_1 \cup W_2$ covers $\partial U$.

We now let $\gamma$ be the parametrization of $\partial U$ by $\gamma(t) = (0, -t)$ for $t < 0$ and $\gamma(t) = (t, 0)$ for $t \geq 0$.
Then $\gamma$ is a unit-speed parametrization with $\dot \gamma(t) = (0, -1)$ for $t < 0$ and $\dot \gamma(t) = (1, 0)$ for $t > 0$.

Let $\kappa$ be the (mean) curvature of $\partial U$. Then one has
$$2\kappa = \sum_k \ddot \gamma_k = 2\delta_0.$$
This is what one would expect.
\end{example}

\begin{example}
Suppose that $U$ is the undergraph of a continuous function $f \in BV_l(\RR)$.

Let $\gamma(t) = (t, f(t))$ be the standard parametrization of $\partial U$.
Then $\nu$ is the direction $\tau$ of $\dot \gamma$ rotated $90$ degrees.
This makes sense since $\dot \gamma(t) = (1, f'(t))$ is a Radon measure (since $f \in BV_l$).
In particular it makes sense even if the derivative of $f$ is a Cantor measure!

We can extend $\nu$ to $\RR^2$ by $\nu(x, y) = \nu(x, f(x))$.
More precisely, if $X$ is a test field, then
$$\langle \nu, X\rangle = \left\langle p_*\nu, \int_{-\infty}^\infty X(\cdot, y) ~dy\right\rangle$$
where $p: \partial U \to \RR$ is the natural projection.

Once this is done we get $\partial_y\nu = 0$ while $\partial_x\nu$ is $\partial_x \tau$ rotated $90$ degrees.
\end{example}

\begin{lemma}\label{derivative of TV}
The Fr\'echet derivative of $|\cdot|_{TV}$ in the weak topology of $T'M$-valued Radon measures is the flattened direction map.
\end{lemma}
\begin{proof}
Let $u, v$ be Radon measures valued in $T'M$, with $\tau$ the direction of $u$, and let $0 < h \lesssim 1$.
Let $u_h = |u + hv|_{TV} - |u|_{TV} - h(v, \tau)$
We must show that for every test function $f \in \mathcal D(M)$,
$$\langle u_h, f\rangle \ll_f |h|.$$
Let $\tau_h$ be the direction of $u + hv$, so that $|u + hv|_{TV} = (u + hv, \tau_h)$ and $\tau = \tau_0$.
Let $X = \langle u, f\rangle^\sharp$ and $Y = \langle v, f\rangle^\sharp$, thus $X/|X| = \tau$ and $(X + hY)/|X + hY| = \tau_h$.
Then
\begin{align*}
\langle u_h, f\rangle &= \langle (u, \tau_h - \tau) + h(v, \tau_h - \tau), f\rangle = g(X, \tau_h - \tau) + hg(Y, \tau_h - \tau)\\
&= g(X + hY, \tau_h) - g(X, \tau) - hg(Y, \tau) = |X + hY|_g - |X|_g - hg(Y, \tau).
\end{align*}
But the classical version of this lemma (for vectors in a finite-dimensional vector space) says that
$$|X + hY|_g - |X|_g - hg(Y, \tau) \ll h,$$
so we conclude the claim.
\end{proof}

\begin{proposition}
Suppose that $U$ has least perimeter.
Then $\partial^*U$ has no mean curvature.
\end{proposition}
\begin{proof}
Let $\nu$ be the Gauss map of $\partial^*U$.
We extend $\nu$ to a vector field $\nu'$ on a neighborhood of $\partial^* U$ by letting $\nu'(x)$ be the mean of $\nu(y)$ taken over the set of all points $y \in \partial^* U$ which minimize $d(x, y)$, if this mean indeed exists.

Let $u \in \mathcal D(M)$.
Applying Lemma \ref{derivative of TV} and the chain rule, if $\tau_h$ denotes the direction of $(\nu')^\flat + d(hu)$, then
$$\partial_h |(\nu')^\flat + d(hu)| = \langle \tau_h, du\rangle.$$
Thus we conclude that $\langle \tau_0, du\rangle = 0$.
But we also have $\nu' = \tau_0$, so $\langle \nu', du\rangle = 0$.
Integrating by parts,
$$0 = \int_M (\nu', du) ~dV = -\int_M u \Div \nu' ~dV.$$
Since $u$ was arbitrary we conclude that $\Div \nu' = 0$, so $H = 0$.
\end{proof}

This gives another proof that the only sets of least perimeter on a surface have geodesic reduced boundaries, since it implies that $\partial^* U$ has no geodesic curvature (which is an ODE and therefore has only smooth solutions).

\section{The honest serious proof}
\subsection{Construction of internal cones}
This is the right way to set up mean curvature.
We also need it to develop the de Giorgi-Nash-Moser stuff.
What I call an internal cone, Giusti calls a tangent cone.

Fix $p \in M$ with injectivity radius $r_0 > 0$.
We write $B'(v, r)$ for a ball in $T_pM$.
In particular, we have a diffeomorphism
$$\exp: B'(0, r_0) \to B(p, r_0).$$
Suppose that $U$ is an open Caccioppoli subset of $B(p, r_0)$, and let $V = \exp^{-1}(U)$.
Then, since diffeomorphisms preserve the class of Radon measures, $V$ is an open Caccioppoli subset of $B'(0, r_0)$.
We let $V_t = \{v \in T_pM: tv \in V\}$, $\psi = 1_V$, and $\psi_t = 1_{V_t}$.

Choose an orthonormal basis $\Lambda = \{\lambda_1, \dots, \lambda_d\}$ of $T_pM$.
The basis $\Lambda$ induces geodesic normal coordinates $x_1, \dots, x_d$ on $M$ centered at $p$ and a Lebesgue volume form
$$d\lambda = d\lambda_1 \wedge \cdots \wedge d\lambda_d$$
on $T_pM$.
One has
\begin{equation}\label{pushforward of flat measure}
\exp_* d\lambda = dx_1 \wedge \cdots \wedge dx_d = \frac{dV}{\sqrt{|\det g|}}
\end{equation}
where $g$ is written in the coordinates $x_1, \dots, x_d$.
Therefore $|g - \delta| \lesssim |x|^2$ and hence, since $\det$ is a smooth map,
\begin{equation}\label{determinant bounds}
|\sqrt{|\det g|} - 1| \lesssim |x|^2.
\end{equation}

Fix $r \in (0, r_0)$; all implied constants below are allowed to depend on $r$, but crucially, are uniform in $t \in (0, 1)$.

\begin{lemma}\label{linearized perimeter bound 1}
One has the bound
\begin{equation}
\label{linearized perimeter bound}
|d\psi_t|_{TV}(B'(0, r)) \lesssim t^{1 - d} |d1_U|_{TV}(B(p, tr))
\end{equation}
uniformly in $t$.
\end{lemma}
\begin{proof}
One has
\begin{align*}
|d\psi_t|_{TV}(B'(0, r)) &= t^{1 - d} |d\psi|_{TV}(B'(0, tr)) = t^{1 - d} |d\psi|_{TV}(\exp^* B(p, tr))\\
&= t^{1 - d} \sup_{X \prec \exp^* B(p, tr)} \langle d\psi, X\rangle\\
&= t^{1 - d} \sup_{Y \prec B(p, tr)} \int_V \mathcal L_{\exp^* Y}d\lambda
\end{align*}
where $d\lambda$ is the Lebesgue volume form on $T_pM$ and we set $Y = \exp_* X$.
As
$$\mathcal L_{\exp^* Y}d\lambda = \exp^* \mathcal L_Y(\exp_* d\lambda)$$
we conclude that, if $x_1, \dots, x_d$ are geodesic normal coordinates centered at $p$,
\begin{align*}
|d\psi_t|_{TV}(B'(0, r)) &= t^{1 - d} \sup_{Y \prec B(p, tr)} \int_U \mathcal L_Y(\exp_* d\lambda)\\
&= t^{1 - d} \sup_{Y \prec B(p, tr)} \int_U \Div Y ~dx_1 \wedge \cdots \wedge dx_d\\
&= t^{1 - d} \sup_{Y \prec B(p, tr)} \int_U |\det g|^{-1} ~\mathcal L_YdV.
\end{align*}
But $g$ is uniformly invertible on $B(p, r)$, so (\ref{linearized perimeter bound}) holds uniformly in $t$.
\end{proof}

\begin{lemma}\label{linearized perimeter bound 2}
Suppose that $U$ has least perimeter. Then for every $t \lesssim 1$,
$$|d1_U|_{TV}(B(p, tr)) \lesssim t^{d - 1}.$$
\end{lemma}
\begin{proof}
Let $A_t$ denote the surface area of $\partial B(p, tr)$ and let $\omega_t$ be the surface area of a $d-1$-sphere of radius $t$, so that $\omega_t \lesssim t^{d - 1}$.
Since $g$ is uniformly bounded on $B(p, tr)$, $A_t \lesssim \omega_t \lesssim t^{d - 1}$.

Let
$$u = 1_U + 1_{B(p, tr)} - 1_{B(p, tr) \cap U}.$$
Then $u - 1_U \in BV_c(M)$, so
$$|d1_U|_{TV}(B(p, tr)) \leq |du|_{TV}(B(p, tr)) \leq A_t \lesssim t^{d - 1}$$
as desired.
\end{proof}

\begin{proposition}
Suppose that $U$ has least perimeter.
Then a subsequence of the $\psi_t$ converge in $L^1_l$ and pointwise almost everywhere as $t \to 0$ to the indicator function of an open cone $V_0 \subseteq T_pM$ with a vertex at $0$.
Besides this, $V_0$ has least perimeter.
\end{proposition}
\begin{proof}
From Lemmata \ref{linearized perimeter bound 1} and \ref{linearized perimeter bound 2}, $||\psi_t||_{BV(B'(0, r))} \lesssim 1$ uniformly in $t \lesssim 1$.
Since the natural map $BV(B'(0, r)) \to L^1_l(B'(0, r))$ is compact, then, by passing to a subsequence we may assume that there exists $\psi_0 \in BV(B'(0, r))$ such that $\psi_t \to \psi_0$ in $L^1_l(B'(0, r))$.
Passing to a further subsequence, we may upgrade to pointwise convergence almost everywhere.
Therefore $\psi_0$ is an indicator function of a subset $V_0$ of $T_pM$.

We now claim that for every $\varepsilon > 0$,
\end{proof}





\section{Garbage}

Suppose that $u$ is a smooth function of least gradient on $U$ which is almost everywhere nonstationary in the sense that $|\nabla u|$ is nonzero almost everywhere.
Then, by definition, $u$ minimizes the action
$$I(u) = \int_U |\nabla u|$$
subject to a Dirichlet condition and so solves the Euler-Lagrange equation
\begin{equation}
\label{strong Euler-Lagrange}
\nabla \cdot \frac{\nabla u}{|\nabla u|} = 0.
\end{equation}
Conversely, $|x|$ is convex in $x$, so if $u$ is a smooth solution to (\ref{strong Euler-Lagrange}), then $u$ has least gradient.
We now show that the integral form of (\ref{strong Euler-Lagrange}) characterizes locally Sobolev minimizers of $I$ \cite[Lemma 1]{BOMBIERI1969}.

\begin{theorem}
Let $u \in W^{1,1}_l(U)$ be almost everywhere nonstationary, and let $N \subseteq U$ be a closed set such that $\mathcal H^{d-1}(N) = 0$.
If, for every $\varphi \in C^1_c(U \setminus N)$,
\begin{equation}
\label{weak Euler-Lagrange}
\int_{U \setminus N} |\nabla u(x)|^{-1} \langle \nabla u(x), \nabla \varphi(x)\rangle ~dx = 0,
\end{equation}
then $u$ has least gradient.
\end{theorem}
\begin{proof}
We will prove that for every $v \in BV_c(U)$ with support $K$,
\begin{equation}
\label{condition for least gradient}
\int_K |\nabla u| \leq \int_K |\nabla u + \nabla v|.
\end{equation}
First suppose that $N$ is empty and $v \in C_c$. Integrating (\ref{weak Euler-Lagrange}) by parts we arrive at (\ref{strong Euler-Lagrange}).
Therefore $u$ is a minimizer of $I$, so (\ref{condition for least gradient}) holds.

Now suppose that $v \in C^1_c(U)$; it follows from the hypothesis on $N$ that $\mathcal H^{n-1}(N \cap K) = 0$.
Since $N$ is closed, $N \cap K$ is compact, so for every $\varepsilon > 0$ there is an open cover $\mathcal B$ of $N \cap K$ by finitely many balls such that
$$\sum_{B \in \mathcal B} (\rad B)^{d - 1} < \varepsilon.$$
Given $B \in \mathcal B$, let $\alpha_B$ be the linear extrapolation of the function which is $1$ on $U \setminus 2B$ and $0$ on $B$.
Let $\alpha$ be a mollification of $\min_B \alpha_B$. Then $\alpha v = 0$ on a neighborhood of $N$ and $\alpha v \in C^1_c$. Thus, by the previous case,
$$\int_K |\nabla u| \leq \int_K |\nabla(u + \alpha v)| \leq \int_K |\nabla (u + v)| + (1-\alpha)|\nabla v| + |v||\nabla \alpha|.$$
We bound
$$\int_K (1-\alpha)|\nabla v| \leq ||v||_{C^1} \int_K (1 - \alpha) \leq ||v||_{C^1} \sum_{B \in \mathcal B} |2B|.$$
Since $|2B| \lesssim (\rad B)^d$ and $\sum_B (\rad B)^{d - 1} < \varepsilon$, this sum is $\ll \varepsilon$.
Similarly, we have
$$\int_K |v||\nabla \alpha| \leq ||v||_{C^1} \sum_{m=1}^M \int_{2B_m \setminus 2B_1 \cup \cdots \cup 2B_{m - 1}} |\nabla \alpha| \lesssim \sum_{B \in \mathcal B} (\rad B)^{d - 1} < \varepsilon.$$
Thus
$$\int_K |\nabla u| \leq \int_K \lim_{\varepsilon \to 0} |\nabla (u + \alpha v)| \leq \int_K |\nabla (u + v)|$$
which implies (\ref{condition for least gradient}).

In the general case, $v \in BV_l(U)$, so $\nabla v$ has a Radon-Nikod\'ym-Lebesgue decomposition
$$\nabla v = w_1 + w_2$$
where $w_1$ is absolutely continuous and $w_2$ is zero almost everywhere.
Since $u$ is locally Sobolev, we may assume that $\nabla u = 0$ on $\supp w_2$, thus
$$\int_K |\nabla(f + g)| = \int_K |\nabla u + w_1| + \int_K |w_2|.$$
Let $\psi_\varepsilon$ be a standard mollifier, $K_\varepsilon = \{x \in A: d(x, K) < \varepsilon\}$; then
$$\int_{K_\varepsilon} |\nabla f| \leq \int_{K_\varepsilon} |\nabla (f + g * \psi_\varepsilon)| \leq \int_{K_\varepsilon} |\nabla u + w_1 * \psi_\varepsilon| + \int_A |w_2 * \psi_\varepsilon|.$$
Taking $\varepsilon \to 0$, we get $w_1 * \psi_\varepsilon \to w_1$ and $w_2 * \psi_\varepsilon \to w_2$ in the weak topology of distributions, thus
$$\int_K |\nabla f| \leq \int_K |\nabla u + w_1| + \int_K |w_2|$$
which was to be shown.
\end{proof}
NOTE: I don't think we actually need this!


\subsection{The coarea formula}
We now prove a coarea formula due to Miranda \cite[Teorema 1.6]{Miranda66}.

The following lemma is also due to Miranda \cite[Proposizione 1.9]{Miranda64}.
\begin{lemma}\label{Miranda coarea baby}
Let $u \in BV_l(M)$ and $E = \{(x, y) \in M \times \RR: y \leq u(x)\}$.
Then for every $U \Subset M$,
$$|\nabla u|_{TV}(U) = |\nabla_x 1_E|_{TV}(U \times \RR).$$
\end{lemma}
\begin{proof}
The triangle inequality applied to Corollary implies
$$|\nabla u|_{TV}(U) \leq \int_{-\infty}^\infty |\nabla 1_{E_y}|_{TV}(U).$$
\end{proof}

We will also need a variant of Fubini's theorem due to Miranda \cite[Teorema 3.3]{Miranda64}.
The point is that $\nabla_x 1_E$ is just a Radon measure, rather than an honest function, so we cannot just directly apply Fubini's theorem.
\begin{lemma}[Fubini-Miranda theorem]\label{Fubini Miranda}
Let $f \in BV_l(U)$ and $E = \{(x, \lambda) \in U \times \RR: \lambda \leq f(x)\}$.
Let $E_\lambda = \{x \in U: (x, \lambda) \in E\}$.
Then for every $V \Subset U$,
$$|\nabla_x 1_E|_{TV}(V \times \RR) = \int_{-\infty}^\infty |\nabla 1_{E_\lambda}|_{TV}(V) ~d\lambda.$$
\end{lemma}
TODO Get the statement of this right (probably the integral in $d\lambda$ is an honest integral). Proof. Obviously this works without using Lebesgue measure.

\begin{theorem}[Miranda's coarea formula]\label{Miranda coarea}
Suppose that $f \in BV_l(U)$, let $E_\lambda = \{f \geq \lambda\}$ denote the superlevel sets of $f$, and let $\varphi_\lambda$ be the indicator function of $E_\lambda$.
Then for almost every $\lambda$, $\varphi_\lambda \in BV_l(U)$, and for every $V \Subset U$,
$$|\nabla f|_{TV}(V) = \int_{-\infty}^\infty |\nabla \varphi_\lambda|_{TV}(V) ~d\lambda.$$
\end{theorem}
\begin{proof}
This is just an immediate concatenation of Lemmata \ref{Miranda coarea baby} and \ref{Fubini Miranda}.
\end{proof}


A useful corollary of Theorem \ref{Miranda IBP} is a relationship between the total variation and the $L^1$ norm:
\begin{corollary}\label{IBP TV}
Let $U \Subset M$, $u \in BV_l(M)$, and suppose that $\partial U$ is a Lipschitz hypersurface such that $|\nabla u|_{TV}(\partial U) = 0$.
Let $\tilde u = (1 - 1_U)u$. Then
$$|\nabla \tilde u|_{TV}(U) = ||u||_{L^1(\partial U)}.$$
\end{corollary}
\begin{proof}
Let $U_\varepsilon = \{x \in U: d(x, U) < \varepsilon\}$.
By Theorem \ref{Miranda IBP},
\begin{equation}
\label{Miranda IBP cor app}
\langle \nabla \tilde u, X\rangle_{U_\varepsilon} + \int_{U_\varepsilon} \tilde u~\mathcal L_XdV = \int_{\partial U_\varepsilon} g(\tilde u X, \nu) ~dA.
\end{equation}
We treat each of the terms in (\ref{Miranda IBP cor app}) in turn.

Letting $X$ range over test fields on $M$,
$$\sup_{||X||_{L^\infty} \leq 1} \langle \nabla \tilde u, X\rangle_{U_\varepsilon} = |\nabla \tilde u|_{TV}(U_\varepsilon).$$
However, $\bigcap_{\varepsilon > 0} U_\varepsilon = \overline U$, so
$$\lim_{\varepsilon \to 0}\sup_{||X||_{L^\infty} \leq 1} \langle \nabla \tilde u, X\rangle_{U_\varepsilon} = |\nabla \tilde u|_{TV}(\overline U).$$

For every test field $X$ on $M$,
$$\left|\int_{U_\varepsilon} \tilde u \mathcal L_XdV\right| \leq ||\nabla X||_{L^\infty} \int_{U_\varepsilon \setminus U} |u| ~dV$$
which vanishes as $\varepsilon \to 0$, since $\bigcap_\varepsilon U_\varepsilon \setminus U = \partial U$, which is null.

Letting $X$ range over test fields on $M$ and $\sigma$ range over measurable signs
$$\sigma: \partial U_\varepsilon \to \{-1, 1\},$$
we have
\begin{align*}
\int_{\partial U_\varepsilon} |u| ~dA &= \sup_\sigma \int_{\partial U_\varepsilon} \sigma u |\nu|_g^2 ~dA = \sup_{||X||_{L^\infty} \leq 1} \int_{\partial U_\varepsilon} g(uX, \nu) ~dA\\
&= \sup_{||X||_{L^\infty} \leq 1} \int_{\partial U_\varepsilon} g(uX, \nu) ~dA = \sup_{||X||_{L^\infty} \leq 1} \int_{\partial U_\varepsilon} g(\tilde uX, \nu) ~dA.
\end{align*}
To see this, we choose $\sigma = \sgn u$ to maximize in $\sigma$, choose a sequence of $X_n$ which converge to $\sigma$ pointwise, and apply the dominated convergence theorem.
We then apply the fact that $u = \tilde u$ on $\partial U_\varepsilon$, a set which does not meet $U$.

Plugging the previous three paragraphs into (\ref{Miranda IBP cor app}) and taking $\varepsilon \to 0$, we conclude the claim.
\end{proof}

\subsection{Geodesics and exorcism}
In this section we let $M$ be a surface and $U$ an open Caccioppoli set of reduced boundary $\Gamma$.

Since $\Gamma$ is a rectifiable curve, there exists a sequence of $C^1$ curves $\gamma_n: [0, 1] \to \Gamma$ and a $1$-null exceptional set $Z \subset \Gamma$ such that $\Gamma$ is the union of $Z$ with the images of the $\gamma_n$.

One really bad thing that could happen is that $Z$ could have positive Hausdorff dimension.
For example, $\Gamma$ could be a Devil's staircase.
Since we want to rule out Devil's staircases, we introduce the following definition:

\begin{definition}
An \dfn{exorcised curve} is a rectifiable curve whose exceptional set is discrete.
\end{definition}

\begin{proposition}[exorcism]
If $U$ has least perimeter then $\Gamma$ is an exorcised curve.
\end{proposition}
\begin{proof}
Let $\varepsilon > 0$.
Then there exist disjoint open intervals $(p_m, q_m) = I_m \subseteq \Gamma$ such that $|I_m| < \varepsilon/2^m$ and $Z \subseteq \bigcup_m I_m$.
If $\varepsilon$ is chosen small enough, then since $M$ is locally contractible, there exists a geodesic $J_m$ from $p_m$ to $q_m$ which is homotopic to $I_m$ by a homotopy $H_m: [0, 1]^2 \to M$.
Let $V_m = H_m((0, 1)^2)$.

Let $\Gamma_0 = \Gamma$, $U_0 = U$, and for each $m$, let $\Gamma_m$ consist of $\Gamma_{m - 1}$, but with $I_m$ removed and replaced by the geodesic $J_m$.
This transforms $U_{m - 1}$ to $U_m$ by removing $U_{m - 1} \cap V_m$ from $U_{m - 1}$ and adjoining $V_m \setminus \overline U_{m - 1}$ to $U_{m - 1}$.

If $\Gamma$ is nonexorcised then there exists $m$ such that $U_{m - 1} \neq U_m$.
Indeed, in a neighborhood of $J_m \subseteq U_m$ the only exceptional points are the two points in $\partial J_m$, but if $\Gamma$ is nonexorcised then there are infinitely many exceptional points in one of the $I_m$.

So if $U_{m - 1} \neq U_m$ then by definition of a geodesic, $\Gamma_m$ is locally shorter than $\Gamma$, and the modification by $H_m$ was a compact perturbation, so $U$ does not have least perimeter.
\end{proof}

\begin{theorem}
If $U$ has least perimeter, then each connected component of $\Gamma$ is a geodesic.
\end{theorem}
\begin{proof}
Since $\Gamma$ is the reduced boundary, every component of $\Gamma$ has positive length, and is closed, therefore is a rectifiable curve.
Consider a connected component $\Gamma_0$, and let $Z_0$ be the exceptional set for $\Gamma_0$.
Since $Z_0$ is discrete, $\nu$ is $C^1$ away from a discrete set. Therefore we have a Lebesgue decomposition
$$\Div \nu = f + \sum_{z \in Z_0} c_z \delta_z$$
where $c_z \in \RR$.
Now we have to check the Euler-Lagrange equations in this case, TODO.
Therefore $\Div \nu = 0$ so $\Gamma_0$ is a solution of the geodesic equation.
\end{proof}



\printbibliography


\end{document}
