\documentclass[reqno,12pt,letterpaper]{amsart}
\RequirePackage{amsmath,amssymb,amsthm,graphicx,mathrsfs,url}
\RequirePackage[usenames,dvipsnames]{color}
\RequirePackage[colorlinks=true,linkcolor=Red,citecolor=Green]{hyperref}
\RequirePackage{amsxtra}
\usepackage{cancel}
\usepackage{tikz-cd}

\setlength{\textheight}{9in} \setlength{\oddsidemargin}{-0.25in}
\setlength{\evensidemargin}{-0.25in} \setlength{\textwidth}{7in}
\setlength{\topmargin}{-0.25in} \setlength{\headheight}{0.18in}
\setlength{\marginparwidth}{1.0in}
\setlength{\abovedisplayskip}{0.2in}
\setlength{\belowdisplayskip}{0.2in}
\setlength{\parskip}{0.05in}
\renewcommand{\baselinestretch}{1.05}

\title[Geodesic laminations by minimal currents]{Geodesic laminations by minimal currents}
\author{Aidan Backus}
\date{July 2021}

\newcommand{\NN}{\mathbf{N}}
\newcommand{\ZZ}{\mathbf{Z}}
\newcommand{\QQ}{\mathbf{Q}}
\newcommand{\RR}{\mathbf{R}}
\newcommand{\CC}{\mathbf{C}}
\newcommand{\DD}{\mathbf{D}}
\newcommand{\PP}{\mathbf P}
\newcommand{\MM}{\mathbf M}
\newcommand{\II}{\mathbf I}
\newcommand{\Hyp}{\mathbf H}
\newcommand{\SL}{\mathrm{SL}}
\newcommand{\evect}{\mathbf e}
\newcommand{\Lagrange}{\mathscr L}

\DeclareMathOperator{\card}{card}
\DeclareMathOperator{\cent}{center}
\DeclareMathOperator{\ch}{ch}
\DeclareMathOperator{\codim}{codim}
\DeclareMathOperator{\diag}{diag}
\DeclareMathOperator{\diam}{diam}
\DeclareMathOperator{\dom}{dom}
\DeclareMathOperator{\Gal}{Gal}
\DeclareMathOperator{\Hom}{Hom}
\DeclareMathOperator{\Jac}{Jac}
\DeclareMathOperator{\Lip}{Lip}
\DeclareMathOperator{\Met}{Met}
\DeclareMathOperator{\id}{id}
\DeclareMathOperator{\rad}{rad}
\DeclareMathOperator{\rank}{rank}
\DeclareMathOperator{\Radon}{Radon}
\DeclareMathOperator*{\Res}{Res}
\DeclareMathOperator{\sgn}{sgn}
\DeclareMathOperator{\singsupp}{sing~supp}
\DeclareMathOperator{\Spec}{Spec}
\DeclareMathOperator{\supp}{supp}
\DeclareMathOperator{\Tan}{Tan}
\newcommand{\tr}{\operatorname{tr}}

\newcommand{\Ric}{\mathrm{Ric}}
\newcommand{\Riem}{\mathrm{Riem}}

\newcommand{\dbar}{\overline \partial}

\DeclareMathOperator{\atanh}{atanh}
\DeclareMathOperator{\arcosh}{arcosh}
\DeclareMathOperator{\csch}{csch}
\DeclareMathOperator{\sech}{sech}

\DeclareMathOperator{\Div}{div}
\DeclareMathOperator{\grad}{grad}
\DeclareMathOperator{\Ell}{Ell}
\DeclareMathOperator{\WF}{WF}

\newcommand{\Hilb}{\mathcal H}
\newcommand{\Homology}{\mathrm H_{\mathrm{dR}}}
\newcommand{\normal}{\mathbf n}
\newcommand{\vol}{\mathrm{vol}}

\newcommand{\Japan}[2]{\left\langle #1 \right\rangle}

\newcommand{\pic}{\vspace{30mm}}
\newcommand{\dfn}[1]{\emph{#1}\index{#1}}

\renewcommand{\Re}{\operatorname{Re}}
\renewcommand{\Im}{\operatorname{Im}}


\newtheorem{theorem}{Theorem}[section]
\newtheorem{badtheorem}[theorem]{``Theorem"}
\newtheorem{prop}[theorem]{Proposition}
\newtheorem{lemma}[theorem]{Lemma}
\newtheorem{claim}[theorem]{Claim}
\newtheorem{proposition}[theorem]{Proposition}
\newtheorem{corollary}[theorem]{Corollary}
\newtheorem{conjecture}[theorem]{Conjecture}
\newtheorem{axiom}[theorem]{Axiom}
\newtheorem{assumption}[theorem]{Assumption}

\theoremstyle{definition}
\newtheorem{definition}[theorem]{Definition}
\newtheorem{remark}[theorem]{Remark}
\newtheorem{example}[theorem]{Example}
\newtheorem{notation}[theorem]{Notation}

\newtheorem{exercise}[theorem]{Discussion topic}
\newtheorem{homework}[theorem]{Homework}
\newtheorem{problem}[theorem]{Problem}

\newtheorem{ack}{Acknowledgements}

\numberwithin{equation}{section}


% Mean
\def\Xint#1{\mathchoice
{\XXint\displaystyle\textstyle{#1}}%
{\XXint\textstyle\scriptstyle{#1}}%
{\XXint\scriptstyle\scriptscriptstyle{#1}}%
{\XXint\scriptscriptstyle\scriptscriptstyle{#1}}%
\!\int}
\def\XXint#1#2#3{{\setbox0=\hbox{$#1{#2#3}{\int}$ }
\vcenter{\hbox{$#2#3$ }}\kern-.6\wd0}}
\def\ddashint{\Xint=}
\def\dashint{\Xint-}

%\usepackage{color}
%\hypersetup{%
%    colorlinks=true, % make the links colored%
%    linkcolor=blue, % color TOC links in blue
%    urlcolor=red, % color URLs in red
%    linktoc=all % 'all' will create links for everything in the TOC
%Ning added hyperlinks to the table of contents 6/17/19
%}

% style=alphabetic
\usepackage[backend=bibtex,maxcitenames=50,maxnames=50]{biblatex}
\addbibresource{topics.bib}
\renewbibmacro{in:}{}
\DeclareFieldFormat{pages}{#1}

\begin{document}
%%%%%%%%%%%%%%%%%%%%%%%%%%%%%%%%%%%%%%%%%%%%%%%%%%%%%%%

% \tableofcontents

\section{The Riemannian minimal surface Lagrangian}
Let $g$ be a metric written in the form
$$g = \begin{bmatrix}g_{11} & \\ & g_{22}\end{bmatrix}$$
where $g_{11}$ is a scalar field and $g_{22}$ is a metric with one dimension down.
That is, there are no cross-terms $g_{12}$.
I think that the Gauss lemma implies that we can always write a metric locally in this form.

Let
$$\mathscr L(x, \omega, p) = \sqrt{g_{11}(x, \omega) + g_{22}(x, \omega)(p, p)}.$$
Then $\mathscr L(x, \omega(x), d\omega(x))$ is the area of the graph of $\omega: \RR^{d - 1} \to \RR$ where $\RR \times \RR^{d - 1}$ is equipped with the metric $g$.
We henceforth suppress the dependence of $g_i = g_{ii}$ on $x, \omega$.

We can rewrite the Lagrangian as
$$\mathscr L(\omega, p) = \sqrt{g_1} \cdot \sqrt{1 + \lambda p^2}$$
where $\lambda p^2$ denotes $g_2(p, p)/g_1$. Since $g_2,g_1$ are positive definite, $p \mapsto \lambda p^2$ is a positive-definite quadratic form.
Note that in isothermal coordinates, $\lambda = 1$.

\begin{example}
With $g$ the hyperbolic metric we get $g_1 = g_2 = y^{-2}$.
In particular $\lambda = 1$ and we get
$$\Lagrange(\omega, p) = \frac{\sqrt{1 + p^2}}{\omega}.$$
\end{example}

We now Taylor expand a difference of Lagrangians.
For every $x, y > -1$ there exists $\kappa \geq 0$ such that
$$\sqrt{1 + x} = \sqrt{1 + y} + \frac{x - y}{2\sqrt{1 + y}} - \kappa(x - y)^2,$$
and $\kappa$ depends continuously on $x,y$.
Setting $x = \lambda p^2$, $y = \lambda q^2$, there exists $\kappa \geq 0$ such that
$$\Lagrange(\omega, p) - \Lagrange(\omega, q) = \lambda \frac{p^2 - q^2}{2\sqrt{1 + \lambda q^2}} \sqrt{g_1} - \kappa \lambda^2 (p^2 - q^2)^2 \sqrt{g_1}.$$
Positivity of $\kappa,\lambda$ implies that
$$\Lagrange(\omega, p) - \Lagrange(\omega, q) \leq \frac{\lambda}{2}(p^2 - q^2) \sqrt{g_1}.$$


\section{De Giorgi lemma in two dimensions}
In this section we let $Af(\rho) = (2\rho)^{-1} \int_{-\rho}^\rho f$ be the mean of $f$ over $(-\rho, \rho)$.

\begin{assumption}
We assume that:
\begin{enumerate}
\item $g = \begin{bmatrix}g_{11} \\ & g_{22}\end{bmatrix}$, as in the previous section.
\item $g(x, y)$ does not depend on $x$.
\end{enumerate}
\end{assumption}

So $g(x, \omega(x))$ is determined by $\omega(x)$ but not $x$ itself.
I don't think this second assumption actually meaningfully changes the below proof.
That said, it does make the below proof cleaner.
Note that this happens for the hyperbolic metric.

\begin{lemma}
Suppose that $\omega_j' \to \Omega$, where $\Omega$ is a constant, in $C^1$.
Let $u_j$ be the function defined by
\begin{align*}
u_j'' &= 0 \\
u_j(\pm \rho_j) &= \omega_j(\pm \rho_j).
\end{align*}
Furthermore, suppose that
$$0 < \rho_j^2 \ll \beta_j \ll 1.$$
If
$$\limsup_{j \to \infty} \frac{1}{\beta_j} \int_{-\rho_j}^{\rho_j} \mathscr L(\omega_j, \omega_j') - \Lagrange(u_j, u_j') \leq 0$$
and
$$\int_{-\rho_j}^{\rho_j} \Lagrange(\omega_j, \omega_j') - \Lagrange(\omega_j, A \omega_j'(\rho)) \leq \beta_j,$$
then
$$\limsup_{j \to \infty} \frac{1}{\beta_j} \int_{-\rho_j/2}^{\rho_j/2} \Lagrange(\omega_j, \omega_j') - \Lagrange(\omega_j, A \omega_j'(\rho)) \leq \frac{1}{8}.$$
\end{lemma}
\begin{proof}
Fix $\varepsilon > 0$.
From properties of the Lagrangian,
$$\int_{-\rho_j/2}^{\rho_j/2} \Lagrange(\omega_j, \omega_j') - \Lagrange(\omega_j, A \omega_j'(\rho_j/2)) \leq \int_{-\rho_j/2}^{\rho_j/2} \frac{\lambda(\omega_j) \sqrt{g_1(\omega_j)}}{2} ((\omega_j')^2 - A \omega_j'(\rho_j/2)^2).$$
We observe that since $g_1$ does not depend on $x$, the convergence of $\omega_j$ to $\Omega$ implies that there is a constant $\Lambda$ such that
$$\lim_{j \to \infty} \lambda(\omega_j) \sqrt{g_1(\omega_j)} = \Lambda$$
uniformly.
In particular, if we set
$$2I = \int_{-\rho_j/2}^{\rho_j/2} ((\omega_j')^2 - A \omega_j'(\rho_j/2)^2)$$
then for every $j$ large enough,
$$\int_{-\rho_j/2}^{\rho_j/2} \Lagrange(\omega_j, \omega_j') - \Lagrange(\omega_j, A \omega_j'(\rho_j/2)) \leq (1 + \varepsilon) \Lambda I.$$

Since $A \omega_j'(\rho_j/2)$ is the mean of $\omega_j$ over the domain of integration
$$\int_{-\rho_j/2}^{\rho_j/2} \omega_j' A\omega_j'(\rho_j/2) = \int_{-\rho_j/2}^{\rho_j/2} A \omega_j'(\rho_j/2)^2$$
which readily implies
$$2I = \int_{-\rho_j/2}^{\rho_j/2} (\omega_j' -  A \omega_j'(\rho_j/2))^2.$$
For any function $f$, $||f - y||_{L^2}$ is minimized exactly when $y$ is the mean of $f$ (just differentiate the square of that norm in $y$ and look for critical points) so
$$2I \leq \int_{-\rho_j/2}^{\rho_j/2} (\omega_j' -  A_j \omega_j'(\rho_j))^2.$$

By Cauchy's inequality
$$|\alpha - \gamma|^2 \leq (1 + \varepsilon) |\alpha - \beta|^2 + (1 + 1/\varepsilon) |\beta - \gamma|^2$$
we have
$$2I \leq (1 + 1/\varepsilon) \int_{-\rho_j/2}^{\rho_j/2} |\omega_j' - u_j'|^2 + (1 + \varepsilon) \int_{-\rho_j/2}^{\rho_j/2} |u_j' - A\omega_j'(\rho_j)|^2.$$
Our boundary condition on $u_j$, and the fundamental theorem of calculus, imply that
$$A u_j'(\rho_j) = A\omega_j'(\rho_j).$$
So by the mean-value property and Cauchy's inequality,
\begin{align*}
\int_{-\rho_j/2}^{\rho_j/2} |u_j' -  A \omega_j'(\rho_j)|^2 &\leq \frac{1}{8} \int_{-\rho}^{\rho} |u_j' -  A \omega_j'(\rho_j)|^2\\
&\leq \frac{1 + \varepsilon}{8} \int_{-\rho_j}^{\rho_j} |\omega_j' -  A \omega_j'(\rho_j)|^2 + \frac{1 + 1/\varepsilon}{8} \int_{-\rho_j}^{\rho_j} |\omega_j' - u_j'|^2
\end{align*}
Thus there exists $C_\varepsilon > 0$ (which does not depend on $j$) such that
\begin{align*}
I &\leq \frac{(1 + \varepsilon)^2}{16} \int_{-\rho_j}^{\rho_j} |\omega_j'(X) -  A \omega_j'(\rho_j)|^2 ~dX + O(1) \int_{-\rho_j}^{\rho_j} |\omega_j' - u_j'|^2\\
&= \frac{(1 + \varepsilon)^2}{16} J + C_\varepsilon K.
\end{align*}

\begin{claim}
If $j$ is large enough then
$$J \leq \frac{2 + \varepsilon}{\Lambda}\beta_j.$$
\end{claim}
\begin{proof}[Proof of claim]
One has
$$J = \int_{-\rho_j}^{\rho_j} (\omega_j')^2 - A \omega_j'(\rho_j)^2.$$
Using the Taylor expansion of the Lagrangian, this implies
\begin{align*}
J &= 2 \int_{-\rho_j}^{\rho_j} \frac{\sqrt{1 + \lambda(\omega_j) A \omega_j'(\rho_j)^2}}{\lambda(\omega_j) \sqrt{g_1(\omega_j)}}\mathscr L(\omega_j, \omega_j') - \mathscr L(\omega_j, A \omega_j'(\rho))\\
&\qquad + 2 \int_{-\rho_j}^{\rho_j} \sqrt{1 + \lambda(\omega_j) A \omega_j'(\rho_j)^2}\kappa(\omega_j', A\omega_j') \lambda(\omega_j) ((\omega_j')^2 - A\omega_j'(\rho_j)^2)^2.
\end{align*}
The first term is, if $j$ is chosen large enough,
$$\leq \frac{2 + \varepsilon/2}{\Lambda} \int_{-\rho_j}^{\rho_j} \Lagrange(\omega_j, \omega_j') - \Lagrange(\omega_j, A \omega_j'(\rho)) \leq \frac{2 + \varepsilon/2}{\Lambda} \beta_j.$$
The convergence of the $\omega_j$ in $C^1$ shows that the second term is bounded for $j$ large by
$$\lesssim \int_{-\rho_j}^{\rho_j} ((\omega_j')^2 - A\omega_j'(\rho_j)^2)^2.$$
We can estimate
$$((\omega_j')^2 - A\omega_j'(\rho_j)^2)^2 \leq ||\omega_j'||_{L^\infty} ((\omega_j')^2 - A \omega_j'(\rho_j)^2)$$
since $|A \omega_j'(\rho_j)| \leq ||\omega_j'||_{L^\infty}$.
But $\omega_j' \to 0$ uniformly, so
$$J \leq \frac{2 + \varepsilon/2}{\Lambda} \beta_j + o(J).$$
Selecting $j$ large enough now completes the claim.
\end{proof}

\begin{claim}
$K \ll \beta_j$.
\end{claim}
\begin{proof}[Proof of claim]
Since $u_j'' = 0$,
$$K = \int_{-\rho_j}^{\rho_j} (\omega_j')^2 - (u_j')^2.$$
Taylor expanding the Lagrangian,
$$K \lesssim \int_{-\rho_j}^{\rho_j} \Lagrange(\omega_j, \omega_j') - \Lagrange(\omega_j, u_j') + \int_{-\rho_j}^{\rho_j} (\Lagrange(\omega_j, \omega_j')^2 - \Lagrange(\omega_j, u_j')^2)^2.$$
Moreover,
\begin{align*}
(\Lagrange(\omega_j, \omega_j')^2 - \Lagrange(\omega_j, u_j')^2)^2 &\lesssim (||\omega_j'||_{L^\infty} + ||u_j'||_{L^\infty}) (\Lagrange(\omega_j, \omega_j')^2 - \Lagrange(\omega_j, u_j')^2)\\
& \lesssim (||\omega_j'||_{L^\infty} + ||u_j'||_{L^\infty}) ((\omega_j')^2 - (u_j')^2).
\end{align*}
By the maximum principle,
$$||\omega_j'||_{L^\infty} + ||u_j'||_{L^\infty} \leq 2||\omega_j'||_{L^\infty} \to 0$$
so
$$\int_{-\rho_j}^{\rho_j} (\Lagrange(\omega_j, \omega_j')^2 - \Lagrange(\omega_j, u_j')^2)^2 \ll K$$
and hence
\begin{align*}
K &\lesssim \int_{-\rho_j}^{\rho_j} \Lagrange(\omega_j, \omega_j') - \Lagrange(\omega_j, u_j')\\
&= \int_{-\rho_j}^{\rho_j} \mathscr L(\omega_j, \omega_j') - \mathscr L(u_j, u_j') + \int_{-\rho_j}^{\rho_j }\mathscr L(u_j, u_j') - \mathscr L(\omega_j, u_j')\\
&\lesssim o(\beta_j) + ||\omega_j - u_j||_{L^1}.
\end{align*}
The second term is the only interesting new term created by the presence of geometry.
To bound it, we observe that $\omega_j - u_j$ is trace-free, so by the Poincar\'e, H\"older, and Cauchy inequalities,
$$||\omega_j - u_j||_{L^1} \lesssim \rho_j ||\omega_j' - u_j'||_{L^1} \leq \rho^{3/2}_j ||\omega_j' - u_j'||_{L^2} \leq \rho^{5/2}_j + \rho^{1/2}_j ||\omega_j' - u_j'||_{L^2}^2 = \rho^{5/2}_j + \rho^{1/2}_j K.$$
Thus
$$K \leq o(\beta_j) + \rho^{5/2}_j + \rho^{1/2}_j K$$
but we assumed that $\rho^2_j \ll \beta_j$ so this is good.
\end{proof}

We now have a bound
$$I \leq \frac{(1 + \varepsilon)^2}{16} \frac{2 + \varepsilon}{\Lambda} \beta_j + o(\beta_j) \leq \frac{1 + O(\varepsilon)}{8\Lambda} \beta_j$$
which gives
$$\int_{-\rho_j/2}^{\rho_j/2} \Lagrange(\omega_j, \omega_j') - \Lagrange(\omega_j, A \omega_j'(\rho_j/2)) \leq \frac{1 + O(\varepsilon)}{8} \beta_j.$$
We conclude
$$\lim_{j \to \infty} \frac{1}{\beta_j} \int_{-\rho_j/2}^{\rho_j/2} \Lagrange(\omega_j, \omega_j') - \Lagrange(\omega_j, A \omega_j'(\rho_j/2)) \leq \frac{1 + O(\varepsilon)}{8}.$$
The left-hand side does not depend on $\varepsilon$, so this is good.
\end{proof}

In what follows we let $\evect_1, \evect_2 \in S^1$ be the standard basis of $\RR^2$, and let $B_r$ be the hyperbolic ball centered on $\evect_2$ of radius $r > 0$, where we view $\Hyp^2$ as the upper half-plane in $\RR^2$.
Since we are favoring a particular system of coordinates we don't need to worry about cleverly defining the excess.

\begin{lemma}
Let $(U_j)$ be a sequence of open sets in $\Hyp^2$ with $C^1$ boundary.
Let $u_j = 1_{U_j}$ and let $\normal_j: \partial U_j \to S^1$ be the Gauss map.
Fix $\delta > 0$ and $\rho_j, \beta_j$ such that0
If
$$\int_{B_{\rho_j}} |du_j| ~\vol - \left|\int_{B_{\rho_j}} du_j ~\vol\right| \leq \beta_j,$$
$$\lim_{j \to \infty} \frac{|\partial U_j \cap B_{\rho_j}| - \eta(U_j, B_{\rho_j})}{\beta_j} = 0,$$
and $\normal_j|\partial B_{\rho_j} \to \evect_2$ uniformly,
then
$$\limsup_{j \to \infty} \frac{1}{\beta_j} \int_{B_{\rho_j/2}} |du_j| ~\vol - \frac{1}{\beta_j}\left|\int_{B_{\rho_j/2}} du_j ~\vol\right| \leq \frac{1}{8}.$$
\end{lemma}
\begin{proof}
Suppose that the lemma is false, thus:

\begin{claim}
There exists a sequence $(U_j)$ meeting the hypotheses of the lemma such that
$$\lim_{j \to \infty} \frac{1}{\beta_j} \int_{B_{\rho_j/2}} |du_j| ~\vol - \frac{1}{\beta_j}\left|\int_{B_{\rho_j/2}} du_j ~\vol\right| > \frac{1}{8}.$$
Moreover, there exist nonempty open sets $I_j \subseteq \RR$ and $C^1$ functions $\omega_j: I_j \to (1/2, 3/2)$ such that the graph of $\omega_j$ is $\partial U_j$ and $\omega_j \to 1$ in $C^1$.
\end{claim}
\begin{proof}[Proof of claim]
Let $(U_j)$ be a sequence which contradicts the lemma, and let $\varepsilon > 0$.
By taking subsequences and using the uniform convergence of $\normal_j$ we may assume that
$$\normal_j|B_{\rho_j} \cdot \evect_2 > 1 - \varepsilon.$$
(Here we use the euclidean dot product.)
If $\varepsilon$ is chosen small enough, there exist open sets $I_j \subseteq \RR$ and $C^1$ functions $\omega_j: I_j \to (0, \infty)$ such that the graph of $\omega_j$ is $\partial U_j$.

If $I_j$ is empty, then $u_j$ is constant and so $du_j = 0$ identically.
Therefore $u_j$ does not contribute to the claimed limit being $> 1/8$ and so by taking subsequences we may discard all empty $I_j$.

Since $\normal_j \to 0$ uniformly, the same holds for $\omega_j'$.
Therefore, by taking subsequences, we may assume that $\omega_j$ tends to a constant function.
Since $I_j$ is nonempty, there always exists $x_j \in I_j$ such that $\omega_j(x_j) \in (1 - O(\rho_j), 1 + O(\rho_j))$ and yet $\rho_j \to 0$.
This gives the desired convergence in $C^1$.
\end{proof}

If $|I| = r$, then
\begin{align*}
\int_I \mathscr L(\omega_j, \omega_j') &= \int_{I \times \RR_+} |du_j| ~\vol,
\end{align*}
since both sides are the hyperbolic length of $\partial U_j$ in $I \times \RR_+$.
Similarly, if $I$ is the $g_j$-ball of radius $r < \rho_j$ then
$$\mathscr L(\omega_j, A_j \omega_j'(r)) = \frac{1}{2r} \left|\int_{(I \times \RR_+) \cap B_{\rho_j}} du_j ~\vol\right|$$
so
$$\int_I \mathscr L(\omega_j, \omega_j') - \mathscr L(\omega_j, A_j \omega_j'(r)) \leq \int_{B_{\rho_j}} |du_j| ~\vol - \left|\int_{B_{\rho_j}} du_j ~\vol\right| \leq \beta_j.$$
Since $I \subset B_{1, j}$ we can take the limit as $I$ grows to $B_{1, j}$, and
$$\int_{B_{1, j}} \mathscr L(\omega_j, \omega_j') - \mathscr L(\omega_j, A_j \omega_j'(\rho_j)) \leq \beta_j.$$

Let $u_j$ be the harmonic function on $B_{1, j}$ such that $u_j - \omega_j$ is trace-free. Then
$$\eta(\partial U_j, B_{\rho_j}) \leq \int_{B_{1, j}} \mathscr L(u_j, u_j')$$
since the right-hand side is the arc length of the graph of $u_j$, which is necessarily longer than the length of the geodesic with the same endpoints.
Therefore
$$\lim_{j \to \infty} \frac{1}{\beta_j} \int_{B_{1, j}} \mathscr L(\omega_j, \omega_j') - \mathscr L(u_j, u_j') \leq \lim_{j \to \infty} \frac{|\partial U_j \cap B_{\rho_j}| - \eta(U_j, B_{\rho_j})}{\beta_j} = 0$$
so by the previous lemma,
$$\limsup_{j \to \infty} \frac{1}{\beta_j} \int_{B_{1/2,j}} \mathscr L(\omega_j, \omega_j') - \mathscr L(\omega_j, A_j \omega_j'(\rho/2)) \leq \frac{1}{8}.$$
This contradicts the claim when we take $I = B_{1/2,j}$.
\end{proof}

\section{Mollification in hyperbolic plane}
Let's assume that the mollification works for now.
I think it's pretty obvious that it does, because I've worked through the mollification argument in very similar circumstances, but there are a lot of annoying details to check (mostly Sobolev embedding stuff).
I haven't had time to write them all down in this particular case because all my time goes to grading midterms :-(

If the mollification argument works, then the above lemma gives:
\begin{lemma}
Let $(U_j)$ be a sequence of sets of hyperbolic least perimeter in $B_j = B(\evect_2, \rho_j)$, $u_j = 1_{U_j}$, where
$$\int_{B_j} |du_j| ~\vol - \left|\int_{B_j} du_j ~\vol\right| \leq \gamma_j$$
and
$$0 < \rho_j^3 \ll \gamma_j \ll 1.$$
Then
$$\limsup_{j \to \infty} \gamma_j^{-1}\left[\int_{B_j/2} |du_j| ~\vol - \left|\int_{B_j/2} du_j ~\vol\right|\right] \leq \frac{1}{8}.$$
\end{lemma}

\section{Continuity of the conormal}
Fix $P \in \Hyp^2$ and write
$$\Lambda(U, \rho) = \frac{1}{\rho}\left[\int_{B(P, \rho)} |du| ~\vol - \left|\int_{B(P, \rho)} du ~\vol\right|\right]$$
whenever $u = 1_U$.
We write
$$\normal_s(P) = \frac{\int_{B(P, s)} du ~\vol}{\int_{B(P, s)} |du| ~\vol}.$$
This $1$-form is clearly continuous.

\begin{lemma}
For every set of hyperbolic least perimeter $U \subseteq \Hyp^2$ such that $P \in \partial U$ and for every sufficiently small dyadic $\rho > 0$, $\Lambda(U, \rho) \ll \rho$, $\normal_s(P) \to \normal(P)$ for dyadic $s$.
\end{lemma}
\begin{proof}
Using the blowup theorem, if the sequence $(\normal_s(P))$, $s$ a subsequence of the dyadics, is Cauchy, then it converges to the desired covector.
Pass to a subsequence of the dyadics; we claim there is a further subsequence $(k_j)$ along which $(\normal_{t_j}(P))$ is Cauchy, so the main subsequence converges to what it should.
Using Giusti--Miranda and the no-cusp lemma, we have
$$|\normal_s(P) - \normal_t(P)|^2 \lesssim 2^k \Lambda(U, 2^{-k})$$
so it suffices to show that there exists a further subsequence $(k_j)$ such that
$$\Lambda(U, 2^{-k_j}) \lesssim 4^{-{k_j}}.$$
If this is not true, then for every $k$,
$$4^k \Lambda(U, 2^k) \geq 1.$$

Either $8^{-k} \ll \gamma_k$, where $\gamma_k = 2^k\Lambda(U, 2^{-k})$ or there are infinitely many $k$ such that
$$\Lambda(U, 2^{-k}) \lesssim 4^{-k}.$$
They then define a Cauchy subsequence. So suppose that $8^{-k} \ll \gamma_k$.
Using the action of $\SL(2, \RR)$ on $\Hyp^2$ we can assume that $P = \evect_2$ and select $U_k$ rotations of $U$ so that
$$\int_{B(P, 2^{-k})} du_k ~\vol = \evect_2 \cdot \int_{B(P, 2^{-k})} |du_k| ~\vol.$$
The hypothesis $\Lambda(U, \rho) \ll \rho$ shows that $\gamma_k \ll 1$, so by the previous lemma, $1 \ll 1$, and everything breaks.
\end{proof}

By fiddling with the above lemma and local compactness it should follow that $\normal_s \to \normal$ is uniformly Cauchy on a subsequence and hence $\normal$ is continuous at $P$.



\printbibliography


\end{document}
