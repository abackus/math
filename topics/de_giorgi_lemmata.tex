\documentclass[reqno,12pt,letterpaper]{amsart}
\RequirePackage{amsmath,amssymb,amsthm,graphicx,mathrsfs,url}
\RequirePackage[usenames,dvipsnames]{color}
\RequirePackage[colorlinks=true,linkcolor=Red,citecolor=Green]{hyperref}
\RequirePackage{amsxtra}
\usepackage{cancel}
\usepackage{tikz-cd}

\setlength{\textheight}{9in} \setlength{\oddsidemargin}{-0.25in}
\setlength{\evensidemargin}{-0.25in} \setlength{\textwidth}{7in}
\setlength{\topmargin}{-0.25in} \setlength{\headheight}{0.18in}
\setlength{\marginparwidth}{1.0in}
\setlength{\abovedisplayskip}{0.2in}
\setlength{\belowdisplayskip}{0.2in}
\setlength{\parskip}{0.05in}
\renewcommand{\baselinestretch}{1.05}

\title[Geodesic laminations by minimal currents]{Geodesic laminations by minimal currents}
\author{Aidan Backus}
\date{July 2021}

\newcommand{\NN}{\mathbf{N}}
\newcommand{\ZZ}{\mathbf{Z}}
\newcommand{\QQ}{\mathbf{Q}}
\newcommand{\RR}{\mathbf{R}}
\newcommand{\CC}{\mathbf{C}}
\newcommand{\DD}{\mathbf{D}}
\newcommand{\PP}{\mathbf P}
\newcommand{\MM}{\mathbf M}
\newcommand{\II}{\mathbf I}
\newcommand{\Hyp}{\mathbf H}
\newcommand{\SL}{\mathrm{SL}}
\newcommand{\evect}{\mathbf e}

\DeclareMathOperator{\card}{card}
\DeclareMathOperator{\cent}{center}
\DeclareMathOperator{\ch}{ch}
\DeclareMathOperator{\codim}{codim}
\DeclareMathOperator{\diag}{diag}
\DeclareMathOperator{\diam}{diam}
\DeclareMathOperator{\dom}{dom}
\DeclareMathOperator{\Gal}{Gal}
\DeclareMathOperator{\Hom}{Hom}
\DeclareMathOperator{\Jac}{Jac}
\DeclareMathOperator{\Lip}{Lip}
\DeclareMathOperator{\Met}{Met}
\DeclareMathOperator{\id}{id}
\DeclareMathOperator{\rad}{rad}
\DeclareMathOperator{\rank}{rank}
\DeclareMathOperator{\Radon}{Radon}
\DeclareMathOperator*{\Res}{Res}
\DeclareMathOperator{\sgn}{sgn}
\DeclareMathOperator{\singsupp}{sing~supp}
\DeclareMathOperator{\Spec}{Spec}
\DeclareMathOperator{\supp}{supp}
\DeclareMathOperator{\Tan}{Tan}
\newcommand{\tr}{\operatorname{tr}}

\newcommand{\Ric}{\mathrm{Ric}}
\newcommand{\Riem}{\mathrm{Riem}}

\newcommand{\dbar}{\overline \partial}

\DeclareMathOperator{\atanh}{atanh}
\DeclareMathOperator{\arcosh}{arcosh}
\DeclareMathOperator{\csch}{csch}
\DeclareMathOperator{\sech}{sech}

\DeclareMathOperator{\Div}{div}
\DeclareMathOperator{\grad}{grad}
\DeclareMathOperator{\Ell}{Ell}
\DeclareMathOperator{\WF}{WF}

\newcommand{\Hilb}{\mathcal H}
\newcommand{\Homology}{\mathrm H_{\mathrm{dR}}}
\newcommand{\normal}{\mathbf n}
\newcommand{\vol}{\mathrm{vol}}

\newcommand{\pic}{\vspace{30mm}}
\newcommand{\dfn}[1]{\emph{#1}\index{#1}}

\renewcommand{\Re}{\operatorname{Re}}
\renewcommand{\Im}{\operatorname{Im}}


\newtheorem{theorem}{Theorem}[section]
\newtheorem{badtheorem}[theorem]{``Theorem"}
\newtheorem{prop}[theorem]{Proposition}
\newtheorem{lemma}[theorem]{Lemma}
\newtheorem{claim}[theorem]{Claim}
\newtheorem{proposition}[theorem]{Proposition}
\newtheorem{corollary}[theorem]{Corollary}
\newtheorem{conjecture}[theorem]{Conjecture}
\newtheorem{axiom}[theorem]{Axiom}
\newtheorem{assumption}[theorem]{Assumption}

\theoremstyle{definition}
\newtheorem{definition}[theorem]{Definition}
\newtheorem{remark}[theorem]{Remark}
\newtheorem{example}[theorem]{Example}
\newtheorem{notation}[theorem]{Notation}

\newtheorem{exercise}[theorem]{Discussion topic}
\newtheorem{homework}[theorem]{Homework}
\newtheorem{problem}[theorem]{Problem}

\newtheorem{ack}{Acknowledgements}

\numberwithin{equation}{section}


% Mean
\def\Xint#1{\mathchoice
{\XXint\displaystyle\textstyle{#1}}%
{\XXint\textstyle\scriptstyle{#1}}%
{\XXint\scriptstyle\scriptscriptstyle{#1}}%
{\XXint\scriptscriptstyle\scriptscriptstyle{#1}}%
\!\int}
\def\XXint#1#2#3{{\setbox0=\hbox{$#1{#2#3}{\int}$ }
\vcenter{\hbox{$#2#3$ }}\kern-.6\wd0}}
\def\ddashint{\Xint=}
\def\dashint{\Xint-}

%\usepackage{color}
%\hypersetup{%
%    colorlinks=true, % make the links colored%
%    linkcolor=blue, % color TOC links in blue
%    urlcolor=red, % color URLs in red
%    linktoc=all % 'all' will create links for everything in the TOC
%Ning added hyperlinks to the table of contents 6/17/19
%}

% style=alphabetic
\usepackage[backend=bibtex,maxcitenames=50,maxnames=50]{biblatex}
\addbibresource{topics.bib}
\renewbibmacro{in:}{}
\DeclareFieldFormat{pages}{#1}

\begin{document}
%%%%%%%%%%%%%%%%%%%%%%%%%%%%%%%%%%%%%%%%%%%%%%%%%%%%%%%

% \tableofcontents

\section{Structuring the proof}
The way that the proof of regularity in Miranda's 1966 paper, or in Giusti's book, is structured does not quite work for our purposes.
The reason is that in the way they structure the proof they lose a certain inequality that they never use but that we need.

Let $U$ be a set of least perimeter in a ball $B$ which is centered on a point $P \in M$.
Let $u$ be the indicator of $U$ and let $\normal$ be the normal vector to $\partial U$.
Let
$$\gamma(\rho) = \frac{1}{\rho^{d - 1}} \left(\int_B |du| ~\vol - \left|\int_B du ~\vol\right|\right).$$
We break into cases based on the decay rate of $\gamma$:

\begin{itemize}
\item[Base] There exists $\sigma > 0$ such that $\gamma \geq \sigma$ for every $\rho$ small.
\item[Inductive] There exists $\sigma' > 0$ such that $\gamma \geq \sigma' \rho$ for every $\rho$ small, but
$$\lim_{\rho \to 0} \gamma(\rho) = 0.$$
\item[Good] $\gamma(\rho) \lesssim \rho$ for every $\rho$ small.
\item[Misc] None of the above.
\end{itemize}

We can always break out of the miscellaneous case by taking subsequences.
If we are in the good case, then one can easily modify Miranda's main theorem (Chapter 8 in Giusti) to show that $\normal$ is continuous at $P$.
In this case, $\partial U$ meets the hypotheses of de Giorgi--Nash--Moser theory and we're done.

If we are in the base case, then the tangent cone $C$ to $U$ at $P$ is a singular minimal cone.
In particular, $\partial U$ has a singularity of codimension $\geq 7$.
Obviously this does not happen on $\Hyp^2$ or $\Hyp^3$, so we can rule out that case.

The interesting case is the inductive case.
In that case we can apply the de Giorgi lemma:

\begin{conjecture}
If $\gamma(\rho^*)$ is small enough and there exists $\sigma' > 0$ such that for every $\rho$, $\gamma(\rho) \geq \sigma' \rho$, then $\gamma(\rho^*/2) \leq \gamma(\rho^*)/4$.
\end{conjecture}

This conjecture clearly contradicts the inductive case, putting us in the good case.

To discuss how we might prove the conjecture, suppose that we have $\rho_n \to 0$, thus $\gamma(\rho_n) \to 0$, but $\gamma(\rho_n/2) > \gamma(\rho_n)/4$.
Then these $\rho_n$ are going to break the below stuff:

\section{De Giorgi lemma in hyperbolic space}
Let $g$ be the hyperbolic metric on $(-R, R) \times (0, \infty)$ given by
$$g = \begin{bmatrix}y^{-2} \\ & y^{-2}\end{bmatrix}.$$
Let
$$\mathscr L(x, \omega, p) = \sqrt{g_{11}(x, \omega) + 2g_{12}(x, u)p + g_{22}(x, \omega) p^2}$$
be the infinitesimal arc length of the graph of $\omega$ where $\omega' = p$.
Thus
$$\mathscr L(\omega, p) = \omega \sqrt{1 + |p|^2}.$$
This gives a metric $g_\omega$ on $(-\rho, \rho)$ by $\omega(x)^2 ~dx$.
We write $\Delta_\omega$ to mean the Laplace-Beltrami operator of $g_\omega$, which after a normalization is
$$\Delta_\omega = \partial^2 + 3(\log \omega)' \partial.$$
This operator is locally uniformly elliptic on $(-R, R)$.

Consider a sequence $(\omega_j)$ in $C^1((-R, R))$ and let $g_j = g_{\omega_j}$.
Fix $\rho_j > 0$ and let $B_{\delta,j}$ be the ball $B_{g_j}(0, \delta \rho_j)$.
We define the averaging operator $A_j$ by letting
$$A_jf(\delta) = \dashint_{B_{\delta,j}} f ~\vol_{g_j}.$$

\begin{lemma}
Suppose that $\omega_j' \to 0$ in $L^\infty(B_{1,j})$, $B_{1,j} \subseteq (-R, R)$, and the images of the $\omega_j$ lie in a uniform compact subset of $(0, \infty)$.
Let $u_j$ be the function defined by
\begin{align*}
\Delta_j u_j &= 0 \\
u_j|\partial B_{1, j} &= \omega_j|\partial B_{1, j}.
\end{align*}
Furthermore, suppose that there exists a small $\delta > 0$ such that
$$\rho_j^{3 - 2\delta} \lesssim \beta_j \ll 1.$$
If
$$\limsup_{j \to \infty} \frac{1}{\beta_j} \int_{B_{1,j}} \mathscr L(x, \omega_j(x), \omega_j'(x)) - \mathscr L(x, u_j(x), u_j'(x)) ~dx \leq 0$$
and
$$\int_{B_{1,j}} \mathscr L(x, \omega_j(x), \omega_j'(x)) - \mathscr L(x, \omega_j(x),  A_j \omega_j'(\rho_j))) ~dx \leq \beta_j,$$
then
$$\limsup_{j \to \infty} \frac{1}{\beta_j} \int_{B_{1/2,j}} \mathscr L(x, \omega_j(x), \omega_j'(x)) - \mathscr L(x, \omega_j(x),  A_j \omega_j'(\rho_j/2))) ~dx \leq \frac{1}{8}.$$
\end{lemma}
\begin{proof}
We first bound for any $p, q \in \RR$
\begin{align*}
\mathscr L(\omega, p) - \mathscr L(\omega, q) &= \omega(\sqrt{1 + p^2} - \sqrt{1 + q^2}) \\
&\leq \frac{\omega}{2\sqrt{1 + q^2}}(p^2 - q^2) \leq \frac{\omega}{2}(p^2 - q^2).
\end{align*}
Therefore
\begin{align*}
\int_{B_{1/2,j}} \mathscr L(\omega_j(x), \omega_j'(x)) - \mathscr L(\omega_j(x),  A_j \omega_j'(\rho_j/2))) ~dx &\leq \frac{1}{2} \int_{B_{1/2,j}} (\omega_j'(x)^2 -  A_j \omega_j'(\rho_j/2)^2 ) \omega_j(x) ~dx.
\end{align*}
We call the right-hand side $I$ and we seek to bound it.

We begin by transforming the domain.
Writing $\vol_j = \omega_j(x) ~dx$ for the $1$-form induced by $g_j$, and using geodesic coordinates centered on $0$ and the fact that the metric $g_j$ is flat (so the Taylor expansion of $g_{ij}$ in geodesic coordinates is just $\delta_{ij}$)
$$\int_{B_{\delta,j}} f ~\vol_j = \int_{-\delta \rho_j}^{\delta \rho_j} f(X) ~dX.$$
Thus we have a change of coordinates $B_{\delta, j } \to (-\delta \rho_j, \delta \rho_j)$ which induces a change of $1$-forms $dX = \omega_j(x) ~dx$.
In such coordinates it follows that
$$2I = \int_{-\rho_j/2}^{\rho_j/2} \omega_j'(X)^2 -  A_j \omega_j'(\rho_j/2)^2 ~dX.$$
In general, we would get error terms here.

Since $A_j \omega_j'$ is the mean of $\omega_j$ over this domain,
$$\int_{-\rho_j/2}^{\rho_j/2} \omega_j'(X) A_j \omega_j'(\rho_j/2) ~dX = \int_{-\rho_j/2}^{\rho_j/2} A_j \omega_j'(\rho_j/2)^2 ~dX$$
which readily implies
$$2I = \int_{-\rho_j/2}^{\rho_j/2} (\omega_j'(X) -  A_j \omega_j'(\rho_j/2))^2 ~dX.$$
For any function $f$, $||f - y||_{L^2}$ is minimized exactly when $y$ is the mean of $f$ (just differentiate the square of that norm in $y$ and look for critical points) so
$$2I \leq \int_{-\rho_j/2}^{\rho_j/2} (\omega_j'(X) -  A_j \omega_j'(\rho_j))^2 ~dX.$$

Let $\varepsilon > 0$.
By Cauchy's inequality
$$|\alpha - \gamma|^2 \leq (1 + \varepsilon) |\alpha - \beta|^2 + (1 + 1/\varepsilon) |\beta - \gamma|^2$$
we have
$$2I \leq (1 + 1/\varepsilon) \int_{B_{1/2,j}} |\omega_j'(X) - u_j'(X)|^2 ~dX + (1 + \varepsilon) \int_{B_{1/2,j}} |u_j'(X) -  A_j \omega_j'(\rho_j)|^2 ~dX.$$
Our boundary condition on $u_j$ says
$$A_j u_j'(\rho_j) = A_j\omega_j'(\rho_j).$$
and in the coordinates $X$, $\Delta_j$ becomes $\partial^2$, thus $u_j'' = 0$.
So by the mean-value property and Cauchy's inequality,
\begin{align*}
\int_{-\rho_j/2}^{\rho_j/2} |u_j'(X) -  A_j \omega_j'(\rho_j)|^2 ~dX &\leq \frac{1}{8} \int_{-\rho}^{\rho} |u_j'(X) -  A_j \omega_j'(\rho_j)|^2 ~dX\\
&\leq \frac{1 + \varepsilon}{8} \int_{-\rho_j}^{\rho_j} |\omega_j'(X) -  A_j \omega_j'(\rho_j)|^2 ~dX \\
&\qquad + \frac{1 + 1/\varepsilon}{8} \int_{-\rho_j}^{\rho_j} |\omega_j'(X) - u_j'(X)|^2 ~dX
\end{align*}
Thus
\begin{align*}
I &\leq \frac{(1 + \varepsilon)^2}{16} \int_{-\rho_j}^{\rho_j} |\omega_j'(X) -  A_j \omega_j'(\rho_j)|^2 ~dX + O(1) \int_{-\rho_j}^{\rho_j} |\omega_j'(X) - u_j'(X)|^2 ~dX\\
&= \frac{(1 + \varepsilon)^2}{16} J + O(1) K.
\end{align*}
Here $O(1)$ denotes a constant that is allowed to depend on $\varepsilon$ but not on $j$.

\begin{claim}
If $j$ is large enough then
$$J \leq (2 + \varepsilon)\beta_j.$$
\end{claim}
\begin{proof}[Proof of claim]
One has
\begin{align*}
J &= \int_{-\rho_j}^{\rho_j} \omega_j'(X)^2 -  A_j \omega_j'(\rho_j)^2 ~dX \\
&= \int_{B_{1, j}} \omega_j'(x)^2 \omega_j(x) -  A_j \omega_j'(\rho_j)^2 \omega_j(x) ~dx \\
&\leq \int_{B_{1, j}} (\omega_j'(x)^2 -  A_j \omega_j'(\rho_j)^2)^2 \omega_j(x)^2 ~dx \\
&\qquad + 2\sqrt{1 +  A_j \omega_j'(\rho_j)^2} \int_{B_{1,j}} \mathscr L(\omega_j, \omega_j') - \mathscr L(\omega_j,  A_j \omega_j'(\rho_j)) \\
&\leq 2 ||\omega_j' ||_{L^\infty}^2 ||\omega_j||_{L^\infty}^2 \int_{-\rho_j}^{\rho_j} \omega_j'(X)^2 -  A_j \omega_j'(\rho_j)^2 ~dX + 2 \beta_j\\
&\leq 2\beta_j + o(1) J. \qedhere
\end{align*}
\end{proof}

\begin{claim}
One has $K \leq o(\beta_j)$.
\end{claim}
\begin{proof}[Proof of claim]
Since $u_j$ is harmonic,
\begin{align*}
K &= \int_{-\rho_j}^{\rho_j} \omega_j'(X)^2 - u_j'(X)^2 ~dX \\
&\leq 2 \sqrt{1 + ||u_j'||_{L^\infty}^2} \int_{B_{1,j}} \mathscr L(\omega_j, \omega_j') - \mathscr L(\omega_j, u_j')\\
&\lesssim \int_{B_{1, j}} \mathscr L(\omega_j, \omega_j') - \mathscr L(u_j, u_j') + \int_{B_{1, j}} \mathscr L(u_j, u_j') - \mathscr L(\omega_j, u_j') \\
&\leq o(\beta_j) + 2||\omega_j - u_j||_{L^1(B_{1,j})}
\end{align*}
for all $j$ large enough.
The second term is the only interesting new term created by the presence of geometry.
To bound it, we observe that $\omega_j - u_j$ is trace-free, so by the Poincar\'e, H\"older, and Cauchy inequalities,
$$||\omega_j - u_j||_{L^1} \lesssim \rho ||\omega_j' - u_j'||_{L^1} \leq \rho^{3/2} ||\omega_j' - u_j'||_{L^2} \leq \rho^{3 - \delta} + \rho^\delta ||\omega_j' - u_j'||_{L^2}^2.$$
Moreover
$$||\omega_j' - u_j'||_{L^2}^2 \lesssim \int_{-\rho_j}^{\rho_j} (\omega_j'(X) - u_j'(X))^2 ~dX = K$$
owing to the uniform boundedness of the $\omega_j$.
Thus
$$K \leq o(\beta_j) + \rho^{3 - \delta} + O(\rho^\delta K)$$
or in other words
\begin{align*}
K &\leq \frac{o(\beta_j) + \rho^{3 - \delta}}{1 - O(\rho^\delta)} \leq \frac{o(\beta_j)}{1 + o(1)}. \qedhere
\end{align*}
\end{proof}

Therefore
$$\frac{I}{\beta_j} \leq \frac{(1 + \varepsilon)^2(1 + \varepsilon/2)}{8} + o(1).$$
Taking the limit as $j \to \infty$ we see the claim.
\end{proof}

In what follows we let $\evect_1, \evect_2 \in S^1$ be the standard basis of $\RR^2$, and let $B_r$ be the hyperbolic ball centered on $\evect_2$ of radius $r > 0$, where we view $\Hyp^2$ as the upper half-plane in $\RR^2$.
Since we are favoring a particular system of coordinates we don't need to worry about cleverly defining the excess.

\begin{lemma}
Let $(U_j)$ be a sequence of open sets in $\Hyp^2$ with $C^1$ boundary.
Let $u_j = 1_{U_j}$ and let $\normal_j: \partial U_j \to S^1$ be the Gauss map.
Fix $\delta > 0$ and $\rho_j, \beta_j$ such that0
If
$$\int_{B_{\rho_j}} |du_j| ~\vol - \left|\int_{B_{\rho_j}} du_j ~\vol\right| \leq \beta_j,$$
$$\lim_{j \to \infty} \frac{|\partial U_j \cap B_{\rho_j}| - \eta(U_j, B_{\rho_j})}{\beta_j} = 0,$$
and $\normal_j|\partial B_{\rho_j} \to \evect_2$ uniformly,
then
$$\limsup_{j \to \infty} \frac{1}{\beta_j} \int_{B_{\rho_j/2}} |du_j| ~\vol - \frac{1}{\beta_j}\left|\int_{B_{\rho_j/2}} du_j ~\vol\right| \leq \frac{1}{8}.$$
\end{lemma}
\begin{proof}
Suppose that the lemma is false, thus:

\begin{claim}
There exists a sequence $(U_j)$ meeting the hypotheses of the lemma such that
$$\lim_{j \to \infty} \frac{1}{\beta_j} \int_{B_{\rho_j/2}} |du_j| ~\vol - \frac{1}{\beta_j}\left|\int_{B_{\rho_j/2}} du_j ~\vol\right| > \frac{1}{8}.$$
Moreover, there exist nonempty open sets $I_j \subseteq \RR$ and $C^1$ functions $\omega_j: I_j \to (1/2, 3/2)$ such that the graph of $\omega_j$ is $\partial U_j$ and $\omega_j \to 1$ in $C^1$.
\end{claim}
\begin{proof}[Proof of claim]
Let $(U_j)$ be a sequence which contradicts the lemma, and let $\varepsilon > 0$.
By taking subsequences and using the uniform convergence of $\normal_j$ we may assume that
$$\normal_j|B_{\rho_j} \cdot \evect_2 > 1 - \varepsilon.$$
(Here we use the euclidean dot product.)
If $\varepsilon$ is chosen small enough, there exist open sets $I_j \subseteq \RR$ and $C^1$ functions $\omega_j: I_j \to (0, \infty)$ such that the graph of $\omega_j$ is $\partial U_j$.

If $I_j$ is empty, then $u_j$ is constant and so $du_j = 0$ identically.
Therefore $u_j$ does not contribute to the claimed limit being $> 1/8$ and so by taking subsequences we may discard all empty $I_j$.

Since $\normal_j \to 0$ uniformly, the same holds for $\omega_j'$.
Therefore, by taking subsequences, we may assume that $\omega_j$ tends to a constant function.
Since $I_j$ is nonempty, there always exists $x_j \in I_j$ such that $\omega_j(x_j) \in (1 - O(\rho_j), 1 + O(\rho_j))$ and yet $\rho_j \to 0$.
This gives the desired convergence in $C^1$.
\end{proof}

If $|I| = r$, then
\begin{align*}
\int_I \mathscr L(\omega_j, \omega_j') &= \int_{I \times \RR_+} |du_j| ~\vol,
\end{align*}
since both sides are the hyperbolic length of $\partial U_j$ in $I \times \RR_+$.
Similarly, if $I$ is the $g_j$-ball of radius $r < \rho_j$ then
$$\mathscr L(\omega_j, A_j \omega_j'(r)) = \frac{1}{2r} \left|\int_{(I \times \RR_+) \cap B_{\rho_j}} du_j ~\vol\right|$$
so
$$\int_I \mathscr L(\omega_j, \omega_j') - \mathscr L(\omega_j, A_j \omega_j'(r)) \leq \int_{B_{\rho_j}} |du_j| ~\vol - \left|\int_{B_{\rho_j}} du_j ~\vol\right| \leq \beta_j.$$
Since $I \subset B_{1, j}$ we can take the limit as $I$ grows to $B_{1, j}$, and
$$\int_{B_{1, j}} \mathscr L(\omega_j, \omega_j') - \mathscr L(\omega_j, A_j \omega_j'(\rho_j)) \leq \beta_j.$$

Let $u_j$ be the harmonic function on $B_{1, j}$ such that $u_j - \omega_j$ is trace-free. Then
$$\eta(\partial U_j, B_{\rho_j}) \leq \int_{B_{1, j}} \mathscr L(u_j, u_j')$$
since the right-hand side is the arc length of the graph of $u_j$, which is necessarily longer than the length of the geodesic with the same endpoints.
Therefore
$$\lim_{j \to \infty} \frac{1}{\beta_j} \int_{B_{1, j}} \mathscr L(\omega_j, \omega_j') - \mathscr L(u_j, u_j') \leq \lim_{j \to \infty} \frac{|\partial U_j \cap B_{\rho_j}| - \eta(U_j, B_{\rho_j})}{\beta_j} = 0$$
so by the previous lemma,
$$\limsup_{j \to \infty} \frac{1}{\beta_j} \int_{B_{1/2,j}} \mathscr L(\omega_j, \omega_j') - \mathscr L(\omega_j, A_j \omega_j'(\rho/2)) \leq \frac{1}{8}.$$
This contradicts the claim when we take $I = B_{1/2,j}$.
\end{proof}

\section{Mollification in hyperbolic plane}
Let's assume that the mollification works for now.
I think it's pretty obvious that it does, because I've worked through the mollification argument in very similar circumstances, but there are a lot of annoying details to check (mostly Sobolev embedding stuff).
I haven't had time to write them all down in this particular case because all my time goes to grading midterms :-(

If the mollification argument works, then the above lemma gives:
\begin{lemma}
Let $(U_j)$ be a sequence of sets of hyperbolic least perimeter in $B_j = B(\evect_2, \rho_j)$, $u_j = 1_{U_j}$, where
$$\int_{B_j} |du_j| ~\vol - \left|\int_{B_j} du_j ~\vol\right| \leq \gamma_j$$
and
$$0 < \rho_j^3 \ll \gamma_j \ll 1.$$
Then
$$\limsup_{j \to \infty} \gamma_j^{-1}\left[\int_{B_j/2} |du_j| ~\vol - \left|\int_{B_j/2} du_j ~\vol\right|\right] \leq \frac{1}{8}.$$
\end{lemma}

\section{Continuity of the conormal}
Fix $P \in \Hyp^2$ and write
$$\Lambda(U, \rho) = \frac{1}{\rho}\left[\int_{B(P, \rho)} |du| ~\vol - \left|\int_{B(P, \rho)} du ~\vol\right|\right].$$

\begin{lemma}
There exists $\sigma > 0$ and $r > 0$ such that for every set of hyperbolic least perimeter $U \subseteq \Hyp^2$ of indicator $u$ such that
$$\Lambda(U, \rho) < \sigma,$$
where $\rho < r$,
it follows that
$$\Lambda(U, \delta \rho) \lesssim \delta^2.$$
\end{lemma}
\begin{proof}
Suppose not, so there exist $U_j$, $\rho_j$, $\gamma_j$, $\delta_j$ such that
\end{proof}

We write
$$\normal_s(P) = \frac{\int_{B(P, s)} du ~\vol}{\int_{B(P, s)} |du| ~\vol}.$$
Then, the Hardy-Littlewood maximal inequality theory I already developed shows that $\normal_s(P)$ is Cauchy as $s \to 0$ iff $P \in \partial^* U$.

\begin{lemma}
Let $P \in \Hyp^2$ and let $U$ meet the hypotheses of the previous lemma, $P \in \partial U$.
Let $\normal_s(P)$ be the $|du|~\vol$-mean of $du~\vol$ on $B(P, s)$. Then if $0 < s < t < \rho$,
$$|\normal_s(P) - \normal_t(P)| \lesssim \sqrt t.$$
\end{lemma}



\printbibliography


\end{document}
