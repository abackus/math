\documentclass[reqno,12pt,letterpaper]{amsart}
\RequirePackage{amsmath,amssymb,amsthm,graphicx,mathrsfs,url}
\RequirePackage[usenames,dvipsnames]{color}
\RequirePackage[colorlinks=true,linkcolor=Red,citecolor=Green]{hyperref}
\RequirePackage{amsxtra}
\usepackage{cancel}
\usepackage{tikz-cd}

\setlength{\textheight}{9in} \setlength{\oddsidemargin}{-0.25in}
\setlength{\evensidemargin}{-0.25in} \setlength{\textwidth}{7in}
\setlength{\topmargin}{-0.25in} \setlength{\headheight}{0.18in}
\setlength{\marginparwidth}{1.0in}
\setlength{\abovedisplayskip}{0.2in}
\setlength{\belowdisplayskip}{0.2in}
\setlength{\parskip}{0.05in}
\renewcommand{\baselinestretch}{1.05}

\title[Geodesic laminations by minimal currents]{Geodesic laminations by minimal currents}
\author{Aidan Backus}
\date{July 2021}

\newcommand{\NN}{\mathbf{N}}
\newcommand{\ZZ}{\mathbf{Z}}
\newcommand{\QQ}{\mathbf{Q}}
\newcommand{\RR}{\mathbf{R}}
\newcommand{\CC}{\mathbf{C}}
\newcommand{\DD}{\mathbf{D}}
\newcommand{\PP}{\mathbf P}
\newcommand{\MM}{\mathbf M}
\newcommand{\II}{\mathbf I}
\newcommand{\Hyp}{\mathbf H}

\DeclareMathOperator{\card}{card}
\DeclareMathOperator{\cent}{center}
\DeclareMathOperator{\ch}{ch}
\DeclareMathOperator{\codim}{codim}
\DeclareMathOperator{\diag}{diag}
\DeclareMathOperator{\diam}{diam}
\DeclareMathOperator{\dom}{dom}
\DeclareMathOperator{\Gal}{Gal}
\DeclareMathOperator{\Hom}{Hom}
\DeclareMathOperator{\Jac}{Jac}
\DeclareMathOperator{\Lip}{Lip}
\DeclareMathOperator{\Met}{Met}
\DeclareMathOperator{\id}{id}
\DeclareMathOperator{\rad}{rad}
\DeclareMathOperator{\rank}{rank}
\DeclareMathOperator{\Radon}{Radon}
\DeclareMathOperator*{\Res}{Res}
\DeclareMathOperator{\sgn}{sgn}
\DeclareMathOperator{\singsupp}{sing~supp}
\DeclareMathOperator{\Spec}{Spec}
\DeclareMathOperator{\supp}{supp}
\DeclareMathOperator{\Tan}{Tan}
\newcommand{\tr}{\operatorname{tr}}

\newcommand{\Ric}{\mathrm{Ric}}
\newcommand{\Riem}{\mathrm{Riem}}

\newcommand{\dbar}{\overline \partial}

\DeclareMathOperator{\atanh}{atanh}
\DeclareMathOperator{\csch}{csch}
\DeclareMathOperator{\sech}{sech}

\DeclareMathOperator{\Div}{div}
\DeclareMathOperator{\grad}{grad}
\DeclareMathOperator{\Ell}{Ell}
\DeclareMathOperator{\WF}{WF}

\newcommand{\Hilb}{\mathcal H}
\newcommand{\Homology}{\mathrm H_{\mathrm{dR}}}
\newcommand{\normal}{\mathbf n}
\newcommand{\vol}{\mathrm{vol}}

\newcommand{\pic}{\vspace{30mm}}
\newcommand{\dfn}[1]{\emph{#1}\index{#1}}

\renewcommand{\Re}{\operatorname{Re}}
\renewcommand{\Im}{\operatorname{Im}}


\newtheorem{theorem}{Theorem}[section]
\newtheorem{badtheorem}[theorem]{``Theorem"}
\newtheorem{prop}[theorem]{Proposition}
\newtheorem{lemma}[theorem]{Lemma}
\newtheorem{claim}[theorem]{Claim}
\newtheorem{proposition}[theorem]{Proposition}
\newtheorem{corollary}[theorem]{Corollary}
\newtheorem{conjecture}[theorem]{Conjecture}
\newtheorem{axiom}[theorem]{Axiom}
\newtheorem{assumption}[theorem]{Assumption}

\theoremstyle{definition}
\newtheorem{definition}[theorem]{Definition}
\newtheorem{remark}[theorem]{Remark}
\newtheorem{example}[theorem]{Example}
\newtheorem{notation}[theorem]{Notation}

\newtheorem{exercise}[theorem]{Discussion topic}
\newtheorem{homework}[theorem]{Homework}
\newtheorem{problem}[theorem]{Problem}

\newtheorem{ack}{Acknowledgements}

\numberwithin{equation}{section}


% Mean
\def\Xint#1{\mathchoice
{\XXint\displaystyle\textstyle{#1}}%
{\XXint\textstyle\scriptstyle{#1}}%
{\XXint\scriptstyle\scriptscriptstyle{#1}}%
{\XXint\scriptscriptstyle\scriptscriptstyle{#1}}%
\!\int}
\def\XXint#1#2#3{{\setbox0=\hbox{$#1{#2#3}{\int}$ }
\vcenter{\hbox{$#2#3$ }}\kern-.6\wd0}}
\def\ddashint{\Xint=}
\def\dashint{\Xint-}

%\usepackage{color}
%\hypersetup{%
%    colorlinks=true, % make the links colored%
%    linkcolor=blue, % color TOC links in blue
%    urlcolor=red, % color URLs in red
%    linktoc=all % 'all' will create links for everything in the TOC
%Ning added hyperlinks to the table of contents 6/17/19
%}

% style=alphabetic
\usepackage[backend=bibtex,maxcitenames=50,maxnames=50]{biblatex}
\addbibresource{topics.bib}
\renewbibmacro{in:}{}
\DeclareFieldFormat{pages}{#1}

\begin{document}
%%%%%%%%%%%%%%%%%%%%%%%%%%%%%%%%%%%%%%%%%%%%%%%%%%%%%%%

% \tableofcontents

\section{De Giorgi lemma in conformal gauge}
Let $g$ be a metric on $[-\rho, \rho] \times \RR$.
Let
$$\mathscr L(x, \omega, p) = \sqrt{g_{11}(x, \omega) + 2g_{12}(x, u)p + g_{22}(x, \omega) p^2}$$
be the infinitesimal arc length of the graph of $u$ where $u' = p$.

We always assume that
$$g = f^2 \begin{bmatrix}1 & \\ & 1\end{bmatrix}$$
for some real function $f: [-\rho, \rho] \times \RR \to \RR$.
We say that $g$ is in \dfn{conformal gauge}.
Clearly any metric on $\RR^2$ can be put in conformal gauge, but this is not true in higher dimensions, where we will need to find a better gauge.

Thus
$$\mathscr L(x, \omega, p) = f(x, \omega)\sqrt{1 + |p|^2}.$$
This gives a metric $g_\omega$ on $[-\rho, \rho]$ by $f(x, \omega(x))^2$.
We write $\Delta_\omega$ to mean the Laplace-Beltrami operator of $g_\omega$, that is,
$$\Delta_\omega u = 3f(\partial_1f + \partial_2f \omega')u' + f^2 u''.$$

Consider a sequence $(\omega_j)$, then $g_{\omega_j}$ thinks that $[-\rho, \rho]$ is a ball.
(This is not true in higher dimensions.)
Let its center be $x_j$.
We write $A_ju(\delta)$ to mean the mean of $u$ with respect to $g_{\omega_j}$ on what $g_{\omega_j}$ thinks is the ball centered on $x_j$ of radius $r$, where $r$ is scaled so that if $\delta = \rho$ then $B_r = [-\rho, \rho]$ and if $\delta = \rho/2$ then $B_r$ has half the radius that $[-\rho, \rho]$ does.

\begin{lemma}
Let $(\omega_j)$ be a sequence in $C^1([-\rho, \rho])$ which tends to $0$ in $W^{1, \infty}$.
Let $u_j$ be the function defined by
\begin{align*}
u_j(-\rho) &= \omega_j(-\rho) \\
u_j(\rho) &= \omega_j(\rho) \\
\Delta_{\omega_j} u_j &= 0.
\end{align*}
Let $a_j,b_j$ bound the ball that $g_{\omega_j}$ thinks is centered on $x_j$ of half-radius to $[-\rho, \rho]$.
If
$$\limsup_{j \to \infty} \frac{1}{\beta_j} \int_{-\rho}^\rho \mathscr L(x, \omega_j(x), \omega_j'(x)) - \mathscr L(x, u_j(x), u_j'(x)) ~dx = 0$$
and
$$\int_{-\rho}^\rho \mathscr L(x, \omega_j(x), \omega_j'(x)) - \mathscr L(x, \omega_j(x), A_{\omega_j} \omega_j'(\rho))) ~dx \leq \beta_j,$$
then
$$\limsup_{j \to \infty} \frac{1}{\beta_j} \int_{a_j}^{b_j} \mathscr L(x, \omega_j(x), \omega_j'(x)) - \mathscr L(x, \omega_j(x), A_{\omega_j} \omega_j'(\rho/2))) ~dx \leq \frac{1}{8}.$$
\end{lemma}
\begin{proof}
The integral we want to bound is
\begin{align*}
& \leq \frac{1}{2} \int_{a_j}^{b_j} (\omega_j'(x)^2 - A_j \omega_j'(\rho/2)^2 ) f(x, \omega_j(x)) ~dx \\
&= \frac{1}{2} \int_{a_j}^{b_j} (\omega_j'(x) - A_j \omega_j'(\rho/2))^2 f(x, \omega_j(x)) ~dx \\
&\leq \frac{1}{2} \int_{a_j}^{b_j} (\omega_j'(x) - A_j \omega_j'(\rho))^2 f(x, \omega_j(x)) ~dx\\
&= I
\end{align*}
To see this you can use a change of variables
$$dy = f(x, \omega_j(x)) ~dx$$
and then painfully write everything out to get the first equality.
The second inequality is basically Poincar\'e's inequality: for a function $h$, the integral of $(h - C)^p$ is minimized when $C$ is the mean of $h$.

By Cauchy's inequality, for every $\varepsilon > 0$,
$$2I \leq (1 + 1/\varepsilon) \int_{a_j}^{b_j} |\omega_j'(x) - u_j'(x)|^2 f(x, \omega_j(x)) ~dx + (1 + \varepsilon) \int_{a_j}^{b_j} |u_j'(x) - A_j \omega_j'(\rho)|^2 f(x, \omega_j(\rho)) ~dx.$$ Now the boundary data specifications of $u_j$ and the mean-value property for $u_j'$ imply that
$$A_j u_j'(\rho/2) = u_j'(x_j) = A_j u_j'(\rho) = A_j\omega_j'(\rho).$$
This is only true up to an error term in higher dimensions. We're using the fact that $\Delta_{\omega_j}$ really is the Laplace operator just written in funny coordinates.
Therefore Miranda's lemma and Cauchy's inequality gives
\begin{align*}
\int_{a_j}^{b_j} |u_j'(x) - A_j \omega_j'(\rho)|^2 f(x, \omega_j(\rho)) ~dx &\leq \frac{1}{8} \int_{-\rho}^{\rho} |u_j'(x) - A_j \omega_j'(\rho)|^2 f(x, \omega_j(\rho)) ~dx\\
&\leq \frac{1 + \varepsilon}{8} \int_{-\rho}^\rho |\omega_j'(x) - A_j \omega_j'(\rho)|^2 f(x, \omega_j(\rho)) ~dx \\
&\qquad + \frac{1 + 1/\varepsilon}{8} \int_{-\rho}^\rho |\omega_j'(x) - u_j'(x)|^2 f(x, \omega_j(\rho)) ~dx.
\end{align*}
Note that Miranda's lemma applies here because we know that $\Delta_{\omega_j}$ really is the Laplace operator.
Thus
\begin{align*}
I &\leq \frac{(1 + \varepsilon)^2}{16} \int_{-\rho}^\rho |\omega_j'(x) - A_j \omega_j(\rho)|^2 f(x, \omega_j(x)) ~dx + O(1) \int_{-\rho}^\rho |\omega_j'(x) - u_j'(x)|^2f(x, \omega_j(x)) ~dx\\
&\leq \frac{(1 + \varepsilon)^2}{16} J + O(1) K.
\end{align*}

To bound $J$ we use the convergence in $W^{1, \infty}$ to get
\begin{align*}
J &= \int_{-\rho}^\rho \omega_j'(x)^2 - A_j \omega_j(\rho)^2f(x, \omega_j(x))  ~dx \\
&\leq \int_{-\rho}^\rho |\omega_j'(x)^2 - A_j \omega_j'(\rho)^2|^2f(x, \omega_j(x))  ~dx \\
&\qquad+ 2\sqrt{1 + \omega_j'(\rho)}\int_{-\rho}^\rho \mathscr L(x, \omega_j(x), \omega_j'(x)) - \mathscr L(x, \omega_j(x), A_j\omega_j'(\rho)) ~dx\\
&\leq o(1) + 2(1 + o(1)) \int_{-\rho}^\rho \mathscr L(x, \omega_j(x), \omega_j'(x)) - \mathscr L(x, \omega_j(x), A_j\omega_j'(\rho)) ~dx \\
&\leq o(1) + 2\beta_j.\end{align*}
Here the asymptotic notation is as $j \to \infty$.

To bound $K$ we use the harmonicity of $u$ and the convergence in $W^{1, \infty}$ as above to get
\begin{align*}
K &= \int_{-\rho}^\rho (\omega_j'(x)^2 - u_j'(x)^2)f(x, u_j(x)) ~dx \\
&\leq O(1) \int_{-\rho}^\rho \mathscr L(x, \omega_j(x), \omega_j'(x)) - \mathscr L(x, u_j(x), u_j'(x)) ~dx \\
&\qquad + O(1) \int_{-\rho}^\rho \mathscr L(x, \omega_j(x), u_j'(x)) - \mathscr L(x, u_j(x), u_j'(x)) \\
&\qquad + o(1) \int_{-\rho}^\rho \mathscr L(x, \omega_j(x), \omega_j'(x)) - \mathscr L(x, \omega_j(x), A_j\omega_j'(\rho)) ~dx.
\end{align*}
The new thing that we have to bound, because of the presence of nontrivial geometry, is the $\omega_j - u_j$ term $L$ which satisfies
\begin{align*}
L &\lesssim \rho \sup_x \sqrt{1 + u_j'(x)^2} |f(x, u_j(x)) - f(x, \omega_j(x)))| \\
&\lesssim \rho^2 ||f'||_{L^\infty} ||u_j - \omega_j||_{L^\infty} \\
&\lesssim \rho^3 ||u_j' - \omega_j'||_{L^\infty} \\
&\lesssim \rho^3 ||u_j'||_{L^\infty} + \rho^3 ||\omega_j'||_{L^\infty}.
\end{align*}
The convergence in $W^{1, \infty}$ kills one of these guys. For the other one, we observe that by the maximum principle and the boundary data, $||u_j'||_{L^\infty} \lesssim ||\omega_j'||_{L^\infty}$ and so the convergence in $W^{1, \infty}$ kills it too.
Summing it all up we get
$$K \lesssim \int_{-\rho}^\rho \mathscr L(x, \omega_j(x), \omega_j'(x)) - \mathscr L(x, u_j(x), u_j'(x)) ~dx + o(1).$$

Therefore
$$I \leq \frac{(1 + \varepsilon)^2\beta_j}{8} + O(1)\int_{-\rho}^\rho \mathscr L(x, \omega_j(x), \omega_j'(x)) - \mathscr L(x, u_j(x), u_j'(x)) ~dx + o(1).$$
Dividing by $\beta_j$ and taking the limit as $j \to \infty$ we see the claim.
\end{proof}

This lemma is not quite the statement of Giusti Lma 6.2 because $[a_j, b_j]$ is not $[-\rho/2, \rho/2]$.
We should be able to change this lemma so that it is $[-\rho/2, \rho/2]$ that is fixed and the bigger interval that varies.
My only worry is if, back in the euclidean metric, the bigger interval will be so big that we won't be able to keep the induction going.
But I think we can.

\section{The higher-dimensional case}
How should we generalize to higher dimensions?
We cannot use conformal gauge. Instead we could assume
$$g = \begin{bmatrix}1 \\ & G\end{bmatrix}$$
in which I think determinant identities imply that the Lagrangian takes the form
$$\mathscr L(x, \omega, p) = \sqrt{1 + G(x, \omega)(p, p)} ~dx.$$
Unfortunately, the Laplace-Beltrami operator induced by $x \mapsto G(x, \omega(x))$ is nasty, as are the balls in question.
However, we still have a lot of freedom in $g$ and we can impose
$$G_{ij} = \delta_{ij} + O(|x|^2 + |\omega(x)|^2).$$
This gauge condition already implies a weak form of Miranda's lemma on Laplace-Beltrami.

If I want to suffer, I can use Bianchi and Young to impose an even stronger gauge condition, probably of the form
$$G_{ij} = \delta_{ij} + S_{ij} + O(|x|^3 + |\omega(x)|^3)$$
where $S$ is some explicit quadratic polynomial.
Anyways, in this gauge we want to integrate
$$\int_B \mathscr L(x, \omega, p) = \int_B \sqrt{1 + G(x, \omega)(p, p)} ~dx = \int_{B_\omega} \sqrt{\frac{1 + |p|_{G(x, \omega)}^2}{\det G(x, \omega)}} ~\widetilde{\vol_{G(x, \omega)}(x)}$$
and we can probably throw away the denominator because it is so close to $1$.
Here $B_\omega$ is $B$ equipped with the metric $x \mapsto G(x, \omega(x))$ rather than the euclidean metric.

\printbibliography


\end{document}
