\documentclass[reqno,12pt,letterpaper]{amsart}
\RequirePackage{amsmath,amssymb,amsthm,graphicx,mathrsfs,url}
\RequirePackage[usenames,dvipsnames]{color}
\RequirePackage[colorlinks=true,linkcolor=Red,citecolor=Green]{hyperref}
\RequirePackage{amsxtra}
\usepackage{cancel}
\usepackage{tikz-cd}

\setlength{\textheight}{9in} \setlength{\oddsidemargin}{-0.25in}
\setlength{\evensidemargin}{-0.25in} \setlength{\textwidth}{7in}
\setlength{\topmargin}{-0.25in} \setlength{\headheight}{0.18in}
\setlength{\marginparwidth}{1.0in}
\setlength{\abovedisplayskip}{0.2in}
\setlength{\belowdisplayskip}{0.2in}
\setlength{\parskip}{0.05in}
\renewcommand{\baselinestretch}{1.05}

\title[Geodesic laminations by minimal currents]{Geodesic laminations by minimal currents}
\author{Aidan Backus}
\date{July 2021}

\newcommand{\NN}{\mathbf{N}}
\newcommand{\ZZ}{\mathbf{Z}}
\newcommand{\QQ}{\mathbf{Q}}
\newcommand{\RR}{\mathbf{R}}
\newcommand{\CC}{\mathbf{C}}
\newcommand{\DD}{\mathbf{D}}
\newcommand{\PP}{\mathbf P}
\newcommand{\MM}{\mathbf M}
\newcommand{\II}{\mathbf I}
\newcommand{\Hyp}{\mathbf H}

\DeclareMathOperator{\card}{card}
\DeclareMathOperator{\cent}{center}
\DeclareMathOperator{\ch}{ch}
\DeclareMathOperator{\codim}{codim}
\DeclareMathOperator{\diag}{diag}
\DeclareMathOperator{\diam}{diam}
\DeclareMathOperator{\dom}{dom}
\DeclareMathOperator{\Gal}{Gal}
\DeclareMathOperator{\Hom}{Hom}
\DeclareMathOperator{\Jac}{Jac}
\DeclareMathOperator{\Lip}{Lip}
\DeclareMathOperator{\Met}{Met}
\DeclareMathOperator{\id}{id}
\DeclareMathOperator{\rad}{rad}
\DeclareMathOperator{\rank}{rank}
\DeclareMathOperator{\Radon}{Radon}
\DeclareMathOperator*{\Res}{Res}
\DeclareMathOperator{\sgn}{sgn}
\DeclareMathOperator{\singsupp}{sing~supp}
\DeclareMathOperator{\Spec}{Spec}
\DeclareMathOperator{\supp}{supp}
\DeclareMathOperator{\Tan}{Tan}
\newcommand{\tr}{\operatorname{tr}}

\newcommand{\Ric}{\mathrm{Ric}}
\newcommand{\Riem}{\mathrm{Riem}}

\newcommand{\dbar}{\overline \partial}

\DeclareMathOperator{\atanh}{atanh}
\DeclareMathOperator{\csch}{csch}
\DeclareMathOperator{\sech}{sech}

\DeclareMathOperator{\Div}{div}
\DeclareMathOperator{\grad}{grad}
\DeclareMathOperator{\Ell}{Ell}
\DeclareMathOperator{\WF}{WF}

\newcommand{\Hilb}{\mathcal H}
\newcommand{\Homology}{\mathrm H_{\mathrm{dR}}}
\newcommand{\normal}{\mathbf n}
\newcommand{\vol}{\mathrm{vol}}

\newcommand{\pic}{\vspace{30mm}}
\newcommand{\dfn}[1]{\emph{#1}\index{#1}}

\renewcommand{\Re}{\operatorname{Re}}
\renewcommand{\Im}{\operatorname{Im}}


\newtheorem{theorem}{Theorem}[section]
\newtheorem{badtheorem}[theorem]{``Theorem"}
\newtheorem{prop}[theorem]{Proposition}
\newtheorem{lemma}[theorem]{Lemma}
\newtheorem{claim}[theorem]{Claim}
\newtheorem{proposition}[theorem]{Proposition}
\newtheorem{corollary}[theorem]{Corollary}
\newtheorem{conjecture}[theorem]{Conjecture}
\newtheorem{axiom}[theorem]{Axiom}
\newtheorem{assumption}[theorem]{Assumption}

\theoremstyle{definition}
\newtheorem{definition}[theorem]{Definition}
\newtheorem{remark}[theorem]{Remark}
\newtheorem{example}[theorem]{Example}
\newtheorem{notation}[theorem]{Notation}

\newtheorem{exercise}[theorem]{Discussion topic}
\newtheorem{homework}[theorem]{Homework}
\newtheorem{problem}[theorem]{Problem}

\newtheorem{ack}{Acknowledgements}

\numberwithin{equation}{section}


% Mean
\def\Xint#1{\mathchoice
{\XXint\displaystyle\textstyle{#1}}%
{\XXint\textstyle\scriptstyle{#1}}%
{\XXint\scriptstyle\scriptscriptstyle{#1}}%
{\XXint\scriptscriptstyle\scriptscriptstyle{#1}}%
\!\int}
\def\XXint#1#2#3{{\setbox0=\hbox{$#1{#2#3}{\int}$ }
\vcenter{\hbox{$#2#3$ }}\kern-.6\wd0}}
\def\ddashint{\Xint=}
\def\dashint{\Xint-}

%\usepackage{color}
%\hypersetup{%
%    colorlinks=true, % make the links colored%
%    linkcolor=blue, % color TOC links in blue
%    urlcolor=red, % color URLs in red
%    linktoc=all % 'all' will create links for everything in the TOC
%Ning added hyperlinks to the table of contents 6/17/19
%}

% style=alphabetic
\usepackage[backend=bibtex,maxcitenames=50,maxnames=50]{biblatex}
\addbibresource{topics.bib}
\renewbibmacro{in:}{}
\DeclareFieldFormat{pages}{#1}

\begin{document}
%%%%%%%%%%%%%%%%%%%%%%%%%%%%%%%%%%%%%%%%%%%%%%%%%%%%%%%

% \tableofcontents

\section{De Giorgi lemma in hyperbolic space}
Let $g$ be the hyperbolic metric on $(-R, R) \times (0, \infty)$ given by
$$g = \begin{bmatrix}y^{-2} \\ & y^{-2}\end{bmatrix}.$$
Let
$$\mathscr L(x, \omega, p) = \sqrt{g_{11}(x, \omega) + 2g_{12}(x, u)p + g_{22}(x, \omega) p^2}$$
be the infinitesimal arc length of the graph of $\omega$ where $\omega' = p$.
Thus
$$\mathscr L(\omega, p) = \omega \sqrt{1 + |p|^2}.$$
This gives a metric $g_\omega$ on $(-\rho, \rho)$ by $\omega(x)^2 ~dx$.
We write $\Delta_\omega$ to mean the Laplace-Beltrami operator of $g_\omega$, that is,
$$\Delta_\omega = \partial^2 - 2(\partial \log \omega) \partial.$$
This operator is locally uniformly elliptic on $(-R, R)$.

Consider a sequence $(\omega_j)$ in $C^1((-R, R))$.
Fix $\rho > 0$ and let $B_{1/2,j} = (a_j, b_j)$ be the ball of radius $\rho/2$ with respect to $g_{\omega_j}$ centered at $0$, and $B_{1,j} = (\alpha_j, \beta_j)$ be the ball of radius $\rho$.
We define the averaging operator $A_j$ by letting
$$A_jf(\delta) = \dashint_{B_{g_{\omega_j}}(0, \delta)} f(x) ~dx.$$

\begin{lemma}
Suppose that $\omega_j \to 0$ in $W^{1, \infty}$ and $B_{1,j} \Subset (-R, R)$.
Let $u_j$ be the function defined by
\begin{align*}
\Delta_{\omega_j} u_j &= 0 \\
u_j|\partial B_1 &= \omega_j|\partial B_1.
\end{align*}
If
$$\limsup_{j \to \infty} \frac{1}{\beta_j} \int_{B_{1,j}} \mathscr L(x, \omega_j(x), \omega_j'(x)) - \mathscr L(x, u_j(x), u_j'(x)) ~dx = 0$$
and
$$\int_{B_{1,j}} \mathscr L(x, \omega_j(x), \omega_j'(x)) - \mathscr L(x, \omega_j(x), A_j \omega_j'(\rho))) ~dx \leq \beta_j,$$
then
$$\limsup_{j \to \infty} \frac{1}{\beta_j} \int_{B_{1/2,j}} \mathscr L(x, \omega_j(x), \omega_j'(x)) - \mathscr L(x, \omega_j(x), A_j \omega_j'(\rho/2))) ~dx \leq \frac{1}{8}.$$
\end{lemma}
\begin{proof}
We first bound for any $p, q \in \RR$
\begin{align*}
\mathscr L(\omega, p) - \mathscr L(\omega, q) &= \omega(\sqrt{1 + p^2} - \sqrt{1 + q^2}) \\
&\leq \frac{\omega}{2\sqrt{1 + q^2}}(p^2 - q^2) \leq \frac{\omega}{2}(p^2 - q^2).
\end{align*}
Therefore
\begin{align*}
\int_{B_{1/2,j}} \mathscr L(\omega_j(x), \omega_j'(x)) - \mathscr L(\omega_j(x), A_j \omega_j'(\rho/2))) ~dx &\leq \frac{1}{2} \int_{B_{1/2,j}} (\omega_j'(x)^2 - A_j \omega_j'(\rho/2)^2 ) \omega_j(x) ~dx.
\end{align*}
We call the right-hand side $I$ and we seek to bound it.
First we set $dX = \omega_j(x) ~dx$, a transformation which maps $B_{\delta,j} = B_{g_{\omega_j}}(0, \delta\rho)$ to $(-\delta\rho, \delta\rho)$ and maps $\Delta_j$ to $\partial^2$.
Thus
$$2I = \int_{-\rho/2}^{\rho/2} \omega_j'(X)^2 - A_j \omega_j'(\rho/2)^2 ~dX.$$
Since $A_j \omega_j'$ is the mean of $\omega_j$ over this domain, the integral of $\omega_j'(X) A_j\omega'(\rho/2) + 2A_j \omega_j'(\rho/2)^2$ is zero.
This implies
$$2I = \int_{-\rho/2}^{\rho/2} (\omega_j'(X) - A_j \omega_j'(\rho/2))^2 ~dX.$$
For any function $f$, $||f - y||_{L^2}$ is minimized exactly when $y$ is the mean of $f$ (just differentiate the square of that norm in $y$ and look for critical points) so
$$2I \leq \int_{-\rho/2}^{\rho/2} (\omega_j'(X) - A_j \omega_j'(\rho))^2 ~dX.$$

Let $\varepsilon > 0$.
By Cauchy's inequality
$$|\alpha - \gamma|^2 \leq (1 + \varepsilon) |\alpha - \beta|^2 + (1 + 1/\varepsilon) |\beta - \gamma|^2$$
we have
$$2I \leq (1 + 1/\varepsilon) \int_{B_{1/2,j}} |\omega_j'(X) - u_j'(X)|^2 ~dX + (1 + \varepsilon) \int_{B_{1/2,j}} |u_j'(X) - A_j \omega_j'(\rho)|^2 ~dX.$$
Our boundary condition on $u_j$ says
$$A_j u_j'(\rho) = A_j\omega_j'(\rho).$$
and in the coordinates $X$, $\Delta_j$ becomes $\partial^2$, thus $u_j'' = 0$.
So by the mean-value property and Cauchy's inequality,
\begin{align*}
\int_{-\rho/2}^{\rho/2} |u_j'(X) - A_j \omega_j'(\rho)|^2 ~dX &\leq \frac{1}{8} \int_{-\rho}^{\rho} |u_j'(X) - A_j \omega_j'(\rho)|^2 ~dX\\
&\leq \frac{1 + \varepsilon}{8} \int_{-\rho}^\rho |\omega_j'(X) - A_j \omega_j'(\rho)|^2 ~dX \\
&\qquad + \frac{1 + 1/\varepsilon}{8} \int_{-\rho}^\rho |\omega_j'(X) - u_j'(X)|^2 ~dX
\end{align*}
Thus
\begin{align*}
I &\leq \frac{(1 + \varepsilon)^2}{16} \int_{-\rho}^\rho |\omega_j'(X) - A_j \omega_j'(\rho)|^2 ~dX + O(1) \int_{-\rho}^\rho |\omega_j'(X) - u_j'(X)|^2 ~dX\\
&= \frac{(1 + \varepsilon)^2}{16} J + O(1) K.
\end{align*}
Here $O(1)$ denotes a constant that is allowed to depend on $\varepsilon$ but not on $j$.

To bound $J$, we write
\begin{align*}
J &= \int_{-\rho}^\rho \omega_j'(X)^2 - A_j \omega_j'(\rho)^2 ~dX \\
&= \int_{B_{1, j}} \omega_j'(x)^2 \omega_j(x) - A_j \omega_j'(\rho)^2 \omega_j(x) ~dx \\
&\leq \int_{B_{1, j}} (\omega_j'(x)^2 - A_j \omega_j'(\rho)^2)^2 \omega_j(x)^2 ~dx \\
&\qquad + 2\sqrt{1 + A_j \omega_j'(\rho)^2} \int_{B_{1,j}} \mathscr L(\omega_j, \omega_j') - \mathscr L(\omega_j, A_j \omega_j'(\rho)) \\
&\leq 2 ||\omega_j' ||_{L^\infty}^2 ||\omega_j||_{L^\infty}^2 \int_{-\rho}^\rho \omega_j'(X)^2 - A_j \omega_j'(\rho)^2 ~dX + 2 \beta_j\\
&\leq 2\beta_j + o(1) J.
\end{align*}
That is,
$$J \leq \frac{2\beta_j}{1 + o(1)},$$
which is suitable.

Since $u_j$ is harmonic,
\begin{align*}
K &= \int_{-\rho}^\rho \omega_j'(X)^2 - u_j'(X)^2 ~dX \\
&\leq 2 \sqrt{1 + ||u_j'||_{L^\infty}^2} \int_{B_{1,j}} \mathscr L(\omega_j, \omega_j') - \mathscr L(\omega_j, u_j')\\
&\lesssim \int_{B_{1, j}} \mathscr L(\omega_j, \omega_j') - \mathscr L(u_j, u_j') + \int_{B_{1, j}} \mathscr L(u_j, u_j') - \mathscr L(\omega_j, u_j') \\
&\leq o(\beta_j) + 2||\omega_j - u_j||_{L^1(B_{1,j})}
\end{align*}
for all $j$ large enough.
The second term is the only interesting new term created by the presence of geometry.
To bound it, we observe that $\omega_j - u_j$ is trace-free, so by the Poincar\'e, H\"older, and Cauchy inequalities, for every $\delta \in (0, 1)$,
$$||\omega_j - u_j||_{L^1} \lesssim \rho ||\omega_j' - u_j'||_{L^1} \leq \rho^{3/2} ||\omega_j' - u_j'||_{L^2} \leq \rho^{3 - \delta} + \rho^\delta ||\omega_j' - u_j'||_{L^2}^2.$$
Moreover
$$||\omega_j' - u_j'||_{L^2}^2 \lesssim \int_{-\rho}^\rho (\omega_j'(X) - u_j'(X))^2 ~dX = K$$
owing to the uniform boundedness of the $\omega_j$. (TODO Add this hypothesis)
Thus
$$K \leq o(\beta_j) + \rho^{3 - \delta} + O(\rho^\delta K)$$
or in other words
$$K \leq \frac{o(\beta_j) + \rho^{3 - \delta}}{1 - O(\rho^\delta)}.$$
TODO add a hypothesis to relate $\rho_j$ and $\beta_j$

Therefore
$$I \leq \frac{(1 + \varepsilon)^2\beta_j}{8} + O(1)\int_{-\rho}^\rho \mathscr L(x, \omega_j(x), \omega_j'(x)) - \mathscr L(x, u_j(x), u_j'(x)) ~dx + o(1).$$
Dividing by $\beta_j$ and taking the limit as $j \to \infty$ we see the claim.
\end{proof}

This lemma is not quite the statement of Giusti Lma 6.2 because $[a_j, b_j]$ is not $[-\rho/2, \rho/2]$.
We should be able to change this lemma so that it is $[-\rho/2, \rho/2]$ that is fixed and the bigger interval that varies.
My only worry is if, back in the euclidean metric, the bigger interval will be so big that we won't be able to keep the induction going.
But I think we can.

\section{The higher-dimensional case}
How should we generalize to higher dimensions?
We cannot use conformal gauge. Instead we could assume
$$g = \begin{bmatrix}1 \\ & G\end{bmatrix}$$
in which I think determinant identities imply that the Lagrangian takes the form
$$\mathscr L(x, \omega, p) = \sqrt{1 + G(x, \omega)(p, p)} ~dx.$$
Unfortunately, the Laplace-Beltrami operator induced by $x \mapsto G(x, \omega(x))$ is nasty, as are the balls in question.
However, we still have a lot of freedom in $g$ and we can impose
$$G_{ij} = \delta_{ij} + O(|x|^2 + |\omega(x)|^2).$$
This gauge condition already implies a weak form of Miranda's lemma on Laplace-Beltrami.

If I want to suffer, I can use Bianchi and Young to impose an even stronger gauge condition, probably of the form
$$G_{ij} = \delta_{ij} + S_{ij} + O(|x|^3 + |\omega(x)|^3)$$
where $S$ is some explicit quadratic polynomial.
Anyways, in this gauge we want to integrate
$$\int_B \mathscr L(x, \omega, p) = \int_B \sqrt{1 + G(x, \omega)(p, p)} ~dx = \int_{B_\omega} \sqrt{\frac{1 + |p|_{G(x, \omega)}^2}{\det G(x, \omega)}} ~\widetilde{\vol_{G(x, \omega)}(x)}$$
and we can probably throw away the denominator because it is so close to $1$.
Here $B_\omega$ is $B$ equipped with the metric $x \mapsto G(x, \omega(x))$ rather than the euclidean metric.

\printbibliography


\end{document}
