\documentclass[reqno,11pt]{amsart}
\usepackage[letterpaper, margin=1in]{geometry}
\RequirePackage{amsmath,amssymb,amsthm,graphicx,mathrsfs,url,slashed,subcaption}
\RequirePackage[usenames,dvipsnames]{xcolor}
\RequirePackage[colorlinks=true,linkcolor=Red,citecolor=Green]{hyperref}
\RequirePackage{amsxtra}
\usepackage{cancel}
\usepackage{tikz-cd}

% \setlength{\textheight}{9.3in} \setlength{\oddsidemargin}{-0.25in}
% \setlength{\evensidemargin}{-0.25in} \setlength{\textwidth}{7in}
% \setlength{\topmargin}{-0.25in} \setlength{\headheight}{0.18in}
% \setlength{\marginparwidth}{1.0in}
% \setlength{\abovedisplayskip}{0.2in}
% \setlength{\belowdisplayskip}{0.2in}
% \setlength{\parskip}{0.05in}
%\renewcommand{\baselinestretch}{1.05}

\title{Level sets of a $1$-harmonic function}
\author{Aidan Backus}
\date{October 2022}

\newcommand{\NN}{\mathbf{N}}
\newcommand{\ZZ}{\mathbf{Z}}
\newcommand{\QQ}{\mathbf{Q}}
\newcommand{\RR}{\mathbf{R}}
\newcommand{\CC}{\mathbf{C}}
\newcommand{\DD}{\mathbf{D}}
\newcommand{\PP}{\mathbf P}
\newcommand{\MM}{\mathbf M}
\newcommand{\II}{\mathbf I}
\newcommand{\Hyp}{\mathbf H}
\newcommand{\Sph}{\mathbf S}
\newcommand{\Group}{\mathbf G}
\newcommand{\GL}{\mathbf{GL}}
\newcommand{\Orth}{\mathbf{O}}
\newcommand{\SpOrth}{\mathbf{SO}}
\newcommand{\Ball}{\mathbf{B}}

\DeclareMathOperator*{\Expect}{\mathbf E}

\DeclareMathOperator{\avg}{avg}
\DeclareMathOperator{\card}{card}
\DeclareMathOperator{\cent}{center}
\DeclareMathOperator{\ch}{ch}
\DeclareMathOperator{\codim}{codim}
\DeclareMathOperator{\Cyl}{Cyl}
\DeclareMathOperator{\diag}{diag}
\DeclareMathOperator{\diam}{diam}
\DeclareMathOperator{\dom}{dom}
\DeclareMathOperator{\Exc}{Exc}
\newcommand{\ext}{\mathrm{ext}}
\DeclareMathOperator{\Gal}{Gal}
\DeclareMathOperator{\Hom}{Hom}
\DeclareMathOperator{\Iso}{Iso}
\DeclareMathOperator{\Jac}{Jac}
\DeclareMathOperator{\Lip}{Lip}
\DeclareMathOperator{\Met}{Met}
\DeclareMathOperator{\id}{id}
\DeclareMathOperator{\rad}{rad}
\DeclareMathOperator{\rank}{rank}
\DeclareMathOperator{\Rm}{Rm}
\DeclareMathOperator{\Hess}{Hess}
\DeclareMathOperator{\Hol}{Hol}
\DeclareMathOperator{\Prop}{Prop}
\DeclareMathOperator{\Radon}{Radon}
\DeclareMathOperator*{\Res}{Res}
\DeclareMathOperator{\sgn}{sgn}
\DeclareMathOperator{\singsupp}{sing~supp}
\DeclareMathOperator{\Spec}{Spec}
\DeclareMathOperator{\supp}{supp}
\DeclareMathOperator{\Tan}{Tan}
\newcommand{\tr}{\operatorname{tr}}

\newcommand{\Mink}{\mathbf m}
\newcommand{\Ric}{\mathrm{Ric}}
\newcommand{\Riem}{\mathrm{Riem}}
\newcommand*\dif{\mathop{}\!\mathrm{d}}
\newcommand*\Dif{\mathop{}\!\mathrm{D}}
\newcommand{\LapQL}{\Delta^{\mathrm{ql}}}

\newcommand{\dbar}{\overline \partial}

\DeclareMathOperator{\atanh}{atanh}
\DeclareMathOperator{\csch}{csch}
\DeclareMathOperator{\sech}{sech}

\DeclareMathOperator{\Div}{div}
\DeclareMathOperator{\Gram}{Gram}
\DeclareMathOperator{\grad}{grad}
\DeclareMathOperator{\dist}{dist}
\DeclareMathOperator{\spn}{span}
\DeclareMathOperator{\Ell}{Ell}
\DeclareMathOperator{\WF}{WF}

\newcommand{\Two}{\mathrm{I\!I}}

\newcommand{\Lagrange}{\mathscr L}
\newcommand{\DirQL}{\mathscr D^{\mathrm{ql}}}
\newcommand{\DirL}{\mathscr D}

\newcommand{\Hilb}{\mathcal H}
\newcommand{\Homology}{\mathrm H}
\newcommand{\normal}{\mathbf n}
\newcommand{\radial}{\mathbf r}
\newcommand{\evect}{\mathbf e}
\newcommand{\vol}{\mathrm{vol}}

\newcommand{\Bmu}{\boldsymbol \mu}
\newcommand{\Bnu}{\boldsymbol \nu}
\newcommand{\Blambda}{\boldsymbol \lambda}

\newcommand{\pic}{\vspace{30mm}}
\newcommand{\dfn}[1]{\emph{#1}\index{#1}}

\renewcommand{\Re}{\operatorname{Re}}
\renewcommand{\Im}{\operatorname{Im}}

\newcommand{\loc}{\mathrm{loc}}
\newcommand{\cpt}{\mathrm{cpt}}

\def\Japan#1{\left \langle #1 \right \rangle}

\newtheorem{theorem}{Theorem}[section]
\newtheorem{badtheorem}[theorem]{``Theorem"}
\newtheorem{prop}[theorem]{Proposition}
\newtheorem{lemma}[theorem]{Lemma}
\newtheorem{sublemma}[theorem]{Sublemma}
\newtheorem{proposition}[theorem]{Proposition}
\newtheorem{corollary}[theorem]{Corollary}
\newtheorem{conjecture}[theorem]{Conjecture}
\newtheorem{axiom}[theorem]{Axiom}
\newtheorem{assumption}[theorem]{Assumption}

\newtheorem{mainthm}{Theorem}
\renewcommand{\themainthm}{\Alph{mainthm}}

% \newtheorem{claim}{Claim}[theorem]
% \renewcommand{\theclaim}{\thetheorem\Alph{claim}}
\newtheorem*{claim}{Claim}

\theoremstyle{definition}
\newtheorem{definition}[theorem]{Definition}
\newtheorem{remark}[theorem]{Remark}
\newtheorem{example}[theorem]{Example}
\newtheorem{notation}[theorem]{Notation}

\newtheorem{exercise}[theorem]{Discussion topic}
\newtheorem{homework}[theorem]{Homework}
\newtheorem{problem}[theorem]{Problem}

\makeatletter
\newcommand{\proofpart}[2]{%
  \par
  \addvspace{\medskipamount}%
  \noindent\emph{Part #1: #2.}
}
\makeatother



\numberwithin{equation}{section}


% Mean
\def\Xint#1{\mathchoice
{\XXint\displaystyle\textstyle{#1}}%
{\XXint\textstyle\scriptstyle{#1}}%
{\XXint\scriptstyle\scriptscriptstyle{#1}}%
{\XXint\scriptscriptstyle\scriptscriptstyle{#1}}%
\!\int}
\def\XXint#1#2#3{{\setbox0=\hbox{$#1{#2#3}{\int}$ }
\vcenter{\hbox{$#2#3$ }}\kern-.6\wd0}}
\def\ddashint{\Xint=}
\def\dashint{\Xint-}

\usepackage[backend=bibtex,style=numeric]{biblatex}
\renewcommand*{\bibfont}{\normalfont\footnotesize}
\addbibresource{topics.bib}
\renewbibmacro{in:}{}
\DeclareFieldFormat{pages}{#1}


\begin{document}
\begin{abstract}
We show that the level sets of a $1$-harmonic function on a manifold of constant sectional curvature are analytic minimal hypersurfaces, extending a classical result of Miranda.
In a companion paper, we will use this fact to establish that the level sets of a $1$-harmonic function form a minimal lamination.
\end{abstract}

\maketitle

%%%%%%%%%%%%%%%%%%%%%%%%%%%%%%%%%%%%%%%%%%%%%%%%%%%%%%%

% \tableofcontents

\section{Introduction}
Throughout this paper, let $M$ be an oriented Riemannian manifold of metric $g$ and dimension $d$.
For a function $u \in BV_\loc(M)$, we write $\star |\dif u|$ for the total variation of the derivative, c.f. (\ref{total variation}).

\begin{definition}\label{main definitions}
A function $u \in BV_\loc(M)$ has \dfn{least gradient}, or is \dfn{$1$-harmonic}, if for every $v \in BV_\cpt(M)$,
\begin{equation}\label{least gradient functional}
\int_M \star |\dif u| \leq \int_M \star |\dif u + \dif v|.
\end{equation}
A set $U$ of locally finite perimeter has \dfn{least perimeter} if $1_U$ has least gradient.
\end{definition}

Aside from being of intrinsic interest, functions of least gradient arise naturally in applied mathematics via magnetic resonance imaging \cite{Tamasan2019, Joy09} and the continuum-time limit of certain combinatorial games \cite{Kohn06}.

The Euler-Lagrange equation for (\ref{least gradient functional}) is the $1$-Laplace equation\footnote{The precise definition for a weak solution of (\ref{1Laplacian}) is given by \cite{Mazon14}, but we will only work with the functional (\ref{least gradient functional}) and not the equation itself, so we will not state the definition of weak solution here.}
\begin{equation}\label{1Laplacian}
\dif^* \left(\frac{\dif u}{|\dif u|}\right) = 0.
\end{equation}
At least formally, (\ref{1Laplacian}) implies that the level sets of $u$ have zero mean curvature, and so one expects them to be minimal hypersurfaces.
It is a classical result that the level sets of a $1$-harmonic function have least perimeter \cite{BOMBIERI1969}, and this proof works in high generality.
The de Giorgi--Miranda regularity theorem \cite{deGiorgi61, Miranda66} shows that for $d \leq 7$ the level sets are then smooth minimal hypersurfaces, as one would formally expect from the $1$-Laplace equation (\ref{1Laplacian}).
We refer to the monograph of Giusti \cite[Part 1]{Giusti77} for an exposition of the proof of the de Giorgi--Miranda theorem.

The proof of the de Giorgi--Miranda regularity theorem strongly uses the flatness of the domain to take averages of $1$-forms in a covariant way.
In this paper, we show how one can take averages of $1$-forms in an approximately covariant way as long as the domain has appropriate symmetries.
Thus we are able to weaken the hypothesis of flatness in the de Giorgi--Miranda theorem and prove:

\begin{theorem}[de Giorgi--Miranda regularity theorem]\label{main lma}
Let $2 \leq d \leq 7$ and suppose that $M$ has constant sectional curvature.
Then every set of least perimeter in $M$ is bounded by embedded analytic stable minimal hypersurfaces.
\end{theorem}

%%%%%%%%%%%%%%%%%%%
\subsection{Applications of the main theorem}
To motivate the study of the level sets of $1$-harmonic functions, we turn to the other side of H\"older duality -- that is, to the $\infty$-Laplacian
$$\Delta_\infty v := (\nabla^\mu \partial^\nu v) \partial_\mu v \partial_\nu v.$$
The equation $\Delta_\infty v = 0$ is invariant under target translations $v \mapsto v + y$ and so Noether's theorem associates a conserved flux\footnote{By a \dfn{conserved flux}, we mean a closed $d-1$-form $\psi$; the equation $\dif \psi = 0$ can be viewed as a conservation law.} $\dif u$ to each $\infty$-harmonic function $v$.
Daskalopolous--Uhlenbeck \cite{daskalopoulos2020transverse,daskalopoulosPrep1} show that if $M = \Hyp^2$, then $u$ is a $1$-harmonic function, and the level sets of $u$ are geodesics contained in the \dfn{maximum stretch locus} $\{|\dif v| = ||\dif v||_{L^\infty}\}$.
Maximum stretch loci of $\infty$-harmonic (or more generally best-Lipschitz) functions on surfaces were studied by Thurston, who showed them to be \dfn{geodesic laminations} -- that is, closed sets which are foliated by geodesics \cite{thurston1979geometry,thurston1998minimal}.
Thus the level sets of a $1$-harmonic function which arises as the potential of a Noetherian flux for the $\infty$-Laplacian form a geodesic lamination as well.

It is natural to expect that the same ideas hold in higher generality.
In general dimension one does not obtain a geodesic lamination but rather a \dfn{minimal lamination} -- that is, a closed set which is foliated by hypersurfaces of zero mean curvature; these hypersurfaces are called \dfn{leaves}.
In the sequel paper \cite{BackusCML}, we will apply Theorem \ref{main lma} to carefully treat the compactness theory of minimal laminations, reconciling the two competing approaches of Thurston \cite{thurston1979geometry} and Colding--Minicozzi \cite{ColdingMinicozziIV}.
We will then apply Theorem \ref{main lma} and the compactness theory to prove the following conjecture of Daskalopoulos--Uhlenbeck \cite[Problem 9.4, Conjecture 9.5]{daskalopoulos2020transverse}:

\begin{theorem}\label{main thm}
Let $2 \leq d \leq 4$ and suppose that $M$ has constant sectional curvature.
\begin{enumerate}
\item Let $u$ be a function of least gradient.
Then $\bigcup_{y \in \RR} \partial \{u > y\}$ is the support of a minimal lamination $\lambda$ whose leaves are the connected components of the hypersurfaces $\partial \{u > y\}$, and $\dif u$ is a Ruelle-Sullivan current for $\lambda$.
\item Conversely, if $\lambda$ is a minimal lamination and $\dif u$ is a Ruelle-Sullivan current for $\lambda$, then $u$ has least gradient.
\end{enumerate}
\end{theorem}

Here a \dfn{Ruelle-Sullivan current} for $\lambda$ is a $d-1$-current $T$ such that for any compactly supported $d-1$-form $\varphi$,
$$\int_M T \wedge \varphi := \int_K \int_{\{k\} \times \RR^{d - 1}} \varphi \dif \mu(k)$$
where $\lambda$ is $K \times \RR^{d - 1}$, thus $\{k\} \times \RR^{d - 1}$ is a leaf of $\lambda$, and $\mu$ is a Radon measure on the space of leaves $K$.
Thus $T = \dif u$ plays the r\^ole of the Noetherian flux for the $\infty$-Laplacian in the case $d = 2$.
However, it is not clear how, or even if, the $\infty$-Laplacian itself should enter the picture if $d \geq 3$.
We hope to return to this point in a later paper.

We believe that Theorem \ref{main thm} will be of interest to topologists, who may take the present paper as a black box, and thus only read \cite{BackusCML}.
Conversely, analysts will find no topology in this paper, besides the elementary theory of currents that we review in \S\ref{Prelims}.

Let us remark on one further application of Theorem \ref{main lma}.
G\'orny \cite[Theorem 1.2]{górny2017planar} uses the euclidean de Giorgi--Miranda theorem to prove a regularity theorem for functions of least gradient.
However, this proof does not elsewhere use the flatness of the domain, so we have:

\begin{theorem}[G\'orny's regularity theorem]
Let $2 \leq d \leq 7$ and suppose that $M$ is a simply connected manifold of constant sectional curvature.
Then any function $u: M \to \RR$ of least gradient can be written $u = u_j + u_c$ where $u_c$ is a continuous function of least gradient and $u_j$ is a jump function of least gradient.
\end{theorem}

This shows that $1$-harmonic functions are somewhat more regular than a general $BV$ function, as $z \mapsto \sqrt z$ on $\CC \setminus \RR_-$ does not admit such a decomposition \cite[Example 4.1]{Ambrosio2000FunctionsOB}.

%%%%%%%%%%%%%%

\subsection{Overview of the proof}
As in \cite{Miranda66, Giusti77}, we prove Theorem \ref{main lma} using the \dfn{de Giorgi lemma}, which controls the oscillation of the conormal $1$-form, or \dfn{excess}, to the reduced boundary to a set of least perimeter.
In this summary, we treat the case $M = \Hyp^d$ for simplicity.

In the euclidean case, the de Giorgi excess of a set $U$ of least perimeter in an open set $A$ is defined by
$$\Exc_A(U) := |\partial^* U \cap A| - \left|\int_A \partial_\mu 1_U \dif x^\mu \star 1\right|.$$
Here the first term is the surface measure of the reduced boundary of $U$ in $A$, and the integral in the second term is $\RR^d$-valued, using the identification $T_x'\RR^d \cong \RR^d$.
This identification is valid exactly because $\RR^d$ is flat; in particular, the excess is preserved by isometries of $\RR^d$, a key fact in the proof of the de Giorgi lemma.

In \S\ref{excess section} we resolve this conundrum.
We fix the coordinate frame $(\partial_\mu)$ obtained from the Poincar\'e ball model of hyperbolic geometry; then we push forward $(\partial_\mu)$ by isometries of $\Hyp^d$ to construct coordinate frames $(\partial_\mu^P)$ centered at each point $P \in \Hyp^d$.
We then define the excess
\begin{equation}\label{excess definition prelim}
\Exc_A(U, P) := |\partial^* U \cap A| - \left|\left[\int_A \partial_\mu^P 1_U \star 1\right] \dif x^\mu_P(P)\right|
\end{equation}
which is an element of $T_P' \Hyp^d$.
One can show that $\Exc_A(U, P)$ does not depend on the choice of isometries, but only on the basepoint $P$.

It remains to show that the excess respects translation along geodesics, possibly up to a perturbative term.
Following the strategy of \cite{daskalopoulosPrep1}, we embed $\Hyp^d$ in the Minkowski spacetime $\RR^{1, d}$ as the future unit hyperboloid.
One then obtains the following key estimate (Proposition \ref{translation invariance excess}): for $P, Q \in A$, $\rho := \diam A$,
\begin{equation}\label{almost translation invariance intro}
|\Exc_A(U, P) - \Exc_A(U, Q)| \lesssim \rho^{d + 1}.
\end{equation}
One can crudely predict this estimate as follows.
Scrutinizing (\ref{excess definition prelim}) we observe that, since we are integrating over the $d-1$-dimensional set $\partial^* U \cap A$, we must estimate the difference of integrands to be $O(\rho^2)$.
On the other hand, the scaling limit $\rho \to 0$ can also be viewed as the nonrelativistic limit, in which $\Hyp^d$ converges to the classical future unit slice $\{t = 1\}$.
So the non-translation-invariant tensor fields $\partial_\mu^P$, $\dif x^\mu_P$ converge to coordinate fields on flat space quadratically fast and we conclude the claim.

In \S\ref{MollifierSection} we recall Miranda's monotonicity formula \cite[Teorema 3.2]{Miranda66} for functions of least gradient (Proposition \ref{Monotone}).
It is used in three ways: to control the surface area of a minimal perimeter, to control the excess of a mollified minimal perimeter, and to show that a minimal perimeter has a tangent cone.

In \S\ref{Plateau section} we prove the de Giorgi lemma, Proposition \ref{de Giorgi}.
We seek to obtain the inductive bound
\begin{equation}\label{de Giorgi lemma intro}
\Exc_{B(P, r/2)}(U, P) \leq \frac{\Exc_{B(P, r)}(U, P)}{2^d} + O(r^{d + 1}),
\end{equation}
for a set $U$ of least perimeter such that $P \in \partial U$, as a standard argument shows that Theorem \ref{main lma} reduces to (\ref{de Giorgi lemma intro}).

We follow \cite[Chapters 6-7]{Giusti77} and begin by considering the case that $U$ is only an approximate minimizer of the area (\ref{least gradient functional}) and with $P$ merely close to $\partial U$, but such that $\partial U$ is $C^1$ and $|\nabla \normal_U|$ is small (Proposition \ref{Miranda44}).
By the approximate translation-invariance (\ref{almost translation invariance intro}), we may actually assume that $P \in \partial U$, and then a Taylor expansion of the metric in the coordinates $(x^\mu_P)$ reduces the problem to the euclidean case.
As in the euclidean case, one can use the monotonicity formula to show that mollification preserves the excess (Proposition \ref{main mollifier lemma}) which completes the proof.


%%%%%%%%%%%%%%%%%%%%%%%%%%%%%%%%%%%%%%%%%%%%%%%%

\subsection{Acknowledgements}
I would like to thank Georgios Daskalopoulos for suggesting this project and for many helpful discussions.
I would also like to thank Karen Uhlenbeck, Richard Schoen, and Christine Breiner for helpful comments.
This work was supported by an NSF Graduate Resarch Fellowship TODO.



%%%%%%%%%%%%%%%%%%%%%%%%%%%%%%%%%%%%%%%%%%%%%%%%%
\section{Preliminaries}\label{Prelims}
\subsection{Notation and conventions}
The operator $\star$ is the Hodge star, thus $\star 1$ is the Riemannian measure.
On a submanifold $\Sigma$ of codimension $\geq 1$, $\vol_\Sigma$ denotes the induced measure and $\star_\Sigma$ denotes the induced Hodge star. We also write $\star_\rho := \star_{\partial B(P, \rho)}$ if $P \in M$ is fixed.

When using the Einstein convention, Greek indices range over $0, 1, \dots$ while Latin indices range over $1, \dots$.
We write $y := x^0$.
Note carefully: we will never sum over indices on a Lorentzian manifold, but only on Riemannian manifolds.
Thus, if we have a timelike vector field $\partial_t$ we do not write $t = x^0$; $\partial_y = \partial_0$ is always \emph{spacelike}.
We use $\sharp, \flat$ for the musical isomorphisms: $(\varphi^\sharp)^\mu := g^{\mu \nu} \varphi_\nu$ and $X^\flat_\mu := g_{\mu \nu} X^\nu$.

We write $\Japan \xi := \sqrt{1 + |\xi|^2}$ for the Japanese norm of a vector $\xi$.

We consider the following manifolds: $\Ball^d$ is the unit ball in $\RR^d$, $\Sph^d$ the unit sphere in $\RR^{d + 1}$, $\Hyp^d$ is the hyperbolic space, and $\RR^{1, d}$ is the Minkowski spacetime.

%%%%%%%%%%%%%%%%%%%%%%%%%%%%%%%%%%%%%%%%%%%%%%%
\subsection{Functions of bounded variation}
An $\ell$-\dfn{current} on an open set $U$ is a continuous linear functional on the space of $C^\infty_\cpt$ differential $\ell$-forms on $U$.
We refer the reader to \cite{simon1983GMT} for a careful exposition of the theory of currents.
We write $\int_U \omega \wedge \psi$ for the pairing of an $\ell$-current $\omega$ with an $\ell$-form $\psi$ with compact support in $U$.
In particular, if $\varphi$ is an $d-\ell$-form, we identify it with its \dfn{Poincar\'e dual}, the $\ell$-current $\psi \mapsto \int_M \varphi \wedge \psi$.

We identify the derivative of a function $u$ with the $d-1$-current
$$\int_M \dif u \wedge \psi := -\int_M u \dif \psi,$$
which is well-defined as long as $u \in L^1_\loc(M)$.
For a vector field $X$, we write $\star (Xu) := \dif u \wedge \star (X^\flat)$.
A function $u$ has \dfn{bounded variation} if its total variation seminorm
\begin{equation}\label{total variation}
\int_M \star |\dif u| := \sup_{\substack{||\psi||_{C^0} \leq 1\\\supp \psi \Subset M}} \int_M \dif u \wedge \psi
\end{equation}
is finite. We write $BV(M)$ for the space of functions of bounded variation.
The local finiteness of $\int \star |\dif u|$, is diffeomorphism-invariant, and hence so is membership in $BV_\loc(M)$.

\begin{proposition}[trace theorem and Stokes formula]
Let $U \subseteq M$ be an open set with nonempty Lipschitz boundary, and $u \in BV(U)$.
Then the trace $v \in L^1(\partial U)$ is well-defined,
%and is characterized by the conditions that for $\vol_{\partial U}$-almost every $x$,
%\begin{equation}\label{convergence of trace}
%\int_{U \cap B(x, \varepsilon)} \star |v(x) - u| \ll \varepsilon^d,
%\end{equation}
and satisfies for every $d - 1$-form $\psi$,
\begin{equation}\label{Miranda IBP}
\int_U \dif u \wedge \psi + \int_U u \dif \psi = \int_{\partial U} v\psi.
\end{equation}
\end{proposition}
\begin{proof}
By a partition of unity argument we can reduce these results to \cite[Teorema 1]{Miranda67}.
\end{proof}

\begin{proposition}[polar decomposition]
For every $u \in BV_\loc(M)$ there exists a $\star |\dif u|$-measurable section $f$ of the cosphere bundle $S'M$ such that for every compactly supported $d-1$-form $\psi$,
\begin{equation}\label{RNy formula}
\int_M \dif u \wedge \psi = \int_M f|\dif u| \wedge \psi.
\end{equation}
\end{proposition}
\begin{proof}
This follows from \cite[Theorem 4.14]{simon1983GMT}.
\end{proof}

Let $f: M \to S'M$ be given by (\ref{RNy formula}).
As in \cite{Miranda66, Giusti77}, most of the technical work in this paper amounts to controlling the oscillation of $f$ at fine scales.
In order to make this precise, we shall need to take ``averages'' of $f$, but $f$ is a section of a curved vector bundle and so averaging is not well-defined.
We show that it is at least well-defined in the fine-scale limit, by proving a form of the Lebesgue differentiation theorem which is manifestly covariant.

To state our Lebesgue differentiation theorem, observe that if $\omega$ is a current with locally finite total variation $|\omega|$, then for any Riemannian metric, $\star|\omega|$ is a Radon measure, and the sheaves $L^p_\loc(\cdot, \star |\omega|)$, $p \in [1, \infty]$, are independent of the metric.
So we write $L^p_\loc(M, \omega)$ for such a sheaf.
We similarly refer to $\omega$-null sets and $\omega$-measurable sets and functions.

\begin{proposition}[Lebesgue differentiation theorem for a vector bundle]\label{LebesgueDiff}
Let $E \to M$ be a vector bundle over an oriented smooth manifold $M$, $\omega$ a current on $M$ with locally finite total variation $|\omega|$, and $f \in L^1_\loc(M, E, \omega)$.
Then there exists an $\omega$-null set $Z \subset M$ such that for every Riemannian metric on $M$, every trivialization $(F_1, \dots, F_\ell)$ of $E$ with dual trivialization $(F'_1, \dots, F'_\ell)$ of $E'$, and every $P \in M \setminus Z$,
$$f(P) = \lim_{r \to 0} \sum_{i=1}^\ell \left[\frac{\int_{B(P, r)} (F'_i, f) \star |\omega|}{\int_{B(P, r)} \star |\omega|}\right] F_i(P).$$
\end{proposition}

We shall apply this proposition with $E := T'M$, $F_\mu = \dif x^\mu$.
Note carefully that the terms inside the limit \emph{are} dependent on the metric and the choice of trivialization, thus the assertion is that the dependence goes away in the limit, and that the set on which the limit converges is independent.
Indeed, the idea is to scrutinize the proof of the Lebesgue differentiation theorem \cite[Chapter 3, Theorem 1.3]{stein2009real} and observe that the null sets constructed in the proof can be covered by null sets which do not depend on the Riemannian metric or the trivialization.

\begin{proof}
Choose a flat Riemannian metric, let $\dif \mu := \star |\omega|$, $\mathcal F = ((F_i), (F_i'))$ a pair of paralellizations of $E, E'$ such that $(F_i', F_j) = \delta_{ij}$, and $\ell$ the rank of $E$.
Then for every $\delta > 0$ there exists $\tilde f \in C_c(M, E)$ such that $||f - \tilde f||_{L^1(\mu)} < \delta$, thus
\begin{align*}
&\left|\sum_{i=1}^\ell \left[(F_i'(x), f(x)) - \dashint_{B(x, r)} (F_i', f) \dif \mu\right] F_i(x)\right| \\
&\qquad \leq \left|\sum_{i=1}^\ell (F_i'(x), f(x) - \tilde f(x)) F_i(x)\right| + \dashint_{B(x, r)} \left|\sum_{i=1}^\ell (F_i', f - \tilde f)F_i(x) \dif \mu \right| \\
&\qquad \qquad + \left|\sum_{i=1}^\ell \left[(F_i'(x), \tilde f(x)) - \dashint_{B(x, r)} (F_i, \tilde f) \dif \mu\right] F_i(x)\right| \\
&\qquad =: I_1(x) + I_{2, r}(x) + I_{3, r}(x).
\end{align*}
Here the integral defining $I_{2, r}(x)$ is valued in the fiber $E_x$.

By the proof of the Lebesgue differentiation theorem, $\{I_1 > \varepsilon\} \subseteq \{|f - \tilde f| > \varepsilon\}$ and $\{I_{2, r} > \varepsilon\} \subseteq \{\dashint_{B(x, r)} |f - \tilde f|\dif \mu\}$, which are $\mathcal F$-independent sets of measure $\lesssim \delta/\varepsilon$.
Meanwhile $I_{3, r} \to 0$ pointwise as $r \to 0$, so for
\begin{equation}\label{definition of null set}
Z_{\varepsilon, \mathcal F} := \left\{x \in M: \limsup_{r \to 0} \left|\sum_{i=1}^\ell \left[(F_i'(x), f(x)) - \dashint_{B(x, r)} (F_i', f) \dif \mu\right] F_i(x)\right| > 2\varepsilon\right\},
\end{equation}
one has
$$Z_{\varepsilon, \mathcal F} \subseteq \bigcap_{r > 0}\bigcup_{s <r} \{I_{1, s} > \varepsilon\} \cup \{I_{2, s} > \varepsilon\}.$$
The right-hand side is independent of $\mathcal F$ and $\delta$, but has $\mu$-measure $\lesssim \delta/\varepsilon$, so it is $\omega$-null.
Thus the union taken over all possible $\mathcal F$ and $\varepsilon$ is also $\omega$-null.

Now let $g$ be a Riemannian metric and $h$ our flat reference metric.
Then $\varphi := \sqrt{\det g/\det h}$ satisfies $\star_g|\omega| = \varphi \dif \mu$, and as $\varphi$ is continuous it does not contribute in the limit superior in (\ref{definition of null set}).
Moreover, the balls $B_g(x, r)$ have bounded eccentricity with respect to $h$, so we can replace $B(x, r)$ with $B_g(x, r)$ in (\ref{definition of null set}) without affecting $Z_{\varepsilon, \mathcal F}$ \cite[Chapter 3, Corollary 1.7]{stein2009real}.
\end{proof}

\begin{corollary}
The section $f: M \to S'M$ in the polar decomposition (\ref{RNy formula}) satisfies
\begin{equation}\label{Lebesgue point definition}
    f(P) = \left[\lim_{r \to 0} \frac{\int_{B(x, r)} \star \partial_\mu u}{\int_{B(x, r)} \star |\dif u|}\right] ~\dif x^\mu(P)
\end{equation}
for any coordinate system $(x^\mu)$ and any Riemannian metric $g$, and $\star|\dif u|$-almost every $P$.
The exceptional set does not depend on $(x^\mu)$ or $g$.
\end{corollary}

It follows from the above corollary that the following definitions, which a priori refer to the metric or to a choice of coordinate system, are actually completely determined by the smooth structure on $M$.

\begin{definition}
Let $U \subseteq M$. We say that $U$ has \dfn{locally finite perimeter} if $1_U \in BV_\loc(M)$.
In that case we make the following definitions:
\begin{enumerate}
\item The \dfn{measure-theoretic boundary} $\partial U$ is the set of points whose Lebesgue density with respect to $M$ is $\in (0, 1)$.
\item The polar section of $1_U$ is called the \dfn{conormal $1$-form} $\normal_U$ to $\partial U$.
\item The set of points $P$ for which $\normal_U(P)$ exists is the \dfn{reduced boundary} $\partial^* U$.
\item The \dfn{perimeter} $|\partial^* U \cap E|$ in a Borel set $E$ is $\int_E \star |\dif u|$.
\end{enumerate}
\end{definition}

Our definition of reduced boundary and conormal $1$-form follows \cite[Definition 3.3]{Giusti77} and is due to \cite{deGiorgi55}.
See \cite[Chapter 6]{Pugh02} for the definition of Lebesgue density.
Choosing a coordinate system on $M$ in which the volume form is $\dif x^0 \wedge \cdots \wedge \dif x^{d - 1}$, we see from \cite[Chapters 1-4]{Giusti77} that the following properties of the reduced boundary hold:

\begin{proposition}\label{locality of Caccioppoli}
    Let $U$ be a set of locally finite perimeter.
    Then:
    \begin{enumerate}
    % \item $\partial^* U$ is either empty or $d-1$-dimensional in the Hausdorff sense, and is $d-1$-rectifiable.
    \item $\partial^* U$ is a dense subset of $\partial U$.
    \item If $\normal_U$ extends to a continuous $1$-form on $\partial U$, then $\partial^* U = \partial U$ is a $C^1$ embedded hypersurface.
    \item If $\partial^* U = \partial U$ is a $C^1$ hypersurface, then $\normal_U$ is the conormal $1$-form on $\partial U$ as defined in differential topology, and $\star |\dif 1_U|$ is the induced measure on $\partial U$.
\end{enumerate}
\end{proposition}

% As a first application of Proposition \ref{locality of Caccioppoli} we recover the following formulation of the coarea formula.

\begin{proposition}[coarea formula]\label{Coarea2}
Let $u \in BV_\loc(M)$ and $E$ an open set. Then
\begin{equation}\label{coarea formula}
\int_E \star |\dif u| = \int_{-\infty}^\infty |E \cap \partial^* \{u > y\}| \dif y.
\end{equation}
\end{proposition}
\begin{proof}
Reasoning identically to \cite[Theorem 1.23]{Giusti77}, we may assume that $u \in C^\infty(M)$.
If this is true and also $u$ has no critical points, then (\ref{coarea formula}) follows from Fubini's theorem, the fact that $|E \cap \partial \{u > y\}|$ is the surface area of $E \cap \{u = y\}$ (by Proposition \ref{locality of Caccioppoli}), and the change-of-variables formula.
However the left-hand side of (\ref{coarea formula}) is unaffected by critical points of $u$, and the right-hand side of (\ref{coarea formula}) is unaffected by critical values of $u$ by Sard's theorem, so (\ref{coarea formula}) holds even if $u \in C^\infty(M)$ has critical points.
\end{proof}

%%%%%%%%%%%%%%%%%%%%%%%
\subsection{Functions of least gradient}
We write
$$\eta(u, U) := \inf_{v \in BV_\cpt(U)} \int_U \star |\dif(u + v)|$$
for $u \in BV_\loc(M)$ and $U \subseteq M$ open with Lipschitz boundary, thus $u$ has least gradient iff $\eta(u, U) = \int_U \star |\dif u|$ for every $U$.
% The Dirichlet problem for functions of least gradient does not depend on whether one optimizes over compactly supported perturbations, or the more general trace-free perturbations \cite{Sternberg93}, so
% $$\eta(u, U) = \inf_{v|_{\partial U} = 0} \int_U \star |\dif(u + v)|.$$
If $u, v \in BV(U)$, then using the coarea formula and reasoning analogously to \cite[Lemma 5.6]{Giusti77}, we obtain the a priori estimates
\begin{align}
|\eta(u, U) - \eta(v, U)| &\leq ||u - v||_{L^1(\partial U)} \label{a priori estimate 1} \\
\eta(u, U) &\leq ||u||_{L^1(\partial U)} \leq |\partial U| \cdot ||u||_{L^\infty(M)}. \label{a priori estimate 2}
\end{align}

\begin{proposition}[Miranda stability theorem]\label{Miranda convergence}
If a sequence of functions $(u_n)$ (not necessarily of the same trace) satisfies for every open $U \Subset M$ with Lipschitz boundary
$$\limsup_{n \to \infty} \int_U \star |\dif u_n| \leq \liminf_{n \to \infty} \eta(u_n, U) < \infty,$$
and $u_n \to u$ in $L^1_\loc(M)$, then $u$ has least gradient, and $\dif u_n \to \dif u$ in the weak topology of measures.
\end{proposition}
\begin{proof}
The proof is similar to \cite[Teorema 3 and Osservazione 3]{Miranda67}, if we observe that we are allowed to add a term of size $o(1)$ to the right-hand side of the inequalities \cite[(2.8), (2.9), (2.13), and (2.14)]{Miranda67}.
The fact that the convergence in the weak topology of measures is equivalent to the convergence in the sense of \cite[Osservazione 3]{Miranda67} follows from the portmanteau theorem \cite[Theorem 13.16]{klenke2013probability}.
\end{proof}

%%%%%%%%%%%%%%%%%%%%%%%%%%%%%%%%%%%%%%%%%%%%%%
\section{Averages of differential forms}\label{excess section}
\subsection{Construction of averages}
The classical de Giorgi lemma is concerned with the rate of convergence in the Lebesgue differentiation theorem of the conormal $1$-form $\normal_U$ to a set $U$ of least perimeter.
As with most tools of harmonic analysis, the Lebesgue differentiation theorem breaks diffeomorphism symmetry.
This is true even when stated carefully as in Proposition \ref{LebesgueDiff}, which asserts that the limiting behavior is diffeomorphism-invariant, but \emph{not} that the rate of convergence of the averages is.

If the Levi-Civita connection $\nabla$ is flat, then $\nabla$ induces canonical isomorphisms $T_P'M \to T_Q'M$ for each $Q$ close enough to $P$ that we can ignore the effects of monodromy.
Hence $\nabla$ identifies $1$-forms $\xi$ defined near $P$ with $T_P'M$-valued functions $\tilde \xi$.
One can then define the average of $\xi$ with respect to a measure $\omega$,
\begin{equation}\label{averages and flat connections}
\avg_U \xi := \dashint_U \tilde \xi \dif \omega
\end{equation}
where the right-hand side is a vector-valued integral and $U \ni P$, and so $\avg_U \xi \in T_P'M$ (but could also be viewed as a $1$-form by parallel translation).
In particular, the choice of $P$ does not matter; in other words this notion of averaging respects translation and rotation symmetries.
This fails, however, if $\nabla$ has curvature, even if it arises from a metric $g$ with translation and rotation symmetries, since the holonomy group of $\nabla$ obstructs the naturality of the isomorphisms $T_P'M \to T_Q'M$.

Now suppose that $\nabla$ is the Levi-Civita connection of a metric $g$ with constant sectional curvature, thus $g$ has translation and rotation symmetries.
Then $\nabla$ still has holonomy, but we can define averages as follows.

Since $g$ has constant sectional curvature $K \in \RR$, then we can cover $M$ by charts $(x^\mu)$ in which $g$ takes the form
\begin{equation}\label{constant sectional curvature metric}
g_{\mu\nu} = \frac{\delta_{\mu\nu}}{(1 + K|x|^2/4)^2}.
\end{equation}
Without loss of generality, $M$ is equal to such a chart.
We fix the origin $O$, where $x = 0$, and introduce a smooth family $(\Phi^P)_{P \in M}$ of oriented isometries $M \to M$, such that $\Phi^O$ is a rotation and $\Phi^P(O) = P$.
We write
$$\partial^P_\mu := \Phi^P_* \partial_{x^\mu}$$
and $x^\mu_P := x^\mu \circ \Phi^P$.
The choices of $(x^\mu)$ and $(\Phi^P)$ amount to selecting an oriented coordinate frame based at $P$ in which the metric takes the form (\ref{constant sectional curvature metric}).
The family of frames $(\partial^P_\mu)$ is uniquely determined up to a \dfn{gauge transformation} -- that is, a section $\chi$ of the bundle $\SpOrth(TM) \to M$.

\begin{definition}
The \dfn{average} of a $1$-form $\xi$ over an open set $U$ based at $P$ with respect to a measure $\omega$ to be
$$\avg_{U, P, \omega} \xi := \left[\dashint_U (\xi, \partial^P_\mu) \dif \omega\right] \dif x^\mu(P).$$
\end{definition}

We sometimes suppress the subscripts $U, P, \omega$ if they are clear or irrelevant.
By definition, $\avg_{U, P} \xi$ is again an element of $T_P'M$, and one can check that if $K = 0$ then it equals (\ref{averages and flat connections}).
Proposition \ref{LebesgueDiff} implies that $\avg_{B(P, r), P} \xi$ converges to $\xi(P)$ as $r \to 0$ unless $P$ is an element of an $\omega$-null set.
It is also gauge invariant (or equivalently rotation invariant): if we obtain $\widetilde{\avg_{U, P}} \xi$ from a coordinate system obtained by rotating $\Phi^P$ by $\chi(P) \in \SpOrth(T_PM)$, then
$$\widetilde{\avg_{U, P, \omega}} \xi = \left[\dashint_{B(P, r)} \chi^\nu_\mu \xi_\nu \dif \omega\right] (\chi(P)^{-1})_\lambda^\mu \dif x^\lambda_P(P) = \avg_{U, P} \xi.$$
The average also satisfies the following sort of weak translation-invariance:

\begin{proposition}\label{translation invariance}
The average in an open set $A$ satisfies
$$||\avg_{A, P} \xi| - |\avg_{A, Q} \xi|| \lesssim (\diam A)^2 ||\xi||_{C^0}.$$
\end{proposition}

%%%%%%%%%%%%%%%%%%
\subsection{Proof of weak translation-invariance}
In the proof of Proposition \ref{translation invariance}, we will fill in the details in the case $K < 0$, as the case $K = 0$ is trivial and the case $K > 0$ is similar and strictly easier; we shall briefly remark on that later.

We first rapidly review the facts about the hyperboloid model that we will need.
For a more thorough discussion which uses many of the same ideas as the proof of Proposition \ref{translation invariance}, see \cite[\S3.1, \S4.1]{daskalopoulosPrep1}.
Let $\RR^{1, d} = \RR_t \times \RR_y \times \RR_x^{d - 1}$ be the Minkowski spacetime with its metric $-\dif t^2 + \dif y^2 + |\dif x|^2$.
Expressing $\Hyp^d$ in the Poincar\'e ball model, we obtain an embedding
\begin{align*}
\Psi: \Hyp^d &\to \RR^{1, d} \\
x &\mapsto \frac{1}{1 - |x|^2/4 - y^2/4} \begin{bmatrix}1 + |x|^2/4 + y^2/4\\y \\ x\end{bmatrix}
\end{align*}
whose image is the unit future hyperboloid, as in the proof of \cite[Proposition 3.5]{lee1997riemannian}, and $\Psi(O) = (1, 0, 0)$.
This embedding induces a split exact sequence of $\SpOrth^+(\RR^{1, d})$-bundles over $\Hyp^d$
\begin{equation}\label{splitting of tangent bundle}
0 \to T\Hyp^d \to \Hyp^d \times \RR^{1, d} \to N\Hyp^d \to 0
\end{equation}
where $N\Hyp^d$ is the normal bundle of the embedding $\Psi$ \cite[(3.4)]{daskalopoulosPrep1}.
Here $\SpOrth^+(\RR^{1, d})$ is the properly orthochronous Lorentz group.

\begin{lemma}
The splitting (\ref{splitting of tangent bundle}) induces canonical isomorphisms of $\SpOrth^+(\RR^{1, d})$-representations
\begin{equation}\label{SES}
\RR^{1, d} = N'_P \Hyp^d \oplus T'_P \Hyp^d.
\end{equation}
for each $P \in \Hyp^d$.
Writing $\xi_O$ for the orthogonal projection of $\xi \in T'_P \Hyp^d \subseteq \RR^{1, d}$ to $T'_O \Hyp^d$,
\begin{equation}\label{Dask estimate}
|\xi| \leq |\xi_O| \leq e^{\dist(O, P)^2/2} |\xi|.
\end{equation}
\end{lemma}
\begin{proof}
We consider the adjoint sequence
$$0 \to N' \Hyp^d \to \Hyp^d \times \RR^{1, d} \to T' \Hyp^d \to 0$$
to (\ref{splitting of tangent bundle}).
Since (\ref{splitting of tangent bundle}) is split exact, we have a direct sum of $\SpOrth^+(\RR^{1, d})$-bundles $\Hyp^d \times \RR^{1, d} = T' \Hyp^d \oplus N' \Hyp^d$.
Here $\Hyp^d \times \RR^{1, d}$ is equipped with its trivial connection $\dif$ and has trivial monodromy group, so the parallel transport maps induce canonical isomorphisms between each fiber of $\Hyp^d \times \RR^{1, d}$ and $\RR^{1, d}$.
Therefore we have canonical isomorphisms (\ref{SES}).

Now (\ref{Dask estimate}) easily follows from \cite[\S4.1]{daskalopoulosPrep1}:\footnote{One may object that \cite[\S4.1]{daskalopoulosPrep1} refers to the splitting of the tangent bundle (\ref{splitting of tangent bundle}) rather than the cotangent bundle, but after conjugating by a musical isomorphism we see that the same estimates hold for the splitting of the cotangent bundle.}
By (\cite[(4.1)]{daskalopoulosPrep1}), $\xi_O$ is just $\xi$ with its timelike part $\xi_t$ set to $0$, thus
$$|\xi_O|^2 = |\xi_x|^2 + |\xi_y|^2 \geq |\xi_x|^2 + |\xi_y|^2 - |\xi_t|^2 = |\xi|^2.$$
On the other hand, by \cite[Lemma 4.2, Lemma 4.1]{daskalopoulosPrep1},
\begin{align*}
|\xi_O| &\leq (1 + |P - O|^2/2) |\xi| \leq (1 + \rho^2 \cosh \rho/2) |\xi| \leq e^{\rho^2/2} |\xi|. \qedhere
\end{align*}
\end{proof}

\begin{lemma}
The pushforward morphism $\Psi_*: T_{(x, y)} \Hyp^d \to T_{\Psi(x, y)} \RR^{1, d}$ is
\begin{equation}\label{pushforward estimates}
\Psi_* = \begin{bmatrix}y & x \\ 1 \\ & \id \end{bmatrix} + O(|x|^2 + y^2).
\end{equation}
\end{lemma}
\begin{proof}
We compute for $f(y, x) := |x|^2/4 + y^2/4$ that
\begin{align*}
\Psi_* &= \frac{1}{1 - f} \begin{bmatrix} \partial_y f & \partial_x f \\ 1 \\ & \id \end{bmatrix} + \frac{1}{(1 - f)^2} \begin{bmatrix}(1 + f) \partial_y f & (1 + f) \partial_x f \\
y \partial_y f & y \partial_x f \\
(\partial_y f) x & x \otimes \partial_x f
\end{bmatrix} \\
&= \begin{bmatrix}y & x \\ 1 \\ & \id \end{bmatrix} + O(|x|^2 + y^2)
\end{align*}
since $f(y, x) = O(|x|^2 + y^2)$ and $2\dif f(y, x) = (y, x)$.
\end{proof}

\begin{lemma}
Let $P \in \Hyp^d$ be given by $x^i = 0$, $y = \rho$. Consider the Lorentz boost
$$\Lambda := \begin{bmatrix}\cosh \rho & \sinh \rho \\ \sinh \rho & \cosh \rho \\ &&\id\end{bmatrix}$$
Up to a gauge transformation, it holds that for each $X \in \Hyp^d$, $Y := (\Phi^P)^{-1}(X)$, the below diagram commutes:
\begin{equation}\label{Lorentz boost diagram}
\begin{tikzcd}[column sep=50pt, row sep=30pt]
\RR^{1, d} \arrow[r, "\Lambda"] & \RR^{1, d} \\
T_Y \Hyp^d \arrow[u, "\dif \Psi(Y)"] \arrow[r, "\dif \Phi^P(Y)"] & T_X \Hyp^d \arrow[u, "\dif \Psi(X)"]
\end{tikzcd}
\end{equation}
\end{lemma}
\begin{proof}
Up to a gauge transformation, we may assume that $\Phi^P$ acts by hyperbolic translation along the $y$-axis.
On the other hand, $\Lambda \in \SpOrth^+(\RR^{1, d})$, so it preserves the unit future hyperboloid $\Psi(\Hyp^d)$ and acts by isometry.
Since $\Lambda$ clearly also preserves $\{(t, y, 0) \in \RR^{1, d}\}$, $\Lambda|\Psi(\Hyp^d)$ must act by hyperbolic translation on the image of the $y$-axis and hence the diagram
$$
\begin{tikzcd}
\RR^{1, d} \arrow[r, "\Lambda"] & \RR^{1, d} \\
\Hyp^d \arrow[u, "\Psi"] \arrow[r, "\Phi^P"] & \Hyp^d \arrow[u, "\Psi"]
\end{tikzcd}
$$
commutes. Linearizing this diagram at $Y$, we obtain (\ref{Lorentz boost diagram}).
\end{proof}

\begin{lemma}\label{DoVF lemma}
Let $A \subseteq \Hyp^d$ be an open set containing $O, P$.
Then up to a gauge transformation,
\begin{equation}\label{difference of vector fields}
||\partial^P_\mu - \partial_\mu||_{C^0(A)} \lesssim (\diam A)^2.
\end{equation}
\end{lemma}
\begin{proof}
We may assume by applying an isometry that $x^i(P) = 0$.
Let $X \in A$; we prove $|\partial^P_\mu(X) - \partial_\mu(X)| \lesssim \varepsilon^2$ for $\varepsilon := \diam A$.
Let $Y = (\Phi^P)^{-1}(X)$, so that $\partial^P_\mu(X) = \dif \Phi^P(Y) \partial_\mu(Y)$.
For $Z \in \Hyp^d$, we assign $T_Z \Hyp^d$ the basis $(\partial_\mu(Z))$.
Then the operator $I: T_Y \Hyp^d \to T_X \Hyp^d$ given by $I \partial_\mu(Y) = \partial_\mu(X)$ is represented by the identity matrix, and
$$|\partial^P_\mu(X) - \partial_\mu(X)| \leq |\dif \Phi^P(Y) - I| = |\dif \Psi(X) \circ \dif \Phi^P(Y) - \dif \Psi(X) \circ I|.$$
Up to a gauge transformation, (\ref{Lorentz boost diagram}) commutes, so
$$|\partial^P_\mu(X) - \partial_\mu(X)| \leq |\Lambda - \dif \Psi(Y) - \dif \Psi(X) \circ I|.$$
If $Y = (x^*, y^*)$, then $X = (x^*, y^* + \rho) + O(\varepsilon^2)$ \cite[(4.5.5)]{ratcliffe2006foundations} since $X \in A$, so
\begin{align*}
\Lambda \circ \dif \Psi(Y) &= \begin{bmatrix}1 & \rho \\ \rho & 1 \\ && \id \end{bmatrix} \begin{bmatrix}y^* & x^* \\ 1 \\ & \id \end{bmatrix} + O(\varepsilon^2) = \begin{bmatrix}y^* + \rho & x^* \\ 1 \\ & \id\end{bmatrix} \begin{bmatrix}1 \\ & \id\end{bmatrix} + O(\varepsilon^2) \\
&= \dif \Psi(X) \circ I + O(\varepsilon^2). \qedhere
\end{align*}
\end{proof}

\begin{proof}[Proof of Proposition \ref{translation invariance} for $K < 0$]
We may assume by rescaling that $K = -1$, by locality that $M = \Hyp^d$, by applying an isometry that $Q = O$, and by applying a gauge transformation that (\ref{difference of vector fields}) holds.
Writing
$$v := \avg_{A, P} \xi, \quad w := \avg_{A, Q} \xi$$
and $\varepsilon := \diam A$, we seek to show $||v| - |w|| \lesssim \varepsilon^2$
Using the splitting of (\ref{SES}) we can view both covectors $v \in T_P' \Hyp^d, w \in T_O' \Hyp^d$ as elements of $\RR^{1, d}$.

We first estimate
$$||v| - |v_O|| \leq \varepsilon^2 |v| \leq \varepsilon^2 ||\xi||_{C^0}$$
using (\ref{Dask estimate}) and the fact that $e^{\varepsilon^2/2} - 1 \leq \varepsilon^2$ for $\varepsilon < 1$.
Moreover, the reverse triangle inequality $||v_O| - |w|| \leq |v_O - w|$ is valid because $v_O, w$ are elements of the spacelike subspace $T'_O \Hyp^d$ of $\RR^{1, d}$, thus
$$||v| - |w|| \leq ||v| - |v_O|| + ||v_O| - |w|| \leq \varepsilon^2 ||\xi||_{C^0} + |v_O - w|.$$
Moreover,
\begin{align*}
|v_O - w| &= \left|\left[\dashint_A (\xi, \partial^P_\mu) \dif \omega\right] (\dif x_P^\mu(P))_O - \left[\dashint_A (\xi, \partial^O_\mu) \dif \omega\right] \dif x^\mu(O)\right| \\
&\leq \left[\dashint_A |\partial^P_\mu - \partial_\mu| \dif \omega\right] \cdot |\dif x^\mu(O)| + \left|\dashint_A (\xi, \partial_\mu^P) \dif \omega\right| \cdot |(\dif x^\mu_P(P))_O - \dif x^\mu(O)|\\
&=: I + J
\end{align*}
It is clear that $I \leq \sum_\mu ||\partial^P_\mu - \partial_\mu||_{C^0(A)} \cdot ||\xi||_{C^0}$, which is $\lesssim \varepsilon^2 ||\xi||_{C^0}$ by (\ref{difference of vector fields}).
By the triangle inequality,
$$J \leq \sum_\mu \left[|(\dif x^\mu_P(P) - \dif x^\mu(P))_O| + |\dif x^\mu(P)_O - \dif x^\mu(O)|\right] ||\xi||_{C^0}.$$
From (\ref{Dask estimate}), the fact that $e^{\varepsilon^2/2} \leq 2$, a musical isomorphism, and (\ref{difference of vector fields}), we dispose of the first term as
$$|(\dif x^\mu_P(P) - \dif x^\mu(P))_O| \leq 2 |\dif x^\mu_P(P) - \dif x^\mu(P)| \leq 2 ||\partial^P_\mu - \partial_\mu||_{C^0(A)} \lesssim \varepsilon^2.$$
Finally, we recall that $\dif x^\mu(P)_O$ is exactly the spacelike part of $\dif \Psi^\mu(P)$.
So as an element of $\RR^{1, d}$, $\dif x^\mu(P)_O$ is the $\mu$th column of the matrix in (\ref{pushforward estimates}) with the first (that is, timelike) row set to $0$, plus $O(\varepsilon^2)$.
Therefore $\dif x^\mu(P)_O = \dif x^\mu(O) + O(\varepsilon^2)$.
\end{proof}

Now we sketch the case $K > 0$.
We identify $\Sph^d \setminus \{\infty\}$ with $\RR^d$ via stereographic projection, giving an embedding $\Psi: \RR^d \to \RR^{d + 1}$ as the unit sphere (minus its south pole).
It is easy to show that $\RR^{d + 1} = T'_P \Sph^d \oplus N'_P \Sph^d$ and $||\xi| - |\xi_O|| \lesssim \dist(O, P)^2 |\xi|$ whenever $\xi \in T_P \Sph^d$.
Here $O$ and $\infty$ are mutual antipodes, and there are no technicalities caused by an indefinite metric.
The pushforward formula (\ref{pushforward estimates}) still holds, and (\ref{Lorentz boost diagram}) holds with the Lorentz boost replaced by a rotation matrix.
We still have
$$\Phi^P(x^*, y^*) = (x^*, y^* + \rho) + O(|x^*|^2 + |y^*|^2),$$
since this just expresses the fact that translation in stereographic projection and spherical translation agree to second order.
Therefore (\ref{difference of vector fields}) holds and the rest of the proof is essentially identical.

%%%%%%%%%%%%%%%%%%%%%%%%%%%%%%%%%%%%%%%%%%%%%%%

\subsection{The excess}
We now introduce the quantity which governs the rate of convergence of the Lebesgue differentiation theorem for $\normal_U$, whenever $U$ is a set of locally finite perimeter.
More precisely, we study the convergence of the approximation
$$\normal_U(P, r) := \avg_{B(P, r), P, \star |\dif 1_U|} \normal_U.$$
Here $\star |\dif 1_U|$ can be viewed as the surface measure of $\partial^* U$, so $\normal_U(\cdot, r)$ is an averaged version of $\normal_U$.
It could be written more explicitly as
$$\normal_U(P, r) = \left[\int_{B(P, r)} \partial_\mu^P 1_U \star 1\right] \dif x_P^\mu(P).$$

\begin{lemma}\label{gauge invariance of the normal}
Let $U$ be a set of locally finite perimeter. Then
$$|\normal_U(P, r)| \leq e^{C|K|r^2}.$$
\end{lemma}
\begin{proof}
We compute for $M := |\partial^* U \cap B(P, r)|$ and $V := (\Phi^P)^{-1}(U)$ that
\begin{align*}
|\normal_U(P, r)|^2 &= |\normal_V(O, r)|^2 = M^{-2} \left|\sum_\mu \int_{B(O, r)} \partial_\mu 1_V \star 1\right|^2 \\
&\leq M^{-2} \max_\mu ||\partial_\mu||_{C^0(B(0, r))}^2 |\partial^* V \cap B(O, r)|^2 \\
&\leq \max_\mu ||g_{\mu\mu}||_{C^0(B(O, r))}.
\end{align*}
The claim now easily follows from the formula (\ref{constant sectional curvature metric}) for the metric.
\end{proof}

\begin{definition}
The \dfn{excess} of a set $U \subset M$ of locally finite perimeter at $P \in \partial U$ and in the open set $A \ni P$ with Lipschitz boundary is
$$\Exc_A(U, P) := |\partial^* U \cap A| - \left|\avg_{A, P, \star |\dif 1_U|} \normal_U\right|.$$
For $\rho > 0$ we write $\Exc_\rho(U, P) := \Exc_{B(P, \rho)}(U, P)$.
\end{definition}

We will control the rate of growth of the excess with the de Giorgi lemma, Proposition \ref{de Giorgi}.
For now, we just check that it has suitable symmetry properties:

\begin{lemma}
Let $U$ be a set of locally finite perimeter, let $A' \subseteq A$ be open sets with Lipschitz boundary, and let $P \in A'$. Then
\begin{equation}\label{approximate monotone}
-C |K| (\diam A')^2 |\partial^* U \cap A'| \leq \Exc_{A'}(U, P) \leq \Exc_A(U, P) + C |K|(\diam A)^2 |\partial^* U \cap A|.
\end{equation}
\end{lemma}
\begin{proof}
To deduce the lower bound on $\Exc_{A'}(U, P)$, we compute for $\rho := \diam A$ that
\begin{align*}
    \left|\int_{A'} \partial^P_\mu 1_U \star 1 \dif x_P^\mu(P)\right|
 & \leq \max_\mu ||\partial^P_\mu||_{C^0(A')} \cdot |\partial^* U \cap A'| \leq e^{CK\rho^2} |\partial^* U \cap A'| \\
 & \leq |\partial^* U \cap A'| + C|K|\rho^2 |\partial^* U \cap A'|.
\end{align*}
The proof of the upper bound is similar.
\end{proof}

\begin{proposition}\label{translation invariance excess}
The excess satisfies for $P, Q \in A$
$$|\Exc_A(U, P) - \Exc_A(U, Q)| \lesssim (\diam A)^2 |\partial^* U \cap A|.$$
\end{proposition}
\begin{proof}
Immediate from Proposition \ref{translation invariance}.
\end{proof}

%%%%%%%%%%%%%%%%%%%%%%%%%%%%%%%%%%%%%

\section{Monotonicity formula}\label{MollifierSection}
Let $u$ be a function of least gradient, at first on $\RR^d$.
Then $u$ satisfies a monotonicity formula \cite[Theorem 5.12]{Giusti77}, and the main idea of the proof is to bound the growth of $\int \star |\dif u|$ using the vector-valued integral of $\dif u$ -- that is,
\begin{equation}\label{integral of du}
I(u, P, r) := \avg_{B(P, r), P, \star |\dif u|} \dif u \cdot \int_{B(P, r)} \star |\dif u|.
\end{equation}
We now apply the above averaging techniques to extend the monotonicity formula to the manifold case.

\begin{proposition}[monotonicity formula]\label{Monotone}
Let $u$ be a function of least gradient on a manifold $M$ of constant sectional curvature $K$.
Then there exists $0 \leq A \lesssim |K|$ such that for $0 < r_1 < r_2 \ll 1$,
\begin{equation}\label{weak monotonicity}
\frac{\dif}{\dif r}\left[e^{Ar^2}r^{1 - d} \int_{B(P, r)} \star |\dif u|\right] \geq 0.
\end{equation}
and
\begin{align*}
&|r_2^{1 - d} I(u, P, r_2) - r_1^{1 - d} I(u, P, r_1)|^2 \\
&\qquad \lesssim \left(1 + (d - 1) \log \frac{r_2}{r_1}\right) \left(r_2^{1 - d}\int_{B(P, r_2)} \star |\dif u| \right)
\left(\int_{r_1}^{r_2} \partial_r \left[e^{Ar^2} r^{1 - d} \int_{B(P, r)} \star |\dif u|\right] \dif r\right)\\
&\qquad \qquad + |K|^2 r_2^{6-2d} \left(\int_{B(P, r_2)} \star |\dif u|\right)^2.
\end{align*}
\end{proposition}

%%%%%%%%%%%%%%%%%%%%%
\subsection{Proof of the monotonicity formula}
Throughout the proof of Proposition \ref{Monotone} we write $\dif \sigma$ for the usual volume form on $\Sph^{d - 1}$.
We begin with an estimate for a smoothed out version of functions of least gradient.
Its euclidean case can be isolated from the proof of \cite[Lemma 5.8]{Giusti77}.

\begin{lemma}\label{monotonicity lemma}
There exists $A$ such that for every $u \in C^1(B_R)$, $0 < r_1 < r_2 < R$, if we let
$$E(r) = \int_{B_r} \star |\dif u| - \eta(u, r),$$
so that $E(R) = 0$ iff $u$ has least gradient, then there exists $A \geq 0$ such that for $R > 0$ small,
\begin{equation}\label{monotonicity lemma eqn}
0 \leq \int_{B_{r_2} \setminus B_{r_1}} \star r^{1 - d}\frac{(\partial_ru)^2}{|\dif u|} \leq 2\int_{r_1}^{r_2} \partial_r \left[e^{Ar^2} r^{1-d}\int_{B_r} \star |\dif u|\right] + \frac{O(E(r))}{r^d} \dif r.
\end{equation}
\end{lemma}
\begin{proof}
This result is coordinate-invariant, so we may use whichever coordinates are convenient: we in fact use normal polar coordinates $(r, \theta)$.
We fix $s \in [r_1, r_2]$ and introduce a competitor $v(r, \theta) = u(s, \theta)$.
From the definition of $\eta$,
\begin{equation}\label{consequence of least gradient monotone}
    \eta(u, s) \leq \int_U \star |\dif v| = \int_0^s \int_{\partial B_r} \star_r |\dif v| \dif r.
\end{equation}
We now recall that
$$\vol_\rho(\theta) = \left[\rho^{d - 1} - \frac{\rho^d}{3} \Ric_P(\theta, \theta) + O(\rho^{d + 1})\right] \dif \sigma(\theta)$$
where the implied constant depends on the curvature of $M$.
Thus we can find $A > 0$ such that for all $\rho$ small enough that $e^{A\rho^2} \sqrt{\det g|_{\partial B_\rho}}$ is monotone in $\rho$ for some $A > 0$, as long as $\rho$ is small enough.
Applying $\partial_r v = 0$ it follows that
\begin{equation}\label{introduce the ricci tensor}
\int_{\partial B_r} \star_r |\dif v| \leq e^{As^2} \frac{\tilde r^{d - 1}}{s^{d - 1}} \int_{\partial B_s} \star_s |\dif v|.
\end{equation}
Applying (\ref{consequence of least gradient monotone}) and Fubini's theorem,
\begin{align*}
\eta(u, s) &\leq e^{As^2} \int_0^s \frac{r^{d - 1}}{s^{d - 1}} \dif r \cdot \int_{\partial B_s} \star_s |\dif v| = \frac{s e^{As^2}}{d} \int_{\partial B_s} \star_s |\dif v|\\
&\leq \frac{s e^{As^2}}{d - 1} \int_{\partial B_s} \star_s |\dif v|.
\end{align*}
By Gauss' lemma, $\dif v$ is the orthogonal projection of $\dif u$ onto $T' \partial B_s$, and its orthocomplement is $\partial_r u$. Therefore by Taylor's theorem,
$$\int_{\partial B_s} \star_s |\dif v| \leq \int_{\partial B_s} \star_s |\dif u| \sqrt{1 - \frac{(\partial_r u)^2}{|\dif u|^2}} \leq \int_{\partial B_s} \star_s \left[|\dif u| - \frac{(\partial_r u)^2}{2 |\dif u|}\right]$$
or in other words
\begin{align*}
\int_{\partial B_s} \star_s \frac{(\partial_r u)^2}{2|\dif u|} &\leq \int_{\partial B_s} \star_s |\dif u| - \frac{d - 1}{s} e^{-As^2} \eta(u, s)\\
&\leq \int_{\partial B_s} \star_s |\dif u| - \frac{d - 1}{s} e^{-As^2} \int_{B_s} \star |\dif u| - O(s^{-1}E(s)).
\end{align*}
We moreover have for $\tilde A \geq 0$ that
$$e^{-\tilde As^2} \partial_s \left[e^{\tilde As^2} s^{1 - d} \int_{B_s} \star |\dif u|\right] = \left[2\tilde As^{2 - d} - \frac{d - 1}{s^d}\right]\int_{B_s} \star |\dif u| + s^{1 - d} \int_{\partial B_s} \star_s |\dif u|$$
so if we choose $\tilde A$ so that
$$-\frac{d - 1}{s} e^{-As^2} = 2\tilde As - \frac{d - 1}{s}$$
then
$$s^{1 - d} \int_{\partial B_s} \star_s |\dif u| - (d - 1)\frac{e^{-\tilde As^2}}{s^d} \int_{B_s} \star|\dif u| \leq e^{-\tilde As^2} \partial_s\left(e^{\tilde As^2} s^{1 - d} \int_{B_s} \star|\dif u|\right).$$
We moreover have $e^{-\tilde As^2} \leq 1$, so we can now integrate with respect to $\dif s$ and rename $\tilde A$ to $A$ to conclude.
\end{proof}

\begin{proof}[Proof of Proposition \ref{Monotone}]
Let
$$I_r := r^{1 - d} I(u, P, r) = r^{1 - d} \left[\int_{B(P, r)} \partial_\mu^P u \star 1\right] \dif x^\mu_P(P).$$
Applying the fundamental theorem of calculus and the fact that $\sqrt{\det g} = e^{-O(Kr^2)}$,
\begin{align*}
I_r &= r^{1 - d} \left[\int_{\partial B(P, r)} u \cdot (\normal, \partial_\mu^P) \star 1\right] \dif x^\mu_P(P) \\
&= \left[\int_{\Sph^{d - 1}} u(r, \theta) \dif \sigma(\theta)\right] \dif x^\mu_P(P) + O(|K|r^{3 - d}) \int_{B(P, r)} \star |\dif u|.
\end{align*}
and hence
\begin{equation}\label{monotone dump the metric}
|I_{r_2} - I_{r_1}| \leq \int_{\Sph^{d - 1}} |u(r_2, \theta) - u(r_1, \theta)| \dif \sigma(\theta) + O(|K|r^2) \int_{B(P, r_2)} \star |\dif u|.
\end{equation}
Henceforth we write $B_r := B(P, r)$.
The metric $g$ plays no role in the dominant term of (\ref{monotone dump the metric}), so we may use \cite[Lemma 5.3]{Giusti77} to bound
$$0 \leq \int_{\Sph^{d - 1}} |u(r_2, \theta) - u(r_1, \theta)| \dif \sigma(\theta) \leq \int_{\Sph^{d - 1}} \int_{r_1}^{r_2} r^{1 - d}|\partial_r u(r, \theta)| \dif r \dif\sigma(\theta).$$
To reintroduce the metric we posit that $r_2$ is small enough that $\dif r \dif \sigma(\theta) \leq \star 2$.
We therefore have
\begin{equation}\label{monotone before cs}
\int_{\Sph^{d - 1}} \int_{r_1}^{r_2} r^{1 - d}|\partial_r u(r, \theta)| \dif r \dif\sigma(\theta) \leq 2 \int_{B_{r_2} \setminus B_{r_1}} \star r^{1 - d}|\partial_r u|
\end{equation}
and if we apply the Cauchy-Schwarz inequality and approximate $u$ by $C^1$ functions as on \cite[pg68]{Giusti77}, it follows from Lemma \ref{monotonicity lemma} that the right-hand side of (\ref{monotone before cs}) is
$$\lesssim \sqrt{\int_{B_{r_2} \setminus B_{r_1}} \star r^{1 - d} |\dif u|} \sqrt{\int_{r_1}^{r_2} \partial_r \left[e^{Ar^2} r^{1-d}\int_{B_r} \star |\dif u|\right] \dif r}.$$
The monotonicity (\ref{weak monotonicity}) follows at once.

Integrating by parts,
\begin{align*}
\int_{B_{r_2} \setminus B_{r_1}} \star r^{1 - d} |\dif u| &= \int_{r_1}^{r_2} r^{1 - d} \partial_r \int_{B_r} \star |\dif u| \dif r \\
&\leq r^{1 - d} \int_{B_r} \star |\dif u| + (d - 1) \int_{r_1}^{r_2} r^{-d} \int_{B_r} \star |\dif u| \dif r.
\end{align*}
Using (\ref{weak monotonicity}) we bound this second integral as
\begin{align*}
\int_{r_1}^{r_2} r^{-d} \int_{B_r} \star |\dif u| \dif r &\leq r^{1 - d} \log \frac{r_2}{r_1} \int_{B_{r_2}} \star |\dif u|.
\end{align*}
If we set
$$J_r := r^{1 - d} \int_{B_r} \star |\dif u|$$
then we can sum up our progress so far as
$$|I_{r_2} - I_{r_1}| \lesssim \sqrt{\left[1 + \log \frac{r_2}{r_1}\right] J_{r_2}} \sqrt{e^{Ar_2^2} J_{r_2} - e^{Ar_1^2} J_{r_1}} + |K|r_2^2 J_{r_2}.$$
The claim now follows by squaring both sides and applying Cauchy-Schwarz.
\end{proof}

%%%%%%%%%%%%%%%%%%%%%%%%%%%%%%%%%%
\subsection{Minimal tangent cones}
Since we are considering the regularity of minimal surfaces, we need to use the $d \leq 7$ hypothesis to show the regularity of tangent cones to minimal surfaces.

\begin{definition}
    For a function $u$ on $M$, $P \in M$ we define the \dfn{blowup} of $u$ at $P$ to be the net of functions $u_t: T_PM \to \RR$, given by
    $$u_t(v) = u\left(\exp_P(tv)\right).$$
\end{definition}

\begin{proposition}\label{blowup theorem}
Suppose that $U$ is an open set with least perimeter in $B(P, r)$, $P \in \partial^* U$, and $u = 1_U$.
Furhermore, suppose that $d \leq 7$.
Then the blowup $(u_t)$ of $u$ converges as $t \to 0$ along a subsequence (that we also denote $t \to 0$) in $L^1_\loc$ and almost everywhere, to the indicator function $v$ of a half-space $C \subset T_PM$ such that $0 \in \partial C$.
Moreover, $\dif u_t \to \dif v$ in the weak topology of measures.
\end{proposition}
\begin{proof}
If we only require that $\partial C$ is a minimal cone rather than a hyperplane, then this is a standard consequence of the Miranda stability theorem (Proposition \ref{Miranda convergence}) and the monotonicity formula (\ref{weak monotonicity}), see for example \cite[Theorem 9.3]{Giusti77} for the euclidean case.
However, a minimal cone of dimension $\leq 6$ is necessarily a hyperplane \cite[Theorem 9.10 and Theorem 10.10]{Giusti77}.
\end{proof}

We now generalize the surface area estimates of \cite[Remark 5.13]{Giusti77}.

\begin{corollary}\label{doubling dimension}
If $d \leq 7$ then there exists $A \geq 0$ such that for every set $U$ of least perimeter in a ball $B_r = B(P, r)$, with $P \in \partial^* U$, and $r > 0$ small,
$$|\Ball^{d - 1}|e^{-Ar^2}r^{d - 1} \leq |\partial^*U \cap B_r| \leq |\Sph^{d - 1}|e^{Ar^2} r^{d - 1}.$$
\end{corollary}
\begin{proof}
The upper bound on $|\partial^* U \cap B_r|$ is obtained by using (\ref{a priori estimate 2}) and the fact that the surface area of $\partial B_r$ is $|\Sph^{d - 1}|(1 + O(r^2))r^{d - 1}$.
This can be seen by integrating $\star_{\partial B_r} 1$ along $\partial B_r$ in normal coordinates and applying the Taylor expansion of the Riemannian measure.
The lower bound is obtained from the monotonicity formula, which implies that
$$\limsup_{\rho \to 0} e^{-A\rho^2} \rho^{1 - d} |\partial^* U \cap B_\rho| \leq |\partial^* U \cap B_r|.$$
To control the left-hand side we take a blowup $(u_\rho)$ of $1_U$.
By Proposition \ref{blowup theorem} we can pass to a subsequence so that $u_\rho \to 1_C$ for $C$ a half-space, which in particular is transverse to $B'_1$, where the prime denotes the euclidean metric on the tangent space.
Then
\begin{align*}
\limsup_{\rho \to 0} e^{-A\rho^2} \rho^{1 - d} |\partial^* U \cap B_\rho| &= \lim_{\rho \to 0} e^{O(\rho^2)} \int_{B'_1} \star'|\dif u_\rho|' = \int_{B'_1} \star'|\dif 1_C|.
\end{align*}
This last term is $|\partial C \cap B'_1|$, the measure of the intersection of the euclidean unit ball with a hyperplane through its origin.
In other words it is the measure $|\Ball^{d - 1}|$ of the unit ball in $\RR^{d - 1}$.
\end{proof}

We will frequently use Corollary \ref{doubling dimension} to bound error terms.
For example, if $u = 1_U$ where $U$ has least perimeter, then the error term in the monotonicity formula is of size $O(|K|r_2^{d + 1})$.



%%%%%%%%%%%%%%%%%%%%%%%%%%%%%%%%%%%%%%%%%%%%%%
\section{Regularity of sets of least perimeter}\label{Plateau section}
\subsection{Induction on scale}
Fix a manifold $M$ of constant sectional curvature and dimension $2 \leq d \leq 7$.
We prove our main theorem, Theorem \ref{main lma}, for $M$.
Following \cite{Miranda66,Giusti77,deGiorgi61}, we proceed by controlling the excess using the following de Giorgi lemma \cite[Theorem 8.1]{Giusti77}:

\begin{proposition}[de Giorgi lemma]\label{de Giorgi}
There exist $C, c, \rho_* > 0$, such that for every $P \in M$, $\rho$ such that $0 < \rho < \rho_*$, and set $U \subset M$ of least perimeter such that
\begin{equation}\label{base case}
\Exc_\rho(U, P) \leq c\rho^{d - 1},
\end{equation}
we have
\begin{equation}\label{dGL concl}
\Exc_{\rho/2}(U, P) \leq 2^{-d} \Exc_\rho(U, P) + C|K|\rho^{d + 1}.
\end{equation}
The constants $C, c, \rho_*$ only depend on $d$ for $|K| \lesssim 1$ (but may explode as $|K| \to \infty$).
\end{proposition}

In order to apply the de Giorgi lemma we recall its consequences: the base case \cite[pg109]{Giusti77} and the inductive case \cite[Corollary 8.3]{Giusti77}.
There are not new ideas in these corollaries, but we do have a new error term in (\ref{dGL concl}) that was not in \cite[Theorem 8.1]{Giusti77}, so for completeness we reprove them.

\begin{corollary}[base case]
Assume the de Giorgi lemma, and let $U \subset M$ have least perimeter.
Then there exists $\rho = \rho(P) < \rho_*$, which is locally uniform in $P \in \partial U$, such that (\ref{base case}) holds.
\end{corollary}
\begin{proof}
Let $Q \in \partial^* U$; we shall choose $\rho$ uniformly in a small neighborhood of $Q$, which is enough since $\partial^* U$ is dense in $\partial U$.
Since $\partial U$ has a tangent space at $Q$ by Proposition \ref{blowup theorem}, $\Exc_r(U, Q) \ll r^{d - 1}$, thus we can choose $r \in (0, \rho_*)$ such that $\Exc_r(U, Q) \leq c(r/2)^{d - 1}$.
For $P \in B(Q, r/4)$ we have $B(P, r/4) \subseteq B(Q, r/2)$ and hence by the de Giorgi lemma, for $\rho := r/4$, (\ref{base case}) holds.
\end{proof}

\begin{corollary}[inductive case]
Assume the de Giorgi lemma, and let $U \subset M$ have least perimeter.
Then $\normal_U$ extends to a continuous $1$-form on $\partial U$, where $\partial U$ is endowed with the subspace topology induced by $M$.
\end{corollary}
\begin{proof}
We use the de Giorgi lemma inductively as in \cite[Theorem 8.2]{Giusti77}.
Suppose that $r/2 < s < r = \rho/2^n$ for some $n$ and some $\rho$ satisfying (\ref{base case}).
To fix notation, let
$\xi := \normal_U(P, r)$, $\eta = \normal_U(P, s)$, $m := |\partial^* U \cap B(P, s)|$, $M := |\partial^* U \cap B(P, r)|$, and $\gamma_n := \Exc_{\rho/2^n}(U, P)$.
We first estimate
$$|\xi - \eta|^2 = |\xi|^2 + |\eta|^2 - 2 g^{-1}(\xi, \eta) \leq 2(1 - g^{-1}(\xi, \eta)) + C|K|r^2.$$
To estimate the right-hand side we write
$$m(1 - g^{-1}(\xi, \eta)) = \int_{B(P, s)} \star(|\dif 1_U| - g^{-1}(\xi, \dif x^\mu_P(P)) \partial^P_\mu 1_U).$$
Now we bound
$$ g^{-1}(\xi, \dif x^\mu_P(P)) \partial^P_\mu 1_U \leq |\xi| \cdot |\dif 1_U| \cdot \max_\mu |\partial^P_\mu| \leq e^{C|K|r^2} |\dif 1_U|,$$
which implies that, since $s \leq r$,
$$\int_{B(P, s)} \star(|\dif 1_U| - g^{-1}(\xi, \dif x^\mu_P(P)) \partial^P_\mu 1_U) \leq \int_{B(P, r)} \star(e^{C|K|r^2} |\dif 1_U| - g^{-1}(\xi, \dif x^\mu_P(P)) \partial^P_\mu 1_U).$$
By Corollary \ref{doubling dimension}, the first term integrates to $\leq M + C|K|r^{d + 1}$, and so by definition of $\xi$,
\begin{align*}
\int_{B(P, r)} \star(e^{CKr^2} |\dif 1_U| - g^{-1}(\xi, \dif x^\mu_P(P)) \partial^P_\mu 1_U) &\leq M(1 - |\xi|^2) + C|K|r^{d + 1}\\
&\leq 2M(1 - |\xi|) + C|K|r^{d + 1}.
\end{align*}
But $M(1 - |\xi|)$ is exactly the definition of $\Exc_r(U, P)$, so putting everything together and using Corollary \ref{doubling dimension} to bound $m \gtrsim s^{d - 1} \gtrsim r^{d - 1}$, we have the bound
\begin{equation}\label{just need the excess}
|\xi - \eta|^2 \lesssim r^{1 - d} \Exc_r(U, P) + |K|r^2.
\end{equation}
Then, since $r = \rho(V)/2^n$, $\Exc_r(U, P) = \gamma_n$, and for $C' := C|K|\rho^{d + 1}$, we have
\begin{equation}\label{induction on gamma}
\gamma_n \leq \frac{\gamma_0}{2^{nd}} + \sum_{k=0}^n \frac{C'}{2^{k(d + 1) + (n - k)d}} \leq \frac{\gamma_0 + C'}{2^{nd}}.
\end{equation}
Indeed, by induction,
\begin{align*}
\gamma_{n + 1}
&\leq \frac{\gamma_0}{2^{(n + 1)d}} + 2^{-d} \sum_{k=0}^n \frac{C'}{2^{k(d + 1) + (n - k)d}} + \frac{C'}{2^{(n + 1)(d + 1)}} \\
&= \frac{\gamma_0}{2^{(n + 1)d}} + \sum_{k=0}^{n + 1} \frac{C'}{2^{k(d + 1) + (n + 1 - k)d}}
\end{align*}
and (\ref{induction on gamma}) follows by summing the geometric series.
Reasoning as on \cite[pg100]{Giusti77} we conclude that for \emph{any} $s < r < \rho$, not just those of the form $r/2 < s < r = \rho/2^n$,
$$|\normal_U(P, r) - \normal_U(P, s)| \lesssim \sqrt{\frac{r}{\rho}}$$
and so $\normal_U(\cdot, r) \to \normal_U$ locally uniformly.
Since $\normal_U(\cdot, r)$ is continuous, the claim follows.
\end{proof}

\begin{proof}[Proof of Theorem \ref{main lma}, assuming the de Giorgi lemma]
If $U$ is a set of least perimeter, then $\normal_U$ is continuous for the subspace topology on $\partial U$, so by Proposition \ref{locality of Caccioppoli} it is a $C^1$ embedded hypersurface.
Since $\partial U$ is locally area minimizing, it must have zero mean curvature and hence, by elliptic bootstrapping, is analytic.
Actually, since $U$ is an absolute minimizer of (\ref{least gradient functional}), the Hessian of the area functional must be positive-definite at $\partial U$, thus $\partial U$ is stable.
\end{proof}

%%%%%%%%%%%%%%%%%%%%%%%%%%%%%%%%%%%%%

\subsection{The smooth case}
We begin the proof of the de Giorgi lemma with the following special case which is analogous to \cite[Lemma 6.4]{Giusti77}.
To state it we fix some constants: $c_0 = c_0(d, K) > 0$ is a small given constant, $\alpha = 1/2 + O(c_0)$ is also given, and $c_1 = c_1(c_0, d)$ will be chosen in the proof of Proposition \ref{Miranda44}.
Constants called $C$, and implied constants, are assumed to only depend on $d$ for $|K| \lesssim 1$ but possibly blow up as $K \to \infty$.
We shall always assume that we are working in a gauge so that
\begin{equation}\label{oscillation of isometries}
||\dif(P \mapsto \Phi^P)||_{C^0} \lesssim 1.
\end{equation}
This can always be ensured by not introducing unnecessary oscillation in $P \mapsto \Phi^P$ when we change coordinates, as we will only change coordinates boundedly many times in each stage of the induction.

\begin{proposition}[de Giorgi lemma, $C^1$ case]\label{Miranda44}
Assume the gauge condition (\ref{oscillation of isometries}).
For every $\rho, \beta$ satisfying $0 < \rho \ll_{c_1} |K|^{-1/2}$ and $0 < \beta \lesssim_{c_0, d} 1$, and every set $U$ with $C^1$ boundary in $B(P, \rho)$, if
$$\Exc_\rho(U, P) \leq \beta$$
and on $B(O, \rho)$, we have
\begin{align}
(\normal_U, \partial_0^P) &\geq e^{-o(c_1^2)}, \label{Miranda44 normal hyp} \\
|\partial^* U \cap B(P, \rho)| &\leq \eta(U, B(P, \rho)) + c_1 \beta, \label{Miranda44 minimality hyp}
\end{align}
where $c_1 = c_1(c_0, d) \ll 1$ is to be chosen later, then
\begin{equation}\label{Miranda44 concl}
\Exc_{\alpha \rho} (U, P) \leq (\alpha^{d + 1} + O(c_0)) \beta + C|K|\rho^{d + 1}.
\end{equation}
\end{proposition}

We introduce the Lagrangian
$$\Lagrange(y, \xi) := \frac{\Japan{\xi}}{(1 + K(|x|^2 + y^2)/4)^{d - 1}} \dif x,$$
for $x \in \RR^{d - 1}$, $y \in \RR$, and $\xi \in T'_x \RR^{d - 1}$.
Here, as always, $\Japan \xi$ denotes the Japanese norm $\sqrt{1 + |\xi|^2}$.
For $K = 0$ this Lagrangian reduces to the usual minimal surface equation, as studied in \cite[Chapter 6]{Giusti77}.

We first show that critical points of $w \mapsto \Lagrange(w, \dif w)$ define minimal surfaces in $M$.
To this end we define for $w \in C^1(\Omega)$, $\Omega \subseteq \RR^{d - 1}$, the map $\Psi_w: \Omega \to M$ given by
$$\Psi_w(x)^i := x^i, \quad \Psi_w(x)^0 := w(x).$$

\begin{lemma}\label{Plateau setup lemma}
Let $w \in C^1(\Omega)$. Then $(\Psi_w^{-1})^* \Lagrange(w, \dif w)$ is the Riemannian measure on $\Psi_w(\Omega)$.
\end{lemma}
\begin{proof}
For $(\partial_i)$ the standard frame on $\Omega$, the metric on the image of $\Psi_w$ satisfies
\begin{align*}
\Psi_w^*(g|\Psi_w(\Omega))(\partial_i, \partial_j) &= g_{\mu\nu} \partial_i \Psi_w^\mu \partial_j \Psi_w^\nu \\
&= \frac{\delta_{\mu\nu}}{(1 + K(|x|^2 + w^2)/4)^2} (\partial_i \Psi_w^\mu \partial_j \Psi_w^\nu) \\
&= \frac{\delta_{ij} + \partial_i w \partial_j w}{(1 + K(|x|^2 + w^2)/4)^2}.
\end{align*}
By \cite[(24)]{Petersen2008} we have $\det((\delta_{ij} + \partial_i w \partial_j w)_{ij}) = 1 + |\dif w|^2$.
\end{proof}

We can view the derivative of a function $w: \Omega \to \RR$ as a map $\dif w: \Omega \to \RR^{d - 1}$.
In particular, it is well-defined to take the average $\avg_\rho \dif w$ of $\dif w$ over the ball $\mathscr B_\rho := \{x \in \RR^{d - 1}: |x| < \rho\}$, c.f. (\ref{averages and flat connections}).
Here and henceforth we use $\mathscr B$ to refer to balls in $\RR^{d - 1}$, and reserve $B$ for balls in $M$.

We now show that $\Lagrange(w, \dif w) - \Lagrange(w, \avg_\rho \dif w)$ is close to the analogous quantity for the euclidean case if $\rho$ is small.

\begin{lemma}
Let $\beta, \rho > 0$, $w \in C^1(B_\rho)$, $w(0) = 0$, $||\dif w||_{C^0} \ll_{c_0, K} 1$, and assume that $P \in M$ satisfies $\Psi_w(0) \in B(P, \rho)$. Then
\begin{align}
\int_{\mathscr B_\rho} \Lagrange(w, \dif w) - \Lagrange(w, \avg_\rho \dif w)
&= \int_{\mathscr B_\rho} \Japan{\dif w} \dif x - \Japan{\avg_\rho \dif w} \dif x + O(|K| \rho^{d + 1}) \label{compare Lagrangians}.
\end{align}
\end{lemma}
\begin{proof}
The left-hand side of (\ref{compare Lagrangians}) is
\begin{align*}
&\left|\int_{\mathscr B_\rho} ((1 + K(|x|^2 + w(x)^2)/4)^{1 - d} - 1)(\Japan{\dif w} - \Japan{\avg_\rho \dif w}) \dif x\right| \\
&\qquad \lesssim_d |K| \int_{\mathscr B_\rho} (|x|^2 + w(x)^2) \cdot \left|\Japan{\dif w} - \Japan{\avg_\rho \dif w}\right| \dif x.
\end{align*}
But $|\mathscr B_\rho| \sim \rho^{d - 1}$, and
$$|\Japan{\dif w} - \Japan{\avg_\rho \dif w}| \lesssim \exp(||\dif w||_{C^0}) \lesssim 1,$$
so
\begin{align*}
(|x|^2 + w(x)^2) &\leq (1 + ||\dif w||_{C^0}^2) \rho^2 \lesssim \rho^2.
\end{align*}
This proves (\ref{compare Lagrangians}).
\end{proof}

We now prove an analogue of \cite[Lemma 6.3]{Giusti77}, which is the de Giorgi lemma for a minimal graph near $O$.
Thanks to (\ref{compare Lagrangians}), the proof is essentially the same as the euclidean case.

\begin{lemma}[de Giorgi lemma, minimal graphs]\label{Miranda43}
If $c_0 \ll_d 1$ then there exists $c_1 = c_1(c_0) > 0$ such that for every $\beta > 0$, $0 < \rho < 1$, and $w \in C^1(\mathscr B_\rho)$, if we denote by
$I_w$ the cylinder in $M$, $I := \{|x| < \rho\}$, and assume that $w(0) = 0$, $||\dif w||_{C^0} \leq c_1$, and
\begin{align}
\int_{\mathscr B_\rho} \Lagrange(w, \dif w) - \Lagrange(w, \avg_\rho \dif w) &\leq \beta \label{Miranda43 oscillation}, \\
\int_{\mathscr B_\rho} \Lagrange(w, \dif w) &\leq \eta(\Psi_w(\Omega), I) + c_1 \beta \label{Miranda43 minimality},
\end{align}
then
\begin{equation}\label{Miranda43 concl}
\int_{\mathscr B_{\alpha \rho}} \Lagrange(w, \dif w) - \Lagrange(w, \avg_{\alpha \rho} \dif w) \leq (\alpha^{d + 1} + c_0) \beta + C|K|\rho^{d + 1}.
\end{equation}
\end{lemma}
\begin{proof}
Let $u$ be the harmonic function on $\mathscr B_\rho$ which equals $w$ on $\partial \mathscr B_\rho$.
By elliptic regularity, the maximum principle for harmonic functions, and (\ref{Miranda43 minimality}),
$$||u||_{C^1} \lesssim ||u||_{C^0} \leq ||w||_{C^0} \leq \rho ||\dif w||_{C^1} \leq c_1.$$
In particular, $\Japan{\dif u} \lesssim 1$ and $u(x)^2 \lesssim \rho^2$, so
\begin{align*}
&\left|\int_{\mathscr B_\rho} \Lagrange(w, \dif w) - \Lagrange(u, \dif u) - \Japan{\dif w} \dif x + \Japan{\dif u} \dif x\right| \\
&\qquad \leq |K| \int_{\mathscr B_\rho} (|x|^2 + w(x)^2) \Japan{\dif w} \dif x + (|x|^2 + u(x)^2) \Japan{\dif u} \dif x
\lesssim_d |K| \rho^{d + 1}.
\end{align*}
Since $u$ and $w$ have the same trace, $|N_u \cap I| \leq \eta(\Psi_w(\Omega), I)$, thus by Lemma \ref{Plateau setup lemma},
\begin{align*}
\int_{\mathscr B_\rho} \Lagrange(w, \dif w) - \Lagrange(u, \dif u) &\leq \int_{\mathscr B_\rho} \Lagrange(w, \dif w) - \eta(\Psi_w(\Omega), I) \leq c_1 \beta.
\end{align*}
Moreover, we meet the hypotheses of (\ref{compare Lagrangians}).
We can moreover replace $\beta$ with some $\beta' \in [\beta, \beta + C|K|\rho^{d + 1}]$ so that $u, w$ meet the hypotheses of \cite[Lemma 6.2]{Giusti77} which gives the result for $c_0 \ll_d 1$.
\end{proof}

In order to apply Lemma \ref{Miranda43} we must show that we can, after applying a suitable isometry, bound a set $U$ of locally finite perimeter by a set $\Psi_w(\Omega)$ for some suitable $w, \Omega$.
However, $\Psi_w(\Omega)$ was defined with reference to a choice of coordinates, and one could imagine that the ball $B$ that $\partial U$ embeds into fails to be convex in such coordinates, thus the domain $\Omega$ could fail to be contractible.
Thus in the Riemannian case we must check that the choice of coordinates does not mangle the convexity of $B$, or equivalently the second fundamental form of $\partial B$.

\begin{lemma}\label{convex balls}
Every ball of radius $\lesssim |K|^{-1/2}$ which contains $O$ appears convex in the Poincar\'e ball model (for $K < 0$) or stereographic projection (for $K > 0$).
\end{lemma}
\begin{proof}
We begin by recalling how the second fundamental form transforms under a conformal change of variables.
Suppose that $g = e^{2\varphi} \tilde g$ for some metric $\tilde g$ and scalar field $\varphi$.
Then by \cite[(12)]{Mondino18}, the second fundamental form of a hypersurface $S$ is
\begin{equation}\label{conformal second form}
\Two^S = \tilde \Two^S - (\tilde \normal_S^\sharp \varphi) \cdot \tilde g.
\end{equation}
In the special case that $g$ has constant sectional curvature $K$, we can take $\tilde g_{ij} = \delta_{ij}$ and
\begin{equation}\label{conformal factor}
\varphi(x) = -\log (1 + K|x|^2/4).
\end{equation}

We now compute $\Two^S$ for $S = \partial B(O, r)$, as follows.
The distance to $\{|x| = \tilde r\}$ from $O$ is
$$r = \int_0^{\tilde r} \sqrt{g_{00}(t, 0, \dots, 0)} \dif t = \int_0^{\tilde r} \frac{\dif t}{1 + \tilde Kt^2} = \frac{1}{\sqrt{\tilde K}} \arctan\left(\sqrt{\tilde K} r\right)$$
where $\tilde K := K/4$ and we have taken suitable holomorphic extensions of $\sqrt \cdot$ and $\arctan(\cdot)$.
Inverting this equation we see that on $S$,
%$$|x| = \frac{1}{\sqrt{\tilde K}} \tan \left(\sqrt{\tilde K} r\right) = -\frac{1}{\sqrt{-\tilde K}} \tanh \left(\sqrt{-\tilde K} r\right).$$
%Regardless of the sign of $K$ we conclude
$|x| = r - O(|K|r^3)$.
From (\ref{conformal second form}) and (\ref{conformal factor}), and the fact that $\tilde \normal_S^\sharp(x) = x$, we conclude
\begin{equation}\label{Two of a sphere}
\Two^S_{ij} = \frac{1}{|x|} \delta_{ij} + \frac{2K|x|}{1 + K|x|^2/2} \delta_{ij} = \frac{\delta_{ij}}{r + O(|K|r^3)} = \frac{g_{ij}}{r + O(|K|r^3)}.
\end{equation}

Now we return to the situation of $S = \partial B$ where $B = B(P, r)$ is not centered on $O$, but only contains $O$.
Since (\ref{Two of a sphere}) is tensorial, it remains true that $\Two^S = g/(r + O(|K|r^3))$.
We let $\tilde \Two$ denote the second fundamental form of $S$ in coordinates, so by (\ref{conformal second form}),
$$|\tilde \Two - \Two| \lesssim |\dif \varphi| \lesssim |K|r.$$
Thus for $r < \min(O(|K|^{-1/2}), 1/2)$, $\tilde \Two \geq r^{-1} I$ is positive-definite, so in coordinates $B$ is locally convex (up to a rotation, $S$ is locally the graph of a convex function).
By the Tietze convexity theorem, it follows that $B$ is convex in coordinates.
\end{proof}

We are now ready to represent a set in $M$ as a function.
Thanks to Lemma \ref{convex balls}, the proof of the next lemma is essentially the same as the construction of the sequence of functions $(\omega_j)$ in the proof of \cite[Lemma 6.4]{Giusti77}, but without the wholly unnecessary appeal to proof by contradiction.

\begin{lemma}\label{rep as a good graph}
For every $\rho \ll_{c_0, c_1, d} |K|^{-1/2}$, every set $U$ of $C^1$ perimeter in $B(P, \rho)$, and every $Q \in B(P, \rho)$, such that
\begin{align}
(\normal_U, \partial_0^Q) &\geq e^{-o(c_1^2)} \label{rep as a good graph hyp}
\end{align}
there exists an open set $\Omega \subset \RR^{d - 1}$, $w \in C^1(\Omega)$, and a ball $\mathscr B \subset \RR^{d - 1}$, such that $\partial U = \Phi^Q(\Psi_w(\Omega))$,
\begin{equation}\label{rep as a good graph small derivative}
||\dif w||_{C^0} \leq c_1,
\end{equation}
and the pullback $\Omega^\alpha$ of $\partial U \cap B(P, \alpha \rho)$ to $\RR^{d - 1}$ satisfies
\begin{equation}\label{rep as a good graph set nests}
    \Omega^\alpha \subseteq (\alpha + c_0) \mathscr B \subset \mathscr B \subseteq \Omega.
\end{equation}
\end{lemma}
\begin{proof}
Without loss of generality we may assume $Q = O$ and $\Phi^O$ is the identity. Then (\ref{rep as a good graph hyp}) simply asserts that $\normal := \normal_U$ satisfies $\normal_0 \geq e^{-o(c_1^2)}$.
By Lemma \ref{convex balls}, $B(P, \rho)$ appears convex in coordinates, so by \cite[Theorem 4.8]{Giusti77} there exists an open set $\Omega \subset \RR^{d - 1}$ and $w \in C^1(\Omega)$ such that $U$ is bounded by $\{y = w\}$ and
$$||\dif w||_{C^0} \leq \sup_{x_1, x_2 \in \Omega} \frac{|w_n(x_1) - w(x_2)|}{|x_1 - x_2|} \leq e^{o(c_1^2)}\sqrt{1 - e^{-o(c_1^2)}} \leq c_1.$$
Therefore (\ref{rep as a good graph small derivative}) holds.

We begin the proof of (\ref{rep as a good graph set nests}) by letting $\Psi_w(x_0, y_0) := P$.

\begin{sublemma}
Let $r \leq \rho$, $\Omega(r) := \Psi_w^{-1}(\partial U \cap B(P, r))$, and $S(r)$ the set of all $x \in \Omega$ such that there exists $y$ such that $(x, y) \in \partial B(P, r)$.
If $\Omega(r)$ is nonempty, then $\sigma^+(r) := \max_{x \in S(r)} |x - x_0|$ and $\sigma^- := \min_{x \in S(r)} |x - x_0|$ satisfy
$$\sigma^\pm(r)^2 = r^2 - (\inf_{\Omega'} w - y_0)^2 + O(c_1 \rho^2).$$
\end{sublemma}
\begin{proof}
Since $\Omega(r)$ is nonempty, $\partial U$ meets $B(P, r)$ and hence $\partial B(P, r)$ (since $\partial U$ also meets $\partial B(P, \rho)$ and is connected).
In particular, there is an element of $\partial U \cap \partial B(P, r)$ which witnesses that $S(r)$ is nonempty.

Writing $w^+ := \sup_{\Omega(r)} w$ and $w^- := \inf_{\Omega(r)} w$, and using (\ref{rep as a good graph small derivative}) and the fact that $\diam \Omega(r) \leq \diam \Omega \lesssim \rho$,
\begin{equation}\label{w is almost constant}
w^+ \leq w^- + ||\dif w||_{C^0} \diam \Omega(r) \leq w^- + O(c_1 \rho).
\end{equation}
After discarding terms of size $O(|K| \rho^2)$ arising from the metric, we may assume that $B(P, r)$ is a euclidean ball and that $\sigma^\pm$ are the radii of disks in $B(P, r)$ centered on a point $(x_0, y)$ where $y \in [w^-, w^+]$, thus $y = w^- + O(c_1 \rho)$ by (\ref{w is almost constant}) and hence
$$(y - y_0)^2 = (w^- - y_0)^2 + (y - w^-)^2 + 2(y - w^-)(w^- - y_0) = (w^- - y_0)^2 + O(c_1 \rho^2).$$
So by the Pythagorean theorem applied to the triangle with legs $[y, y_0]$ and $[x_0, x_0 + \sigma^\pm]$,
\begin{align*}
r^2 &= (\sigma^\pm)^2 + (y - y_0)^2 + O(|K| \rho^2) = (\sigma^\pm)^2 + (w^- - y_0)^2 + O(c_1 \rho^2). \qedhere
\end{align*}
\end{proof}

From the definitions, $\mathscr B(x_0, \sigma^-(\rho)) \subseteq \Omega$ and $\mathscr B(x_0, \sigma^+(\alpha \rho)) \subseteq \Omega^\alpha$.
Since
$$\inf_{\Omega^\alpha} w \leq \sup_\Omega w \leq \inf_\Omega w + \diam \Omega \cdot ||\dif w||_{C^0} \leq \inf_\Omega w + O(c_1 \rho^2)$$
and $|K| \rho^2 \leq c_1 \rho$ for $\rho \ll |K|^{-1}$, we have for $v := \inf_\Omega w - y_0$ that
$$\sigma^+(\alpha \rho)^2 = \alpha^2\left(\rho^2 - \frac{v^2}{\alpha^2}\right) + O(c_1 \rho^2) \leq \alpha^2(\rho^2 - v^2) + O(c_1 \rho^2) \leq \alpha^2 \sigma^-(\rho)^2 + O(c_1 \rho^2).$$
Moreover, since $\partial U \cap B(P, \alpha \rho)$ is nonempty,
$$\sigma^-(\rho)^2 \geq \rho^2 - (\alpha \rho - |K|\rho^3)^2 - O(c_1 \rho^2) \gtrsim \rho^2$$
which implies that
$$\sigma(\alpha \rho) \leq \sqrt{\alpha^2 \sigma^-(\rho)^2 + O(c_1 \rho^2)} \leq \alpha \sigma^-(\rho) + O(c_1) \frac{\rho^2}{\alpha \sigma^-(\rho)} \leq \alpha \sigma^-(\rho) + O(c_1 \rho).$$
In particular,
$$\Omega^\alpha \subseteq \mathscr B(x_0, \alpha \sigma^-(\rho) + O(c_1 \rho)) \subseteq \mathscr B(x_0, \sigma^-(\rho)) \subseteq \Omega$$
which, along with the fact that $O(c_1) \leq c_0$, gives the claim with $\mathscr B := \mathscr B(x_0, \sigma^-(\rho))$.
\end{proof}

The above setup allows us to show the $C^1$ case of the de Giorgi lemma, an analogue of \cite[Lemma 6.4]{Giusti77}.
We cannot quite proceed as in \cite{Giusti77}, however, as we need to carefully apply the approximate translation symmetry.

\begin{proof}[Proof of Proposition \ref{Miranda44}]
Throughout this proof we assume that there exists some $Q \in \partial U \cap B(P, \alpha \rho)$.
If not, then (\ref{Miranda44 concl}) is vacuous since then $\Exc_{\alpha \rho} (U, P) = 0$.

Let $I: T_YM \to T_XM$ be the identity matrix with respect to the coordinate frame $(\partial_\mu)$, where $(\Phi^Q)^{-1}(Y) = (\Phi^P)^{-1}(X)$ and $X \in B(P, \rho)$ is given.
By our gauge assumption (\ref{oscillation of isometries}),
$$|\Phi^Q \circ (\Phi^P)^{-1} - I| \lesssim \rho^2 + ||\dif(Z \mapsto \Phi^Z)||_{C^0} \rho \lesssim \rho.$$
If $\rho$ is small enough depending on $c_1$ and the implicit constant in our gauge assumption, then $\rho \ll c_1^2$.
So by (\ref{Miranda44 normal hyp}), it follows that on $B(Q, \rho)$,
$$(\normal_U, \partial^Q_0) \geq O(\rho) + (\normal_U, \partial^P_0) \geq O(\rho) + e^{-o(c_1^2)} \geq e^{-o(c_1^2)}.$$
Therefore, by Lemma \ref{rep as a good graph},
there exist $\mathscr B, \Omega \subset \RR^{d - 1}$ satisfying (\ref{rep as a good graph set nests}), and $w: \Omega \to \RR$ such that $||\dif w||_{C^0} \leq c_1$, and such that the image of $\Phi^Q \circ \Psi_w$ is
$$\Gamma := \partial U \cap \{(x_Q^1, \dots, x_Q^{d - 1}) \in \Omega\}.$$
Since $Q \in \Gamma$, $w(0) = 0$.

Let $\mathscr B' := (\alpha + c_0) \mathscr B$, and for $W \subseteq \RR^{d - 1}$ open, introduce the cylinder
$$I_W := B(P, \rho) \cap \{(x^1_Q, \dots, x^{d - 1}_Q) \in W\}.$$
We write $\star'$ and $|\cdot|'$ for the euclidean Hodge star and absolute value with respect to the coordinates $(x^\mu_Q)$.
If we put
$$\Exc_A'(U) := \int_A \star' |\dif 1_U|' - \left|\int_A \star' (\normal_U, \partial^Q_\mu) \dif x^\mu_Q(Q)\right|',$$
then a Taylor expansion of the metric (\ref{constant sectional curvature metric}) gives for $A \subseteq B(Q, r)$ that
$$\Exc_A(U) = \Exc_A'(U) + O(r^{d + 1}).$$
On the other hand, by the proof of \cite[Lemma 6.4]{Giusti77}, for any ball $\mathscr D$ there is a dilate of $\mathscr D$ by a dilation of size $e^{O(c_0)}$ such that
\begin{align*}
\Exc_{I_{\mathscr D}}'(U) &= \int_{e^{O(c_0)} \mathscr D} \langle \dif w\rangle \dif x - \langle \avg_{e^O(c_0) \mathscr D} \dif w\rangle \dif x.
\end{align*}
Applying (\ref{compare Lagrangians}) we arrive at
\begin{equation}\label{excess versus lagrangian}
\Exc_{I_{\mathscr D}}(U) = \int_{e^{O(c_0)} \mathscr D} \Lagrange(w, \dif w) - \Lagrange(w, \avg_{e^{O(c_0)} \mathscr D} \dif w) + O(|K| \rho^{d + 1}).
\end{equation}

We are ready to prove (\ref{Miranda44 concl}).
By (\ref{rep as a good graph set nests}) and (\ref{excess versus lagrangian}),
\begin{align*}
\int_{e^{-O(c_0)} \mathscr B} \Lagrange(w, \dif w) - \Lagrange(w, \avg_{e^{-O(c_0)} \mathscr B} \dif w)
&\leq \Exc_{I_{\mathscr B}}(U) + O(|K| \rho^{d + 1}) \\
&\leq \Exc_\rho(U, P) + O(|K| \rho^{d + 1}) \\
&\leq \beta + O(|K| \rho^{d + 1}).
\end{align*}
By Lemma \ref{Miranda43}, it follows that
$$\int_{e^{O(c_0)} \mathscr B'} \Lagrange(w, \dif w) - \Lagrange(w, \avg_{e^{O(c_0)} \mathscr B'} \dif w) \leq (\alpha^d + O(c_0)) \beta + O(|K| \rho^{d + 1}).$$
Now applying (\ref{rep as a good graph set nests}) and (\ref{excess versus lagrangian}) again,
\begin{align*}
\Exc_{\alpha \rho}(U, P)
&\leq \Exc_{I_{\mathscr B'}}(U) + O(|K| \rho^{d + 1}) \\
&\leq \int_{e^{O(c_0)} \mathscr B'} \Lagrange(w, \dif w) - \Lagrange(w, \avg_{e^{O(c_0)} \mathscr B'} \dif w) + O(|K| \rho^{d + 1}) \\
&\leq (\alpha^d + O(c_0)) \beta + O(|K| \rho^{d + 1}). \qedhere
\end{align*}
\end{proof}

%%%%%%%%%%%%%%%%%%%%%%%%%%%%%%%%%%%%%%%%%%%
\subsection{Mollification}
We now reduce the de Giorgi lemma to its $C^1$ case.
To do so we need to introduce a suitable convolution kernel, but in order to carry out convolution we must work in coordinates.
To be more precise, the convolution $f * g$ of two functions defined near $O$, and the subtraction $x - y$ of two points near $O$, are defined in terms of the affine structure induced by the coordinates $(x^\mu)$.
In particular, neither of these operations are even approximately translation-invariant, and unlike in \cite{Giusti77} we must take care to ensure that this will not create unacceptable error terms.

Following \cite[Chapter 7]{Giusti77} we define the convolution kernel
$$\chi_\varepsilon(x) := \frac{d + 1}{|\Ball^d|} \varepsilon^{-d}1_{|x| < \varepsilon} \left(1 - \frac{|x|}{\varepsilon}\right)$$
We write $B_\varepsilon := B(O, \varepsilon)$ and $u_\varepsilon := u * \chi_\varepsilon$ whenever $u \in BV(B_{2\varepsilon})$.
When applied to sets of least perimeter with small excess, this convolution operator satisfies the following analogue of \cite[Theorem 7.3]{Giusti77}, which asserts that the convolution has a normal vector which is $C^0$ close to a constant.

\begin{proposition}\label{main mollifier lemma}
Let $q := \min(1/4, 1/(2(d - 1))$.
There exists $c > 0$ such that for every $0 < \rho \lesssim 1$, every $0 < \gamma \lesssim 1$, and every set $U$ of least perimeter such that
\begin{equation}\label{hypothesis on main mollifier lemma}
\Exc_\rho(U, O) \leq \gamma \rho^{d - 1},
\end{equation}
if we let $\varepsilon := \gamma^4\rho$, $\sigma := \gamma^{1/(2(d - 1))}\rho$, and $\varphi := (1_U)_\varepsilon$, then:
\begin{enumerate}
\item For every $y \in (c\gamma^2, 1 - c\gamma^2)$, the level set $\partial \{\varphi > y\} \cap B_{\rho - 2\sigma}$ is a $C^1$ hypersurface.
\item Possibly after applying a rotation around $O$, for every $x \in B_{\rho - 2\sigma}$ such that $c\gamma^2 < \varphi(x) < 1 - c\gamma^2$,
\begin{equation}\label{claim on main mollifier lemma}
\frac{\dif \varphi}{|\dif \varphi|} \geq 1 - O(\gamma^q) - O(\rho^2).
\end{equation}
\end{enumerate}
\end{proposition}

The estimate (\ref{claim on main mollifier lemma}) says that the normal vector to level sets $\{\varphi = y\}$, where $y$ is far from $\{0, 1\}$, is approximately constant in the coordinates $(x^\mu)$.
In particular, if $\gamma, \rho$ is small depending on $c_1$, then the level sets will satisfy (\ref{Miranda44 normal hyp}).
To deal with the somewhat large number of parameters involved here, it may helpful to think of the application to the de Giorgi lemma, in which case $c_1$ is frozen and $\gamma \sim \rho^2$, hence
$$1 \gg \gamma^q \gg \sigma \gg \rho \gg \gamma \gg \varepsilon \gg \varepsilon \delta > 0.$$
Since we can take losses of size $O(\rho^2)$, we shall mean by $|\cdot|$ the euclidean norm in the coordinates, and we shall work with the euclidean measure rather than $\star 1$.

The idea behind the proof of Proposition \ref{main mollifier lemma}, as in \cite{Giusti77}, is to cover $\partial^* U$ by small balls and apply the monotonicity formula in each ball.
Our next lemma thus estimates the behavior of $u$ in a single ball.

\begin{lemma}\label{mollifier sublemma}
Let $\delta := \gamma^d$, and let $u := 1_U$ satisfy
\begin{equation}\label{hypothesis on mollifier sublemma}
\int_{B_\rho} (|\dif u| - \partial_0 u)(z) \dif z \leq \rho^{d - 1} \gamma.
\end{equation}
Then for every $P \in \partial^* U \cap B_\varepsilon$ and $x \in B_{\rho - 2\sigma}$,
$$(1_{B(P, 2\delta\varepsilon)}(|\dif u| - \partial_0 u))_\varepsilon(x) \lesssim \gamma^q (1_{B(P, \delta\varepsilon)} |\dif u|)_\varepsilon(x).$$
\end{lemma}
\begin{proof}
We first claim that for $r > 0$ so small that $B(P, 2r) \subseteq B_\varepsilon$,
\begin{equation}\label{bound the kernel}
\sup_{y \in B(P, 2r)} \chi_\varepsilon(x - y) \lesssim \inf_{y \in B(P, r)} \chi_\varepsilon(x - y).
\end{equation}
In the euclidean case, this result can be isolated from the proof of \cite[Theorem 7.3]{Giusti77}.
Otherwise, we can use the smallness of $\varepsilon$ and the Taylor expansion of the metric to approximate $g$-balls by euclidean balls.
This suffices to prove (\ref{bound the kernel}), since $\chi_\varepsilon$ is uniformly continuous.

Now let $V := B(P, 2\delta\varepsilon)$.
Integrating (\ref{bound the kernel}) against $1_V(|\dif u| - \partial_0)$,
\begin{equation}\label{kernel bounded}
    (1_V(|\dif u| - \partial_0))_\varepsilon(x) \lesssim \inf_{y \in B(P, \delta\varepsilon)} \chi_\varepsilon(x - y) \int_V (|\dif u| - \partial_0 u)(z) \dif z.
\end{equation}
To estimate the right-hand side of (\ref{kernel bounded}), we assume that $\gamma$ is chosen so small that $\sigma > 2\delta\varepsilon$.
We then set $W := B(P, \sigma)$ and apply the weak monotonicity formula (\ref{weak monotonicity}) to get
$$(2\delta\varepsilon)^{1 - d} \int_V |\dif u|(z) \dif z \leq e^{O(|K| \sigma^2)} \sigma^{1 - d} \int_W |\dif u|(z) \dif z$$
and hence
\begin{align*}
(2\delta\varepsilon)^{1 - d} \int_V (|\dif u| - \partial_0 u)(z) \dif z
&\leq \sigma^{1 - d} \int_W (|\dif u| - \partial_0 u)(z) \dif z + O(|K|) \sigma^{3 - d} \int_W |\dif u|(z) \dif z \\
&\qquad + \sigma^{1 - d} \int_W \partial_0 u(z) \dif z - (2\delta\varepsilon)^{1 - d} \int_V \partial_0 u(z) \dif z\\
&=: I_1 + I_2 + I_3 - I_4.
\end{align*}

Before estimating the $I_i$, we next claim that we may assume that
\begin{equation}\label{submollifier gauge assumption}
|\partial_0 - \partial_0^P| \lesssim \gamma^{1/(d - 1)}
\end{equation}
on $W$. Indeed, the statement of Lemma \ref{mollifier sublemma} is independent of the choice of gauge $\Phi^P$, so we can make a gauge transformation, apply Lemma \ref{DoVF lemma}, and recall $O \in W$ (since $\sigma > \varepsilon$) to get $|\partial_0 - \partial_0^P| \lesssim \sigma^2$
which gives (\ref{submollifier gauge assumption}).

We then use (\ref{hypothesis on mollifier sublemma}) to bound
$$I_1 \leq \sigma^{1 - d} \int_{B_\rho} (|\dif u| - \partial_0 u)(z) \dif z \leq \gamma^{\frac{1 - d}{2(d - 1)}} \gamma = \gamma^{1/2}.$$
Also by Corollary \ref{doubling dimension} we have $I_2 \lesssim |K| \gamma^{1/(d - 1)}$.
We can ignore the $K$ since we are allowing constants to blow up as $K \to \infty$.

To estimate $I_3 - I_4$, we recall the notation (\ref{integral of du}) for vector-valued integrals, and apply the monotonicity formula, Proposition \ref{Monotone}, to compute
\begin{align*}
    I_3 - I_4 &= \sigma^{1 - d} I(u, Q, \sigma)_0 - (2 \delta \varepsilon)^{1 - d} I(u, Q, 2\delta\varepsilon)_0 + O(|K| \sigma^2) \\
    &\leq |\sigma^{1 - d} I(u, Q, \sigma) - (2 \delta \varepsilon)^{1 - d} I(u, Q, 2 \delta \varepsilon)| + O(|K| \sigma^2)) \\
    &\lesssim \sqrt{1 + (d - 1) \log \frac{\sigma}{2\delta\varepsilon}} \sqrt{\sigma^{1 - d} \int_W \star |\dif u|} \sqrt{\int_{2\delta\varepsilon}^\sigma \partial_r \left[e^{Ar^2} r^{1 - d} \int_{B(Q, r)} \star |\dif u|\right] \dif r}\\
&\qquad + |K| \sigma^{3 - d} \int_W \star |\dif u| + |K| \sigma^2 \\
&=: J_1 J_2 J_3 + J_4 + J_5.
\end{align*}
Here $0 \leq A \lesssim |K|$ is as in Proposition \ref{Monotone}.
By definition we have $J_1 \lesssim -\log \gamma$, $J_4 = I_1 \lesssim \gamma^{1/(d - 1)}$, and $J_5 \lesssim \gamma^{1/(d - 1)}$.
From Corollary \ref{doubling dimension} we have $J_2 \lesssim 1$.
We then need to bound $J_3$:

\begin{sublemma}
    We have $J_3 \lesssim \gamma^q$.
\end{sublemma}
\begin{proof}
We bound
\begin{align*}
J_3^2 &\leq \sigma^{1 - d} \int_W |\dif u|(z) \dif z - (2 \delta \varepsilon)^{1 - d} \int_V |\dif u|(z) \dif z + O(|K|) \sigma^{3 - d} \int_W \star |\dif u| \\
&= \sigma^{1 - d} \int_W (|\dif u| - \partial_0 u)(z) \dif z + \sigma^{1 - d} \int_W (\partial_0 - \partial_0^P)u(z) \dif z + \sigma^{1 - d} \int_W \partial_0^Pu(z) \dif z \\
  &\qquad - (2 \delta\varepsilon)^{1 - d} \int_V \partial_0^P u(z) \dif z + O(|K|) \sigma^{3 - d} \int_W \star |\dif u| \\
&=: K_1 + K_2 + K_3 - K_4 + K_5.
\end{align*}
Then $K_1 = I_1 \leq \gamma^{1/2}$, $K_2 = I_2 \lesssim \gamma^{1/(d - 1)}$, and $K_5 = J_4 \lesssim \gamma^{1/(d - 1)}$.

To estimate $K_3 - K_4$, we introduce the $d - 1$-form
$$\psi := \dif x^1_P \wedge \cdots \wedge \dif x^{d - 1}_P.$$
Then
\begin{equation}\label{K3 calculus}
K_3 = \sigma^{1 - d} \int_W \dif u \wedge \psi = \sigma^{1 - d} \int_{U \cap \partial W} \psi.
\end{equation}
We decompose
$$\partial W = \Gamma_+ \cup \Gamma_0 \cup \Gamma_-$$
where $\pm x^0_P > 0$ on the hemispheres $\Gamma_\pm$ and $\Gamma_0$ is the equator.
Then all positive contributions to the integral in the right-hand side of (\ref{K3 calculus}) come from $\Gamma_+$.
Moreover, as $d-1$-chains in $M$, $\partial \Gamma_+ = \Gamma_0$.
However, if we set $N := \{x^0_P = 0\}$ and $W_0 := W \cap N$, then $\Gamma_0 = \partial W_0$, and $\Gamma_+$ and $W_0$ have the same homology class relative to $\Gamma_0$.
But $\dif \psi = 0$, so by Stokes' theorem,
$$K_3 \leq \sigma^{1 - d} \int_{\Gamma_+ \cap U} \psi \leq \sigma^{1 - d} \int_{\Gamma_+} \psi = \sigma^{1 - d} \int_{W_0} \psi.$$
Since $\psi$ is the euclidean volume form on $W_0$, and $W_0$ is a $d-1$-ball whose euclidean radius is $\leq \sigma + O(|K| \sigma^3)$, it follows that
\begin{equation}\label{K3 calculus 2}
K_3 \leq |\Ball^{d - 1}| + O(|K| \sigma^2).
\end{equation}
By Corollary \ref{doubling dimension}, the right-hand side of (\ref{K3 calculus 2}) is $\leq K_4 + O(\gamma^{1/(d - 1)})$.
Adding up all the $K_i$ we complete the proof.
\end{proof}

By (\ref{kernel bounded}),
$$(1_V(|\dif u| - \partial_0 u))_\varepsilon(x) \lesssim (\delta\varepsilon)^{d - 1} \gamma^q \inf_{y \in B(P, \delta\varepsilon)} \chi_\varepsilon(x - y).$$
We finally apply Corollary \ref{doubling dimension} to prove
\begin{align*}
(\delta\varepsilon)^{d - 1} \inf_{y \in B(P, \delta\varepsilon)} \chi_\varepsilon(x - y)
&\lesssim \int_{B(P, \delta \varepsilon)} \chi_\varepsilon(x - y) |\dif u|(y) \dif y = |\dif u|_\varepsilon(x). \qedhere
\end{align*}
\end{proof}

\begin{proof}[Proof of Proposition \ref{main mollifier lemma}]
Let $\delta := \gamma^d > 0$ and $u := 1_U$.
We greedily construct a cover $\mathcal V = \{V_n: 1 \leq n \leq N\}$ of $\partial^* U \cap B_{\varepsilon(1 - 2\delta)}$ by balls of radius $2\delta\varepsilon$, centered on points $Q_n \in \partial^* U \cap B_{\varepsilon(1 - \delta)}$, which is \dfn{efficient} in the sense that each $Q \in \partial^*U \cap B_{\varepsilon(1 - \delta)}$ only lies in $O_d(1)$ balls $V_n$.
We set $V_0 := B_\varepsilon \setminus B_{\varepsilon(1 - 2\delta)}$.
Then $\supp \dif u \subseteq \bigcup_n V_n$, so
\begin{equation}\label{sum over cover}
(|\dif u| - \partial_0 u)_\varepsilon \leq \sum_{n = 0}^N (1_{V_n} (|\dif u| - \partial_0 u))_\varepsilon.
\end{equation}
After rotating the coordinate frame we may assume that $\dif u = \partial_0 u \dif x^0$. Then
\begin{equation}\label{excess is d0u}
    \int_{B(P, \rho)} (|\dif u| - \partial_0 u) \dif x = \Exc_\rho(U, P) + O(\rho^{d + 1}) \lesssim \rho^{d - 1}(\gamma + \rho^2).
\end{equation}
Applying (\ref{excess is d0u}) and reasoning similarly to \cite[pg92]{Giusti77}, one can show that there exists $c > 0$ such that
\begin{equation}\label{V0 case}
    (1_{V_0} (|\dif u| - \partial_0 u))_\varepsilon \lesssim (\gamma + \rho^2) |\dif u|_\varepsilon
\end{equation}
on $B_\sigma \cap \{c\gamma^2 < \varphi < 1 - c\gamma^2\}$.
% The proof of (\ref{V0 case}) is essentially given by \cite[pg92]{Giusti77}, so we just sketch it.
% For $y \in V_0$, $\chi_\varepsilon(x - y) \lesssim \delta/\varepsilon^d$, so using Corollary \ref{doubling dimension}, one can show
% $$\int_{V_0} \chi_\varepsilon(x - y)(|\psi| \cdot |\dif u| - \star(\dif u \wedge \psi))(y) \dif y \lesssim \frac{\gamma^d}{\varepsilon}.$$
% Here we used $||\psi||_{L^\infty} \lesssim 1$.
% One can use \cite[Lemma 7.1]{Giusti77}, the assumption $c\gamma^2 < \varphi < 1 - c\gamma^2$, and the fact that $g$ is a perturbation of the euclidean metric to obtain
% $$\int_{B_\varepsilon} \chi_\varepsilon(x - y) |\dif u|(y) \dif y \gtrsim \frac{\gamma^{d - 1}}{\varepsilon}$$
% which then implies (\ref{V0 case}).
Since $\mathcal V$ is efficient, we can sum (\ref{V0 case}) and Lemma \ref{mollifier sublemma} (with $\gamma$ replaced by $\gamma + \rho^2$) over $n$ in (\ref{sum over cover}) to show that
$$(|\dif u| - \partial_0 u)_\varepsilon \lesssim (\gamma^q + \rho^2) |\dif u|_\varepsilon$$
which implies (\ref{claim on main mollifier lemma}).
In particular near $\varphi^{-1}(y) \cap B_{\rho - 2\sigma}$, where $y \in (c\gamma^2, 1 - c\gamma^2)$, one has $|\dif u| > 0$.
Therefore $u$ is a $C^1$ submersion by \cite[Lemma 7.1]{Giusti77}, which completes the proof.
\end{proof}

%%%%%%%%%%%%%%%%%%%%%%%%%%%
\subsection{Reducing to the smooth case}
In order to combine the above results we use the following analogue of \cite[Lemma 7.5]{Giusti77}.
We do not assert that the $C^1$ manifold $\partial V$ is close to $\partial U$ in $C^0$, and in general it may only be close to a rotation of $\partial U$. The relevant fact is the estimate (\ref{single mollify excess}), that their excesses are close.

\begin{lemma}\label{single mollify}
For every $\varepsilon > 0$ there exists $\delta = \delta(d, K, \varepsilon) > 0$ and $r = r(d, K, \varepsilon) > 0$ such that for any ball $B(P, \rho)$ such that $\rho < r$ and any set $U$ of least perimeter in $B(P, \rho)$ such that
$$\Exc_\rho (U, P) \leq \delta \rho^{d - 1},$$
there exists a set $V$ of $C^1$ perimeter in $B(P, (1 - \varepsilon)\rho)$ such that
\begin{align}
((\Phi^P)^* \normal_V)_0 &\geq e^{-o(\varepsilon^2)}, \label{single mollify normal}\\
|\partial V \cap B(P, (1 - \varepsilon)\rho)| &\leq \eta(V, B(P, (1 - \varepsilon)\rho)) + \varepsilon \Exc_\rho (U, P), \label{single mollify minimality}
\end{align}
and for $\rho/10 \leq \varpi \leq (1 - \varepsilon)\rho$,
\begin{equation}
|\Exc_\varpi (U, P) - \Exc_\varpi (V, P)| \leq \varepsilon \Exc_\rho (U, P) + C|K| \rho^{d + 1}. \label{single mollify excess}
\end{equation}
\end{lemma}
\begin{proof}
If not, then there exist balls $B_n := B(P_n, \rho_n)$ and sets $U_n$ of least perimeter in $B_n$ such that
\begin{equation}\label{single mollify Excess assumption}
\gamma_n := \rho_n^{1 - d} \Exc_{\rho_n} (U_n, P_n) \leq n^{-2},
\end{equation}
and $\rho_n \leq 1/n$, but such that for every set $V_n$ of $C^1$ perimeter in $B(P_n, (1 - \varepsilon) \rho_n)$, at least one of (\ref{single mollify normal}), (\ref{single mollify minimality}), or (\ref{single mollify excess}) fail.
Applying an isometry, we may assume that $P_n = O$.

Now let $w_n := (u_n)_{\gamma_n^4 \rho_n}$ be the mollification of $u_n$, draw $t \in [0, 1]$ uniformly at random, and let $c, q > 0$ be as in Proposition \ref{main mollifier lemma} (which we apply with $\gamma := n^{-2}$ and $\rho := \rho_n$), $a_n := c\gamma_n^2$, $b_n := 1 - c\gamma_n^2$.
By the coarea formula, Proposition \ref{Coarea2},
$$\int_{tB_n} \star |\dif w_n| = \int_0^1 |\partial^* \{w_n > y\} \cap tB_n| \dif y \geq \int_{a_n}^{b_n} |\partial^* \{w_n > y\} \cap tB_n| \dif y,$$
so by the mean value theorem, there exists $y_n \in (a_n, b_n)$ such that
$$|\partial^* \{w_n > y_n\} \cap tB_n| \leq \frac{1}{b_n - a_n} \int_{tB_n} \star |\dif w_n|.$$
We then set $V_n := \{w_n > y_n\}$, $v_n := 1_{V_n}$, so that $V_n$ has $C^1$ boundary in $tB_n$ and
\begin{equation}\label{MVT mollifier}
|V_n \cap tB_n| \leq \frac{1}{b_n - a_n} \int_{tB_n} \star |\dif w_n|.
\end{equation}
Since $\grad w_n$ is normal to the level sets of $w_n$, $\normal_{V_n} = \dif w_n/|\dif w_n|$.
So by (\ref{claim on main mollifier lemma}),
$$(\normal_{V_n})_0 \geq 1 - O(n^{-2q} + n^{-2}) = 1 - O(n^{-2q}).$$
Thus for $n$ large, $V_n$ satisfies (\ref{single mollify normal}).

Let $\Gamma_n := \partial(tB_n)$.
Arguing completely analogously to the proofs of \cite[(7.23), (7.22)]{Giusti77}, respectively, we see that almost surely,
\begin{align}
||u_n - v_n||_{L^1(\Gamma_n)} &\ll \gamma_n \label{trace of vn} \\
|\partial V_n \cap sB_n| &\leq |\partial^* U_n \cap sB_n| + \gamma_n. \label{difference of surface area}
\end{align}
Here we are using the fact that $\sum_n n^{-2} = \pi^2/6$ is finite.
The conjunction of (\ref{trace of vn}), (\ref{difference of surface area}), and (\ref{a priori estimate 1}) implies
\begin{equation}
||\partial^* U_n \cap tB_n| - |\partial V_n \cap tB_n|| \ll \gamma_n, \label{mollifier quant2}
\end{equation}
and the conjunction of (\ref{mollifier quant2}), (\ref{a priori estimate 1}), the fact that $U_n$ has least perimeter, (\ref{single mollify Excess assumption}) and (\ref{trace of vn}) implies (\ref{single mollify minimality}).

To derive a contradiction, we must show that $V_n$ satisfies (\ref{single mollify excess}) for $n$ large.
To this end we fix $\varepsilon > 0$ and $\varpi \in [\rho/10, t\rho]$ where $t > 1 - \varepsilon$ can be chosen at random almost surely.
Then
\begin{align*}
    |\Exc_\varpi(U, O) - \Exc_\varpi(V, O)|
    &\leq ||\partial^* U_n \cap t B_n| - |\partial V_n \cap t B_n||\\
    &\qquad + \left|\left[\int_{B_\varpi} \star \partial_\mu(u_n - v_n) \right] \dif x^\mu(P)\right| + O(|K| \rho_n^{d + 1}) \\
    &=: I_1 + I_2 + I_3.
\end{align*}
By (\ref{mollifier quant2}), $I_1 < \varepsilon \Exc_\varpi(U_n, P_n)/2$ if $n$ is large, and $I_3$ is irrelevant.
Since $\dif \psi_\mu = 0$, by Stokes' theorem and (\ref{trace of vn}), if $n$ is large then
\begin{align*}
    I_2 &\leq \left|\left[\int_{\partial B_\varpi} \star_\varpi (\normal_{B_\varpi})_\mu (u_n - v_n)\right] \dif x^\mu(O)\right| \leq \frac{\varepsilon}{2} \Exc_\varpi(U_n, P_n) + O(|K| \rho_n^{d + 1}).
\end{align*}
This implies (\ref{single mollify excess}) for $n$ large and so contradicts our assumptions.
\end{proof}

\begin{proof}[Proof of the de Giorgi lemma]
Choose $A = A(d) \geq 1$ to be larger than the implied constant in the $O(c_0)$ in (\ref{Miranda44 concl}).
Choose $c_0 = c_0(A, d, K) \leq 2^{-(d + 3)}/A$ so small that $c_0$ meets the hypotheses of Lemma \ref{Miranda43}, and let $\alpha := 1/(2(1 - c_0))$.
Choose $c_1 = c_1(c_0, d, K) \leq c_0$ so small that $c_1$ meets the hypotheses of Proposition \ref{Miranda44}.

Given a set $U$ of least perimeter, let $V$ be the set obtained from Lemma \ref{single mollify} with $\varepsilon := c_1$, and let $\beta := \Exc_{B(P, \rho)} U$.
Then $(\normal_V, \partial_0^P) \geq e^{-o(c_1^2)}$ on $B(P, (1 - c_0)\rho)$, and
$$|\partial V \cap B(P, (1 - c_0)\rho)| \leq \eta(V, B(P, (1 - c_0)\rho)) + c_1 \beta.$$
By (\ref{single mollify excess}) and (\ref{approximate monotone}),
\begin{align*}
\Exc_{(1 - c_0) \rho} (V, P) &\leq \Exc_{(1 - c_0) \rho} (U, P) + c_0 \Exc_\rho (U, P) + C|K| \rho^{d + 1} \\
&\leq (1 + c_0) \beta + C |K| \rho^{d + 1}.
\end{align*}
Since $1/2 = \alpha (1 - c_0)$, by Proposition \ref{Miranda44}, it follows that
\begin{align*}
    \Exc_{\rho/2} (V, P) &\leq (2^{-(d + 1)} + Ac_0) (1 + c_0) \beta + C |K| \rho^{d + 1} \\
    &\leq (2^{-(d + 1)} + 2^{-(d + 2)}) \beta + C |K| \rho^{d + 1}.
\end{align*}
Finally, by (\ref{single mollify excess}) and (\ref{approximate monotone}),
\begin{align*}
    \Exc_{\rho/2} (U, P)
    &\leq \Exc_{\rho/2} (V, P) + c_0 \Exc_\rho (U, P) + C|K| \rho^{d + 1} \\
    &\leq (2^{-(d + 1)} + 2^{-(d + 2)} + 2^{-(d + 3)}) \beta + C |K| \rho^{d + 1}\\
    &\leq 2^{-d} \beta + C |K| \rho^{d + 1}. \qedhere
\end{align*}
\end{proof}


\printbibliography

\end{document}
