\documentclass[reqno,10pt]{amsart}
\usepackage[letterpaper, margin=1in]{geometry}
\RequirePackage{amsmath,amssymb,amsthm,graphicx,mathrsfs,url,slashed,subcaption}
\RequirePackage[usenames,dvipsnames]{xcolor}
\RequirePackage[colorlinks=true,linkcolor=Red,citecolor=Green]{hyperref}
\RequirePackage{amsxtra}
\usepackage{cancel}
\usepackage{tikz-cd}

% \setlength{\textheight}{9.3in} \setlength{\oddsidemargin}{-0.25in}
% \setlength{\evensidemargin}{-0.25in} \setlength{\textwidth}{7in}
% \setlength{\topmargin}{-0.25in} \setlength{\headheight}{0.18in}
% \setlength{\marginparwidth}{1.0in}
% \setlength{\abovedisplayskip}{0.2in}
% \setlength{\belowdisplayskip}{0.2in}
% \setlength{\parskip}{0.05in}
%\renewcommand{\baselinestretch}{1.05}

\title{Functions of least gradient and minimal laminations in constant curvature}
\author{Aidan Backus}
\date{July 2022}

\newcommand{\NN}{\mathbf{N}}
\newcommand{\ZZ}{\mathbf{Z}}
\newcommand{\QQ}{\mathbf{Q}}
\newcommand{\RR}{\mathbf{R}}
\newcommand{\CC}{\mathbf{C}}
\newcommand{\DD}{\mathbf{D}}
\newcommand{\PP}{\mathbf P}
\newcommand{\MM}{\mathbf M}
\newcommand{\II}{\mathbf I}
\newcommand{\Hyp}{\mathbf H}
\newcommand{\Sph}{\mathbf S}
\newcommand{\Group}{\mathbf G}
\newcommand{\GL}{\mathbf{GL}}
\newcommand{\Orth}{\mathbf{O}}
\newcommand{\SpOrth}{\mathbf{SO}}
\newcommand{\Ball}{\mathbf{B}}

\DeclareMathOperator*{\Expect}{\mathbf E}

\DeclareMathOperator{\avg}{avg}
\DeclareMathOperator{\card}{card}
\DeclareMathOperator{\cent}{center}
\DeclareMathOperator{\ch}{ch}
\DeclareMathOperator{\codim}{codim}
\DeclareMathOperator{\Cyl}{Cyl}
\DeclareMathOperator{\diag}{diag}
\DeclareMathOperator{\diam}{diam}
\DeclareMathOperator{\dom}{dom}
\DeclareMathOperator{\Exc}{Exc}
\newcommand{\ext}{\mathrm{ext}}
\DeclareMathOperator{\Gal}{Gal}
\DeclareMathOperator{\Hom}{Hom}
\DeclareMathOperator{\Iso}{Iso}
\DeclareMathOperator{\Jac}{Jac}
\DeclareMathOperator{\Lip}{Lip}
\DeclareMathOperator{\Met}{Met}
\DeclareMathOperator{\id}{id}
\DeclareMathOperator{\rad}{rad}
\DeclareMathOperator{\rank}{rank}
\DeclareMathOperator{\Rm}{Rm}
\DeclareMathOperator{\Hess}{Hess}
\DeclareMathOperator{\Hol}{Hol}
\DeclareMathOperator{\Prop}{Prop}
\DeclareMathOperator{\Radon}{Radon}
\DeclareMathOperator*{\Res}{Res}
\DeclareMathOperator{\sgn}{sgn}
\DeclareMathOperator{\singsupp}{sing~supp}
\DeclareMathOperator{\Spec}{Spec}
\DeclareMathOperator{\supp}{supp}
\DeclareMathOperator{\Tan}{Tan}
\newcommand{\tr}{\operatorname{tr}}

\newcommand{\Mink}{\mathbf m}
\newcommand{\Ric}{\mathrm{Ric}}
\newcommand{\Riem}{\mathrm{Riem}}
\newcommand*\dif{\mathop{}\!\mathrm{d}}
\newcommand*\Dif{\mathop{}\!\mathrm{D}}
\newcommand{\LapQL}{\Delta^{\mathrm{ql}}}

\newcommand{\dbar}{\overline \partial}

\DeclareMathOperator{\atanh}{atanh}
\DeclareMathOperator{\csch}{csch}
\DeclareMathOperator{\sech}{sech}

\DeclareMathOperator{\Div}{div}
\DeclareMathOperator{\Gram}{Gram}
\DeclareMathOperator{\grad}{grad}
\DeclareMathOperator{\dist}{dist}
\DeclareMathOperator{\spn}{span}
\DeclareMathOperator{\Ell}{Ell}
\DeclareMathOperator{\WF}{WF}

\newcommand{\Two}{\mathrm{I\!I}}

\newcommand{\Lagrange}{\mathscr L}
\newcommand{\DirQL}{\mathscr D^{\mathrm{ql}}}
\newcommand{\DirL}{\mathscr D}

\newcommand{\Hilb}{\mathcal H}
\newcommand{\Homology}{\mathrm H}
\newcommand{\normal}{\mathbf n}
\newcommand{\radial}{\mathbf r}
\newcommand{\evect}{\mathbf e}
\newcommand{\vol}{\mathrm{vol}}

\newcommand{\Bmu}{\boldsymbol \mu}
\newcommand{\Bnu}{\boldsymbol \nu}
\newcommand{\Blambda}{\boldsymbol \lambda}

\newcommand{\pic}{\vspace{30mm}}
\newcommand{\dfn}[1]{\emph{#1}\index{#1}}

\renewcommand{\Re}{\operatorname{Re}}
\renewcommand{\Im}{\operatorname{Im}}

\newcommand{\loc}{\mathrm{loc}}
\newcommand{\cpt}{\mathrm{cpt}}

\def\Japan#1{\left \langle #1 \right \rangle}

\newtheorem{theorem}{Theorem}[section]
\newtheorem{badtheorem}[theorem]{``Theorem"}
\newtheorem{prop}[theorem]{Proposition}
\newtheorem{lemma}[theorem]{Lemma}
\newtheorem{sublemma}[theorem]{Sublemma}
\newtheorem{proposition}[theorem]{Proposition}
\newtheorem{corollary}[theorem]{Corollary}
\newtheorem{conjecture}[theorem]{Conjecture}
\newtheorem{axiom}[theorem]{Axiom}
\newtheorem{assumption}[theorem]{Assumption}

\newtheorem{mainthm}{Theorem}
\renewcommand{\themainthm}{\Alph{mainthm}}

\newtheorem{claim}{Claim}[theorem]
\renewcommand{\theclaim}{\thetheorem\Alph{claim}}

\theoremstyle{definition}
\newtheorem{definition}[theorem]{Definition}
\newtheorem{remark}[theorem]{Remark}
\newtheorem{example}[theorem]{Example}
\newtheorem{notation}[theorem]{Notation}

\newtheorem{exercise}[theorem]{Discussion topic}
\newtheorem{homework}[theorem]{Homework}
\newtheorem{problem}[theorem]{Problem}

\makeatletter
\newcommand{\proofpart}[2]{%
  \par
  \addvspace{\medskipamount}%
  \noindent\emph{Part #1: #2.}
}
\makeatother

\newtheorem{ack}{Acknowledgements}

\numberwithin{equation}{section}


% Mean
\def\Xint#1{\mathchoice
{\XXint\displaystyle\textstyle{#1}}%
{\XXint\textstyle\scriptstyle{#1}}%
{\XXint\scriptstyle\scriptscriptstyle{#1}}%
{\XXint\scriptscriptstyle\scriptscriptstyle{#1}}%
\!\int}
\def\XXint#1#2#3{{\setbox0=\hbox{$#1{#2#3}{\int}$ }
\vcenter{\hbox{$#2#3$ }}\kern-.6\wd0}}
\def\ddashint{\Xint=}
\def\dashint{\Xint-}

\usepackage[backend=bibtex,style=numeric]{biblatex}
\renewcommand*{\bibfont}{\normalfont\footnotesize}
\addbibresource{topics.bib}
\renewbibmacro{in:}{}
\DeclareFieldFormat{pages}{#1}


\begin{document}
\begin{abstract}
The least-gradient maximum principle, essentially due to Miranda and de Giorgi in the 1960s, shows that least-gradient functions on euclidean space define a minimal lamination of the support of their derivative.
We show that this result holds on manifolds of constant sectional curvature by studying a Plateau-type PDE.
We then apply the least-gradient maximum principle to answer some questions of Daskalopoulos--Uhlenbeck concerning best-Lipschitz maps, generalize a result of G\'orny decomposing least-gradient functions, and propose some future applications.
\end{abstract}

\maketitle

%%%%%%%%%%%%%%%%%%%%%%%%%%%%%%%%%%%%%%%%%%%%%%%%%%%%%%%

% \tableofcontents

\section{Introduction}
Throughout this paper, let $M$ be an oriented Riemannian manifold of metric $g$ and dimension $d$.
For a function $u \in BV_\loc(M)$, we write $\star |\dif u|$ for the total variation of the derivative, c.f. (\ref{total variation}).

\begin{definition}\label{main definitions}
A function $u \in BV_\loc(M)$ has \dfn{least gradient} if for every open $U \Subset M$ and every $v \in BV_\cpt(U)$,
\begin{equation}\label{least gradient functional}
\int_U \star |\dif u| \leq \int_U \star |\dif u + \dif v|.
\end{equation}
A set $U$ of locally finite perimeter has \dfn{least perimeter} if $1_U$ has least gradient.
\end{definition}

Functions of least gradient are the formal limits of $p$-harmonic functions as $p \to 1$.
They arise naturally in conductivity and magnetic resonance imaging \cite{Tamasan2019, Joy09}, in the numerical analysis of mean curvature flow \cite{Thomas05}, in the continuum-time limit of certain combinatorial games \cite{Kohn06}, and as duals to $\infty$-harmonic functions \cite{daskalopoulos2020transverse}.
Since $\infty$-harmonic functions on hyperbolic surfaces induce geodesic laminations \cite{daskalopoulos2020transverse}, it is natural to conjecture some relationship between functions of least gradient and geodesic laminations.
The goal of this paper is to make this relationship precise.

\begin{definition}
A \dfn{minimal lamination}\footnote{Here ``minimal'' refers to the minimality (with respect to the area functional, if it is indeed convex) of the leaves, and does not refer to minimality of the lamination with respect to inclusion as in the notion of minimal lamination in \cite[Theorem 4.7]{casson_bleiler_1988}.} $\lambda$ in $M$ is a partition of a closed subset of $M$ into smooth connected hypersurfaces, called \dfn{leaves}, with zero mean curvature.
We call $\lambda$ a \dfn{geodesic lamination} if $M$ is a surface, and \dfn{analytic} if every leaf of $\lambda$ is analytic.
\end{definition}

\begin{mainthm}[maximum principle]\label{main thm}
Let $2 \leq d \leq 7$ and suppose that $M$ has constant sectional curvature.
\begin{enumerate}
\item Let $u$ be a function of least gradient, $\lambda := \bigcup_{y \in \RR} \partial \{u > y\}$.
Then, if $u$ is a function of least gradient,
\begin{enumerate}
\item $\lambda$ is an analytic minimal lamination in $M$,
\item for every $y \in \RR$, $\partial \{u > y\}$ is either empty or a locally finite union of connected hypersurfaces with boundary, and 
\item if $M$ is the interior of a surface-with-boundary $\overline M$ and $u \in BV_\loc(\overline M)$, then $\lambda$ extends to a geodesic lamination of $\overline M$.
\end{enumerate}
\item Conversely, if $\lambda$ is a minimal lamination, $u \in BV_\loc(M)$, and every superlevel set $\{u > y\}$ is bounded by leaves of $\lambda$, then $u$ has least gradient and $\lambda$ is the analytic minimal lamination induced by $u$.
\end{enumerate}
\end{mainthm}

We now recall some special cases that were already known.
If $M$ is flat, this result essentially appears in \cite{Miranda66} but was formally spelled out and christened the \dfn{least-gradient maximum principle}\footnote{Theorem \ref{main thm} is called a maximum principle because it implies that if $M$ is the interior of a manifold-with-boundary $\overline M = M \cup \partial M$, then the level sets of any function of least gradient on $M$ must extend all the way to the boundary, just as the maximum principle implies that the level sets of the solution to an elliptic Dirichlet problem should behave.} in \cite[Proposition 3.4, Corollary 3.5]{górny2017planar}.
It also essentially appears if $M$ is a hyperbolic surface and $u$ is dual to an $\infty$-harmonic function in \cite[Theorem 5.2, Theorem 6.10]{daskalopoulos2020transverse}.
In that same paper, Daskalopoulos--Uhlenbeck conjectured that the maximum principle should hold for $M$ a hyperbolic surface, with no duality assumption \cite[Problem 9.4, Conjecture 9.5]{daskalopoulos2020transverse}.

The maximum principle is a local statement.
To apply it in a global setting we recall that an affine line bundle $E \to M$ is \dfn{flat with structure group $\RR$} if we have a privileged system of trivializations called \dfn{flat trivializations} such that the transition function $s_{ij}$ takes the form $s_{ij}(y) = y + c_{ij}$ for a constant $c_{ij} \in \RR$.
Thus flat trivializations $u, v$ of the same section satisfy $u = v + c$ on overlaps and the difference of two global sections is a global function.
We sometimes leave the bundle $E$ implicit.
We say that a section has \dfn{locally least gradient} if its flat trivializations have least gradient.
Since the leaves of a minimal lamination need not be globally area-minimizing, the maximum principle also applies to sections with locally least gradient.

The locality of the maximum principle also allows us to use it to generalize a theorem of G\'orny \cite[Theorem 1.2]{górny2017planar}, which decomposes functions of least gradient.
Our statement avoids the convexity assumptions of \cite{górny2017planar}, but instead captures the topology of $M$ using an affine line bundle:

\begin{mainthm}[G\'orny decomposition]\label{Gorny regularity}
Let $2 \leq d \leq 7$, suppose that $M$ has constant sectional curvature, and let $u: M \to \RR$ have least gradient.
Then there exists an affine line bundle $E \to M$ which is flat with structure group $\RR$, sections $Ju, Cu: M \to E$ of locally least gradient, and a discrete minimal lamination $\lambda$ in $M$, such that:
\begin{enumerate}
\item $Cu$ is continuous,
\item $Ju$ is locally constant on $M \setminus \lambda$,
\item the traces of $Ju$ and $u$ are constant on each side of both leaves of $\lambda$,
\item and $u = Ju - Cu$.
\end{enumerate}
\end{mainthm}

One should compare and contrast Theorem \ref{Gorny regularity} to \cite[Proposition 8.8]{daskalopoulos2020transverse}, which can be viewed as the special case that $M$ is a closed hyperbolic surface.
In that case one show that $Cu$ is a Devil's staircase, thus $\dif Cu$ is mutually singular with Lebesgue measure.
However, general minimal laminations are nowhere near as rigid as geodesic laminations on negatively curvatured manifolds and so we do not pursue this train of thought further here.

In \S\ref{open problems} we conjecture some further applications of the maximum principle.

Let us now discuss the main ingredients in the proof of the maximum principle.
The first is a generalization of Miranda's monotonicity formula \cite[Theorem 2.8]{Miranda66}.
This formula is stronger than monotonicity formulae for minimal surfaces in Riemannian manifolds that we are aware of (e.g. \cite[\S7]{MarquesXX}) in that it gives a lower bound on the rate of growth of the monotone quantity.
We will never use the conclusion about the Ricci curvature, which we include as a curiosity item.

\begin{mainthm}[monotonicity formula]\label{monotonicity prestate}
For every $P \in M$ there exists $A \geq 0$ depending continuously on $P$ such that for every function $u$ of least gradient defined near $P$, every exponential normal coordinate system $(x^\mu)$ based at $P$, and $0 < r_1 < r_2 \lesssim 1$,
\begin{align*}
&\left|\int_{r_1}^{r_2} \partial_r \left[r^{1 - d} \int_{B(P, r)} \dif u \wedge \dif x^1 \wedge \cdots \wedge \dif x^{d - 1}\right] \dif r\right|^2 \\
&\qquad \lesssim \left(1 + (d - 1) \log \frac{r_2}{r_1}\right) \left(r_2^{1 - d}\int_{B(P, r_2)} \star |\dif u| \right)\left(\int_{r_1}^{r_2} \partial_r \left[e^{Ar^2} r^{1 - d} \int_{B(P, r)} \star |\dif u|\right] \dif r\right).
\end{align*}
In particular,
\begin{equation}\label{weak monotonicity}
\int_{r_1}^{r_2} \partial_r \left[e^{Ar^2} r^{1 - d} \int_{B(P, r)} \star |\dif u|\right] \dif r \geq 0.
\end{equation}
Moreover, if $M$ is flat or has negative Ricci curvature, then we can take $A = 0$.
\end{mainthm}

Using Theorem \ref{monotonicity prestate}, we generalize Miranda's proof \cite{Miranda66} of de Giorgi's regularity theorem \cite{deGiorgi61} to manifolds\footnote{De Giorgi's original paper \cite{deGiorgi61} is notoriously difficulty to find a copy of, as noted by \cite{Miranda66, Giusti77}; moreover, \cite{Miranda66} is in Italian. Therefore, in the proof of Theorem \ref{main lma}, we follow the English-language monograph of Giusti \cite[Part 1]{Giusti77}.}.
From Theorem \ref{main lma} it is straightforward to show Theorem \ref{main thm}.
In \S\ref{open problems} we propose an alternate proof and a generalization of Theorem \ref{main lma}.

\begin{mainthm}[regularity]\label{main lma}
Let $2 \leq d \leq 7$ and suppose that $M$ has constant sectional curvature.
Then every set of least perimeter in $M$ is bounded by analytic minimal hypersurfaces.
\end{mainthm}

As in \cite{Miranda66, Giusti77}, we prove Theorem \ref{main lma} by reducing it to a de Giorgi-type lemma, Proposition \ref{de Giorgi}, which controls the oscillation of the conormal $1$-form, or \dfn{excess}, to the reduced boundary to a set of least perimeter.
However, the proof that the excess defined in \cite{Miranda66, Giusti77} is coordinate-independent relies on the flatness of the Levi-Civita connection of $\RR^d$ to allow for the integration of a $1$-form against a volume form.
Since that is manifestly impossible in our case, we instead construct the excess by choosing a family of stereographic projections (for sectional curvature $K < 0$) or Poincar\'e ball models (for $K > 0$) to cover $M$.
Such a family is well-defined up to the local rotation symmetries of $M$ and so gives a coordinate-invariant definition of the excess.
It is not quite translation-invariant, but by identifying tangent vectors to $\Hyp^d$ or $\Sph^d$ with elements of $\RR^{1, d}$ or $\RR^{d + 1}$ respectively, we are able to prove the ``next best thing'' to translation invariance, namely Proposition \ref{translation invariance}.
From here the proof of Proposition \ref{de Giorgi} is remarkably similar to the euclidean case and we omit many of the remaining details from there on.

%%%%%%%%%%%%%%%%%%%%%%%%%%%%%%%%%%%%%%%%%%%%%%%
\subsection{Preliminaries on functions of bounded variation}\label{prelims}
We start by fixing some notation. The operator $\star$ is the Hodge star, thus $\star 1$ is the Riemannian measure.
On a submanifold $\Sigma$ of codimension $\geq 1$, $\vol_\Sigma$ denotes the induced volume form and $\star_\Sigma$ denotes the induced Hodge star. We also write $\star_\rho := \star_{B(P, \rho)}$ if $P \in M$ is fixed.

Recall the definition of a de Rham current \cite{simon1983GMT}.
We write $\int_U \omega \wedge \psi$ for the pairing of an $\ell$-current $\omega$ with a compactly supported $\ell$-form $\psi$ in an open set $U$, and identify any $d - \ell$-form $\omega$ with its Poincar\'e dual, the $\ell$-current $\psi \mapsto \int_U \omega \wedge \psi$.

We now recall the Riemannian analogue of the theory in \cite[Chapter 1]{Giusti77}.
We identify the derivative of a function $u$ with the $d-1$-current
$$\int_U \dif u \wedge \psi = -\int_U u \dif \psi.$$
For a vector field $X$, we write $\star (Xu) := \dif u \wedge \star (X^\flat)$.
By definition, the space of functions of \dfn{bounded variation} $BV(U)$ is the space of functions $u$ for which
\begin{equation}\label{total variation}
\int_U \star |\dif u| := \sup_{\substack{||\psi||_{C^0} \leq 1\\\supp \psi \Subset V}} \int_U \dif u \wedge \psi
\end{equation}
is finite.
The local finiteness of $\int \star |\dif u|$, and hence the sheaf $BV_\loc$, is diffeomorphism-invariant.
Thanks to the diffeomorphism-invariance, one can reduce the following theorem to the euclidean case \cite[Teorema 1]{Miranda67}:

\begin{proposition}[Miranda trace theorem]\label{traces}
Let $U \subseteq M$ be an open set with nonempty Lipschitz boundary.
For every $u \in BV_\loc(U)$ there exists $v \in L^1_\loc(\partial U)$ such that for every $d-1$-form $\psi$,
\begin{equation}\label{Miranda IBP}
\int_U \dif u \wedge \psi + \int_U u \dif \psi = \int_{\partial U} v\psi.
\end{equation}
Moreover, for almost every $x \in \partial U$,
\begin{equation}\label{convergence of trace}
\int_{U \cap B(x, \varepsilon)} \star |v(x) - u| \ll \varepsilon^d.
\end{equation}
\end{proposition}

By \cite[Theorem 4.14]{simon1983GMT}, for every $u \in BV_\loc(M)$, there exists a $\star |\dif u|$-measurable section $f$ of the cosphere bundle $S'M$ such that for every compactly supported $d-1$-form $\psi$,
\begin{equation}\label{RNy formula}
\int_M \dif u \wedge \psi = \int_M f|\dif u| \wedge \psi.
\end{equation}
The section $f$ of (\ref{RNy formula}) is given pointwise $\star |\dif u|$-almost everywhere, in any local coordinates $(x^\mu)$, by
\begin{equation}\label{Lebesgue point definition}
    f(P) = \left[\lim_{r \to 0} \frac{\int_{B(x, r)} \star \partial_\mu u}{\int_{B(x, r)} \star |\dif u|}\right] ~\dif x^\mu,
\end{equation}
according to the Besicovitch differentiation theorem; here we view $(\dif x^\mu)$ as a basis of $T'_PM$.
Whether the limit $f(P)$ in (\ref{Lebesgue point definition}) exists, or indeed its value as a point of $S'_PM$, is diffeomorphism-invariant.

\begin{definition}
Let $U \subseteq M$ and $u = 1_U$. Then we say that $U$ has \dfn{locally finite perimeter} if $u \in BV_\loc(M)$.
In that case we make the following definitions:
\begin{enumerate}
\item The \dfn{measure-theoretic boundary} $\partial U$ is the set of points whose Lebesgue density with respect to $M$ is $\in (0, 1)$.
\item The set of points $P$ for which the limit (\ref{Lebesgue point definition}) exists and satisfies $|f(P)| = 1$ is the \dfn{reduced boundary} $\partial^* U$.
\item The $\star |\dif u|$-measurable $1$-form $f$ defined by (\ref{Lebesgue point definition}) is the \dfn{conormal $1$-form} $\normal_U$ to $\partial U$.
\item The \dfn{perimeter} $|\partial^* U \cap E|$ in a Borel set $E$ is $\int_E \star |\dif u|$.
\end{enumerate}
\end{definition}

Our definition of reduced boundary and conormal $1$-form follows \cite[Definition 3.3]{Giusti77} and is due to \cite{deGiorgi55}.
See \cite{Battista_2021} for an equivalent definition of reduced boundary on Riemannian manifolds, and see \cite[Chapter 6]{Pugh02} for the definition of Lebesgue density.

\begin{proposition}\label{locality of Caccioppoli}
    Let $U$ be a set of locally finite perimeter with conormal $1$-form $\normal$.
    Then:
    \begin{enumerate}
    \item $\partial^* U$ is either empty or $d-1$-dimensional in the Hausdorff sense, and is $d-1$-rectifiable.
    \item $\partial^* U$ is a dense subset of $\partial U$.
    \item If $\normal$ extends to a continuous $1$-form on $\partial U$, then $\partial^* U = \partial U$ is a $C^1$ hypersurface.
    \item If $\partial^* U = \partial U$ is a $C^1$ hypersurface, then $\normal$ is the conormal $1$-form on $\partial U$ as defined in differential topology, and $\star |\dif 1_U|$ is the induced volume form on $\partial U$.
\end{enumerate}
\end{proposition}
\begin{proof}
Most of the assertions of this proposition are diffeomorphism-invariant, so we may assume that $M = \RR^d$ and appeal to \cite[Chapters 2-4]{Giusti77}.
The proof that $\star |\dif 1_U|$ is the induced volume form is identical to \cite[Example 1.4]{Giusti77}.
\end{proof}

\begin{proposition}[coarea formula]\label{Coarea2}
Let $u \in BV_\loc(M)$ and $E$ an open set. Then
\begin{equation}\label{coarea formula}
\int_E \star |\dif u| = \int_{-\infty}^\infty |E \cap \partial^* \{u > y\}| \dif y.
\end{equation}
\end{proposition}
\begin{proof}
We follow \cite[Theorem 1.23]{Giusti77}, which first proves (\ref{coarea formula}) for $u \in C^\infty(\RR^d)$ using piecewise linear functions.
Such functions are not available for our purposes; instead we note that if $u \in C^\infty(\RR^d)$ and $u$ has no critical points then (\ref{coarea formula}) follows from Fubini's theorem, the fact that $|E \cap \partial \{u > y\}|$ is the surface area of $E \cap \{u = y\}$ (by Proposition \ref{locality of Caccioppoli}), and the change-of-variables formula.
However the left-hand side of (\ref{coarea formula}) is unaffected by critical points of $u$, and the right-hand side of (\ref{coarea formula}) is unaffected by critical values of $u$ by Sard's theorem.
So (\ref{coarea formula}) holds for $u \in C^\infty(\RR^d)$.
The rest of the proof is identical to \cite[Theorem 1.23]{Giusti77}, so we omit it.
% Taking a sequence in $C^\infty(M)$ that converges to $u$ in $L^1_\loc(M)$\footnote{Recall that $C^\infty(M)$ is not dense in $BV_\loc(M)$.}, and applying Fatou's lemma and the semicontinuity of total variation, we conclude the $\geq$ direction of (\ref{coarea formula}).
% Moreover, Stokes' theorem gives that for every $d-1$-form $\psi$ such that $||\psi||_{L^\infty} \leq 1$ and $\supp \psi \Subset E$,
% $$\int_E u \wedge \dif \psi = \int_{-\infty}^\infty \int_E |\psi| \star |\dif 1_{\partial \{u > y\}}| \dif y \leq \int_{-\infty}^\infty |E \cap \partial \{u > y\}| \dif y.$$
% Taking the supremum over $\psi$ we obtain the direction $\leq$ in (\ref{coarea formula}).
\end{proof}

We define
$$\eta(u, U) := \inf_{v \in BV_\cpt(U)} \int_U \star |\dif(u + v)|$$
for $u \in BV_\loc(M)$ and $U \Subset M$ with Lipschitz boundary, so that $u$ has least gradient iff $\eta(u, U) = \int_U \star |\dif u|$ for every $U$.
It is straightforward to generalize \cite[Lemma 5.6]{Giusti77}, thus for $u, v \in BV_\loc(U)$,
\begin{equation}
|\eta(u, U) - \eta(v, U)| \leq ||u - v||_{L^1(\partial U)}. \label{a priori estimate 1}
\end{equation}
In case $v = 0$, we note that by (\ref{convergence of trace}), the trace map is a contraction in $L^\infty$ norm, thus
\begin{equation}
\eta(u, U) \leq ||u||_{L^1(\partial U)} \leq |\partial U| \cdot ||u||_{L^\infty(M)}. \label{a priori estimate 2}
\end{equation}

\begin{definition}
A sequence $(u_n)$ in $BV_\loc(M)$ has \dfn{approximately least gradient} if for every open $U \Subset M$,
$$\limsup_{n \to \infty} \int_U \star |\dif u_n| \leq \liminf_{n \to \infty} \eta(u_n, U) < \infty.$$
\end{definition}

\begin{proposition}[Miranda stability theorem]\label{Miranda convergence}
If a sequence of functions $(u_n)$ has approximately least gradient and $u_n \to u$ in $L^1_\loc(M)$, then $u$ has least gradient, and for every open set $U \Subset M$ with Lipschitz boundary such that $\int_{\partial U} \star |\dif u| = 0$, one has
\begin{equation}\label{convergence in total variation}
\lim_{n \to \infty} \int_U \star |\dif u_n| = \int_U \star |\dif u|.
\end{equation}
\end{proposition}
\begin{proof}
The proof is similar to Teorema 3 and Osservazione 3 in \cite{Miranda67}; we just note the necessary modifications.
Suitable generalizations of Teorema 2 and Osservazione 2 follow from Proposition \ref{traces}.
One needs to add a term of size $o(1)$ to the right-hand side of the inequalities (2.8), (2.9), (2.13), and (2.14); however, in the limit, this term vanishes and so the conclusions (2.15) and (2.16) are unaffected.
\end{proof}

\begin{remark}\label{transversality}
The somewhat unusual condition $\int_{\partial U} \star |\dif u| = 0$ refers to the same Radon measure $\star |\dif u|$ that acts on the open sets of $M$, not on a measure that acts on the relatively open subsets of $\partial U$.
It should be interpreted as a transversality condition: if $u$ is the indicator function of a set $Z$ with $C^\infty$ boundary, then $\int_{\partial U} \star |\dif u| = 0$ if $\partial U$ is transverse to $\partial Z$.
In particular, the notion of convergence in the Miranda stability theorem is weaker than convergence in $BV_\loc(M)$.\footnote{Therefore the Dirichlet problem for functions of least gradient is ill-posed in general, as the trace morphism is discontinuous with respect to the topology for this notion of convergence. See \cite{spradlin2013traces} for an explicit counterexample.}
\end{remark}

\begin{corollary}\label{compactness}
Every sequence $(u_n)$ of approximately least gradient converges in $L^1_\loc$ and almost everywhere along a subsequence to a function of least gradient $u$ such that for every open set $U \Subset M$ of Lipschitz boundary such that $\int_{\partial U} \star |\dif u| = 0$, one has (\ref{convergence in total variation}).
\end{corollary}
\begin{proof}
By compactness of the natural map $BV_\loc \to L^1_\loc$, $(u_n)$ has a convergent subsequence in $L^1_\loc$ and almost everywhere.
The conditions on the limit $u$ follow from the Miranda stability theorem.
\end{proof}

\begin{proposition}\label{level sets are minimal}
For every function $u$ of least gradient, the superlevel sets $\{u > y\}$ have least perimeter.
If we instead have a sequence $(u_n)$ of approximately least gradient, then $(\{u_n > y\})$ has approximately least perimeter.
\end{proposition}
\begin{proof}
In the proof of \cite[Theorem 1]{BOMBIERI1969}, we replace the coarea formula (replace \cite[Theorem 1.6]{Miranda66} with Proposition \ref{Coarea2}) and the Miranda stability theorem (replace \cite[Teorema 3]{Miranda67} with Proposition \ref{Miranda convergence}).
\end{proof}

% There is an annoying technicality with normal coordinates: they are not isothermal.\footnote{Suppose that the exponential map is conformal. By checking on planes in the tangent space we may assume that $d = 2$, and hence that $g$ is locally K\"ahler. But every K\"ahler metric with a holomorphic exponential map is flat \cite{MO68766}.} 
% To get around this, we introduce coordinates which are isothermal, but still satisfy the estimate (\ref{expand metric}):

% \begin{definition}
% We call two coordinate systems $(x^\mu)$ and $(\tilde x^\mu)$, based at $(P, V)$ and $(Q, W)$ respectively, \dfn{compatible} if: TODO update this
% \begin{enumerate}
% \item both $(x^\mu)$ and $(\tilde x^\mu)$ are normal coordinates, or they are both are stereographic coordinates, and 
% \item when written out in the coordinates $(x^\mu)$ and $(\tilde x^\mu)$ respectively, $V$ is a scalar multiple of $W$.
% \end{enumerate}
% \end{definition}

\begin{definition}
For a function $u$ on $M$, $P \in M$ we define the \dfn{tangent rescaling} of $u$ at $P$ to be the net of functions $u_t: T_PM \to \RR$, given by
$$u_t(v) = u\left(\exp_P(tv)\right).$$
\end{definition}

\begin{proposition}\label{blowup theorem}
Suppose that $U$ is an open set with least perimeter in $B(P, r)$, $P \in \partial^* U$, and $u = 1_U$.
Furthermore, suppose that $d \leq 7$.
Then the tangent rescaling of $u$ converges as $t \to 0$ along a subsequence (that we also denote $t \to 0$) in $L^1_\loc$ and almost everywhere, to the indicator function $v$ of a half-space $C \subset T_PM$ such that $0 \in \partial C$.
Moreover, for every open set $V \Subset T_PM$ of Lipschitz boundary such that $\partial V$ is transverse to $\partial C$,
$$\lim_{t \to 0} \int_V \star |\dif u_t| = \int_V \star |\dif v|.$$
\end{proposition}
\begin{proof}
We first observe that the tangent rescaling $(u_t)$ has approximately least gradient in $T_PM$ (where we give $T_PM$ its euclidean metric).
This follows from the Taylor expansion of the volume form and reasoning as in the proof of \cite[Theorem 9.3]{Giusti77}, and so by Corollary \ref{compactness}, there exists a set $C$ of least perimeter in $T_PM$, such that the tangent rescaling converges to $v := 1_C$ in the desired sense.
But $T_PM$ is isometric to $\RR^d$, $d \leq 7$, so by the Bernstein--Fleming theorem \cite[Theorem 17.3]{Giusti77}, $\partial C$ is a hyperplane.
The fact that $0 \in \partial C$ follows from the fact that $P \in \partial^* U$.
% To prove the claim, write $|\cdot|'$, $\star'$ for the notions taken in the tangent space with its euclidean geometry, and write $U_t$ for the set indicated by $u_t$.
% If $V$ is a precompact open subset of $T_PM$, $V_t = \{v \in T_PM: tv \in V\}$, then we have the scale-invariance
% \begin{equation}\label{almost blowup scale invariance}
% |\partial^* U_t \cap V|' = t^{1 - d}|\partial^* U_1 \cap V_{1/t}|'.
% \end{equation}
% From (\ref{almost blowup scale invariance}) and the Taylor expansion (\ref{expand volume form}),
% $$t^{d - 1} |\partial^* U_t \cap V|' = |\partial^* U_1 \cap V_{1/t}|' \leq e^{O(t^2)} |\partial^* U \cap \exp_P(V_{1/t})|.$$
% For every $w \in BV_\cpt(V)$, the least-gradient nature of $u$ gives
% $$|\partial^* U \cap \exp_P(V_{1/t})| \leq \int_{(\exp_P)_* V_{1/t}} \star |\dif(u + (\exp_P)_* w_{1/t})| \leq e^{O(t^2)} \int_{V_{1/t}} \star'|\dif(u_1 + w_{1/t})|'.$$
% Therefore, after applying (\ref{almost blowup scale invariance}) and (\ref{expand volume form}) again,
% $$|\partial^* U_t \cap V|' \leq e^{O(t^2)} t^{1 - d} \int_{V_{1/t}} \star' |\dif (u_1 + w_{1/t})| = e^{O(t^2)} \int_V \star' |\dif (u_t + w)|.$$
% Since $V,w$ were arbitrary, we conclude that $(u_t)$ has approximately least gradient.
\end{proof}

%%%%%%%%%%%%%%%%%%%%%%%%%%%%%%%%%%%%%%%%%%%%%%%

% \subsection{Applications to hyperbolic geometry}\label{hyperbolicApps}
% Let $M$ be a closed hyperbolic surface.
% Daskalopoulos--Uhlenbeck \cite{daskalopoulos2020transverse} considered best-Lipschitz maps $M \to \Sph^1$. They identified a particularly important class of such maps, the $\infty$-harmonic maps, which are particularly significant because they induce geodesic laminations of $M$.

% \begin{definition}\label{BestLipDfn}
% Write $L_f$ for the Lipschitz constant of a map $f: M \to N$.
% A \dfn{best-Lipschitz map} $u: M \to N$ is a minimizer of $L_u$ in a given homotopy class.
% For such a map we define the \dfn{maximum-stretch locus}
% $$\lambda_u := \{x \in M: L(x) = \sup L\}$$
% where $L(x)$ denotes the local Lipschitz constant of $u$ at $x$.
% If a best-Lipschitz map $u: M \to \Sph^1$ is the weak limit in $L^r$ for $r > d$ of $p$-harmonic maps as $p \to \infty$, we call $u$ \dfn{$\infty$-harmonic}.
% \end{definition}

% \begin{theorem}[Daskalopoulos--Uhlenbeck]\label{infinity harmonic laminations}
% Suppose that $M$ is a closed hyperbolic surface and $u: M \to \Sph^1$ is $\infty$-harmonic. Then the maximum-stretch locus $\lambda_u$ is a geodesic lamination in $M$.
% \end{theorem}

% In \cite[\S5]{daskalopoulos2020transverse}, Daskalopoulos--Uhlenbeck prove Theorem \ref{infinity harmonic laminations} by considering the viscosity solution theory of $\infty$-Laplace equation
% \begin{equation}\label{infinity laplace}
%     \Hess u(\grad u, \grad u) = 0.
% \end{equation}
% However, the theory of viscosity solutions of (\ref{infinity laplace}) is still nascent, and Daskalopoulos--Uhlenbeck ask \cite[Problem 9.5]{daskalopoulos2020transverse} for a proof of Theorem \ref{infinity harmonic laminations} that bypasses (\ref{infinity laplace}) altogether.

% We give a partial resolution of this problem by proving \cite[Theorem-Conjecture 9.6]{daskalopoulos2020transverse}.
% Before we state it, we recall from \cite[\S6]{daskalopoulos2020transverse} that to any $\infty$-harmonic map $u: M \to \Sph^1$, one may associate a section $v$ of locally least gradient, which we call a \dfn{Daskalopoulos-Uhlenbeck dual} of $u$, such that (among other properties) $\supp \dif v \subseteq \lambda_u$.
% The proof that a Daskalopoulos-Uhlenbeck dual exists does not use Theorem \ref{infinity harmonic laminations}.
% Since $\supp \dif v$ is a geodesic lamination, we conclude from the maximum principle that:

% \begin{corollary}\label{maximum stretch contains lamination}
% The maximum-stretch locus of an $\infty$-harmonic function on a closed hyperbolic surface contains a geodesic lamination.
% \end{corollary}

% The general results in \S\ref{prelims}, which were known before this paper, allow us to prove \cite[Conjecture 9.5]{daskalopoulos2020transverse} by showing that $\dif v$ endows $\lambda_u$ with the structure of an oriented measured lamination.
% For the definitions, see \cite[\S8]{daskalopoulos2020transverse} or \cite{Ruelle75}.

% \begin{proposition}\label{ruelle sullivan antiderivative}
% Let $\lambda$ be an oriented, transversely measured geodesic lamination on a closed hyperbolic surface $M$, and let $\dif v$ be the Ruelle-Sullivan $1$-current induced by $\lambda$.
% Then $\dif v$ is the derivative of a section of locally least gradient on $M$.
% \end{proposition}
% \begin{proof}
% As observed in \cite[\S9]{daskalopoulos2020transverse}, if we lift $\dif v$ to a $1$-current $\dif \tilde v$ on the universal cover $\Hyp^2$, then $\dif \tilde v$ is exact and any antiderivative $\tilde v$ of $\dif \tilde v$ has superlevel sets $\{\tilde v \geq y\}$ which are bounded by geodesics.
% Moreover we can choose $\tilde v$ to be $\pi_1(M)$-equivariant.
% Then the lift $\tilde \lambda$ of the geodesic lamination $\lambda$ to a fundamental domain of $M$ is a geodesic lamination in an open subset of $\Hyp^2$ and so is discrete, owing the chaoticness of the geodesic flow on the cosphere bundle $S'\Hyp^2$.
% Thus we obtain a section of least gradient on $M$, by the maximum principle. TODO coarea formula once we know it's discrete
% \end{proof}

%%%%%%%%%%%%%%%%%%%%%%%%%%%%%%%%%%%%%
\subsection{Some future directions}\label{open problems}
The hypothesis that $M = (M, g)$ has constant sectional curvature is very technically convenient, but it seems probably unnecessary, and with moderately more work, one should be able to remove or significantly weaken it.
A natural starting point would be to prove Theorem \ref{main lma} in the case that $M$ admits a cover by isothermal charts, because in this setting the obvious analogoue of Lemma \ref{Plateau setup lemma} would hold with an identical proof.
It seems unlikely that an analogue of Proposition \ref{translation invariance} would hold for any notion of excess defined without reference to a family of isometries; on the other hand, one could probably accept a de Giorgi lemma with error terms of size $O(\rho^d)$, as (\ref{LC Cauchy}) should still hold in that case.

A completely different approach to the proof of Theorem \ref{main lma} would be through the phase transition theory. 
To this end, let $W$ be a double-well potential, e.g. $W(x) = (1 - x^2)^2/4$.
If $u^\varepsilon(t)$ is a solution of the Allen-Cahn heat flow
$$\varepsilon^2(\partial_t u^\varepsilon(t) - \Delta u^\varepsilon(t)) + W'(u^\varepsilon(t)) = 0,$$
then the level sets $\{u^\varepsilon = y\}$, $-1 < y < 1$, evolve by an approximate mean curvature flow in the limit $\varepsilon \to 0$ \cite{Guaraco}, and so it is natural to try to approximate $u := 1_U - 1_{U^c}$, where $U$ has least periemter, by solutions $u^\varepsilon$ of the elliptic Allen-Cahn equation $\varepsilon^2 \Delta u^\varepsilon = W'(u^\varepsilon)$.
This approach was used to construct (necessarily unstable) minimal surfaces in threefolds by Chodosh--Mantoulidis \cite{Chodosh_2020}.
The regularity theory already established for the $\varepsilon \to 0$ limit in \cite[Appendix B]{Guaraco} implies that Theorem \ref{main lma} follows from the following:

\begin{conjecture}
Let $u := 1_U - 1_{U^c}$ where $U$ has least perimeter. Then there exists $u^\varepsilon$ solving the elliptic Allen-Cahn equation such that $u^\varepsilon$ has zero Morse index, and the level sets $\{u^\varepsilon = y\}$, $-1 < y < 1$, converge as varifolds as $\varepsilon \to 0$ to $\partial U$.
\end{conjecture}

Of course, the tricky bit is that $u^\varepsilon$ may not, a priori, have zero Morse index.
This approach seems most plausible when $M$ has negative curvature, so as to force the Allen-Cahn energy to have better convexity properties.

As far as the applications to hyperbolic geometry are concerned, a possible lead was suggested in private communication and \cite[Problem 9.11]{daskalopoulos2020transverse} by Daskalopoulos--Uhlenbeck.
Consider a closed hyperbolic threefold $M$ and a cohomology class $\xi \in H^1(M, \ZZ)$.
By the Hurcewiz theorem, $\xi$ determines a homomorphism $\pi_1(M) \to \ZZ = \pi_1(\Sph^1)$ and hence a homotopy class $[u]$ of maps $M \to \Sph^1$.
The duality theory of \cite{daskalopoulos2020transverse} suggests that for each $1$-form $v$ which is the limit of closed $p$-harmonic $1$-forms $v_p$
as $p \to \infty$, and $[v_p] = \xi$, there should be a map $u: M \to \Sph^1$ which is ``dual'' to $v$.
In particular, $u$ would have least gradient and would have homotopy class induced by $\xi$.
If this construction works as conjectured, then by Theorem \ref{main thm}, it provides a mechanism for constructing minimal laminations in $M$ with prescribed topological properties.

In the euclidean case, one can also show a well-posedness result \cite[Theorem 1.1]{górny2017planar} using Theorem \ref{Gorny regularity}.
The generalization of this fact is not obvious as we have not extended the Sternberg--Williams--Ziemer theorem \cite{ZiemerWilliamsSternberg1992} to the Riemannian case, so we state it as a conjecture:

\begin{conjecture}\label{Sternberg}
Let $\overline M$ be a compact, strictly convex surface-with-boundary with constant sectional curvature.
Then for each $f \in BV(\partial M)$ there is a unique function of least gradient on $M$ with trace $f$.
\end{conjecture}

Finally we remark that, thanks to a recent theorem of Loisel \cite{Loisel20}, Theorem \ref{main thm} may have applications in computational geometry.
To be more precise, in order to find a minimal surface $N$ in a manifold-with-boundary $\overline M$ with prescribed boundary data $N \cap \partial M$, it should suffice to find a function of least gradient with suitable boundary data.
Loisel's theorem ensures that we can find $u$ which minimizes the least-gradient functional in a suitable finite-element space in sublinear time.
It remains to show that for an appropriately fine triangulation, every simplex $T$ intersects the actual minimal surface $N$ iff $1/4 < u < 3/4$ on $T$ (say), and that we can efficiently compute the set of simplices on which $1/4 < u < 3/4$ (that is, we do not need to brute force search the set of simplices).
These conjectures seem reasonable and should furnish the minimal surface $N$ in sublinear time, avoiding the need to run the mean curvature flow for long times as in more standard approaches such as \cite{Thomas05}.

%%%%%%%%%%%%%%%%%%%%%%%%%%%%%%%%%%%%%
\subsection{Overview of the paper}
In \S\ref{MollifierSection}, we prove the monotonicity formula, Theorem \ref{monotonicity prestate}, and review its standard consequences.
Then in \S\ref{Plateau section}, we prove a de Giorgi-type lemma, Proposition \ref{de Giorgi}, and conclude Theorem \ref{main lma}.
In \S\ref{GornySec}, we use Theorems \ref{monotonicity prestate} and \ref{main lma} to prove the maximum principle, Theorem \ref{main thm}, and then as a consequence of the maximum principle and a topological argument we conclude Theorem \ref{Gorny regularity}.

%%%%%%%%%%%%%%%%%%%%%%%%%%%%%%%%%%%%%%%%%%%%%%%%

\subsection{Acknowledgements}
I would like to thank Georgios Daskalopoulos, Karen Uhlenbeck, Trent Lucas, Christine Breiner, NSF...



%%%%%%%%%%%%%%%%%%%%%%%%%%%%%%


%%%%%%%%%%%%%%%%%%%%%%%%%%%%%%%%%%%%%%%%%%%%%%%%%%%
\section{Monotonicity and mollification}\label{MollifierSection}
We have two purposes in this section: to prove Theorem \ref{monotonicity prestate} and to show that we can approximate minimal perimeters by $C^1$, approximately minimal perimeters.
Fix normal coordinates $(x^\mu)$, $\mu = 0, \dots, d - 1$, centered on a point $P \in M$.
We also write $B_r := B(P, r)$.
We use spherical coordinates $(\theta^i)$, $i = 1, \dots, d - 1$, on each sphere $\partial B_r$ which are compatible with $(x^\mu)$ in the sense that $x^0 = r \cos \theta^1$.
This is possible because
\begin{equation}\label{partial Br is a variety}
\partial B_r = \{(x^0)^2 + \cdots + (x^{d - 1})^2 = r^2\}.
\end{equation}
We write $\dif \sigma$ for the standard measure on $\Sph^{d - 1}$.

Following \cite{Giusti77}, we often use the freedom to choose coordinates to assume that the normal vectors to certain hypersurfaces $N$ are approximately horizontal with respect to a coordinate system $(x^\mu)$.
We make this precise by asserting that the conormal $1$-form to $N$, when wedged with the \dfn{vertical $d-1$-form}
\begin{equation}\label{d1 form}
\psi := \dif x^1 \wedge \dif x^2 \wedge \cdots \wedge \dif x^{d - 1}.
\end{equation}
should be close to the volume form.

%%%%%%%%%%%%%%%%%%%%%%%%%%%%%%%%%
\subsection{Monotonicity formula}
To prepare for the monotonicity formula, we first generalize an estimate that can be isolated from the proof of \cite[Lemma 5.8]{Giusti77}.

\begin{lemma}\label{monotonicity lemma}
There exists $A \geq 0$ such that for every $u \in C^1(B_R)$, $0 < r_1 < r_2 < R$, if we let
$$E(r) = \int_{B_r} \star |\dif u| - \eta(u, r),$$
so that $E(R) = 0$ iff $u$ has least gradient, then there exists $A \geq 0$ such that for $R > 0$ small,
\begin{equation}\label{monotonicity lemma eqn}
0 \leq \int_{B_{r_2} \setminus B_{r_1}} \star r^{1 - d}\frac{(\partial_ru)^2}{|\dif u|} \leq 2\int_{r_1}^{r_2} \partial_r \left[e^{Ar^2} r^{1-d}\int_{B_r} \star |\dif u|\right] + \frac{O(E(r))}{r^d} \dif r.
\end{equation}
If $g$ has negative Ricci curature or is flat, then $A = 0$.
\end{lemma}
\begin{proof}
We fix $s \in [r_1, r_2]$, work in normal coordinates, and introduce a competitor $v(r, \theta) = u(s, \theta)$.
From the definition of $\eta$,
\begin{equation}\label{consequence of least gradient monotone}
    \eta(u, s) \leq \int_U \star |\dif v| = \int_0^s \int_{\partial B_r} \star_r |\dif v| \dif r.
\end{equation}
Using the Taylor expansion of the metric in normal coordinates, and applying $\partial_r v = 0$, we obtain the existence of $A \geq 0$ such that
\begin{equation}\label{introduce the ricci tensor}
\int_{\partial B_r} \star_r |\dif v| \leq e^{As^2} \frac{\tilde r^{d - 1}}{s^{d - 1}} \int_{\partial B_s} \star_s |\dif v|.
\end{equation}
Here if $g$ has negative Ricci curvature or is flat, then $A = 0$, thanks to the facts that $\sqrt{\det g} = 1 + O(|x|^2)$ and $\Hess \sqrt{\det g(P)} = -\Ric(P)/3$ \cite[Lemma 3.4]{schoen1994lectures}.
Applying (\ref{consequence of least gradient monotone}) and Fubini's theorem,
\begin{align*}
\eta(u, s) &\leq  e^{As^2} \int_0^s \frac{r^{d - 1}}{s^{d - 1}} \dif r \cdot \int_{\partial B_s} \star_s |\dif v| = \frac{s e^{As^2}}{d} \int_{\partial B_s} \star_s |\dif v|\\
&\leq \frac{s e^{As^2}}{d - 1} \int_{\partial B_s} \star_s |\dif v|.
\end{align*}
By Gauss' lemma, $\dif v$ is the orthogonal projection of $\dif u$ onto $T' \partial B_s$, and its orthocomplement is $\partial_r u$. Therefore by Taylor's theorem,
$$\int_{\partial B_s} \star_s |\dif v| \leq \int_{\partial B_s} \star_s |\dif u| \sqrt{1 - \frac{(\partial_r u)^2}{|\dif u|^2}} \leq \int_{\partial B_s} \star_s \left[|\dif u| - \frac{(\partial_r u)^2}{2 |\dif u|}\right]$$
or in other words
\begin{align*}
\int_{\partial B_s} \star_s \frac{(\partial_r u)^2}{2|\dif u|} &\leq \int_{\partial B_s} \star_s |\dif u| - \frac{d - 1}{s} e^{-As^2} \eta(u, s)\\
&\leq \int_{\partial B_s} \star_s |\dif u| - \frac{d - 1}{s} e^{-As^2} \int_{B_s} \star |\dif u| - O(s^{-1}E(s)).
\end{align*}
We moreover have for $\tilde A \geq 0$ that
$$e^{-\tilde As^2} \partial_s \left[e^{\tilde As^2} s^{1 - d} \int_{B_s} \star |\dif u|\right] = \left[2\tilde As^{2 - d} - \frac{d - 1}{s^d}\right]\int_{B_s} \star |\dif u| + s^{1 - d} \int_{\partial B_s} \star_s |\dif u|$$
so if we choose $\tilde A$ so that
$$-\frac{d - 1}{s} e^{-As^2} = 2\tilde As - \frac{d - 1}{s}$$
then $A = 0$ implies $\tilde A = 0$, and
$$s^{1 - d} \int_{\partial B_s} \star_s |\dif u| - (d - 1)\frac{e^{-As^2}}{s^d} \int_{B_s} \star|\dif u| \leq e^{-As^2} \partial_s\left(e^{As^2} s^{1 - d} \int_{B_s} \star|\dif u|\right).$$
We moreover have $e^{-As^2} \leq 1$, so we can now integrate with respect to $\dif s$ to conclude the result.
\end{proof}

\begin{proof}[Proof of Theorem \ref{monotonicity prestate}]
We first compute $\dif u \wedge \psi = \partial_0 u \dif x$
where $\dif x$ is the natural euclidean volume form on $T_PM$.
Moreover, the radial part of $\partial_0$ is $\cos \theta^1$, and $\iota_{\partial_r} \dif x = r^{d - 1} \dif \sigma$.
Thus by (\ref{partial Br is a variety}),
$$\int_{B_r} \dif u \wedge \psi = r^{d - 1}\int_{\partial B_r} u \cos \theta^1 \dif \sigma(\theta)$$
and hence, since $|\cos \theta^1| \leq 1$,
\begin{align}
\left|\int_{r_1}^{r_2} \partial_r \left[r^{1 - d}\int_{B_r} \dif u \wedge \psi\right] \dif r\right|
&= \left|\int_{\Sph^{d - 1}} (u(r_2, \theta) - u(r_1, \theta)) \cos \theta^1 \dif \sigma(\theta)\right| \\
&\leq \int_{\Sph^{d - 1}} |u(r_2, \theta) - u(r_1, \theta)| \dif \sigma(\theta). \label{monotone dump the metric}
\end{align}
The metric $g$ plays no role in (\ref{monotone dump the metric}), so we may use \cite[Lemma 5.3]{Giusti77} to bound
$$\int_{\Sph^{d - 1}} |u(r_2, \theta) - u(r_1, \theta)| \dif \sigma(\theta) \leq \int_{\Sph^{d - 1}} \int_{r_1}^{r_2} r^{1 - d}|\partial_r u(r, \theta)| \dif r \dif\sigma(\theta).$$
To reintroduce the metric we posit that $R$ is small enough that $\dif r \dif \sigma(\theta) \leq \star 2$.
We therefore have
\begin{equation}\label{monotone before cs}
\int_{\Sph^{d - 1}} \int_{r_1}^{r_2} r^{1 - d}|\partial_r u(r, \theta)| \dif r \dif\sigma(\theta) \leq 2 \int_{B_{r_2} \setminus B_{r_1}} \star r^{1 - d}|\partial_r u|
\end{equation}
and if we apply the Cauchy-Schwarz inequality and approximate $u$ by $C^1$ functions (see \cite[pg68]{Giusti77}), it follows from Lemma \ref{monotonicity lemma} that the right-hand side of (\ref{monotone before cs}) is
$$\lesssim \sqrt{\int_{B_{r_2} \setminus B_{r_1}} \star r^{1 - d} |\dif u|} \sqrt{\int_{r_1}^{r_2} \partial_r \left[e^{Ar^2} r^{1-d}\int_{B_r} \star |\dif u|\right] \dif r}.$$
The monotonicity (\ref{weak monotonicity}) follows at once. To strengthen it we just need to bound $r^{1 - d} |\dif u|$.
Integrating by parts,
\begin{align*}
\int_{B_{r_2} \setminus B_{r_1}} r^{1 - d} |\dif u| &= \int_{r_1}^{r_2} r^{1 - d} \partial_r \int_{B_r} \star |\dif u| \dif r \\
&\leq r^{1 - d} \int_{B_r} \star |\dif u| + (d - 1) \int_{r_1}^{r_2} r^{-d} \int_{B_r} \star |\dif u| \dif r.
\end{align*}
Using (\ref{weak monotonicity}) we bound this second integral as
\begin{align*}
\int_{r_1}^{r_2} r^{-d} \int_{B_r} \star |\dif u| \dif r &\leq r^{1 - d} \log \frac{r_2}{r_1} \int_{B_{r_2}} \star |\dif u|. \qedhere
\end{align*}
\end{proof}

As an application, let us control the surface area of a minimal perimeter, generalizing \cite[Remark 5.13]{Giusti77}.
This gives a quantitative sense in which a minimal perimeter has codimension $1$.
Write $|\Ball^\ell|$ for the volume of the unit ball in $\RR^\ell$.

\begin{corollary}\label{doubling dimension}
If $d \leq 7$ then there exists $A \geq 0$ such that for every set $U$ of least perimeter in a ball $B_r = B(P, r)$, with $P \in \partial^* U$, and $r > 0$ small,
$$|\Ball^{d - 1}|e^{-Ar^2}r^{d - 1} \leq |\partial^*U \cap B_r| \leq |\Sph^{d - 1}|e^{Ar^2} r^{d - 1}.$$
\end{corollary}
\begin{proof}
The upper bound on $|\partial^* U \cap B_r|$ is obtained by using (\ref{a priori estimate 2}) and the fact that the surface area of $\partial B_r$ is $|\Sph^{d - 1}|(1 + O(r^2))r^{d - 1}$.
This can be seen by integrating $\star_{\partial B_r} 1$ along $\partial B_r$ in normal coordinates and applying the Taylor expansion of the volume form.
The lower bound is obtained from the monotonicity formula, which implies that
$$\limsup_{\rho \to 0} e^{-A\rho^2} \rho^{1 - d} |\partial^* U \cap B_\rho| \leq |\partial^* U \cap B_r|.$$
To control the left-hand side we take a tangent rescaling $(u_\rho)$ of $1_U$.
By Proposition \ref{blowup theorem} we can pass to a subsequence so that $u_\rho \to 1_C$ for $C$ a half-space, which in particular is transverse to $B'_1$, where as usual the prime denotes the euclidean metric on the tangent space.
Then
\begin{align*}
\limsup_{\rho \to 0} e^{-A\rho^2} \rho^{1 - d} |\partial^* U \cap B_\rho| &= \lim_{\rho \to 0} e^{O(\rho^2)} \int_{B'_1} \star'|\dif u_\rho|' = \int_{B'_1} \star'|\dif 1_C|' = |\partial C \cap B'_1|' = |\Ball^{d - 1}|. \qedhere
\end{align*}
\end{proof}

%%%%%%%%%%%%%%%%%%%%%%%%%%%%%%%%%%%%%%%%%%%%%%%%%%%%%%%%%%%%%%%%
\subsection{Mollification of sets of least perimeter}
Our goal for this section is to generalize \cite[Lemma 7.5]{Giusti77}, which reduces the study of sets of least perimeter to that of sets with $C^1$ perimeter.

In this section only, the convolution $f * g$ of two functions defined near $P$, and the subtraction $x - y$ of two points near $P$, are meant in the sense of the coordinates $(x^\mu)$\footnote{and crucially, not in terms of the coordinates $(\tilde x^\mu)$ that we introduce in the proof of Lemma \ref{mollifier sublemma}} and the volume form $\dif x := \dif x^0 \wedge \cdots \wedge \dif x^{d - 1}$ obtained from them. Following \cite[Chapter 7]{Giusti77} we define the convolution kernel
$$\chi_\varepsilon(x) := \frac{d + 1}{|\Ball^d|} \varepsilon^{-d}1_{|x| < \varepsilon} \left(1 - \frac{|x|}{\varepsilon}\right)$$
We write $u_\varepsilon := u * \chi_\varepsilon$ whenever $u \in BV$ is defined near $P$.

Our first goal is to prove a Riemannian analogue of \cite[Theorem 7.3]{Giusti77}, which estimates $(1_U)_\varepsilon$ for $U$ a set of least perimeter.
The argument there required one to cover $\partial^* U$ by small balls and apply the monotonicity formula in each ball.
To deal with the somewhat large number of parameters involved here, it may helpful to think of the case $\Delta = 1$, in which case we have
$$1 \gg \gamma^q \gg \sigma \gg \gamma \gg \varepsilon \gg \varepsilon \delta > 0.$$
The main new ingredients in the proof are estimates to deal with the lack of translation invariance and the new terms from the Taylor expansion of normal coordinates, but the idea is essentially the same as on \cite[pg89--92]{Giusti77}.

\begin{lemma}[control on $u_\varepsilon$ in each ball]\label{mollifier sublemma}
Let $q < \min(1/8, 1/(4(d - 1)))$.
For every $0 < \Delta, \gamma \lesssim 1$, if we let $\varepsilon = \gamma^4 \Delta$, $\sigma = \gamma^{1/(2(d - 1))} \Delta$, $\delta = \gamma^d$, and let $u$ be the indicator function of a set $U$ of least perimeter such that
\begin{equation}\label{hypothesis on mollifier sublemma}
\int_{B_\Delta} \star(|\psi| \cdot |\dif u| - \dif u \wedge \psi) \leq \Delta^{d - 1} \gamma,
\end{equation}
then on $B_{\Delta - 2\sigma}$, if $Q \in \partial^* U$,
$$(1_{B(Q, 2\delta\varepsilon)}(|\psi| \cdot |\dif u| - \star(\dif u \wedge \psi)))_\varepsilon \ll \gamma^q (1_{B(Q, \delta\varepsilon)} |\dif u|)_\varepsilon.$$
\end{lemma}
\begin{proof}
We first claim that for $r > 0$ so small that $B(Q, 2r) \subseteq B_\varepsilon$,
\begin{equation}\label{bound the kernel}
\sup_{y \in B(Q, 2r)} \chi_\varepsilon(x - y) \lesssim \inf_{y \in B(Q, r)} \chi_\varepsilon(x - y).
\end{equation}
In the euclidean case (with constant equal to $4$) this result can be isolated from the proof of \cite[Theorem 7.3]{Giusti77}.
Otherwise, we can use the smallness of $\varepsilon$ and the Taylor expansion of the metric to approximate $g$-balls by euclidean balls.
This suffices to prove (\ref{bound the kernel}), since $\chi_\varepsilon$ is uniformly continuous.

Now let $V := B(Q, 2\delta\varepsilon)$.
Integrating (\ref{bound the kernel}) against $1_V(|\psi| \cdot |\dif u| - \star(\dif u \wedge \psi))$,
\begin{equation}\label{kernel bounded}
(1_V(|\psi| \cdot |\dif u| - \star(\dif u \wedge \psi)))_\varepsilon(x) \lesssim \inf_{y \in B(Q, \delta\varepsilon)} \chi_\varepsilon(x - y) \int_V \star |\psi| \cdot |\dif u| - \dif u \wedge \psi.
\end{equation}
To estimate the right-hand side of (\ref{kernel bounded}) we introduce a new coordinate system $(\tilde x^\mu)$ of normal coordinates based at $Q$ which are \dfn{compatible} with $(x^\mu)$ in the sense that $\dif \tilde x^0(Q)$ is a scalar multiple of $\dif x^0(Q)$.
We write $\tilde g$ for the metric written in these new coordinates, and write $\tilde \psi$ for their vertical $d-1$-form.
By compatibility, $\tilde \psi = \psi + O(\varepsilon)$, so
\begin{equation}\label{split up T Ttilde}
\int_V \star |\psi| \cdot |\dif u| - \dif u \wedge \psi \leq \int_V \star |\psi| \cdot |\dif u| - \dif u \wedge \tilde \psi + O(\varepsilon) \int_V \star |\dif u|.
\end{equation}
The error term here is given by Corollary \ref{doubling dimension} and the assumption $\Delta \lesssim 1$ as $\lesssim \gamma^2 \int_{B(Q, \delta\varepsilon)} \star |\dif u|$.
To estimate the dominant term in (\ref{split up T Ttilde}) we assume that $\gamma$ is chosen so small that $\sigma > 2\delta\varepsilon$, so that if we set $W := B(Q, \sigma)$ and apply the monotonicity formula and compatibility to obtain $A \geq 0$ such that
\begin{align*}
(2\delta\varepsilon)^{1 - d} &\int_V \star |\psi| \cdot |\dif u| - \dif u \wedge \psi \\
&\leq \sigma^{1 - d}\int_W \star |\dif u| + (2\delta\varepsilon)^{1 - d} \int_V \star(|\psi| - 1)|\dif u| + 2A\sigma^{3 - d} \int_W \star |\dif u| - (2\delta\varepsilon)^{1 - d}\int_V \dif u \wedge \psi\\
&\leq \sigma^{1 - d}\int_W \star |\dif u| - \dif u \wedge \psi + (2\delta\varepsilon)^{1 - d} \int_V \star(|\psi| - 1)|\dif u| + 2A\sigma^{3 - d} \int_W \star |\dif u| \\
&\qquad + O(\varepsilon \sigma^{1 - d}) \int_W \star |\dif u| + \sigma^{1 - d}\int_W \dif u \wedge \tilde \psi - (2\delta\varepsilon)^{1 - d}\int_V \dif u \wedge \tilde \psi\\
&=: I_1 + I_2 + I_3 + I_4 + I_5 - I_6.
\end{align*}
We apply (\ref{hypothesis on mollifier sublemma}) to bound $I_1 \leq \gamma^{1/2}$, and we use the Taylor expansion of the metric on $V$ to bound $|\psi| - 1 \lesssim \varepsilon^2$ on $B_\varepsilon$.
By Corollary \ref{doubling dimension} and the assumption $\Delta \lesssim 1$, it follows that $I_2 + I_3 + I_4 \lesssim \gamma^{1/(d - 1)}$.

To estimate $I_5 - I_6$, we apply the monotonicity formula and compatibility again:
\begin{align*}
&\sigma^{1 - d} \int_W \dif u \wedge \tilde \psi - (2\delta\varepsilon)^{1 - d} \int_V \dif u \wedge \tilde \psi \\
&\qquad \lesssim \sqrt{1 + (d - 1) \log \frac{\sigma}{2\delta\varepsilon}} \sqrt{\sigma^{1 - d} \int_W \star |\dif u|} \sqrt{\int_{2\delta\varepsilon}^\sigma \partial_r \left[e^{Ar^2} r^{1 - d} \int_{B(Q, r)} \star |\dif u|\right] \dif r}\\
&\qquad \qquad + \varepsilon \sigma^{1 - d} \int_W \star |\dif u| + \varepsilon (2\delta\varepsilon)^{1 - d} \int_V \star |\dif u| \\
&\qquad =: J_1 J_2 J_3 + J_4 + J_5.
\end{align*}
We have $J_1 \lesssim -\log \gamma$, from Corollary \ref{doubling dimension} we have $J_2 \lesssim 1$, and finally $J_4 + J_5 \lesssim \gamma^4$.
So, we need to get a gain from $J_3$, which we do as follows:
\begin{align*}
J_3^2 &\leq \sigma^{1 - d} \int_W \star |\dif u| - (2 \delta \varepsilon)^{1 - d} \int_V \star |\dif u| + 2A\sigma^{3 - d} \int_W \star |\dif u| \\
&= \sigma^{1 - d} \int_W \star |\dif u| - \dif u \wedge \psi + \sigma^{1 - d} \int_W \dif u \wedge (\psi - \tilde \psi) + \sigma^{1 - d} \int_W \dif u \wedge \tilde \psi \\
&\qquad - (2 \delta\varepsilon)^{1 - d} \int_V \star |\dif u| + 2A \sigma^{3 - d} \int_W \star |\dif u| \\
&=: K_1 + K_2 + K_3 - K_4 + K_5.
\end{align*}
Then $K_1 = I_1 \leq \gamma^{1/2}$, $K_2 \lesssim I_3 \lesssim \gamma^4$, and $K_5 = I_2 \lesssim \gamma^{1/(d - 1)}$.

To estimate $K_3 - K_4$ we observe that for $u = 1_U$,
\begin{equation}\label{K3 calculus}
K_3 = \sigma^{1 - d} \int_W \dif u \wedge \tilde \psi = \sigma^{1 - d} \int_{U \cap \partial W} \tilde \psi.
\end{equation}
We decompose
$$\partial W = \Gamma_+ \cup \Gamma_0 \cup \Gamma_-$$
where $\tilde x^0 > 0$ on $\Gamma_+$ and $\tilde x^0 < 0$ on $\Gamma_-$. Then all positive contributions to the integral in the right-hand side of (\ref{K3 calculus}) come from $\Gamma_+$.
Moreover, as $d-1$-cells in $M$, $\partial \Gamma_+ = \Gamma_0$, but also if we set $N = \{\tilde x^0 = 0\}$ and $W_0 = W \cap N$, then $\Gamma_0 = \partial W_0$.
In particular, there is a homotopy relating $\Gamma_+$ and $\partial W_0$ which holds $\Gamma_0$ fixed.
Since $\dif \psi = 0$, we can use (\ref{partial Br is a variety}) as follows:
\begin{align*}
K_3 &\leq \sigma^{1 - d} \int_{\Gamma_+} \tilde \psi = \sigma^{1 - d} \int_{W_0} \tilde \psi \leq |\Ball^{d - 1}| + O(\sigma^2).
\end{align*}
Hence by Corollary \ref{doubling dimension},
$$K_3 \leq |\Ball^{d - 1}| + O(\sigma^2) \leq K_4 e^{O(\varepsilon\delta)^2} + O(\sigma^2) \leq K_4 + O(\sigma^2) \leq K_4 + O(\gamma^{1/(d - 1)}).$$
In conclusion, $J_3 \lesssim \gamma^{\min(1/4, 1/(d - 1))}$, and hence by (\ref{kernel bounded}) we have for some $q > 0$ that
$$(1_V(|\dif u| - \star(\dif u \wedge \psi)))_\varepsilon(x) \ll (\delta\varepsilon)^{d - 1} \gamma^q \inf_{y \in B(Q, \delta\varepsilon)} \chi_\varepsilon(x - y).$$
Finally, by Corollary \ref{doubling dimension},
\begin{align*}
(\delta\varepsilon)^{d - 1} \inf_{y \in B(Q, \delta\varepsilon)} \chi_\varepsilon(x - y) &\lesssim (1_{B(Q, \delta\varepsilon)} |\dif u|)_\varepsilon(x). \qedhere
\end{align*}
\end{proof}

\begin{figure}
\centering
\begin{subfigure}[b]{0.4\linewidth}
\includegraphics[width=\linewidth]{estimating_scales.png}
\end{subfigure}
\begin{subfigure}[b]{0.4\linewidth}
\includegraphics[width=\linewidth]{estimate homotopy.png}
\end{subfigure}
\caption{An illustration of the proof of Lemma \ref{mollifier sublemma}. On the left, the relative scales of balls involved in the proof of the lemma for $\gamma \ll 1$: $V$ is much smaller than $B_\varepsilon$, which in turn is tiny compared to $W$. On the right, $\Gamma_+$ and $W_0$ have the same homology class and the same boundary $\Gamma_0$, and $W_0$ is a slight deformation of a small ball in $\RR^{d - 1}$, so an integral $\int_{\Gamma_+} \omega$ can be approximated by the integral of $\star_{W_0} \omega$ over a small ball.}
\label{estimating figures}
\end{figure}

\begin{lemma}[control on $u_\varepsilon$]\label{main mollifier lemma}
Let $q < \min(1/8, 1/(4(d - 1)))$. There exists $c > 0$ such that for every $0 < \Delta \lesssim 1$ such that for every $0 < \gamma \lesssim 1$ and every indicator function $u$ of a set $U$ of least perimeter such that
\begin{equation}\label{hypothesis on main mollifier lemma}
\int_{B_\Delta} \star |\psi| \cdot |\dif u| - \dif u \wedge \psi \leq \gamma \Delta^{d - 1},
\end{equation}
if we let $\varepsilon = \gamma^4\Delta$, $\sigma = \gamma^{1/(2(d - 1))}\Delta$, and $\varphi = u_\varepsilon$, then on $B_{\Delta - 2\sigma} \cap \{c\gamma^2 < \varphi < 1 - c\gamma^2\}$,
\begin{equation}\label{claim on main mollifier lemma}
(1 - o(\gamma^q)) |\psi| \cdot |\dif \varphi| \leq \star(\dif \varphi \wedge \psi)
\end{equation}
and for every $y \in (c\gamma^2, 1 - c\gamma^2)$,
\begin{equation}\label{claim 2 on main mollifier lemma}
\text{the level set } \partial \{\varphi > y\} \cap B_{\Delta - 2\sigma} \text{ is a }C^1\text{ hypersurface}.
\end{equation}
\end{lemma}
\begin{proof}
Let $\delta = \gamma^d > 0$.
By running a greedy algorithm, we construct a cover $\mathcal V = \{V_n: 1 \leq n \leq N\}$ of $\partial^* U \cap B_{\varepsilon(1 - 2\delta)}$ by balls of radius $2\delta\varepsilon$, centered on points $Q_n \in \partial^* U \cap B_{\varepsilon(1 - \delta)}$, which is \dfn{efficient} in the sense that the dilates $V_n/2 := B(Q_n, \delta\varepsilon)$ are disjoint.
We set $V_0 := B_\varepsilon \setminus B_{\varepsilon(1 - 2\delta)}$.

To bring the $\psi$ inside the convolution we compute
\begin{align*}
(|\psi| \cdot |\dif u|)_\varepsilon
&= \int_{B_\varepsilon} |\psi|(x - y) |\dif u|(x - y) \chi_\varepsilon(y) \dif y \\
&= \int_{B_\varepsilon} |\psi|(x) |\dif u|(x - y) \chi_\varepsilon(y) \dif y + \int_{B_\varepsilon} ||\psi|(x - y) - |\psi|(x)| \cdot |\dif u|(x - y) \chi_\varepsilon(y) \dif y
\end{align*}
and observe that $||\psi|(x - y) - |\psi|(x)| \lesssim \dist(x, y) \lesssim \varepsilon \lesssim \gamma^4$.
Since $\dif u$ is supported in $\bigcup_n V_n$, it follows that
\begin{equation}\label{sum over cover}
|\psi| \cdot |\dif \varphi| - \star(\dif \varphi \wedge \psi)
\leq O(\gamma^4) |\dif \varphi| + \sum_{n=0}^N (1_{V_n}(|\psi| \cdot |\dif u| - \star(\dif u \wedge \psi)))_\varepsilon.
\end{equation}

We claim that there exists $c > 0$ such that on $B_\sigma \cap \{c\gamma^2 < \varphi < 1 - c\gamma^2\}$,
\begin{equation}\label{V0 case}
(1_{V_0}(|\psi| \cdot |\dif u| - \star(\dif u \wedge \psi)))_\varepsilon \lesssim \gamma |\dif u|_\varepsilon.
\end{equation}
The proof of (\ref{V0 case}) is essentially given by \cite[pg92]{Giusti77}, so we just sketch it.
For $y \in V_0$, $\chi_\varepsilon(x - y) \lesssim \delta/\varepsilon^d$, so using Corollary \ref{doubling dimension}, one can show
$$\int_{V_0} \chi_\varepsilon(x - y)(|\psi| \cdot |\dif u| - \star(\dif u \wedge \psi))(y) \dif y \lesssim \frac{\gamma^d}{\varepsilon}.$$
Here we used $||\psi||_{L^\infty} \lesssim 1$.
One can use \cite[Lemma 7.1]{Giusti77}, the assumption $c\gamma^2 < \varphi < 1 - c\gamma^2$, and the fact that $g$ is a perturbation of the euclidean metric to obtain
$$\int_{B_\varepsilon} \chi_\varepsilon(x - y) |\dif u|(y) \dif y \gtrsim \frac{\gamma^{d - 1}}{\varepsilon}$$
which then implies (\ref{V0 case}).

Since $\mathcal V$ is efficient, we can sum (\ref{V0 case}) and Lemma \ref{mollifier sublemma} over $n$ in (\ref{sum over cover}) to show that (\ref{claim on main mollifier lemma}) holds.
In particular near $\varphi^{-1}(y) \cap B_{\Delta - 2\sigma}$, where $y \in (c\gamma^2, 1 - c\gamma^2)$, one has $|\dif u| > 0$.
Therefore $u$ is a $C^1$ submersion by \cite[Lemma 7.1]{Giusti77}, thus (\ref{claim 2 on main mollifier lemma}) holds.
\end{proof}

% \begin{lemma}
% Suppose that $u = 1_U$ where $U$ has locally finite perimeter, $w = u_\varepsilon$, and $V = \{w > y\}$ where $y \in (0, 1)$.
% Then for any submanifold $\iota: E \to M$ and $0 < \varepsilon \lesssim 1$,
% \begin{equation}\label{Giusti125}
% x|E \cap (U \Delta V)| \leq \frac{1}{\min(y, 1 - y)} \int_E |w - u| \iota^*(\star 1).
% \end{equation}
% Moreover, if $0 < \tau \lesssim 1$, then
% \begin{align}
% \int_{B_\tau} \star |w - u| &\lesssim \varepsilon |B_{\tau + \varepsilon} \cap \partial^* U|, \label{Giusti711}\\
% \int_{B_\tau} \star (|\dif w| - |\dif u|) &\lesssim |(B_{\tau + \varepsilon} \setminus B_\tau) \cap \partial^* U|. \label{Giusti712}
% \end{align}
% \end{lemma}
% \begin{proof}
% The estimate (\ref{Giusti125}) is a straightforward modification of \cite[Lemma 1.25]{Giusti77}.
% The estimates (\ref{Giusti711}, \ref{Giusti712}) follow from \cite[Lemma 7.2]{Giusti77} where we use the fact that for $\tau \lesssim 1$, we can impose normal coordinates and compare $\star 1$ to the euclidean volume form using Taylor expansion.
% \end{proof}

We now state our main mollification result.
On first reading, it may help to take $P_n = P$, $(x^\mu_n) = (x^\mu)$, $\omega = \star \dif x^\mu$, and $\Delta_n = 1$.
The proof can be viewed as generalizing \cite[Lemma 7.5]{Giusti77}, and the main difference is that we phrase the estimate (\ref{mollifier quant3}) in a somewhat more coordinate-independent manner.

\begin{proposition}\label{mollifier quant}
Let $q < \min(1/8, 1/(4(d - 1)))$.
Let $(P_n)$ be a precompact sequence in $M$ and $0 < \Delta_n \lesssim 1$.
Let $U_n$ be a set of least perimeter in $B_n := B(P_n, \Delta_n)$, let $\psi^n$ be given by (\ref{d1 form}) for some normal coordinates $(x^\mu_n)$ based at $P_n$, and
$$\gamma_n := \Delta_n^{1 - d} \int_{B_n} \star |\psi^n| \cdot |\dif 1_{U_n}| - \dif 1_{U_n} \wedge \psi^n.$$
If $(\gamma_n) \in \ell^1$ then for almost every $t \in (0, 1)$ and every $n \in \NN$ there exists a set $V_n$ with $C^1$ boundary in $tB_n := B(P_n, t\Delta_n)$ such that for $n \gg 1$,
\begin{align}
|\partial V_n \cap tB_n| &\leq \eta(V_n, t\Delta_n) + o(\gamma_n \Delta_n^{d - 1}), \label{mollifier quant1}\\
||\partial^* U_n \cap tB_n| - |\partial V_n \cap tB_n|| &\ll \gamma_n \Delta_n^{d - 1}, \label{mollifier quant2}
\end{align}
on $B(P_n, t(1 - 2\sigma_n)\Delta_n)$, where $\sigma_n := \gamma_n^{1/(2(d - 1))}$, one has
\begin{equation}
\star(\normal_{V_n} \wedge \psi^n) \geq |\psi^n| \cdot (1 - o(\gamma^q)), \label{mollifier quant4}
\end{equation}
and for almost every $s \in (0, t]$ and every $d-1$-form $\omega_n$ defined near $P_n$,
\begin{equation}\label{mollifier quant3}
\left|\int_{sB_n} \dif(1_{U_n} - 1_{V_n}) \wedge \omega_n\right| \ll \gamma_n \Delta_n^{d - 1} ||\omega_n||_{C^1}.
\end{equation}
\end{proposition}
\begin{proof}
The precompactness of $(P_n)$ allows us to use the fact that all implied constants in this sections are clearly locally uniform in $P$.
Draw $t$ uniformly at random, let $w_n := (u_n)_{\Delta_n \gamma_n^4}$, let $c, q$ be as in Lemma \ref{main mollifier lemma}, and let $a_n = c\gamma_n^2$, $b_n = 1 - c\gamma_n^2$.
By the coarea formula, Proposition \ref{Coarea2},
$$\int_{tB_n} \star |\dif w_n| = \int_0^1 |\partial^* \{w_n > y\} \cap tB_n| \dif y \geq \int_{a_n}^{b_n} |\partial^* \{w_n > y\} \cap tB_n| \dif y,$$
so by the mean value theorem, there exists $y_n \in (a_n, b_n)$ such that
$$|\partial^* \{w_n > y_n\} \cap tB_n| \leq \frac{1}{b_n - a_n} \int_{tB_n} \star |\dif w_n|.$$
We then set $V_n := \{w_n > y_n\}$, $v_n := 1_{V_n}$, so $V_n$ has $C^1$ boundary in $tB_n$ by (\ref{claim 2 on main mollifier lemma}), and by the above computation,
\begin{equation}\label{MVT mollifier}
|V_n \cap tB_n| \leq \frac{1}{1 - 2c\gamma_n^2} \int_{tB_n} \star |\dif w_n|.
\end{equation}
Since $\grad w_n$ is normal to the level sets of $w_n$, $\normal_{V_n} = \dif w_n/|\dif w_n|$.
Hence (\ref{mollifier quant4}) is a consequence of (\ref{claim on main mollifier lemma}).

Arguing completely analogously to the proofs of \cite[(7.23), (7.22)]{Giusti77}, respectively, we see that
for $s \in (0, t]$ drawn uniformly at random and $\Gamma_n := \partial(sB_n)$, we have almost surely that
\begin{align}
||u_n - v_n||_{L^1(\Gamma_n)} &\ll \Delta_n^{d - 1} \gamma_n \label{trace of vn} \\
|\partial V_n \cap sB_n| &\leq |\partial^* U_n \cap sB_n| + o(\Delta_n^{d - 1} \gamma_n). \label{difference of surface area}
\end{align}
% By (\ref{Giusti711}) and Corollary \ref{doubling dimension},
% $$\limsup_{n \to \infty} \Delta_n^{1 - d} \gamma_n^{-4} \int_{sB_n} \star |u_n - w_n| \lesssim \limsup_{n \to \infty} \Delta_n^{2-d} |\partial^* U_n \cap sB_n| \lesssim \sup_n \Delta_n \lesssim 1.$$
% Differentiating in $s$, we see that
% $$||u_n - w_n||_{L^1(\Gamma_n)} \ll \Delta_n^{d - 1} \gamma_n^3$$
% almost surely. So by (\ref{Giusti125}),
% $$||u_n - v_n||_{L^1(\Gamma_n)} \lesssim \gamma_n^{-2} ||u_n - w_n||_{L^1(\Gamma_n)} \ll \Delta_n^{d - 1} \gamma_n$$
% almost surely, proving (\ref{trace of vn}).
% Now let
% $$f(r) = \sum_{n=1}^\infty \gamma_n \Delta_n^{1 - d} \int_{rB_n} \star |\dif u_n|.$$
% Then $f' \geq 0$, and by Corollary \ref{doubling dimension}, $f(1) \lesssim \sum_n \gamma_n < \infty$.
% So almost surely,
% $$f(s + \gamma_n^4) - f(s) \lesssim \gamma_n^4$$
% and hence
% $$\int_{(s + \gamma_n^4)B_n \setminus sB_n} \star |\dif u_n| \lesssim \gamma_n^3 \Delta_n^{d - 1}.$$
% From (\ref{Giusti712}) it follows that
% $$\int_{sB_n} \star |\dif w_n| \leq \int_{sB_n} \star |\dif u_n| + o(\gamma_n^2).$$
% By (\ref{MVT mollifier}) we conclude that (\ref{difference of surface area}) holds almost surely.
For $s = t$, the estimate (\ref{mollifier quant2}) is the conjunction of (\ref{trace of vn}), (\ref{difference of surface area}), and (\ref{a priori estimate 1});
(\ref{mollifier quant1}) is the conjunction of (\ref{mollifier quant2}), (\ref{a priori estimate 1}), the fact that $U_n$ has least perimeter, and (\ref{trace of vn}).

It remains to establish (\ref{mollifier quant3}).
Integrating by parts,
$$\left|\int_{sB_n} \dif (u_n - v_n) \wedge \omega_n\right| \leq ||\omega_n||_{L^\infty} ||u_n - v_n||_{L^1(\Gamma_n)} + ||\dif \omega_n||_{L^\infty} \int_0^t ||u_n - v_n||_{L^1(\partial(sB_n))} \dif s.$$
By (\ref{trace of vn}),
$$\limsup_{n \to \infty} ||\omega_n||_{C^1}^{-1} \gamma_n^{-1} \Delta_n^{1 - d} \left|\int_{sB_n} \dif(u_n - v_n) \wedge \omega_n\right| \leq \limsup_{n \to \infty} \int_0^s \gamma_n^{-1} \Delta_n^{1 - d} ||u_n - v_n||_{L^1(\partial(rB_n))} \dif r.$$
Moreover, (\ref{trace of vn}) holds with $s$ replaced by almost any $r$, so
$$f_n(r) := \gamma_n^{-1} \Delta_n^{1 - d} ||u_n - v_n||_{L^1(\partial(rB_n))}$$
satisfies $(f_n) \in \ell^\infty([0, 1] \to L^\infty)$, and $f_n \to 0$ almost everywhere.
So by Fatou's lemma,
\begin{align*}
0 \leq \limsup_{n \to \infty} ||\omega_n||_{C^1}^{-1} \gamma_n^{-1} \Delta_n^{1 - d} \left|\int_{tB_n} \dif(u_n - v_n) \wedge \omega_n\right| &\leq \int_0^t \lim_{n \to \infty} f_n(r) \dif r = 0. \qedhere
\end{align*}
\end{proof}


%%%%%%%%%%%%%%%%%%%%%%%%%%%%%%%%%%%%%%%%%%%%%%
\section{Regularity of minimal hypersurfaces}\label{Plateau section}
In this section we prove Theorem \ref{main lma}.
To set up the proof, recall that we can cover any manifold $M$ of constant sectional curvature $K \in \RR$ by charts $(x^\mu)$ in which the metric takes the form 
\begin{equation}\label{constant sectional curvature metric}
g_{\mu\nu} = \frac{\delta_{\mu\nu}}{(1 + K|x|^2/4)^2}.
\end{equation}
Without loss of generality, $M$ is equal to such a charts.
We fix the origin $O$, where $x = 0$.

When using the Einstein convention, Greek indices range over $0, 1, \dots$ while Latin indices range over $1, \dots$.
However, even when we have a hyperbolic time $t$ we do \emph{not} sum over it, and do not write $x^0 = t$.
Rather, we write $y := x^0$, which is a spacelike index.
We write $\Japan \xi := \sqrt{1 + |\xi|^2}$ for the Japanese norm of a vector $\xi$.

\subsection{Approximate conormal one-form}
Consider a smooth family $(\Phi^P)_{P \in M}$ of oriented isometries $M \to M$, such that $\Phi^O$ is a rotation and $\Phi^P(O) = P$.
We write 
$$\partial^P_\mu := \Phi^P_* \partial_{x^\mu}$$
and $x^\mu_P := x^\mu \circ \Phi^P$.
The choices of $(x^\mu)$ and $(\Phi^P)$ amount to selecting an oriented coordinate frame based at $P$ in which the metric takes the form (\ref{constant sectional curvature metric}).
The family of frames $(\partial^P_\mu)$ is uniquely determined up to a gauge transformation $\chi: M \to \SpOrth(TM)$, and so we will call a quantity \dfn{gauge-invariant} if it does not depend on the choice of $(x^\mu)$ or $(\Phi^P)$.

\begin{definition}
Let $U \subset M$ be a set of locally finite perimeter. Its \dfn{approximate conormal $1$-form} is defined for $P \in \partial U$,
$$\normal_U(P, r) := \frac{\int_{B(P, r)} \partial^P_\mu 1_U \star 1}{|\partial^* U \cap B(P, r)|} \dif x^\mu_P(P) \in T'_PM.$$
\end{definition}
 
\begin{lemma}\label{gauge invariance of the normal}
Let $U$ be a set of locally finite perimeter. Then $\normal_U(P, r)$ is gauge-invariant, and
$$|\normal_U(P, r)| \leq e^{C|K|r^2}.$$
\end{lemma}
\begin{proof}
For the gauge-invariance, let $\chi \in \SpOrth(T_PM)$ be a rotation.
Acting $\chi$ on $\normal_U(P, r)$ we obtain
$$\frac{\int_{B(P, r)} \chi^\nu_\mu \partial^P_\nu 1_U \star 1}{|\partial^* U \cap B(P, r)|} (\chi^{-1})_\lambda^\mu \dif x^\lambda_P(P) = \normal_U(P, r).$$ 
Now we compute for $M := |\partial^* U \cap B(P, r)|$ and $V := (\Phi^P)^{-1}(U)$ that 
\begin{align*}
|\normal_U(P, r)|^2 &= |\normal_V(P, r)|^2 = \delta_{\mu\nu} M^{-2} \left(\int_{\Phi^{-1}(B(P, r))} \partial_\mu 1_V \star 1\right) \left(\int_{B(O, r)} \partial_\nu 1_V \star 1\right) \\
&= M^{-2} \left|\sum_\mu \int_{B(O, r)} \partial_\mu 1_V \star 1\right|^2 \\
&\leq M^{-2} \max_\mu ||\partial_\mu||_{C^0(B(0, r))} |\partial^* V \cap B(O, r)| \\
&\leq \max_\mu ||g_{\mu\mu}||_{C^0(B(O, r))}.
\end{align*}
The claim now easily follows from the formula (\ref{constant sectional curvature metric}) for the metric.
\end{proof}

As in \cite[Chapters 8-9]{Giusti77} we need to estimate the oscillation of $\normal_U(P, \cdot)$, and the quantity that governs this oscillation is the excess:

\begin{definition}
The \dfn{excess} of a set $U \subset M$ of locally finite perimeter at $P \in \partial U$ and in the open set $A \ni P$ is 
$$\Exc_A(U, P) := |\partial^* U \cap A| - \left|\int_A \partial^P_\mu 1_U \star 1 \dif x_P^\mu(P)\right|.$$
For $\rho > 0$ we write $\Exc_\rho(U, P) := \Exc_{B(P, \rho)}(U, P)$.
\end{definition}

\begin{lemma}
Let $U$ be a set of locally finite perimeter, let $A' \subseteq A$ be open sets with Lipschitz boundary such that $|\partial^* U \cap A| \sim (\diam A)^{d - 1}$,
and let $P \in A'$. Then
\begin{equation}\label{approximate monotone}
-C |K| (\diam A)^{d + 1} \leq \Exc_{A'}(U, P) \leq \Exc_A(U, P) + C |K|(\diam A)^{d + 1}.
\end{equation}
The hypotheses of this lemma are met if $A$ is a ball and $|\partial U \cap A| \leq 2\eta(U, A)$.
\end{lemma}
\begin{proof}
To deduce the positivity, we compute for $\rho := \diam A'$ that
\begin{align*}
    \left|\int_{A'} \partial^P_\mu 1_U \star 1 \dif x_P^\mu(P)\right|
 & \leq \max_\mu ||\partial^P_\mu||_{C^0(A')} \cdot |\partial^* U \cap A'| \leq e^{CK\rho^2} |\partial^* U \cap A'| \\
 & \leq |\partial^* U \cap A'| + C|K|\rho^{d + 1}.
\end{align*}
The proof of monotonicity is similar.

If $A$ is a ball and $|\partial^* U \cap A| \leq 2\eta(U, A)$, then we appeal to Corollary \ref{doubling dimension}.
Since $A$ is small, there is an absolutely minimizing hypersurface $\partial^* V$ in $A$ such that $\partial^* U \cap \partial A = \partial^* V \cap \partial A$, so
$$\eta(U, A) = |\partial^* V \cap A| \sim (\diam A)^{d - 1}.$$
Since $|\partial^* U \cap A| \geq \eta(U, A)$, it follows that $|\partial^* U \cap A| \sim (\diam A)^{d - 1}$.
\end{proof}

%%%%%%%%%%%%%%%%%%%%%%%%%%%%%%%%%%%%%%

\subsection{Approximate symmetries of the excess} \label{translation appendix}
The excess of a set $U$ of locally finite perimeter is clearly invariant under gauge transformations; in the euclidean case, it is additionally true that $\Exc_A(U, P)$ is independent of $P$.
Here this is false because the curvature of $M$ implies that $\partial_\mu^P$ actually depends on $P$.
In fact, the failure of translation-invariance is the single most significant difference between the present paper and the classical paper of Miranda \cite{Miranda66}.
We substitute the translation-invariance with the following:

\begin{proposition}\label{translation invariance}
If $U$ is a set of least perimeter in an open set $A$ with Lipschitz boundary, and $P, Q \in A$, then
$$|\Exc_A(U, P) - \Exc_A(U, Q)| \leq C|K|(\diam A)^2 |\partial^* U \cap A|.$$
\end{proposition}

In the proof of Proposition \ref{translation invariance}, we will fill in the details in the case $K < 0$, as the case $K = 0$ is trivial and the case $K > 0$ is similar and strictly easier; we shall briefly remark on that later.

We first rapidly review the facts about the hyperboloid model that we will need.
For a more thorough discussion which uses many of the same ideas as the proof of Proposition \ref{translation invariance}, see \cite[\S3.1, \S4.1]{daskalopoulosPrep1}.
Let $\RR^{1, d} = \RR_t \times \RR_y \times \RR_x^{d - 1}$ be the Minkowski spacetime with its metric $-\dif t^2 + \dif y^2 + |\dif x|^2$.
Expressing $\Hyp^d$ in the Poincar\'e ball model, we obtain an embedding 
\begin{align*}
\Psi: \Hyp^d &\to \RR^{1, d} \\
x &\mapsto \frac{1}{1 - |x|^2/4 - y^2/4} \begin{bmatrix}1 + |x|^2/4 + y^2/4\\y \\ x\end{bmatrix}
\end{align*}
whose image is the unit future hyperboloid, as in the proof of \cite[Proposition 3.5]{lee1997riemannian}.
This embedding induces a split exact sequence of $\SpOrth^+(\RR^{1, d})$-bundles over $\Hyp^d$
\begin{equation}\label{splitting of tangent bundle}
0 \to T\Hyp^d \to \Hyp^d \times \RR^{1, d} \to N\Hyp^d \to 0
\end{equation}
where $N\Hyp^d$ is the normal bundle of the embedding $\Psi$ \cite[(3.4)]{daskalopoulosPrep1}.
Here $\SpOrth^+(\RR^{1, d})$ is the properly orthochronous Lorentz group.

\begin{lemma}
The splitting (\ref{splitting of tangent bundle}) induces canonical isomorphisms of $\SpOrth^+(\RR^{1, d})$-representations 
\begin{equation}\label{SES}
\RR^{1, d} = N'_P \Hyp^d \oplus T'_P \Hyp^d.
\end{equation}
for each $P \in \Hyp^d$.
Writing $\xi_O$ for the orthogonal projection of $\xi \in T'_P \Hyp^d \subseteq \RR^{1, d}$ to $T'_O \Hyp^d$,
\begin{equation}\label{Dask estimate}
|\xi| \leq |\xi_O| \leq e^{\dist(O, P)^2/2} |\xi|.
\end{equation}
\end{lemma}
\begin{proof}
We consider the adjoint sequence
$$0 \to N' \Hyp^d \to \Hyp^d \times \RR^{1, d} \to T' \Hyp^d \to 0$$
to (\ref{splitting of tangent bundle}).
Since (\ref{splitting of tangent bundle}) is split exact, we have a direct sum of $\SpOrth^+(\RR^{1, d})$-bundles $\Hyp^d \times \RR^{1, d} = T' \Hyp^d \oplus N' \Hyp^d$.
Here $\Hyp^d \times \RR^{1, d}$ is equipped with its trivial connection $\dif$ and has trivial monodromy group, so the parallel transport maps induce canonical isomorphisms between each fiber of $\Hyp^d \times \RR^{1, d}$ and $\RR^{1, d}$.
Therefore we have canonical isomorphisms (\ref{SES}).

Now (\ref{Dask estimate}) easily follows from \cite[\S4.1]{daskalopoulosPrep1}:\footnote{One may object that \cite[\S4.1]{daskalopoulosPrep1} refers to the splitting of the tangent bundle (\ref{splitting of tangent bundle}) rather than the cotangent bundle, but after conjugating by a musical isomorphism we see that the same estimates hold for the splitting of the cotangent bundle.}
By (\cite[(4.1)]{daskalopoulosPrep1}), $\xi_O$ is just $\xi$ with its timelike part $\xi_t$ set to $0$, thus 
$$|\xi_O|^2 = |\xi_x|^2 + |\xi_y|^2 \geq |\xi_x|^2 + |\xi_y|^2 - |\xi_t|^2 = |\xi|^2.$$
On the other hand, by \cite[Lemma 4.2, Lemma 4.1]{daskalopoulosPrep1},
\begin{align*}
|\xi_O| &\leq (1 + |P - O|^2/2) |\xi| \leq (1 + \rho^2 \cosh \rho/2) |\xi| \leq e^{\rho^2/2} |\xi|. \qedhere 
\end{align*}
\end{proof}

\begin{lemma}
The pushforward morphism $\Psi_*: T_{(x, y)} \Hyp^d \to T_{\Psi(x, y)} \RR^{1, d}$ is
\begin{equation}\label{pushforward estimates}
\Psi_* = \begin{bmatrix}y & x \\ 1 \\ & \id \end{bmatrix} + O(|x|^2 + y^2).
\end{equation}
\end{lemma}
\begin{proof}
We compute for $f(y, x) := |x|^2/4 + y^2/4$ that
\begin{align*}
\Psi_* &= \frac{1}{1 - f} \begin{bmatrix} \partial_y f & \partial_x f \\ 1 \\ & \id \end{bmatrix} + \frac{1}{(1 - f)^2} \begin{bmatrix}(1 + f) \partial_y f & (1 + f) \partial_x f \\
y \partial_y f & y \partial_x f \\
(\partial_y f) x & x \otimes \partial_x f
\end{bmatrix} \\
&= \begin{bmatrix}y & x \\ 1 \\ & \id \end{bmatrix} + O(|x|^2 + y^2)
\end{align*}
since $f(y, x) = O(|x|^2 + y^2)$ and $2\dif f(y, x) = (y, x)$.
\end{proof}

\begin{lemma}
Let $P \in \Hyp^d$ be given by $x^i = 0$, $y = \rho$. Consider the Lorentz boost
$$\Lambda := \begin{bmatrix}\cosh \rho & \sinh \rho \\ \sinh \rho & \cosh \rho \\ &&\id\end{bmatrix}$$
Up to a gauge transformation, it holds that for each $X \in \Hyp^d$, $Y := (\Phi^P)^{-1}(X)$, the below diagram commutes:
\begin{equation}\label{Lorentz boost diagram}
\begin{tikzcd}[column sep=50pt, row sep=30pt]
\RR^{1, d} \arrow[r, "\Lambda"] & \RR^{1, d} \\
T_Y \Hyp^d \arrow[u, "\dif \Psi(Y)"] \arrow[r, "\dif \Phi^P(Y)"] & T_X \Hyp^d \arrow[u, "\dif \Psi(X)"]
\end{tikzcd}
\end{equation}
\end{lemma}
\begin{proof}
Up to a gauge transformation, we may assume that $\Phi^P$ acts by hyperbolic translation along the $y$-axis.
On the other hand, $\Lambda \in \SpOrth^+(\RR^{1, d})$, so it preserves the unit future hyperboloid $\Psi(\Hyp^d)$ and acts by isometry.
Since $\Lambda$ clearly also preserves $\{(t, y, 0) \in \RR^{1, d}\}$, $\Lambda|\Psi(\Hyp^d)$ must act by hyperbolic translation on the image of the $y$-axis and hence the diagram
$$
\begin{tikzcd}
\RR^{1, d} \arrow[r, "\Lambda"] & \RR^{1, d} \\
\Hyp^d \arrow[u, "\Psi"] \arrow[r, "\Phi^P"] & \Hyp^d \arrow[u, "\Psi"]
\end{tikzcd}
$$
commutes. Linearizing this diagram at $Y$, we obtain (\ref{Lorentz boost diagram}).
\end{proof}

\begin{lemma}
Let $A \subseteq \Hyp^d$ be an open set containing $O, P$.
Then up to a gauge transformation,
\begin{equation}\label{difference of vector fields}
||\partial^P_\mu - \partial_\mu||_{C^0(A)} \lesssim (\diam A)^2.
\end{equation}
\end{lemma}
\begin{proof}
We may assume by applying an isometry that $x^i(P) = 0$.
Let $X \in A$; we prove $|\partial^P_\mu(X) - \partial_\mu(X)| \lesssim \varepsilon^2$ for $\varepsilon := \diam A$.
Let $Y = (\Phi^P)^{-1}(X)$, so that $\partial^P_\mu(X) = \dif \Phi^P(Y) \partial_\mu(Y)$.
For $Z \in \Hyp^d$, we assign $T_Z \Hyp^d$ the basis $(\partial_\mu(Z))$.
Then the operator $I: T_Y \Hyp^d \to T_X \Hyp^d$ given by $I \partial_\mu(Y) = \partial_\mu(X)$ is represented by the identity matrix, and
$$|\partial^P_\mu(X) - \partial_\mu(X)| \leq |\dif \Phi^P(Y) - I| = |\dif \Psi(X) \circ \dif \Phi^P(Y) - \dif \Psi(X) \circ I|.$$
Up to a gauge transformation, (\ref{Lorentz boost diagram}) commutes, so
$$|\partial^P_\mu(X) - \partial_\mu(X)| \leq |\Lambda - \dif \Psi(Y) - \dif \Psi(X) \circ I|.$$
If $Y = (x^*, y^*)$, then $X = (x^*, y^* + \rho) + O(\varepsilon^2)$ \cite[(4.5.5)]{ratcliffe2006foundations} since $X \in A$, so
\begin{align*}
\Lambda \circ \dif \Psi(Y) &= \begin{bmatrix}1 & \rho \\ \rho & 1 \\ && \id \end{bmatrix} \begin{bmatrix}y^* & x^* \\ 1 \\ & \id \end{bmatrix} + O(\varepsilon^2) = \begin{bmatrix}y^* + \rho & x^* \\ 1 \\ & \id\end{bmatrix} \begin{bmatrix}1 \\ & \id\end{bmatrix} + O(\varepsilon^2) \\
&= \dif \Psi(X) \circ I + O(\varepsilon^2). \qedhere
\end{align*}
\end{proof}

\begin{proof}[Proof of Proposition \ref{translation invariance} for $K < 0$]
We may assume by rescaling that $K = -1$, by locality that $M = \Hyp^d$, by applying an isometry that $Q = O$, and by applying a gauge transformation that (\ref{difference of vector fields}) holds.
Writing
$$v := \int_A \partial_\mu^P 1_U \star 1 \dif x^\mu_P(P), \quad w := \int_A \partial_\mu 1_U \star 1 \dif x^\mu(O)$$
and $\varepsilon := \diam A$, we observe that 
$$|\Exc_A(U, P) - \Exc_A(U, O)| = ||\partial^* U \cap A| - |v| - |\partial^* U \cap A| + |w|| = ||v| - |w||$$
so it suffices to show that
$$||v| - |w|| \lesssim \varepsilon^2 |\partial^* U \cap A|.$$
Using the splitting of (\ref{SES}) we can view both covectors $v \in T_P' \Hyp^d, w \in T_O' \Hyp^d$ as elements of $\RR^{1, d}$.

We first estimate
$$||v| - |v_O|| \leq \varepsilon^2 |v| \leq \varepsilon^2 |\partial^* U \cap A|$$
using (\ref{Dask estimate}) and the fact that $e^{\varepsilon^2/2} - 1 \leq \varepsilon^2$ for $\varepsilon < 1$.
Moreover, the reverse triangle inequality $||v_O| - |w|| \leq |v_O - w|$ is valid because $v_O, w$ are elements of the spacelike subspace $T'_O \Hyp^d$ of $\RR^{1, d}$, thus
$$||v| - |w|| \leq ||v| - |v_O|| + ||v_O| - |w|| \leq \varepsilon^2 |\partial^* U \cap A| + |v_O - w|.$$
Moreover,
\begin{align*}
|v_O - w| &= \left|\left[\int_A \partial^P_\mu 1_U \star 1\right] (\dif x_P^\mu(P))_O - \left[\int_A \partial_\mu 1_U \star 1\right] \dif x^\mu(O)\right| \\
&\leq \left[\int_A |\partial^P_\mu - \partial_\mu| \star |\dif 1_U|\right] \cdot |\dif x^\mu(O)| + \left|\int_A \partial^P_\mu 1_U \star 1\right| \cdot |(\dif x^\mu_P(P))_O - \dif x^\mu(O)|\\
&=: I + J
\end{align*}
It is clear that $I \leq \sum_\mu ||\partial^P_\mu - \partial_\mu||_{C^0(A)} \cdot |\partial^*U \cap A|$, which is $\lesssim \varepsilon^2 |\partial^* U \cap A|$ by (\ref{difference of vector fields}).

To estimate $J$, we bound
$$\left|\int_A \partial^P_\mu 1_U \star 1\right| \cdot |(\dif x^\mu_P(P))_O - \dif x^\mu(O)| \leq \sum_\mu \left[|(\dif x^\mu_P(P) - \dif x^\mu(P))_O| + |\dif x^\mu(P)_O - \dif x^\mu(O)|\right] |\partial^* U \cap A|.$$
From (\ref{Dask estimate}), the fact that $e^{\varepsilon^2/2} \leq 2$, a musical isomorphism, and (\ref{difference of vector fields}), we dispose of the first term as 
$$|(\dif x^\mu_P(P) - \dif x^\mu(P))_O| \leq 2 |\dif x^\mu_P(P) - \dif x^\mu(P)| \leq 2 ||\partial^P_\mu - \partial_\mu||_{C^0(A)} \lesssim \varepsilon^2.$$
Finally, we recall that $\dif x^\mu(P)_O$ is exactly the spacelike part of $\dif \Psi^\mu(P)$.
So as an element of $\RR^{1, d}$, $\dif x^\mu(P)_O$ is the $\mu$th column of the matrix in (\ref{pushforward estimates}) with the first (that is, timelike) row set to $0$, plus $O(\varepsilon^2)$.
Therefore $\dif x^\mu(P)_O = \dif x^\mu(O) + O(\varepsilon^2)$.
\end{proof}

Now we sketch the case $K > 0$.
The embedding $\Psi$ should be replaced with the inverse of the stereographic projection $\Sph^d \setminus \{\infty\} \to \RR^{d + 1}$.
It is easy to show that $\RR^{d + 1} = T'_P \Sph^d \oplus N'_P \Sph^d$ and $||\xi| - |\xi_O|| \lesssim \dist(O, P)^2 |\xi|$ whenever $\xi \in T_P \Sph^d$.
Here $O$ and $\infty$ are mutual antipodes, and there are no technicalities caused by an indefinite metric.
The pushforward formula (\ref{pushforward estimates}) still holds, and (\ref{Lorentz boost diagram}) holds with the Lorentz boost replaced by a rotation matrix.
If $Y = (x^*, y^*)$, then $\Phi^P(Y) = (x^*, y^* + \rho) + O(|x^*|^2 + |y^*|^2)$ still, since this just expresses the fact that translation in stereographic projection and spherical translation agree to second order.
Therefore (\ref{difference of vector fields}) holds and the rest of the proof is essentially identical.

%%%%%%%%%%%%%%%%%%%%%%%%%%%%%%%%%%%%%%

\subsection{Reduction to a de Giorgi lemma}
Following \cite{Miranda66,Giusti77,deGiorgi61}, it is natural to control the excess using the following analogue of the famous de Giorgi lemma \cite[Theorem 8.1]{Giusti77}:

\begin{proposition}[de Giorgi lemma]\label{de Giorgi}
There exist $C, c, \rho_* > 0$, such that for every $P \in M$, $\rho$ such that $0 < \rho < \rho_*$, and set $U \subset M$ of least perimeter such that 
$$\Exc_\rho(U, P) \leq c\rho^{d - 1},$$
we have 
\begin{equation}\label{dGL concl}
\Exc_{\rho/2}(U, P) \leq 2^{-d} \Exc_\rho(U, P) + C|K|\rho^{d + 1}.
\end{equation}
The constants $C, c, \rho_*$ only depend on $d$ for $|K| \lesssim 1$ (but may explode as $|K| \to \infty$).
\end{proposition}

\begin{proof}[Proof of Theorem \ref{main lma}, assuming the de Giorgi lemma]
Let $U$ be a set of least perimeter and $P \in \partial U$.
By Proposition \ref{blowup theorem}, $U$ has a nonsingular tangent cone at $P$ if $P \in \partial^* U$, and by Proposition \ref{locality of Caccioppoli}, $\partial^* U$ is dense in $\partial U$.
As on \cite[pg109]{Giusti77}, we see that $\Exc_\rho(U, P) \leq c\rho^{d - 1}$ for $\rho$ which is locally uniformly bounded from below.
So, reasoning identically to \cite[Chapters 8-9]{Giusti77}, we just need to show the following analogue of \cite[Theorem 8.2]{Giusti77}: with $c, \rho_*$ as in the de Giorgi lemma, if $0 < \rho < \rho_*$ such that $\Exc_\rho(U, P) \leq c\rho^{d - 1}$, then
\begin{equation}\label{LC Cauchy}
|\normal_U(P, r) - \normal_U(P, s)| \lesssim \sqrt{\frac{r}{\rho}}.
\end{equation}
Moreover, by \cite[pg100]{Giusti77}, we may assume that $r/2 < s < r = \rho/2^n$ for some $n$.

To fix notation, let
$\xi := \normal_U(P, r)$, $\eta = \normal_U(P, s)$, $m := |\partial^* U \cap B(P, s)|$, $M := |\partial^* U \cap B(P, r)|$, and $\gamma_n := \Exc_{\rho/2^n}(U, P)$.
We first estimate
$$|\xi - \eta|^2 = |\xi|^2 + |\eta|^2 - 2 g^{-1}(\xi, \eta) \leq 2(1 - g^{-1}(\xi, \eta)) + C|K|r^2.$$
To estimate the right-hand side we write 
$$m(1 - g^{-1}(\xi, \eta)) = \int_{B(P, s)} \star(|\dif 1_U| - g^{-1}(\xi, \dif x^\mu_P(P)) \partial^P_\mu 1_U).$$
Now we bound 
$$ g^{-1}(\xi, \dif x^\mu_P(P)) \partial^P_\mu 1_U \leq |\xi| \cdot |\dif 1_U| \cdot \max_\mu |\partial^P_\mu| \leq e^{C|K|r^2} |\dif 1_U|,$$
which implies that, since $s \leq r$,
$$\int_{B(P, s)} \star(|\dif 1_U| - g^{-1}(\xi, \dif x^\mu_P(P)) \partial^P_\mu 1_U) \leq \int_{B(P, r)} \star(e^{C|K|r^2} |\dif 1_U| - g^{-1}(\xi, \dif x^\mu_P(P)) \partial^P_\mu 1_U).$$
By Corollary \ref{doubling dimension}, the first term integrates to $\leq M + C|K|r^{d + 1}$, and so by definition of $\xi$,
\begin{align*}
\int_{B(P, r)} \star(e^{CKr^2} |\dif 1_U| - g^{-1}(\xi, \dif x^\mu_P(P)) \partial^P_\mu 1_U) &\leq M(1 - |\xi|^2) + C|K|r^{d + 1}\\
&\leq 2M(1 - |\xi|) + C|K|r^{d + 1}.
\end{align*}
But $M(1 - |\xi|)$ is exactly the definition of $\Exc_r(U, P)$, so putting everything together and using Corollary \ref{doubling dimension} to bound $m \gtrsim s^{d - 1} \gtrsim r^{d - 1}$, we have the bound 
\begin{equation}\label{just need the excess}
|\xi - \eta|^2 \lesssim r^{1 - d} \Exc_r(U, P) + |K|r^2.
\end{equation}
Then, since $r = \rho(V)/2^n$, $\Exc_r(U, P) = \gamma_n$, and for $C' := C|K|\rho^{d + 1}$ we have the recurrence relation 
$$\gamma_{k + 1} \leq \frac{\gamma_k}{2^d} + \frac{C'}{2^{k{d + 1}}},$$
by the de Giorgi lemma.
By induction, then,
$$\gamma_n \leq \frac{\gamma_0}{2^{nd}} + C' \sum_{k=0}^n \frac{1}{2^{k(d + 1)} 2^{(n - k)d}} = \frac{{\gamma_0}}{2^{nd}} + C' \frac{2^{n + 1} - 1}{2^{n(d + 1)}} \leq \frac{c\rho^{d - 1} + 2C'}{2^{nd}}.$$
Combining this estimate with (\ref{just need the excess}), we deduce (\ref{LC Cauchy}).
\end{proof}



% \subsection{Euclidean minimal graphs}
% Before proving the de Giorgi lemma, we need to recall some facts about the Plateau equation for minimal hypersurfaces in $\RR^d$.
% We remind the reader that $\Japan\xi$ denotes the Japanese norm $\sqrt{1 + |\xi|^2}$.

% If $v$ is a function on $\RR^{d - 1}$, then we can use diffeomorphism $x \mapsto (x, v(x))$ to impose coordinates on the graph $N$ of $v$; in these coordinates the metric tensor on $N$ can be easily seen to be $1 + \dif v \otimes \dif v$.
% Then by \cite[(24)]{Petersen2008}, the surface area of $N$ in $\Omega \times \RR$ is $\int_\Omega \Japan{\dif v} \dif x$, where $\dif x := \dif x^1 \wedge \cdots \wedge \dif x^{d - 1}$.
% The Lagrangian $\Japan{\dif v} \dif x$ is invariant under the scaling
% \begin{equation}\label{rescaled solution}
% v_\lambda(x) := v(\lambda x)/\lambda.
% \end{equation}

% As in \cite{deGiorgi61, Miranda66}, de Giorgi's lemma is based on the below key estimate.
% To state it, we define for a $1$-form $\xi: \RR^{d - 1} \to (\RR^{d - 1})'$ and $\rho > 0$, the $1$-form $\avg_\rho \xi$ which acts on vectors $v \in T_x\RR^{d - 1} = \RR^{d - 1}$ by
% \begin{equation}\label{definition of averaging}
% (\avg_\rho \xi(x), v) := \dashint_{B_\rho} (\xi(x), v) \dif x
% \end{equation}
% where $B_\rho \subset \RR^{d - 1}$.\footnote{(\ref{definition of averaging}) relies on the flatness of the Levi-Civita connection of $\RR^{d - 1}$ to make the identification $T'\RR^{d - 1} = \RR^{d - 1} \times (\RR^{d - 1})'$, so that we cannot replace $\RR^{d - 1}$ with a curved space in this definition.}
% Now the below lemma follows from \cite[Lemma 4.2]{Miranda66} and the scale-invariance (\ref{rescaled solution}):

% \begin{lemma}[de Giorgi lemma, euclidean minimal graphs]\label{Miranda42 quant}
% For every $c_0 > 0$ small there exists $c_1 = c_1(d, c_0) > 0$ such that the following holds.
% Let $\beta, \rho > 0$, $0 < \alpha < 1$, $w \in C^1(B_\rho)$, and let $u$ be the harmonic function on $B_\rho$ with $u|\partial B_\rho = w|\partial B_\rho$.
% If $||\dif w||_{C^0} \leq c_1$,
% \begin{align}
% \int_{B_\rho} \Japan{\dif w} - \Japan{\avg_\rho \dif w} \dif x &\leq \beta, \label{Miranda42 hyp} \\
% \int_{B_\rho} \Japan{\dif w} - \Japan{\dif u} \dif x &\leq c_1 \beta,
% \end{align}
% then 
% \begin{equation}\label{Miranda42 concl}
% \int_{B_{\alpha \rho}} \Japan{\dif w} - \Japan{\avg_{\alpha \rho} \dif w} \dif x \leq (\alpha^{d + 1} + c_0) \beta.
% \end{equation}
% \end{lemma}


\subsection{Proof of the de Giorgi lemma}
We set up the proof of the de Giorgi lemma by establishing some conventions.
We require that any implied constant or constant called $C$ must only depend on $d$ in the limit $K \to 0$.
We fix $c_0 = c_0(d) \in (0, 1)$ to be a small constant to be chosen later.

We introduce the Lagrangian
$$\Lagrange(y, \xi) := \frac{\Japan{\xi}}{(1 + K(|x|^2 + y^2)/4)^{d - 1}} \dif x,$$
for $x \in \RR^{d - 1}$, $y \in \RR$, and $\xi \in T'_x \RR^{d - 1}$.
Notice that for $K = 0$ this Lagrangian reduces to the usual minimal surface equation, as studied in \cite[\S6]{Giusti77}.

We first show that critical points of $w \mapsto \Lagrange(w, \dif w)$ define minimal surfaces in $M$.
To this end we define for $w \in C^1(\Omega)$ and $\Omega \subseteq \RR^{d - 1}$ open the map $\Psi_w: \Omega \to M$ given by 
$$\Psi_w(x)^i := x^i, \quad \Psi_w(x)^0 := w(x).$$

\begin{lemma}\label{Plateau setup lemma}
Let $w \in C^1(\Omega)$. Then $(\Psi_w^{-1})^* \Lagrange(w, \dif w)$ is the Riemannian measure on $\Psi_w(\Omega)$.
\end{lemma}
\begin{proof}
For $(\partial_i)$ the standard frame on $\Omega$, the metric on the image of $\Psi_w$ satisfies
\begin{align*}
\Psi_w^*(g|N_w)(\partial_i, \partial_j) &= g_{\mu\nu} \partial_i \Psi_w^\mu \partial_j \Psi_w^\nu \\
&= \frac{\delta_{\mu\nu}}{(1 + K(|x|^2 + w^2)/4)^2} (\partial_i \Psi_w^\mu \partial_j \Psi_w^\nu) \\
&= \frac{\delta_{ij} + \partial_i w \partial_j w}{(1 + K(|x|^2 + w^2)/4)^2}.
\end{align*}
By \cite[(24)]{Petersen2008} we have $\det((\delta_{ij} + \partial_i w \partial_j w)_{ij}) = 1 + |\dif w|^2$.
\end{proof}

We can view the derivative of a function $w: \Omega \to \RR$ as a map $\dif w: \Omega \to \RR^{d - 1}$.
In particular, it is well-defined to take the average $\avg_\rho \dif w$ of $\dif w$ over the ball $B_\rho := \{x \in \RR^{d - 1}: |x| < \rho\}$.
Moreover, $\Lagrange(w, \dif w) - \Lagrange(w, \avg_\rho \dif w)$ is close to the analogous quantity for the euclidean case if $\rho$ is small:

\begin{lemma}
Let $\beta, \rho > 0$, $w \in C^1(B_\rho)$, $w(0) = 0$, and $||\dif w||_{C^0} \leq 1$. Then
\begin{align}
\int_{B_\rho} \Lagrange(w, \dif w) - \Lagrange(w, \avg_\rho \dif w) &= \int_{B_\rho} \Japan{\dif w} \dif x - \Japan{\avg_\rho \dif w} \dif x + O(|K| \rho^{d + 1}) \label{compare Lagrangians}, \\
\int_{B_\rho} \Lagrange(w, \dif w) - \Lagrange(w, \avg_\rho \dif w) &= e^{O(c_0)} \Exc_{B(P, \rho)} (U, \Psi_w(0)) + O(|K| \rho^{d + 1}). \label{compare Lagrangians 2}
\end{align}
\end{lemma}
\begin{proof}
The left-hand side of (\ref{compare Lagrangians}) is
\begin{align*}
&\left|\int_{B_\rho} ((1 + K(|x|^2 + w(x)^2)/4)^{1 - d} - 1)(\Japan{\dif w} - \Japan{\avg_\rho \dif w}) \dif x\right| \\
&\qquad \lesssim_d |K| \int_{B_\rho} (|x|^2 + w(x)^2) \cdot \left|\Japan{\dif w} - \Japan{\avg_\rho \dif w}\right| \dif x.
\end{align*}
But $|B_\rho| \sim \rho^{d - 1}$, $|\Japan{\dif w} - \Japan{\avg_\rho \dif w}| \lesssim \exp(||\dif w||_{C^0}) \lesssim 1$,
\begin{align*}
(|x|^2 + w(x)^2) &\leq (1 + ||\dif w||_{C^0}^2) \rho^2 \lesssim \rho^2.
\end{align*}
This proves (\ref{compare Lagrangians}). Taylor expanding the definition of the excess we see that up to a term of size $O(|K| \rho^{d + 1})$, we can replace the right-hand side of (\ref{compare Lagrangians 2}) with its euclidean analogue, which by \cite[pg83]{Giusti77} is equal to the right-hand side of (\ref{compare Lagrangians}), proving (\ref{compare Lagrangians 2}).
\end{proof}

We are now ready to prove an analogue of \cite[Lemma 6.3]{Giusti77}, which is the de Giorgi lemma for a minimal graph near $O$.
Thanks to (\ref{compare Lagrangians}), the proof is essentially the same as the euclidean case.

\begin{lemma}[de Giorgi lemma, minimal graphs]\label{Miranda43}
If $c_0 \ll_d 1$ then there exists $c_1 = c_1(c_0) > 0$ such that for every $\beta > 0$, $0 < \rho < 1$, and $w \in C^1(B_\rho)$, if we denote by 
$I_w$ the cylinder in $M$, $I := \{|x| < \rho\}$, and assume that $w(0) = 0$, $||\dif w||_{C^0} \leq c_1$, and
\begin{align}
\int_{B_\rho} \Lagrange(w, \dif w) - \Lagrange(w, \avg_\rho \dif w) &\leq \beta \label{Miranda43 oscillation}, \\
\int_{B_\rho} \Lagrange(w, \dif w) &\leq \eta(N_w, I) + c_1 \beta \label{Miranda43 minimality},
\end{align}
then for any $0 < \alpha < 1$,
\begin{equation}\label{Miranda43 concl}
\int_{B_{\alpha \rho}} \Lagrange(w, \dif w) - \Lagrange(w, \avg_{\alpha \rho} \dif w) \leq (\alpha^{d + 1} + c_0) \beta + C|K|\rho^{d + 1}.
\end{equation}
\end{lemma}
\begin{proof}
Let $u$ be the harmonic function on $B_\rho$ which equals $w$ on $\partial B_\rho$.
By elliptic regularity, the maximum principle for harmonic functions, and (\ref{Miranda43 minimality}),
$$||u||_{C^1} \lesssim ||u||_{C^0} \leq ||w||_{C^0} \leq \rho ||\dif w||_{C^1} \leq c_1.$$
In particular, $\Japan{\dif u} \lesssim 1$ and $u(x)^2 \lesssim \rho^2$, so
\begin{align*}
&\left|\int_{B_\rho} \Lagrange(w, \dif w) - \Lagrange(u, \dif u) - \Japan{\dif w} \dif x + \Japan{\dif u} \dif x\right| \\
&\qquad \leq |K| \int_{B_\rho} (|x|^2 + w(x)^2) \Japan{\dif w} \dif x + (|x|^2 + u(x)^2) \Japan{\dif u} \dif x 
\lesssim_d |K| \rho^{d + 1}.
\end{align*}
Since $u$ and $w$ have the same trace, $|N_u \cap I| \leq \eta(N_w, I)$, thus by Lemma \ref{Plateau setup lemma},
\begin{align*}
\int_{B_\rho} \Lagrange(w, \dif w) - \Lagrange(u, \dif u) &\leq \int_{B_\rho} \Lagrange(w, \dif w) - \eta(N_w, I) \leq c_1 \beta.
\end{align*}
Moreover, we meet the hypotheses of (\ref{compare Lagrangians}).
We can moreover replace $\beta$ with some $\beta' \in [\beta, \beta + C|K|\rho^{d + 1}]$ so that $u, w$ meet the hypotheses of \cite[Lemma 4.2]{Giusti77} which gives the result for $c_0 \ll_d 1$.
\end{proof}

In order to apply Lemma \ref{Miranda43} we must show that we can, after applying a suitable isometry, bound a set $U$ of locally finite perimeter by a set $\Psi_w(\Omega)$ for some suitable $w, \Omega$.
To do so we wish to appeal to \cite[Theorem 4.8]{Giusti77}, which requires that we check the below convexity condition.

\begin{lemma}\label{convex balls}
Let $B \subset M$ be a ball such that $O \in B$.
If $\diam B \lesssim |K|^{-1/2}$, then $B$ appears convex in the coordinates from the Poincar\'e ball model (for $K < 0$) or the stereographic projection (for $K > 0$).
\end{lemma}
\begin{proof}
We may rescale $K$ so that
$$\dist(P, Q) = \mathrm{arsinh} \sqrt{\delta(P, Q)}, \qquad \delta(P, Q) := \frac{|P - Q|^2}{(1 \pm |P|^2)(1 \pm |Q|^2)}$$
where $\pm = \sgn K$ and $|\cdot|$ is the euclidean norm.
In particular, if $B = B(Q, r)$, $\delta := \delta(\cdot, Q)$, and $\delta_* := \sinh^2 r$, then $\partial B$ is cut out by the equation $\delta = \delta_*$.
Therefore $B$ appears convex in coordinates if its second fundamental form $\Two = \nabla(\dif \delta/|\dif \delta|)$ is positive-definite (where the second fundamental form, Levi-Civita connection, and cometric are all euclidean).
For $Q$ fixed, $\dif \delta$ is proportional to 
$$\xi := (1 \pm |P|^2)^{-1} \left[\left(1 + \frac{|P - Q|^2}{1 \pm |P|^2}\right)P - Q\right] = P - Q + O(|P|^2 + |Q|^2),$$
and $\nabla \xi = \id + O(|P| + |Q|)$.
In particular,
\begin{align*}
\Two &= \nabla\frac{\xi}{|\xi|} = \frac{\nabla \xi}{|\xi|} - \frac{\xi \otimes \nabla|\xi|}{|\xi|^2} \\
&= |\xi|^{-1} \left[\nabla \xi - \frac{\xi \otimes \tr(\xi \otimes \nabla \xi)}{|\xi|}\right] \\
&= |\xi|^{-1} [\id + O(|\xi| + |P| + |Q|)] = |\xi|^{-1} [\id + O(r)]
\end{align*}
which is positive-definite for $r \ll 1$.
\end{proof}

We are now ready to represent a set in $M$ as a function.
We fix $\alpha = \alpha(c_0)$ to be chosen later.
Thanks to Lemma \ref{convex balls}, the proof of the next lemma is essentially the same as the construction of the sequence of functions $(\omega_j)$ in the proof of \cite[Lemma 6.4]{Giusti77}, but without the wholly unnecessary appeal to proof by contradiction.

\begin{lemma}\label{rep as a good graph}
For every $\rho \ll_{c_0, c_1, d} |K|^{-1}$ and every set $U$ of $C^1$ perimeter in $B(P, \rho)$ such that $O \in B(P, \rho)$,
\begin{align}
\star(\normal_U \wedge \psi) &\geq e^{-o(c_1^2)} |\psi| \label{rep as a good graph hyp}
\end{align}
there exists an open set $\Omega \subset \RR^{d - 1}$, and $w \in C^1(\Omega)$, such that $\partial U = \{y = w\}$,
\begin{equation}\label{rep as a good graph small derivative}
||\dif w||_{C^0} \leq c_1.
\end{equation}
If $\Omega^\alpha := \{x \in \Omega: (x, w(x)) \in B(P, \alpha \rho)\}$ is nonempty, then there exists a ball $V \subset \RR^{d - 1}$ centered on $0$ such that
\begin{equation}\label{rep as a good graph set nests}
    \Omega^\alpha \subseteq V \subset (\alpha - c_0) V \subseteq \Omega.
\end{equation}
\end{lemma}
\begin{proof}
Since $\dist(O, P) \leq \rho$, we obtain by Taylor expanding the Hodge star for $\normal := \normal_U$ that 
$$\normal_0 \geq \frac{1 - O(|K|\rho^2)}{|\psi|} \star (\normal \wedge \psi) \geq e^{-o(c_1^2)}.$$
Moreover, by Lemma \ref{convex balls}, $B(P, \rho)$ appears convex in coordinates, so by \cite[Theorem 4.8]{Giusti77} there exists an open set $\Omega \subset \RR^{d - 1}$ and $w \in C^1(\Omega)$ such that $U$ is bounded by $\{y = w\}$ and
$$||\dif w||_{C^0} \leq \sup_{x_1, x_2 \in \Omega} \frac{|w_n(x_1) - w(x_2)|}{|x_1 - x_2|} \leq e^{o(c_1^2)}\sqrt{1 - e^{-o(c_1^2)}} \leq c_1.$$
Therefore (\ref{rep as a good graph small derivative}) holds.
We begin the proof of (\ref{rep as a good graph set nests}) by letting $(x_0, y_0) := P$.

\begin{claim}
Let $r \leq \rho$, $\Omega' := \{x \in \Omega: (x, w(x)) \in B(P, r)\}$, and
$$S := \left\{x \in \RR^{d - 1}: \exists y \in \left[\inf_{\Omega'} w, \sup_{\Omega'} w\right] \text{ such that } (x, y) \in B(P, r)\right\}.$$
If $\Omega'$ is nonempty, then $\sigma^+ := \max_{x \in S} |x - x_0|$ and $\sigma^- := \min_{x \in S} |x - x_0|$ satisfy 
$$\sigma^\pm(r) = \sqrt{r^2 - (\inf_{\Omega'} w - y_0)^2} + O(c_1  \rho) + O(|K| \rho^2).$$
\end{claim}
\begin{proof}[Proof of claim]
Since $\Omega'$ is nonempty, so is $S$. But $\diam \Omega' \lesssim r \Japan K$.
Writing $w^+ := \sup_{\Omega'} w$ and $w^- := \inf_{\Omega'} w$,
$$w^+ \leq w^- + ||\dif w||_{C^0} \diam \Omega' \leq w^- + O(c_1 \rho \Japan K).$$
Up to terms of size $O(|K| \rho^2)$ arising from the metric, we may assume that $B(P, r)$ is a euclidean ball and that $\sigma^\pm$ are the radii of disks centered on the $y$-axis of $B(P, r)$ with $y$-coordinates in $[w^-, w^+]$.
The claim now follows from the Pythagorean theorem and the fact that $\Japan K \lesssim 1$.
\end{proof}

Applying the claim first when $r := \rho$, we see that $\Omega$ contains $B_{\sigma^-(\rho)}$.
On the other hand, applying the claim when $r := \alpha \rho$, we see that $\Omega^\alpha$ is contained in $B_{\sigma^+(\alpha \rho)}$.
Since 
$$\inf_{\Omega^\alpha} w \leq \sup_\Omega w \leq \inf_\Omega w + O(c_1 \rho)$$
and $|K| \rho^2 \leq c_1 \rho$ for $\rho \ll |K|^{-1}$, we have for $v := \inf_\Omega w - y_0$ that
$$\sigma(\alpha \rho)^2 = \alpha^2\left(\rho^2 - \frac{v^2}{\alpha^2}\right) + O(c_1^2 \rho^2) \leq \alpha^2(\rho^2 - v^2) + O(c_1^2 \rho^2) \leq \alpha^2 \sigma(\rho)^2 + O(c_1^2 + \rho^2).$$
In conclusion $\Omega^\alpha$ is contained in $B_{\sigma(\alpha \rho)} \subseteq B_{\alpha \sigma(\rho) + O(c_1 \rho)}$ while $\Omega$ contains $B_{\sigma(\rho)}$.
For $\rho$ small, $c_1 \rho \ll c_0$.
\end{proof}

The above setup allows us to show the $C^1$ case of the de Giorgi lemma, an analogue of \cite[Lemma 6.4]{Giusti77}.
We cannot quite proceed as in \cite{Giusti77} however as we need to carefully apply the gauge invariance and approximate translation invariance of the excess.

\begin{lemma}[de Giorgi lemma, $C^1$ case]\label{Miranda44}
Let $0 < \rho \ll_{c_1, d} |K|^{-1}$ and $0 < \beta \lesssim_{c_0, d} 1$, and let $U$ be a set of $C^1$ perimeter in $B(P, \rho)$ such that $\Exc_\rho (U, P) \leq \beta$. Assume that
\begin{align}
\star(\normal_U \wedge (\Phi^P)^* \psi) &\geq (1 - o(c_1^2)) |(\Phi^P)^* \psi|, \label{Miranda44 normal hyp} \\
|\partial^* U \cap B(P, \rho)| &\leq \eta(U, B(P, \rho)) + c_1 \beta. \label{Miranda44 minimality hyp}
\end{align}
Then
\begin{equation}\label{Miranda44 concl}
\Exc_{\alpha \rho} (U, P) \leq (\alpha^{d + 1} + c_0) \beta + C|K|\rho^{d + 1}.
\end{equation}
\end{lemma}
\begin{proof}
Throughout this proof we assume that $\partial U \cap B(P, \alpha \rho)$ is nonempty.
If not, then (\ref{Miranda44 concl}) is vacuous since then $\Exc_{\alpha \rho} (U, P) = 0$.
Taylor expanding (\ref{Miranda44 normal hyp}) and applying \cite[Theorem 4.8]{Giusti77}, we see that $\partial U$ is the graph of a function and hence there is a unique $Q \in B(P, \rho)$ such that 
$$\{Q\} = \partial U \cap \{x_P^1 = \cdots = x_P^{d - 1} = 0\}.$$
The hypotheses and conclusion of this lemma are preserved by gauge transformations $\chi$ such that $\chi^P = \id$, so after applying an appropriately chosen $\chi$, we may assume that $\Phi^{PQ} := \Phi^Q \circ (\Phi^P)^{-1}$ is a hyperbolic translation along the geodesic containing $P, Q$. In that case, since $\dist(P, Q) < \rho$ and $|K| \rho \ll c_1$, (\ref{Miranda44 minimality hyp}) gives
\begin{align*}
\star(\normal_U \wedge (\Phi^Q)^* \psi) &\geq \star(\normal_U \wedge (\Phi^{PQ})^* (\Phi^P)^* \psi) + O(|K| \rho^2) \geq \star (\normal_U \wedge (\Phi^P)^* \psi) + O(|K| \rho) \\
&\geq (1 - o(c_1^2)) |(\Phi^P)^* \psi| \geq (1 - o(c_1^2)) |(\Phi^Q)^* \psi|.
\end{align*}
Therefore by Lemma \ref{rep as a good graph}, there exist $V, \Omega \subset \RR^d$ satisfying (\ref{rep as a good graph set nests}), and $w: \Omega \to \RR$ such that $||\dif w||_{C^0} \leq c_1$, and the image of $\Phi^Q \circ \Psi_w$ is
$$\Gamma := \partial U \cap \{(x_Q^1, \dots, x_Q^{d - 1}) \in \Omega\}.$$
Since $Q \in \Gamma$, $w(0) = 0$.

Let $V' := (\alpha - c_0) V$, and for $W \subseteq \RR^{d - 1}$ open, introduce the cylinder $I_W := \{(x^1_Q, \dots, x^{d - 1}_Q) \in W\}$.
By Proposition \ref{translation invariance} and (\ref{approximate monotone}),
$$\tilde \beta := \Exc_{I_{V'}} (U, Q) \leq \Exc_{B(P, \rho)} (U, Q) + C|K| \rho^{d + 1} \leq \beta + 2C|K| \rho^{d + 1}.$$
Replacing $\tilde \beta$ with $\max(\tilde \beta, \beta)$ as necessary, we see from (\ref{Miranda44 minimality hyp}) that
$$|\partial^* U \cap I_{V'}| \leq \eta(U, I_{V'}) + c_1 \tilde \beta.$$
Replacing $\tilde \beta$ with $e^{O(c_0)} \tilde \beta$ we have by Lemma \ref{Miranda43}, (\ref{compare Lagrangians 2}), Proposition \ref{translation invariance}, and (\ref{approximate monotone}) that 
\begin{align*}
\Exc_{B(P, \alpha \rho)} (U, P) &\leq \Exc_{B(P, \alpha \rho)} (U, Q) + C|K| \rho^{d + 1} \leq \Exc_{I_V} (U, Q) + C|K| \rho^{d + 1}\\
&\leq \left[\frac{1}{(\alpha + c_0)^{d + 1}} + c_0\right] \tilde \beta + C|K| \rho^{d + 1} \leq \left[\frac{1}{\alpha^{d + 1}} + c_0\right] \tilde \beta + C|K| \rho^{d + 1} \\
&\leq \left[\frac{1}{\alpha^{d + 1}} + c_0\right] \beta + C|K| \rho^{d + 1}. 
\qedhere 
\end{align*}
\end{proof}

We now use a compactness argument to apply Proposition \ref{mollifier quant} and complete the proof of the de Giorgi lemma.
The proof from this point onwards is essentially identical to the euclidean case.

\begin{lemma}\label{single mollify}
There exists $c = c(d, K, c_0, c_1) > 0$ and $r = r(d, K, c_0, c_1) > 0$ such that for any ball $B(P, \rho)$ such that $\rho < r$ and any set $U$ of least perimeter in $B(P, \rho)$ such that
$$\Exc_\rho (U, P) \leq c\rho^{d - 1},$$
there exists a set $V$ of $C^1$ perimeter in $B(P, (1 - c_0)\rho)$ such that
\begin{align}
\star(\normal_V \wedge (\Phi^P)^* \psi) &\geq (1 - o(c_1^2))|(\Phi^P)^* \psi|, \label{single mollify normal}\\
|\partial V \cap B(P, (1 - c_0)\rho)| &\leq \eta(V, B(P, (1 - c_0)\rho)) + c_1 \Exc_\rho (U, P) + C|K| \rho^{d + 1}, \label{single mollify minimality}
\end{align}
and for $0 < \varpi \leq (1 - c_0)\rho$,
\begin{equation}
|\Exc_\varpi (U, P) - \Exc_\varpi (V, P)| \leq c_0 \Exc_\rho (U, P). \label{single mollify excess}
\end{equation}
\end{lemma}
\begin{proof}
If not, then there exist balls $B_n := B(P_n, \rho_n)$ and sets $U_n$ of least perimeter in $B_n$ such that
\begin{equation}\label{single mollify Excess assumption}
\Exc_{\rho_n} (U_n, P_n) \leq n^{-2} \rho_n^{d - 1},
\end{equation}
and $\rho_n \leq 1/n$, but such that for every set $V_n$ of $C^1$ perimeter in $B(P_n, (1 - c_0) \rho_n)$, at least one of (\ref{single mollify normal}), (\ref{single mollify minimality}), or (\ref{single mollify excess}) fail.
Applying an isometry, we may assume that $P_n = O$, and after applying a gauge transformation $\chi_n$ we may assume that
$$\Exc_{\rho_n} (U_n, P) = \int_{B_n} \star (|\dif 1_{U_n}| - \partial_0 1_{U_n}).$$
Moreover, if $\psi'$ is the vertical $d-1$-form induced by normal coordinates based at $P$, then $\partial_0 1_{U_n} = \dif 1_{U_n} \wedge \psi' + O(\rho_n^2)$, thus by (\ref{single mollify Excess assumption}),
$$\gamma_n := \rho_n^{1 - d} \int_{B(P, \rho_n)} |\psi'| \cdot |\dif 1_{U_n}| - \dif 1_{U_n} \wedge \psi' \lesssim n^{-2}.$$
In particular $(\gamma_n) \in \ell^1$, so by Proposition \ref{mollifier quant}, for almost every $t \in (1 - c_0, 1)$ there exist sets $V_n$ of $C^1$ perimeter in $B(P, t\rho_n)$ such that (\ref{mollifier quant1}), (\ref{mollifier quant2}), (\ref{mollifier quant4}), and (\ref{mollifier quant3}) hold, but at least one of (\ref{single mollify normal}), (\ref{single mollify minimality}), or (\ref{single mollify excess}) fail.
Since $t > 1 - c_0$, we can restrict to consideration of $B(P, (1 - c_0) \rho_n)$, in which case (\ref{mollifier quant4}) implies (\ref{single mollify normal}) and (\ref{mollifier quant1}) implies (\ref{single mollify minimality}).
Therefore after taking a subsequence, (\ref{single mollify excess}) fails for every $n$ and some $\varpi$, that is,
\begin{equation}\label{single mollify Excess assumption 2}
|\Exc_\varpi (U, P) - \Exc_\varpi (V, P)| > \frac{c_0}{2} \gamma_n \rho_n^{d - 1}.
\end{equation}
But plugging in $\omega := \star \dif x^\mu$ into (\ref{mollifier quant3}), we conclude that
$$\left|\int_{B(P, \varpi)} \star \partial_\mu (1_{U_n} - 1_{V_n})\right| \ll \rho_n^{d - 1} \gamma_n.$$
Combined with (\ref{mollifier quant2}) and (\ref{single mollify Excess assumption}) this contradicts the failure of (\ref{single mollify Excess assumption 2}).
\end{proof}

\begin{proof}[Proof of the de Giorgi lemma]
Let $c_0 \leq 2^{-(d + 1)}/10$ be small enough to meet the hypotheses of Lemma \ref{Miranda43}, let $\alpha := 1/(2(1 - c_0))$, and let $V$ be the set obtained from Lemma \ref{single mollify}.
By (\ref{single mollify excess}) and (\ref{approximate monotone}), for $\beta := \Exc_{B(P, \rho)} U$,
\begin{align*}
\Exc_{(1 - c_0) \rho} (V, P) &\leq \Exc_{(1 - c_0) \rho} (U, P) + c_0 \Exc_\rho (U, P) + C|K| \rho^{d + 1} \leq (1 + c_0) \Exc_\rho (U, P) + C |K| \rho^{d + 1} \\
&\leq (1 + c_0) \beta + C |K| \rho^{d + 1} =: \beta'.
\end{align*}
So by Lemma \ref{Miranda44} and our choice of $\alpha$,
\begin{align*}
\Exc_{\rho/2} (V, P) &\leq (2^{-(d + 1)} (1 - c_0)^{-(d + 1)} + c_0) \beta' + C |K| \rho^{d + 1} \\
&\leq (2^{-(d + 1)} + 5c_0) \beta + C |K| \rho^{d + 1}.
\end{align*}
Applying (\ref{single mollify excess}) again, we obtain $\Exc_{\rho/2} (U, P) \leq (2^{-(d + 1)} + 7 c_0) \beta + C |K| \rho^{d +1}$ which implies (\ref{dGL concl}).
\end{proof}


%%%%%%%%%%%%%%%%%%%%%%%%%%%%%%%%%%%%
\section{Applications of regularity}\label{GornySec}

\subsection{The maximum principle}\label{Max Princip}
We now prove Theorem \ref{main thm}, following \cite[pg7]{górny2017planar}.
Let $u$ have least gradient, $A_y = \partial \{u > y\}$, and $\lambda := \bigcup_{y \in \RR} A_y$.
Since $\lambda = \supp \dif u$, $\lambda$ is closed.
The sets $\{u > y\}$ are totally ordered by $\subseteq$, so either $A_y = A_{y'}$ or $A_y \cap A_{y'} = \emptyset$.
By Proposition \ref{level sets are minimal}, $\{u > y\}$ has least perimeter, so by Theorem \ref{main lma}, $A_y$ is a disjoint union of $C^\infty$ minimal hypersurfaces.
Moreover, by Corollary \ref{doubling dimension}, $A_y$ can only have locally finitely many connected components (or else it would have infinite measure, since each component has measure $\gtrsim r^{d - 1}$ in a ball of radius $r$).

We now have to address the special case that $\overline M$ is a surface-with-boundary; we need to show that $\lambda$ extends to a geodesic lamination of $\overline M$.
If this fails, then there are geodesics $\gamma_1, \gamma_2$ in $\lambda$ which intersect on $\partial M$, say at $P \in \partial M$.
Then there is a contractible neighborhood of $P$ in $\overline M$ which contains a geodesic triangle $\Delta$, so that two of the sides of $\Delta$ are $\gamma_1, \gamma_2$, and the third side is strictly contained in $M$.
Reasoning identically to the proof of \cite[Proposition 3.5]{górny2017planar}, we see that this contradicts that $u$ has least gradient.

Now we address the converse.
Let $\lambda$ be a minimal lamination and $u \in BV_\loc(M)$ have superlevel sets which are all bounded by leaves of $\lambda$.
Moreover, let $U \Subset M$ and let $v \in BV_\cpt(U)$, thus $V = U \setminus \supp v$ is the intersection of an open neighborhood of $\partial U$ with $U$.
If $v$ is not identically zero, then there is a positive measure set $Y \subseteq \RR$ such that for $y \in Y$, $\partial \{u > y\} \neq \partial^* \{u + v > y\}$.
But
$$V \cap \partial \{u > y\} = V \cap \partial^* \{u + v > y\}$$
so $1_{\{u > y\}}$ and $1_{\{u + v > y\}}$ are equalized by the trace map $BV_\loc(U) \to L^1_\loc(\partial U)$.
Now $\{u > y\}$ has least perimeter in $U$, so if $|U \cap \partial \{u > y\}| = |U \cap \partial^* \{u + v > y\}|$ then $\{u + v > y\}$ has least perimeter in $U$. Then by Theorem \ref{main lma}, $\partial \{u > y\}, \partial^* \{u + v > y\}$ are analytic and hence are equal by analytic continuation.
This contradicts that $y \in Y$, so for each $y \in Y$,
$$|U \cap \partial \{u > y\}| < |U \cap \partial^* \{u + v > y\}|.$$
Integrating both sides in $y$ using the coarea formula, we conclude that 
\begin{align*}
\int_U \star |\dif u + \dif v| - \star |\dif u| &= \int_{-\infty}^\infty |U \cap \partial^* \{u + v > y\}| - |U \cap \partial \{u > y\}| \dif y \\
&= \int_Y |U \cap \partial^* \{u + v > y\}| - |U \cap \partial \{u > y\}| \dif y > 0,
\end{align*}
and since $v$ was an arbitrary perturbation, $u$ must have least gradient.

% \subsubsection{Surfaces with boundary}
% Now suppose that $\overline M$ is a surface-with-boundary.
% We must show that if $\gamma_1, \gamma_2$ are two distinct geodesics in the lamination $\lambda$, then their extensions $\overline \gamma_i$ to $\overline M$ do not intersect on $\partial M$.
% If they do intersect at $P \in \partial M$, then $\gamma_1, \gamma_2$ bound the same superlevel set $\{u > y\}$.
% Let $Q, R$ be points on $\gamma_1, \gamma_2$ respectively which are so close to $P$ that $P, Q, R$ bound a nondegenerate, contractible geodesic triangle $T$.
% By local finiteness, we may assume that no geodesics in $A_y$ pass through the interior of $T$.
% Since $\{u > y\}$ has least perimeter, $v = 1_{\{u > y\}}$ has least gradient, and without loss of generality we may assume that $v = 1$ inside $T$ and $v = 0$ on the opposite sides of $\gamma_1, \gamma_2$.
% Moreover $\int_M \star |\dif v|$ is the sum of lengths of geodesics in $A_y$.
% If we define $\tilde v = v$ away from the interior of $T$ and $\tilde v = 0$ on $T$, then $\int_M \star |\dif \tilde v|$ is the sum of lengths of geodesics in $A_y$, but with $\overline{PQ}$ and $\overline{PR}$ replaced by $\overline{QR}$.
% So by the triangle inequality and the fact that $T$ is nondegenerate,
% $$\int_M \star |\dif \tilde v| < \int_M \star |\dif v|,$$
% yet by construction $\tilde v$ and $v$ have the same trace (which is well-defined by Proposition \ref{traces}).
% This is a contradiction since $v$ has least gradient.

% \begin{figure}
% \centering
% \begin{subfigure}[b]{0.4\linewidth}
% \includegraphics[width=\linewidth]{geodesic_triangle.png}
% \end{subfigure}
% \begin{subfigure}[b]{0.4\linewidth}
% \includegraphics[width=\linewidth]{converse_max.png}
% \end{subfigure}
% \caption{The left picture illustrates the $d = 2$ case.
% The geodesic triangle $T$ is lightly shaded, bounded by the geodesics $\gamma_1, \gamma_2, \overline{QR}$. Here $\{u > y\} = \{v > 0\}$ is shaded, and $\{\tilde v > 0\}$ is darkly shaded.
% The right picture illustrates the converse to the maximum principle. The function $u$ is constant on each colored piece of $\partial U$, and its level sets $A_1, A_2$ form a geodesic lamination.
% The competitor $v$ has the same trace on $\partial U$, but its level sets (the lighter curves) are not geodesics and so by the coarea formula, $u$ is smaller.}
% \label{max_princip_graphs}
% \end{figure}

% For the converse, suppose that $\lambda$ is a minimal lamination whose set of leaves is discrete, so every set $A_y$ is minimal and $y$ ranges over a discrete set $\Gamma$.
% Fix $U \Subset M$ with Lipschitz boundary, let $T: BV(U) \to L^1(\partial U)$ be the trace map, and let $v$ be a competitor in $U$, thus $v \in BV(U)$ and $Tu = Tv$.
% In particular, for every $y \in \Gamma$, $\{Tu \geq y\} = \{Tv \geq y\}$.

% By (\ref{convergence of trace}), for any $w \in BV(U)$, $x \in \partial U$, and $z \in \RR$, $T(1_{w > z})(x)$ is the density of $\{w > z\}$ in an infinitesimal neighborhood of $x$.
% Let $\varepsilon > 0$ be so small that $\{Tu > y\} = \{Tu > y - 2\varepsilon\}$, which exists since $\Gamma$ is discrete.
% If $Tv(x) > y$, then the density of $\{v > y - \varepsilon/2\}$ near $x$ is $1$, so $T(1_{\{v > y - \varepsilon/2\}})(x) = 1$, so 
% $$1_{\{Tu > y\}} = 1_{\{Tv > y\}} \leq T(1_{\{v > y - \varepsilon/2\}}).$$
% Conversely, if $Tv(x) \leq y - \varepsilon$, then the density of $\{v > y - \varepsilon/2\}$ near $x$ is $0$. Thus 
% $$T(1_{\{v > y - \varepsilon/2\}}) \leq T(1_{\{v > y - \varepsilon\}}) \leq 1_{\{Tv > y - \varepsilon\}} = 1_{\{Tu > y - \varepsilon\}} = 1_{\{Tu > y\}}.$$
% The inequalities collapse and imply that $1_{\{Tu > y - \varepsilon\}} = T(1_{\{v > y - \varepsilon\}})$.
% This is true for $\varepsilon$ arbitrarily small, so
% $$1_{\{Tu \geq y\}} = T(1_{\{v \geq y\}}).$$
% The left-hand side is $T(1_{\{u > y\}})$ since $u$ is locally constant away from $\lambda$ (since $\Gamma$ is discrete).
% Therefore $A_y$ and $1_{\{v \geq y\}}$ are competitors, thus since $A_y$ is minimal, 
% \begin{equation}\label{laminationwise least gradient}
% |A_y \cap U| \leq |\partial^* \{v \geq y\} \cap U|.
% \end{equation}
% We now integrate both sides of (\ref{laminationwise least gradient}) against $\dif y$ and apply the coarea formula, Proposition \ref{coarea formula}, to see that
% $$\int_U \star |\dif u| = \int_{-\infty}^\infty |A_y \cap U| \dif y \leq \int_{-\infty}^\infty |\partial^* \{v \geq y\} \cap U| \dif y = \int_U \star |\dif v|,$$
%     implying that $u$ has least gradient.

% TODO: If $M$ has negative curvature then maybe the chaoticness of the geodesic flow implies that $\lambda$ is a priori discrete, so we can drop that hypothesis?
% Since that's the only case we care about, maybe remove this from the max princip and just put it in the section on hyperbolic geometry

\subsection{G\'orny decomposition}
We next prove Theorem \ref{Gorny regularity}, following \cite[pg8--11]{górny2017planar}.
We begin by decomposing functions of least gradient into sections using the following proposition.
It essentially appears for contractible sets in \cite[pg10--11]{górny2017planar} but we found it more elegant to reprove the result in a way which better shows the importance of the topology.

\begin{definition}
Let $u \in BV(M)$. The \dfn{jumpset} of $u$ is the set $J_u$ of $x \in M$ such that there exists $\normal \in T_x'M$ and $u(x-) < u(x+) \in \RR$ such that 
\begin{align*}
\lim_{r \to 0} \dashint_{B(x, r) \cap (\exp_x)_* \{v \in T_xM: (\normal, v) > 0\}} \star |u - u(x+)|  & = 0, \\
\lim_{r \to 0} \dashint_{B(x, r) \cap (\exp_x)_* \{v \in T_xM: (\normal, v) < 0\}} \star |u - u(x-)|  & = 0.
\end{align*}
\end{definition}

\begin{proposition}\label{existence of jump graphs}
Let $u \in BV(M)$ satisfy:
\begin{enumerate}
\item $u$ only has jump discontinuities,
\item the jumpset of $u$ is a lamination $\lambda$ of $M$ with countably many leaves,
\item each leaf of $\lambda$ is the locally finite union of submanifolds of $M$ without boundary, and
\item the traces of $u$ along each leaf $N$ of $\lambda$ are constant on $N$.
\end{enumerate}
Then there exists an affine line bundle $E \to M$, which is flat with structure group $\RR$, and a section $Ju: M \to E$ such that $Ju$ is locally constant on $M \setminus \lambda$ and for each leaf $N$ of $\lambda$, the amounts that $u, Ju$ jump along $N$ are equal.
\end{proposition}
\begin{proof}
We first observe that $\pi_0(M \setminus \lambda)$ is nonempty and countable, since $\lambda$ is a countable union of null sets and so $M \setminus \lambda$ is nonempty, and since $\lambda$ only has countably many connected components.
We define a new lamination $\tilde \lambda$, as follows. Initialize $\tilde \lambda := \lambda$ and replace each leaf in $\lambda$ with its connected components.
We then adjoin further leaves to $\tilde \lambda$ so that for each $U \in \pi_0(M \setminus \tilde \lambda)$, if $\mathcal V(U)$ denotes the set of components which are adjacent to $U$, then $\overline{U \cup \bigcup_{V \in \mathcal V(U)} V}$ is contained in a simply connected subset of $M$.

We set $\{U_i: i \in I\} = \pi_0(M \setminus \tilde \lambda)$ for a countable set $I$.
We endow $I$ with the structure of a weighted directed multigraph, where the set of edges $E_{ij}$ from $i$ to $j$ is $E_{ij} := \pi_0(\partial U_i \cap \partial U_j)$ if $i \neq j$.
Thus each element of $E_{ij}$ is a leaf of $\tilde \lambda$.
We weight an edge $N \in E_{ij}$ by the amount that $u$ jumps along any curve $\gamma$ from $U_i$ to $U_j$ transverse to $N$ when $\gamma$ passes through $N$.
The leaves we adjoined all have weight $0$.

We choose $x_i \in U_i$ for each $i \in I$.
We call a path $\sum_k N_k$ through $I$, where $N_k$ is an edge $i_k \to j_k$, a \dfn{boundary} if there are curves $\gamma_1, \dots, \gamma_m$, where $\gamma_k$ is a curve $x_{i_k} \to x_{j_k}$ that is transverse to $\tilde \lambda$ and passes through $N_k$ exactly once, and the homology class $[\sum_k \gamma_k]$ is zero.

\begin{claim}
The total weight of any boundary is $0$.
\end{claim}
\begin{proof}[Proof of claim]
Let $k = 1, \dots, n$, let $\gamma := \sum_k \gamma_k$ be the associated $1$-chain to a boundary $\sum_k N_k$, and let $w_k$ be the weight of $N_k$.
If we set $v$ to be $u$ on $U_{i_1}$, $u - w_1$ on $U_{i_2}$, etc., $u - \sum_{k < n} w_k$ on $U_{i_n}$, then $v$ is continuous away from $N_{i_n}$ where it has a jump discontinuity of size $\sum_k w_k$.
Then $\int_\gamma \dif v = 0$ but also $\int_\gamma \dif v = \sum_k w_k$.
\end{proof}

Let $P_{ij}$ be the set of paths $i \to j$.
It follows that if $p, q \in P_{ij}$ are \dfn{homologous} in the sense that $p - q$ is a boundary, then $p, q$ have the same total weight.
On the other hand, if $i, j$ are adjacent, then for $N, N' \in E_{ij}$ and $\gamma, \gamma': x_i \to x_j$ which are respectively transverse to $N, N'$ and do not leave $\overline U_i \cup \overline U_j$, $[\gamma - \gamma'] = 0$, so $N, N'$ are homologous and hence have the same weight.

We now define $V_i$ to be the interior of $\overline U_i \cup \bigcup_{E_{ij} \neq \emptyset} \overline U_j$.
Then $(V_i)$ is an open cover of $M$.
For $x \in V_i \setminus \tilde \lambda$, let $j(x)$ be such that $x \in U_{j(x)}$, and let $Ju_i(x)$ be the weight of some (and hence any) edge in $E_{ij(x)}$ if $j(x) \neq i$, or $Ju_i(x) = 0$ if $j(x) = i$.
Then $Ju_i$ is well-defined, locally constant, and jumps by the same amount along each leaf of $\tilde \lambda$ in $V_i$ as $u$.
Moreover, on $V_i \cap V_j$, $Ju_i - Ju_j = Ju_i(x_j)$ which is constant.
Now the desired bundle $E$ exists with flat trivializations $V_i$ and transition functions $s_{ij} := Ju_i(x_j)$.
If $V_i \cap V_j \cap V_k$ is nonempty, then it is contained in a simply connected set and so $s_{ik} = s_{ij} + s_{jk}$.
\end{proof}

\begin{figure}
\centering
\begin{subfigure}[b]{0.4\linewidth}
\includegraphics[width=\linewidth]{sample torus.png}
\end{subfigure}
\begin{subfigure}[b]{0.4\linewidth}
\includegraphics[width=\linewidth]{torus graph.png}
\end{subfigure}
\caption{The proof of Proposition \ref{existence of jump graphs} in case $M = \mathbf T^2$ and $u$ has two jump discontinuities, along each of the black loops. Note that we have to adjoin green leaves in order to ensure the simple connectedness condition, and that if we were to retract the green edges of the graph, then there would be a vertex of $I$ with a self-loop.}
\label{torus graphs}
\end{figure}

\begin{proof}[Proof of Theorem \ref{Gorny regularity}]
Let $u$ be a function of least gradient on $M$ with jumpset $J_u$.
Reasoning identically to the proof of \cite[Proposition 3.9]{górny2017planar} we see that $x \in J_u$ iff
$$\inf\left\{t \in \RR: \lim_{r \to 0} \frac{|\{u \leq t\} \cap B(x, r)|}{|B(x, r)|} = 1\right\} \neq \sup\left\{t \in \RR: \lim_{r \to 0} \frac{|\{u \geq t\} \cap B(x, r)|}{|B(x, r)|} = 1\right\}.$$
By \cite[Theorem 4.1]{HakkarainenKorteLahtiShanmugalingam+2015}, it follows that $u$ only has jump discontinuities.
Moreover, $J_u$ is a sublamination of the minimal lamination $\lambda$ furnished by the maximum principle, and along each connected component of any leaf in $J_u$, the trace of $u$ from each side is constant along $N$.
So by Proposition \ref{existence of jump graphs}, there exists a unique (up to isomorphism) decomposition $u = u_j - u_c$ into sections $u_j, u_c: M \to E$, where $E$ is a flat affine line bundle with structure group $\RR$, such that $u_j$ is locally constant on $M \setminus \lambda$, $u_j$ has the same jumpset, with the same traces on each leaf, as $u$, and $u_c$ has no jump discontinuties.
Reasoning identically to \cite[pg11]{górny2017planar} we see that $u_j, u_c$ have locally least gradient.
Repeating the reasoning from the start of this proof and using the fact that $u_c$ has no jump discontinuities, it follows that $u_c$ is continuous.
\end{proof}


%%%%%%%%%%%%%%%%%%%%%%
% \subsection{Numerical analysis of minimal laminations}\label{numerics}
% The maximum principle, when combined with recent work of Loisel \cite{Loisel20}, furnishes a large class of minimal laminations which are inexpensive to numerically compute.
% For simplicity, we consider the model case that $M$ is a closed space form such that $\pi_1(M)$ is nonzero, but the results of this section can be easily extended to other manifolds $M$ of constant sectional curvature.
% We give an algorithm for constructing minimal laminations in $M$, and provide the results of some numerical experiments used to construct some examples of minimal laminations in three dimensions.

% We begin by recalling Loisel's theorem \cite[Theorem 1]{Loisel20} on the numerical analysis of the $p$-Laplacian in the limiting case $p = 1$.

% \begin{definition}
% By the \dfn{PL finite element space} associated to a triangulation $\mathcal T$ of a polytope $\overline \Omega \subset \RR^d$ we mean the space of all continuous functions $u: \overline \Omega \to \RR$ whose restrictions $u|T$, $T \in \mathcal T$, are linear.
% We equip a PL finite element space with the norm inherited from $W^{1, 1}(\Omega) \subset BV(\Omega)$.
% By a \dfn{PL trace} we mean a continuous function $f: \partial \Omega \to \RR$ such that if $T \in \mathcal T$ intersects $\partial \Omega$, then $f|T \cap \partial \Omega$ is linear.
% \end{definition} 

% Following \cite[\S3.2]{Loisel20}, we identify any PL trace $f$ with its \dfn{discrete harmonic prolongation}, that is, the minimizer $v$ of $\int_\Omega \star |\dif v|^2$ in the PL finite element space subject to $v|\partial \Omega = f$.
% Thus $||v||_V \lesssim ||f||_{L^1(\partial \Omega)}$.

% \begin{theorem}[Loisel]
% Let $\Omega$ be a polytope in $\RR^d$ and let $\mathcal T$ be a triangulation of $\Omega$ with quasiuniformity parameter $\lesssim 1$.
% Let $V$ be the PL finite element space associated to $\mathcal T$ and let $f$ be a PL trace.
% Let $u \in V$ minimize $\int_\Omega \star |\dif u|$ subject to the constraints $u \in V$, $u|\partial \Omega = f$.
% Then the barrier method of \cite[\S2.3]{Loisel20} returns $u$ with runtime $\lesssim |\mathcal T|^{1/2} \log (|\mathcal T|\Japan{||f||_V})$.
% \end{theorem}

% The same argument shows the above result holds when the Hodge star is not necessarily the euclidean Hodge star.
% TODO: Confirm this.

% Let $M$ be a closed space form, let $\xi \in H^1(M, \ZZ)$ be a cohomology class, and let $u: M \to \Sph^1$ be a map of least gradient such that $\xi = [\dif u]$.
% Then we associate to $\xi$ the minimal lamination induced by $u$.
% Note that $u$ may not be unique.

% TODO: How are all these things stored in memory?

% TODO: The algorithm. Turn $\xi$ into Dirichlet data, compute a tolerance, run Loisel's algorithm, and then use the fact that every leaf runs through the boundary to get the leaves.

% \begin{proposition}\label{application to Loisel}
% Let $M$ be a closed space form, and let $\xi \in H^1(M, \ZZ)$.
% Then Algorithm TODO returns the minimal lamination associated to $\xi$.
% Moreover, the runtime is TODO.
% \end{proposition}

% TODO: Do some numerical experiments, show what minimal laminations in a fundamental polytope in $\Hyp^3$ or $\Sph^3$ look like



% \begin{lemma}
% Let $P \in M$ and let $\gamma$ be the unique geodesic through $O, P$.
% Let $\Phi^P: M \to M$ be the unique oriented isometry of $M$ such that $\Phi^P(P) = O$ and $\Phi^P$ maps $\gamma$ into itself.
% Then on $T_QM$, $Q \in M$,
% $$|\dif \Phi^P - \id| \lesssim |K|(|P|^2 + |P| \cdot |Q|).$$
% \end{lemma}
% \begin{proof}
% We prove this in case $K = -1$, in which case we have 
% $$\varphi(Q) := (\Phi^P)^{-1}(Q) = \frac{(1 - |P|^2)Q + (|Q|^2 + 2P \cdot Q + 1)P}{|P|^2 |Q|^2 + 2P \cdot Q + 1}$$
% by \cite[(4.5.5)]{ratcliffe2006foundations}.
% The cited proof only relies on the fact that the isometry group of $M$ is Coxeter and generated by spherical inversions, so we can adapt it to the case $K = +1$ as well. TODO: Write out the details for $K = +1$.
% The case of general $K$ follows by a scaling argument.

% We now compute 
% \begin{align*}
% \dif \varphi &= (1 - |P|^2) \dif \frac{Q}{|P|^2|Q|^2 + 2P \cdot Q + 1} + P \otimes \dif \frac{|Q|^2 + 2P \cdot Q + 1}{|P|^2 |Q|^2 + 2P\cdot Q + 1} \\
% &= \frac{1 - |P|^2}{|P|^2 |Q|^2 + 2 P \cdot Q + 1} \id - \frac{1 - |P|^2}{(|P|^2 |Q|^2 + 2P \cdot Q + 1)^2}(2|P|^2Q^{\otimes 2} + 2P \otimes Q) \\
% &\qquad + \frac{2Q \otimes P + 2P^{\otimes 2}}{|P|^2 |Q|^2 + 2P \cdot Q + 1} - \frac{|Q|^2 + 2P \cdot Q + 1}{(|P|^2 |Q|^2 + 2P \cdot Q + 1)^2} (2Q \otimes P + 2P^{\otimes 2}) \\
% &= \id + O(|P|^2 + |P| \cdot |Q|). 
% \end{align*}
% Inverting $\dif \varphi$ using a Neumann series completes the proof.
% \end{proof}




\printbibliography

\end{document}
