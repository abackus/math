\documentclass[reqno,12pt,letterpaper]{amsart}
\RequirePackage{amsmath,amssymb,amsthm,graphicx,mathrsfs,url,slashed}
\RequirePackage[usenames,dvipsnames]{xcolor}
\RequirePackage[colorlinks=true,linkcolor=Red,citecolor=Green]{hyperref}
\RequirePackage{amsxtra}
\usepackage{cancel}
\usepackage{tikz-cd}

\setlength{\textheight}{9.3in} \setlength{\oddsidemargin}{-0.25in}
\setlength{\evensidemargin}{-0.25in} \setlength{\textwidth}{7in}
\setlength{\topmargin}{-0.25in} \setlength{\headheight}{0.18in}
\setlength{\marginparwidth}{1.0in}
\setlength{\abovedisplayskip}{0.2in}
\setlength{\belowdisplayskip}{0.2in}
\setlength{\parskip}{0.05in}
\renewcommand{\baselinestretch}{1.05}

\title[Riemannian least-gradient maximum principle]{The least-gradient maximum principle on Riemannian manifolds}
\author{Aidan Backus}
\date{May 2022}

\newcommand{\NN}{\mathbf{N}}
\newcommand{\ZZ}{\mathbf{Z}}
\newcommand{\QQ}{\mathbf{Q}}
\newcommand{\RR}{\mathbf{R}}
\newcommand{\CC}{\mathbf{C}}
\newcommand{\DD}{\mathbf{D}}
\newcommand{\PP}{\mathbf P}
\newcommand{\MM}{\mathbf M}
\newcommand{\II}{\mathbf I}
\newcommand{\Hyp}{\mathbf H}
\newcommand{\Sph}{\mathbf S}
\newcommand{\GL}{\mathbf{GL}}
\newcommand{\Orth}{\mathbf{O}}
\newcommand{\SpOrth}{\mathbf{SO}}
\newcommand{\Ball}{\mathbf{B}}

\DeclareMathOperator{\avg}{avg}
\DeclareMathOperator{\card}{card}
\DeclareMathOperator{\cent}{center}
\DeclareMathOperator{\ch}{ch}
\DeclareMathOperator{\codim}{codim}
\DeclareMathOperator{\diag}{diag}
\DeclareMathOperator{\diam}{diam}
\DeclareMathOperator{\dom}{dom}
\DeclareMathOperator{\Exc}{Exc}
\DeclareMathOperator{\Gal}{Gal}
\DeclareMathOperator{\Hom}{Hom}
\DeclareMathOperator{\Iso}{Iso}
\DeclareMathOperator{\Jac}{Jac}
\DeclareMathOperator{\Lip}{Lip}
\DeclareMathOperator{\Met}{Met}
\DeclareMathOperator{\id}{id}
\DeclareMathOperator{\rad}{rad}
\DeclareMathOperator{\rank}{rank}
\DeclareMathOperator{\Rm}{Rm}
\DeclareMathOperator{\Hess}{Hess}
\DeclareMathOperator{\Radon}{Radon}
\DeclareMathOperator*{\Res}{Res}
\DeclareMathOperator{\sgn}{sgn}
\DeclareMathOperator{\singsupp}{sing~supp}
\DeclareMathOperator{\Spec}{Spec}
\DeclareMathOperator{\supp}{supp}
\DeclareMathOperator{\Tan}{Tan}
\newcommand{\tr}{\operatorname{tr}}

\newcommand{\Ric}{\mathrm{Ric}}
\newcommand{\Riem}{\mathrm{Riem}}
\newcommand*\dif{\mathop{}\!\mathrm{d}}
\newcommand{\LapQL}{\Delta^{\mathrm{ql}}}

\newcommand{\dbar}{\overline \partial}

\DeclareMathOperator{\atanh}{atanh}
\DeclareMathOperator{\csch}{csch}
\DeclareMathOperator{\sech}{sech}

\DeclareMathOperator{\Div}{div}
\DeclareMathOperator{\Gram}{Gram}
\DeclareMathOperator{\grad}{grad}
\DeclareMathOperator{\dist}{dist}
\DeclareMathOperator{\spn}{span}
\DeclareMathOperator{\Ell}{Ell}
\DeclareMathOperator{\WF}{WF}

\newcommand{\Lagrange}{\mathscr L}
\newcommand{\DirQL}{\mathscr D^{\mathrm{ql}}}
\newcommand{\DirL}{\mathscr D}

\newcommand{\Hilb}{\mathcal H}
\newcommand{\Homology}{\mathrm H}
\newcommand{\normal}{\mathbf n}
\newcommand{\radial}{\mathbf r}
\newcommand{\evect}{\mathbf e}
\newcommand{\vol}{\mathrm{vol}}

\newcommand{\pic}{\vspace{30mm}}
\newcommand{\dfn}[1]{\emph{#1}\index{#1}}

\renewcommand{\Re}{\operatorname{Re}}
\renewcommand{\Im}{\operatorname{Im}}

\newcommand{\loc}{\mathrm{loc}}
\newcommand{\cpt}{\mathrm{cpt}}

\def\Japan#1{\left \langle #1 \right \rangle}

\newtheorem{theorem}{Theorem}[section]
\newtheorem{badtheorem}[theorem]{``Theorem"}
\newtheorem{prop}[theorem]{Proposition}
\newtheorem{lemma}[theorem]{Lemma}
\newtheorem{sublemma}[theorem]{Sublemma}
\newtheorem{claim}[theorem]{Claim}
\newtheorem{proposition}[theorem]{Proposition}
\newtheorem{corollary}[theorem]{Corollary}
\newtheorem{conjecture}[theorem]{Conjecture}
\newtheorem{axiom}[theorem]{Axiom}
\newtheorem{assumption}[theorem]{Assumption}

\theoremstyle{definition}
\newtheorem{definition}[theorem]{Definition}
\newtheorem{remark}[theorem]{Remark}
\newtheorem{example}[theorem]{Example}
\newtheorem{notation}[theorem]{Notation}

\newtheorem{exercise}[theorem]{Discussion topic}
\newtheorem{homework}[theorem]{Homework}
\newtheorem{problem}[theorem]{Problem}

\makeatletter
\newcommand{\proofpart}[2]{%
  \par
  \addvspace{\medskipamount}%
  \noindent\emph{Part #1: #2.}
}
\makeatother

\newtheorem{ack}{Acknowledgements}

\numberwithin{equation}{section}


% Mean
\def\Xint#1{\mathchoice
{\XXint\displaystyle\textstyle{#1}}%
{\XXint\textstyle\scriptstyle{#1}}%
{\XXint\scriptstyle\scriptscriptstyle{#1}}%
{\XXint\scriptscriptstyle\scriptscriptstyle{#1}}%
\!\int}
\def\XXint#1#2#3{{\setbox0=\hbox{$#1{#2#3}{\int}$ }
\vcenter{\hbox{$#2#3$ }}\kern-.6\wd0}}
\def\ddashint{\Xint=}
\def\dashint{\Xint-}

\usepackage[backend=bibtex,style=alphabetic]{biblatex}
\renewcommand*{\bibfont}{\normalfont\footnotesize}
\addbibresource{topics.bib}
\renewbibmacro{in:}{}
\DeclareFieldFormat{pages}{#1}


\begin{document}
\begin{abstract}
The least-gradient maximum principle, essentially due to Miranda and de Giorgi in the 1960s, shows that least-gradient functions define a minimal lamination of the support of their derivative.
We show that this result still holds on hyperbolic manifolds.
As a consequence we answer some questions of Daskalopoulos--Uhlenbeck concerning the $L^\infty$-Teichm\"uller theory.
\end{abstract}

\maketitle

%%%%%%%%%%%%%%%%%%%%%%%%%%%%%%%%%%%%%%%%%%%%%%%%%%%%%%%

% \tableofcontents

\section{Introduction}
Throughout this paper, let $M$ be an oriented Riemannian manifold of metric $g$ and dimension $d$.
For a function $u \in BV_\loc(M)$, we write $\star |\dif u|$ for the total variation of the derivative, c.f. (\ref{total variation}).

\begin{definition}\label{main definitions}
A function $u \in BV_\loc(M)$ has \dfn{least gradient} if for every $v \in BV_\cpt(M)$ such that $\supp v \subseteq U \Subset M$,
$$\int_U \star |\dif u| \leq \int_U \star |\dif u + \dif v|.$$
A set $U$ of locally finite perimeter has \dfn{least perimeter} if $1_U$ has least gradient.
\end{definition}

Functions of least gradient arise naturally as solutions to an inverse problem in magnetic resonance imaging (MRI) \cite{Nachman2009, Tamasan2019, Joy09} as well as the formal limit of $p$-harmonic functions as $p \to 1$, but we are interested in them primarily because of their application to the $L^\infty$-Teichm\"uller theory of Thurston and Daskalopoulos--Uhlenbeck \cite{thurston1998minimal, daskalopoulos2020transverse}.
In that context one is interested in the existence and regularity of minimal laminations in a closed hyperbolic manifold, and our main theorem, Theorem \ref{main thm}, constructs them from function of least gradient.

\begin{definition}
A \dfn{minimal lamination} in $M$ is a partition of a closed subset of $M$ into smooth hypersurfaces with zero mean curvature.
\end{definition}

\begin{theorem}[maximum principle]\label{main thm}
Suppose that $2 \leq d \leq 7$.
Let $u: M \to \RR$ be a function of least gradient, and $A_y = \partial \{u > y\}$.
Then $\lambda = (A_y)_{y \in \RR}$ is a minimal lamination in $M$, and each $A_y$ is a locally finite union of connected minimal hypersurfaces.
\end{theorem}

Theorem \ref{main thm} generalizes \cite[Proposition 3.4]{górny2017planar} and partially solves a problem \cite[Problem 9.5]{daskalopoulos2020transverse} of Daskalopoulos--Uhlenbeck; we state this as Corollary \ref{maximum stretch contains lamination}.
As a byproduct we obtain Corollary \ref{ruelle sullivan antiderivative} which is a solution to \cite[Problem 9.7]{daskalopoulos2020transverse}.
Moreovevr, it was shown by G\'orny \cite[Theorem 1.2]{górny2017planar} that functions of least gradient on $\RR^d$, $d \leq 7$, have a decomposition into continuous and jump parts.
Given Theorem \ref{main thm}, the proof of this result goes through verbatim: TODO CHECK ME

\begin{theorem}[G\'orny's regularity theorem]\label{Gorny regularity}
Let $M$ be an open bounded strictly convex manifold of dimension $\leq 7$.
Then any function of least gradient $u: M \to \RR$ can be written as the sum of a continuous function of least gradient, and a function of least gradient with only jump discontinuities.
\end{theorem}

%%%%%%%%%%%%%%%%%%%%%%%%%%%%%%%%%%%%%
\subsection{Overview of the proof}
Regularity proofs in geometric measure theory usually have three components:
\begin{enumerate}
\item A monotonicity formula that controls the behavior of a singular set at fine scales;
\item Study of the tangent cone obtained by blowing up a singular set;
\item A $\varepsilon$-regularity lemma which asserts that a singular set of ``small oscillation'' is smooth.
\end{enumerate}
Experts in PDE can understand the study of the tangent cone as a proof of ``subcriticality'', as these tangent cones should be thought of as ``zero solutions''; then the $\varepsilon$-regularity lemma is a sort of ``small data well-posedness''.
In particular, this strategy was used by Miranda to show the least-gradient maximum principle on euclidean space \cite{Miranda64, Miranda66, Miranda67}.
We shall follow Giusti's exposition \cite{Giusti77} of Miranda's proof, and shall often refer to Giusti when the proof of a lemma is straightforward to generalize from the euclidean case.

Before the proof, we give some applications of Theorem \ref{main thm} and some open problems in this present section.
In \S\ref{LeastGradientFunctions} we recall elementary facts about functions of least gradient and their level sets, the so-called sets of least perimeter, c.f. Proposition \ref{level sets are minimal}.
We then show that the tangent cone of a set of least perimeter exists and is actually a hyperplane, Proposition \ref{blowup theorem}; this can easily be shown by reduction to the Bernstein-Fleming theorem which classifies minimal cones in $\RR^d$ \cite[Theorem 17.3]{Giusti77}.

In \S\ref{MollifierSection} we prove the following monotonicity formula which may be of independent interest:

\begin{theorem}[monotonicity formula]\label{monotonicity prestate}
There exists $A \geq 0$ depending continuously on $P \in M$ such that for every function $u$ of least gradient defined near $P \in M$, every normal coordinate system $(x^\mu)$ based at $P$, and $0 < r_1 < r_2 \lesssim 1$,
\begin{align*}
&\left|\int_{r_1}^{r_2} \partial_r \left[r^{1 - d} \int_{B(P, r)} \dif u \wedge \dif x^1 \wedge \cdots \wedge \dif x^{d - 1}\right] \dif r\right|^2 \\
&\qquad \lesssim \left(1 + (d - 1) \log \frac{r_2}{r_1}\right) \left(r_2^{1 - d}\int_{B(P, r_2)} \star |\dif u| \right)\left(\int_{r_1}^{r_2} \partial_r \left[e^{Ar^2} r^{1 - d} \int_{B(P, r)} \star |\dif u|\right] \dif r\right).
\end{align*}
\end{theorem}

This formula generalizes \cite[Theorem 2.8]{Miranda66} and is stronger than monotonicity formulae for minimal surfaces in Riemannian manifolds that the author is aware of (see e.g. the notes of Marques \cite[\S7]{MarquesXX}) in that it gives a lower bound on the rate of growth of the monotone quantity $e^{Ar^2} r^{1 - d} \int_{B_r} \star |\dif u|$.
As a consequence, we show Proposition \ref{mollifier quant}, which allows for the mollification of sets of least perimeter.

We thus restrict to $C^1$ sets of least perimeter in \S\ref{Plateau section}, and represent them as graphs of solutions $\omega: \Omega \to \RR$ of a Plateau equation (Proposition \ref{construction of Plateau energy}) where $\Omega$ is a certain Riemannian manifold constructed using normal coordinates on $M$.
We then arrive at the fundamental inequality of this paper: Proposition \ref{dGL Laplace} generalizes \cite[Teorema 4.3]{Miranda66} and shows that one obtains a multiplicative gain in the oscillation of $\omega$ when one passes from a scale $r$ to $r/2$.

In \S\ref{de Giorgi section} our goal is to show:

\begin{theorem}[regularity of minimal hypersurfaces]\label{main lma}
Suppose that $2 \leq d \leq 7$.
Then every set of least perimeter in $M$ is bounded by a minimal hypersurface.
\end{theorem}

Theorem \ref{main lma} follows from the de Giorgi $\varepsilon$-regularity lemma, Proposition \ref{dGL final}, which generalizes \cite[Teorema 5.7]{Miranda66} and itself follows by a mollification of Proposition \ref{dGL Laplace}.

Let us now explain why Theorem \ref{main thm} follows from Theorem \ref{main lma}.
Let $u$ have least gradient,
\begin{equation}\label{lamination union}
A = \bigcup_y \partial \{u > y\},
\end{equation}
let $B$ be the interior of $\{du = 0\}$, and let $x \in M$.
Then $x \in B$ iff $u = u(x)$ near $x$, but that happens iff for every $y < u(x)$, $x$ is interior to $\{u > y\}$ and for every $y \geq u(x)$, $x$ is exterior to $\{u > y\}$.
This happens iff for every $y \in \RR$, $x$ is either interior or exterior to $\{u > y\}$, thus $x \notin \partial \{u > y\}$, which happens iff $x \notin A$.
Thus $\{A, B\}$ is a partition of $M$, so $A$ is closed.
Moreover, the sets $\{u > y\}$ are totally ordered by $\subseteq$, so the sets $\partial \{u > y\}$ are disjoint.
They are also hypersurfaces with the desired amount of regularity, by Theorem \ref{main lma}.
By Proposition \ref{local finiteness} the decomposition of $\partial \{u > y\}$ into connected components is locally finite.

%%%%%%%%%%%%%%%%%%%%%%%%%%%%%%%%%%%%%%%%%%%%%%%

\subsection{Applications to hyperbolic geometry}
The Thurston asymmetric metric, first defined in \cite{thurston1998minimal}, is constructed from best-Lipschitz maps between two closed hyperbolic manifolds.
As a first step towards an analytic understanding of the Thurston asymmetric metric, Daskalopoulos--Uhlenbeck \cite{daskalopoulos2020transverse} considered best-Lipschitz maps $M \to \Sph^1$ where $M$ is a closed hyperbolic surface.
They identified a particularly important class of such maps, the $\infty$-harmonic maps, defined as follows, which are particularly significant because they induce geodesic laminations of $M$.

\begin{definition}
If a function $u$ is the weak limit in $L^r$ for $r > d$ of $p$-harmonic functions as $p \to \infty$, we call $u$ \dfn{$\infty$-harmonic}.
For an $\infty$-harmonic function $u$ we define the \dfn{maximum-stretch locus}
$$\lambda_u := \{x \in M: L(x) = \sup L\}$$
where $L(x)$ denotes the local Lipschitz constant of $u$ at $x$.
\end{definition}

\begin{theorem}\label{infinity harmonic laminations}
Suppose that $M$ is a closed hyperbolic surface and $u$ is an $\infty$-harmonic function. Then the maximum-stretch locus $\lambda_u$ is a geodesic lamination in $M$.
\end{theorem}

In \cite[\S5]{daskalopoulos2020transverse}, Daskalopoulos--Uhlenbeck prove Theorem \ref{infinity harmonic laminations} by considering the viscosity solution theory of $\infty$-Laplace equation
\begin{equation}\label{infinity laplace}
    \Hess u(\grad u, \grad u) = 0.
\end{equation}
However, the theory of viscosity solutions of (\ref{infinity laplace}) is still nascent, and Daskalopoulos--Uhlenbeck ask for a proof of Theorem \ref{infinity harmonic laminations} that bypasses (\ref{infinity laplace}) altogether, c.f. \cite[Problem 9.5]{daskalopoulos2020transverse}.
We give a partial resolution of this problem by proving \cite[Theorem-Conjecture 9.6]{daskalopoulos2020transverse}.
Before stating it we note that by a \dfn{section of least gradient} on $M$ we mean a section that lifts to a function of least gradient on $\Hyp^d$.

\begin{corollary}\label{maximum stretch contains lamination}
Let $u$ be an $\infty$-harmonic function on a closed hyperbolic surface $M$.
Then the maximum-stretch locus $\lambda_u$ contains a geodesic lamination.
\end{corollary}
\begin{proof}
By \cite[\S6]{daskalopoulos2020transverse}\footnote{The proof that such a section exists does not use Theorem \ref{infinity harmonic laminations}.}, there exists an affine bundle $E \to M$ and a section $v$ of least gradient of $E$ such that $\supp \dif v \subseteq \lambda_u$.
By Theorem \ref{main thm}, $\supp \dif v$ is a geodesic lamination.
\end{proof}

It is not clear that $\supp \dif v = \lambda_u$ in the above corollary, but see Conjecture \ref{two laminations agree}.

Daskalopoulos--Uhlenbeck also ask for \cite[Problem 9.7]{daskalopoulos2020transverse} a partial converse to the fact that $\dif v$ endows $\lambda_u$ with the structure of an oriented measured lamination, which we now prove.
For the definitions, see \cite[\S8]{daskalopoulos2020transverse} or the original paper of Ruelle--Sullivan \cite{Ruelle75}.

\begin{corollary}\label{ruelle sullivan antiderivative}
Let $\lambda$ be an oriented, transversely measured geodesic lamination on a closed hyperbolic surface $M$, and let $\dif v$ be the Ruelle-Sullivan $1$-current induced by $\lambda$.
Then $\dif v$ is the derivative of a section of least gradient $v: M \to E$, for some affine bundle $E \to M$.
\end{corollary}
\begin{proof}
As observed in \cite[\S9]{daskalopoulos2020transverse}, if we lift $\dif v$ to a $1$-current $\dif \tilde v$ on $\Hyp^2$, then $\dif \tilde v$ is exact and any antiderivative $\tilde v$ of $\dif \tilde v$ has superlevel sets $\{\tilde v \geq y\}$ which are bounded by geodesics.
Moreover we can choose $\tilde v$ to be $\pi_1(M)$-equivariant.
The claim now follows from Proposition \ref{minimal bounding implies least gradient} and the realization of $\pi_1(M)$-equivariant functions on $\Hyp^2$ as sections of an affine bundle \cite[\S4]{daskalopoulos2020transverse}.
\end{proof}

The analogous theory for threefolds is also quite interesting, as closed minimal surfaces in hyperbolic threefolds have been studied intensely since the seminal work of Uhlenbeck \cite{Uhlenbeck1983ClosedMS}.
Daskalopoulos--Uhlenbeck have suggested \cite[Problem 9.13]{daskalopoulos2020transverse} that for a line bundle $E \to M$ over a closed hyperbolic threefold $M$, it would be particularly fruitful to study sections of $E$ of least gradient.
However, we shall not pursue this line of inquiry here.

Theorem \ref{main thm} also furnishes a large class of minimal laminations which are inexpensive to numerically compute.

\begin{proposition}\label{cohomology makes laminations}
Let $M$ be a closed hyperbolic manifold of dimension $d \leq 7$, and let $\alpha \in H^1(M, \RR)$.
Then there is a natural minimal lamination $\lambda$ in $M$ induced by $\alpha$, which can be obtained by solving a least-gradient Dirichlet problem on a fundamental polytope of $M$.
\end{proposition}
\begin{proof}
Let $\Gamma = \pi_1(M)$. The Hurcewiz theorem implies that $\alpha$ pulls back to a map $\Gamma \to \RR$.
Then $\alpha$ is a representation of $\Gamma$ and thus induces a space $E$ of equivariant functions on the boundary $\Omega$ of each fundamental polytope $\Omega$ of $\Gamma$. To be more precise, for every $f \in E$ and $\gamma \in \Gamma$,
\begin{equation}\label{boundary data for Loisel}
f(\gamma x) = f(x) + \alpha(\gamma),
\end{equation}
and $f$ is constant on each face of $\partial \Omega$.
The relation (\ref{boundary data for Loisel}) is an underdetermined boundary condition for functions in $E$, and so we consider the subspace $E'$ of functions which in addition are zero on a maximal set of faces such that we do not determine $f|\partial \Omega = 0$, thus any function $E' \subseteq E$ has a completely determined trace $t$ on $\partial \Omega$.
It follows from \cite[Theorem 4.4]{daskalopoulos2020transverse} that there exists a function $u$ of least gradient on $\Omega$ whose trace is $t$.
The minimal lamination induced by a function $u \in E$ of least gradient from Theorem \ref{main thm} does not depend on the class of $u$ in $E/E'$, since that is a space of constants; so we obtain a uniquely defined minimal lamination of $M$.
\end{proof}

\begin{corollary}
With $M, \alpha, \lambda$ as in Proposition \ref{cohomology makes laminations}, and every quasiuniform triangulation $T$ of $M$, there exists a numerical algorithm for computing $\lambda$ with finite elements in $T$ with time complexity $O(|T|^{1/2} \log |T|)$ where $|\cdot|$ is cardinality.
\end{corollary}
\begin{proof}
By \cite[Theorem 1]{Loisel20}, one can solve the least-gradient Dirichlet problem with finite elements in $T$ with time complexity $O(|T|^{1/2} \log |T|)$.
\end{proof}

TODO: Do some numerical experiments, show what minimal laminations in a fundamental polytope in $\Hyp^3$ look like


%%%%%%%%%%%%%%%%%%%%%%%%%%%%%%%%%%%%%%%%%%%%%%%

\subsection{Some open problems}
% We believe that Theorem \ref{main lma} could be extended to a wider class of Riemannian manifolds:
%
% \begin{conjecture}\label{main conj}
% Suppose that $2 \leq d \leq 7$ and $M$ is a simply connected Riemannian $d$-fold. Then Theorem \ref{main lma} holds with $\Hyp^d$ replaced by $M$.
% \end{conjecture}
%
% This conjecture implies Theorem \ref{main thm} for arbitrary $M$ by passing to the universal cover.
% Since our proof strongly uses the symmetries of $\Hyp^d$, it would be reasonable to first try to prove Conjecture \ref{main conj} for locally homogeneous manifolds, and especially those locally homogeneous manifolds $M$ such that for every $P \in M$ there is a natural action of $\Orth(\RR^d)$ on $M$ by isometry that fixes $P$.
% Another line of attack would be to show that our methods are perturbative, so that if $(M, g)$ is a simply connected Riemannian manifold such that $g$ is ``close to constant negative curvature'' in some sense, then Conjecture \ref{main conj} holds for $(M, g)$.

It seems highly unlikely that Theorem \ref{main lma} can be extended to any manifold $M$ of dimension $8$, due to the existence of Simons cones in $\RR^8$ \cite[Theorem A]{BOMBIERI1969}.

\begin{problem}
    For each Riemannian manifold $M$ of dimension $8$, construct a set $U$ of least perimeter in $M$ such that $\partial U$ has a singularity of codimension $8$.
\end{problem}

Owing to the G\'orny regularity theorem, Theorem \ref{Gorny regularity}, we suspect that its consequence \cite[Theorem 1.1]{górny2017planar} holds, giving well-posedness for the Dirichlet problem in strictly convex planar domains.
However, this is not obvious as we have not extended the Sternberg--Williams--Ziemer theorem \cite{ZiemerWilliamsSternberg1992} to the hyperbolic case.

\begin{conjecture}
Let $\overline M$ be an bounded, convex Riemannian manifold with nonempty boundary.
Then there exists a solution of the least-gradient Dirichlet problem on $M$ with data in $BV(\partial M)$.
\end{conjecture}

We would also like to know that the $\infty$-harmonic/least-gradient duality gives a complete proof of Theorem \ref{infinity harmonic laminations}, which follows from the below conjecture.

\begin{conjecture}\label{two laminations agree}
Let $u$ be an $\infty$-harmonic function with dual least-gradient section $v$.
Then $\supp \dif v = \lambda_u$.
\end{conjecture}


%%%%%%%%%%%%%%%%%%%%%%%%%%%%%%%%%%%%%%%%%%%%%%%%

\subsection{Acknowledgements}
I would like to thank Georgios Daskalopoulos for suggesting this project and for many helpful discussions.
I would also like to thank Trent Lucas for help with the proof of Proposition \ref{cohomology makes laminations} and Joshua Lin for help with the proof of Proposition \ref{construction of Plateau energy}.

%%%%%%%%%%%%%%%%%%%%%%%%%%%%%%%%%%%%%%%%%%%%%%%%%%

\section{Functions of least gradient}\label{LeastGradientFunctions}
\subsection{Notation}
The statements $A \lesssim B$ and $A = O(B)$ are equivalent and mean that there exists $C > 0$ independent of $A, B \geq 0$ such that $A \leq BC$.
Similarly the statements $A \ll B$ and $A = o(B)$ mean that there exists a function $f: \RR_+ \to \RR_+$ with $\lim_{x \to 0} f(x) = 0$, such that $A = f(B)$.

The operator $\star$ is the Hodge star, thus $\star 1$ is the Riemannian volume form.
For tensor fields we shall use the musical isomorphisms $\sharp, \flat$ and the Einstein convention.
When using the Einstein convention, Greek indices range over $0, \dots, d - 1$ while Latin indices range over $1, \dots, d - 1$.
However, all of our metrics are Riemannian, rather than Lorentz.

%%%%%%%%%%%%%%%%%%%%%%%%%
\subsection{Riemannian measure theory}
Let us recall some measure-theoretic facts.
See \cite[Chapter 1]{Giusti77} for analogous results over $\RR^d$, and see \cite{simon1983GMT} for the definition of a de Rham current.
We write $\int_U \omega \wedge \psi$ for the pairing of a de Rham $\ell$-current $\omega$ with a compactly supported $\ell$-form $\psi$ in an open set $U$.
We identify the distributional derivative of a function $u$ with the $d-1$-current
$$\int_U \dif u \wedge \psi = -\int_U u \dif \psi.$$
A function $u$ is in $BV(U)$ iff its derivative $\dif u$ has finite total variation
\begin{equation}\label{total variation}
\int_U \star |\dif u| := \sup_{\substack{||\psi||_{L^\infty} \leq 1\\\supp \psi \Subset V}} \int_U \dif u \wedge \psi.
\end{equation}
Whether a current has locally finite total variation is independent of the Riemannian metric and so $BV_\loc(M)$ is also independently defined.

Now let $u \in BV_\loc(M)$.
Then by \cite[Theorem 4.14]{simon1983GMT}, there exists a $\star |\dif u|$-measurable section $f$ of the cosphere bundle $S'M$ such that for every compactly supported $d-1$-form $\psi$,
\begin{equation}\label{RNy formula}
\int_M \dif u \wedge \psi = \int_M f|\dif u| \wedge \psi.
\end{equation}

For a vector field $X$, we write $\star (Xu) := \dif u \wedge \star (X^\flat)$.
The section $f$ of (\ref{RNy formula}) is given pointwise $\star |\dif u|$-almost everywhere, in any local coordinates $(y^\mu)$, by
\begin{equation}\label{Lebesgue point definition}
    f(x) = \left[\lim_{r \to 0} \frac{\int_{B(x, r)} \star \partial_\mu u}{\int_{B(x, r)} \star |\dif u|}\right] ~\dif y^\mu,
\end{equation}
according to the Besicovitch differentiation theorem; here we view $(\dif y^\mu)$ as a basis of $T'_xM$.
Whether the limit $f(x)$ in (\ref{Lebesgue point definition}) exists, or indeed its value as a point of $S'_xM$, do not depend on the Riemannian metric or the choice of coordinates.

\begin{definition}
The points $x$ for which the limit (\ref{Lebesgue point definition}) exists and satisfies $|f(x)| = 1$ are called the \dfn{Lebesgue points} of $\dif u$.
\end{definition}

\begin{definition}
Let $U$ be a set of locally finite perimeter, and let $u = 1_U$. Then:
\begin{enumerate}
\item The \dfn{measure-theoretic boundary} $\partial U$ is the set of points whose Lebesgue density with respect to $M$ is $\in (0, 1)$.
\item The set of Lebesgue points of $\dif u$ is the \dfn{reduced boundary} $\partial^* U$.
\item The $*|\dif u|$-measurable $1$-form $f$ defined by (\ref{Lebesgue point definition}) is the \dfn{conormal $1$-form} $\normal_U$ to $\partial U$.
\end{enumerate}
\end{definition}

Our definition of reduced boundary and conormal $1$-form follows \cite[Definition 3.3]{Giusti77} and is due to \cite{deGiorgi55}.
See \cite{Battista_2021} for an equivalent definition of reduced boundary on Riemannian manifolds, and see \cite[Chapter 6]{Pugh02} for the definition of Lebesgue density.

\begin{proposition}\label{locality of Caccioppoli}
    Let $U$ be a set of locally finite perimeter with conormal $1$-form $\normal$.
    Then:
    \begin{enumerate}
    \item $\partial^* U$ is either empty or $d-1$-dimensional in the Hausdorff sense, and is $d-1$-rectifiable.
    \item $\partial^* U$ is a dense subset of $\partial U$.
    \item If $\normal$ extends to a continuous $1$-form on $\partial U$, then $\partial^* U = \partial U$ is a $C^1$ hypersurface.
    \item If $\partial^* U = \partial U$ is a smooth hypersurface, then $\normal$ is the conormal $1$-form on $\partial U$ as defined in differential topology, and $\star |\dif 1_U|$ is the induced volume form on $\partial U$.
\end{enumerate}
\end{proposition}
\begin{proof}
Most of the assertions of this proposition are diffeomorphism-invariant, so we may assume that $M = \RR^d$ and appeal to \cite[Chapters 2-4]{Giusti77}.
The proof that $\star |\dif 1_U|$ is the induced volume form is identical to \cite[Example 1.4]{Giusti77}.
\end{proof}

\begin{definition}
Let $M$ be a Riemannian manifold, let $U$ be a set of locally finite perimeter, and let $E$ be a Borel set.
The \dfn{perimeter} of $U$ in $E$ is
$$|E \cap \partial^* U| := \int_E \star |\dif 1_U|.$$
\end{definition}

\begin{proposition}[coarea formula]\label{Coarea2}
Let $M$ be a Riemannian manifold and $u \in BV_\loc(M)$. Then for every open set $E$,
\begin{equation}\label{coarea formula}
\int_E \star |\dif u| = \int_{-\infty}^\infty |E \cap \partial \{u > y\}| \dif y.
\end{equation}
\end{proposition}
\begin{proof}
We follow \cite[Theorem 1.23]{Giusti77}, which first proves (\ref{coarea formula}) for $u \in C^\infty(\RR^d)$ using piecewise linear functions.
Such functions are not available for our purposes; instead we note that if $u \in C^\infty(\RR^d)$ and $u$ has no critical points then (\ref{coarea formula}) follows from Fubini's theorem, the fact that $|E \cap \partial \{u > y\}|$ is the surface area of $E \cap \{u = y\}$ (by Proposition \ref{locality of Caccioppoli}), and the change-of-variables formula.
However the left-hand side of (\ref{coarea formula}) is unaffected by critical points of $u$, and the right-hand side of (\ref{coarea formula}) is unaffected by critical values of $u$ by Sard's theorem.
So (\ref{coarea formula}) holds for $u \in C^\infty(\RR^d)$.

The rest of the proof is identical to \cite[Theorem 1.23]{Giusti77}, so we omit the details.
Taking a sequence in $C^\infty(M)$ that converges to $u$ in $L^1_\loc(M)$\footnote{Recall that $C^\infty(M)$ is not dense in $BV_\loc(M)$.}, and applying Fatou's lemma and the semicontinuity of total variation, we conclude the $\geq$ direction of (\ref{coarea formula}).
Moreover, Stokes' theorem gives that for every $d-1$-form $\psi$ such that $||\psi||_{L^\infty} \leq 1$ and $\supp \psi \Subset E$,
$$\int_E u \wedge \dif \psi = \int_{-\infty}^\infty \int_E |\psi| \star |\dif 1_{\partial \{u > y\}}| \dif y \leq \int_{-\infty}^\infty |E \cap \partial \{u > y\}| \dif y.$$
Taking the supremum over $\psi$ we obtain the direction $\leq$ in (\ref{coarea formula}).
\end{proof}

%%%%%%%%%%%%%%%%%%%%

\subsection{Trace and stability}
We now assert the trace theorem for $BV$ functions and stability theorem for least-gradient functions.
We also recall some a priori estimates for least-gradient functions.

\begin{proposition}[Miranda trace theorem]\label{traces}
Let $U \Subset M$ be an open set with Lipschitz boundary.
For every $u \in BV_\loc(U)$ there exists $v \in L^1_\loc(\partial U)$ such that for every $d-1$-form $\psi$,
\begin{equation}\label{Miranda IBP}
\int_U \dif u \wedge \psi + \int_U u \wedge \dif \psi = \int_{\partial U} v\psi.
\end{equation}
Moreover, for almost every $x \in \partial U$,
\begin{equation}\label{convergence of trace}
\int_{U \cap B(x, \varepsilon)} \star |v(x) - u| \ll \varepsilon^d.
\end{equation}
\end{proposition}
\begin{proof}
The assertion (\ref{Miranda IBP}) is diffeomorphism-invariant and so follows from \cite[Teorema 1]{Miranda67}, and (\ref{convergence of trace}) also follows from that result if we are willing to drop a constant factor.
\end{proof}

To state our a priori estimates we define
$$\eta(u, U) := \inf_{v \in BV_\cpt(U)} \int_U \star |\dif(u + v)|$$
for $u \in BV_\loc(M)$ and $U \Subset M$, so that $u$ has least gradient iff $\eta(u, U) = \int_U \star |\dif u|$ for every $U$.

Suppose that $u, v \in BV_\loc(M)$ and $U \Subset M$ is bounded by a Lipschitz hypersurface $N$. Armed with the Miranda trace theorem, it is straightforward to generalize \cite[Lemma 5.6]{Giusti77}, thus
\begin{equation}
|\eta(u, U) - \eta(v, U)| \leq ||u - v||_{L^1(N)}. \label{a priori estimate 1}
\end{equation}
In case $v = 0$, we note that by (\ref{convergence of trace}), the trace map is a contraction in $L^\infty$ norm, thus
\begin{equation}
\eta(u, U) \leq ||u||_{L^1(N)} \leq |N| \cdot ||u||_{L^\infty(M)}. \label{a priori estimate 2}
\end{equation}

\begin{definition}
A sequence $(u_n)$ in $BV_\loc(M)$ has \dfn{approximately least gradient} if for every open $U \Subset M$,
$$\limsup_{n \to \infty} \int_U \star |\dif u_n| \leq \liminf_{n \to \infty} \eta(u_n, U).$$
\end{definition}

\begin{proposition}[Miranda stability theorem]\label{Miranda convergence}
If a sequence of functions $(u_n)$ has approximately least gradient and $u_n \to u$ in $L^1_\loc(M)$, then $u$ has least gradient, and for every open set $U \Subset M$ with Lipschitz boundary such that $\int_{\partial U} \star |\dif u| = 0$, one has
\begin{equation}\label{convergence in total variation}
\lim_{n \to \infty} \int_U \star |\dif u_n| = \int_U \star |\dif u|.
\end{equation}
\end{proposition}
\begin{proof}
The proof is similar to Teorema 3 and Osservazione 3 in \cite{Miranda67}; we just note the necessary modifications.
Suitable generalizations of Teorema 2 and Osservazione 2 follow from Proposition \ref{traces}.
One needs to add a term of size $o(1)$ to the right-hand side of the inequalities (2.8), (2.9), (2.13), and (2.14); however, in the limit, this term vanishes and so the conclusions (2.15) and (2.16) are unaffected.
\end{proof}

The somewhat unusual condition $\int_{\partial U} \star |\dif u| = 0$ refers to the same Radon measure $\star |\dif u|$ that acts on the open sets of $M$, not on a measure that acts on the relatively open subsets of $\partial U$.
It should be interpreted as a transversality condition: if $u$ is the indicator function of a set $Z$ with $C^\infty$ boundary, then $\int_{\partial U} \star |\dif u| = 0$ if $\partial U$ is transverse to $\partial Z$.

The Miranda stability theorem and the compactness of the natural map $BV_\loc \to L^1_\loc$ imply the following.

\begin{corollary}\label{compactness}
Every sequence $(u_n)$ of approximately least gradient converges in $L^1_\loc$ and almost everywhere along a subsequence to a function of least gradient $u$ such that for every open set $U \Subset M$ of Lipschitz boundary such that $\int_{\partial U} \star |\dif u| = 0$, one has (\ref{convergence in total variation}).
\end{corollary}

\begin{proposition}\label{level sets are minimal}
For every function $u$ of least gradient, the superlevel sets $\{u > y\}$ have least perimeter.
If we instead have a sequence $(u_n)$ of approximately least gradient, then $(\{u_n > y\})$ has approximately least perimeter.
\end{proposition}
\begin{proof}
In the proof of \cite[Theorem 1]{BOMBIERI1969}, replace the coarea formula \cite[Theorem 1.6]{Miranda66} with Proposition \ref{Coarea2} and replace \cite[Teorema 3]{Miranda67} with Proposition \ref{Miranda convergence}.
\end{proof}

\begin{proposition}\label{minimal bounding implies least gradient}
Let $u \in BV_\loc(M)$, and suppose that the level sets $\partial \{u > y\}$ define a minimal lamination.
Then $u$ is a function of least gradient.
\end{proposition}
\begin{proof}
Fix $U \Subset M$ with Lipschitz boundary, let $T: BV(U) \to L^1(\partial U)$ be the trace map, and let $v$ be a competitor in $U$, thus $v \in BV(U)$ and $Tu = Tv$, so in particular for every $y \in \RR$, $\{Tu > y\} = \{Tv > y\}$.
But for every $w \in BV(U)$, $T(1_{\{w > y\}})$ indicates the set of $x \in \partial U$ such that $w > y$ in a neighborhood of $x$.
By (\ref{convergence of trace}), this is the set $\{Tw > y\}$.
Therefore $T(1_{\{u > y\}}) = T(1_{\{v > y\}})$, so $\partial^* \{v > y\}$ is a competitor to $\partial^* \{u > y\}$ in $U$.
Since $\partial^* \{u > y\}$ is minimal,
\begin{equation}\label{laminationwise least gradient}
|\partial^* \{u > y\} \cap U| \leq |\partial^* \{v > y\} \cap U|.
\end{equation}
We now integrate both sides of (\ref{laminationwise least gradient}) against $\dif y$ and apply Proposition \ref{coarea formula} to see that $\int \star |\dif u| \leq \int \star |\dif v|$.
\end{proof}

%%%%%%%%%%%%%%%%%%%%%%%%%%%%%%%%%%%%%%%%%%%%

\subsection{Blowup of the reduced boundary}
Now let us study the blowup of $M$ at a point $p$ on the reduced boundary of a set $U$ of least perimeter, giving a generalization of the conjunction of \cite[Theorem 9.3]{Giusti77} and \cite[Theorem 6.2.2]{Simons68}.

\begin{definition}
For a function $u$ on $M$, $P \in M$ we define the \dfn{tangent rescaling} of $u$ at $P$ to be the net of functions $u_t: T_PM \to \RR$, given by
$$u_t(v) = u\left(\exp_P(tv)\right)$$
\end{definition}

\begin{proposition}\label{blowup theorem}
Suppose that $U$ is an open set with least perimeter in $B(P, r)$, $P \in \partial^* U$, and $u = 1_U$.
Furthermore, suppose that $d \leq 7$.
Then the tangent rescaling of $u$ converges as $t \to 0$ along a subsequence (that we also denote $t \to 0$) in $L^1_\loc$ and almost everywhere, to the indicator function $v$ of a half-space $C \subset T_PM$ such that $0 \in \partial C$.
Moreover, for every open set $V \Subset T_PM$ of Lipschitz boundary such that $\partial V$ is transverse to $\partial C$,
$$\lim_{t \to 0} \int_V \star |\dif u_t| = \int_V \star |\dif v|.$$
\end{proposition}
\begin{proof}
We claim that the tangent rescaling $(u_t)$ has approximately least gradient in $T_PM$ (where we give $T_PM$ its euclidean metric). If this true, then by Corollary \ref{compactness}, there exists a set $C$ of least perimeter in $T_PM$, such that the tangent rescaling converges to $v := 1_C$ in the desired sense.
But $T_PM$ is isometric to $\RR^d$, $d \leq 7$, so by the Bernstein--Fleming theorem \cite[Theorem 17.3]{Giusti77}, $\partial C$ is a hyperplane.
The fact that $0 \in \partial C$ follows from the fact that $P \in \partial^* U$.

To prove the claim, write $|\cdot|'$, $\star'$ for the notions taken in the tangent space with its euclidean geometry, and write $U_t$ for the set indicated by $u_t$.
If $V$ is a precompact open subset of $T_PM$, $V_t = \{v \in T_PM: tv \in V\}$, then we have the scale-invariance
\begin{equation}\label{almost blowup scale invariance}
|\partial^* U_t \cap V|' = t^{1 - d}|\partial^* U_1 \cap V_{1/t}|'.
\end{equation}
From (\ref{almost blowup scale invariance}) and the Taylor expansion of $g$ in normal coordinates \cite[Lemma 3.4]{schoen1994lectures},
$$t^{d - 1} |\partial^* U_t \cap V|' = |\partial^* U_1 \cap V_{1/t}|' \leq e^{O(t^2)} |\partial^* U \cap \exp_P(V_{1/t})|.$$
For every $w \in BV_\cpt(V)$, the least-gradient nature of $u$ gives
$$|\partial^* U \cap \exp_P(V_{1/t})| \leq \int_{(\exp_P)_* V_{1/t}} \star '|\dif(u + (\exp_P)_* w_{1/t})| \leq e^{O(t^2)} \int_{V_{1/t}} \star'|\dif(u_1 + w_{1/t})|.$$
Therefore, after applying (\ref{almost blowup scale invariance}) and the Taylor expansion again,
$$|\partial^* U_t \cap V|' \leq e^{O(t^2)} t^{1 - d} \int_{V_{1/t}} \star' |\dif (u_1 + w_{1/t})| = e^{O(t^2)} \int_V \star' |\dif (u_t + w)|.$$
Since $V,w$ were arbitrary, we conclude that $(u_t)$ has approximately least gradient.
\end{proof}

%%%%%%%%%%%%%%%%%%%%%%%%%%%%%%%%%%%%%%%%%%%%%%%%%%%
\section{Monotonicity and mollification}\label{MollifierSection}
We have two purposes in this section: to prove Theorem \ref{monotonicity prestate} and to show that we can approximate minimal perimeters by $C^1$, approximately minimal perimeters.
Fix normal coordinates $(x^\mu)$, $\mu = 0, \dots, d - 1$, centered on a point $P \in M$ and define the closed $d-1$-form
\begin{equation}\label{d1 form}
\psi := \dif x^1 \wedge \dif x^2 \wedge \cdots \wedge \dif x^{d - 1}.
\end{equation}
We also write $B_r := B(P, r)$.
We use spherical coordinates $(\theta^i)$, $i = 1, \dots, d - 1$, on each sphere $\partial B_r$ which are compatible with $(x^\mu)$ in the sense that $x^0 = r \cos \theta^1$, which is possible because
\begin{equation}\label{partial Br is a variety}
\partial B_r = \{(x^0)^2 + \cdots + (x^{d - 1})^2 = r^2\}.
\end{equation}
We write $\dif \sigma$ for the standard measure on $\Sph^{d - 1}$.

\begin{remark}\label{independence of constants}
Let $K$ be a compact set in $M$.
Then all below implied constants may be chosen independently of the choice of $(x^\mu)$ as long as $P \in K$.
This holds because the space of all normal coordinate systems with basepoint in $K$ is the compact set $K \times \SpOrth_d$.
\end{remark}

%%%%%%%%%%%%%%%%%%%%%%%%%%%%%%%%%
\subsection{Monotonicity formula}
To prepare for the monotonicity formula, we first generalize an estimate that can be isolated from the proof of \cite[Lemma 5.8]{Giusti77}.

\begin{lemma}\label{monotonicity lemma}
Let $u \in C^1(B_R)$, $0 < r_1 < r_2 < R$, and let
$$E(r) = \int_{B_r} \star |\dif u| - \eta(u, r),$$
so that $E(R) = 0$ iff $u$ has least gradient. Then there exists $A \geq 0$ such that for $R > 0$ small,
\begin{equation}\label{monotonicity lemma eqn}
0 \leq \int_{B_{r_2} \setminus B_{r_1}} \star r^{1 - d}\frac{(\partial_ru)^2}{|\dif u|} \leq 2\int_{r_1}^{r_2} \partial_r \left[e^{Ar^2} r^{1-d}\int_{B_r} \star |\dif u|\right] + \frac{O(E(r))}{r^d} \dif r.
\end{equation}
\end{lemma}
\begin{proof}
We fix $s \in [r_1, r_2]$, introduce a competitor $v(r, \theta) = u(s, \theta)$, and allow $A \geq 0$ to be a constant which may vary from line to line.
From the definition of $\eta$,
\begin{equation}\label{consequence of least gradient monotone}
    \eta(u, s) \leq \int_U \star |\dif v| = \int_0^s \int_{\partial B_r} \star_r |\dif v| \dif r.
\end{equation}
Taylor expanding the volume form and the metric \cite[Lemma 3.4]{schoen1994lectures}, and applying $\partial_r v = 0$, we obtain the existence of $A \geq 0$ such that
\begin{equation}\label{introduce the ricci tensor}
\int_{\partial B_r} \star_r |\dif v| \leq e^{As^2} \frac{\tilde r^{d - 1}}{s^{d - 1}} \int_{\partial B_s} \star_s |\dif v|.\
\end{equation}
Applying (\ref{consequence of least gradient monotone}) and Fubini's theorem,
\begin{align*}
\eta(u, s) &\leq  e^{As^2} \int_0^s \frac{r^{d - 1}}{s^{d - 1}} \dif r \cdot \int_{\partial B_s} \star_s |\dif v| = \frac{s e^{As^2}}{d} \int_{\partial B_s} \star_s |\dif v|\\
&\leq \frac{s e^{As^2}}{d - 1} \int_{\partial B_s} \star_s |\dif v|.
\end{align*}
By Gauss' lemma, $\dif v$ is the orthogonal projection of $\dif u$ onto $T' \partial B_s$, and its orthocomplement is $\partial_r u$. Therefore by Taylor's theorem,
$$\int_{\partial B_s} \star_s |\dif v| \leq \int_{\partial B_s} \star_s |\dif u| \sqrt{1 - \frac{(\partial_r u)^2}{|\dif u|^2}} \leq \int_{\partial B_s} \star_s \left[|\dif u| - \frac{(\partial_r u)^2}{2 |\dif u|}\right]$$
or in other words
\begin{align*}
\int_{\partial B_s} \star_s \frac{(\partial_r u)^2}{2|\dif u|} &\leq \int_{\partial B_s} \star_s |\dif u| - \frac{d - 1}{s} e^{-As^2} \eta(u, s)\\
&\leq \int_{\partial B_s} \star_s |\dif u| - \frac{d - 1}{s} e^{-As^2} \int_{B_s} \star |\dif u| - O(s^{-1}E(s)).
\end{align*}
We moreover have for $\tilde A \geq 0$ that
$$e^{-\tilde As^2} \partial_s \left[e^{\tilde As^2} s^{1 - d} \int_{B_s} \star |\dif u|\right] = \left[2\tilde As^{2 - d} - \frac{d - 1}{s^d}\right]\int_{B_s} \star |\dif u| + s^{1 - d} \int_{\partial B_s} \star_s |\dif u|$$
so if we choose $\tilde A$ so that
$$-\frac{d - 1}{s} e^{-As^2} = 2\tilde As - \frac{d - 1}{s}$$
then
$$s^{1 - d} \int_{\partial B_s} \star_s |\dif u| - (d - 1)\frac{e^{-As^2}}{s^d} \int_{B_s} \star|\dif u| \leq e^{-As^2} \partial_s\left(e^{As^2} s^{1 - d} \int_{B_s} \star|\dif u|\right).$$
We moreover have $e^{-As^2} \leq 1$, so we can now integrate with respect to $\dif s$ to conclude the result.
\end{proof}

\begin{proposition}[monotonicity formula]\label{Monotonicity Formula}\label{sharp monotonicity}
There exists $A \geq 0$ (depending continuously on $P$) such that for every function $u$ in least gradient in $B_R$ where $R$ is small, and every $0 < r < R$,
\begin{equation}\label{weak monotone}
\partial_r e^{Ar^2} r^{1 - d} \int_{B_r} \star|\dif u| \geq 0.
\end{equation}
Stronger,
\begin{align}\label{strong monotone}
&\left|\int_{r_1}^{r_2} \partial_r \left[r^{1 - d}\int_{B_r} \dif u \wedge \psi\right] \dif r\right|^2 \\
&\qquad \lesssim \left(1 + (d - 1)\log\frac{r_2}{r_1}\right)r_2^{1 - d}\int_{B_r} \star |\dif u| \int_{r_1}^{r_2} \partial_r \left[e^{Ar^2} r^{1 - d}\int_{B_r} \star |\dif u|\right] \dif r.
\end{align}
\end{proposition}
\begin{proof}
We first compute $\dif u \wedge \psi = \partial_0 u \dif x$
where $\dif x$ is the natural euclidean volume form on $T_PM$.
Moreover, the radial part of $\partial_0$ is $\cos \theta^1$, and $\iota_{\partial_r} \dif x = r^{d - 1} \dif \sigma$.
Thus by (\ref{partial Br is a variety}),
$$\int_{B_r} \dif u \wedge \psi = r^{d - 1}\int_{\partial B_r} u \cos \theta^1 \dif \sigma(\theta)$$
and hence, since $|\cos \theta^1| \leq 1$,
\begin{align}
\left|\int_{r_1}^{r_2} \partial_r \left[r^{1 - d}\int_{B_r} \dif u \wedge \psi\right] \dif r\right|
&= \left|\int_{\Sph^{d - 1}} (u(r_2, \theta) - u(r_1, \theta)) \cos \theta^1 \dif \sigma(\theta)\right| \\
&\leq \int_{\Sph^{d - 1}} |u(r_2, \theta) - u(r_1, \theta)| \dif \sigma(\theta). \label{monotone dump the metric}
\end{align}
The metric $g$ plays no role in (\ref{monotone dump the metric}), so we may use \cite[Lemma 5.3]{Giusti77} to bound
$$\int_{\Sph^{d - 1}} |u(r_2, \theta) - u(r_1, \theta)| \dif \sigma(\theta) \leq \int_{\Sph^{d - 1}} \int_{r_1}^{r_2} r^{1 - d}|\partial_r u(r, \theta)| \dif r \dif\sigma(\theta).$$
To reintroduce the metric we posit that $R$ is small enough that $\dif r \dif \sigma(\theta) \leq \star 2$.
We therefore have
\begin{equation}\label{monotone before cs}
\int_{\Sph^{d - 1}} \int_{r_1}^{r_2} r^{1 - d}|\partial_r u(r, \theta)| \dif r \dif\sigma(\theta) \leq 2 \int_{B_{r_2} \setminus B_{r_1}} \star r^{1 - d}|\partial_r u|
\end{equation}
and if we apply the Cauchy-Schwarz inequality and approximate $u$ by $C^1$ functions (see \cite[pg68]{Giusti77}), it follows from Lemma \ref{monotonicity lemma} that the right-hand side of (\ref{monotone before cs}) is
$$\lesssim \sqrt{\int_{B_{r_2} \setminus B_{r_1}} \star r^{1 - d} |\dif u|} \sqrt{\int_{r_1}^{r_2} \partial_r \left[e^{Ar^2} r^{1-d}\int_{B_r} \star |\dif u|\right] \dif r}.$$
The monotonicity (\ref{weak monotone}) follows at once. To strengthen it we just need to bound $r^{1 - d} |\dif u|$.
Integrating by parts,
\begin{align*}
\int_{B_{r_2} \setminus B_{r_1}} r^{1 - d} |\dif u| &= \int_{r_1}^{r_2} r^{1 - d} \partial_r \int_{B_r} \star |\dif u| \dif r \\
&\leq r^{1 - d} \int_{B_r} \star |\dif u| + (d - 1) \int_{r_1}^{r_2} r^{-d} \int_{B_r} \star |\dif u| \dif r.
\end{align*}
Using (\ref{weak monotone}) we bound this second integral as
\begin{align*}
\int_{r_1}^{r_2} r^{-d} \int_{B_r} \star |\dif u| \dif r &\leq r^{1 - d} \log \frac{r_2}{r_1} \int_{B_{r_2}} \star |\dif u|. \qedhere
\end{align*}
\end{proof}

In fact, if $g$ has negative Ricci curvature, then we can take $A = 0$ in (\ref{introduce the ricci tensor}) and hence in (\ref{strong monotone}).
However, we will never need this fact.

%%%%%%%%%%%%%%%%%%%%%%%%%%%%%%%%%%%%%%%%%%%%%%%%%%%%%%%
\subsection{Applications of the monotonicity formula}
We now give several consequences of the monotonicity formula; the surface area estimate generalizes \cite[Remark 5.13]{Giusti77}.
Write $\Ball^\ell$ for the unit ball in $\RR^\ell$.

\begin{lemma}\label{least perimeter minimal size}
For a set $U$ of least perimeter, if $P \in \partial^* U$ and $d \leq 7$, one has
$$\lim_{r \to 0} r^{1 - d} |\partial^* U \cap B(P, r)| = |\Ball^{d - 1}|.$$
\end{lemma}
\begin{proof}
Choose a sequence of $r \to 0$; then there is a subsequence along which the limit in Proposition \ref{blowup theorem} exists for $u = 1_U$.
With notation as in the proof of Proposition \ref{blowup theorem},
$$r^{1 - d} |\partial^* U \cap B(P, r)| = e^{O(r^2)} r^{1 - d}\int_{B'_r} \star'|\dif u_1|' = e^{O(r^2)} \int_{B'_1} \star'|\dif u_r|'.$$
Then $u_r \to 1_C$ for $C$ a half-space, which in particular is transverse to $B'_1$.
So by the Miranda stability theorem, Proposition \ref{Miranda convergence},
\begin{align*}
\lim_{r \to 0} e^{O(r^2)} \int_{B'_1} \star'|\dif u_r|' &= \int_{B'_1} \star'|\dif 1_C|' = |\partial C \cap B'_1|' = |\Ball^{d - 1}|. \qedhere
\end{align*}
\end{proof}

\begin{proposition}[surface area estimate]\label{doubling dimension}
If $d \leq 7$ then there exists $A \geq 0$ such that for every set $U$ of least perimeter in a ball $B_r = B(P, r)$, with $P \in \partial^* U$, and $r > 0$ small,
$$|\Ball^{d - 1}|e^{-Ar^2}r^{d - 1} \leq |\partial^*U \cap B_r| \leq |\Sph^{d - 1}|e^{Ar^2} r^{d - 1}.$$
\end{proposition}
\begin{proof}
The upper bound on $|\partial^* U \cap B_r|$ is obtained by using (\ref{a priori estimate 2}) and the fact that the surface area of $\partial B_r$ is $|\Sph^{d - 1}|(1 + O(r^2))r^{d - 1}$ (see e.g. \cite{gray1974volume}).
The lower bound is obtained from Proposition \ref{Monotonicity Formula}, which implies that
$$\lim_{\rho \to 0} e^{-A\rho^2} \rho^{1 - d} |\partial^* U \cap B_\rho| \leq |\partial^* U \cap B_r|.$$
The left-hand side is given by Lemma \ref{least perimeter minimal size}.
\end{proof}

\begin{proposition}[local finiteness]\label{local finiteness}
Let $U$ have least perimeter in $M$, $K \Subset M$ compact, and $d \leq 7$. Then only finitely many connected components of $\partial U$ meet $K$.
\end{proposition}
\begin{proof}
If not, then there exists $P \in K \cap \overline{\partial U}$ and a sequence of connected components $(D_n)$, such that there exists $P_n \in D_n \cap K$ with $P_n \to P$.
Since $\partial^* U$ is dense in $\partial U$ we can in fact take $P_n \in D_n^* := D_n \cap \partial^* U$.
Now we take $\varepsilon > 0$ so small that for $A$ as in Proposition \ref{doubling dimension}, $e^{-A\varepsilon^2} \geq 1/2$, and choose $N$ so large that if $n \geq N$ then $B_n := B(P_n, \varepsilon/2)$ is contained in $B := B(P, \varepsilon)$.
Thus
$$|\partial U \cap B| \geq \sum_{n=N}^\infty |D_n^* \cap B_n| \geq \frac{\varepsilon^{d - 1}}{2^d} \sum_{n=N}^\infty |\Ball^{d - 1}| = \infty$$
which violates that $U$ has locally finite perimeter.
\end{proof}

Local finiteness is not quite an obvious fact, since the measure-theoretic boundary need not be closed, and so even if we take $\partial U$ to be a union of lines, the lines could possibly cluster if $U$ is not assumed to have least perimeter.

%%%%%%%%%%%%%%%%%%%%%%%%%%%%%%%%%%%%%%%%%%%%%%%%%%%%%%%%%%%%%%%%
\subsection{Convolution estimates}
In what follows, the convolution $f * g$ of two functions defined near $P$, or the subtraction $x - y$ of two points near $P$, are meant in the sense of the normal coordinates $(x^\mu)$\footnote{and crucially, \emph{not} in terms of the normal coordinates $(\tilde x^\mu)$ that we introduce in the proof of Lemma \ref{mollifier sublemma}}. Following \cite[Chapter 7]{Giusti77} we define the convolution kernel
$$\chi_\varepsilon(x) = \frac{d + 1}{|\Ball^d|} \varepsilon^{-d}1_{|x| < \varepsilon} \left(1 - \frac{|x|}{\varepsilon}\right)$$
For this section only, we write $u_\varepsilon = u * \chi_\varepsilon$ whenever $u \in BV$ is defined near $P$.
TODO: Replace $\chi_\varepsilon$ with a Gaussian.

Our purpose now is to prove a Riemannian analogue of \cite[Theorem 7.3]{Giusti77}, which estimates $(1_U)_\varepsilon$ for $U$ a set of least perimeter.
The argument there required one to cover $\partial^* U$ by small balls and apply the monotonicity formula in each ball.
To deal with the somewhat large number of parameters involved here, it may helpful to think of the case $\Delta = 1$, in which case we have 
$$1 \gg \gamma^q \gg \sigma \gg \gamma \gg \varepsilon \gg \varepsilon \delta > 0.$$

\begin{lemma}[control on $u_\varepsilon$ in each ball]\label{mollifier sublemma}
Let $q < \min(1/8, 1/(4(d - 1)))$.
For every $0 < \Delta, \gamma \lesssim 1$, if we let $\varepsilon = \gamma^4 \Delta$, $\sigma = \gamma^{1/(2(d - 1))} \Delta$, $\delta = \gamma^d$, and let $u$ be the indicator function of a set $U$ of least perimeter such that
\begin{equation}\label{hypothesis on mollifier sublemma}
\int_{B_\Delta} \star(|\dif u| - \dif u \wedge \psi) \leq \Delta^{d - 1} \gamma,
\end{equation}
then on $B_{\Delta - 2\sigma}$, if $Q \in \partial^* U$,
$$(1_{B(Q, 2\delta\varepsilon)}(|\dif u| - \star(\dif u \wedge \psi)))_\varepsilon \ll \gamma^q (1_{B(Q, \delta\varepsilon)} |\dif u|)_\varepsilon.$$
\end{lemma}
\begin{proof}
We first claim that for $r > 0$ so small that $B(Q, 2r) \subseteq B_\varepsilon$,
\begin{equation}\label{bound the kernel}
\sup_{y \in B(Q, 2r)} \chi_\varepsilon(x - y) \lesssim \inf_{y \in B(Q, r)} \chi_\varepsilon(x - y).
\end{equation}
In the euclidean case (with constant equal to $4$) this result can be isolated from the proof of \cite[Theorem 7.3]{Giusti77}.
Otherwise, we can use the smallness of $\varepsilon$ to approximate geodesic balls by euclidean balls.
This suffices to prove (\ref{bound the kernel}), since $\chi_\varepsilon$ is uniformly continuous.

Now let $V := B(Q, 2\delta\varepsilon)$.
Integrating (\ref{bound the kernel}) against $1_V(|\dif u| - \star(\dif u \wedge \psi))$,
\begin{equation}\label{kernel bounded}
(1_V(|\dif u| - \star(\dif u \wedge \psi)))_\varepsilon(x) \lesssim \inf_{y \in B(Q, \delta\varepsilon)} \chi_\varepsilon(x - y) \int_V \star |\dif u| - \dif u \wedge \psi.
\end{equation}
To estimate the right-hand side of (\ref{kernel bounded}) we introduce a new coordinate system $(\tilde x^\mu)$ of normal coordinates centered on $Q$ which are compatible with $(x^\mu)$ in the sense that $\dif \tilde x^0$ is a scalar multiple of $\dif x^0$ at $Q$.
We write $\tilde g$ for the metric written in these new coordinates, and write
$$\tilde \psi := \dif \tilde x^1 \wedge \cdots \wedge \dif \tilde x^{d - 1}.$$
Then a Taylor expansion gives the bound
\begin{align}
|\tilde \psi - \psi| &= |\star(\tilde \psi - \psi)| = |\sqrt{\det g} g^{11} \cdots g^{(d-1)(d-1)} \dif x^0 - \sqrt{\det \tilde g} \tilde g^{11} \cdots \tilde g^{(d-1)(d-1)} \dif \tilde x^0|\\
&\lesssim |\det g - \det \tilde g| + |g^{11} \cdots g^{(d-1)(d-1)} - \tilde g^{11} \cdots \tilde g^{(d-1)(d-1)}| + |\dif(x^0 - \tilde x^0)| \lesssim \varepsilon. \label{T vs Ttilde}
\end{align}
In particular,
\begin{equation}\label{split up T Ttilde}
\int_V \star |\dif u| - \dif u \wedge \psi \leq \int_V \star |\dif u| - \dif u \wedge \tilde \psi + O(\varepsilon) \int_V \star |\dif u|.
\end{equation}
The error term here is given by Proposition \ref{doubling dimension} and the assumption $\Delta \lesssim 1$ as $\lesssim \gamma^4 \int_{B(Q, \delta\varepsilon)} \star |\dif u|$.
To estimate the dominant term in (\ref{split up T Ttilde}) we assume that $\gamma$ is chosen so small that $\sigma > 2\delta\varepsilon$, so that if we set $W := B(Q, \sigma)$ and apply Proposition \ref{Monotonicity Formula} and (\ref{T vs Ttilde}) to obtain $A \geq 0$ such that
\begin{align*}
(2\delta\varepsilon)^{1 - d} \int_V \star |\dif u| - \dif u \wedge \psi
&\leq \sigma^{1 - d}\int_W \star |\dif u| + 2A\sigma^{3 - d} \int_W \star |\dif u| - (2\delta\varepsilon)^{1 - d}\int_V \dif u \wedge \psi\\
&\leq \sigma^{1 - d}\int_W \star |\dif u| - \dif u \wedge \psi + 2A\sigma^{3 - d} \int_W \star |\dif u| \\
&\qquad + O(\varepsilon \sigma^{1 - d}) \int_W \star |\dif u| + \sigma^{1 - d}\int_W \dif u \wedge \tilde \psi - (2\delta\varepsilon)^{1 - d}\int_V \dif u \wedge \tilde \psi\\
&=: I_1 + I_2 + I_3 + I_4 - I_5.
\end{align*}
We apply (\ref{hypothesis on mollifier sublemma}) to bound $I_1 \leq \gamma^{1/2}$.
By Proposition \ref{doubling dimension} and the assumption $\Delta \lesssim 1$, $I_2 \lesssim \gamma^{1/(d - 1)}$ and $I_3 \lesssim \gamma^4$.

To estimate $I_4 - I_5$, we apply Proposition \ref{sharp monotonicity}:
\begin{align*}
&\sigma^{1 - d} \int_W \dif u \wedge \tilde \psi - (2\delta\varepsilon)^{1 - d} \int_V \dif u \wedge \tilde \psi \\
&\qquad \lesssim \sqrt{1 + (d - 1) \log \frac{\sigma}{2\delta\varepsilon}} \sqrt{\sigma^{1 - d} \int_W \star |\dif u|} \sqrt{\int_{2\delta\varepsilon}^\sigma \partial_r \left[e^{Ar^2} r^{1 - d} \int_{B(Q, r)} \star |\dif u|\right] \dif r}\\
&\qquad =: J_1 J_2 J_3.
\end{align*}
We have $J_1 \lesssim -\log \gamma$, and from Proposition \ref{doubling dimension} we have $J_2 \lesssim 1$.
So, we need to get a gain from $J_3$, which we do as follows:
\begin{align*}
J_3^2 &\leq \sigma^{1 - d} \int_W \star |\dif u| - (2 \delta \varepsilon)^{1 - d} \int_V \star |\dif u| + 2A\sigma^{3 - d} \int_W \star |\dif u| \\
&= \sigma^{1 - d} \int_W \star |\dif u| - \dif u \wedge \psi + \sigma^{1 - d} \int_W \dif u \wedge (\psi - \tilde \psi) + \sigma^{1 - d} \int_W \dif u \wedge \tilde \psi \\
&\qquad - (2 \delta\varepsilon)^{1 - d} \int_V \star |\dif u| + 2A \sigma^{3 - d} \int_W \star |\dif u| \\
&=: K_1 + K_2 + K_3 - K_4 + K_5.
\end{align*}
Then $K_1 = I_1 \leq \gamma^{1/2}$, $K_2 \lesssim I_3 \lesssim \gamma^4$, and $K_5 = I_2 \lesssim \gamma^{1/(d - 1)}$.

To estimate $K_3 - K_4$ we observe that for $u = 1_U$,
\begin{equation}\label{K3 calculus}
K_3 = \sigma^{1 - d} \int_W \dif u \wedge \tilde \psi = \sigma^{1 - d} \int_{U \cap \partial W} \tilde \psi.
\end{equation}
We decompose
$$\partial W = \Gamma_+ \cup \Gamma_0 \cup \Gamma_-$$
where $\tilde x^0 > 0$ on $\Gamma_+$ and $\tilde x^0 < 0$ on $\Gamma_-$. Then all positive contributions to the integral in the right-hand side of (\ref{K3 calculus}) come from $\Gamma_+$.
Moreover, as $d-1$-cells in $M$, $\partial \Gamma_+ = \Gamma_0$, but also if we set $N = \{\tilde x^0 = 0\}$ and $W_0 = W \cap N$, then $\Gamma_0 = \partial W_0$.
In particular, there is a homotopy relating $\Gamma_+$ and $\partial W_0$ which holds $\Gamma_0$ fixed.
Since $\dif \psi = 0$, we can use (\ref{partial Br is a variety}) as follows:
\begin{align*}
K_3 &\leq \sigma^{1 - d} \int_{\Gamma_+} \tilde \psi = \sigma^{1 - d} \int_{W_0} \tilde \psi = |\Ball^{d - 1}|.
\end{align*}
Hence by Proposition \ref{doubling dimension},
$$K_3 \leq |\Ball^{d - 1}| \leq K_4 e^{O(\varepsilon\delta)^2} \leq K_4 + O(\varepsilon\delta)^2 \leq K_4 + O(\gamma^{8 + 2d}).$$
In conclusion, $J_3 \lesssim \gamma^{\min(1/4, 1/(2(d - 1)))}$, and hence by (\ref{kernel bounded}) we have
$$(1_V(|\dif u| - \star(\dif u \wedge \psi)))_\varepsilon(x) \ll (\delta\varepsilon)^{d - 1} \gamma^q \inf_{y \in B(Q, \delta\varepsilon)} \chi_\varepsilon(x - y).$$
Finally, by Proposition \ref{doubling dimension},
\begin{align*}
(\delta\varepsilon)^{d - 1} \inf_{y \in B(Q, \delta\varepsilon)} \chi_\varepsilon(x - y) &\lesssim (1_{B(Q, \delta\varepsilon)} |\dif u|)_\varepsilon(x). \qedhere
\end{align*}
\end{proof}

\begin{proposition}[control on $u_\varepsilon$]\label{main mollifier lemma}
Let $q < \min(1/8, 1/(4(d - 1)))$. There exists $c > 0$ such that for every $0 < \Delta \lesssim 1$ such that for every $0 < \gamma \lesssim 1$ and every indicator function $u$ of a set $U$ of least perimeter such that
\begin{equation}\label{hypothesis on main mollifier lemma}
\int_{B_\Delta} \star |\dif u| - \dif u \wedge \psi \leq \gamma \Delta^{d - 1},
\end{equation}
if we let $\varepsilon = \gamma^4\Delta$, $\sigma = \gamma^{1/(2(d - 1))}\Delta$, and $\varphi = u_\varepsilon$, then on $B_{\Delta - 2\sigma} \cap \{c\gamma^2 < \varphi < 1 - c\gamma^2\}$,
\begin{equation}\label{claim on main mollifier lemma}
(1 - o(\gamma^q)) |\dif \varphi| \leq \star(\dif \varphi \wedge \psi)
\end{equation}
and for every $y \in (c\gamma^2, 1 - c\gamma^2)$,
\begin{equation}\label{claim 2 on main mollifier lemma}
\text{the level set } \partial \{\varphi > y\} \cap B_{\Delta - 2\sigma} \text{ is a }C^1\text{ hypersurface}.
\end{equation}
\end{proposition}
\begin{proof}
Let $\delta = \gamma^d > 0$.
By running a greedy algorithm, we construct a cover $\mathcal V = \{V_n: 1 \leq n \leq N\}$ of $\partial^* U \cap B_{\varepsilon(1 - 2\delta)}$ by balls of radius $2\delta\varepsilon$, centered on points $Q_n \in \partial^* U \cap B_{\varepsilon(1 - \delta)}$, which is \dfn{efficient} in the sense that the dilates $V_n/2 := B(Q_n, \delta\varepsilon)$ are disjoint.
We set $V_0 := B_\varepsilon \setminus B_{\varepsilon(1 - 2\delta)}$.

Since $\dif u$ is supported in $\bigcup_n V_n$,
$$|\dif \varphi| - \star(\dif \varphi \wedge \psi) = (|\dif u| - \star(\dif \varphi \wedge \psi))_\varepsilon \leq \sum_{n=0}^N (1_{V_n}(|\dif u| - \star(\dif u \wedge \psi)))_\varepsilon.$$
From an argument identical to \cite[pg92]{Giusti77} there exists $c > 0$ such that on $B_\sigma \cap \{c\gamma^2 < \varphi < 1 - c\gamma^2\}$,
\begin{equation}\label{V0 case}
(1_{V_0}(|\dif u| - \star(\dif u \wedge \psi)))_\varepsilon \lesssim \gamma |\dif u|_\varepsilon.
\end{equation}
Since $\mathcal V$ is efficient, we can sum (\ref{V0 case}) and Lemma \ref{mollifier sublemma} over $n$ to show that (\ref{claim on main mollifier lemma}) holds.
Thus we may fix $x \in \partial^* U \cap B_\sigma$ such that $c\gamma^2 < \varphi(x) < 1 - c\gamma^2$, and observe that then $|\dif u(x)| > 0$.
So $\varphi$ is a $C^1$ submersion near $x$ by \cite[Lemma 7.1]{Giusti77}, thus (\ref{claim 2 on main mollifier lemma}) holds.
\end{proof}

% \begin{figure}[ht]\label{covering diagram}
% \caption{The sets $V_0, V_1/2 \dots, V_n/2$ (in dark grey) are an annulus and several small balls of radius $\delta \varepsilon$, which approximately cover the boundary of the set $U$ (in light grey).}
% \includegraphics[width=0.4\textwidth]{covering lemma}
% \end{figure}

% \begin{sublemma}
% On $B_{1 - 2\sigma} \cap \{c\gamma^2 < \varphi < 1 - c\gamma^2\}$,
% $$(1_{V_0}(|\dif u| - Tu))_\varepsilon \lesssim_p \gamma |\dif u|_\varepsilon.$$
% \end{sublemma}
% \begin{proof}
% It follows from the definitions that for every $P \in B_\varepsilon$ and $Q \in V_0$, $\chi_\varepsilon(P - Q) \lesssim \varepsilon^{-d} \delta$,
% whence, by Proposition \ref{doubling dimension},
% \begin{align*}
% (1_{V_0}(|\dif u| - x\partial_zu))_\varepsilon &\lesssim \frac{\delta}{\varepsilon^d} \int_{B_\varepsilon} *|\dif u| \lesssim \frac{\delta}{\varepsilon}.
% \end{align*}
% Let $P \in B_\varepsilon$. Then by Lemma \ref{Giusti71}, there exists $c > 0$ such that if $\varphi \in (c\gamma^2, 1 - c\gamma^2)$, then there exists $Q \in \partial^* U$ such that $d(P, Q) < \varepsilon(1 - \gamma)$.
% If $d(Q, R) < \gamma\varepsilon/2$, then $d(P, R) < \varepsilon - \gamma\varepsilon/2$ and so $\chi_\varepsilon(P - R) \gtrsim \varepsilon^{-d}\gamma$ whenever $R \in B(Q, \gamma\varepsilon/2) =: W_0(P)$.
% We thus estimate
% $$\varepsilon^{-1} \gamma^{d - 1} \lesssim (1_{W_0(P)}|\dif u|)_\varepsilon \leq |\dif u|_\varepsilon$$
% and hence we obtain
% $$(1_{V_0}(|\dif u| - x\partial_zu))_\varepsilon(P) \lesssim \delta \gamma^{1 - d} |\dif u|_\varepsilon$$
% uniformly in $P$. Since $\delta = \gamma^d$, the claim holds.
% \end{proof}

%%%%%%%%%%%%%%%%%%%%%%%%%%%%%%%%%%%%%
\subsection{Mollification of sets of least perimeter}
We now generalize \cite[Lemma 7.5]{Giusti77}, which reduces the study of sets of least perimeter to that of sets with $C^1$ perimeter.
On first reading it may help to fix $P_n = P$, $(x^\mu_n) = (x^\mu)$, $\omega = \star \dif x^\mu$, and $\Delta_n = 1$.

\begin{proposition}\label{mollifier quant}
Let $(P_n)$ be a precompact sequence in $M$ and $0 < \Delta_n \lesssim 1$.
Let $U_n$ be a set of least perimeter in $B_n := B(P_n, \Delta_n)$, $(x^\mu_n)$ be a normal coordinate system at $P_n$, let $\psi^n$ be given by (\ref{d1 form}), and 
$$\gamma_n := \Delta_n^{1 - d} \int_{B_n} \star |\dif 1_U| - \dif 1_U \wedge \psi^n.$$
If $(\gamma_n) \in \ell^1$ then for almost every $t \in (0, 1)$ and every $n \in \NN$ there exists a set $V_n$ with $C^1$ boundary in $tB_n := B(P_n, t\Delta_n)$ such that
\begin{align}
|V_n \cap tB_n| &\leq \eta(V_n, t\Delta_n) + o(\gamma_n \Delta_n^{d - 1}), \label{mollifier quant1}\\
||U_n \cap tB_n| - |V_n \cap tB_n|| &\ll \gamma_n \Delta_n^{d - 1}, \label{mollifier quant2}\\
g^{-1}(\normal_{V_n}, \dif x^0_n) &\geq 1 - O(|x_n|^2) - o(\gamma_n^q), \label{mollifier quant4}
\end{align}
and for every $d-1$-form $\omega_n$ defined near $P_n$,
\begin{equation}\label{mollifier quant3}
\left|\int_{tB_n} \dif(1_{U_n} - 1_{V_n}) \wedge \omega_n\right| \ll \gamma_n \Delta_n^{d - 1} ||\omega_n||_{C^1}.
\end{equation}
\end{proposition}
\begin{proof}
In this proof, we shall assume that the constants furnished by the above results are uniform in $n$; this is possible by Remark \ref{independence of constants}.

\proofpart{1}{Construction of $V_n$}
Draw $t$ uniformly at random, let $w_n := (u_n)_{\Delta_n \gamma_n^4}$, let $c, q$ be as in Proposition \ref{main mollifier lemma}, and let $a_n = c\gamma_n^2$, $b_n = 1 - c\gamma_n^2$.
By the coarea formula, Proposition \ref{Coarea2},
$$\int_{tB_n} \star |\dif w_n| = \int_0^1 |\partial^* \{w_n > y\} \cap tB_n| \dif y \geq \int_{a_n}^{b_n} |\partial^* \{w_n > y\} \cap tB_n| \dif y,$$
so by the mean value theorem, there exists $y_n \in (a_n, b_n)$ such that
\begin{equation}\label{MVT mollifier}
|\partial^* \{w_n > y_n\} \cap tB_n| \leq \frac{1}{b_n - a_n} \int_{tB_n} \star |\dif w_n|.
\end{equation}
We then set $V_n := \{w_n > y_n\}$, $v_n := 1_{V_n}$, so $V_n$ has $C^1$ boundary in $tB_n$ by (\ref{claim 2 on main mollifier lemma}).

\proofpart{2}{Proof of (\ref{mollifier quant4})}
Since $\grad w_n$ is normal to the level sets of $w_n$, $\normal^n := \normal_{V_n} = \dif w_n/|\dif w_n|$ and hence by (\ref{claim on main mollifier lemma}) and a Taylor expansion, on $t(1 - 2\sigma_n)B_n$, $\sigma_n := \gamma_n^{1/(2(d - 1))}$,
$$1 - o(\gamma^q_n) \leq \star(\normal^n \wedge \psi^n) \leq (1 + O(|x_n|^2)) g^{-1}(\normal^n, \dif x_n^0).$$
As the term $O(|x_n|^2)$ degenerates near the boundary of $t(1 - 2\sigma)B_n$ anyways, we can discard the constraint $|x_n| < t(1 - 2\sigma_n)\Delta_n$ and thus we have proven (\ref{mollifier quant4}).

\proofpart{3}{Auxiliary estimates}
Let $\Gamma_n := \partial(tB_n)$; we claim that almost surely,
\begin{align}
||u_n - v_n||_{L^1(\Gamma_n)} &\ll \Delta_n^{d - 1} \gamma_n \label{trace of vn} \\
|\partial V_n \cap tB_n| &\leq |\partial^* U_n \cap tB_n| + o(\Delta_n^{d - 1} \gamma_n). \label{difference of surface area}
\end{align}
To establish the claim, observe that by \cite[Lemma 7.2]{Giusti77} and Proposition \ref{doubling dimension},
$$\limsup_{n \to \infty} \Delta_n^{1 - d} \gamma_n^{-4} \int_{tB_n} \star |u_n - w_n| \leq \limsup_{n \to \infty} \Delta_n^{2-d} |\partial^* U_n \cap tB_n| \lesssim \sup_n \Delta_n \lesssim 1$$
and hence almost surely,
$$||u_n - v_n||_{L^1(\Gamma_n)} \lesssim \gamma_n^{-2} ||u_n - w_n||_{L^1(\Gamma_n)} \ll \Delta_n^{d - 1} \gamma_n.$$
So (\ref{trace of vn}) holds by \cite[Lemma 1.25]{Giusti77} and the fact that $y_n \in (a_n, b_n)$.
Now let
$$f(s) = \sum_{n=1}^\infty \gamma_n \Delta_n^{1 - d} \int_{sB_n} \star |\dif u_n|.$$
Then $f' \geq 0$, and by Proposition \ref{doubling dimension}, $f(1) \lesssim \sum_n \gamma_n < \infty$.
So almost surely,
$$f(t + \gamma_n^4) - f(t) \lesssim \gamma_n^4$$
and hence
$$\int_{(t + \gamma_n^4)B_n \setminus tB_n} \star |\dif u_n| \lesssim \gamma_n^3 \Delta_n^{d - 1}.$$
From \cite[Lemma 7.2]{Giusti77} it follows that
$$\int_{tB_n} \star |\dif w_n| \leq \int_{tB_n} \star |\dif w_n| + O(\gamma_n^2).$$
By (\ref{MVT mollifier}) we conclude that (\ref{difference of surface area}) holds almost surely.

\proofpart{4}{Proof of (\ref{mollifier quant1}) and (\ref{mollifier quant2})}
The estimate (\ref{mollifier quant2}) is the conjunction of (\ref{trace of vn}), (\ref{difference of surface area}), and (\ref{a priori estimate 1}); (\ref{mollifier quant1}) is the conjunction of (\ref{mollifier quant2}), (\ref{a priori estimate 1}), the fact that $U_n$ has least perimeter, and (\ref{trace of vn}).

\proofpart{5}{Proof of (\ref{mollifier quant3})}
Integrating by parts,
$$\left|\int_{tB_n} \dif (u_n - v_n) \wedge \omega_n\right| \leq ||\omega_n||_{L^\infty} ||u_n - v_n||_{L^1(\Gamma_n)} + ||\dif \omega_n||_{L^\infty} \int_0^1 ||u_n - v_n||_{L^1(\partial(sB_n))} \dif s.$$
By (\ref{trace of vn}),
$$\limsup_{n \to \infty} ||\omega_n||_{C^1}^{-1} \gamma_n^{-1} \Delta_n^{1 - d} \left|\int_{tB_n} \dif(u_n - v_n) \wedge \omega_n\right| \leq \limsup_{n \to \infty} \int_0^1 \gamma_n^{-1} \Delta_n^{1 - d} ||u_n - v_n||_{L^1(\partial(sB_n))} \dif s.$$
Moreover, (\ref{trace of vn}) holds with $t$ replaced by almost any $s$, so
$$f_n(s) := \gamma_n^{-1} \Delta_n^{1 - d} ||u_n - v_n||_{L^1(\partial(sB_n))}$$
satisfies $(f_n) \in \ell^\infty([0, 1] \to L^\infty)$, and $f_n \to 0$ almost everywhere.
So by Fatou's lemma,
\begin{align*}
0 \leq \limsup_{n \to \infty} ||\omega_n||_{C^1}^{-1} \gamma_n^{-1} \Delta_n^{1 - d} \left|\int_{tB_n} \dif(u_n - v_n) \wedge \omega_n\right| &\leq \int_0^1 \lim_{n \to \infty} f_n(s) \dif s = 0. \qedhere
\end{align*}
\end{proof}



%%%%%%%%%%%%%%%%%%%%%%%%%%%%%%%%%%%%%%%%%%%%%%%

\section{Plateau's equation}\label{Plateau section}
Having shown that we can reduce the study of sets of least perimeter to sets with $C^1$ and approximately least perimeter, our next task is to represent such sets as graphs of approximate solutions to a Plateau-type equation.
It is natural to expect the Plateau equation to a PDE on a manifold $\Omega$ of dimension $d - 1$, so we now construct $\Omega$.

\subsection{Plateau energy}
Let $(x^\mu)$ and $\psi$ be as above, let $\overline M \subseteq M$ be the domain of $(x^\mu)$, and let $\Omega \subseteq \RR^{d - 1}$ be a small open neighborhood of $0$ (on which we have not yet imposed a metric).
We introduce the projection
\begin{align*}
    \Pi: \overline M &\to \Omega \subseteq \RR^{d - 1}\\
    x &\mapsto \overline x := (x^1, \dots, x^{d - 1}).
\end{align*}
If $\omega$ is a $C^r$ function $\Omega \to \RR$, we introduce as a section of $\Pi$ the locally closed $C^r$ embedding
\begin{align*}
    \Psi_\omega: \Omega &\to M \\
    x &\mapsto (\omega(x), x^1, x^2, \dots, x^{d - 1})
\end{align*}
which identifies $\Omega$ with the ``graph'' $N_\omega$ of $\omega$, which is a hypersurface in $M$.

We shall need to pull back component functions by a diffeomorphism $\Psi$. Thus for a tensor field $T$ and multiiindex $I$,
$(\Psi^* T)_I$ denotes the $I$th component of the pullback tensor field $\Psi^* T$, while $\Psi^* (T_I) := T_I \circ \Psi$.
In particular $(\Psi^* T)_I \neq \Psi^* (T_I)$.

\begin{lemma}\label{pullback metric to Omega}
Let $\omega: \Omega \to \RR$ and $\Psi = \Psi_\omega$. Then
$$(\Psi^* g)_{ij} = \Psi^* (g_{00}) \omega_{,i} \omega_{,j} + 2 \Psi^* (g_{i0}) \omega_{,j} + \Psi^* (g_{ij}).$$
\end{lemma}
\begin{proof}
Let us suppress pullbacks on component functions. Let $h = \Psi^* g$, thus 
$$h_{ij} = g_{\mu\nu} (\dif \Psi \cdot \partial_i)^\mu (\dif \Psi \cdot \partial_j)^\nu = g_{\mu\nu} (\dif \Psi)^\mu_i (\dif \Psi)^\nu_j.$$
As $\Psi^\mu = \delta^\mu_0 \omega + x^\mu$,
it follows that 
$$h_{ij} = g_{\mu\nu}(\delta^\mu_0 \omega_{,i} + \delta^\mu_i)(\delta^\nu_0 \omega_{,j} + \delta^\mu_j)$$
which implies the claim.
\end{proof}

The previous lemma involves pullbacks of components $g_{ij}$, so for a fixed $\omega$, we define the \dfn{reduced metric} $\slashed g = \slashed g^{(\omega)}$ to be the metric on $\Omega$ with components
$$\slashed g_{ij} = \Psi_\omega^*(g_{ij}).$$
We note carefully that $\slashed g$ is not the pullback of $g|N$ by $\Psi$ unless $\Psi = \Psi_\omega$ and $\dif \omega = 0$; roughly speaking, $\slashed g^{(\omega)}$ is the pullback of the metric on $N$ if we ``ignore the curvature induced by $\omega$.''
Henceforth when we have a function $\omega$ in mind, $\slashed g^{(\omega)}$ will be our metric of choice on $\Omega$.

\begin{definition}\label{definition of Plateau energy}
Let $\omega: \Omega \to \RR$ be $C^1$ and let $\sigma$ be a continuous $1$-form on $\Omega$.
Then the \dfn{Plateau energy} is the $d-1$-form on $\Omega$,
\begin{equation}\label{Plateau energy}
\Lagrange[\omega, \sigma] := \star_{\slashed g} \sqrt{1 + \Psi_\omega^* (g_{00}) \cdot |\sigma|_{\slashed g^{-1}}^2 + (\sigma, L)}
\end{equation}
where $\slashed g = \slashed g^{(\omega)}$ is the reduced metric and $L$ is the vector field
\begin{equation}\label{Plateau error}
L^i := 2\slashed g^{ij} \Psi^*_\omega (g_{0j}).
\end{equation}
The \dfn{Plateau equation} is the Euler-Lagrange equation for minimizers of $\int \Lagrange[\omega, \dif \omega]$.
\end{definition}

\begin{proposition}\label{construction of Plateau energy}
For $\omega \in C^1(\Omega)$,
$$\Lagrange[\omega, \dif \omega] = \Psi_\omega^* \vol_{N_\omega}.$$
In particular, $N_\omega$ is a nonparametric minimal hypersurface iff $\omega$ is a weak solution of the Plateau equation.
\end{proposition}
\begin{proof}
Let $\xi = (\Psi_\omega^* \vol_{N_\omega})/(\dif \overline x)$ (where $\dif \overline x = \dif \overline x^1 \wedge \cdots \wedge \dif \overline x^{d - 1}$); then by Lemma \ref{pullback metric to Omega},
$$\xi^2 = \det(\Psi^* (g_{00}) \dif \omega \otimes \dif \omega + L^\flat \otimes \dif \omega + \slashed g)$$
where the $\flat$ was taken using $\slashed g$.
We want to show $\xi \dif \overline x = \Lagrange[\omega, \dif \omega]$.
Moreover, by the Weinstein-Aronsazjn theorem,
\begin{align*}
\xi^2 &= \det(\slashed g) \cdot \det(1 + \slashed g^{-1}(\Psi^* (g_{00}) \dif \omega \otimes \dif \omega + L^\flat \otimes \dif \omega))\\
&= \det (\slashed g) \cdot \det(1 + \Psi^*(g_{00}) \slashed g^{-1}(\dif \omega, \dif \omega) + \slashed g^{-1}(L^\flat, \dif \omega))\\
&= \det (\slashed g) \cdot 1 + \Psi^*(g_{00}) |\dif \omega|_{\slashed g^{-1}}^2 + (L, \dif \omega).
\end{align*}
Taking square roots and using $\sqrt{\det(\slashed g)} \dif \overline x = \star_{\slashed g} 1$, we see the desired result.
\end{proof}

\begin{corollary}[elliptic bootstrap]\label{C1 implies smooth}
Let $U$ be a set of least perimeter. If $\normal_U$ extends to a continuous $1$-form on $\partial U$, then $\partial U$ is a smooth minimal hypersurface.
\end{corollary}
\begin{proof}
By Proposition \ref{locality of Caccioppoli}, $\partial U$ is a $C^1$ minimal hypersurface.
Selecting appropriate normal coordinate charts centered on $\partial U$, we realize $\partial U$ locally as the graph of a $C^1$ function $\omega$ which is a weak solution of the Plateau equation by Proposition \ref{construction of Plateau energy}.
Therefore $\omega$ is smooth (see \cite[\S8.3.2]{evans2010partial} and the references therein).
\end{proof}


%%%%%%%%%%%%%%%%%%%%%%%%%%%%%%%%%%%%%%%%%%%%%%%%%%%%%%%%%%%%%
\subsection{A multiplicative gain}
We now seek to obtain a multiplicative gain on the oscillation of the Plateau energy $\dif \omega \mapsto \Lagrange[\omega, \dif \omega]$ when one passes from a dyadic scale $r$ to its child $r/2$.
Such gains appeared in the work of Miranda \cite[Teorema 4.3]{Miranda66}, who proved this result by linearizing the Plateau equation to obtain the Laplace equation on $\Omega$.
We run into two obstructions to this strategy:
\begin{enumerate}
\item $\Omega$ is not flat, so we instead obtain a Laplace-Beltrami equation.
\item In the limit $\dif \omega \to 0$, $\Lagrange[\omega, \dif \omega]$ has a term $L\omega$ which is pointwise linear in $\dif \omega$, and therefore dominates the Dirichlet energy term $\Psi_\omega^* (g_{00}) \cdot |\dif \omega|_{\slashed g^{-1}}^2$.
\end{enumerate}

Fix $\omega$, and write $\overline B_r$ for the $\slashed g^{(\omega)}$-ball of radius $r$ centered on $0$.
Write 
$$\DirL[\sigma] := \star_{\slashed g} |\sigma|^2_{\slashed g^{-1}}/2$$
for the Dirichlet energy of a continuous $1$-form $\sigma$.
One has the curvature estimate 
\begin{equation}\label{bounds on the curvature}
||\Rm_{\slashed g}||_{C^r} \lesssim_g 1.
\end{equation}
Indeed, the derivatives of $\slashed g$ which appear in (\ref{bounds on the curvature}) do not include any derivatives in the $0$ direction, but $\slashed g(\overline x)_{ij} = g(\overline x, \omega(\overline x))_{ij}$ so for $k = 1, \dots, d - 1$, $\Psi = \Psi_\omega$,
$$\slashed g(\overline x)_{ij,k} = \dif \Psi_k^\mu \cdot \Psi^* (g_{ij,\mu}) = \delta_k^\mu \Psi^* (g_{ij,\mu}) = \Psi^* (g_{ij,k}),$$
and hence no derivatives of $\omega$ appear.

For this reason, we may fix a normal coordinate system $(y^i)$ for $\slashed g$ based at $0$, and the constants in the Taylor expansion of $\slashed g$ at $0$ will not depend on $\omega$ (and as usual are locally uniform in the basepoint $P$ of $M$ we chose to define $(x^\mu)$ at, and uniform in the choice of pair $((x^\mu), (y^i)) \in \SpOrth_d \times \SpOrth_{d - 1}$).
In these coordinates,
$$[\partial_i, \Delta_{\slashed g}] = -\slashed g^{jk}_{,i} \partial_j \partial_k - \slashed g^{jk} (\log \sqrt{\det \slashed g})_{,ik} \partial_j$$
and so we have an estimate on the symbol $p$ of the commutator:
\begin{equation}\label{commutator symbol}
    |p(y, \eta)| \lesssim |y| \cdot |\eta|^2 + |\eta|.
\end{equation}
We also have a mean-value formula
\begin{equation}\label{MVP}
    \dashint_{\overline B_r} \star_{\slashed g} u = u(0) + \frac{\Delta_{\slashed g} u(0)}{2(d + 2)}r^2 + O(r^4) ||u||_{L^\infty(\overline B_r)}.
\end{equation}
Indeed, by the classical Riemannian mean-value formula \cite[Theorem 3]{cuhk13},
$$\dashint_{\overline B_r} \star_{\slashed g} u = u(0) + \frac{\Delta_{\slashed g} u(0)}{2(d + 2)}r^2 + O(r^4 ||u||_{C^4(\overline B_{r/2})} \cdot ||\Ric_{\slashed g}||_{L^\infty} \cdot ||\nabla_{\slashed g} \tr_{\slashed g} \Ric_{\slashed g}||_{L^\infty}).$$
We can neglect the curvature factors using (\ref{bounds on the curvature}), and using an interior Schauder estimate \cite[Theorem 7.18]{zworski2012semiclassical}, we can also bound $||u||_{C^4(\overline B_{r/2})} \lesssim ||u||_{L^\infty(\overline B_r)}$.

\begin{definition}
Let us work in the coordinate system $(y^i)$.
The \dfn{averaging operator} $\avg_r$ is defined on tensor fields $T$ on $\Omega \subseteq \RR^{d - 1}$ by declaring that for every multiindex $I$,
$$(\avg_r T)_I(y) := \dashint_{\overline B_r} \star_{\slashed g} (T_I).$$
\end{definition}

We claim that
\begin{equation}\label{MVP averaged}
|\avg_r (\dif u)|_{\slashed g^{-1}}^2 = |\dif u(0)|_{\slashed g^{-1}}^2 + O(||\dif u||_{L^\infty}^2 r^2 + ||\dif u||_{L^\infty} r^3 + r^6).
\end{equation}
Indeed, by (\ref{MVP}),
$$\avg_r(\dif u)_i = \dashint_{B_r} \star_{\slashed g} \partial_i u = \partial_i u(0) + \frac{[\Delta_{\slashed g}, \partial_i] u(0)}{2(d + 2)} r^2 + O(r^4 ||\dif u||_{L^\infty}),$$
and by (\ref{commutator symbol}), the commutator term is $O(r^2 ||\dif u||_{L^\infty} + r^3)$.

We now generalize \cite[Lemma 4.1]{Miranda66} to the Laplace-Beltrami equation, which is the source of the multiplicative gain we seek: more precisely,
\begin{equation}\label{DGL harmonics}
\int_{\overline B_{r/2}} \DirL[\dif u] - \DirL[\avg_{r/2}(\dif u)] \leq \frac{1}{2^{d + 1}} \int_{\overline B_r} \DirL[\dif u] - \DirL[\avg_r(\dif u)].
\end{equation}
To prove this inequality we may assume that $u(0) = 0$. If in addition $\dif u(0) = 0$ then 

\begin{lemma}
Let $\sigma$ be a continuous $1$-form and $\underline \sigma = \avg_r \sigma$. Then, uniformly in $\omega$ with $||\omega||_{C^1} \lesssim 1$,
$$\left|\int_{\overline B_r} \star_{\slashed g} (L, \sigma - \underline \sigma) \right| \lesssim r^d ||\sigma||_{L^\infty(\overline B_r)}.$$
\end{lemma}
\begin{proof}
Let $\underline L = \avg_r L$. Then
$$\int_{\overline B_r} \star_{\slashed g} (L, \sigma - \underline \sigma) = \underline L^i \int_{\overline B_r} \star_{\slashed g} (\sigma_i - \underline \sigma_i) + \int_{\overline B_r} \star_{\slashed g} (L^i - \underline L^i)(\sigma_i - \underline \sigma_i)$$
and the first term on the right-hand side vanishes by definition of $\underline \sigma$.
As for the second term we apply the Poincar\'e inequality:
\begin{align*}
\left|\int_{\overline B_r} \star_{\slashed g} (L^i - \underline L^i)(\sigma_i - \underline \sigma_i)\right| &\leq ||L^i - \underline L^i||_{L^1} \cdot ||\sigma_i - \underline \sigma_i||_{L^\infty} \lesssim r^d \cdot ||\sigma||_{L^\infty} \cdot ||\nabla L||_{L^\infty}.
\end{align*}
Moreover one can check $||\nabla L||_{L^\infty} \lesssim_{g, ||\omega||_{C^1}} 1$.
\end{proof}

For the remainder of the paper we shall need some small constants $(c_\ell)$.
We start by fixing $0 < c_0 \ll 1$, which we will later choose to depend only on $(M, g, P)$ (and which will be locally uniform in $P$).

\begin{proposition}\label{dGL Laplace}
For $0 < r \lesssim_{c_0} 1$, with $\overline B_r \subseteq \Omega$, let $0 < \kappa, \beta < 1$ and suppose that 
\begin{align}
||\dif \omega||_{L^\infty(\overline B_r)} &\leq \kappa, \label{dGL Linfty}\\
\int_{\overline B_r} \Lagrange[\omega, \dif \omega] - \Lagrange[\omega, \avg_r \dif \omega] &\leq \beta, \label{dGL osc}\\
\int_{\overline B_r} \Lagrange[\omega, \dif \omega] &\leq \eta(U_\omega, \overline B_r \times (-\ell, \ell)) + \beta\kappa, \label{dGL mean curv}
\end{align}
where $\ell > ||\omega||_{L^\infty}$ and $\partial U_\omega = N_\omega$. Then 
$$\int_{\overline B_{r/2}} \Lagrange[\omega, \dif \omega] - \Lagrange[\omega, \avg_{r/2} \dif \omega] \leq \frac{e^{c_0}}{2^{d + 1}} \beta + O(\kappa (\beta + r^d)).$$
\end{proposition}
\begin{proof}

\end{proof}



%%%%%%%%%%%%%%%%%%%%%%%%%%%%%%%%%%%%%%%%
\subsection{The conormal $1$-form}
It follows from Corollary \ref{C1 implies smooth} that $\normal_U$ is now of interest when $U$ is a set of least perimeter, so we turn to a study of a such $1$-forms.

\begin{definition}
The \dfn{cross product} $v_1 \times \cdots \times v_{d - 1}$ of vectors $v_1, \dots, v_{d - 1} \in T_QM$ is the unique vector $v_0$
such that:
\begin{enumerate}
\item $g(v_0, v_i) = 0$,
\item $((-1)^{d - 1} v_0, v_1, \dots, v_{d - 1})$ is positively oriented or linearly dependent, and
\item $g(v_0, v_0) = |\det \Gram(v_1, \dots, v_{d - 1})|$.
\end{enumerate}
\end{definition}

\begin{lemma}
Let $\omega$ and $\phi_i = \dif \Psi_\omega \cdot \partial_i$ be as above.
The cross product $\phi_0 := \phi_1 \times \cdots \phi_{d - 1}$ is 
\begin{equation}\label{phi0 components}
\phi_0^\mu = e^{O(|x|^2 + |\omega(x)|^2)} (g^{\mu 0} + g^{\mu i}\omega_{,i}).
\end{equation}
\end{lemma}


\begin{proposition}
The conormal $1$-form to $N = N_\omega$ is
\begin{equation}\label{conormal}
\normal_\mu(x, \omega(x)) = \frac{e^{O(|x|^2 + |\omega(x)|^2)}}{\sqrt{1 + |\dif \omega(x)|^2_{(\slashed g^{(\omega(x))})^{-1}(x)}}} (\delta^0_\mu - \delta^i_\mu \partial_i \omega(x)).
\end{equation}
\end{proposition}
\begin{proof}
It is clear from (\ref{definition of slashed g}) that (\ref{decay of slashed metric}) holds.
Moreover, if we let
$$h_{ij} = g(\phi_i, \phi_j)$$
be the pullback by $\Psi_\omega$ of the Riemannian metric on $N_\omega$, then $h = \Gram(\phi_1, \dots, \phi_{d - 1})$ and so
$$\Lagrange[\omega, \grad_{\slashed g} \omega] = \Psi_\omega^* |\phi_0|_{\slashed g^{(\omega)}} \dif x^1 \wedge \cdots \wedge \dif x^{d - 1}$$
which gives the formula (\ref{Plateau energy}).

Finally, it follows from the definition of cross product that $\normal = \phi_0/|\phi_0|$.
We therefore claim that

which implies (\ref{conormal}). To prove (\ref{phi0 components}) we observe that $\phi_0$ depends smoothly on $g$ and so if we replace $g$ with the metric $g'$ defined by $g'_{00} = g_{00}$, $g'_{ij} = g_{ij}$, $g'_{0i} = 0$, then it suffices to show that the new cross product $\phi_0'$ satisfies
\begin{equation}\label{perturbed phi0 components}
(\phi_0')^\mu = g^{\mu\nu}(\delta_\nu^0 - \delta_\nu^i\omega_{,i}).
\end{equation}
With $g$ replaced by $g'$ we obtain the exact formula
\begin{equation}\label{perturbed norm of phi0}
g(\phi_0', \phi_0') = \sqrt{\det \slashed g} \sqrt{1 + |\dif \omega|_{\slashed g^{(\omega)}}}
\end{equation}
in place of (\ref{norm of phi0}), and so we just need to show that if $\phi_0'$ satisfies (\ref{perturbed phi0 components}) then it is $g'$-orthogonal to the span of $\phi_1, \dots, \phi_{d - 1}$ and has norm (\ref{perturbed norm of phi0}).
Orthogonality follows from the computation
$$g'(\phi_0', \phi_i) = g'_{\mu\nu} (\phi_0')\mu \phi_i^\nu = \delta_\mu^\nu (\delta_0^\mu \omega_{,i} + \delta_i^\mu) (\delta_\nu^0 - \delta_\nu^i \omega_{,i}) = 0.$$
To prove (\ref{perturbed norm of phi0}) we just compute
\begin{align*}
g'(\phi_0', \phi_0')
&= g'_{\mu \nu} (g')^{\mu \lambda}(\delta^0_\lambda - \delta^i_\lambda \omega_{,i}) (g')^{\nu \kappa}(\delta_\kappa^0 - \delta_\kappa^j \omega_{,j})\\
&= (\delta_\mu^0 - \delta_\nu^i \omega_{,i})((g')^{0 \nu} - (g')^{j \nu} \omega_{,j})\\
&= (g')^{00} - (g')^{0j} \omega_{,j} - (g')^{0i} \omega_{,i} + (g')^{ij} \omega_{,i} \omega_{,j}.
\end{align*}
From the definition of $g'$ we can rewrite this last expression as
$$g'(\phi_0', \phi_0') = (g_{00})^{-1} + (g_{\hat 0 \hat 0})^{-1} \dif \omega^{\otimes 2}.$$
But $g_{00} (g_{\hat 0 \hat 0})^{-1} = \slashed g^{-1}$ so this is as desired.
\end{proof}

%%%%%%%%%%%%%%%%%%%%%%%%%%%%%%%%%%%%%%

\section{Regularity of minimal hypersurfaces}\label{de Giorgi section}
\subsection{Excess: first properties}
\subsection{Excess and Plateau's equation}
\subsection{de Giorgi's lemma}
\begin{proposition}[de Giorgi's lemma]\label{dGL final}
hello
\end{proposition}
\subsection{Induction on scale}


\printbibliography

\end{document}
