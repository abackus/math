\documentclass[reqno,10pt]{amsart}
\usepackage[letterpaper, margin=1in]{geometry}
\RequirePackage{amsmath,amssymb,amsthm,graphicx,mathrsfs,url,slashed}
\RequirePackage[usenames,dvipsnames]{xcolor}
\RequirePackage[colorlinks=true,linkcolor=Red,citecolor=Green]{hyperref}
\RequirePackage{amsxtra}
\usepackage{cancel}
\usepackage{tikz-cd}

% \setlength{\textheight}{9.3in} \setlength{\oddsidemargin}{-0.25in}
% \setlength{\evensidemargin}{-0.25in} \setlength{\textwidth}{7in}
% \setlength{\topmargin}{-0.25in} \setlength{\headheight}{0.18in}
% \setlength{\marginparwidth}{1.0in}
% \setlength{\abovedisplayskip}{0.2in}
% \setlength{\belowdisplayskip}{0.2in}
% \setlength{\parskip}{0.05in}
%\renewcommand{\baselinestretch}{1.05}

\title{Functions of least gradient in negative curvature}
\author{Aidan Backus}
\date{July 2022}

\newcommand{\NN}{\mathbf{N}}
\newcommand{\ZZ}{\mathbf{Z}}
\newcommand{\QQ}{\mathbf{Q}}
\newcommand{\RR}{\mathbf{R}}
\newcommand{\CC}{\mathbf{C}}
\newcommand{\DD}{\mathbf{D}}
\newcommand{\PP}{\mathbf P}
\newcommand{\MM}{\mathbf M}
\newcommand{\II}{\mathbf I}
\newcommand{\Hyp}{\mathbf H}
\newcommand{\Sph}{\mathbf S}
\newcommand{\Group}{\mathbf G}
\newcommand{\GL}{\mathbf{GL}}
\newcommand{\Orth}{\mathbf{O}}
\newcommand{\SpOrth}{\mathbf{SO}}
\newcommand{\Ball}{\mathbf{B}}

\DeclareMathOperator*{\Expect}{\mathbf E}

\DeclareMathOperator{\avg}{avg}
\DeclareMathOperator{\card}{card}
\DeclareMathOperator{\cent}{center}
\DeclareMathOperator{\ch}{ch}
\DeclareMathOperator{\codim}{codim}
\DeclareMathOperator{\Cyl}{Cyl}
\DeclareMathOperator{\diag}{diag}
\DeclareMathOperator{\diam}{diam}
\DeclareMathOperator{\dom}{dom}
\DeclareMathOperator{\Exc}{Exc}
\newcommand{\ext}{\mathrm{ext}}
\DeclareMathOperator{\Gal}{Gal}
\DeclareMathOperator{\Hom}{Hom}
\DeclareMathOperator{\Iso}{Iso}
\DeclareMathOperator{\Jac}{Jac}
\DeclareMathOperator{\Lip}{Lip}
\DeclareMathOperator{\Met}{Met}
\DeclareMathOperator{\id}{id}
\DeclareMathOperator{\rad}{rad}
\DeclareMathOperator{\rank}{rank}
\DeclareMathOperator{\Rm}{Rm}
\DeclareMathOperator{\Hess}{Hess}
\DeclareMathOperator{\Hol}{Hol}
\DeclareMathOperator{\Radon}{Radon}
\DeclareMathOperator*{\Res}{Res}
\DeclareMathOperator{\sgn}{sgn}
\DeclareMathOperator{\singsupp}{sing~supp}
\DeclareMathOperator{\Spec}{Spec}
\DeclareMathOperator{\supp}{supp}
\DeclareMathOperator{\Tan}{Tan}
\newcommand{\tr}{\operatorname{tr}}

\newcommand{\Mink}{\mathbf m}
\newcommand{\Ric}{\mathrm{Ric}}
\newcommand{\Riem}{\mathrm{Riem}}
\newcommand*\dif{\mathop{}\!\mathrm{d}}
\newcommand*\Dif{\mathop{}\!\mathrm{D}}
\newcommand{\LapQL}{\Delta^{\mathrm{ql}}}

\newcommand{\dbar}{\overline \partial}

\DeclareMathOperator{\atanh}{atanh}
\DeclareMathOperator{\csch}{csch}
\DeclareMathOperator{\sech}{sech}

\DeclareMathOperator{\Div}{div}
\DeclareMathOperator{\Gram}{Gram}
\DeclareMathOperator{\grad}{grad}
\DeclareMathOperator{\dist}{dist}
\DeclareMathOperator{\spn}{span}
\DeclareMathOperator{\Ell}{Ell}
\DeclareMathOperator{\WF}{WF}

\newcommand{\Lagrange}{\mathscr L}
\newcommand{\DirQL}{\mathscr D^{\mathrm{ql}}}
\newcommand{\DirL}{\mathscr D}

\newcommand{\Hilb}{\mathcal H}
\newcommand{\Homology}{\mathrm H}
\newcommand{\normal}{\mathbf n}
\newcommand{\radial}{\mathbf r}
\newcommand{\evect}{\mathbf e}
\newcommand{\vol}{\mathrm{vol}}

\newcommand{\pic}{\vspace{30mm}}
\newcommand{\dfn}[1]{\emph{#1}\index{#1}}

\renewcommand{\Re}{\operatorname{Re}}
\renewcommand{\Im}{\operatorname{Im}}

\newcommand{\loc}{\mathrm{loc}}
\newcommand{\cpt}{\mathrm{cpt}}

\def\Japan#1{\left \langle #1 \right \rangle}

\newtheorem{theorem}{Theorem}[section]
\newtheorem{badtheorem}[theorem]{``Theorem"}
\newtheorem{prop}[theorem]{Proposition}
\newtheorem{lemma}[theorem]{Lemma}
\newtheorem{sublemma}[theorem]{Sublemma}
\newtheorem{claim}[theorem]{Claim}
\newtheorem{proposition}[theorem]{Proposition}
\newtheorem{corollary}[theorem]{Corollary}
\newtheorem{conjecture}[theorem]{Conjecture}
\newtheorem{axiom}[theorem]{Axiom}
\newtheorem{assumption}[theorem]{Assumption}

\theoremstyle{definition}
\newtheorem{definition}[theorem]{Definition}
\newtheorem{remark}[theorem]{Remark}
\newtheorem{example}[theorem]{Example}
\newtheorem{notation}[theorem]{Notation}

\newtheorem{exercise}[theorem]{Discussion topic}
\newtheorem{homework}[theorem]{Homework}
\newtheorem{problem}[theorem]{Problem}

\makeatletter
\newcommand{\proofpart}[2]{%
  \par
  \addvspace{\medskipamount}%
  \noindent\emph{Part #1: #2.}
}
\makeatother

\newtheorem{ack}{Acknowledgements}

\numberwithin{equation}{section}


% Mean
\def\Xint#1{\mathchoice
{\XXint\displaystyle\textstyle{#1}}%
{\XXint\textstyle\scriptstyle{#1}}%
{\XXint\scriptstyle\scriptscriptstyle{#1}}%
{\XXint\scriptscriptstyle\scriptscriptstyle{#1}}%
\!\int}
\def\XXint#1#2#3{{\setbox0=\hbox{$#1{#2#3}{\int}$ }
\vcenter{\hbox{$#2#3$ }}\kern-.6\wd0}}
\def\ddashint{\Xint=}
\def\dashint{\Xint-}

\usepackage[backend=bibtex,style=numeric]{biblatex}
\renewcommand*{\bibfont}{\normalfont\footnotesize}
\addbibresource{topics.bib}
\renewbibmacro{in:}{}
\DeclareFieldFormat{pages}{#1}


\begin{document}
\begin{abstract}
The least-gradient maximum principle, essentially due to Miranda and de Giorgi in the 1960s, shows that least-gradient functions define a minimal lamination of the support of their derivative.
We show that this result still holds on hyperbolic manifolds.
As a consequence we answer some questions of Daskalopoulos--Uhlenbeck concerning the $L^\infty$-Teichm\"uller theory.
\end{abstract}

\maketitle

%%%%%%%%%%%%%%%%%%%%%%%%%%%%%%%%%%%%%%%%%%%%%%%%%%%%%%%

% \tableofcontents

\section{Introduction}
Throughout this paper, let $M$ be an oriented Riemannian manifold of metric $g$ and dimension $d$.
For a function $u \in BV_\loc(M)$, we write $\star |\dif u|$ for the total variation of the derivative, c.f. (\ref{total variation}).

\begin{definition}\label{main definitions}
A function $u \in BV_\loc(M)$ has \dfn{least gradient} if for every $v \in BV_\cpt(M)$ such that $\supp v \subseteq U \Subset M$,
$$\int_U \star |\dif u| \leq \int_U \star |\dif u + \dif v|.$$
A set $U$ of locally finite perimeter has \dfn{least perimeter} if $1_U$ has least gradient.
\end{definition}

Functions of least gradient arise naturally as solutions to an inverse problem in magnetic resonance imaging (MRI) \cite{Nachman2009, Tamasan2019, Joy09} as well as the formal limit of $p$-harmonic functions as $p \to 1$, but we are interested in them primarily because of their application to the $L^\infty$-Teichm\"uller theory of Thurston, which deals with best-Lipschitz maps between closed hyperbolic manifolds \cite{thurston1998minimal}.
An analytic approach to best-Lipschitz maps was introduced by Daskalopoulos--Uhlenbeck \cite{daskalopoulos2020transverse, daskalopoulosPrep1}, where they are viewed as the limit of $p$-harmonic maps as $p \to \infty$, and therefore are ``dual'' in a suitable sense to functions of least gradient.

%%%%%%%%%%%%%%%%%%%%%%%%%%%%%%%%%%%%%%%%%

\subsection{Main theorems}
As a first step towards understanding functions of least gradient, we prove the following regularity theorem, which generalizes the euclidean case that was proven by Miranda \cite{Miranda64, Miranda66, Miranda67} (c.f. the monograph of Giusti \cite{Giusti77} which exposits Miranda's argument; we shall frequently appeal to Giusti when the generalization of some lemma to the hyperbolic case is straightforward).

\begin{theorem}[regularity of minimal hypersurfaces]\label{main lma}
Let $2 \leq d \leq 7$ and suppose that $M$ has constant sectional curvature $\leq 0$.
Then every set of least perimeter in $M$ is bounded by $C^\infty$ minimal hypersurfaces.
\end{theorem}

We believe that our arguments could be modified to the case $K > 0$, by replacing the unit hyperboloid $\Hyp^d \to \RR^{1, d}$ with the unit sphere $\Sph^d \to \RR^{d + 1}$, but we will not pursure this line of inquiry here as we are mainly interested in the application to $L^\infty$-Teichm\"uller theory.

Central to the proof of Theorem \ref{main lma} is the following monotonicity formula for functions of least gradient. It generalizes the monotonicity formula of Miranda \cite[Theorem 2.8]{Miranda66} and is stronger than monotonicity formulae for minimal surfaces in Riemannian manifolds that we are aware of (see e.g. the notes of Marques \cite[\S7]{MarquesXX}) in that it gives a lower bound on the rate of growth of the monotone quantity. We will never use the conclusion about the Ricci curvature, and only include it as a curiosity item.
Unlike our other theorems, Theorem \ref{monotonicity prestate} makes no hypothesis on the dimension $d$ and does not require constant sectional curvature.

\begin{theorem}[monotonicity formula]\label{monotonicity prestate}
There exists $A \geq 0$ depending continuously on $P \in M$ such that for every function $u$ of least gradient defined near $P \in M$, every normal coordinate system $(x^\mu)$ based at $P$, and $0 < r_1 < r_2 \lesssim 1$,
\begin{align*}
&\left|\int_{r_1}^{r_2} \partial_r \left[r^{1 - d} \int_{B(P, r)} \dif u \wedge \dif x^1 \wedge \cdots \wedge \dif x^{d - 1}\right] \dif r\right|^2 \\
&\qquad \lesssim \left(1 + (d - 1) \log \frac{r_2}{r_1}\right) \left(r_2^{1 - d}\int_{B(P, r_2)} \star |\dif u| \right)\left(\int_{r_1}^{r_2} \partial_r \left[e^{Ar^2} r^{1 - d} \int_{B(P, r)} \star |\dif u|\right] \dif r\right).
\end{align*}
Moreover, if $M$ is flat or has Ricci curvature $< 0$, then we can take $A = 0$.
\end{theorem}

\begin{definition}
A \dfn{minimal lamination} in $M$ is a partition of a closed subset of $M$ into smooth hypersurfaces with zero mean curvature.
If $M$ is a surface we also call $M$ a \dfn{geodesic lamination}.
\end{definition}

Our main theorem is Theorem \ref{main thm}, and is concerned with the construction of minimal laminations.
As such, it has several applications to hyperbolic geometry that we will discuss in Section \ref{hyperbolicApps}.
Theorem \ref{main thm} generalizes the maximum principle stated by G\'orny \cite[Proposition 3.4, Corollary 3.5]{górny2017planar}.
It is called a maximum principle because it implies that if $M$ is the interior of a manifold-with-boundary $\overline M = M \cup \partial M$, then the level sets of any function of least gradient on $M$ must extend all the way to the boundary, just as one expects the level sets of the solution to an elliptic Dirichlet problem to behave.

\begin{theorem}[maximum principle]\label{main thm}
Let $2 \leq d \leq 7$ and suppose that $M$ has constant sectional curvature $\leq 0$.
Let $u \in BV_\loc(M)$, $A_y = \partial \{u > y\}$, $\bigcup_{y \in \RR} A_y$.
Then, if $u$ is a function of least gradient:
\begin{enumerate}
\item $\lambda$ is a minimal lamination in $M$,
\item $A_y$ is either empty or a locally finite union of connected hypersurfaces with boundary, and 
\item if $M$ is the interior of a surface-with-boundary $\overline M$, then $\lambda$ extends to a geodesic lamination of $\overline M$.
\end{enumerate}
Conversely, if $\lambda$ is a minimal lamination in $M$, then $u$ has least gradient.
\end{theorem}

Our last theorem was shown by G\'orny \cite[Theorem 1.2]{górny2017planar} in the euclidean case, and shows that the Radon-Nikod\'ym decomposition of a function of least gradient has no Cantor part, and that the decomposition preserves the least-gradient condition.

\begin{theorem}[Radon-Nikod\'ym decomposition]\label{Gorny regularity}
Let $\overline M$ be an compact convex manifold-with-boundary of dimension $\leq 7$ and constant sectional curvature $\leq 0$.
Then any function of least gradient $u: \overline M \to \RR$ can be written as the sum of a continuous function of least gradient, and a function of least gradient with only jump discontinuities.
\end{theorem}


%%%%%%%%%%%%%%%%%%%%%%%%%%%%%%%%%%%%%%%%%%%%%%%

\subsection{Applications to hyperbolic geometry}\label{hyperbolicApps}
The Thurston asymmetric metric, first defined in \cite{thurston1998minimal}, is constructed from best-Lipschitz maps between two closed hyperbolic manifolds.

\begin{definition}
Let $M, N$ be marked closed Riemannian manifolds, thus we have a fixed homotopy class $[F]$ of maps $M \to N$.
Write $L_f$ for the Lipschitz constant of a map $f: M \to N$.
A \dfn{best-Lipschitz map} $F: M \to N$ is a minimizer of $L_F$ in the homotopy class $[F]$.
The \dfn{Thurston asymmetric metric} on the Teichm\"uller space $\mathcal M_{g, 1}$ of marked closed hyperbolic surfaces of genus $g \geq 2$ is defined by setting $\dist(M, N) := L_F$, where $F$ is any best-Lipschitz map in $[F]$.
\end{definition} 

As a first step towards an analytic understanding of the Thurston asymmetric metric, Daskalopoulos--Uhlenbeck \cite{daskalopoulos2020transverse} considered best-Lipschitz maps $M \to \Sph^1$, for $M$ a closed hyperbolic surfaces.
They identified a particularly important class of such maps, the $\infty$-harmonic maps, defined as follows, which are particularly significant because they induce geodesic laminations of $M$.

\begin{definition}
If a best-Lipschitz function $u$ is the weak limit in $L^r$ for $r > d$ of $p$-harmonic functions as $p \to \infty$, we call $u$ \dfn{$\infty$-harmonic}.
For an $\infty$-harmonic function $u$ we define the \dfn{maximum-stretch locus}
$$\lambda_u := \{x \in M: L(x) = \sup L\}$$
where $L(x)$ denotes the local Lipschitz constant of $u$ at $x$.
\end{definition}

\begin{theorem}\label{infinity harmonic laminations}
Suppose that $M$ is a closed hyperbolic surface and $u$ is an $\infty$-harmonic function. Then the maximum-stretch locus $\lambda_u$ is a geodesic lamination in $M$.
\end{theorem}

In \cite[\S5]{daskalopoulos2020transverse}, Daskalopoulos--Uhlenbeck prove Theorem \ref{infinity harmonic laminations} by considering the viscosity solution theory of $\infty$-Laplace equation
\begin{equation}\label{infinity laplace}
    \Hess u(\grad u, \grad u) = 0.
\end{equation}
However, the theory of viscosity solutions of (\ref{infinity laplace}) is still nascent, and Daskalopoulos--Uhlenbeck ask for a proof of Theorem \ref{infinity harmonic laminations} that bypasses (\ref{infinity laplace}) altogether, c.f. \cite[Problem 9.5]{daskalopoulos2020transverse}.
We give a partial resolution of this problem by proving \cite[Theorem-Conjecture 9.6]{daskalopoulos2020transverse}.
Before stating it we note that by a \dfn{section of least gradient} on $M$ we mean a section that lifts to a function of least gradient on $\Hyp^d$.

\begin{corollary}\label{maximum stretch contains lamination}
Let $u$ be an $\infty$-harmonic function on a closed hyperbolic surface $M$.
Then the maximum-stretch locus $\lambda_u$ contains a geodesic lamination.
\end{corollary}
\begin{proof}
By \cite[\S6]{daskalopoulos2020transverse}\footnote{The proof that such a section exists does not use Theorem \ref{infinity harmonic laminations}.}, there exists an affine bundle $E \to M$ and a section $v$ of least gradient of $E$ such that $\supp \dif v \subseteq \lambda_u$.
By Theorem \ref{main thm}, $\supp \dif v$ is a geodesic lamination.
\end{proof}

It is not clear that $\supp \dif v = \lambda_u$ in the above corollary, but see Conjecture \ref{two laminations agree}.

Daskalopoulos--Uhlenbeck also ask for \cite[Problem 9.7]{daskalopoulos2020transverse} a partial converse to the fact that $\dif v$ endows $\lambda_u$ with the structure of an oriented measured lamination, which we now prove.
For the definitions, see \cite[\S8]{daskalopoulos2020transverse} or the original paper of Ruelle--Sullivan \cite{Ruelle75}.

\begin{corollary}\label{ruelle sullivan antiderivative}
Let $\lambda$ be an oriented, transversely measured geodesic lamination on a closed hyperbolic surface $M$, and let $\dif v$ be the Ruelle-Sullivan $1$-current induced by $\lambda$.
Then $\dif v$ is the derivative of a section of least gradient $v: M \to E$, for some affine bundle $E \to M$.
\end{corollary}
\begin{proof}
As observed in \cite[\S9]{daskalopoulos2020transverse}, if we lift $\dif v$ to a $1$-current $\dif \tilde v$ on $\Hyp^2$, then $\dif \tilde v$ is exact and any antiderivative $\tilde v$ of $\dif \tilde v$ has superlevel sets $\{\tilde v \geq y\}$ which are bounded by geodesics.
Moreover we can choose $\tilde v$ to be $\pi_1(M)$-equivariant.
We can realize the $\pi_1(M)$-equivariant function $\tilde v$ on $\Hyp^2$ as a section $v$ of an affine bundle, which has least gradient by Theorem \ref{main thm}.
\end{proof}

The analogous theory for threefolds is also quite interesting, as closed minimal surfaces in hyperbolic threefolds have been studied intensely since the work of Uhlenbeck \cite{Uhlenbeck1983ClosedMS}.
Daskalopoulos--Uhlenbeck have suggested \cite[Problem 9.13]{daskalopoulos2020transverse} that for a line bundle $E \to M$ over a closed hyperbolic threefold $M$, it would be particularly fruitful to study sections of $E$ of least gradient, as one could presumably foliate $M$ by minimal surfaces.
But we leave this line of inquiry for a later paper.

Theorem \ref{main thm} also furnishes a large class of minimal laminations which are inexpensive to numerically compute.

\begin{proposition}\label{cohomology makes laminations}
Let $M$ be a closed hyperbolic manifold of dimension $d \leq 7$, and let $\alpha \in H^1(M, \RR)$.
Then there is a natural minimal lamination $\lambda$ in $M$ induced by $\alpha$, which can be obtained by solving a least-gradient Dirichlet problem on a fundamental polytope of $M$.
\end{proposition}
\begin{proof}
Let $\Gamma = \pi_1(M)$. The Hurcewiz theorem implies that $\alpha$ pulls back to a map $\Gamma \to \RR$.
Then $\alpha$ is a representation of $\Gamma$ and thus induces a space $E$ of equivariant functions on the boundary $\Omega$ of each fundamental polytope $\Omega$ of $\Gamma$. To be more precise, for every $f \in E$ and $\gamma \in \Gamma$,
\begin{equation}\label{boundary data for Loisel}
f(\gamma x) = f(x) + \alpha(\gamma),
\end{equation}
and $f$ is constant on each face of $\partial \Omega$.
The relation (\ref{boundary data for Loisel}) is an underdetermined boundary condition for functions in $E$, and so we consider the subspace $E'$ of functions which in addition are zero on a maximal set of faces such that we do not determine $f|\partial \Omega = 0$, thus any function $E' \subseteq E$ has a completely determined trace $t$ on $\partial \Omega$.
It follows from \cite[Theorem 4.4]{daskalopoulos2020transverse} that there exists a function $u$ of least gradient on $\Omega$ whose trace is $t$.
The minimal lamination induced by a function $u \in E$ of least gradient from Theorem \ref{main thm} does not depend on the class of $u$ in $E/E'$, since that is a space of constants; so we obtain a uniquely defined minimal lamination of $M$.
\end{proof}

\begin{corollary}\label{application to Loisel}
With $M, \alpha, \lambda$ as in Proposition \ref{cohomology makes laminations}, and every quasiuniform triangulation $T$ of $M$, there exists a numerical algorithm for computing $\lambda$ with finite elements in $T$ with time complexity $O(|T|^{1/2} \log |T|)$ where $|\cdot|$ is cardinality.
\end{corollary}
\begin{proof}
By \cite[Theorem 1]{Loisel20}, one can solve the least-gradient Dirichlet problem with finite elements in $T$ with time complexity $O(|T|^{1/2} \log |T|)$.
\end{proof}

TODO: Need to check that the laminations induced by the finite element space converge to a minimal lamination

TODO: Do some numerical experiments, show what minimal laminations in a fundamental polytope in $\Hyp^3$ look like


%%%%%%%%%%%%%%%%%%%%%%%%%%%%%%%%%%%%%%%%%%%%%%%

\subsection{Some open problems}
We believe that our results could be extended to a wider class of Riemannian manifolds:

\begin{conjecture}\label{main conj}
The main theorems of this paper hold for any Riemannian manifold of dimension $2 \leq d \leq 7$.
\end{conjecture}

However, it seems highly unlikely that Theorem \ref{main lma} can be extended to any manifold $M$ of dimension $8$, due to the existence of Simons cones in $\RR^8$ \cite[Theorem A]{BOMBIERI1969}.
We would therefore appreciate a general class of counterexampels:

\begin{problem}
For each Riemannian manifold $M$ of dimension $8$, construct a set $U$ of least perimeter in $M$ such that $\partial U$ has a singularity of codimension $8$.
\end{problem}

Owing to the G\'orny decomposition, Theorem \ref{Gorny regularity}, we suspect that its consequence \cite[Theorem 1.1]{górny2017planar} holds, giving well-posedness for the Dirichlet problem in strictly convex planar domains.
However, this is not obvious as we have not extended the Sternberg--Williams--Ziemer theorem \cite{ZiemerWilliamsSternberg1992} to the Riemannian case.

\begin{conjecture}
Let $\overline M = M \cup \partial M$ be a compact, strictly convex surface-with-boundary and constant scalar curvature $\leq 0$.
Then there exists a solution of the least-gradient Dirichlet problem on $M$ with data in $BV(\partial M)$.
\end{conjecture}

We would also like to know that the $\infty$-harmonic/least-gradient duality gives a complete proof of Theorem \ref{infinity harmonic laminations}, which follows from the below conjecture.

\begin{conjecture}\label{two laminations agree}
Let $u$ be an $\infty$-harmonic function on a hyperbolic surface, with dual least-gradient section $v$.
Then $\supp \dif v = \lambda_u$.
\end{conjecture}

Finally we would like to know that one can foliate a hyperbolic threefold fibered over $\Sph^1$ into minimal surfaces, which follows from the following construction:

\begin{problem}
Let $M$ be a hyperbolic threefold and fix a homotopy class of maps $[u]: M \to \Sph^1$.
Construct a map of least gradient $u$ in $[u]$ such that $\dif u$ has full support.
\end{problem}

%%%%%%%%%%%%%%%%%%%%%%%%%%%%%%%%%%%%%
\subsection{Overview of the paper}
In \S\ref{prelims} we shall discuss preliminary facts about geometric measure theory on Riemannian manifolds and the unit hyperboloid embedding $\Hyp^d \to \RR^{1, d}$.
This embedding will be quite convenient for us, as the Levi-Civita connection on $\RR^{1, d}$ is flat, and the curvature of the Levi-Civita connection on $\Hyp^d$ will frequently be a technical obstacle.
Then we recall elementary facts about functions of least gradient, and show that the tangent cone of a set of least perimeter exists and is actually a hyperplane, Proposition \ref{blowup theorem}; this can easily be shown by reduction to the Bernstein-Fleming theorem which classifies minimal cones in $\RR^d$ \cite[Theorem 17.3]{Giusti77}.

In \S\ref{MollifierSection} we prove Theorem \ref{monotonicity prestate} and several consequences that we will use later, including Proposition \ref{mollifier quant}, which allows for the mollification of sets of least perimeter.

In \S\ref{Plateau section}, we use the mollification result to restrict to $C^1$ minimal hypersurfaces, and represent them as graphs of solutions $\omega: \Omega \to \RR$ of a Plateau equation where $\Omega$ is an open subset of $\Hyp^{d - 1}$ (if $d \geq 3$) or $\RR$ (if $d = 2$). More precisely, we have the quasilinear elliptic PDE
\begin{equation}\label{Plateau intro}
\frac{F}{\sqrt{1 + F|\nabla \omega|^2}} \Delta \omega + g\left(\nabla \frac{F}{\sqrt{1 + F|\nabla \omega|^2}}, \nabla \omega\right) = 0
\end{equation}
for a suitable weight $F$. The main point is that if $M$ is hyperbolic then we may assume that the metric on $M$ is written in ``Killing gauge", a gauge condition which implies that $\Omega$ does not depend on the given $C^1$ set.
A study of the elliptic regularity theory of (\ref{Plateau intro}) yields the fundamental inequality of this paper: Proposition \ref{dGL Laplace} generalizes de Giorgi--Miranda's elliptic regularity estimate on the Plateau equation \cite[Teorema 4.3]{Miranda66} and shows that one obtains a multiplicative gain in the oscillation of $\omega$ when one passes from a scale $r$ to $r/2$, as long as $g$ is in Killing gauge.

In \S\ref{de Giorgi section} our goal is to show Theorem \ref{main lma}.
This follows from a standard induction-on-scale argument and the de Giorgi $\varepsilon$-regularity lemma, Proposition \ref{dGL final}, which generalizes \cite[Teorema 5.7]{Miranda66} and itself follows by mollification and Proposition \ref{dGL Laplace}.

In \S\ref{Gorny regularity}, we prove Theorems \ref{main thm} and \ref{Gorny regularity}.


%%%%%%%%%%%%%%%%%%%%%%%%%%%%%%%%%%%%%%%%%%%%%%%%

\subsection{Acknowledgements}
I would like to thank Georgios Daskalopoulos for suggesting this project and for many helpful discussions.
I would also like to thank Trent Lucas for comments on Proposition \ref{cohomology makes laminations} and Joshua Lin for providing me with Proposition \ref{Josh helped}.



%%%%%%%%%%%%%%%%%%%%%%%%%%%%%%%%%%%%%%%%%%%%%%%%%%
\section{Preliminaries}\label{prelims}
\subsection{Notation and conventions}
For tensor fields we shall use the musical isomorphisms $\sharp, \flat$ and the Einstein convention.
When using the Einstein convention, Greek indices range over $0, 1, \dots$ while Latin indices range over $1, \dots$.
All such indices range over spacelike components; when we use a Lorentz metric we reserve $t$ for a timelike component.
If we have a fixed coordinate system $(x^\mu)$, we sometimes suppress it and write $\partial_\mu := \partial_{x^\mu}$.
The operator $\star$ is the Hodge star, thus $\star 1$ is the Riemannian volume form.
On a submanifold $\Sigma$ of codimension $\geq 1$, $\vol_\Sigma$ denotes the induced volume form and $\star_\Sigma$ denotes the induced Hodge star.

The statements $A \lesssim B$ and $A = O(B)$ are equivalent and mean that there exists $C > 0$ independent of $A, B \geq 0$ such that $A \leq BC$.
Similarly the statements $A \ll B$ and $A = o(B)$ mean that there exists a function $f: \RR_+ \to \RR_+$ with $\lim_{x \to 0} f(x) = 0$, such that $A \leq Bf(B)$.
We also abuse notation slightly and write $A \ll 1$ to mean that $\lim A = 0$ where the limit is taken over some implied parameter.
We write $W^{s, p}(M, N)$ for the Sobolev space of maps $M \to N$ with regularity and integrability exponents $s, p$.

We write $\int_U \omega \wedge \psi$ for the pairing of a de Rham $\ell$-current $\omega$ with a compactly supported $\ell$-form $\psi$ in an open set $U$.
In particular, we have a Poincar\'e duality map which identifies a $d - \ell$-form $\omega$ with the $\ell$-current $\psi \mapsto \int_U \omega \wedge \psi$.
See \cite{simon1983GMT} for the definition of a de Rham current.

We identify the distributional derivative of a function $u$ with the $d-1$-current
$$\int_U \dif u \wedge \psi = -\int_U u \dif \psi.$$
A function $u$ is in $BV(U)$ iff its derivative $\dif u$ has finite total variation
\begin{equation}\label{total variation}
\int_U \star |\dif u| := \sup_{\substack{||\psi||_{L^\infty} \leq 1\\\supp \psi \Subset V}} \int_U \dif u \wedge \psi.
\end{equation}
Whether a current has locally finite total variation is independent of the Riemannian metric and so $BV_\loc(M)$ is also independently defined.

\begin{definition}
A \dfn{hyperbolic manifold} is a Riemannian manifold (possibly not geodesically complete) with constant sectional curvature $-1$.
The \dfn{Japanese norm} of an element $\xi$ of a normed space is $\Japan\xi := \sqrt{1 + |\xi|^2}$.
\end{definition}

%%%%%%%%%%%%%%%%%%%%%%%%%%%%%%%%%%%%%%%%%%%%%%%%%%
\subsection{Riemannian measure theory}
We recall the Riemannian analogue of the main results of \cite[Chapter 1]{Giusti77}.
By \cite[Theorem 4.14]{simon1983GMT}, for every $u \in BV_\loc(M)$, there exists a $\star |\dif u|$-measurable section $f$ of the cosphere bundle $S'M$ such that for every compactly supported $d-1$-form $\psi$,
\begin{equation}\label{RNy formula}
\int_M \dif u \wedge \psi = \int_M f|\dif u| \wedge \psi.
\end{equation}

For a vector field $X$, we write $\star (Xu) := \dif u \wedge \star (X^\flat)$.
The section $f$ of (\ref{RNy formula}) is given pointwise $\star |\dif u|$-almost everywhere, in any local coordinates $(y^\mu)$, by
\begin{equation}\label{Lebesgue point definition}
    f(x) = \left[\lim_{r \to 0} \frac{\int_{B(x, r)} \star \partial_\mu u}{\int_{B(x, r)} \star |\dif u|}\right] ~\dif y^\mu,
\end{equation}
according to the Besicovitch differentiation theorem; here we view $(\dif y^\mu)$ as a basis of $T'_xM$.
Whether the limit $f(x)$ in (\ref{Lebesgue point definition}) exists, or indeed its value as a point of $S'_xM$, denotes not depend on the Riemannian metric or the choice of coordinates.

\begin{definition}
The points $x$ for which the limit (\ref{Lebesgue point definition}) exists and satisfies $|f(x)| = 1$ are called the \dfn{Lebesgue points} of $\dif u$.
\end{definition}

\begin{definition}
Let $U$ be a set of locally finite perimeter, and let $u = 1_U$. Then:
\begin{enumerate}
\item The \dfn{measure-theoretic boundary} $\partial U$ is the set of points whose Lebesgue density with respect to $M$ is $\in (0, 1)$.
\item The set of Lebesgue points of $\dif u$ is the \dfn{reduced boundary} $\partial^* U$.
\item The $\star |\dif u|$-measurable $1$-form $f$ defined by (\ref{Lebesgue point definition}) is the \dfn{conormal $1$-form} $\normal_U$ to $\partial U$.
\end{enumerate}
\end{definition}

Our definition of reduced boundary and conormal $1$-form follows \cite[Definition 3.3]{Giusti77} and is due to \cite{deGiorgi55}.
See \cite{Battista_2021} for an equivalent definition of reduced boundary on Riemannian manifolds, and see \cite[Chapter 6]{Pugh02} for the definition of Lebesgue density.

\begin{proposition}\label{locality of Caccioppoli}
    Let $U$ be a set of locally finite perimeter with conormal $1$-form $\normal$.
    Then:
    \begin{enumerate}
    \item $\partial^* U$ is either empty or $d-1$-dimensional in the Hausdorff sense, and is $d-1$-rectifiable.
    \item $\partial^* U$ is a dense subset of $\partial U$.
    \item If $\normal$ extends to a continuous $1$-form on $\partial U$, then $\partial^* U = \partial U$ is a $C^1$ hypersurface.
    \item If $\partial^* U = \partial U$ is a smooth hypersurface, then $\normal$ is the conormal $1$-form on $\partial U$ as defined in differential topology, and $\star |\dif 1_U|$ is the induced volume form on $\partial U$.
\end{enumerate}
\end{proposition}
\begin{proof}
Most of the assertions of this proposition are diffeomorphism-invariant, so we may assume that $M = \RR^d$ and appeal to \cite[Chapters 2-4]{Giusti77}.
The proof that $\star |\dif 1_U|$ is the induced volume form is identical to \cite[Example 1.4]{Giusti77}.
\end{proof}

\begin{definition}
Let $M$ be a Riemannian manifold, let $U$ be a set of locally finite perimeter, and let $E$ be a Borel set.
The \dfn{perimeter} of $U$ in $E$ is
$$|E \cap \partial^* U| := \int_E \star |\dif 1_U|.$$
\end{definition}

\begin{proposition}[coarea formula]\label{Coarea2}
Let $M$ be a Riemannian manifold and $u \in BV_\loc(M)$. Then for every open set $E$,
\begin{equation}\label{coarea formula}
\int_E \star |\dif u| = \int_{-\infty}^\infty |E \cap \partial \{u > y\}| \dif y.
\end{equation}
\end{proposition}
\begin{proof}
We follow \cite[Theorem 1.23]{Giusti77}, which first proves (\ref{coarea formula}) for $u \in C^\infty(\RR^d)$ using piecewise linear functions.
Such functions are not available for our purposes; instead we note that if $u \in C^\infty(\RR^d)$ and $u$ has no critical points then (\ref{coarea formula}) follows from Fubini's theorem, the fact that $|E \cap \partial \{u > y\}|$ is the surface area of $E \cap \{u = y\}$ (by Proposition \ref{locality of Caccioppoli}), and the change-of-variables formula.
However the left-hand side of (\ref{coarea formula}) is unaffected by critical points of $u$, and the right-hand side of (\ref{coarea formula}) is unaffected by critical values of $u$ by Sard's theorem.
So (\ref{coarea formula}) holds for $u \in C^\infty(\RR^d)$.

The rest of the proof is identical to \cite[Theorem 1.23]{Giusti77}, so we omit the details.
Taking a sequence in $C^\infty(M)$ that converges to $u$ in $L^1_\loc(M)$\footnote{Recall that $C^\infty(M)$ is not dense in $BV_\loc(M)$.}, and applying Fatou's lemma and the semicontinuity of total variation, we conclude the $\geq$ direction of (\ref{coarea formula}).
Moreover, Stokes' theorem gives that for every $d-1$-form $\psi$ such that $||\psi||_{L^\infty} \leq 1$ and $\supp \psi \Subset E$,
$$\int_E u \wedge \dif \psi = \int_{-\infty}^\infty \int_E |\psi| \star |\dif 1_{\partial \{u > y\}}| \dif y \leq \int_{-\infty}^\infty |E \cap \partial \{u > y\}| \dif y.$$
Taking the supremum over $\psi$ we obtain the direction $\leq$ in (\ref{coarea formula}).
\end{proof}

%%%%%%%%%%%%%%%%%%%%%%%%%%%%%%%%%%%%%%%%%%%%%%%%%%
\subsection{A flat connection}
Crucial to the proof of the regularity Theorem \ref{main lma} is a mean-value property for certain elliptic PDE (\ref{Plateau eqn}).
To formulate it, we recall that if $E$ is a finite-dimensional vector space and we have an $E$-valued function $\xi$, it is well-defined to take the mean of $\xi$; to be precise, we use the Bochner-Lebesgue integral of $\xi$ to define its mean, as in \cite{Rieffel70}.
If $E$ is instead a vector bundle with flat connection, one can identify all the fibers of $E$ with a particular fiber $E'$ and view a section of $E$ as an $E'$-valued function; thus we define the mean for sections of $E$.

However, to prove Theorem \ref{main lma} we will need to take means of sections of a cotangent bundle $T'M$ equipped with Levi-Civita connection $\nabla$.
This is not well-defined, as $\nabla$ is not a flat connection, but in the case $M = \Hyp^d$ we can embed $T'M$ in a larger bundle which has a flat connection, which is well-defined up to a natural action of the properly orthochronous Lorentz group $\SpOrth_{1, d}^+$.
This construction already appears in \cite{daskalopoulosPrep1}\footnote{At the time of writing, \cite{daskalopoulosPrep1} is still in preparation. We are grateful to Georgios Daskalopoulos for providing us with an early draft.} where it is used to study Schatten-von Neumann harmonic maps.

\begin{definition}
A \dfn{quadratic bundle} is a vector bundle $E$ equipped with a nondegenerate quadratic form, i.e. a section $q$ of $E' \otimes E'$ such that the contraction $v \mapsto q(v, \cdot)$ is injective.
\end{definition}

We adopt the convention that Minkowski spacetime $\RR^{1, d}$ is $\RR \times \RR^d$ equipped with the Lorentz metric $\Mink(\partial_t, \partial_t) = -1$ where $t$ is the coordinate on $\RR$, $\Mink(\partial_\mu, \partial_\mu) = 1$ for $\mu = 0, \dots, d - 1$ the coordinate indices on $\RR^d$, and all other components $0$.

For the remainder of this paper we fix an identification of $\Hyp^d$ with the future branch of the unit timelike hyperboloid:
\begin{equation}\label{hyperboloid choice}
\Hyp^m \cong \{x \in \RR^{1,d}: \Mink(x, x) = -1, x^0 > 1\}.
\end{equation}
Since the Christoffel symbols of $\RR^{1,d}$ vanish, the Levi-Civita connection on $T\RR^{1, d}$ is $\dif$, and once we have fixed (\ref{hyperboloid choice}), the restriction of the bundle $T\RR^{1, d}$ induces an isomorphism
$$T\RR^{1, d}|\Hyp^d = \Hyp^d \times \RR^{1, d},$$
a flat connection $\Dif$ on $\Hyp^d \times \RR^{1, d}$ defined by restricting $\dif$, and an exact sequence of quadratic bundles over $\Hyp^d$:
$$\begin{tikzcd}
0 \to (T\Hyp^d)^\perp \to \Hyp^d \times \RR^{1, d} \arrow[r,"\Pi"] & T\Hyp^d \to 0
\end{tikzcd}$$
so that the Christoffel symbols of $\Dif$ vanish with respect to any frame for $\Hyp^d \times \RR^{1, d}$ induced by a basis of $\RR^{1, d}$.
%
% \begin{definition}
% We call $\Dif$ the \dfn{adapted connection} on $\Hyp^m \times \RR^{1, m}$.
% \end{definition}
%
% To study the form the metric tensor on $\Hyp^m$ takes in the above coordinates it will be recall that $\Hyp^m$ is spacelike and work in slightly higher generality.
%
% \begin{definition}
% Let $\Sigma$ be a spacelike hypersurface in $\RR^{1, m}$. The \dfn{defining function} $\tau \in C^\infty(\RR^m)$ of $\Sigma$ is the function such that
% $$\Sigma = \{(t, x) \in \RR^{1, m}: x \in \RR^m, ~t = \tau(x)\}.$$
% \end{definition}
%
% It is straightforward to show that $\tau$ is well-defined for any spacelike hypersurface and induces a diffeomorphism
% $$\RR^m \ni x \mapsto (\tau(x), x) \in \Sigma.$$
% With respect to the coordinates induced by this diffeomorphism the metric on $\Sigma$ is
% \begin{equation}\label{spacelike metric}
% g_{\mu\nu} = \delta_{\mu\nu} - \partial_\mu \tau \partial_\nu \tau.
% \end{equation}
% In the case of hyperbolic space, one has
% \begin{equation}\label{defining function}
% \tau(x) = \sqrt{|x|^2 + 1}.
% \end{equation}

We henceforth adopt the convention that any $(\ell_1, \ell_2)$-tensor field on $\Hyp^d$ is to be identified with a map
$$\Hyp^d \to (\RR^{1, d})^{\otimes(\ell_1 + \ell_2)}$$
equipped with the connection $\Dif$.
In particular, it makes sense to define the Bochner-Lebesgue integral
\begin{equation}\label{definition of vectorial integral}
\int_{\Hyp^d} T \otimes \vol \in (\RR^{1, d})^{\otimes (\ell_1 + \ell_2)}
\end{equation}
for every volume form $\vol$ and every tensor field $T \in L^1(\vol)$. More precisely, for every linear functional
$$\varphi: (\RR^{1, d})^{\otimes(\ell_1, \ell_2)} \to \RR,$$
we define
$$\left(\varphi, \int_{\Hyp^d} T \otimes \vol\right) := \int_{\Hyp^d} (\varphi, T(x)) \vol(x)$$
where $(\varphi, T(x))$ makes sense since $T(x) \in (\RR^{1, d})^{\otimes (\ell_1 + \ell_2)}$.
Since $\Dif$ is flat, and  $\SpOrth^+_{1, d}$ acts transitively on $\Hyp^m$, the integral (\ref{definition of vectorial integral}) is well-defined in the sense that any two different identifications (\ref{hyperboloid choice}) would merely transform the value of the integral by the action of $\SpOrth^+_{1, d}$.

\begin{definition}
The \dfn{average} in a ball $B \subset \Hyp^d$ is defined for tensor fields $T$ on $\Hyp^d$ by
$$\avg_B T := \frac{1}{|B|} \int_B T \otimes (\star 1).$$
If a basepoint $P$ is fixed we write $\avg_r := \avg_{B(P, r)}$.
\end{definition}

The average is not a tensor field on $\Hyp^d$, but it can be transformed into one by applying the projection $\Pi$.
We write $\Pi_Q$ for the fiber of $\Pi$ over $Q \in \Hyp^d$, thus
\begin{equation}\label{projection formula}
\Pi_Q V = V + \eta(V, Q)Q.
\end{equation}

Since $\Hyp^d$ is the graph 
$$\{(t, x) \in \RR^{1, d} t = \sqrt{1 + |x|^2},$$
$\Hyp^d$ is transverse to any time-slice $\{t = c\}$ with angle 
\begin{equation}\label{angle of slice}
\tan \varphi = \frac{\dif}{\dif r} \sqrt{1 + |x|^2} = \frac{|x|}{\sqrt{1 + |x|^2}}.
\end{equation}
In particular, if $(r, \theta)$ are polar coordinates on $\Hyp^d$ based at $O$, and $v = v^\mu \partial_\mu + v^t \partial_t$ is tangent to $\Hyp^d$ at $(r, \theta)$, then 
\begin{equation}\label{bound the timelike part}
|v_t| = |v_x| \tan \varphi \leq r|v_x|.
\end{equation}

% We now estimate $\Pi_Q: T_P\Hyp^d \to T_Q\Hyp^d$.
% In the $d = 2$ case, these estimates already appear in \cite{daskalopoulosPrep1}.

% \begin{lemma}
% Let $P, Q \in \Hyp^d$ and $t = \dist(P, Q)$. Then $P - Q$ is spacelike and
% $$\eta(P - Q, P - Q) = 2(\cosh t - 1) \in [t^2, t^2 \cosh t].$$
% \end{lemma}
% \begin{proof}
% By acting $\SpOrth^+_{1, d}$ on $\RR^{1, d}$ we may assume that $P, Q$ are any two points connected by a geodesic of length $t$.
% The geodesic $\gamma: [0, t] \to \Hyp^m$ such that
% $$\gamma(s) = (\cosh s, \sinh s, 0, \dots, 0)$$
% is spacelike and therefore satisfies
% $$|\gamma| = \int_0^t \sqrt{\eta(\gamma'(s), \gamma'(s))} \dif s = \int_0^t \sqrt{\sinh^2 s - \cosh^2 s} \dif s = t.$$
% Therefore we may assume that
% \begin{align*}
% P &= (1, 0, \dots, 0) \\
% Q &= (\cosh t, \sinh t, 0, \dots, 0)
% \end{align*}
% and hence
% \begin{align*}
% \eta(P - Q, P - Q) &= \sinh^2 t - (\cosh t - 1)^2 = 2(\cosh t - 1). \qedhere
% \end{align*}
% \end{proof}

% \begin{lemma}
%     Let $P, Q \in \Hyp^d$, $V \in T_P\Hyp^d$, and $W = \Pi_Q V$. Then
% \begin{align*}
%     |W| &\leq (1 + \eta(P - Q, P - Q)/2)|V|, \\
%     |W - V| &\leq |V|\sqrt{\eta(P - Q, P - Q)} \sqrt{1 + \frac{1}{4} \eta(P - Q, P - Q)}.
% \end{align*}
% \end{lemma}
% \begin{proof}
% By (\ref{projection formula}),
% $$|W|^2 = \eta(W, W) = \eta(V, V) + \eta(V, Q)^2$$
% and $|W - V| = \eta(V, Q)$.
% We therefore estimate
% \begin{align*}
% \eta(V, Q)^2 &= \eta(V, \Pi_Q(P - Q))^2 = \eta(V, P - Q + \eta(P - Q, Q)Q).
% \end{align*}
% Since every tangent vector to $Q$ is spacelike,
% $$\eta(V, P - Q + \eta(P - Q, Q)Q) \leq \eta(W, W)(\eta(P - Q, P - Q) + \eta(Q - P, Q)^2)$$
% and as
% $$\eta(Q - P, Q) = \frac{1}{2}\eta(P - Q, P - Q),$$
% we conclude
% \begin{align*}
% \eta(V, Q)^2 &\leq \eta(V, V) \eta(P - Q, P - Q) \left[1 + \frac{1}{4} \eta(P - Q, P - Q)\right]. \qedhere
% \end{align*}
% \end{proof}



%%%%%%%%%%%%%%%%%%%%%%%%%%%%%%%%%%%%%%%%%%%%%%%%%%

\subsection{Functions of least gradient}\label{LeastGradientFunctions}
Recall that a function $u \in BV_\loc$ has least gradient iff $u$ is a critical point of the action
$$\int_B \star |\dif u|.$$

\begin{proposition}[Miranda trace theorem]\label{traces}
Let $U \Subset M$ be an open set with Lipschitz boundary.
For every $u \in BV_\loc(U)$ there exists $v \in L^1_\loc(\partial U)$ such that for every $d-1$-form $\psi$,
\begin{equation}\label{Miranda IBP}
\int_U \dif u \wedge \psi + \int_U u \dif \psi = \int_{\partial U} v\psi.
\end{equation}
Moreover, for almost every $x \in \partial U$,
\begin{equation}\label{convergence of trace}
\int_{U \cap B(x, \varepsilon)} \star |v(x) - u| \ll \varepsilon^d.
\end{equation}
\end{proposition}
\begin{proof}
The assertion (\ref{Miranda IBP}) is diffeomorphism-invariant and so follows from \cite[Teorema 1]{Miranda67}, and (\ref{convergence of trace}) also follows from that result if we are willing to drop a constant factor.
\end{proof}

To state our a priori estimates we define
$$\eta(u, U) := \inf_{v \in BV_\cpt(U)} \int_U \star |\dif(u + v)|$$
for $u \in BV_\loc(M)$ and $U \Subset M$, so that $u$ has least gradient iff $\eta(u, U) = \int_U \star |\dif u|$ for every $U$.

Suppose that $u, v \in BV_\loc(M)$ and $U \Subset M$ is bounded by a Lipschitz hypersurface $N$. Armed with the Miranda trace theorem, it is straightforward to generalize \cite[Lemma 5.6]{Giusti77}, thus
\begin{equation}
|\eta(u, U) - \eta(v, U)| \leq ||u - v||_{L^1(N)}. \label{a priori estimate 1}
\end{equation}
In case $v = 0$, we note that by (\ref{convergence of trace}), the trace map is a contraction in $L^\infty$ norm, thus
\begin{equation}
\eta(u, U) \leq ||u||_{L^1(N)} \leq |N| \cdot ||u||_{L^\infty(M)}. \label{a priori estimate 2}
\end{equation}

\begin{definition}
A sequence $(u_n)$ in $BV_\loc(M)$ has \dfn{approximately least gradient} if for every open $U \Subset M$,
$$\limsup_{n \to \infty} \int_U \star |\dif u_n| \leq \liminf_{n \to \infty} \eta(u_n, U).$$
\end{definition}

\begin{proposition}[Miranda stability theorem]\label{Miranda convergence}
If a sequence of functions $(u_n)$ has approximately least gradient and $u_n \to u$ in $L^1_\loc(M)$, then $u$ has least gradient, and for every open set $U \Subset M$ with Lipschitz boundary such that $\int_{\partial U} \star |\dif u| = 0$, one has
\begin{equation}\label{convergence in total variation}
\lim_{n \to \infty} \int_U \star |\dif u_n| = \int_U \star |\dif u|.
\end{equation}
\end{proposition}
\begin{proof}
The proof is similar to Teorema 3 and Osservazione 3 in \cite{Miranda67}; we just note the necessary modifications.
Suitable generalizations of Teorema 2 and Osservazione 2 follow from Proposition \ref{traces}.
One needs to add a term of size $o(1)$ to the right-hand side of the inequalities (2.8), (2.9), (2.13), and (2.14); however, in the limit, this term vanishes and so the conclusions (2.15) and (2.16) are unaffected.
\end{proof}

The somewhat unusual condition $\int_{\partial U} \star |\dif u| = 0$ refers to the same Radon measure $\star |\dif u|$ that acts on the open sets of $M$, not on a measure that acts on the relatively open subsets of $\partial U$.
It should be interpreted as a transversality condition: if $u$ is the indicator function of a set $Z$ with $C^\infty$ boundary, then $\int_{\partial U} \star |\dif u| = 0$ if $\partial U$ is transverse to $\partial Z$.

\begin{corollary}\label{compactness}
Every sequence $(u_n)$ of approximately least gradient converges in $L^1_\loc$ and almost everywhere along a subsequence to a function of least gradient $u$ such that for every open set $U \Subset M$ of Lipschitz boundary such that $\int_{\partial U} \star |\dif u| = 0$, one has (\ref{convergence in total variation}).
\end{corollary}
\begin{proof}
By compactness of the natural map $BV_\loc \to L^1_\loc$ we know that $(u_n)$ has a convergent subsequence in $L^1_\loc$ and almost everywhere.
The conditions on the limit $u$ follow from the Miranda stability theorem.
\end{proof}

\begin{proposition}\label{level sets are minimal}
For every function $u$ of least gradient, the superlevel sets $\{u > y\}$ have least perimeter.
If we instead have a sequence $(u_n)$ of approximately least gradient, then $(\{u_n > y\})$ has approximately least perimeter.
\end{proposition}
\begin{proof}
In the proof of \cite[Theorem 1]{BOMBIERI1969}, replace the coarea formula \cite[Theorem 1.6]{Miranda66} with Proposition \ref{Coarea2} and replace \cite[Teorema 3]{Miranda67} with Proposition \ref{Miranda convergence}.
\end{proof}

%%%%%%%%%%%%%%%%%%%%%%%%%%%%%%%%%%%%%%%%%%%%

\subsection{Blowup of the reduced boundary}
Now let us study the blowup of $M$ at a point $p$ on the reduced boundary of a set $U$ of least perimeter, giving a generalization of the conjunction of \cite[Theorem 9.3]{Giusti77} and \cite[Theorem 6.2.2]{Simons68}.

\begin{definition}
For a function $u$ on $M$, $P \in M$ we define the \dfn{tangent rescaling} of $u$ at $P$ to be the net of functions $u_t: T_PM \to \RR$, given by
$$u_t(v) = u\left(\exp_P(tv)\right)$$
\end{definition}

\begin{proposition}\label{blowup theorem}
Suppose that $U$ is an open set with least perimeter in $B(P, r)$, $P \in \partial^* U$, and $u = 1_U$.
Furthermore, suppose that $d \leq 7$.
Then the tangent rescaling of $u$ converges as $t \to 0$ along a subsequence (that we also denote $t \to 0$) in $L^1_\loc$ and almost everywhere, to the indicator function $v$ of a half-space $C \subset T_PM$ such that $0 \in \partial C$.
Moreover, for every open set $V \Subset T_PM$ of Lipschitz boundary such that $\partial V$ is transverse to $\partial C$,
$$\lim_{t \to 0} \int_V \star |\dif u_t| = \int_V \star |\dif v|.$$
\end{proposition}
\begin{proof}
We claim that the tangent rescaling $(u_t)$ has approximately least gradient in $T_PM$ (where we give $T_PM$ its euclidean metric). If this true, then by Corollary \ref{compactness}, there exists a set $C$ of least perimeter in $T_PM$, such that the tangent rescaling converges to $v := 1_C$ in the desired sense.
But $T_PM$ is isometric to $\RR^d$, $d \leq 7$, so by the Bernstein--Fleming theorem \cite[Theorem 17.3]{Giusti77}, $\partial C$ is a hyperplane.
The fact that $0 \in \partial C$ follows from the fact that $P \in \partial^* U$.

To prove the claim, write $|\cdot|'$, $\star'$ for the notions taken in the tangent space with its euclidean geometry, and write $U_t$ for the set indicated by $u_t$.
If $V$ is a precompact open subset of $T_PM$, $V_t = \{v \in T_PM: tv \in V\}$, then we have the scale-invariance
\begin{equation}\label{almost blowup scale invariance}
|\partial^* U_t \cap V|' = t^{1 - d}|\partial^* U_1 \cap V_{1/t}|'.
\end{equation}
From (\ref{almost blowup scale invariance}) and the Taylor expansion of $g$ in normal coordinates \cite[Lemma 3.4]{schoen1994lectures},
$$t^{d - 1} |\partial^* U_t \cap V|' = |\partial^* U_1 \cap V_{1/t}|' \leq e^{O(t^2)} |\partial^* U \cap \exp_P(V_{1/t})|.$$
For every $w \in BV_\cpt(V)$, the least-gradient nature of $u$ gives
$$|\partial^* U \cap \exp_P(V_{1/t})| \leq \int_{(\exp_P)_* V_{1/t}} \star |\dif(u + (\exp_P)_* w_{1/t})| \leq e^{O(t^2)} \int_{V_{1/t}} \star'|\dif(u_1 + w_{1/t})|'.$$
Therefore, after applying (\ref{almost blowup scale invariance}) and the Taylor expansion again,
$$|\partial^* U_t \cap V|' \leq e^{O(t^2)} t^{1 - d} \int_{V_{1/t}} \star' |\dif (u_1 + w_{1/t})| = e^{O(t^2)} \int_V \star' |\dif (u_t + w)|.$$
Since $V,w$ were arbitrary, we conclude that $(u_t)$ has approximately least gradient.
\end{proof}

  %%%%%%%%%%%%%%%%%%%%%%%%%%%%%%%%%%%%%%%%%%%%%%%%%%%
\section{Monotonicity and mollification}\label{MollifierSection}
We have two purposes in this section: to prove Theorem \ref{monotonicity prestate} and to show that we can approximate minimal perimeters by $C^1$, approximately minimal perimeters.
Fix normal coordinates $(x^\mu)$, $\mu = 0, \dots, d - 1$, centered on a point $P \in M$ and define the closed $d-1$-form
\begin{equation}\label{d1 form}
\psi := \dif x^1 \wedge \dif x^2 \wedge \cdots \wedge \dif x^{d - 1}.
\end{equation}
We also write $B_r := B(P, r)$.
We use spherical coordinates $(\theta^i)$, $i = 1, \dots, d - 1$, on each sphere $\partial B_r$ which are compatible with $(x^\mu)$ in the sense that $x^0 = r \cos \theta^1$, which is possible because
\begin{equation}\label{partial Br is a variety}
\partial B_r = \{(x^0)^2 + \cdots + (x^{d - 1})^2 = r^2\}.
\end{equation}
We write $\dif \sigma$ for the standard measure on $\Sph^{d - 1}$.

\begin{remark}\label{independence of constants}
Let $K$ be a compact set in $M$.
Then all below implied constants may be chosen independently of the choice of $(x^\mu)$ as long as $P \in K$.
This holds because the space of all normal coordinate systems with basepoint in $K$ is the compact set $K \times \SpOrth_d$.
\end{remark}

%%%%%%%%%%%%%%%%%%%%%%%%%%%%%%%%%
\subsection{Monotonicity formula}
To prepare for the monotonicity formula, we first generalize an estimate that can be isolated from the proof of \cite[Lemma 5.8]{Giusti77}.

\begin{lemma}\label{monotonicity lemma}
Let $u \in C^1(B_R)$, $0 < r_1 < r_2 < R$, and let
$$E(r) = \int_{B_r} \star |\dif u| - \eta(u, r),$$
so that $E(R) = 0$ iff $u$ has least gradient. Then there exists $A \geq 0$ such that for $R > 0$ small,
\begin{equation}\label{monotonicity lemma eqn}
0 \leq \int_{B_{r_2} \setminus B_{r_1}} \star r^{1 - d}\frac{(\partial_ru)^2}{|\dif u|} \leq 2\int_{r_1}^{r_2} \partial_r \left[e^{Ar^2} r^{1-d}\int_{B_r} \star |\dif u|\right] + \frac{O(E(r))}{r^d} \dif r.
\end{equation}
\end{lemma}
\begin{proof}
We fix $s \in [r_1, r_2]$, introduce a competitor $v(r, \theta) = u(s, \theta)$, and allow $A \geq 0$ to be a constant which may vary from line to line.
From the definition of $\eta$,
\begin{equation}\label{consequence of least gradient monotone}
    \eta(u, s) \leq \int_U \star |\dif v| = \int_0^s \int_{\partial B_r} \star_r |\dif v| \dif r.
\end{equation}
Taylor expanding the volume form and the metric \cite[Lemma 3.4]{schoen1994lectures}, and applying $\partial_r v = 0$, we obtain the existence of $A \geq 0$ such that
\begin{equation}\label{introduce the ricci tensor}
\int_{\partial B_r} \star_r |\dif v| \leq e^{As^2} \frac{\tilde r^{d - 1}}{s^{d - 1}} \int_{\partial B_s} \star_s |\dif v|.\
\end{equation}
Applying (\ref{consequence of least gradient monotone}) and Fubini's theorem,
\begin{align*}
\eta(u, s) &\leq  e^{As^2} \int_0^s \frac{r^{d - 1}}{s^{d - 1}} \dif r \cdot \int_{\partial B_s} \star_s |\dif v| = \frac{s e^{As^2}}{d} \int_{\partial B_s} \star_s |\dif v|\\
&\leq \frac{s e^{As^2}}{d - 1} \int_{\partial B_s} \star_s |\dif v|.
\end{align*}
By Gauss' lemma, $\dif v$ is the orthogonal projection of $\dif u$ onto $T' \partial B_s$, and its orthocomplement is $\partial_r u$. Therefore by Taylor's theorem,
$$\int_{\partial B_s} \star_s |\dif v| \leq \int_{\partial B_s} \star_s |\dif u| \sqrt{1 - \frac{(\partial_r u)^2}{|\dif u|^2}} \leq \int_{\partial B_s} \star_s \left[|\dif u| - \frac{(\partial_r u)^2}{2 |\dif u|}\right]$$
or in other words
\begin{align*}
\int_{\partial B_s} \star_s \frac{(\partial_r u)^2}{2|\dif u|} &\leq \int_{\partial B_s} \star_s |\dif u| - \frac{d - 1}{s} e^{-As^2} \eta(u, s)\\
&\leq \int_{\partial B_s} \star_s |\dif u| - \frac{d - 1}{s} e^{-As^2} \int_{B_s} \star |\dif u| - O(s^{-1}E(s)).
\end{align*}
We moreover have for $\tilde A \geq 0$ that
$$e^{-\tilde As^2} \partial_s \left[e^{\tilde As^2} s^{1 - d} \int_{B_s} \star |\dif u|\right] = \left[2\tilde As^{2 - d} - \frac{d - 1}{s^d}\right]\int_{B_s} \star |\dif u| + s^{1 - d} \int_{\partial B_s} \star_s |\dif u|$$
so if we choose $\tilde A$ so that
$$-\frac{d - 1}{s} e^{-As^2} = 2\tilde As - \frac{d - 1}{s}$$
then
$$s^{1 - d} \int_{\partial B_s} \star_s |\dif u| - (d - 1)\frac{e^{-As^2}}{s^d} \int_{B_s} \star|\dif u| \leq e^{-As^2} \partial_s\left(e^{As^2} s^{1 - d} \int_{B_s} \star|\dif u|\right).$$
We moreover have $e^{-As^2} \leq 1$, so we can now integrate with respect to $\dif s$ to conclude the result.
\end{proof}

\begin{proposition}[monotonicity formula]\label{Monotonicity Formula}\label{sharp monotonicity}
There exists $A \geq 0$ (depending continuously on $P$) such that for every function $u$ in least gradient in $B_R$ where $R$ is small, and every $0 < r < R$,
\begin{equation}\label{weak monotone}
\partial_r e^{Ar^2} r^{1 - d} \int_{B_r} \star|\dif u| \geq 0.
\end{equation}
Stronger,
\begin{align}\label{strong monotone}
&\left|\int_{r_1}^{r_2} \partial_r \left[r^{1 - d}\int_{B_r} \dif u \wedge \psi\right] \dif r\right|^2 \\
&\qquad \lesssim \left(1 + (d - 1)\log\frac{r_2}{r_1}\right)r_2^{1 - d}\int_{B_r} \star |\dif u| \int_{r_1}^{r_2} \partial_r \left[e^{Ar^2} r^{1 - d}\int_{B_r} \star |\dif u|\right] \dif r.
\end{align}
\end{proposition}
\begin{proof}
We first compute $\dif u \wedge \psi = \partial_0 u \dif x$
where $\dif x$ is the natural euclidean volume form on $T_PM$.
Moreover, the radial part of $\partial_0$ is $\cos \theta^1$, and $\iota_{\partial_r} \dif x = r^{d - 1} \dif \sigma$.
Thus by (\ref{partial Br is a variety}),
$$\int_{B_r} \dif u \wedge \psi = r^{d - 1}\int_{\partial B_r} u \cos \theta^1 \dif \sigma(\theta)$$
and hence, since $|\cos \theta^1| \leq 1$,
\begin{align}
\left|\int_{r_1}^{r_2} \partial_r \left[r^{1 - d}\int_{B_r} \dif u \wedge \psi\right] \dif r\right|
&= \left|\int_{\Sph^{d - 1}} (u(r_2, \theta) - u(r_1, \theta)) \cos \theta^1 \dif \sigma(\theta)\right| \\
&\leq \int_{\Sph^{d - 1}} |u(r_2, \theta) - u(r_1, \theta)| \dif \sigma(\theta). \label{monotone dump the metric}
\end{align}
The metric $g$ plays no role in (\ref{monotone dump the metric}), so we may use \cite[Lemma 5.3]{Giusti77} to bound
$$\int_{\Sph^{d - 1}} |u(r_2, \theta) - u(r_1, \theta)| \dif \sigma(\theta) \leq \int_{\Sph^{d - 1}} \int_{r_1}^{r_2} r^{1 - d}|\partial_r u(r, \theta)| \dif r \dif\sigma(\theta).$$
To reintroduce the metric we posit that $R$ is small enough that $\dif r \dif \sigma(\theta) \leq \star 2$.
We therefore have
\begin{equation}\label{monotone before cs}
\int_{\Sph^{d - 1}} \int_{r_1}^{r_2} r^{1 - d}|\partial_r u(r, \theta)| \dif r \dif\sigma(\theta) \leq 2 \int_{B_{r_2} \setminus B_{r_1}} \star r^{1 - d}|\partial_r u|
\end{equation}
and if we apply the Cauchy-Schwarz inequality and approximate $u$ by $C^1$ functions (see \cite[pg68]{Giusti77}), it follows from Lemma \ref{monotonicity lemma} that the right-hand side of (\ref{monotone before cs}) is
$$\lesssim \sqrt{\int_{B_{r_2} \setminus B_{r_1}} \star r^{1 - d} |\dif u|} \sqrt{\int_{r_1}^{r_2} \partial_r \left[e^{Ar^2} r^{1-d}\int_{B_r} \star |\dif u|\right] \dif r}.$$
The monotonicity (\ref{weak monotone}) follows at once. To strengthen it we just need to bound $r^{1 - d} |\dif u|$.
Integrating by parts,
\begin{align*}
\int_{B_{r_2} \setminus B_{r_1}} r^{1 - d} |\dif u| &= \int_{r_1}^{r_2} r^{1 - d} \partial_r \int_{B_r} \star |\dif u| \dif r \\
&\leq r^{1 - d} \int_{B_r} \star |\dif u| + (d - 1) \int_{r_1}^{r_2} r^{-d} \int_{B_r} \star |\dif u| \dif r.
\end{align*}
Using (\ref{weak monotone}) we bound this second integral as
\begin{align*}
\int_{r_1}^{r_2} r^{-d} \int_{B_r} \star |\dif u| \dif r &\leq r^{1 - d} \log \frac{r_2}{r_1} \int_{B_{r_2}} \star |\dif u|. \qedhere
\end{align*}
\end{proof}

In fact, if $g$ has negative Ricci curvature, then we can take $A = 0$ in (\ref{introduce the ricci tensor}) and hence in (\ref{strong monotone}). Indeed, one has for $\xi = x/|x|$,
$$\frac{\star_r 1}{\star_s 1}(\xi) = \frac{r^{d - 1}}{s^{d - 1}} \frac{1 - r^2 \Ric_{\mu \nu}(0) \xi^\mu \xi^\nu/3 + O(r^3)}{1 - s^2 \Ric_{\mu \nu}(0) \xi^\mu \xi^\nu/3 + O(s^3)}$$
which is $\leq 1$ for $r \leq s \lesssim 1$ and $(\Ric_{\mu\nu})$ negative-definite.
This could be used to make the proof of (\ref{strong monotone}) slightly easier; however, we prefer to work in this higher generality for now, as a hypothesis on the curvature in Theorem \ref{monotonicity prestate} seems rather unnatural.

%%%%%%%%%%%%%%%%%%%%%%%%%%%%%%%%%%%%%%%%%%%%%%%%%%%%%%%
\subsection{Applications of the monotonicity formula}
We now give several consequences of the monotonicity formula; the surface area estimate generalizes \cite[Remark 5.13]{Giusti77}.
Write $\Ball^\ell$ for the unit ball in $\RR^\ell$.

\begin{lemma}\label{least perimeter minimal size}
For a set $U$ of least perimeter, if $P \in \partial^* U$ and $d \leq 7$, one has
$$\lim_{r \to 0} r^{1 - d} |\partial^* U \cap B(P, r)| = |\Ball^{d - 1}|.$$
\end{lemma}
\begin{proof}
Choose a sequence of $r \to 0$; then there is a subsequence along which the limit in Proposition \ref{blowup theorem} exists for $u = 1_U$.
With notation as in the proof of Proposition \ref{blowup theorem},
$$r^{1 - d} |\partial^* U \cap B(P, r)| = e^{O(r^2)} r^{1 - d}\int_{B'_r} \star'|\dif u_1|' = e^{O(r^2)} \int_{B'_1} \star'|\dif u_r|'.$$
Then $u_r \to 1_C$ for $C$ a half-space, which in particular is transverse to $B'_1$.
So by the Miranda stability theorem, Proposition \ref{Miranda convergence},
\begin{align*}
\lim_{r \to 0} e^{O(r^2)} \int_{B'_1} \star'|\dif u_r|' &= \int_{B'_1} \star'|\dif 1_C|' = |\partial C \cap B'_1|' = |\Ball^{d - 1}|. \qedhere
\end{align*}
\end{proof}

\begin{proposition}[surface area estimate]\label{doubling dimension}
If $d \leq 7$ then there exists $A \geq 0$ such that for every set $U$ of least perimeter in a ball $B_r = B(P, r)$, with $P \in \partial^* U$, and $r > 0$ small,
$$|\Ball^{d - 1}|e^{-Ar^2}r^{d - 1} \leq |\partial^*U \cap B_r| \leq |\Sph^{d - 1}|e^{Ar^2} r^{d - 1}.$$
\end{proposition}
\begin{proof}
The upper bound on $|\partial^* U \cap B_r|$ is obtained by using (\ref{a priori estimate 2}) and the fact that the surface area of $\partial B_r$ is $|\Sph^{d - 1}|(1 + O(r^2))r^{d - 1}$ (see e.g. \cite{gray1974volume}).
The lower bound is obtained from Proposition \ref{Monotonicity Formula}, which implies that
$$\lim_{\rho \to 0} e^{-A\rho^2} \rho^{1 - d} |\partial^* U \cap B_\rho| \leq |\partial^* U \cap B_r|.$$
The left-hand side is given by Lemma \ref{least perimeter minimal size}.
\end{proof}

%%%%%%%%%%%%%%%%%%%%%%%%%%%%%%%%%%%%%%%%%%%%%%%%%%%%%%%%%%%%%%%%
\subsection{Mollification of sets of least perimeter}
Our goal for this section is to generalize \cite[Lemma 7.5]{Giusti77}, which reduces the study of sets of least perimeter to that of sets with $C^1$ perimeter.

In this section, the convolution $f * g$ of two functions defined near $P$, or the subtraction $x - y$ of two points near $P$, are meant in the sense of the normal coordinates $(x^\mu)$\footnote{and crucially, \emph{not} in terms of the normal coordinates $(\tilde x^\mu)$ that we introduce in the proof of Lemma \ref{mollifier sublemma}} and the volume form $\dif x := \dif x^0 \wedge \cdots \wedge \dif x^{d - 1}$ obtained from them. Following \cite[Chapter 7]{Giusti77} we define the convolution kernel
$$\chi_\varepsilon(x) = \frac{d + 1}{|\Ball^d|} \varepsilon^{-d}1_{|x| < \varepsilon} \left(1 - \frac{|x|}{\varepsilon}\right)$$
For this section only, we write $u_\varepsilon = u * \chi_\varepsilon$ whenever $u \in BV$ is defined near $P$.

Our first goal is to prove a Riemannian analogue of \cite[Theorem 7.3]{Giusti77}, which estimates $(1_U)_\varepsilon$ for $U$ a set of least perimeter.
The argument there required one to cover $\partial^* U$ by small balls and apply the monotonicity formula in each ball.
To deal with the somewhat large number of parameters involved here, it may helpful to think of the case $\Delta = 1$, in which case we have
$$1 \gg \gamma^q \gg \sigma \gg \gamma \gg \varepsilon \gg \varepsilon \delta > 0.$$

\begin{lemma}[control on $u_\varepsilon$ in each ball]\label{mollifier sublemma}
Let $q < \min(1/8, 1/(4(d - 1)))$.
For every $0 < \Delta, \gamma \lesssim 1$, if we let $\varepsilon = \gamma^4 \Delta$, $\sigma = \gamma^{1/(2(d - 1))} \Delta$, $\delta = \gamma^d$, and let $u$ be the indicator function of a set $U$ of least perimeter such that
\begin{equation}\label{hypothesis on mollifier sublemma}
\int_{B_\Delta} \star(|\psi| \cdot |\dif u| - \dif u \wedge \psi) \leq \Delta^{d - 1} \gamma,
\end{equation}
then on $B_{\Delta - 2\sigma}$, if $Q \in \partial^* U$,
$$(1_{B(Q, 2\delta\varepsilon)}(|\psi| \cdot |\dif u| - \star(\dif u \wedge \psi)))_\varepsilon \ll \gamma^q (1_{B(Q, \delta\varepsilon)} |\dif u|)_\varepsilon.$$
\end{lemma}
\begin{proof}
We first claim that for $r > 0$ so small that $B(Q, 2r) \subseteq B_\varepsilon$,
\begin{equation}\label{bound the kernel}
\sup_{y \in B(Q, 2r)} \chi_\varepsilon(x - y) \lesssim \inf_{y \in B(Q, r)} \chi_\varepsilon(x - y).
\end{equation}
In the euclidean case (with constant equal to $4$) this result can be isolated from the proof of \cite[Theorem 7.3]{Giusti77}.
Otherwise, we can use the smallness of $\varepsilon$ to approximate $g$-balls by euclidean balls.
This suffices to prove (\ref{bound the kernel}), since $\chi_\varepsilon$ is uniformly continuous.

Now let $V := B(Q, 2\delta\varepsilon)$.
Integrating (\ref{bound the kernel}) against $1_V(|\psi| \cdot |\dif u| - \star(\dif u \wedge \psi))$,
\begin{equation}\label{kernel bounded}
(1_V(|\psi| \cdot |\dif u| - \star(\dif u \wedge \psi)))_\varepsilon(x) \lesssim \inf_{y \in B(Q, \delta\varepsilon)} \chi_\varepsilon(x - y) \int_V \star |\psi| \cdot |\dif u| - \dif u \wedge \psi.
\end{equation}
To estimate the right-hand side of (\ref{kernel bounded}) we introduce a new coordinate system $(\tilde x^\mu)$ of normal coordinates centered on $Q$ which are compatible with $(x^\mu)$ in the sense that $\dif \tilde x^0$ is a scalar multiple of $\dif x^0$ at $Q$.
We write $\tilde g$ for the metric written in these new coordinates, and write
$$\tilde \psi := \dif \tilde x^1 \wedge \cdots \wedge \dif \tilde x^{d - 1}.$$
Then a Taylor expansion gives the bound
\begin{align}
|\tilde \psi - \psi| &= |\star(\tilde \psi - \psi)| = |\sqrt{\det g} g^{11} \cdots g^{(d-1)(d-1)} \dif x^0 - \sqrt{\det \tilde g} \tilde g^{11} \cdots \tilde g^{(d-1)(d-1)} \dif \tilde x^0|\\
&\lesssim |\det g - \det \tilde g| + |g^{11} \cdots g^{(d-1)(d-1)} - \tilde g^{11} \cdots \tilde g^{(d-1)(d-1)}| + |\dif(x^0 - \tilde x^0)| \lesssim \varepsilon. \label{T vs Ttilde}
\end{align}
In particular,
\begin{equation}\label{split up T Ttilde}
\int_V \star |\psi| \cdot |\dif u| - \dif u \wedge \psi \leq \int_V \star |\psi| \cdot |\dif u| - \dif u \wedge \tilde \psi + O(\varepsilon) \int_V \star |\dif u|.
\end{equation}
The error term here is given by Proposition \ref{doubling dimension} and the assumption $\Delta \lesssim 1$ as $\lesssim \gamma^4 \int_{B(Q, \delta\varepsilon)} \star |\dif u|$.
To estimate the dominant term in (\ref{split up T Ttilde}) we assume that $\gamma$ is chosen so small that $\sigma > 2\delta\varepsilon$, so that if we set $W := B(Q, \sigma)$ and apply Proposition \ref{Monotonicity Formula} and (\ref{T vs Ttilde}) to obtain $A \geq 0$ such that
\begin{align*}
(2\delta\varepsilon)^{1 - d} &\int_V \star |\psi| \cdot |\dif u| - \dif u \wedge \psi \\
&\leq \sigma^{1 - d}\int_W \star |\dif u| + (2\delta\varepsilon)^{1 - d} \int_V \star(|\psi| - 1)|\dif u| + 2A\sigma^{3 - d} \int_W \star |\dif u| - (2\delta\varepsilon)^{1 - d}\int_V \dif u \wedge \psi\\
&\leq \sigma^{1 - d}\int_W \star |\dif u| - \dif u \wedge \psi + (2\delta\varepsilon)^{1 - d} \int_V \star(|\psi| - 1)|\dif u| + 2A\sigma^{3 - d} \int_W \star |\dif u| \\
&\qquad + O(\varepsilon \sigma^{1 - d}) \int_W \star |\dif u| + \sigma^{1 - d}\int_W \dif u \wedge \tilde \psi - (2\delta\varepsilon)^{1 - d}\int_V \dif u \wedge \tilde \psi\\
&=: I_1 + I_2 + I_3 + I_4 + I_5 - I_6.
\end{align*}
We apply (\ref{hypothesis on mollifier sublemma}) to bound $I_1 \leq \gamma^{1/2}$, and we use the Taylor expansion of the metric on $V$ to bound $|\psi| - 1 \lesssim \varepsilon^2$ on $B_\varepsilon$.
By Proposition \ref{doubling dimension} and the assumption $\Delta \lesssim 1$, $I_2 \lesssim \gamma^8$, $I_3 \lesssim \gamma^{1/(d - 1)}$, and $I_4 \lesssim \gamma^4$.

To estimate $I_5 - I_6$, we apply Proposition \ref{sharp monotonicity}:
\begin{align*}
&\sigma^{1 - d} \int_W \dif u \wedge \tilde \psi - (2\delta\varepsilon)^{1 - d} \int_V \dif u \wedge \tilde \psi \\
&\qquad \lesssim \sqrt{1 + (d - 1) \log \frac{\sigma}{2\delta\varepsilon}} \sqrt{\sigma^{1 - d} \int_W \star |\dif u|} \sqrt{\int_{2\delta\varepsilon}^\sigma \partial_r \left[e^{Ar^2} r^{1 - d} \int_{B(Q, r)} \star |\dif u|\right] \dif r}\\
&\qquad =: J_1 J_2 J_3.
\end{align*}
We have $J_1 \lesssim -\log \gamma$, and from Proposition \ref{doubling dimension} we have $J_2 \lesssim 1$.
So, we need to get a gain from $J_3$, which we do as follows:
\begin{align*}
J_3^2 &\leq \sigma^{1 - d} \int_W \star |\dif u| - (2 \delta \varepsilon)^{1 - d} \int_V \star |\dif u| + 2A\sigma^{3 - d} \int_W \star |\dif u| \\
&= \sigma^{1 - d} \int_W \star |\dif u| - \dif u \wedge \psi + \sigma^{1 - d} \int_W \dif u \wedge (\psi - \tilde \psi) + \sigma^{1 - d} \int_W \dif u \wedge \tilde \psi \\
&\qquad - (2 \delta\varepsilon)^{1 - d} \int_V \star |\dif u| + 2A \sigma^{3 - d} \int_W \star |\dif u| \\
&=: K_1 + K_2 + K_3 - K_4 + K_5.
\end{align*}
Then $K_1 = I_1 \leq \gamma^{1/2}$, $K_2 \lesssim I_3 \lesssim \gamma^4$, and $K_5 = I_2 \lesssim \gamma^{1/(d - 1)}$.

To estimate $K_3 - K_4$ we observe that for $u = 1_U$,
\begin{equation}\label{K3 calculus}
K_3 = \sigma^{1 - d} \int_W \dif u \wedge \tilde \psi = \sigma^{1 - d} \int_{U \cap \partial W} \tilde \psi.
\end{equation}
We decompose
$$\partial W = \Gamma_+ \cup \Gamma_0 \cup \Gamma_-$$
where $\tilde x^0 > 0$ on $\Gamma_+$ and $\tilde x^0 < 0$ on $\Gamma_-$. Then all positive contributions to the integral in the right-hand side of (\ref{K3 calculus}) come from $\Gamma_+$.
Moreover, as $d-1$-cells in $M$, $\partial \Gamma_+ = \Gamma_0$, but also if we set $N = \{\tilde x^0 = 0\}$ and $W_0 = W \cap N$, then $\Gamma_0 = \partial W_0$.
In particular, there is a homotopy relating $\Gamma_+$ and $\partial W_0$ which holds $\Gamma_0$ fixed.
Since $\dif \psi = 0$, we can use (\ref{partial Br is a variety}) as follows:
\begin{align*}
K_3 &\leq \sigma^{1 - d} \int_{\Gamma_+} \tilde \psi = \sigma^{1 - d} \int_{W_0} \tilde \psi = |\Ball^{d - 1}|.
\end{align*}
Hence by Proposition \ref{doubling dimension},
$$K_3 \leq |\Ball^{d - 1}| \leq K_4 e^{O(\varepsilon\delta)^2} \leq K_4 + O(\varepsilon\delta)^2 \leq K_4 + O(\gamma^{8 + 2d}).$$
In conclusion, $J_3 \lesssim \gamma^{\min(1/4, 1/(2(d - 1)))}$, and hence by (\ref{kernel bounded}) we have
$$(1_V(|\dif u| - \star(\dif u \wedge \psi)))_\varepsilon(x) \ll (\delta\varepsilon)^{d - 1} \gamma^q \inf_{y \in B(Q, \delta\varepsilon)} \chi_\varepsilon(x - y).$$
Finally, by Proposition \ref{doubling dimension},
\begin{align*}
(\delta\varepsilon)^{d - 1} \inf_{y \in B(Q, \delta\varepsilon)} \chi_\varepsilon(x - y) &\lesssim (1_{B(Q, \delta\varepsilon)} |\dif u|)_\varepsilon(x). \qedhere
\end{align*}
\end{proof}

\begin{lemma}[control on $u_\varepsilon$]\label{main mollifier lemma}
Let $q < \min(1/8, 1/(4(d - 1)))$. There exists $c > 0$ such that for every $0 < \Delta \lesssim 1$ such that for every $0 < \gamma \lesssim 1$ and every indicator function $u$ of a set $U$ of least perimeter such that
\begin{equation}\label{hypothesis on main mollifier lemma}
\int_{B_\Delta} \star |\psi| \cdot |\dif u| - \dif u \wedge \psi \leq \gamma \Delta^{d - 1},
\end{equation}
if we let $\varepsilon = \gamma^4\Delta$, $\sigma = \gamma^{1/(2(d - 1))}\Delta$, and $\varphi = u_\varepsilon$, then on $B_{\Delta - 2\sigma} \cap \{c\gamma^2 < \varphi < 1 - c\gamma^2\}$,
\begin{equation}\label{claim on main mollifier lemma}
(1 - o(\gamma^q)) |\psi| \cdot |\dif \varphi| \leq \star(\dif \varphi \wedge \psi)
\end{equation}
and for every $y \in (c\gamma^2, 1 - c\gamma^2)$,
\begin{equation}\label{claim 2 on main mollifier lemma}
\text{the level set } \partial \{\varphi > y\} \cap B_{\Delta - 2\sigma} \text{ is a }C^1\text{ hypersurface}.
\end{equation}
\end{lemma}
\begin{proof}
Let $\delta = \gamma^d > 0$.
By running a greedy algorithm, we construct a cover $\mathcal V = \{V_n: 1 \leq n \leq N\}$ of $\partial^* U \cap B_{\varepsilon(1 - 2\delta)}$ by balls of radius $2\delta\varepsilon$, centered on points $Q_n \in \partial^* U \cap B_{\varepsilon(1 - \delta)}$, which is \dfn{efficient} in the sense that the dilates $V_n/2 := B(Q_n, \delta\varepsilon)$ are disjoint.
We set $V_0 := B_\varepsilon \setminus B_{\varepsilon(1 - 2\delta)}$.

To bring the $\psi$ inside the convolution we compute
\begin{align*}
(|\psi| \cdot |\dif u|)_\varepsilon
&= \int_{B_\varepsilon} |\psi|(x - y) |\dif u|(x - y) \chi_\varepsilon(y) \dif y \\
&= \int_{B_\varepsilon} |\psi|(x) |\dif u|(x - y) \chi_\varepsilon(y) \dif y + \int_{B_\varepsilon} ||\psi|(x - y) - |\psi|(x)| \cdot |\dif u|(x - y) \chi_\varepsilon(y) \dif y
\end{align*}
and observe that since $\psi$ is continuous, $||\psi|(x - y) - |\psi|(x)| \lesssim \varepsilon \lesssim \gamma^4$.
Since $\dif u$ is supported in $\bigcup_n V_n$, it follows that
\begin{equation}\label{sum over cover}
|\psi| \cdot |\dif \varphi| - \star(\dif \varphi \wedge \psi)
\leq O(\gamma^4) |\dif \varphi| + \sum_{n=0}^N (1_{V_n}(|\psi| \cdot |\dif u| - \star(\dif u \wedge \psi)))_\varepsilon.
\end{equation}

We claim that there exists $c > 0$ such that on $B_\sigma \cap \{c\gamma^2 < \varphi < 1 - c\gamma^2\}$,
\begin{equation}\label{V0 case}
(1_{V_0}(|\psi| \cdot |\dif u| - \star(\dif u \wedge \psi)))_\varepsilon \lesssim \gamma |\dif u|_\varepsilon.
\end{equation}
The proof of (\ref{V0 case}) is essentially given by \cite[pg92]{Giusti77}, so we just sketch it.
For $y \in V_0$, $\chi_\varepsilon(x - y) \lesssim \delta/\varepsilon^d$, so using Proposition \ref{doubling dimension}, one can show
$$\int_{V_0} \chi_\varepsilon(x - y)(|\psi| \cdot |\dif u| - \star(\dif u \wedge \psi))(y) \dif y \lesssim \frac{\gamma^d}{\varepsilon}.$$
Here we used $||\psi||_{L^\infty} \lesssim 1$.
One can use \cite[Lemma 7.1]{Giusti77}, the assumption $c\gamma^2 < \varphi < 1 - c\gamma^2$, and the fact that $g$ is a perturbation of the euclidean metric to obtain
$$\int_{B_\varepsilon} \chi_\varepsilon(x - y) |\dif u|(y) \dif y \gtrsim \frac{\gamma^{d - 1}}{\varepsilon}$$
which then implies (\ref{V0 case}).

Since $\mathcal V$ is efficient, we can sum (\ref{V0 case}) and Lemma \ref{mollifier sublemma} over $n$ in (\ref{sum over cover}) to show that (\ref{claim on main mollifier lemma}) holds.
In particular near $\varphi^{-1}(y) \cap B_{\Delta - 2\sigma}$, where $y \in (c\gamma^2, 1 - c\gamma^2)$, one has $|\dif u| > 0$.
Therefore $u$ is a $C^1$ submersion by \cite[Lemma 7.1]{Giusti77}, thus (\ref{claim 2 on main mollifier lemma}) holds.
\end{proof}

\begin{lemma}[some known inequalities]
Suppose that $u = 1_U$ where $U$ has locally finite perimeter, $w = u_\varepsilon$, and $V = \{w > y\}$ where $y \in (0, 1)$.
Then for any submanifold $\iota: E \to M$ and $0 < \varepsilon \lesssim 1$,
\begin{equation}\label{Giusti125}
x|E \cap (U \Delta V)| \leq \frac{1}{\min(y, 1 - y)} \int_E |w - u| \iota^*(\star 1).
\end{equation}
Moreover, if $0 < \tau \lesssim 1$, then
\begin{align}
\int_{B_\tau} \star |w - u| &\lesssim \varepsilon |B_{\tau + \varepsilon} \cap \partial^* U|, \label{Giusti711}\\
\int_{B_\tau} \star (|\dif w| - |\dif u|) &\lesssim |(B_{\tau + \varepsilon} \setminus B_\tau) \cap \partial^* U|. \label{Giusti712}
\end{align}
\end{lemma}
\begin{proof}
The estimate (\ref{Giusti125}) is a straightforward modification of \cite[Lemma 1.25]{Giusti77}.
The estimates (\ref{Giusti711}, \ref{Giusti712}) follow from \cite[Lemma 7.2]{Giusti77} where we use the fact that for $\tau \lesssim 1$, we can impose normal coordinates in which $\star 1$ is approximately the euclidean volume form.
\end{proof}

On first reading of the next lemma, it may help to take $P_n = P$, $(x^\mu_n) = (x^\mu)$, $\omega = \star \dif x^\mu$, and $\Delta_n = 1$.

\begin{lemma}\label{mollifier quant}
Let $q < \min(1/8, 1/(4(d - 1)))$.
Let $(P_n)$ be a precompact sequence in $M$ and $0 < \Delta_n \lesssim 1$.
Let $U_n$ be a set of least perimeter in $B_n := B(P_n, \Delta_n)$, $(x^\mu_n)$ be a normal coordinate system at $P_n$, let $\psi^n$ be given by (\ref{d1 form}), and
$$\gamma_n := \Delta_n^{1 - d} \int_{B_n} \star |\psi^n| \cdot |\dif 1_{U_n}| - \dif 1_{U_n} \wedge \psi^n.$$
If $(\gamma_n) \in \ell^1$ then for almost every $t \in (0, 1)$ and every $n \in \NN$ there exists a set $V_n$ with $C^1$ boundary in $tB_n := B(P_n, t\Delta_n)$ such that for $n \gg 1$,
\begin{align}
|\partial V_n \cap tB_n| &\leq \eta(V_n, t\Delta_n) + o(\gamma_n \Delta_n^{d - 1}), \label{mollifier quant1}\\
||\partial^* U_n \cap tB_n| - |\partial V_n \cap tB_n|| &\ll \gamma_n \Delta_n^{d - 1}, \label{mollifier quant2}
\end{align}
on $B(P_n, t(1 - 2\sigma_n)\Delta_n)$, where $\sigma_n := \gamma_n^{1/(2(d - 1))}$, one has
\begin{equation}
\star(\normal_{V_n} \wedge \psi^n) \geq |\psi^n| \cdot (1 - o(\gamma^q)), \label{mollifier quant4}
\end{equation}
and for every $d-1$-form $\omega_n$ defined near $P_n$,
\begin{equation}\label{mollifier quant3}
\left|\int_{tB_n} \dif(1_{U_n} - 1_{V_n}) \wedge \omega_n\right| \ll \gamma_n \Delta_n^{d - 1} ||\omega_n||_{C^1}.
\end{equation}
\end{lemma}
\begin{proof}
In this proof, we shall assume that the constants furnished by the above results are uniform in $n$; this is possible by Remark \ref{independence of constants} and the compactness of $\overline{\{P_n: n \in \NN\}}$.

\proofpart{1}{Construction of $V_n$ and proof of (\ref{mollifier quant4})}
Draw $t$ uniformly at random, let $w_n := (u_n)_{\Delta_n \gamma_n^4}$, let $c, q$ be as in Lemma \ref{main mollifier lemma}, and let $a_n = c\gamma_n^2$, $b_n = 1 - c\gamma_n^2$.
By the coarea formula, Proposition \ref{Coarea2},
$$\int_{tB_n} \star |\dif w_n| = \int_0^1 |\partial^* \{w_n > y\} \cap tB_n| \dif y \geq \int_{a_n}^{b_n} |\partial^* \{w_n > y\} \cap tB_n| \dif y,$$
so by the mean value theorem, there exists $y_n \in (a_n, b_n)$ such that
$$|\partial^* \{w_n > y_n\} \cap tB_n| \leq \frac{1}{b_n - a_n} \int_{tB_n} \star |\dif w_n|.$$
We then set $V_n := \{w_n > y_n\}$, $v_n := 1_{V_n}$, so $V_n$ has $C^1$ boundary in $tB_n$ by (\ref{claim 2 on main mollifier lemma}), and by the above computation,
\begin{equation}\label{MVT mollifier}
|V_n \cap tB_n| \leq \frac{1}{1 - 2c\gamma_n^2} \int_{tB_n} \star |\dif w_n|.
\end{equation}
Since $\grad w_n$ is normal to the level sets of $w_n$, $\normal_{V_n} = \dif w_n/|\dif w_n|$.
Hence (\ref{mollifier quant4}) is a consequence of (\ref{claim on main mollifier lemma}).

\proofpart{2}{Auxiliary estimates}
Let $\Gamma_n := \partial(tB_n)$; we claim that almost surely,
\begin{align}
||u_n - v_n||_{L^1(\Gamma_n)} &\ll \Delta_n^{d - 1} \gamma_n \label{trace of vn} \\
|\partial V_n \cap tB_n| &\leq |\partial^* U_n \cap tB_n| + o(\Delta_n^{d - 1} \gamma_n). \label{difference of surface area}
\end{align}
To establish the claim, observe that by (\ref{Giusti711}) and Proposition \ref{doubling dimension},
$$\limsup_{n \to \infty} \Delta_n^{1 - d} \gamma_n^{-4} \int_{tB_n} \star |u_n - w_n| \lesssim \limsup_{n \to \infty} \Delta_n^{2-d} |\partial^* U_n \cap tB_n| \lesssim \sup_n \Delta_n \lesssim 1.$$
Differentiating in $t$, we see that
$$||u_n - w_n||_{L^1(\Gamma_n)} \ll \Delta_n^{d - 1} \gamma_n^3$$
almost surely. So by (\ref{Giusti125}),
$$||u_n - v_n||_{L^1(\Gamma_n)} \lesssim \gamma_n^{-2} ||u_n - w_n||_{L^1(\Gamma_n)} \ll \Delta_n^{d - 1} \gamma_n$$
almost surely, proving (\ref{trace of vn}).
Now let
$$f(s) = \sum_{n=1}^\infty \gamma_n \Delta_n^{1 - d} \int_{sB_n} \star |\dif u_n|.$$
Then $f' \geq 0$, and by Proposition \ref{doubling dimension}, $f(1) \lesssim \sum_n \gamma_n < \infty$.
So almost surely,
$$f(t + \gamma_n^4) - f(t) \lesssim \gamma_n^4$$
and hence
$$\int_{(t + \gamma_n^4)B_n \setminus tB_n} \star |\dif u_n| \lesssim \gamma_n^3 \Delta_n^{d - 1}.$$
From (\ref{Giusti712}) it follows that
$$\int_{tB_n} \star |\dif w_n| \leq \int_{tB_n} \star |\dif u_n| + o(\gamma_n^2).$$
By (\ref{MVT mollifier}) we conclude that (\ref{difference of surface area}) holds almost surely.

\proofpart{3}{Proof of (\ref{mollifier quant1}) and (\ref{mollifier quant2})}
The estimate (\ref{mollifier quant2}) is the conjunction of (\ref{trace of vn}), (\ref{difference of surface area}), and (\ref{a priori estimate 1});
(\ref{mollifier quant1}) is the conjunction of (\ref{mollifier quant2}), (\ref{a priori estimate 1}), the fact that $U_n$ has least perimeter, and (\ref{trace of vn}).

\proofpart{4}{Proof of (\ref{mollifier quant3})}
Integrating by parts,
$$\left|\int_{tB_n} \dif (u_n - v_n) \wedge \omega_n\right| \leq ||\omega_n||_{L^\infty} ||u_n - v_n||_{L^1(\Gamma_n)} + ||\dif \omega_n||_{L^\infty} \int_0^t ||u_n - v_n||_{L^1(\partial(sB_n))} \dif s.$$
By (\ref{trace of vn}),
$$\limsup_{n \to \infty} ||\omega_n||_{C^1}^{-1} \gamma_n^{-1} \Delta_n^{1 - d} \left|\int_{tB_n} \dif(u_n - v_n) \wedge \omega_n\right| \leq \limsup_{n \to \infty} \int_0^t \gamma_n^{-1} \Delta_n^{1 - d} ||u_n - v_n||_{L^1(\partial(sB_n))} \dif s.$$
Moreover, (\ref{trace of vn}) holds with $t$ replaced by almost any $s$, so
$$f_n(s) := \gamma_n^{-1} \Delta_n^{1 - d} ||u_n - v_n||_{L^1(\partial(sB_n))}$$
satisfies $(f_n) \in \ell^\infty([0, 1] \to L^\infty)$, and $f_n \to 0$ almost everywhere.
So by Fatou's lemma,
\begin{align*}
0 \leq \limsup_{n \to \infty} ||\omega_n||_{C^1}^{-1} \gamma_n^{-1} \Delta_n^{1 - d} \left|\int_{tB_n} \dif(u_n - v_n) \wedge \omega_n\right| &\leq \int_0^t \lim_{n \to \infty} f_n(s) \dif s = 0. \qedhere
\end{align*}
\end{proof}

TODO: Compactness and contradiction


%%%%%%%%%%%%%%%%%%%%%%%%%%%%%%%%%%%%%%%%%%%%%%
\section{Plateau's equation}\label{Plateau section}
Having shown that we can reduce the study of sets of least perimeter to sets with $C^1$ and approximately least perimeter, our next task is to represent such sets as graphs of approximate solutions to a Plateau-type equation.
For hyperbolic manifolds, Plateau's equation will be posed on $\Hyp^{d - 1}$, where we adopt the convention that $\Hyp^m = \RR^m$ for $m = 0, 1$.


\subsection{Plateau energy}
We construct Plateau's equation as the Euler-Lagrange equation of a certain Lagrangian on $\Hyp^{d - 1}$.
To accomplish this we first identify a coordinate condition which makes Plateau's equation take a particularly simple form.

\begin{definition}
We call a vector field $X$ on a hyperbolic manifold $M$ an \dfn{infinitesimal generator of translations} if there are coordinates $(x^\mu)$ such that $X = \partial_{x^0}$ and 
\begin{equation}\label{half space form}
g = \frac{(\dif x^0)^2 + \cdots + (\dif x^{d - 1})^2}{(x^1)^2}.
\end{equation}
\end{definition}

\begin{definition}
Let $M = (M, g)$ be a hyperbolic manifold. We say that a coordinate system $(x^\mu)$ is a \dfn{Killing gauge} if
$$g = g_{00} \dif x^0 \otimes \dif x^0 + g_{ij} \dif x^i \otimes \dif x^j$$
for $i, j = 1, \dots, d - 1$, and $\partial_0$ is an infinitesimal generator of translations.
If $(x^\mu)$ is understood, then we just say that $g$ is \dfn{in Killing gauge}.
\end{definition}

\begin{lemma}
If $(x^\mu)$ is a Killing gauge, then $\partial_0 g_{\mu\nu} = 0$ and $\{x^0 = c\}$ is either hyperbolic or $1$-dimensional.
\end{lemma}
\begin{proof}
The equation $\partial_0 g_{\mu\nu} = 0$ is equivalent to $\partial_0$ being a Killing field. This follows from the fact that $(\partial_0)^\flat_\mu = g_{\mu 0}$ and the Killing equation
$$\partial_\mu g_{\nu 0} + \partial_\nu g_{\mu 0} - 2\Gamma_{\mu \nu}^\lambda g_{\lambda 0} = 0,$$
where we can expand the Christoffel symbols as
$$2 \Gamma^\lambda_{\mu \nu} g_{\lambda 0} = \partial_\nu g_{0 \mu} + \partial_\mu g_{0 \nu} - \partial_0 g_{\mu \nu}.$$
Since $\partial_0$ is an infinitesimal generator of translations, it is a Killing field and we can replace $x^i$ with coordinates for which the metric takes the form (\ref{half space form}), implying that $\{x^0 = c\}$ has the induced metric
$$(g_{ij}) = \frac{(\dif x^1)^2 + \cdots + (\dif x^{d -1})^2}{(x^1)^2},$$
which is either hyperbolic or $1$-dimensional.
\end{proof}

Suppose that $(x^\mu)$ is a Killing gauge, $\Omega \subseteq \Hyp^{d - 1}$ is open, and we have an identification
$$\Omega \cong \{x^0 = 0\};$$
we will not explicitly write pullbacks or pushforwards for this identification.
Suppose further that
$$\omega: \Omega \to \RR$$
is a $C^1$ function. Then one has a graph
$$N = \{(\omega(x), x) \in M: x \in \Omega\}.$$
Let $\Psi: \Omega \to N$ be the diffeomorphism $\Psi(x) = (\omega(x), x)$.

\begin{lemma}
The only independent component of the volume form of the pullback metric $\Psi^* g$ is
$$\sqrt{\det \Psi^* g} = \sqrt{\det ((g_{ij}))} \sqrt{1 + g_{00} |\dif \omega|_{g^{-1}}^2}$$
\end{lemma}
\begin{proof}
One has
$$\dif \Psi = \begin{bmatrix}
\dif \omega \\
I
\end{bmatrix}$$
so
\begin{align*}
(\Psi^* g)_{ij} &= g(\dif \Psi \cdot \partial_i, \dif \Psi \cdot \partial_j) = g_{\mu\nu} \dif \Psi^\mu_i \dif \Psi^\nu_j \\
&= g_{\mu\nu} (\delta^\mu_0 \partial_i \omega + \delta^\mu_i)(\delta^\nu_0 \partial_j \omega + \delta^\mu_j)\\
&= \Psi^* (g_{ij}) + \Psi^* (g_{00}) \partial_i \omega \partial_j \omega.
\end{align*}
Here $(\Psi^* g)_{ij}$ are components of the pullback metric but $\Psi^* (g_{\mu\nu})$ are pullbacks of the components of the metric.
Using the Killing nature of $\partial_0$ we can eliminate the pullbacks on the components, thus
$$\Psi^* g = g_{ij} \dif x^i \otimes \dif x^j + g_{00} \dif \omega \otimes \dif \omega.$$
TODO,
by the Weinstein-Aronsazjn theorem \cite{Tao13}.
\end{proof}

\begin{definition}
Suppose that $(M, g)$ is hyperbolic and $(x^\mu)$ is a Killing gauge.
Let $\Omega \subseteq \Hyp^{d - 1}$ be identified with $\{x^0 = c\}$ and let $\omega: \Omega \to \RR$ be $C^1$.
Then the \dfn{Plateau energy} of $\omega$ is the $d-1$-form
$$\Lagrange[\dif \omega] = \star \sqrt{1 + F |\dif \omega|^2},$$
on $\Omega$ and $F$ is the pullback of $g_{00}$ on $\{x^0 = 0\} \subset M$ to $\Omega$.
The \dfn{Plateau equation} is the Euler-Lagrange equation for the Plateau energy, and a variational solution of the Plateau equation is called a \dfn{minimal graph}.
\end{definition}

So if $N$ is the graph of $\omega$, then $\Lagrange[\dif \omega]$ is the pullback of the volume form on $N$ by $\Psi$.
In particular, minimal graphs have no mean curvature.

\begin{lemma}
The Plateau equation is the elliptic PDE on $\Hyp^{d - 1}$,
\begin{equation}\label{Plateau eqn}
\frac{F}{\sqrt{1 + F |\nabla \omega|^2}} \Delta \omega + g\left(\nabla \frac{F}{\sqrt{1 + F |\nabla \omega|^2}}, \nabla \omega\right) = 0.
\end{equation}
\end{lemma}
\begin{proof}
Consider a variation $\omega$ with $\phi = -\dot \omega$ compactly supported in $\Hyp^{d - 1}$:
\begin{align*}
\frac{\dif}{\dif t} \Lagrange[\dif \omega] &= \int_{\Hyp^{d - 1}} \star \partial_t \sqrt{1 + F |\nabla \omega|^2} \\
&= -\int_{\Hyp^{d - 1}} \star \frac{F g(\nabla \phi, \nabla \omega)}{\sqrt{1 + F|\nabla \omega|^2}} \\
&= \int_{\Hyp^{d - 1}} \star \phi\left[\frac{F}{\sqrt{1 + F |\nabla \omega|^2}} \Delta \omega + g\left(\nabla \frac{F}{\sqrt{1 + F |\nabla \omega|^2}}, \nabla \omega\right)\right]. \qedhere
\end{align*}
\end{proof}

Of course, the euclidean Plateau equation is (\ref{Plateau eqn}) with $F = 1$ and $g$, $|\cdot|$ replaced by the euclidean metric.

%%%%%%%%%%%%%%%%%%%%%%%%%%%%%%%%%
\subsection{de Giorgi lemma for the Plateau equation}
This section is dedicated to the proof of a de Giorgi-type lemma for the Plateau equation, generalizing \cite[Teorema 4.3]{Miranda66}.
More precisely we have:

\begin{proposition}\label{dGL Laplace}
Fix $\beta, \varepsilon > 0$.
Let $w \in C^1(\Omega)$ where $\Omega \subseteq \Hyp^{d - 1}$ is open, suppose that $||\dif w||_{L^\infty} \leq \varepsilon$,
\begin{equation}\label{dGL Laplace 1}
\int_{B_\rho} \Lagrange(\dif w) - \Lagrange(\avg_\rho \dif w) \leq \beta,
\end{equation}
and let $\tilde w$ be the solution of the Plateau equation (\ref{Plateau eqn}) on $B_\rho$ with the same trace on $\partial B_\rho$ as $\omega$.
If
\begin{equation}\label{dGL Laplace 2}
\int_{B_\rho} \Lagrange(\dif w) - \Lagrange(\dif \tilde w) \leq \beta\varepsilon,
\end{equation}
then one has
\begin{equation}\label{dGL Laplace concl}
\int_{B_{\rho/2}} \Lagrange(\dif w) - \Lagrange(\avg_{\rho/2} \dif w) \leq \frac{\beta}{2^{d + 1}} + O(\beta + \rho^d)\varepsilon.
\end{equation}
\end{proposition}

Before we prove Proposition \ref{dGL Laplace} we should remark on the role of the various parameters that appear in it.
In fact, once a set of least perimeter has been fixed, then $\beta,\varepsilon$ should be viewed as functions of $\rho$, and we shall later show that $\varepsilon \lesssim \rho^\delta$ for some small $\delta > 0$.
In our application, we shall require $\beta \ll \rho^d$, and this is ensured by the following elementary lemma:

\begin{lemma}[induction on scale]\label{effectiveness of dGL Laplace}
Let $\beta, \varepsilon > 0$ be functions of $\rho > 0$, such that $\beta \lesssim 1$ and for some $0 < \delta \leq 1$,
$$\beta(\rho) \leq \frac{\beta(2\rho)}{2^{d + 1}} + O(\rho^{d + \delta}).$$
Then for every $0 < \delta' < \delta$, there exists $\rho_0 = \rho_0(\delta') > 0$ so that for $\rho < \rho_0$,
$$\beta(\rho) \lesssim_{\delta'} \rho^{d + \delta'}.$$
\end{lemma}
\begin{proof}
Let $A, B > 0$. By hypothesis, we may choose $B$ so that for every $\rho > 0$,
$$\beta(\rho) \leq \frac{\beta(2\rho)}{2^{d + 1}} + \rho^{d + \delta}B.$$
We now choose $A$ so that we have the inductive hypothesis
$$\beta(2\rho) \leq 2^{d + \delta'} \rho^{d + \delta'} A.$$
Then
$$\beta(\rho) \leq \frac{A}{2^{1 - \delta'}} \rho^{d + \delta'} + \rho^{d + \delta}B.$$
If we choose $\rho_0$ small enough depending on $\delta'$ and $B$, then
$$\rho^{d + \delta}B < A\left(1 - \frac{1}{2^{1 - \delta'}}\right) \rho^{\delta'}$$
since $\delta' < \delta$.
Therefore $\beta(\rho) \leq A\rho^{d + \delta'}$.
\end{proof}

In order to prove Proposition \ref{dGL Laplace}, we replace $\avg$ with a coordinate-dependent operator $\widetilde \avg$.
More precisely, we fix normal coordinates $(x^i)$ on $\Hyp^{d - 1}$ and set for a $1$-form $\xi$,
$$(\widetilde \avg_r \xi)_i = \dashint_{B_r} \xi_i(x) \dif x.$$
Thus $\widetilde \avg_r \xi$ is a $1$-form on $B_r$, unlike $\avg_r \xi$ which is a section of $\Hyp^{d - 1} \times \RR^{1, d - 1}$.
We shall show that $\avg_r \xi$ is apparoximated by $\widetilde \avg_r \xi$ in the limit $r \to 0$.

\begin{lemma}\label{mapping properties of averages}
Let $P = \Pi \circ \avg_r$ or $P = \widetilde \avg_r$.
Then $P$ is a bounded linear operator
\begin{equation}\label{mapping properties bound}
P: C^\infty(M, T'M) \to C^\infty(M, T'M)
\end{equation}
and there exists a connection $\nabla_P$ which annihilates the image of $P$.
\end{lemma}
\begin{proof}
Clearly $P$ is bounded
$$P: C^\infty(M, T'M) \to C^0(M, T'M)$$
and if $\nabla_P$ exists, then this bound implies the bound (\ref{mapping properties bound}) (since for $\xi$ in the image of $P$,
$$||\xi||_{C^{r + 1}} \leq \Japan{\Gamma_r} ||\xi||_{C^r}$$
where $\Gamma_r$ is an expression involving the Christoffel symbols of $\nabla_P$; so we may induct on $r$).
We claim that for $P = \Pi \circ \avg_r$ we can take
$$\nabla_P = \nabla + \Pi [\Pi, \Dif],$$
where $\nabla$ is the Levi-Civita connection on $\Hyp^{d - 1}$, and for $P = \widetilde \avg_r$, we can take $\nabla_P$ to be the pullback of the flat connection on $\RR^{d - 1}$ along the normal coordinates used to define $\widetilde \avg_r$.
The latter is clear.
For the former, we recall that $\nabla = \Pi \Dif \Pi$ and so
$$\nabla_P P = \Pi \Dif \Pi \Pi \avg_r + \Pi [\Pi, \Dif] \Pi \avg_r = \Pi \Dif \avg_r = 0$$
where $\Dif \avg_r = 0$ by definition of $\avg_r$.
\end{proof}

\begin{lemma}\label{comparing averages}
For $w \in C^1(\Omega)$ with $||\dif w||_{L^\infty} \leq \varepsilon$,
$$\left|\Lagrange(\Pi \circ \avg_r \dif w) - \Lagrange(\widetilde \avg_r \dif w)\right| \lesssim r\varepsilon^2.$$
\end{lemma}
\begin{proof}
From Lemma \ref{mapping properties of averages}, $\Pi \circ \avg_r \dif$ and $\widetilde \avg_r \dif$ are $C^\infty$ one-parameter families of operators on $C^\infty(M, T'M)$, and
$$\lim_{r \to 0} \delta_0 \Pi \circ \avg_r \dif = -\dif \delta_0 = \lim_{r \to 0} \delta_0 \widetilde \avg_r \dif$$
where $\dif \delta_0$ is meant in the distributional sense.
Since the image of such operators is given by their behavior at $0$ it follows that
\begin{align*}
|\Lagrange(\Pi \avg_r \dif w) - \Lagrange(\widetilde \avg_r \dif w)|
&\lesssim |\avg_r \dif w|^2 - |\widetilde \avg_r \dif w|^2 + |(1 - \Pi) \avg_r \dif w|^2
\end{align*}
TODO... this seems sus
\end{proof}

TODO....

%%%%%%%%%%%%%%%%%%%%%%%%%%%%%%%%%%%%%%%%%%%%%%%%

\subsection{From sets of least perimeter to minimal graphs}
We now discuss how to pass from sets of least perimeter to functions which meet the bootstrap assumptions of Proposition \ref{dGL Laplace}.
To do so, we will need \cite[Proposition 4.8]{Giusti77}, which can be stated on a manifold as follows:

\begin{lemma}\label{sets are graphs}
Let $U$ be a set of locally finite perimeter in a $d$-fold $M$, and consider an open embedding
\begin{equation}\label{coordinates for making a graph}
(\Phi, \Psi): \tilde \Omega \times I \to M
\end{equation}
where $\tilde \Omega \subset \RR^{d - 1}$ and $I \subset \RR$ are open convex bounded sets.
Let $y$ be the coordinate function given by $\Psi$.
If $(\normal_U, \partial_y) \geq 1/2$ on $|\dif 1_U|$-almost all of $\partial^* U$, then there exists an open set $\Omega \subseteq \tilde \Omega$ and a function $\omega \in W^{1, \infty}(\Omega, I)$ such that 
$$\partial U \cap (\Phi, \Psi)(\tilde \Omega \times I) = \{(\Phi(x), \Psi(\omega(x))): x \in \Omega\}$$
and satisfying the estimate 
\begin{equation}\label{estimate on dif omega}
||\dif \omega||_{L^\infty} \lesssim ||\sqrt{1 - (\normal_U, \partial_y)}||_{L^\infty}.
\end{equation}
In particular, if $\partial U$ is $C^1$, then $\omega \in C^1(\Omega)$.
\end{lemma}

\begin{lemma}[elliptic bootstrapping]\label{C1 implies smooth}
Let $U$ be a set of least perimeter in a hyperbolic manifold, and assume that $\normal_U$ extends to a continuous $1$-form on $\partial U$.
Then $\partial U$ is a $C^\infty$ minimal hypersurface.
\end{lemma}
\begin{proof}
By Proposition \ref{locality of Caccioppoli}, $\partial U$ is $C^1$ and has no mean curvature.
Let $P \in \partial U$, and choose a Killing gauge based at $P$ with $\partial_0(P) = \normal_U(P)^\sharp$.
Then in a small neighborhood of $P$, $\partial U$ is a $C^1$ minimal graph by Lemma \ref{sets are graphs}.
So, by bootstrapping (\ref{Plateau eqn}), $\partial U$ is $C^\infty$ near $P$.
\end{proof}

\begin{definition}
Fix an infinitesimal generator of translations $X$ on a hyperbolic manifold $M$, with flow $\varphi$.
Let $B$ be a ball in a hypersurface perpendicular to $X$, and let $I \subset \RR$ be an interval centered on $0$.
Then the \dfn{translation cylinder} is the set
$$\Cyl_X(B, I) := \{\varphi_y(x): (x, y) \in B \times I\}.$$
If $\Gamma$ is a translation cylinder, we denote by $\Gamma/2$ the translation cylinder obtained by shrinking $B$ and $I$ by a factor of $2$.\
We let the \dfn{diameter} of a translation cylinder $\Gamma$ be 
$$\diam \Gamma := \sup_{x, y \in \Gamma} \dist(x, y).$$
\end{definition}

\begin{lemma}\label{sets are graphs 2}
Let $X$ be an infinitesimal generator of translations on a hyperbolic manifold, with flow $\varphi$.
Let $U$ be a set of locally finite perimeter in a translation cylinder $\Cyl_X(B, I)$, and suppose that:
\begin{enumerate}
\item $(\normal_U, X) \geq 1/2$ everywhere, and 
\item $U$ contains either the top or bottom of $\Cyl_X(B, I)$.
\end{enumerate}
Then there exists $\omega \in W^{1, \infty}(B, I)$ such that
\begin{align}
\partial U &= \{\varphi_{\omega(x)}(x): x \in B\} \\
||\dif \omega||_{L^\infty} &\lesssim ||\sqrt{1 - (\normal_U, X)}||_{L^\infty}.\label{estimate on dif omega 2}
\end{align}
In particular, if $\partial U$ is $C^1$, then $\omega \in C^1(\Omega)$.
\end{lemma}
\begin{proof}
This lemma is immediate from Lemma \ref{sets are graphs} where $X = \partial_y$, and (\ref{estimate on dif omega 2}) follows from (\ref{estimate on dif omega}) and the fact that we can put normal coordinates on $B$, thus whether we define $||\dif \omega||_{L^\infty}$ using the hyperbolic or euclidean metrics on $B$ will not affect its behavior up to a constant factor.
\end{proof}




%%%%%%%%%%%%%%%%%%%%%%%%%%%%%%%%%%%%%%%%%%%%%%%%%%%

\section{Regularity of minimal hypersurfaces}\label{de Giorgi section}
In this section we shall prove Theorem \ref{main lma}.
Actually, we shall not prove this theorem in such high generality: Miranda's arguments already work for the case that the sectional curvature $K = 0$, and if $K < 0$ then we may rescale and assume that $K = -1$.
Then, since the statement is local, we may assume that in fact $M$ is an open subset of $\Hyp^d$.
So, by Lemma \ref{C1 implies smooth}, it suffices to show the following proposition:

\begin{proposition}[Theorem \ref{main lma}, precise statement]
Let $U$ be a set of least perimeter in an open submanifold $M$ of $\Hyp^d$, and let $2 \leq d \leq 7$.
Then the conormal $1$-form $\normal_U$ to $\partial^* U$ extends to a continuous map $\partial U \to \RR^{1, d}$.
\end{proposition}

%%%%%%%%%%%%%%%%%%%%%%%%%%%%%%%%%%%%%%%%%%%

\subsection{The excess}
The key step, as in \cite{Miranda66}, is to prove a de Giorgi lemma for sets of least perimeter by exploiting the gains of Proposition \ref{dGL Laplace}.
The quantity of interest in the de Giorgi lemma is known as ``excess'', and is defined now.
We remind the reader of the definition (\ref{definition of vectorial integral}) of an integral of a tensor field.

\begin{definition}
The \dfn{excess} of a set $U$ of locally finite perimeter in a translation cylinder $\Gamma$, with $u = 1_U$ and $r = \diam \Gamma$, is
$$\Exc_\Gamma U := r^{1 - d}\left[e^{r^2}\int_\Gamma \star |\dif u| - \left|\int_\Gamma \dif u \otimes (\star 1)\right|\right].$$
\end{definition}

The notion of excess here is the one which appears in the work of Miranda \cite{Miranda66}, which is not equivalent to the notion of excess popularized by Allard \cite{Allard72} and which appears in the reference of Colding--Minicozzi \cite{colding2011course}.

It at first appears that the excess depends on the choice of identification (\ref{hyperboloid choice}) of an open subset of $M$ with a subset of $\RR^{1, d}$.
Such an identification unique up to the action of the isometry group $\SpOrth_{1, d}^+$.
Clearly $\int_B \star |\dif u|$ is left invariant by isometries.
Meanwhile, $\int_B \dif u \otimes (\star 1)$ is \emph{equivariant}: it is only defined up to the action of $\SpOrth_{1, d}^+$ on $\RR^{1, d}$.
However, elements of $\SpOrth_{1, d}^+$ are Lorentz transformations and so preserve the length $|\int_B \dif u \otimes (\star 1)|$.
Moreover, any $1$-form on $\Hyp^d$ induces a spacelike section of $\Hyp^d \times \RR^{1, d}$, and since the spacelike vectors form a cone, the integral of such a spacelike section is spacelike.
Therefore its length $|\int_B \dif u \otimes (\star 1)|$ is well-defined as a real number.
So, we have proven the below lemma:

\begin{lemma}
The excess of a set of locally finite perimeter is a well-defined real number which is invariant under isometries of $M$.
\end{lemma}

In the euclidean case, we do not have a prefactor of $e^{r^2}$ (see e.g. \cite[Chapter 6]{Giusti77}).
For us, the factor $e^{r^2}$ is necessary to make the excess a nonnegative, monotone quantity, because the triangle inequality is false for spacelike vectors in $\RR^{1, d}$.

\begin{lemma}
If $0 < \diam \Gamma \lesssim 1$, then for every set $U$ of least perimeter,
$$\Exc_\Gamma U \geq 0.$$
\end{lemma}
\begin{proof}
We let $P \in \Gamma$, apply an isometry so we may assume $P = O$, and let $r = \diam \Gamma$.
Viewing $\dif 1_U$ as a section of $M \times \RR^{1, d}$, we decompose it as
$$\dif 1_U = \xi + \tau$$
where $\xi$ lies in the spacelike hyperplane $\{t = 0\}$ and $\tau$ is a multiple of $\dif t$.
Then, since $\tau$ is timelike and $\{t = 0\}$ is spacelike,
$$\left|\int_{B(P, r)} \dif 1_U \otimes (\star 1)\right| \leq \left|\int_{B(P, r)} \xi \otimes (\star 1)\right| \leq \int_{B(P, r)} \star |\xi|.$$
Let $|\tau|$ denote $|\tilde \tau|$ where $\tau = \tilde \tau \dif t$.
We can estimate from above if $|\tau| < |\xi|/10$ that
$$\int_{B(P, r)} \star |\dif 1_U| = \int_{B(P, r)} \star |\xi| \sqrt{1 - \frac{|\tau|^2}{|\xi|^2}} \geq \frac{1}{2} \int_{B(P, r)} \star |\xi| - \int_{B(P, r)} \star \frac{|\tau|^2}{|\xi|}.$$
In other words, if $|\tau| < |\xi|/10$,
$$\left|\int_{B(P, r)} \dif 1_U \otimes (\star 1)\right| \leq \int_{B(P, r)} \star |\dif 1_U| + \frac{1}{2} \int_{B(P, r)} \star \frac{|\tau|^2}{|\xi|}.$$
From (\ref{bound the timelike part}),
$$\frac{|\tau|^2}{2|\xi|} \leq \frac{r^2}{2} |\xi|;$$
in particular $|\tau| < |\xi|/10$ if $r$ is small.
Moreover since $|\tau| < |\xi|/10$, (\ref{bound the timelike part}) gives $|\xi| \leq 2|\dif 1_U|$ and hence
\begin{align*}
\left|\int_{B(P, r)} \dif 1_U \otimes (\star 1)\right| &\leq r^2 \int_{B(P, r)} \star |\dif 1_U| \leq e^{r^2} \int_{B(P, r)} \star |\dif 1_U|. \qedhere
\end{align*}
\end{proof}


%%%%%%%%%%%%%%%%%%%%%%%%%%%%%%%%%%%%

\subsection{de Giorgi lemma}
We are finally ready to state the de Giorgi lemma:

\begin{proposition}[de Giorgi lemma]\label{dGL final}
Suppose that $2 \leq d \leq 7$ and $M$ is an open submanifold of $\Hyp^d$.
Then there exist dimensional constants $\sigma, R > 0$ such that for every $P \in M$ and $U$ a set of least perimeter in a neighborhood of $P$, if $r < R$ and $\Exc_B(P, r) < \sigma$, then
$$\Exc_{B(P, r/2)} U < \frac{1}{2} \Exc_{B(P, r)} U.$$
\end{proposition}

We begin its proof by proving the $C^1$ case:

\begin{lemma}\label{dGL C1}
Let $U$ be a set of $C^1$ perimeter in $B := B(P, \rho) \subseteq \Hyp^d$, $2 \leq d \leq 7$, and suppose that
\begin{align}
\Exc_B U &< \gamma, \label{dGL C1 hyp1} \\
|\partial U \cap B| &\leq \eta(U, B) + \rho^{d - 1}\gamma\delta. \label{dGL C1 hyp2}
\end{align}
If there exists an infinitesimal generator of translations $X$ such that 
\begin{equation}
(\normal_U, X) \geq 1 - \delta, \label{dGL C1 hyp3}
\end{equation}
then
\begin{equation}
\Exc_{B/2} U < \frac{\gamma}{2} + O(\gamma + \rho)\rho^{d - 1}\delta. \label{dGL C1 concl}
\end{equation}
\end{lemma}
\begin{proof}

\end{proof}



%%%%%%%%%%%%%%%%%%%%%%%%%%%%%%%%%%%%

\subsection{Induction on scale}


%%%%%%%%%%%%%%%%%%%%%%%%%%%%%%%%%%%%
\section{Proofs of main theorems}\label{GornySec}
Having completed the proof of Theorems \ref{main lma} and \ref{monotonicity prestate}, their consequences, Theorems \ref{main thm} and \ref{Gorny regularity}, fall out quite easily; we essentially can follow the arguments of G\'orny \cite{górny2017planar}.
However, we include the details for completeness.

\subsection{The maximum principle}
We now prove Theorem \ref{main thm}.

\proofpart{1}{Least gradient implies minimal lamination}
Let $u$ have least gradient, and observe that $\lambda = \supp \dif u$, so $\lambda$ is closed.
The sets $\{u > y\}$ are totally ordered by $\subseteq$, so the sets $\partial \{u > y\}$ are disjoint.
By Proposition \ref{level sets are minimal}, $\{u > y\}$ has least perimeter, so by Theorem \ref{main lma} it is a disjoint union of $C^\infty$ minimal hypersurfaces.

\proofpart{2}{Local finiteness}
We now claim that the decomposition of $\{u > y\}$ into minimal hypersurfaces is finite in any compact subset $K$ of $M$.
If not, then there exists $P \in K \cap \overline{\partial U}$ and a sequence of connected components $(D_n)$, such that there exists $P_n \in D_n \cap K$ with $P_n \to P$.
Since $\partial^* U$ is dense in $\partial U$ we can in fact take $P_n \in D_n^* := D_n \cap \partial^* U$.
Now we take $\varepsilon > 0$ so small that $e^{-A\varepsilon^2} \geq 1/2$, and choose $N$ so large that if $n \geq N$ then $B_n := B(P_n, \varepsilon/2)$ is contained in $B := B(P, \varepsilon)$.
Thus by Proposition \ref{doubling dimension},
$$|\partial U \cap B| \geq \sum_{n=N}^\infty |D_n^* \cap B_n| \geq \sum_{n=N}^\infty \frac{\varepsilon^{d - 1}}{2^d} |\Ball^{d - 1}| = \infty,$$
which violates that $U$ has locally finite perimeter. TODO: Make a picture of a contradictory set

\proofpart{3}{Surfaces with boundary}
Now suppose that $\overline M$ is a surface-with-boundary.
We must show that if $\gamma_1, \gamma_2$ are two distinct geodesics in the lamination $\lambda$, then their extensions $\overline \gamma_i$ to $\overline M$ do not intersect on $\partial M$.
If they do intersect at $P \in \partial M$, then $\gamma_1, \gamma_2$ bound the same superlevel set $\{u > y\}$.
Let $Q, R$ be points on $\gamma_1, \gamma_2$ respectively which are so close to $P$ that $P, Q, R$ bound a nondegenerate, contractible geodesic triangle $T$.
By local finiteness, we may assume that no geodesics in $A_y$ pass through the interior of $T$.
Since $\{u > y\}$ has least perimeter, $v = 1_{\{u > y\}}$ has least gradient, and without loss of generality we may assume that $v = 1$ inside $T$ and $v = 0$ on the opposite sides of $\gamma_1, \gamma_2$.
Moreover $\int_M \star |\dif v|$ is the sum of lengths of geodesics in $A_y$.
If we define $\tilde v = v$ away from the interior of $T$ and $\tilde v = 0$ on $T$, then $\int_M \star |\dif \tilde v|$ is the sum of lengths of geodesics in $A_y$, but with $\overline{PQ}$ and $\overline{PR}$ replaced by $\overline{QR}$.
So by the triangle inequality and the fact that $T$ is nondegenerate,
$$\int_M \star |\dif \tilde v| < \int_M \star |\dif v|,$$
yet by construction $\tilde v$ and $v$ have the same trace (which is well-defined by Proposition \ref{traces}).
This is a contradiction since $v$ has least gradient. TODO: Make a picture of the geodesic triangle

\proofpart{4}{Minimal lamination implies least gradient}
Now suppose that $\lambda$ is a minimal lamination, so every set $A_y$ is minimal.
Fix $U \Subset M$ with Lipschitz boundary, let $T: BV(U) \to L^1(\partial U)$ be the trace map, and let $v$ be a competitor in $U$, thus $v \in BV(U)$ and $Tu = Tv$, so in particular for every $y \in \RR$, $\{Tu > y\} = \{Tv > y\}$.
But for every $w \in BV(U)$, $T(1_{\{w > y\}})$ indicates the set of $x \in \partial U$ such that $w > y$ in a neighborhood of $x$.
By (\ref{convergence of trace}), this is the set $\{Tw > y\}$.
Therefore $T(1_{\{u > y\}}) = T(1_{\{v > y\}})$, so $\partial^* \{v > y\}$ is a competitor to $A_y$ in $U$.
Since $A_y$ is minimal,
\begin{equation}\label{laminationwise least gradient}
|A_y \cap U| \leq |\partial^* \{v > y\} \cap U|.
\end{equation}
We now integrate both sides of (\ref{laminationwise least gradient}) against $\dif y$ and apply Proposition \ref{coarea formula} to see that
$$\int \star |\dif u| \leq \int \star |\dif v|,$$
implying that $u$ has least gradient.
TODO: Check this very carefully and draw a picture.

\subsection{Proof of Radon-Nikod\'ym decomposition}
We finally prove Theorem \ref{Gorny regularity}.
TODO


\printbibliography

\end{document}
