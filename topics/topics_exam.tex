\documentclass[10pt]{beamer}

\usetheme[progressbar=frametitle, block=fill]{metropolis}
\RequirePackage{amsmath,amssymb,amsthm,graphicx,mathrsfs,url,slashed,subcaption}
\usepackage{appendixnumberbeamer}

\usepackage{booktabs}
\usepackage[scale=2]{ccicons}

\usepackage{pgfplots}
\usepgfplotslibrary{dateplot}

\newcommand{\NN}{\mathbf{N}}
\newcommand{\ZZ}{\mathbf{Z}}
\newcommand{\QQ}{\mathbf{Q}}
\newcommand{\RR}{\mathbf{R}}
\newcommand{\CC}{\mathbf{C}}
\newcommand{\DD}{\mathbf{D}}
\newcommand{\PP}{\mathbf P}
\newcommand{\MM}{\mathbf M}
\newcommand{\II}{\mathbf I}
\newcommand{\Hyp}{\mathbf H}
\newcommand{\Sph}{\mathbf S}
\newcommand{\Group}{\mathbf G}
\newcommand{\GL}{\mathbf{GL}}
\newcommand{\Orth}{\mathbf{O}}
\newcommand{\SpOrth}{\mathbf{SO}}
\newcommand{\Ball}{\mathbf{B}}

\newcommand*\dif{\mathop{}\!\mathrm{d}}
\newcommand{\dfn}[1]{\emph{#1}\index{#1}}

\newcommand{\normal}{\mathbf n}

\newcommand{\loc}{\mathrm{loc}}
\newcommand{\cpt}{\mathrm{cpt}}

\newtheorem{proposition}{Proposition}
\newtheorem{question}{Question}


\usepackage{xspace}
\newcommand{\themename}{\textbf{\textsc{metropolis}}\xspace}

\title{Functions of least gradient, minimal laminations, and averaging of differential forms}
% \subtitle{A modern beamer theme}
% \date{\today}
\date{October 2022}
\author{Aidan Backus}
\institute{Brown University}
% \titlegraphic{\hfill\includegraphics[height=1.5cm]{logo.pdf}}

\begin{document}

\maketitle

\begin{frame}{Table of contents}
  \setbeamertemplate{section in toc}[sections numbered]
  \tableofcontents%[hideallsubsections]
\end{frame}

\section{The one-Laplacian}

\begin{frame}{The one-Laplacian}
We shall study the one-Laplace equation
$$\Delta_1 u := \dif^* \frac{\dif u}{|\dif u|} = 0.$$
Solutions are called \dfn{one-harmonic} or are said to have \dfn{least gradient}.

The good news:
\begin{itemize}
    \item $\Delta_1$ is in divergence form.
    \item If $y$ is a regular value of $u$, and $u(x) = y$, then $\Delta_1 u(x)$ is the mean curvature of $\{u = y\}$ at $x$, so $\{u = y\}$ is a minimal surface.
\end{itemize}

The bad news:
\begin{itemize}
    \item $\Delta_1 u$ is not elliptic near $\{\dif u = 0\}$.
    \item The weak formulation $$\int_M \left(\frac{\dif u}{|\dif u|}, \nabla \varphi\right) \dif V = 0$$ makes no sense if $\{\dif u = 0\}$ has positive measure.
\end{itemize}
\end{frame}

\begin{frame}{Weak and variational formulations}

\begin{definition}[de L\'eon, Maz\'on, Rosser '14]
A scalar field $u \in BV_\loc(M)$ is a \dfn{weak solution} of $\Delta_1 u = 0$ if there exists a vector field $X \in L^\infty(M, |\dif u|\dif V)$ such that as currents, $|\dif u| = (\dif u, X)$,
and for every $\varphi \in H^1_\cpt(M)$,
$$\int_M (\dif \varphi, X) \dif V = 0.$$
\end{definition}

\begin{theorem}[de L\'eon, Maz\'on, Rosser '14]
A scalar field $u \in BV(M)$ is a weak solution of $\Delta_1 u = 0$ iff $u$ minimizes 
$$\int_M |\dif u| \dif V := \sup_{X \in C^0_\cpt} \frac{1}{||X||_{C^0}} \int_M (\dif u, X) \dif V$$
among all scalar fields with the same trace.
\end{theorem}
\end{frame}

\begin{frame}{Minimal surfaces}
Since the natural space for solving $\Delta_1 u = 0$ is $BV$, solutions may have jumps, and in fact could even be indicator functions:

\begin{definition}
A set $U$ has \dfn{locally finite perimeter} if $1_U \in BV_\loc$.
It has \dfn{least perimeter} if, in addition, $\Delta_1 1_U = 0$.
\end{definition}

\begin{theorem}[de Giorgi--Miranda regularity theorem, '60s]
If $M$ is an open subset of $\RR^d$ and $d \leq 7$, then for every set $U$ of least perimeter in $M$, $\partial U$ is an (analytic, embedded, stable) minimal hypersurface.
\end{theorem}

This is false for $d = 8$ because $\{(x, y) \in (\RR^4)^2: |x|^2 = |y|^2\}$ is area-minimizing and singular. \textbf{Henceforth we assume $d \leq 7$ without explicitly stating it.}
    
\end{frame}

\begin{frame}{Minimal surfaces}
\begin{theorem}[de Giorgi--Miranda regularity theorem, '60s]
If $M$ is an open subset of $\RR^d$ and $d \leq 7$, then for every set $U$ of least perimeter in $M$, $\partial U$ is an (analytic, embedded, stable) minimal hypersurface.
\end{theorem}

\begin{question}
Let $M = M^d$ be a Riemannian manifold.
Under what hypotheses on $M$ does the de Giorgi--Miranda regularity theorem hold?
\end{question}

\end{frame}

\begin{frame}{Geodesic laminations}
Consider the infinity-Laplace equation
$$\Delta_\infty v := \nabla^\mu \partial^\nu v \cdot \partial_\mu v \cdot \partial_\nu v = 0.$$
Solutions are minimizers of $||\dif v||_{L^\infty}$.

Daskalopolous and Uhlenbeck constructed a \dfn{conjugate one-harmonic} function $u$ to an $\infty$-harmonic function.

\begin{definition}
A \dfn{geodesic lamination} is a closed subset of $M$ equipped with a Lipschitz local product structure $K \times N$ where $K \subset \RR$ is compact and the fibers $\{k\} \times N$ are complete geodesics.
\end{definition}

\begin{theorem}[Daskalopolous, Uhlenbeck '20]
Let $M$ be a closed Riemann surface of genus $\geq 2$, $\Delta_\infty v = 0$, and $u$ the conjugate $1$-harmonic function to $v$.
Then the support $\lambda$ of $\dif u$ is a geodesic lamination, and the geodesics are level sets of $u$.
Moreover, $|\dif v| = ||\dif v||_{L^\infty}$ on $\lambda$.
\end{theorem}

\end{frame}

\begin{frame}{Geodesic laminations}
\begin{theorem}[Daskalopolous, Uhlenbeck '20]
Let $M$ be a closed Riemann surface of genus $\geq 2$, $\Delta_\infty v = 0$, and $u$ the conjugate $1$-harmonic function to $v$.
Then the support $\lambda$ of $\dif u$ is a geodesic lamination, and the geodesics are level sets of $u$.
Moreover, $|\dif v| = ||\dif v||_{L^\infty}$ on $\lambda$.
\end{theorem}

\begin{question}
On a more general Riemannian manifold $M$, what is the topological structure of the support $\lambda$ of $\dif u$, where $u$ is $1$-harmonic?
What is the analogue of the conjugate $\infty$-harmonic $v$, and how does it interact with $\lambda$?
\end{question}

\end{frame}

\begin{frame}{Computational geometry}
Currently algorithms for computing minimal surfaces proceed by running a parabolic PDE solver on the mean curvature flow and taking $t \to \infty$.
Can we do better?

\begin{theorem}[Loisel '20]
Let $\mathcal T$ be a quasiuniform triangulation of $M$ with $n$ simplices.
Using a convex optimization method, one can minimize $\int_M |\dif u| \dif V$ among all functions of the same trace in $PL(\mathcal T)$, in $O(n^{1/2} \log n)$ Newton iterations.
\end{theorem}

\begin{question}
In what sense does the minimizer in $PL(\mathcal T)$ converge to a $1$-harmonic function as we refine $\mathcal T$?
Does this give an algorithm for computing minimal surfaces, geodesic laminations, and similar objects, from their Dirichlet data, which improves over mean curvature flow methods?
\end{question}
\end{frame}

\section{The regularity theorem}
\begin{frame}{The regularity theorem}
\begin{theorem}[--, '22]
If $M$ is a \textbf{manifold of constant sectional curvature} and $d \leq 7$, then for every set $U$ of least perimeter in $M$, $\partial U$ is an (analytic, embedded, stable) minimal hypersurface.
\end{theorem}

To set up the proof and see what's different from Miranda's case ($M \subseteq \RR^d$), we need to scrutinize the Lebesgue differentiation theorem to see how to construct the conormal $1$-form to $\partial U$.
\end{frame}

\begin{frame}{Lebesgue differentiation theorem}
We need to take averages of differential forms:
\begin{itemize}
    \item If $f \in L^1_\loc(\RR^d)$ and $\mu$ is Radon, then for $\mu$-almost every $x$,
$$f(x) := \lim_{\varepsilon \to 0} \frac{1}{\mu(B(x, \varepsilon))} \int_{B(x, \varepsilon)} f \dif \mu$$
is well-defined.
    \item If $f = \dif u/|\dif u|$ where $u = 1_U$, $U$ locally finite perimeter, one hopes to define the conormal $1$-form to $\partial U$, $|\dif u|\dif V$-almost everywhere.
    \item If $M$ is not flat, then $f(x), f(y)$ live in \textbf{different vector spaces} for $x \neq y$! One has to choose a trivialization of the cotangent bundle to take the average of $f$.
\end{itemize}
\end{frame}

\begin{frame}{Lebesgue differentiation theorem}
\begin{proposition}
Let $\omega$ be a distribution of locally finite total variation, and $f: M \to T'M$ a $1$-form.
Then there exists an $\omega$-null set $Z$ such that for every Riemannian metric on $M$, every frame $(\partial_\mu)$ and dual coframe $(\dif x^\mu)$, and every $p \notin Z$,
$$f(p) := \lim_{\varepsilon \to 0} \frac{\int_{B(p, \varepsilon)} (f, \partial_\mu) |\omega| \dif V}{\int_{B(p, \varepsilon)} |\omega| \dif V} \dif x^\mu(p)$$
is well-defined.
\end{proposition}
    
The content here is that $Z$ is independent of the choice of Riemannian metric, frame, or coframe.
In other words, \textbf{the notion of Lebesgue point of a differential form is well-defined}.

\end{frame}

\begin{frame}{de Giorgi's reduced boundary}
\begin{definition}
Let $U$ be a set of locally finite perimeter, $u := 1_U$, $\normal := \dif u/|\dif u|$, and $\partial^* U$ the set of $|\dif u|$-Lebesgue points of $\normal$.
Then we call $\normal$ the \dfn{conormal one-form} to the \dfn{reduced boundary} $\partial^* U$.
\end{definition}

It is now easy to show (c.f. Giusti, Chapters 3 and 4):
\begin{itemize}
    \item $|\dif u|\dif V$ is the codimension-$1$ Hausdorff measure $\dif S$ on the measure-theoretic boundary $\partial U$.
    \item $\partial^* U$ is dense and $\dif S$-full measure in $\partial U$.
    \item If $\normal$ extends to a continuous $1$-form along $\partial U$, then $\partial U = \partial^* U$ is an embedded $C^1$ hypersurface.
    \item Elliptic bootstrapping: If $\partial U$ is $C^1$ and $U$ has least perimeter, then $\partial U$ is an analytic stable minimal hypersurface.
\end{itemize}

The upshot is that we just need to show that $\normal$ is continuous.
\end{frame}

\begin{frame}{Minkowski spacetime}
    $so(2, 1)$
\end{frame}

\end{document}
