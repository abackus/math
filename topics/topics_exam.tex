\documentclass[10pt]{beamer}

\usetheme[progressbar=frametitle, block=fill]{metropolis}
\RequirePackage{amsmath,amssymb,amsthm,graphicx,mathrsfs,url,slashed,subcaption}
\usepackage{appendixnumberbeamer}

\usepackage{booktabs}
\usepackage[scale=2]{ccicons}

\usepackage{pgfplots}
\usepgfplotslibrary{dateplot}

\newcommand{\NN}{\mathbf{N}}
\newcommand{\ZZ}{\mathbf{Z}}
\newcommand{\QQ}{\mathbf{Q}}
\newcommand{\RR}{\mathbf{R}}
\newcommand{\CC}{\mathbf{C}}
\newcommand{\DD}{\mathbf{D}}
\newcommand{\PP}{\mathbf P}
\newcommand{\MM}{\mathbf M}
\newcommand{\II}{\mathbf I}
\newcommand{\Hyp}{\mathbf H}
\newcommand{\Sph}{\mathbf S}
\newcommand{\Group}{\mathbf G}
\newcommand{\GL}{\mathbf{GL}}
\newcommand{\Orth}{\mathbf{O}}
\newcommand{\SpOrth}{\mathbf{SO}}
\newcommand{\Ball}{\mathbf{B}}

\newcommand*\dif{\mathop{}\!\mathrm{d}}
\newcommand{\dfn}[1]{\emph{#1}\index{#1}}

\newcommand{\avg}{\mathrm{avg}}
\newcommand{\normal}{\mathbf n}

\newcommand{\loc}{\mathrm{loc}}
\newcommand{\cpt}{\mathrm{cpt}}

\newtheorem{proposition}{Proposition}
\newtheorem{question}{Question}


\usepackage{xspace}
\newcommand{\themename}{\textbf{\textsc{metropolis}}\xspace}

\title{Functions of least gradient, minimal laminations, and averaging of differential forms}
% \subtitle{A modern beamer theme}
% \date{\today}
\date{October 2022}
\author{Aidan Backus}
\institute{Brown University}
% \titlegraphic{\hfill\includegraphics[height=1.5cm]{logo.pdf}}

\begin{document}

\maketitle

\begin{frame}{Table of contents}
  \setbeamertemplate{section in toc}[sections numbered]
  \tableofcontents%[hideallsubsections]
\end{frame}

\section{The one-Laplacian}

\begin{frame}{The one-Laplacian}
We shall study the one-Laplace equation
$$\Delta_1 u := \dif^* \frac{\dif u}{|\dif u|} = 0.$$
Solutions are called \dfn{one-harmonic}.

\pause

The good news:
\begin{itemize}
    \item $\Delta_1$ is in divergence form.
    \item If $y$ is a regular value of $u$, and $u(x) = y$, then $\Delta_1 u(x)$ is the mean curvature of $\{u = y\}$ at $x$, so $\{u = y\}$ is a minimal surface.
\end{itemize}

\pause

The bad news:
\begin{itemize}
    \item $\Delta_1 u$ is not elliptic near $\{\dif u = 0\}$.
    \item The weak formulation $$\int_M \left(\frac{\dif u}{|\dif u|}, \nabla \varphi\right) \dif V = 0$$ makes no sense if $\{\dif u = 0\}$ has positive measure.
\end{itemize}
\end{frame}

\begin{frame}{Weak and variational formulations}

\begin{definition}[de L\'eon, Maz\'on, Rosser '14]
A scalar field $u \in BV_\loc(M)$ is a \dfn{weak solution} of $\Delta_1 u = 0$ if there exists a measurable vector field $X$ such that $|\dif u| = (\dif u, X)$,
and for every $\varphi \in H^1_\cpt(M)$,
$$\int_M (\dif \varphi, X) \dif V = 0.$$
\end{definition}

\pause

\begin{theorem}[de L\'eon, Maz\'on, Rosser '14]
A scalar field $u \in BV(M)$ is a weak solution of $\Delta_1 u = 0$ iff $u$ is a critical point of the Lagrangian
$$\int_M |\dif u| \dif V := \sup_{X \in C^0_\cpt} \frac{1}{||X||_{C^0}} \int_M (\dif u, X) \dif V.$$
\end{theorem}
\end{frame}

\begin{frame}{More on the variational formulation}
\begin{definition}
A function $u \in BV$ has \dfn{least gradient} if it is a minimizer of $\int |\dif u| \dif V$.  
\end{definition}

\pause

By convexity of the Lagrangian, a critical point is locally the same thing as a minimizer.

\pause

\begin{definition}
A set $U$ of locally finite perimeter has \dfn{least perimeter} if $1_U$ has least gradient.
\end{definition}
\end{frame}


\begin{frame}{Minimal surfaces}
\begin{proposition}[Bombieri, de Giorgi, and Giusti '69]
Let $u$ be a function of least gradient and $y \in \RR$.
Then the measure-theoretic level set $\partial \{u > y\}$ has least perimeter.
\end{proposition}

\pause

It follows from the coarea formula
$$\int_M |\dif u| \dif V = \int_{-\infty}^\infty |\partial \{u > y\}| \dif y.$$

\pause

\begin{theorem}[de Giorgi--Miranda regularity theorem, '60s]
If $M$ is an open subset of $\RR^d$ and $d \leq 7$, then for every set $U$ of least perimeter in $M$, $\partial U$ is an (analytic, embedded, stable) minimal hypersurface.
\end{theorem}

\pause 

This is false for $d = 8$ because $\{(x, y) \in (\RR^4)^2: |x|^2 < |y|^2\}$ has least perimeter (Bombieri, de Giorgi, Giusti '69). \pause \textbf{Henceforth we assume $d \leq 7$ without explicitly stating it.}

\end{frame}

\begin{frame}{Minimal surfaces}
\begin{theorem}[de Giorgi--Miranda regularity theorem, '60s]
If $M$ is an open subset of $\RR^d$ and $d \leq 7$, then for every set $U$ of least perimeter in $M$, $\partial U$ is an (analytic, embedded, stable) minimal hypersurface.
\end{theorem}

\pause

\begin{question}
Let $M = M^d$ be a Riemannian manifold.
Under what hypotheses on $M$ does the de Giorgi--Miranda regularity theorem hold?
\end{question}

\pause

I will partially answer this question later.
\end{frame}

\begin{frame}{Duality with the $\infty$-Laplacian}
For a second we consider the $\infty$-Laplace equation 
$$\Delta_\infty v := \nabla^\mu \partial^\nu v \cdot \partial_\mu v \cdot \partial_\nu v = 0.$$
Solutions are minimizers of $||\dif v||_{L^\infty}$, so are called \dfn{best Lipschitz} (or \dfn{$\infty$-harmonic}).

\pause

If $M$ is a surface, Daskalopoulos and Uhlenbeck '20 construct a \dfn{conjugate $q$-harmonic} $u_q$ to a $p$-harmonic $v_p$, $p > 2$, $1/p + 1/q = 1$, by 
$$\dif u_q = |\dif v_p|^{p - 2} \star \dif v_p, ~\dif v_p = |\dif u_q|^{q - 2} \star \dif u_q.$$
In the limit $p \to \infty$ (up to some normalization) we see that $\infty$-harmonic and $1$-harmonic functions also come in conjugate pairs.
\end{frame}

\begin{frame}{Geodesic laminations}
\begin{definition}
Let $M$ be a surface.
A \dfn{geodesic lamination} is a closed subset of $M$ which can be locally be flattened by a Lipschitz coordinate transformation to take the form $K \times \RR$ where $K \subset \RR$ is compact and for each $k \in K$, $\{k\} \times \RR$ is a geodesic.
\end{definition}

\pause

\begin{theorem}[Daskalopoulos, Uhlenbeck '20]
Let $M$ be a closed hyperbolic surface, $\Delta_\infty v = 0$, and $u$ the conjugate $1$-harmonic function to $v$.
Then the support $\lambda$ of $\dif u$ is a geodesic lamination, and the geodesics are level sets of $u$.
Moreover, $|\dif v| = ||\dif v||_{L^\infty}$ on $\lambda$.
\end{theorem}
\end{frame}

\begin{frame}{Geodesic laminations}
\begin{theorem}[Daskalopoulos, Uhlenbeck '20]
Let $M$ be a closed hyperbolic surface, $\Delta_\infty v = 0$, and $u$ the conjugate $1$-harmonic function to $v$.
Then the support $\lambda$ of $\dif u$ is a geodesic lamination, and the geodesics are level sets of $u$.
Moreover, $|\dif v| = ||\dif v||_{L^\infty}$ on $\lambda$.
\end{theorem}

\pause

\begin{question}
Let $M$ be a more general Riemannian manifold.
For a $1$-harmonic function $u$ on $M$, what is the topological structure of the level sets of $u$?
\end{question}

\pause

\begin{question}
What sort of convex duality is there between the $p$-Laplacian and $q$-Laplacian, $1/p + 1/q = 1$, on a general Riemannian manifold?
\end{question}

\pause

I can partially answer the first question, but the second is a goal of mine.
\end{frame}

\begin{frame}{Computational geometry}
Currently algorithms for computing minimal surfaces proceed by running a parabolic PDE solver on the mean curvature flow and taking $t \to \infty$.

\pause

\begin{theorem}[Loisel '20]
Let $\mathcal T$ be a quasiuniform triangulation of $M$ with $n$ simplices.
Using a convex optimization method, one can minimize $\int_M |\dif u| \dif V$ in $PL(\mathcal T)$, in $O(n^{1/2} \log n)$ Newton iterations.
\end{theorem}

\pause

\begin{question}
In what sense does the minimizer in $PL(\mathcal T)$ converge to a $1$-harmonic function as we refine $\mathcal T$? What about its level sets?
\end{question}

\pause

\begin{question}
Is it practical to apply Loisel's algorithm to compute minimal surfaces from their Dirichlet data?
\end{question}

\pause

More future goals of mine.
\end{frame}

%%%%%%%%%%%%%%%%%%%%%%%%%%%%%%%%%%%%%
\section{The regularity theorem}

\begin{frame}{The regularity theorem}
\begin{theorem}[--, '22]
If $M$ is a \textbf{manifold of constant sectional curvature} and $d \leq 7$, then for every set $U$ of least perimeter in $M$, $\partial U$ is an (analytic, embedded, stable) minimal hypersurface.
\end{theorem}

\pause

To set up the proof and see what's different from Miranda's case ($M \subseteq \RR^d$), we need to scrutinize the Lebesgue differentiation theorem to see how to construct the conormal $1$-form to $\partial U$.
\end{frame}

\begin{frame}{Lebesgue differentiation theorem}
We need to take averages of differential forms: \pause
\begin{itemize}
    \item If $f \in L^1_\loc(\RR^d)$ and $\mu$ is Radon, then for $\mu$-almost every $x$,
$$f(x) := \lim_{\varepsilon \to 0} \frac{1}{\mu(B(x, \varepsilon))} \int_{B(x, \varepsilon)} f \dif \mu$$
is well-defined.\pause 
    \item If $f = \dif u/|\dif u|$ where $u = 1_U$, $U$ locally finite perimeter, one hopes to define the conormal $1$-form to $\partial U$, $|\dif u|\dif V$-almost everywhere.\pause 
    \item If $M$ is not flat, then $f(x), f(y)$ live in \textbf{different vector spaces} for $x \neq y$! One has to choose a trivialization of the cotangent bundle to take the average of $f$.
\end{itemize}
\end{frame}

\begin{frame}{Lebesgue differentiation theorem}
\begin{proposition}
Let $\omega$ be a distribution of locally finite total variation, and $f$ an $L^1_\loc$ differential $1$-form.
Then there exists an $\omega$-null set $Z$ such that for every Riemannian metric on $M$, every frame $(\partial_\mu)$ and dual coframe $(\dif x^\mu)$, and every $p \notin Z$,
$$f(p) = \lim_{\varepsilon \to 0} \frac{\int_{B(p, \varepsilon)} (f, \partial_\mu) |\omega| \dif V}{\int_{B(p, \varepsilon)} |\omega| \dif V} \dif x^\mu(p).$$
\end{proposition}

Content here is that $Z$ is independent of the choice of Riemannian metric, frame, or coframe: \textbf{the notion of Lebesgue point of a differential form is well-defined} as an invariant of $M$ as a $C^\infty$ manifold equipped with a $L^1_\loc$ differential $1$-form.
\end{frame}

\begin{frame}{Lebesgue differentiation theorem}
Why is the set of Lebesgue points well-defined?\pause

\begin{itemize}
    \item Choice of metric is immaterial because $\dif V$ cancels itself, and balls have bounded eccentricity.\pause
    \item Choice of coordinates: approximate $f$ by $\tilde f \in C_C(M, T'M)$, then $Z$ is covered by sets which have arbitrarily small measure (the Hardy-Littlewood maximal inequality) and are independent of the choice of coorinates (possibly after enlarging them a little).\pause
\end{itemize}

However, different choices of coordinates give averages that converge at different rates.
Need to choose coordinates later on so that averages converge fast enough.
\end{frame}

\begin{frame}{de Giorgi's reduced boundary}
\begin{definition}
Let $U$ be a set of locally finite perimeter, $u := 1_U$, $\normal := \dif u/|\dif u|$, and $\partial^* U$ the set of $|\dif u|$-Lebesgue points of $\normal$.
Then we call $\normal$ the \dfn{conormal one-form} to the \dfn{reduced boundary} $\partial^* U$.
\end{definition}\pause

It is now easy to show (c.f. Giusti '77, Chapters 3 and 4):\pause
\begin{itemize}
    \item $|\dif u|\dif V$ is the codimension-$1$ Hausdorff measure $\dif S$ on the measure-theoretic boundary $\partial U$.\pause
    \item $\partial^* U$ is dense and $\dif S$-full measure in $\partial U$.\pause
    \item If $\normal$ extends to a continuous $1$-form along $\partial U$, then $\partial U = \partial^* U$ is an embedded $C^1$ hypersurface.\pause
    \item Elliptic bootstrapping: If $\partial U$ is $C^1$ and $U$ has least perimeter, then $\partial U$ is an analytic stable minimal hypersurface.\pause
\end{itemize}

The upshot is that we just need to show that $\normal$ is continuous.
\end{frame}

\begin{frame}{Continuity of the conormal}
We just need to show that $\normal$ is continuous.
Miranda '66 proceeds as follows:\pause

\begin{itemize}
\item Take averages $\normal_r$ over $B(x, r)$ of $\normal$. \pause
\item Show that $(\normal_r)$ is locally uniformly Cauchy using induction on scale.\pause
\end{itemize}

In order for us to take averages we need a way to embed all the tangent spaces of $M$ in a single vector space, in a way that respects the geometry enough to not give error terms when we do induction on scale.\pause
\begin{itemize}
\item A natural choice is to use the exponential map.\pause
\begin{itemize}
\item This gives very messy computations that seem to lead nowhere.\pause
\item Exponential map is not conformal (unless $M$ is flat) which causes technical issues.\pause
\end{itemize}
\item But if $M$ has constant sectional curvature, all its tangent spaces embed in a certain space in a very natural way...
\end{itemize}
\end{frame}

\begin{frame}{Isomorphisms of tangent spaces}
We work out the case $M = \Hyp^d$ explicitly (other cases similar).\pause
\begin{itemize}
\item Embed $\Hyp^d$ as the future unit hyperboloid in Minkowski spacetime $\RR^{1, d}$: $\Hyp^d = \{t^2 = |x|^2 + 1, t > 0\}$.\pause
\item $\RR^{1, d}$ is flat, so all its tangent spaces are identified with $\RR^{1, d}$.\pause
\item So the tangent spaces $T_x\Hyp^d$ all embed in $\RR^{1, d}$ and are identified up to Lorentz transformations ($SO^+(d, 1)$).\pause
\begin{itemize}
\item In the nonrelativistic limit $c \to \infty$ ($c$ the lightspeed), an observer at the origin of $\RR^{1, d}$ would think that $M$ is $\{t = 1\}$.\pause
\item The error the observer would make would only be quadratically small in $c^{-1}$.
\end{itemize}
\end{itemize}
\end{frame}

\begin{frame}{Isomorphisms of tangent spaces}
To define an averaging operator on $\Omega^1\Hyp^d$, we used the embedding $\Hyp^d \subset \RR^{1, d}$.\pause
\begin{itemize}
\item Up to a quadratically small error in the scale $c^{-1}$, everything is $SO(d, 1)$-equivariant.\pause
\item Precedent for this trick: Daskalopoulos and Uhlenbeck '22 propose to study a PDE on $\Hyp^d$ using the action of $SO(d, 1)$ on $\RR^{1, d}$.\pause
\begin{itemize}
\item In the case of D-U, they want to study Schatten-von Neumann $p$-harmonic maps $u: \Hyp^2 \to \Hyp^2$.\pause
\item To construct a stress tensor (or ``energy-momentum tensor") for $u$, they have to apply N\"other's theorem to $so(2, 1)$.
\end{itemize}
\end{itemize}
\end{frame}

\begin{frame}{Definition of the averaging operator}
Let $(x^\mu)$ be Poincar\'e ball coordinates on $\Hyp^d$.
Using a family of isometries (uniquely defined up to a gauge transformation $\chi: \Hyp^d \to \SpOrth(\Hyp^d)$) we define translates $(x^\mu_p)$ based at any point $p \in \Hyp^d$.\pause

\begin{definition}
For a $1$-form $\xi$ on $\Hyp^d$, $p \in \Hyp^d$, $A \subseteq \Hyp^d$, and a measure $\omega$, define
$$\avg_{p, A, \omega} \xi := \left[\frac{1}{\omega(A)} \int_A (\xi, \partial_\mu^p) \dif \omega\right] \dif x^\mu_p(p).$$
\end{definition}
\end{frame}

\begin{frame}{Translation invariance of the averaging operator}
\begin{definition}
For a $1$-form $\xi$ on $\Hyp^d$, $p \in \Hyp^d$, $A \subseteq \Hyp^d$, and a measure $\omega$, define
$$\avg_{p, A, \omega} \xi := \left[\frac{1}{\omega(A)} \int_A (\xi, \partial_\mu^p) \dif \omega\right] \dif x^\mu_p(p).$$
\end{definition}\pause

\begin{itemize}
\item This choice of averaging operator agrees with averaging of maps into $\RR^{d, 1}$ up to second order.\pause
\item It has the advantage of mapping into a cotangent space of $M$ rather than a bigger space.\pause
\end{itemize}

\begin{proposition}
The average is gauge-invariant, and almost translation-invariant:
$$|\avg_{p, A} \xi - \avg_{q, A} \xi| \lesssim (\mathrm{diam} A)^2 ||\xi||_{C^0}.$$
\end{proposition} 
\end{frame}

\begin{frame}{de Giorgi's lemma}
\begin{definition}
Let $U$ be a set of locally finite perimeter.
The \dfn{excess} of $U$ in an open set $A$ based at $p$ is
$$\gamma_{A, p} := |\partial^* U \cap A|\left(1 - \left|\avg_{p, A, dS} \dif 1_U\right|\right).$$
\end{definition}\pause

\begin{proposition}[de Giorgi's lemma]
If $U$ has least perimeter, $p \in \partial U$, and $r \ll 1$, then
$$\gamma_{B(p, r/2), p} \leq \frac{\gamma_{B(p, r), p}}{2^d} + O(r^{d + 1}).$$
\end{proposition}\pause

From de Giorgi's lemma and an induction on scale, $\gamma_{B(p, r), p} \lesssim r^d$, and it follows that 
$$\normal_r(p) := \avg_{p, B(p, r), dS} \dif 1_U$$
is locally uniformly Cauchy as $r \to 0$, so we win.
\end{frame}

\begin{frame}{Proof of de Giorgi's lemma}
Trying to estimate $\gamma_{B(p, r), p}$.\pause

\begin{itemize}
\item Since $d \leq 7$ and there are no singular cones of least perimeter in $\RR^7$, every tangent cone to $\partial U$ is a hyperplane.\pause
\item Comparing $\partial U$ to its tangent cones, we can make the bootstrap assumption $\gamma_{B(p, r), p} \ll r^{d - 1}$.\pause
\item By a mollification argument, wlog $\partial U$ is $C^1$. This preserves $\gamma_{B(p, r), p}$, but maybe now $p \notin \partial U$.\pause
\item Since we can take a loss of size $O(r^{d + 1})$, we can replace $\gamma_{B(p, r), p}$ with $\gamma_{B(p, r), q}$ where $q \in \partial U \cap B(p, r)$.
\end{itemize}
\end{frame}

\begin{frame}{Proof of de Giorgi's lemma}
Trying to estimate $\gamma_{B(p, r/2), q}$ where $\partial U$ is $C^1$, $q \in \partial U$, and $\gamma_{B(p, r/2), q} \ll r^{d - 1}$.\pause

\begin{itemize}
\item By the bootstrap assumption, $\normal_U$ doesn't oscillate much.\pause
\item After rotation, can assume that $\partial U$ is the graph of a $C^1$ function $w$ such that:\pause
\begin{itemize}
\item $w(0) = 0$ (since $q \in \partial U$),\pause
\item $||\dif w||_{C^0}$ is smaller than an arbitrary absolute constant, and \pause
\item $w$ approximately solves the minimal surface equation, $Pw = 0$, for $\Hyp^d$.\pause
\end{itemize}
\item Let $\mathscr L$ be the Lagrangian for the minimal surface operator $P$. Then 
$$\int_{B_\rho} \mathscr L(w, \dif w) - \mathscr L(w, \avg_\rho \dif w) \approx \gamma_{B(p, r), q}.$$
\end{itemize}
\end{frame}

\begin{frame}{An elliptic estimate}
\begin{lemma}
For each $\varepsilon > 0$, there exists $\delta > 0$ such that for every $w \in C^1(B_\rho)$ with $w(0) = 0$ and $||\dif w||_{C^0} \leq \delta$ and every $\beta > 0$, if 
$$\int_{B_\rho} \mathscr L(w, \dif w) - \mathscr L(w, \avg_\rho \dif w) \leq \beta,$$
and $w$ is approximately a solution in the sense that there exists $\tilde w$ with the same trace such that $P\tilde w = 0$ and
$$\int_{B_\rho} \mathscr L(w, \dif w) \leq \delta \beta + \int_{B_\rho} \mathscr L(\tilde w, \dif \tilde w),$$
then for any $\alpha = 0.5 + O(\varepsilon)$,
$$\int_{B_{\alpha \rho}} \mathscr L(w, \dif w) - \mathscr L(w, \avg_{\alpha \rho} \dif w) \leq (\alpha^{d + 1} + \varepsilon) \beta + O(\rho^{d + 1}).$$
\end{lemma}
\end{frame}

\begin{frame}{Proof of the elliptic estimate}
Recall $w(0) = 0$ - equivalent to the fact that the excess is based at $\partial U$ \textbf{and the reason why averaging must be translation-invariant up to quadratic error}.\pause

\begin{itemize}
\item Since $w(0) = 0$ and $||\dif w||_{C^0} \leq \delta$, we are only interested in near the origin of hyperbolic space.\pause
\item Taylor expanding the hyperbolic metric at the origin
$$g_{\mu\nu} = \frac{\delta_{\mu\nu}}{(1 - |x|^2/4)^2} = \delta_{\mu\nu} + O(|x|^2)$$
we see that up to $O(\rho^{d + 1})$, we can replace $P$ with the euclidean minimal surface equation.\pause
\item Reduced the problem to its euclidean case (Miranda '66).\pause
\end{itemize}

With this estimate in hand, we have proven the de Giorgi lemma and hence the de Giorgi--Miranda regularity theorem for $M = \Hyp^d$!!
The proof for more general manifolds of constant sectional curvature is very similar.
\end{frame}

%%%%%%%%%%%%%%%%%%%%%%%%%%
\section{Topological structure of functions of least gradient}

\begin{frame}{Minimal laminations}
\begin{definition}
A \dfn{minimal lamination} is a closed subset of $M$ which can be locally be flattened by a Lipschitz coordinate transformation to take the form $K \times \RR^{d - 1}$ where $K \subset \RR$ is compact and for each $k \in K$, the \dfn{leaf} $\{k\} \times \RR^{d - 1}$ is a minimal surface.
\end{definition}\pause

\begin{theorem}[--, '22]
    Suppose that $M$ has constant sectional curvature, $2 \leq d \leq 4$, and $u \in BV(M)$ has least gradient.
    Then the support of $\dif u$ is a minimal lamination $\lambda$ whose leaves are the level sets of $u$.
    \end{theorem}
\end{frame}

\begin{frame}{Proof of the topological structure theorem}
Let $u$ have least gradient.\pause
\begin{itemize}
\item Level sets of $u$ are stable embedded minimal hypersurfaces, and if $\partial U, \partial V$ are level sets then either $U \subseteq V$ or $V \subseteq U$.\pause
\item Minimal surface equation is elliptic, so by the maximum principle, $\partial U, \partial V$ are disjoint.\pause
\item Given minimal hypersurfaces, when we can we glue them together in a Lipschitz way?\pause
\begin{itemize}
\item For $d = 2$ (geodesics), D-U '20 found Lipschitz charts using the exponential map, since it sends geodesics to lines.\pause
\item Exponential map doesn't send minimal hypersurfaces to hyperplanes, so need a new idea.
\end{itemize}
\end{itemize}
\end{frame}

\begin{frame}{Flattening the level sets}

\begin{theorem}[Colding, Minicozzi '02]
Let $(N_i)$ be a sequence of minimal hypersurfaces in a precompact subset of $M$, and assume that the second fundamental forms are locally uniformly bounded.
Then there exists a minimal lamination $\lambda$ such that $N_i \to \lambda$ along a subsequence, in the sense that if we think of $N_i$ as a lamination with finitely many leaves, the local product structure of $N_i$ converges in a H\"older space to the local product structure of $\lambda$.
\end{theorem}\pause

\begin{itemize}
\item By Schauder and Harnack, a bound on the second fundamental forms implies a bound on the derivatives of the local product structures.\pause
\item By Ascoli, a subsequence converges.\pause
\item Finally, a regularity theorem for minimal laminations implies that a continuous lamination is Lipschitz.
\end{itemize}
\end{frame}

\begin{frame}{Flattening the level sets}

\begin{theorem}[stable Bernstein theorem]
Assume that $2 \leq d \leq 4$.
The second fundamental form of a \textbf{stable} minimal hypersurface $N$ in $M$ is locally uniformly bounded independently of $N$.
\end{theorem}\pause

The proof of this is very dimension-specific:\pause
\begin{itemize}
\item For $d = 2$ this is vacuous (the second fundamental form of a geodesic is $0$).\pause
\item For $d = 3$, we can use the fact that $N$ has quadratic area growth to bound its Gauss curvature.\pause
\item For $d = 4$, Chodosh and Li '21 slice $N$ by level sets of the fundamental solution of the Laplacian on $M$; the slices are $2$-dimensional, hence have controlled Gauss curvature by Gauss-Bonnet.
\end{itemize}
\end{frame}

\begin{frame}{The punchline}
Let $u$ be a function of least gradient.\pause

\begin{itemize}
\item The level sets of $u$ are smooth disjoint stable embedded minimal hypersurfaces.\pause
\item By the stable Bernstein theorem, the level sets all satisfy the same bound on their second fundamental forms.\pause
\item So by the compactness theorem of Colding--Minicozzi, if we choose a dense sequence $(\tilde N_i)$ of level sets, and let
$$N_i := \tilde N_1 \cup \cdots \cup \tilde N_i,$$
then along a subsequence, $N_i$ converges to a lamination.\pause
\end{itemize}

That's what we wanted to show!
\end{frame}

\begin{frame}[allowframebreaks]{References}
\begin{itemize}
\item --, in preparation: ``Functions of least gradient and minimal laminations''
\item --, in preparation: ``Modes of convergence of minimal laminations''
\item Bombieri, de Giorgi, and Giusti '69: ``Minimal cones and the Bernstein problem''
\item Chodosh and Li '21: ``Stable minimal hypersurfaces in $\RR^4$''
\item Colding and Minicozzi '02: ``The space of embedded minimal surfaces of fixed genus in a 3-manifold IV''
\item Colding and Minicozzi '11: ``A course in minimal surfaces''
\item Daskalopoulos and Uhlenbeck '20: ``Transverse measures and best Lipschitz and least gradient maps''
\item Daskalopoulos and Uhlenbeck '22: ``Analytic properties of stretch maps and geodesic laminations''
\item Daskalopoulos and Uhlenbeck, in preparation: ``Lie algebra valued laminations correspond to earthquake flows''
\item de Giorgi, '55: ``Frontiere orientate di misura minima''
\item de L\'eon, Maz\'on, and Rosser '14: ``Functions of least gradient and 1-harmonic functions''
\item Giusti '77: ``Functions of bounded variation and minimal surfaces''
\item Loisel '20: ``Efficient algorithms for solving the p-Laplacian in polynomial time''
\item Miranda '64: ``Superfici cartesiane generalizzate ed insiemi di perimetro localmente finito sui prodotti cartesiani''
\item Miranda '66: ``Sul minimo dell'integrale del gradiente di una funzione''
\item Miranda '67: ``Comportamento delle successioni convergenti di frontiere minimali''
\end{itemize}
\end{frame}

\begin{frame}{Acknowledgments}
Special thanks to:
\begin{itemize}
\item George Daskalopoulos, for advising this work 
\item Richard Schoen and Chao Li, for suggesting the application of the stable Bernstein theorem
\item Tai Borges and Haram Ko, for comments on a draft version of this talk
\item and you, for coming!
\end{itemize}
\end{frame}

\end{document}
