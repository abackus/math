\documentclass[reqno,10pt]{amsart}
\usepackage[letterpaper, margin=1in]{geometry}
\RequirePackage{amsmath,amssymb,amsthm,graphicx,mathrsfs,url,slashed,subcaption}
\RequirePackage[usenames,dvipsnames]{xcolor}
\RequirePackage[colorlinks=true,linkcolor=Red,citecolor=Green]{hyperref}
\RequirePackage{amsxtra}
\usepackage{cancel}
\usepackage{tikz-cd}

% \setlength{\textheight}{9.3in} \setlength{\oddsidemargin}{-0.25in}
% \setlength{\evensidemargin}{-0.25in} \setlength{\textwidth}{7in}
% \setlength{\topmargin}{-0.25in} \setlength{\headheight}{0.18in}
% \setlength{\marginparwidth}{1.0in}
% \setlength{\abovedisplayskip}{0.2in}
% \setlength{\belowdisplayskip}{0.2in}
% \setlength{\parskip}{0.05in}
%\renewcommand{\baselinestretch}{1.05}

\title{Modes of convergence of minimal laminations}
\author{Aidan Backus}
\date{October 2022}

\newcommand{\NN}{\mathbf{N}}
\newcommand{\ZZ}{\mathbf{Z}}
\newcommand{\QQ}{\mathbf{Q}}
\newcommand{\RR}{\mathbf{R}}
\newcommand{\CC}{\mathbf{C}}
\newcommand{\DD}{\mathbf{D}}
\newcommand{\PP}{\mathbf P}
\newcommand{\MM}{\mathbf M}
\newcommand{\II}{\mathbf I}
\newcommand{\Hyp}{\mathbf H}
\newcommand{\Sph}{\mathbf S}
\newcommand{\Group}{\mathbf G}
\newcommand{\GL}{\mathbf{GL}}
\newcommand{\Orth}{\mathbf{O}}
\newcommand{\SpOrth}{\mathbf{SO}}
\newcommand{\Ball}{\mathbf{B}}

\newcommand*\dif{\mathop{}\!\mathrm{d}}

\DeclareMathOperator{\dist}{dist}
\DeclareMathOperator{\MeasLam}{MeasLam}
\DeclareMathOperator{\MinLam}{MinLam}
\DeclareMathOperator{\Lam}{Lam}
\DeclareMathOperator{\supp}{supp}

\newcommand{\Leaves}{\mathscr L}
\newcommand{\Hypspace}{\mathscr H}

\newcommand{\Two}{\mathrm{I\!I}}


\newcommand{\Hilb}{\mathcal H}
\newcommand{\Homology}{\mathrm H}
\newcommand{\normal}{\mathbf n}
\newcommand{\radial}{\mathbf r}
\newcommand{\evect}{\mathbf e}
\newcommand{\vol}{\mathrm{vol}}

\newcommand{\Bmu}{\boldsymbol \mu}
\newcommand{\Bnu}{\boldsymbol \nu}
\newcommand{\Blambda}{\boldsymbol \lambda}

\newcommand{\pic}{\vspace{30mm}}
\newcommand{\dfn}[1]{\emph{#1}\index{#1}}

\renewcommand{\Re}{\operatorname{Re}}
\renewcommand{\Im}{\operatorname{Im}}

\newcommand{\loc}{\mathrm{loc}}
\newcommand{\cpt}{\mathrm{cpt}}

\def\Japan#1{\left \langle #1 \right \rangle}

\newtheorem{theorem}{Theorem}[section]
\newtheorem{badtheorem}[theorem]{``Theorem"}
\newtheorem{prop}[theorem]{Proposition}
\newtheorem{lemma}[theorem]{Lemma}
\newtheorem{sublemma}[theorem]{Sublemma}
\newtheorem{proposition}[theorem]{Proposition}
\newtheorem{corollary}[theorem]{Corollary}
\newtheorem{conjecture}[theorem]{Conjecture}
\newtheorem{axiom}[theorem]{Axiom}
\newtheorem{assumption}[theorem]{Assumption}

\newtheorem{mainthm}{Theorem}
\renewcommand{\themainthm}{\Alph{mainthm}}

% \newtheorem{claim}{Claim}[theorem]
% \renewcommand{\theclaim}{\thetheorem\Alph{claim}}
\newtheorem*{claim}{Claim}

\theoremstyle{definition}
\newtheorem{definition}[theorem]{Definition}
\newtheorem{remark}[theorem]{Remark}
\newtheorem{example}[theorem]{Example}
\newtheorem{notation}[theorem]{Notation}

\newtheorem{exercise}[theorem]{Discussion topic}
\newtheorem{homework}[theorem]{Homework}
\newtheorem{problem}[theorem]{Problem}

\makeatletter
\newcommand{\proofpart}[2]{%
  \par
  \addvspace{\medskipamount}%
  \noindent\emph{Part #1: #2.}
}
\makeatother



\numberwithin{equation}{section}


% Mean
\def\Xint#1{\mathchoice
{\XXint\displaystyle\textstyle{#1}}%
{\XXint\textstyle\scriptstyle{#1}}%
{\XXint\scriptstyle\scriptscriptstyle{#1}}%
{\XXint\scriptscriptstyle\scriptscriptstyle{#1}}%
\!\int}
\def\XXint#1#2#3{{\setbox0=\hbox{$#1{#2#3}{\int}$ }
\vcenter{\hbox{$#2#3$ }}\kern-.6\wd0}}
\def\ddashint{\Xint=}
\def\dashint{\Xint-}

\usepackage[backend=bibtex,style=numeric]{biblatex}
\renewcommand*{\bibfont}{\normalfont\footnotesize}
\addbibresource{topics.bib}
\renewbibmacro{in:}{}
\DeclareFieldFormat{pages}{#1}


\begin{document}
\begin{abstract}
We collect several results relating different notions of convergence of laminations, especially minimal laminations.
We also give a condition for a partition of a closed set into minimal hypersurfaces to be a lamination.
\end{abstract}

\maketitle

%%%%%%%%%%%%%%%%%%%%%%%%%%%%%%%%%%%%%%%%%%%%%%%%%%%%%%%

% \tableofcontents

\section{Introduction}
The purpose of this paper is to record several modes of convergence for laminations, and in particular for minimal laminations, for use in the companion papers \cite{BackusFLG, DaskalopoulosPrep2}.
In such papers it is crucial that a disjoint family of minimal hypersurfaces without boundary -- which we refer to as a \dfn{minimal partition} -- is in fact a minimal lamination.
However this is false in general, as we will discuss, and so we will use the compactness theory discussed in this paper as well as the stable Bernstein theorem \cite{Schoen2016, Chodosh2021} to give a criterion under which a minimal partition is a lamination.

Most of the modes of convergence in this paper and their compactness theorems have been known to various sources \cite{ColdingMinicozziIV, ColdingMinicozziV, thurston1979geometry}.
Our contribution is to synthesize the known results, explaining under what circumstances different modes of convergence are stronger than others, as well as under what circumstances a sequence of laminations has a convergent subsequence.

%%%%%%%%%%%%%%%%%%%%%%%%

\subsection{Acknowledgements}
I would like to thank Georgios Daskalopolous for suggesting this project and for helpful discussions, and NSF...

%%%%%%%%%%%%%%%%%%%%%%%%%%%

\section{Preliminaries}
\begin{definition}
Let $S$ be a nonempty closed subset of a smooth manifold $M$.
\begin{enumerate}
\item A (codimension $1$) \dfn{flow box} for $S$ is a $C^0$ chart in which $S$ can be expressed as $K \times \RR^{d - 1} \subseteq \RR^d$ for some closed $K \subseteq [-1, 1]$.
\item The closed set $K$ is called a \dfn{leaf space}. Its fibers are called \dfn{leaves}, and the \dfn{label} of the fiber $\{k\} \times \RR^{d - 1}$ is $k$.
\item A \dfn{lamination} $\lambda$ in $M$, with support $S$, consists of an atlas of lamination flow boxes for $S$, whose transition maps preserve the local product structure, such that the leaves are $C^2$ hypersurfaces in $M$.
\item A \dfn{plaque} of $\lambda$ is a connected component of $M \setminus S$.
\end{enumerate}
\end{definition}

We shall write a flow box as 
$$F_\alpha: [-1, 1] \times \RR^{d - 1} \to M$$
where $\alpha$ ranges over some atlas $A$, and denote the associated leaf space by $K_\alpha \subseteq [-1, 1]$.
The transition maps will be denoted $\psi_{\alpha \beta}$, and are characterized by
\begin{equation}\label{transition relation}
F_\alpha = F_\beta \circ \psi_{\alpha \beta}
\end{equation}
wherever both sides of (\ref{transition relation}) are defined.
Since the transition maps are required to preserve the local product structure $K_\alpha \times \RR^{d - 1}$, they induce self-maps of $K_\alpha \cap K_\beta$ that we also denote by $\psi_{\alpha \beta}$ if no confusion can result.

We assume that the leaves are $C^2$ in order to ensure that the normal vectors to each leaves are well-defined in $C^1$.
In \cite{Morgan88}, such laminations are called $C^2$ \dfn{along leaves}.
This is not the same thing as assuming that the lamination admits a $C^2$ atlas, as it may not be able to extend the normal vectors to each leaf to a $C^1$ vector field on $M$ even locally; see \S\ref{RegularitySec} for more precise assertions about regularity.
This assumption in particular ensures that the second fundamental forms, and hence mean curvatures, to each leaf are well-defined in $C^0$.

It seems rather difficult to say much about convergence of general laminations, as we will give several examples to illustrate.
However, the situation improves significantly if the laminations in question are minimal:

\begin{definition}
Let $M$ be a Riemannian manifold.
A lamination $\lambda$ is \dfn{minimal} if every leaf of $\lambda$ has zero mean curvature, and is \dfn{geodesic} if in addition $M$ is a surface.
\end{definition}

A lamination is geodesic iff its leaves are geodesics.

Control on the second fundamental forms will be a crucial assumption throughout this paper and so we make the following definition:

\begin{definition}
The \dfn{maximal curvature} of a lamination $\lambda$ is
$$R_\lambda := \sup_N ||\Two_N||_{C^0}$$
where $N$ ranges over leaves of $\lambda$ and $\Two_N$ is the second fundamental form of $N$.
\end{definition}

The maximal curvature of a geodesic lamination is $0$.
For a minimal lamination in a threefold $M$, $R_\lambda$ is instead controlled by the Gauss curvature of the leaves of $\lambda$ and the Ricci curvature of $M$.

%%%%%%%%%%%%%%%%%%%%%%%%%%%%%%%%%%%%%
\subsection{Topological conditions on laminations}
Let us now identify some useful topological conditions one may put on laminations.
Before doing so, we recall some elementary point-set topology, as in \cite[Chapter 2]{Pugh02}.
A \dfn{Polish space} is a second-countable topological space which admits a complete metric. 
This condition shall be useful for us because every leaf space $K_\alpha$ is Polish.
A Polish space is \dfn{perfect} if it has no isolated points, and \dfn{totally disconnected} if every component of $X$ is singleton.

\begin{definition}
Let $\lambda$ be a lamination in $M$.
\begin{enumerate}
\item $\lambda$ is \dfn{discrete}, \dfn{perfect}, or \dfn{totally disconnected}, if the same property holds of every leaf space of $\lambda$.
\item $\lambda$ is \dfn{finite} if every leaf of $\lambda$ is closed, and $\lambda$ has only finitely many leaves.
\item $\lambda$ is a \dfn{foliation} if $\supp \lambda = M$.
\item $\lambda$ is \dfn{absolutely continuous} if we can choose an atlas so that every leaf space of $\lambda$ is connected.
\end{enumerate}
\end{definition}

Every finite lamination is discrete, and every discrete lamination is totally disconnected.
Conversely, a discrete lamination in a closed manifold, or a perfect discrete lamination, is finite.
However, a lamination with finitely many leaves need not be finite, even if it is geodesic:

\begin{example}\label{two geodesics}
Let $M = \RR \times \Sph^1$ be a cylinder.
We may choose a metric on $M$ so that there is a geodesic $\gamma_1$ which is a simple closed curve looping around $M$, and that there is another geodesic $\gamma_2$ which winds around $M$ infinitely many times, converging to $\gamma_1$.
TODO: Make a picture.
Then $\gamma_1, \gamma_2$ form a geodesic lamination with two leaves which is not finite.
This example, and slight modifications thereof, will be frequently useful as a counterexample throughout this paper.
\end{example}



%%%%%%%%%%%%%%%%%%%%%%%%%%%%%%

\subsection{Regularity of minimal laminations}\label{RegularitySec}
Though we impose that laminations are $C^0$ (in the sense that their flow boxes are $C^0$) we can improve this regularity in the case of minimal laminations.
In this section we follow \cite{Solomon86}, which treats the special case that $\lambda$ is a minimal foliation of $M \subseteq \RR^d$.

\begin{theorem}\label{regularity theorem}
Let $\lambda$ be a minimal lamination. Then $\lambda$ admits a Lipschitz atlas, and $\normal_\lambda$ is Lipschitz.
\end{theorem}
\begin{proof}
TODO
\end{proof}

\begin{example}
Lipschitz regularity is optimal, even in the nicest possible case of a geodesic foliation of a complete Riemannian manifold.
Indeed, let $M = B((2, 0), 1)$ in $\RR^2$. Then the chords $\{y = ax\}$ for $a > 0$ and $\{y = a\}$ for $a \leq 0$ define a foliation $\lambda$ of $M$.
If we equip $M$ with the Beltrami-Klein metric, then, since the leaves of $\lambda$ are chords, $\lambda$ defines a geodesic foliation of $M \cong \Hyp^2$.
Moreover the conormal to the foliation is
$$\normal = \frac{1_{y > 0}}{\sqrt{1 + y^2/x^2}} \left[\dif y - \frac{y}{x} \dif x\right] + 1_{y \leq 0} \dif y$$
which is not $C^1$.
TODO: Make a picture
\end{example}



%%%%%%%%%%%%%%%%%%%%%
\section{Thurston's geometric topology}
\subsection{Global space of leaves}
We now put a topology on the set $\Leaves \lambda$ of all leaves of a lamination $\lambda$ in a closed manifold, so that $\Leaves \lambda$ is a compact Hausdorff space.

\begin{definition}
Let $X$ be a compact metric space. The \dfn{Hausdorff distance} between two closed sets $A, B \subset X$ is
$$\dist(A, B) := \max\left(\max_{a \in A} \min_{b \in B} \dist(a, b), \max_{b \in B} \min_{a \in A} \dist(a, b)\right).$$
The space of closed subsets of $X$ is the \dfn{hyperspace} $\Hypspace X$.
\end{definition}

The fact that we restrict $X$ to be compact is motivated by the following example.

\begin{example}
Consider $\Hyp^2$ with its disk model in $\CC$, and let $\gamma$ be the horizontal geodesic through the origin $0$.
Consider also a sequence of geodesics $(\gamma_j)$ through $i/j$ which are horizontal at $i/j$.
Then $\lambda_j \to \lambda$ in Thurston's geometric topology and in Hausdorff distance.
However,
$$\dist_{\mathscr H(\Hyp^2)}(\gamma_j, \gamma) = \infty.$$
Indeed, one can find points $x$ on $\gamma$ which are arbitrarily close to the circle at infinity $\partial \Hyp^2$.
Let $\rho$ be the geodesic which is orthogonal to $\gamma$ at $x$.
Then the segment of $\rho$ between $\gamma$ and $\gamma_j$ would become arbitrarily long as $x \to \partial \Hyp^2$.
\end{example}

For compact $X$, however, one can give a homeomorphism-invariant definition of the topology of $\Hypspace X$ \cite[Chapter 4]{nadler2017continuum}, and from that definition it follows that $\Hypspace X$ is a compact metric space.
In particular, $\Hypspace$ is a self-map of the class of compact metrizable spaces.
Moreover, if $K_i \to K$ in $\Hypspace X$, then $K$ is the set of limits of sequences $(k_i)$ such that $k_i \in K_i$.

These properties make it suitable to define $\Leaves \lambda$ to be a certain closed subset of $\Hypspace M$:

\begin{definition}
If $\lambda$ is a lamination in a closed manifold $M$, we let $\Leaves \lambda$ be the subset of $\Hypspace M$ whose elements are closed sets $\overline N$ where $N$ is a leaf of $\lambda$.
\end{definition}

\begin{proposition}
For any lamination $\lambda$, $\Leaves \lambda$ is in canonical bijection with the set of all leaves of $\lambda$, and is a closed subset of $\Hypspace M$.
\end{proposition}
\begin{proof}
Clearly the map $N \mapsto \overline N$ is a canonical surjection from the set of leaves of $\lambda$ to $\Leaves \lambda$.
In fact, it is injective as well: if $\overline N = \overline{N'}$, then for any sequence $(x_i)$ in $N$ such that $x_i \to x$ in $M$, there exists a sequence $(x_i')$ in $N'$ such that $x_i' \to x$.
Moreover (possibly after discarding finitely many entries in the sequence) we may assume that these limits take place inside some flow box with image $U_\alpha$, so there are co-Cauchy sequences $(k, y_i)$ and $(k', y_i')$ in $\RR \times \RR^{d - 1}$, where $k, k'$ are the labels of $N, N'$ respectively.
This is only possible if $k = k'$ and so $N \cap U_\alpha = N' \cap U_\alpha$.
After covering $M$ by flow boxes we see that $N = N'$ and so the canonical map is injective.

Now if $(N_i)$ is a sequence in $\Leaves \lambda$ which converges to some closed set $N \in \Hypspace M$, then $N \in \Hypspace(\supp \lambda)$.
Indeed, $\supp \lambda$ is a compact subset of $M$, so $\Hypspace(\supp \lambda)$ is compact, and hence is a closed subset of $\Hypspace M$.
In any flow box $U$, the sets $N_i \cap U$ take the form $\{k_i\} \times \RR^{d - 1}$, and any limit of such sets must of the form $\{k\} \times \RR^{d - 1}$.
So $N \cap U$ is a leaf of the restriction of $\lambda$ to $U$.
It follows that $N \in \Leaves \lambda$.
\end{proof}

%%%%%%%%%%%%%%%%%%%%%%%%%
\subsection{Definition}

\begin{definition}
A sequence of laminations $\lambda_i$ converges to $\lambda$ in \dfn{Thurston's geometric topology} if, for every leaf $N$ of $\lambda$, every $x \in N$, and every $\varepsilon > 0$, there exists $i_\varepsilon \in \NN$ such that for every $i \geq i_\varepsilon$, $\supp \lambda_i$ intersects $B(x, \varepsilon)$, and for $x_i \in B(x, \varepsilon) \cap \supp \lambda_i$,
\begin{equation}\label{convergence of normals}
\dist_{S' M}(\normal_{\lambda_i}(x_i), \normal_\lambda(x)) < 2\varepsilon.
\end{equation}
\end{definition}

It is straightforward to see that this topology does not depend on the Riemannian metric.
The distnace in (\ref{convergence of normals}) is in the cosphere bundle $S'M$.
It is also straightforward to see that if $\lambda_i \to \lambda$ and $\lambda' \subseteq \lambda$ then $\lambda_i \to \lambda'$.
Thus Thurston's geometric topology is very far from Hausdorff.
To deal with the lack of uniqueness of limits we single out the special case that we have a privileged choice of limit:

\begin{definition}
If $\lambda_i \to \lambda$ in Thurston's geometric topology, and for every lamination $\lambda'$, $\lambda_i \to \lambda'$ iff $\lambda' \subseteq \lambda$, we say that $\lambda$ is the \dfn{maximal limit} of $(\lambda_i)$ in Thurston's geometric topology.
\end{definition}

If a maximal limit exists, it is unique.
However one may consider the case that there are two limiting laminations whose leaves intersect to see that there need not exist a maximal limit.

Since Thurston's geometric topology requires convergence of normal vector fields pointwise, it ``respects the regularity of the flow boxes" in some sense.
On the other hand this makes it somewhat hard to prove convergence in this topology.

\begin{lemma}\label{transverse at one implies transverse at all}
Let $\lambda$ be a minimal lamination, $N$ a leaf of $\lambda$, $x \in N$, and $\gamma$ a geodesic which intersects $N$ at $x$ transversely.
Then for every leaf $\tilde N$ whose label is close to $N$ in a Lipschitz flow box near $x$, $\gamma$ intersects $\tilde N$ transversely at a point near $x$.
\end{lemma}
\begin{proof}
Since Lipschitz maps preserve transversality TODO, we may replace $\lambda$ with a lamination in $[0, 1] \times \RR^{d - 1}$, so that $N$ corresponds to $\{0\} \times \RR^{d - 1}$ and $N'$ corresponds to $\{k\} \times \RR$, where $0 < k \ll 1$ depending on $\gamma$.
In this flow box, $\gamma$ corresponds to the graph $\{t = f(k)\}$ of a Lipschitz map $f$ taking a neighborhood of $0 \in \RR$ into $\RR^{d - 1}$.
If $k$ is small enough, then $k$ is contained in the domain of $f$, so $\gamma$ is transverse to the fiber of $k$.
\end{proof}

\begin{lemma}
Let $(\lambda_i)$ be a sequence of geodesic laminations, and $\lambda$ a geodesic lamination, such that for every leaf $N$ of $\lambda$ and every $x \in \NN$, there exists a transverse geodesic $\gamma$ to $N$ at $x$ such that for every $\varepsilon > 0$, we can find $i \in \NN$, a leaf $N_i$ of $\lambda_i$, and $x_i \in N_i$, such that $\dist(x_i, x) < \varepsilon$ and $\gamma$ is transverse to $N_i$ at $x_i$.
\end{lemma}
\begin{proof}
By Lemma \ref{transverse at one implies transverse at all} and Theorem \ref{regularity theorem}, $\gamma$ is transverse to leaves of $\lambda$ (possibly after replacing $M$ with a small neighborhood of $x$).
For each $k$ such that $\gamma(k) \in \supp \lambda$, we let $X(k) \in T_{\gamma(k)} M$ be unit length and tangent to the leaf of $\lambda$ which contains $\gamma(k)$.
Then we may extend $X$ to a Lipschitz vector field along $\gamma$, and set
$$F(k, t) := \exp_{\gamma(k)}(tX(k)).$$
So $F$ is a Lipschitz flow box.
Working in the coordinates given by $F$, $N_i$ is transverse to $\{t = 0\}$, so it is the graph $\{k = f_i(t)\}$ of a Lipschitz function $f_i$ defined near $0 \in \RR$.
Moreover, $f_i(0) \to f(0)$.
We claim that $\dif f_i(0) \to \dif f(0)$; if not,
\end{proof}

%%%%%%%%%%%%%%%%%%%%%%%%%%%%
\subsection{Compactness}

\begin{lemma}
Let $(\lambda_i)$ be a sequence of geodesic laminations in a closed surface $M$, and $\lambda$ a lamination such that $\Leaves \lambda_i \to \Leaves \lambda$ in $\Hypspace^2 M$.
Then $\lambda$ is a maximal limit of $(\lambda_i)$ in Thurston's geometric topology.
\end{lemma}
\begin{proof}

\end{proof}

From this lemma and the compactness of $\Hypspace^2 M$ (we can prove compactness of $\Hypspace^k M$ by induction on $k$) we have:

\begin{proposition}
Every sequence of geodesic laminations in a closed surface has a subsequence which converges to a maximal limit in Thurston's geometric topology.
\end{proposition}

Minimality is a crucial hypothesis in the characterization of Thurston's geometric topology by Hausdorff distance:

\begin{example}
Let $f_i$ be a smooth approximation to the Dirac measure $\delta_{1/2-1/i}$ at $1/2 - 1/i$ with compact support in $(1/2 - 2/i, 1/2)$, so $f_i \to 0$ pointwise.
Let $\lambda_i$ be the lamination consisting of the graph of $f_i$ in $\RR^2$ and $\lambda$ the lamination consisting of the horizontal line through $0$.
Then $\lambda_i$ converges to $\lambda$ in Thurston's geometric topology but not in Hausdorff distance.
TODO: Include a picture
\end{example}

\begin{example}\label{Hausdorff does not imply Thurston}
Consider the sequence of continuous functions $f_i: \RR \to \RR$ where $f_i$ is constant away from $[0, 1/i^2]$, is $0$ on $(-\infty, 0]$, is $1/i$ on $[1/i^2, \infty)$, and is linear on $[0, 1/i^2]$.
By smoothing out the graphs of $f_i$ appropriately one obtains a sequence of $C^1$ curves $\gamma_i$ in $\RR^2$, whose geodesic curvatures blow up near $(0, 0)$, and if we set $\lambda_i$ to be the union of $\gamma_1, \dots, \gamma_i$, then $(\lambda_i)$ is a sequence of laminations which converges in Hausdorff distance but not in Thurston's geometric topology.
TODO: Include a picture.
\end{example}

%%%%%%%%%%%%%%%%%%%%%%%%%%%%%%%%%%%

\section{The weak topology of measures}
Another topology in common use is the weak topology on measures on the space of measured laminations, due to Thurston \cite[Chapter 8]{thurston1998minimal}.
To state it, we recall the notion of a transverse measure.
These are certain Radon measures on the leaf spaces; to study them it will be useful to recall some facts about Radon measures on Polish spaces TODO Cite it.

\subsection{Preliminaries}
Let $X$ be a Polish space.
The space $C_0(X)$ of continuous functions $f: X \to \RR$ such that $\lim_{x \to \infty} f(x) = 0$ is canonically isomorphic to the direct limit $\varinjlim C(Y)$ of continuous functions $f: Y \to \RR$ where $Y$ ranges over all compact subsets of $X$ and the bonding maps are inclusions.
Since $C(Y)$ is a Banach space, the identification $C_0(X) = \varinjlim C(Y)$ endows $C_0(X)$ with the structure of a locally convex space.
If $X$ is in fact compact, then this direct limit is trivial and so $C(X) = C_0(X)$.
Its dual $C_0(X)'$ is canonically isomorphic to the space of signed Radon measures on $X$.

\begin{definition}
The weak topology on $C_0(X)'$ is known as the \dfn{weak topology of measures}.
\end{definition}

Unpacking the definitions, a sequence $(\mu_i)$ of Radon measures converges to $\mu$ in the weak topology of measures iff for every compact $Y \subseteq X$ and every continuous function $f: Y \to \RR$,
$$\lim_{i \to \infty} \int_Y f \dif \mu_i = \int_Y f \dif \mu.$$
The same thing works for currents.
The advantage of the weak topology is its compactness.

\begin{proposition}[Prokohorov compactness theorem]
If $X$ is a compact Polish space, then any closed set $S \subseteq C(X)'$ such that $\sup_{\mu \in S} \mu(X) < \infty$ is compact.
\end{proposition}
\begin{proof}
TODO Cite 
\end{proof}

Having topologized the space of Radon measures, we define what it means for such a measure to be transverse to a lamination:

% If we have a second locally compact Polish space $Y$, equipped with a Radon measure $\mu \in C_0(Y)'$ and a Borel map $\pi: Y \to X$, we can view $Y$ as a ``measure-theoretic fiber bundle'' over $X$ by equipping $X$ with the pushforward measure
% $$\pi_* \mu(E) := \mu(\pi^{-1}(E)).$$

% \begin{lemma}[disintegration theorem]
% Each fiber $Y_x := \pi^{-1}(x)$ naturally carries the structure of a measured space, where the measure $\mu_x$ on $Y_x$ characterized by the formula
% $$\int_Y f \dif \mu = \int_X \int_{Y_x} f(y) \dif \mu_x(y) \dif \pi_* \mu(x)$$
% for every $f \in C_0(Y)$.
% \end{lemma}
% TODO: Cite this, if we actually need it

\begin{definition}
Let $\lambda$ be a lamination with atlas $A$.
A \dfn{transverse measure} to $\lambda$ consists of Radon measures $\mu_\alpha$ with $\supp \mu_\alpha = K_\alpha$, $\alpha \in A$, such that each transition map $\psi_{\alpha \beta}$ is measure-preserving:
$$\mu_\alpha|_{K_\alpha \cap K_\beta} = \psi_{\alpha \beta}^* (\mu_\beta|_{K_\alpha \cap K_\beta}).$$
The pair $(\lambda, \mu)$ is called a \dfn{measured lamination}.
\end{definition}

One must be careful here: in \cite{thurston1979geometry}, it is assumed that $\supp \mu_\alpha = K_\alpha$, but in \cite{daskalopoulos2020transverse}, it is only assumed that $\supp \mu_\alpha \subseteq K_\alpha$.
In particular, not every lamination admits a transverse measure:

\begin{example}
Let $\lambda$ be the pathological lamination of Example \ref{two geodesics}.
Let $\mu$ be transverse to $\lambda$, and let $F_\alpha$ be a flow box centered on a point of $\gamma_1$.
Then up to homeomorphism we may identify $K_\alpha$ with the one-point compactification of $\NN$, where $\infty$ corresponds to $\gamma_1$, and each of the natural numbers corresponds to a component of the intersection of $\gamma_2$ with the image of $F_\alpha$.
TODO: Make a picture.

The condition that $\mu_\alpha$ has full support implies that for each $n \in \NN$, $\mu_\alpha(\{n\}) > 0$, and the condition that $\psi_{\alpha \beta}$ is measure-preserving implies that $\mu_\alpha(\{n\})$ is independent of $n$, since $n, n + 1$ are both indices of the same leaf.
So any neighborhood of $\infty$ has infinite measure, which contradicts that $\mu_\alpha$ is a Radon measure on the compact set $K_\alpha$.
\end{example}

%%%%%%%%%%%%%%%%%%%%%%%%%%%%%%%%%%

\subsection{Ruelle-Sullivan currents}
The definition of transverse measure in terms of Radon measures on $K_\alpha$ is convenient because $K_\alpha$ is compact.
However, the definition is not intrinsic, and this causes problems when considering questions of convergence: the fact that the flow boxes of a convergent sequence of measured laminations converge should be a consequence of, not a part of, the definition!

To rectify this, we first observe that in the definition of a transverse measure, we cannot define a transverse measure to be one on the underlying manifold $M$ itself.
Indeed, Lebesgue measure is ``transverse'' to all foliations; thus such a definition forgets the ``direction'' the measure points in.
However, the notion of Ruelle-Sullivan current allows us to speak of a measure-theoretic object on $M$ which has a well-defined local product structure.

\begin{definition}
A lamination is \dfn{oriented} if one can choose its transition maps to all be orientation-preserving.
\end{definition}

It is clear that a lamination $\lambda$ is locally orientable, since if one replaces $M$ by a small open set, then $\lambda$ has a global flow box.

\begin{definition}
Let $(\lambda, \mu)$ be a measured oriented lamination and let $(\chi_\alpha)_{\alpha \in A}$ be a subordinate partition of unity.
The \dfn{Ruelle-Sullivan current} associated to $(\lambda, \mu)$ is defined for all compactly supported $d-1$-forms $\varphi$ by
\begin{equation}\label{RS current}
\int_M T_\mu \wedge \varphi := \sum_{\alpha \in A} \int_{K_\alpha} \left[\int_{\RR^{d - 1} \times \{k\}} (F_\alpha^{-1})^* (\chi_\alpha \varphi) \right] \dif \mu_\alpha(k).
\end{equation}
\end{definition}

\begin{lemma}
The Ruelle-Sullivan current $T_\mu$ is well-defined; it is honestly a $d-1$-current, and does not depend on the choice of partition of unity.
Moreover, $\dif T_\mu = 0$.
\end{lemma}
\begin{proof}
We first claim that the right-hand side of (\ref{RS current}) is always finite, and is continuous in $\varphi$.
In fact, possibly after refining $(\chi_\alpha)$, we may assume that it is a locally finite partition of unity.
In particular, we just need to check the continuity in a single flow box:
$$\left|\int_{K_\alpha} \left[\int_{\RR^{d - 1} \times \{k\}} (F_\alpha^{-1})^* (\chi_\alpha \varphi) \right] \dif \mu_\alpha(k)\right| \leq \int_{K_\alpha} \int_{\RR^{d - 1} \times \{k\}} |(F_\alpha^{-1})^* (\chi_\alpha \varphi)| \dif \mu_\alpha(k).$$
The inner integral is controlled by $||\varphi||_{C^0(U_\alpha)}$ where $U_\alpha$ is the image of $F_\alpha$.
The outer integral is then well-defined because it is against a Radon measure.

The independence of choices is like in \cite[Theorem 8.2]{daskalopoulos2020transverse} TODO.
Also if a $d-2$-form $\psi$ has compact support in a single flow box, then
$$\int_{\RR^{d - 1} \times \{k\}} (F_\alpha^{-1})^* \dif \psi = \int_{\RR^{d - 1} \times \{k\}} \dif((F_\alpha^{-1})^* \psi) = 0$$
by Stokes' theorem, so 
\begin{align*}
\int_M \dif T_\mu \wedge \psi &= -\int_M T_\mu \wedge \dif \psi \\
&= -\int_{K_\alpha} \int_{\RR^{d - 1} \times \{k\}} (F_\alpha^{-1})^* \dif \psi \dif \mu_\alpha(k) = 0. \qedhere
\end{align*}
\end{proof}

Though (\ref{RS current}) is the more traditional way of stating the definition of a Ruelle-Sullivan current, there is a more intrinsic way as well.
We first observe that if $\mu$ is a transverse measure, then $\mu$ defines a measure on $\supp \lambda$: in each flow box $F_\alpha$, an open set $U$ has measure 
\begin{equation}\label{transverse measure of an open set}
\mu(U) := \int_{K_\alpha} |F_\alpha(\RR^{d - 1} \times \{k\}) \cap U| \dif \mu_\alpha(k).
\end{equation}

\begin{lemma}
For an oriented measured lamination $(\lambda, \mu)$, the polar decomposition of $T_\mu$ is 
\begin{equation}\label{polar ruelle sullivan}
T_\mu = \normal_\lambda \mu.
\end{equation}
\end{lemma}
\begin{proof}
For an open set $U \subseteq M$ in a flow box $F_\alpha$, the total variation measure $|T_\mu|$ satisfies
$$|T_\mu|(U) = \sup_{||\varphi||_{C^0} \leq 1} \int_{K_\alpha} \int_{\RR^{d - 1} \times \{k\}} \varphi \dif \mu_\alpha(k)$$
where the supremum ranges over $d-1$-forms $\varphi$ with compact support in $U$.
However, $\star \normal_\lambda$ is the Riemannian measure on $F_\alpha(\RR^{d - 1} \times \{k\})$, so
$$\int_{\RR^{d - 1} \times \{k\}} \varphi \leq \int_{\RR^{d - 1} \times \{k\}} (F_\alpha^{-1})^*(\star \normal_\lambda).$$
Since $||\normal^\lambda||_{C^0} = 1$, it follows that a sequence of cutoffs of $\star \normal_\lambda$ to more and more of $U$ is a maximizing sequence.
Therefore $\normal_\lambda$ is the polar part of (\ref{polar ruelle sullivan}), and
$$|T_\mu|(U) = \int_{K_\alpha} \int_{\RR^{d - 1} \times \{k\}} (F_\alpha^{-1})^*(1_U \star \normal_\lambda) \dif \mu_\alpha(k).$$
The inner integral is the Riemannian measure of $F_\alpha(\RR^{d - 1} \times \{k\}) \cap U$, so by (\ref{transverse measure of an open set}), $|T_\mu| = \mu$.
\end{proof}

We are now ready to define the weak topology of measures.

\begin{definition}
A sequence of measured laminations $(\lambda_i, \mu_i)$ converges in the \dfn{weak topology of measures} to a measure $(\lambda, \mu)$ if it is possible to cover $M$ by open sets $(U_\beta)$ so that in each $U_\beta$, $\lambda$ is orientable, $\lambda_i$ is eventually orientable, and we can choose orientations in $U_\beta$ and a subordinate partition of unity $(\chi_{\alpha \beta})$ in $U_\beta$ such that, if we choose $\lambda_i, \lambda$ to be cooriented in $U_\beta$, the local Ruelle-Sullivan currents $T_{\mu_i}^\beta$ converge to $T_\mu^\beta$ in the weak topology of measures.
\end{definition}

Concretely, this means that after replacing $M$ with an open set which is so small that $\lambda_i, \lambda$ all share a common orientation, one has 
$$\lim_{i \to \infty} \int_M T_{\mu_i} \wedge \varphi = \int_M T_\mu \wedge \varphi$$
for every compactly supported $d-1$-form $\varphi$.
If we can in addition choose identical flow boxes of $\lambda_i, \lambda$, then the characterization is even simpler:

\begin{lemma}
If there exists a common atlas $(F_\alpha)$ of flow boxes for $\lambda_i$ and $\lambda$, and $\mu_i, \mu$ are transverse to $\lambda_i, \lambda$, then $(\lambda_i, \mu_i) \to (\lambda, \mu)$ in the weak topology of measures iff for every $\alpha \in A$, $\mu_{i\alpha} \to \mu_\alpha$ in the weak topology of measures.
\end{lemma}
\begin{proof}
Since they have common flow boxes, we can realize a test function $f$ for $\dif \mu_\alpha$ by choosing $\star \varphi$ to be conormal to the leaves.
Or we can build a test form $\varphi$ by noticing that only the Hodge dual to the conormal part matters, and the conormal part is given by $f$.
\end{proof}

%%%%%%%%%%%%%%%%%%%%%%%%%%%%%

\subsection{Measures on transverse curves}
Here is another characterization of the weak topology of measures in terms of transverse curves.

\begin{definition}
For a flow box $F: I \times \RR^{d - 1} \to M$, we define its \dfn{labelling map}
$$\Pi_F: F(I \cap \RR^{d - 1}) \to I$$
to be the projection of $F^{-1}$ onto $I$:
% https://q.uiver.app/?q=WzAsMyxbMCwwLCJGKEkgXFxjYXAgXFxtYXRoYmIgUl57ZCAtIDF9KSJdLFsyLDAsIkkgXFxjYXAgXFxtYXRoYmIgUl57ZCAtIDF9Il0sWzIsMiwiSSJdLFswLDEsIkZeey0xfSJdLFsxLDIsIiIsMCx7InN0eWxlIjp7ImhlYWQiOnsibmFtZSI6ImVwaSJ9fX1dLFswLDIsIlxcUGlfRiIsMl1d
\[\begin{tikzcd}
	{F(I \cap \mathbb R^{d - 1})} && {I \cap \mathbb R^{d - 1}} \\
	\\
	&& I
	\arrow["{F^{-1}}", from=1-1, to=1-3]
	\arrow[two heads, from=1-3, to=3-3]
	\arrow["{\Pi_F}"', from=1-1, to=3-3]
\end{tikzcd}\]
\end{definition}

\begin{definition}
For a smooth path $\gamma: I \to M$ and a lamination $\lambda$, we say that $\gamma$ is \dfn{transverse} to $\lambda$ if, for every $t \in I$ such that $\gamma(t) \in \supp \lambda$, there exists a flow box $F$ in a neighborhood of $\gamma(t)$ and a neighborhood $J \subseteq I$ of $t$ such that
$$\Pi_F \circ \gamma|_J: J \to I$$
is a topological embedding, i.e. a homeomorphism onto its image.
If $\Pi_F \circ \gamma|_J$ is in addition increasing, we say that $\gamma$ is \dfn{positively transverse}.

% By a \dfn{transverse homotopy} of transverse curves to $\lambda$ we mean a homotopy of curves $H: I^2 \to M$ such that every curve $\gamma_s := H(s, \cdot)$ is transverse to $\lambda$, and if $H(s, t)$ is an element of some leaf $N$ of $\lambda$, then so is $H(s', t)$ for any $s'$.
\end{definition}

\begin{definition}
For a transverse curve $\gamma$ to a measured lamination $(\lambda, \mu)$, we get a Radon measure $\gamma^! \mu$ on $I$, as follows.
Near $t \in I$, choose a flow box $F$ near $\gamma(t)$ and a neighborhood $J$ of $t$, such that $\Pi_F \circ \gamma|_J$ is a topological embedding.
Then
$$(\gamma^! \mu)|_J := ((\Pi_F \circ \gamma|_J)^{-1})_* \mu_F.$$
\end{definition}

The measure $\gamma^! \mu$ is well-defined, essentially since the transition maps are measure-preserving (so the only thing that actually matters is the weight assigned to each leaf that $\gamma$ passes through).

% \begin{lemma}
% Let $\mu$ be a transverse measure and $\gamma$ a transverse curve.
% Then if $\rho$ is transversely homotopic to, or a reparametrization of, $\gamma$, then $\mu^! \rho = \mu^! \gamma$.
% \end{lemma}
% \begin{proof}
% TODO; see \cite[\S7]{daskalopoulos2020transverse}.
% \end{proof}

% \begin{lemma}
% Let $\gamma$ be a positively transverse curve to an oriented geodesic lamination $\lambda$.
% Then up to transverse homotopy, we may assume that $\gamma$ is $C^1$ and satisfies
% $$(\gamma', \normal_\lambda) \geq (1 - \varepsilon) |\gamma'|$$
% where $\varepsilon > 0$ is arbitrary.
% \end{lemma}
% \begin{proof}
% TODO
% \end{proof}

% \begin{lemma}
% Let $\gamma: I \to M$ be a $C^1$ curve with $|\gamma'| > 0$, and let $(\lambda, \mu)$ be an oriented measured geodesic lamination in an oriented measured flow box $F_\alpha$. The following are equivalent:
% \begin{itemize}
% \item $\gamma$ is positively transverse to $\lambda$.
% \item For every $t \in I$ such that $\gamma(t) \in \supp \lambda$, it is possible to extend $\gamma'$ to a continuous vector field $X$ on $B(\gamma(t), \varepsilon)$ for arbitrarily small $\varepsilon > 0$, and
% $$N_k(\varepsilon) := F_\alpha(\{k\} \times \RR^{d - 1}) \cap B(\gamma(t), \varepsilon),$$
% then 
% \begin{equation}\label{quantitative transversality}
% \int_{B(\gamma(t), \varepsilon)} T_\lambda \wedge \star X^\flat \geq \frac{9}{10} \int_{K_\alpha} \left[\int_{N_k(\varepsilon)} |X| \star_{N_k(\varepsilon)} 1\right] \dif \mu_\alpha(k).
% \end{equation}
% \end{itemize}
% \end{lemma}
% \begin{proof}
% Let $\star_k := \star_{N_k(\varepsilon)}$.
% If $\gamma$ is positively transverse, then up to homotopy we may assume that $(\gamma', \normal_\lambda) \geq .999 |\gamma'|$.
% When we extend $\gamma'$ to a continuous vector field $X$, we may then continue to enforce $(X, \normal_\lambda) \geq .998 |X|$, since $\normal_\lambda$ is continuous.
% The tangential part $X_k$ to $N_k(\varepsilon)$ satisfies 
% $$X = (X, \normal_\lambda) \normal_\lambda^\sharp + X_k$$
% on $N_k(\varepsilon)$. In particular,
% $$|X_k| \leq \sqrt{1 - .998^2} |X| \leq .07 |X|.$$
% Since $\star \normal_\lambda$ is the Riemannian measure $\star_k 1$ on $N_k$, we obtain 
% $$\int_{N_k(\varepsilon)} \star X^\flat \geq \int_{N_k(\varepsilon)} \star_k (.998 - .07)|X| \geq \int_{N_k(\varepsilon)} \star_k .9|X|.$$
% Integrating in $k$, we obtain (\ref{quantitative transversality}).

% Conversely, if there exist $X_\varepsilon$ extending $\gamma'$ and satisfying (\ref{quantitative transversality}) for each small $\varepsilon > 0$, then taking the limit inferior as $\varepsilon \to 0$ of (\ref{quantitative transversality}), we see that $\gamma'(t)^\flat$ forms a angle with $\normal_\lambda(\gamma(t))$ with positive cosine, so $\gamma$ is positively transverse.
% \end{proof}

\begin{lemma}
Let $(\lambda, \mu)$ be a measured oriented lamination, $\gamma$ a curve, and $\gamma(t) \in \supp \lambda$.
Then $\gamma$ is positively transverse to $\lambda$ at $\gamma(t)$ iff 
\begin{equation}\label{transverse means rs}
\liminf_{\varepsilon \to 0} \frac{1}{\mu(B(\gamma(t), \varepsilon))} \int_{B(\gamma(t), \varepsilon)} T_\mu \wedge \star (\gamma')^\flat > 0.
\end{equation}
\end{lemma}
\begin{proof}
We work in flow box coordinates $(k, y)$, so that the labelling map $\Pi$ is $\Pi(k, y) = k$.
In such coordinates, leaves $N$ take the form $\{k = k_N\}$ for some $k_N$, so $\dif k$ is conormal to each leaf.
In particular, $\normal_\lambda = (g^{kk})^{-1/2} \dif k$.
Also $(\Pi \circ \gamma)' = (\dif k, \gamma')$, so (since $(g^{kk})^{-1/2}$ is positive), $\gamma$ is positively transverse at $\gamma(t)$ iff
\begin{equation}\label{transverse means dk}
(\normal_\lambda, \gamma')(\gamma(t)) > 0.
\end{equation}
But $\gamma(t) \in \supp \mu$, and $\mu$ is a Radon measure.
So by the Lebesgue differentiation theorem, (\ref{transverse means dk}) happens iff 
$$\liminf_{\varepsilon \to 0} \frac{1}{\mu(B(\gamma(t), \varepsilon))} \int_{B(\gamma(t), \varepsilon)} (\normal_\lambda, \gamma') \dif \mu > 0.$$
Applying the polar decomposition (\ref{polar ruelle sullivan}) of $T_\mu$ we conclude that this last condition is equivalent to (\ref{transverse means rs}).
\end{proof}

\begin{proposition}
Let $(\lambda_i, \mu_i) \to (\lambda, \mu)$ in the weak topology of measures, and let $\gamma$ be transverse to $\lambda$.
Then for all sufficiently large $i$, $\gamma$ is transverse to $\lambda_i$.
\end{proposition}
\begin{proof}
TODO, clean this up. But it's basically right.
Let $t, t_i \in I$ satisfy $\gamma(t_i) \in \supp \lambda_i$, $\gamma(t) \in \supp \lambda$, and $t_i \to t$, and $\psi := \star (\gamma')^\flat$.
Let $B_\varepsilon^i := B(\gamma(t_i), \varepsilon)$.
Then by the portmanteau theorem,
$$\liminf_{i \to \infty} \frac{1}{\mu_i(B_\varepsilon^i)} \geq \frac{1}{2\mu(B_\varepsilon)}.$$
So by (\ref{transverse means rs}), 
$$\liminf_{\varepsilon \to 0} \liminf_{i \to \infty} \frac{1}{\mu_i(B_\varepsilon^i)} \int_{B_\varepsilon^i} T_{\mu_i} \wedge \psi \geq \liminf_{\varepsilon \to 0} \frac{1}{2\mu(B_\varepsilon)} \int_{B_\varepsilon} T_\mu \wedge \psi > 0.$$
To commute the limits, we fix $\delta > 0$ and $\varepsilon^* > 0$ such that if $\varepsilon \leq \varepsilon^*$, then
$$\liminf_{i \to \infty} \frac{1}{\mu_i(B_\varepsilon^i)} \int_{B_\varepsilon} T_{\mu_i} \wedge \psi \geq \delta.$$
In particular, after finitely many $i$, 
\begin{equation}\label{need to take a liminf}
\frac{1}{\mu_i(B_\varepsilon^i)} \int_{B_\varepsilon^i} T_{\mu_i} \wedge \psi \geq \frac{\delta}{2}.
\end{equation}
Since $\varepsilon \leq \varepsilon^*$ is arbitrary, (\ref{need to take a liminf}) still holds if we take a limit inferior in $\varepsilon$ on the left-hand side.
Applying (\ref{transverse means rs}) in reverse, we see that $\gamma$ is transverse to $\lambda_i$ at $\gamma(t_i)$.
By continuity of
$$s \mapsto \frac{1}{\mu_i(B(\gamma(s), \varepsilon))} \int_{B(\gamma(s), \varepsilon)} T_{\mu_i} \wedge \psi,$$
it follows that $\gamma$ is transverse to $\lambda_i$ at $\gamma(s)$ for all $s$ satisfying $|s - t| < \rho_t$ with $\gamma(s) \in \supp \lambda_i$, and all $i \geq i_t$.
The sets $\{s \in I: |s - t| < \rho_t\}$ form an open cover of the compact set $\gamma^{-1}(\supp \lambda)$, so we can find $i^*$ independent of $t$ such that if $\gamma(s) \in \supp \lambda_i$ is close to a point of $\supp \lambda$ and $i \geq i^*$, then $\gamma$ is transverse to $\lambda_i$ at $\gamma(s)$.
\end{proof}

%%%%%%%%%%%%%%%%%%%%%%%%%%%%%%%%%%%%%%%
\subsection{Weak topology of measures versus Thurston's geometric topology}
We now show that for a geodesic lamination, convergence in the weak topology of measures implies convergence in Thurston's geometric topology.
Thurston claimed this fact \cite[Proposition 8.10.3]{thurston1979geometry} but his proof left something to be desired as it did not justify why the limit is geodesic, or why the convergence respects the normal vectors.

\begin{lemma}
Let $(\lambda_i, \mu_i)$ be measured geodesic laminations.
If $(\lambda_i, \mu_i) \to (\lambda, \mu)$ in the weak topology of measures, then for every $x \in \supp \lambda$ there exists $x_i \in \supp \lambda_i$ such that $x_i \to x$.
\end{lemma}
\begin{proof}
For each $\varepsilon > 0$, choose an $d-1$-form $\varphi_\varepsilon$ in $L^\infty_\cpt(B(x, \varepsilon))$ such that for each leaf $\tilde N$ of $\lambda$ which meets $B(x, \varepsilon/2)$,
\begin{equation}\label{test form is conormal}
\int_{B(x, \varepsilon/2) \cap \tilde N} \varphi_\varepsilon \wedge \normal_{\tilde N} \geq 1,
\end{equation}
and choose a flow box $F_\alpha$ for $\lambda$ at $x$.
For any $\varepsilon$ small enough, $\lambda_i, \lambda$ are coorientable, so we can choose Ruelle-Sullivan currents $T_{\mu_i}, T_\mu$ defined on $B(x, \varepsilon)$ such that for $i$ large enough, 
\begin{align*}
\int_{B(x, \varepsilon)} T_{\mu_i} \wedge \varphi_\varepsilon
&\geq \frac{1}{2} \int_{B(x, \varepsilon)} T_\mu \wedge \varphi_\varepsilon\\
&\geq \frac{1}{2} \int_{K_\alpha} \left[\int_{\RR^{d - 1} \times \{k\}} (F_\alpha^{-1})^* (1_{B(x, \varepsilon/2)} \varphi_\varepsilon)\right] \dif \mu_\alpha(k).
\end{align*}
By (\ref{test form is conormal}), there exists $k_0 > 0$ such that for $|k| < k_0$ in $K_\alpha$,
$$\int_{\RR^{d - 1} \times \{k\}} (F_\alpha^{-1})^* (1_{B(x, \varepsilon/2)} \varphi_\varepsilon) \geq 1,$$
and since $\mu_\alpha$ is supported near $0$ in $K_\alpha$ it follows that $\int T_{\mu_i} \wedge \varphi_\varepsilon \gtrsim 1$.
Therefore there exists
$$x_i \in B(x, \varepsilon) \cap \supp T_{\mu_i} = B(x, \varepsilon) \cap \supp \lambda_i.$$
Since $\varepsilon$ was arbitrary, we conclude $x_i \to x$.
\end{proof}

\begin{lemma}
Let $(u_i)$ be a sequence of functions of least gradient which is bounded in $L^1$, and $T$ a current such that $\dif u_i \to T$ weakly.
Then there exists a function $u$ of least gradient such that along a subsequence, $u_i \to u$ weakly in $BV$, and $\dif u = T$.
\end{lemma}
\begin{proof}
By the Miranda stability theorem, $(u_i)$ has a subsequential limit $u$ in $L^1$, which has least gradient.
For any compactly supported $d-1$-form $\varphi$,
\begin{align*}
\int_M T \wedge \varphi
&= \lim_{i \to \infty} \int_M \dif u_i \wedge \varphi 
= -\lim_{i \to \infty} \int_M u_i \dif \varphi = - \int_M u \dif \varphi.
\end{align*}
where the last equality is because the Radon-Nikod\'ym derivative of $\dif \varphi$ with respect to Lebesgue measure is in $L^\infty$ and hence $L^1(\dif \varphi)$ is a weaker space than $L^1$. The last expression here is $\int_M \dif u \wedge \varphi$.
So $T = \dif u$ and $\dif u_i \to \dif u$ in the weak topology of measures; that is, $u_i \to u$ weakly in $BV$.
\end{proof}

\begin{lemma}
Let $(\lambda_i, \mu_i)$ be measured geodesic laminations.
If $(\lambda_i, \mu_i) \to (\lambda, \mu)$ in the weak topology of measures, then $\lambda$ is geodesic.
\end{lemma}
\begin{proof}
Let $x \in \supp \lambda$ and $r > 0$ such that $B := B(x, r)$ is contractible.
In $B$, we can write $T_{\mu_i} = \dif u_i$ for some sequence of functions of least gradient $u_i \in BV(B)$.
Since $u_i$ is only defined up to a constant, we impose $\int_M \star u_i = 0$, so by Poincar\'e's inequality,
$$||u_i||_{L^1(B)} \lesssim r\int_B \star |T_{\mu_i}| \lesssim 1$$
where the last bound is because $(T_{\mu_i})$ is compact in the weak topology of measures.
So $u_i \to u$ where $u$ has least gradient and satisfies $T_\mu = \dif u$.
So by \cite{BackusFLG}, the leaf $N$ containing $x$ is a geodesic and $x$ is not an endpoint of $N$.
But $x$ was arbitrary, so $N$ is in fact complete.
\end{proof}

\begin{proposition}\label{measured implies Thurston}
Let $(\lambda_i, \mu_i)$ be measured geodesic laminations.
If $(\lambda_i, \mu_i) \to (\lambda, \mu)$ in the weak topology of measures, then $\lambda$ is a geodesic lamination and $\lambda_i \to \lambda$ in Thurston's geometric topology.
\end{proposition}
\begin{proof}
Suppose that $x \in N$, where $N$ is a leaf of $\lambda$, and $x_i \to x$ where $x_i \in N_i$, $N_i$ a leaf of $\lambda_i$.
It remains to show that $\normal_i := \normal_{N_i}(x_i)$ converges to $\normal := \normal_N(x)$ in the cosphere bundle $S'M$.
If this fails, then we work in normal coordinates based at $x$, so we can think of points of $M$ as vectors in $\RR^d$, and points of $S'M$ as unit vectors in $\RR^d$.
In such coordinates we may view $N$ as the first axis.
Possibly after taking a subsequence of $(\lambda_i)$, we may assume that no matter what sequence $(x_i)$ we choose,
$$|\sin \angle(\normal_i, \normal)| \geq \varepsilon.$$
The geodesic curvature of $N_i$ with respect to the euclidean metric on the tangent space is bounded independently of $i$ in terms of the scalar curvature $R$ of $M$ near $x$, so there exists $r = r(R) > 0$ independent of $i$ such that $N_i$ \dfn{avoids cones} in the sense that for every $v \in \RR^d$ such that $0 < |v| < r$ and $|\sin v| < \varepsilon/2$, $v + x_i$ does not lie in $N_i$.
After shrinking $\varepsilon$, we may assume that $\varepsilon < \max(r/2, 1/100)$.

To obtain a contradiction, we choose $y$ to lie on $N$ and satisfy $\dist(x, y) = \varepsilon$.
For $i$ large, $|x_i| < \varepsilon^2/10$, so
$$0 < \varepsilon - \varepsilon^2 \leq |y - x_i| < 2\varepsilon = r$$
and hence (using a superscript $j$ to indicate the $j$th coordinate)
$$|y^1 - x_i^1| \geq |y - x_i| - |x_i^2| \geq \varepsilon - 2\varepsilon^2.$$
Consider the triangle $\Delta$ whose vertices are $x_i, y$, and $z_i := (y^1, x_i^2)$.
Then $\Delta$ is a right triangle, and its smallest angle $\theta_\Delta$ satisfies
$$|\sin \theta_\Delta| = \frac{|x_i^2|}{|y^1 - x_i^1|} < \frac{\varepsilon^2}{\varepsilon - 2\varepsilon^2} < \frac{\varepsilon}{4}.$$
Thus for every large $i$, $y$ is contained in the cone avoided by $N_i$, and in fact for any $y_i \in N_i$, $\dist(y, y_i) \geq \varepsilon/4$.
But we argued above that any point of $N$ could be approximated by points of $N_i$ for $i$ large, so this is a contradiction.
\end{proof}


%%%%%%%%%%%%%%%%%%%%%%%

\section{Convergence in flow boxes}
Let $C^{1-}$ be the Fr\'echet space $\bigcap_{\alpha < 1} C^\alpha$, where $C^\alpha$ are H\"older spaces.
The following definition is one possible interpretation of the vague definition of \cite[Lemma II.1.2]{ColdingMinicozziV}.

\begin{definition}
A sequence $(\lambda_i)$ of laminations \dfn{converges on the level of flow boxes} to $\lambda$ if it converges in Thurston's geometric topology, and we can find atlases
$$F_{i\alpha}: U_\alpha \to [-1, 1] \times \RR^{d - 1}$$
of flow boxes for $\lambda_i$, and an atlas $F_\alpha: U_\alpha \to [-1, 1] \times \RR^{d - 1}$ of flow boxes for $\lambda$, such that $F_{i\alpha} \to F_\alpha$ and $F_{i\alpha}^{-1} \to F_\alpha^{-1}$ in $C^{1-}$.
\end{definition}

Just as for Thurston's geometric topology, convergence on the level of flow boxes does not satisfy uniqueness of limits, and hence we speak of \dfn{maximal limits} to allow us to speak of a unique limit of a sequence.

%%%%%%%%%%%%%%%%%%%%%%%%%
\subsection{Convergence in flow boxes versus the weak topology of measures}
Let $W^{1, \infty}_p(I, M)$ be the space of partial Lipschitz maps $I \to M$.
That is, an element of $W^{1, \infty}_p(I, M)$ consists of a set $S \subseteq I$ and a Lipschitz map $f: S \to M$.
We say that a sequence $(f_i)$ converges to $f \in W^{1, \infty}_p(I, M)$ if...

\begin{lemma}
There is an extension operator 
$$E: W^{1, \infty}_p(I, M) \to W^{1, \infty}(I, M)$$
such that $E$ is continuous, and $||(Ef)'||_{L^\infty} = ||f'||_{L^\infty}$.
\end{lemma}
\begin{proof}
???
\end{proof}

\begin{lemma}
If $\lambda_i \to \lambda$ in Thurston's geometric topology and $x \in N$, $N$ a leaf of $\lambda$, then for any sequence of leaves $N_i$ of $\lambda_i$ which contain $x_i \in N_i$ such that $x_i \to x$, $\normal_{N_i} \to \normal_N$ uniformly in a neighborhood of $x$.
\end{lemma}
\begin{proof}
By definition of Thurston's geometric topology, $\normal_{N_i}(x_i) \to \normal_N(x)$.
However, $N_i, N$ are geodesics, so $\normal_{N_i}, \normal_N$ are covariantly constant.
In particular they remain close on small sets.
\end{proof}

\begin{lemma}
If $\lambda$ is a geodesic lamination, and $\gamma$ is a transverse curve, there exists a Lipschitz unit-length vector field $X$ along $\gamma$ such that if $\gamma(t) \in N$ for some leaf $N$ of $\lambda$, then $X(\gamma(t))$ is tangent to $N$.
Moreover, the map that sends a geodesic lamination to a Lipschitz vector field is continuous for Thurston's geometric topology.
\end{lemma}
\begin{proof}
By Theorem \ref{regularity theorem}, $\normal_\lambda$ is Lipschitz, so $(\normal_\lambda^\sharp)^\perp$ is Lipschitz, unit-length, tangent to the leaves of $\lambda$, and defined on $\supp \lambda \cap \gamma(I)$.
We then let
\begin{equation}\label{tangent frame to a lamination}
	X := E(\normal_\lambda^\sharp)^\perp.
\end{equation}

To see the continuity, suppose that $\lambda_i \to \lambda$ in Thurston's geometric topology.
Let $x \in \supp \lambda$; by taking a subsequence we can find $x_i \in \supp \lambda_i$ such that $x_i \to x$ and $\normal_{\lambda_i}(x_i) \to \normal_\lambda(x)$.
So if $N_i, N$ are the leaves containing $x_i, x$, then $\normal_{N_i} \to \normal_N$ uniformly in a neighborhood of $x$.
Thus (TODO) $\normal_{\lambda_i} \circ \gamma \to \normal_\lambda \circ \gamma$ in $W^{1, \infty}_p(I, M)$, hence if we define $X_i$ by (\ref{tangent frame to a lamination}), $X_i \to X$ in $W^{1, \infty}(I, M)$.
\end{proof}

\begin{theorem}
Let $(\lambda_i, \mu_i)$ be measured geodesic laminations.
If $(\lambda_i, \mu_i) \to (\lambda, \mu)$ in the weak topology of measures, then $\lambda_i \to \lambda$ on the level of flow boxes.
\end{theorem}
\begin{proof}
By Proposition \ref{measured implies Thurston}, $\lambda_i \to \lambda$ in Thurston's geometric topology and $\lambda$ is a geodesic lamination.
We extend $\normal_\lambda^\sharp$ to a Lipschitz vector field $Y$ on $M$; then any integral curve $\gamma$ of $Y$ is transverse to $\lambda$.
In particular, we can cover $\supp \lambda$ by the images of curves whcih are transverse to $\gamma$.

For each such transverse curve $\gamma$, we choose a Lipschitz unit vector field $X$ along $\gamma$ which is tangent to the leaves of $\lambda$.
Then we set
$$F(k, y) := \exp_{\gamma(k)}(yX(\gamma(k))).$$
Since $\exp_{\gamma(k)}$ sends lines through the origin to geodesics through $\gamma(k)$, and $X$ is Lipschitz, $F$ is a flow box for $\lambda$.
Since $\gamma$ is eventually transverse to $\lambda_i$, we obtain flow boxes $F_i$ for each of the $\lambda_i$, with associated vector fields $X_i$ as well, such that $X_i \to X$ in $W^{1, \infty}$. Therefore $F_i \to F$ in $W^{1, \infty}$.
\end{proof}

This is a little suspicious; maybe the convergence really is in $W^{1, \infty}$ but I was just expecting it in $C^\alpha$. 
Maybe partial Lipschitz maps don't work the way I think they do... or maybe Colding--Minicozzi's theorem wasn't sharp.

Using the above prop we know that the normal vectors converge and the flow boxes are good, so we can maybe prove something like:

\begin{proposition}\label{measured implies transversality}
Let $(\lambda, \mu)$ and $(\lambda_i, \mu_i)$ be measured oriented laminations. The following are equivalent:
\begin{itemize}
\item $(\lambda_i, \mu_i) \to (\lambda, \mu)$ in the weak topology of measures.
\item For every positively transverse curve $\gamma$ to $\lambda$, and every sufficiently large $i$, $\gamma$ is transverse to $\lambda_i$, and $\gamma^! \mu_i \to \gamma^! \mu$ in the weak topology of measures.
\end{itemize}
\end{proposition}

%%%%%%%%%%%%%%%%%%%%%%%%%%%%%%
\subsection{Compactness}
\begin{definition}
A sequence of laminations $(\lambda_i)$ is \dfn{tight} if ... a sequence of transverse measures is \dfn{tight}...
\end{definition}

It follows from the definitions that if $M$ is closed, then every sequence of laminations in $M$ is tight.
A sequence of measured laminations $(\lambda_i, \mu_i)$ in a closed manifold such that $(\mu_i)_\gamma([-1, 1]) \lesssim 1$ independently of $i$ or the homotopy class $\gamma$ is also tight.

\begin{theorem}
Let $(\lambda_i)$ be a tight sequence of minimal laminations with bounded curvature. Then:
\begin{enumerate}
\item After passing to a subsequence, we may assume that $(\lambda_i)$ converges to a maximal limit $\lambda$ on the level of flow boxes, and hence in Thurston's geometric topology.
\item The limit $\lambda$ is a Lipschitz minimal lamination satisfying the same curvature bounds as $(\lambda_i)$.
\item If $\mu_i$ is a transverse measure to $\lambda_i$ such that $(\mu_i)$ is tight, then after passing to a further subsequence, we may assume that $(\lambda_i, \mu_i)$ in fact converges to some measured minimal lamination $(\lambda, \mu)$ in the weak topology of measures.
\end{enumerate}
\end{theorem}
\begin{proof}
By a density argument we may assume that $\lambda_i$ has finitely many leaves.
We now select $x_0 \in M$ and construct flow boxes $F_i$ on $B(x_0, r)$ where $r > 0$ is to be determined, but is at least smaller than the injectivity radius $R(x_0)$.

Let $\varepsilon > 0$.
After selecting $r$ small depending on $g$, $\varepsilon$, and $\sup_i ||\lambda_i||_{B(x_0, R(x_0))}$, and rescaling $g$, we may assume that the leaves $N_{i1}, \dots, N_{ik_i}$ have all second fundamental forms of size $\leq \varepsilon/2$ on $B(x_0, r)$.
Then if $r$ is small enough depending on $g$, the second fundamental forms with respect to the euclidean metric on $x_0$-normal coordinates are of size $\leq \varepsilon$ on $B(x_0, r)$.
Moreover, because there is a global Lipschitz normal vector field to $\lambda_i$, we can represent $N_{i1}, \dots, N_{ik_i}$ as graphs over the equatorial hyperplane in $B(x_0, r)$, possibly after applying a rotation $R_i$.
Thus we have coordinates $(x_i, y_i)$ on $B(x_0, r)$ where
$$N_{ij} = \{(y_i = f_{ij}(x_i))\}.$$
Somewhere in here we have to replace $B(x_0, r)$ with a cylinder, just like in the proof of de Giorgi's lemma.

By the maximum principle, the $f_{ij}$ satisfy $f_{ij}(x_i) = f_{ij'}(x_i)$ implies $j = j'$.
It follows that after applying a permutation to $\{1, \dots, k_i\}$ we may assume that $j < j'$ implies $f_{ij} < f_{ij'}$.
Then if we set $u_{ij} := f_{i,j+1} - f_{ij}$, then $u_{ij} > 0$ and we get a second-order operator $P_{ij}$ such that
$$\Delta u_{ij} = P_{ij} u_{ij}$$
and $||P_{ij}|| \lesssim \varepsilon$.
Then for $\delta > 0$, the Schauder and Harnack inequalities give
$$\delta r ||\dif u_{ij}||_{C^0(B_{\delta r})} \lesssim \sup_{B(\delta r)} u_{ij} \leq e^{c_1(\varepsilon) \delta^{c_2(\varepsilon)}} \inf_{B(\delta r)} u_{ij}.$$
We now set $\eta_{ij} := f_{ij}(0)$ and define for $\xi_i \in B_r$ and $\eta_i \in [\eta_{ij}, \eta_{i,j+1}]$
$$\varphi_i(\xi_i, \eta_i) := f_{ij}(\xi_i) + \frac{\eta_i - \eta_{ij}}{\eta_{i,j+1} - \eta_{ij}} f_{ij}(x_i).$$
That is, the change of coordinates
$$(x_i, y_i) = (\xi_i, \varphi_i(\xi_i, \eta_i))$$
flattens out the graphs of the $f_{ij}$ to hyperplanes.
Moreover
$$\dif \varphi_i = \dif f_{ij} + \frac{\eta_i - \eta_{ij}}{\eta_{i,j+1} - \eta_{ij}} \dif u_{ij} + \frac{u_{ij}}{\eta_{i,j+1} - \eta_{ij}} \dif \eta_i.$$
If we chose $R_i$ correctly, then $\dif f_{ij}$ is small, and if we chose $\varepsilon, \delta$ small enough then $\dif u_{ij}$ is small too.
Also we cooked up these coordinates so that $f_{ij} \approx \eta_{ij}$ on $B_{\delta r}$ if $\delta$ is small enough, thus the coefficient on the $\dif \eta_i$ is basically unity.
In particular $\dif \varphi_i \approx \dif \eta_i$ in $L^\infty$.

In conclusion, we can get a flow box $F_i$ by rotating the diffeomorphism $(x_i, y_i) = (\xi_i, \varphi_i(\xi_i, \eta_i))$ by $R_i$, and $F_i$ will have Lipschitz norm and conorm both very close to $1$.
So by a compactness argument we see that a subsequence converges in $C^{1-}$ to a flow box $F$.
The leaves are $C^{1-}$-limits of hypersurfaces with uniformly small second fundamental form so they are $C^{1-}$ as well.
But a $C^{1-}$ limit of minimal hypersurfaces is minimal, and also $C^\infty$.
Finally, Theorem \ref{regularity theorem} implies that we can upgrade from $C^{1-}$ regularity of $F$ to Lipschitz regularity.
The Prokohorov compactness theorem lets us upgrade it to convergence in the weak topology of measures.

TODO, That's very sketchy and there's a lot of details to fill in. But that's the idea.
\end{proof}

TODO: Can we upgrade this all the way to convergence in $BV$?
The converse holds anyways


%%%%%%%%%%%%%%%%%%%%%%%%%%%

\subsection{Construction of minimal laminations}
We conclude this paper by applying the above theory to prove a condition for a collection of minimal hypersurfaces to form a minimal lamination.
This condition will be used in the companion papers \cite{BackusFLG, DaskalopoulosPrep2}.

\begin{definition}
A \dfn{minimal partition} is a closed subset $\lambda$ of $M$ which has been partitioned into (disjoint, embedded, without boundary) minimal hypersurfaces, called the \dfn{leaves} of $\lambda$.
\end{definition}

\begin{theorem}
Let $\lambda$ be a minimal partition such that the second fundamental forms of the leaves of $\lambda$ are locally uniformly bounded.
Then $\lambda$ is a minimal lamination.
\end{theorem}
\begin{proof}
Let $(x_i)$ be a dense sequence in $\supp \lambda$, and let $N_i$ be the leaf containing $x_i$.
Then the union of $N_1, \dots, N_i$ is the support of a lamination $\lambda_i$.
By assumption, $||\lambda_i||_K \lesssim_K 1$.
So $(\lambda_i)$ has a minimal subsequential limit $\tilde \lambda$ on the level of flow boxes.
TODO we have to use Thurston's geometric topology to show that $\tilde \lambda = \lambda$.
\end{proof}

The above result is particularly useful when it is combined with the stable Bernstein theorem.\footnote{so-called because this theorem is essentially equivalent to the assertion that a stable two-sided minimal hypersurface in $\RR^d$ is a hyperplane.}

\begin{proposition}[stable Bernstein theorem]
Let $2 \leq d \leq 4$ and let $N$ be a two-sided stable minimal hypersurface in $B_r \subseteq M$, $r \lesssim 1$, where $M$ has bounded geometry and dimension $d$.
Then on $B_{r/2}$, $|\Two_N| \lesssim_M r^{-1}$.
\end{proposition}
\begin{proof}
If $d = 2$ this is trivial since $\Two_N$ has zero independent components and hence $\Two_N = 0$, if $d = 3$ then it was proven by Schoen \cite[Corollary 2.11]{colding2011course}, and if $d = 4$ it was proven by Chodosh--Li \cite{Chodosh2021}.
\end{proof}

\begin{corollary}
Let $\lambda$ be a minimal partition in $M$ whose leaves are two-sided and stable, where $M$ has dimension $2 \leq d \leq 4$.
Then $\lambda$ is a minimal lamination.
\end{corollary}

TODO: Construction of $\dif u$ on $\lambda$ from some data on $\lambda$


\printbibliography

\end{document}
