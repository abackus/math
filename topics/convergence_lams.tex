\documentclass[reqno,11pt]{amsart}
\usepackage[letterpaper, margin=1in]{geometry}
\RequirePackage{amsmath,amssymb,amsthm,graphicx,mathrsfs,url,slashed,subcaption}
\RequirePackage[usenames,dvipsnames]{xcolor}
\RequirePackage[colorlinks=true,linkcolor=Red,citecolor=Green]{hyperref}
\RequirePackage{amsxtra}
\usepackage{cancel}
\usepackage{tikz-cd}

% \setlength{\textheight}{9.3in} \setlength{\oddsidemargin}{-0.25in}
% \setlength{\evensidemargin}{-0.25in} \setlength{\textwidth}{7in}
% \setlength{\topmargin}{-0.25in} \setlength{\headheight}{0.18in}
% \setlength{\marginparwidth}{1.0in}
% \setlength{\abovedisplayskip}{0.2in}
% \setlength{\belowdisplayskip}{0.2in}
% \setlength{\parskip}{0.05in}
%\renewcommand{\baselinestretch}{1.05}

\title{Modes of convergence of minimal laminations}
\author{Aidan Backus}
\date{October 2022}

\newcommand{\NN}{\mathbf{N}}
\newcommand{\ZZ}{\mathbf{Z}}
\newcommand{\QQ}{\mathbf{Q}}
\newcommand{\RR}{\mathbf{R}}
\newcommand{\CC}{\mathbf{C}}
\newcommand{\DD}{\mathbf{D}}
\newcommand{\PP}{\mathbf P}
\newcommand{\MM}{\mathbf M}
\newcommand{\II}{\mathbf I}
\newcommand{\Hyp}{\mathbf H}
\newcommand{\Sph}{\mathbf S}
\newcommand{\Group}{\mathbf G}
\newcommand{\GL}{\mathbf{GL}}
\newcommand{\Orth}{\mathbf{O}}
\newcommand{\SpOrth}{\mathbf{SO}}
\newcommand{\Ball}{\mathbf{B}}

\newcommand*\dif{\mathop{}\!\mathrm{d}}

\DeclareMathOperator{\dist}{dist}
\DeclareMathOperator{\MeasLam}{MeasLam}
\DeclareMathOperator{\MinLam}{MinLam}
\DeclareMathOperator{\Lam}{Lam}
\DeclareMathOperator{\supp}{supp}

\newcommand{\Leaves}{\mathscr L}
\newcommand{\Hypspace}{\mathscr H}

\newcommand{\Two}{\mathrm{I\!I}}


\newcommand{\Hilb}{\mathcal H}
\newcommand{\Homology}{\mathrm H}
\newcommand{\normal}{\mathbf n}
\newcommand{\radial}{\mathbf r}
\newcommand{\evect}{\mathbf e}
\newcommand{\vol}{\mathrm{vol}}

\newcommand{\Bmu}{\boldsymbol \mu}
\newcommand{\Bnu}{\boldsymbol \nu}
\newcommand{\Blambda}{\boldsymbol \lambda}

\newcommand{\pic}{\vspace{30mm}}
\newcommand{\dfn}[1]{\emph{#1}\index{#1}}

\renewcommand{\Re}{\operatorname{Re}}
\renewcommand{\Im}{\operatorname{Im}}

\newcommand{\loc}{\mathrm{loc}}
\newcommand{\cpt}{\mathrm{cpt}}

\def\Japan#1{\left \langle #1 \right \rangle}

\newtheorem{theorem}{Theorem}[section]
\newtheorem{badtheorem}[theorem]{``Theorem"}
\newtheorem{prop}[theorem]{Proposition}
\newtheorem{lemma}[theorem]{Lemma}
\newtheorem{sublemma}[theorem]{Sublemma}
\newtheorem{proposition}[theorem]{Proposition}
\newtheorem{corollary}[theorem]{Corollary}
\newtheorem{conjecture}[theorem]{Conjecture}
\newtheorem{axiom}[theorem]{Axiom}
\newtheorem{assumption}[theorem]{Assumption}

\newtheorem{mainthm}{Theorem}
\renewcommand{\themainthm}{\Alph{mainthm}}

% \newtheorem{claim}{Claim}[theorem]
% \renewcommand{\theclaim}{\thetheorem\Alph{claim}}
\newtheorem*{claim}{Claim}

\theoremstyle{definition}
\newtheorem{definition}[theorem]{Definition}
\newtheorem{remark}[theorem]{Remark}
\newtheorem{example}[theorem]{Example}
\newtheorem{notation}[theorem]{Notation}

\newtheorem{exercise}[theorem]{Discussion topic}
\newtheorem{homework}[theorem]{Homework}
\newtheorem{problem}[theorem]{Problem}

\makeatletter
\newcommand{\proofpart}[2]{%
  \par
  \addvspace{\medskipamount}%
  \noindent\emph{Part #1: #2.}
}
\makeatother



\numberwithin{equation}{section}


% Mean
\def\Xint#1{\mathchoice
{\XXint\displaystyle\textstyle{#1}}%
{\XXint\textstyle\scriptstyle{#1}}%
{\XXint\scriptstyle\scriptscriptstyle{#1}}%
{\XXint\scriptscriptstyle\scriptscriptstyle{#1}}%
\!\int}
\def\XXint#1#2#3{{\setbox0=\hbox{$#1{#2#3}{\int}$ }
\vcenter{\hbox{$#2#3$ }}\kern-.6\wd0}}
\def\ddashint{\Xint=}
\def\dashint{\Xint-}

\usepackage[backend=bibtex,style=numeric]{biblatex}
\renewcommand*{\bibfont}{\normalfont\footnotesize}
\addbibresource{topics.bib}
\renewbibmacro{in:}{}
\DeclareFieldFormat{pages}{#1}


\begin{document}
\begin{abstract}
We collect several results relating different notions of convergence of laminations, especially minimal laminations.
We also give a condition for a partition of a closed set into minimal hypersurfaces to be a lamination.
\end{abstract}

\maketitle

%%%%%%%%%%%%%%%%%%%%%%%%%%%%%%%%%%%%%%%%%%%%%%%%%%%%%%%

% \tableofcontents

\section{Introduction}
The purpose of this paper is to record several modes of convergence for laminations, and in particular for minimal laminations, for use in the companion papers \cite{BackusFLG, DaskalopoulosPrep2}.
In such papers it is crucial that a disjoint family of minimal hypersurfaces without boundary -- which we refer to as a \dfn{minimal partition} -- is in fact a minimal lamination.
However this is false in general, as we will discuss, and so we will use the compactness theory discussed in this paper as well as the stable Bernstein theorem \cite{Schoen2016, Chodosh2021} to give a criterion under which a minimal partition is a lamination.

Most of the modes of convergence in this paper and their compactness theorems have been known to various sources \cite{ColdingMinicozziIV, ColdingMinicozziV, thurston1979geometry}.
Our contribution is to synthesize the known results, explaining under what circumstances different modes of convergence are stronger than others, as well as under what circumstances a sequence of laminations has a convergent subsequence.

%%%%%%%%%%%%%%%%%%%%%%%%

\subsection{Acknowledgements}
I would like to thank Georgios Daskalopolous for suggesting this project and for helpful discussions, and NSF...

%%%%%%%%%%%%%%%%%%%%%%%%%%%

\section{Preliminaries}
Throughout we write $I$ for a compact interval.

\begin{definition}
Let $S$ be a nonempty closed subset of a smooth manifold $M$.
\begin{enumerate}
\item A (codimension $1$) \dfn{flow box} for $S$ is a $C^0$ chart $F: I \times \RR^{d - 1} \to M$ and a closed set $K \subseteq I$ such that $S = F(K \times \RR^{d - 1})$, and for every $k \in K$, $F(\{k\} \times \RR^{d - 1})$ is $C^2$.
\item A \dfn{transition map} between two flow boxes $(F_\alpha, K_\alpha), (F_\beta, K_\beta)$ is a $C^0$ map
$$\psi_{\alpha \beta}: I \times \RR^{d - 1} \to I \times \RR^{d - 1}$$
such that
\begin{equation}\label{transition relation}
F_\alpha = F_\beta \circ \psi_{\alpha \beta},
\end{equation}
and in addition $\psi_{\alpha \beta}(K_\alpha) = K_\beta$.
\item A \dfn{laminar atlas} for $S$ consists of a cover of an open neighborhood of $S$ by images of compatible flow boxes.
\item A \dfn{lamination} $\lambda$ with support $S$ consists of the data of $S$ and a maximal laminar atlas for $S$.
\item A \dfn{leaf} of $\lambda$ consists of a subset of $S$ which locally takes the form $F_\alpha(\{k\} \times \RR^{d - 1})$, for some flow box $(F_\alpha, K_\alpha)$ and $k \in K_\alpha$.
\item If $N$ is a leaf of $\lambda$, which takes the form $F_\alpha(\{k\} \times \RR^{d - 1})$, we call $k \in K_\alpha$ the \dfn{label} of $N$ in $(F_\alpha, K_\alpha)$.
\item A \dfn{plaque} of $\lambda$ consists of a connected component of $M \setminus S$.
\end{enumerate}
\end{definition}

We assume that the leaves are $C^2$ in order to ensure that the normal vectors to each leaves are well-defined in $C^1$.
In \cite{Morgan88}, such laminations are called $C^2$ \dfn{along leaves}.
This is not the same thing as assuming that the lamination admits a $C^2$ atlas, as it may not be able to extend the normal vectors to each leaf to a $C^1$ vector field on $M$ even locally; see \S\ref{RegularitySec} for more precise assertions about regularity.
This assumption in particular ensures that the second fundamental forms, and hence mean curvatures, to each leaf are well-defined in $C^0$.

It seems rather difficult to say much about convergence of general laminations, as we will give several examples to illustrate.
However, the situation improves significantly if the laminations in question are minimal:

\begin{definition}
Let $M$ be a Riemannian manifold.
A lamination $\lambda$ is \dfn{minimal} if every leaf of $\lambda$ has zero mean curvature, and is \dfn{geodesic} if in addition $M$ is a surface.
\end{definition}

A lamination is geodesic iff its leaves are geodesics.

Control on the second fundamental forms will be a crucial assumption throughout this paper and so we make the following definition:

\begin{definition}
The \dfn{maximal curvature} of a lamination $\lambda$ is
$$R_\lambda := \sup_N ||\Two_N||_{C^0}$$
where $N$ ranges over leaves of $\lambda$ and $\Two_N$ is the second fundamental form of $N$.
\end{definition}

The maximal curvature of a geodesic lamination is $0$.
For a minimal lamination in a threefold $M$, $R_\lambda$ is instead controlled by the Gauss curvature of the leaves of $\lambda$ and the Ricci curvature of $M$.

\begin{definition}
Let $\lambda$ be a lamination in $M$.
\begin{enumerate}
\item $\lambda$ is \dfn{discrete} if every leaf space of $\lambda$ is finite.
\item $\lambda$ is \dfn{finite} if $\lambda$ is discrete and admits a finite laminar atlas.
\item $\lambda$ is a \dfn{foliation} if $\supp \lambda = M$.
\end{enumerate}
\end{definition}

Every finite lamination is discrete, and conversely, a discrete lamination in a closed manifold is finite.
However, a lamination with finitely many leaves need not be finite, even if it is geodesic:

\begin{example}\label{two geodesics}
Let $M = \RR \times \Sph^1$ be a cylinder.
We may choose a metric on $M$ so that there is a geodesic $\gamma_1$ which is a simple closed curve looping around $M$, and that there is another geodesic $\gamma_2$ which winds around $M$ infinitely many times, converging to $\gamma_1$.
TODO: Make a picture.
Then $\gamma_1, \gamma_2$ form a geodesic lamination with two leaves which is not finite.
This example, and slight modifications thereof, will be frequently useful as a counterexample throughout this paper.
\end{example}



%%%%%%%%%%%%%%%%%%%%%%%%%%%%%%

\subsection{Regularity of minimal laminations}\label{RegularitySec}
Though we impose that laminations are $C^0$ (in the sense that their flow boxes are $C^0$) we can improve this regularity in the case of minimal laminations.
In this section we follow \cite{Solomon86}, which treats the special case that $\lambda$ is a minimal foliation of $M \subseteq \RR^d$.

\begin{theorem}\label{regularity theorem}
Let $\lambda$ be a minimal lamination. Then $\lambda$ admits a Lipschitz atlas, and $\normal_\lambda$ is Lipschitz.
\end{theorem}
\begin{proof}
TODO
\end{proof}

\begin{example}
Lipschitz regularity is optimal, even in the nicest possible case of a geodesic foliation of a complete Riemannian manifold.
Indeed, let $M = B((2, 0), 1)$ in $\RR^2$. Then the chords $\{y = ax\}$ for $a > 0$ and $\{y = a\}$ for $a \leq 0$ define a foliation $\lambda$ of $M$.
If we equip $M$ with the Beltrami-Klein metric, then, since the leaves of $\lambda$ are chords, $\lambda$ defines a geodesic foliation of $M \cong \Hyp^2$.
Moreover the conormal to the foliation is
$$\normal = \frac{1_{y > 0}}{\sqrt{1 + y^2/x^2}} \left[\dif y - \frac{y}{x} \dif x\right] + 1_{y \leq 0} \dif y$$
which is not $C^1$.
TODO: Make a picture
\end{example}


%%%%%%%%%%%%%%%%
\subsection{Hausdorff distance}

\begin{definition}
Let $X$ be a compact metric space. The \dfn{Hausdorff distance} between two closed sets $A, B \subset X$ is
$$\dist(A, B) := \max\left(\max_{a \in A} \min_{b \in B} \dist(a, b), \max_{b \in B} \min_{a \in A} \dist(a, b)\right).$$
The space of closed subsets of $X$ is the \dfn{hyperspace} $\Hypspace X$.
\end{definition}

For compact $X$, however, one can give a homeomorphism-invariant definition of the topology of $\Hypspace X$ \cite[Chapter 4]{nadler2017continuum}, and from that definition it follows that $\Hypspace X$ is a compact metric space.
In particular, $\Hypspace$ is a self-map of the class of compact metrizable spaces.
Moreover, if $K_i \to K$ in $\Hypspace X$, then $K$ is the set of limits of sequences $(k_i)$ such that $k_i \in K_i$.


%%%%%%%%%%%%%%%%%%%%%
\section{Thurston's geometric topology}
% \subsection{Global space of leaves}
% We now put a topology on the set $\Leaves \lambda$ of all leaves of a lamination $\lambda$ in a closed manifold, so that $\Leaves \lambda$ is a compact Hausdorff space.
%
%
% The fact that we restrict $X$ to be compact is motivated by the following example.
%
% \begin{example}
% Consider $\Hyp^2$ with its disk model in $\CC$, and let $\gamma$ be the horizontal geodesic through the origin $0$.
% Consider also a sequence of geodesics $(\gamma_j)$ through $i/j$ which are horizontal at $i/j$.
% Then $\lambda_j \to \lambda$ in Thurston's geometric topology and in Hausdorff distance.
% However,
% $$\dist_{\mathscr H(\Hyp^2)}(\gamma_j, \gamma) = \infty.$$
% Indeed, one can find points $x$ on $\gamma$ which are arbitrarily close to the circle at infinity $\partial \Hyp^2$.
% Let $\rho$ be the geodesic which is orthogonal to $\gamma$ at $x$.
% Then the segment of $\rho$ between $\gamma$ and $\gamma_j$ would become arbitrarily long as $x \to \partial \Hyp^2$.
% \end{example}
%
%
% These properties make it suitable to define $\Leaves \lambda$ to be a certain closed subset of $\Hypspace M$:
%
% \begin{definition}
% If $\lambda$ is a lamination in a closed manifold $M$, we let $\Leaves \lambda$ be the subset of $\Hypspace M$ whose elements are closed sets $\overline N$ where $N$ is a leaf of $\lambda$.
% \end{definition}
%
% \begin{proposition}
% For any lamination $\lambda$, $\Leaves \lambda$ is in canonical bijection with the set of all leaves of $\lambda$, and is a closed subset of $\Hypspace M$.
% \end{proposition}
% \begin{proof}
% Clearly the map $N \mapsto \overline N$ is a canonical surjection from the set of leaves of $\lambda$ to $\Leaves \lambda$.
% In fact, it is injective as well: if $\overline N = \overline{N'}$, then for any sequence $(x_i)$ in $N$ such that $x_i \to x$ in $M$, there exists a sequence $(x_i')$ in $N'$ such that $x_i' \to x$.
% Moreover (possibly after discarding finitely many entries in the sequence) we may assume that these limits take place inside some flow box with image $U_\alpha$, so there are co-Cauchy sequences $(k, y_i)$ and $(k', y_i')$ in $\RR \times \RR^{d - 1}$, where $k, k'$ are the labels of $N, N'$ respectively.
% This is only possible if $k = k'$ and so $N \cap U_\alpha = N' \cap U_\alpha$.
% After covering $M$ by flow boxes we see that $N = N'$ and so the canonical map is injective.
%
% Now if $(N_i)$ is a sequence in $\Leaves \lambda$ which converges to some closed set $N \in \Hypspace M$, then $N \in \Hypspace(\supp \lambda)$.
% Indeed, $\supp \lambda$ is a compact subset of $M$, so $\Hypspace(\supp \lambda)$ is compact, and hence is a closed subset of $\Hypspace M$.
% In any flow box $U$, the sets $N_i \cap U$ take the form $\{k_i\} \times \RR^{d - 1}$, and any limit of such sets must of the form $\{k\} \times \RR^{d - 1}$.
% So $N \cap U$ is a leaf of the restriction of $\lambda$ to $U$.
% It follows that $N \in \Leaves \lambda$.
% \end{proof}

\begin{definition}
A sequence of laminations $\lambda_i$ converges to $\lambda$ in \dfn{Thurston's geometric topology} if, for every leaf $N$ of $\lambda$, every $x \in N$, and every $\varepsilon > 0$, there exists $i_\varepsilon \in \NN$ such that for every $i \geq i_\varepsilon$, $\supp \lambda_i$ intersects $B(x, \varepsilon)$, and for $x_i \in B(x, \varepsilon) \cap \supp \lambda_i$,
\begin{equation}\label{convergence of normals}
\dist_{S' M}(\normal_{\lambda_i}(x_i), \normal_\lambda(x)) < 2\varepsilon.
\end{equation}
\end{definition}

It is straightforward to see that this topology does not depend on the Riemannian metric.
The distnace in (\ref{convergence of normals}) is in the cosphere bundle $S'M$.
It is also straightforward to see that if $\lambda_i \to \lambda$ and $\lambda' \subseteq \lambda$ then $\lambda_i \to \lambda'$.
Thus Thurston's geometric topology is very far from Hausdorff.
To deal with the lack of uniqueness of limits we single out the special case that we have a privileged choice of limit:

\begin{definition}
If $\lambda_i \to \lambda$ in Thurston's geometric topology, and for every lamination $\lambda'$, $\lambda_i \to \lambda'$ iff $\lambda' \subseteq \lambda$, we say that $\lambda$ is the \dfn{maximal limit} of $(\lambda_i)$ in Thurston's geometric topology.
\end{definition}

If a maximal limit exists, it is unique.
However one may consider the case that there are two limiting laminations whose leaves intersect to see that there need not exist a maximal limit.

% Since Thurston's geometric topology requires convergence of normal vector fields pointwise, it ``respects the regularity of the flow boxes" in some sense.
% On the other hand this makes it somewhat hard to prove convergence in this topology.

% \begin{lemma}
% Let $(\lambda_i)$ be a sequence of geodesic laminations, and $\lambda$ a geodesic lamination, such that for every leaf $N$ of $\lambda$ and every $x \in \NN$, there exists a transverse geodesic $\gamma$ to $N$ at $x$ such that for every $\varepsilon > 0$, we can find $i \in \NN$, a leaf $N_i$ of $\lambda_i$, and $x_i \in N_i$, such that $\dist(x_i, x) < \varepsilon$ and $\gamma$ is transverse to $N_i$ at $x_i$.
% \end{lemma}
% \begin{proof}
% By Lemma \ref{transverse at one implies transverse at all} and Theorem \ref{regularity theorem}, $\gamma$ is transverse to leaves of $\lambda$ (possibly after replacing $M$ with a small neighborhood of $x$).
% For each $k$ such that $\gamma(k) \in \supp \lambda$, we let $X(k) \in T_{\gamma(k)} M$ be unit length and tangent to the leaf of $\lambda$ which contains $\gamma(k)$.
% Then we may extend $X$ to a Lipschitz vector field along $\gamma$, and set
% $$F(k, t) := \exp_{\gamma(k)}(tX(k)).$$
% So $F$ is a Lipschitz flow box.
% Working in the coordinates given by $F$, $N_i$ is transverse to $\{t = 0\}$, so it is the graph $\{k = f_i(t)\}$ of a Lipschitz function $f_i$ defined near $0 \in \RR$.
% Moreover, $f_i(0) \to f(0)$.
% We claim that $\dif f_i(0) \to \dif f(0)$; if not,
% \end{proof}

% \begin{example}
% Let $f_i$ be a smooth approximation to the Dirac measure $\delta_{1/2-1/i}$ at $1/2 - 1/i$ with compact support in $(1/2 - 2/i, 1/2)$, so $f_i \to 0$ pointwise.
% Let $\lambda_i$ be the lamination consisting of the graph of $f_i$ in $\RR^2$ and $\lambda$ the lamination consisting of the horizontal line through $0$.
% Then $\lambda_i$ converges to $\lambda$ in Thurston's geometric topology but not in Hausdorff distance.
% TODO: Include a picture
% \end{example}
%
% \begin{example}\label{Hausdorff does not imply Thurston}
% Consider the sequence of continuous functions $f_i: \RR \to \RR$ where $f_i$ is constant away from $[0, 1/i^2]$, is $0$ on $(-\infty, 0]$, is $1/i$ on $[1/i^2, \infty)$, and is linear on $[0, 1/i^2]$.
% By smoothing out the graphs of $f_i$ appropriately one obtains a sequence of $C^1$ curves $\gamma_i$ in $\RR^2$, whose geodesic curvatures blow up near $(0, 0)$, and if we set $\lambda_i$ to be the union of $\gamma_1, \dots, \gamma_i$, then $(\lambda_i)$ is a sequence of laminations which converges in Hausdorff distance but not in Thurston's geometric topology.
% TODO: Include a picture.
% \end{example}


%%%%%%%%%%%%%%%%%%%%%%%%%%%%%%%%%%%

\section{The weak topology of measures}
Another topology in common use is the weak topology on measures on the space of measured laminations, due to Thurston \cite[Chapter 8]{thurston1998minimal}.
To state it, we recall the notion of a transverse measure.
These are certain Radon measures on the leaf spaces; to study them it will be useful to recall some facts about Radon measures on Polish spaces TODO Cite it.

\subsection{Preliminaries}
Let $X$ be a Polish space.
The space $C_0(X)$ of continuous functions $f: X \to \RR$ such that $\lim_{x \to \infty} f(x) = 0$ is canonically isomorphic to the direct limit $\varinjlim C(Y)$ of continuous functions $f: Y \to \RR$ where $Y$ ranges over all compact subsets of $X$ and the bonding maps are inclusions.
Since $C(Y)$ is a Banach space, the identification $C_0(X) = \varinjlim C(Y)$ endows $C_0(X)$ with the structure of a locally convex space.
If $X$ is in fact compact, then this direct limit is trivial and so $C(X) = C_0(X)$.
Its dual $C_0(X)'$ is canonically isomorphic to the space of signed Radon measures on $X$.

\begin{definition}
The weak topology on $C_0(X)'$ is known as the \dfn{weak topology of measures}.
\end{definition}

Unpacking the definitions, a sequence $(\mu_i)$ of Radon measures converges to $\mu$ in the weak topology of measures iff for every compact $Y \subseteq X$ and every continuous function $f: Y \to \RR$,
$$\lim_{i \to \infty} \int_Y f \dif \mu_i = \int_Y f \dif \mu.$$
The same thing works for currents.

Having topologized the space of Radon measures, we define what it means for such a measure to be transverse to a lamination:

\begin{definition}
Let $\lambda$ be a lamination with atlas $A$.
A \dfn{transverse measure} to $\lambda$ consists of Radon measures $\mu_\alpha$ with $\supp \mu_\alpha = K_\alpha$, $\alpha \in A$, such that each transition map $\psi_{\alpha \beta}$ is measure-preserving:
$$\mu_\alpha|_{K_\alpha \cap K_\beta} = \psi_{\alpha \beta}^* (\mu_\beta|_{K_\alpha \cap K_\beta}).$$
The pair $(\lambda, \mu)$ is called a \dfn{measured lamination}.
\end{definition}

One must be careful here: in \cite{thurston1979geometry} and here, it is assumed that $\supp \mu_\alpha = K_\alpha$, but in \cite{daskalopoulos2020transverse}, it is only assumed that $\supp \mu_\alpha \subseteq K_\alpha$.
In particular, not every lamination admits a transverse measure, as we will later see.

%%%%%%%%%%%%%%%%%%%%%%%%%%%%%%%%%%

\subsection{Ruelle-Sullivan currents}
The definition of transverse measure in terms of Radon measures on $K_\alpha$ is convenient because $K_\alpha$ is compact.
However, the definition is not intrinsic, and this causes problems when considering questions of convergence: the fact that the flow boxes of a convergent sequence of measured laminations converge should be a consequence of, not a part of, the definition!

To rectify this, we first observe that in the definition of a transverse measure, we cannot define a transverse measure to be one on the underlying manifold $M$ itself.
Indeed, Lebesgue measure is ``transverse'' to all foliations; thus such a definition forgets the ``direction'' the measure points in.
However, the notion of Ruelle-Sullivan current allows us to speak of a measure-theoretic object on $M$ which has a well-defined local product structure.

\begin{definition}
A lamination is \dfn{oriented} if one can choose its transition maps to all be orientation-preserving.
\end{definition}

It is clear that a lamination $\lambda$ is locally orientable, since if one replaces $M$ by a small open set, then $\lambda$ has a global flow box.

\begin{definition}
Let $(\lambda, \mu)$ be a measured oriented lamination and let $(\chi_\alpha)_{\alpha \in A}$ be a subordinate partition of unity.
The \dfn{Ruelle-Sullivan current} associated to $(\lambda, \mu)$ is defined for all compactly supported $d-1$-forms $\varphi$ by
\begin{equation}\label{RS current}
\int_M T_\mu \wedge \varphi := \sum_{\alpha \in A} \int_{K_\alpha} \left[\int_{\RR^{d - 1} \times \{k\}} (F_\alpha^{-1})^* (\chi_\alpha \varphi) \right] \dif \mu_\alpha(k).
\end{equation}
\end{definition}

\begin{lemma}
The Ruelle-Sullivan current $T_\mu$ is well-defined; it is honestly a $d-1$-current, and does not depend on the choice of partition of unity.
Moreover, $\dif T_\mu = 0$.
\end{lemma}
\begin{proof}
We first claim that the right-hand side of (\ref{RS current}) is always finite, and is continuous in $\varphi$.
In fact, possibly after refining $(\chi_\alpha)$, we may assume that it is a locally finite partition of unity.
In particular, we just need to check the continuity in a single flow box:
$$\left|\int_{K_\alpha} \left[\int_{\RR^{d - 1} \times \{k\}} (F_\alpha^{-1})^* (\chi_\alpha \varphi) \right] \dif \mu_\alpha(k)\right| \leq \int_{K_\alpha} \int_{\RR^{d - 1} \times \{k\}} |(F_\alpha^{-1})^* (\chi_\alpha \varphi)| \dif \mu_\alpha(k).$$
The inner integral is controlled by $||\varphi||_{L^1(U_\alpha)} \cdot |U_\alpha|$ where $U_\alpha$ is the image of $F_\alpha$.
The outer integral is then well-defined because it is against a Radon measure.

We next observe that the choice of partition of unity is irrelevant, thus if $\varphi$ has compact support in $U_\alpha \cap U_\beta$, then
\begin{equation}\label{well-defined of Ruelle-Sullivan}
\int_{K_\alpha} \int_{\RR^{d - 1} \times \{k\}} (F_\alpha^{-1})^* \varphi \dif \mu_\alpha(k) = \int_{K_\beta} \int_{\RR^{d - 1} \times \{k\}} (F_\beta^{-1})^* \varphi \dif \mu_\beta(k).
\end{equation}
Indeed,
\begin{align*}
\int_{K_\alpha} \int_{\RR^{d - 1} \times \{k\}} (F_\alpha^{-1})^* \varphi \dif \mu_\alpha(k)
&= \int_{K_\beta} (F_\alpha F_\beta^{-1})^* \left[\int_{\RR^{d - 1} \times \{k\}} (F_\alpha^{-1})^* \varphi \dif \mu_\alpha(k)\right] \\
&= \int_{K_\beta} \left[\int_{\RR^{d - 1} \times \{k\}} (F_\beta^{-1})^* \varphi\right] (F_\alpha F_\beta^{-1})^* \dif \mu_\beta(k) \\
&= \int_{K_\beta} \int_{\RR^{d - 1} \times \{k\}} (F_\beta^{-1})^* \varphi \dif \mu_\beta(k)
\end{align*}
where the last equation is because of the measure-preserving nature of the transition maps; this proves (\ref{well-defined of Ruelle-Sullivan}).

Finally, if a $d-2$-form $\psi$ has compact support in a single flow box, then
$$\int_{\RR^{d - 1} \times \{k\}} (F_\alpha^{-1})^* \dif \psi = \int_{\RR^{d - 1} \times \{k\}} \dif((F_\alpha^{-1})^* \psi) = 0$$
by Stokes' theorem, so
\begin{align*}
\int_M \dif T_\mu \wedge \psi &= -\int_M T_\mu \wedge \dif \psi \\
&= -\int_{K_\alpha} \int_{\RR^{d - 1} \times \{k\}} (F_\alpha^{-1})^* \dif \psi \dif \mu_\alpha(k) = 0. \qedhere
\end{align*}
\end{proof}

Though (\ref{RS current}) is the more traditional way of stating the definition of a Ruelle-Sullivan current, there is a more intrinsic way as well.
We first observe that if $\mu$ is a transverse measure, then $\mu$ defines a measure on $\supp \lambda$: in each flow box $F_\alpha$, an open set $U$ has measure
\begin{equation}\label{transverse measure of an open set}
\mu(U) := \int_{K_\alpha} |F_\alpha(\RR^{d - 1} \times \{k\}) \cap U| \dif \mu_\alpha(k).
\end{equation}

\begin{lemma}
For an oriented measured lamination $(\lambda, \mu)$, the polar decomposition of $T_\mu$ is
\begin{equation}\label{polar ruelle sullivan}
T_\mu = \normal_\lambda \mu.
\end{equation}
\end{lemma}
\begin{proof}
For an open set $U \subseteq M$ in a flow box $F_\alpha$, the total variation measure $|T_\mu|$ satisfies
$$|T_\mu|(U) = \sup_{||\varphi||_{C^0} \leq 1} \int_{K_\alpha} \int_{\RR^{d - 1} \times \{k\}} \varphi \dif \mu_\alpha(k)$$
where the supremum ranges over $d-1$-forms $\varphi$ with compact support in $U$.
However, $\star \normal_\lambda$ is the Riemannian measure on $F_\alpha(\RR^{d - 1} \times \{k\})$, so
$$\int_{\RR^{d - 1} \times \{k\}} \varphi \leq \int_{\RR^{d - 1} \times \{k\}} (F_\alpha^{-1})^*(\star \normal_\lambda).$$
Since $||\normal^\lambda||_{C^0} = 1$, it follows that a sequence of cutoffs of $\star \normal_\lambda$ to more and more of $U$ is a maximizing sequence.
Therefore $\normal_\lambda$ is the polar part of (\ref{polar ruelle sullivan}), and
$$|T_\mu|(U) = \int_{K_\alpha} \int_{\RR^{d - 1} \times \{k\}} (F_\alpha^{-1})^*(1_U \star \normal_\lambda) \dif \mu_\alpha(k).$$
The inner integral is the Riemannian measure of $F_\alpha(\RR^{d - 1} \times \{k\}) \cap U$, so by (\ref{transverse measure of an open set}), $|T_\mu| = \mu$.
\end{proof}

\begin{proposition}
If $\lambda$ is a minimal lamination which admits a transverse measure, then every leaf of $\lambda$ is closed.
\end{proposition}
\begin{proof}
If $N$ is a leaf of $\lambda$ which is not closed, but $\mu$ is a transverse measure, then since $\supp \lambda$ is closed, we can find a sequence $(x_i)$ which converges to a point $x \in \supp \lambda \setminus N$.
Considering flow box coordinates $F_\alpha$ at $x$, we see that there exist infinitely many different labels $(k_i)$ of $N$.
Indeed, if it is possible to express $N$ locally near $x$ as $J \times \RR^{d - 1}$ where $J \subset I$ is finite, then $N$ would be locally closed in a neighborhood of $x$, a contradiction.

In the image $U_\alpha$ of $F_\alpha$, $\lambda$ is orientable and so we can write $T_\mu = \dif u$ for some function $u$ of least gradient.
In particular, $N \cap U_\alpha$ is a level set of $u$ and hence bounds a set of least perimeter, which then must have infinitely many connected components in any neighborhood of $x$.
However, sets of least perimeter only have locally finitely many connected components, so this case is still a contradiction.
\end{proof}

In particular, the pathological lamination of Example \ref{two geodesics} does not admit a transverse measure.

We are now ready to define the weak topology of measures.

\begin{definition}
A sequence of measured laminations $(\lambda_i, \mu_i)$ converges in the \dfn{weak topology of measures} to a measure $(\lambda, \mu)$ if it is possible to cover $M$ by open sets $(U_\beta)$ so that in each $U_\beta$, $\lambda$ is orientable, $\lambda_i$ is eventually orientable, and we can choose orientations in $U_\beta$ and a subordinate partition of unity $(\chi_{\alpha \beta})$ in $U_\beta$ such that, if we choose $\lambda_i, \lambda$ to be cooriented in $U_\beta$, the local Ruelle-Sullivan currents $T_{\mu_i}^\beta$ converge to $T_\mu^\beta$ in the weak topology of measures.
\end{definition}

Concretely, this means that after replacing $M$ with an open set which is so small that $\lambda_i, \lambda$ all share a common orientation, one has
$$\lim_{i \to \infty} \int_M T_{\mu_i} \wedge \varphi = \int_M T_\mu \wedge \varphi$$
for every compactly supported $d-1$-form $\varphi$.
% If we can in addition choose identical flow boxes of $\lambda_i, \lambda$, then the characterization is even simpler:

% \begin{lemma}
% If there exists a common atlas $(F_\alpha)$ of flow boxes for $\lambda_i$ and $\lambda$, and $\mu_i, \mu$ are transverse to $\lambda_i, \lambda$, then $(\lambda_i, \mu_i) \to (\lambda, \mu)$ in the weak topology of measures iff for every $\alpha \in A$, $\mu_{i\alpha} \to \mu_\alpha$ in the weak topology of measures.
% \end{lemma}
% \begin{proof}
% Since they have common flow boxes, we can realize a test function $f$ for $\dif \mu_\alpha$ by choosing $\star \varphi$ to be conormal to the leaves.
% Or we can build a test form $\varphi$ by noticing that only the Hodge dual to the conormal part matters, and the conormal part is given by $f$.
% TODO fill in the details here
% \end{proof}

%%%%%%%%%%%%%%%%%%%%%%%%%%%%%

\subsection{Measures on transverse curves}
Here is another characterization of the weak topology of measures in terms of transverse curves.

\begin{definition}
For a flow box $F: I \times \RR^{d - 1} \to M$, we define its \dfn{labelling map}
$$\Pi_F: F(I \cap \RR^{d - 1}) \to I$$
to be the projection of $F^{-1}$ onto $I$:
% https://q.uiver.app/?q=WzAsMyxbMCwwLCJGKEkgXFxjYXAgXFxtYXRoYmIgUl57ZCAtIDF9KSJdLFsyLDAsIkkgXFxjYXAgXFxtYXRoYmIgUl57ZCAtIDF9Il0sWzIsMiwiSSJdLFswLDEsIkZeey0xfSJdLFsxLDIsIiIsMCx7InN0eWxlIjp7ImhlYWQiOnsibmFtZSI6ImVwaSJ9fX1dLFswLDIsIlxcUGlfRiIsMl1d
\[\begin{tikzcd}
	{F(I \cap \mathbb R^{d - 1})} && {I \cap \mathbb R^{d - 1}} \\
	\\
	&& I
	\arrow["{F^{-1}}", from=1-1, to=1-3]
	\arrow[two heads, from=1-3, to=3-3]
	\arrow["{\Pi_F}"', from=1-1, to=3-3]
\end{tikzcd}\]
\end{definition}

\begin{definition}
For a smooth path $\gamma: I \to M$ and a lamination $\lambda$, we say that $\gamma$ is \dfn{transverse} to $\lambda$ if, for every $t \in I$ such that $\gamma(t) \in \supp \lambda$, there exists a flow box $F$ in a neighborhood of $\gamma(t)$ and a neighborhood $J \subseteq I$ of $t$ such that
$$\Pi_F \circ \gamma|_J: J \to I$$
is a topological embedding, i.e. a homeomorphism onto its image.
If $\Pi_F \circ \gamma|_J$ is in addition increasing, we say that $\gamma$ is \dfn{positively transverse}.
% By a \dfn{transverse homotopy} of transverse curves to $\lambda$ we mean a homotopy of curves $H: I^2 \to M$ such that every curve $\gamma_s := H(s, \cdot)$ is transverse to $\lambda$, and if $H(s, t)$ is an element of some leaf $N$ of $\lambda$, then so is $H(s', t)$ for any $s'$.
\end{definition}

\begin{definition}
For a transverse curve $\gamma$ to a measured lamination $(\lambda, \mu)$, we get a Radon measure $\gamma^! \mu$ on $I$, as follows.
Near $t \in I$, choose a flow box $F$ near $\gamma(t)$ and a neighborhood $J$ of $t$, such that $\Pi_F \circ \gamma|_J$ is a topological embedding.
Then
$$(\gamma^! \mu)|_J := ((\Pi_F \circ \gamma|_J)^{-1})_* \mu_F.$$
\end{definition}

The measure $\gamma^! \mu$ is well-defined, essentially since the transition maps are measure-preserving (so the only thing that actually matters is the weight assigned to each leaf that $\gamma$ passes through).

\begin{lemma}
Let $(\lambda, \mu)$ be a measured oriented lamination, $\gamma$ a curve, and $\gamma(t) \in \supp \lambda$.
Then $\gamma$ is positively transverse to $\lambda$ at $\gamma(t)$ iff
\begin{equation}\label{transverse means rs}
\lim_{\varepsilon \to 0} \frac{1}{\mu(B(\gamma(t), \varepsilon))} \int_{B(\gamma(t), \varepsilon)} T_\mu \wedge \star (\gamma')^\flat > 0.
\end{equation}
\end{lemma}
\begin{proof}
We work in flow box coordinates $(k, y)$, so that the labelling map $\Pi$ is $\Pi(k, y) = k$.
In such coordinates, leaves $N$ take the form $\{k = k_N\}$ for some $k_N$, so $\dif k$ is conormal to each leaf.
In particular, $\normal_\lambda = (g^{kk})^{-1/2} \dif k$.
Also $(\Pi \circ \gamma)' = (\dif k, \gamma')$, so (since $(g^{kk})^{-1/2}$ is positive), $\gamma$ is positively transverse at $\gamma(t)$ iff
\begin{equation}\label{transverse means dk}
(\normal_\lambda, \gamma')(\gamma(t)) > 0.
\end{equation}
But $\gamma(t) \in \supp \mu$, and $\mu$ is a Radon measure.
So by the Lebesgue differentiation theorem, (\ref{transverse means dk}) happens iff
$$\lim_{\varepsilon \to 0} \frac{1}{\mu(B(\gamma(t), \varepsilon))} \int_{B(\gamma(t), \varepsilon)} (\normal_\lambda, \gamma') \dif \mu > 0.$$
Applying the polar decomposition (\ref{polar ruelle sullivan}) of $T_\mu$ we conclude that this last condition is equivalent to (\ref{transverse means rs}).
\end{proof}

\begin{proposition}
Let $(\lambda, \mu)$ be a measured oriented lamination and $\gamma$ a transverse curve to $\lambda$.
Then there exists an open set $U \supseteq \supp \lambda$ in $M$ and an open set $\mathcal U \ni (\lambda, \mu)$ in the space of measured oriented laminations, such that if $(\kappa, \nu) \in \mathcal U$ has support in $U$, then $\gamma$ is transverse to $\kappa$.
\end{proposition}
\begin{proof}
Suppose not. Then there exist laminations $\lambda_i$ with
$$\supp \lambda_i \subseteq \{x \in M: \dist(x, \supp \lambda) < i^{-1}\},$$
and transverse measures $\mu_i$ to $\lambda_i$, such that $(\lambda_i, \mu_i) \to (\lambda, \mu)$ in the weak topology of measures, but with $\gamma$ not transverse to all the $\lambda_i$, say at $\gamma(t_i) \in \supp \lambda_i$.
By taking a subsequence, we may assume that $\gamma(t_i) \to \gamma(t)$ for some $t \in I$.
Since $\dist(\gamma(t_i), \supp \lambda) < i^{-1}$, $\gamma(t) \in \supp \lambda$.
Since $\gamma$ is transverse to $\lambda$, there exist $\varepsilon_*, \delta > 0$ such that for each $0 < \varepsilon \leq \varepsilon_*$,
$$\frac{1}{\mu(B(\gamma(t), \varepsilon))} \int_{B(\gamma(t), \varepsilon)} T_\mu \wedge \psi \geq \delta.$$
From the definition of the weak topology of measures,
$$\lim_{i \to \infty} \frac{1}{\mu(B(\gamma(t_i), \varepsilon))} \int_{B(\gamma(t_i), \varepsilon)} T_{\mu_i} \wedge \psi \geq \delta.$$
By the portmanteau theorem, this implies
$$\liminf_{i \to \infty} \frac{1}{\mu_i(B(\gamma(t_i), \varepsilon))} \int_{B(\gamma(t_i), \varepsilon)} T_{\mu_i} \wedge \psi \geq \delta.$$
In particular, for every $i$ large enough,
$$\frac{1}{\mu_i(B(\gamma(t_i), \varepsilon))} \int_{B(\gamma(t_i), \varepsilon)} T_{\mu_i} \wedge \psi > 0,$$
which contradicts the fact that $\gamma$ is not transverse to $\lambda_i$ at $t_i$.
\end{proof}

\begin{example}
There exist measured oriented laminations $(\lambda, \mu)$ and $(\kappa, \nu)$ which are arbitrarily close in the weak topology of measures, such that $\lambda$ admits a transverse curve $\gamma$ which is not transverse to $\kappa$.
Working in $\RR^2$, let $\lambda$ consist of the horizontal axis, $\gamma$ the vertical axis, and $\kappa$ the horizontal axis plus a curve $\rho$ in the upper half-plane which meets $\gamma$ tangentially.
Then take $\mu$ to be the usual measure on $\lambda$, $\nu$ to be $\mu$ plus a small measure on $\rho$.
We can take $\nu$ to be arbitrarily close to $\mu$ while preserving the fact that $\supp \nu = \supp \kappa$.
\end{example}

%%%%%%%%%%%%%%%%%%%%%%%%%%%%%%%%%%%%%%%
\subsection{Weak topology of measures versus Thurston's geometric topology}
We now show that for a geodesic lamination, convergence in the weak topology of measures implies convergence in Thurston's geometric topology.
Thurston claimed this fact \cite[Proposition 8.10.3]{thurston1979geometry} but his proof left something to be desired as it did not justify why the limit is geodesic, or why the convergence respects the normal vectors.

\begin{lemma}\label{measured limits are almost thurston}
Let $(\lambda_i, \mu_i)$ be measured geodesic laminations.
If $(\lambda_i, \mu_i) \to (\lambda, \mu)$ in the weak topology of measures, then for every $x \in \supp \lambda$ there exists $x_i \in \supp \lambda_i$ such that $x_i \to x$.
\end{lemma}
\begin{proof}
For every $\varepsilon > 0$, $\mu(B(x, \varepsilon)) > 0$, so by the portmanteau theorem,
$$\liminf_{i \to \infty} \mu_i(B(x, 2\varepsilon)) \geq \mu(B(x, \varepsilon)).$$
Therefore there exists $x_i \in \supp \lambda_i \cap B(x, 2\varepsilon)$ if $i$ is large enough.
\end{proof}

\begin{lemma}\label{weak convergence of FLG}
Let $(u_i)$ be a sequence of functions of least gradient which is bounded in $L^1$, and $T$ a current such that $\dif u_i \to T$ weakly.
Then there exists a function $u$ of least gradient such that along a subsequence, $u_i \to u$ weakly in $BV$, and $\dif u = T$.
\end{lemma}
\begin{proof}
By the Miranda stability theorem, $(u_i)$ has a subsequential limit $u$ in $L^1$, which has least gradient.
For any compactly supported $d-1$-form $\varphi$,
\begin{align*}
\int_M T \wedge \varphi
&= \lim_{i \to \infty} \int_M \dif u_i \wedge \varphi
= -\lim_{i \to \infty} \int_M u_i \dif \varphi = - \int_M u \dif \varphi.
\end{align*}
where the last equality is because the Radon-Nikod\'ym derivative of $\dif \varphi$ with respect to Lebesgue measure is in $L^\infty$ and hence $L^1(\dif \varphi)$ is a weaker space than $L^1$. The last expression here is $\int_M \dif u \wedge \varphi$.
So $T = \dif u$ and $\dif u_i \to \dif u$ in the weak topology of measures; that is, $u_i \to u$ weakly in $BV$.
\end{proof}

\begin{lemma}\label{limits of measured geodesic lams are geodesic}
Let $(\lambda_i, \mu_i)$ be measured geodesic laminations.
If $(\lambda_i, \mu_i) \to (\lambda, \mu)$ in the weak topology of measures, then $\lambda$ is geodesic.
\end{lemma}
\begin{proof}
Let $x \in \supp \lambda$ and $r > 0$ such that $B := B(x, r)$ is contractible.
In $B$, we can write $T_{\mu_i} = \dif u_i$ for some sequence of functions of least gradient $u_i \in BV(B)$.
Since $u_i$ is only defined up to a constant, we impose $\int_M \star u_i = 0$, so by Poincar\'e's inequality,
$$||u_i||_{L^1(B)} \lesssim r\int_B \star |T_{\mu_i}| \lesssim 1$$
where the last bound is because $(T_{\mu_i})$ is compact in the weak topology of measures.
So by Lemma \ref{weak convergence of FLG}, $u_i \to u$ where $u$ has least gradient and satisfies $T_\mu = \dif u$.
So by \cite{BackusFLG}, the leaf $N$ containing $x$ is a geodesic and $x$ is not an endpoint of $N$.
But $x$ was arbitrary, so $N$ is in fact complete.
\end{proof}

\begin{proposition}\label{measured implies Thurston}
Let $(\lambda_i, \mu_i)$ be measured geodesic laminations.
If $(\lambda_i, \mu_i) \to (\lambda, \mu)$ in the weak topology of measures, then $\lambda$ is a geodesic lamination and $\lambda_i \to \lambda$ in Thurston's geometric topology.
\end{proposition}
\begin{proof}
By Lemmata \ref{measured limits are almost thurston} and \ref{convergence of geodesic lams in thurston},
$\lambda$ is a geodesic lamination and for every $x \in \supp \lambda$, $\varepsilon > 0$, and large $i$, there exists $y \in \supp \lambda_i \cap B(x, \varepsilon)$.
To get convergence of the normals we choose a Lipschitz $d-1$-form $\varphi$ which extends $\star \normal$.
Then $\int_K T_\mu \wedge \varphi = \mu(K)$ for any $\mu$-continuity set $K$, so by the portmanteau theorem,
$$\lim_{i \to \infty} \frac{\int_K T_{\mu_i} \wedge \varphi}{\mu_i(K)} = \frac{\int_K T_\mu \wedge \varphi}{\mu(K)} = 1.$$
On the other hand, if we assume that there exists $\delta, \varepsilon > 0$ such that for every $y \in \supp \lambda_i \cap B(x, \varepsilon)$,
$$|\sin(\normal_i(y) - \normal(x))| \geq \delta,$$
then possibly after shrinking $\varepsilon$ we may assume that $B(x, \varepsilon)$ is a $\mu$-continuity set, so that
$$\int_K T_{\mu_i} \wedge \varphi = \int_K \normal_i\mu_i \wedge \star \normal \leq (1 - O(\delta)) \mu_i(K)$$
which contradicts $\delta > 0$.
\end{proof}

%%%%%%%%%%%%%%%%%%%%%%%

\section{Convergence in flow boxes}
Let $C^{1-}$ be the Fr\'echet space $\bigcap_{\alpha < 1} C^\alpha$, where $C^\alpha$ are H\"older spaces.
The following definition is one possible interpretation of the vague definition of \cite[Lemma II.1.2]{ColdingMinicozziV}.

\begin{definition}
A sequence $(\lambda_i)$ of laminations \dfn{converges on the level of flow boxes} to $\lambda$ if it converges in Thurston's geometric topology, and there exists a laminar atlas $(F_\alpha)$ for $\lambda$ such that for each $\alpha$, $F_\alpha$ and $(F_\alpha)^{-1}$ are limits in $C^{1-}$ of flow boxes $F_\alpha^i$, $(F_\alpha^i)^{-1}$ for $\lambda_i$.
\end{definition}

We allow the limiting flow box to not be Lipschitz, but just $C^{1-}$, if necessary.
Just as for Thurston's geometric topology, convergence on the level of flow boxes does not satisfy uniqueness of limits, and hence we speak of \dfn{maximal limits} to allow us to speak of a unique limit of a sequence.

\begin{proposition}
Let $(\lambda, \mu)$ be a measured geodesic lamination.
Then there exist finite sublaminations $\lambda_i$ of $\lambda$ such that $\lambda$ is a maxima limit of $(\lambda_i)$ on the level of flow boxes.
\end{proposition}
\begin{proof}
Let $(x_i)$ be a dense sequence in $\supp \lambda$ and $N_i$ the leaf containing $x_i$.
Since $\lambda$ admits a transverse measure, $N_i$ is closed, so setting $\lambda_i := \bigcup_{j < i} N_j$, $\lambda_i$ is a finite lamination.
Now if $N$ is a leaf in $\lambda$, either $N = N_i$ for some $i$, in which case clearly $N$ is a limiting leaf of $\lambda_i$, or for every $x \in N$ there exists a subsequence $(x_{i_j})$ such that $x_{i_j} \to x$.
In partiular, $N_{i_j}$ approximates $N$ in Thurston's geometric topology (here the convergence of normal vectors is because it's a geodesic lamination).
Moreover, if $F$ is a flow box for $\lambda$ near $x$, then $F$ is also a flow box for $N_i$ for every $i$, hence the convergence in flow boxes.
\end{proof}


%%%%%%%%%%%%%%%%%%%%%%%%%
\subsection{Compactness}
\begin{definition}
A sequence of laminations $(\lambda_i)$ is \dfn{tight} if there exists a compact set $K \subseteq M$ such that for every $i$ there exists a leaf $N_i$ of $\lambda_i$, such that $N_i \cap K$ is nonempty.
\end{definition}

\begin{lemma}\label{limit of minimals is minimal}
Let $(N_i)$ be a sequence of complete embedded minimal hypersurfaces of uniformly bounded curvature and $N_i \to N$ in $\Hypspace M$.
Then $N$ is a complete minimal hypersurface.
\end{lemma}
\begin{proof}
In a neighborhood of any point on $N$, we can write $N_i = \partial U_i$ where $U_i$ is a set of least perimeter.
Let $u_i := 1_{U_i}$; then $u_i$ is a function of least gradient.

If we set $N = \partial U$, $u := 1_U$, then $u_i \to u$ we claim that pointwise away from $N$, and hence almost everywhere.
This can be seen by writing $N_i$ as the graph of $f_i$, so $U_i = \{y < f_i(x)\}$, and this is possible due to the assumption of bounded curvature.
Similarly we write $U = \{y < f(x)\}$.
Now if $(x, y) \in U$ but $(x, y) \notin U_i$ for arbitrarily large $i$, then $f_i(x) \leq y < f(x)$ so $(x, f_i(x)) \in N_i$ converges to a point $(x, \tilde y)$ with $\tilde y \neq y$, thus $(x, \tilde y) \notin N$.
This violates the convergence $N_i \to N$ in $\Hypspace N$.

So $u_i \to u$ almost everywhere, and hence by dominated convergence also converges in $L^1$.
By the Miranda stability theorem it follows that $U$ has least perimeter, and so by \cite{BackusFLG}, $N$ is a complete minimal hypersurface.
\end{proof}

\begin{lemma}\label{convergence of geodesic lams in thurston}
Assume that $(\lambda_i)$ is a sequence of geodesic laminations, and $\lambda$ a geodesic lamination, such that for every leaf $N$ of $\lambda$, there is a leaf $N_i$ of $\lambda_i$ such that $N_i \to N$ pointwise.
Then $\lambda_i \to \lambda$ in Thurston's geometric topology.
\end{lemma}
\begin{proof}
We must show that for $x \in N$ and $x_i \in N_i$, $N_i$ a leaf of $\lambda_i$ such that $x_i \to x$, $\normal_i := \normal_{N_i}(x_i)$ converges to $\normal := \normal_N(x)$ in the cosphere bundle $S'M$.
If this fails, then we work in normal coordinates based at $x$, so we can think of points of $M$ as vectors in $\RR^d$, and points of $S'M$ as unit vectors in $\RR^d$.
In such coordinates we may view $N$ as the first axis.
Possibly after taking a subsequence of $(\lambda_i)$, we may assume that no matter what sequence $(x_i)$ we choose,
$$|\sin \angle(\normal_i, \normal)| \geq \varepsilon.$$
The geodesic curvature of $N_i$ with respect to the euclidean metric on the tangent space is bounded independently of $i$ in terms of the scalar curvature $R$ of $M$ near $x$, so there exists $r = r(R) > 0$ independent of $i$ such that $N_i$ \dfn{avoids cones} in the sense that for every $v \in \RR^d$ such that $0 < |v| < r$ and $|\sin v| < \varepsilon/2$, $v + x_i$ does not lie in $N_i$.
After shrinking $\varepsilon$, we may assume that $\varepsilon < \max(r/2, 1/100)$.

To obtain a contradiction, we choose $y$ to lie on $N$ and satisfy $\dist(x, y) = \varepsilon$.
For $i$ large, $|x_i| < \varepsilon^2/10$, so
$$0 < \varepsilon - \varepsilon^2 \leq |y - x_i| < 2\varepsilon = r$$
and hence (using a superscript $j$ to indicate the $j$th coordinate)
$$|y^1 - x_i^1| \geq |y - x_i| - |x_i^2| \geq \varepsilon - 2\varepsilon^2.$$
Consider the triangle $\Delta$ whose vertices are $x_i, y$, and $z_i := (y^1, x_i^2)$.
Then $\Delta$ is a right triangle, and its smallest angle $\theta_\Delta$ satisfies
$$|\sin \theta_\Delta| = \frac{|x_i^2|}{|y^1 - x_i^1|} < \frac{\varepsilon^2}{\varepsilon - 2\varepsilon^2} < \frac{\varepsilon}{4}.$$
Thus for every large $i$, $y$ is contained in the cone avoided by $N_i$, and in fact for any $y_i \in N_i$, $\dist(y, y_i) \geq \varepsilon/4$.
But we argued above that any point of $N$ could be approximated by points of $N_i$ for $i$ large, so this is a contradiction.
\end{proof}

\begin{theorem}\label{compactness in flow boxes and Thurston}
Let $(\lambda_i)$ be a tight sequence of finite minimal laminations with bounded curvature. Then:
\begin{enumerate}
\item After passing to a subsequence, we may assume that $(\lambda_i)$ converges to a maximal limit $\lambda$ on the level of flow boxes, and hence in Thurston's geometric topology.
\item The limit $\lambda$ is a Lipschitz minimal lamination satisfying the same curvature bounds as $(\lambda_i)$.
\item If $\mu_i$ is a transverse measure to $\lambda_i$ and for every compact $K \subseteq M$, $\mu_i(K) \lesssim_K 1$, then after passing to a further subsequence, we may assume that $(\lambda_i, \mu_i)$ in fact converges to some measured minimal lamination $(\tilde \lambda, \mu)$ in the weak topology of measures, where $\tilde \lambda$ is a sublamination of the maximal lamination $\lambda$.
\end{enumerate}
\end{theorem}
\begin{proof}
We first select $x_0 \in M$; we will construct suitable flow boxes $F_i$ for $\lambda_i$ with image $B(x_0, r)$.
Here $r$ is to be determined, but at least is smaller than $R(x_0)/2$, where $R(x_0)$ is the injectivity radius of $M$ at $x_0$.

Let $\varepsilon > 0$.
After selecting $r$ small depending on $g$, $\varepsilon$, and the curvature bounds on $\lambda_i$, and rescaling $g$, we may assume that the leaves $N_{i1}, \dots, N_{ik_i}$ have all second fundamental forms of size $\leq \varepsilon$ on $B(x_0, r)$.
Then if $r$ is small enough depending on $g$, the second fundamental forms $\Two_{ij}$ of $N_{ij}$ with respect to the euclidean metric in $x_0$-normal coordinates are of size $\leq 2\varepsilon$ on $B(x_0, r)$.

If $r$ is small enough, then because there is a Lipschitz conormal $1$-form $\normal_i$ to $\lambda_i$ in $B(x_0, r)$, we can represent $N_{i1}, \dots, N_{ik_i}$ as graphs over the equatorial hyperplane in $B(x_0, r)$, possibly after applying a rotation $R_i$.
The uniform bounds on $\Two_{ij}$ ensure that $r$ is independent of $i$.
Thus we have $x_0$-normal coordinates $(x_i, y_i)$ on $B(x_0, r)$ where
$$N_{ij} = \{y_i = f_{ij}(x_i)\}$$
and $\normal_i(0, 0) = \dif y_i$.
Since $|\Two_{ij}| \leq 2\varepsilon$, if $\varepsilon \leq 1$,
\begin{equation}\label{bound on derivatives}
||\dif f_{ij}||_{C^0} \leq |\normal_i(0, f_{ij}(0))| + O(\varepsilon r) \lesssim (1 + \varepsilon) r \lesssim r.
\end{equation}

By the maximum principle, if $f_{ij}(x_i) = f_{ij'}(x_i)$ for some $i, j, j'$, then $j = j'$.
So after applying a permutation to $\{1, \dots, k_i\}$ we may assume that $j < j'$ implies $f_{ij} < f_{ij'}$.
Then if we set $u_{ij} := f_{i,j+1} - f_{ij}$, then $u_{ij} > 0$ by the maximum principle, and we get a second-order divergence-form operator $P_{ij}(\varepsilon, r)$ such that
\begin{equation}\label{elliptic PDE}
-\Delta u_{ij} = P_{ij}(\varepsilon, r) u_{ij}
\end{equation}
and $P_{ij}(\varepsilon, r)$ is perturbative as $\varepsilon, r \to 0$ in the sense that its coefficients can be made arbitrarily small in $C^0(B_{r/2})$ by taking $\varepsilon, r$ small.
In fact, by \cite[Chapter 7]{colding2011course}, if we set
$$F(x, y, p, q) := h^{ij} (q_{ij} + \Gamma_{ij}^0 + p_i \Gamma_{0j}^0 + p_j \Gamma^0_{i0} + p_i p_j \Gamma^0_{00}) - p_m h^{ij} (\Gamma_{ij}^m + p_i \Gamma_{0j}^m + p_i \Gamma_{i0}^m + q_{ij} \Gamma_{00}^m),$$
where $\Gamma_{\mu \nu}^\lambda$ are Christoffel symbols for $M$ at $(x, y)$, $x_0 := (0, 0)$, the zero index corresponds to $y$, and $h^{ij}$ are the components of the inverse of
$$h_{ij} = g_{ij} + p_i g_{j0} + p_j g_{0i} + p_i p_j g_{00}$$
evaluted at $(x, y)$, then $h_{ij} = \delta_{ij} + O(|x|^2 + y^2 + |p|^2)$, $\Gamma_{\mu \nu}^\lambda = O(|x|^2 + y^2)$, and
$$F(x, f_{i, j + 1}(x), \dif f_{i, j + 1}(x), \partial^2 f_{i, j + 1}(x)) - F(x, f_{ij}(x), \dif f_{ij}(x), \partial^2 f_{ij}(x)) = 0.$$
By (\ref{bound on derivatives}), we conclude that the dominant term is $\Delta u_{ij}$ and the perturbative terms are $O(r + \varepsilon)$ in $C^0$, as required by (\ref{elliptic PDE}).
TODO: Reindex all this, don't want the ij's to be Einstein indices and leaf indices
It follows that if $\varepsilon, r$ are small, then $-\Delta - P$ is uniformly elliptic, with ellipticity constant close to that of $-\Delta$.
So by the Harnack inequality, for any $0 < \delta < 1/2$,
\begin{equation}\label{Harnack bound}
\sup_{B_{\delta r}} u_{ij} \leq e^{O(\delta)} \inf_{B_{\delta r}} u_{ij}.
\end{equation}

We now set $\eta_{ij} := f_{ij}(0)$ and define
$$\varphi_i(\xi_i, \eta_i) := \sum_{j=1}^{k_j - 1} 1_{[\eta_{ij}, \eta_{i,j+1})}(\eta_i) \left[f_{ij}(\xi_i) + \frac{\eta_i - \eta_{ij}}{\eta_{i,j+1} - \eta_{ij}} u_{ij}(x_i)\right].$$
That is, the change of coordinates
$$(x_i, y_i) = (\xi_i, \varphi_i(\xi_i, \eta_i))$$
flattens out the graphs of the $f_{ij}$ to hyperplanes.
Moreover
\begin{align*}
\dif \varphi_i &= \sum_{j=1}^{k_j - 1} 1_{[\eta_{ij}, \eta_{i, j + 1})}(\eta_i) \left[\dif f_{ij} + \frac{\eta_i - \eta_{ij}}{\eta_{i,j+1} - \eta_{ij}} \dif u_{ij} + \frac{u_{ij}}{\eta_{i,j+1} - \eta_{ij}} \dif \eta_i\right] \\
&\qquad + \sum_{j = 2}^{k_j - 1} \delta_{\eta_{ij}}(\eta_i)\left[f_{i,j-1}(\xi_i) + u_{i,j-1}(\xi_i) - f_{i,j}(\xi_i)\right] \\
&= \sum_{j=1}^{k_j - 1} 1_{[\eta_{ij}, \eta_{i, j + 1})}(\eta_i) \left[\dif f_{ij} + \frac{\eta_i - \eta_{ij}}{\eta_{i,j+1} - \eta_{ij}} \dif u_{ij} + \frac{u_{ij}}{\eta_{i,j+1} - \eta_{ij}} \dif \eta_i\right].
\end{align*}
Now $\eta_i - \eta_{ij} \in [0, \eta_{i, j + 1} - \eta_{ij}]$ for $\eta_i \in [\eta_{ij}, \eta_{i,j+1})$, so the first two terms are bounded by (\ref{bound on derivatives}) as
$$\left|\left|\dif f_{ij} + \frac{\eta_i - \eta_{ij}}{\eta_{i,j+1} - \eta_{ij}} \dif u_{ij}\right|\right|_{C^0(B_{\delta r})}
\leq 3\sup_{j' \in \{1, \dots, k_i\}} ||\dif f_{ij'}||_{C^0(B_{\delta r})}
\lesssim \delta r.$$
From (\ref{Harnack bound}),
$$\frac{u_{ij}}{\eta_{i,j+1} - \eta_{ij}} = \frac{u_{ij}(0) + \sup_{B_{\delta r}} u_{ij} - \inf_{B_{\delta r}} u_{ij}}{u_{ij}(0)} = 1 + O(\delta r).$$
In conclusion
$$\dif \varphi_i = \dif \eta_i + O(\delta r).$$
Therefore the flow boxes $F_i(\xi_i, \eta_i) := (x_i, y_i)$ have Lipschitz norm and conorm in $(1/2, 3/2)$ if $\delta$ is chosen small enough independently of $i$.
So by a compactness argument, a subsequence of flow boxes and their inverses converge in $C^{1-}$ to a local diffeomorphism $F$.
Since $x_0$ was arbitrary, a diagonal argument ensures that we can cover $M$ by sets $U_\alpha := B(x_0, \delta r)$ in which these flow boxes $F_{i\alpha}$ converge to some local diffeomorphism $F_\alpha$.
By taking a further subsequence and rotating, we may assume that $R_i$ converges to the identity as well.

Since $(\lambda_i)$ is tight, we can find a precompact ball $B$ in $M$ so that every leaf of $\lambda_i$ passes through $B/2$.
Then by compactness of $\Hypspace \overline B$, we obtain limits of every sequence $(N_i)$ of $\lambda_i$ in $\Hypspace \overline B$.
By Lemma \ref{limit of minimals is minimal}, such a limit $N$ is a geodesic.
Then $F_{i\alpha}^{-1}(N_i \cap U_\alpha) = \{\eta = \eta_i\}$ for some $\eta_i$, and by the $C^{1-}$-convergence of the $F_i$, the hyperplanes $\{\eta = \eta_i\}$ must converge to some hyperplane $\{\eta = \eta_\infty\}$, so that $F_\alpha(\{\eta = \eta_\infty\}) = N \cap U_\alpha$.
So the limiting geodesics $N$ form a geodesic lamination $\lambda$ and $F_\alpha$ is a flow box for $\lambda$.
Diagonalizing against larger and larger compact balls, we extend the geodesic lamination $\lambda$ in $B$ to a lamination in $M$.

The above convergence is in Thurston's geometric topology.
Indeed, if $N$ is a leaf of $\lambda$, then we can find leaves $N_i$ such that $N_i \to N$ pointwise.
So by Lemma \ref{convergence of geodesic lams in thurston} and the fact that $N$ is a geodesic, we conclude the desired convergence.

Finally, if $\mu_i$ is transverse to $\lambda_i$, then we observe that since $(\lambda_i)$ is tight, we can find a compact subset $K$ of $M$ which meets every leaf of $\lambda$.
Then $\lambda, \lambda_i$ are orientable on $K \cap U_\alpha$, so we can take $T_{\mu_i}|_{K \cap U_\alpha}$ and obtain a convergent subsequence to some $T_\mu|_{K \cap U_\alpha}$; here we use the uniform bounds on $\mu_i(K)$.
Diagonalizing again, we obtain $\mu := |T_\mu|$, which is a transverse measure to $\lambda|_K$, and is globally defined since it is not oriented.
Since $K$ meets every leaf of $\lambda$, $\mu$ extends uniquely to all of $M$ as a transverse measure to $\lambda$.
\end{proof}

%%%%%%%%%%%%%%%%%%%%%%%%%
\subsection{Convergence in flow boxes versus the weak topology of measures}
\begin{theorem}\label{measures implies flow boxes}
Let $(\lambda_i, \mu_i)$ be measured geodesic laminations.
If $(\lambda_i, \mu_i) \to (\lambda, \mu)$ in the weak topology of measures, then $\lambda_i \to \lambda$ on the level of flow boxes.
\end{theorem}
\begin{proof}
The question is local, so we may remove all leaves of $\lambda$ except those that meet some precompact open set.
Then, for each $i$, let $(\lambda_{ij})_j$ be finite sublaminations of $\lambda_i$ converging to $\lambda_i$ on the level of flow boxes.
Then any subsequence $(i_k)$, has a further subsequence $(i_{k_\ell})$ such that $\lambda_{i_{k_\ell} i_{k_\ell}} \to \lambda$ on the level of flow boxes by Theorem \ref{compactness in flow boxes and Thurston}.
Indeed, as geodesic laminations, they have bounded curvature, and $(\lambda_i)$ is tight by construction.
So $\lambda_{i_{k_\ell}} \to \lambda$ on the level of flow boxes, but $(i_k)$ was arbitrary, so $\lambda_i \to \lambda$ on the level of flow boxes.
\end{proof}

% \begin{lemma}
% For every geodesic lamination $\lambda$ and every $x \in \supp \lambda$, there is a transverse curve to $\lambda$ through $x$.
% \end{lemma}
% \begin{proof}
% Since $\normal_\lambda^\sharp$ is Lipschitz with no zeroes, it extends to a Lipschitz vector field $X$ with no zeroes in a neighborhood of $\supp \lambda$.
% Then any integral curve of $X$ through $x$ is transverse to $\lambda$.
% \end{proof}

% \begin{lemma}
% Let $\lambda$ be a geodesic lamination and $\gamma$ a transverse curve to $\lambda$.
% Then $\lambda$ extends to a geodesic foliation $\overline \lambda$ in a neighborhood of $\gamma(I)$, such that the Lipschitz constant of $\normal_{\overline \lambda}$ is the Lipschitz constant of $\normal_\lambda$.
% \end{lemma}
% \begin{proof}
% Extend $(\normal_\lambda^\sharp)^\perp$, which is tangent to the leaves of $\lambda$, to a Lipschitz vector field $Y$ on a neighborhood of $\gamma(I)$ with a comparable Lipschitz constant to $(\normal_\lambda^\sharp)^\perp$. Then
% $$F(k, y) = \exp_{\gamma(k)}(yY(\gamma(k)))$$
% is a flow box for $\lambda$ in a neighborhood of $\gamma$ such that every fiber $\{k\} \times \RR$, not just those for which $k \in K_F$, is a geodesic.
% So $F$ is a flow box for a geodesic foliation $\overline \lambda$ which extends $\lambda$ near $\gamma(I)$.
% Moreover, $\normal_{\overline \lambda} = -(Y^\perp)^\flat$ has a comparable Lipschitz constant to $\normal_\lambda$ by construction.
% \end{proof}

% \begin{lemma}
% If $\lambda_i \to \lambda$ in Thurston's geometric topology and $x \in N$, $N$ a leaf of $\lambda$, then for any sequence of leaves $N_i$ of $\lambda_i$ which contain $x_i \in N_i$ such that $x_i \to x$, $\normal_{N_i} \to \normal_N$ uniformly in a neighborhood of $x$.
% \end{lemma}
% \begin{proof}
% By definition of Thurston's geometric topology, $\normal_{N_i}(x_i) \to \normal_N(x)$.
% However, $N_i, N$ are geodesics, so $\normal_{N_i}, \normal_N$ are covariantly constant.
% In particular they remain close on small sets.
% \end{proof}

% \begin{lemma}
% If $\lambda$ is a geodesic lamination, and $\gamma$ is a transverse curve, there exists a Lipschitz unit-length vector field $X$ along $\gamma$ such that if $\gamma(t) \in N$ for some leaf $N$ of $\lambda$, then $X(\gamma(t))$ is tangent to $N$.
% Moreover, the map that sends a geodesic lamination to a Lipschitz vector field is continuous for Thurston's geometric topology.
% \end{lemma}
% \begin{proof}
% First, $(\normal_\lambda^\sharp)^\perp$ is Lipschitz, unit-length, tangent to the leaves of $\lambda$, and defined on $\supp \lambda \cap \gamma(I)$.
% We then let
% \begin{equation}\label{tangent frame to a lamination}
% 	X := E(\normal_\lambda^\sharp)^\perp.
% \end{equation}

% To see the continuity, suppose that $\lambda_i \to \lambda$ in Thurston's geometric topology.
% Let $x \in \supp \lambda$; by taking a subsequence we can find $x_i \in \supp \lambda_i$ such that $x_i \to x$ and $\normal_{\lambda_i}(x_i) \to \normal_\lambda(x)$.
% So if $N_i, N$ are the leaves containing $x_i, x$, then $\normal_{N_i} \to \normal_N$ uniformly in a neighborhood of $x$.
% Thus (TODO) $\normal_{\lambda_i} \circ \gamma \to \normal_\lambda \circ \gamma$ in $W^{1, \infty}_p(I, M)$, hence if we define $X_i$ by (\ref{tangent frame to a lamination}), $X_i \to X$ in $W^{1, \infty}(I, M)$.
% \end{proof}


% \begin{proof}
% By Proposition \ref{measured implies Thurston}, $\lambda_i \to \lambda$ in Thurston's geometric topology and $\lambda$ is a geodesic lamination.
% Let $\gamma$ be a transverse curve to $\lambda$, and choose a Lipschitz unit vector field $X$ along $\gamma$ which is tangent to the leaves of $\lambda$.
% Then we set
% $$F(k, y) := \exp_{\gamma(k)}(yX(\gamma(k))).$$
% Since $\exp_{\gamma(k)}$ sends lines through the origin to geodesics through $\gamma(k)$, and $X$ is Lipschitz, $F$ is a flow box for $\lambda$.
% Since $\gamma$ is eventually transverse to $\lambda_i$, we obtain flow boxes $F_i$ for each of the $\lambda_i$, with associated vector fields $X_i$ as well, such that $X_i \to X$ in $W^{1, \infty}$. Therefore $F_i \to F$ in $W^{1, \infty}$.
% \end{proof}



%%%%%%%%%%%%%%%%%%%%%%%%%%%

\subsection{Construction of minimal laminations}
We conclude this paper by applying the above theory to prove a condition for a collection of minimal hypersurfaces to form a minimal lamination.
This condition will be used in the companion papers \cite{BackusFLG, DaskalopoulosPrep2}.

\begin{definition}
A \dfn{minimal partition} is a closed subset $\lambda$ of $M$ which has been partitioned into (disjoint, embedded, without boundary) minimal hypersurfaces, called the \dfn{leaves} of $\lambda$.
\end{definition}

\begin{theorem}\label{partition implies lamination}
Let $\lambda$ be a minimal partition such that the second fundamental forms of the leaves of $\lambda$ are locally uniformly bounded.
Then $\lambda$ is a minimal lamination.
\end{theorem}
\begin{proof}
Let $(x_i)$ be a dense sequence in $\supp \lambda$, and let $N_i$ be the leaf containing $x_i$.
Then the union of $N_1, \dots, N_i$ is the support of a lamination $\lambda_i$, so that the curvatures of $\lambda_i$ are locally uniformly bounded.
So $(\lambda_i)$ has a maximal subsequential limit $\tilde \lambda$ on the level of flow boxes by Theorem \ref{compactness in flow boxes and Thurston}.
In particular, the leaves of $\tilde \lambda$ are exactly the limits in Hausdorff distance of the leaves of $\lambda$, but $\Leaves \lambda$ is closed so $\lambda = \tilde \lambda$.
\end{proof}

\begin{corollary}
Let $\lambda$ be a geodesic partition. Then $\lambda$ is a geodesic lamination.
\end{corollary}

The above result is also useful when combined with the below stable Bernstein theorem.\footnote{so-called because this theorem is essentially equivalent to the assertion that a stable two-sided minimal hypersurface in $\RR^d$ is a hyperplane.}

\begin{proposition}[stable Bernstein theorem]
Let $3 \leq d \leq 4$ and let $N$ be a two-sided stable minimal hypersurface in $B_r \subseteq M$, $r \lesssim 1$, where $M$ has bounded geometry and dimension $d$.
Then on $B_{r/2}$, $|\Two_N| \lesssim_M r^{-1}$.
\end{proposition}
\begin{proof}
For $d = 3$ see \cite[Corollary 2.11]{colding2011course}, and for $d = 4$ see \cite{Chodosh2021}.
\end{proof}

\begin{corollary}
Let $\lambda$ be a minimal partition in $M$ whose leaves are two-sided and stable, where $M$ has dimension $3 \leq d \leq 4$.
Then $\lambda$ is a minimal lamination.
\end{corollary}

\printbibliography

\end{document}
