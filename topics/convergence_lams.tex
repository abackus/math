\documentclass[reqno,10pt]{amsart}
\usepackage[letterpaper, margin=1in]{geometry}
\RequirePackage{amsmath,amssymb,amsthm,graphicx,mathrsfs,url,slashed,subcaption}
\RequirePackage[usenames,dvipsnames]{xcolor}
\RequirePackage[colorlinks=true,linkcolor=Red,citecolor=Green]{hyperref}
\RequirePackage{amsxtra}
\usepackage{cancel}
\usepackage{tikz-cd}

% \setlength{\textheight}{9.3in} \setlength{\oddsidemargin}{-0.25in}
% \setlength{\evensidemargin}{-0.25in} \setlength{\textwidth}{7in}
% \setlength{\topmargin}{-0.25in} \setlength{\headheight}{0.18in}
% \setlength{\marginparwidth}{1.0in}
% \setlength{\abovedisplayskip}{0.2in}
% \setlength{\belowdisplayskip}{0.2in}
% \setlength{\parskip}{0.05in}
%\renewcommand{\baselinestretch}{1.05}

\title{Modes of convergence of minimal laminations}
\author{Aidan Backus}
\date{October 2022}

\newcommand{\NN}{\mathbf{N}}
\newcommand{\ZZ}{\mathbf{Z}}
\newcommand{\QQ}{\mathbf{Q}}
\newcommand{\RR}{\mathbf{R}}
\newcommand{\CC}{\mathbf{C}}
\newcommand{\DD}{\mathbf{D}}
\newcommand{\PP}{\mathbf P}
\newcommand{\MM}{\mathbf M}
\newcommand{\II}{\mathbf I}
\newcommand{\Hyp}{\mathbf H}
\newcommand{\Sph}{\mathbf S}
\newcommand{\Group}{\mathbf G}
\newcommand{\GL}{\mathbf{GL}}
\newcommand{\Orth}{\mathbf{O}}
\newcommand{\SpOrth}{\mathbf{SO}}
\newcommand{\Ball}{\mathbf{B}}

\newcommand*\dif{\mathop{}\!\mathrm{d}}

\DeclareMathOperator{\dist}{dist}
\DeclareMathOperator{\MeasLam}{MeasLam}
\DeclareMathOperator{\MinLam}{MinLam}
\DeclareMathOperator{\Lam}{Lam}
\DeclareMathOperator{\supp}{supp}

\newcommand{\Two}{\mathrm{I\!I}}

\newcommand{\Lagrange}{\mathscr L}
\newcommand{\DirQL}{\mathscr D^{\mathrm{ql}}}
\newcommand{\DirL}{\mathscr D}

\newcommand{\Hilb}{\mathcal H}
\newcommand{\Homology}{\mathrm H}
\newcommand{\normal}{\mathbf n}
\newcommand{\radial}{\mathbf r}
\newcommand{\evect}{\mathbf e}
\newcommand{\vol}{\mathrm{vol}}

\newcommand{\Bmu}{\boldsymbol \mu}
\newcommand{\Bnu}{\boldsymbol \nu}
\newcommand{\Blambda}{\boldsymbol \lambda}

\newcommand{\pic}{\vspace{30mm}}
\newcommand{\dfn}[1]{\emph{#1}\index{#1}}

\renewcommand{\Re}{\operatorname{Re}}
\renewcommand{\Im}{\operatorname{Im}}

\newcommand{\loc}{\mathrm{loc}}
\newcommand{\cpt}{\mathrm{cpt}}

\def\Japan#1{\left \langle #1 \right \rangle}

\newtheorem{theorem}{Theorem}[section]
\newtheorem{badtheorem}[theorem]{``Theorem"}
\newtheorem{prop}[theorem]{Proposition}
\newtheorem{lemma}[theorem]{Lemma}
\newtheorem{sublemma}[theorem]{Sublemma}
\newtheorem{proposition}[theorem]{Proposition}
\newtheorem{corollary}[theorem]{Corollary}
\newtheorem{conjecture}[theorem]{Conjecture}
\newtheorem{axiom}[theorem]{Axiom}
\newtheorem{assumption}[theorem]{Assumption}

\newtheorem{mainthm}{Theorem}
\renewcommand{\themainthm}{\Alph{mainthm}}

% \newtheorem{claim}{Claim}[theorem]
% \renewcommand{\theclaim}{\thetheorem\Alph{claim}}
\newtheorem*{claim}{Claim}

\theoremstyle{definition}
\newtheorem{definition}[theorem]{Definition}
\newtheorem{remark}[theorem]{Remark}
\newtheorem{example}[theorem]{Example}
\newtheorem{notation}[theorem]{Notation}

\newtheorem{exercise}[theorem]{Discussion topic}
\newtheorem{homework}[theorem]{Homework}
\newtheorem{problem}[theorem]{Problem}

\makeatletter
\newcommand{\proofpart}[2]{%
  \par
  \addvspace{\medskipamount}%
  \noindent\emph{Part #1: #2.}
}
\makeatother



\numberwithin{equation}{section}


% Mean
\def\Xint#1{\mathchoice
{\XXint\displaystyle\textstyle{#1}}%
{\XXint\textstyle\scriptstyle{#1}}%
{\XXint\scriptstyle\scriptscriptstyle{#1}}%
{\XXint\scriptscriptstyle\scriptscriptstyle{#1}}%
\!\int}
\def\XXint#1#2#3{{\setbox0=\hbox{$#1{#2#3}{\int}$ }
\vcenter{\hbox{$#2#3$ }}\kern-.6\wd0}}
\def\ddashint{\Xint=}
\def\dashint{\Xint-}

\usepackage[backend=bibtex,style=numeric]{biblatex}
\renewcommand*{\bibfont}{\normalfont\footnotesize}
\addbibresource{topics.bib}
\renewbibmacro{in:}{}
\DeclareFieldFormat{pages}{#1}


\begin{document}
\begin{abstract}
We collect several results relating different notions of convergence of laminations, especially minimal laminations.
We also give a condition for a partition of a closed set into minimal hypersurfaces to be a lamination.
\end{abstract}

\maketitle

%%%%%%%%%%%%%%%%%%%%%%%%%%%%%%%%%%%%%%%%%%%%%%%%%%%%%%%

% \tableofcontents

\section{Introduction}
The purpose of this paper is to record several modes of convergence for laminations, and in particular for minimal laminations, for use in the companion papers \cite{BackusFLG, DaskalopoulosPrep2}.
In such papers it is crucial that a disjoint family of minimal hypersurfaces without boundary -- which we refer to as a \dfn{minimal partition} -- is in fact a minimal lamination.
However this is false in general, as we will discuss, and so we will use the compactness theory discussed in this paper as well as the stable Bernstein theorem \cite{Schoen2016, Chodosh2021} to give a criterion under which a minimal partition is a lamination.

Most of the modes of convergence in this paper and their compactness theorems have been known to various sources \cite{ColdingMinicozziIV, ColdingMinicozziV, thurston1979geometry}.
Our contribution is to synthesize the known results, explaining under what circumstances different modes of convergence are stronger than others, as well as under what circumstances a sequence of laminations has a convergent subsequence.

%%%%%%%%%%%%%%%%%%%%%%%%

\subsection{Acknowledgements}
I would like to thank Georgios Daskalopolous for suggesting this project and for helpful discussions, and NSF...

%%%%%%%%%%%%%%%%%%%%%%%%%%%

\section{Preliminaries}
\begin{definition}
Let $S$ be a closed subset of a smooth manifold $M$.
\begin{enumerate}
\item A (codimension $1$ lamination) \dfn{flow box} for $S$ is a $C^0$ chart in which $S$ can be expressed as $K \times N \subseteq \RR^d$ for some closed $K \subseteq \RR$ and open connected $N \subseteq \RR^{d - 1}$; we call $K$ a \dfn{leaf space}.
\item A (codimension $1$) \dfn{lamination} $\lambda$ in $M$, with support $S$, consists of an atlas of lamination flow boxes whose transition maps preserve the local product structure, such that the preimage of the sets $N$ are $C^2$ hypersurfaces in $M$.
\item A \dfn{leaf} of $\lambda$ is a connected subset of $\supp \lambda = S$ which is locally modeled on the fibers $\{k\} \times N$, $k \in K$.
\item A \dfn{plaque} of $\lambda$ is a connected component of $M \setminus S$.
\end{enumerate}
\end{definition}

Here we assume that the leaves are $C^2$ in order to ensure that the normal vectors to each leaves are well-defined in $C^1$.
This in particular ensures that the second fundamental forms to each leaf are well-defined in $C^0$.
Control on the second fundamental forms will be a crucial assumption throughout this paper and so we define the seminorms
$$||\lambda||_K := \sup_{N \in \mathscr L(\lambda)} \sup_{x \in N \cap K} |\Two_N|$$
where $\mathscr L(\lambda)$ is the set of all leaves of $\lambda$, $K$ ranges over compact subsets of $M$, $\Two_N$ is the second fundamental form of $N$, and $|\Two_N|$ is taken with respect to some fixed Riemannian metric on $M$.
Control on $||\lambda||_K$ in turn allows one to control the Lipschitz seminorm of the flow boxes of $\lambda$ in $K$.

Tracing the second fundamental forms we obtain the mean curvature, another crucial invariant through this paper because it defines minimal laminations.

\begin{definition}
Let $M$ be a Riemannian manifold.
A lamination $\lambda$ is \dfn{minimal} if every leaf of $\lambda$ has zero mean curvature.
A \dfn{geodesic lamination} is a minimal lamination in a surface.
\end{definition}

In most applications one is interested in geodesic laminations, or minimal laminations in threefolds.
In those cases one can easily control $||\lambda||_K$ in terms of simpler invariants: for geodesic laminations, $||\lambda||_K = 0$, and for minimal laminations in threefolds, $||\lambda||_K$ is controlled by the Gauss curvature of the leaves of $\lambda$ on $K$.

%%%%%%%%%%%%%%%%%%%%%%%%%%%%%%

\subsection{Regularity of minimal laminations}
Though we impose that laminations are $C^0$ (in the sense that their flow boxes are $C^0$) we can improve this regularity slightly in the case of minimal laminations.
In this section we follow \cite{Solomon86}, which treats the special case that $\lambda$ is a minimal foliation ($\supp \lambda = M$) of $M \subseteq \RR^d$.

\begin{proposition}
Let $\lambda$ be a minimal lamination. Then $\lambda$ admits a Lipschitz atlas.
\end{proposition}

\begin{example}
Lipschitz regularity is optimal, even in the nicest possible case of a geodesic foliation of a complete Riemannian manifold.
Indeed, let $M = B((2, 0), 1)$ in $\RR^2$. Then the chords $\{y = ax\}$ for $a > 0$ and $\{y = a\}$ for $a \leq 0$ define a foliation $\lambda$ of $M$.
If we equip $M$ with the Beltrami-Klein metric, then, since the leaves of $\lambda$ are chords, $\lambda$ defines a geodesic foliation of $M \cong \Hyp^2$.
Moreover the conormal to the foliation is
$$\normal = \frac{1_{y > 0}}{\sqrt{1 + y^2/x^2}} \left[\dif y - \frac{y}{x} \dif x\right] + 1_{y \leq 0} \dif y$$
which is clearly not $C^1$.
\end{example}

%%%%%%%%%%%%%%%%%%%%%%%%%%%%%%
\subsection{Transverse measures}
Let $X$ be a locally compact Polish space, and $C_0(X)$ the space of continuous functions on $X$ which vanish at infinity.
Then $C_0(X)$ naturally has a locally convex structure, induced by the natural isomorphism
$$C_0(X) = \varinjlim C(K)$$
where $K$ ranges over all compact subsets of $X$, thus $C(K)$ is a Banach space with respect to the supremum norm.
Moreover, by the Riesz-Markov theorem, $C_0(X)'$ is naturally isomorphic to the space of (signed) Radon measures on $X$.
The weak topology on $C_0(X)'$ is therefore known as the \dfn{weak topology of measures}.
Unpacking the definitions, this means that a sequence $(\mu_i)$ of Radon measures converges to $\mu$ iff for every $f \in C_0(X)$,
$$\lim_{i \to \infty} \int_X f \dif \mu_i = \int_X f \dif \mu.$$

If we have a second locally compact Polish space $Y$, equipped with a Radon measure $\mu \in C_0(Y)'$ and a Borel map $\pi: Y \to X$, we can view $Y$ as a ``measure-theoretic fiber bundle'' over $X$ by equipping $X$ with the pushforward measure
$$\pi_* \mu(E) := \mu(\pi^{-1}(E)).$$
By the disintegration theorem, each fiber $Y_x := \pi^{-1}(x)$ naturally carries the structure of a measured space, where the measure $\mu_x$ on $Y_x$ characterized by the formula
$$\int_Y f \dif \mu = \int_X \int_{Y_x} f(y) \dif \mu_x(y) \dif \pi_* \mu(x)$$
for every $f \in C_0(Y)$.


\begin{definition}
Let $\lambda$ be a lamination.
\begin{enumerate}
\item A path $\gamma: [0, 1] \to M$ is \dfn{transverse} to $\lambda$ if for every flow box $U \times N$, the projection of $\gamma$ onto $U$ is a homeomorphism onto its image, and the endpoints $\gamma(0), \gamma(1)$ lie in plaques of $\lambda$.
\item A Radon measure $\mu$ on $M$ is \dfn{transverse} to $\lambda$ if $\supp \mu \subseteq \supp \lambda$ and, for every homotopy of paths $(\gamma_t)$ which is transverse to $\lambda$,
$$\frac{\dif}{\dif t} \mu(\gamma_t([0, 1])) = 0.$$
\item The pair $(\lambda, \mu)$ is a \dfn{measured lamination} (with full support) if $\mu$ is a transverse measure to $\lambda$ and $\supp \mu = \supp \lambda$.
\end{enumerate}
\end{definition}

Owing to the disintegration theorem, there is an equivalent characterization of transverse measures in terms of local coordinates.

\begin{proposition}
Let $\lambda$ be a lamination, equipped with an atlas of flow boxes with transition maps $\psi_{\alpha\beta}$ and leaf spaces $K_\alpha$, $\alpha, \beta \in A$.
\begin{enumerate}
\item If $\mu$ is a transverse measure to $\lambda$, and $\pi_\alpha$ is the natural projection of the $\alpha$th flow box onto $K_\alpha$, then the pushforward measures
\begin{equation}\label{pushforward formula}
\mu_\alpha(E) := (\pi_\alpha)_* \mu(E)
\end{equation}
are supported on the leaf space $K_\alpha$ and satisfy $(\psi_{\alpha\beta})^* \mu_\alpha = \mu_\beta$.
\item Conversely, if we have a family of Radon measures $\mu_\alpha \in C_0(K_\alpha)'$ with full support satisfying $(\psi_{\alpha\beta})^* \mu_\alpha = \mu_\beta$, then there is a transverse measure $\mu$ to $\lambda$ such that (\ref{pushforward formula}) holds.
\end{enumerate}
\end{proposition}

TODO: Currents and orientations. What does it mean for a current to be normal? and cooriented?
Alternatively, maybe all this stuff should go in the FLG paper. I'm not sure.


%%%%%%%%%%%%%%%%%%%%%

\section{The modes of convergence}


%%%%%%%%%%%%%%%%%%%%%%%%%%%%%%%%%%%%%%%

\subsection{Thurston's geometric topology}
We now recall the ``geometric topology'' introduced by Thurston \cite[\S8.10]{thurston1998minimal}.
To state it we let $\normal_N(x)$ denote the unit-length conormal $1$-form to a hypersurface $N$ at a point $x$ with respect to some Riemannian metric.

\begin{definition}
A sequence of laminations $\lambda_i$ converges to $\lambda$ in \dfn{Thurston's geometric topology} if, for every leaf $N$ in $\lambda$ and every $x \in N$ there exist leaves $N_i$ of $\lambda_i$, and $x_i \in N_i$, such that $x_i \to x$ and $\normal_{N_i}(x_i) \to \normal_N(x)$.
\end{definition}

It is straightforward to see that this topology does not depend on the Riemannian metric, or on the choice of trivialization that was used to make sense of the convergence $\normal_{N_i}(x_i) \to \normal_N(x)$.

I think we actually want to define the limit to be the MAXIMAL lamination to which all the leaves converge to.
Otherwise (as Thurston already observed) the limits are nonunique (and in fact everything converges to the empty lamination, so ``every sequence has a convergent subsequence'' is a vacuous statement).

Since Thurston's geometric topology requires convergence of normal vector fields pointwise, it ``respects the regularity of the flow boxes" in some sense.
On the other hand this makes it somewhat hard to prove convergence in this topology.
To overcome this issue we use the following lemma:

\begin{lemma}
Let $(\lambda_i)$ be a sequence of minimal laminations, and $\lambda$ a lamination, such that for every leaf $N$ in $\lambda$ and every $x \in N$ there exist leaves $N_i$ of $\lambda_i$, and $x_i \in N_i$, such that $x_i \to x$.
Then $\lambda_i \to \lambda$ in Thurston's geometric topology, and $\lambda$ is minimal.
\end{lemma}
\begin{proof}
We must show that $\normal_N(x)$ is the limit of $\normal_{N_i}(x_i)$, possibly along a subsequence.
If this is false then
\end{proof}

%%%%%%%%%%%%%%%%%%%%%%%%%%%%%%%%%

\subsection{Hausdorff distance}
Convergence in Hausdorff distance is also called ``convergence of the leaves as sets''.
In order to give the definition of Hausdorff distance between two laminations, we need to first recall the definition of Hausdorff distance for closed subsets of a compact metric space \cite[Chapter 4]{nadler2017continuum}.

\begin{definition}
Let $X$ be a compact metric space. The \dfn{Hausdorff distance} between two closed sets $A, B \subset X$ is
$$\dist(A, B) := \max\left(\max_{a \in A} \min_{b \in B} \dist(a, b), \max_{b \in B} \min_{a \in A} \dist(a, b)\right).$$
The space of closed subsets of $X$ is the \dfn{hyperspace} $\mathscr H(X)$.
\end{definition}

For a compact metric space $X$, $\mathscr H(X)$ is a compact metric space, and its topology is determined by the topology on $X$ \cite[Corollary 4.6, Theorem 4.13]{nadler2017continuum}.

It is natural to phrase definitions that involve minimax conditions (or equivalently, that have a high quantifier complexity) in terms of games.
In the case of Hausdorff distance, we consider the following game:
\begin{enumerate}
\item Alice chooses either $A$ or $B$, and a point $x$ in the set she chose.
\item Bob chooses a path $\gamma$ from $x$ to any point in the set Alice did not choose.
\end{enumerate}
The payoff for Alice is the length of $\gamma$.
It is straightforward to see that if Alice and Bob both play optimally then Alice's payoff is $\dist(A, B)$.

We now upgrade this topology on the space of closed sets to a topology on the space of laminations.
To do so, recall that a \dfn{compact exhaustion} of a manifold $M$ consists of an increasing chain $(U_n)$ of open subsets of $M$, such that the closures $\overline{U_n}$ are compact and such that $\bigcup_n U_n = M$.
We shall in addition assume that $\partial U_n$ is $C^\infty$, and that if $M$ is compact, then $U_n = M$.

\begin{definition}
Let $M$ be a Riemannian manifold and $(U_n)$ a compact exhaustion of $M$.
We say that a sequence $(\lambda_i)$ of laminations converges to a lamination $\lambda$ in \dfn{Hausdorff distance} if, for every leaf $N$ of $\lambda$, there exists leaves $N_i$ of $\lambda_i$ such that for every $n$, $N_i \cap \overline{U_n} \to N \cap \overline{U_n}$ in $\mathscr H(\overline{U_n})$.
\end{definition}

Analogously to the definition of the topology on a Fr\'echet space, we say that a sequence of laminations converges for Hausdorff distance if it converges in the $n$th Hausdorff distance for every $n$ (but possibly not uniformly as $n \to \infty$).
Since the topology induced by Hausdorff distance on the level of closed sets is independent of the metric, the convergence in Hausdorff distance on the level of laminations is also independent of any Riemannian structure one may put on $M$.
One can also show that the topology is independent of the choice of compact exhaustion.

There is another reasonable-sounding definition of Hausdorff distance for laminations, where one allows Hausdorff distance to be defined on \emph{arbitrary} metric spaces and considers the Hausdorff distance on closed subsets of $M$, rather than closed subsets of the closures of entries of a compact exhaustion of $M$.
One would of course need to replace certain maxes and mins with sups and infs as necessary.
We call this topology the \dfn{global Hausdorff distance}.
However, we view this topology as somewhat pathological if $M$ is noncompact, and so we will never use it.
For example, convergence of minimal laminations in Thurston's geometric topology does not imply convergence in global Hausdorff distance:

\begin{example}
Consider $\Hyp^2$ with its disk model in $\CC$, and let $\gamma$ be the horizontal geodesic through the origin $0$.
Consider also a sequence of geodesics $(\gamma_j)$ through $i/j$ which are horizontal at $i/j$.
Then we can define a geodesic lamination $\lambda$ consisting of each of these geodesics, and sublaminations $\lambda_j$ which contain $\gamma_1, \dots, \gamma_j$.
Then $\lambda_j \to \lambda$ in Thurston's geometric topology and in Hausdorff distance.
However,
$$\dist_{\mathscr H(\Hyp^2)}(\gamma_j, \gamma) = \infty.$$
Indeed, Alice can choose points $x$ on $\gamma$ which are arbitrarily close to the circle $\partial \Hyp^2$ at infinity.
Then Bob would play optimally by choosing the geodesic $\rho$ which is orthogonal to $\gamma$ at $x$.
Then the segment of $\rho$ between $\gamma$ and $\gamma_j$ would become arbitrarily long as $x \to \partial \Hyp^2$.
So in global Hausdorff distance, $\lambda_j$ and $\lambda$ are infinitely far apart!
\end{example}

%%%%%%%%%%%%%%%%%%%%%%%%%%%%%%%%%%%

\subsection{The weak topology of measures}
Another topology in common use is the weak topology on measures on the space of measured laminations, due to Thurston \cite[Chapter 8]{thurston1998minimal}.

\begin{definition}
A sequence $(\lambda_i, \mu_i)$ of measured laminations (with full support) converges in the \dfn{weak topology of measures} to a measured lamination $(\lambda, \mu)$ if $(\mu_i)$ converges to $\mu$ in the weak topology of measures.
\end{definition}

Let us extoll some advantages and disadvantages of the weak topology of measures.
First, a lamination does not come equipped with a canonical transverse measure, and in fact may have \emph{no} transverse mesures whatsoever (since we require that they have full support).
This can happen in the nicest possible case of geodesic laminations:

\begin{example}
Not every geodesic lamination, admits a transverse measure.
Let $M$ be a punctured torus.
Then we can select a Riemannian metric on $M$ so that an unpunctured neck of $M$ admits a simple closed geodesic $\gamma_1$, and in addition a geodesic $\gamma_2$ from the puncture which wraps around the neck infinitely many times and converges to $\gamma_1$.
Then $\gamma_1, \gamma_2$ define a geodesic lamination $\lambda$, since the union of their images is closed.
But any transverse path $\rho$ through $\gamma_1$ passes through $\gamma_2$ infinitely many times, and each time, if $\lambda$ admits a transverse measure $\mu$, the measure of $\rho$ increases by $\mu_\alpha(\{\gamma_2\}) > 0$.
Thus $\mu$ is not locally finite but is Radon, a contradiction.
\end{example}

Second, the weak topology of measures is not directly related to $||\lambda||_K$ (at least if $\lambda$ is nonminimal).
This is a blessing and a curse: it makes convergence in the weak topology of measures easy to prove, but also not particularly useful without a minimality assumption.

Finally, in the companion papers \cite{BackusFLG, DaskalopoulosPrep2}, the weak topology of measures will be useful because we are interested in Ruelle-Sullivan currents, defined in terms of a transverse measure and a global normal vector field as below:

\begin{proposition}\label{construction of ruelle sullivan currents}
Let $\lambda$ be a lamination, $\omega$ a $d-1$-current which is normal to $\lambda$, and $(\chi_\alpha)_{\alpha \in A}$ a partition of unity subordinate to an atlas of flow boxes for $\lambda$ which is cooriented with $\omega$ and leaf spaces $(K_\alpha)$.
Then $\mu := \star |\omega|$ is a transverse measure to $\lambda$, such that for every continuous $d-1$-form $\varphi$ of compact support,
$$\int_M \omega \wedge \varphi = \sum_{\alpha \in A} \int_{K_\alpha} \int_{N_{\alpha k}} \chi_\alpha \varphi \dif \mu_\alpha(k)$$
where $N_{\alpha k}$ denotes the $k$th leaf in the $\alpha$th flow box.
\end{proposition}

\begin{definition}
The current $\omega$ of Proposition \ref{construction of ruelle sullivan currents} associated to the oriented measured lamination $(\lambda, \mu)$ is called the \dfn{Ruelle-Sullivan current} associated to $(\lambda, \mu)$.
\end{definition}

%%%%%%%%%%%%%%%%%%%%%%%

\subsection{Convergence in flow boxes}
Let $C^{1-}$ be the Fr\'echet space $\bigcap_{\alpha < 1} C^\alpha$, where $C^\alpha$ are H\"older spaces.
The following definition is one possible interpretation of the vague definition of \cite[Appendix B]{ColdingMinicozziIV}.

\begin{definition}
A sequence $(\lambda_i)$ in $\Lam M$ \dfn{converges on the level of flow boxes} to $\lambda$ if we can find atlases
$$F_{i\alpha}: U_\alpha \to V_\alpha \times N_\alpha$$
of flow boxes for $\lambda_i$, and an atlas $F_\alpha: U_\alpha \to V_\alpha \times N_\alpha$ of flow boxes for $\lambda$, such that $F_{i\alpha} \to F_\alpha$ in $C^{1-}$, and the leaf space $K_\alpha \subseteq V_\alpha$ of $\lambda$ is the set of of all $\lim_{i \to \infty} k_{i\alpha}$ where $k_{i\alpha} \in K_{i\alpha}$.
\end{definition}


%%%%%%%%%%%%%%%%%%%%%%%

\section{Implications between modes}

\subsection{Thurston's geometric topology versus Hausdorff distance}


\begin{example}
Let $f_i$ be a smooth approximation to the Dirac measure $\delta_{1/2-1/i}$ at $1/2 - 1/i$ with compact support in $(1/2 - 2/i, 1/2)$, so $f_i \to 0$ pointwise.
Let $\lambda_i$ be the lamination consisting of the graph of $f_i$ in $\RR^2$ and $\lambda$ the lamination consisting of the horizontal line through $0$.
Then $\lambda_i$ converges to $\lambda$ in Thurston's geometric topology but not in Hausdorff distance.
TODO: Include a picture
\end{example}

\begin{example}\label{Hausdorff does not imply Thurston}
Consider the sequence of continuous functions $f_i: \RR \to \RR$ where $f_i$ is constant away from $[0, 1/i^2]$, is $0$ on $(-\infty, 0]$, is $1/i$ on $[1/i^2, \infty)$, and is linear on $[0, 1/i^2]$.
By smoothing out the graphs of $f_i$ appropriately one obtains a sequence of $C^1$ curves $\gamma_i$ in $\RR^2$, whose geodesic curvatures blow up near $(0, 0)$, and if we set $\lambda_i$ to be the union of $\gamma_1, \dots, \gamma_i$, then $(\lambda_i)$ is a sequence of laminations which converges in Hausdorff distance but not in Thurston's geometric topology.
TODO: Include a picture.
\end{example}

On the other hand, if we assume minimality, then we can prove equivalence.

\begin{proposition}
Let $(\lambda_i)$ be a sequence of minimal laminations.
Then $(\lambda_i)$ converges in Thurston's geometric topology iff it converges in Hausdorff distance, and then its limit lamination is minimal.
\end{proposition}
\begin{proof}

\end{proof}


%%%%%%%%%%%%%%%%%%%%%%%%%%%%%%%%%%%%%%%%

\subsection{The weak topology of measures}
TODO: Cite Thurston, give ounterexamples to show we need minimality

\begin{proposition}
Let $(\lambda_i, \mu_i)$ be a sequence of measured minimal laminations which converge in the weak topology of measures to $(\lambda, \mu)$.
Then $\lambda$ is minimal and $\lambda_i \to \lambda$ in Thurston's geometric topology and Hausdorff distance.
\end{proposition}
\begin{proof}

\end{proof}

%%%%%%%%%%%%%%%%%%%%%%%%%%%%%%%%%%

\subsection{Flow boxes and Thurston's geometric topology}
\begin{proposition}
Let $(\lambda_i)$ be a sequence of minimal laminations which converges on the level of flow boxes to $\lambda$.
Then $\lambda_i \to \lambda$ in Thurston's geometric topology.
\end{proposition}
\begin{proof}
The statement is local so we work in a single flow box.
Let $N$ be a leaf of $\lambda$, and let $k$ be the index of $N$.
Then there are indices $k_i$ of leaves $N_i$ of $\lambda_i$ such that $k_i \to k$.
So there exists $x_i \in x$ such that $x_i \to x$.
So the normal vectors converge.
\end{proof}

I guess the converse should fail, because presumably there's a sequence of minimal surfaces (in a ball of $\RR^3$ or $\Hyp^3$, say) where the curvature blows up.
Then these guys would have a subsequence in Hausdorff distance by compactness of $\mathscr H$ but not in flow boxes.


%%%%%%%%%%%%%%%%%%%%%%%%%%%
\section{The compactness theorem}
Let us give a precise and general statement of the compactness theorem, \cite[Proposition B.1]{ColdingMinicozziIV}.

\begin{theorem}
Let $(\lambda_i)$ be a sequence of minimal laminations such that for each compact $K$, $||\lambda_i||_K \lesssim_K 1$.
Then after passing to a subsequence, there exists a minimal lamination $\lambda$ such that $\lambda_i \to \lambda$ on the level of flow boxes.
\end{theorem}
\begin{proof}
By a density argument we may assume that $\lambda_i$ has finitely many leaves.
We now select $x_0 \in M$ and construct flow boxes $F_i$ on $B(x_0, r)$ where $r > 0$ is to be determined, but is at least smaller than the injectivity radius $R(x_0)$.

Let $\varepsilon > 0$.
After selecting $r$ small depending on $g$, $\varepsilon$, and $\sup_i ||\lambda_i||_{B(x_0, R(x_0))}$, and rescaling $g$, we may assume that the leaves $N_{i1}, \dots, N_{ik_i}$ have all second fundamental forms of size $\leq \varepsilon/2$ on $B(x_0, r)$.
Then if $r$ is small enough depending on $g$, the second fundamental forms with respect to the euclidean metric on $x_0$-normal coordinates are of size $\leq \varepsilon$ on $B(x_0, r)$.
Moreover, because there is a global Lipschitz normal vector field to $\lambda_i$, we can represent $N_{i1}, \dots, N_{ik_i}$ as graphs over the equatorial hyperplane in $B(x_0, r)$, possibly after applying a rotation $R_i$.
Thus we have coordinates $(x_i, y_i)$ on $B(x_0, r)$ where
$$N_{ij} = \{(y_i = f_{ij}(x_i))\}.$$
Somewhere in here we have to replace $B(x_0, r)$ with a cylinder, just like in the proof of de Giorgi's lemma.

By the maximum principle, the $f_{ij}$ satisfy $f_{ij}(x_i) = f_{ij'}(x_i)$ implies $j = j'$.
It follows that after applying a permutation to $\{1, \dots, k_i\}$ we may assume that $j < j'$ implies $f_{ij} < f_{ij'}$.
Then if we set $u_{ij} := f_{i,j+1} - f_{ij}$, then $u_{ij} > 0$ and we get a second-order operator $P_{ij}$ such that
$$\Delta u_{ij} = P_{ij} u_{ij}$$
and $||P_{ij}|| \lesssim \varepsilon$.
Then for $\delta > 0$, the Schauder and Harnack inequalities give
$$\delta r ||\dif u_{ij}||_{C^0(B_{\delta r})} \lesssim \sup_{B(\delta r)} u_{ij} \leq e^{c_1(\varepsilon) \delta^{c_2(\varepsilon)}} \inf_{B(\delta r)} u_{ij}.$$
We now set $\eta_{ij} := f_{ij}(0)$ and define for $\xi_i \in B_r$ and $\eta_i \in [\eta_{ij}, \eta_{i,j+1}]$
$$\varphi_i(\xi_i, \eta_i) := f_{ij}(\xi_i) + \frac{\eta_i - \eta_{ij}}{\eta_{i,j+1} - \eta_{ij}} f_{ij}(x_i).$$
That is, the change of coordinates
$$(x_i, y_i) = (\xi_i, \varphi_i(\xi_i, \eta_i))$$
flattens out the graphs of the $f_{ij}$ to hyperplanes.
Moreover
$$\dif \varphi_i = \dif f_{ij} + \frac{\eta_i - \eta_{ij}}{\eta_{i,j+1} - \eta_{ij}} \dif u_{ij} + \frac{u_{ij}}{\eta_{i,j+1} - \eta_{ij}} \dif \eta_i.$$
If we chose $R_i$ correctly, then $\dif f_{ij}$ is small, and if we chose $\varepsilon, \delta$ small enough then $\dif u_{ij}$ is small too.
Also we cooked up these coordinates so that $f_{ij} \approx \eta_{ij}$ on $B_{\delta r}$ if $\delta$ is small enough, thus the coefficient on the $\dif \eta_i$ is basically unity.
In particular $\dif \varphi_i \approx \dif \eta_i$ in $L^\infty$.

In conclusion, we can get a flow box $F_i$ by rotating the diffeomorphism $(x_i, y_i) = (\xi_i, \varphi_i(\xi_i, \eta_i))$ by $R_i$, and $F_i$ will have Lipschitz norm and conorm both very close to $1$.
So by a compactness argument we see that a subsequence converges in $C^{1-}$ to a flow box $F$.
The leaves are $C^{1-}$-limits of hypersurfaces with uniformly small second fundamental form so they are $C^{1-}$ as well.
But a $C^{1-}$ limit of minimal hypersurfaces is minimal, and also $C^\infty$.
Finally, the Solomon regularity theory implies that we can upgrade from $C^{1-}$ regularity of $F$ to Lipschitz regularity.

That's very sketchy and there's a lot of details to fill in. But that's the idea.
\end{proof}


%%%%%%%%%%%%%%%%%%%%%%%%%%%

\subsection{Construction of minimal laminations}
We conclude this paper by applying the above theory to prove a condition for a collection of minimal hypersurfaces to form a minimal lamination.
This condition will be used in the companion papers \cite{BackusFLG, DaskalopoulosPrep2}.

\begin{definition}
A \dfn{minimal partition} is a closed subset $\lambda$ of $M$ which has been partitioned into (disjoint, embedded, without boundary) minimal hypersurfaces, called the \dfn{leaves} of $\lambda$.
\end{definition}

\begin{theorem}
Let $\lambda$ be a minimal partition such that the second fundamental forms of the leaves of $\lambda$ are locally uniformly bounded.
Then $\lambda$ is a minimal lamination.
\end{theorem}
\begin{proof}
Let $(x_i)$ be a dense sequence in $\supp \lambda$, and let $N_i$ be the leaf containing $x_i$.
Then the union of $N_1, \dots, N_i$ is the support of a lamination $\lambda_i$.
By assumption, $||\lambda_i||_K \lesssim_K 1$.
So $(\lambda_i)$ has a minimal subsequential limit $\tilde \lambda$ on the level of flow boxes.
We then have to use Thurston's geometric topology to show that $\tilde \lambda = \lambda$.
\end{proof}

The above result is particularly useful when it is combined with the stable Bernstein theorem.\footnote{so-called because this theorem is essentially equivalent to the assertion that a stable two-sided minimal hypersurface in $\RR^d$ is a hyperplane.}

\begin{proposition}[stable Bernstein theorem]
Let $2 \leq d \leq 4$ and let $N$ be a two-sided stable minimal hypersurface in $B_r \subseteq M$, $r \lesssim 1$, where $M$ has bounded geometry and dimension $d$.
Then on $B_{r/2}$, $|\Two_N| \lesssim_M r^{-1}$.
\end{proposition}
\begin{proof}
If $d = 2$ this is trivial since $\Two_N$ has zero independent components and hence $\Two_N = 0$, if $d = 3$ then it was proven by Schoen \cite[Corollary 2.11]{colding2011course}, and if $d = 4$ it was proven by Chodosh--Li \cite{Chodosh2021}.
\end{proof}

\begin{corollary}
Let $\lambda$ be a minimal partition in $M$ whose leaves are two-sided and stable, where $M$ has dimension $2 \leq d \leq 4$.
Then $\lambda$ is a minimal lamination.
\end{corollary}


\printbibliography

\end{document}
