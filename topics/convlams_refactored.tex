\documentclass[reqno,11pt]{amsart}
\usepackage[letterpaper, margin=1in]{geometry}
\RequirePackage{amsmath,amssymb,amsthm,graphicx,mathrsfs,url,slashed,subcaption}
\RequirePackage[usenames,dvipsnames]{xcolor}
\RequirePackage[colorlinks=true,linkcolor=Red,citecolor=Green]{hyperref}
\RequirePackage{amsxtra}
\usepackage{cancel}
\usepackage{tikz-cd}

% \setlength{\textheight}{9.3in} \setlength{\oddsidemargin}{-0.25in}
% \setlength{\evensidemargin}{-0.25in} \setlength{\textwidth}{7in}
% \setlength{\topmargin}{-0.25in} \setlength{\headheight}{0.18in}
% \setlength{\marginparwidth}{1.0in}
% \setlength{\abovedisplayskip}{0.2in}
% \setlength{\belowdisplayskip}{0.2in}
% \setlength{\parskip}{0.05in}
%\renewcommand{\baselinestretch}{1.05}

\title{Minimal laminations and level sets of $1$-harmonic functions}
\author{Aidan Backus}
\address{Department of Mathematics, Brown University}
\email{aidan\_backus@brown.edu}
\date{\today}

\newcommand{\NN}{\mathbf{N}}
\newcommand{\ZZ}{\mathbf{Z}}
\newcommand{\QQ}{\mathbf{Q}}
\newcommand{\RR}{\mathbf{R}}
\newcommand{\CC}{\mathbf{C}}
\newcommand{\DD}{\mathbf{D}}
\newcommand{\PP}{\mathbf P}
\newcommand{\MM}{\mathbf M}
\newcommand{\II}{\mathbf I}
\newcommand{\Hyp}{\mathbf H}
\newcommand{\Sph}{\mathbf S}
\newcommand{\Group}{\mathbf G}
\newcommand{\GL}{\mathbf{GL}}
\newcommand{\Orth}{\mathbf{O}}
\newcommand{\SpOrth}{\mathbf{SO}}
\newcommand{\Ball}{\mathbf{B}}

\newcommand*\dif{\mathop{}\!\mathrm{d}}

\DeclareMathOperator{\card}{card}
\DeclareMathOperator{\dist}{dist}
\DeclareMathOperator{\MeasLam}{MeasLam}
\DeclareMathOperator{\MinLam}{MinLam}
\DeclareMathOperator{\Lam}{Lam}
\DeclareMathOperator{\supp}{supp}
\DeclareMathOperator{\tr}{tr}

\newcommand{\Leaves}{\mathscr L}
\newcommand{\Lagrange}{\mathcal L}
\newcommand{\Hypspace}{\mathscr H}

\newcommand{\Two}{\mathrm{I\!I}}


\newcommand{\Hilb}{\mathcal H}
\newcommand{\Homology}{\mathrm H}
\newcommand{\normal}{\mathbf n}
\newcommand{\radial}{\mathbf r}
\newcommand{\evect}{\mathbf e}
\newcommand{\vol}{\mathrm{vol}}

\newcommand{\diam}{\mathrm{diam}}
\newcommand{\Ell}{\mathrm{Ell}}
\newcommand{\inj}{\mathrm{inj}}
\newcommand{\Lip}{\mathrm{Lip}}
\newcommand{\Riem}{\mathrm{Riem}}

\newcommand{\Bmu}{\boldsymbol \mu}
\newcommand{\Bnu}{\boldsymbol \nu}
\newcommand{\Blambda}{\boldsymbol \lambda}

\newcommand{\pic}{\vspace{30mm}}
\newcommand{\dfn}[1]{\emph{#1}\index{#1}}

\renewcommand{\Re}{\operatorname{Re}}
\renewcommand{\Im}{\operatorname{Im}}

\newcommand{\loc}{\mathrm{loc}}
\newcommand{\cpt}{\mathrm{cpt}}

\def\Japan#1{\left \langle #1 \right \rangle}

\newtheorem{theorem}{Theorem}[section]
\newtheorem{badtheorem}[theorem]{``Theorem"}
\newtheorem{prop}[theorem]{Proposition}
\newtheorem{lemma}[theorem]{Lemma}
\newtheorem{sublemma}[theorem]{Sublemma}
\newtheorem{proposition}[theorem]{Proposition}
\newtheorem{corollary}[theorem]{Corollary}
\newtheorem{conjecture}[theorem]{Conjecture}
\newtheorem{axiom}[theorem]{Axiom}
\newtheorem{assumption}[theorem]{Assumption}

\newtheorem{mainthm}{Theorem}
\renewcommand{\themainthm}{\Alph{mainthm}}

% \newtheorem{claim}{Claim}[theorem]
% \renewcommand{\theclaim}{\thetheorem\Alph{claim}}
\newtheorem*{claim}{Claim}

\theoremstyle{definition}
\newtheorem{definition}[theorem]{Definition}
\newtheorem{remark}[theorem]{Remark}
\newtheorem{example}[theorem]{Example}
\newtheorem{notation}[theorem]{Notation}

\newtheorem{exercise}[theorem]{Discussion topic}
\newtheorem{homework}[theorem]{Homework}
\newtheorem{problem}[theorem]{Problem}

\makeatletter
\newcommand{\proofpart}[2]{%
  \par
  \addvspace{\medskipamount}%
  \noindent\emph{Part #1: #2.}
}
\makeatother



\numberwithin{equation}{section}


% Mean
\def\Xint#1{\mathchoice
{\XXint\displaystyle\textstyle{#1}}%
{\XXint\textstyle\scriptstyle{#1}}%
{\XXint\scriptstyle\scriptscriptstyle{#1}}%
{\XXint\scriptscriptstyle\scriptscriptstyle{#1}}%
\!\int}
\def\XXint#1#2#3{{\setbox0=\hbox{$#1{#2#3}{\int}$ }
\vcenter{\hbox{$#2#3$ }}\kern-.6\wd0}}
\def\ddashint{\Xint=}
\def\dashint{\Xint-}

\usepackage[backend=bibtex,style=alphabetic,giveninits=true]{biblatex}
\renewcommand*{\bibfont}{\normalfont\footnotesize}
\addbibresource{topics.bib}
\renewbibmacro{in:}{}
\DeclareFieldFormat{pages}{#1}

\newcommand\todo[1]{\textcolor{red}{TODO: #1}}


\begin{document}
\begin{abstract}
We collect several results concerning regularity of minimal laminations, and governing the various modes of convergence for sequences of minimal laminations.
We then apply this theory to prove that a function is $1$-harmonic iff its level sets are a minimal lamination; this resolves an open problem of Daskalopoulos and Uhlenbeck.
\end{abstract}

\maketitle

%%%%%%%%%%%%%%%%%%%%%%%%%%%%%%%%%%%%%%%%%%%%%%%%%%%%%%%

% \tableofcontents

\section{Introduction}
The space of codimension-$1$ minimal laminations on a Riemannian manifold has been topologized in several different ways.
Thurston \cite[Chapter 8]{thurston1979geometry} introduced both his geometric topology as well as the weak topology of measures on the space of measured geodesic laminations.
Independently of Thurston, Colding and Minicozzi \cite[Appendix B]{ColdingMinicozziIV} introduced a topology that emphasized not the laminations themselves, but rather the coordinate charts which flatten them.
We shall explain how these three modes of convergence are related, as well as the regularity and compactness theorems associated to each such mode.

We then turn to our main goal.
We show that a current is Ruelle-Sullivan with respect to a minimal lamination if and only if it is locally the exterior derivative of a $1$-harmonic function.
This generalizes a theorem of Daskalopoulos and Uhlenbeck \cite[Theorem 6.1]{daskalopoulos2020transverse} and strengthens an unpublished result of Auer and Bangert \cite{Auer01}.

%%%%%%%%%%%%%%%%%
\subsection{Minimal laminations}\label{Lams sections}
Throughout this paper, we fix an interval $I \subset \RR$, a box $J \subset \RR^{d - 1}$, and a Riemannian manifold $M = (M, g)$ of dimension $d \geq 2$.

\begin{definition}
A (codimension-$1$) \dfn{laminar flow box} is a $C^0$ coordinate chart $F: I \times J \to M$ and a compact set $K \subseteq I$, such that for each $k \in K$, $F|_{\{k\} \times J}$ is a $C^1$ embedding, and the \dfn{leaf} $F(\{k\} \times J)$ is a $C^1$ complete hypersurface in $F(I \times J)$.

Two laminar flow boxes $(F_\alpha, K_\alpha)$ and $(F_\beta, K_\beta)$ belong to the same \dfn{laminar atlas} if the transition map $\psi_{\alpha \beta}$ between $F_\alpha$ and $F_\beta$ maps each leaf $\{k\} \times J$, $k \in K_\alpha$, to a leaf $\{\psi_{\alpha \beta}(k)\} \times J$, so that $\psi_{\alpha \beta}$ is a homeomorphism $K_\alpha \to K_\beta$.
\end{definition}

\begin{definition}
A \dfn{lamination} $\lambda$ consists of a nonempty closed set $S \subseteq M$, called its \dfn{support}, and a maximal laminar atlas $\{(F_\alpha, K_\alpha): \alpha \in A\}$ such that in the image $U_\alpha$ of each flow box $F_\alpha$,
$$S \cap U_\alpha = F_\alpha(K_\alpha \times J).$$
If $\lambda$ is a lamination in the image of a flow box $F$, and $N := F(\{k\} \times J)$ is a leaf of $\lambda$, we call $k$ the \dfn{label} of $N$.
A \dfn{foliation} is a lamination with support $S = M$.
\end{definition}

Summarizing the above definitions, a lamination is a nonempty closed set $S$ with a $C^0$ local product structure which locally realizes it as $K \times J$ for some compact set $K \subset \RR$.

\begin{definition}
We call a lamination $C^r$ (resp. \dfn{Lipschitz}) if its flow boxes are $C^r$ (resp. Lipschitz) coordinate charts, and say that it is \dfn{tangentially $C^r$} if for each flow box $(F, K)$, $F|_{\{k\} \times J}$ is a $C^r$ embedding for $k \in K$.\footnote{Such laminations are also known as $C^r$ \dfn{along leaves} \cite{Morgan88}.}
\end{definition}

In particular, we assume that laminations are $C^0$ and tangentially $C^1$; the latter assertion implies that the flow box can push forward the normal vector to each leaf, and in particular that the mean curvature to each leaf is well-defined as a signed Radon measure.

In this paper we shall focus on laminations with minimal leaves.\footnote{The word ``minimal'' is overloaded. In \cite{daskalopoulos2020transverse}, a \dfn{minimal lamination} is a lamination $\lambda$ in which every leaf is dense in $\supp \lambda$.
We adopt the terminology of \cite{Ohshika90}.}

\begin{definition}
A lamination $\lambda$ is \dfn{minimal} if its leaves $F_\alpha(\{k\} \times J)$ have zero mean curvature, and is \dfn{geodesic} if, in addition, $d = 2$.
\end{definition}

%%%%%%%%%%%%%%%%%%
\subsection{Regularity of minimal laminations}
The definitions of \S\ref{Lams sections} are tedious to work with, both because one has to prove the existence of flow boxes which flatten sets which may be extremely rough, and because one has no quantitative control on said flow boxes.
However, if we have curvature bounds on the leaves and on the underlying manifold $M$, our first main theorem drastically changes the story: it shows that the lamination $\lambda$ can be reconstructed from its set of leaves, in such a way that the flow boxes for $\lambda$ are under control in the Lipschitz and tangentially $C^\infty$ sense.
Here, a sequence converges \dfn{tangentially in $C^\infty$} if it converges in $C^0$ and all higher derivatives tangent to any leaves of $\lambda$ converge in $C^0$ as well.

\begin{mainthm}\label{regularity theorem}
Let $K := \|\Riem_M\|_{C^0}$ and let $i$ be the injectivity radius of $M$, and suppose that $K < \infty$, $i > 0$.
Let $\mathcal S$ be a set of disjoint minimal hypersurfaces in $M$, such that for every $N \in \mathcal S$,
\begin{equation}\label{curvature bound in regularity}
	\|\Two_N\|_{C^0} \leq A,
\end{equation}
and that $\bigcup_{N \in \mathcal S} N$ is a closed subset of $M$. Then:
\begin{enumerate}
\item There exists a Lipschitz minimal lamination $\lambda$ whose leaves are exactly the elements of $\mathcal S$.
\item There exists a Lipschitz line bundle on $M$ which is normal to every leaf of $\lambda$.
\item There exist constants $L = L(A, K, i) > 0$ and $r = r(A, K, i) > 0$, and a Lipschitz laminar atlas $(F_\alpha)$ for $\lambda$, such that for every $\alpha$,
\begin{equation}\label{conorm of flow box}
	\max(\Lip(F_\alpha), \Lip(F_\alpha^{-1})) \leq L,
\end{equation}
and the image of $F_\alpha$ contains a ball of radius $r$.
\item $F_\alpha$ and $F_\alpha^{-1}$ are tangentially $C^\infty$, with seminorms only depending on $A, K, i$.
\end{enumerate}
\end{mainthm}

Several similar results to Theorem \ref{regularity theorem} have appeared in the literature already, but Theorem \ref{regularity theorem} strengthens and clarifies them.
To our knowledge, the first related result is due to Solomon \cite[Theorem 1.1]{Solomon86}, which we improve on in several ways:
\begin{enumerate}
\item \label{foliation to lamination} Solomon's proof is only valid for minimal foliations in $\RR^d$.
\item We obtain estimates which only depend on the curvatures of the leaves and $M$, and on the injectivity radius $i$; they do not depend on the regularity of a given $C^0$ laminar atlas.
\item In fact, we do not even assume the existence of a $C^0$ laminar atlas.
\end{enumerate}
As Solomon notes, it is easy to extend his proof to minimal foliations of a Riemannian manifold $M$; the key point of (\ref{foliation to lamination}) is that we would like Theorem \ref{regularity theorem} to be true for minimal \emph{laminations}.

Using completely different techniques, Daskalopoulos and Uhlenbeck \cite[Proposition 7.3]{daskalopoulos2020transverse} obtained a version of Theorem \ref{regularity theorem} without any $C^0$ dependence, under the assumption that $M$ is a closed hyperbolic surface.
The key point of their argument is that the exponential map sends lines to geodesics, so it provides a much shorter proof of Theorem \ref{regularity theorem}, at the price of only working in dimension $2$.

Our work is closest to the proof of Colding and Minicozzi \cite[Appendix B]{ColdingMinicozziIV} for minimal laminations of a Riemannian manifold.
Their new idea is to fill in the gaps between the leaves in Solomon's constructions by linear interpolation.
However, Colding and Minicozzi take it as given that one can use the curvature bound (\ref{curvature bound in regularity}) to show that all the leaves can be represented as graphs at once.
Indeed, \emph {a priori}, the leaves could fail to be close to parallel, and then it would not be possible to construct a coordinate chart in which they are all graphs.

We fill in the aforementioned gap: the elements of $\mathcal S$ must be ``close to parallel on small scales'', where the scale is governed by $A, K$.
Otherwise, since the scale is small, we may replace the elements of $\mathcal S$ by their tangent spaces, which would then intersect, contradicting the disjointness of $\mathcal S$.
This approach was already suggested by Thurston \cite[\S8.5]{thurston1979geometry} in the case of geodesic laminations, though he omitted the details. 


%%%%%%%%%%%%%%%%%%
\subsection{Spaces of minimal laminations}\label{LamSpace section}
In the literature, there are at least three different topologies on the space of laminations on a Riemannian manifold $M$, which we now recall.

Thurston's geometric topology \cite[Chapter 8]{thurston1979geometry} says that a lamination $\lambda'$ is close to a lamination $\lambda$ if every leaf of $\lambda$ is close to a leaf of $\lambda'$ at least locally, and the same holds for their normal vectors $\normal$.

\begin{definition}
We define the basic open sets in \dfn{Thurston's geometric topology} to be defined by a lamination $\lambda$, $x \in M$, and $\varepsilon > 0$: the basic open set $\mathscr N(\lambda, x, \varepsilon)$ is the set of all laminations $\kappa$ such that there exists $y \in \supp \kappa \cap B(x, \varepsilon)$ such that the normal vectors are close: $\dist(\normal_\lambda(x), \normal_\kappa(y)) < \varepsilon$.
\end{definition}

A sequence of laminations $(\lambda_i)$ converges to a lamination $\lambda$ in Thurston's geometric topology iff, for every leaf $N$ of $\lambda$, every $x \in N$, and every $\varepsilon > 0$, there exists $i_{\varepsilon, x} \in \NN$ such that for every $i \geq i_{\varepsilon, x}$, $\supp \lambda_i$ intersects $B(x, \varepsilon)$, and for $x_i \in B(x, \varepsilon) \cap \supp \lambda_i$,
$$\dist_{SM}(\normal_{\lambda_i}(x_i), \normal_\lambda(x)) < 2\varepsilon.$$
It is straightforward to show that Thurston's geometric topology does not depend on the choice of Riemannian metric on $M$, or the choice of extension of the distance function on $M$ to its sphere bundle $SM$, which are implicit in the statement thereof.
However, the limiting lamination is not unique, as if $\lambda_i \to \lambda$ and $\lambda'$ is a sublamination of $\lambda$, then $\lambda_i \to \lambda'$.
In particular, Thurston's topology is not Hausdorff, and we say that $\lambda$ is a \dfn{maximal limit} of a sequence $(\lambda_i)$ if $\lambda_i \to \lambda$ and for every $\lambda'$ such that $\lambda_i \to \lambda'$, $\lambda'$ is a sublamination of $\lambda$.

Independently of Thurston, Colding and Minicozzi \cite[Appendix B]{ColdingMinicozziIV} defined a sequence of laminations to converge ``if the corresponding coordinate maps converge;'' that is, if the laminar atlases converge.
This of course says nothing about the limiting set of leaves and in the sequel paper \cite{ColdingMinicozziV} they additionally impose that the sets of leaves converge ``as sets.''

In this paper we consider a similar condition to the one in \cite{ColdingMinicozziV}, which we believe to be more natural: that the laminar atlases converge and that the laminations themselves converge in Thurston's geometric topology.
To be more precise:

\begin{definition}
A sequence $(\lambda_i)$ of laminations \dfn{flow-box converges} in a function space $X$ to $\lambda$ if it converges in Thurston's geometric topology, and there exists a laminar atlas $(F_\alpha)$ for $\lambda$ such that for each $\alpha$, $F_\alpha$ and $(F_\alpha)^{-1}$ are limits in $X$ of flow boxes $F_\alpha^i$, $(F_\alpha^i)^{-1}$ in laminar atlases for $\lambda_i$.
\end{definition}

The notion of flow-box convergence is mainly useful for tangential $C^\infty$ and the Fr\'echet space $C^{1-} := \bigcap_{0 \leq \theta < 1} C^\theta$, where $C^\theta$ are H\"older spaces.

We now define convergence of laminations equipped with transverse measures.\footnote{We assume that $\supp \mu_\alpha = K_\alpha$, but in \cite{daskalopoulos2020transverse}, it is only assumed that $\supp \mu_\alpha \subseteq K_\alpha$.
In particular, not every lamination admits a transverse measure.}

\begin{definition}
Let $\lambda$ be a lamination with atlas $A$.
A \dfn{transverse measure} to $\lambda$ consists of Radon measures $\mu_\alpha$ with $\supp \mu_\alpha = K_\alpha$, $\alpha \in A$, such that each transition map $\psi_{\alpha \beta}$ is measure-preserving:
$$\mu_\alpha|_{K_\alpha \cap K_\beta} = \psi_{\alpha \beta}^* (\mu_\beta|_{K_\alpha \cap K_\beta}).$$
The pair $(\lambda, \mu)$ is called a \dfn{measured lamination}.
\end{definition}

\begin{definition}
Let $(\lambda, \mu)$ be a measured, oriented lamination, and let $(\chi_\alpha)_{\alpha \in A}$ be a subordinate partition of unity.
The \dfn{Ruelle-Sullivan current} $T_\mu$ associated to $(\lambda, \mu)$ is defined for all compactly supported $d-1$-forms $\varphi$ by
\begin{equation}\label{RS current}
\int_M T_\mu \wedge \varphi := \sum_{\alpha \in A} \int_{K_\alpha} \left[\int_{\{k\} \times J} (F_\alpha^{-1})^* (\chi_\alpha \varphi) \right] \dif \mu_\alpha(k).
\end{equation}
\end{definition}

The Ruelle-Sullivan current was introduced by \cite{Ruelle75}, and we review its properties in \S\ref{RS prelims}.
In particular we show that $T_\mu$ makes sense (as a distributional sections of a different line bundle) even if $\lambda$ is not orientable.
Hence we may define:

\begin{definition}
A sequence of measured laminations $(\lambda_i, \mu_i)$ \dfn{converges} to $(\lambda, \mu)$ if their Ruelle-Sullivan currents $T_{\mu_i} \to T_\mu$ converge in the weak topology of measures.
\end{definition}

% The convergence of Ruelle-Sullivan currents, which is very convenient to work with analytically, is equivalent to a definition of measure convergence that may be more familiar to topologists, namely convergence of the transverse measure along each transverse curve, as we explain in Appendix \ref{transverse curves}.

Filling in some of the details of the argument of Colding and Minicozzi \cite[Appendix B]{ColdingMinicozziIV}, it follows from the regularity theorem, Theorem \ref{regularity theorem}, that once we have a bound on the curvatures of the leaves, every sequence of laminations has convergent subsequences in each of the above modes of convergence.
More precisely we have:

\begin{definition}
A sequence $(\lambda_n)$ of laminations has \dfn{bounded curvature} if there exists $C > 0$ such that for any $n$ and any leaf $N$ of $\lambda_n$, the second fundamental form satisfies $\|\Two_N\|_{C^0} \leq C$.
\end{definition}

\begin{mainthm}\label{compactness theorem}
Let $(\lambda_n)$ be a sequence of minimal laminations of bounded curvature, and assume that for some compact set $E \Subset M$ and every leaf $N$ of $\lambda_n$, $N \cap E$ is nonempty. Then:
\begin{enumerate}
\item A subsequence converges in the $C^{1-}$ and tangentially $C^\infty$ flow box topology, and in particular in Thurston's geometric topology, to a minimal lamination.
\item If $\mu_n$ is transverse to $\lambda_n$ and there exists $C > 0$ such that $\mu_n(M) \leq C$, then a further subsequence converges in the measure topology.
\end{enumerate}
\end{mainthm}

We use Theorem \ref{compactness theorem} to explain how the above modes of convergence are related.
Modulo details to be made precise later in the paper, convergence in measure implies convergence in flow boxes implies Thurston convergence.

%%%%%%%%%%%%%%%%%%
\subsection{Applications to \texorpdfstring{$1$-harmonic}{one-harmonic} functions}\label{FLG section}
% Geodesic laminations are of interest to the Thurston school of Teichm\"uller theory \cite[Chapter 8]{thurston1979geometry}.
% Later Thurston introduced \dfn{best Lipschitz maps}, namely maps $v: M \to N$ between closed manifolds which minimize their Lipschitz constant $\Lip(v)$ subject to a constraint on their homotopy class \cite{thurston1998minimal}.
% If $M, N$ are closed hyperbolic surfaces of the same genus, then $\log \Lip(v)$ is the distance between $M$ and $N$ in \dfn{Thurston's asymmetric metric} on Teichm\"uller space.
% This circle of ideas has been developed by the Thurston school \cite{papadopoulos:hal-00129729, Gu_ritaud_2017} but has recently also made contact with geometric PDE through the work of Daskalopoulos and Uhlenbeck \cite{daskalopoulos2020transverse,daskalopoulosPrep1,DaskalopoulosPrep2}, which we now recall.
% Let $M$ be a closed hyperbolic surface.
% The Euler-Lagrange equation for best Lipschitz maps $v: M \to \Sph^1$ is the $\infty$-Laplace equation
% \begin{equation}\label{infinity laplacian}
% 	\langle \nabla^2 v, \nabla v \otimes \nabla v\rangle = 0.
% \end{equation}
% Thus, every \dfn{$\infty$-harmonic function} -- that is, a viscosity solution of (\ref{infinity laplacian}) -- is best Lipschitz \cite{daskalopoulos2020transverse}.
% From an $\infty$-harmonic function we may associate two pieces of data:
% \begin{enumerate}
% \item The set $\{\dif v = \|\dif v\|_{L^\infty}\}$ where the best Lipschitz constant $\Lip(v)$ is attained is the support of a geodesic lamination, called the \dfn{maximum stretch lamination} $\lambda$.
% \item The $\infty$-Laplace equation is invariant under addition of constants $v \mapsto v + y$, so by Noether's theorem, $v$ is associated to a \dfn{conserved flux} $\dif u$.
% \end{enumerate}
% If $d = 2$, the associated conservation law is the $1$-Laplace equation (see Definition \ref{1harmonic dfn})
% \begin{equation}\label{1Laplacian}
% \dif^* \left(\frac{\dif u}{|\dif u|}\right) = 0.
% \end{equation}
The main result of this paper concerns \dfn{$1$-harmonic functions}, that is, solutions of the \dfn{$1$-Laplace equation}
\begin{equation}\label{1Laplacian}
	\nabla \cdot \left(\frac{\nabla u}{|\nabla u|}\right) = 0
\end{equation}
in the weak sense of Definition \ref{1harmonic dfn} (namely, sections of the sheafification of the presheaf of $1$-harmonic functions in the sense of Maz\'on, Rossi, and Segura de L\'eon \cite{Mazon14}).
Such functions are also known as \dfn{functions of locally least gradient}, and purely on the level of formal manipulations, their level sets are minimal hypersurfaces.
We show that this is true in a particularly strong sense: the level sets actually define a minimal lamination.

\begin{mainthm}\label{main thm}
Suppose that $\dim M \leq 4$.
\begin{enumerate}
\item Let $u$ be a $1$-harmonic function on $M$.
Then:
\begin{enumerate}
\item $\bigcup_{y \in \RR} \partial \{u > y\}$ is the support of a Lipschitz minimal lamination $\lambda$.
\item The leaves of $\lambda$ are the connected components of the level sets $\partial \{u > y\}$.
\item There is a measured oriented structure on $\lambda$ whose Ruelle-Sullivan current is $\dif u$.
\end{enumerate}
\item Conversely, if $\lambda$ is a minimal measured oriented lamination with Ruelle-Sullivan current $T$, and $M$ is simply connected, then there exists a $1$-harmonic function $u$ such that $T = \dif u$.
\end{enumerate}
\end{mainthm}

The main ingredients in the proof of Theorem \ref{main thm} are Theorem \ref{regularity theorem}, the regularity theory of minimal hypersurfaces, and curvature estimates on stable minimal hypersurfaces established by Schoen \cite{Schoen2016} for $d = 3$ and Chodosh and Li \cite{Chodosh2021} for $d = 4$.
In particular, a key step in the proof is to establish that the stability radii of the level sets of a $1$-harmonic function are bounded from below, so that they satisfy uniform curvature estimates.

A similar result to Theorem \ref{main thm}, proven with somewhat different methods, was announced but never published by Auer and Bangert \cite{Auer01}, who claimed to establish that a locally minimal $d - 1$-current is Ruelle-Sullivan for a lamination in a much weaker sense than ours.
In particular, it does not seem that one cannot extract Lipschitz regularity, or indeed the existence of even $C^0$ flow boxes, directly from their methods.

Aside from giving a complete proof of the claimed theorem of \cite{Auer01}, our motivation for Theorem \ref{main thm} is to generalize the work of Daskalopoulos and Uhlenbeck on $\infty$-harmonic maps from a closed hyperbolic surface to $\Sph^1$ \cite{daskalopoulos2020transverse}, which associates to each such map a geodesic lamination $\lambda$ and a $1$-harmonic function $v$ on the universal cover such that $\dif v$ drops to a Ruelle-Sullivan current for a sublamination of $\lambda$.
Inspired by this theorem, Daskalopoulos and Uhlenbeck conjectured that for any $1$-harmonic function on $\Hyp^2$, $\dif u$ should be Ruelle-Sullivan for some (possibly not maximum-stretch) geodesic lamination \cite[Problem 9.4]{daskalopoulos2020transverse}, and conversely that if $T$ is a Ruelle-Sullivan current for some geodesic lamination, then local primitives of $T$ are $1$-harmonic \cite[Conjecture 9.5]{daskalopoulos2020transverse}.
Of course such results are special cases of Theorem \ref{main thm}.
We shall revisit the connection between Theorem \ref{main thm} and the $\infty$-Laplacian in \cite{BackusInfinityMaxwell1}, where we explain how one can view the $1$-Laplacian as the convex dual problem to the problem of constructing a calibration of a minimal lamination, which is given by a system of ``$\infty$-elliptic'' equations.

In \S\ref{1harmonic apps} we discuss some easy consequences of Theorem \ref{main thm}.
We disprove a conjecture on the uniqueness of the superlevel sets of Maz\'on, Rossi, and Segura de Le\'on \cite[Remark 2.8]{Mazon14}, and prove a version of the G\'orny decomposition of a $1$-harmonic function \cite[Theorem 1.2]{górny2017planar}.

We would also like to highlight a possible application of Theorem \ref{main thm} that we shall not address here.
The associated parabolic flow to (\ref{1Laplacian}) is known as \dfn{level set flow} and acts on the level sets of a smooth submersion $u$ by mean curvature flow.
As such, it arises as a model of interfaces with minimal area, as a means of continuing mean curvature flow past its singular times, and in the \dfn{level set method} of computing minimal surfaces \cite{Evans91,Sethian90,Chen89,Thomas05}.
Since Theorem \ref{main thm} is concerned exactly with the level sets of a $1$-harmonic function, it would be very interesting to apply it to rigorously justify certain properties of the level set method, and in particular to understand the limiting behavior of level set flow, even in the case that $u$ is not a smooth submersion.

%%%%%%%%%%%%%%%%%%%%%%

\subsection{Notation and conventions}
The operator $\star$ is the Hodge star, thus $\star 1$ is the Riemannian measure.
We denote the musical isomorphisms by $\sharp, \flat$.
If $U$ is an open set, we write $|U| := \int_U \star 1$ for the volume of $U$, but if $U$ is a submanifold or rectifiable set of positive codimension, we instead write $|U|$ for its surface measure.
We write $\normal_N$ for the normal vector (or conormal $1$-form) for a hypersurface $N$, $\nabla_N$ for the Levi-Civita connection, and $\Two_N := \nabla_N \normal_N$ for the second fundamental form.

We consider the following manifolds: $\Ball^d$ is the unit ball in $\RR^d$, $\Sph^d$ the unit sphere in $\RR^{d + 1}$, and $\Hyp^d$ is the hyperbolic space.

For a map $F: X \to Y$ between metric spaces, we write $\Lip(F)$ for its Lipschitz constant.
If $X, Y$ are connected Riemannian manifolds, one of which is $1$-dimensional, then we have $\Lip(F) = \|\dif F\|_{L^\infty}$.

%%%%%%%%%%%%%%%%%%%%%%%
\subsection{Outline of the paper}
The rest of the paper is organized as follows:
\begin{itemize}
\item In \S\ref{Regularity}, we establish some curvature bounds on disjoint families of minimal surfaces, and use them to prove the regularity theorem, Theorem \ref{regularity theorem}.
\item In \S\ref{Prelims}, we develop basic facts about the measure topology, and $1$-harmonic functions, that we shall use throughout the remainder of the paper. This section is independent of \S\ref{Regularity}.
\item In \S\ref{CompactnessSec}, we prove the compactness theorem, Theorem \ref{compactness theorem} (the compactness theorem) and explore the consequences for how the different modes of convergence are related to each other. This section applies both \S\ref{Regularity} and \S\ref{Prelims}.
\item In \S\ref{1harmonic sec}, we prove the equivalence of $1$-harmonic functions and measured oriented minimal laminations, Theorem \ref{main thm}, and apply it to prove sundry facts about $1$-harmonic functions. This section relies on \S\ref{Regularity} and \S\ref{Prelims}, but it is independent of \S\ref{CompactnessSec}.
\end{itemize}

%%%%%%%%%%%%%%%%%%%%%%%%

\subsection{Acknowledgements}
I would like to thank Georgios Daskalopoulos for suggesting this project and for many helpful discussions.
I would also like to thank Chao Li for help understanding \cite{Chodosh2021}, Victor Bangert for providing me with a draft of the proof announced in \cite{Auer01}, Stephen Obinna for suggesting the brief proof of Lemma \ref{cardinality appendix}, and Jeremy Kahn and JiaHua Zou for helpful discussions.

This research was supported by the National Science Foundation's Graduate Research Fellowship Program under Grant No. DGE-2040433.



%%%%%%%%%%%%%%%%%%%%%%%%%%%


% %%%%%%%%%%%%%%%%%%%%
% \subsection{Measured convergence on closed hyperbolic surfaces}
% Thurston studied measured geodesic laminations on closed hyperbolic surfaces.
% In that setting, he gave a different definition of measure convergence \cite[\S8.10]{thurston1979geometry}.
% We shall show that his definition is equivalent to ours \emph{in the setting of closed hyperbolic surfaces}.
% Under those hypotheses, each measured geodesic lamination has null support \cite[\S8.5]{thurston1979geometry}.

% Following \cite[\S7]{daskalopoulos2020transverse}, given an oriented geodesic lamination $\lambda$ on the closed hyperbolic surface $M$, we say that an oriented curve $\gamma$ is \dfn{good} if $\gamma$ is transverse to every geodesic in $\lambda$, $\partial \gamma$ does not meet $\supp \lambda$, and $\langle \gamma', \normal_\lambda\rangle$ does not change sign.
% By taking double covers of a train track neighborhood of a geodesic lamination $\lambda$, we may always assume that $\lambda$ is oriented.

% \begin{lemma}
% Let $\lambda$ be an oriented geodesic lamination in a closed hyperbolic surface and $\gamma$ a transverse curve to $\lambda$.
% Then, possibly after extending $\gamma$ slightly, $\gamma$ subdivides into finitely many good subcurves with respect to $\lambda$.
% \end{lemma}
% \begin{proof}
% By extending $\gamma$ we may assume that $\partial \gamma$ does not meet the null set $\supp \lambda$.
% Thus $\gamma$ is an \dfn{admissible transversal} in the sense of \cite[\S7]{daskalopoulos2020transverse}.
% So by \cite[Lemma 7.9]{daskalopoulos2020transverse}, $\gamma$ is the sum of finitely many good $\lambda$-good curves.
% \end{proof}

% If $\gamma: I \to M$ is a $\lambda$-good curve, and $\mu$ is a transverse measure to $\lambda$ with Ruelle-Sullivan current $\dif u$, then we can define a Radon measure $\gamma^! \mu$ on $I$, by declaring that for any interval $[\alpha, \beta] \subseteq I$ with $\gamma_*([\alpha, \beta])$ a simply connected $\lambda$-good curve,
% $$\mu([\alpha, \beta]) = |u(\gamma(\beta)) - u(\gamma(\alpha))|.$$
% We call $\gamma^! \mu$ the \dfn{exceptional pullback} of $\mu$ by $\gamma$, since normally measures push forward, rather than pull back; it is straightforward to check that the exceptional pullback is a well-defined Radon measure.

% The exceptional pullback allows us to define Thurston's notion of measured convergence \cite[\S8.10]{thurston1979geometry}.
% To avoid confusion with our definition of the measure topology we refer to it as the ``transverse cocycle topology", following the terminology of \cite[\S7.2]{daskalopoulos2020transverse}.

% \begin{definition}
% Let $(\lambda, \mu)$ be a measured oriented geodesic lamination, $\varepsilon > 0$, $\gamma_1, \dots, \gamma_k$ good curves with respect to $\lambda$, and $f_1, \dots, f_k$ continuous functions on $I$.
% Let
% \begin{equation}
% \mathscr N = \mathscr N(\lambda, \mu, \varepsilon, \gamma_1, \dots, \gamma_k, f_1, \dots, f_k) \label{cocycle basic open set}
% \end{equation}
% be the set of measured oriented geodesic laminations $(\lambda', \mu')$ such that for every $i = 1, \dots, k$, $\gamma_i$ is $\lambda'$-good, and
% $$\left|\int_I f_i \dif(\gamma_i^! \mu - \gamma_i^! \mu')\right| < \varepsilon.$$
% Then $\mathscr N$ is a basic open set in the \dfn{transverse cocycle topology}.
% \end{definition}

% We could develop the theory of exceptional pullback measures for more general laminations and foliations, but this would take us too far afield, as it would require a more careful application of the geometric measure theory of the $BV_\loc$ function $u$.
% Thus we content ourselves to proving:

% \begin{proposition}
% Let $M$ be a closed hyperbolic surface.
% A sequence $(\lambda_n, \mu_n)$ converges in the measure topology to $(\lambda, \mu)$ iff it converges in the transverse cocycle topology.
% \end{proposition}
% \begin{proof}
% Suppose that $(\lambda_n, \mu_n) \to (\lambda, \mu)$ in the measure topology, and let $\mathscr N$ be the basic open set for the transverse cocycle topology given by (\ref{cocycle basic open set}).
% We must show that for $n$ large enough, $(\lambda_n, \mu_n) \in \mathscr N$.

% Since $\gamma_i$ is $\lambda$-good, from Lemma \ref{convergence of normals}, for $n \geq N$ (where $N \in \NN$ is independent of $i$, since $k$ is finite), $\gamma_i$ is $\lambda_n$-good.
% Now let
% $$Z := \supp \lambda \cup \bigcup_{n \geq N} \supp \lambda_n.$$
% Then $Z$ is null, and since the $\gamma_i$ are transverse to $Z$, Fubini's theorem implies that $\gamma_i^{-1}(\gamma_i \cap Z)$ is Lebesgue null in $I$.

% Let $u_n, u$ be the primitives of the Ruelle-Sullivan currents for $\mu_n, \mu$, chosen so that for some $x \in M$, $u_n(x) = u(x) = 0$.
% Along any subsequence, there is a further subsequence on which $u_n \to u$ almost everywhere, and in particular since $u_n, u$ are locally constant on $M \setminus Z$, we actually can conclude that $u_n \to u$ pointwise on $M \setminus Z$.
% Therefore, for almost every point of $I$, $u_n \to u$ pointwise.
% So $\gamma_i^! \mu_n \to \gamma_i^! \mu$ weakly.
% It follows that for $n$ large enough depending on $\varepsilon$, $(\lambda_n, \mu_n) \in \mathscr N$.

% Conversely, if $(\lambda_n, \mu_n) \to (\lambda, \mu)$ in the transverse cocycle topology, and $\varphi$ is a $d-1$-form, we can approximate in the weak topology of measures the projection of $\varphi$ to the volume bundle of $\lambda$ by $\varphi^\ell := \sum_i f_i [\gamma_i]$ where $[\gamma_i]$ is the $d-1$-current given by integration along the curve $\gamma_i$.
% Without loss of generality, $\gamma_i$ is $\lambda$-good, and therefore also $\lambda_n$-good.
% Then
% $$\int_M T_{\mu_n} \wedge \varphi^\ell \to \int_M T_\mu \wedge \varphi^\ell.$$ 
% \end{proof}

%%%%%%%%%%%%%%%%%%%%%%%%%%%%%%%%%%%%%%%%%%
\section{Regularity of laminations}\label{Regularity}
The goal of this section is to prove the regularity theorem, Theorem \ref{regularity theorem}, for minimal laminations.

\subsection{Elliptic estimates on leaves}\label{Leaf estimates}
We first recall some estimates on minimal surfaces in normal coordinates, following \cite[\S7.1]{colding2011course}.

Let $\mu, \nu, \dots$ range over indices $0, \dots, d - 1$, $i, j, \dots$ range over $1, \dots, d - 1$, $x := (x^1, \dots, x^{d - 1})$, and $y := x^0$.
Let $g$ be a metric on $\RR^{d - 1}_x \times \RR_y$ satisfying the normal coordinates condition 
$$g_{\mu\nu} = \delta_{\mu\nu} + O(|x|^2 + y^2)$$
and a curvature bound $\|\Riem_g\|_{C^0} \leq K_0$ for some $0 < K_0 \leq 1$.
Under these hypotheses, the Christoffel symbols $\Gamma$ of the Levi-Civita connection $\nabla$ satisfy $|\Gamma| \lesssim K_0(|x|^2 + y^2)$.
Let $N = \{y = u(x)\}$ for some function $u$ defined on $4\Ball^{d - 1}$.
By \cite[(7.21)]{colding2011course}, if we define for a $2$-jet $(y, \xi, H)$ on $\RR^{d - 1}$,
$$h_{ij}(x; y, \xi) := g_{ij}(x, y) + \xi_i g_{j0}(x, y) + \xi_j g_{i0}(x, y) + \xi_i \xi_j g_{00}(x, y),$$
denote by $(h^{ij})$ the inverse of $(h_{ij})$, and let
\begin{align*}
	F(x; y, \xi, H) 
	&:= h^{ij}(H_{ij} + \Gamma^0_{ij} + \xi_i \Gamma^0_{0j} + \xi_j \Gamma^0_{0i} + \xi_i \xi_j \Gamma^0_{00}) \\
	&\qquad + \xi_m h^{ij}(\Gamma^m_{ij} + \xi_i \Gamma^m_{0j} + \xi_j \Gamma^m_{0i} + \xi_i \xi_j \Gamma^m_{00})
\end{align*}
(where we have taken an Einstein sum)
then $h^{ij} = \delta^{ij} + O(K_0(|x|^2 + y^2) + |\xi|)$ so we have the estimate
$$F(x, y, \xi, H) = \tr H + O((K_0 |H| + |\xi|)(|x| + |y| + |\xi|)).$$
Moreover, if we denote $Pu(x) := F(x, u(x), \dif u(x), \nabla^2 u(x))$, then $Pu = 0$ is the minimal surface equation with respect to the metric $g$.
For $\|u\|_{C^1} \lesssim 1$, this equation is uniformly elliptic, so that by Schauder estimates \cite[Theorem 6.2]{gilbarg2015elliptic}, for any $r \geq 0$,
\begin{equation}\label{norms on uk}
	\|u\|_{C^r(3\Ball^{d - 1})} \lesssim_r 1.
\end{equation}

\begin{lemma}
Suppose that $u_2 \geq u_1$ satisfy $Pu_1 = Pu_2 = 0$ on $4\Ball^{d - 1}$ and $v := u_2 - u_1$.
Then for $K_0 \ll 1$, $\|u_i\|_{C^1} \lesssim 1$, 
\begin{equation}\label{Schauder Harnack}
	\|\dif v\|_{C^0(\Ball^{d - 1})} \lesssim \sup_{2\Ball^{d - 1}} v \lesssim \inf_{\Ball^{d - 1}} v.
\end{equation}
\end{lemma}
\begin{proof}
Let
$$Qv := \int_0^1 \frac{\partial F(s)}{\partial H_{ij}} \partial_i \partial_j v + \frac{\partial F(s)}{\partial \xi_i} \partial_i v + \frac{\partial F(s)}{\partial y} v \dif s$$
where 
$$F(s) := F(x, su_2(x) + (1 - s)u_1(x), \dif(su_2 + (1 - s) u_1)(x), \nabla^2(su_2 + (1 - s)u_1)(x)).$$
Then $Q$ is a linear differential operator, and $Qv = 0$ \cite[(7.25)]{colding2011course}.
The principal symbol $q$ of $Q$ satisfies
\begin{align*}
q_{ij} &= \int_0^1 \frac{\partial F(s)}{\partial H_{ij}} \dif s = \int_0^1 h_{ij}(s) \dif s \\
&= \delta_{ij} + O(K_0(|x|^2 + \|u_1\|_{C^0}^2 + \|u_2\|_{C^0}^2) + \|\dif u_1\|_{C^0} + \|\dif u_2\|_{C^0}).
\end{align*}
Then if $K_0$ is chosen smaller than an absolute constant, the first eigenvalue $\lambda_1(q) \geq \frac{1}{2}$ on $3\Ball^{d - 1}$.
By (\ref{norms on uk}), the coefficients of $Q$ are bounded in $C^1$ on $3\Ball^{d - 1}$.
The claim now follows from Schauder estimates \cite[Theorem 6.2]{gilbarg2015elliptic} and the Harnack inequality \cite[Corollary 9.25]{gilbarg2015elliptic}.
\end{proof}

For the remainder of the paper we fix a constant $K_0$ satisfying the hypotheses of the above lemma.

\subsection{A preliminary choice of coordinates}
We now construct normal coordinates in which the leaves of $\lambda$ are $C^1$-close to hyperplanes $\{y = y_0\}$.
The utility of this fact is that, if $f: \RR^{d - 1}_x \to \RR_y$ is a $C^1$ function, and its graph has normal vector $\normal$, then
\begin{equation}\label{nabla as a normal}
	\normal = \frac{\partial_y - \nabla f}{\sqrt{1 + |\nabla f|^2}}.
\end{equation}
So the leaves of $\lambda$ are minimal graphs which are small in $C^1$ and so we may apply (\ref{Schauder Harnack}) uniformly among all of the leaves at once.

An analogous result was proven by \cite{Solomon86} (without the quantitative dependence) by a very different means, using the regularity theory for integral flat convergence of minimal currents \cite[Theorem 5.3.14]{federer2014geometric}.
We did not do this, both because \cite{Solomon86} requires that $\lambda$ be a foliation, and because it does not seem particularly easy to recover quantitative bounds from the highly general theory of \cite[Chapter 5]{federer2014geometric}.

We begin with the following bootstrapping argument:

\begin{lemma}\label{existence of tubes}
	Let $N$ be a connected $C^2$ hypersurface in $\RR^d = \RR^{d - 1}_x \times \RR_y$ which is tangent to $\{y = 0\}$ at the origin.
	If $\|\Two_N\|_{C^0} \leq \frac{1}{8}$, then $N \cap B(0, 1)$ is the graph over $\{y = 0\}$ of a function $f$ with
	$$|f(x)| \leq \|\Two_N\|_{C^0} |x|^2.$$
\end{lemma}
\begin{proof}
	Near $0$, $N$ can be represented a graph $\{y = f(x)\}$, since it is tangent to $\{y = 0\}$.
	This representation is valid on the component of the set $\{|\nabla f(x)| < \infty\}$ containing $0$, and it is related to the unit normal by (\ref{nabla as a normal}).
	Rearranging (\ref{nabla as a normal}) and taking derivatives,
	$$-\nabla^2 f(x) = \frac{\nabla \normal(x, f(x)) \cdot (\partial_x \otimes \partial_x + \nabla f(x) \otimes \partial_y)}{\sqrt{1 + |\nabla f(x)|^2}} - \frac{\nabla^2 f(x) \cdot (\nabla f(x) \otimes \normal(x, f(x)))}{(1 + |\nabla f|^2)^{3/2}}.$$
	Here $-\nabla^2$ denotes the negative Hessian, not the Laplacian.
	Since
	$$|\partial_x \otimes \partial_x + \nabla f(x) \otimes \partial_y| \leq \sqrt{1 + |\nabla f(x)|^2},$$
	and $\nabla \normal = \Two_N$, we conclude
\begin{equation}\label{bound Hessian by Two}
	|\nabla^2 f(x)| \leq |\Two_N(x, f(x))| + |\nabla^2 f(x)| |\nabla f(x)|.
\end{equation}
	In order to control the error terms in (\ref{bound Hessian by Two}), we make the \dfn{bootstrap assumption}
\begin{equation}\label{bootstrap}
	|\nabla f(x)| \leq \frac{1}{2},
\end{equation}
	which is at least valid in some small neighborhood $B_R$ of $0$ since (\ref{nabla as a normal}) and the fact that $N$ is tangent to $\{y = 0\}$ at $0$ imply that $\nabla f(0) = 0$.
	By (\ref{bound Hessian by Two}),
$$|\nabla^2 f(x)| \leq 2|\Two_N(x, f(x))|,$$
	and integrating this inequality one obtains for $|x| \leq R$ that
\begin{equation}\label{closed bootstrap}
	|\nabla f(x)| \leq 2|\Two_N(x, f(x))| |x| \leq \frac{1}{4}.
\end{equation}
	In particular, since $\nabla f \in C^1$, either $R \geq 1$ or there exists $R' > R$ such that the bootstrap assumption (\ref{bootstrap}) is valid on $B_{R'}$.
	Therefore (\ref{bootstrap}) is valid with $R = 1$.
	Integrating (\ref{closed bootstrap}), we obtain the desired conclusion.
\end{proof}


\begin{lemma}\label{lams have C0 fields}
	Suppose that $\delta > 0$ is small enough depending on $K$.
	Then there exists $r = r(\delta, K, i, A) > 0$ such that for every disjoint family of minimal surfaces $\mathcal S$ satisfying the curvature bound (\ref{curvature bound in regularity}) and every $p \in \bigcup_{N \in \mathcal S} N$, we can choose normal coordinates $(x, y) \in \RR^{d - 1} \times \RR$ based at $p$ so that
\begin{equation}\label{normal is basically dy}
	\sup_{N \in \mathcal S} \|\normal_\lambda - \partial_y\|_{C^0(B(p, r))} \leq \delta.
\end{equation}
\end{lemma}
\begin{proof}
Consider normal coordinates $(x, y)$ based at $p$, and write $\Two_N'$ for the second fundamental form of $N \in \mathcal S$ taken with respect to the euclidean metric from those coordinates, $\normal_N'$ the normal in coordinates, and $\Gamma$ the Christoffel symbols.
In particular, since $\normal_N^\flat$ is the conormal and satisfies $\normal_N^\flat = (\normal_N')^\flat/|\normal_N'|$, 
$$\Two_N' = \dif (\normal_N')^\flat = (\nabla - \Gamma) |\normal_N'| \normal_N^\flat = |\normal_N'| (\Two_N - |\normal_N'| \Gamma \otimes \normal_N^\flat) + \dif \normal_N' \otimes \normal_N^\flat.$$
Using estimates on normal coordinates we conclude that for every $0 < s < i$ and some absolute $C > 0$,
$$\|\Two_N'\|_{B(p, s)} \leq A + CKs.$$
After rescaling we may assume that $A \leq 1/16$, $K \leq 1/(32C)$, and $i \geq 2$, so $\|\Two_N'\|_{C^0(B(p, 2))} \leq 1/8$.
Then we apply Lemma \ref{existence of tubes}: for $q \in N \cap B(p, 1)$, $B(q, 1) \subseteq B(p, 2)$, and $\tilde x$ the euclidean coordinate on $T_q N$ induced by the normal coordinates $(x, y)$, $N \cap B(q, 1)$ is the graph of a function $f$ on $T_q N$ satisfying
\begin{equation}\label{living in a tube}
|f(\tilde x)| \leq A|\tilde x|^2.
\end{equation}
Here, and for the remainder of this proof, we use $|\cdot|$ to mean the euclidean metric only.

Let $0 < r < s\delta^2$ for some small absolute $s > 0$ to be chosen later, and suppose that for every choice of normal coordinates $(x, y)$ at $p$, (\ref{normal is basically dy}) fails.
Then, since every coordinate system fails to have the desired properties, we might as well choose one such that for some $N \in \mathcal S$ and some $q \in B(p, r) \cap N$, $\normal_N(q)$ is a scalar multiple of $\partial_y$.
By the contradiction assumption, we can choose $N' \in \mathcal S$ and $q \in B(p, r) \cap N'$ such that
$$|\normal_{N'}(q') - \partial_y| > \delta.$$
In particular, since $|\normal_{N'}(q')| = 1 + O(r^2)$ and $|\partial_y| = 1$, the angle $\theta$ between these two vectors is given by the law of cosines as 
$$1^2 + (1 + O(r^2))^2 - 1(1 + O(r^2)) \cos \theta = |\normal_{N'}(q') - \partial_y|^2$$
which can be neatly estimated for $s$ small enough as
$$\cos \theta < 1 - \frac{\delta^2}{2} + O(r^2) \leq 1 - \frac{\delta^2}{4}.$$
But $\theta$ is the angle between the tangent planes $T_q N$ and $T_{q'} N'$.
We consider the triangle $\Delta(q, q', r)$ where $r$ is a point of intersection of $P := T_q N$ and $P' := T_{q'} N'$, so again by the law of cosines, if $\alpha := |q - r|$ and $\beta := |q' - r|$,
$$\alpha^2 + \beta^2 - 2\alpha\beta \cos \theta = |q - q'|^2 \leq r^2.$$
By Young's inequality, it follows that 
$$r^2 \geq (\alpha^2 + \beta^2)(1 - \cos \theta) > (\alpha^2 + \beta^2) \frac{\delta^2}{4}$$
or in other words 
$$\alpha^2 + \beta^2 < \frac{4r^2}{\delta^2} < 4s^2 \delta^2$$
which means for $\delta$ small that $\max(\alpha, \beta) < 2c\delta < s/4$.
Hence $P, P'$ intersect in $B(p, s/4 + r) \subseteq B(p, s/2)$.

Now consider the tubes $\mathcal T, \mathcal T'$ of all points which are within $s^2/16$ of $P, P'$.
Since $P, P'$ intersect in $B(p, s/2)$, if $s$ is small, any graphs over $P, P'$ in $\mathcal T, \mathcal T'$ must intersect in $B(p, s)$.
In particular we can take $s < 1$ and conclude from (\ref{living in a tube}) that $N, N'$ are not disjoint, contradicting the definition of $\mathcal S$.
\end{proof}

\subsection{Proof of Theorem \ref{regularity theorem}}
Fix $\delta > 0$ to be chosen later, and $P \in M$.
By Lemma \ref{lams have C0 fields}, if $\delta \leq \delta_*$ for some $\delta_* = \delta_*(i, K) > 0$, there exists $r = r(\delta, i, K, A) > 0$ such that $B(P, r)$ admits rescaled normal coordinates $(x, y) \in 5\Ball^{d - 1} \times (-2, 2)$ in which the curvature of the rescaled metric has a $C^0$ norm $\leq K_0$ and
\begin{equation}\label{normal is almost constant}
\|\normal - \partial_y\|_{C^0(B(P, r))} \leq \delta.
\end{equation}
Moreover,
$$|\normal \cdot \partial_y| \geq 1 - |\normal - \partial_y| \geq 1 - \delta,$$
so if we select $\delta := \min(\delta_*, \frac{1}{4})$, then in $5\Ball^{d - 1} \times (-1, 1)$, then every leaf is the graph of a function, say $u_k: 5\Ball^{d - 1} \to (-2, 2)$ where $u_k(0) = k$, and
$$\|\dif u_k\|_{C^0} \leq \frac{1 - (1 - \delta)^2}{1 - \delta} \leq 1.$$
If $r$ is chosen small enough depending on $g$, then the metric $\tilde g$ induced by $g$ on $5\Ball^{d - 1} \times (-2, 2)$ satisfies $\|\Riem_{\tilde g}\|_{C^0} \leq K_0$.
Moreover, $\|u_k\|_{C^0} \leq 2$, and $u_k$ has a minimal graph, so the elliptic estimates stated in \S\ref{Leaf estimates} apply to $u_k$ uniformly in $k$.

Now let $-1 < k < \ell < 1$, and let $v_{\ell k} := u_\ell - u_k$.
By (\ref{Schauder Harnack}) with $v := v_{\ell k}$, for every $x \in \Ball^{d - 1}$,
\begin{equation}\label{bound on du}
|\dif u_\ell(x) - \dif u_k(x)| \lesssim |u_\ell(x) - u_k(x)|
\end{equation}
and it follows that
\begin{equation}\label{vertical Lipschitz}
|\normal(x, u_\ell(x)) - \normal(x, u_k(x))| \lesssim |u_\ell(x) - u_k(x)|.
\end{equation}

To extend (\ref{vertical Lipschitz}) to a Lipschitz bound on $\normal$, let $X_1, X_2 \in (\Ball^{d - 1} \times (-1, 1)) \cap \supp \lambda$.
Then there exist $x_1, x_2 \in \Ball^{d - 1}$ and $k_1, k_2 \in (-1, 1)$ such that $X_i = (x_i, u_{k_i}(x_i))$.
Setting $Y := (x_2, u_{k_1}(x_2))$,
$$|\normal(X_1) - \normal(X_2)| \leq |\normal(X_1) - \normal(Y)| + |\normal(Y) - \normal(X_2)|.$$
Then by (\ref{norms on uk}) and the mean value theorem,
$$|\normal(X_1) - \normal(Y)| \lesssim |\dif u_{k_1}(x_1) - \dif u_{k_1}(x_2)| \lesssim |X_1 - Y|.$$
Moreover, by (\ref{vertical Lipschitz}),
$$|\normal(Y) - \normal(X_2)| \lesssim |u_{k_1}(x) - u_{k_2}(x)| = |Y - X_2|.$$
Since $\delta \leq \frac{1}{4}$, by (\ref{normal is almost constant}),
$$|\sin \angle(X_1 - Y, X_2 - Y)| > 1 - O(\delta)$$
and we conclude by the Pythagorean theorem that
$$|Y - X_2|^2 + |X_1 - Y|^2 \lesssim |X_1 - X_2|^2.$$
In conclusion,
$$|\normal(X_1) - \normal(X_2)| \lesssim |X_1 - X_2|$$
which implies that $\normal$ is Lipschitz on $V \cap \supp \lambda$, where $V$ is the neighborhood of $P$ which was mapped to $\Ball^{d - 1} \times (-1, 1)$ by the cylindrical coordinates $(x, y)$.
In particular, $V$ contains a ball of the form $B(P, s)$, where $s$ only depends on $r$ (and $r$ only depends on $g$ and $A$).
Taking a Lipschitz extension of $\normal$ to $V$ we obtain the desired Lipschitz normal subbundle.

Following \cite[Appendix B]{ColdingMinicozziIV}, we construct the laminar flow box
\begin{align*}
	F: \RR^{d - 1}_\xi \times \RR_\eta &\to V \subseteq \RR^{d - 1}_x \times \RR_y \\
	(\xi, \eta) &\mapsto (\xi, f(\xi, \eta))
\end{align*}
by setting
$$f(\xi, \eta) := u_\eta(\xi)$$
if $u_\eta$ exists, and if $k < \eta < \ell$ and there does not $k < \eta' < \ell$ such that $u_{\eta'}$ exists, then
$$f(\xi, \eta) := u_k(\xi) + \frac{\eta - k}{\ell - k} v_{\ell k}(\xi)$$
is the linear interpolant of $u_k$ and $u_\ell$.

By (\ref{norms on uk}), $F$ is bounded in tangential $C^\infty$.
In particular, if $V$ is a vector field tangent to $\{\eta = k\}$, then the pushforward
$$F_* V = V^i \partial_{x^i} + (Vf) \partial_y$$
is well-defined, and pushforwards of such vector fields span the tangent bundle of the graph of $u_k$. 
The bound
\begin{equation}\label{xiLip of f}
	\|\partial_\xi f\|_{C^0} \lesssim \sup_k \|u_k\|_{C^1} \lesssim 1,
\end{equation}
a consequence of (\ref{norms on uk}), establishes that $\|F_* V\|_{C^0} \sim \|V\|_{C^0}$, and then 
$$\|(F_* V) F^{-1}\|_{C^0} \lesssim \|V(F \circ F^{-1})\|_{C^0} \leq \|V\|_{C^0} \sim \|F_* V\|_{C^0}.$$
Since $V$ was arbitrary we conclude that $F^{-1}$ is bounded in tangential $C^1$, hence in tangential $C^\infty$ by the inverse function theorem. 

It remains to show that $F$ is a Lipschitz isomorphism.
To do this, we first claim that $\Lip(f) \sim 1$.
In the $\xi$ direction, we use (\ref{xiLip of f}).
If $-1 < k < \ell < 1$, then by (\ref{bound on du}) and (\ref{Schauder Harnack}),
\begin{equation}\label{f lip}
	|f(\xi, k) - f(\xi, \ell)| \lesssim |u_k(\xi) - u_\ell(\xi)| \lesssim \ell - k.
\end{equation}
This shows that $f$ is Lipschitz in the $\eta$ direction on the leaves with constant comparable to $1$, and hence on its entire domain by linear interpolation, proving the claim.
We can then estimate using (\ref{f lip})
$$|F(\xi_1, \eta_1) - F(\xi_2, \eta_2)| \lesssim |\xi_1 - \xi_2| + \Lip(f)(|\xi_1 - \xi_2| + |\eta_1 + \eta_2|)$$
so that $\Lip(F) \lesssim 1 + \Lip(f) \lesssim 1$.

To obtain a bound on $\Lip(F^{-1})$, we observe that
\begin{equation}\label{F is coLip in xi}
|\xi_1 - \xi_2|^2
\leq |\xi_1 - \xi_2|^2 + |f(\xi_1, \eta_1) - f(\xi_2, \eta_1)|^2 
= |F(\xi_1, \eta) - F(\xi_2, \eta)|^2.
\end{equation}
By Harnack's inequality with $\eta_1 = k$ and $\eta_2 = \ell$, or $k \leq \eta_1 < \eta_2 \leq \ell$ if $\eta_1, \eta_2$ lie in the plaque between leaves $k, \ell$,
$$\frac{|f(\xi_1, \eta_1) - f(\xi_1, \eta_2)|}{|\eta_1 - \eta_2|} \gtrsim \frac{v_{\ell k}(\xi_1)}{\ell - k} \gtrsim \frac{v_{\ell k}(0)}{\ell - k} = 1$$
whence by the mean value theorem and (\ref{F is coLip in xi}),
\begin{align*}
	|\eta_1 - \eta_2| 
	&\lesssim |f(\xi_1, \eta_1) - f(\xi_1, \eta_2)| \\
	&\leq |f(\xi_1, \eta_1) - f(\xi_2, \eta_2)| + \|\partial_\xi f\|_{C^0} |\xi_1 - \xi_2| \\
	&\leq (1 + \|\partial_\xi f\|_{C^0}) |F(\xi_1, \eta_1) - F(\xi_2, \eta_2)|.
\end{align*}
By (\ref{norms on uk}) and the fact that either $\partial_\xi f = \partial_\xi u_\eta$, or there are $k,\ell$ such that $\partial_\xi f$ is the linear interpolation of $\partial_\xi u_k$ and $\partial_\xi u_\ell$, $\|\partial_\xi f\|_{C^0} \lesssim 1$.
Thus
$$|F(\xi_1, \eta_1) - F(\xi_2, \eta_2)| \gtrsim |\xi_1 - \xi_2|^2 + |\eta_1 - \eta_2|^2.$$
It follows that $\Lip(F^{-1}) \lesssim 1$, so $F$ is a Lipschitz isomorphism with constants comparable to $1$.

Finally, we compose $F$ with the change of coordinates at the start of this proof to obtain a laminar flow box in a small neighborhood of $(0, 0)$ whose image has radius $O(r)$, and whose Lipschitz constants are comparable to $O(r^{-1})$.

%%%%%%%%%%%%%%%%%%%%%%%%%%%%%%%%%%%%%%%%%

\section{The measure topology}\label{Prelims}
\subsection{Preliminaries}
Let $X$ be a metrizable space, and let $C_\cpt(X)$ be the space of compactly supported continuous functions $f: X \to \RR$.
Its dual $C_\cpt(X)'$ is canonically isomorphic to the space of signed Radon measures on $X$, where the bilinear pairing is given by integration.
The weak topology on $C_\cpt(X)'$ is known as the \dfn{weak topology of measures}.
Unpacking the definitions, a sequence $(\mu_n)$ of Radon measures converges to $\mu$ in the weak topology of measures iff for every continuous function $f: X \to \RR$,
$$\lim_{n \to \infty} \int_X f \dif \mu_n = \int_X f \dif \mu.$$

\begin{proposition}[portmanteau theorem]
	Let $(\mu_n)$ be a sequence of positive Radon measures on a compact metrizable space $X$ with $\mu_n(X) \lesssim 1$, and let $\mu$ be a Radon measure on $X$. The following are equivalent:
\begin{enumerate}
	\item $\mu_n \to \mu$ in the weak topology of measures.
	\item $\liminf_{n \to \infty} \mu_n(X) \geq \mu(X)$ and for every closed $Y \subseteq X$, $\limsup_{n \to \infty} \mu_n(Y) \leq \mu(Y)$.
	\item $\limsup_{n \to \infty} \mu_n(X) \leq \mu(X)$ and for every open $Z \subseteq X$, $\liminf_{n \to \infty} \mu_n(Z) \geq \mu(Z)$.
	\item For every $W \subseteq X$ with $\mu(\partial W) = 0$, $\lim_{n \to \infty} \mu_n(W) = \mu(W)$.
\end{enumerate}
	If we choose a metric on $X$, then the above conditions imply:
\begin{enumerate}
	\setcounter{enumi}{4}
	\item For every $x \in X$ and all but countably many $\varepsilon > 0$, $\lim_{n \to \infty} \mu_n(B(x, \varepsilon)) = \mu(B(x, \varepsilon))$.
\end{enumerate}
\end{proposition}
\begin{proof}
	See \cite[Theorem 13.16]{klenke2013probability} for the equivalence of (1)--(4); \cite{klenke2013probability} deals with subprobability measures, but this is equivalent to measures of bounded total mass by a rescaling.

	We then must show that (4) implies (5); to do so, it suffices to show that for all but countably many $\varepsilon$, $\mu(\partial B(x, \varepsilon)) = 0$.
	Let
	$$A := \{\varepsilon > 0: \mu(\partial B(x, \varepsilon)) > 0\}.$$
	Since the sets $\partial B(x, \varepsilon)$ are disjoint, for every countable $A' \subseteq A$,
	$$\sum_{\varepsilon \in A'} \mu(\partial B(x, \varepsilon)) \leq \mu(X) < \infty,$$
	where $\mu(X) < \infty$ since $X$ is compact.
	It follows from (the contrapositive of) Lemma \ref{cardinality appendix} below that $A$ is countable.
\end{proof}

\begin{lemma}\label{cardinality appendix}
Let $S$ be an uncountable set and $f: S \to (0, \infty)$. Then there exists a countable set $S' \subset S$ such that
$$\sum_{x \in S'} f(x) = \infty.$$
\end{lemma}
\begin{proof}
Define $S_n := f^{-1}([\frac{1}{n + 1}, \frac{1}{n}))$ for $n \in \NN$ (where we take the convention $1/n = \infty$).
Since $S$ is uncountable but $\NN$ is countable, it follows from the infinite pigeonhole principle that there exists $n \in \NN$ such that $S_n$ is infinite.
In particular there exists an infinite countable set $S' \subseteq S_n$, which then satisfies
\begin{align*}
\sum_{x \in S'} f(x) &\geq \sum_{x \in S'} \frac{1}{n + 1} = \infty. \qedhere 
\end{align*}
\end{proof}

There are subtleties involved in the portmanteau theorem for noncompact $X$.
However, this will never be an issue, as we shall only use it locally, in small precompact balls.

\subsection{Ruelle-Sullivan currents}\label{RS prelims}
If $X = M$ is a manifold, then we can consider instead the space $C_\cpt(M, \Omega^\ell)$ of compactly supported continuous $\ell$-forms.
An $\ell$-\dfn{current} is an element of the dual space $C_\cpt(M, \Omega^\ell)'$ \cite{simon1983GMT}.\footnote{Strictly speaking, $C_\cpt(M, \Omega_\ell)'$ is the space of $\ell$-currents of locally finite total variation. However, we will never need to consider $\ell$-currents that do not have locally finite total variation, so we suppress this technicality.}
We denote the pairing of an $\ell$-current $T$ and an $\ell$-form $\varphi$ by $\int_M T \wedge \varphi$.
Again we have the \dfn{weak topology of measures} on the space of $\ell$-currents.

To any $\ell$-current $T$ we may associate a positive Radon measure, its \dfn{total variation} $\star |T|$, which satisfies for any function $f$,
$$\int_M f \star |T| := \sup_{|\varphi| \leq |f|} \int_M T \wedge \varphi,$$
and a $|T|$-measurable $d - \ell$-form $\psi$, the \dfn{polar part} \cite[Theorem 4.14]{simon1983GMT}, which satisfies $T = \psi |T|$, $|T|$-almost everywhere.
Moreover, if $\psi \mapsto -\int_M T \wedge \dif \psi$ is an $\ell - 1$-current, then it is the \dfn{exterior derivative} $\dif T$.
We write $\|T\|_{TV} := \int_M \star |T|$ for the total variation norm.

With the measure-theoretic machinery above in place, we recall some facts about Ruelle-Sullivan currents.
Let $(\lambda, \mu)$ be a measured oriented lamination.
Then the Ruelle-Sullivan current $T_\mu$ is a well-defined closed $d-1$-current \cite[Theorem 8.2]{daskalopoulos2020transverse}. 
In particular, we may lift $T_\mu$ to the universal cover $\tilde M$, where it is exact \cite[Theorem 8.3]{daskalopoulos2020transverse}.
Moreover, $T_\mu$ has an intrinsic definition as the unique $d-1$-current with a certain polar decomposition.
To be more precise, recall that $\mu$ defines a measure on $\supp \lambda$: in each flow box $F_\alpha$, an open set $U$ has measure
\begin{equation}\label{transverse measure of an open set}
\mu(U) := \int_{K_\alpha} |F_\alpha(\{k\} \times J) \cap U| \dif \mu_\alpha(k).
\end{equation}

% \begin{proof}
% We first claim that the right-hand side of (\ref{RS current}) is always finite, and is continuous in $\varphi$.
% In fact, possibly after refining $(\chi_\alpha)$, we may assume that it is a locally finite partition of unity.
% In particular, we just need to check the continuity in a single flow box:
% $$\left|\int_{K_\alpha} \left[\int_{\RR^{d - 1} \times \{k\}} (F_\alpha^{-1})^* (\chi_\alpha \varphi) \right] \dif \mu_\alpha(k)\right| \leq \int_{K_\alpha} \int_{\RR^{d - 1} \times \{k\}} |(F_\alpha^{-1})^* (\chi_\alpha \varphi)| \dif \mu_\alpha(k).$$
% The inner integral is controlled by $\|\varphi\|_{C^0(U_\alpha)} \cdot |U_\alpha|$ where $U_\alpha$ is the image of $F_\alpha$.
% The outer integral is then well-defined because it is against a Radon measure.

% We next observe that the choice of partition of unity is irrelevant, thus if $\varphi$ has compact support in $U_\alpha \cap U_\beta$, then
% \begin{equation}\label{well-defined of Ruelle-Sullivan}
% \int_{K_\alpha} \int_{\RR^{d - 1} \times \{k\}} (F_\alpha^{-1})^* \varphi \dif \mu_\alpha(k) = \int_{K_\beta} \int_{\RR^{d - 1} \times \{k\}} (F_\beta^{-1})^* \varphi \dif \mu_\beta(k).
% \end{equation}
% Indeed,
% \begin{align*}
% \int_{K_\alpha} \int_{\RR^{d - 1} \times \{k\}} (F_\alpha^{-1})^* \varphi \dif \mu_\alpha(k)
% &= \int_{K_\beta} (F_\alpha F_\beta^{-1})^* \left[\int_{\RR^{d - 1} \times \{k\}} (F_\alpha^{-1})^* \varphi \dif \mu_\alpha(k)\right] \\
% &= \int_{K_\beta} \left[\int_{\RR^{d - 1} \times \{k\}} (F_\beta^{-1})^* \varphi\right] (F_\alpha F_\beta^{-1})^* \dif \mu_\beta(k) \\
% &= \int_{K_\beta} \int_{\RR^{d - 1} \times \{k\}} (F_\beta^{-1})^* \varphi \dif \mu_\beta(k)
% \end{align*}
% where the last equation is because of the measure-preserving nature of the transition maps; this proves (\ref{well-defined of Ruelle-Sullivan}).

% Finally, if a $d-2$-form $\psi$ has compact support in a single flow box, then
% $$\int_{\RR^{d - 1} \times \{k\}} (F_\alpha^{-1})^* \dif \psi = \int_{\RR^{d - 1} \times \{k\}} \dif((F_\alpha^{-1})^* \psi) = 0$$
% by Stokes' theorem, so
% \begin{align*}
% \int_M \dif T_\mu \wedge \psi &= -\int_M T_\mu \wedge \dif \psi \\
% &= -\int_{K_\alpha} \int_{\RR^{d - 1} \times \{k\}} (F_\alpha^{-1})^* \dif \psi \dif \mu_\alpha(k) = 0. \qedhere
% \end{align*}
% \end{proof}

\begin{lemma}
For a measured oriented lamination $(\lambda, \mu)$, with Lipschitz normal vector $\normal_\lambda$, the polar decomposition of $T_\mu$ is
\begin{equation}\label{polar ruelle sullivan}
T_\mu = \normal_\lambda \mu.
\end{equation}
\end{lemma}
\begin{proof}
For an open set $U \subseteq M$ in a flow box $F_\alpha$, the total variation satisfies
$$\int_U \star |T_\mu| = \sup_{\|\varphi\|_{C^0} \leq 1} \int_{K_\alpha} \int_{\{k\} \times J} \varphi \dif \mu_\alpha(k)$$
where the supremum ranges over $d-1$-forms $\varphi$ with compact support in $U$.
However, $\star \normal_\lambda^\flat$ is the Riemannian measure on $F_\alpha(\{k\} \times J)$, so
$$\int_{\{k\} \times J} \varphi \leq \int_{\{k\} \times J} (F_\alpha^{-1})^*(\star \normal_\lambda^\flat).$$
Since $\|\normal^\lambda\|_{C^0} = 1$, it follows that a sequence of cutoffs of $\star \normal_\lambda^\flat$ to more and more of $U$ is a maximizing sequence.
Therefore $\normal_\lambda$ is the polar part of (\ref{polar ruelle sullivan}), and
$$\int_U \star |T_\mu| = \int_{K_\alpha} \int_{\{k\} \times J} (F_\alpha^{-1})^*(1_U \star \normal_\lambda^\flat) \dif \mu_\alpha(k).$$
The inner integral is the Riemannian measure of $F_\alpha(\{k\} \times J) \cap U$, so by (\ref{transverse measure of an open set}), $|T_\mu| = \mu$.
\end{proof}

The above computation motivates the definition of Ruelle-Sullivan current of a \emph{nonorientable} lamination.
To be more precise, if $\lambda$ is a nonorientable lamination with normal vector field $\normal_\lambda$, then we can view $\normal_\lambda$ as a section of a (necessarily twisted) line bundle $L$ over $M$.
We can then define $T_\mu$ to be $\normal_\lambda \mu$, which makes sense as a distributional section of $L$, and can be tested against any continuous $d-1$-form on $M$ whose support is contained in a contractible set.
In particular, we shall speak of the Ruelle-Sullivan current of any measured lamination, even if it is nonorientable.

\begin{lemma}\label{convergence of normals}
If $(\lambda_n, \mu_n) \to (\lambda, \mu)$, $x_n \in \supp \lambda_n$ converges to $x \in \supp \lambda$, and $(\lambda_n), \lambda$ have continuous normal vector fields $(\normal_n), \normal$, then $\normal_n(x_n) \to \normal(x)$ pointwise.
\end{lemma}
\begin{proof}
	Choose a continuous $d-1$-form $\varphi$ which extends $\star \normal^\flat$.
	Then for every $\varepsilon > 0$,
	$$\int_{B(x, \varepsilon)} T_\mu \wedge \varphi = \mu(B(x, \varepsilon))$$
	so by the portmanteau theorem, for almost every $\varepsilon > 0$,
	\begin{equation}\label{epsilon is a continuity set}
		\lim_{n \to \infty} \frac{\int_{B(x, \varepsilon)} T_{\mu_n} \wedge \varphi}{\mu_n(B(x, \varepsilon))} = \frac{\int_{B(x, \varepsilon)} T_\mu \wedge \varphi}{\mu(B(x, \varepsilon))} = 1.
	\end{equation}
	On the other hand, if we assume that there exist $\delta, \varepsilon > 0$ and a coordinate system such that for every $y \in \supp \lambda_n \cap B(x, \varepsilon)$,
	$$|\normal_n - \normal| \geq \delta,$$
	then possibly after shrinking $\varepsilon$ we may assume that (\ref{epsilon is a continuity set}) holds, hence by (\ref{polar ruelle sullivan}),
	$$\int_{B(x, \varepsilon)} T_{\mu_n} \wedge \varphi = \int_{B(x, \varepsilon)} \normal_n^\flat \wedge \star \normal^\flat \dif \mu_i \leq (1 - O(\delta)) \mu_n(B(x, \varepsilon))$$
	and therefore $\delta = 0$, a contradiction.
\end{proof}

%%%%%%%%%%%%%%%%%%%%%%%%%%%
\subsection{\texorpdfstring{$1$-harmonic}{One-harmonic} functions}
The appropriate notion of solution of a $1$-harmonic function on a bounded Lipschitz domain in $\RR^d$ was introduced by Maz\'on, Rossi, and Segura de Le\'on \cite{Mazon14}: a function $u \in BV(U)$ is $1$-harmonic if $\nabla u/|\nabla u|$ extends off the support of $\nabla u$ to a divergence-free vector field $X$ on $U$ with $\|X\|_{L^\infty} \leq 1$.
Unfortunately this notion is not quite suitable on a closed manifold:

\begin{example}
Let $u$ be the indicator function of the upper hemisphere $\Sph^3_+$ of $\Sph^3$.
Then the only level set of $u$ is the equatorial $\Sph^2$, which is a minimal surface, so $u$ ``should'' be $1$-harmonic.
However, by the divergence theorem, if $X$ is an extension of $\nabla u/|\nabla u|$, then
$$\int_{\Sph^3_+} \nabla \cdot X \dif V = \int_{\Sph^2} \dif A = |\Sph^2|$$
so that $X$ cannot be divergence-free.
\end{example}

This issue can be avoided by allowing $\nabla u/|\nabla u|$ to be divergence-free on a neighborhood of $\supp \nabla u$.
This is arguably a more natural definition, since the solutions of a PDE should form a sheaf.
Indeed, what we refer to as a ``$1$-harmonic function'' is a section of the sheafification of the presheaf of $1$-harmonic functions in the sense of \cite{Mazon14}.

\begin{definition}\label{1harmonic dfn}
A \dfn{$1$-harmonic function} is a function $u \in BV_\loc(M)$ such that, on a neighborhood of any point of $\supp \nabla u$, $\nabla u/|\nabla u|$ extends to a divergence-free vector field $X$ such that $\|X\|_{L^\infty} \leq 1$.
\end{definition}

A function $u$ on a ball $U$ has \dfn{least gradient} if it is a minimizer of the total variation $\|\dif u\|_{TV(U)}$ among all $BV$ functions with the same trace along $\partial U$.
A subset $V \subset U$ has \dfn{least perimeter} if $1_V$ has least gradient.
Conversely, if $u$ has least gradient, then every superlevel set $\{u > y\}$ has least perimeter \cite[Theorem 1]{BOMBIERI1969}.
We say that $u$ has \dfn{locally least gradient} if we can cover $M$ by open sets $U$ such that $u|_U$ has least gradient.
A straightforward modification of \cite[Theorem 1.1]{Mazon14} gives:

\begin{theorem}
A function $u \in BV_\loc(M)$ is $1$-harmonic iff it has locally least gradient.
\end{theorem}

\begin{example}
It is possible to have locally least gradient but not least gradient.
Again we consider the indicator function $u$ of the upper hemisphere of $\Sph^3$.
It is clear that $u$ is $1$-harmonic and hence has locally least gradient, but since $\Sph^3$ is a closed manifold, the trace of $u$ is $0$ and so the competition class of $u$ contains the trivial function, which is the only function of least gradient on $\Sph^3$.
\end{example}

The de Giorgi--Miranda regularity theorem \cite{deGiorgi61,Miranda66} asserts that every set of least perimeter in an open subset of $\RR^d$, $d \leq 7$, is bounded by stable minimal hypersurfaces.
It is folklore that the same result holds for Riemannian manifolds; for completeness, in the companion paper \cite{BackusFLG} we established the de Giorgi--Miranda theorem in this generality, hence:

\begin{theorem}\label{main thm of old paper}
Suppose that $\dim M \leq 7$, $M$ is a compact manifold, possibly with boundary, and $u: M \to \RR$ is a function of least gradient.
Then every superlevel set $\partial \{u > y\}$ of $u$ is bounded by complete stable embedded minimal hypersurfaces.
\end{theorem}

The following compactness theorem for $1$-harmonic functions in the weak topology of measures will frequently be useful.
It is a straightforward consequence of the compactness of the forgetful map $BV(M) \to L^1(M)$ and the proof of \cite[Osservazione 3]{Miranda67}, and we omit the details.

\begin{proposition}\label{MirandaStability}
  Suppose that $M$ is a compact manifold, possibly with boundary.
	If a sequence of functions $(u_n)$ (not necessarily of the same trace) is bounded in $L^1(M)$ and satisfies
\begin{equation}\label{boundedness in Miranda}
	\limsup_{n \to \infty} \int_M \star |\dif u_n| \leq \liminf_{n \to \infty} \inf_{v|_{\partial M} = 0} \int_M \star |\dif(u_n + v)| < \infty,
\end{equation}
	then there exists a $1$-harmonic function $u$ such that along a subsequence, $u_n \to u$ in $L^1(M)$ and $\dif u_n \to \dif u$ in the weak topology of measures.
\end{proposition}



%%%%%%%%%%%%%%%%%%%%%%%%%%%%%%%%%%%%%%%%%
\section{Compactness of the space of laminations}\label{CompactnessSec}
In this section we prove Theorem \ref{compactness theorem}, the compactness theorem.
We then apply it to explore the implications between the different modes of convergence.

\subsection{The Hausdorff topology}
Throughout this section we use the Hausdorff topology on closed subsets of a topological space \cite[Chapter IV]{nadler2017continuum}:

\begin{definition}
Let $X$ be a topological space, and $(Y_n)$ a sequence of closed subsets of $X$.
\begin{enumerate}
\item The \dfn{limit inferior} $\liminf_{n \to \infty} Y_n$ is the set of all $x \in X$ such that for every open neighborhood $U \ni x$, $U \cap Y_n$ is eventually nonempty.
\item The \dfn{limit superior} $\limsup_{n \to \infty} Y_n$ is the set of all $x \in X$ such that for every open neighborhood $U \ni x$, $U \cap Y_n$ is nonempty for infinitely many $n$.
\item If $\liminf_{n \to \infty} Y_n = \limsup_{n \to \infty} Y_n$, we call that set the \dfn{limit} $\lim_{n \to \infty} Y_n$.
\end{enumerate}
\end{definition}

\begin{lemma}
Suppose that $\lambda_n \to \lambda$ in the weak topology of measures or Thurston's geometric topology.
Then 
\begin{equation}\label{supports shrink in the limit}
\supp \lambda \subseteq \liminf_{n \to \infty} \supp \lambda_n.
\end{equation}
\end{lemma}
\begin{proof}
Let $x \in \supp \lambda$.
If the convergence is in the weak topology of measures, let $\mu, \mu_n$ be the transverse measures.
Then by the portmanteau theorem, for any $\varepsilon > 0$,
$$\liminf_{n \to \infty} \mu_n(B(x, \varepsilon)) \geq \mu(B(x, \varepsilon)) > 0$$
so $\mu_n(B(x, \varepsilon)) \gtrsim 1$.
So for any $\varepsilon$ we can find $n$ and $x_n \in \supp \lambda_n \cap B(x, \varepsilon)$.
If instead the convergence is in the Thurston topology, we pass to a subsequence which realizes the limit inferior in (\ref{supports shrink in the limit}).
Then by definition of a basic open set, for every $\varepsilon > 0$ we can find $n$ and $x_n$ such that $x_n \in \supp \lambda_n \cap B(x, \varepsilon)$.
Either way, we conclude (\ref{supports shrink in the limit}).
\end{proof}

Given a connected oriented hypersurface $N$, we denote by $[N]$ the $d-1$-current defined by 
$$\int_M [N] \wedge \psi = \int_N \psi.$$
In particular, if we view $N$ as a lamination with one leaf and transverse measure $\mu$ assigning that leaf mass $1$, then $[N] = T_\mu$ is the Ruelle-Sullivan current for $(N, \mu)$.
Moreover, if $N = F_* \sigma$ for some simplex $\sigma$ and $C^1$ map $F$, then 
$$\int_M [N] \wedge \psi = \int_\sigma F^* \psi,$$
so if $N_1, N_2$ are $C^1$-close, then $F_1^*, F_2^*$ are $C^0$-close, and hence $[N_1], [N_2]$ are close in the weak topology of measures.

\begin{lemma}\label{measured convergence is smooth convergence}
Let $C > 0$, let $(N_n)$ be a sequence of minimal surfaces with $\|\Two_{N_n}\|_{C^0} \leq C$, and let $N$ be a surface.
If $[N_n] \to [N]$ in the weak topology of measures, then $N$ is a minimal surface.
\end{lemma}
\begin{proof}
Let $p \in N = \supp [N]$; by (\ref{supports shrink in the limit}), there exist $p_n \in N_n$ with $p_n \to p$.
Then by Lemma \ref{convergence of normals}, $\normal_{N_n}(p_n) \to \normal_N(p)$.
By assumption, $\|\nabla \normal_{N_n}\|_{C^0} \leq C$; moreover, we can choose normal coordinates $(x, y) \in \RR^{d - 1} \times \RR$ at $p$ with $\partial_y|_p = \normal_N(p)$.
So in a neighborhood of $p$, for every $n$ large, $\normal_{N_n}$ is $C^0$ close to $\partial_y$.
In particular, $N_n$ are the graphs of functions $u_n: \RR^{d - 1}_x \to \RR_y$ which are bounded in $C^1$ and solve $Pu_n = 0$.
By (\ref{norms on uk}), $(u_n)$ is precompact in $C^\infty$.
Similarly, $N$ is the graph of some $u$, and along a subsequence $u_{n_k} \to \tilde u$ in $C^\infty$ for some $\tilde u$, which then has a graph $\tilde N$.
In particular, $[N_{n_k}] \to [\tilde N]$ in the weak topology of measures; it follows that $\tilde N = N$, so $\tilde u = u$.
Therefore $u_{n_k} \to u$ in $C^\infty$, hence $Pu = 0$.
\end{proof}

%%%%%%%%%%%%%%%%%%

\subsection{Proof of Theorem \texorpdfstring{\ref{compactness theorem}}{B}}
\subsubsection{Construction of the limiting flow box}
Let $P \in M$, and let $(\lambda_n)$ be a sequence of minimal laminations of bounded curvature, such that every leaf of every lamination meets a compact set.

By Theorem \ref{regularity theorem}, there exist $r > 0$ and $L \geq 1$ such that for every large $n \in \NN$, $B(P, r)$ is contained in the image of a flow box $F_n$ for $\lambda_n$ with Lipschitz constant $L$, such that $F_n(0, 0) = P$.
By the Arzela-Ascoli theorem, along a subsequence $F_n \to F$ in $C^0$ for some map $F: I \times J \to B(P, r)$ and some $I \subseteq \RR$, $J \subseteq \RR^{d - 1}$, such that on the image $V$ of $F$, we also have the convergence $F_n^{-1} \to F^{-1}$.
Moreover, $F(0, 0) = P$, so that $F: I \times J \to V$ is a homeomorphism onto a set which contains $P$.
Since
$$\max(\Lip(F), \Lip(F^{-1})) \leq \limsup_{n \to \infty} \max(\Lip(F_n), \Lip(F_n^{-1})) \leq L,$$
it follows that $\max(\Lip(F), \Lip(F^{-1})) \leq L$, and for any $\theta \in (0, 1)$,
\begin{align*}
	\|F - F_n\|_{C^\theta}
	&\leq \Lip(F - F_n)^\theta \|F - F_n\|_{C^0}^{1 - \theta} \leq (2L)^\theta \|F - F_n\|_{C^0}^{1 - \theta}.
\end{align*}
It follows that $F_n \to F$ in $C^\theta$, hence in $C^{1-}$, and similarly for $F^{-1}$.
Since $(F_n)$ and $(F_n^{-1})$ are bounded in tangential $C^\infty$, a similar compactness argument to the above shows that $F_n \to F$ and $F_n^{-1} \to F^{-1}$ in tangential $C^\infty$.

Since $P$ was arbitrary, it follows that we can find laminar atlases $(F_\alpha^n, K_\alpha^n)$ for each large $n \in \NN$ such that $F_\alpha^n \to F_\alpha$ in $C^{1-}$, where the images of $F_\alpha$ and $F_\alpha^n$ are an open cover $(U_\alpha)$ of $M$ independent of $n$, and $(F_\alpha)$ satisfies the usual transition relations, and $F_\alpha$ is a Lipschitz isomorphism.

%%%%%%%%%%%%%%%%%%%%%%%

\subsubsection{Construction of the limiting lamination}
We now construct the limiting lamination.
We employ the Hausdorff hyperspace $\Hypspace I$ of closed subsets of $I$ to accomplish this.
Since $I$ is a compact metric space, so is $\Hypspace I$ \cite[Theorem 4.17]{nadler2017continuum}, so we may diagonalize so that for every $\alpha$, either $K^n_\alpha \to K_\alpha$ for some nonempty $K_\alpha$ in the Hausdorff distance on $I$, or for all $n \geq n^*(\alpha)$, $K_\alpha^n$ is empty (in which case we define $K_\alpha = \emptyset$).

In order to ensure that the laminations $\lambda_n$ do not escape to infinity, fix a compact set $E \subseteq M$ such that every leaf of every $\lambda_n$ meets $E$.
Then there exists a finite set $A_E \subseteq A$ such that $E \subseteq \bigcup_{\alpha \in A_E} U_\alpha$.

\begin{lemma}\label{label sets are nonempty}
	There exists $\alpha$ such that $K_\alpha$ is nonempty.
\end{lemma}
\begin{proof}
	Suppose not; then for
	$$n \geq \max_{\alpha \in A_E} n^*(\alpha)$$
	and $\alpha \in A_E$, $K_\alpha^n = \emptyset$, so no leaves of $\lambda_n$ meet $U_\alpha$, and hence no leaves of $\lambda_n$ meet $E$.
	This is a contradiction since $\lambda_n$ has a leaf.
\end{proof}

In each flow box $F_\alpha$ with $K_\alpha$ nonempty, we thus have the leaves of a lamination, namely $K_\alpha \times J$.
We now check the transition relations to ensure that they glue to a global lamination; this is straightforward but we include it for completeness.

Thus let $\psi_{\alpha \beta}$ and $\psi_{\alpha \beta}^n$ be the transition maps, thus $\psi_{\alpha \beta}^n$ induces a map
$$\psi_{\alpha \beta}^n: K_\alpha^n \to K_\beta^n.$$
By convergence of $(F_\alpha^n)$, $\psi_{\alpha \beta}$ induces a map $K_\alpha \to K_\beta$.

\begin{definition}
	A \dfn{cocycle of labels} $(k_\alpha)_{\alpha \in A'}$ is a set $A' \subseteq A$ and an element of $\prod_{\alpha \in A'} K_\alpha$, such that:
\begin{enumerate}
	\item The cocycle condition: $k_\beta = \psi_{\alpha \beta}(k_\alpha)$ for $\alpha, \beta \in A'$.
	\item For every $\alpha \in A'$, if $\psi_{\alpha \beta}(k_\alpha)$ is well-defined, then $\beta \in A'$.
\end{enumerate}
\end{definition}

\begin{lemma}
	Every cocycle of labels $(k_\alpha)_{\alpha \in A'}$ defines a complete minimal hypersurface $N$ such that
	$$N \cap U_\alpha = F_\alpha(\{k_\alpha\} \times J).$$
\end{lemma}
\begin{proof}
We have the cocycle condition
$$(N \cap U_\alpha) \cap U_\beta = (N \cap U_\beta) \cap U_\alpha$$
which follows from the fact that
\begin{align*}
F_\alpha(\{k_\alpha\} \times J) \cap U_\beta
&= F_\beta(\psi_{\alpha \beta}(\{k_\beta\} \times J)) \cap U_\alpha \cap U_\beta \\
&= F_\beta(\psi_{\alpha \beta}(\{k_\beta\} \times J)) \cap U_\alpha.
\end{align*}
From the cocycle condition, it follows that $N$ honestly defines a Lipschitz hypersurface in $M$, which is complete in $\bigcup_{\alpha \in A'} U_\alpha$.
If $\overline N$ intersects $U_\alpha$ for some $\alpha \notin A'$, then $N$ intersects $U_\beta$ for some $\beta \in A'$ so that $U_\beta \cap U_\alpha \cap \overline N$ is nonempty.
But then $\psi_{\beta \alpha}(k_\beta)$ must be defined, so $\alpha \in A'$, a contradiction.
Therefore $N$ is complete in $M$.

To prove minimality, let
$$u_\alpha(k, x) = (F_\alpha)_* 1_{k > k_\alpha}$$
and similarly $u_\alpha^n(k, x) = (F_\alpha^n)_* 1_{k > k_\alpha^n}$ where $(k_\alpha^n) \in \prod_n K_\alpha^n$ converges to $k_\alpha$.
Since $F_\alpha \circ (F_\alpha^n)^{-1}$ converges to the identity map in $C^{1-}$, and $F_\alpha^{-1}(N \cap U_\alpha)$ has zero measure, it follows that $u_\alpha^n \to u_\alpha$ almost everywhere, and hence in $L^1(I \times J)$ by the dominated convergence theorem.
But $u_\alpha^n$ has least gradient, so by Proposition \ref{MirandaStability}, $\dif u_\alpha^n \to \dif u_\alpha$ in the weak topology of measures.
Clearly $\dif u_\alpha = [N \cap U_\alpha]$ and similarly for $u_\alpha^n$, so by Lemma \ref{measured convergence is smooth convergence}, $N \cap U_\alpha$ is minimal.
\end{proof}

\begin{lemma}
	Let $\lambda$ be the lamination with laminar atlas $(F_\alpha, K_\alpha)$.
	Then $\lambda$ is well-defined and minimal.
\end{lemma}
\begin{proof}
Since 
$$\supp \lambda \cap U_\alpha = K_\alpha \times J$$
and $K_\alpha$ is compact, $\supp \lambda$ is closed.
Now if we choose $\alpha$ such that $K_\alpha$ is nonempty, every element of $K_\alpha$ uniquely determines a cocycle of labels, and hence a leaf of $\lambda$.
So $\supp \lambda$ is nonempty, and since all of its leaves are complete minimal, $\lambda$ is minimal.
\end{proof}

\subsubsection{Convergence in Thurston's geometric topology}
At this stage of the argument we have constructed a limiting lamination with limiting flow boxes; we now check that the sequence of laminations actually converges to the limiting lamination.

If $K_\alpha$ is nonempty, then any $k_\alpha \in K_\alpha$ is the limit of some sequence $(k_\alpha^n)_n \in \prod_n K_\alpha^n$ \cite[Theorem 4.11]{nadler2017continuum}.
Thus $\{k_\alpha\} \times J$ can be written as the set of limits of sequences $(k_\alpha^n, x)_n \in \prod_n K_\alpha^n \times J$, and so any leaf $N$ of $\lambda$ takes the form $N = \lim_{n \to \infty} N_n$ for some sequence $(N_n) \in \prod_n \Leaves \lambda_n$, where $\Leaves \lambda_n$ is the set of leaves of $\lambda_n$.
In other words, leaves of $\lambda$ are pointwise limits of leaves in $\lambda_n$.

So it suffices to show that for $N \in \Leaves \lambda$, $P \in N$, and $P_n \to P$, where $P_n \in N_n$ and $N_n \in \Leaves \lambda_n$, $\normal_{N_n}(P_n) \to \normal_N(P)$.
To do this, suppose that $P \in U_\alpha$; $F_\alpha^n$ is close in tangential $C^\infty$ to $F_\alpha$, and the label $k^n_\alpha$ of $N_n$ is close to the label $k_\alpha$ of $N$.
In particular, if we consider $N$ and $N_n$ as graphs of functions $u, u_n$ in the coordinates induced by $F_\alpha$, then $u_n \to u$ in $C^\infty$; however, in such coordinates, $u$ is a constant.
A bootstrapping argument based on (\ref{nabla as a normal}) then shows that, since $\dif u_n \to 0$ in $C^0$, $\normal_{N_n} \to \partial_y = \normal_N$ in $C^0$ near $P$.

\subsubsection{Convergence in the measure topology}
Suppose that $\mu_n$ is transverse to $\lambda_n$.
After possibly shrinking the $U_\alpha$ slightly for $\alpha \in A_E$, we may assume that they are precompact in $M$ and still form an open cover of $E$.
Then $K := \bigcup_{\alpha \in A_E} \overline{U_\alpha}$ is compact, so by Prohorov's theorem \cite[Theorem 13.29]{klenke2013probability}, there is a subsequence of $(T_{\mu_n})$ which converges to some $T_\mu|_K$ on $K$.
Moreover, by the portmanteau theorem,
$$\supp T_\mu|_K \subseteq \liminf_{n \to \infty} \supp T_{\mu_n}|_K \subseteq \liminf_{n \to \infty} \supp \lambda_n \cap K.$$
Here the $(\lambda_n)$ in the limit inferior refers to the subsequence which already converges in the Thurston topology (and has converging Ruelle-Sullivan currents).
In particular, the limit inferior is actually a limit and we conclude
$$\supp T_\mu|_K \subseteq \supp \lambda \cap K.$$
We may assume that $\mu_\alpha^n \to \mu_\alpha$ weakly for every $\alpha \in A_E$ and some positive Radon measures $\mu_\alpha$ (whose support is necessarily then contained in $K_\alpha$).
Taking the limit as $n \to \infty$ of the equation 
$$\int_{U_\alpha} T_{\mu_n} \wedge \varphi = \int_I \int_{\{k\} \times J} (F_\alpha^n)^* \varphi \dif \mu_\alpha^n(k),$$
we conclude that
$$\int_{U_\alpha} T_\mu|_K \wedge \varphi = \int_I \int_{\{k\} \times J} F_\alpha^* \varphi \dif \mu_\alpha(k).$$
In other words, $T_\mu|_K$ is Ruelle-Sullivan for $\lambda|_K$, possibly after shrinking $\lambda|_K$ so that their supports match.
By the measure-preserving condition in the definition of transverse measure, $T_\mu|_K$ extends uniquely to a Ruelle-Sullivan current $T_\mu$ on all of $M$, which then necessarily is a weak limit of $(T_{\mu_n})$.
This completes the proof of Theorem \ref{compactness theorem}.


%%%%%%%%%%%%%%%%%%%%%%%%%%%%%%%%%%%%%%
\subsection{Consequences of measured convergence}
We now apply Theorem \ref{compactness theorem} to explain how the different modes of convergence are related to each other.
It is clear from the definitions that flow-box convergence implies Thurston convergence.
Moreover, for $d = 2$, Thurston claimed that that measure convergence implies Thurston convergence \cite[Proposition 8.10.3]{thurston1979geometry}, though he did not explicitly justify why the limit was geodesic, or why the convergence preserves the normal vectors.
We complete the proof that measure convergence implies Thurston convergence, and show that flow-box convergence sits in the middle of the chain of implications.

\begin{lemma}\label{limits of measured geodesic lams are geodesic}
Suppose that $\dim M \leq 7$. The set of minimal measured laminations of bounded curvature is closed in the weak topology of measures.
\end{lemma}
\begin{proof}
Let $(\lambda, \mu)$ be a measured lamination and suppose that $(\lambda_i, \mu_i) \to (\lambda, \mu)$ in the weak topology of measures, where $(\lambda_i, \mu_i)$ are measured minimal and of bounded curvature.
Let $x \in \supp \lambda$ and $r > 0$ such that $B := B(x, r)$ is contractible.
In $B$, we can write $T_{\mu_i} = \dif u_i$ for some sequence of functions of least gradient $u_i \in BV(B)$.
Since $u_i$ is only defined up to a constant, we impose $\int_M \star u_i = 0$, so by Poincar\'e's inequality,
$$\|u_i\|_{L^1(B)} \lesssim r\mu_i(B) \leq 2r \mu(B) < \infty$$
for $i$ large.
So by Proposition \ref{MirandaStability}, there exists a $1$-harmonic function $u$ such that along a subsequence, $\dif u_i \to \dif u$ in the weak topology of measures.
Then $T = \dif \mu$, so the leaves of $\lambda$ are level sets of $u$.
By Theorem \ref{main thm of old paper} and the fact that $\dim M \leq 7$, the leaves of $\lambda$ are minimal hypersurfaces, as desired.
\end{proof}

\begin{proposition}
Suppose that $\dim M \leq 7$.
Let $(\lambda_n, \mu_n)$ be measured minimal laminations in $M$, and $(\lambda_n, \mu_n) \to (\lambda, \mu)$.
Then $\lambda_n \to \lambda$ in Thurston's geometric topology.
\end{proposition}
\begin{proof}
By (\ref{supports shrink in the limit}), for every $x \in \supp \lambda$, $\varepsilon > 0$, and large $n$, $\supp \lambda_n \cap B(x, \varepsilon)$ is nonempty, and by Lemma \ref{limits of measured geodesic lams are geodesic}, $\lambda$ is a minimal lamination.
By Theorem \ref{regularity theorem}, $\lambda, \lambda_n$ admit Lipschitz normal vectors, so by Lemma \ref{convergence of normals}, $\lambda_n \to \lambda$ in Thurston's geometric topology.
\end{proof}

\begin{proposition}
Suppose that $\dim M \leq 7$.
Let $(\lambda_n, \mu_n)$ be measured minimal laminations in $M$ of bounded curvature, and $(\lambda_n, \mu_n) \to (\lambda, \mu)$.
Then $\lambda_n \to \lambda$ in the $C^{1-}$ and tangentially $C^\infty$ flow box topology.
\end{proposition}
\begin{proof}
We first observe that $\lambda_n \to \lambda$ in Thurston's geometric topology.
After discarding some leaves of $\lambda_n$ we may assume that $\lambda$ is a maximal limit for the Thurston topology.
Moreover, every subsequence $(\lambda_{n_k})$ has a further subsequence $(\lambda_{n_{k_\ell}})$ which converges to some maximal limit $\tilde \lambda$ in the $C^{1-}$ flow box topology by Theorem \ref{compactness theorem}.
But convergence in the flow box topology implies convergence in Thurston's topology, so $\tilde \lambda = \lambda$.
Since $(\lambda_{n_k})$ was arbitrary, it follows that $\lambda_n \to \lambda$ in the $C^{1-}$ flow box topology.
\end{proof}


%%%%%%%%%%%%%%%%%%%%
\section{Application to \texorpdfstring{$1$-harmonic}{one-harmonic} functions}\label{1harmonic sec}
The purpose of this section is to prove Theorem \ref{main thm}, and explore some of its consequences.
Throughout, we shall assume that $2 \leq \dim M \leq 4$, as we shall appeal to the stable Bernstein theorem (Theorem \ref{stable Bernstein}) in the proof that a $1$-harmonic function induces a minimal lamination.

%%%%%%%%%%%%%
\subsection{Stability of minimal surfaces}
Here we collect some results about the stability of minimal surfaces that will be useful in the proof of Theorem \ref{main thm}.
We begin with an estimate on the curvature of a stable minimal surface.
Since, by Theorem \ref{main thm of old paper}, the level sets of a function of least gradient are stable, we can in particular obtain estimates on the curvature of the level sets of a function of least gradient.
If $\dim M = 3$, the stable Bernstein theorem was proven by Schoen \cite{Schoen2016} and Colding and Minicozzi \cite{ColdingMinicozziParametric}; see also \cite[Theorem 2.10]{colding2011course}.
If $\dim M = 4$ the stable Bernstein theorem was recently proven by Chodosh and Li \cite{Chodosh2021}.
The statement here is not quite the formulation of Chodosh and Li, but it is clear that our formulation reduces to \cite[Theorem 1]{Chodosh2021} using the point-picking argument of \cite[\S5]{Chodosh2021}.

\begin{theorem}[stable Bernstein theorem]\label{stable Bernstein}
	Let $M$ be a manifold of bounded geometry, possibly with boundary, such that $2 \leq \dim M \leq 4$.
	Then there exists $A \geq 0$ such that for every complete stable minimal immersion $N \to M$ of codimension $1$ and trivial normal bundle,
	$$|\Two_N(P)| \leq \frac{A}{\dist(P, \partial M)}.$$
\end{theorem}

This estimate only has content when $\dim M = 3, 4$.
If one proved it for $\dim M = 5, 6, 7$, then Theorem \ref{main thm} would follow for such dimensions.

In the converse direction, it will be occasionally useful to know that $1$-harmonic functions have stability radii bounded from below independently of the function.

\begin{lemma} \label{LLG implies LG}
Let $u$ be a $1$-harmonic function, $x \in M$, and $r > 0$ is at most the injectivity radius of $M$ at $x$.
Then $u|_{B(x, r)}$ has least gradient.
\end{lemma}
\begin{proof}
Let $(U_\alpha)$ be an open cover of $B(x, r)$ by sets in which $u$ has least gradient, and let $(u_t)$ be a variation of $u$ for the Dirichlet problem for the total variation energy $J$.
Since $\overline{B(x, r)}$ is a complete contractible manifold-with-boundary, one can show that $J$ is convex.
By taking a partition of unity, we may write $u_t = \sum_\alpha u_t^\alpha$ where $u_t^\alpha$ is a variation of $u$ for the Dirichlet problem on $U_\alpha$.
Then at $t = 0$,
$$\partial_t J(u_t^\alpha) = \partial_t \int_{B(x, r)} \star |\dif u_t^\alpha| = \partial_t \int_{U_\alpha} \star |\dif u_t^\alpha| = 0.$$
Summing in $\alpha$, we see that $\partial_t J(u_t) = 0$.
So $u$ is a critical point of the convex function $J$, and hence is a minimum.
\end{proof}

\begin{lemma}\label{minimal implies 1 harmonic}
Let $N$ be a minimal hypersurface which bounds an open set $U$.
Then $1_U$ is $1$-harmonic.
\end{lemma}
\begin{proof}
Since $N$ is minimal, it is locally area-minimizing, in the sense that we can find an open cover $\mathcal V$ of a tubular neighborhood $T$ of $N$, such that for $V \in \mathcal V$, $N \cap V$ is minimal for the Dirichlet problem in $V$.
This follows from \cite[\S2]{Lawlor96} and the references therein.
In particular, $1_U|_V$ has least gradient, so $1_U$ has locally least gradient on $T$.
Since $1_U|_{M \setminus T}$ is constant, it follows that $1_U$ is $1$-harmonic.
\end{proof}

%%%%%%%%%%%%%%
\subsection{Proof of Theorem \texorpdfstring{\ref{main thm}}{C}}
\subsubsection{\texorpdfstring{$1$-harmonic}{One-harmonic} function induces minimal lamination}
Let $u$ be a $1$-harmonic function on $M$.
The statement is local, so we may always replace $M$ with a small open set.
In particular, we may assume that $u$ has least gradient, so that by Theorem \ref{main thm of old paper}, the level sets of $u$ are complete embedded stable minimal hypersurfaces in $M$.
Let
$$Y = \{y \in \RR: \partial \{u > y\} \neq \emptyset\}$$
index the level sets of $u$.

Let $y, z \in Y$. If $y > z$, then $\{u > y\} \subseteq \{u > z\}$, so $\partial \{u > y\}$ lies on one side of $\partial \{u > z\}$.
By the maximum principle, it follows that either $\partial \{u > y\}$ and $\partial \{u > z\}$ are disjoint, or are equal.
Moreover, $\dif u$ is conormal to $\partial \{u > y\}$, so $\partial \{u > y\}$ has trivial normal bundle.
By the stable Bernstein theorem,
\begin{equation}\label{curvature estimate on 1 harmonic}
	\sup_{y \in Y} \|\Two_{\partial \{u > y\}}\|_{C^0} \leq A.
\end{equation}
Then by Theorem \ref{regularity theorem}, $\supp \dif u$ is the support of a lamination $\lambda$ with leaves given by the level sets of $u$.
It follows that $\lambda$ is minimal and $\dif u$ is conormal to $\lambda$.
In particular, we obtain an orientation on $\lambda$ from $\dif u$.

We now construct the transverse measure to $\lambda$.
In any oriented laminar coordinates $(k, x) \in K \times J$ for $\lambda$, $\partial_x u = 0$, so $\star |\dif u|$ defines a measure $\mu$ on $K$: given $\alpha < \beta$, let
$$\mu([\alpha, \beta] \cap K) := u(\beta, x) - u(\alpha, x)$$
for any (and hence every, since $\partial_x u = 0$) $x \in J$.
Since $(k, x)$ are oriented laminar coordinates, $u(\cdot, x)$ is nondecreasing, so $\mu$ is a positive measure.

If $(k', x') \in K' \times J$ is a different laminar coordinate system, and the transition map carries $\alpha, \beta$ to $\alpha', \beta'$, then
$$\mu'([\alpha', \beta'] \cap K') := u'(\beta', x') - u(\alpha', x') = u(\beta, x_1) - u(\alpha, x_2)$$
for some $x_1, x_2 \in J$. Since $\partial_x u = 0$,
$$u(\beta, x_1) - u(\alpha, x_2) = u(\beta, x_1) - u(\alpha, x_1) = \mu([\alpha, \beta] \cap K).$$
It follows that $\mu$ is transverse, and by construction $\mu$ lifts to $\star |\dif u|$ in $M$.
So by (\ref{polar ruelle sullivan}), $\dif u = \normal_\lambda |\dif u|$ is the Ruelle-Sullivan current for the measured oriented structure we just imposed on $\lambda$.

%%%%%%%%%%%%%%%%

\subsubsection{Minimal lamination induces \texorpdfstring{$1$-harmonic}{one-harmonic} function}
Suppose that we are given a measured oriented minimal lamination $\lambda$, which then has a Ruelle-Sullivan current $T$.
Since $\dif T = 0$, we may assume, possibly after replacing $M$ with its universal cover, that $T$ is exact, say $T = \dif u$, and we must show that $u$ has locally least gradient.
If this is not true, then by \cite[Theorem 2.2]{Sternberg93}, we can choose a small open ball $E \subseteq M$ and a function $v \in BV_\cpt(E)$ such that
$$\int_E \star |\dif u + \dif v| < \int_E \star |\dif u| < \infty.$$
Since $v$ has compact support, there exists a collar neighborhood $F \subset E$ of $\partial E$ such that for every $y \in \RR$,
\begin{equation}\label{collar condition}
	\partial \{u > y\} \cap F = \partial^* \{u + v > y\} \cap F.
\end{equation}
Here $\partial^*$ is the reduced boundary operator (see \cite[Chapter 3]{Giusti77} or \cite[\S2]{BackusFLG}).

The function $1_{\{u > y\}}$ is $1$-harmonic by Lemma \ref{minimal implies 1 harmonic}, so by Lemma \ref{LLG implies LG}, $1_{\{u > y\}}|_E$ has least gradient.
By (\ref{collar condition}), $1_{\{u + v > y\}} - 1_{\{u > y\}}$ has compact support in $E$, so 
$$\int_E \star |\dif 1_{\{u > y\}}| \leq \int_E \star |\dif 1_{\{u + v > y\}}|.$$
Since $1_{\{u > y\}}$ has least gradient in $E$, we estimate using the coarea formula (see \cite[Proposition 2.5]{BackusFLG} for a proof at this regularity)
\begin{align*}
\int_E \star |\dif u| &= \int_{-\infty}^\infty \int_E \star |\dif 1_{\{u > y\}}| \dif y \leq \int_{-\infty}^\infty \int_E \star |\dif 1_{\{u + v > y\}}| \dif y \\
&= \int_E \star |\dif u + \dif v| < \int_E \star |\dif u|
\end{align*}
which is a contradiction.
This completes the proof of Theorem \ref{main thm}.

%%%%%%%%%%%%%%%%%%%%%%%%%%%%
\subsection{Conclusions for \texorpdfstring{$1$-harmonic}{one-harmonic} functions}\label{1harmonic apps}
Let us explore some consequences of the lamination perspective on $1$-harmonic functions.

\subsubsection{Nonuniqueness of the Dirichlet problem}
We first recall that the Dirichlet problem for the $1$-Laplacian need not have a unique solution \cite[Example 2.7]{Mazon14}.\footnote{The formulation of the Dirichlet problem is subtle. Here we take the relaxation formulation of \cite[Definition 2.3]{Mazon14}, though the same argument should work for any reasonable formulation.}
However, Maz\'on, Rossi, and Segura de Le\'on \cite[Remark 2.8]{Mazon14} conjectured that two solutions of the Dirichlet problem will have the same ``frame of superlevel sets,'' a notion they do not define.
In view of Theorem \ref{main thm} it is natural to interpret their conjecture to mean that the (unmeasured) lamination associated to a $1$-harmonic function is uniquely determined by the Dirichlet data.
For this interpretation, their conjecture is false:

\begin{example}
Consider the euclidean ball $\Ball^3$, with coordinates $(x, y) \in \RR^2 \times \RR$.
Let $N$ be a vertical catenoid in $\Ball^3$, so that $N$ intersects $\partial \Ball^3 = \Sph^2$ in two circles $\{y = y_*\} \cap \Sph^2$.
Then $N$ separates $\Ball^3$ into the interior $U$ of the catenoid, and an exterior annular region.
We can then set $u := 1_U$; the trace of $u$ is $h(x, y) := 1_{|y| > y_*}$, and the only level set of $u$ is a minimal surface, so $u$ is $1$-harmonic by Lemma \ref{minimal implies 1 harmonic}.
However, we can also extend $h$ to the function $v(x, y) := 1_{|y| > y_*}$, which obviously has least gradient since its only level set is the union of two flat disks $\{y = \pm y_*\}$.
\end{example}

\subsubsection{The G\'orny decomposition}
We now consider an analogue of the G\'orny decomposition \cite[Theorem 1.2]{górny2017planar} of a function of least gradient.
Recall that a \dfn{Cantor function} is a continuous function whose exterior derivative is mutually singular with Lebesgue measure.
A \dfn{jump function} is a function $u: M \to \RR$ such that $\dif u = \sum_i a_i \delta_{N_i}$, where $(N_i)$ is a (possibly finite) sequence of hypersurfaces, $a_i \in \RR$, and $\delta_{N_i}$ is the Dirac measure for $N_i$, defined by
$$\int_M f \dif \delta_{N_i} = \int_{N_i} f \dif S.$$
Thus $u$ jumps across each $N_i$ by $a_i$.
In general, it is not possible to decompose a function $u$ of bounded variation into an absolutely continuous (that is, $W^{1, 1}_\loc$) part, a Cantor part, and a jump part \cite[Example 4.1]{Ambrosio2000FunctionsOB}.
G\'orny showed that for a function of least gradient on euclidean space, such a decomposition exists.
We give a new proof using Theorem \ref{main thm} which applies on curved domains.

\begin{proposition}
Let $u$ be a $1$-harmonic function and suppose that $H^1(M, \RR) = 0$. Then there exists a decomposition of $u$ into $1$-harmonic functions 
$$u = u_{ac} + u_C + u_j,$$
with mutually singular exterior derivatives, such that $u_{ac} \in W^{1, 1}_\loc(M)$, $u_C$ is a Cantor function, and $u_j$ is a jump function.
Up to addition of additive constants, this decomposition is unique.
\end{proposition}
\begin{proof}
By considering the minimal lamination associated to $u$, we obtain a laminar atlas $(F_\alpha)$.
We then prove the decomposition in a single flow box $F_\alpha: U_\alpha \to V_\alpha$, where
$$U_\alpha = I \times J \subset \RR_x \times \RR^{d - 1}_y.$$
We use Lemma \ref{LLG implies LG} to assume that $V_\alpha$ is contained in a ball so small that $u|_{V_\alpha}$ has least gradient.

In the flow box coordinates, $u(x, y) = \tilde u^\alpha(x)$ for some $\tilde u_\alpha: I \to \RR$. 
Since $\dif \tilde u^\alpha$ is the transverse measure, it is a Radon measure, so $\tilde u^\alpha$ has bounded variation and hence we have the Lebesgue decomposition on an interval \cite[Corollary 3.33]{Ambrosio2000FunctionsOB} 
$$\tilde u^\alpha = \tilde u^\alpha_{ac} + \tilde u^\alpha_C + \tilde u^\alpha_j$$
where $u^\alpha_{ac} \in W^{1, 1}_\loc(I)$, $u^\alpha_C$ is a Cantor function, and $u_j^\alpha$ is a jump function.
This decomposition is unique up to an addition of constants, and induces a decomposition of $\dif \tilde u^\alpha$ into mutually singular measures.
We then write $u^\alpha_\sigma(x, y) = \tilde u^\alpha_\sigma(x)$ to obtain a function on $V_\alpha \subseteq M$, where $\sigma \in \{ac, C, j\}$.

We next claim that $\dif u^\alpha_{ac}, \dif u^\alpha_C, \dif u^\alpha_j$ are mutually singular.
Since $\dif \tilde u^\alpha_{ac}, \dif \tilde u^\alpha_C, \dif \tilde u^\alpha_j$ are mutually singular, we have a decomposition
$$I = I_{ac} \sqcup I_C \sqcup I_j$$
such that for $\sigma \neq \tau$, $I_\tau$ is a $\dif \tilde u^\alpha_\sigma$-null set.
Applying Fubini's theorem, the same decomposition holds for $\dif u^\alpha_\sigma$ and $I \times J \cong V_\alpha$, implying mutual singularity of the $\dif u^\alpha_\sigma$.

We now claim that $u^\alpha_\sigma$ have least gradient on $V_\alpha$.
To ease notation we do this for $\sigma = j$; the other cases are similar.
If the claim fails, then there is some $v \in BV_\cpt(V_\alpha)$ such that $\int \star |\dif u^\alpha_j| > \int \star |\dif (u^\alpha_j + v)|$.
But if so, then by mutual singularity and the fact that $u$ has least gradient on $V_\alpha$,
\begin{align*}
\int_{V_\alpha} \star |\dif u| &= \int_{V_\alpha} \star (|\dif u^\alpha_j| + |\dif u^\alpha_C| + |\dif u^\alpha_{ac}|) \\
&>  \int_{V_\alpha} \star (|\dif u^\alpha_j + \dif v| + |\dif u^\alpha_C| + |\dif u^\alpha_{ac}|) \\
&\geq \int_{V_\alpha} \star |\dif u + \dif v| \geq \int_{V_\alpha} \star |\dif u|,
\end{align*}
a contradiction.

Finally we glue the local decompositions together.
The measure-preserving property of transition maps and the uniqueness of the Lebesgue decomposition implies that
$$\dif u^\alpha_\sigma|_{V_\alpha \cap V_\beta} = \dif u^\beta_\sigma|_{V_\alpha \cap V_\beta}.$$
As closed currents form a sheaf, it follows that there exist unique closed currents $\dif u_\sigma$ on all of $M$ such that $\dif u_\sigma|_{V_\alpha} = \dif u^\alpha_\sigma$.
Since $H^1(M, \RR)$, $\dif u_\sigma$ has an antiderivative $u_\sigma$, which is $1$-harmonic since $u_\sigma|_{V_\alpha} = u_\sigma^\alpha$ has least gradient.
\end{proof}

%%%%%%%%%%%%%%%%%%%%%%%%%%%%%%
% \subsection{The Euler characteristic of a minimal lamination}
% Suppose that $\dim M = 3$, so that the leaves of a measured lamination $\lambda$ are surfaces.
% Under these hypotheses, Morgan and Shelan \cite[\S5]{Morgan88} defined the Euler characteristic $\chi(\lambda)$.

% \begin{definition}
% A measured lamination $\lambda$ is a \dfn{parallel family of compact leaves} if there is a compact surface $N \subset M$, possibly with boundary, such that leaves of $\lambda$ are sections of the normal bundle of $N$.
% \end{definition}

% \begin{definition}
% The \dfn{Euler characteristic} $\chi(\lambda)$ of a measured lamination $\lambda$ in a compact $3$-manifold $M$, possibly with boundary, has the following properties:
% \begin{enumerate}
% \item Suppose that $\lambda$ is a parallel family of compact leaves. Then for any leaf $N$ of $\lambda$,
% $$\chi(\lambda) := |\lambda| \chi(N).$$
% \item Suppose that we can write $M = \bigcup_i M_i$ such that:
% \begin{enumerate}
% \item $M_i$ is a compact $3$-manifold, possibly with boundary.
% \item The $M_i$ are in general position with respect to each other.
% \item The intersections $M_{i_1 \cdots i_k} := \bigcap_j M_{i_j}$ have boundary transverse to $\lambda$.
% \end{enumerate}
% Then
% $$\chi(\lambda) := \sum_{k=1}^\infty (-1)^k \sum_{i_1 < \cdots < i_k} \chi(\lambda|_{M_{i_1 \cdots i_k}}).$$
% \end{enumerate}
% \end{definition}

%%%%%%%%%%%%%%%%%%%%%%%%%%%%%


%%%%%%%%%%%%%%%%%%%%%%%%%%%%%%%
% \appendix \section{Transverse cocycles}\label{transverse curves}
% In this appendix, we show that convergence in the weak topology of measures as we have stated it (that is, the weak topology on the space of Ruelle-Sullivan currents) is equivalent to the formulation of Thurston \cite[\S8.6]{thurston1979geometry} that is more familiar to geometric topologists.

% Let $\lambda$ be an oriented minimal lamination.
% By Theorem \ref{regularity theorem}, $\lambda$ has a global Lipschitz normal vector field $\normal_\lambda$ and is tangentially $C^\infty$.
% We shall assume that $\supp \lambda$ is a Lebesgue null set.
% This assumption is harmless, because in the application of interest to geometric topologists, $M$ is a closed hyperbolic surface, and then by the Gauss-Bonnet formula it is indeed true that $\supp \lambda$ is null \cite[\S8.5]{thurston1979geometry}.

% We now review preliminaries that were established in \cite[\S7.2]{daskalopoulos2020transverse}.
% A curve $\gamma: I \to M$ is said to be \dfn{positively transverse} to $\lambda$ if $\langle \gamma', \normal_\lambda \rangle > 0$ on $\supp \lambda$; by taking pushforwards by the inverse of a laminar flow box, this is equivalent to the condition of \cite[Definition 7.7]{daskalopoulos2020transverse}.
% We define negatively transverse curves similarly.
% The transverse curve $\gamma$ is \dfn{admissibly transverse} if, in addition, the endpoints of $\gamma$ lie in $M \setminus \supp \lambda$.

% % A sum of admissibly transverse curves is known as a \dfn{transverse $1$-chain}.
% % We write $C_1(M, \lambda)$ for the group of transverse $1$-chains, modulo transverse homotopies to $\lambda$.
% % A \dfn{transverse $1$-cocycle} is a representation $C_1(M, \lambda) \to \RR$ \cite[Definition 7.12]{daskalopoulos2020transverse}.

% % If $\mu$ is a transverse measure, then we can define a transverse $1$-cocycle, which we also denote by $\mu$, as follows.
% % By subdividing $\gamma \in C_1(M, \lambda)$, we may assume that $\gamma$ is a positively transverse curve in the image $U_\alpha$ of a flow box $F_\alpha$.
% % The projection of $(F_\alpha^{-1})_* \gamma$ to $I \subset K_\alpha$ is strictly increasing since $\gamma$ is positively transverse, so $\mu_\alpha$ induces a measure $\tilde \mu$ on $\gamma(I)$. We then define
% % \begin{equation}\label{definition of cocycle}
% % 	\mu(\gamma) := \int_{\gamma(I)} \langle \gamma', \normal_\lambda \rangle \dif \tilde \mu.
% % \end{equation}
% % This formula defines a positively transverse cocycle, and every positively transverse cocycle arises in this way; these facts are easy modifications of the discussion in \cite[\S7.2]{daskalopoulos2020transverse}.

% If $\gamma: I \to M$ is an admissibly transverse curve, and $\mu$ is a transverse measure, then we define a measure $\gamma^! \mu$, the \dfn{exceptional pullback}\footnote{One can only push forward measures in general.} of $\mu$, on $I$.
% First, we may assume that $\gamma$ is an alternating sum of admissibly positively transverse curves, and then by restricting to any such summand we may assume that $\gamma$ is admissibly positively transverse.
% Such alternating sums are known as \dfn{good subdivisions} and constructed in \cite[Lemma 7.9]{daskalopoulos2020transverse}.

% If $\gamma$ is positively admissibly transverse, we consider all decompositions $\gamma = \sum_i \gamma_i$ where $\gamma_i$ is positively admissibly transverse and has domain $I_i = [t_i, t_{i + 1}] \subseteq I$.
% Then, in a neighborhood of $\gamma_i$, $T_\mu$ is exact, say $T_\mu = \dif u$. Then we may set 
% $$\gamma^! \mu(I_i) := u(\gamma(t_{i + 1})) - u(\gamma(t_i)).$$
% Since $\supp \lambda$ is Lebesgue null, the set of $t$ such that $u(\gamma(t))$ is ill-defined is also Lebesgue null, and $u$ is a nondecreasing function since $\gamma$ is positively admissibly transverse.
% So $\gamma^! \mu$ is a well-defined Radon measure on $I$ whose (distributional) Radon-Nikod\'ym derivative is $\dif(\gamma^* u)$ (which itself is well-defined since $u$ is nondecreasing and hence $BV(I)$).
% We define $\gamma^! \mu$ for inadmissible transverse curves $\gamma$ by extending $\gamma$ slightly to a transverse curve and then taking intersections over extensions and applying continuity from above.

% Taking exceptional pullbacks, we see that we may view $\mu$ as the data of a Radon measure on every transverse curve satisfying a compatibility condition; this is the definition of transverse measure which appears in the definition of measure convergence in \cite[\S8.6]{thurston1979geometry}.


% \begin{proposition}\label{characterization of measure convergence}
% 	Let $(\lambda_n, \mu_n)$ and $(\lambda, \mu)$ be oriented measured minimal laminations with Lebesgue null supports. Then $(\lambda_n, \mu_n) \to (\lambda, \mu)$ iff for every positively transverse curve $\gamma$ to $\lambda$ defined in a small neighborhood of $\supp \lambda$:
% \begin{enumerate}
% \item $\gamma$ is eventually positively transverse to $\lambda_n$, and 
% \item $\gamma^! \mu_n \to \gamma^! \mu$ in the weak topology of measures.
% \end{enumerate}
% \end{proposition}
% \begin{proof}
% 	First assume that $(\lambda_n, \mu_n) \to (\lambda, \mu)$ and $\gamma$ is positively transverse to $\lambda$.
% 	By Lemma \ref{convergence of normals}, in a neighborhood of $\supp \lambda$, we have $\langle \normal_{\lambda_n}, \gamma' \rangle > 0$ for $n$ large, where $n$ can be chosen uniformly on $M$ since the image of $\gamma$ is compact. So $\gamma$ is eventually positively transverse to $\lambda_n$.

% 	By working locally we may assume that $T_\mu = \dif u$ and $T_{\mu_n} = \dif u_n$, and by possibly extending $\gamma$ if necessary we may assume that it is admissibly transverse.
% 	Since $\dif u_n \to \dif u$ in the weak topology of measures, the portmanteau theorem implies that for any curve $\rho$ from $x$ to $y$ such that $u$ is continuous near $x, y$, $u_n(y) - u_n(x) \to u(y) - u(x)$.
% 	Taking $\rho$ to range over admissible subcurves of $\gamma$, and applying the portmanteau theorem again, we conclude that $\dif(\gamma^* u_n) \to \dif(\gamma^* u)$ and hence $\gamma^! \mu_n \to \gamma^! \mu$.

% 	Conversely, if $\gamma^! \mu_n \to \gamma^! \mu$ for every positively transverse $\gamma$, then by the existence of flow box coordinates for $\lambda$ (given by Proposition \ref{regularity theorem}), we can actually foliate a neighborhood of any point $x$ of $\supp \lambda$ by admissibly positively transverse curves.
% 	These curves depend continuously on their intersection point with the leaf of $\supp \lambda$ containing $x$, so the fact that one of them is positively transverse to $\lambda_n$ for $n \geq N/2$ and some even integer $N$ implies that in a neighborhood of $x$, every curve in the foliation is positively transverse to $\lambda_n$ for $n \geq N$.
% 	Let $J$ be the set of such curves, equipped with the measure $\nu$ obtained by disintegrating the Riemannian measure into arc length measures $s_\gamma$ on each $\gamma$.
	
% 	If $\varphi$ is a continuous $d-1$-form with support near $x$, we may define the restriction $f_\gamma$ of $\varphi \wedge (\gamma')^\flat$ to $\gamma$.
% 	By the disintegration theorem,
% 	$$\int_M T_\mu \wedge \varphi = \int_J \int_I f_\gamma \frac{\dif(\gamma^! \mu)}{\dif s_\gamma} \dif s_\gamma \dif \nu(\gamma) = \int_J \int_I f_\gamma \dif(\gamma^! \mu) \dif \nu(\gamma).$$
% 	One can show using the dominated convergence theorem and the fact that $\gamma^! \mu_n \to \gamma^! \mu$ weakly that 
% 	$$\int_M T_\mu \wedge \varphi = \lim_{n \to \infty} \int_J \int_I f_\gamma \dif(\gamma^! \mu_n) \dif \nu(\gamma) = \lim_{n \to \infty} \int_M T_{\mu_n} \wedge \varphi$$
% 	but since $\varphi$ is arbitrary, it holds that $T_{\mu_n} \to T_\mu$ in the weak topology of measures.
% \end{proof}


\printbibliography

\end{document}
