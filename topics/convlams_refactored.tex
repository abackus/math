\documentclass[reqno,11pt]{amsart}
\usepackage[letterpaper, margin=1in]{geometry}
\RequirePackage{amsmath,amssymb,amsthm,graphicx,mathrsfs,url,slashed,subcaption}
\RequirePackage[usenames,dvipsnames]{xcolor}
\RequirePackage[colorlinks=true,linkcolor=Red,citecolor=Green]{hyperref}
\RequirePackage{amsxtra}
\usepackage{cancel}
\usepackage{tikz-cd}

% \setlength{\textheight}{9.3in} \setlength{\oddsidemargin}{-0.25in}
% \setlength{\evensidemargin}{-0.25in} \setlength{\textwidth}{7in}
% \setlength{\topmargin}{-0.25in} \setlength{\headheight}{0.18in}
% \setlength{\marginparwidth}{1.0in}
% \setlength{\abovedisplayskip}{0.2in}
% \setlength{\belowdisplayskip}{0.2in}
% \setlength{\parskip}{0.05in}
%\renewcommand{\baselinestretch}{1.05}

\title{Minimal laminations and level sets of $1$-harmonic functions}
\author{Aidan Backus}
\address{Department of Mathematics, Brown University}
\email{aidan\_backus@brown.edu}
\date{\today}

\newcommand{\NN}{\mathbf{N}}
\newcommand{\ZZ}{\mathbf{Z}}
\newcommand{\QQ}{\mathbf{Q}}
\newcommand{\RR}{\mathbf{R}}
\newcommand{\CC}{\mathbf{C}}
\newcommand{\DD}{\mathbf{D}}
\newcommand{\PP}{\mathbf P}
\newcommand{\MM}{\mathbf M}
\newcommand{\II}{\mathbf I}
\newcommand{\Hyp}{\mathbf H}
\newcommand{\Sph}{\mathbf S}
\newcommand{\Group}{\mathbf G}
\newcommand{\GL}{\mathbf{GL}}
\newcommand{\Orth}{\mathbf{O}}
\newcommand{\SpOrth}{\mathbf{SO}}
\newcommand{\Ball}{\mathbf{B}}

\newcommand*\dif{\mathop{}\!\mathrm{d}}

\DeclareMathOperator{\card}{card}
\DeclareMathOperator{\dist}{dist}
\DeclareMathOperator{\MeasLam}{MeasLam}
\DeclareMathOperator{\MinLam}{MinLam}
\DeclareMathOperator{\Lam}{Lam}
\DeclareMathOperator{\supp}{supp}
\DeclareMathOperator{\tr}{tr}

\newcommand{\Leaves}{\mathscr L}
\newcommand{\Lagrange}{\mathcal L}
\newcommand{\Hypspace}{\mathscr H}

\newcommand{\Two}{\mathrm{I\!I}}


\newcommand{\Hilb}{\mathcal H}
\newcommand{\Homology}{\mathrm H}
\newcommand{\normal}{\mathbf n}
\newcommand{\radial}{\mathbf r}
\newcommand{\evect}{\mathbf e}
\newcommand{\vol}{\mathrm{vol}}

\newcommand{\diam}{\mathrm{diam}}
\newcommand{\Ell}{\mathrm{Ell}}
\newcommand{\inj}{\mathrm{inj}}
\newcommand{\Lip}{\mathrm{Lip}}
\newcommand{\Riem}{\mathrm{Riem}}
\newcommand{\Ric}{\mathrm{Ric}}

\DeclareMathOperator*{\essinf}{ess\,inf}
\DeclareMathOperator*{\esssup}{ess\,sup}

\DeclareMathOperator{\Div}{div}

\newcommand{\Bmu}{\boldsymbol \mu}
\newcommand{\Bnu}{\boldsymbol \nu}
\newcommand{\Blambda}{\boldsymbol \lambda}

\newcommand{\pic}{\vspace{30mm}}
\newcommand{\dfn}[1]{\emph{#1}\index{#1}}

\renewcommand{\Re}{\operatorname{Re}}
\renewcommand{\Im}{\operatorname{Im}}

\newcommand{\loc}{\mathrm{loc}}
\newcommand{\cpt}{\mathrm{cpt}}

\def\Japan#1{\left \langle #1 \right \rangle}

\newtheorem{theorem}{Theorem}[section]
\newtheorem{badtheorem}[theorem]{``Theorem"}
\newtheorem{prop}[theorem]{Proposition}
\newtheorem{lemma}[theorem]{Lemma}
\newtheorem{sublemma}[theorem]{Sublemma}
\newtheorem{proposition}[theorem]{Proposition}
\newtheorem{corollary}[theorem]{Corollary}
\newtheorem{conjecture}[theorem]{Conjecture}
\newtheorem{axiom}[theorem]{Axiom}
\newtheorem{assumption}[theorem]{Assumption}

\newtheorem{mainthm}{Theorem}
\renewcommand{\themainthm}{\Alph{mainthm}}

% \newtheorem{claim}{Claim}[theorem]
% \renewcommand{\theclaim}{\thetheorem\Alph{claim}}
\newtheorem*{claim}{Claim}

\theoremstyle{definition}
\newtheorem{definition}[theorem]{Definition}
\newtheorem{remark}[theorem]{Remark}
\newtheorem{example}[theorem]{Example}
\newtheorem{notation}[theorem]{Notation}

\newtheorem{exercise}[theorem]{Discussion topic}
\newtheorem{homework}[theorem]{Homework}
\newtheorem{problem}[theorem]{Problem}

\makeatletter
\newcommand{\proofpart}[2]{%
  \par
  \addvspace{\medskipamount}%
  \noindent\emph{Part #1: #2.}
}
\makeatother



\numberwithin{equation}{section}


% Mean
\def\Xint#1{\mathchoice
{\XXint\displaystyle\textstyle{#1}}%
{\XXint\textstyle\scriptstyle{#1}}%
{\XXint\scriptstyle\scriptscriptstyle{#1}}%
{\XXint\scriptscriptstyle\scriptscriptstyle{#1}}%
\!\int}
\def\XXint#1#2#3{{\setbox0=\hbox{$#1{#2#3}{\int}$ }
\vcenter{\hbox{$#2#3$ }}\kern-.6\wd0}}
\def\ddashint{\Xint=}
\def\dashint{\Xint-}

\usepackage[backend=bibtex,style=alphabetic,giveninits=true]{biblatex}
\renewcommand*{\bibfont}{\normalfont\footnotesize}
\addbibresource{topics.bib}
\renewbibmacro{in:}{}
\DeclareFieldFormat{pages}{#1}

\newcommand\todo[1]{\textcolor{red}{TODO: #1}}


\begin{document}
\begin{abstract}
We collect several results concerning regularity of minimal laminations, and governing the various modes of convergence for sequences of minimal laminations.
We then apply this theory to prove that a function has locally least gradient (is $1$-harmonic) iff its level sets are a minimal lamination; this resolves an open problem of Daskalopoulos and Uhlenbeck.
\end{abstract}

\maketitle

%%%%%%%%%%%%%%%%%%%%%%%%%%%%%%%%%%%%%%%%%%%%%%%%%%%%%%%

% \tableofcontents

\section{Introduction}
The space of codimension-$1$ minimal laminations on a Riemannian manifold has been topologized in several different ways.
Thurston \cite[Chapter 8]{thurston1979geometry} introduced both his geometric topology as well as the weak topology of measures on the space of measured geodesic laminations.
Independently of Thurston, Colding and Minicozzi \cite[Appendix B]{ColdingMinicozziIV} introduced a topology that emphasized not the laminations themselves, but rather the coordinate charts which flatten them.

We establish a regularity theorem for minimal laminations, which implies compactness properties for the aforementioned topologies.
We also show that a current is Ruelle-Sullivan with respect to a minimal lamination if and only if it is locally the exterior derivative of a function of least gradient, generalizing a theorem of Daskalopoulos and Uhlenbeck \cite[Theorem 6.1]{daskalopoulos2020transverse} and strengthing an unpublished result of Auer and Bangert \cite{Auer01, Auer12}.

%%%%%%%%%%%%%%%%%
\subsection{Minimal laminations}\label{Lams sections}
Throughout this paper, we fix an interval $I \subset \RR$, a box $J \subset \RR^{d - 1}$, and a smooth Riemannian manifold $M = (M, g)$ of dimension $d \geq 2$.

\begin{definition}
A (codimension-$1$) \dfn{laminar flow box} is a $C^0$ coordinate chart $F: I \times J \to M$ and a compact set $K \subseteq I$, such that for each $k \in K$, $F|_{\{k\} \times J}$ is a $C^1$ embedding, and the \dfn{leaf} $F(\{k\} \times J)$ is a $C^1$ complete hypersurface in $F(I \times J)$.

Two laminar flow boxes $(F_\alpha, K_\alpha)$ and $(F_\beta, K_\beta)$ belong to the same \dfn{laminar atlas} if the transition map $\psi_{\alpha \beta}$ between $F_\alpha$ and $F_\beta$ maps each leaf $\{k\} \times J$, $k \in K_\alpha$, to a leaf $\{\psi_{\alpha \beta}(k)\} \times J$, so that $\psi_{\alpha \beta}$ is a homeomorphism $K_\alpha \to K_\beta$.
\end{definition}

\begin{definition}
A \dfn{lamination} $\lambda$ consists of a nonempty closed set $S \subseteq M$, called its \dfn{support}, and a maximal laminar atlas $\{(F_\alpha, K_\alpha): \alpha \in A\}$ such that in the image $U_\alpha$ of each flow box $F_\alpha$,
$$S \cap U_\alpha = F_\alpha(K_\alpha \times J).$$
If $\lambda$ is a lamination in the image of a flow box $F$, and $N := F(\{k\} \times J)$ is a leaf of $\lambda$, we call $k$ the \dfn{label} of $N$.
A \dfn{foliation} is a lamination with support $S = M$.
\end{definition}

Summarizing the above definitions, a lamination is a nonempty closed set $S$ with a $C^0$ local product structure which locally realizes it as $K \times J$ for some compact set $K \subset \RR$.
In some sources, including \cite{Auer01}, laminations are not required to have a $C^0$ local product structure, but are only required to have disjoint leaves.

\begin{definition}
We call a lamination $C^r$ (resp. \dfn{Lipschitz}) if its flow boxes are $C^r$ (resp. Lipschitz) coordinate charts, and say that it is \dfn{tangentially $C^r$} if for each flow box $(F, K)$, $F|_{\{k\} \times J}$ is a $C^r$ embedding for $k \in K$.\footnote{Such laminations are also known as $C^r$ \dfn{along leaves} \cite{Morgan88}.}
\end{definition}

In particular, we assume that laminations are $C^0$ and tangentially $C^1$; the latter assertion implies that the flow box can push forward the normal vector to each leaf, and in particular that the mean curvature to each leaf is well-defined in a distributional sense.

In this paper we shall focus on laminations with minimal leaves\footnote{The word ``minimal'' is overloaded. In \cite{daskalopoulos2020transverse}, a \dfn{minimal lamination} is a lamination $\lambda$ in which every leaf is dense in $\supp \lambda$.
We adopt the terminology of \cite{ColdingMinicozziIV}.} and transverse measures.

\begin{definition}
A lamination $\lambda$ is \dfn{minimal} if its leaves $F_\alpha(\{k\} \times J)$ have zero mean curvature, and is \dfn{geodesic} if, in addition, $d = 2$.
\end{definition}

\begin{definition}
Let $\lambda$ be a lamination with atlas $A$.
A \dfn{transverse measure} to $\lambda$ consists of Radon measures $\mu_\alpha$ with $\supp \mu_\alpha = K_\alpha$, $\alpha \in A$, such that each transition map $\psi_{\alpha \beta}$ is measure-preserving:
$$\mu_\alpha|_{K_\alpha \cap K_\beta} = \psi_{\alpha \beta}^* (\mu_\beta|_{K_\alpha \cap K_\beta}).$$
The pair $(\lambda, \mu)$ is called a \dfn{measured lamination}.
\end{definition}

\begin{definition}
Let $(\lambda, \mu)$ be a measured oriented lamination, with atlas $A$ and a subordinate partition of unity $(\chi_\alpha)$.
The \dfn{Ruelle-Sullivan current} $T_\mu$ associated to $(\lambda, \mu)$ is defined for all compactly supported $d-1$-forms $\varphi$ by
\begin{equation}\label{RS current}
\int_M T_\mu \wedge \varphi := \sum_{\alpha \in A} \int_{K_\alpha} \left[\int_{\{k\} \times J} (F_\alpha^{-1})^* (\chi_\alpha \varphi) \right] \dif \mu_\alpha(k).
\end{equation}
\end{definition}

We assume that every transverse measure has full support, $\supp \mu_\alpha = K_\alpha$, but in \cite{daskalopoulos2020transverse}, it is only assumed that $\supp \mu_\alpha \subseteq K_\alpha$.
In particular, for our convention, not every lamination admits a transverse measure.

The Ruelle-Sullivan current was introduced by \cite{Ruelle75}, and we review its properties in \S\ref{RS prelims}.
In particular we show that $T_\mu$ makes sense (as a distributional section of a suitable line bundle) even if $\lambda$ is not orientable.

%%%%%%%%%%%%%%%%%%
\subsection{Regularity of minimal laminations}
The definitions of \S\ref{Lams sections} are tedious to work with, both because one has to prove the existence of flow boxes which flatten sets which may be extremely rough, and because one has no quantitative control on said flow boxes.
However, if we have curvature bounds on the leaves and on the underlying manifold $M$, our first main theorem drastically changes the story: it shows that the lamination $\lambda$ can be reconstructed from its set of leaves, in such a way that the flow boxes for $\lambda$ are under control in the Lipschitz and tangentially $C^\infty$ sense.
Here, \dfn{tangential $C^\infty$} is the topology defined by seminorms $f \mapsto \|\nabla_N^m f\|_{C^0}$, where $N$ ranges over leaves of the given lamination $\lambda$, $\nabla_N$ is the Levi-Civita connection on $N$, $m$ ranges over $\NN$ (including $0$); thus we ignore derivatives normal to $N$.

\begin{mainthm}\label{regularity theorem}
Let $K := \|\Riem_M\|_{C^0}$ and let $i$ be the injectivity radius of $M$, and suppose that $K < \infty$, $i > 0$.
Let $\mathcal S$ be a set of disjoint minimal hypersurfaces in $M$, such that for every $N \in \mathcal S$,
\begin{equation}\label{curvature bound in regularity}
	\|\Two_N\|_{C^0} \leq A,
\end{equation}
and that $\bigcup_{N \in \mathcal S} N$ is a closed subset of $M$. Then:
\begin{enumerate}
\item There exists a Lipschitz minimal lamination $\lambda$ whose leaves are exactly the elements of $\mathcal S$.
\item There exists a Lipschitz line bundle on $M$ which is normal to every leaf of $\lambda$.
\item There exist constants $L = L(A, K, i) > 0$ and $r = r(A, K, i) > 0$, and a Lipschitz laminar atlas $(F_\alpha)$ for $\lambda$, such that for every $\alpha$,
\begin{equation}\label{conorm of flow box}
	\max(\Lip(F_\alpha), \Lip(F_\alpha^{-1})) \leq L,
\end{equation}
and the image of $F_\alpha$ contains a ball of radius $r$.
\item $F_\alpha$ and $F_\alpha^{-1}$ are tangentially $C^\infty$, with seminorms only depending on $A, K, i$.
\end{enumerate}
\end{mainthm}

In the remainder of this paper we prove two consequences of Theorem \ref{regularity theorem}: a characterization of minimal laminations (Theorem \ref{main thm}) and a compactness theorem (Theorem \ref{compactness theorem}), which we state below.
In both theorems, a bound on the curvature will be necessary in order to invoke Theorem \ref{regularity theorem}.

\begin{definition}
A sequence $(\lambda_n)$ of laminations has \dfn{bounded curvature} if there exists $C > 0$ such that for any $n$ and any leaf $N$ of $\lambda_n$, the second fundamental form satisfies $\|\Two_N\|_{C^0} \leq C$.
\end{definition}

Several similar results to Theorem \ref{regularity theorem} have appeared in the literature already, but Theorem \ref{regularity theorem} strengthens and clarifies them.
To our knowledge, the first related result is due to Solomon \cite[Theorem 1.1]{Solomon86}, which we improve on in several ways:
\begin{enumerate}
\item \label{foliation to lamination} Solomon's proof is for minimal foliations in $\RR^d$.
\item We obtain estimates which only depend on the curvatures of the leaves and $M$, and on the injectivity radius $i$; they do not depend on the regularity of a given $C^0$ laminar atlas.
\item In fact, we do not even assume the existence of a $C^0$ laminar atlas.
\end{enumerate}
As Solomon notes, it is easy to extend his proof to minimal foliations in a Riemannian manifold $M$; the key point of (\ref{foliation to lamination}) is that we would like Theorem \ref{regularity theorem} to be true for minimal \emph{laminations}.

Our work is closest to a compactness theorem due to Colding and Minicozzi for minimal laminations of a Riemannian manifold \cite[Appendix B]{ColdingMinicozziIV}.
Their new idea is to fill in the gaps between the leaves in Solomon's constructions by linear interpolation.
However, Colding and Minicozzi assume that the laminations have finitely many leaves, and that the curvature bound (\ref{curvature bound in regularity}) implies that all of the leaves can be represented as graphs at once.
Indeed, \emph {a priori}, the leaves could fail to be close to parallel, and then it would not be possible to construct a coordinate chart in which they are all graphs.

We eliminate such assumptions by showing that members of $\mathcal S$ must be ``close to parallel on small scales'', where the scale is governed by $A, K$.
Otherwise, since the scale is small, we may replace the elements of $\mathcal S$ by their tangent spaces, which would then intersect, contradicting the disjointness of $\mathcal S$.
This approach was already suggested by Thurston \cite[\S8.5]{thurston1979geometry} in the case of geodesic laminations, though he omitted the details. 

Using completely different techniques, Daskalopoulos and Uhlenbeck \cite[Proposition 7.3]{daskalopoulos2020transverse} obtained a version of Theorem \ref{regularity theorem} without any $C^0$ dependence, under the assumption that $M$ is a closed hyperbolic surface.
The key point of their argument is that the exponential map sends lines to geodesics, so it provides a much shorter proof of Theorem \ref{regularity theorem}, at the price of only working in dimension $2$.


%%%%%%%%%%%%%%%%%%
\subsection{Applications to \texorpdfstring{$1$-harmonic}{one-harmonic} functions}\label{FLG section}
The main result of this paper realizes the Ruelle-Sullivan current of a minimal lamination as the exterior derivative of a function of locally least gradient, and vice versa.

\begin{definition}
Let $U \subseteq M$ be an open set.
A function $u \in BV(U)$ has \dfn{least gradient} in $U$ if $u$ minimizes its total variation $\int_U \star |\dif u|$ among all functions with the same trace as $u$ along $\partial U$.
A function $u \in BV_\loc(M)$ has \dfn{locally least gradient} if we can cover $M$ by open sets $U$ such that $u|_U$ has least gradient in $U$.
\end{definition}

One can think of functions of locally least gradient as weak solutions of the \dfn{$1$-Laplace equation} \cite{Mazon14}
\begin{equation}\label{1Laplacian}
	\nabla \cdot \left(\frac{\nabla u}{|\nabla u|}\right) = 0
\end{equation}
and so we also call functions of locally least gradient \dfn{$1$-harmonic functions}, though for our purposes, we shall only use the PDE (rather than the minimization problem) as a heuristic.
Observe that formally, (\ref{1Laplacian}) implies that the level sets of $u$ are minimal hypersurfaces.

The reader should carefully note that not every function of locally least gradient has least gradient, essentially because a minimal hypersurface need not be absolutely area-minimizing.
A simple example is the indicator function of the upper hemisphere of $\Sph^2$, which has locally least gradient, but does not have least gradient (for, since $\Sph^2$ is a closed manifold, the only function of least gradient on $\Sph^2$ is $0$).

\begin{mainthm}\label{main thm}
Suppose that $d \leq 7$.
\begin{enumerate}
\item Let $u$ be a function of locally least gradient on $M$.
Then:
\begin{enumerate}
\item $\overline{\bigcup_{y \in \RR} \partial \{u > y\}}$ is the support of a Lipschitz minimal lamination $\lambda$.
\item Every connected component of a level set $\partial \{u > y\}$ is a leaf of $\lambda$.
\item There exists a measured oriented structure on $\lambda$ whose Ruelle-Sullivan current is $\dif u$.
\end{enumerate}
\item Conversely, if $H^1(M, \RR) = 0$ and $\lambda$ is a minimal measured oriented minimal lamination, then:
\begin{enumerate}
\item If $\lambda$ has bounded curvature, then any primitive $u$ of the Ruelle-Sullivan current of $\lambda$ has locally least gradient.
\item If the leaves of $\lambda$ are absolutely area-minimizing, then $u$ has least gradient.
\end{enumerate}
\end{enumerate}
\end{mainthm}

We call the leaves of $\lambda$ \dfn{generalized level sets} of $u$.
The structure of the generalized level sets $N$ is studied in \S\ref{genSets}: it turns out that either $N$ is a component of $\partial \{u > y\}$ or $\partial \{u < y\}$, or $u$ attains a local extremum along $N$.
This needs to be formulated carefully, however, since a priori $u$ is not defined pointwise.

The main ingredients in the proof of Theorem \ref{main thm} are Theorem \ref{regularity theorem}, the regularity theory of minimal hypersurfaces, and curvature estimates on stable minimal hypersurfaces due to Schoen, Simon, and Yau \cite{Schoen75,Schoen81}.
With these ingredients in place, it remains to show that the stability radii of the level sets of a function of locally least gradient are bounded from below, and locally the area of the level sets is bounded from above; this gives uniform curvature estimates on the level sets.

A similar result to Theorem \ref{main thm}, proven with somewhat different methods, was announced but never published by Auer and Bangert \cite{Auer01, Auer12}, who claimed to establish that a locally minimal $d - 1$-current is Ruelle-Sullivan for a lamination in a weaker sense than ours.
In particular, it does not seem that one can extract Lipschitz regularity directly from their methods.

Our motivation for Theorem \ref{main thm} is to generalize the work of Daskalopoulos and Uhlenbeck on $\infty$-harmonic maps from a closed hyperbolic surface to $\Sph^1$ \cite{daskalopoulos2020transverse}, which associates to each such map a geodesic lamination $\lambda$ and function $v$ of locally least gradient on the universal cover such that $\dif v$ drops to a Ruelle-Sullivan current for a sublamination of $\lambda$.
Inspired by this theorem, Daskalopoulos and Uhlenbeck conjectured that for any function of locally least gradient on $\Hyp^2$, $\dif u$ should be Ruelle-Sullivan for some (possibly not maximum-stretch) geodesic lamination \cite[Problem 9.4]{daskalopoulos2020transverse}, and conversely that if $T$ is a Ruelle-Sullivan current for some geodesic lamination, then local primitives of $T$ have locally least gradient \cite[Conjecture 9.5]{daskalopoulos2020transverse}.
Of course such results are special cases of Theorem \ref{main thm}.
We shall revisit the connection between Theorem \ref{main thm} and the $\infty$-Laplacian in \cite{BackusInfinityMaxwell1}, where we explain how one can view the $1$-Laplacian as the convex dual problem to the problem of constructing a calibration of a minimal lamination, which is given by a system of ``$\infty$-elliptic'' equations.

In \S\ref{1harmonic apps} we discuss some easy consequences of Theorem \ref{main thm}: some remarks on the uniqueness theory for the Dirichlet problem for the $1$-Laplacian, and a version of the G\'orny decomposition of a function of least gradient \cite[Theorem 1.2]{górny2017planar}.

% We would also like to highlight a possible further direction of study which we shall not address here.
% The associated parabolic flow to (\ref{1Laplacian}),
% \begin{equation}\label{level set flow}
% \partial_t u = |\nabla u| \nabla \cdot \left(\frac{\nabla u}{|\nabla u|}\right),
% \end{equation}
% is known as \dfn{level set flow} and acts on the level sets of $u$ by mean curvature flow.
% As such, it arises as a model of interfaces with minimal area, as a means of continuing mean curvature flow past its singular times, and in the \dfn{level set method} of computing minimal surfaces \cite{Chen89,Thomas05}.
% Previous work on the level set flow has been especially concerned with the evolution of the hypersurface $N(t) := \{u(t) = 0\}$ under the assumption that $N(0)$ is mean convex and $u$ is a (necessarily continuous) viscosity solution of (\ref{level set flow}) \cite{Evans91,Colding2016RegularityOT,sun2022generic}.
% It would be very interesting to prove a parabolic version of Theorem \ref{main thm} which asserts that under appropriate hypotheses on a measured oriented lamination $\lambda$ with Ruelle-Sullivan current $\dif u$, the level set flow of $u$ corresponds to a flow of $\lambda$ by mean curvature flow, even if $u$ is discontinuous.

%%%%%%%%%%%%%%%%%%
\subsection{Spaces of minimal laminations}\label{LamSpace section}
In the literature, there are at least three different topologies on the space of laminations on a Riemannian manifold $M$, which we now recall.

Thurston's geometric topology \cite[Chapter 8]{thurston1979geometry} says that a lamination $\lambda'$ is close to a lamination $\lambda$ if every leaf of $\lambda$ is close to a leaf of $\lambda'$ at least locally, and the same holds for their normal vectors $\normal$.

\begin{definition}
We define the basic open sets in \dfn{Thurston's geometric topology} to be defined by a lamination $\lambda$, $x \in M$, and $\varepsilon > 0$: the basic open set $\mathscr N(\lambda, x, \varepsilon)$ is the set of all laminations $\kappa$ such that there exists $y \in \supp \kappa \cap B(x, \varepsilon)$ such that the normal vectors are close: $\dist(\normal_\lambda(x), \normal_\kappa(y)) < \varepsilon$.
\end{definition}

A sequence of laminations $(\lambda_i)$ converges to a lamination $\lambda$ in Thurston's geometric topology iff, for every leaf $N$ of $\lambda$, every $x \in N$, and every $\varepsilon > 0$, there exists $i_{\varepsilon, x} \in \NN$ such that for every $i \geq i_{\varepsilon, x}$, $\supp \lambda_i$ intersects $B(x, \varepsilon)$, and for $x_i \in B(x, \varepsilon) \cap \supp \lambda_i$,
$$\dist_{SM}(\normal_{\lambda_i}(x_i), \normal_\lambda(x)) < 2\varepsilon.$$
It is straightforward to show that Thurston's geometric topology does not depend on the choice of Riemannian metric on $M$, or the choice of extension of the distance function on $M$ to its sphere bundle $SM$, which are implicit in the statement thereof.
However, the limiting lamination is not unique, as if $\lambda_i \to \lambda$ and $\lambda'$ is a sublamination of $\lambda$, then $\lambda_i \to \lambda'$.
In particular, Thurston's topology is not Hausdorff, and we say that $\lambda$ is a \dfn{maximal limit} of a sequence $(\lambda_i)$ if $\lambda_i \to \lambda$ and for every $\lambda'$ such that $\lambda_i \to \lambda'$, $\lambda'$ is a sublamination of $\lambda$.

Independently of Thurston, Colding and Minicozzi \cite[Appendix B]{ColdingMinicozziIV} defined a sequence of laminations to converge ``if the corresponding coordinate maps converge;'' that is, if the laminar atlases converge.
This of course says nothing about the limiting set of leaves and in the sequel paper \cite{ColdingMinicozziV} they additionally impose that the sets of leaves converge ``as sets.''

In this paper we consider a similar condition to the one in \cite{ColdingMinicozziV}, which we believe to be more natural: that the laminar atlases converge and that the laminations themselves converge in Thurston's geometric topology.
To be more precise:

\begin{definition}
A sequence $(\lambda_i)$ of laminations \dfn{flow-box converges} in a function space $X$ to $\lambda$ if it converges in Thurston's geometric topology, and there exists a laminar atlas $(F_\alpha)$ for $\lambda$ such that for each $\alpha$, $F_\alpha$ and $(F_\alpha)^{-1}$ are limits in $X$ of flow boxes $F_\alpha^i$, $(F_\alpha^i)^{-1}$ in laminar atlases for $\lambda_i$.
\end{definition}

The notion of flow-box convergence is mainly useful for tangential $C^\infty$ and the Fr\'echet space $C^{1-} := \bigcap_{0 \leq \theta < 1} C^\theta$, where $C^\theta$ are H\"older spaces.

Next we recall convergence of laminations equipped with transverse measures.
We remind the reader that in \S\ref{RS prelims} we define the Ruelle-Sullivan current of a possibly nonorientable measured lamination.

\begin{definition}
A sequence of measured laminations $(\lambda_i, \mu_i)$ \dfn{converges} to $(\lambda, \mu)$ if their Ruelle-Sullivan currents $T_{\mu_i} \to T_\mu$ converge in the weak topology of measures.
\end{definition}

Filling in some of the details of the argument of Colding and Minicozzi \cite[Appendix B]{ColdingMinicozziIV}, it follows from the regularity theorem, Theorem \ref{regularity theorem}, that once we have a bound on the curvatures of the leaves, every sequence of laminations has convergent subsequences in each of the above modes of convergence.

\begin{mainthm}\label{compactness theorem}
Let $(\lambda_n)$ be a sequence of minimal laminations of bounded curvature, and let $E \subseteq M$ be a compact set. Then:
\begin{enumerate}
\item Suppose that for every $n$ and every leaf $N$ of $\lambda_n$, $N \cap E$ is nonempty. Then a subsequence of $(\lambda_n)$ converges in the $C^{1-}$ and tangentially $C^\infty$ flow box topology, and in particular in Thurston's geometric topology, to a minimal lamination.
\item Suppose that, in addition, $\mu_n$ is transverse to $\lambda_n$, there exists $\varepsilon > 0$ such that $\mu_n(E) > \varepsilon$, and there exists $C > 0$ such that $\mu_n(M) \leq C$, then a further subsequence converges in the measure topology.
\end{enumerate}
\end{mainthm}

In \S\ref{relationships between modes}, we use Theorem \ref{compactness theorem} to explain how the above modes of convergence are related.
Here is a (not quite sharp) statement of these results:

\begin{corollary}
Let $(\lambda_n, \mu_n)$ be a sequence of measured minimal laminations of bounded curvature and $(\lambda, \mu)$ a measured minimal lamination.
If $d \leq 7$ and $(\lambda_n, \mu_n) \to (\lambda, \mu)$, then $\lambda_n \to \lambda$ in the $C^{1-}$ and tangential $C^\infty$ flow box topologies, hence in Thurston's geometric topology.
\end{corollary}

%%%%%%%%%%%%%%%%%%%%%%

\subsection{Notation and conventions}
The operator $\star$ is the Hodge star, thus $\star 1$ is the Riemannian measure.
We denote the musical isomorphisms by $\sharp, \flat$.
If $U$ is an open set, we write $|U| := \int_U \star 1$ for the volume of $U$, but if $U$ is a submanifold or rectifiable set of positive codimension, we instead write $|U|$ for its surface measure.
We write $\normal_N$ for the normal vector (or conormal $1$-form) for a hypersurface $N$, $\nabla_N$ for the Levi-Civita connection, and $\Two_N := \nabla_N \normal_N$ for the second fundamental form.

We consider the following manifolds: $\Ball^d$ is the unit ball in $\RR^d$ (and $r\Ball^d$ is the ball of radius $r$), $\Sph^d$ the unit sphere in $\RR^{d + 1}$, and $\Hyp^d$ is the hyperbolic space.

For a map $F: X \to Y$ between metric spaces, we write $\Lip(F)$ for its Lipschitz constant.
If $X, Y$ are connected Riemannian manifolds, one of which is $1$-dimensional, then we have $\Lip(F) = \|\dif F\|_{L^\infty}$.

%%%%%%%%%%%%%%%%%%%%%%%
\subsection{Outline of the paper}
The rest of the paper is organized as follows:
\begin{itemize}
\item In \S\ref{Regularity}, we prove the regularity theorem, Theorem \ref{regularity theorem}.
\item In \S\ref{Prelims}, we develop basic facts about Ruelle-Sullivan currents, and $1$-harmonic functions, that we shall use throughout the remainder of the paper. This section is independent of \S\ref{Regularity}, but depends on Appendix \ref{boundary appendix}.
\item In \S\ref{1harmonic sec}, we prove the equivalence of $1$-harmonic functions and measured oriented minimal laminations, Theorem \ref{main thm}, and apply it to study $1$-harmonic functions. This section relies on \S\ref{Regularity}, \S\ref{Prelims}, and Appendices \ref{boundary appendix} and \ref{locally minimizing appendix}.
\item In \S\ref{CompactnessSec}, we prove the compactness theorem, Theorem \ref{compactness theorem}, and explore the consequences for how the different modes of convergence are related to each other. This section applies \S\ref{Regularity}, \S\ref{Prelims}, and Appendix \ref{boundary appendix} for the proof of Theorem \ref{compactness theorem}, but the consequences of it also apply \S\ref{1harmonic sec}.
\item In Appendix \ref{boundary appendix}, we recall various technical results of geometric measure theory that we shall need.
\item In Appendix \ref{locally minimizing appendix}, we give a short proof that the radius of a ball in which a minimal hypersurface is absolutely area-minimizing is controlled from below by the curvature. The proof applies both \S\ref{Regularity} and \S\ref{Prelims}.
\end{itemize}

%%%%%%%%%%%%%%%%%%%%%%%%

\subsection{Acknowledgements}
I would like to thank Georgios Daskalopoulos for suggesting this project and for many helpful discussions, Victor Bangert for providing me with the manuscript \cite{Auer12} and other helpful comments, and Stephen Obinna for suggesting the brief proof of Lemma \ref{cardinality appendix}.

An earlier draft of this paper only considered the cases $d = 2$, $d = 3$.
I would like to thank Chao Li for suggesting the reference \cite{chodosh2022complete} which extended the main result to $d \leq 4$, and William Minicozzi for suggesting the references \cite{Schoen75, Schoen81} which extended the main result to its natural hypothesis $d \leq 7$.

This research was supported by the National Science Foundation's Graduate Research Fellowship Program under Grant No. DGE-2040433.



%%%%%%%%%%%%%%%%%%%%%%%%%%%%%%%%%%%%%%%%%%
\section{Regularity of laminations}\label{Regularity}
\subsection{Elliptic estimates on leaves}\label{Leaf estimates}
Before we prove Theorem \ref{regularity theorem} we recall some well-known estimates on minimal surfaces in normal coordinates.
Let $g$ be a metric on $\RR^{d - 1}_x \times \RR_y$ satisfying the normal coordinates condition $g - I = O(K_0(|x|^2 + y^2))$, and a curvature bound $\|\Riem_g\|_{C^0} \leq K_0$, where $I$ is the identity matrix.
For a function $u \in C^1(4\Ball^{d - 1})$, let
$$Pu(x) = F(x, u(x), \nabla u(x), \nabla^2 u(x)) = 0$$
be the minimal surface equation.
Then by \cite[(7.21)]{colding2011course}, the coefficient $F$ has the form
$$F(x, y, \xi, H) = \tr H + O(K_0(|x| + |y|) + |\xi|)(1 + |H|)$$
at least for $|x| + |y| + |\xi| \lesssim 1$.
Thus for $\|u\|_{C^1(4\Ball^{d - 1})} \lesssim 1$, the minimal surface equation is uniformly elliptic, so that by Schauder estimates \cite[Theorem 6.2]{gilbarg2015elliptic}, for any $r \geq 0$,
\begin{equation}\label{norms on uk}
	\|u\|_{C^r(3\Ball^{d - 1})} \lesssim_r 1.
\end{equation}

\begin{lemma}
Suppose that $u_2 \geq u_1$ satisfy $Pu_1 = Pu_2 = 0$ on $4\Ball^{d - 1}$ and $v := u_2 - u_1$.
Then for $K_0 \ll 1$, $\|u_i\|_{C^1} \lesssim 1$, 
\begin{equation}\label{Schauder Harnack}
	\|\dif v\|_{C^0(\Ball^{d - 1})} \lesssim \sup_{2\Ball^{d - 1}} v \lesssim \inf_{\Ball^{d - 1}} v.
\end{equation}
\end{lemma}
\begin{proof}
By the proof of \cite[Theorem 7.3]{colding2011course}, there exists a linear partial differential operator $Q$ such that $Qv = 0$, and if $K_0$ is small enough and $u_1, u_2$ are bounded in $C^1$, then on $3\Ball^{d - 1}$, $Q$ is uniformly elliptic, and the coefficients are bounded in $C^1$.
The claim now follows from Schauder estimates and the Harnack inequality \cite[Corollary 9.25]{gilbarg2015elliptic}.
\end{proof}

For the remainder of the paper we fix a constant $K_0$ satisfying the hypotheses of the above lemma.

\subsection{A preliminary choice of coordinates}
We now construct normal coordinates in which the leaves of $\lambda$ are $C^1$-close to hyperplanes $\{y = y_0\}$.
The utility of this fact is that, if $f: \RR^{d - 1}_x \to \RR_y$, and its graph has normal vector $\normal$, then
\begin{equation}\label{nabla as a normal}
	\normal = \frac{\partial_y f - \nabla f}{\sqrt{1 + |\nabla f|^2}}.
\end{equation}
So if $Pf = 0$, then the leaves of $\lambda$ are minimal graphs which are small in $C^1$ and so we may apply (\ref{Schauder Harnack}) uniformly among all of the leaves at once.

A similar result was proven by \cite{Solomon86} (without the quantitative dependence) using the regularity theory for integral flat convergence of minimal currents \cite[Theorem 5.3.14]{federer2014geometric}.
We did not do this because it does not seem particularly easy to recover quantitative bounds from the highly general theory of \cite[Chapter 5]{federer2014geometric}.

\begin{lemma}\label{existence of tubes}
	Let $N$ be an embedded $C^2$ hypersurface in $\RR^d = \RR^{d - 1}_x \times \RR_y$ which is tangent to $\{y = 0\}$ at the origin.
	If $\|\Two_N\|_{C^0} \leq \frac{1}{8}$, then the connected component of $N \cap B(0, 1)$ containing $0$ is the graph over $\{y = 0\}$ of a function $f$ with
	$$|f(x)| \leq \|\Two_N\|_{C^0} |x|^2.$$
\end{lemma}
\begin{proof}
	Near $0$, $N$ can be represented a graph $\{y = f(x)\}$, since it is tangent to $\{y = 0\}$.
	This representation is valid on the component of the set $\{|\nabla f(x)| < \infty\}$ containing $0$, and it is related to the unit normal by (\ref{nabla as a normal}).
	Rearranging (\ref{nabla as a normal}) and taking derivatives,
	$$-\nabla^2 f(x) = \frac{\nabla \normal(x, f(x)) \cdot (\partial_x \otimes \partial_x + \nabla f(x) \otimes \partial_y)}{\sqrt{1 + |\nabla f(x)|^2}} - \frac{\nabla^2 f(x) \cdot (\nabla f(x) \otimes \normal(x, f(x)))}{(1 + |\nabla f|^2)^{3/2}}.$$
	Here $-\nabla^2$ denotes the negative Hessian, not the Laplacian.
	Since
	$$|\partial_x \otimes \partial_x + \nabla f(x) \otimes \partial_y| \leq \sqrt{1 + |\nabla f(x)|^2},$$
	and $\nabla \normal = \Two_N$, we conclude
\begin{equation}\label{bound Hessian by Two}
	|\nabla^2 f(x)| \leq |\Two_N(x, f(x))| + |\nabla^2 f(x)| |\nabla f(x)|.
\end{equation}
	In order to control the error terms in (\ref{bound Hessian by Two}), we make the \dfn{bootstrap assumption}
\begin{equation}\label{bootstrap}
	|\nabla f(x)| \leq \frac{1}{2},
\end{equation}
	which is at least valid in some small neighborhood $B_R$ of $0$ since (\ref{nabla as a normal}) and the fact that $N$ is tangent to $\{y = 0\}$ at $0$ imply that $\nabla f(0) = 0$.
	By (\ref{bound Hessian by Two}),
$$|\nabla^2 f(x)| \leq 2|\Two_N(x, f(x))|,$$
	and integrating this inequality one obtains for $|x| \leq R$ that
\begin{equation}\label{closed bootstrap}
	|\nabla f(x)| \leq 2|\Two_N(x, f(x))| |x| \leq \frac{1}{4}.
\end{equation}
	In particular, since $\nabla f \in C^1$, either $R \geq 1$ or there exists $R' > R$ such that the bootstrap assumption (\ref{bootstrap}) is valid on $B_{R'}$.
	Therefore (\ref{bootstrap}) is valid with $R = 1$.
	Integrating (\ref{closed bootstrap}), we obtain the desired conclusion.
\end{proof}


\begin{lemma}\label{lams have C0 fields}
	Suppose that $\delta > 0$ is small enough depending on $K$.
	Then there exists $r = r(\delta, K, i, A) > 0$ such that for every disjoint family of hypersurfaces $\mathcal S$ satisfying the curvature bound (\ref{curvature bound in regularity}) and every $p \in \bigcup_{N \in \mathcal S} N$, we can choose normal coordinates $(x, y) \in \RR^{d - 1} \times \RR$ based at $p$ so that
\begin{equation}\label{normal is basically dy}
	\sup_{N \in \mathcal S} \|\normal_\lambda - \partial_y\|_{C^0(B(p, r))} \leq \delta.
\end{equation}
\end{lemma}
\begin{proof}
Consider normal coordinates $(x, y)$ based at $p$, and write $\Two_N'$ for the second fundamental form of $N \in \mathcal S$ taken with respect to the euclidean metric from those coordinates, $\normal_N'$ the euclidean normal, $\nabla'$ the euclidean Levi-Civita connection, and $\Gamma$ the Christoffel symbols.
In particular, since $\normal_N^\flat$ is the conormal and satisfies $\normal_N^\flat = (\normal_N')^\flat/|\normal_N'|$, 
$$\Two_N' = \nabla' (\normal_N')^\flat = (\nabla - \Gamma) |\normal_N'| \normal_N^\flat = |\normal_N'| (\Two_N - |\normal_N'| \Gamma \otimes \normal_N^\flat) + \nabla' \normal_N' \otimes \normal_N^\flat.$$
Using estimates on normal coordinates we conclude that for every $0 < s < i$ and some absolute $C > 0$,
$$\|\Two_N'\|_{B(p, s)} \leq A + CKs.$$
After rescaling we may assume that $A \leq 1/16$, $K \leq 1/(32C)$, and $i \geq 2$, so $\|\Two_N'\|_{C^0(B(p, 2))} \leq 1/8$.
Then we apply Lemma \ref{existence of tubes}: for $q \in N \cap B(p, 1)$, $B(q, 1) \subseteq B(p, 2)$, and $\tilde x$ the euclidean coordinate on $T_q N$ induced by the normal coordinates $(x, y)$, $N \cap B(q, 1)$ is the graph of a function $f$ on $T_q N$ satisfying
\begin{equation}\label{living in a tube}
|f(\tilde x)| \leq A|\tilde x|^2.
\end{equation}
Here, and for the remainder of this proof, we use $|\cdot|$ to mean the euclidean metric only.

Let $0 < r < s\delta^2$ for some small absolute $s > 0$ to be chosen later, and suppose that for every choice of normal coordinates $(x, y)$ at $p$, (\ref{normal is basically dy}) fails.
Then, since every coordinate system fails to have the desired properties, we might as well choose one such that for some $N \in \mathcal S$ and some $q \in B(p, r) \cap N$, $\normal_N(q)$ is a scalar multiple of $\partial_y$.
By the contradiction assumption, we can choose $N' \in \mathcal S$ and $q \in B(p, r) \cap N'$ such that
$$|\normal_{N'}(q') - \partial_y| > \delta.$$
In particular, since $|\normal_{N'}(q')| = 1 + O(r^2)$ and $|\partial_y| = 1$, the angle $\theta$ between these two vectors is given by the law of cosines as 
$$1^2 + (1 + O(r^2))^2 - 1(1 + O(r^2)) \cos \theta = |\normal_{N'}(q') - \partial_y|^2$$
which can be neatly estimated for $s$ small enough as
$$\cos \theta < 1 - \frac{\delta^2}{2} + O(r^2) \leq 1 - \frac{\delta^2}{4}.$$
But $\theta$ is the angle between the tangent planes $T_q N$ and $T_{q'} N'$.
We consider the triangle $\Delta(q, q', r)$ where $r$ is a point of intersection of $P := T_q N$ and $P' := T_{q'} N'$, so again by the law of cosines, if $\alpha := |q - r|$ and $\beta := |q' - r|$,
$$\alpha^2 + \beta^2 - 2\alpha\beta \cos \theta = |q - q'|^2 \leq r^2.$$
By Young's inequality, it follows that 
$$r^2 \geq (\alpha^2 + \beta^2)(1 - \cos \theta) > (\alpha^2 + \beta^2) \frac{\delta^2}{4}$$
or in other words 
$$\alpha^2 + \beta^2 < \frac{4r^2}{\delta^2} < 4s^2 \delta^2$$
which means for $\delta$ small that $\max(\alpha, \beta) < 2c\delta < s/4$.
Hence $P, P'$ intersect in $B(p, s/4 + r) \subseteq B(p, s/2)$.

Now consider the tubes $\mathcal T, \mathcal T'$ of all points which are within $s^2/16$ of $P, P'$.
Since $P, P'$ intersect in $B(p, s/2)$, if $s$ is small, any graphs over $P, P'$ in $\mathcal T, \mathcal T'$ must intersect in $B(p, s)$.
In particular we can take $s < 1$ and conclude from (\ref{living in a tube}) that $N, N'$ are not disjoint, contradicting the definition of $\mathcal S$.
\end{proof}

\subsection{Proof of Theorem \ref{regularity theorem}}
Fix $\delta > 0$ to be chosen later, and $P \in M$.
By Lemma \ref{lams have C0 fields}, if $\delta \leq \delta_*$ for some $\delta_* = \delta_*(i, K) > 0$, there exists $r = r(\delta, i, K, A) > 0$ such that $B(P, r)$ admits rescaled normal coordinates $(x, y) \in 5\Ball^{d - 1} \times (-2, 2)$ in which the curvature of the rescaled metric has a $C^0$ norm $\leq K_0$ and
\begin{equation}\label{normal is almost constant}
\|\normal - \partial_y\|_{C^0(B(P, r))} \leq \delta.
\end{equation}
Moreover,
$$|\normal \cdot \partial_y| \geq 1 - |\normal - \partial_y| \geq 1 - \delta,$$
so if we select $\delta := \min(\delta_*, \frac{1}{4})$, then in $5\Ball^{d - 1} \times (-1, 1)$, then every leaf is the graph of a function, say $u_k: 5\Ball^{d - 1} \to (-2, 2)$ where $u_k(0) = k$, and
$$\|\dif u_k\|_{C^0} \leq \frac{1 - (1 - \delta)^2}{1 - \delta} \leq 1.$$
If $r$ is chosen small enough depending on $g$, then the metric $\tilde g$ induced by $g$ on $5\Ball^{d - 1} \times (-2, 2)$ satisfies $\|\Riem_{\tilde g}\|_{C^0} \leq K_0$.
Moreover, $\|u_k\|_{C^0} \leq 2$, and $u_k$ has a minimal graph, so the elliptic estimates stated in \S\ref{Leaf estimates} apply to $u_k$ uniformly in $k$.

Now let $-1 < k < \ell < 1$, and let $v_{\ell k} := u_\ell - u_k$.
By (\ref{Schauder Harnack}) with $v := v_{\ell k}$, for every $x \in \Ball^{d - 1}$,
\begin{equation}\label{bound on du}
|\dif u_\ell(x) - \dif u_k(x)| \lesssim |u_\ell(x) - u_k(x)|
\end{equation}
and it follows that
\begin{equation}\label{vertical Lipschitz}
|\normal(x, u_\ell(x)) - \normal(x, u_k(x))| \lesssim |u_\ell(x) - u_k(x)|.
\end{equation}

To extend (\ref{vertical Lipschitz}) to a Lipschitz bound on $\normal$, let $X_1, X_2 \in (\Ball^{d - 1} \times (-1, 1)) \cap \supp \lambda$.
Then there exist $x_1, x_2 \in \Ball^{d - 1}$ and $k_1, k_2 \in (-1, 1)$ such that $X_i = (x_i, u_{k_i}(x_i))$.
Setting $Y := (x_2, u_{k_1}(x_2))$,
$$|\normal(X_1) - \normal(X_2)| \leq |\normal(X_1) - \normal(Y)| + |\normal(Y) - \normal(X_2)|.$$
Then by (\ref{norms on uk}) and the mean value theorem,
$$|\normal(X_1) - \normal(Y)| \lesssim |\dif u_{k_1}(x_1) - \dif u_{k_1}(x_2)| \lesssim |X_1 - Y|.$$
Moreover, by (\ref{vertical Lipschitz}),
$$|\normal(Y) - \normal(X_2)| \lesssim |u_{k_1}(x) - u_{k_2}(x)| = |Y - X_2|.$$
Since $\delta \leq \frac{1}{4}$, by (\ref{normal is almost constant}),
$$|\sin \angle(X_1 - Y, X_2 - Y)| > 1 - O(\delta)$$
and we conclude by the Pythagorean theorem that
$$|Y - X_2|^2 + |X_1 - Y|^2 \lesssim |X_1 - X_2|^2.$$
In conclusion,
$$|\normal(X_1) - \normal(X_2)| \lesssim |X_1 - X_2|$$
which implies that $\normal$ is Lipschitz on $V \cap \supp \lambda$, where $V$ is the neighborhood of $P$ which was mapped to $\Ball^{d - 1} \times (-1, 1)$ by the cylindrical coordinates $(x, y)$.
In particular, $V$ contains a ball of the form $B(P, s)$, where $s$ only depends on $r$ (and $r$ only depends on $g$ and $A$).
Taking a Lipschitz extension of $\normal$ to $V$ we obtain the desired Lipschitz normal subbundle.

Following \cite[Appendix B]{ColdingMinicozziIV}, we construct the laminar flow box
\begin{align*}
	F: \RR^{d - 1}_\xi \times \RR_\eta &\to V \subseteq \RR^{d - 1}_x \times \RR_y \\
	(\xi, \eta) &\mapsto (\xi, f(\xi, \eta))
\end{align*}
by setting
$$f(\xi, \eta) := u_\eta(\xi)$$
if $u_\eta$ exists, and if $k < \eta < \ell$ and there does not $k < \eta' < \ell$ such that $u_{\eta'}$ exists, then
$$f(\xi, \eta) := u_k(\xi) + \frac{\eta - k}{\ell - k} v_{\ell k}(\xi)$$
is the linear interpolant of $u_k$ and $u_\ell$.

By (\ref{norms on uk}), $F$ is bounded in tangential $C^\infty$.
In particular, if $V$ is a vector field tangent to $\{\eta = k\}$, then the pushforward
$$F_* V = V^i \partial_{x^i} + (Vf) \partial_y$$
is well-defined, and pushforwards of such vector fields span the tangent bundle of the graph of $u_k$. 
The bound
\begin{equation}\label{xiLip of f}
	\|\partial_\xi f\|_{C^0} \lesssim \sup_k \|u_k\|_{C^1} \lesssim 1,
\end{equation}
a consequence of (\ref{norms on uk}), establishes that $\|F_* V\|_{C^0} \sim \|V\|_{C^0}$, and then 
$$\|(F_* V) F^{-1}\|_{C^0} \lesssim \|V(F \circ F^{-1})\|_{C^0} \leq \|V\|_{C^0} \sim \|F_* V\|_{C^0}.$$
Since $V$ was arbitrary we conclude that $F^{-1}$ is bounded in tangential $C^1$, hence in tangential $C^\infty$ by the inverse function theorem. 

It remains to show that $F$ is a Lipschitz isomorphism.
To do this, we first claim that $\Lip(f) \sim 1$.
In the $\xi$ direction, we use (\ref{xiLip of f}).
If $-1 < k < \ell < 1$, then by (\ref{bound on du}) and (\ref{Schauder Harnack}),
\begin{equation}\label{f lip}
	|f(\xi, k) - f(\xi, \ell)| \lesssim |u_k(\xi) - u_\ell(\xi)| \lesssim \ell - k.
\end{equation}
This shows that $f$ is Lipschitz in the $\eta$ direction on the leaves with constant comparable to $1$, and hence on its entire domain by linear interpolation, proving the claim.
We can then estimate using (\ref{f lip})
$$|F(\xi_1, \eta_1) - F(\xi_2, \eta_2)| \lesssim |\xi_1 - \xi_2| + \Lip(f)(|\xi_1 - \xi_2| + |\eta_1 + \eta_2|)$$
so that $\Lip(F) \lesssim 1 + \Lip(f) \lesssim 1$.

To obtain a bound on $\Lip(F^{-1})$, we observe that
\begin{equation}\label{F is coLip in xi}
|\xi_1 - \xi_2|^2
\leq |\xi_1 - \xi_2|^2 + |f(\xi_1, \eta_1) - f(\xi_2, \eta_1)|^2 
= |F(\xi_1, \eta) - F(\xi_2, \eta)|^2.
\end{equation}
By Harnack's inequality with $\eta_1 = k$ and $\eta_2 = \ell$, or $k \leq \eta_1 < \eta_2 \leq \ell$ if $\eta_1, \eta_2$ lie in the plaque between leaves $k, \ell$,
$$\frac{|f(\xi_1, \eta_1) - f(\xi_1, \eta_2)|}{|\eta_1 - \eta_2|} \gtrsim \frac{v_{\ell k}(\xi_1)}{\ell - k} \gtrsim \frac{v_{\ell k}(0)}{\ell - k} = 1$$
whence by the mean value theorem and (\ref{F is coLip in xi}),
\begin{align*}
	|\eta_1 - \eta_2| 
	&\lesssim |f(\xi_1, \eta_1) - f(\xi_1, \eta_2)| \\
	&\leq |f(\xi_1, \eta_1) - f(\xi_2, \eta_2)| + \|\partial_\xi f\|_{C^0} |\xi_1 - \xi_2| \\
	&\leq (1 + \|\partial_\xi f\|_{C^0}) |F(\xi_1, \eta_1) - F(\xi_2, \eta_2)|.
\end{align*}
By (\ref{norms on uk}) and the fact that either $\partial_\xi f = \partial_\xi u_\eta$, or there are $k,\ell$ such that $\partial_\xi f$ is the linear interpolation of $\partial_\xi u_k$ and $\partial_\xi u_\ell$, $\|\partial_\xi f\|_{C^0} \lesssim 1$.
Thus
$$|F(\xi_1, \eta_1) - F(\xi_2, \eta_2)| \gtrsim |\xi_1 - \xi_2|^2 + |\eta_1 - \eta_2|^2.$$
It follows that $\Lip(F^{-1}) \lesssim 1$, so $F$ is a Lipschitz isomorphism with constants comparable to $1$.

Finally, we compose $F$ with the change of coordinates at the start of this proof to obtain a laminar flow box in a small neighborhood of $(0, 0)$ whose image has radius $O(r)$, and whose Lipschitz constants are comparable to $O(r^{-1})$.

%%%%%%%%%%%%%%%%%%%%%%%%%%%%%%%%%%%%%%%%%

\section{Ruelle-Sullivan currents and functions of least gradient}\label{Prelims}
\subsection{Ruelle-Sullivan currents}\label{RS prelims}
Let $(\lambda, \mu)$ be a measured oriented lamination.
Then the Ruelle-Sullivan current $T_\mu$ is a well-defined closed $d-1$-current \cite[Theorem 8.2]{daskalopoulos2020transverse}. 
In particular, we may lift $T_\mu$ to the universal cover $\tilde M$, where it is exact \cite[Theorem 8.3]{daskalopoulos2020transverse}.
Moreover, $T_\mu$ has an intrinsic definition as the unique $d-1$-current with a certain polar decomposition.
To be more precise, recall that $\mu$ defines a measure on $\supp \lambda$: in each flow box $F_\alpha$, an open set $U$ has measure
\begin{equation}\label{transverse measure of an open set}
\mu(U) := \int_{K_\alpha} |F_\alpha(\{k\} \times J) \cap U| \dif \mu_\alpha(k).
\end{equation}

\begin{lemma}
For a measured oriented lamination $(\lambda, \mu)$, with Lipschitz normal vector $\normal_\lambda$, the polar decomposition of $T_\mu$ is
\begin{equation}\label{polar ruelle sullivan}
T_\mu = \normal_\lambda \mu.
\end{equation}
\end{lemma}
\begin{proof}
For an open set $U \subseteq M$ in a flow box $F_\alpha$, the total variation satisfies
$$\int_U \star |T_\mu| = \sup_{\|\varphi\|_{C^0} \leq 1} \int_{K_\alpha} \int_{\{k\} \times J} \varphi \dif \mu_\alpha(k)$$
where the supremum ranges over $d-1$-forms $\varphi$ with compact support in $U$.
However, $\star \normal_\lambda^\flat$ is the Riemannian measure on $F_\alpha(\{k\} \times J)$, so
$$\int_{\{k\} \times J} \varphi \leq \int_{\{k\} \times J} (F_\alpha^{-1})^*(\star \normal_\lambda^\flat).$$
Since $\|\normal^\lambda\|_{C^0} = 1$, it follows that a sequence of cutoffs of $\star \normal_\lambda^\flat$ to more and more of $U$ is a maximizing sequence.
Therefore $\normal_\lambda$ is the polar part of (\ref{polar ruelle sullivan}), and
$$\int_U \star |T_\mu| = \int_{K_\alpha} \int_{\{k\} \times J} (F_\alpha^{-1})^*(1_U \star \normal_\lambda^\flat) \dif \mu_\alpha(k).$$
The inner integral is the Riemannian measure of $F_\alpha(\{k\} \times J) \cap U$, so by (\ref{transverse measure of an open set}), $|T_\mu| = \mu$.
\end{proof}

The above computation motivates the definition of Ruelle-Sullivan current of a \emph{nonorientable} lamination.
To be more precise, if $\lambda$ is a nonorientable lamination with normal vector field $\normal_\lambda$, then we can view $\normal_\lambda$ as a section of a line bundle $L$ over $M$ of structure group $\ZZ/2$.
We can then define $T_\mu$ to be $\normal_\lambda \mu$, which makes sense as a distributional section of $L$, and can be tested against any continuous $d-1$-form on $M$ whose support is contained in a trivializing chart of $L$.
In particular, we shall speak of the Ruelle-Sullivan current of any measured lamination, even if it is nonorientable.

\begin{lemma}\label{convergence of normals}
If $(\lambda_n, \mu_n) \to (\lambda, \mu)$, $x_n \in \supp \lambda_n$ converges to $x \in \supp \lambda$, and $(\lambda_n), \lambda$ have continuous normal vector fields $(\normal_n), \normal$, then $\normal_n(x_n) \to \normal(x)$ pointwise.
\end{lemma}
\begin{proof}
	Choose a continuous $d-1$-form $\varphi$ which extends $\star \normal^\flat$.
	Then for every $\varepsilon > 0$,
	$$\int_{B(x, \varepsilon)} T_\mu \wedge \varphi = \mu(B(x, \varepsilon))$$
	so by Proposition \ref{portmanteau}, for almost every $\varepsilon > 0$,
	\begin{equation}\label{epsilon is a continuity set}
		\lim_{n \to \infty} \frac{\int_{B(x, \varepsilon)} T_{\mu_n} \wedge \varphi}{\mu_n(B(x, \varepsilon))} = \frac{\int_{B(x, \varepsilon)} T_\mu \wedge \varphi}{\mu(B(x, \varepsilon))} = 1.
	\end{equation}
	On the other hand, if we assume that there exist $\delta, \varepsilon > 0$ and a coordinate system such that for every $y \in \supp \lambda_n \cap B(x, \varepsilon)$,
	$$|\normal_n - \normal| \geq \delta,$$
	then possibly after shrinking $\varepsilon$ we may assume that (\ref{epsilon is a continuity set}) holds, hence by (\ref{polar ruelle sullivan}),
	$$\int_{B(x, \varepsilon)} T_{\mu_n} \wedge \varphi = \int_{B(x, \varepsilon)} \normal_n^\flat \wedge \star \normal^\flat \dif \mu_i \leq (1 - O(\delta)) \mu_n(B(x, \varepsilon))$$
	and therefore $\delta = 0$, a contradiction.
\end{proof}

%%%%%%%%%%%%%%%%%%%%%%%%%%%
\subsection{Functions of least gradient}
Here we record sundry facts about functions of least gradient that we shall need later.

\begin{theorem}\label{main thm of old paper}
Let $u \in BV(M)$ have least gradient in $M$ and $y \in \RR$. Then $1_{\{u > y\}}$ has least gradient in $M$.
In particular, if $d \leq 7$, then $\partial \{u > y\}$ is the sum of complete disjoint embedded oriented minimal hypersurfaces.
\end{theorem}
\begin{proof}
Let $v := 1_{\{u > y\}}$.
Then $v$ has least gradient \cite[Theorem 1]{BOMBIERI1969}, so by Theorem \ref{regularity}, $\{u > y\}$ is bounded by disjoint embedded minimal hypersurfaces.
These hypersurfaces inherit an orientation from the current $\dif v$.
\end{proof}

\begin{proposition}[maximum principle]
Suppose that $M$ is a compact manifold, possibly with boundary.
Let $u \in BV(M)$ be a function of least gradient.
Then 
\begin{equation}\label{least gradient maximum principle}
\esssup_M u = \esssup_{\partial M} u.
\end{equation}
\end{proposition}
\begin{proof}
Suppose that (\ref{least gradient maximum principle}) fails, and let $y := \esssup_{\partial M} u$; then $\{u > y\}$ has positive measure.
Let
$$v(x) := \begin{cases}
u(x), \text{ if } u(x) \leq y \\
y, \text{ if } u(x) > y 
\end{cases}$$
so $u|_{\partial M} = v|_{\partial M}$ and
\begin{equation}\label{least gradient maximum principle 2}
\int_M \star |\dif v| = \int_M \star |\dif u| - \int_{\partial \{u > y\}} \star_{\partial \{u > y\}} |\dif u| - \int_{\{u > y\}} \star |\dif u|.
\end{equation}
The second integral on the right-hand side of (\ref{least gradient maximum principle 2}) is given by the amount that $u$ jumps across $\partial \{u > y\}$.
If this is zero, then $u$ is continuous across $\partial \{u > y\}$, so in order for $\{u > y\}$ to have positive measure, the third integral on the right-hand side of (\ref{least gradient maximum principle 2}) must be nonzero (so that $u|_{\overline{\{u > y\}}}$ is not almost everywhere $y$).
Therefore we have 
$$\int_M \star |\dif v| < \int_M \star |\dif u|$$
which is impossible if $u$ has least gradient.
\end{proof}

\begin{proposition}[Miranda compactness]\label{MirandaStability}
  Suppose that $M$ is a compact manifold, possibly with boundary.
	If a sequence of functions $(u_n)$ (not necessarily of the same trace) is bounded in $L^1(M)$ and satisfies
\begin{equation}\label{boundedness in Miranda}
	\limsup_{n \to \infty} \int_M \star |\dif u_n| \leq \liminf_{n \to \infty} \inf_{v|_{\partial M} = 0} \int_M \star |\dif(u_n + v)| < \infty,
\end{equation}
	then there exists a function $u$ of least gradient such that along a subsequence, $u_n \to u$ in $L^1(M)$ and $\dif u_n \to \dif u$ in the weak topology of measures.
\end{proposition}
\begin{proof}
This follows from the compactness of the forgetful map $BV(M) \to L^1(M)$ and the proof of \cite[Osservazione 3]{Miranda67}.
\end{proof}


%%%%%%%%%%%%%%%%%%%
\subsection{The Hausdorff topology on closed sets}
In the sequel we shall use the Hausdorff topology on closed subsets of a topological space \cite[Chapter IV]{nadler2017continuum}:

\begin{definition}
Let $X$ be a topological space, and $(Y_n)$ a sequence of closed subsets of $X$.
\begin{enumerate}
\item The \dfn{limit inferior} $\liminf_{n \to \infty} Y_n$ is the set of all $x \in X$ such that for every open neighborhood $U \ni x$, $U \cap Y_n$ is eventually nonempty.
\item The \dfn{limit superior} $\limsup_{n \to \infty} Y_n$ is the set of all $x \in X$ such that for every open neighborhood $U \ni x$, $U \cap Y_n$ is nonempty for infinitely many $n$.
\item If $\liminf_{n \to \infty} Y_n = \limsup_{n \to \infty} Y_n$, we call that set the \dfn{limit} $\lim_{n \to \infty} Y_n$.
\end{enumerate}
\end{definition}

\begin{lemma}
Suppose that $\lambda_n \to \lambda$ in the weak topology of measures or Thurston's geometric topology.
Then 
\begin{equation}\label{supports shrink in the limit}
\supp \lambda \subseteq \liminf_{n \to \infty} \supp \lambda_n.
\end{equation}
\end{lemma}
\begin{proof}
Let $x \in \supp \lambda$.
If the convergence is in the weak topology of measures, let $\mu, \mu_n$ be the transverse measures.
Then by Proposition \ref{portmanteau}, for any $\varepsilon > 0$,
$$\liminf_{n \to \infty} \mu_n(B(x, \varepsilon)) \geq \mu(B(x, \varepsilon)) > 0$$
so $\mu_n(B(x, \varepsilon)) \gtrsim 1$.
So for any $\varepsilon$ we can find $n$ and $x_n \in \supp \lambda_n \cap B(x, \varepsilon)$.
If instead the convergence is in the Thurston topology, we pass to a subsequence which realizes the limit inferior in (\ref{supports shrink in the limit}).
Then by definition of a basic open set, for every $\varepsilon > 0$ we can find $n$ and $x_n$ such that $x_n \in \supp \lambda_n \cap B(x, \varepsilon)$.
Either way, we conclude (\ref{supports shrink in the limit}).
\end{proof}

Given a connected oriented submanifold $N$, we let $[N]$ denote the integral current defined by integration along $N$, or equivalently the Ruelle-Sullivan current of the lamination whose only leaf is $N$.

\begin{lemma}\label{C1 close implies measure close}
Let $N_1, N_2$ be connected oriented submanifolds which are $C^1$-close.
Then $[N_1], [N_2]$ are close in the weak topology of measures.
\end{lemma}
\begin{proof}
Since $N_1, N_2$ are $C^1$, they admit triangulations, so we can reduce to the case that $N_i = (F_i)_* \sigma$, where $\sigma$ is the standard simplex and $F_i$ are $C^1$ maps.
Then the pullback maps $F_1^*$ and $F_2^*$ are close in $C^0$, and for any form $\psi$,
$$\int_M [N_i] \wedge \psi = \int_\sigma F^*_i \psi.$$
Therefore $[N_1], [N_2]$ are close in the weak topology of measures.
\end{proof}

\begin{lemma}\label{measured convergence is smooth convergence}
Let $C > 0$, let $(N_n)$ be a sequence of minimal hypersurfaces with $\|\Two_{N_n}\|_{C^0} \leq C$, and let $N$ be a $C^1$ hypersurface.
If $[N_n] \to [N]$ in the weak topology of measures, then $N$ is a minimal hypersurface.
\end{lemma}
\begin{proof}
Let $p \in N = \supp [N]$; by (\ref{supports shrink in the limit}), there exist $p_n \in N_n$ with $p_n \to p$.
Then by Lemma \ref{convergence of normals}, $\normal_{N_n}(p_n) \to \normal_N(p)$.
By assumption, $\|\nabla \normal_{N_n}\|_{C^0} \leq C$; moreover, we can choose normal coordinates $(x, y) \in \RR^{d - 1} \times \RR$ at $p$ with $\partial_y|_p = \normal_N(p)$.
So in a neighborhood of $p$, for every $n$ large, $\normal_{N_n}$ is $C^0$ close to $\partial_y$.
In particular, $N_n$ are the graphs of functions $u_n: \RR^{d - 1}_x \to \RR_y$ which are bounded in $C^1$ and solve $Pu_n = 0$.
By (\ref{norms on uk}), $(u_n)$ is precompact in $C^\infty$.
Similarly, $N$ is the graph of some $u$, and along a subsequence $u_{n_k} \to \tilde u$ in $C^\infty$ for some $\tilde u$, which then has a graph $\tilde N$.
By Lemma \ref{C1 close implies measure close}, $[N_{n_k}] \to [\tilde N]$ in the weak topology of measures; it follows that $\tilde N = N$, so $\tilde u = u$.
Therefore $u_{n_k} \to u$ in $C^\infty$, hence $Pu = 0$.
\end{proof}

%%%%%%%%%%%%%%%%%%%%%%%

\section{Application to \texorpdfstring{$1$-harmonic}{one-harmonic} functions}\label{1harmonic sec}
The purpose of this section is to prove Theorem \ref{main thm}, and explore some of its consequences.
Throughout, we shall assume that the dimension of $M$ is $2 \leq d \leq 7$.

%%%%%%%%%%%%%
\subsection{Stability of \texorpdfstring{$1$-harmonic functions}{one-harmonic functions}}
Here we collect some results about the stability of $1$-harmonic functions that will be useful in the proof of Theorem \ref{main thm}.

\begin{lemma}
There exists a continuous function $R: M \to \RR_+$ such that the following holds.
Let $p \in M$, $y \in \RR$, $0 < r \leq R(p)$, $u$ a function of least gradient on $B(p, r)$, and $N := \partial \{u > y\} \cap B(p, r)$.
Then 
\begin{equation}\label{least gradient area bound}
|N| \leq \frac{4\pi^{d/2}}{\Gamma(d/2)} r^{d - 1}.
\end{equation}
\end{lemma}
\begin{proof}
By Theorem \ref{main thm of old paper}, $v := 1_{\{u > y\}}|_{B(p, r)}$ is a function of least gradient, and the components of $\supp \dif v$ are minimal hypersurfaces whose surface area in $B(p, r)$ sums to $\int_{B(p, r)} \star |\dif v|$, so it suffices to estimate the total variation of a function of least gradient.
Reasoning identically to \cite[Lemma 5.6]{Giusti77}, we have 
$$\int_{B(p, r)} \star |\dif v| \leq \|v\|_{L^1(\partial B(p, r))} \leq |\partial B(p, r)|$$
where the trace of $v$ is defined along $\partial B(p, r)$ and has $L^\infty$ norm $\leq 1$ by \cite[Theorem 2.10]{Giusti77}.
We can estimate the surface area of $\partial B(p, r)$ as $r^{d - 1}$ times twice the volume of $\Sph^{d - 1}$, at least if $r$ is small enough depending on the curvature of $M$ near $p$.
\end{proof}

\begin{lemma}\label{choose balls for main thm}
There exist continuous functions $R, K: M \to \RR_+$ such that the following holds.
Let $p \in M$, $0 < r \leq R(p)$, $u$ a function of least gradient on $B(p, 2r)$, and $N = \partial \{u > y\}$ a level set of $u$.
Then
\begin{equation}\label{least gradient curvature bound}
\|\Two_N\|_{B(p, r)} \leq \frac{K(p)}{r}.
\end{equation}
\end{lemma}
\begin{proof}
Let $q \in N \cap B(p, r)$ (so $B(q, r) \subseteq B(p, 2r)$).
By (\ref{least gradient area bound}), $|N \cap B(p, r)| \lesssim r^{d - 1}$.
Since $N$ bounds a set $\{u > y\}$ such that $v := 1_{\{u > y\}}|_{B(p, 2r)}$ has least gradient (by Theorem \ref{main thm of old paper}), the components of $N$ are stable.
(Indeed, if $(N_t)$ is a normal variation of $N$ with compact support in $B(p, 2r)$, then for $t$ small, $N_t$ bounds an open set whose indicator function $v_t$ is a competitor to $v$, so $\int_{B(p, 2r)} \star |\dif v_t| \geq \int_{B(p, 2r)} \star |\dif v|$.)
So by \cite[pg785, Corollary 1]{Schoen81}\footnote{See also \cite[Theorem 3]{Schoen75} for an easier proof when $M$ has nonpositive curvature and dimension $d \leq 6$, or \cite[Chapter 2, \S\S4-5]{colding2011course} for a textbook treatment of a similar estimate. By \cite[Lemma 2.4]{chodosh2022complete}, we may remove the dependence on the volume bound if $d \leq 4$.}
\begin{align*}
|\Two_N(q)| &\leq \|\Two_N\|_{C^0(B(q, r/2))} \lesssim_{d, \|\Riem_g\|_{C^0(B(p, 2r))}} \frac{1}{r}. \qedhere 
\end{align*}
\end{proof}

%%%%%%%%%%%%%%
\subsection{Proof of Theorem \texorpdfstring{\ref{main thm}}{C}}
\subsubsection{\texorpdfstring{$1$-harmonic}{One-harmonic} function induces minimal lamination}
Let $u$ be a function of locally least gradient.
By Theorem \ref{main thm of old paper}, the level sets of $u$ are complete embedded minimal hypersurfaces in $M$.
The theorem is local, so we may replace $M$ by the balls $B(p, R(p))$ defined by Lemma \ref{choose balls for main thm}. 

Let $y, z \in \RR$. If $y > z$, then $\{u > y\} \subseteq \{u > z\}$, so $\partial \{u > y\}$ lies on one side of $\partial \{u > z\}$.
By the maximum principle for minimal hypersurfaces, it follows that either $\partial \{u > y\}$ and $\partial \{u > z\}$ are disjoint, or are equal.
If we shrank $M$ enough, then by (\ref{least gradient curvature bound}),
\begin{equation}\label{curvature bound on level sets}
\sup_{y \in \RR} \|\Two_{\partial \{u > y\}}\|_{C^0} < \infty.
\end{equation}

Let us now partition $\supp \dif u$ into leaves. 
By (\ref{level sets define support}), $S := \bigcup_{y \in \RR} \partial \{u > y\}$ is dense in $\overline S = \supp \dif u$.
To describe the rest of $\supp \dif u$, suppose that we have $x \in \overline S \setminus S$, and we choose $x_n \in \partial \{u > y_n\}$ converging to $x$.
Then $(y_n)$ must be a bounded sequence, since by the maximum principle (\ref{least gradient maximum principle}) $u$ is bounded on $B(p, R(p))$, so after passing to a subsequence, we may assume that $(y_n)$ converges monotonically to some $y \in \RR$.

After passing to a smaller ball and rescaling, we may use (\ref{curvature bound on level sets}) to assume that the the minimal hypersurfaces $N_n := \partial \{u > y_n\}$ all have small curvature in $C^0$.
Thus by Lemmata \ref{existence of tubes} and \ref{lams have C0 fields}, each minimal hypersurface $N_n$ can locally be viewed as a graph of a function $f_n$ which is small in $C^2$, in normal coordinates centered on $x$.
By (\ref{norms on uk}), $\|f_n\|_{C^3} \lesssim 1$ and so along a subsequence, $f_n \to f$ in $C^2$, for some $f$ whose graph is a minimal hypersurface $N \ni x$ satisfying the same curvature bound.
Moreover, since $(y_n)$ is a monotone sequence and the $N_n$ are disjoint, $(f_n)$ is also a monotone sequence.
Therefore, after taking a reflection of the coordinate system if necessary, we may assume that $(f_n)$ is an increasing sequence.

Thus $f \geq f_n$ for every $n$, and $f_n(0) < f(0) = 0$, since $x = (0, 0)$ is contained in $N$ but not any of the $N_n$.
So by the maximum principle for minimal hypersurfaces, $N$ is disjoint from all the $N_n$.
This argument works for any sequence $(x_n)$ which approximates $x$ with $(u(x_n))$ monotone, so $N$ is disjoint from $\bigcup_{y \in \RR} \partial \{u > y\}$.

\begin{definition}
We call a hypersurface $N$ a \dfn{level set} of $u$ if it is a component of $\partial \{u > y\}$.
We say that $N$ is a \dfn{generalized level set} if it is a level set, or a $C^2$ limit of level sets as above.
\end{definition}

\begin{lemma}
Let $N, N'$ be generalized level sets of $u$.
Then $N = N'$, or $N \cap N'$ is empty.
\end{lemma}
\begin{proof}
As above, we assume that $N, N'$ are not equal, and are graphs of functions $f, f'$, which are approximated in $C^2$ by sequences $f_n, f_n'$ whose graphs are level sets.
Now suppose that $N \cap N'$ is nonempty, hence for some $x$ we have $f(x) = f'(x)$.
By the maximum principle for minimal hypersurfaces and the fact that $f, f'$ are distinct, $x$ is a saddle point of $f - f'$, so there exist $x_+, x_-$ close to $x$ with $f(x_+) > f'(x_+)$ and $f(x_-) < f'(x_-)$.
Therefore for $n$ large enough, $f_n(x_+) > f_n'(x_+)$ and $f_n(x_-) < f'_n(x_-)$, so by the intermediate value theorem there exist $x_n$ with $f_n(x_n) = f_n(x_n')$.
But the level sets of $u$ are disjoint, so this is a contradiction.
\end{proof}

Summing up, we have partitioned the closed set $\supp \dif u$ into generalized level sets, which satisfy a curvature bound (\ref{curvature bound on level sets}).
So by Theorem \ref{regularity theorem}, the generalized level sets are the leaves of a lamination $\lambda$, which by (\ref{level sets define support}) has support equal to $\supp \dif u$.
Since the level sets are minimal, so is $\lambda$; since $\dif u$ is conormal to, and orients, every level set, $\dif u$ is conormal to, and orients, $\lambda$.
We can locally approximate a generalized level set by a level set in $C^2$, hence in the Hausdorff topology.

We now construct the transverse measure to $\lambda$.
In any oriented laminar coordinates $(k, x) \in K \times J$ for $\lambda$, $\partial_x u = 0$, so $\star |\dif u|$ defines a measure $\mu$ on $K$: given $\alpha < \beta$, let
$$\mu([\alpha, \beta] \cap K) := u(\beta, x) - u(\alpha, x)$$
for any (and hence every, since $\partial_x u = 0$) $x \in J$.
Since $(k, x)$ are oriented laminar coordinates, $u(\cdot, x)$ is nondecreasing, so $\mu$ is a positive measure.

If $(k', x') \in K' \times J$ is a different laminar coordinate system, and the transition map carries $\alpha, \beta$ to $\alpha', \beta'$, then
$$\mu'([\alpha', \beta'] \cap K') := u'(\beta', x') - u(\alpha', x') = u(\beta, x_1) - u(\alpha, x_2)$$
for some $x_1, x_2 \in J$. Since $\partial_x u = 0$,
$$u(\beta, x_1) - u(\alpha, x_2) = u(\beta, x_1) - u(\alpha, x_1) = \mu([\alpha, \beta] \cap K).$$
It follows that $\mu$ is transverse, and by construction $\mu$ lifts to $\star |\dif u|$ in $M$.
So by (\ref{polar ruelle sullivan}), $\dif u = \normal_\lambda |\dif u|$ is the Ruelle-Sullivan current for the measured oriented structure we just imposed on $\lambda$.

%%%%%%%%%%%%%%%%

\subsubsection{Minimal lamination induces \texorpdfstring{$1$-harmonic}{one-harmonic} function}
Suppose that $H^1(M, \RR) = 0$ and we are given a measured oriented minimal lamination $\lambda$, which then has a Ruelle-Sullivan current $\dif u$.
If $u$ does not have locally least gradient, then there exists $p \in M$ such that for any open set $E \ni p$, $u|_E$ does not have least gradient.
Since $\supp \dif u = \supp \lambda$, we may assume that $p \in \supp \lambda$ (for if $p \notin \supp \dif u$, then $u$ is actually constant in a neighborhood of $p$, hence trivially has least gradient there).

By Proposition \ref{minimal implies locally minimizing}, if $\lambda$ has bounded curvature, then there exists $\delta > 0$ depending on $p$, $g$, and the curvature of $\lambda$, such that for any $q \in B(p, \delta)$ and any leaf $N$ of $\lambda$, $N \cap B(q, 2\delta)$ is absolutely area-minimizing in $B(q, 2\delta)$.
In particular, $N \cap B(p, \delta)$ is absolutely area-minimizing in $B(p, \delta) \subseteq B(q, 2\delta)$, and let $E := B(p, \delta)$.
On the other hand, if the leaves of $\lambda$ are absolutely area-minimizing, then we can take $E := M$.

By construction, $u|_E$ does not have least gradient, so there exists a perturbation $v$ which satisfies
\begin{equation}\label{not least gradient compact support}
\int_M \star |\dif u| > \int_M \star |\dif u + \dif v|
\end{equation}
and has compact support in the interior $M$ \cite[Theorem 2.2]{Sternberg93}.
In particular, there exists a collar neighborhood $F$ of the boundary such that $v|_F = 0$.
Then for each $y \in \RR$,
$$\partial \{u > y\} \cap F = \partial^* \{u > y\} \cap F = \partial^* \{u + v > y\} \cap F,$$
so that $1_{\{u + v > y\}} - 1_{\{u > y\}}$ has compact support in $E$.
Since $\partial \{u > y\}$ is absolutely area-minimizing in $E$, $1_{\{u > y\}}$ has least gradient in $E$.
So we may estimate using the coarea formula (see \cite[Proposition 2.5]{BackusFLG} for a proof at this regularity)
\begin{align*}
\int_E \star |\dif u| &= \int_{-\infty}^\infty \int_E \star |\dif 1_{\{u > y\}}| \dif y \leq \int_{-\infty}^\infty \int_E \star |\dif 1_{\{u + v > y\}}| \dif y = \int_E \star |\dif u + \dif v|
\end{align*}
which is a contradiction of (\ref{not least gradient compact support}).
This completes the proof of Theorem \ref{main thm}.


%%%%%%%%%%%%%%%%%%%%%%%%%%%%
\subsection{Conclusions for \texorpdfstring{$1$-harmonic}{one-harmonic} functions}\label{1harmonic apps}
Let us explore some consequences of the lamination perspective on $1$-harmonic functions.

\subsubsection{Not every leaf is a level set}
We briefly give an example of a function of locally least gradient which has a leaf which is not the boundary of a superlevel set.
Let $u(x, y) := |x|$ on $\Ball^2$; then $u$ has locally least gradient, since its level sets are lines.
Moreover, $u$ attains a minimum on $\{x = 0\}$, which is a generalized level set of $u$.
But $\{x = 0\}$ is not the boundary of any superlevel set $\{u > z\}$, since for $z > 0$, $\partial \{u > z\} = \{|x| = z\}$, and for $z \leq 0$, $\{u > z\} = \Ball^2$ has no boundary in $\Ball^2$.

%%%%%%%%%%%%%%%%%%%

\subsubsection{The Dirichlet problem}
Let $\overline M$ be a compact Riemannian manifold with nonempty Lipschitz boundary $\partial M$, and interior $M$.
A consequence of our formulation of the $1$-Laplacian, and the nonuniqueness and nonstability of minimal hypersurfaces, is that there are \emph{two} reasonable formulations for the Dirichlet problem for the $1$-Laplacian on $M$:
\begin{enumerate}
\item The Dirichlet problem for functions of least gradient. This needs to be formulated carefully for discontinuous boundary data because otherwise solutions will not exist \cite{spradlin2013traces}, but the relaxation formulation of \cite{Mazon14} suffices for boundary data in $L^1$.
\item The Dirichlet problem for functions of locally least gradient. Given a function $h \in L^1(\partial M)$, find a function $u \in BV(M)$, so that we can cover $\overline M$ by open sets $U_\alpha$ equipped with data $h_\alpha \in L^1(\partial U_\alpha)$, such that $u$ solves the Dirichlet problem for functions of least gradient on $U_\alpha$ with data $h_\alpha$, and $h_\alpha|_{\partial M} = h$.
\end{enumerate}
Under certain convexity hypotheses on $\partial M$, the Dirichlet problem for functions of least gradient with $C^0$ data has a unique solution \cite{ZiemerWilliamsSternberg1992}.
However, the Dirichlet problem for functions of least gradient is highly nonlocal, unlike the Dirichlet problem for solutions of a PDE; moreover, restricting to functions of least gradient causes Theorem \ref{main thm} to fail.

For discontinuous data, the Dirichlet problem for functions of least gradient does not have a unique solution.
When combined with Theorem \ref{main thm}, \cite[Remark 2.8]{Mazon14} could be interpreted as the following conjecture:

\begin{conjecture}
The (unmeasured) lamination associated to a solution of the Dirichlet problem for functions of least gradient is determined by the data.
\end{conjecture}

This conjecture is morally equivalent to uniqueness for the Plateau problem for absolutely minimizing oriented minimal hypersurfaces.
We are not aware of uniqueness results concerning the absolutely minimizing Plateau problem, so we will not address this conjecture here.

In contrast, the Dirichlet problem for functions of locally least gradient with boundary data fails to have a unique unmeasured lamination except in the most trivial situations (such as $M = \Ball^2$), because this is is morally equivalent to uniqueness for the Plateau problem (for possibly unstable minimal hypersurfaces).\footnote{For example, on $\Sph^2 \subset \RR^3_{xyz}$, take $h(x, y, z) := 1_{|z| > 0.5}$. Then the absolutely minimizing lamination for the Dirichlet problem on $\Ball^3$ consists of two disks; there is another minimal lamination with the same Plateau data, which consists of a single catenoid.}

\subsubsection{The G\'orny decomposition}
We now consider an analogue of the G\'orny decomposition \cite[Theorem 1.2]{górny2017planar} of a function of least gradient.
Recall that a \dfn{Cantor function} is a continuous function whose exterior derivative is mutually singular with Lebesgue measure.
A \dfn{jump function} is a function $u: M \to \RR$ such that $\dif u = \sum_i a_i \delta_{N_i}$, where $(N_i)$ is a (possibly finite) sequence of $C^0$ hypersurfaces, $a_i \in \RR$, and $\delta_{N_i}$ is the Dirac measure for $N_i$, defined by
$$\int_M f \dif \delta_{N_i} = \int_{N_i} f \dif S.$$
Thus $u$ jumps across each $N_i$ by $a_i$.
In general, it is not possible to decompose a function $u$ of bounded variation into an absolutely continuous (that is, $W^{1, 1}_\loc$) part, a Cantor part, and a jump part \cite[Example 4.1]{Ambrosio2000FunctionsOB}.
G\'orny showed that for a function of least gradient on euclidean space, such a decomposition exists.
We give a new proof using Theorem \ref{main thm} which applies on curved domains.

\begin{proposition}
Let $u$ be a function of locally least gradient, and suppose that $H^1(M, \RR) = 0$. Then there exists a decomposition of $u$ into functions of locally least gradient
$$u = u_{ac} + u_C + u_j,$$
with mutually singular exterior derivatives, such that $u_{ac} \in W^{1, 1}_\loc(M)$, $u_C$ is a Cantor function, and $u_j$ is a jump function.
Up to addition of additive constants, this decomposition is unique.
\end{proposition}
\begin{proof}
By considering the minimal lamination associated to $u$ by Theorem \ref{main thm}, we obtain a laminar atlas $(F_\alpha)$.
We then prove the decomposition in a single Lipschitz flow box $F_\alpha: U_\alpha \to V_\alpha$, where
$$U_\alpha = I \times J \subset \RR_k \times \RR^{d - 1}_x.$$
Possibly after refining the open cover $(V_\alpha)$, we may assume that $V_\alpha$ is contained in a ball so small that $u|_{V_\alpha}$ has least gradient.

In the flow box coordinates, $u(k, x) = \tilde u^\alpha(k)$ for some $\tilde u_\alpha: I \to \RR$. 
Since $\dif \tilde u^\alpha$ is the transverse measure, it is a Radon measure, so $\tilde u^\alpha$ has bounded variation and hence we have the Lebesgue decomposition on an interval \cite[Corollary 3.33]{Ambrosio2000FunctionsOB} 
$$\tilde u^\alpha = \tilde u^\alpha_{ac} + \tilde u^\alpha_C + \tilde u^\alpha_j$$
where $u^\alpha_{ac} \in W^{1, 1}_\loc(I)$, $u^\alpha_C$ is a Cantor function, and $u_j^\alpha$ is a jump function.
This decomposition is unique up to an addition of constants, and induces a decomposition of $\dif \tilde u^\alpha$ into mutually singular measures.
We then write $u^\alpha_\sigma(k, x) = \tilde u^\alpha_\sigma(k)$ to obtain a function on $V_\alpha \subseteq M$, where $\sigma \in \{ac, C, j\}$.
The functions $u^\alpha_{ac}, u^\alpha_C, u^\alpha_j$ are $W^{1,1}_\loc$, Cantor, and jump respectively, since Lipschitz isomorphisms preserve these conditions.

We next claim that $\dif u^\alpha_{ac}, \dif u^\alpha_C, \dif u^\alpha_j$ are mutually singular.
Since $\dif \tilde u^\alpha_{ac}, \dif \tilde u^\alpha_C, \dif \tilde u^\alpha_j$ are mutually singular, we have a decomposition
$$I = I_{ac} \sqcup I_C \sqcup I_j$$
such that for $\sigma \neq \tau$, $I_\tau$ is a $\dif \tilde u^\alpha_\sigma$-null set.
Applying Fubini's theorem, the same decomposition holds for $\dif u^\alpha_\sigma$ and $I \times J \cong V_\alpha$, implying mutual singularity of the $\dif u^\alpha_\sigma$.

We now claim that $u^\alpha_\sigma$ have least gradient on $V_\alpha$.
To ease notation we do this for $\sigma = j$; the other cases are similar.
If the claim fails, then there is some $v \in BV_\cpt(V_\alpha)$ such that $\int \star |\dif u^\alpha_j| > \int \star |\dif (u^\alpha_j + v)|$.
But if so, then by mutual singularity and the fact that $u$ has least gradient on $V_\alpha$,
\begin{align*}
\int_{V_\alpha} \star |\dif u| &= \int_{V_\alpha} \star (|\dif u^\alpha_j| + |\dif u^\alpha_C| + |\dif u^\alpha_{ac}|) \\
&>  \int_{V_\alpha} \star (|\dif u^\alpha_j + \dif v| + |\dif u^\alpha_C| + |\dif u^\alpha_{ac}|) \\
&\geq \int_{V_\alpha} \star |\dif u + \dif v| \geq \int_{V_\alpha} \star |\dif u|,
\end{align*}
a contradiction.

Finally we glue the local decompositions together.
The measure-preserving property of transition maps and the uniqueness of the Lebesgue decomposition implies that
$$\dif u^\alpha_\sigma|_{V_\alpha \cap V_\beta} = \dif u^\beta_\sigma|_{V_\alpha \cap V_\beta}.$$
As closed currents form a sheaf, it follows that there exist unique closed currents $\dif u_\sigma$ on all of $M$ such that $\dif u_\sigma|_{V_\alpha} = \dif u^\alpha_\sigma$.
Since $H^1(M, \RR) = 0$, $\dif u_\sigma$ has an antiderivative $u_\sigma$, which has locally least gradient, since $u_\sigma|_{V_\alpha} = u_\sigma^\alpha$ has least gradient.
\end{proof}

%%%%%%%%%%%%%%%%%%%%%%%%

\subsection{Structure of the generalized level sets}\label{genSets}
Suppose that $u$ has locally least gradient, and let $\lambda$ be the associated lamination.
A byproduct of the proof of Theorem \ref{main thm} is that the leaves of $\lambda$ are generalized level sets of $u$, thus they are $C^2$ limits of components of level sets $\partial \{u > y\}$.
We show that, away from the local extrema of $u$, these are the only generalized level sets.

To be more precise, introduce the \dfn{lower semicontinuous envelope}
$$u_{\rm lsc}(x) := \sup \{f(x): \text{$f$ is a lower semicontinuous function and $f \geq u$ almost everywhere}\}.$$
Since the supremum of lower semicontinuous functions is lower semicontinuous, so is $u_{\rm lsc}$.
By the G\'orny decomposition, $u = u_{\rm lsc}$ almost everywhere away from the hypersurfaces along which $u$ jumps, and if $u$ jumps along a hypersurface $N$, say from $y_-$ to $y_+$, where $y_- < y_+$, then 
$$u_{\rm lsc}|_N = y_-.$$
The set on which a lower semicontinuous function attains its minimum is a well-defined closed set, so we say that $u$ has a \dfn{local minimum} at $x \in M$ if $u_{\rm lsc}$ has a local minimum at $x$.
We dually define the \dfn{upper semicontinuous envelope} $u_{\rm usc}$ and \dfn{local maxima} of $u$.

Let $N$ be a generalized level set of $u$.
The restriction of $N$ to a flow box takes the form $\{k = k_*\}$ for some $k_*$ in the local leaf space $K$.
By writing $\tilde u_{\rm lsc}(k) = u_{\rm lsc}(k, z)$ and $\tilde u_{\rm usc}(k) = u_{\rm usc}(k, z)$, for $k \in K$, we see that $u_{\rm usc}|_N, u_{\rm lsc}|_N$ are constant, say $u_{\rm usc}|_N = y_+$ and $u_{\rm lsc}|_N = y_-$.
In particular $y_+ \geq y_-$, with strict inequality iff $u$ jumps along $N$.

\begin{proposition}
Let $u$ be a function of locally least gradient, and let $N \subset M$ be a generalized level set of $u$.
Then either there exists $y \in \RR$ such that $N$ is a component of $\partial \{u > y\}$ or $\partial \{u < y\}$, or $u$ has a local extremum along $N$.
\end{proposition}
\begin{proof}
Suppose that $N_n \subseteq \partial \{u > y_n\}$ converge (in $C^2$, hence in the Hausdorff sense) to $N$.
By passing to a flow box and suppressing the $d - 1$ directions on which $u$ is constant, we may replace $u, u_{\rm usc}, u_{\rm lsc}$ by functions $\tilde u, \tilde u_{\rm usc}, \tilde u_{\rm lsc}$ on $\RR$, and $N_n$ by a point $k_n \in [0, 1]$, such $\tilde u_{\rm usc}, \tilde u_{\rm lsc}$ are semicontinuous envelopes of $\tilde u$.
After taking a subsequence we can take $k_n \to k_*$ to see that $N$ corresponds to the point $k_*$.

Suppose that $N$ is not a component of $\partial \{u > y\}$ or $\partial \{u < y\}$ for any $y$, so $k_*$ is not a point of $\partial \{\tilde u > y\}$ or $\partial \{\tilde u < y\}$.
In particular if we take $y_-, y_+$ to be the values of $\tilde u_{\rm lsc}, \tilde u_{\rm usc}$ at $k_*$, then $k_*$ is not a point of $\partial \{\tilde u > y_-\}$ or $\partial \{\tilde u < y_+\}$.

Suppose that $k_*$ is an interior point of $\{\tilde u > y_-\}$.
Then $k_*$ is a local minimum, for on a punctured neighborhood $0 < |k - k_*| < \delta$ of $k_*$,
$$\tilde u_{\rm lsc}(k) > y_- = \tilde u_{\rm lsc}(k_*).$$
Similarly, if $k_*$ is an interior point of $\{\tilde u < y_+\}$, then $k_*$ is a local maximum.

The alternative is that $k_*$ is interior to $\{\tilde u \leq y_-\} \cap \{\tilde u \geq y_+\}$.
Therefore $y := y_- = y_+$, and $\tilde u$ is identically $y$ in a neighborhood of $k_*$.
This is impossible, since $k_n \to k_*$ and $\tilde u$ is not constant along the sequence of $k_n$.
\end{proof}

%%%%%%%%%%%%%%%%%%%%%%%%%%%%%%%%%%%%%%%%%
\section{Compactness of the space of laminations}\label{CompactnessSec}
In this section we prove Theorem \ref{compactness theorem}, the compactness theorem.
We then apply it to explore the implications between the different modes of convergence.

%%%%%%%%%%%%%%%%%%

\subsection{Proof of Theorem \texorpdfstring{\ref{compactness theorem}}{B}}
\subsubsection{Construction of the limiting flow box}
Let $P \in M$, and let $(\lambda_n)$ be a sequence of minimal laminations of bounded curvature, such that every leaf of every lamination meets a compact set.

By Theorem \ref{regularity theorem}, there exist $r > 0$ and $L \geq 1$ such that for every large $n \in \NN$, $B(P, r)$ is contained in the image of a flow box $F_n$ for $\lambda_n$ with Lipschitz constant $L$, such that $F_n(0, 0) = P$.
By the Arzela-Ascoli theorem, along a subsequence $F_n \to F$ in $C^0$ for some map $F: I \times J \to B(P, r)$ and some $I \subseteq \RR$, $J \subseteq \RR^{d - 1}$, such that on the image $V$ of $F$, we also have the convergence $F_n^{-1} \to F^{-1}$.
Moreover, $F(0, 0) = P$, so that $F: I \times J \to V$ is a homeomorphism onto a set which contains $P$.
Since
$$\max(\Lip(F), \Lip(F^{-1})) \leq \limsup_{n \to \infty} \max(\Lip(F_n), \Lip(F_n^{-1})) \leq L,$$
it follows that $\max(\Lip(F), \Lip(F^{-1})) \leq L$, and for any $\theta \in (0, 1)$,
\begin{align*}
	\|F - F_n\|_{C^\theta}
	&\leq \Lip(F - F_n)^\theta \|F - F_n\|_{C^0}^{1 - \theta} \leq (2L)^\theta \|F - F_n\|_{C^0}^{1 - \theta}.
\end{align*}
It follows that $F_n \to F$ in $C^\theta$, hence in $C^{1-}$, and similarly for $F^{-1}$.
Since $(F_n)$ and $(F_n^{-1})$ are bounded in tangential $C^\infty$, a similar compactness argument to the above shows that $F_n \to F$ and $F_n^{-1} \to F^{-1}$ in tangential $C^\infty$.

Since $P$ was arbitrary, it follows that we can find laminar atlases $(F_\alpha^n, K_\alpha^n)$ for each large $n \in \NN$ such that $F_\alpha^n \to F_\alpha$ in $C^{1-}$, where the images of $F_\alpha$ and $F_\alpha^n$ are an open cover $(U_\alpha)$ of $M$ independent of $n$, and $(F_\alpha)$ satisfies the usual transition relations, and $F_\alpha$ is a Lipschitz isomorphism.

%%%%%%%%%%%%%%%%%%%%%%%

\subsubsection{Construction of the limiting lamination}
We now construct the limiting lamination.
We employ the Hausdorff hyperspace $\Hypspace I$ of closed subsets of $I$ to accomplish this.
Since $I$ is a compact metric space, so is $\Hypspace I$ \cite[Theorem 4.17]{nadler2017continuum}, so we may diagonalize so that for every $\alpha$, either $K^n_\alpha \to K_\alpha$ for some nonempty $K_\alpha$ in the Hausdorff distance on $I$, or for all $n \geq n^*(\alpha)$, $K_\alpha^n$ is empty (in which case we define $K_\alpha = \emptyset$).

In order to ensure that the laminations $\lambda_n$ do not escape to infinity, fix a compact set $E \subseteq M$ such that every leaf of every $\lambda_n$ meets $E$.
Then there exists a finite set $A_E \subseteq A$ such that $E \subseteq \bigcup_{\alpha \in A_E} U_\alpha$.

\begin{lemma}\label{label sets are nonempty}
	There exists $\alpha$ such that $K_\alpha$ is nonempty.
\end{lemma}
\begin{proof}
	Suppose not; then for
	$$n \geq \max_{\alpha \in A_E} n^*(\alpha)$$
	and $\alpha \in A_E$, $K_\alpha^n = \emptyset$, so no leaves of $\lambda_n$ meet $U_\alpha$, and hence no leaves of $\lambda_n$ meet $E$.
	This is a contradiction since $\lambda_n$ has a leaf.
\end{proof}

In each flow box $F_\alpha$ with $K_\alpha$ nonempty, we thus have the leaves of a lamination, namely $K_\alpha \times J$.
We now check the transition relations to ensure that they glue to a global lamination; this is straightforward but we include it for completeness.

Thus let $\psi_{\alpha \beta}$ and $\psi_{\alpha \beta}^n$ be the transition maps, thus $\psi_{\alpha \beta}^n$ induces a map
$$\psi_{\alpha \beta}^n: K_\alpha^n \to K_\beta^n.$$
By convergence of $(F_\alpha^n)$, $\psi_{\alpha \beta}$ induces a map $K_\alpha \to K_\beta$.

\begin{definition}
	A \dfn{cocycle of labels} $(k_\alpha)_{\alpha \in A'}$ is a set $A' \subseteq A$ and an element of $\prod_{\alpha \in A'} K_\alpha$, such that:
\begin{enumerate}
	\item The cocycle condition: $k_\beta = \psi_{\alpha \beta}(k_\alpha)$ for $\alpha, \beta \in A'$.
	\item For every $\alpha \in A'$, if $\psi_{\alpha \beta}(k_\alpha)$ is well-defined, then $\beta \in A'$.
\end{enumerate}
\end{definition}

\begin{lemma}
	Every cocycle of labels $(k_\alpha)_{\alpha \in A'}$ defines a complete minimal hypersurface $N$ such that
	$$N \cap U_\alpha = F_\alpha(\{k_\alpha\} \times J).$$
\end{lemma}
\begin{proof}
We have the cocycle condition
$$(N \cap U_\alpha) \cap U_\beta = (N \cap U_\beta) \cap U_\alpha$$
which follows from the fact that
\begin{align*}
F_\alpha(\{k_\alpha\} \times J) \cap U_\beta
&= F_\beta(\psi_{\alpha \beta}(\{k_\beta\} \times J)) \cap U_\alpha \cap U_\beta \\
&= F_\beta(\psi_{\alpha \beta}(\{k_\beta\} \times J)) \cap U_\alpha.
\end{align*}
From the cocycle condition, it follows that $N$ honestly defines a Lipschitz hypersurface in $M$, which is complete in $\bigcup_{\alpha \in A'} U_\alpha$.
If $\overline N$ intersects $U_\alpha$ for some $\alpha \notin A'$, then $N$ intersects $U_\beta$ for some $\beta \in A'$ so that $U_\beta \cap U_\alpha \cap \overline N$ is nonempty.
But then $\psi_{\beta \alpha}(k_\beta)$ must be defined, so $\alpha \in A'$, a contradiction.
Therefore $N$ is complete in $M$.

To prove minimality, let
$$u_\alpha(k, x) = (F_\alpha)_* 1_{k > k_\alpha}$$
and similarly $u_\alpha^n(k, x) = (F_\alpha^n)_* 1_{k > k_\alpha^n}$ where $(k_\alpha^n) \in \prod_n K_\alpha^n$ converges to $k_\alpha$.
Since $F_\alpha \circ (F_\alpha^n)^{-1}$ converges to the identity map in $C^{1-}$, and $F_\alpha^{-1}(N \cap U_\alpha)$ has zero measure, it follows that $u_\alpha^n \to u_\alpha$ almost everywhere, and hence in $L^1(I \times J)$ by the dominated convergence theorem.
But $u_\alpha^n$ has least gradient, so by Miranda compactness (Proposition \ref{MirandaStability}), $\dif u_\alpha^n \to \dif u_\alpha$ in the weak topology of measures.
Clearly $\dif u_\alpha = [N \cap U_\alpha]$ and similarly for $u_\alpha^n$, so by Lemma \ref{measured convergence is smooth convergence}, $N \cap U_\alpha$ is minimal.
\end{proof}

\begin{lemma}
	Let $\lambda$ be the lamination with laminar atlas $(F_\alpha, K_\alpha)$.
	Then $\lambda$ is well-defined and minimal.
\end{lemma}
\begin{proof}
Since 
$$\supp \lambda \cap U_\alpha = K_\alpha \times J$$
and $K_\alpha$ is compact, $\supp \lambda$ is closed.
Now if we choose $\alpha$ such that $K_\alpha$ is nonempty, every element of $K_\alpha$ uniquely determines a cocycle of labels, and hence a leaf of $\lambda$.
So $\supp \lambda$ is nonempty, and since all of its leaves are complete minimal, $\lambda$ is minimal.
\end{proof}

\subsubsection{Convergence in Thurston's geometric topology}
At this stage of the argument we have constructed a limiting lamination with limiting flow boxes; we now check that the sequence of laminations actually converges to the limiting lamination.

If $K_\alpha$ is nonempty, then any $k_\alpha \in K_\alpha$ is the limit of some sequence $(k_\alpha^n)_n \in \prod_n K_\alpha^n$ \cite[Theorem 4.11]{nadler2017continuum}.
Thus $\{k_\alpha\} \times J$ can be written as the set of limits of sequences $(k_\alpha^n, x)_n \in \prod_n K_\alpha^n \times J$, and so any leaf $N$ of $\lambda$ takes the form $N = \lim_{n \to \infty} N_n$ for some sequence $(N_n) \in \prod_n \Leaves \lambda_n$, where $\Leaves \lambda_n$ is the set of leaves of $\lambda_n$.
In other words, leaves of $\lambda$ are pointwise limits of leaves in $\lambda_n$.

So it suffices to show that for $N \in \Leaves \lambda$, $P \in N$, and $P_n \to P$, where $P_n \in N_n$ and $N_n \in \Leaves \lambda_n$, $\normal_{N_n}(P_n) \to \normal_N(P)$.
To do this, suppose that $P \in U_\alpha$; $F_\alpha^n$ is close in tangential $C^\infty$ to $F_\alpha$, and the label $k^n_\alpha$ of $N_n$ is close to the label $k_\alpha$ of $N$.
In particular, if we consider $N$ and $N_n$ as graphs of functions $u, u_n$ in the coordinates induced by $F_\alpha$, then $u_n \to u$ in $C^\infty$; however, in such coordinates, $u$ is a constant.
A bootstrapping argument based on (\ref{nabla as a normal}) then shows that, since $\dif u_n \to 0$ in $C^0$, $\normal_{N_n} \to \partial_y = \normal_N$ in $C^0$ near $P$.

\subsubsection{Convergence in the measure topology}
Suppose that $\mu_n$ is transverse to $\lambda_n$.
After possibly shrinking the $U_\alpha$ slightly for $\alpha \in A_E$, we may assume that they are precompact in $M$ and still form an open cover of $E$.
Then $K := \bigcup_{\alpha \in A_E} \overline{U_\alpha}$ is compact, so by Prohorov's theorem \cite[Theorem 13.29]{klenke2013probability}, there is a subsequence of $(T_{\mu_n})$ which converges to some $T_\mu|_K$ on $K$.
Since $\mu_n(E) \gtrsim 1$, $T_\mu$ is nonzero.
Moreover, by Proposition \ref{portmanteau}
$$\supp T_\mu|_K \subseteq \liminf_{n \to \infty} \supp T_{\mu_n}|_K \subseteq \liminf_{n \to \infty} \supp \lambda_n \cap K.$$
Here the $(\lambda_n)$ in the limit inferior refers to the subsequence which already converges in the Thurston topology (and has converging Ruelle-Sullivan currents).
In particular, the limit inferior is actually a limit and we conclude
$$\supp T_\mu|_K \subseteq \supp \lambda \cap K.$$
We may assume that $\mu_\alpha^n \to \mu_\alpha$ weakly for every $\alpha \in A_E$ and some positive Radon measures $\mu_\alpha$ (whose support is necessarily then contained in $K_\alpha$).
Taking the limit as $n \to \infty$ of the equation 
$$\int_{U_\alpha} T_{\mu_n} \wedge \varphi = \int_I \int_{\{k\} \times J} (F_\alpha^n)^* \varphi \dif \mu_\alpha^n(k),$$
we conclude that
$$\int_{U_\alpha} T_\mu|_K \wedge \varphi = \int_I \int_{\{k\} \times J} F_\alpha^* \varphi \dif \mu_\alpha(k).$$
In other words, $T_\mu|_K$ is Ruelle-Sullivan for $\lambda|_K$, possibly after shrinking $\lambda|_K$ so that their supports match.
By the measure-preserving condition in the definition of transverse measure, $T_\mu|_K$ extends uniquely to a Ruelle-Sullivan current $T_\mu$ on all of $M$, which then necessarily is a weak limit of $(T_{\mu_n})$.
This completes the proof of Theorem \ref{compactness theorem}.


%%%%%%%%%%%%%%%%%%%%%%%%%%%%%%%%%%%%%%
\subsection{Consequences of measured convergence}\label{relationships between modes}
We now apply Theorems \ref{main thm} and \ref{compactness theorem} to explain how the different modes of convergence are related to each other.
It is clear from the definitions that flow-box convergence implies Thurston convergence.
Moreover, for $d = 2$, Thurston claimed that that measure convergence implies Thurston convergence \cite[Proposition 8.10.3]{thurston1979geometry}, though he did not explicitly justify why the limit was geodesic, or why the convergence preserves the normal vectors.
We complete the proof that measure convergence implies Thurston convergence, and show that flow-box convergence sits in the middle of the chain of implications.

\begin{lemma}\label{limits of measured geodesic lams are geodesic}
Suppose that $d \leq 7$. The set of minimal measured laminations of bounded curvature is closed in the weak topology of measures.
\end{lemma}
\begin{proof}
Let $(\lambda, \mu)$ be a measured lamination and suppose that $(\lambda_i, \mu_i) \to (\lambda, \mu)$ in the weak topology of measures, where $(\lambda_i, \mu_i)$ are measured minimal and of bounded curvature.
By Proposition \ref{minimal implies locally minimizing} and the curvature bound, for every $x \in M$ there exists $r > 0$ such that every leaf of every lamination $\lambda_i$ is absolutely area-minimizing in $B(x, r)$.
After shrinking $r$ if necessary, we may assume that $H^1(B(x, r), \RR) = 0$.
Then, by Theorem \ref{main thm}, the Ruelle-Sullivan currents $T_{\mu_i}$ on $B(x, r)$ are the exterior derivatives of functions $u_i$ of least gradient.
Since $u_i$ is only defined up to a constant, we impose $\int_M \star u_i = 0$, so by Poincar\'e's inequality,
$$\|u_i\|_{L^1(B)} \lesssim r\mu_i(B(x, r)) \lesssim r^d < \infty$$
for $i$ large.
So by Miranda compactness (Proposition \ref{MirandaStability}), there exists a function $u$ of least gradient such that along a subsequence, $\dif u_i \to \dif u$ in the weak topology of measures.
Then $T = \dif \mu$, so the leaves of $\lambda$ are level sets of $u$.
By Theorem \ref{main thm of old paper}, the leaves of $\lambda$ are minimal hypersurfaces, as desired.
\end{proof}

\begin{proposition}
Suppose that $d \leq 7$.
Let $(\lambda_n, \mu_n)$ be measured minimal laminations in $M$, and $(\lambda_n, \mu_n) \to (\lambda, \mu)$.
Then $\lambda_n \to \lambda$ in Thurston's geometric topology.
\end{proposition}
\begin{proof}
By (\ref{supports shrink in the limit}), for every $x \in \supp \lambda$, $\varepsilon > 0$, and large $n$, $\supp \lambda_n \cap B(x, \varepsilon)$ is nonempty, and by Lemma \ref{limits of measured geodesic lams are geodesic}, $\lambda$ is a minimal lamination.
By Theorem \ref{regularity theorem}, $\lambda, \lambda_n$ admit Lipschitz normal vectors, so by Lemma \ref{convergence of normals}, $\lambda_n \to \lambda$ in Thurston's geometric topology.
\end{proof}

\begin{proposition}\label{convergence of traansverse measures means flow box convergence}
Suppose that $d \leq 7$.
Let $(\lambda_n, \mu_n)$ be measured minimal laminations in $M$ of bounded curvature, and $(\lambda_n, \mu_n) \to (\lambda, \mu)$.
Then $\lambda_n \to \lambda$ in the $C^{1-}$ and tangentially $C^\infty$ flow box topology.
\end{proposition}
\begin{proof}
We first observe that $\lambda_n \to \lambda$ in Thurston's geometric topology.
After discarding some leaves of $\lambda_n$ we may assume that $\lambda$ is a maximal limit for the Thurston topology.
Moreover, every subsequence $(\lambda_{n_k})$ has a further subsequence $(\lambda_{n_{k_\ell}})$ which converges to some maximal limit $\tilde \lambda$ in the $C^{1-}$ flow box topology by Theorem \ref{compactness theorem}.
But convergence in the flow box topology implies convergence in Thurston's topology, so $\tilde \lambda = \lambda$.
Since $(\lambda_{n_k})$ was arbitrary, it follows that $\lambda_n \to \lambda$ in the $C^{1-}$ flow box topology.
\end{proof}

%%%%%%%%%%%%%%%%%%%%%%
\appendix 
\section{Geometric measure theory}\label{boundary appendix}
\subsection{Weak topology of measures}
Let $X$ be a metrizable space, and let $C_\cpt(X)$ be the space of compactly supported continuous functions $f: X \to \RR$.
Its dual $C_\cpt(X)'$ is canonically isomorphic to the space of signed Radon measures on $X$, where the bilinear pairing is given by integration.
The weak-star topology on $C_\cpt(X)'$ is known as the \dfn{weak topology of measures}.
Unpacking the definitions, a sequence $(\mu_n)$ of Radon measures converges to $\mu$ in the weak topology of measures iff for every continuous function $f: X \to \RR$,
$$\lim_{n \to \infty} \int_X f \dif \mu_n = \int_X f \dif \mu.$$

\begin{proposition}[portmanteau theorem]\label{portmanteau}
	Let $(\mu_n)$ be a sequence of positive Radon measures on a compact metrizable space $X$ with $\mu_n(X) \lesssim 1$, and let $\mu$ be a Radon measure on $X$. The following are equivalent:
\begin{enumerate}
	\item $\mu_n \to \mu$ in the weak topology of measures.
	\item $\liminf_{n \to \infty} \mu_n(X) \geq \mu(X)$ and for every closed $Y \subseteq X$, $\limsup_{n \to \infty} \mu_n(Y) \leq \mu(Y)$.
	\item $\limsup_{n \to \infty} \mu_n(X) \leq \mu(X)$ and for every open $Z \subseteq X$, $\liminf_{n \to \infty} \mu_n(Z) \geq \mu(Z)$.
	\item For every $W \subseteq X$ with $\mu(\partial W) = 0$, $\lim_{n \to \infty} \mu_n(W) = \mu(W)$.
\end{enumerate}
	If we choose a metric on $X$, then the above conditions imply:
\begin{enumerate}
	\setcounter{enumi}{4}
	\item For every $x \in X$ and all but countably many $\varepsilon > 0$, $\lim_{n \to \infty} \mu_n(B(x, \varepsilon)) = \mu(B(x, \varepsilon))$.
\end{enumerate}
\end{proposition}

\begin{lemma}\label{cardinality appendix}
Let $S$ be an uncountable set and $f: S \to (0, \infty)$. Then there exists a countable set $S' \subset S$ such that
$$\sum_{x \in S'} f(x) = \infty.$$
\end{lemma}
\begin{proof}
Define $S_n := f^{-1}([\frac{1}{n + 1}, \frac{1}{n}))$ for $n \in \NN$ (where we take the convention $1/n = \infty$).
Since $S$ is uncountable but $\NN$ is countable, it follows from the infinite pigeonhole principle that there exists $n \in \NN$ such that $S_n$ is infinite.
In particular there exists an infinite countable set $S' \subseteq S_n$, which then satisfies
\begin{align*}
\sum_{x \in S'} f(x) &\geq \sum_{x \in S'} \frac{1}{n + 1} = \infty. \qedhere 
\end{align*}
\end{proof}

\begin{proof}[Proof of Proposition \ref{portmanteau}]
	See \cite[Theorem 13.16]{klenke2013probability} for the equivalence of (1)--(4); \cite{klenke2013probability} deals with subprobability measures, but this is equivalent to measures of bounded total mass by a rescaling.

	We then must show that (4) implies (5); to do so, it suffices to show that for all but countably many $\varepsilon$, $\mu(\partial B(x, \varepsilon)) = 0$.
	Let
	$$A := \{\varepsilon > 0: \mu(\partial B(x, \varepsilon)) > 0\}.$$
	Since the sets $\partial B(x, \varepsilon)$ are disjoint, for every countable $A' \subseteq A$,
	$$\sum_{\varepsilon \in A'} \mu(\partial B(x, \varepsilon)) \leq \mu(X) < \infty,$$
	where $\mu(X) < \infty$ since $X$ is compact.
	It follows from (the contrapositive of) Lemma \ref{cardinality appendix} that $A$ is countable.
\end{proof}

There are subtleties involved in the portmanteau theorem for noncompact $X$.
However, this will never be an issue, as we shall only use it locally, in small precompact balls.

%%%%%%%%%%%%%%%%%%%%%5
\subsection{Currents of locally finite mass}
Let $M$ be a manifold, and consider instead the space $C_\cpt(M, \Omega^\ell)$ of compactly supported continuous $\ell$-forms.
An $\ell$-\dfn{current of locally finite mass} (which we will simply call an \dfn{$\ell$-current}) is an element of the dual space $C_\cpt(M, \Omega^\ell)'$ \cite{simon1983GMT}.
We denote the pairing of an $\ell$-current $T$ and an $\ell$-form $\varphi$ by $\int_M T \wedge \varphi$; this defines the weak topology of measures on the space of currents.
To any $\ell$-current $T$ we may associate a positive Radon measure, its \dfn{mass measure} $\star |T|$, which satisfies for any function $f$,
$$\int_M f \star |T| := \sup_{|\varphi| \leq |f|} \int_M T \wedge \varphi,$$
and a $|T|$-measurable $d - \ell$-form $\psi$, the \dfn{polar part} \cite[Theorem 4.14]{simon1983GMT}, which satisfies $T = \psi |T|$, $|T|$-almost everywhere.

A function $u$ has \dfn{locally bounded variation} (denoted $u \in BV_\loc(M)$) if its distributional derivative $\dif u$ is a $d - 1$-current of locally finite mass, and a measurable set $E \subseteq M$ has \dfn{locally finite perimeter} if $1_E \in BV_\loc(M)$.

%%%%%%%%%%%%%%%%%%%%%%
\subsection{Boundaries}
If $X$ is a metric measure space and $E \subseteq X$ is a measurable set, then it is standard dogma in measure theory that $E$ should be viewed not as a set, but as the \emph{equivalence class} of sets up to symmetric difference with sets of measure zero.
Therefore the boundary $\partial^{\rm top} E := \overline E \cap \overline{M \setminus E}$ in the sense of point-set topology is ill-defined, unless $E$ has some canonical representative (such as an open set).
This issue is rectified by the following definition.

\begin{definition}
Let $(X, \mu)$ be a metric measure space and $E \subseteq X$ a measurable set.
The \dfn{measure-theoretic boundary}\footnote{There are multiple, nonequivalent definitions of the boundary operator on metric measure spaces \cite[Chapter 6]{Pugh02}.} $\partial^{\rm meas} E$ is the set of $x \in X$ such that for every sufficiently small $\varepsilon > 0$,
$$0 < \frac{\mu(E \cap B(x, \varepsilon))}{\mu(B(x, \varepsilon))} < 1.$$
\end{definition}

If $X = M$ is a Riemannian manifold, then there exists a representative $\tilde E$ of the equivalence class $E$ such that $\partial^{\rm top} \tilde E = \partial^{\rm meas} E$ \cite[Theorem 3.1]{Giusti77}; it is straightforward to check that $\partial^{\rm meas} E$ and $\tilde E$ do not depend on the choice of Riemannian metric.
\emph{We adopt the convention throughout this paper that we are working with representatives $\tilde E$ such that $\partial^{\rm top} \tilde E = \partial^{\rm meas} E$, and simply write $\partial E$ for the boundary.}
Since $\partial E = \partial^{\rm top} \tilde E$, $\partial E$ is a closed set. 

Now suppose that $M$ is a Riemannian manifold, and $E$ is a set of locally finite perimeter.
Then $\dif 1_E$ has locally finite mass, so we can consider its polar part, a $1$-form which we identify with the unit conormal $\normal_E^\flat$.
Recall that the \dfn{Lebesgue points} of $f$ are those points $x \in M$ such that the components $(\normal_E^\flat)_i(x)$ satisfy 
$$(\normal_E^\flat)_i(x) = \lim_{\varepsilon \to 0} \frac{1}{\int_{B(x, \varepsilon)} \star |\dif 1_E|} \int_{B(x, \varepsilon)} \star (\normal_E^\flat)_i |\dif 1_E|.$$
The set of Lebesgue points does not depend on the Riemannian metric, or the choice of trivialization of the cotangent bundle used to define the components $(\normal_E^\flat)_i$ \cite[Corollary 2.2]{BackusFLG}, so the following definition makes sense:

\begin{definition}
Let $M$ be a smooth manifold, and let $E \subseteq M$ be a set of locally finite perimeter.
The \dfn{reduced boundary} $\partial^* E$ is the set of Lebesgue points of the unit conormal $\normal_E^\flat$.
\end{definition}

The $d - 1$-dimensional Hausdorff measure $\mathcal H^{d - 1}$ on $\partial E$ equals $\star |\dif 1_E|$, and $\partial^* E$ is both dense and $\mathcal H^{d - 1}$-almost all of $\partial E$ \cite[Theorem 4.4]{Giusti77}.
If $\normal_E^\flat$ extends continuously to all of $\partial E$, then $\partial E = \partial^* E$ is $C^1$ \cite[Theorem 4.11]{Giusti77}.

%%%%%%%%%%%%%%%%%%%%%%%
\subsection{Level sets}
Let $u \in BV_\loc(M)$.
Since $u$ is only defined almost everywhere, we cannot discuss its values pointwise.
Instead, we work with the superlevel sets $\{u > y\}$, $y \in \RR$, which are well-defined as (equivalence classes of) measurable sets.
For almost every $y \in \RR$, it follows from the coarea formula for $BV_\loc$ functions \cite[Theorem 1.23]{Giusti77} that $\{u > y\}$ is a set of locally finite perimeter, and we consider the ``level sets'' $\partial \{u > y\}$.

\begin{proposition}
Let $u \in BV_\loc(M)$. Then 
\begin{equation}\label{level sets define support}
	\supp \dif u = \overline{\bigcup_{y \in \RR} \partial \{u > y\}}.
\end{equation}
\end{proposition}
\begin{proof}
First observe that $M \setminus \supp \dif u$ is the set of $x \in M$ such that for some open neighborhood $V \ni x$, $u|_V$ is almost constant.
Hence if $\partial \{u > y\} \cap V$ is nonempty, we can write $V = V_+ \sqcup V_-$, where both $V_\pm$ have positive measure, $u|_{V_-} \leq y$, and $u|_{V_+} > y$, which contradicts that $u|_V$ is almost constant.
Therefore $\partial \{u > y\} \subseteq \supp \dif u$.

Conversely, if $x \notin \overline{\bigcup_{y \in \RR} \partial \{u > y\}}$, then there exists an open neighborhood $V \ni x$ such that for every $y \in \RR$, $\partial \{u > y\} \cap V$ is empty.
Therefore either $|\{u > y\} \cap V| = |V|$ or $|\{u \leq y\} \cap V| = |V|$.
Let $y$ be the real number at which $u$ transitions from $\{u > y\}$ meeting $V$ to $\{u \leq y\}$ meeting $V$.
(Such a number must exist, or else $|u|_V| = +\infty$, violating that $u \in L^1_\loc$.)
Then $u|_V = y$ almost everywhere, so $u|_V$ is almost constant, hence $x \notin \supp \dif u$.
\end{proof}

The purpose of the closure in (\ref{level sets define support}) is the existence of local extrema of $u$.
This is illustrated by the following example which will also be critical in \S\ref{genSets}:

\begin{example}
Consider $u: \RR \to \RR$, $x \mapsto |x|$.
\emph{As a measurable set}, $\{u > 0\}$ is the entire real line, so the level set $\partial \{u > 0\}$ is empty.
Of course, this defies our usual intuition, which says that the level set $\{u = 0\}$ is the point $\{0\}$; the issue is that we are using machinery designed for $BV_\loc$ functions, which are usually discontinuous, to the continuous function $u$.
\end{example}

%%%%%%%%%%%%%%%%%%%%
\subsection{Regularity of minimizing boundaries}
The regularity of codimension-$1$ hypersurfaces is well-known: see Giusti \cite[Theorem 10.11]{Giusti77} for an exposition of Miranda \cite{Miranda66}'s proof using functions of least gradient on $\RR^d$; Morgan \cite[Chapter 8]{Morgan88} for a history of the result and more references, including proofs on Riemannian manifolds; and my expository note \cite{BackusFLG} for a direct generalization of \cite{Miranda66} to Riemannian manifolds.
For the reader's convenience, we record here a short argument which reduces the regularity theorem to Nash's embedding theorem and a weak form of Allard's $\varepsilon$-regularity theorem on $\RR^d$ which is easily established in lecture notes of de Lellis \cite[Theorem 3.2]{DeLellis18}.

\begin{theorem}\label{regularity}
Let $d \leq 7$. Let $E$ be a set of locally finite perimeter in the Riemannian manifold $M$, such that $1_E$ has least gradient.
Then $\partial E$ is a smooth area-minimizing minimal hypersurface.
\end{theorem}

To set up the proof, recall that the mean curvature $H_N$ of a rectifiable set $N \subset \RR^D$ satisfies, for every $C^\infty_\cpt(\RR^D)$ vector field $X$,
$$\delta N(X) = -\int_N X \cdot H \dif \mu_N$$
where $\delta N(X)$ is the first variation of area in the direction $X$, and $\dif \mu_N$ is the surface measure on $N$.
By \cite[Proposition 1.5]{DeLellis18}, 
$$\delta N(X) = \int_N \Div_{TN} X \dif \mu_N$$
where, given a measurable subbundle $E$ of the tangent bundle $T\RR^D$ with an orthonormal frame $v_1, \dots, v_\delta$, the divergence is 
$$\Div_E X := \sum_{i = 1}^\delta \langle v_i, \nabla X\rangle.$$

\begin{theorem}[Allard's $\varepsilon$-regularity theorem]
For any integers $1 \leq \delta < D$ there exist $\varepsilon, \gamma, \alpha > 0$ such that the following holds:

Let $N \subset \RR^D$ be a $\delta$-rectifiable set, $x \in N$, and $0 < r < 1$.
Suppose that
\begin{align}
\mu_N(B(x, r)) &\leq (|\Ball^\delta| + \varepsilon) r^\delta, \label{measure excess}\\
\|H_N\|_{L^\infty(N \cap B(x, r))} &\leq \varepsilon/r. \label{mean curvature excess}
\end{align}
Then $N \cap B(x, \gamma r)$ is a $C^{1 + \alpha}$ submanifold of $B(x, \gamma r)$ without boundary.
\end{theorem}

\begin{proof}[Proof of Theorem \ref{regularity}]
By the Nash embedding theorem, we may assume that $M$ is embedded in $\RR^D$ for some $D \geq d + 1$.
Let $N := \partial^* E$. 
By \cite[Theorem 4.11]{Giusti77} and the diffeomorphism invariance of $\partial^*$, $N$ is a $d - 1$-rectifiable subset of $\RR^D$.

Let $x \in N$.
For $0 < r \ll 1$, the dilation $F_r$ by $r$ of $(\exp_x)^{-1}(N \cap B(x, r))$ is an approximately mass-minimizing $d - 1$-rectifiable subset of the unit ball in $T_x M$ (since on small scales the exponential map is close to an isometry).
By a suitable modification of \cite[Chapters 9-10]{Giusti77}, $d \leq 7$ implies that $[F_r]$ converges in the weak topology of measures on $T_x M$ to a hyperplane $T_x N$.
In particular, $r^{1 - d} \mu_N(B(x, r))$ is approximated arbitrarily well by $|\Ball^{d - 1}|$ as $r \to 0$, so for $r$ small, (\ref{measure excess}) holds.

Let $X$ be a $C^\infty_\cpt(B(x, r))$ vector field on $\RR^D$.
If $X$ is tangent to $M$, then $\delta N(X) = 0$ by minimality of $N$.
So we assume that $X$ is normal to $M$.
By the Gauss-Codazzi theorem, $\nabla_M X$ is given by a contraction of the second fundamental form $\Two_M$ with $X$.
Moreover, $\nabla_N X$ is a minor of $\nabla_M X$, so $\Div_{TN} X$ is a partial trace of $\Two_M \otimes X$.
In particular,
$$\left|\int_N \Div_{TN} X \dif \mu_N\right| \lesssim_{D, d} \int_N |\Two_M| |X| \dif \mu_N \leq \|\Two_M\|_{C^0} \|X\|_{L^1(N)}.$$
Therefore 
$$\|H_{N \to \RR^D}\|_{L^\infty} \lesssim_{D, d} \|\Two_M\|_{C^0}.$$
So for $r$ small, (\ref{mean curvature excess}) holds.

So by Allard's theorem, $N \cap B(x, \gamma r)$ is $C^{1 + \alpha}$ and a closed set, implying that the normal vector extends continuously to $\partial E \cap B(x, \gamma r)$.
By elliptic bootstrapping, it follows that $\partial E \cap B(x, \gamma r)$ is $C^\infty$.
\end{proof}

%%%%%%%%%%%%%%%%%%%%%%
\section{Minimal hypersurfaces are locally area-minimizing} \label{locally minimizing appendix}
\begin{proposition}\label{minimal implies locally minimizing}
Let $d \leq 7$, let $i, A > 0$, and let $g$ be a Riemannian metric on $M$ with $\|g\|_{C^{2 + \alpha}} < \infty$ and injectivity radius $\geq i$.
Then there exists $\delta_* > 0$ which only depends on $\|g\|_{C^{2 + \alpha}}, i, A$, such that the following holds:

For every ball $B(p, \delta) \subset M$ with $\delta \leq \delta_*$ and every oriented minimal hypersurface $N \subset B(p, \delta)$ with $p \in N$, $\partial N \subset \partial B(p, \delta)$, $\|\Two_N\|_{C^0} \leq A$, and trivial normal bundle, $N$ is the unique absolute area-minimizer among all hypersurfaces with boundary $\partial N$.
\end{proposition}

This proposition (or at least, a qualitative version of it) seems to be well-known, but we are not aware of a suitable reference.
We give a version of the argument sketched by Otis Chodosh in a MathOverflow post \cite{MathOverflowMinimalLocal}.
We take the results of \S\S\ref{Regularity}-\ref{Prelims} as given in this appendix, as we shall not use Proposition \ref{minimal implies locally minimizing} in those sections.

\begin{lemma}\label{existence of absolute minimizers}
Let $d \leq 7$, let $U \subseteq M$ be a ball of finite radius and $H^1(U, \RR) = 0$, and let $N \subset M$ be a hypersurface of trivial normal bundle and $\Two_N$ bounded.
Then there exists a smooth hypersurface $N'$ with $N \cap \partial U = N' \cap \partial U$, which is absolutely area-minimizing.
\end{lemma}
\begin{proof}
Since $N$ has trivial normal bundle and $H^1(U, \RR) = 0$, integration along $N$ defines an exact $d - 1$-current $\dif u$, and we can normalize $u$ to be $0$ somewhere.
Since $\Two_N$ is bounded and $U$ has finite radius, the surface area of $N$ is finite, so $u \in BV(U)$, hence $u \in L^1(\partial U)$.
So by \cite[Theorem 1.20]{Giusti77} there exists a $\{0, 1\}$-valued function $u'$ of least gradient with $u'|_{\partial U} = u|_{\partial U}$, and since $d \leq 7$ it follows from Theorem \ref{main thm of old paper} that $\dif u'$ is the current associated to an absolutely area-minimizing smooth hypersurface $N'$ which is a competitor to $N$.
\end{proof}

\begin{lemma}
Let $Q_n$ be a sequence of second order linear elliptic operators on $\Ball^d$, whose coefficients (in nondivergence form) converge in $C^0$ to those of a second order linear elliptic operator $Q$.
Let $\lambda_1^{(n)}, \lambda_1 > 0$ be the spectral gaps of $Q_n, Q$.
Then 
\begin{equation}\label{spectral gap}
\liminf_{n \to \infty} \lambda_1^{(n)} \geq \lambda_1.
\end{equation}
\end{lemma}
\begin{proof}
Let $u \in W^{2, 2}(\Ball^d)$ satisfy $\|u\|_{L^2} = 1$; then $\langle u, Qu\rangle_{L^2} \geq \lambda_1$ by the Rayleigh-Ritz formula.
Moreover, if $a_n,a,b_n,b,c_n,c$ are the coefficients,
\begin{align*}
\langle u, (Q_n - Q)u\rangle_{L^2}
&= \langle u, (a_n - a) \cdot \nabla^2 u\rangle_{L^2} + \langle u, (b_n - b) \cdot \nabla u\rangle_{L^2} + \langle u, (c_n - c) u\rangle_{L^2} \\
&\leq (\|a_n - a\|_{C^0} + \|b_n - b\|_{C^0} + \|c_n - c\|_{C^0}) \|u\|_{L^2} \|u\|_{W^{2, 2}} \to 0.
\end{align*}
Therefore
$$\lim_{n \to \infty} \langle u, Q_n u\rangle_{L^2} = \langle u, Qu\rangle_{L^2} \geq \lambda_1.$$
Taking $u$ to be the first eigenfunction of $u$, we get (\ref{spectral gap}).
\end{proof}

Given a hypersurface $\Sigma \subset \Ball^d$ and a metric $h$ on $\Ball^d$, let
\begin{equation}\label{stability formula}
Q_{\Sigma, h} = -\Delta_{\Sigma, h} - |\Two_{\Sigma, h}|^2 - \Ric_h(\normal_{\Sigma, h}, \normal_{\Sigma, h})
\end{equation}
be the stability operator for $\Sigma$, and let $H_{\Sigma, h}$ be the mean curvature.
Then the second variation of area in the direction of a normal variation $s$ is \cite[Chapter 1, \S8.1]{colding2011course}
\begin{equation}\label{second variation formula}
\delta^2_s |\Sigma|_h = \langle s, Q_{\Sigma, h} s\rangle_{L^2(\Sigma, h)} + \|sH_{\Sigma, h}\|_{L^2(\Sigma, h)}^2.
\end{equation}

\begin{lemma}\label{uniform continuity of stability}
Suppose that $s \in W^{1, 2}(\Ball^{d - 1})$ satisfy $s|_{\partial \Ball^{d - 1}} = 0$.
Let $\mathscr N$ be the set of graphs in $\Ball^d$ of functions lying in some compact subset of $C^2(\Ball^{d - 1})$ (so $s$ induces a normal variation on each member of $\mathscr N$).
Let $\mathscr G$ be a compact set of Riemannian metrics on $\Ball^d$ for the $C^2$ topology.
Then
\begin{align*}
\mathscr N \times \mathscr G &\to \RR, \\
(\Sigma, h) &\mapsto \delta^2_s |\Sigma|_h
\end{align*}
is uniformly continuous on $\mathscr N$, with modulus of continuity only depending on $\|s\|_{W^{1, 2}}, \mathscr N, \mathscr G$.
\end{lemma}
\begin{proof}
From (\ref{second variation formula}) and the fact that $s \mapsto \langle s, Q_{\Sigma, h} s\rangle_{L^2(\Sigma, h)}$ is a continuous quadratic form on $W^{1, 2}$, it will suffice to show that
\begin{equation}\label{stability curvature map}
(\Sigma, h) \mapsto (Q_{\Sigma, h}, H_{\Sigma, h})
\end{equation}
is uniformly equicontinuous, where the topology on the space of differential operators $Q$ is the $C^0$ topology on their nondivergence form coefficients.
Moreover, since $\mathscr N \times \mathscr G$ is compact, it suffices to show that (\ref{stability curvature map}) is continuous.
This is clear from (\ref{stability formula}) and the fact that $H_{\Sigma, h}$ only depends on second derivatives of $(\Sigma, h)$.
\end{proof}

\begin{proof}[Proof of Proposition \ref{minimal implies locally minimizing}]
We proceed by compactness and contradiction.
Let $(g_j)$ be a sequence of metrics bounded in $C^{2 + \alpha}(\Ball^d)$, which are increasingly severe violations of the proposition, in the sense that for some sequence $\delta_j \to 0$, there exist oriented $g_j$-minimal hypersurfaces $N_j \subset B_j := B_{g_j}(0, \delta_j)$ which contain $0$ and are not uniquely area-minimizing, but such that $\|\Two_{N_j}\|_{C^0(B_j)} \leq A$.
Since $(g_j)$ is bounded in $C^{2 + \alpha}$, $(g_j)$ is contained in a compact subset of $C^2$ and their curvatures are bounded in $C^0$.

% and after rescaling and shrinking $\delta_j$ if necessary, we may assume that $\|\Riem_{g_j}\|_{C^0} \leq K_0$ as in \S\ref{Regularity}.
% After applying isometries as necessary, we may assume that $g_j$ is written in normal coordinates centered on $0$, hence $|g_j - I| \lesssim K_0 |x|^2$.
% By compactness of the forgetful map $C^{2 + \alpha} \to C^2$, we may further assume that $(g_j)$ converges in $C^2$ to a $2$-tensor $g$ on $\Ball^d$ such that $|g - I| \lesssim K_0 |x|^2$.
% Taking $K_0$ small enough, it follows that $g$ is positive-definite and hence is a Riemannian metric.

By Lemma \ref{existence of absolute minimizers}, there exist $g_j$-absolutely minimizing competitors $N_j'$ to $N_j$ in $B_j$.
Let $L_j, L_j' \subset \Ball^d$ be the rescalings of $N_j, N_j'$ by $\delta_j^{-1}$.
Since $A < \infty$, $(\Riem_{g_j})$ is bounded in $C^0$, and $\delta_j \to 0$, $L_j$ is an oriented hypersurface in $\Ball^d$ such that $0 \in L_j$ and $\Two_{L_j} \to 0$ uniformly, where $\Two_{L_j}$ is taken in the flat metric on $\Ball^d$.
By Lemma \ref{existence of tubes}, for $j$ large enough, $L_j$ is a graph, possibly after rotation, of some function $f_j$ such that $\|f_j\|_{C^2} \to 0$.
Let $L$ be the graph of $0$, so $L_j \to L$ as $C^2$ graphs.
Since $L$ is a disk, $L$ is the unique absolutely minimizer among its competitors for the flat metric.

Let $\eta_j$ be the area of an absolutely minimizing competitor of $L_j$ (for the flat metric).
Since $N_j'$ is absolutely minimizing for $g_j$, and $L_j \to L$ in $C^0$, the normal coordinates condition on $g_j$ implies that
$$|L| - o(1) \leq |L_j'| \leq \eta_j + o(1) \leq |L| + o(1)$$
where areas are taken in the flat metric.
So along a subsequence, by Miranda compactness (Proposition \ref{MirandaStability}), the current defined by $L_j'$ converges in the weak topology of measures to the current of an absolutely minimizing competitor $L'$ of $L$.
By uniqueness of $L$, $L = L'$; moreover, $|L_j'| \to |L|$, so (\ref{supports shrink in the limit}), $L_j' \to L$ in the Hausdorff topology.
Therefore, by Lemma \ref{convergence of normals}, any sequence of points $p_j \in L_j'$ converges to some $p \in L$, with $\normal_{L_j'}(p_j) \to \normal_L(p)$.
So by Lemma \ref{existence of tubes}, near $p$, $L_j'$ is a graph over $L$, of a function $f_j'$ such that $\|f_j'\|_{C^1} \to 0$.

The size of the ball around $p$ in which $L_j'$ is a graph is bounded from below by a quantity depending on $\|f_j'\|_{C^2}$.
But by (\ref{norms on uk}), $\|f_j'\|_{C^2} \to 0$.
So by a bootstrapping argument and Lemma \ref{existence of tubes}, $L_j'$ is globally a graph over $L_j'$ for $j$ large enough.
Since $L_j$ is close to $L$ in $C^2$, if $j$ is large, then $L_j'$ is a graph over $L_j$.

Thus, for $j \gg 1$, we may view $L_j'$ as a normal variation of $L_j$ in the rescaled metric $g_j' := \delta_j^{1/2} g_j$, associated to a function $s_j$ on $L_j$.
But $\|f_j - f_j'\|_{C^2} \to 0$, so $s_j \to 0$ in $C^2 \subset W^{1, 2}$.
Let $Q_j$ be the stability operator $Q_{L_j, g_j'}$, and observe that since $L_j, L_j'$ are converging to $L$ in $C^2$, they lie in a compact subset of $C^2$. 
Moreover, $L_j$ has zero $g_j'$-mean curvature, so the $g_j'$-area of $L_j'$ satisfies
\begin{equation}\label{stability area bound}
|L_j'|_{g_j'} \geq |L_j|_{g_j'} + \langle s_j, Q_j s_j\rangle_{L^2(L_j, g_j')} - o(\|s_j\|_{W^{1, 2}(L_j, g_j')}^2)
\end{equation}
where the rate of convergence in the error term is independent of $j$ by Lemma \ref{uniform continuity of stability}.
By (\ref{norms on uk}), $\|s_j\|_{W^{1, 2}} \sim \|s_j\|_{L^2}$, which we use to replace the error term in (\ref{stability area bound}).

Let $Q = -\Delta$ be the stability operator of $L$, so that the nondivergence form coefficients of $Q_j$ converge in $C^0$ to coefficients of $Q$.
The spectral gap of $Q$ is the first Dirichlet eigenvalue $\lambda_1 > 0$ of $\Ball^{d - 1}$.
By (\ref{spectral gap}), if $j$ is large enough, then $Q_j$ has spectral gap $\geq \lambda_1/2$.
Since $L_j'$ is an absolute minimizer, we obtain from (\ref{stability area bound}) that for $j$ large enough,
$$|L_j'|_{g_j'} \geq |L_j|_{g_j'} + \frac{\lambda_1}{2} \|s_j\|_{L^2}^2 - o(\|s_j\|_{L^2}^2) \geq |L_j'|_{g_j'} + \frac{\lambda_1}{3} \|s_j\|_{L^2}^2.$$
Rearranging terms and exploiting $\lambda_1 > 0$, we see that $s_j = 0$.
Therefore $L_j = L_j'$ is both an absolute minimizer and not an absolute minimizer, a contradiction.
\end{proof}

\printbibliography

\end{document}
