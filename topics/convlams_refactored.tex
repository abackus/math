\documentclass[reqno,11pt]{amsart}
\usepackage[letterpaper, margin=1in]{geometry}
\RequirePackage{amsmath,amssymb,amsthm,graphicx,mathrsfs,url,slashed,subcaption}
\RequirePackage[usenames,dvipsnames]{xcolor}
\RequirePackage[colorlinks=true,linkcolor=Red,citecolor=Green]{hyperref}
\RequirePackage{amsxtra}
\usepackage{cancel}
\usepackage{tikz-cd}

% \setlength{\textheight}{9.3in} \setlength{\oddsidemargin}{-0.25in}
% \setlength{\evensidemargin}{-0.25in} \setlength{\textwidth}{7in}
% \setlength{\topmargin}{-0.25in} \setlength{\headheight}{0.18in}
% \setlength{\marginparwidth}{1.0in}
% \setlength{\abovedisplayskip}{0.2in}
% \setlength{\belowdisplayskip}{0.2in}
% \setlength{\parskip}{0.05in}
%\renewcommand{\baselinestretch}{1.05}

\title{Spaces of minimal laminations and an application to 1-harmonic functions}
\author{Aidan Backus}
\date{October 2022}

\newcommand{\NN}{\mathbf{N}}
\newcommand{\ZZ}{\mathbf{Z}}
\newcommand{\QQ}{\mathbf{Q}}
\newcommand{\RR}{\mathbf{R}}
\newcommand{\CC}{\mathbf{C}}
\newcommand{\DD}{\mathbf{D}}
\newcommand{\PP}{\mathbf P}
\newcommand{\MM}{\mathbf M}
\newcommand{\II}{\mathbf I}
\newcommand{\Hyp}{\mathbf H}
\newcommand{\Sph}{\mathbf S}
\newcommand{\Group}{\mathbf G}
\newcommand{\GL}{\mathbf{GL}}
\newcommand{\Orth}{\mathbf{O}}
\newcommand{\SpOrth}{\mathbf{SO}}
\newcommand{\Ball}{\mathbf{B}}

\newcommand*\dif{\mathop{}\!\mathrm{d}}

\DeclareMathOperator{\dist}{dist}
\DeclareMathOperator{\MeasLam}{MeasLam}
\DeclareMathOperator{\MinLam}{MinLam}
\DeclareMathOperator{\Lam}{Lam}
\DeclareMathOperator{\supp}{supp}

\newcommand{\Leaves}{\mathscr L}
\newcommand{\Hypspace}{\mathscr H}

\newcommand{\Two}{\mathrm{I\!I}}


\newcommand{\Hilb}{\mathcal H}
\newcommand{\Homology}{\mathrm H}
\newcommand{\normal}{\mathbf n}
\newcommand{\radial}{\mathbf r}
\newcommand{\evect}{\mathbf e}
\newcommand{\vol}{\mathrm{vol}}

\newcommand{\diam}{\mathrm{diam}}
\newcommand{\inj}{\mathrm{inj}}
\newcommand{\Lip}{\mathrm{Lip}}
\newcommand{\Riem}{\mathrm{Riem}}

\newcommand{\Bmu}{\boldsymbol \mu}
\newcommand{\Bnu}{\boldsymbol \nu}
\newcommand{\Blambda}{\boldsymbol \lambda}

\newcommand{\pic}{\vspace{30mm}}
\newcommand{\dfn}[1]{\emph{#1}\index{#1}}

\renewcommand{\Re}{\operatorname{Re}}
\renewcommand{\Im}{\operatorname{Im}}

\newcommand{\loc}{\mathrm{loc}}
\newcommand{\cpt}{\mathrm{cpt}}

\def\Japan#1{\left \langle #1 \right \rangle}

\newtheorem{theorem}{Theorem}[section]
\newtheorem{badtheorem}[theorem]{``Theorem"}
\newtheorem{prop}[theorem]{Proposition}
\newtheorem{lemma}[theorem]{Lemma}
\newtheorem{sublemma}[theorem]{Sublemma}
\newtheorem{proposition}[theorem]{Proposition}
\newtheorem{corollary}[theorem]{Corollary}
\newtheorem{conjecture}[theorem]{Conjecture}
\newtheorem{axiom}[theorem]{Axiom}
\newtheorem{assumption}[theorem]{Assumption}

\newtheorem{mainthm}{Theorem}
\renewcommand{\themainthm}{\Alph{mainthm}}

% \newtheorem{claim}{Claim}[theorem]
% \renewcommand{\theclaim}{\thetheorem\Alph{claim}}
\newtheorem*{claim}{Claim}

\theoremstyle{definition}
\newtheorem{definition}[theorem]{Definition}
\newtheorem{remark}[theorem]{Remark}
\newtheorem{example}[theorem]{Example}
\newtheorem{notation}[theorem]{Notation}

\newtheorem{exercise}[theorem]{Discussion topic}
\newtheorem{homework}[theorem]{Homework}
\newtheorem{problem}[theorem]{Problem}

\makeatletter
\newcommand{\proofpart}[2]{%
  \par
  \addvspace{\medskipamount}%
  \noindent\emph{Part #1: #2.}
}
\makeatother



\numberwithin{equation}{section}


% Mean
\def\Xint#1{\mathchoice
{\XXint\displaystyle\textstyle{#1}}%
{\XXint\textstyle\scriptstyle{#1}}%
{\XXint\scriptstyle\scriptscriptstyle{#1}}%
{\XXint\scriptscriptstyle\scriptscriptstyle{#1}}%
\!\int}
\def\XXint#1#2#3{{\setbox0=\hbox{$#1{#2#3}{\int}$ }
\vcenter{\hbox{$#2#3$ }}\kern-.6\wd0}}
\def\ddashint{\Xint=}
\def\dashint{\Xint-}

\usepackage[backend=bibtex,style=numeric]{biblatex}
\renewcommand*{\bibfont}{\normalfont\footnotesize}
\addbibresource{topics.bib}
\renewbibmacro{in:}{}
\DeclareFieldFormat{pages}{#1}

\newcommand\todo[1]{\textcolor{red}{TODO: #1}}


\begin{document}
\begin{abstract}
We collect several results governing the various modes of convergence for sequences of minimal laminations.
We then apply the theory that we develop in order to give a sufficient condition for a collection of disjoint minimal surfaces to form a minimal lamination.
Finally, we show that a function is 1-harmonic iff it is the Ruelle-Sullivan current of a minimal lamination; this resolves an open problem of Daskalopoulos--Uhlenbeck.
\end{abstract}

\maketitle

%%%%%%%%%%%%%%%%%%%%%%%%%%%%%%%%%%%%%%%%%%%%%%%%%%%%%%%

% \tableofcontents

\section{Introduction}
The space of codimension-$1$ minimal laminations on a Riemannian manifold has been topologized in several different ways.
Thurston \cite[Chapter 8]{thurston1979geometry} introduced both his geometric topology as well as the weak topology of measures on the space of measured geodesic laminations.
Independently of Thurston, Colding--Minicozzi \cite[Appendix B]{ColdingMinicozziIV} introduced a topology that emphasized not the laminations themselves, but rather the coordinate charts which flatten them.
Our first goal in this paper is to explain how these three modes of convergence are related, as well as the regularity and compactness theorems associated to each such mode.

Our second goal is to give a sufficient condition under which a set of disjoint minimal hypersurfaces is actually a minimal lamination.
In dimension $d = 2$, any such set is a lamination \cite[Proposition 7.3]{daskalopoulos2020transverse} but in general one needs uniform bounds on the curvature of the leaves in order to apply the compactness theory and prevent the formation of a singularity in the limit \cite[Lemma II.1.1]{ColdingMinicozziV}.
In dimensions $d = 3, 4$, if one in addition knows that the leaves are stable, then the curvature bounds are instead provided by the stable Bernstein theorem of Schoen \cite{Schoen2016} and Chodosh--Li \cite{Chodosh2021}.

We then turn to the main goal of this series of papers, which also includes the prequel paper \cite{BackusFLG}.
We show that any $1$-harmonic function gives rise to a Ruelle-Sullivan current for a minimal lamination, and conversely.
This generalizes a theorem of Daskalopoulos--Uhlenbeck \cite[Theorem 6.1]{daskalopoulos2020transverse} and resolves the outstanding problems \cite[Problem 9.4]{daskalopoulos2020transverse} and \cite[Conjecture 9.5]{daskalopoulos2020transverse}.

%%%%%%%%%%%%%%%%%
\subsection{Minimal laminations}\label{Lams sections}
Let $I \subseteq \RR$ be an interval, and $M$ a Riemannian manifold of dimension $d \geq 2$.
A (codimension-$1$) \dfn{laminar flow box} is a $C^0$ coordinate chart $F: I \times \RR^{d - 1} \to M$ and a compact set $K \subseteq I$ such that each \dfn{leaf} $F(\{k\} \times \RR^{d - 1})$ is $C^2$.
A \dfn{laminar transition map} between two laminar flow boxes $(F_\alpha, K_\alpha), (F_\beta, K_\beta)$ is a $C^0$ map
$$\psi_{\alpha \beta}: I \times \RR^{d - 1} \to I \times \RR^{d - 1}$$
satisfying the usual transition relation
\begin{equation}\label{transition relation}
F_\alpha = F_\beta \circ \psi_{\alpha \beta},
\end{equation}
which maps each leaf $\{k\} \times \RR^{d - 1}$, $k \in K_\alpha$, to a leaf $\{\psi_{\alpha \beta}(k)\} \times \RR^{d - 1}$, so that $\psi_{\alpha \beta}$ is a homeomorphism $K_\alpha \to K_\beta$.
By a \dfn{laminar atlas} we shall mean an atlas for $M$, such that the coordinate charts are all laminar flow boxes and the transition maps are also laminar.

\begin{definition}
A \dfn{lamination} $\lambda$ consists of a nonempty closed set $S \subseteq M$, called its \dfn{support}, and a maximal laminar atlas $\{(F_\alpha, K_\alpha): \alpha \in A\}$ such that in the image $U_\alpha$ of each flow box $F_\alpha$,
$$S \cap U_\alpha = F_\alpha(K_\alpha \times \RR^{d - 1}).$$
If $\lambda$ is a lamination in the image of a flow box $F$, and $N := F(\{k\} \times \RR^{d - 1})$ is a leaf of $\lambda$, we call $k$ the \dfn{label} of $N$.
\end{definition}

Summarizing the above definitions, a lamination is a nonempty closed set $S$ with a $C^0$ local product structure which realizes it as $K \times \RR^{d - 1}$ for some compact set $K \subset \RR$.

We assume that the leaves are $C^2$ in order to ensure that the normal vectors to each leaves are well-defined in $C^1$, and in particular the second fundamental form and mean curvature of each leaf is well-defined.
Such laminations are sometimes called $C^2$ \dfn{along leaves} \cite{Morgan88}.
This is not the same thing as assuming that the lamination admits a $C^2$ atlas, as it may not be able to extend the normal vectors to each leaf to a $C^1$ vector field on $M$ even locally.
In any case, we will make more precise assertions about the existence of flow boxes of various regularities in \S\ref{Regularity}.

\begin{definition}
A lamination $\lambda$ is \dfn{minimal} if its leaves $F_\alpha(\{k\} \times \RR^{d - 1})$ have zero mean curvature, and is \dfn{geodesic} if, in addition, $d = 2$.
\end{definition}

Geodesic laminations are of great interest to the Thurston school of geometric topology \cite[Chapter 8]{thurston1979geometry}.
Later Thurston introduced \dfn{best Lipschitz maps}, namely maps $v: M \to N$ between closed manifolds which minimize their Lipschitz constant $\Lip(v)$ subject to a constraint on their homotopy class.
These maps define a geodesic lamination whose support is the set of points $x$ so that the local Lipschitz constant of $v$ at $x$ is equal to $\Lip(v)$ \cite{thurston1998minimal}.
If $M, N$ are hyperbolic surfaces of the same genus $g$, then $\Lip(v)$ is the distance between $M$ and $N$ in \dfn{Thurston's asymmetric metric} on Teichm\"uller space.
This circle of ideas has been developed by the Thurston school \cite{papadopoulos:hal-00129729} but has recently also made contact with geometric PDE through the work of Daskalopoulos--Uhlenbeck \cite{daskalopoulos2020transverse,daskalopoulosPrep1,DaskalopoulosPrep2}, as we recall in \S\ref{FLG section}.

We prove a sufficient condition for a collection of minimal hypersurfaces to define a minimal lamination.

\begin{theorem}\label{building a minimal lamination}
Let $2 \leq d \leq 7$ and suppose that $M$ has constant sectional curvature.
If $\mathcal N$ is a set of disjoint embedded minimal hypersurfaces in $M$, and
\begin{equation}\label{bounding Two}
\sup_{N \in \mathcal N} ||\Two_N||_{C^0} < \infty,
\end{equation}
then $\mathcal N$ is the set of leaves of a minimal lamination.
\end{theorem}

The condition (\ref{bounding Two}) is frequently met in applications, owing to the following:

\begin{theorem}[stable Bernstein theorem]
Let $3 \leq d \leq 4$ and let $N$ be a two-sided stable minimal hypersurface in $B_r \subseteq M$, $r \lesssim 1$, where $M$ has bounded geometry and dimension $d$.
Then on $B_{r/2}$, $|\Two_N| \lesssim_M r^{-1}$.
\end{theorem}
\begin{proof}
For $d = 3$ see \cite[Corollary 2.11]{colding2011course}, and for $d = 4$ see \cite{Chodosh2021}.
\end{proof}

\begin{corollary}\label{building stable}
Let $M$ be a manifold of constant sectional curvature, $\mathcal N$ a set of disjoint embedded minimal hypersurfaces in $M$, and either:
\begin{enumerate}
\item $d = 2$, or
\item $d \in \{3, 4\}$ and every leaf of $\lambda$ is stable.
\end{enumerate}
Then $\mathcal N$ is the set of leaves of a minimal lamination.
\end{corollary}
\begin{proof}
The hypotheses of this corollary, plus the stable Bernstein theorem, imply that $|\Two_N|$ is locally uniformly bounded, so we may apply Theorem \ref{building a minimal lamination}.
\end{proof}

%%%%%%%%%%%%%%%%%%
\subsection{Spaces of minimal laminations}\label{LamSpace section}
In the literature there are at least three different topologies on the space of laminations on $M$.
The first is Thurston's geometric topology \cite[Chapter 8]{thurston1979geometry}, which says that a lamination $\lambda'$ is close to a lamination $\lambda$ if every leaf of $\lambda$ is close to a leaf of $\lambda'$ at least locally, and the same holds for their normal vectors $\normal$.

\begin{definition}
A sequence of laminations $\lambda_i$ converges to a lamination $\lambda$ in \dfn{Thurston's geometric topology} if, for every leaf $N$ of $\lambda$, every $x \in N$, and every $\varepsilon > 0$, there exists $i_\varepsilon \in \NN$ such that for every $i \geq i_\varepsilon$, $\supp \lambda_i$ intersects $B(x, \varepsilon)$, and for $x_i \in B(x, \varepsilon) \cap \supp \lambda_i$,
\begin{equation}\label{convergence of normals}
\dist_{SM}(\normal_{\lambda_i}(x_i), \normal_\lambda(x)) < 2\varepsilon.
\end{equation}
\end{definition}

It is straightforward to show that Thurston's geometric topology does not depend on the choice of Riemannian metric on $M$, or the choice of extension of the distance function on $M$ to its sphere bundle $SM$, which are implicit in the statement thereof.
However, the limiting lamination is not unique, as if $\lambda_i \to \lambda$ and $\lambda'$ is a sublamination of $\lambda$, then $\lambda_i \to \lambda'$.
In particular, Thurston's topology is not Hausdorff, and we say that $\lambda$ is a \dfn{maximal limit} of a sequence $(\lambda_i)$ if $\lambda_i \to \lambda$ and for every $\lambda'$ such that $\lambda_i \to \lambda'$, $\lambda'$ is a sublamination of $\lambda$.

Independently of Thurston, Colding--Minicozzi \cite[Appendix B]{ColdingMinicozziIV} defined a sequence of laminations to converge ``if the corresponding coordinate maps converge;'' that is, if the laminar atlases converge.
This of course says nothing about the limiting set of leaves and in the sequel paper \cite{ColdingMinicozziV} they additionally impose that the sets of leaves converge (say, in Hausdorff distance).
The following is equivalent to requiring that the sets of leaves converge.

\begin{definition}
A sequence $(\lambda_i)$ of laminations \dfn{flow-box converges} in a function space $X$ to $\lambda$ if it converges in Thurston's geometric topology, and there exists a laminar atlas $(F_\alpha)$ for $\lambda$ such that for each $\alpha$, $F_\alpha$ and $(F_\alpha)^{-1}$ are limits in $X$ of flow boxes $F_\alpha^i$, $(F_\alpha^i)^{-1}$ in laminar atlases for $\lambda_i$.
\end{definition}

We now define convergence of laminations equipped with transverse measures.

\begin{definition}
Let $\lambda$ be a lamination with atlas $A$.
A \dfn{transverse measure} to $\lambda$ consists of Radon measures $\mu_\alpha$ with $\supp \mu_\alpha = K_\alpha$, $\alpha \in A$, such that each transition map $\psi_{\alpha \beta}$ is measure-preserving:
$$\mu_\alpha|_{K_\alpha \cap K_\beta} = \psi_{\alpha \beta}^* (\mu_\beta|_{K_\alpha \cap K_\beta}).$$
The pair $(\lambda, \mu)$ is called a \dfn{measured lamination}.
\end{definition}

Caveat lector: we assume that $\supp \mu_\alpha = K_\alpha$, but in \cite{daskalopoulos2020transverse}, it is only assumed that $\supp \mu_\alpha \subseteq K_\alpha$.
In particular, not every lamination admits a transverse measure.

The definition of transverse measure in terms of Radon measures on $K_\alpha$ is convenient because $K_\alpha$ is compact.
However, the definition is not intrinsic, and this causes problems when considering questions of convergence: the fact that the flow boxes of a convergent sequence of measured laminations converge should be a consequence of, not a part of, the definition!

To rectify this, we first observe that in the definition of a transverse measure, we cannot define a transverse measure to be one on the underlying manifold $M$ itself.
Indeed, Lebesgue measure is ``transverse'' to all foliations; thus such a definition forgets the ``direction'' the measure points in.
However, the notion of Ruelle-Sullivan current allows us to speak of a measure-theoretic object on $M$ which has a well-defined local product structure.

\begin{definition}
A lamination is \dfn{oriented} if one can choose its transition maps to all be orientation-preserving.
\end{definition}

\begin{definition}
Let $(\lambda, \mu)$ be a measured oriented lamination and let $(\chi_\alpha)_{\alpha \in A}$ be a subordinate partition of unity.
The \dfn{Ruelle-Sullivan current} $T_\mu$ associated to $(\lambda, \mu)$ is defined for all compactly supported $d-1$-forms $\varphi$ by
\begin{equation}\label{RS current}
\int_M T_\mu \wedge \varphi := \sum_{\alpha \in A} \int_{K_\alpha} \left[\int_{\{k\} \times \RR^{d - 1}} (F_\alpha^{-1})^* (\chi_\alpha \varphi) \right] \dif \mu_\alpha(k).
\end{equation}
\end{definition}

It is clear that any lamination is locally orientable, so the next definition makes sense.

\begin{definition}
A sequence of measured laminations $(\lambda_i, \mu_i)$ \dfn{converges} to $(\lambda, \mu)$ if locally, their Ruelle-Sullivan currents $T_{\mu_i} \to T_\mu$ converge in the weak topology of measures.
\end{definition}

The convergence of Ruelle-Sullivan currents, which is very convenient to work with analytically, is equivalent to a definition of measure convergence that may be more familiar to topologists, namely convergence of the transverse measure along each transverse curve.

It is clear from the definitions that flow-box convergence implies Thurston convergence, and it is well-known that measure convergence implies Thurston convergence \cite[Proposition 8.10.3]{thurston1979geometry}.
We show that flow-box convergence actually sits in the middle of the chain of implications:

\begin{theorem}\label{implication theorem}
Suppose that $M$ has constant sectional curvature and $2 \leq d \leq 7$.
Let $(\lambda_n, \mu_n)$ be measured minimal laminations, and $(\lambda_n, \mu_n) \to (\lambda, \mu)$. Then $\lambda_n \to \lambda$ in the $C^{1-}$ flow box topology.
\end{theorem}

We also prove some compactness results for the above modes of convergence.
In general a sequence of laminations may escape to infinity or blow up in curvature, and we must rule these possibilities out.

\begin{definition}
A sequence $(\lambda_n)$ of laminations is \dfn{tight} if there exists a compact set $K \subseteq M$ such that for every $n$, $\supp \lambda_n$ intersects $K$.
The sequence has \dfn{bounded curvature} if there exists $C > 0$ such that for any $n$ and any leaf $N$ of $\lambda_n$, the second fundamental form satisfies $||\Two_N||_{C^0} \leq C$.
\end{definition}

We write $C^{1-}$ for the Fr\'echet space $\bigcap_{\alpha < 1} C^\alpha$, where $C^\alpha$ are H\"older spaces.

\begin{theorem}\label{compactness theorem}
Suppose that $M$ has constant sectional curvature and $2 \leq d \leq 7$.
Let $(\lambda_n)$ be a tight sequence of minimal laminations of locally bounded curvature.
Then a subsequence converges as flow boxes in $C^{1-}$, and in particular in Thurston's topology, to a minimal lamination.
Furthermore, if $\mu_n$ is transverse to $\lambda_n$ and for each compact set $K \Subset M$, $\mu_n(K) \lesssim_K 1$, then a further subsequence converges in the measure topology.
\end{theorem}



%%%%%%%%%%%%%%%%%%
\subsection{Best Lipschitz and least gradient maps}\label{FLG section}
If $M$ is a closed hyperbolic manifold, then the Euler-Lagrange equation for best Lipschitz maps $v: M \to \Sph^1$ is the $\infty$-Laplace equation \cite{daskalopoulos2020transverse}
\begin{equation}\label{infinity laplacian}
(\nabla^\mu \partial^\nu v) \partial_\mu v \partial_\nu v = 0.
\end{equation}
This equation is invariant under translations $v \mapsto v + y$, so by Noether's theorem, it has a conserved flux $\dif u$.
If $d = 2$, the associated conservation law is the $1$-Laplace equation.
We studied the $1$-Laplacian in the prequel paper \cite{BackusFLG}; here we recall the main result of that paper.

\begin{definition}
A function $u \in BV_\loc(M)$ has \dfn{least gradient}, or is \dfn{$1$-harmonic}, if for every $w \in BV_\cpt(M)$,
\begin{equation}\label{least gradient functional}
\int_M \star |\dif u| \leq \int_M \star |\dif u + \dif w|.
\end{equation}
\end{definition}

Here $\star |\dif u|$ is the total variation of the current $\dif u$; we refer to \S\ref{MeasurePrelims} for the precise definition.
The formal Euler-Lagrange equation for (\ref{least gradient functional}) is the $1$-Laplace equation
\begin{equation}\label{1Laplacian}
\dif^* \left(\frac{\dif u}{|\dif u|}\right) = 0.
\end{equation}
Formally, the $1$-Laplace equation (\ref{1Laplacian}) asserts that the level sets are indeed minimal, but since the derivation of the Euler-Lagrange equation for (\ref{least gradient functional}) is only formal, and the precise definition of weak solution \cite{Mazon14} does not directly imply that the level sets have zero mean curvature, this has to be checked separately.
This is the main result of the prequel paper \cite{BackusFLG}:

\begin{theorem}\label{main thm of old paper}
Let $M$ be a manifold of constant sectional curvature and $2 \leq d \leq 7$.
Then for every $1$-harmonic function $u: M \to \RR$ and $y \in \RR$, the level set $\partial \{u > y\}$ is an analytic embedded stable minimal hypersurface in $M$.
\end{theorem}
\begin{proof}[Proof sketch]
By a straightfoward modification of \cite[Theorem 1]{BOMBIERI1969}, the superlevel sets $\{u > y\}$ have least perimeter, that is their indicator functions have least gradient.
The regularity of boundaries of sets of least perimeter was established for $M = \RR^d$ by the classical work of de Giorgi and Miranda \cite{deGiorgi61, Miranda66} but their proof does not generalize nicely because it relies on the invariance of tangent vectors under parallel transport in order to define averages of normal vectors to sets of least perimeter.
In \cite[\S3]{BackusFLG} we establish a suitable method of taking averages of the normal vector provided that $M = \Hyp^d$ or $M = \Sph^d$.
This allows us to modify the relevant parts of \cite{Miranda66}.
See \cite[\S1]{BackusFLG} for a more detailed summary of \cite{BackusFLG}.
\end{proof}

We use Theorem \ref{main thm of old paper} as a regularity result at various points in this paper, which allows us to show that the limiting laminations have smooth leaves (rather than currents for leaves, say).
However, Theorem \ref{main thm of old paper} is of more interest to this paper rather than just as a regularity lemma; to motivate why, we recall the relationship between best Lipschitz maps, maps of least gradient, and geodesic laminations:

\begin{theorem}[Daskalopoulos--Uhlenbeck]\label{DU theorem}
Let $M$ be a closed hyperbolic surface and $v: M \to \Sph^1$ an $\infty$-harmonic function with conserved flux $\dif u$ and maximal stretch geodesic lamination $\lambda$.
Then any local primitive $u$ of $\dif u$ is a $1$-harmonic function, and $\dif u$ is Ruelle-Sullivan for $\lambda$.
Moreover, the level sets of $u$ are geodesics in $\lambda$.
\end{theorem}

We refer to the original paper \cite{daskalopoulos2020transverse} for a more precise statement.
Inspired by this theorem, Daskalopoulos--Uhlenbeck conjectured that for any $1$-harmonic function on $\Hyp^2$, $\dif u$ should be Ruelle-Sullivan for some (possibly not maximum-stretch) geodesic lamination \cite[Problem 9.4]{daskalopoulos2020transverse}, and conversely that if $T$ is a Ruelle-Sullivan current for some geodesic lamination, then local primitives of $T$ are $1$-harmonic \cite[Conjecture 9.5]{daskalopoulos2020transverse}.
Of course, if $d \geq 3$, then the level sets will be minimal hypersurfaces rather than geodesics.

Using Theorem \ref{building a minimal lamination}, the stable Bernstein theorem \cite{Schoen2016, Chodosh2021}, and Theorem \ref{main thm of old paper}, we prove the conjectures of Daskalopoulos--Uhlenbeck:

\begin{theorem}\label{main thm}
Let $2 \leq d \leq 4$ and suppose that $M$ has constant sectional curvature.
\begin{enumerate}
\item Let $u$ be a $1$-harmonic function.
Then $\bigcup_{y \in \RR} \partial \{u > y\}$ is the support of a minimal lamination $\lambda$ whose leaves are the connected components of the hypersurfaces $\partial \{u > y\}$, and $\dif u$ is a Ruelle-Sullivan current for $\lambda$.
\item Conversely, if $\lambda$ is a minimal lamination and $\dif u$ is a Ruelle-Sullivan current for $\lambda$, then $u$ is $1$-harmonic.
\end{enumerate}
\end{theorem}

However, Theorem \ref{main thm} leaves a key point -- the role of the $\infty$-Laplacian -- open, and we have not attempted to address this point here.
The Daskalopoulos--Uhlenbeck theorem was our main motivation for this series of papers; however, our work works in dimensions $d = 3, 4$ while Theorem \ref{DU theorem} is purely a statement about $d = 2$.
Indeed, the $\infty$-Laplacian gives geodesic laminations, but the $1$-Laplacian gives codimension-$1$ laminations, which are not geodesic if $d \geq 3$, so there is not an obvious generalization of Theorem \ref{DU theorem} for $d = 3, 4$.

It is tantalizing to think that a suitable system of coupled $\infty$-Laplacians will satisfy a generalized maximum-stretch condition on a codimension-$1$ minimal lamination $\lambda$, and that the conservation law for this conjectural system will be the $1$-Laplacian.
However, even if one was to derive such a system of $\infty$-Laplacians formally, the analysis for studying such a system would likely not be in place.
Indeed, the study of $\infty$-harmonic functions is almost entirely carried out in the language of viscosity solutions and comparison-with-cones, which only make sense when the target is $\RR$ rather than $\RR^m$ or a vector bundle.
Even so, we hope to return to this question in a later work: what is the correct generalization of the Daskalopoulos--Uhlenbeck theorem to $d = 3$?

%%%%%%%%%%%%%%%%%%%%%%%
\subsection{Outline of the paper}
In \S\ref{Prelims} we recall preliminaries.

In \S\ref{Regularity} we establish that every $C^0$ minimal lamination has a Lipschitz laminar atlas and a Lipschitz normal bundle.
We will use this regularity frequently through the remainder of the paper.

In \S\ref{CompactnessSec} we prove Theorem \ref{compactness theorem},
and use it to prove Theorems \ref{implication theorem} and \ref{building a minimal lamination}.
Then we use Theorem \ref{building a minimal lamination} to prove Theorem \ref{main thm}.

%%%%%%%%%%%%%%%%%%%%%%%%

\subsection{Acknowledgements}
I would like to thank Georgios Daskalopoulos for suggesting this project and for many helpful discussions.
I would also like to thank Chao Li for helpful comments.
\todo{This work was supported by an NSF Graduate Resarch Fellowship}



%%%%%%%%%%%%%%%%%%%%%%%%%%%

\section{Preliminaries}\label{Prelims}
\subsection{Notation and conventions}
The operator $\star$ is the Hodge star, thus $\star 1$ is the Riemannian measure.
If $U$ is an open set, we write $|U| := \int_U \star 1$ for the volume of $U$, but if $U$ is a submanifold or rectifiable set of positive codimension, we instead write $|U|$ for its surface measure.

For a map $F: X \to Y$ between metric spaces, we write $\Lip(F)$ for its Lipschitz constant.
If $X, Y$ are connected Riemannian manifolds, one of which is $1$-dimensional, then we have $\Lip(F) = ||\dif F||_{L^\infty}$.

We write $\normal_N$ for the normal vector (or conormal $1$-form) for a hypersurface $N$, $\nabla_N$ for the Levi-Civita connection, and $\Two_N := \nabla_N \normal_N$ for the second fundamental form.

We let $\Leaves \lambda$ denote the set of leaves of a lamination $\lambda$.

% \subsection{Laminations}
% \begin{definition}
% Let $\lambda$ be a lamination in $M$.
% \begin{enumerate}
% \item $\lambda$ is \dfn{discrete} if every leaf space of $\lambda$ is finite.
% \item $\lambda$ is \dfn{finite} if $\lambda$ is discrete and admits a finite laminar atlas.
% \item $\lambda$ is a \dfn{foliation} if $\supp \lambda = M$.
% \end{enumerate}
% \end{definition}

% Every finite lamination is discrete, and conversely, a discrete lamination in a closed manifold is finite.
% However, a lamination with finitely many leaves need not be finite, even if it is geodesic:

% \begin{example}\label{two geodesics}
% Let $M = \RR \times \Sph^1$ be a cylinder.
% We may choose a metric on $M$ so that there is a geodesic $\gamma_1$ which is a simple closed curve looping around $M$, and that there is another geodesic $\gamma_2$ which winds around $M$ infinitely many times, converging to $\gamma_1$.
% Then $\gamma_1, \gamma_2$ form a geodesic lamination with two leaves which is not finite.
% This example, and slight modifications thereof, will be frequently useful as a counterexample throughout this paper.
% \end{example}


%%%%%%%%%%%%%%%%
\subsection{Hausdorff distance}
In order to measure when two leaves are ``close'', we shall consider the Hausdorff distance on the space of leaves, as defined in \cite[Chapter 4]{nadler2017continuum}.

\begin{definition}
Let $X$ be a compact metric space. The \dfn{Hausdorff distance} between two nonempty closed sets $A, B \subset X$ is
\begin{equation}\label{compact hausdorff distance}
	\dist(A, B) := \max\left(\max_{a \in A} \min_{b \in B} \dist(a, b), \max_{b \in B} \min_{a \in A} \dist(a, b)\right).
\end{equation}
The space of closed subsets of $X$ is the \dfn{hyperspace} $\Hypspace X$.
\end{definition}

\begin{definition}
If $X$ is a metrizable space and $(A_n)$ is a sequence of closed subsets of $X$, we define
\begin{equation}\label{Hausdorff is the limit set}
\lim_{n \to \infty} A_n := \left \{ \lim_{n \to \infty} x_n: (x_n) \in \prod_{n=1}^\infty A_n\right\}.
\end{equation}
\end{definition}

\begin{proposition}\label{Hausdorff on a CMS}
Let $X$ be a compact metric space. Then:
\begin{enumerate}
\item The topology on $\Hypspace X$ is the topology for the convergence mode (\ref{Hausdorff is the limit set}).
\item The topology on $\Hypspace X$ is completely determined by the topology on $X$.
\item $\Hypspace X$ is a compact metric space.
\end{enumerate}
In particular, $\Hypspace$ is a self-map of the set of compact metrizable spaces.
\end{proposition}
\begin{proof}
By \cite[Theorem 4.11]{nadler2017continuum}, a sequence $(A_n)$ of closed subsets of $X$, the limit $A$ in $\Hypspace X$ of $A_n$ is the set of $x \in X$ such that for every $U \ni x$ open, there are $x_n \in A_n$ such that eventually $x_n \in U$.
But this is exactly the characterization of $\lim_n A_n$ given by (\ref{Hausdorff is the limit set}).
It is also independent of the metric on $X$, so as a metrizable space, $\Hypspace X$ is determined by the metrizable (rather than metric) structure on $X$.
The compactness now follows from \cite[Theorem 4.17]{nadler2017continuum}, and the fact that $\Hypspace$ is a self-map of the space of compact metrizable spaces follows.
\end{proof}

\begin{definition}
Let $M$ be a manifold. We define $\Hypspace_\loc(M)$ to be the set of closed subsets of $M$, equipped with the mode of convergence (\ref{Hausdorff is the limit set}).
\end{definition}

\begin{proposition}
For a manifold $M$, $\Hypspace_\loc(M)$ is a compact metrizable space.
\end{proposition}
\begin{proof}
Choose a Riemannian metric on $M$, and let $(X_m)$ be a compact exhaustion of $M$ such that $\diam(X_m) \leq 2^m$.
Let $\dist_m$ be the metric on $\Hypspace X_m$, and observe that $\diam(\Hypspace X_m) \leq \diam X_m \leq 2^m$.
For $A, B \in \Hypspace_\loc(M)$, we define
$$\dist(A, B) := \sum_{m=1}^\infty \frac{\dist_m(A \cap X_m, B \cap X_m)}{4^m}.$$
Here, we set $\dist_m(A \cap X_m, B \cap X_m) = 0$ if $A$ or $B$ does not intersect $X_m$.

By Proposition \ref{Hausdorff on a CMS}, $A_n \to A$ in $\Hypspace_\loc(M)$ iff for every $m$, $\dist_m(A_n \cap X_m, A \cap X_m) \to 0$.
It also holds that $\dist_m(A_n \cap X_m, A \cap X_m)/4^m \leq 2^{-m}$, so by dominated convergence,
$$\lim_{n \to \infty} \dist(A_n, A) = \sum_{m=1}^\infty \frac{1}{4^m} \lim_{n \to \infty} \dist_m(A_n \cap X_m, A \cap X_m).$$
Thus $A_n \to A$ in $\Hypspace_\loc(M)$ iff $\dist(A_n, A) \to 0$.
Thus $\dist_m$ is a metrization of $\Hypspace_\loc(M)$.

To see the compactness, let $(A_n)$ be a sequence in $\Hypspace_\loc(M)$.
By a diagonal argument, we may pass to a subsequence and find a closed set $A$ such that for every $m$, either $A \cap X_m$ is nonempty and $A_n \cap X_m \to A \cap X_m$, or $A \cap X_m$ is empty.
If there exists $m$ such that $A \cap X_m$ is nonempty, then the same holds for any index $\geq m$, and so $A_n \to A$ in $\Hypspace_\loc(M)$ and $A$ is nonempty.
Otherwise, $A$ is empty and for every $m$, $A_n \cap X_m$ is eventually empty.
But then there are no limit points of any sequence $(x_n) \in \prod_n A_n$ and so $A_n \to A$.
\end{proof}

Note carefully that we allow $\lim_n A_n$ to be empty in $\Hypspace_\loc(M)$. This accounts for the possibility of a mass escaping to infinity.

Since $\Hypspace_\loc(M)$ is a compact metrizable space, we may by Proposition \ref{Hausdorff on a CMS} define $\Hypspace^2_\loc(M) := \Hypspace(\Hypspace_\loc(M))$ to be the set of nonempty closed sets of closed subsets of $M$, which will also be a compact metrizable space.

\todo{Do we actually use the compactness of $\Hypspace^2_\loc(M)$?}

%%%%%%%%%%%%%%%%%%%%%
\subsection{Compactness in function spaces}
\begin{proposition}[Arzela-Ascoli theorem]\label{AA Holder}
Suppose that $(u_n)$ is a sequence of Lipschitz functions with bounded Lipschitz norms on a totally bounded metric space. Then there is a subsequence of $(u_n)$ which converges in $C^{1-}$ to a Lipschitz function.
\end{proposition}
\begin{proof}
By the classical Arzela-Ascoli theorem, along a subsequence we have $u_n \to u$ in $C^0$, where
$$|u(x) - u(y)| \leq \liminf_{n \to \infty} |u_n(x) - u_n(y)| + 2 ||u_n - u||_{C^0} \leq \liminf_{n \to \infty} \Lip(u_n)$$
and hence
$$\Lip(u) \leq \liminf_{n \to \infty} \Lip(u_n) < \infty.$$
Now let $0 < \alpha < 1$, and assume without loss of generality that $u = 0$, so that we must show that $||u_n||_{C^\alpha} \to 0$.
In fact
$$\frac{|u_n(x) - u_n(y)|}{\dist(x, y)^\alpha} = \left|\frac{u_n(x) - u_n(y)}{\dist(x, y)}\right|^\alpha \cdot |u_n(x) - u_n(y)|^\alpha \leq \Lip(u_n)^\alpha \cdot ||u_n||_{C^0}^{1 - \alpha}.$$
Since $u_n \to 0$ in $C^0$, and $(\Lip(u_n))$ is uniformly bounded, the claim follows.
\end{proof}


%%%%%%%%%%%%%%%%%%%%%
\subsection{Measure theory}\label{MeasurePrelims}
Let $X$ be a locally compact Polish space, and let $C_\cpt(X)$ be the space of compactly supported continuous functions $f: X \to \RR$.
Its dual $C_\cpt(X)'$ is canonically isomorphic to the space of signed Radon measures on $X$, where the bilinear pairing is given by integration.
The weak topology on $C_\cpt(X)'$ is known as the \dfn{weak topology of measures}.
Unpacking the definitions, a sequence $(\mu_n)$ of Radon measures converges to $\mu$ in the weak topology of measures iff for every compact $Y \subseteq X$ and every continuous function $f: Y \to \RR$,
$$\lim_{n \to \infty} \int_Y f \dif \mu_n = \int_Y f \dif \mu.$$
We shall frequently use the following characterization of weak convergence:
\todo{There are multiple definitions of weak limits of probability spaces, I need to check that this is compatible with the portmanteau theorem}

\begin{proposition}[portmanteau theorem]
	Let $(\mu_n)$ be a sequence of Radon measures on a locally compact Polish space $X$ with $\mu_n(X) \lesssim 1$, and let $\mu$ be a Radon measure on $X$. The following are equivalent:
\begin{enumerate}
	\item $\mu_n \to \mu$ in the weak topology of measures.
	\item $\liminf_{n \to \infty} \mu_n(X) \geq \mu(X)$ and for every closed $Y \subseteq X$, $\limsup_{n \to \infty} \mu_n(Y) \leq \mu(Y)$.
	\item $\limsup_{n \to \infty} \mu_n(X) \leq \mu(X)$ and for every open $Z \subseteq X$, $\liminf_{n \to \infty} \mu_n(Z) \geq \mu(Z)$.
	\item For every $W \subseteq X$ with $\mu(\partial W) = 0$, $\lim_{n \to \infty} \mu_n(W) = \mu(W)$.
\end{enumerate}
	If $X$ is a manifold, these conditions are equivalent to:
\begin{enumerate}
	\setcounter{enumi}{4}
	\item For every $x \in X$ and almost every $0 < \varepsilon \ll 1$, $\lim_{n \to \infty} \mu_n(B(x, \varepsilon)) = \mu(B(x, \varepsilon))$.
\end{enumerate}
\end{proposition}
\begin{proof}
	See \cite[Theorem 13.16]{klenke2013probability} for the metrizable case.
	For the manifold case, we just observe that almost every $\varepsilon > 0$ satisfies $\mu(\partial B(x, \varepsilon)) = 0$. Indeed, if not, then we can find a set of real numbers $A$ such that $0$ is a condensation point of $A$, and for every $\varepsilon \in A$, $\mu(\partial B(x, \varepsilon)) > 0$.
	In particular, for every $\delta > 0$, $A \cap (0, \delta)$ is uncountable, so
	$$\mu(B(x, \delta)) \geq \sum_{\varepsilon \in A \cap (0, \delta)} \mu(\partial B(x, \delta)) = \infty,$$
	but since $X$ is a manifold, if $\delta$ is small enough then $B(x, \delta)$ is precompact.
	This contradicts that $\mu$ is a Radon measure.
\end{proof}

If $X = M$ is a manifold, then we can consider instead the space $C_\cpt(M, \Omega^\ell)$ of compactly supported continuous $\ell$-forms.
An $\ell$-\dfn{current} is an element of the dual space $C_\cpt(M, \Omega^\ell)'$.\footnote{Strictly speaking, $C_\cpt(M, \Omega_\ell)'$ is the space of $\ell$-currents of locally finite total variation. However, we will never need to consider $\ell$-currents that do not have locally finite total variation, so we suppress this technicality.}
We denote the pairing of an $\ell$-current $T$ and an $\ell$-form $\varphi$ by $\int_M T \wedge \varphi$.
Any $d-\ell$-form $\psi$ gives rise to an $\ell$-current $T$, the \dfn{Poincar\'e dual} of $\psi$, by $\int_M T \wedge \varphi = \int_M \psi \wedge \varphi$.
In particular, the Poincar\'e dual of any function is a $d$-current.
See \cite{simon1983GMT} for more on the theory of currents.

Again we have the weak topology on the space of $\ell$-currents; we also have the \dfn{exterior derivative}
$$\int_M \dif T \wedge \psi := -\int_M T \wedge \dif \psi$$
defined for any $\ell$-current $T$ such that $\psi \mapsto \int_M T \wedge \dif \psi$ extends from $C^1_\cpt(M, \Omega^{\ell - 1})$ to an $\ell - 1$-current.

With this machinery in place, we can talk about Ruelle-Sullivan currents and their limits.

\begin{lemma}
Let $(\lambda, \mu)$ be a measured oriented lamination.
Then the Ruelle-Sullivan current $T_\mu$ is well-defined; it is honestly a $d-1$-current, and does not depend on the choice of partition of unity.
Moreover, $\dif T_\mu = 0$.
\end{lemma}
\begin{proof}
We first claim that the right-hand side of (\ref{RS current}) is always finite, and is continuous in $\varphi$.
In fact, possibly after refining $(\chi_\alpha)$, we may assume that it is a locally finite partition of unity.
In particular, we just need to check the continuity in a single flow box:
$$\left|\int_{K_\alpha} \left[\int_{\RR^{d - 1} \times \{k\}} (F_\alpha^{-1})^* (\chi_\alpha \varphi) \right] \dif \mu_\alpha(k)\right| \leq \int_{K_\alpha} \int_{\RR^{d - 1} \times \{k\}} |(F_\alpha^{-1})^* (\chi_\alpha \varphi)| \dif \mu_\alpha(k).$$
The inner integral is controlled by $||\varphi||_{C^0(U_\alpha)} \cdot |U_\alpha|$ where $U_\alpha$ is the image of $F_\alpha$.
The outer integral is then well-defined because it is against a Radon measure.

We next observe that the choice of partition of unity is irrelevant, thus if $\varphi$ has compact support in $U_\alpha \cap U_\beta$, then
\begin{equation}\label{well-defined of Ruelle-Sullivan}
\int_{K_\alpha} \int_{\RR^{d - 1} \times \{k\}} (F_\alpha^{-1})^* \varphi \dif \mu_\alpha(k) = \int_{K_\beta} \int_{\RR^{d - 1} \times \{k\}} (F_\beta^{-1})^* \varphi \dif \mu_\beta(k).
\end{equation}
Indeed,
\begin{align*}
\int_{K_\alpha} \int_{\RR^{d - 1} \times \{k\}} (F_\alpha^{-1})^* \varphi \dif \mu_\alpha(k)
&= \int_{K_\beta} (F_\alpha F_\beta^{-1})^* \left[\int_{\RR^{d - 1} \times \{k\}} (F_\alpha^{-1})^* \varphi \dif \mu_\alpha(k)\right] \\
&= \int_{K_\beta} \left[\int_{\RR^{d - 1} \times \{k\}} (F_\beta^{-1})^* \varphi\right] (F_\alpha F_\beta^{-1})^* \dif \mu_\beta(k) \\
&= \int_{K_\beta} \int_{\RR^{d - 1} \times \{k\}} (F_\beta^{-1})^* \varphi \dif \mu_\beta(k)
\end{align*}
where the last equation is because of the measure-preserving nature of the transition maps; this proves (\ref{well-defined of Ruelle-Sullivan}).

Finally, if a $d-2$-form $\psi$ has compact support in a single flow box, then
$$\int_{\RR^{d - 1} \times \{k\}} (F_\alpha^{-1})^* \dif \psi = \int_{\RR^{d - 1} \times \{k\}} \dif((F_\alpha^{-1})^* \psi) = 0$$
by Stokes' theorem, so
\begin{align*}
\int_M \dif T_\mu \wedge \psi &= -\int_M T_\mu \wedge \dif \psi \\
&= -\int_{K_\alpha} \int_{\RR^{d - 1} \times \{k\}} (F_\alpha^{-1})^* \dif \psi \dif \mu_\alpha(k) = 0. \qedhere
\end{align*}
\end{proof}

Though (\ref{RS current}) is the more traditional way of stating the definition of a Ruelle-Sullivan current, there is a more intrinsic way as well.
We first observe that if $\mu$ is a transverse measure, then $\mu$ defines a measure on $\supp \lambda$: in each flow box $F_\alpha$, an open set $U$ has measure
\begin{equation}\label{transverse measure of an open set}
\mu(U) := \int_{K_\alpha} |F_\alpha(\RR^{d - 1} \times \{k\}) \cap U| \dif \mu_\alpha(k).
\end{equation}

\begin{lemma}
For an oriented measured lamination $(\lambda, \mu)$, the polar decomposition of $T_\mu$ is
\begin{equation}\label{polar ruelle sullivan}
T_\mu = \normal_\lambda \mu.
\end{equation}
\end{lemma}
\begin{proof}
For an open set $U \subseteq M$ in a flow box $F_\alpha$, the total variation measure $|T_\mu|$ satisfies
$$|T_\mu|(U) = \sup_{||\varphi||_{C^0} \leq 1} \int_{K_\alpha} \int_{\RR^{d - 1} \times \{k\}} \varphi \dif \mu_\alpha(k)$$
where the supremum ranges over $d-1$-forms $\varphi$ with compact support in $U$.
However, $\star \normal_\lambda$ is the Riemannian measure on $F_\alpha(\RR^{d - 1} \times \{k\})$, so
$$\int_{\RR^{d - 1} \times \{k\}} \varphi \leq \int_{\RR^{d - 1} \times \{k\}} (F_\alpha^{-1})^*(\star \normal_\lambda).$$
Since $||\normal^\lambda||_{C^0} = 1$, it follows that a sequence of cutoffs of $\star \normal_\lambda$ to more and more of $U$ is a maximizing sequence.
Therefore $\normal_\lambda$ is the polar part of (\ref{polar ruelle sullivan}), and
$$|T_\mu|(U) = \int_{K_\alpha} \int_{\RR^{d - 1} \times \{k\}} (F_\alpha^{-1})^*(1_U \star \normal_\lambda) \dif \mu_\alpha(k).$$
The inner integral is the Riemannian measure of $F_\alpha(\RR^{d - 1} \times \{k\}) \cap U$, so by (\ref{transverse measure of an open set}), $|T_\mu| = \mu$.
\end{proof}

%%%%%%%%%%%%%%%%%%%%%%%%%%%%
\subsection{Nonorientable laminations}\label{nonorientable section}
Throughout \S\ref{CompactnessSec} we shall assume that the lamination $\lambda$ is oriented, so that we may consider the Ruelle-Sullivan current $T_\mu$ associated to a transverse measure $\mu$ to $\lambda$.
However, this assumption may be dropped by passing to the orientable double cover of $\lambda$; \todo{see George's new paper for more about this}

%%%%%%%%%%%%%%%%%%%%%%%%%%%%
\subsection{Functions of least gradient}
Here we recall some facts about functions of least gradient.

\begin{proposition}[Miranda stability theorem]
	If a sequence of functions $(u_n)$ (not necessarily of the same trace) is bounded in $L^1_\loc(M)$ and satisfies for every open $U \Subset M$ with Lipschitz boundary
\begin{equation}\label{boundedness in Miranda}
	\limsup_{n \to \infty} \int_U \star |\dif u_n| \leq \liminf_{n \to \infty} \inf_{v \in BV_\cpt(U)} \int_U \star |\dif(u_n + v)| < \infty,
\end{equation}
	then there exists a function $u$ of least gradient such that along a subsequence, $u_n \to u$ in $L^1_\loc(M)$ and $\dif u_n \to \dif u$ in the weak topology of measures.
\end{proposition}
\begin{proof}
The forgetful map $BV_\loc(M) \to L^1_\loc(M)$ is compact, so (\ref{boundedness in Miranda}) and the bounds in $L^1_\loc(M)$ imply that $(u_n)$ has a convergent subsequence.
The rest of the proof is similar to \cite[Teorema 3 and Osservazione 3]{Miranda67}; see \todo{\cite{BackusFLG}} for the straightforward modifications.
\end{proof}

\begin{proposition}[maximum principle]\label{max princip}
Let $u$ be a $1$-harmonic function.
If $u$ attains a local maximum, then $u$ is constant.
\end{proposition}
\begin{proof}
Let $y$ be a local maximum of $u$; then $\partial \{u \geq y\}$ is a minimal hypersurface by Theorem \ref{main thm of old paper}, or else it is empty. If it is empty, then $M = \{u \geq y\}$ and $u$ is constant, so we exclude this case.
In a neighborhood of any point of $\partial \{u \geq y\}$ we may decrease $\int \star |\dif u|$ by decreasing $y$ a small amount, which violates that $u$ has least gradient.
\end{proof}

%%%%%%%%%%%%%%%%%%%%%%%%%%%%%%%%%%%%%%%%%%
\section{Regularity of flow boxes}\label{Regularity}
The goal of this section is to prove the following regularity theorem for minimal laminations that we will use several times.
The proof is based on \cite[Theorem 1.1]{Solomon86}, which addresses the case that $\lambda$ is a minimal foliation, and does not explictly spell out the $W^{1, \infty}$ norm and conorm of the laminar flow box.
Lipschitz regularity is optimal even for the nicest case of a geodesic foliation of $\Hyp^2$ \cite[\S1]{Solomon86}, so this result is sharp.

\begin{proposition}\label{regularity theorem}
Let $\lambda$ be a minimal lamination in $M$. Then:
\begin{enumerate}
\item There exists a Lipschitz subbundle of $TM$ which restricts to a normal bundle to each of leaves of $\lambda$.
\item There exist $L, r > 0$ which only depend on $g$ and $\sup_{N \in \Leaves \lambda} ||\Two_N||_{C^0}$, and a Lipschitz laminar atlas $(F_\alpha)$ for $\lambda$, such that for every $\alpha$,
\begin{equation}\label{conorm of flow box}
	\max(\Lip(F_\alpha), \Lip(F_\alpha^{-1})) \leq L,
\end{equation}
and the image of $F_\alpha$ contains a ball of size $r$.
\end{enumerate}
\end{proposition}

To begin the proof we first consider when we can represent the leaves of $\lambda$ as graphs in a uniform way.

\begin{lemma}\label{existence of tubes}
	Let $N$ be a connected hypersurface in $\RR^d = \RR^{d - 1}_x \times \RR_y$ which is tangent to $\{y = 0\}$ at the origin.
	If $||\Two_N||_{C^0} \leq \log(5/4)$, then for every $(x, y) \in N \cap B_1$,
	$$\max(|y|, 0.5 \cdot |\normal_N(x, y) - \partial_y|) \leq ||\Two_N||_{C^0}.$$
\end{lemma}
\begin{proof}
	Near $0$, $N$ can be represented a graph $\{y = f(x)\}$, since it is tangent to $\{y = 0\}$.
	This representation is valid on the component of the set $\{|\nabla f(x)| < \infty\}$ containing $0$, and it is related to the unit normal by
\begin{equation}\label{nabla as a normal}
	\nabla f(x) = \frac{\partial_y - \normal(x, f(x))}{\sqrt{1 + |\nabla f(x)|^2}}.
\end{equation}
	Taking derivatives of both sides,
	$$-\nabla^2 f(x) = \frac{\nabla n(x, f(x)) \cdot (\partial_x \otimes \partial_x + \nabla f(x) \otimes \partial_y)}{\sqrt{1 + |\nabla f(x)|^2}} + (\partial_y - \normal(x, f(x))) \cdot \frac{\nabla^2 f(x) \cdot \nabla f(x)}{(1 + |\nabla f|^2)^{3/2}}.$$
	Here $-\nabla^2$ denotes the negative Hessian, not the Laplacian.
	We can use (\ref{nabla as a normal}) to bound
	$$|\partial_y - \normal(x, f(x))| \leq |\nabla f(x)|\sqrt{1 + |\nabla f(x)|^2} \leq |\nabla f(x)| + |\nabla f(x)|^2.$$
	In particular,
	$$|\nabla^2 f(x)| \leq |\Two_N(x, f(x))| + |\nabla^2 f(x)| (|\nabla f(x)|^2 + |\nabla f(x)|^3).$$
	We now make the bootstrap assumption $|\nabla f(x)| \leq 1/4$, which is at least valid in some small neighborhood of $0$ since (\ref{nabla as a normal}) and the fact that $N$ is tangent to $\{y = 0\}$ at $0$ imply that $\nabla f(x) = 0$. Under that assumption, we have
	$$|\nabla^2 f(x)| \leq \frac{||\Two_N||_{C^0}}{1 - |\nabla f(x)|^2 - |\nabla f(x)|^3} \leq ||\Two_N||_{C^0} \cdot (1 + |\nabla f(x)|).$$
	If we write $x = (r, \theta)$, then by the mean value theorem, it follows that
\begin{align*}
	|\nabla f(x)| &\leq ||\Two_N||_{C^0} \int_0^r 1 + |\nabla f(s, \theta)| \dif s \\
	&= r||\Two_N||_{C^0} + \int_0^r |\nabla f(s, \theta)| \dif(||\Two_N||_{C^0} s).
\end{align*}
	So by Gr\"onwall's inequality and the fact that $||\Two_N||_{C^0} \leq \log(5/4)$ and $r < 1$,
\begin{align*}
	|\nabla f(x)| &\leq r ||\Two_N||_{C^0} + ||\Two_N||_{C^0}^2 \int_0^r s \exp(||\Two_N||_{C^0} \cdot (r - s)) \dif s \\
	&= r ||\Two_N||_{C^0} + \exp(r ||\Two_N||_{C^0}) - r ||\Two_N||_{C^0} - 1 \\
	&= \exp(r ||\Two_N||_{C^0}) - 1 \leq 0.25.
\end{align*}
	In particular, we recover the bootstrap assumption, and by the mean value theorem again,
\begin{align*}
	|f(x)| &\leq \int_0^r |\nabla f(s, \theta)| \dif s \leq \int_0^r \exp(s ||\Two_N||_{C^0}) - 1 \dif s \\
	&= ||\Two_N||_{C^0}^{-1} (\exp(r ||\Two_N||_{C^0}) - r||\Two_N|| - 1) \\
	&\leq r^2 ||\Two_N||_{C^0}
\end{align*}
	where we again used the bound $||\Two_N||_{C^0} \leq \log(5/4) < 0.25$ to control the higher-order terms in the Taylor expansion of $\exp(r ||\Two_N||_{C^0})$. Since $r < 1$ was arbitrary we conclude $||f||_{C^0} \leq 1$. Moreover, by (\ref{nabla as a normal}),
\begin{align*}
	|\normal(x, f(x)) - \partial_y| &\leq |\nabla f(x)| \leq 2r ||\Two_N||_{C^0}. \qedhere
\end{align*}
\end{proof}

\begin{lemma}\label{lams have C0 fields}
	Let $\lambda$ be a minimal lamination.
	For every $\delta > 0$ there exist
\begin{equation}\label{dependency of r}
	r = r\left(\delta, g, \sup_{N \in \Leaves \lambda} ||\Two_N||_{C^0}\right) > 0
\end{equation}
	and normal coordinates $(x, y) \in \RR^{d - 1} \times \RR$ on $B(P, r)$ so that
\begin{equation}\label{normal is basically dy}
	||\normal_\lambda - \partial_y||_{C^0(B(P, r))} \leq \delta.
\end{equation}
\end{lemma}
\begin{proof}
	Suppose not.
	Then, after rescaling, there exist minimal laminations $\lambda_n'$ such that
	$$\sup_{N \in \Leaves \lambda_n'} ||\Two_N||_{C^0} \leq 1,$$
	but no matter how we rotate normal coordinates based at $P$, (\ref{normal is basically dy}) fails for $\lambda_n'$.

\begin{sublemma}
	For each $n \geq \inj(g)$ there exists a lamination $\lambda_n$ in $\RR^d$ so that every rotation of $B_1$ fails (\ref{normal is basically dy}), but satisfies
\begin{equation}\label{bounds on Two in representation}
	\sup_{N \in \Leaves \lambda_n} ||\Two_N||_{C^0} \lesssim_g \frac{1}{n^2}.
\end{equation}
\end{sublemma}

We remark carefully that the lamination in the above sublemma may not be minimal.
However this fact will be irrelevant.

\begin{proof}
	Since $n \geq \inj(g)$ we can pass to the tangent space and rescale $B(P, 1/n)$ to obtain $B_1$. The scaling makes the second fundamental form $\Two_N'$ with respect to $g$ satisfy
$$||\Two_N'||_{C^0} \leq \frac{e^{-O(n^2)}}{n^2} \lesssim \frac{1}{n^2}.$$
	Here and always all implied constants may depend on $g$.
	In normal coordinates,
	$$g_{\mu \nu} = \delta_{\mu \nu} + \sum_{k=2}^\infty O(|n|^{-k}) |x|^k,$$
	and differentiating this expression gives that the connection $1$-form $\Gamma$ for the Levi-Civita connection $\nabla_g$ satisfies $|\Gamma| \lesssim n^{-2}$.
	But $||\Two_N - \Two_N'||_{C^0} \lesssim ||\Gamma||_{C^0}$, since $\Two_N - \Two_N'$ is the difference between the euclidean and $\nabla_g$ derivatives of the conormal.
\end{proof}

	By Lemma \ref{bounds on Two in representation}, every leaf of $\lambda_n$ is $O(n^{-2})$-close to its tangent spaces in $C^1$.
	In particular, if $n \gg \delta^{-1/2}$, and the normal vector at some point to $N \in \Leaves \lambda_n$ is $\partial_y$, then
	$$||\normal_N - \partial_y||_{C^0(B(P, r))} < \frac{\delta}{10}.$$
	We can always impose this for some $N$ by applying a rotation.
	But by our contradiction assumption, there exists some leaf $N'$ and some $P \in N'$ so that $|\normal_{N'}(P) - \partial_y| \geq \delta$ and hence
	$$\inf_{N'} |\normal_{N'} - \partial_y| \geq \frac{9}{10}\delta.$$
	By the reverse triangle inequality it follows that
	$$\inf_{\substack{P \in N\\ P' \in N'}} \sin \angle(\normal_N(P), \normal_{N'}(P')) \geq \frac{\delta}{5},$$
	at least if $\delta$ is smaller than an absolute constant.
	Suppose that $N, N'$ both meet $B_{\delta/50}$, say at $P, P'$. Then $T_PN, T_{P'} N'$ meet in $B_{\delta/10}$, but $N, N'$ stay $o(\delta)$-close to $T_PN, T_PN'$, so $N, N'$ meet in $B_\delta$. This is a contradiction since they are leaves of the same lamination.
\end{proof}

We should point out that \cite{Solomon86} proves a variant of Lemma \ref{lams have C0 fields} (without the quantitative dependence) by a very different means, using the regularity theory for integral flat convergence of minimal currents \cite[Theorem 5.3.14]{federer2014geometric}.
We could do this as well, but it did not seem particularly easy to recover quantitative bounds from the highly general theory of \cite[Chapter 5]{federer2014geometric}, so we found it easier to reprove the above lemma from scratch.

\begin{proof}[Proof of Proposition \ref{regularity theorem}]
Fix $\delta > 0$ to be chosen later, and $P \in M$.
By Lemma \ref{lams have C0 fields}, there exists $r > 0$ with dependencies (\ref{dependency of r}) such that $B(P, r)$ admits cylindrical coordinates $(x, y) \in 3\Ball^{d - 1} \times (-2, 2)$, and
\begin{equation}\label{normal is almost constant}
||\normal - \partial_y||_{C^0(B(P, r))} < \delta.
\end{equation}
If $\delta$ is chosen small enough depending on $g$, then we may assume that in $2\Ball^{d - 1} \times (-1, 1)$,
every leaf is the graph of a function, say $u_k: 3\Ball^{d - 1} \to (-2, 2)$ where $u_k(0) = k$.
Then $u_k$ solves a quasilinear elliptic PDE $Pu_k = 0$, so that if $r$ is chosen small enough depending on $g$ and $\sup_{N \in \Leaves \lambda} ||\Two_N||_{C^0}$, then the ellipticity of $P$ and H\"older norms of the coefficients of $P$ only depend on $g$, but not on $k$ or $\delta$.
So by a straightforward modification of \cite[Corollary 16.7]{gilbarg2015elliptic}, for every $m \geq 0$ there exist $C_m = C_m(g) > 0$ such that
\begin{equation}\label{norms on uk}
\sup_{|k| < 1} ||u_k||_{C^m} \leq C_m.
\end{equation}
Now for $k \in (-1, 1)$ fixed, let $k < \ell < 1$, and let $v_{\ell k} := u_\ell - u_k$.
Then $v_{\ell k}$ is the difference of two elements of the kernel of $P$, so $v_{\ell k}$ solves a linear elliptic PDE $Q_k v_{\ell k} = 0$, where the ellipticity of $Q_k$ and the H\"older norms of the coefficients of $Q_k$ only depend on $g$ and $C_m$ for some $m$, but not on $\ell, k, \delta$.
Moreover, $v_{\ell k} > 0$: clearly $v_{\ell k} \geq 0$, and if $v_{\ell k}(x) = 0$ for some $x$, then $v_{\ell k} = 0$ by the maximum principle, which implies $k = \ell$, a contradiction.
By the Schauder \cite[Theorem 6.2]{gilbarg2015elliptic} and Harnack \cite[Theorem 9.25]{gilbarg2015elliptic} inequalities, for every $x \in \Ball^{d - 1}$,
\begin{equation}\label{Schauder Harnack}
	||\dif v_{\ell k}||_{C^0(\Ball^{d - 1})} \lesssim_{C_m, g} ||v_{\ell k}||_{C^0(2 \Ball^{d - 1})} \lesssim_{C_m, g} \inf_{C^0(\Ball^{d - 1})} v_{\ell k} \leq v_{\ell k}(x).
\end{equation}
In particular, for every $x$,
$$|\dif u_\ell(x) - \dif u_k(x)| \lesssim_{C_m, g} |u_\ell(x) - u_k(x)|$$
so there exists $C = C'(C_m, g)$ such that
\begin{equation}\label{vertical Lipschitz}
|\normal(x, u_\ell(x)) - \normal_k(x, u_k(x))| \leq C |u_\ell(x) - u_k(x)|.
\end{equation}

To extend (\ref{vertical Lipschitz}) to a Lipschitz bound on $\normal$, let $X_1, X_2 \in (\Ball^{d - 1} \times (-1, 1)) \cap \supp \lambda$.
Then there exist $x_1, x_2 \in \Ball^{d - 1}$ and $k_1, k_2 \in (-1, 1)$ such that $X_i = (x_i, u_{k_i}(x_i))$.
Setting $Y := (x_2, u_{k_1}(x_2))$,
$$|\normal(X_1) - \normal(X_2)| \leq |\normal(X_1) - \normal(Y)| + |\normal(Y) - \normal(X_2)|.$$
Then by (\ref{norms on uk}) and the mean value theorem,
$$|\normal(X_1) - \normal(Y)| \lesssim |\dif u_{k_1}(x_1) - \dif u_{k_1}(x_2)| \leq C_2 |X_1 - Y|.$$
Moreover, by (\ref{vertical Lipschitz}),
$$|\normal(Y) - \normal(X_2)| \leq C|u_{k_1}(x) - u_{k_2}(x)| = C|Y - X_2|.$$
If $\delta$ is chosen smaller than some absolute constant, then by (\ref{normal is almost constant}),
$$|\sin \angle(X_1 - Y, X_2 - Y)| > 1 - O(\delta)$$
and we conclude by the Pythagorean theorem that
$$|Y - X_2|^2 + |X_1 - Y|^2 \lesssim |X_1 - X_2|^2.$$
In conclusion,
$$|\normal(X_1) - \normal(X_2)| \lesssim_g |X_1 - X_2|$$
which implies that $\normal$ is Lipschitz on $V \cap \supp \lambda$, where $V$ is the neighborhood of $P$ which was mapped to $\Ball^{d - 1} \times (-1, 1)$ by the cylindrical coordinates $(x, y)$.
In particular, $V$ contains a ball of the form $B(P, s)$, where $s$ only depends on $r$.
Note that $r$ only depends on $g$ and $\sup_{N \in \Leaves \lambda} ||\Two_N||_{C^0}$.
Moreover, $|X_1 - X_2| \sim r \dist(X_1, X_2)$ by definition of the coordinates $(x, y)$, so we obtain
\begin{equation}\label{lipschitz normal}
	|\Lip(\normal)| \lesssim r^{-1}.
\end{equation}
Taking a Lipschitz extension of $\normal$ to $V$ we obtain the desired Lipschitz normal subbundle.

Following \cite[Appendix B]{ColdingMinicozziIV}, we construct the laminar flow box
\begin{align*}
	F: \RR^{d - 1}_\xi \times \RR_\eta &\to V \subseteq \RR^{d - 1}_x \times \RR_y \\
	(\xi, \eta) &\mapsto (\xi, f(\xi, \eta))
\end{align*}
by setting
$$f(\xi, \eta) := u_\eta(\xi)$$
if $u_\eta$ exists, and if $k < \eta < \ell$ and there does not $k < \eta' < \ell$ such that $u_{\eta'}$ exists, then
$$f(\xi, \eta) := u_k(\xi) + \frac{\eta - k}{\ell - k} v_{\ell k}(\xi)$$
is the linear interpolant of $u_k$ and $u_\ell$.
Then it suffices to estimate $||\dif f||_{L^\infty}$ and $||\dif f^{-1}||_{L^\infty}$ in order to control $\Lip(F)$ and $\Lip(F^{-1})$.

First, in the interpolated region $k < \eta < \ell$,
$$\dif f = \dif u_k + \frac{\eta - k}{\ell - k} \dif v_{\ell k} + \frac{v_{\ell k}}{\ell - k} \dif \eta.$$
From (\ref{norms on uk}), $|\dif u_k| \leq C_1 \lesssim 1$, and from (\ref{Schauder Harnack}) with $x = 0$,
$$\max(|v_{\ell, k}|, |\dif v_{\ell k}|) \lesssim \ell - k;$$
we conclude that on the interpolated regions, $|\dif f| \lesssim 1$.
In the other direction, since $\dif u_k$ and $\dif v_{\ell k}$ are orthogonal to $\dif \eta$, (\ref{Schauder Harnack}) gives
$$|\dif f| \geq \inf \frac{v_{\ell k}}{\ell - k} \gtrsim \frac{v_{\ell k}(0)}{\ell - k} = 1.$$

We now turn to the foliated region. If $u_\eta$ exists, then
$$\dif f = \dif u_\eta + \frac{\partial u_\eta}{\partial \eta} \dif \eta.$$
We again have $|\dif u_\eta| \lesssim 1$, and from (\ref{Schauder Harnack}) the second term is $\lesssim 1$ as well, hence $|\dif f| \lesssim 1$.
Moreover,
$$|\dif f| \geq \left|\frac{\partial u_\eta}{\partial \eta}\right| \geq \inf_{k < \ell} \frac{v_{\ell k}(x)}{\ell - k}.$$
By (\ref{Schauder Harnack}),
\begin{align*}
	v_{\ell k}(\xi) &\geq v_{\ell k}(0) - ||\dif v_{\ell k}||_{C^0} \cdot |\xi| \\
	&\geq \ell - k - O(v_{\ell k}(0)) |\xi| \\
	&\geq (1 - O(|\xi|)) (\ell - k).
\end{align*}
Thus for $|\xi|$ smaller than an absolute constant, $|\dif f| \geq 1/2$.
This implies that $F$ (or rather, the composition of $F$ with the change of coordinates at the start of this proof) is a laminar flow box in a small neighborhood of $(0, 0)$ whose image has radius $cr$ for some small $c > 0$, and whose Lipschitz constants are comparable to $O(r^{-1})$.
\end{proof}

%%%%%%%%%%%%%%%%%%%%%%%%%%%%%%%%%%%%%%%%%
\section{Proofs of main theorems}\label{CompactnessSec}
In this section we prove Theorems \ref{compactness theorem} and \ref{implication theorem}.
Recall that throughout this section we assume that $\lambda$ is oriented, c.f. \S\ref{nonorientable section}.

\subsection{Compactness}
Convergence in Thurston's geometric topology can be annoying to check, as it requires one to check the convergence of the normal vectors as well as the leaves themselves.
Here we eliminate the need to check convergence of the normal vectors under certain hypotheses.

\begin{lemma}\label{convergence of geodesic lams in thurston}
Assume that $(\lambda_i)$ is a sequence of minimal laminations, and $\lambda$ a minimal lamination, such that for every leaf $N$ of $\lambda$ and every large $i$, there is a leaf $N_i$ of $\lambda_i$ such that $N_i \to N$ in $\Hypspace M$.
Furthermore, suppose that $(\lambda_i)$ has bounded curvature.
Then $\lambda_i \to \lambda$ in Thurston's geometric topology.
\end{lemma}
\begin{proof}
Suppose not. Then, there exists $P \in N$ such that for every $P_i \in N_i$ such that $P_i \to P$, $\normal_i := \normal_{N_i}(P_i)$ does not converge to $\normal := \normal_N(P)$ in the cosphere bundle $S'M$.
In normal coordinates $(x, y')$ based at $P$, $N$ can be viewed as the graph of a function $\{y' = w(x)\}$; we then set $y := y' - w(x)$, so that $N = \{y = 0\}$.
Using the coordinates $(x, y)$ we identify $M$, and each of its cotangent spaces, with $\RR^d$.

Possibly after taking a subsequence of $(\lambda_i)$, we may choose $\varepsilon > 0$ such that no matter what sequence $(P_i)$ we choose,
\begin{equation}\label{thurston angle defect}
	|\sin \angle(\normal_i, \normal)| \geq \varepsilon.
\end{equation}
The second fundamental form $\Two_i'$ of $N_i$ with respect to the euclidean metric on $\RR^d$ is bounded in terms of the curvature tensors of $M$ and $N$, and the second fundamental form $\Two_i$ of $N_i$ with respect to $M$, in $B(P, 2\varepsilon)$, as long as $\varepsilon$ is small.
But $\Two_i$ is uniformly bounded since $(\lambda_i)$ has bounded curvature. \todo{Spell this out. What does ``sin'' even mean in this context?}
The resulting uniform bound on $\Two_i'$, combined with (\ref{thurston angle defect}), furnishes a small $\delta > 0$ independent of $i$ such that $N_i$ \dfn{avoids cones} in the sense that for every $v \in \RR^d$ such that $0 < |v| < \delta$ and $|\sin v| < \varepsilon/2$, $P_i + v \notin N_i$.
After shrinking $\varepsilon$, we may assume that $\varepsilon < \max(\delta/2, 1/100)$.

To obtain a contradiction, we choose $Q \in N$ such that $\dist(P, Q) = \varepsilon$, which can be done as long as $\varepsilon$ is small.
For $i$ large, $|P_i| < \varepsilon^2/10$, so
$$0 < \varepsilon - \varepsilon^2 \leq |Q - P_i| < 2\varepsilon < \delta$$
and hence (using a superscript $j$ to indicate the $j$th coordinate)
$$|y^1 - x_i^1| \geq |y - x_i| - |x_i^2| \geq \varepsilon - 2\varepsilon^2.$$
Consider the triangle $\Delta$ whose vertices are $x_i, y$, and $z_i := (y^1, x_i^2)$.
Then $\Delta$ is a right triangle, and its smallest angle $\theta_\Delta$ satisfies
$$|\sin \theta_\Delta| = \frac{|x_i^2|}{|y^1 - x_i^1|} < \frac{\varepsilon^2}{\varepsilon - 2\varepsilon^2} < \frac{\varepsilon}{4}.$$
Thus for every large $i$, $y$ is contained in the cone avoided by $N_i$, and in fact for any $y_i \in N_i$, $\dist(y, y_i) \geq \varepsilon/4$.
But we argued above that any point of $N$ could be approximated by points of $N_i$ for $i$ large, so this is a contradiction.
\end{proof}

\begin{lemma}\label{limit of minimals is minimal}
Let $(N_n)$ be a sequence of complete embedded minimal hypersurfaces of uniformly bounded curvature and $N_n \to N$ in $\Hypspace M$.
Then $N$ is a complete minimal hypersurface.
\end{lemma}
\begin{proof}
In a neighborhood of any point on $N$, we can write $N_n = \partial U_n$ where $U_n$ is a set of least perimeter.
Let $u_n := 1_{U_n}$; then $u_n$ is a $1$-harmonic function.

If we set $N = \partial U$, $u := 1_U$, then we claim that $u_n \to u$ almost everywhere.
To prove this, we use Lemma \ref{lams have C0 fields} to write $U_n = \{y < f_n(x)\}$ for some sequence $(f_n)$ of smooth functions and some coordinates $(x, y)$.
Similarly we write $U = \{y < f(x)\}$.
(If this is not possible in a small neighborhood of some point $P \in N$, then $N$ must be almost orthogonal to $N_n$ at $P$, which violates the assumption of convergence.)
Now if $(x, y) \in U$ but $(x, y) \notin U_n$ for arbitrarily large $n$, then $f_n(x) \leq y < f(x)$ so $(x, f_n(x)) \in N_i$ converges to a point $(x, \tilde y)$ with $\tilde y \neq y$, thus $(x, \tilde y) \notin N$, which is impossible.
The same argument works for the complement of $U$, and shows that $u_n \to u$ pointwise on $M \setminus N$, which proves the claim.

So by dominated convergence, $u_n \to u$ in $L^1_\loc(M)$.
By \todo{\cite{BackusFLG}} and the fact that $(u_n)$ has least gradient, for $P \in N$ and $r > 0$,
$$\liminf_{n \to \infty} \inf_{v \in BV_\cpt(B(P, r))} \int_{B(P, r)} \star |\dif (u_n + v)| = \liminf_{n \to \infty} \int_{B(P, r)} \star |\dif u_n| \lesssim r^{d - 1} < \infty.$$
By the Miranda stability theorem, it follows that $U$ has least perimeter, and so by Theorem \ref{main thm of old paper}, that $N$ is a complete minimal hypersurface.
\end{proof}

\begin{proof}[Proof of Theorem \ref{compactness theorem}]
Let $P \in M$.
By Proposition \ref{regularity theorem}, there exist $r > 0$ and $L \geq 1$ such that for every large $n \in \NN$, $B(P, r)$ is contained in the image of a flow box $F_n$ for $\lambda_n$ with Lipschitz constant $L$, such that $F_n(0, 0) = P$.
By Proposition \ref{AA Holder}, along a subsequence $F_n \to F$ in $C^{1-}$ for some $C^{1-}$ map $F: I \times J \to B(P, r)$ and some $I \subseteq \RR$, $J \subseteq \RR^{d - 1}$, such that on the image $V$ of $F$, $F^{-1}$ is also $C^{1-}$.
Moreover, $F(0, 0) = P$, so that $F: I \times J \to V$ is a Lipschitz isomorphism onto a set which contains $P$.
Since $P$ was arbitrary, it follows that we can find laminar atlases $(F_\alpha^n. K_\alpha^n)$ for each large $n \in \NN$ such that $F_\alpha^n \to F_\alpha$, where the images of $F_\alpha$ are an open cover $(U_\alpha)$ of $M$, and $(F_\alpha)$ satisfies the usual transition relations.

We now construct the limiting lamination.
By Proposition \ref{Hausdorff on a CMS} and the fact that $I$ is a compact metric space, we may diagonalize so that for every $\alpha$, either $K^n_\alpha \to K_\alpha$ for some nonempty $K_\alpha$ in $\Hypspace I$, or for all sufficiently large $n$ depending on $\alpha$, $K_\alpha^n$ is empty.
However, since $(\lambda_n)$ is tight, we may choose a subsequence so that for some $\alpha$, $K_\alpha^n$ is not eventually empty.

Now let $\psi_{\alpha \beta}$ and $\psi_{\alpha \beta}^n$ be the transition maps, thus $\psi_{\alpha \beta}^n$ induces a map
$$\psi_{\alpha \beta}^n: K_\alpha^n \to K_\beta^n.$$
By convergence of $(F_\alpha^n)$, $\psi_{\alpha \beta}$ induces a map $K_\alpha \to K_\beta$.
We then define for $(k_\alpha) \in \prod_\alpha K_\alpha$
$$N \cap U_\alpha = F_\alpha(\{k_\alpha\} \times \RR^{d - 1}).$$
We then have the cocycle condition
$$(N \cap U_\alpha) \cap U_\beta = (N \cap U_\beta) \cap U_\alpha$$
which follows from the fact that
\begin{align*}
F_\alpha(\{k_\alpha\} \times \RR^{d - 1}) \cap U_\beta
&= F_\beta(\psi_{\alpha \beta}(\{k_\beta\} \times \RR^{d - 1}) \cap U_\alpha \cap U_\beta \\
&= F_\beta(\psi_{\alpha \beta}(\{k_\beta\} \times \RR^{d - 1})) \cap U_\alpha.
\end{align*}
From the cocycle condition, it follows that $N$ honestly defines a Lipschitz hypersurface in $M$.
Since homeomorphisms preserve Hausdorff convergence, it follows that if we have labels $(k_\alpha^n) \in \prod_\alpha K_\alpha$ of leaves $N_n \in \Leaves \lambda_n$, then $N_n \to N$ in $\Hypspace_\loc(M)$.
By Lemma \ref{limit of minimals is minimal}, $N$ is a minimal hypersurface, so if we set $\lambda$ to be the lamination with laminar atlas $(F_\alpha, K_\alpha)$, then $\lambda$ is a minimal lamination.
So by Lemma \ref{convergence of geodesic lams in thurston}, $\lambda_n \to \lambda$ in Thurston's geometric topology.

Finally, if $\mu_n$ is transverse to $\lambda_n$, then we assume, possibly after shrinking the $U_\alpha$s, that $U_\alpha$ is precompact.
By Prohorov's theorem \cite[Theorem 13.29]{klenke2013probability}, there is a subsequence of $(T_{\mu_n})$ which converges to some $T_\mu|_{U_\alpha}$, and by diagonalizing again we may assume that the convergence holds for every $\alpha$ and defines a global current $T_\mu$.
Moreover, by the portmanteau theorem,
$$\supp T_\mu \subseteq \liminf_{n \to \infty} \supp T_{\mu_n} \subseteq \liminf_{n \to \infty} \supp \lambda_n.$$
Here the $(\lambda_n)$ in the limit inferior refers to the subsequence which already converges in the Thurston topology (and has converging Ruelle-Sullivan currents).
In particular, the limit inferior is actually a limit and we conclude
$$\supp T_\mu \subseteq \supp \lambda.$$
In each flow box, the definition of weak limit implies that
$$\int_{U_\alpha} T_\mu \wedge \varphi = \int_I \int_{\{k\} \times \RR^{d - 1}} F_\alpha^* \varphi \dif \mu(k)$$
where $\mu$ is a weak limit of $(\mu_n)$.
In particular, $T_\mu$ is Ruelle-Sullivan for $\lambda$, possibly after shrinking $\lambda$ so that their supports match.
\end{proof}

%%%%%%%%%%%%%%%%%%%%%%%%%%%%%%%%%%%%%%
\subsection{Measured convergence implies flow box convergence}
In order to show that convergence in the weak topology of measures implies convergence in the $C^{1-}$ flow box topology, we first show a weaker mode of convergence, namely Thurston's geometric topology.
Thurston claimed this fact \cite[Proposition 8.10.3]{thurston1979geometry} in case $d = 2$, but his proof left something to be desired as it did not justify why the limit is geodesic, or why the convergence respects the normal vectors.

\begin{lemma}\label{limits of measured geodesic lams are geodesic}
	The set of minimal measured laminations is closed in the weak topology of measures.
\end{lemma}
\begin{proof}
Let $(\lambda, \mu)$ be a measured lamination and suppose that $(\lambda_i, \mu_i) \to (\lambda, \mu)$ in the weak topology of measures, where $(\lambda_i, \mu_i)$ are measured minimal.
Let $x \in \supp \lambda$ and $r > 0$ such that $B := B(x, r)$ is contractible.
In $B$, we can write $T_{\mu_i} = \dif u_i$ for some sequence of functions of least gradient $u_i \in BV(B)$.
Since $u_i$ is only defined up to a constant, we impose $\int_M \star u_i = 0$, so by Poincar\'e's inequality,
$$||u_i||_{L^1(B)} \lesssim r\mu_i(B) \leq 2r \mu(B) < \infty$$
for $i$ large.
So by the Miranda stability theorem, there exists a $1$-harmonic function $u$ such that along a subsequence, $\dif u_i \to \dif u$ in the weak topology of measures.
But then we must have $T = \dif u$, so $\lambda$ is minimal by Theorem \ref{main thm of old paper}.
\end{proof}

\begin{lemma}\label{measured implies Thurston}
Let $(\lambda_i, \mu_i)$ be measured minimal laminations.
If $(\lambda_i, \mu_i) \to (\lambda, \mu)$ in the weak topology of measures, then $\lambda$ is a minimal lamination and $\lambda_i \to \lambda$ in Thurston's geometric topology.
\end{lemma}
\begin{proof}
We first show that
\begin{equation}\label{support is nonincreasing}
	\supp \lambda \subseteq \liminf_{i \to \infty} \supp \lambda_i.
\end{equation}
Indeed, if $x \in \supp \lambda$ then for all $\varepsilon > 0$, $\mu(B(x, \varepsilon)) > 0$.
Then by the portmanteau theorem, for all $i$ large, $\mu_i(B(x, \varepsilon)) > 0$; this proves (\ref{support is nonincreasing}).
It follows that for every $x \in \supp \lambda$, $\varepsilon > 0$, and large $i$, there exists $y \in \supp \lambda_i \cap B(x, \varepsilon)$.
By Lemma \ref{limits of measured geodesic lams are geodesic}, $\lambda$ is a minimal lamination.

To get convergence of the normals we use Proposition \ref{regularity theorem} to choose a Lipschitz $d-1$-form $\varphi$ which extends $\star \normal$.
Then for every $\varepsilon > 0$,
$$\int_{B(x, \varepsilon)} T_\mu \wedge \varphi = \mu(B(x, \varepsilon))$$
so by the portmanteau theorem, for almost every $\varepsilon > 0$,
\begin{equation}\label{epsilon is a continuity set}
	\lim_{i \to \infty} \frac{\int_{B(x, \varepsilon)} T_{\mu_i} \wedge \varphi}{\mu_i(B(x, \varepsilon))} = \frac{\int_{B(x, \varepsilon)} T_\mu \wedge \varphi}{\mu(B(x, \varepsilon))} = 1.
\end{equation}
On the other hand, if we assume that there exists $\delta, \varepsilon > 0$ such that for every $y \in \supp \lambda_i \cap B(x, \varepsilon)$,
$$|\sin \angle(\normal_i, \normal)| \geq \delta,$$
then possibly after shrinking $\varepsilon$ we may assume that (\ref{epsilon is a continuity set}) holds, hence
$$\int_{B(x, \varepsilon)} T_{\mu_i} \wedge \varphi = \int_{B(x, \varepsilon)} \normal_i\mu_i \wedge \star \normal \leq (1 - O(\delta)) \mu_i(B(x, \varepsilon))$$
and therefore $\delta = 0$, a contradiction.
\end{proof}

\begin{lemma}\label{Thurston implies bounds on Two}
Suppose that $\lambda_n \to \lambda$ in Thurston's geometric topology. Then in any small ball $B$ which meets $\supp \lambda$,
$$\limsup_{n \to \infty} \sup_{N \in \Leaves \lambda_n} ||\Two_N||_{C^0(B)} < \infty.$$
\end{lemma}
\begin{proof}
Suppose not; then after passing to a slightly larger ball and a subsequence, we can find $P_n \in N_n \in \Leaves \lambda_n$ such that $|\Two_N(P_n)| \to \infty$ and $P_n \to P$ for some $P$.
We claim that, in fact, we may assume that $P \in \supp \lambda$. \todo{This has to do with Schauder and Harnack estimates controlling $\Two_N$ near $P$.}
Then \todo{we can choose the principal curvatures to make $N_n$ arbitrarily convex in the span of the eigenvector and the normal, but then $N_n$ has to converge on a subsequence to a line segment which is orthogonal to $\lambda$, even though its normal vector is in the same direction as $\lambda$...}
\end{proof}

\begin{proof}[Proof of Theorem \ref{implication theorem}]
By the portmanteau theorem, $\supp \lambda_n$ eventually meets a small neighborhood of any point of $\supp \lambda$, so $(\lambda_n)$ is tight.
Moreover, by Lemma \ref{measured implies Thurston}, $\lambda_n \to \lambda$ in Thurston's geometric topology.
So if $N$ is a leaf of $\lambda$ and $P \in N$, there are leaves $N_n$ of $\lambda_n$ and $P_n \in N_n$ such that $P_n \to P$ and $\normal_{N_n}(P_n) \to \normal_N(P)$.
By Lemma \ref{Thurston implies bounds on Two}, $(\lambda_n)$ has locally uniformly bounded curvature if $n$ is large \todo{on the leaves that contain $\lambda$, which is what actually matters...}
Therefore every subsequence $(\lambda_{n_k})$ has a further subsequence $(\lambda_{n_{k_\ell}})$ which converges to some $\tilde \lambda$ by Theorem \ref{compactness theorem}.
But convergence in the flow box topology implies convergence in Thurston's topology, so possibly after shrinking one or both of the limiting laminations, $\tilde \lambda = \lambda$.
Since $(\lambda_{n_k})$ was arbitrary, $\lambda_n \to \lambda$ in the flow box topology.
\end{proof}


%%%%%%%%%%%%%%%%%%%%
\subsection{Construction of minimal laminations}

\begin{proof}[Proof of Theorem \ref{building a minimal lamination}]
Let $(x_i)$ be a dense sequence in $\supp \lambda$, and let $N_i$ be the leaf containing $x_i$.
Then the union of $N_1, \dots, N_i$ is the support of a lamination $\lambda_i$, so that the curvatures of $\lambda_i$ are locally uniformly bounded.
So $(\lambda_i)$ has a maximal subsequential limit $\tilde \lambda$ in the $C^{1-}$ flow box topology by Theorem \ref{compactness theorem}.
In particular, the leaves of $\tilde \lambda$ are exactly the limits in $\Hypspace_\loc(M)$ of the leaves of $\lambda$, but $\Leaves \lambda$ is closed so $\lambda = \tilde \lambda$.
\todo{Make precise the idea that $\Leaves \lambda$ is closed.}
\end{proof}

\begin{proof}[Proof of Theorem \ref{main thm}]
Let $u$ be a $1$-harmonic function.
By Theorem \ref{main thm of old paper}, the level sets of $u$ are stable minimal hypersurfaces.
Moreover, if $y > z$, then $\{u > y\} \subseteq \{u > z\}$, so $\partial \{u > y\}$ lies on one side of $\partial \{u > z\}$.
By the maximum principle for minimal surfaces \cite[Corollary 1.28]{colding2011course}, it follows that either $\partial \{u > y\}$ and $\partial \{u > z\}$ are disjoint, or are equal.
So the set of level sets of $u$ meets the hypotheses of Corollary \ref{building stable}, and so defines a minimal lamination $\lambda$.
Since the leaves of $\lambda$ are level sets, $\dif u$ is conormal to $\lambda$.

We now construct the transverse measure to $\lambda$.
In any oriented laminar coordinates $(k, x) \in K \times \RR^{d - 1}$ for $\lambda$, $\partial_x u = 0$, so $\star |\dif u|$ defines a measure $\mu$ on $K$: given $\alpha < \beta$, let
$$\mu([\alpha, \beta] \cap K) := u(\beta, x) - u(\alpha, x)$$
for any (and hence every, since $\partial_x u = 0$) $x \in \RR^{d - 1}$.
By Proposition \ref{max princip}, $u(\beta, x) > u(\alpha, x)$, so $\mu$ is a positive measure.
If $(k', x') \in K' \times \RR^{d - 1}$ is a different laminar coordinate system, and the transition map carries $\alpha, \beta$ to $\alpha', \beta'$, then
$$\mu'([\alpha', \beta'] \cap K') := u'(\beta', x') - u(\alpha', x') = u(\beta, x_1) - u(\alpha, x_2)$$
for some $x_1, x_2 \in \RR^{d - 1}$. Since $\partial_x u = 0$,
$$u(\beta, x_1) - u(\alpha, x_2) = u(\beta, x_1) - u(\alpha, x_1) = \mu([\alpha, \beta] \cap K).$$
It follows that $\mu$ is transverse, and by construction $\mu$ lifts to $\star |\dif u|$ in $M$.
Therefore $\dif u$ is Ruelle-Sullivan for $\lambda$.

Now for the converse, if $\dif u$ is an exact Ruelle-Sullivan current for a minimal lamination $\lambda$, we must show that $u$ has least gradient.
If not, we can choose an open set $E \subseteq M$ with $C^\infty$ boundary and a function $v \in BV_\cpt(E)$ such that
$$\int_E \star |\dif u + \dif v| < \int_E \star |\dif u| < \infty.$$
Since $v$ has compact support, there exists a collar neighborhood $F \subseteq E$ of $\partial E$ such that for every $y \in \RR$,
$$\partial \{u > y\} \cap F = \partial^* \{u + v > y\} \cap F.$$
But by Theorem \ref{main thm of old paper}, the level sets $\partial \{u > y\}$ are stable minimal, so it follows that
$$|\partial \{u > y\} \cap E| \leq |\partial^* \{u + v > y\} \cap E|.$$
So by the coarea formula (see \todo{\cite{BackusFLG}}  for a proof at this regularity),
\begin{align*}
\int_E \star |\dif u| &= \int_{-\infty}^\infty |\partial \{u > y\} \cap E| \dif y \leq \int_{-\infty}^\infty |\partial^* \{u + v > y\} \cap E| \dif y \\
&= \int_E \star |\dif u + \dif v| < \int_E \star |\dif u|
\end{align*}
which is a contradiction.
\end{proof}


\printbibliography

\end{document}
