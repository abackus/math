\documentclass[reqno,11pt]{amsart}
\usepackage[letterpaper, margin=1in]{geometry}
\RequirePackage{amsmath,amssymb,amsthm,graphicx,mathrsfs,url,slashed,subcaption}
\RequirePackage[usenames,dvipsnames]{xcolor}
\RequirePackage[colorlinks=true,linkcolor=Red,citecolor=Green]{hyperref}
\RequirePackage{amsxtra}
\usepackage{cancel}
\usepackage{tikz-cd}

% \setlength{\textheight}{9.3in} \setlength{\oddsidemargin}{-0.25in}
% \setlength{\evensidemargin}{-0.25in} \setlength{\textwidth}{7in}
% \setlength{\topmargin}{-0.25in} \setlength{\headheight}{0.18in}
% \setlength{\marginparwidth}{1.0in}
% \setlength{\abovedisplayskip}{0.2in}
% \setlength{\belowdisplayskip}{0.2in}
% \setlength{\parskip}{0.05in}
%\renewcommand{\baselinestretch}{1.05}

\title{1-harmonic functions and sequences of minimal laminations}
\author{Aidan Backus}
\date{October 2022}

\newcommand{\NN}{\mathbf{N}}
\newcommand{\ZZ}{\mathbf{Z}}
\newcommand{\QQ}{\mathbf{Q}}
\newcommand{\RR}{\mathbf{R}}
\newcommand{\CC}{\mathbf{C}}
\newcommand{\DD}{\mathbf{D}}
\newcommand{\PP}{\mathbf P}
\newcommand{\MM}{\mathbf M}
\newcommand{\II}{\mathbf I}
\newcommand{\Hyp}{\mathbf H}
\newcommand{\Sph}{\mathbf S}
\newcommand{\Group}{\mathbf G}
\newcommand{\GL}{\mathbf{GL}}
\newcommand{\Orth}{\mathbf{O}}
\newcommand{\SpOrth}{\mathbf{SO}}
\newcommand{\Ball}{\mathbf{B}}

\newcommand*\dif{\mathop{}\!\mathrm{d}}

\DeclareMathOperator{\dist}{dist}
\DeclareMathOperator{\MeasLam}{MeasLam}
\DeclareMathOperator{\MinLam}{MinLam}
\DeclareMathOperator{\Lam}{Lam}
\DeclareMathOperator{\supp}{supp}

\newcommand{\Leaves}{\mathscr L}
\newcommand{\Hypspace}{\mathscr H}

\newcommand{\Two}{\mathrm{I\!I}}


\newcommand{\Hilb}{\mathcal H}
\newcommand{\Homology}{\mathrm H}
\newcommand{\normal}{\mathbf n}
\newcommand{\radial}{\mathbf r}
\newcommand{\evect}{\mathbf e}
\newcommand{\vol}{\mathrm{vol}}

\newcommand{\inj}{\mathrm{inj}}
\newcommand{\Lip}{\mathrm{Lip}}
\newcommand{\Riem}{\mathrm{Riem}}

\newcommand{\Bmu}{\boldsymbol \mu}
\newcommand{\Bnu}{\boldsymbol \nu}
\newcommand{\Blambda}{\boldsymbol \lambda}

\newcommand{\pic}{\vspace{30mm}}
\newcommand{\dfn}[1]{\emph{#1}\index{#1}}

\renewcommand{\Re}{\operatorname{Re}}
\renewcommand{\Im}{\operatorname{Im}}

\newcommand{\loc}{\mathrm{loc}}
\newcommand{\cpt}{\mathrm{cpt}}

\def\Japan#1{\left \langle #1 \right \rangle}

\newtheorem{theorem}{Theorem}[section]
\newtheorem{badtheorem}[theorem]{``Theorem"}
\newtheorem{prop}[theorem]{Proposition}
\newtheorem{lemma}[theorem]{Lemma}
\newtheorem{sublemma}[theorem]{Sublemma}
\newtheorem{proposition}[theorem]{Proposition}
\newtheorem{corollary}[theorem]{Corollary}
\newtheorem{conjecture}[theorem]{Conjecture}
\newtheorem{axiom}[theorem]{Axiom}
\newtheorem{assumption}[theorem]{Assumption}

\newtheorem{mainthm}{Theorem}
\renewcommand{\themainthm}{\Alph{mainthm}}

% \newtheorem{claim}{Claim}[theorem]
% \renewcommand{\theclaim}{\thetheorem\Alph{claim}}
\newtheorem*{claim}{Claim}

\theoremstyle{definition}
\newtheorem{definition}[theorem]{Definition}
\newtheorem{remark}[theorem]{Remark}
\newtheorem{example}[theorem]{Example}
\newtheorem{notation}[theorem]{Notation}

\newtheorem{exercise}[theorem]{Discussion topic}
\newtheorem{homework}[theorem]{Homework}
\newtheorem{problem}[theorem]{Problem}

\makeatletter
\newcommand{\proofpart}[2]{%
  \par
  \addvspace{\medskipamount}%
  \noindent\emph{Part #1: #2.}
}
\makeatother



\numberwithin{equation}{section}


% Mean
\def\Xint#1{\mathchoice
{\XXint\displaystyle\textstyle{#1}}%
{\XXint\textstyle\scriptstyle{#1}}%
{\XXint\scriptstyle\scriptscriptstyle{#1}}%
{\XXint\scriptscriptstyle\scriptscriptstyle{#1}}%
\!\int}
\def\XXint#1#2#3{{\setbox0=\hbox{$#1{#2#3}{\int}$ }
\vcenter{\hbox{$#2#3$ }}\kern-.6\wd0}}
\def\ddashint{\Xint=}
\def\dashint{\Xint-}

\usepackage[backend=bibtex,style=numeric]{biblatex}
\renewcommand*{\bibfont}{\normalfont\footnotesize}
\addbibresource{topics.bib}
\renewbibmacro{in:}{}
\DeclareFieldFormat{pages}{#1}


\begin{document}
\begin{abstract}
We collect several results relating different notions of convergence of laminations, especially minimal laminations.
We also give a condition for a partition of a closed set into minimal hypersurfaces to be a lamination.
\end{abstract}

\maketitle

%%%%%%%%%%%%%%%%%%%%%%%%%%%%%%%%%%%%%%%%%%%%%%%%%%%%%%%

% \tableofcontents

\section{Introduction}
The space of codimension-$1$ minimal laminations on a Riemannian manifold has been topologized in several different ways.
Thurston \cite[Chapter 8]{thurston1979geometry} introduced both his geometric topology as well as the weak topology of measures on the space of measured geodesic laminations.
Seemingly independently of Thurston, Colding--Minicozzi \cite[Appendix B]{ColdingMinicozziIV} introduced a topology that emphasized not the laminations themselves, but rather the coordinate charts which flatten them.
One goal of this paper is to explain how these three modes of convergence are related, and fill in the details of the compactness theorems available for each mode of convergence.
Our results will consider compactness of sequences of laminations on manifolds that are not necessarily closed and will explicitly rule out the possibility that a sequence of laminations converges to a lamination of lower regularity.

Our second goal is to give a sufficient condition under which a set of disjoint minimal hypersurfaces is actually a minimal lamination.
In dimension $d = 2$, any such set is a lamination \cite[Proposition 7.3]{daskalopoulos2020transverse} but in general one needs uniform bounds on the curvature of the leaves in order to apply the compactness theory and prevent the formation of a singularity in the limit.
In dimensions $d = 3, 4$, if one in addition knows that the leaves are stable, then the curvature bounds are instead provided by the stable Bernstein theorem of Schoen \cite{Schoen2016} and Chodosh--Li \cite{Chodosh2021}.

The application of the stable Bernstein theorem completes the third and final goal of this paper, and the main goal of this series of papers, which also includes the prequel paper \cite{BackusFLG}.
We show that any $1$-harmonic function gives rise to a Ruelle-Sullivan current for a minimal lamination, and conversely.
This generalizes a theorem of Daskalopoulos--Uhlenbeck \cite[Theorem 6.1]{daskalopoulos2020transverse} and resolves \cite[Problem 9.4]{daskalopoulos2020transverse} and \cite[Conjecture 9.5]{daskalopoulos2020transverse}.

%%%%%%%%%%%%%%%%%
\subsection{Minimal laminations}\label{Lams sections}
Let $I \subseteq \RR$ be an interval, and $M$ a Riemannian manifold of dimension $d \geq 2$.
A (codimension-$1$) \dfn{laminar flow box} is a $C^0$ coordinate chart $F: I \times \RR^{d - 1} \to M$ and a compact set $K \subseteq I$ such that each \dfn{leaf} $F(\{k\} \times \RR^{d - 1})$ is $C^2$.
A \dfn{laminar transition map} between two laminar flow boxes $(F_\alpha, K_\alpha), (F_\beta, K_\beta)$ is a $C^0$ map
$$\psi_{\alpha \beta}: I \times \RR^{d - 1} \to I \times \RR^{d - 1}$$
satisfying the usual transition relation
\begin{equation}\label{transition relation}
F_\alpha = F_\beta \circ \psi_{\alpha \beta},
\end{equation}
which maps each leaf $\{k\} \times \RR^{d - 1}$, $k \in K_\alpha$, to a leaf $\{\psi_{\alpha \beta}(k)\} \times \RR^{d - 1}$, so that $\psi_{\alpha \beta}$ is a homeomorphism $K_\alpha \to K_\beta$.
By a \dfn{laminar atlas} we shall mean an atlas for $M$, such that the coordinate charts are all laminar flow boxes and the transition maps are also laminar.

\begin{definition}
A \dfn{lamination} $\lambda$ consists of a nonempty closed set $S \subseteq M$, called its \dfn{support}, and a maximal laminar atlas $\{(F_\alpha, K_\alpha): \alpha \in A\}$ such that in the image $U_\alpha$ of each flow box $F_\alpha$,
$$S \cap U_\alpha = F_\alpha(K \times \RR^{d - 1}).$$
If $\lambda$ is a lamination in the image of a flow box $F$, and $N := F(\{k\} \times \RR^{d - 1})$ is a leaf of $\lambda$, we call $k$ the \dfn{label} of $\lambda$.
\end{definition}

Summarizing the above definitions, a lamination is a nonempty closed set $S$ with a $C^0$ local product structure which realizes it as $K \times \RR^{d - 1}$ for some compact set $K \subset \RR$.

We assume that the leaves are $C^2$ in order to ensure that the normal vectors to each leaves are well-defined in $C^1$, and in particular the second fundamental form and mean curvature of each leaf is well-defined.
Such laminations are sometimes called $C^2$ \dfn{along leaves} \cite{Morgan88}.
This is not the same thing as assuming that the lamination admits a $C^2$ atlas, as it may not be able to extend the normal vectors to each leaf to a $C^1$ vector field on $M$ even locally; see \S\ref{RegularitySec} for more precise assertions about regularity.

\begin{definition}
A lamination $\lambda$ is \dfn{minimal} if its leaves $F_\alpha(\{k\} \times \RR^{d - 1})$ have zero mean curvature, and is \dfn{geodesic} if, in addition, $d = 2$.
\end{definition}

Geodesic laminations are of great interest to the Thurston school of geometric topology \cite[Chapter 8]{thurston1979geometry}.
Later Thurston introduced \dfn{best Lipschitz maps}, namely maps $v: M \to N$ between closed manifolds which minimize their Lipschitz constant $L$ subject to a constraint on their homotopy class.
These maps define a geodesic lamination whose support is the set of points $x$ so that the local Lipschitz constant of $v$ at $x$ is equal to $L$ \cite{thurston1998minimal}.
If $M, N$ are hyperbolic surfaces of the same genus $g$, then the best Lipschitz constant $L$ is the distance between $M$ and $N$ in \dfn{Thurston's asymmetric metric} on Teichm\"uller space.
This circle of ideas has been developed by the Thurston school \cite{papadopoulos:hal-00129729} but has recently also made contact with geometric PDE through the work of Daskalopoulos--Uhlenbeck \cite{daskalopoulos2020transverse,daskalopoulosPrep1,DaskalopoulosPrep2}, as we recall in \S\ref{FLG section}.

We prove a sufficient condition for a collection of minimal hypersurfaces to define a minimal lamination.

\begin{theorem}\label{building a minimal lamination}
Let $2 \leq d \leq 7$ and suppose that $M$ has constant sectional curvature.
If $\mathcal N$ is a set of disjoint embedded minimal hypersurfaces in $M$, and
\begin{equation}\label{bounding Two}
\sup_{N \in \mathcal N} ||\Two_N||_{C^0} < \infty,
\end{equation}
then $\mathcal N$ is the set of leaves of a minimal lamination.
\end{theorem}

The condition (\ref{bounding Two}) is frequently met in applications, owing to the following:

\begin{theorem}[stable Bernstein theorem]
Let $3 \leq d \leq 4$ and let $N$ be a two-sided stable minimal hypersurface in $B_r \subseteq M$, $r \lesssim 1$, where $M$ has bounded geometry and dimension $d$.
Then on $B_{r/2}$, $|\Two_N| \lesssim_M r^{-1}$.
\end{theorem}
\begin{proof}
For $d = 3$ see \cite[Corollary 2.11]{colding2011course}, and for $d = 4$ see \cite{Chodosh2021}.
\end{proof}

\begin{corollary}
Let $M$ be a manifold of constant sectional curvature, $\mathcal N$ a set of disjoint embedded minimal hypersurfaces in $M$, and either:
\begin{enumerate}
\item $d = 2$, or
\item $d \in \{3, 4\}$ and every leaf of $\lambda$ is stable.
\end{enumerate}
Then $\mathcal N$ is the set of leaves of a minimal lamination.
\end{corollary}
\begin{proof}
The hypotheses of this corollary, plus the stable Bernstein theorem, imply that $|\Two_N|$ is locally uniformly bounded, so we may apply Theorem \ref{building a minimal lamination}.
\end{proof}

%%%%%%%%%%%%%%%%%%
\subsection{Spaces of minimal laminations}\label{LamSpace section}
In the literature there are at least three different topologies on the space of laminations on $M$.
The first is Thurston's geometric topology \cite[Chapter 8]{thurston1979geometry}, which says that a lamination $\lambda'$ is close to a lamination $\lambda$ if every leaf of $\lambda$ is close to a leaf of $\lambda'$ at least locally, and the same holds for their normal vectors $\normal$.

\begin{definition}
A sequence of laminations $\lambda_i$ converges to a lamination $\lambda$ in \dfn{Thurston's geometric topology} if, for every leaf $N$ of $\lambda$, every $x \in N$, and every $\varepsilon > 0$, there exists $i_\varepsilon \in \NN$ such that for every $i \geq i_\varepsilon$, $\supp \lambda_i$ intersects $B(x, \varepsilon)$, and for $x_i \in B(x, \varepsilon) \cap \supp \lambda_i$,
\begin{equation}\label{convergence of normals}
\dist_{SM}(\normal_{\lambda_i}(x_i), \normal_\lambda(x)) < 2\varepsilon.
\end{equation}
\end{definition}

It is straightforward to show that Thurston's geometric topology does not depend on the choice of Riemannian metric on $M$, or the choice of extension of the distance function on $M$ to its sphere bundle $SM$, which are implicit in the statement thereof.
However, the limiting lamination is not unique, as if $\lambda_i \to \lambda$ and $\lambda'$ is a sublamination of $\lambda$, then $\lambda_i \to \lambda'$.
In particular, Thurston's topology is not Hausdorff, and we say that $\lambda$ is a \dfn{maximal limit} of a sequence $(\lambda_i)$ if $\lambda_i \to \lambda$ and for every $\lambda'$ such that $\lambda_i \to \lambda'$, $\lambda'$ is a sublamination of $\lambda$.

Independently of Thurston, Colding--Minicozzi \cite[Appendix B]{ColdingMinicozziIV} defined a sequence of laminations to converge ``if the corresponding coordinate maps converge;'' that is, if the laminar atlases converge.
This of course says nothing about the limiting set of leaves and in the sequel paper \cite{ColdingMinicozziV} they additionally impose that the sets of leaves converge (say, in Hausdorff distance).
The following is equivalent to requiring that the sets of leaves converge.

\begin{definition}
A sequence $(\lambda_i)$ of laminations \dfn{flow-box converges} in a function space $X$ to $\lambda$ if it converges in Thurston's geometric topology, and there exists a laminar atlas $(F_\alpha)$ for $\lambda$ such that for each $\alpha$, $F_\alpha$ and $(F_\alpha)^{-1}$ are limits in $X$ of flow boxes $F_\alpha^i$, $(F_\alpha^i)^{-1}$ in laminar atlases for $\lambda_i$.
\end{definition}

We now define convergence of laminations equipped with transverse measures.

\begin{definition}
Let $\lambda$ be a lamination with atlas $A$.
A \dfn{transverse measure} to $\lambda$ consists of Radon measures $\mu_\alpha$ with $\supp \mu_\alpha = K_\alpha$, $\alpha \in A$, such that each transition map $\psi_{\alpha \beta}$ is measure-preserving:
$$\mu_\alpha|_{K_\alpha \cap K_\beta} = \psi_{\alpha \beta}^* (\mu_\beta|_{K_\alpha \cap K_\beta}).$$
The pair $(\lambda, \mu)$ is called a \dfn{measured lamination}.
\end{definition}

Caveat lector: we assume that $\supp \mu_\alpha = K_\alpha$, but in \cite{daskalopoulos2020transverse}, it is only assumed that $\supp \mu_\alpha \subseteq K_\alpha$.
In particular, not every lamination admits a transverse measure.

The definition of transverse measure in terms of Radon measures on $K_\alpha$ is convenient because $K_\alpha$ is compact.
However, the definition is not intrinsic, and this causes problems when considering questions of convergence: the fact that the flow boxes of a convergent sequence of measured laminations converge should be a consequence of, not a part of, the definition!

To rectify this, we first observe that in the definition of a transverse measure, we cannot define a transverse measure to be one on the underlying manifold $M$ itself.
Indeed, Lebesgue measure is ``transverse'' to all foliations; thus such a definition forgets the ``direction'' the measure points in.
However, the notion of Ruelle-Sullivan current allows us to speak of a measure-theoretic object on $M$ which has a well-defined local product structure.

\begin{definition}
A lamination is \dfn{oriented} if one can choose its transition maps to all be orientation-preserving.
\end{definition}

It is clear that a lamination $\lambda$ is locally orientable, since if one replaces $M$ by a small open set, then $\lambda$ has a global flow box.

\begin{definition}
Let $(\lambda, \mu)$ be a measured oriented lamination and let $(\chi_\alpha)_{\alpha \in A}$ be a subordinate partition of unity.
The \dfn{Ruelle-Sullivan current} $T_\mu$ associated to $(\lambda, \mu)$ is defined for all compactly supported $d-1$-forms $\varphi$ by
\begin{equation}\label{RS current}
\int_M T_\mu \wedge \varphi := \sum_{\alpha \in A} \int_{K_\alpha} \left[\int_{\{k\} \times \RR^{d - 1}} (F_\alpha^{-1})^* (\chi_\alpha \varphi) \right] \dif \mu_\alpha(k).
\end{equation}
\end{definition}

It is clear that any lamination is locally orientable, so the next definition makes sense.

\begin{definition}
A sequence of measured laminations $(\lambda_i, \mu_i)$ \dfn{converges} to $(\lambda, \mu)$ if locally, their Ruelle-Sullivan currents $T_{\mu_i} \to T_\mu$ converge in the weak topology of measures.
\end{definition}

The convergence of Ruelle-Sullivan currents, which is very convenient to work with analytically, is equivalent to a definition of measure convergence that may be more familiar to topologists, namely convergence of the transverse measure along each transverse curve.

It is clear from the definitions that flow-box convergence implies Thurston convergence, and it is well-known that measure convergence implies Thurston convergence \cite[Proposition 8.10.3]{thurston1979geometry}.
We show that flow-box convergence actually sits in the middle of the chain of implications:

\begin{theorem}\label{implication theorem}
Suppose that $M$ has constant sectional curvature and $2 \leq d \leq 7$.
Let $(\lambda_i, \mu_i)$ be measured minimal laminations, and $(\lambda_i, \mu_i) \to (\lambda, \mu)$. Then $\lambda_i \to \lambda$ as flow boxes in $^{1-}$.
\end{theorem}

We also prove some compactness results for the above modes of convergence.
In general a sequence of laminations may escape to infinity or blow up in curvature, and we must rule these possibilities out.

\begin{definition}
A sequence $(\lambda_i)$ of laminations is \dfn{tight} if there exists a compact set $K \subseteq M$ such that for every $i$, $\supp \lambda_i$ intersects $K$.
The sequence has \dfn{bounded curvature} if there exists $C > 0$ such that for any $i$ and any leaf $N$ of $\lambda_i$, the second fundamental form satisfies $||\Two_N||_{C^0} \leq C$.
\end{definition}

We write $C^{1-}$ for the Fr\'echet space $\bigcap_{\alpha < 1} C^\alpha$, where $C^\alpha$ are H\"older spaces.

\begin{theorem}\label{compactness theorem}
Suppose that $M$ has constant sectional curvature and $2 \leq d \leq 7$.
Let $(\lambda_i)$ be a tight sequence of minimal laminations of bounded curvature.
Then a subsequence converges as flow boxes in $C^{1-}$, and in particular in Thurston's topology, to a minimal lamination.
If $\lambda_i$ are measured, then a further subsequence converges as measures.
\end{theorem}



%%%%%%%%%%%%%%%%%%
\subsection{Best Lipschitz and least gradient maps}\label{FLG section}
If $M$ is a closed hyperbolic manifold, then the Euler-Lagrange equation for best Lipschitz maps $v: M \to \Sph^1$ is the $\infty$-Laplace equation \cite{daskalopoulos2020transverse}
\begin{equation}\label{infinity laplacian}
(\nabla^\mu \partial^\nu v) \partial_\mu v \partial_\nu v = 0.
\end{equation}
This equation is invariant under translations $v \mapsto v + y$, so by Noether's theorem, it has a conserved flux $\dif u$.
If $d = 2$, the associated conservation law is the $1$-Laplace equation.
We studied the $1$-Laplacian in the prequel paper \cite{BackusFLG}; here we recall the main result of that paper.

\begin{definition}
A function $u \in BV_\loc(M)$ has \dfn{least gradient}, or is \dfn{$1$-harmonic}, if for every $w \in BV_\cpt(M)$,
\begin{equation}\label{least gradient functional}
\int_M \star |\dif u| \leq \int_M \star |\dif u + \dif w|.
\end{equation}
\end{definition}

Here $\star |\dif u|$ is the total variation of the current $\dif u$; we refer to \S\ref{MeasurePrelims} for the precise definition.
The formal Euler-Lagrange equation for (\ref{least gradient functional}) is the $1$-Laplace equation
\begin{equation}\label{1Laplacian}
\dif^* \left(\frac{\dif u}{|\dif u|}\right) = 0.
\end{equation}
Formally, the $1$-Laplace equation (\ref{1Laplacian}) asserts that the level sets are indeed minimal, but since the derivation of the Euler-Lagrange equation for (\ref{least gradient functional}) is only formal, and the precise definition of weak solution \cite{Mazon14} does not directly imply that the level sets have zero mean curvature, this has to be checked separately.
This is the main result of the prequel paper \cite{BackusFLG}:

\begin{theorem}\label{main thm of old paper}
Let $M$ be a manifold of constant sectional curvature and $2 \leq d \leq 7$.
Then for every $1$-harmonic function $u: M \to \RR$ and $y \in \RR$, the level set $\partial \{u > y\}$ is an analytic embedded stable minimal hypersurface in $M$.
\end{theorem}
\begin{proof}[Proof sketch]
By a straightfoward modification of \cite[Theorem 1]{BOMBIERI1969}, the superlevel sets $\{u > y\}$ have least perimeter, that is their indicator functions have least gradient.
The regularity of boundaries of sets of least perimeter was established for $M = \RR^d$ by the classical work of de Giorgi and Miranda \cite{deGiorgi61, Miranda66} but their proof does not generalize nicely because it relies on the invariance of tangent vectors under parallel transport in order to define averages of normal vectors to sets of least perimeter.
In \cite[\S3]{BackusFLG} we establish a suitable method of taking averages of the normal vector provided that $M = \Hyp^d$ or $M = \Sph^d$.
This allows us to modify the relevant parts of \cite{Miranda66}.
See \cite[\S1]{BackusFLG} for a more detailed summary of \cite{BackusFLG}.
\end{proof}

We use Theorem \ref{main thm of old paper} as a regularity result at various points in this paper, which allows us to show that the limiting laminations have smooth leaves (rather than currents for leaves, say).
However, Theorem \ref{main thm of old paper} is of more interest to this paper rather than just as a regularity lemma; to motivate why, we recall the  relationship between best Lipschitz maps, maps of least gradient, and geodesic laminations:

\begin{theorem}[Daskalopoulos--Uhlenbeck]\label{DU theorem}
Let $M$ be a closed hyperbolic surface and $v: M \to \Sph^1$ an $\infty$-harmonic function with conserved flux $\dif u$ and maximal stretch geodesic lamination $\lambda$.
Then any local primitive $u$ of $\dif u$ is a $1$-harmonic function, and $\dif u$ is Ruelle-Sullivan for $\lambda$.
Moreover, the level sets of $u$ are geodesics in $\lambda$.
\end{theorem}

We refer to the original paper \cite{daskalopoulos2020transverse} for a more precise statement.
Inspired by this theorem, Daskalopoulos--Uhlenbeck conjectured that for any $1$-harmonic function on $\Hyp^2$, $\dif u$ should be Ruelle-Sullivan for some (possibly not maximum-stretch) geodesic lamination \cite[Problem 9.4]{daskalopoulos2020transverse}, and conversely that if $T$ is a Ruelle-Sullivan current for some geodesic lamination, then local primitives of $T$ are $1$-harmonic \cite[Conjecture 9.5]{daskalopoulos2020transverse}.
Of course, if $d \geq 3$, then the level sets will be minimal hypersurfaces rather than geodesics.

Using Theorem \ref{building a minimal lamination}, the stable Bernstein theorem \cite{Schoen2016, Chodosh2021}, and Theorem \ref{main thm of old paper}, we prove the conjectures of Daskalopoulos--Uhlenbeck:

\begin{theorem}\label{main thm}
Let $2 \leq d \leq 4$ and suppose that $M$ has constant sectional curvature.
\begin{enumerate}
\item Let $u$ be a $1$-harmonic function.
Then $\bigcup_{y \in \RR} \partial \{u > y\}$ is the support of a minimal lamination $\lambda$ whose leaves are the connected components of the hypersurfaces $\partial \{u > y\}$, and $\dif u$ is a Ruelle-Sullivan current for $\lambda$.
\item Conversely, if $\lambda$ is a minimal lamination and $\dif u$ is a Ruelle-Sullivan current for $\lambda$, then $u$ is $1$-harmonic.
\end{enumerate}
\end{theorem}

However, Theorem \ref{main thm} leaves a key point -- the role of the $\infty$-Laplacian -- open, and we have not attempted to address this point here.
The Daskalopoulos--Uhlenbeck theorem was our main motivation for this series of papers; however, our work works in dimensions $d = 3, 4$ while Theorem \ref{DU theorem} is purely a statement about $d = 2$.
Indeed, the $\infty$-Laplacian gives geodesic laminations, but the $1$-Laplacian gives codimension-$1$ laminations, which are not geodesic if $d \geq 3$, so there is not an obvious generalization of Theorem \ref{DU theorem} for $d = 3, 4$.

It is tantalizing to think that a suitable system of coupled $\infty$-Laplacians will satisfy a generalized maximum-stretch condition on a codimension-$1$ minimal lamination $\lambda$, and that the conservation law for this conjectural system will be the $1$-Laplacian.
However, even if one was to derive such a system of $\infty$-Laplacians formally, the analysis for studying such a system would likely not be in place.
Indeed, the study of $\infty$-harmonic functions is almost entirely carried out in the language of viscosity solutions and comparison-with-cones, which only make sense when the target is $\RR$ rather than $\RR^m$ or a vector bundle.
Even so, we hope to return to this question in a later work: what is the correct generalization of the Daskalopoulos--Uhlenbeck theorem to $d = 3$?

%%%%%%%%%%%%%%%%%%%%%%%
\subsection{Outline of the paper}
In \S\ref{Prelims} we recall preliminaries.

In \S\ref{Regularity} we establish that every $C^0$ minimal lamination has a Lipschitz laminar atlas and a Lipschitz normal bundle.
We will use this regularity frequently through the remainder of the paper.

In \S\ref{CompactnessSec} we prove the compactness Theorem \ref{compactness theorem} and the convergence implications Theorem \ref{implication theorem} simultaneously.

In \S\ref{construction} we prove Theorem \ref{building a minimal lamination} establishing which sets of minimal hypersurfaces are minimal laminations, using Theorem \ref{compactness theorem}.
We then use Theorem \ref{building a minimal lamination} to prove Theorem \ref{main thm}.


%%%%%%%%%%%%%%%%%%%%%%%%

\subsection{Acknowledgements}
I would like to thank Georgios Daskalopoulos for suggesting this project and for many helpful discussions.
I would also like to thank Chao Li for helpful comments.
This work was supported by an NSF Graduate Resarch Fellowship TODO.



%%%%%%%%%%%%%%%%%%%%%%%%%%%

\section{Preliminaries}\label{Prelims}
\subsection{Laminations}\label{RegularitySec}
\begin{definition}
Let $\lambda$ be a lamination in $M$.
\begin{enumerate}
\item $\lambda$ is \dfn{discrete} if every leaf space of $\lambda$ is finite.
\item $\lambda$ is \dfn{finite} if $\lambda$ is discrete and admits a finite laminar atlas.
\item $\lambda$ is a \dfn{foliation} if $\supp \lambda = M$.
\end{enumerate}
\end{definition}

Every finite lamination is discrete, and conversely, a discrete lamination in a closed manifold is finite.
However, a lamination with finitely many leaves need not be finite, even if it is geodesic:

\begin{example}\label{two geodesics}
Let $M = \RR \times \Sph^1$ be a cylinder.
We may choose a metric on $M$ so that there is a geodesic $\gamma_1$ which is a simple closed curve looping around $M$, and that there is another geodesic $\gamma_2$ which winds around $M$ infinitely many times, converging to $\gamma_1$.
Then $\gamma_1, \gamma_2$ form a geodesic lamination with two leaves which is not finite.
This example, and slight modifications thereof, will be frequently useful as a counterexample throughout this paper.
\end{example}

Though we impose that laminations are $C^0$ (in the sense that their flow boxes are $C^0$) we will later establish that they are, in fact, Lipschitz.
This regularity is optimal even for the nicest case of a geodesic foliation of $\Hyp^2$ \cite[\S1]{Solomon86}.


%%%%%%%%%%%%%%%%
\subsection{Thurston's geometric topology}

\begin{definition}
Let $X$ be a compact metric space. The \dfn{Hausdorff distance} between two closed sets $A, B \subset X$ is
$$\dist(A, B) := \max\left(\max_{a \in A} \min_{b \in B} \dist(a, b), \max_{b \in B} \min_{a \in A} \dist(a, b)\right).$$
The space of closed subsets of $X$ is the \dfn{hyperspace} $\Hypspace X$.
\end{definition}

For compact $X$, however, one can give a homeomorphism-invariant definition of the topology of $\Hypspace X$ \cite[Chapter 4]{nadler2017continuum}, and from that definition it follows that $\Hypspace X$ is a compact metric space.
In particular, $\Hypspace$ is a self-map of the class of compact metrizable spaces.
Moreover, if $K_i \to K$ in $\Hypspace X$, then $K$ is the set of limits of sequences $(k_i)$ such that $k_i \in K_i$.


%%%%%%%%%%%%%%%%%%%%%
\subsection{Measure theory}\label{MeasurePrelims}
Let $X$ be a locally compact Polish space, and let $C_\cpt(X)$ be the space of compactly supported continuous functions $f: X \to \RR$.
Its dual $C_\cpt(X)'$ is canonically isomorphic to the space of signed Radon measures on $X$, where the bilinear pairing is given by integration.
The weak topology on $C_\cpt(X)'$ is known as the \dfn{weak topology of measures}.
Unpacking the definitions, a sequence $(\mu_i)$ of Radon measures converges to $\mu$ in the weak topology of measures iff for every compact $Y \subseteq X$ and every continuous function $f: Y \to \RR$,
$$\lim_{i \to \infty} \int_Y f \dif \mu_i = \int_Y f \dif \mu.$$
We shall frequently use the following characterization of weak convergence:

\begin{proposition}[portmanteau theorem]
	Let $(\mu_n)$ be a sequence of Radon measures on a locally compact Polish space $X$ with $\mu_n(X) \lesssim 1$, and let $\mu$ be a Radon measure on $X$. The following are equivalent:
\begin{enumerate}
	\item $\mu_n \to \mu$ in the weak topology of measures.
	\item $\liminf_{n \to \infty} \mu_n(X) \geq \mu(X)$ and for every closed $Y \subseteq X$, $\limsup_{n \to \infty} \mu_n(Y) \leq \mu(Y)$.
	\item $\limsup_{n \to \infty} \mu_n(X) \leq \mu(X)$ and for every open $Z \subseteq X$, $\liminf_{n \to \infty} \mu_n(Z) \geq \mu(Z)$.
	\item For every $W \subseteq X$ with $\mu(\partial W) = 0$, $\lim_{n \to \infty} \mu_n(W) = \mu(W)$.
\end{enumerate}
	If $X$ is a manifold, these conditions are equivalent to:
\begin{enumerate}
	\setcounter{enumi}{4}
	\item For every $x \in X$ and almost every $0 < \varepsilon \ll 1$, $\lim_{n \to \infty} \mu_n(B(x, \varepsilon)) = \mu(B(x, \varepsilon))$.
\end{enumerate}
\end{proposition}
\begin{proof}
	See \cite[Theorem 13.16]{klenke2013probability} for the metrizable case.
	For the manifold case, we just observe that almost every $\varepsilon > 0$ satisfies $\mu(\partial B(x, \varepsilon)) = 0$. Indeed, if not, then we can find a set of real numbers $A$ such that $0$ is a condensation point of $A$, and for every $\varepsilon \in A$, $\mu(\partial B(x, \varepsilon)) > 0$.
	In particular, for every $\delta > 0$, $A \cap (0, \delta)$ is uncountable, so
	$$\mu(B(x, \delta)) \geq \sum_{\varepsilon \in A \cap (0, \delta)} \mu(\partial B(x, \delta)) = \infty,$$
	but since $X$ is a manifold, if $\delta$ is small enough then $B(x, \delta)$ is precompact.
	This contradicts that $\mu$ is a Radon measure.
\end{proof}

If $X = M$ is a manifold, then we can consider instead the space $C_\cpt(M, \Omega^\ell)$ of compactly supported continuous $\ell$-forms.
An $\ell$-\dfn{current of locally finite total variation} is an element of the dual space $C_\cpt(M, \Omega^\ell)'$.
We shall never need to consider $\ell$-currents whose total variation is not locally finite, so we simply refer to such objects as $\ell$-currents.
We denote the pairing of an $\ell$-current $T$ and an $\ell$-form $\varphi$ by $\int_M T \wedge \varphi$.
Any $d-\ell$-form $\psi$ gives rise to an $\ell$-current $T$, the \dfn{Poincar\'e dual} of $\psi$, by $\int_M T \wedge \varphi = \int_M \psi \wedge \varphi$.
In particular, the Poincar\'e dual of any function is a $d$-current.
See \cite{simon1983GMT} for more on the theory of currents.

Again we have the weak topology on the space of $\ell$-currents; we also have the \dfn{derivative}
$$\int_M \dif T \wedge \psi := -\int_M T \wedge \dif \psi$$
defined for any $\ell$-current $T$ such that $\psi \mapsto \int_M T \wedge \dif \psi$ extends from $C^1_\cpt(M, \Omega^{\ell - 1})$ to an $\ell - 1$-current.

With this machinery in place, we can talk about Ruelle-Sullivan currents and their limits.

\begin{lemma}
Let $(\lambda, \mu)$ be a measured oriented lamination.
Then the Ruelle-Sullivan current $T_\mu$ is well-defined; it is honestly a $d-1$-current, and does not depend on the choice of partition of unity.
Moreover, $\dif T_\mu = 0$.
\end{lemma}
\begin{proof}
We first claim that the right-hand side of (\ref{RS current}) is always finite, and is continuous in $\varphi$.
In fact, possibly after refining $(\chi_\alpha)$, we may assume that it is a locally finite partition of unity.
In particular, we just need to check the continuity in a single flow box:
$$\left|\int_{K_\alpha} \left[\int_{\RR^{d - 1} \times \{k\}} (F_\alpha^{-1})^* (\chi_\alpha \varphi) \right] \dif \mu_\alpha(k)\right| \leq \int_{K_\alpha} \int_{\RR^{d - 1} \times \{k\}} |(F_\alpha^{-1})^* (\chi_\alpha \varphi)| \dif \mu_\alpha(k).$$
The inner integral is controlled by $||\varphi||_{C^0(U_\alpha)} \cdot |U_\alpha|$ where $U_\alpha$ is the image of $F_\alpha$.
The outer integral is then well-defined because it is against a Radon measure.

We next observe that the choice of partition of unity is irrelevant, thus if $\varphi$ has compact support in $U_\alpha \cap U_\beta$, then
\begin{equation}\label{well-defined of Ruelle-Sullivan}
\int_{K_\alpha} \int_{\RR^{d - 1} \times \{k\}} (F_\alpha^{-1})^* \varphi \dif \mu_\alpha(k) = \int_{K_\beta} \int_{\RR^{d - 1} \times \{k\}} (F_\beta^{-1})^* \varphi \dif \mu_\beta(k).
\end{equation}
Indeed,
\begin{align*}
\int_{K_\alpha} \int_{\RR^{d - 1} \times \{k\}} (F_\alpha^{-1})^* \varphi \dif \mu_\alpha(k)
&= \int_{K_\beta} (F_\alpha F_\beta^{-1})^* \left[\int_{\RR^{d - 1} \times \{k\}} (F_\alpha^{-1})^* \varphi \dif \mu_\alpha(k)\right] \\
&= \int_{K_\beta} \left[\int_{\RR^{d - 1} \times \{k\}} (F_\beta^{-1})^* \varphi\right] (F_\alpha F_\beta^{-1})^* \dif \mu_\beta(k) \\
&= \int_{K_\beta} \int_{\RR^{d - 1} \times \{k\}} (F_\beta^{-1})^* \varphi \dif \mu_\beta(k)
\end{align*}
where the last equation is because of the measure-preserving nature of the transition maps; this proves (\ref{well-defined of Ruelle-Sullivan}).

Finally, if a $d-2$-form $\psi$ has compact support in a single flow box, then
$$\int_{\RR^{d - 1} \times \{k\}} (F_\alpha^{-1})^* \dif \psi = \int_{\RR^{d - 1} \times \{k\}} \dif((F_\alpha^{-1})^* \psi) = 0$$
by Stokes' theorem, so
\begin{align*}
\int_M \dif T_\mu \wedge \psi &= -\int_M T_\mu \wedge \dif \psi \\
&= -\int_{K_\alpha} \int_{\RR^{d - 1} \times \{k\}} (F_\alpha^{-1})^* \dif \psi \dif \mu_\alpha(k) = 0. \qedhere
\end{align*}
\end{proof}

Though (\ref{RS current}) is the more traditional way of stating the definition of a Ruelle-Sullivan current, there is a more intrinsic way as well.
We first observe that if $\mu$ is a transverse measure, then $\mu$ defines a measure on $\supp \lambda$: in each flow box $F_\alpha$, an open set $U$ has measure
\begin{equation}\label{transverse measure of an open set}
\mu(U) := \int_{K_\alpha} |F_\alpha(\RR^{d - 1} \times \{k\}) \cap U| \dif \mu_\alpha(k).
\end{equation}

\begin{lemma}
For an oriented measured lamination $(\lambda, \  mu)$, the polar decomposition of $T_\mu$ is
\begin{equation}\label{polar ruelle sullivan}
T_\mu = \normal_\lambda \mu.
\end{equation}
\end{lemma}
\begin{proof}
For an open set $U \subseteq M$ in a flow box $F_\alpha$, the total variation measure $|T_\mu|$ satisfies
$$|T_\mu|(U) = \sup_{||\varphi||_{C^0} \leq 1} \int_{K_\alpha} \int_{\RR^{d - 1} \times \{k\}} \varphi \dif \mu_\alpha(k)$$
where the supremum ranges over $d-1$-forms $\varphi$ with compact support in $U$.
However, $\star \normal_\lambda$ is the Riemannian measure on $F_\alpha(\RR^{d - 1} \times \{k\})$, so
$$\int_{\RR^{d - 1} \times \{k\}} \varphi \leq \int_{\RR^{d - 1} \times \{k\}} (F_\alpha^{-1})^*(\star \normal_\lambda).$$
Since $||\normal^\lambda||_{C^0} = 1$, it follows that a sequence of cutoffs of $\star \normal_\lambda$ to more and more of $U$ is a maximizing sequence.
Therefore $\normal_\lambda$ is the polar part of (\ref{polar ruelle sullivan}), and
$$|T_\mu|(U) = \int_{K_\alpha} \int_{\RR^{d - 1} \times \{k\}} (F_\alpha^{-1})^*(1_U \star \normal_\lambda) \dif \mu_\alpha(k).$$
The inner integral is the Riemannian measure of $F_\alpha(\RR^{d - 1} \times \{k\}) \cap U$, so by (\ref{transverse measure of an open set}), $|T_\mu| = \mu$.
\end{proof}

%%%%%%%%%%%%%%%%%%%%%%%%%%%%
\subsection{Functions of bounded variation and least gradient}
\begin{proposition}[Miranda stability theorem]
	If a sequence of functions $(u_n)$ (not necessarily of the same trace) is bounded in $L^1_\loc(M)$ and satisfies for every open $U \Subset M$ with Lipschitz boundary
\begin{equation}\label{boundedness in Miranda}
	\limsup_{n \to \infty} \int_U \star |\dif u_n| \leq \liminf_{n \to \infty} \eta(u_n, U) < \infty,
\end{equation}
	then there exists a function $u$ of least gradient such that along a subsequence, $u_n \to u$ in $L^1_\loc(M)$ and $\dif u_n \to \dif u$ in the weak topology of measures.
\end{proposition}
\begin{proof}
The forgetful map $BV_\loc(M) \to L^1_\loc(M)$ is compact, so (\ref{boundedness in Miranda}) and the bounds in $L^1_\loc(M)$ imply that $(u_n)$ has a convergent subsequence.
The rest of the proof is similar to \cite[Teorema 3 and Osservazione 3]{Miranda67}; see \cite[\S2]{BackusFLG} for the straightforward modifications.
\end{proof}

\begin{proposition}\label{doubling dimension}
If $d \leq 7$ then there exists $A \geq 0$ such that for every set $U$ of least perimeter in a ball $B_r = B(P, r)$, $P \in \partial^* U$,
$$|\Ball^{d - 1}| e^{-Ar^2} r^{d - 1} \leq |\partial^* U \cap B_r| \leq |\Sph^{d - 1}| e^{Ar^2} r^{d - 1}.$$
\end{proposition}
\begin{proof}
This consequence of the monotonicity formula for $1$-harmonic functions is proven in \cite[TODO]{BackusFLG}.
\end{proof}

\begin{proposition}[maximum principle]\label{max princip}
Let $u$ be a $1$-harmonic function.
If $u$ attains a local maximum, then $u$ is constant.
\end{proposition}
\begin{proof}
Let $y$ be a local maximum of $u$; then $\partial \{u \geq y\}$ is a minimal hypersurface by Theorem \ref{main thm of old paper}, or else it is empty. If it is empty, then $M = \{u \geq y\}$ and $u$ is constant, so we exclude this case.
In a neighborhood of any point of $\partial \{u \geq y\}$ we may decrease $\int \star |\dif u|$ by decreasing $y$ a small amount, which violates that $u$ has least gradient.
\end{proof}

%%%%%%%%%%%%%%%%%%%%%%%%%%%%%%%%%%%%%%%%%%
\section{Regularity of flow boxes}\label{Regularity}
The goal of this section is to prove the following regularity theorem for minimal laminations that we will use several times.
The proof is based on \cite[Theorem 1.1]{Solomon86}, which addresses the case that $\lambda$ is a minimal foliation, and does not explictly spell out the $W^{1, \infty}$ norm and conorm of the laminar flow box.

\begin{proposition}\label{regularity theorem}
Let $\lambda$ be a minimal lamination in $M$. Then:
\begin{enumerate}
\item There exists a Lipschitz subbundle of $TM$ which restricts to a normal bundle to each of leaves of $\lambda$.
\item There exist $A, r > 0$ which only depend on $g$ and $\sup_{N \in \Leaves \lambda} ||\Two_N||_{C^0}$, and a Lipschitz laminar atlas $(F_\alpha)$ for $\lambda$, such that for every $\alpha$,
\begin{equation}\label{conorm of flow box}
	\max(\Lip(F_\alpha), \Lip(F_\alpha^{-1})) \leq A,
\end{equation}
and the image of $F_\alpha$ contains a ball of size $r$.
\end{enumerate}
\end{proposition}

To begin the proof we first consider when we can represent the leaves of $\lambda$ as graphs in a uniform way.

\begin{definition}
	A minimal lamination $\lambda$ is \dfn{graph-representable} in $B(P, r)$ if there exist exponential normal coordinates $(x, y) \in \RR^{d - 1} \times \RR$ based at $P$ and a cylinder $Z = \{|x| < s, |y| < t\}$ containing $B(P, r/5)$ so that for every $N \in \Leaves \lambda$, there exists a function $f$ such that
	$$N \cap Z = \{y = f(x)\}.$$
\end{definition}

\begin{lemma}
	Let $\lambda$ be a minimal lamination.
	There exists $r > 0$ which only depends on $g$ and $\sup_{N \in \Leaves \lambda} ||\Two_N||_{C^0}$ such that $\lambda$ is graph-representable in $B(P, r)$.
\end{lemma}
\begin{proof}
	Suppose not. Then there exist minimal laminations $\lambda_n'$ such that $\sup_{N \in \Leaves \lambda_n'} ||\Two_N||_{C^0} \leq 1$ but $\lambda_n'$ is not graph-representable in $B(P, 1/n)$.
	After rescaling and applying the exponential map, we see that this contradiction assumption implies the following:

\begin{sublemma}
	For each $n \geq \inj(g)$ there exists a minimal lamination $\lambda_n$ in $\RR^d$ with respect to the metric
	$$\tilde g_{n|\mu \nu} = \delta_{\mu \nu} - \frac{1}{3n^2} \Riem_{\mu \alpha \beta \nu}(g) x^\alpha x^\beta - \cdots$$
	which is not graph-representable in $B_1$, but satisfies
\begin{equation}\label{bounds on Two in representation}
	\sup_{N \in \Leaves \lambda_n} ||\Two_N||_{C^0} \leq \frac{1}{n^2}.
\end{equation}
\end{sublemma}

\begin{sublemma}
	There exists $n_* = n_*(g)$ such that for every $n \geq n_*$, and every leaf $N$ of $\lambda_n$, there exist tubular neighborhoods $T_N$ of hyperplanes in $B_1$ of radius $O(n^{-1})$, so that $N$ is contained in $T_N$. Furthermore, if $N, N'$ are leaves of $\lambda_n$ which pass through $B_{1/4}$, then $T_N \cap B_{1/2}$ and $T_{N'} \cap B_{1/2}$ are disjoint.
\end{sublemma}
\begin{proof}
The bounds (\ref{bounds on Two in representation}), plus the fact that $\tilde g_{n|\mu \nu} \to \delta_{\mu \nu}$ quadratically fast depending on $g$, imply that for $n \geq n(g)$, there exists a tubular neighborhood $T_N$ of a hyperplane $L_N$ in $B_1$ of radius $O(n^{-1})$ which contains $N$ TODO.

To prove the disjointness, let $N, N' \in \Leaves \lambda_n$.
Since $T_N$ has radius $O(n^{-1})$, if $T_N \cap T_{N'} \cap B_{1/3}$ is nonempty and $n$ is larger than an absolute constant, then $L_N \cap L_{N'} \cap B_{1/2}$ is nonempty.
But $N \to L_N$ in $C^0$ by construction, so then $N \cap N'$ is nonempty, and hence $N = N'$ since $\lambda_n$ is a lamination.
\end{proof}

By (\ref{bounds on Two in representation}), if $n$ is large enough, then each leaf $N \in \Leaves \lambda_n$ which meets $B_{1/4}$ can individually be represented as a graph, possibly after rotating $\RR^d$.
Such a rotation $R_N$ can be chosen so that $T_N$ takes the form $\{|y - y_N| \lesssim n^{-1}\}$ for some $y_N \in (-1, 1)$.

\end{proof}

\begin{lemma}\label{lams have C0 fields}
Let $\lambda$ be a minimal lamination. Then there locally exists a continuous normal vector field to $\lambda$.
\end{lemma}
\begin{proof}
Let $N$ be a leaf of $\lambda$, and $P \in N$.
We show the continuity at $P$.
The assertion is clear if $P$ is an isolated leaf, so let $P_n \in N_n$ and $P_n \to P$.
Close to $P$, the leaves $N_n$ are stable minimal, and hence bound sets $U_n$ of least perimeter.
So by the Miranda stability theorem and the uniform bounds given by Proposition \ref{doubling dimension}, $\dif 1_{U_n}$ converges to an exact $d-1$-current $\dif 1_U$ in the weak topology of measures, where $U$ has least perimeter.
By a consequence of the portmanteau theorem, $U$ must be bounded by $N$.
Now if we set $\normal_n := \normal_{N_n}(P_n)$, $\normal := \normal_N(P)$, and $\varphi$ an extension of $\star \normal$ to a tubular neighborhood of $M$, and we assume that there exists $\delta > 0$ such that on some subsequence,
$$|\sin \angle(\normal_n, \normal)| \geq \delta,$$
then for every $\varepsilon > 0$,
$$\int_{B(P, \varepsilon)} \dif 1_U \wedge \varphi = |N \cap B(P, \varepsilon))|,$$
so by the portmanteau theorem, for almost every small $\varepsilon$,
$$\lim_{n \to \infty} \frac{\int_{B(P, \varepsilon)} \dif 1_{U_n} \wedge \varphi}{|N_n \cap B(x, \varepsilon)|} = \frac{\int_{B(P, \varepsilon)} \dif 1_U \wedge \varphi}{|N \cap B(x, \varepsilon)|} = 1,$$
but at the same time, for $n$ large,
$$\int_{B(P, \varepsilon)} \dif 1_{U_n} \wedge \varphi = \int_{N_n \cap B(P, \varepsilon)} \normal_n \star |\dif 1_{U_n}| \wedge \normal \leq (1 - O(\delta)) |N_n \cap B(P, \varepsilon)|,$$
which implies $\delta = 0$, a contradiction.
Therefore $\normal_n \to \normal$, as desired.
\end{proof}

TODO: Replace the above with a theorem that says that the charts have uniform size wrt $\Two$

We should point out that \cite{Solomon86} proves the above lemma by a very different means, using the regularity theory for integral flat convergence of minimal currents \cite[Theorem 5.3.14]{federer2014geometric}.
We could do this as well, but for the sake of exposition, we prefer to avoid the highly technical machinery of \cite[Chapter 5]{federer2014geometric}.

\begin{proof}[Proof of Proposition \ref{regularity theorem}]
By Lemma \ref{lams have C0 fields}, we may work in a small open set, which admits cylindrical coordinates $(x, y) \in 3\Ball^{d - 1} \times (-2, 2)$ whose origin $P$ satisfies $\normal(P) = \partial_y$, and
\begin{equation}\label{normal is almost constant}
||\normal - \partial_y||_{C^0} < \delta.
\end{equation}
If $\delta$ is chosen small enough, then we may assume that in $2\Ball^{d - 1} \times (-1, 1)$,
every leaf is the graph of a function, say $u_k: 3\Ball^{d - 1} \to (-2, 2)$ where $u_k(0) = k$.
Then $u_k$ solves a quasilinear elliptic PDE $Pu_k = 0$, so that the ellipticity of $P$ and H\"older norms of the coefficients of $P$ only depend on $g$, but not on $k$ or $\delta$.
By a straightforward modification of \cite[Corollary 16.7]{gilbarg2015elliptic}, for every $m \geq 0$ there exist $C_m = C_m(g) > 0$ such that
\begin{equation}\label{norms on uk}
\sup_{|k| < 1} ||u_k||_{C^m} \leq C_m.
\end{equation}
Now for $k \in (-1, 1)$ fixed, let $k < \ell < 1$, and let $v_{\ell k} := u_\ell - u_k$.
Then $v_{\ell k}$ is the difference of two elements of the kernel of $P$, so $v_{\ell k}$ solves a linear elliptic PDE $Q_k v_{\ell k} = 0$, where the ellipticity of $Q_k$ and the H\"older norms of the coefficients of $P$ only depend on $g$ and $C_m$ for some $m$, but not on $\ell, k, \delta$.
Moreover, $v_{\ell k} > 0$: clearly $v_{\ell k} \geq 0$, and if $v_{\ell k}(x) = 0$ for some $x$, then $v_{\ell k} = 0$ by the maximum principle, which implies $k = \ell$, a contradiction.
By the Schauder \cite[Theorem 6.2]{gilbarg2015elliptic} and Harnack \cite[Theorem 9.25]{gilbarg2015elliptic} inequalities, for every $x \in \Ball^{d - 1}$,
$$||\dif v_{\ell k}||_{C^0(\Ball^{d - 1})} \lesssim_{C_m, g} ||v_{\ell k}||_{C^0(2 \Ball^{d - 1})} \lesssim_{C_m, g} \inf_{C^0(\Ball^{d - 1})} v_{\ell k} \leq v_{\ell k}(x).$$
In particular, for every $x$,
$$|\dif u_\ell(x) - \dif u_k(x)| \lesssim_{C_m, g} |u_\ell(x) - u_k(x)|$$
so there exists $C = C'(C_m, g)$ such that
\begin{equation}\label{vertical Lipschitz}
|\normal(x, u_\ell(x)) - \normal_k(x, u_k(x))| \leq C |u_\ell(x) - u_k(x)|.
\end{equation}

To extend (\ref{vertical Lipschitz}) to a Lipschitz bound on $\normal$, let $X_1, X_2 \in (\Ball^{d - 1} \times (-1, 1)) \cap \supp \lambda$.
Then there exist $x_1, x_2 \in \Ball^{d - 1}$ and $k_1, k_2 \in (-1, 1)$ such that $X_i = (x_i, u_{k_i}(x_i))$.
Setting $Y := (x_2, u_{k_1}(x_2))$,
$$|\normal(X_1) - \normal(X_2)| \leq |\normal(X_1) - \normal(Y)| + |\normal(Y) - \normal(X_2)|.$$
Then by (\ref{norms on uk}) and the mean value theorem,
$$|\normal(X_1) - \normal(Y)| \lesssim |\dif u_{k_1}(x_1) - \dif u_{k_1}(x_2)| \leq C_2 |X_1 - Y|.$$
Moreover, by (\ref{vertical Lipschitz}),
$$|\normal(Y) - \normal(X_2)| \leq C|u_{k_1}(x) - u_{k_2}(x)| = C|Y - X_2|.$$
If $\delta$ is chosen small enough, then by (\ref{normal is almost constant}),
$$|\sin \angle(X_1 - Y, X_2 - Y)| > 1 - O(\delta)$$
and we conclude by the Pythagorean theorem that
$$|Y - X_2|^2 + |X_1 - Y|^2 \lesssim |X_1 - X_2|^2.$$
In conclusion,
\begin{equation}\label{lipschitz normal}
|\normal(X_1) - \normal(X_2)| \lesssim_g |X_1 - X_2|
\end{equation}
which implies that $\normal$ is Lipschitz on $V \cap \supp \lambda$, where $V$ is the neighborhood of $P$ which was mapped to $\Ball^{d - 1} \times (-1, 1)$ by the cylindrical coordinates $(x, y)$.
Taking a Lipschitz extension, we define $\normal$ on all of $V$.
Covering $M$ by such neighborhoods $V(P)$ we obtain the desired Lipschitz subbundle.

Now we build a laminar flow box $F$ on a neighborhood of $P$ satisfying (\ref{conorm of flow box}).
Let
\end{proof}

%%%%%%%%%%%%%%%%%%%%%%%%%%%%%%%%%%%%%%%%%
\section{Compactness}\label{CompactnessSec}
In this section we prove Theorems \ref{compactness theorem} and \ref{implication theorem}.

%%%%%%%%%%%%%%%%%%%%%%%%%%%%%%%%%%%%%%%
\subsection{Weak topology of measures versus Thurston's geometric topology}
We first show that for a minimal lamination, convergence in the weak topology of measures implies convergence in Thurston's geometric topology.
Thurston claimed this fact \cite[Proposition 8.10.3]{thurston1979geometry} in case $d = 2$, but his proof left something to be desired as it did not justify why the limit is geodesic, or why the convergence respects the normal vectors.

\begin{lemma}\label{limits of measured geodesic lams are geodesic}
	The set of minimal measured laminations is closed in the weak topology of measures.
\end{lemma}
\begin{proof}
Let $(\lambda, \mu)$ be a measured lamination and suppose that $(\lambda_i, \mu_i) \to (\lambda, \mu)$ in the weak topology of measures, where $(\lambda_i, \mu_i)$ are measured minimal.
Let $x \in \supp \lambda$ and $r > 0$ such that $B := B(x, r)$ is contractible.
In $B$, we can write $T_{\mu_i} = \dif u_i$ for some sequence of functions of least gradient $u_i \in BV(B)$.
Since $u_i$ is only defined up to a constant, we impose $\int_M \star u_i = 0$, so by Poincar\'e's inequality,
$$||u_i||_{L^1(B)} \lesssim r\mu_i(B) \leq 2r \mu(B) < \infty$$
for $i$ large.
So by the Miranda stability theorem, there exists a $1$-harmonic function $u$ such that along a subsequence, $\dif u_i \to \dif u$ in the weak topology of measures.
But then we must have $T = \dif u$, so $\lambda$ is minimal by Theorem \ref{main thm of old paper}.
\end{proof}

\begin{proposition}\label{measured implies Thurston}
Let $(\lambda_i, \mu_i)$ be measured minimal laminations.
If $(\lambda_i, \mu_i) \to (\lambda, \mu)$ in the weak topology of measures, then $\lambda$ is a minimal lamination and $\lambda_i \to \lambda$ in Thurston's geometric topology.
\end{proposition}
\begin{proof}
We first show that
\begin{equation}\label{support is nonincreasing}
	\supp \lambda \subseteq \liminf_{i \to \infty} \supp \lambda_i.
\end{equation}
Indeed, if $x \in \supp \lambda$ then for all $\varepsilon > 0$, $\mu(B(x, \varepsilon)) > 0$.
Then by the portmanteau theorem, for all $i$ large, $\mu_i(B(x, \varepsilon)) > 0$; this proves (\ref{support is nonincreasing}).
It follows that for every $x \in \supp \lambda$, $\varepsilon > 0$, and large $i$, there exists $y \in \supp \lambda_i \cap B(x, \varepsilon)$.
By Lemma \ref{limits of measured geodesic lams are geodesic}, $\lambda$ is a minimal lamination.

To get convergence of the normals we choose a Lipschitz $d-1$-form $\varphi$ which extends $\star \normal$ (TODO: This is the regularity theorem).
Then for every $\varepsilon > 0$,
$$\int_{B(x, \varepsilon)} T_\mu \wedge \varphi = \mu(B(x, \varepsilon))$$
so by the portmanteau theorem, for almost every $\varepsilon > 0$,
\begin{equation}\label{epsilon is a continuity set}
	\lim_{i \to \infty} \frac{\int_{B(x, \varepsilon)} T_{\mu_i} \wedge \varphi}{\mu_i(B(x, \varepsilon))} = \frac{\int_{B(x, \varepsilon)} T_\mu \wedge \varphi}{\mu(B(x, \varepsilon))} = 1.
\end{equation}
On the other hand, if we assume that there exists $\delta, \varepsilon > 0$ such that for every $y \in \supp \lambda_i \cap B(x, \varepsilon)$,
$$|\sin \angle(\normal_i, \normal)| \geq \delta,$$
then possibly after shrinking $\varepsilon$ we may assume that (\ref{epsilon is a continuity set}) holds, hence
$$\int_{B(x, \varepsilon)} T_{\mu_i} \wedge \varphi = \int_{B(x, \varepsilon)} \normal_i\mu_i \wedge \star \normal \leq (1 - O(\delta)) \mu_i(B(x, \varepsilon))$$
and therefore $\delta = 0$, a contradiction.
\end{proof}

%%%%%%%%%%%%%%%%%%%%%%%

\subsection{Hausdorff distance versus Thurston's geometric topology}
Convergence in Thurston's geometric topology can be annoying to check, as it requires one to check the convergence of the normal vectors as well as the leaves themselves.
Here we eliminate the need to check convergence of the normal vectors under certain hypotheses.

\begin{proposition}\label{convergence of geodesic lams in thurston}
Assume that $(\lambda_i)$ is a sequence of minimal laminations, and $\lambda$ a minimal lamination, such that for every leaf $N$ of $\lambda$ and every large $i$, there is a leaf $N_i$ of $\lambda_i$ such that $N_i \to N$ in $\Hypspace M$.
Furthermore, suppose that $(\lambda_i)$ has bounded curvature.
Then $\lambda_i \to \lambda$ in Thurston's geometric topology.
\end{proposition}
\begin{proof}
Suppose not. Then, there exists $P \in N$ such that for every $P_i \in N_i$ such that $P_i \to P$, $\normal_i := \normal_{N_i}(P_i)$ does not converge to $\normal := \normal_N(P)$ in the cosphere bundle $S'M$.
In normal coordinates $(x, y')$ based at $P$, $N$ can be viewed as the graph of a function $\{y' = w(x)\}$; we then set $y := y' - w(x)$, so that $N = \{y = 0\}$.
Using the coordinates $(x, y)$ we identify $M$, and each of its cotangent spaces, with $\RR^d$.

Possibly after taking a subsequence of $(\lambda_i)$, we may choose $\varepsilon > 0$ such that no matter what sequence $(P_i)$ we choose,
\begin{equation}\label{thurston angle defect}
	|\sin \angle(\normal_i, \normal)| \geq \varepsilon.
\end{equation}
The second fundamental form $\Two_i'$ of $N_i$ with respect to the euclidean metric on $\RR^d$ is bounded in terms of the curvature tensors of $M$ and $N$, and the second fundamental form $\Two_i$ of $N_i$ with respect to $M$, in $B(P, 2\varepsilon)$, as long as $\varepsilon$ is small.
But $\Two_i$ is uniformly bounded since $(\lambda_i)$ has bounded curvature. TODO: Spell this out. What does ``sin'' even mean in this context?
The resulting uniform bound on $\Two_i'$, combined with (\ref{thurston angle defect}), furnishes a small $\delta > 0$ independent of $i$ such that $N_i$ \dfn{avoids cones} in the sense that for every $v \in \RR^d$ such that $0 < |v| < \delta$ and $|\sin v| < \varepsilon/2$, $P_i + v \notin N_i$.
After shrinking $\varepsilon$, we may assume that $\varepsilon < \max(\delta/2, 1/100)$.

To obtain a contradiction, we choose $Q \in N$ such that $\dist(P, Q) = \varepsilon$, which can be done as long as $\varepsilon$ is small.
For $i$ large, $|P_i| < \varepsilon^2/10$, so
$$0 < \varepsilon - \varepsilon^2 \leq |Q - P_i| < 2\varepsilon < \delta$$
and hence (using a superscript $j$ to indicate the $j$th coordinate)
$$|y^1 - x_i^1| \geq |y - x_i| - |x_i^2| \geq \varepsilon - 2\varepsilon^2.$$
Consider the triangle $\Delta$ whose vertices are $x_i, y$, and $z_i := (y^1, x_i^2)$.
Then $\Delta$ is a right triangle, and its smallest angle $\theta_\Delta$ satisfies
$$|\sin \theta_\Delta| = \frac{|x_i^2|}{|y^1 - x_i^1|} < \frac{\varepsilon^2}{\varepsilon - 2\varepsilon^2} < \frac{\varepsilon}{4}.$$
Thus for every large $i$, $y$ is contained in the cone avoided by $N_i$, and in fact for any $y_i \in N_i$, $\dist(y, y_i) \geq \varepsilon/4$.
But we argued above that any point of $N$ could be approximated by points of $N_i$ for $i$ large, so this is a contradiction.
\end{proof}

%%%%%%%%%%%%%%%%%%%%%
\subsection{Compactness}

\begin{lemma}\label{limit of minimals is minimal}
Let $(N_i)$ be a sequence of complete embedded minimal hypersurfaces of uniformly bounded curvature and $N_i \to N$ in $\Hypspace M$.
Then $N$ is a complete minimal hypersurface.
\end{lemma}
\begin{proof}
In a neighborhood of any point on $N$, we can write $N_i = \partial U_i$ where $U_i$ is a set of least perimeter.
Let $u_i := 1_{U_i}$; then $u_i$ is a function of least gradient.

If we set $N = \partial U$, $u := 1_U$, then $u_i \to u$ we claim that pointwise away from $N$, and hence almost everywhere.
This can be seen by writing $N_i$ as the graph of $f_i$, so $U_i = \{y < f_i(x)\}$, and this is possible due to the assumption of bounded curvature.
Similarly we write $U = \{y < f(x)\}$.
Now if $(x, y) \in U$ but $(x, y) \notin U_i$ for arbitrarily large $i$, then $f_i(x) \leq y < f(x)$ so $(x, f_i(x)) \in N_i$ converges to a point $(x, \tilde y)$ with $\tilde y \neq y$, thus $(x, \tilde y) \notin N$.
This violates the convergence $N_i \to N$ in $\Hypspace N$.

So $u_i \to u$ almost everywhere, and hence by dominated convergence also converges in $L^1$.
By the Miranda stability theorem it follows that $U$ has least perimeter, and so Theorem \ref{main thm of old paper}, $N$ is a complete minimal hypersurface.
\end{proof}


\begin{proposition}\label{compactness in flow boxes and Thurston}
Let $(\lambda_i)$ be a tight sequence of finite minimal laminations with bounded curvature. Then:
\begin{enumerate}
\item After passing to a subsequence, we may assume that $(\lambda_i)$ converges to a maximal limit $\lambda$ on the level of flow boxes, and hence in Thurston's geometric topology.
\item The limit $\lambda$ is a Lipschitz minimal lamination satisfying the same curvature bounds as $(\lambda_i)$.
\item If $\mu_i$ is a transverse measure to $\lambda_i$ and for every compact $K \subseteq M$, $\mu_i(K) \lesssim_K 1$, then after passing to a further subsequence, we may assume that $(\lambda_i, \mu_i)$ in fact converges to some measured minimal lamination $(\tilde \lambda, \mu)$ in the weak topology of measures, where $\tilde \lambda$ is a sublamination of the maximal lamination $\lambda$.
\end{enumerate}
\end{proposition}
\begin{proof}
We first select $x_0 \in M$; we will construct suitable flow boxes $F_i$ for $\lambda_i$ with image $B(x_0, r)$.
Here $r$ is to be determined, but at least is smaller than $R(x_0)/2$, where $R(x_0)$ is the injectivity radius of $M$ at $x_0$.

Let $\varepsilon > 0$.
After selecting $r$ small depending on $g$, $\varepsilon$, and the curvature bounds on $\lambda_i$, and rescaling $g$, we may assume that the leaves $N_{i1}, \dots, N_{ik_i}$ have all second fundamental forms of size $\leq \varepsilon$ on $B(x_0, r)$.
Then if $r$ is small enough depending on $g$, the second fundamental forms $\Two_{ij}$ of $N_{ij}$ with respect to the euclidean metric in $x_0$-normal coordinates are of size $\leq 2\varepsilon$ on $B(x_0, r)$.

If $r$ is small enough, then because there is a Lipschitz conormal $1$-form $\normal_i$ to $\lambda_i$ in $B(x_0, r)$, we can represent $N_{i1}, \dots, N_{ik_i}$ as graphs over the equatorial hyperplane in $B(x_0, r)$, possibly after applying a rotation $R_i$.
The uniform bounds on $\Two_{ij}$ ensure that $r$ is independent of $i$.
Thus we have $x_0$-normal coordinates $(x_i, y_i)$ on $B(x_0, r)$ where
$$N_{ij} = \{y_i = f_{ij}(x_i)\}$$
and $\normal_i(0, 0) = \dif y_i$.
Since $|\Two_{ij}| \leq 2\varepsilon$, if $\varepsilon \leq 1$,
\begin{equation}\label{bound on derivatives}
||\dif f_{ij}||_{C^0} \leq |\normal_i(0, f_{ij}(0))| + O(\varepsilon r) \lesssim (1 + \varepsilon) r \lesssim r.
\end{equation}

By the maximum principle, if $f_{ij}(x_i) = f_{ij'}(x_i)$ for some $i, j, j'$, then $j = j'$.
So after applying a permutation to $\{1, \dots, k_i\}$ we may assume that $j < j'$ implies $f_{ij} < f_{ij'}$.
Then if we set $u_{ij} := f_{i,j+1} - f_{ij}$, then $u_{ij} > 0$ by the maximum principle, and we get a second-order divergence-form operator $P_{ij}(\varepsilon, r)$ such that
\begin{equation}\label{elliptic PDE}
-\Delta u_{ij} = P_{ij}(\varepsilon, r) u_{ij}
\end{equation}
and $P_{ij}(\varepsilon, r)$ is perturbative as $\varepsilon, r \to 0$ in the sense that its coefficients can be made arbitrarily small in $C^0(B_{r/2})$ by taking $\varepsilon, r$ small.
In fact, by \cite[Chapter 7]{colding2011course}, if we set
$$F(x, y, p, q) := h^{ij} (q_{ij} + \Gamma_{ij}^0 + p_i \Gamma_{0j}^0 + p_j \Gamma^0_{i0} + p_i p_j \Gamma^0_{00}) - p_m h^{ij} (\Gamma_{ij}^m + p_i \Gamma_{0j}^m + p_i \Gamma_{i0}^m + q_{ij} \Gamma_{00}^m),$$
where $\Gamma_{\mu \nu}^\lambda$ are Christoffel symbols for $M$ at $(x, y)$, $x_0 := (0, 0)$, the zero index corresponds to $y$, and $h^{ij}$ are the components of the inverse of
$$h_{ij} = g_{ij} + p_i g_{j0} + p_j g_{0i} + p_i p_j g_{00}$$
evaluted at $(x, y)$, then $h_{ij} = \delta_{ij} + O(|x|^2 + y^2 + |p|^2)$, $\Gamma_{\mu \nu}^\lambda = O(|x|^2 + y^2)$, and
$$F(x, f_{i, j + 1}(x), \dif f_{i, j + 1}(x), \partial^2 f_{i, j + 1}(x)) - F(x, f_{ij}(x), \dif f_{ij}(x), \partial^2 f_{ij}(x)) = 0.$$
By (\ref{bound on derivatives}), we conclude that the dominant term is $\Delta u_{ij}$ and the perturbative terms are $O(r + \varepsilon)$ in $C^0$, as required by (\ref{elliptic PDE}).
TODO: Reindex all this, don't want the ij's to be Einstein indices and leaf indices
It follows that if $\varepsilon, r$ are small, then $-\Delta - P$ is uniformly elliptic, with ellipticity constant close to that of $-\Delta$.
So by the Harnack inequality, for any $0 < \delta < 1/2$,
\begin{equation}\label{Harnack bound}
\sup_{B_{\delta r}} u_{ij} \leq e^{O(\delta)} \inf_{B_{\delta r}} u_{ij}.
\end{equation}

We now set $\eta_{ij} := f_{ij}(0)$ and define
$$\varphi_i(\xi_i, \eta_i) := \sum_{j=1}^{k_j - 1} 1_{[\eta_{ij}, \eta_{i,j+1})}(\eta_i) \left[f_{ij}(\xi_i) + \frac{\eta_i - \eta_{ij}}{\eta_{i,j+1} - \eta_{ij}} u_{ij}(x_i)\right].$$
That is, the change of coordinates
$$(x_i, y_i) = (\xi_i, \varphi_i(\xi_i, \eta_i))$$
flattens out the graphs of the $f_{ij}$ to hyperplanes.
Moreover
\begin{align*}
\dif \varphi_i &= \sum_{j=1}^{k_j - 1} 1_{[\eta_{ij}, \eta_{i, j + 1})}(\eta_i) \left[\dif f_{ij} + \frac{\eta_i - \eta_{ij}}{\eta_{i,j+1} - \eta_{ij}} \dif u_{ij} + \frac{u_{ij}}{\eta_{i,j+1} - \eta_{ij}} \dif \eta_i\right] \\
&\qquad + \sum_{j = 2}^{k_j - 1} \delta_{\eta_{ij}}(\eta_i)\left[f_{i,j-1}(\xi_i) + u_{i,j-1}(\xi_i) - f_{i,j}(\xi_i)\right] \\
&= \sum_{j=1}^{k_j - 1} 1_{[\eta_{ij}, \eta_{i, j + 1})}(\eta_i) \left[\dif f_{ij} + \frac{\eta_i - \eta_{ij}}{\eta_{i,j+1} - \eta_{ij}} \dif u_{ij} + \frac{u_{ij}}{\eta_{i,j+1} - \eta_{ij}} \dif \eta_i\right].
\end{align*}
Now $\eta_i - \eta_{ij} \in [0, \eta_{i, j + 1} - \eta_{ij}]$ for $\eta_i \in [\eta_{ij}, \eta_{i,j+1})$, so the first two terms are bounded by (\ref{bound on derivatives}) as
$$\left|\left|\dif f_{ij} + \frac{\eta_i - \eta_{ij}}{\eta_{i,j+1} - \eta_{ij}} \dif u_{ij}\right|\right|_{C^0(B_{\delta r})}
\leq 3\sup_{j' \in \{1, \dots, k_i\}} ||\dif f_{ij'}||_{C^0(B_{\delta r})}
\lesssim \delta r.$$
From (\ref{Harnack bound}),
$$\frac{u_{ij}}{\eta_{i,j+1} - \eta_{ij}} = \frac{u_{ij}(0) + \sup_{B_{\delta r}} u_{ij} - \inf_{B_{\delta r}} u_{ij}}{u_{ij}(0)} = 1 + O(\delta r).$$
In conclusion
$$\dif \varphi_i = \dif \eta_i + O(\delta r).$$
Therefore the flow boxes $F_i(\xi_i, \eta_i) := (x_i, y_i)$ have Lipschitz norm and conorm in $(1/2, 3/2)$ if $\delta$ is chosen small enough independently of $i$.
So by a compactness argument, a subsequence of flow boxes and their inverses converge in $C^{1-}$ to a local diffeomorphism $F$.
Since $x_0$ was arbitrary, a diagonal argument ensures that we can cover $M$ by sets $U_\alpha := B(x_0, \delta r)$ in which these flow boxes $F_{i\alpha}$ converge to some local diffeomorphism $F_\alpha$.
By taking a further subsequence and rotating, we may assume that $R_i$ converges to the identity as well.

Since $(\lambda_i)$ is tight, we can find a precompact ball $B$ in $M$ so that every leaf of $\lambda_i$ passes through $B/2$.
Then by compactness of $\Hypspace \overline B$, we obtain limits of every sequence $(N_i)$ of $\lambda_i$ in $\Hypspace \overline B$.
By Lemma \ref{limit of minimals is minimal}, such a limit $N$ is a geodesic.
Then $F_{i\alpha}^{-1}(N_i \cap U_\alpha) = \{\eta = \eta_i\}$ for some $\eta_i$, and by the $C^{1-}$-convergence of the $F_i$, the hyperplanes $\{\eta = \eta_i\}$ must converge to some hyperplane $\{\eta = \eta_\infty\}$, so that $F_\alpha(\{\eta = \eta_\infty\}) = N \cap U_\alpha$.
So the limiting geodesics $N$ form a geodesic lamination $\lambda$ and $F_\alpha$ is a flow box for $\lambda$.
Diagonalizing against larger and larger compact balls, we extend the geodesic lamination $\lambda$ in $B$ to a lamination in $M$.

The above convergence is in Thurston's geometric topology.
Indeed, if $N$ is a leaf of $\lambda$, then we can find leaves $N_i$ such that $N_i \to N$ pointwise.
So by Lemma \ref{convergence of geodesic lams in thurston} and the fact that $N$ is a geodesic, we conclude the desired convergence.

Finally, if $\mu_i$ is transverse to $\lambda_i$, then we observe that since $(\lambda_i)$ is tight, we can find a compact subset $K$ of $M$ which meets every leaf of $\lambda$.
Then $\lambda, \lambda_i$ are orientable on $K \cap U_\alpha$, so we can take $T_{\mu_i}|_{K \cap U_\alpha}$ and obtain a convergent subsequence to some $T_\mu|_{K \cap U_\alpha}$; here we use the uniform bounds on $\mu_i(K)$.
Diagonalizing again, we obtain $\mu := |T_\mu|$, which is a transverse measure to $\lambda|_K$, and is globally defined since it is not oriented.
Since $K$ meets every leaf of $\lambda$, $\mu$ extends uniquely to all of $M$ as a transverse measure to $\lambda$.
\end{proof}

%%%%%%%%%%%%%%%%%%%%%%%%%
\subsection{Convergence in flow boxes versus the weak topology of measures}
\begin{proposition}\label{measures implies flow boxes}
Let $(\lambda_i, \mu_i)$ be measured geodesic laminations.
If $(\lambda_i, \mu_i) \to (\lambda, \mu)$ in the weak topology of measures, then $\lambda_i \to \lambda$ on the level of flow boxes.
\end{proposition}
\begin{proof}
The question is local, so we may remove all leaves of $\lambda$ except those that meet some precompact open set.
Then, for each $i$, let $(\lambda_{ij})_j$ be finite sublaminations of $\lambda_i$ converging to $\lambda_i$ on the level of flow boxes.
Then any subsequence $(i_k)$, has a further subsequence $(i_{k_\ell})$ such that $\lambda_{i_{k_\ell} i_{k_\ell}} \to \lambda$ on the level of flow boxes by Theorem \ref{compactness in flow boxes and Thurston}.
Indeed, as geodesic laminations, they have bounded curvature, and $(\lambda_i)$ is tight by construction.
So $\lambda_{i_{k_\ell}} \to \lambda$ on the level of flow boxes, but $(i_k)$ was arbitrary, so $\lambda_i \to \lambda$ on the level of flow boxes.
\end{proof}

%%%%%%%%%%%%%%%%%%%%%%%%%
\subsection{Putting it all together}
We now complete the proof of Theorems \ref{compactness theorem} and \ref{implication theorem}.

\begin{lemma}\label{reduction to finite case}
Let $(\lambda, \mu)$ be a measured minimal lamination.
Then there exist finite sublaminations $\lambda_i$ of $\lambda$ such that $\lambda$ is a maximal limit of $(\lambda_i)$ on the level of flow boxes.
\end{lemma}
\begin{proof}
Let $(x_i)$ be a dense sequence in $\supp \lambda$ and $N_i$ the leaf containing $x_i$.
Since $\lambda$ admits a transverse measure, $N_i$ is closed, so setting $\lambda_i := \bigcup_{j < i} N_j$, $\lambda_i$ is a finite lamination.
Now if $N$ is a leaf in $\lambda$, either $N = N_i$ for some $i$, in which case clearly $N$ is a limiting leaf of $\lambda_i$, or for every $x \in N$ there exists a subsequence $(x_{i_j})$ such that $x_{i_j} \to x$.
We claim that $N$ is a maximal limit of $N_{i_j}$ in Thurston's geometric topology.
To see why, we observe that $\normal_\lambda \in W^{1, \infty}$ TODO: regularity theorem so that $||\Two_{\lambda_i}||_{L^\infty} \lesssim 1$.
Since the second fundamental form of each individual leaf is continuous, it follows that $||\Two_{N_i}||_{C^0} \lesssim 1$ and so $(\lambda_i)$ has bounded curvature.
The desired convergence in Thurston's geometric topology now follows from Proposition \ref{convergence of geodesic lams in thurston}.
Finally, if $F$ is a flow box for $\lambda$ near $x$, then $F$ is also a flow box for $N_i$ for every $i$, hence the convergence in flow boxes.
\end{proof}

%%%%%%%%%%%%%%%%%%%%%%%%%%%%%%%%%%%%%%%%%
%%%%%%%%%%%%%%%%%%%%%%%%%%%%%%%%%%%%%%%%%
\section{Construction of minimal laminations}\label{construction}
We now prove Theorems \ref{building a minimal lamination} and \ref{main thm}.

\begin{proof}[Proof of Theorem \ref{building a minimal lamination}]
Let $(x_i)$ be a dense sequence in $\supp \lambda$, and let $N_i$ be the leaf containing $x_i$.
Then the union of $N_1, \dots, N_i$ is the support of a lamination $\lambda_i$, so that the curvatures of $\lambda_i$ are locally uniformly bounded.
So $(\lambda_i)$ has a maximal subsequential limit $\tilde \lambda$ on the level of flow boxes by Theorem \ref{compactness in flow boxes and Thurston}.
In particular, the leaves of $\tilde \lambda$ are exactly the limits in Hausdorff distance of the leaves of $\lambda$, but $\Leaves \lambda$ is closed so $\lambda = \tilde \lambda$.
\end{proof}

\begin{proof}[Proof of Theorem \ref{main thm}]
Let $u$ be a $1$-harmonic function.
By Theorem \ref{main thm of old paper}, the level sets of $u$ are stable minimal hypersurfaces.
Moreover, if $y > z$, then $\{u > y\} \subseteq \{u > z\}$, so $\partial \{u > y\}$ lies on one side of $\partial \{u > z\}$.
By the maximum principle for minimal surfaces \cite[Corollary 1.28]{colding2011course}, it follows that either $\partial \{u > y\}$ and $\partial \{u > z\}$ are disjoint, or are equal.
So the set of level sets of $u$ meets the hypotheses of Theorem \ref{building a minimal lamination}, and so defines a minimal lamination $\lambda$.
Since the leaves of $\lambda$ are level sets, $\dif u$ is conormal to $\lambda$.

We now construct the transverse measure to $\lambda$.
In any oriented laminar coordinates $(k, x) \in K \times \RR^{d - 1}$ for $\lambda$, $\partial_x u = 0$, so $\star |\dif u|$ defines a measure $\mu$ on $K$: given $\alpha < \beta$, let
$$\mu([\alpha, \beta] \cap K) := u(\beta, x) - u(\alpha, x)$$
for any (and hence every, since $\partial_x u = 0$) $x \in \RR^{d - 1}$.
By Proposition \ref{max princip}, $u(\beta, x) > u(\alpha, x)$, so $\mu$ is a positive measure.
If $(k', x') \in K' \times \RR^{d - 1}$ is a different laminar coordinate system, and the transition map carries $\alpha, \beta$ to $\alpha', \beta'$, then
$$\mu'([\alpha', \beta'] \cap K') := u'(\beta', x') - u(\alpha', x') = u(\beta, x_1) - u(\alpha, x_2)$$
for some $x_1, x_2 \in \RR^{d - 1}$. Since $\partial_x u = 0$ this is exactly $\mu([\alpha, \beta] \cap K)$.
It follows that $\mu$ is transverse, and by construction $\mu$ lifts to $\star |\dif u|$ in $M$.
Therefore $\dif u$ is Ruelle-Sullivan for $\lambda$.

Now for the converse, if $\dif u$ is an exact Ruelle-Sullivan current for a minimal lamination $\lambda$, we must show that $u$ has least gradient.
If not, we can choose $v \in BV_\cpt(E)$ and an open set $E \subseteq M$ such that
$$\int_E \star |\dif u + \dif v| < \int_E \star |\dif u| < \infty.$$
By the coarea formula (see \cite[\S2]{BackusFLG} for a proof at this regularity) and the fact that the level sets of $u$ are stable minimal,
\begin{align*}
\int_E \star |\dif u| &= \int_{-\infty}^\infty |\partial \{u > y\} \cap E| \dif y \leq \int_{-\infty}^\infty |\partial^* \{u + v > y\}| \dif y \\
&= \int_E \star |\dif u + \dif v| < \int_E \star |\dif u|
\end{align*}
which is a contradiction.
\end{proof}


\printbibliography

\end{document}
