\documentclass[reqno,11pt]{amsart}
\usepackage[letterpaper, margin=1in]{geometry}
\RequirePackage{amsmath,amssymb,amsthm,graphicx,mathrsfs,url,slashed,subcaption}
\RequirePackage[usenames,dvipsnames]{xcolor}
\RequirePackage[colorlinks=true,linkcolor=Red,citecolor=Green]{hyperref}
\RequirePackage{amsxtra}
\usepackage{cancel}
\usepackage{tikz-cd}

% \setlength{\textheight}{9.3in} \setlength{\oddsidemargin}{-0.25in}
% \setlength{\evensidemargin}{-0.25in} \setlength{\textwidth}{7in}
% \setlength{\topmargin}{-0.25in} \setlength{\headheight}{0.18in}
% \setlength{\marginparwidth}{1.0in}
% \setlength{\abovedisplayskip}{0.2in}
% \setlength{\belowdisplayskip}{0.2in}
% \setlength{\parskip}{0.05in}
%\renewcommand{\baselinestretch}{1.05}

\title{Minimal laminations and level sets of $1$-harmonic functions}
\author{Aidan Backus}
\address{Department of Mathematics, Brown University}
\email{aidan\_backus@brown.edu}
\date{\today}

\newcommand{\NN}{\mathbf{N}}
\newcommand{\ZZ}{\mathbf{Z}}
\newcommand{\QQ}{\mathbf{Q}}
\newcommand{\RR}{\mathbf{R}}
\newcommand{\CC}{\mathbf{C}}
\newcommand{\DD}{\mathbf{D}}
\newcommand{\PP}{\mathbf P}
\newcommand{\MM}{\mathbf M}
\newcommand{\II}{\mathbf I}
\newcommand{\Hyp}{\mathbf H}
\newcommand{\Sph}{\mathbf S}
\newcommand{\Group}{\mathbf G}
\newcommand{\GL}{\mathbf{GL}}
\newcommand{\Orth}{\mathbf{O}}
\newcommand{\SpOrth}{\mathbf{SO}}
\newcommand{\Ball}{\mathbf{B}}

\newcommand*\dif{\mathop{}\!\mathrm{d}}

\DeclareMathOperator{\card}{card}
\DeclareMathOperator{\dist}{dist}
\DeclareMathOperator{\MeasLam}{MeasLam}
\DeclareMathOperator{\MinLam}{MinLam}
\DeclareMathOperator{\Lam}{Lam}
\DeclareMathOperator{\supp}{supp}
\DeclareMathOperator{\tr}{tr}

\newcommand{\Leaves}{\mathscr L}
\newcommand{\Lagrange}{\mathcal L}
\newcommand{\Hypspace}{\mathscr H}

\newcommand{\Two}{\mathrm{I\!I}}


\newcommand{\Hilb}{\mathcal H}
\newcommand{\Homology}{\mathrm H}
\newcommand{\normal}{\mathbf n}
\newcommand{\radial}{\mathbf r}
\newcommand{\evect}{\mathbf e}
\newcommand{\vol}{\mathrm{vol}}

\newcommand{\diam}{\mathrm{diam}}
\newcommand{\Ell}{\mathrm{Ell}}
\newcommand{\inj}{\mathrm{inj}}
\newcommand{\Lip}{\mathrm{Lip}}
\newcommand{\Riem}{\mathrm{Riem}}

\newcommand{\Bmu}{\boldsymbol \mu}
\newcommand{\Bnu}{\boldsymbol \nu}
\newcommand{\Blambda}{\boldsymbol \lambda}

\newcommand{\pic}{\vspace{30mm}}
\newcommand{\dfn}[1]{\emph{#1}\index{#1}}

\renewcommand{\Re}{\operatorname{Re}}
\renewcommand{\Im}{\operatorname{Im}}

\newcommand{\loc}{\mathrm{loc}}
\newcommand{\cpt}{\mathrm{cpt}}

\def\Japan#1{\left \langle #1 \right \rangle}

\newtheorem{theorem}{Theorem}[section]
\newtheorem{badtheorem}[theorem]{``Theorem"}
\newtheorem{prop}[theorem]{Proposition}
\newtheorem{lemma}[theorem]{Lemma}
\newtheorem{sublemma}[theorem]{Sublemma}
\newtheorem{proposition}[theorem]{Proposition}
\newtheorem{corollary}[theorem]{Corollary}
\newtheorem{conjecture}[theorem]{Conjecture}
\newtheorem{axiom}[theorem]{Axiom}
\newtheorem{assumption}[theorem]{Assumption}

\newtheorem{mainthm}{Theorem}
\renewcommand{\themainthm}{\Alph{mainthm}}

% \newtheorem{claim}{Claim}[theorem]
% \renewcommand{\theclaim}{\thetheorem\Alph{claim}}
\newtheorem*{claim}{Claim}

\theoremstyle{definition}
\newtheorem{definition}[theorem]{Definition}
\newtheorem{remark}[theorem]{Remark}
\newtheorem{example}[theorem]{Example}
\newtheorem{notation}[theorem]{Notation}

\newtheorem{exercise}[theorem]{Discussion topic}
\newtheorem{homework}[theorem]{Homework}
\newtheorem{problem}[theorem]{Problem}

\makeatletter
\newcommand{\proofpart}[2]{%
  \par
  \addvspace{\medskipamount}%
  \noindent\emph{Part #1: #2.}
}
\makeatother



\numberwithin{equation}{section}


% Mean
\def\Xint#1{\mathchoice
{\XXint\displaystyle\textstyle{#1}}%
{\XXint\textstyle\scriptstyle{#1}}%
{\XXint\scriptstyle\scriptscriptstyle{#1}}%
{\XXint\scriptscriptstyle\scriptscriptstyle{#1}}%
\!\int}
\def\XXint#1#2#3{{\setbox0=\hbox{$#1{#2#3}{\int}$ }
\vcenter{\hbox{$#2#3$ }}\kern-.6\wd0}}
\def\ddashint{\Xint=}
\def\dashint{\Xint-}

\usepackage[backend=bibtex,style=alphabetic,giveninits=true]{biblatex}
\renewcommand*{\bibfont}{\normalfont\footnotesize}
\addbibresource{topics.bib}
\renewbibmacro{in:}{}
\DeclareFieldFormat{pages}{#1}

\newcommand\todo[1]{\textcolor{red}{TODO: #1}}


\begin{document}
\begin{abstract}
We collect several results concerning regularity of minimal laminations, and governing the various modes of convergence for sequences of minimal laminations.
We then apply this theory to prove that a function is $1$-harmonic iff its level sets are a minimal lamination; this resolves an open problem of Daskalopoulos--Uhlenbeck.
\end{abstract}

\maketitle

%%%%%%%%%%%%%%%%%%%%%%%%%%%%%%%%%%%%%%%%%%%%%%%%%%%%%%%

% \tableofcontents

\section{Introduction}
The space of codimension-$1$ minimal laminations on a Riemannian manifold has been topologized in several different ways.
Thurston \cite[Chapter 8]{thurston1979geometry} introduced both his geometric topology as well as the weak topology of measures on the space of measured geodesic laminations.
Independently of Thurston, Colding--Minicozzi \cite[Appendix B]{ColdingMinicozziIV} introduced a topology that emphasized not the laminations themselves, but rather the coordinate charts which flatten them.
We shall explain how these three modes of convergence are related, as well as the regularity and compactness theorems associated to each such mode.

We then turn to the main goal of this series of papers, which also includes the prequel paper \cite{BackusFLG}.
We show that a current is Ruelle-Sullivan with respect to a minimal lamination if and only if it is locally the exterior derivative of a $1$-harmonic function.
This generalizes a theorem of Daskalopoulos--Uhlenbeck \cite[Theorem 6.1]{daskalopoulos2020transverse} and resolves the open problems \cite[Problem 9.4]{daskalopoulos2020transverse} and \cite[Conjecture 9.5]{daskalopoulos2020transverse}.

%%%%%%%%%%%%%%%%%
\subsection{Minimal laminations}\label{Lams sections}
Throughout this paper, we fix an interval $I \subset \RR$, a box $J \subset \RR^{d - 1}$, and a Riemannian manifold $M = (M, g)$ of dimension $d \geq 2$.

\begin{definition}
A (codimension-$1$) \dfn{laminar flow box} is a $C^0$ coordinate chart $F: I \times J \to M$ and a compact set $K \subseteq I$, such that for each $k \in K$, $F|_{\{k\} \times J}$ is a $C^1$ embedding, and the \dfn{leaf} $F(\{k\} \times J)$ is a $C^1$ complete hypersurface in $F(I \times J)$.
\end{definition}

\begin{definition}
Let $(F_\alpha, K_\alpha)$ and $(F_\beta, K_\beta)$ be laminar flow boxes.
We say that they belong to the same \dfn{laminar atlas} if the transition map $\psi_{\alpha \beta}$ between $F_\alpha$ and $F_\beta$ maps each leaf $\{k\} \times J$, $k \in K_\alpha$, to a leaf $\{\psi_{\alpha \beta}(k)\} \times J$, so that $\psi_{\alpha \beta}$ is a homeomorphism $K_\alpha \to K_\beta$.
\end{definition}

\begin{definition}
A \dfn{lamination} $\lambda$ consists of a nonempty closed set $S \subseteq M$, called its \dfn{support}, and a maximal laminar atlas $\{(F_\alpha, K_\alpha): \alpha \in A\}$ such that in the image $U_\alpha$ of each flow box $F_\alpha$,
$$S \cap U_\alpha = F_\alpha(K_\alpha \times J).$$
If $\lambda$ is a lamination in the image of a flow box $F$, and $N := F(\{k\} \times J)$ is a leaf of $\lambda$, we call $k$ the \dfn{label} of $N$.
A \dfn{foliation} is a lamination with support $S = M$.
\end{definition}

Summarizing the above definitions, a lamination is a nonempty closed set $S$ with a $C^0$ local product structure which locally realizes it as $K \times J$ for some compact set $K \subset \RR$.

\begin{definition}
We call a lamination $C^r$ (resp. \dfn{Lipschitz}) if its flow boxes are $C^r$ (resp. Lipschitz) coordinate charts, and say that it is \dfn{tangentially $C^r$} if for each flow box $(F, K)$, $F|_{\{k\} \times J}$ is a $C^r$ embedding for $k \in K$.\footnote{Such laminations are also known as $C^r$ \dfn{along leaves} \cite{Morgan88}.}
\end{definition}

In particular, we assume that laminations are $C^0$ and tangentially $C^1$; the latter assertion implies that the flow box can push forward the normal vector to each leaf, and in particular that the mean curvature to each leaf is well-defined as a signed Radon measure.

In this paper we shall focus on laminations with minimal leaves.\footnote{The word ``minimal'' is overloaded. In \cite{casson_bleiler_1988,daskalopoulos2020transverse}, a \dfn{minimal lamination} is a lamination $\lambda$ in which every leaf is dense in $\supp \lambda$.
We adopt the terminology of \cite{Ohshika90}, which unfortunately clashes with the rest of the literature.}

\begin{definition}
A lamination $\lambda$ is \dfn{minimal} if its leaves $F_\alpha(\{k\} \times J)$ have zero mean curvature, and is \dfn{geodesic} if, in addition, $d = 2$.
\end{definition}

%%%%%%%%%%%%%%%%%%
\subsection{Regularity of minimal laminations}
The definitions of \S\ref{Lams sections} are tedious to work with, both because one has to prove the existence of flow boxes which flatten sets which may be extremely rough, and because one has no quantitative control on said flow boxes.
However, if we have curvature bounds on the leaves and on the underlying manifold $M$, our first main theorem drastically changes the story: it shows that the lamination $\lambda$ can be reconstructed from its set of leaves, in such a way that the flow boxes for $\lambda$ are under control in the Lipschitz and tangentially $C^\infty$ sense.
Here, a sequence converges \dfn{tangentially in $C^\infty$} if it converges in $C^0$ and all higher derivatives tangent to any leaves of $\lambda$ converge in $C^0$ as well.

\begin{mainthm}\label{regularity theorem}
Let $K := \|\Riem_M\|_{C^0}$ and let $i$ be the injectivity radius of $M$, and suppose that $K < \infty$, $i > 0$.
Let $\mathcal S$ be a set of disjoint minimal hypersurfaces in $M$, such that for every $N \in \mathcal S$,
\begin{equation}\label{curvature bound in regularity}
	\|\Two_N\|_{C^0} \leq A,
\end{equation}
and that $\bigcup_{N \in \mathcal S} N$ is a closed subset of $M$. Then:
\begin{enumerate}
\item There exists a Lipschitz minimal lamination $\lambda$ whose leaves are exactly the elements of $\mathcal S$.
\item There exists a Lipschitz line bundle on $M$ which is normal to every leaf of $\lambda$.
\item There exist constants $L = L(A, K, i) > 0$ and $r = r(A, K, i) > 0$, and a Lipschitz laminar atlas $(F_\alpha)$ for $\lambda$, such that for every $\alpha$,
\begin{equation}\label{conorm of flow box}
	\max(\Lip(F_\alpha), \Lip(F_\alpha^{-1})) \leq L,
\end{equation}
and the image of $F_\alpha$ contains a ball of radius $r$.
\item $F_\alpha$ and $F_\alpha^{-1}$ are tangentially $C^\infty$, with seminorms only depending on $A, K, i$.
\end{enumerate}
\end{mainthm}

Several similar results to Theorem \ref{regularity theorem} have appeared in the literature already, but Theorem \ref{regularity theorem} significantly strengthens them.
To our knowledge, the first related result is due to Solomon \cite[Theorem 1.1]{Solomon86}, which we improve on in several ways:
\begin{enumerate}
\item \label{foliation to lamination} Solomon's proof is only valid for minimal foliations in $\RR^d$.
\item We obtain estimates which only depend on the curvatures of the leaves and $M$, and on the injectivity radius $i$; they do not depend on the regularity of a given $C^0$ laminar atlas.
\item In fact, we do not even assume the existence of a $C^0$ laminar atlas.
\end{enumerate}
As Solomon notes, it is easy to extend his proof to minimal foliations of a Riemannian manifold $M$; the key point of (\ref{foliation to lamination}) is that we would like Theorem \ref{regularity theorem} to be true for minimal \emph{laminations}.
Colding--Minicozzi \cite[Appendix B]{ColdingMinicozziIV} sketched a proof for minimal laminations of a Riemannian manifold which have finitely many leaves by filling in the gaps between the leaves in Solomon's constructions by linear interpolation.
However, Colding--Minicozzi again assume the existence of a $C^0$ atlas when they implicitly appeal to the argument of Solomon to obtain the existence of normal coordinates in which all leaves are graphical (a key step in the construction); this of course implies that their estimates are dependent on the $C^0$ structure of $\lambda$.

Using completely different techniques, Daskalopoulos--Uhlenbeck \cite[Proposition 7.3]{daskalopoulos2020transverse} obtained a version of Theorem \ref{regularity theorem} without any $C^0$ dependence, under the assumption that $M$ is a closed hyperbolic surface.
The key point of their argument is that the exponential map sends lines to geodesics, so it provides a much shorter proof of Theorem \ref{regularity theorem}, at the price of only working in dimension $2$.

The main point of our proof of Theorem \ref{regularity theorem} is that the elements of $\mathcal S$ must be ``close to parallel on small scales'', where the scale is governed by $A, K$.
Otherwise, since the scale is small, we may replace the elements of $\mathcal S$ by their tangent spaces, which would then intersect, contradicting the disjointness of $\mathcal S$.
Once they are close to parallel, one can show that the normal vector of each leaf is $C^0$-close to a fixed vector field, and we may proceed as in Colding--Minicozzi; this does not require that we assume that the normal vector to $\lambda$ is $C^0$, but only that the normal vector to each leaf is.


%%%%%%%%%%%%%%%%%%
\subsection{Spaces of minimal laminations}\label{LamSpace section}
In the literature, there are at least three different topologies on the space of laminations on a Riemannian manifold $M$, which we now recall.

Thurston's geometric topology \cite[Chapter 8]{thurston1979geometry} says that a lamination $\lambda'$ is close to a lamination $\lambda$ if every leaf of $\lambda$ is close to a leaf of $\lambda'$ at least locally, and the same holds for their normal vectors $\normal$.

\begin{definition}
We define the basic open sets in \dfn{Thurston's geometric topology} to be defined by a lamination $\lambda$, $x \in M$, and $\varepsilon > 0$: the basic open set $\mathscr N(\lambda, x, \varepsilon)$ is the set of all laminations $\kappa$ such that there exists $y \in \supp \kappa \cap B(x, \varepsilon)$ such that the normal vectors are close: $\dist(\normal_\lambda(x), \normal_\kappa(y)) < \varepsilon$.
\end{definition}

A sequence of laminations $(\lambda_i)$ converges to a lamination $\lambda$ in Thurston's geometric topology iff, for every leaf $N$ of $\lambda$, every $x \in N$, and every $\varepsilon > 0$, there exists $i_{\varepsilon, x} \in \NN$ such that for every $i \geq i_{\varepsilon, x}$, $\supp \lambda_i$ intersects $B(x, \varepsilon)$, and for $x_i \in B(x, \varepsilon) \cap \supp \lambda_i$,
$$\dist_{SM}(\normal_{\lambda_i}(x_i), \normal_\lambda(x)) < 2\varepsilon.$$
It is straightforward to show that Thurston's geometric topology does not depend on the choice of Riemannian metric on $M$, or the choice of extension of the distance function on $M$ to its sphere bundle $SM$, which are implicit in the statement thereof.
However, the limiting lamination is not unique, as if $\lambda_i \to \lambda$ and $\lambda'$ is a sublamination of $\lambda$, then $\lambda_i \to \lambda'$.
In particular, Thurston's topology is not Hausdorff, and we say that $\lambda$ is a \dfn{maximal limit} of a sequence $(\lambda_i)$ if $\lambda_i \to \lambda$ and for every $\lambda'$ such that $\lambda_i \to \lambda'$, $\lambda'$ is a sublamination of $\lambda$.

Independently of Thurston, Colding--Minicozzi \cite[Appendix B]{ColdingMinicozziIV} defined a sequence of laminations to converge ``if the corresponding coordinate maps converge;'' that is, if the laminar atlases converge.
This of course says nothing about the limiting set of leaves and in the sequel paper \cite{ColdingMinicozziV} they additionally impose that the sets of leaves converge ``as sets.''

In this paper we consider a similar condition to the one in \cite{ColdingMinicozziV}, which we believe to be more natural: that the laminar atlases converge and that the laminations themselves converge in Thurston's geometric topology.
To be more precise:

\begin{definition}
A sequence $(\lambda_i)$ of laminations \dfn{flow-box converges} in a function space $X$ to $\lambda$ if it converges in Thurston's geometric topology, and there exists a laminar atlas $(F_\alpha)$ for $\lambda$ such that for each $\alpha$, $F_\alpha$ and $(F_\alpha)^{-1}$ are limits in $X$ of flow boxes $F_\alpha^i$, $(F_\alpha^i)^{-1}$ in laminar atlases for $\lambda_i$.
\end{definition}

The notion of flow-box convergence is mainly useful for tangential $C^\infty$ and the Fr\'echet space $C^{1-} := \bigcap_{0 \leq \theta < 1} C^\theta$, where $C^\theta$ are H\"older spaces.

We now define convergence of laminations equipped with transverse measures.\footnote{We assume that $\supp \mu_\alpha = K_\alpha$, but in \cite{daskalopoulos2020transverse}, it is only assumed that $\supp \mu_\alpha \subseteq K_\alpha$.
In particular, not every lamination admits a transverse measure.}

\begin{definition}
Let $\lambda$ be a lamination with atlas $A$.
A \dfn{transverse measure} to $\lambda$ consists of Radon measures $\mu_\alpha$ with $\supp \mu_\alpha = K_\alpha$, $\alpha \in A$, such that each transition map $\psi_{\alpha \beta}$ is measure-preserving:
$$\mu_\alpha|_{K_\alpha \cap K_\beta} = \psi_{\alpha \beta}^* (\mu_\beta|_{K_\alpha \cap K_\beta}).$$
The pair $(\lambda, \mu)$ is called a \dfn{measured lamination}.
\end{definition}

Now suppose that $\lambda$ is \dfn{oriented} -- that is, all its transition maps are orientation-preserving.
Then we can define measure convergence in terms of its Ruelle-Sullivan current \cite{Ruelle75}.

\begin{definition}
Let $(\lambda, \mu)$ be a measured, oriented lamination, and let $(\chi_\alpha)_{\alpha \in A}$ be a subordinate partition of unity.
The \dfn{Ruelle-Sullivan current} $T_\mu$ associated to $(\lambda, \mu)$ is defined for all compactly supported $d-1$-forms $\varphi$ by
\begin{equation}\label{RS current}
\int_M T_\mu \wedge \varphi := \sum_{\alpha \in A} \int_{K_\alpha} \left[\int_{\{k\} \times J} (F_\alpha^{-1})^* (\chi_\alpha \varphi) \right] \dif \mu_\alpha(k).
\end{equation}
\end{definition}

\begin{definition}
A sequence of measured laminations $(\lambda_i, \mu_i)$ \dfn{converges} to $(\lambda, \mu)$ if their local Ruelle-Sullivan currents $T_{\mu_i} \to T_\mu$ converge in the weak topology of measures.
\end{definition}

% The convergence of Ruelle-Sullivan currents, which is very convenient to work with analytically, is equivalent to a definition of measure convergence that may be more familiar to topologists, namely convergence of the transverse measure along each transverse curve, as we explain in Appendix \ref{transverse curves}.

Filling in some of the details of the argument of Colding--Minicozzi \cite[Appendix B]{ColdingMinicozziIV}, it follows from the regularity theorem, Theorem \ref{regularity theorem}, that once we have a bound on the curvatures of the leaves, every sequence of laminations has convergent subsequences in each of the above modes of convergence.
More precisely we have:

\begin{definition}
A sequence $(\lambda_n)$ of laminations has \dfn{bounded curvature} if there exists $C > 0$ such that for any $n$ and any leaf $N$ of $\lambda_n$, the second fundamental form satisfies $\|\Two_N\|_{C^0} \leq C$.
\end{definition}

\begin{mainthm}\label{compactness theorem}
Let $(\lambda_n)$ be a sequence of minimal laminations of bounded curvature, and assume that for some compact set $E \Subset M$ and every leaf $N$ of $\lambda_n$, $N \cap E$ is nonempty. Then:
\begin{enumerate}
\item A subsequence converges in the $C^{1-}$ and tangentially $C^\infty$ flow box topology, and in particular in Thurston's geometric topology, to a minimal lamination.
\item If $\mu_n$ is transverse to $\lambda_n$ and there exists $C > 0$ such that $\mu_n(M) \leq C$, then a further subsequence converges in the measure topology.
\end{enumerate}
\end{mainthm}

We now use Theorem \ref{compactness theorem} to explain how the above modes of convergence are related.
It is clear from the definitions that flow-box convergence implies Thurston convergence.
Moreover, for $d = 2$, Thurston claimed that that measure convergence implies Thurston convergence \cite[Proposition 8.10.3]{thurston1979geometry}, though he did not explicitly justify why the limit was geodesic, or why the convergence preserves the normal vectors.
We complete the proof that measure convergence implies Thurston convergence, and show that flow-box convergence sits in the middle of the chain of implications:

\begin{mainthm}\label{implication theorem}
Suppose that $\dim M \leq 7$.
Let $(\lambda_n, \mu_n)$ be measured minimal laminations in $M$, and $(\lambda_n, \mu_n) \to (\lambda, \mu)$.
Then:
\begin{enumerate}
	\item $\lambda_n \to \lambda$ in Thurston's geometric topology.
	\item If $(\lambda_n)$ has bounded curvature, then $\lambda_n \to \lambda$ in the $C^{1-}$ and tangentially $C^\infty$ flow box topology.
\end{enumerate}
\end{mainthm}

In Theorem \ref{implication theorem}, we appeal to the measure-theoretic results of \cite{BackusFLG} in order to justify the regularity of the limit; this ultimately relies on the nonexistence of $\leq 6$-dimensional singular minimal tangent cones, and is where the hypothesis on dimension enters.



%%%%%%%%%%%%%%%%%%
\subsection{Application to \texorpdfstring{$p$-harmonic}{p-harmonic} maps and Teichm\"uller theory}\label{FLG section}
Geodesic laminations are of interest to the Thurston school of Teichm\"uller theory \cite[Chapter 8]{thurston1979geometry}.
Later Thurston introduced \dfn{best Lipschitz maps}, namely maps $v: M \to N$ between closed manifolds which minimize their Lipschitz constant $\Lip(v)$ subject to a constraint on their homotopy class.
If $M, N$ are closed hyperbolic surfaces of the same genus, then $\log \Lip(v)$ is the distance between $M$ and $N$ in \dfn{Thurston's asymmetric metric} on Teichm\"uller space.
This circle of ideas has been developed by the Thurston school \cite{papadopoulos:hal-00129729, Gu_ritaud_2017} but has recently also made contact with geometric PDE through the work of Daskalopoulos--Uhlenbeck \cite{daskalopoulos2020transverse,daskalopoulosPrep1}, which we now recall.

Let $M$ be a closed hyperbolic surface.
The Euler-Lagrange equation for best Lipschitz maps $v: M \to \Sph^1$ is the $\infty$-Laplace equation
\begin{equation}\label{infinity laplacian}
	\langle \nabla^2 v, \nabla v \otimes \nabla v\rangle = 0.
\end{equation}
Thus, every \dfn{$\infty$-harmonic function} -- that is, a viscosity solution of (\ref{infinity laplacian}) -- is best Lipschitz \cite{daskalopoulos2020transverse}.
From an $\infty$-harmonic function we may associate two pieces of data:
\begin{enumerate}
\item The set $\{\dif v = \|\dif v\|_{L^\infty}\}$ where the best Lipschitz constant $\Lip(v)$ is attained is the support of a geodesic lamination, called the \dfn{maximum stretch lamination} $\lambda$.
\item The $\infty$-Laplace equation is invariant under addition of constants $v \mapsto v + y$, so by Noether's theorem, $v$ is associated to a \dfn{conserved flux} $\dif u$.
\end{enumerate}
If $d = 2$, the associated conservation law is the $1$-Laplace equation
\begin{equation}\label{1Laplacian}
\dif^* \left(\frac{\dif u}{|\dif u|}\right) = 0.
\end{equation}
A \dfn{$1$-harmonic function} $u$ is a weak solution of (\ref{1Laplacian}), in the sense that for some divergence-free vector field $X \in L^\infty(M, TM)$,
$$|\dif u| = (\dif u, X)$$
in the sense of currents. 
It is equivalent \cite{Mazon14} to demand that $u$ is a \dfn{function of least gradient} in the sense that $u$ is a minimizer of the total variation $\|\dif u\|_{TV}$ among all $BV_\loc$ functions with the same trace.

Daskalopoulos--Uhlenbeck \cite{daskalopoulos2020transverse,daskalopoulosPrep1} showed that the Noetherian flux $\dif u$ is a Ruelle-Sullivan current for the maximum stretch lamination. More precisely:

\begin{theorem}[Daskalopoulos--Uhlenbeck]\label{DU theorem}
Let $M$ be a closed hyperbolic surface and $v: M \to \Sph^1$ an $\infty$-harmonic map with maximal stretch geodesic lamination $\lambda$ and conserved flux $T$. Then:
\begin{enumerate}
\item $T$ is the Ruelle-Sullivan current for a measured oriented structure on $\lambda$.
\item There exists a $1$-harmonic, $\pi_1(M)$-equivariant function $u: \Hyp^2 \to \RR$ such that $\dif u$ is a lift of $T$ to the universal cover $\Hyp^2$.
\item Every level set of $u$ is the lift of a geodesic of $\lambda$ to $\Hyp^2$.
\end{enumerate}
\end{theorem}

Inspired by this theorem, Daskalopoulos--Uhlenbeck conjectured that for any $1$-harmonic function on $\Hyp^2$, $\dif u$ should be Ruelle-Sullivan for some (possibly not maximum-stretch) geodesic lamination \cite[Problem 9.4]{daskalopoulos2020transverse}, and conversely that if $T$ is a Ruelle-Sullivan current for some geodesic lamination, then local primitives of $T$ are $1$-harmonic \cite[Conjecture 9.5]{daskalopoulos2020transverse}.
Of course, if $\dim M \geq 3$, then the level sets will be minimal hypersurfaces rather than geodesics.
We prove these conjectures:

\begin{mainthm}\label{main thm}
Suppose that $\dim M \leq 4$.
\begin{enumerate}
\item Let $u$ be a $1$-harmonic function on $M$.
Then:
\begin{enumerate}
\item $\bigcup_{y \in \RR} \partial \{u > y\}$ is the support of a minimal lamination $\lambda$.
\item The leaves of $\lambda$ are the connected components of the level sets $\partial \{u > y\}$.
\item There is a measured oriented structure on $\lambda$ whose Ruelle-Sullivan current is $\dif u$.
\end{enumerate}
\item Conversely, if $\lambda$ is a minimal measured oriented lamination with Ruelle-Sullivan current $T$, and $M$ is simply connected, then there exists a $1$-harmonic function $u$ such that $T = \dif u$.
\end{enumerate}
\end{mainthm}

\subsubsection{Overview of the proof}
The first step in the proof of Theorem \ref{main thm} is to show that the level sets of $u$ are in fact smooth.
Since $u$ has least gradient, its superlevel sets have \dfn{least perimeter} \cite[Theorem 1]{BOMBIERI1969} in the sense that their level sets have least gradient, and the boundary of a set of least perimeter is a smooth stable minimal hypersurface \cite{Miranda66,deGiorgi61,BackusFLG}.
Thus we can use the stable Bernstein theorem:

\begin{theorem}[stable Bernstein theorem]
	Let $M$ be a manifold of bounded geometry such that $\dim M \leq 4$.
	Then there exists $A \geq 0$ such that for every complete stable minimal hypersurface $N$ in $M$ with trivial normal bundle,
	$$|\Two_N(P)| \leq \frac{A}{1 + \dist(P, \partial M)}.$$
\end{theorem}
\begin{proof}
	If $d = 2$ this clearly holds with $A = 0$.
	If $d = 3$ this was proven in \cite{Schoen2016}, and we refer to \cite[Theorem 2.10]{colding2011course} for a modern proof.
	If $d = 4$ this result can be derived from \cite[Theorem 1]{Chodosh2021} using the sort of techniques discussed in \cite[\S3]{White13}.
\end{proof}

It follows that the level sets of $u$ have bounded curvature.
Then Theorem \ref{main thm} follows from the general theory developed previously in this paper.

\subsubsection{Possible future directions}
Theorem \ref{main thm} leaves a key point -- the role of the $\infty$-Laplacian -- open.
We will come back to this point in \cite{BackusInfinityMaxwell1}, where we explain how one can view the $1$-Laplacian as the convex dual problem to the problem of constructing a calibration of a minimal lamination, which is given by a system of ``$\infty$-elliptic'' equations.

We would also like to highlight a possible application of Theorem \ref{main thm} outside of Teichm\"uller theory.
The associated parabolic flow to (\ref{1Laplacian}) is known as \dfn{level set flow} and acts on the level sets of a smooth submersion $u$ by mean curvature flow.
As such, it arises as a model of interfaces with minimal area, as a means of continuing mean curvature flow past its singular times, and in the \dfn{level set method} of computing minimal surfaces \cite{Evans91,Sethian90,Chen89,Thomas05}.
Since Theorem \ref{main thm} is concerned exactly with the level sets of a $1$-harmonic function, it would be very interesting to apply it to rigorously justify the level set method, and in particular to understand the limiting behavior of level set flow, even in the case that $u$ is not a smooth submersion.

%%%%%%%%%%%%%%%%%%%%%%%
\subsection{Outline of the paper}
In \S\ref{Prelims} we recall preliminaries.

In \S\ref{Regularity} we prove the regularity theorem, Theorem \ref{regularity theorem}.

In \S\ref{CompactnessSec} we prove Theorems \ref{compactness theorem} (the compactness theorem), \ref{implication theorem} (the implication theorem), and \ref{main thm} (the equivalence of $1$-harmonic functions and measured minimal laminations) using Theorem \ref{regularity theorem}.

% We also include Appendix \ref{transverse curves}, where we explain why our definition of convergence in the weak topology of measures is equivalent to a more standard definition.

%%%%%%%%%%%%%%%%%%%%%%%%

\subsection{Acknowledgements}
I would like to thank Georgios Daskalopoulos for suggesting this project and for many helpful discussions.
I would also like to thank Chao Li for help understanding \cite{Chodosh2021}, Stephen Obinna for suggesting the brief proof of Lemma \ref{cardinality appendix}, and Jeremy Kahn for helpful discussions about laminations of threefolds.

This research was supported by the National Science Foundation's Graduate Research Fellowship Program under Grant No. DGE-2040433.



%%%%%%%%%%%%%%%%%%%%%%%%%%%

\section{Preliminaries}\label{Prelims}
\subsection{Notation and conventions}
The operator $\star$ is the Hodge star, thus $\star 1$ is the Riemannian measure.
We denote the musical isomorphisms by $\sharp, \flat$.
If $U$ is an open set, we write $|U| := \int_U \star 1$ for the volume of $U$, but if $U$ is a submanifold or rectifiable set of positive codimension, we instead write $|U|$ for its surface measure.
We write $\normal_N$ for the normal vector (or conormal $1$-form) for a hypersurface $N$, $\nabla_N$ for the Levi-Civita connection, and $\Two_N := \nabla_N \normal_N$ for the second fundamental form.

We consider the following manifolds: $\Ball^d$ is the unit ball in $\RR^d$, $\Sph^d$ the unit sphere in $\RR^{d + 1}$, and $\Hyp^d$ is the hyperbolic space.

For a map $F: X \to Y$ between metric spaces, we write $\Lip(F)$ for its Lipschitz constant.
If $X, Y$ are connected Riemannian manifolds, one of which is $1$-dimensional, then we have $\Lip(F) = \|\dif F\|_{L^\infty}$.

We let $\Leaves \lambda$ denote the set of leaves of a lamination $\lambda$.

We use the Hausdorff notions of limit and limit inferior of a sequence of closed sets \cite[Chapter IV]{nadler2017continuum}:

\begin{definition}
Let $X$ be a topological space, and $(Y_n)$ a sequence of closed subsets of $X$.
\begin{enumerate}
\item The \dfn{limit inferior} $\liminf_{n \to \infty} Y_n$ is the set of all $x \in X$ such that for every open neighborhood $U \ni x$, $U \cap Y_n$ is eventually nonempty.
\item The \dfn{limit superior} $\limsup_{n \to \infty} Y_n$ is the set of all $x \in X$ such that for every open neighborhood $U \ni x$, $U \cap Y_n$ is nonempty for infinitely many $n$.
\item If $\liminf_{n \to \infty} Y_n = \limsup_{n \to \infty} Y_n$, we call that set the \dfn{limit} $\lim_{n \to \infty} Y_n$.
\end{enumerate}
\end{definition}

%%%%%%%%%%%%%%%%%%%%%
\subsection{Measure theory}\label{MeasurePrelims}
Let $X$ be a metrizable space, and let $C_\cpt(X)$ be the space of compactly supported continuous functions $f: X \to \RR$.
Its dual $C_\cpt(X)'$ is canonically isomorphic to the space of signed Radon measures on $X$, where the bilinear pairing is given by integration.
The weak topology on $C_\cpt(X)'$ is known as the \dfn{weak topology of measures}.
Unpacking the definitions, a sequence $(\mu_n)$ of Radon measures converges to $\mu$ in the weak topology of measures iff for every continuous function $f: X \to \RR$,
$$\lim_{n \to \infty} \int_X f \dif \mu_n = \int_X f \dif \mu.$$

\subsubsection{The portmanteau theorem}
We shall frequently use the following characterization of weak convergence:

\begin{proposition}[portmanteau theorem]
	Let $(\mu_n)$ be a sequence of positive Radon measures on a compact metrizable space $X$ with $\mu_n(X) \lesssim 1$, and let $\mu$ be a Radon measure on $X$. The following are equivalent:
\begin{enumerate}
	\item $\mu_n \to \mu$ in the weak topology of measures.
	\item $\liminf_{n \to \infty} \mu_n(X) \geq \mu(X)$ and for every closed $Y \subseteq X$, $\limsup_{n \to \infty} \mu_n(Y) \leq \mu(Y)$.
	\item $\limsup_{n \to \infty} \mu_n(X) \leq \mu(X)$ and for every open $Z \subseteq X$, $\liminf_{n \to \infty} \mu_n(Z) \geq \mu(Z)$.
	\item For every $W \subseteq X$ with $\mu(\partial W) = 0$, $\lim_{n \to \infty} \mu_n(W) = \mu(W)$.
\end{enumerate}
	If we choose a metric on $X$, then the above conditions imply:
\begin{enumerate}
	\setcounter{enumi}{4}
	\item For every $x \in X$ and all but countably many $\varepsilon > 0$, $\lim_{n \to \infty} \mu_n(B(x, \varepsilon)) = \mu(B(x, \varepsilon))$.
\end{enumerate}
\end{proposition}
\begin{proof}
	See \cite[Theorem 13.16]{klenke2013probability} for the equivalence of (1)--(4); \cite{klenke2013probability} deals with subprobability measures, but this is equivalent to measures of bounded total mass by a rescaling.

	We then must show that (4) implies (5); to do so, it suffices to show that for all but countably many $\varepsilon$, $\mu(\partial B(x, \varepsilon)) = 0$.
	Let
	$$A := \{\varepsilon > 0: \mu(\partial B(x, \varepsilon)) > 0\}.$$
	Since the sets $\partial B(x, \varepsilon)$ are disjoint, for every countable $A' \subseteq A$,
	$$\sum_{\varepsilon \in A'} \mu(\partial B(x, \varepsilon)) \leq \mu(X) < \infty,$$
	where $\mu(X) < \infty$ since $X$ is compact.
	It follows from (the contrapositive of) Lemma \ref{cardinality appendix} below that $A$ is countable.
\end{proof}

\begin{lemma}\label{cardinality appendix}
Let $S$ be an uncountable set and $f: S \to (0, \infty)$. Then there exists a countable set $S' \subset S$ such that
$$\sum_{x \in S'} f(x) = \infty.$$
\end{lemma}
\begin{proof}
Define $S_n := f^{-1}([\frac{1}{n + 1}, \frac{1}{n}))$ for $n \geq 1$, and $S_0 := f^{-1}([1, \infty))$.
Since $S$ is uncountable but $\NN$ is countable, it follows from the infinite pigeonhole principle that there exists $n \in \NN$ such that $S_n$ is infinite.
In particular there exists an infinite countable set $S' \subseteq S_n$, which then satisfies
\begin{align*}
\sum_{x \in S'} f(x) &\geq \sum_{x \in S'} \frac{1}{n} = \infty. \qedhere 
\end{align*}
\end{proof}

There are subtleties involved in the portmanteau theorem for noncompact $X$.
However, this will never be an issue, as we shall only use it locally, in small precompact balls.

\subsubsection{Ruelle-Sullivan currents}
If $X = M$ is a manifold, then we can consider instead the space $C_\cpt(M, \Omega^\ell)$ of compactly supported continuous $\ell$-forms.
An $\ell$-\dfn{current} is an element of the dual space $C_\cpt(M, \Omega^\ell)'$ \cite{simon1983GMT}.\footnote{Strictly speaking, $C_\cpt(M, \Omega_\ell)'$ is the space of $\ell$-currents of locally finite total variation. However, we will never need to consider $\ell$-currents that do not have locally finite total variation, so we suppress this technicality.}
We denote the pairing of an $\ell$-current $T$ and an $\ell$-form $\varphi$ by $\int_M T \wedge \varphi$.
Any $d-\ell$-form $\psi$ gives rise to an $\ell$-current $T$, the \dfn{Poincar\'e dual} of $\psi$, by $\int_M T \wedge \varphi = \int_M \psi \wedge \varphi$.
In particular, the Poincar\'e dual of any function is a $d$-current.
Again we have the \dfn{weak topology of measures} on the space of $\ell$-currents.

To any $\ell$-current $T$ we may associate a positive Radon measure, its \dfn{total variation} $\star |T|$, which satisfies for any function $f$,
$$\int_M f \star |T| := \sup_{|\varphi| \leq |f|} \int_M T \wedge \varphi,$$
and a $|T|$-measurable $d - \ell$-form $\psi$, the \dfn{polar part} \cite[Theorem 4.14]{simon1983GMT}, which satisfies $T = \psi |T|$, $|T|$-almost everywhere.
Moreover, if $\psi \mapsto -\int_M T \wedge \dif \psi$ is an $\ell - 1$-current, then it is the \dfn{exterior derivative} $\dif T$.
We write $\|T\|_{TV} := \int_M \star |T|$ for the total variation norm.

With the measure-theoretic machinery above in place, we recall some facts about Ruelle-Sullivan currents.
Let $(\lambda, \mu)$ be a measured oriented lamination.
Then the Ruelle-Sullivan current $T_\mu$ is a well-defined closed $d-1$-current \cite[Theorem 8.2]{daskalopoulos2020transverse}. 
In particular, we may lift $T_\mu$ to the universal cover $\tilde M$, where it is exact \cite[Theorem 8.3]{daskalopoulos2020transverse}.
Moreover, $T_\mu$ has an intrinsic definition as the unique $d-1$-current with a certain polar decomposition.
To be more precise, recall that $\mu$ defines a measure on $\supp \lambda$: in each flow box $F_\alpha$, an open set $U$ has measure
\begin{equation}\label{transverse measure of an open set}
\mu(U) := \int_{K_\alpha} |F_\alpha(\{k\} \times J) \cap U| \dif \mu_\alpha(k).
\end{equation}

% \begin{proof}
% We first claim that the right-hand side of (\ref{RS current}) is always finite, and is continuous in $\varphi$.
% In fact, possibly after refining $(\chi_\alpha)$, we may assume that it is a locally finite partition of unity.
% In particular, we just need to check the continuity in a single flow box:
% $$\left|\int_{K_\alpha} \left[\int_{\RR^{d - 1} \times \{k\}} (F_\alpha^{-1})^* (\chi_\alpha \varphi) \right] \dif \mu_\alpha(k)\right| \leq \int_{K_\alpha} \int_{\RR^{d - 1} \times \{k\}} |(F_\alpha^{-1})^* (\chi_\alpha \varphi)| \dif \mu_\alpha(k).$$
% The inner integral is controlled by $\|\varphi\|_{C^0(U_\alpha)} \cdot |U_\alpha|$ where $U_\alpha$ is the image of $F_\alpha$.
% The outer integral is then well-defined because it is against a Radon measure.

% We next observe that the choice of partition of unity is irrelevant, thus if $\varphi$ has compact support in $U_\alpha \cap U_\beta$, then
% \begin{equation}\label{well-defined of Ruelle-Sullivan}
% \int_{K_\alpha} \int_{\RR^{d - 1} \times \{k\}} (F_\alpha^{-1})^* \varphi \dif \mu_\alpha(k) = \int_{K_\beta} \int_{\RR^{d - 1} \times \{k\}} (F_\beta^{-1})^* \varphi \dif \mu_\beta(k).
% \end{equation}
% Indeed,
% \begin{align*}
% \int_{K_\alpha} \int_{\RR^{d - 1} \times \{k\}} (F_\alpha^{-1})^* \varphi \dif \mu_\alpha(k)
% &= \int_{K_\beta} (F_\alpha F_\beta^{-1})^* \left[\int_{\RR^{d - 1} \times \{k\}} (F_\alpha^{-1})^* \varphi \dif \mu_\alpha(k)\right] \\
% &= \int_{K_\beta} \left[\int_{\RR^{d - 1} \times \{k\}} (F_\beta^{-1})^* \varphi\right] (F_\alpha F_\beta^{-1})^* \dif \mu_\beta(k) \\
% &= \int_{K_\beta} \int_{\RR^{d - 1} \times \{k\}} (F_\beta^{-1})^* \varphi \dif \mu_\beta(k)
% \end{align*}
% where the last equation is because of the measure-preserving nature of the transition maps; this proves (\ref{well-defined of Ruelle-Sullivan}).

% Finally, if a $d-2$-form $\psi$ has compact support in a single flow box, then
% $$\int_{\RR^{d - 1} \times \{k\}} (F_\alpha^{-1})^* \dif \psi = \int_{\RR^{d - 1} \times \{k\}} \dif((F_\alpha^{-1})^* \psi) = 0$$
% by Stokes' theorem, so
% \begin{align*}
% \int_M \dif T_\mu \wedge \psi &= -\int_M T_\mu \wedge \dif \psi \\
% &= -\int_{K_\alpha} \int_{\RR^{d - 1} \times \{k\}} (F_\alpha^{-1})^* \dif \psi \dif \mu_\alpha(k) = 0. \qedhere
% \end{align*}
% \end{proof}

\begin{lemma}
For a measured oriented lamination $(\lambda, \mu)$, with Lipschitz normal vector $\normal_\lambda$, the polar decomposition of $T_\mu$ is
\begin{equation}\label{polar ruelle sullivan}
T_\mu = \normal_\lambda \mu.
\end{equation}
\end{lemma}
\begin{proof}
For an open set $U \subseteq M$ in a flow box $F_\alpha$, the total variation satisfies
$$\int_U \star |T_\mu| = \sup_{\|\varphi\|_{C^0} \leq 1} \int_{K_\alpha} \int_{\{k\} \times J} \varphi \dif \mu_\alpha(k)$$
where the supremum ranges over $d-1$-forms $\varphi$ with compact support in $U$.
However, $\star \normal_\lambda^\flat$ is the Riemannian measure on $F_\alpha(\{k\} \times J)$, so
$$\int_{\{k\} \times J} \varphi \leq \int_{\{k\} \times J} (F_\alpha^{-1})^*(\star \normal_\lambda^\flat).$$
Since $\|\normal^\lambda\|_{C^0} = 1$, it follows that a sequence of cutoffs of $\star \normal_\lambda^\flat$ to more and more of $U$ is a maximizing sequence.
Therefore $\normal_\lambda$ is the polar part of (\ref{polar ruelle sullivan}), and
$$\int_U \star |T_\mu| = \int_{K_\alpha} \int_{\{k\} \times J} (F_\alpha^{-1})^*(1_U \star \normal_\lambda^\flat) \dif \mu_\alpha(k).$$
The inner integral is the Riemannian measure of $F_\alpha(\{k\} \times J) \cap U$, so by (\ref{transverse measure of an open set}), $|T_\mu| = \mu$.
\end{proof}

The above computation motivates the definition of Ruelle-Sullivan current of a \emph{nonorientable} lamination.
To be more precise, if $\lambda$ is a nonorientable lamination with normal vector field $\normal_\lambda$, then we can view $\normal_\lambda$ as a section of a (necessarily twisted) line bundle $L$ over $M$.
We can then define $T_\mu$ to be $\normal_\lambda \mu$, which makes sense as a distributional section of $L$, and can be tested against any continuous $d-1$-form on $M$ whose support is contained in a contractible set.
In particular, we shall speak of the Ruelle-Sullivan current of any measured lamination, even if it is nonorientable.

\begin{lemma}\label{convergence of normals}
If $(\lambda_n, \mu_n) \to (\lambda, \mu)$, $x_n \in \supp \lambda_n$ converges to $x \in \supp \lambda$, and $(\lambda_n), \lambda$ have continuous normal vector fields $(\normal_n), \normal$, then $\normal_n(x_n) \to \normal(x)$ pointwise.
\end{lemma}
\begin{proof}
	Choose a continuous $d-1$-form $\varphi$ which extends $\star \normal^\flat$.
	Then for every $\varepsilon > 0$,
	$$\int_{B(x, \varepsilon)} T_\mu \wedge \varphi = \mu(B(x, \varepsilon))$$
	so by the portmanteau theorem, for almost every $\varepsilon > 0$,
	\begin{equation}\label{epsilon is a continuity set}
		\lim_{n \to \infty} \frac{\int_{B(x, \varepsilon)} T_{\mu_n} \wedge \varphi}{\mu_n(B(x, \varepsilon))} = \frac{\int_{B(x, \varepsilon)} T_\mu \wedge \varphi}{\mu(B(x, \varepsilon))} = 1.
	\end{equation}
	On the other hand, if we assume that there exist $\delta, \varepsilon > 0$ and a coordinate system such that for every $y \in \supp \lambda_n \cap B(x, \varepsilon)$,
	$$|\normal_n - \normal| \geq \delta,$$
	then possibly after shrinking $\varepsilon$ we may assume that (\ref{epsilon is a continuity set}) holds, hence by (\ref{polar ruelle sullivan}),
	$$\int_{B(x, \varepsilon)} T_{\mu_n} \wedge \varphi = \int_{B(x, \varepsilon)} \normal_n^\flat \wedge \star \normal^\flat \dif \mu_i \leq (1 - O(\delta)) \mu_n(B(x, \varepsilon))$$
	and therefore $\delta = 0$, a contradiction.
\end{proof}

\begin{lemma}
Suppose that $\lambda_n \to \lambda$ in the weak topology of measures or Thurston's geometric topology.
Then 
\begin{equation}\label{supports shrink in the limit}
\supp \lambda \subseteq \liminf_{n \to \infty} \supp \lambda_n.
\end{equation}
\end{lemma}
\begin{proof}
Let $x \in \supp \lambda$.
If the convergence is in the weak topology of measures, let $\mu, \mu_n$ be the transverse measures.
Then by the portmanteau theorem, for any $\varepsilon > 0$,
$$\liminf_{n \to \infty} \mu_n(B(x, \varepsilon)) \geq \mu(B(x, \varepsilon)) > 0$$
so $\mu_n(B(x, \varepsilon)) \gtrsim 1$.
So for any $\varepsilon$ we can find $n$ and $x_n \in \supp \lambda_n \cap B(x, \varepsilon)$.
If instead the convergence is in the Thurston topology, we pass to a subsequence which realizes the limit inferior in (\ref{supports shrink in the limit}).
Then by definition of a basic open set, for every $\varepsilon > 0$ we can find $n$ and $x_n$ such that $x_n \in \supp \lambda_n \cap B(x, \varepsilon)$.
Either way, we conclude (\ref{supports shrink in the limit}).
\end{proof}

%%%%%%%%%%%%%%%%%%%%%%%%%%
\subsection{\texorpdfstring{$1$-harmonic}{One-harmonic} functions}
A subset $U \subset M$ has \dfn{least perimeter} if $1_U$ has least gradient.
By \cite[Theorem 1]{BOMBIERI1969}, if $u$ has least gradient, then every superlevel set $\{u > y\}$ has least perimeter.
Moreover, a function has least gradient if and only if it is $1$-harmonic \cite{Mazon14}.

The de Giorgi--Miranda regularity theorem \cite{deGiorgi61,Miranda66} asserts that every set of least perimeter in an open subset of $\RR^d$, $d \leq 7$, is bounded by stable minimal hypersurfaces.
It is folklore that the same result holds for Riemannian manifolds; for completeness, in the companion paper \cite{BackusFLG} we established the de Giorgi--Miranda theorem in this generality, hence:

\begin{theorem}\label{main thm of old paper}
Suppose that $\dim M \leq 7$ and $u: M \to \RR$ is a $1$-harmonic function.
Then every level set $\partial \{u > y\}$ of $U$ is bounded by stable minimal hypersurfaces.
\end{theorem}

The following compactness theorem for $1$-harmonic functions in the weak topology of measures will frequently be useful.
It is a straightforward consequence of the compactness of the forgetful map $BV(M) \to L^1(M)$ and the proof of \cite[Osservazione 3]{Miranda67}, and we omit the details.

\begin{proposition}[Miranda stability theorem]
  Suppose that $M$ is a compact manifold, possibly with boundary.
	If a sequence of functions $(u_n)$ (not necessarily of the same trace) is bounded in $L^1(M)$ and satisfies
\begin{equation}\label{boundedness in Miranda}
	\limsup_{n \to \infty} \int_M \star |\dif u_n| \leq \liminf_{n \to \infty} \inf_{v|_{\partial M} = 0} \int_M \star |\dif(u_n + v)| < \infty,
\end{equation}
	then there exists a $1$-harmonic function $u$ such that along a subsequence, $u_n \to u$ in $L^1(M)$ and $\dif u_n \to \dif u$ in the weak topology of measures.
\end{proposition}

%%%%%%%%%%%%%%%%%%%%%%%%%%%%%%%%%%%%%%%%%%
\subsection{Elliptic estimates on leaves}\label{Leaf estimates}
We now recall some estimates on leaves of a lamination in normal coordinates, following \cite[\S7.1]{colding2011course}.
Since the leaves of a lamination clearly become graphs when written in (suitably rotated) flow box coordinates, this establishes estimates on leaves that we shall use throughout the remainder of the paper.

Let $\mu, \nu, \dots$ range over indices $0, \dots, d - 1$, $i, j, \dots$ range over $1, \dots, d - 1$, $x := (x^1, \dots, x^{d - 1})$, and $y := x^0$.
Let $g$ be a metric on $\RR^{d - 1}_x \times \RR_y$ satisfying the normal coordinates condition 
$$g_{\mu\nu} = \delta_{\mu\nu} + O(|x|^2 + y^2)$$
and a curvature bound $\|\Riem_g\|_{C^0} \leq K_0$ for some $0 < K_0 \leq 1$.
Under these hypotheses, the Christoffel symbols $\Gamma$ of the Levi-Civita connection $\nabla$ satisfy $|\Gamma| \lesssim K_0(|x|^2 + y^2)$.
Let $N = \{y = u(x)\}$ for some function $u$ defined on $4\Ball^{d - 1}$.
By \cite[(7.21)]{colding2011course}, if we define for a $2$-jet $(y, \xi, H)$ on $\RR^{d - 1}$,
$$h_{ij}(x; y, \xi) := g_{ij}(x, y) + \xi_i g_{j0}(x, y) + \xi_j g_{i0}(x, y) + \xi_i \xi_j g_{00}(x, y),$$
denote by $(h^{ij})$ the inverse of $(h_{ij})$, and let
\begin{align*}
	F(x; y, \xi, H) 
	&:= h^{ij}(H_{ij} + \Gamma^0_{ij} + \xi_i \Gamma^0_{0j} + \xi_j \Gamma^0_{0i} + \xi_i \xi_j \Gamma^0_{00}) \\
	&\qquad + \xi_m h^{ij}(\Gamma^m_{ij} + \xi_i \Gamma^m_{0j} + \xi_j \Gamma^m_{0i} + \xi_i \xi_j \Gamma^m_{00})
\end{align*}
(where we have taken an Einstein sum)
then $h^{ij} = \delta^{ij} + O(K_0(|x|^2 + y^2) + |\xi|)$ so we have the estimate
$$F(x, y, \xi, H) = \tr H + O((K_0 |H| + |\xi|)(|x| + |y| + |\xi|)).$$
Moreover, if we denote $Pu(x) := F(x, u(x), \dif u(x), \nabla^2 u(x))$, then $Pu = 0$ is the minimal surface equation with respect to the metric $g$.
For $\|u\|_{C^1} \lesssim 1$, this equation is uniformly elliptic, so that by Schauder estimates \cite[Theorem 6.2]{gilbarg2015elliptic}, for any $r \geq 0$,
\begin{equation}\label{norms on uk}
	\|u\|_{C^r(3\Ball^{d - 1})} \lesssim_r 1.
\end{equation}

\begin{lemma}
Suppose that $u_2 \geq u_1$ satisfy $Pu_1 = Pu_2 = 0$ on $4\Ball^{d - 1}$ and $v := u_2 - u_1$.
Then for $K_0 \ll 1$, $\|u_i\|_{C^1} \lesssim 1$, 
\begin{equation}\label{Schauder Harnack}
	\|\dif v\|_{C^0(\Ball^{d - 1})} \lesssim \sup_{2\Ball^{d - 1}} v \lesssim \inf_{\Ball^{d - 1}} v.
\end{equation}
\end{lemma}
\begin{proof}
Let
$$Qv := \int_0^1 \frac{\partial F(s)}{\partial H_{ij}} \partial_i \partial_j v + \frac{\partial F(s)}{\partial \xi_i} \partial_i v + \frac{\partial F(s)}{\partial y} v \dif s$$
where 
$$F(s) := F(x, su_2(x) + (1 - s)u_1(x), \dif(su_2 + (1 - s) u_1)(x), \nabla^2(su_2 + (1 - s)u_1)(x)).$$
Then $Q$ is a linear differential operator, and $Qv = 0$ \cite[(7.25)]{colding2011course}.
The principal symbol $q$ of $Q$ satisfies
\begin{align*}
q_{ij} &= \int_0^1 \frac{\partial F(s)}{\partial H_{ij}} \dif s = \int_0^1 h_{ij}(s) \dif s \\
&= \delta_{ij} + O(K_0(|x|^2 + \|u_1\|_{C^0}^2 + \|u_2\|_{C^0}^2) + \|\dif u_1\|_{C^0} + \|\dif u_2\|_{C^0}).
\end{align*}
Then if $K_0$ is chosen smaller than an absolute constant, the first eigenvalue $\lambda_1(q) \geq \frac{1}{2}$ on $3\Ball^{d - 1}$.
By (\ref{norms on uk}), the coefficients of $Q$ are bounded in $C^1$ on $3\Ball^{d - 1}$.
The claim now follows from Schauder estimates \cite[Theorem 6.2]{gilbarg2015elliptic} and the Harnack inequality \cite[Corollary 9.25]{gilbarg2015elliptic}.
\end{proof}

For the remainder of the paper we fix a constant $K_0$ satisfying the hypotheses of the above lemma.

Given a connected oriented hypersurface $N$, we denote by $[N]$ the $d-1$-current defined by 
$$\int_M [N] \wedge \psi = \int_N \psi.$$
In particular, if we view $N$ as a lamination with one leaf and transverse measure $\mu$ assigning that leaf mass $1$, then $[N] = T_\mu$ is the Ruelle-Sullivan current for $(N, \mu)$.
Moreover, if $N = F_* \sigma$ for some simplex $\sigma$ and $C^1$ map $F$, then 
$$\int_M [N] \wedge \psi = \int_\sigma F^* \psi,$$
so if $N_1, N_2$ are $C^1$-close, then $F_1^*, F_2^*$ are $C^0$-close, and hence $[N_1], [N_2]$ are close in the weak topology of measures.

\begin{lemma}\label{measured convergence is smooth convergence}
Let $C > 0$, let $(N_n)$ be a sequence of minimal surfaces with $\|\Two_{N_n}\|_{C^0} \leq C$, and let $N$ be a surface.
If $[N_n] \to [N]$ in the weak topology of measures, then $N$ is a minimal surface.
\end{lemma}
\begin{proof}
Let $p \in N = \supp [N]$; as in Lemma \ref{supports shrink in the limit} there exist $p_n \in N_n$ with $p_n \to p$.
Then by Lemma \ref{convergence of normals}, $\normal_{N_n}(p_n) \to \normal_N(p)$.
Rescaling $M$ by an amount that depends on $C$ and working in normal coordiantes $(x, y) \in \RR^{d - 1} \times \RR$ at $p$ with $\partial_y|_p = \normal_N(p)$, it follows that in a neighborhood of $p$, for every $n$ large, $\normal_{N_n}$ is $C^0$ close to $\partial_y$.
In particular, $N_n$ are the graphs of functions $u_n: \RR^{d - 1}_x \to \RR_y$ which are bounded in $C^1$ and solve $Pu_n = 0$.
By (\ref{norms on uk}), $(u_n)$ is precompact in $C^\infty$.
Similarly, $N$ is the graph of some $u$, and along a subsequence $u_{n_k} \to \tilde u$ in $C^\infty$ for some $\tilde u$, which then has a graph $\tilde N$.
In particular, $[N_{n_k}] \to [\tilde N]$ in the weak topology of measures; it follows that $\tilde N = N$, so $\tilde u = u$.
Therefore $u_{n_k} \to u$ in $C^\infty$, hence $Pu = 0$.
\end{proof}




%%%%%%%%%%%%%%%%%%%%%%%%%%%%%%%%%%%%%%%%%%
\section{Regularity of laminations}\label{Regularity}
The goal of this section is to prove the regularity theorem, Theorem \ref{regularity theorem}, for minimal laminations.
The proof is based on \cite[Theorem 1.1]{Solomon86}, which addresses the case that $\lambda$ is a minimal foliation, and does not explictly spell out the quantitative regularity of the laminar atlas.

\subsection{Representation as graphs}
We first consider when we can represent the leaves of $\lambda$ as graphs in a uniform way, or equivalently when we can choose coordinates so that the leaves of $\lambda$ are uniformly far from vertical.
If $f: \RR^{d - 1}_x \to \RR_y$ is a $C^1$ function, and its graph has normal vector $\normal$, then
\begin{equation}\label{nabla as a normal}
	\normal = \frac{\partial_y - \nabla f}{\sqrt{1 + |\nabla f|^2}},
\end{equation}
so by a bootstrapping argument we may approximate $\normal$ by $\partial_y - \nabla f$.
To be more precise:

\begin{lemma}\label{existence of tubes}
	Let $N$ be a connected $C^2$ hypersurface in $\RR^d = \RR^{d - 1}_x \times \RR_y$ which is tangent to $\{y = 0\}$ at the origin.
	If $\|\Two_N\|_{C^0} \leq \frac{1}{8}$, then for every $(x, y) \in N \cap B_1$,
	$$\max(|y|, |\normal(x, y) - \partial_y|) \leq 3\|\Two_N\|_{C^0}.$$
\end{lemma}
\begin{proof}
	Near $0$, $N$ can be represented a graph $\{y = f(x)\}$, since it is tangent to $\{y = 0\}$.
	This representation is valid on the component of the set $\{|\nabla f(x)| < \infty\}$ containing $0$, and it is related to the unit normal by (\ref{nabla as a normal}).
	Rearranging (\ref{nabla as a normal}),
	$$\nabla f = \partial_y - \frac{\normal}{\sqrt{1 + |\nabla f|^2}},$$
	and then taking derivatives,
	$$-\nabla^2 f(x) = \frac{\nabla \normal(x, f(x)) \cdot (\partial_x \otimes \partial_x + \nabla f(x) \otimes \partial_y)}{\sqrt{1 + |\nabla f(x)|^2}} - \frac{\nabla^2 f(x) \cdot (\nabla f(x) \otimes \normal(x, f(x)))}{(1 + |\nabla f|^2)^{3/2}}.$$
	Here $-\nabla^2$ denotes the negative Hessian, not the Laplacian.
	Since
	$$|\partial_x \otimes \partial_x + \nabla f(x) \otimes \partial_y| \leq \sqrt{1 + |\nabla f(x)|^2},$$
	and $\nabla \normal = \Two_N$, we conclude
\begin{equation}\label{bound Hessian by Two}
	|\nabla^2 f(x)| \leq |\Two_N(x, f(x))| + |\nabla^2 f(x)| |\nabla f(x)|.
\end{equation}
	In order to control the error terms in (\ref{bound Hessian by Two}), we make the \dfn{bootstrap assumption}
\begin{equation}\label{bootstrap}
	|\nabla f(x)| \leq \frac{1}{2},
\end{equation}
	which is at least valid in some small neighborhood $B_R$ of $0$ since (\ref{nabla as a normal}) and the fact that $N$ is tangent to $\{y = 0\}$ at $0$ imply that $\nabla f(0) = 0$.
	By (\ref{bound Hessian by Two}),
$$|\nabla^2 f(x)| \leq 2|\Two_N(x, f(x))|,$$
	and integrating this inequality one obtains for $|x| \leq R$ that
\begin{equation}\label{closed bootstrap}
	|\nabla f(x)| \leq 2|\Two_N(x, f(x))| \leq \frac{1}{4}.
\end{equation}
	In particular, since $\nabla f \in C^1$, either $R \geq 1$ or there exists $R' > R$ such that the bootstrap assumption (\ref{bootstrap}) is valid on $B_{R'}$, so (\ref{bootstrap}) is valid with $R = 1$.
	Integrating (\ref{closed bootstrap}),
$$|f(x)| \leq 2\|\Two_N\|_{C^0}.$$
	Moreover, by (\ref{nabla as a normal}) and (\ref{closed bootstrap}),
\begin{align*}
	|\normal(x, f(x)) - \partial_y| &\leq |\nabla f(x)| + \left|1 - \frac{1}{\sqrt{1 + |\nabla f(x)|^2}}\right| \\
	&\leq 2\|\Two_N\|_{C^0} + 4\|\Two_N\|_{C^0}^2 \leq 3\|\Two_N\|_{C^0}. \qedhere
\end{align*}
\end{proof}

Our next lemma guarantees the existence of normal coordinates in which the leaves of $\lambda$ are close to hyperplanes $\{y = y_0\}$.
An analogous result was proven by \cite{Solomon86} (without the quantitative dependence) by a very different means, using the regularity theory for integral flat convergence of minimal currents \cite[Theorem 5.3.14]{federer2014geometric}.
We did not do this, both because \cite{Solomon86} requires that $\lambda$ be a foliation, and because it does not seem particularly easy to recover quantitative bounds from the highly general theory of \cite[Chapter 5]{federer2014geometric}.

\begin{lemma}\label{lams have C0 fields}
	For every sufficiently small $\delta > 0$ there exist $r = r(\delta, g, A) > 0$ such that for every disjoint family of minimal surfaces $\mathcal S$ satisfying the curvature bound (\ref{curvature bound in regularity}), we can choose normal coordinates $(x, y) \in \RR^{d - 1} \times \RR$ on $B(P, r)$ so that
\begin{equation}\label{normal is basically dy}
	\sup_{N \in \mathcal S} \|\normal_\lambda - \partial_y\|_{C^0(B(P, r))} \leq \delta.
\end{equation}
\end{lemma}
\begin{proof}
	Suppose not.
	Then, after rescaling to set $A = 1$, there exist disjoint families of minimal surfaces $\mathcal S_n'$ in $B(P, 1/n)$ such that
	$$\sup_{N \in \mathcal S_n'} \|\Two_N\|_{C^0} \leq 1,$$
	but no matter how we rotate normal coordinates based at $P$, (\ref{normal is basically dy}) fails for $\mathcal S_n'$.
	By passing to normal coordinates and rescaling by $n$, we can find disjoint families of (possibly nonminimal) surfaces $\mathcal S_n$ in $B_1 \subset \RR^d$ such that
\begin{equation}\label{bounds on Two in representation}
	\sup_{N \in \mathcal S_n} \|\Two_N\|_{C^0} \lesssim_g \frac{1}{n^2}
\end{equation}
	but for every rotation of $\RR^d$, (\ref{normal is basically dy}) fails.
	Here we used the fact that if $\Two_N$ is $C^0$-small, then in normal coordinates and small enough curvature (which holds for $n \gg 1$, because of the rescaling), the second fundamental form of the representation of $N$ in coordinates is $\lesssim \|\Two_N\|_{C^0}$.

	By Lemma \ref{existence of tubes} and (\ref{bounds on Two in representation}), for every $n \gg 1$, every member of $\mathcal S_n$ is $O(n^{-2})$-close to its tangent spaces in $C^1$.
	In particular, if $n \gg \delta^{-1/2}$, and the normal vector at some point to $N \in \mathcal S_n$ is $\partial_y$, then
	$$\|\normal_N - \partial_y\|_{C^0(B(P, r))} \ll \delta.$$
	We can always impose this for some $N$ by applying a rotation of $\RR^d$.
	But by our contradiction assumption, there exists some leaf $N'$ of $\mathcal S_n$ and some $P \in N'$ so that $|\normal_{N'}(P) - \partial_y| \geq \delta$ and hence if $n$ is large enough,
	$$\inf_{N'} |\normal_{N'} - \partial_y| \geq \frac{\delta}{2}.$$
	By the reverse triangle inequality it follows that
\begin{equation}\label{discrepancy in normals}
	\inf_{\substack{P \in N\\ P' \in N'}} \sin \angle(\normal_N(P), \normal_{N'}(P')) \gtrsim \delta
\end{equation}
	at least if $\delta$ is smaller than an absolute constant.
	
	In order to obtain a contradiction, we may assume that $r$ is small depending on $\delta$, and consider the restriction of $\mathcal S_n$ to $B_r$.
	In particular, we may assume that there are points $P \in N \cap B_r$ and $P' \in N' \cap B_r$ for some $r \ll \delta$.
	By (\ref{discrepancy in normals}), the angle $\theta$ between the tangent spaces $T_P N$ and $T_{P'} N'$ is $\gtrsim \delta$, but $|P - P'| \ll \delta$.
	This is only possible if $T_P N$ and $T_{P'} N'$ intersect in some ball $B_{r'}$ for some $r' \ll 1$, say $r' = 1/2$.
	But then, since $N, N'$ are $O(n^{-2})$-close to $T_P N$ and $T_{P'} N'$ in $C^0$, then $N$ and $N'$ intersect in $B_1$ if $n$ is large enough.
	However, $N, N'$ were assumed to be disjoint, so this is a contradiction.
\end{proof}

\subsection{Proof of Theorem \ref{regularity theorem}}
Fix $\delta > 0$ to be chosen later, and $P \in M$.
By Lemma \ref{lams have C0 fields}, if $\delta \leq \delta_*$ for some $\delta_* = \delta_*(g) > 0$, there exists $r = r(\delta, g, A) > 0$ such that $B(P, r)$ admits rescaled normal coordinates $(x, y) \in 5\Ball^{d - 1} \times (-2, 2)$ in which the curvature of the rescaled metric has a $C^0$ norm $\leq K_0$ and
\begin{equation}\label{normal is almost constant}
\|\normal - \partial_y\|_{C^0(B(P, r))} \leq \delta.
\end{equation}
Moreover,
$$|\normal \cdot \partial_y| \geq 1 - |\normal - \partial_y| \geq 1 - \delta,$$
so if we select $\delta := \min(\delta_*, \frac{1}{4})$, then in $5\Ball^{d - 1} \times (-1, 1)$, then every leaf is the graph of a function, say $u_k: 5\Ball^{d - 1} \to (-2, 2)$ where $u_k(0) = k$, and
$$\|\dif u_k\|_{C^0} \leq \frac{1 - (1 - \delta)^2}{1 - \delta} \leq 1.$$
If $r$ is chosen small enough depending on $g$, then the metric $\tilde g$ induced by $g$ on $5\Ball^{d - 1} \times (-2, 2)$ satisfies $\|\Riem_{\tilde g}\|_{C^0} \leq K_0$.
Moreover, $\|u_k\|_{C^0} \leq 2$, and $u_k$ has a minimal graph, so the elliptic estimates stated in \S\ref{Leaf estimates} apply to $u_k$ uniformly in $k$.

Now let $-1 < k < \ell < 1$, and let $v_{\ell k} := u_\ell - u_k$.
By (\ref{Schauder Harnack}) with $v := v_{\ell k}$, for every $x \in \Ball^{d - 1}$,
\begin{equation}\label{bound on du}
|\dif u_\ell(x) - \dif u_k(x)| \lesssim |u_\ell(x) - u_k(x)|
\end{equation}
and it follows that
\begin{equation}\label{vertical Lipschitz}
|\normal(x, u_\ell(x)) - \normal(x, u_k(x))| \lesssim |u_\ell(x) - u_k(x)|.
\end{equation}

To extend (\ref{vertical Lipschitz}) to a Lipschitz bound on $\normal$, let $X_1, X_2 \in (\Ball^{d - 1} \times (-1, 1)) \cap \supp \lambda$.
Then there exist $x_1, x_2 \in \Ball^{d - 1}$ and $k_1, k_2 \in (-1, 1)$ such that $X_i = (x_i, u_{k_i}(x_i))$.
Setting $Y := (x_2, u_{k_1}(x_2))$,
$$|\normal(X_1) - \normal(X_2)| \leq |\normal(X_1) - \normal(Y)| + |\normal(Y) - \normal(X_2)|.$$
Then by (\ref{norms on uk}) and the mean value theorem,
$$|\normal(X_1) - \normal(Y)| \lesssim |\dif u_{k_1}(x_1) - \dif u_{k_1}(x_2)| \lesssim |X_1 - Y|.$$
Moreover, by (\ref{vertical Lipschitz}),
$$|\normal(Y) - \normal(X_2)| \lesssim |u_{k_1}(x) - u_{k_2}(x)| = |Y - X_2|.$$
Since $\delta \leq \frac{1}{4}$, by (\ref{normal is almost constant}),
$$|\sin \angle(X_1 - Y, X_2 - Y)| > 1 - O(\delta)$$
and we conclude by the Pythagorean theorem that
$$|Y - X_2|^2 + |X_1 - Y|^2 \lesssim |X_1 - X_2|^2.$$
In conclusion,
$$|\normal(X_1) - \normal(X_2)| \lesssim |X_1 - X_2|$$
which implies that $\normal$ is Lipschitz on $V \cap \supp \lambda$, where $V$ is the neighborhood of $P$ which was mapped to $\Ball^{d - 1} \times (-1, 1)$ by the cylindrical coordinates $(x, y)$.
In particular, $V$ contains a ball of the form $B(P, s)$, where $s$ only depends on $r$ (and $r$ only depends on $g$ and $A$).
Taking a Lipschitz extension of $\normal$ to $V$ we obtain the desired Lipschitz normal subbundle.

Following \cite[Appendix B]{ColdingMinicozziIV}, we construct the laminar flow box
\begin{align*}
	F: \RR^{d - 1}_\xi \times \RR_\eta &\to V \subseteq \RR^{d - 1}_x \times \RR_y \\
	(\xi, \eta) &\mapsto (\xi, f(\xi, \eta))
\end{align*}
by setting
$$f(\xi, \eta) := u_\eta(\xi)$$
if $u_\eta$ exists, and if $k < \eta < \ell$ and there does not $k < \eta' < \ell$ such that $u_{\eta'}$ exists, then
$$f(\xi, \eta) := u_k(\xi) + \frac{\eta - k}{\ell - k} v_{\ell k}(\xi)$$
is the linear interpolant of $u_k$ and $u_\ell$.

By (\ref{norms on uk}), $F$ is bounded in tangential $C^\infty$.
In particular, if $V$ is a vector field tangent to $\{\eta = k\}$, then the pushforward
$$F_* V = V^i \partial_{x^i} + (Vf) \partial_y$$
is well-defined, and pushforwards of such vector fields span the tangent bundle of the graph of $u_k$. 
The bound
\begin{equation}\label{xiLip of f}
	\|\partial_\xi f\|_{C^0} \lesssim \sup_k \|u_k\|_{C^1} \lesssim 1,
\end{equation}
a consequence of (\ref{norms on uk}), establishes that $\|F_* V\|_{C^0} \sim \|V\|_{C^0}$, and then 
$$\|(F_* V) F^{-1}\|_{C^0} \lesssim \|V(F \circ F^{-1})\|_{C^0} \leq \|V\|_{C^0} \sim \|F_* V\|_{C^0}.$$
Since $V$ was arbitrary we conclude that $F^{-1}$ is bounded in tangential $C^1$, hence in tangential $C^\infty$ by the inverse function theorem. 

It remains to show that $F$ is a Lipschitz isomorphism.
To do this, we first claim that $\Lip(f) \sim 1$.
In the $\xi$ direction, we use (\ref{xiLip of f}).
If $-1 < k < \ell < 1$, then by (\ref{bound on du}) and (\ref{Schauder Harnack}),
\begin{equation}\label{f lip}
	|f(\xi, k) - f(\xi, \ell)| \lesssim |u_k(\xi) - u_\ell(\xi)| \lesssim \ell - k.
\end{equation}
This shows that $f$ is Lipschitz in the $\eta$ direction on the leaves with constant comparable to $1$, and hence on its entire domain by linear interpolation, proving the claim.
We can then estimate using (\ref{f lip})
$$|F(\xi_1, \eta_1) - F(\xi_2, \eta_2)| \lesssim |\xi_1 - \xi_2| + \Lip(f)(|\xi_1 - \xi_2| + |\eta_1 + \eta_2|)$$
so that $\Lip(F) \lesssim 1 + \Lip(f) \lesssim 1$.

To obtain a bound on $\Lip(F^{-1})$, we observe that
\begin{equation}\label{F is coLip in xi}
|\xi_1 - \xi_2|^2
\leq |\xi_1 - \xi_2|^2 + |f(\xi_1, \eta_1) - f(\xi_2, \eta_1)|^2 
= |F(\xi_1, \eta) - F(\xi_2, \eta)|^2.
\end{equation}
By Harnack's inequality with $\eta_1 = k$ and $\eta_2 = \ell$, or $k \leq \eta_1 < \eta_2 \leq \ell$ if $\eta_1, \eta_2$ lie in the plaque between leaves $k, \ell$,
$$\frac{|f(\xi_1, \eta_1) - f(\xi_1, \eta_2)|}{|\eta_1 - \eta_2|} \gtrsim \frac{v_{\ell k}(\xi_1)}{\ell - k} \gtrsim \frac{v_{\ell k}(0)}{\ell - k} = 1$$
whence by the mean value theorem and (\ref{F is coLip in xi}),
\begin{align*}
	|\eta_1 - \eta_2| 
	&\lesssim |f(\xi_1, \eta_1) - f(\xi_1, \eta_2)| \\
	&\leq |f(\xi_1, \eta_1) - f(\xi_2, \eta_2)| + \|\partial_\xi f\|_{C^0} |\xi_1 - \xi_2| \\
	&\leq (1 + \|\partial_\xi f\|_{C^0}) |F(\xi_1, \eta_1) - F(\xi_2, \eta_2)|.
\end{align*}
By (\ref{norms on uk}) and the fact that either $\partial_\xi f = \partial_\xi u_\eta$, or there are $k,\ell$ such that $\partial_\xi f$ is the linear interpolation of $\partial_\xi u_k$ and $\partial_\xi u_\ell$, $\|\partial_\xi f\|_{C^0} \lesssim 1$.
Thus
$$|F(\xi_1, \eta_1) - F(\xi_2, \eta_2)| \gtrsim |\xi_1 - \xi_2|^2 + |\eta_1 - \eta_2|^2.$$
It follows that $\Lip(F^{-1}) \lesssim 1$, so $F$ is a Lipschitz isomorphism with constants comparable to $1$.

Finally, we compose $F$ with the change of coordinates at the start of this proof to obtain a laminar flow box in a small neighborhood of $(0, 0)$ whose image has radius $O(r)$, and whose Lipschitz constants are comparable to $O(r^{-1})$.

%%%%%%%%%%%%%%%%%%%%%%%%%%%%%%%%%%%%%%%%%
\section{Proofs of main theorems}\label{CompactnessSec}
\subsection{Compactness}
We now prove Theorem \ref{compactness theorem}.

\subsubsection{Construction of the limiting flow box}
Let $P \in M$.
By Theorem \ref{regularity theorem}, there exist $r > 0$ and $L \geq 1$ such that for every large $n \in \NN$, $B(P, r)$ is contained in the image of a flow box $F_n$ for $\lambda_n$ with Lipschitz constant $L$, such that $F_n(0, 0) = P$.
By the Arzela-Ascoli theorem, along a subsequence $F_n \to F$ in $C^0$ for some map $F: I \times J \to B(P, r)$ and some $I \subseteq \RR$, $J \subseteq \RR^{d - 1}$, such that on the image $V$ of $F$, we also have the convergence $F_n^{-1} \to F^{-1}$.
Moreover, $F(0, 0) = P$, so that $F: I \times J \to V$ is a homeomorphism onto a set which contains $P$.
Since
$$\max(\Lip(F), \Lip(F^{-1})) \leq \limsup_{n \to \infty} \max(\Lip(F_n), \Lip(F_n^{-1})) \leq L,$$
it follows that $\max(\Lip(F), \Lip(F^{-1})) \leq L$, and for any $\theta \in (0, 1)$,
\begin{align*}
	\|F - F_n\|_{C^\theta}
	&\leq \Lip(F - F_n)^\theta \|F - F_n\|_{C^0}^{1 - \theta} \leq (2L)^\theta \|F - F_n\|_{C^0}^{1 - \theta}.
\end{align*}
It follows that $F_n \to F$ in $C^\theta$, hence in $C^{1-}$, and similarly for $F^{-1}$.
Since $(F_n)$ and $(F_n^{-1})$ are bounded in tangential $C^\infty$, a similar compactness argument to the above shows that $F_n \to F$ and $F_n^{-1} \to F^{-1}$ in tangential $C^\infty$.

Since $P$ was arbitrary, it follows that we can find laminar atlases $(F_\alpha^n, K_\alpha^n)$ for each large $n \in \NN$ such that $F_\alpha^n \to F_\alpha$ in $C^{1-}$, where the images of $F_\alpha$ and $F_\alpha^n$ are an open cover $(U_\alpha)$ of $M$ independent of $n$, and $(F_\alpha)$ satisfies the usual transition relations, and $F_\alpha$ is a Lipschitz isomorphism.

\subsubsection{Construction of the limiting lamination}
We now construct the limiting lamination.
Let $\Hypspace I$ be the space of closed subsets of $I$ with its Hausdorff distance; since $I$ is a compact metric space, so is $\Hypspace I$ \cite[Theorem 4.17]{nadler2017continuum}, so we may diagonalize so that for every $\alpha$, either $K^n_\alpha \to K_\alpha$ for some nonempty $K_\alpha$ in the Hausdorff distance on $I$, or for all $n \geq n^*(\alpha)$, $K_\alpha^n$ is empty (in which case we define $K_\alpha = \emptyset$).

In order to ensure that the laminations $\lambda_n$ do not escape to infinity, fix a compact set $E \subseteq M$ such that every leaf of every $\lambda_n$ meets $E$.
Then there exists a finite set $A_E \subseteq A$ such that $E \subseteq \bigcup_{\alpha \in A_E} U_\alpha$.

\begin{lemma}\label{label sets are nonempty}
	There exists $\alpha$ such that $K_\alpha$ is nonempty.
\end{lemma}
\begin{proof}
	Suppose not; then for
	$$n \geq \max_{\alpha \in A_E} n^*(\alpha)$$
	and $\alpha \in A_E$, $K_\alpha^n = \emptyset$, so no leaves of $\lambda_n$ meet $U_\alpha$, and hence no leaves of $\lambda_n$ meet $E$.
	This is a contradiction since $\lambda_n$ has a leaf.
\end{proof}

In each flow box $F_\alpha$ with $K_\alpha$ nonempty, we thus have the leaves of a lamination, namely $K_\alpha \times J$.
We now check the transition relations to ensure that they glue to a global lamination; this is straightforward but we include it for completeness.

Thus let $\psi_{\alpha \beta}$ and $\psi_{\alpha \beta}^n$ be the transition maps, thus $\psi_{\alpha \beta}^n$ induces a map
$$\psi_{\alpha \beta}^n: K_\alpha^n \to K_\beta^n.$$
By convergence of $(F_\alpha^n)$, $\psi_{\alpha \beta}$ induces a map $K_\alpha \to K_\beta$.

\begin{definition}
	A \dfn{cocycle of labels} $(k_\alpha)_{\alpha \in A'}$ is a set $A' \subseteq A$ and an element of $\prod_{\alpha \in A'} K_\alpha$, such that:
\begin{enumerate}
	\item The cocycle condition: $k_\beta = \psi_{\alpha \beta}(k_\alpha)$ for $\alpha, \beta \in A'$.
	\item For every $\alpha \in A'$, if $\psi_{\alpha \beta}(k_\alpha)$ is well-defined, then $\beta \in A'$.
\end{enumerate}
\end{definition}

\begin{lemma}
	Every cocycle of labels $(k_\alpha)_{\alpha \in A'}$ defines a complete minimal hypersurface $N$ such that
	$$N \cap U_\alpha = F_\alpha(\{k_\alpha\} \times J).$$
\end{lemma}
\begin{proof}
We have the cocycle condition
$$(N \cap U_\alpha) \cap U_\beta = (N \cap U_\beta) \cap U_\alpha$$
which follows from the fact that
\begin{align*}
F_\alpha(\{k_\alpha\} \times J) \cap U_\beta
&= F_\beta(\psi_{\alpha \beta}(\{k_\beta\} \times J)) \cap U_\alpha \cap U_\beta \\
&= F_\beta(\psi_{\alpha \beta}(\{k_\beta\} \times J)) \cap U_\alpha.
\end{align*}
From the cocycle condition, it follows that $N$ honestly defines a Lipschitz hypersurface in $M$, which is complete in $\bigcup_{\alpha \in A'} U_\alpha$.
If $\overline N$ intersects $U_\alpha$ for some $\alpha \notin A'$, then $N$ intersects $U_\beta$ for some $\beta \in A'$ so that $U_\beta \cap U_\alpha \cap \overline N$ is nonempty.
But then $\psi_{\beta \alpha}(k_\beta)$ must be defined, so $\alpha \in A'$, a contradiction.
Therefore $N$ is complete in $M$.

To prove minimality, let
$$u_\alpha(k, x) = (F_\alpha)_* 1_{k > k_\alpha}$$
and similarly $u_\alpha^n(k, x) = (F_\alpha^n)_* 1_{k > k_\alpha^n}$ where $(k_\alpha^n) \in \prod_n K_\alpha^n$ converges to $k_\alpha$.
Since $F_\alpha \circ (F_\alpha^n)^{-1}$ converges to the identity map in $C^{1-}$, and $F_\alpha^{-1}(N \cap U_\alpha)$ has zero measure, it follows that $u_\alpha^n \to u_\alpha$ almost everywhere, and hence in $L^1(I \times J)$ by the dominated convergence theorem.
But $u_\alpha^n$ has least gradient, so by the Miranda stability theorem, $\dif u_\alpha^n \to \dif u_\alpha$ in the weak topology of measures.
Clearly $\dif u_\alpha = [N \cap U_\alpha]$ and similarly for $u_\alpha^n$, so by Lemma \ref{measured convergence is smooth convergence}, $N \cap U_\alpha$ is minimal.
\end{proof}

\begin{lemma}
	Let $\lambda$ be the lamination with laminar atlas $(F_\alpha, K_\alpha)$.
	Then $\lambda$ is well-defined and minimal.
\end{lemma}
\begin{proof}
Since 
$$\supp \lambda \cap U_\alpha = K_\alpha \times J$$
and $K_\alpha$ is compact, $\supp \lambda$ is closed.
Now if we choose $\alpha$ such that $K_\alpha$ is nonempty, every element of $K_\alpha$ uniquely determines a cocycle of labels, and hence a leaf of $\lambda$.
So $\supp \lambda$ is nonempty, and since all of its leaves are complete minimal, $\lambda$ is minimal.
\end{proof}

\subsubsection{Convergence in Thurston's geometric topology}
At this stage of the argument we have constructed a limiting lamination with limiting flow boxes; we now check that the convergence occurs in other modes as well.

If $K_\alpha$ is nonempty, then any $k_\alpha \in K_\alpha$ is the limit of some sequence $(k_\alpha^n)_n \in \prod_n K_\alpha^n$ \cite[Theorem 4.11]{nadler2017continuum}.
Thus $\{k_\alpha\} \times J$ can be written as the set of limits of sequences $(k_\alpha^n, x)_n \in \prod_n K_\alpha^n \times J$, and so any leaf $N$ of $\lambda$ takes the form $N = \lim_{n \to \infty} N_n$ for some sequence $(N_n) \in \prod_n \Leaves \lambda_n$.
In other words, leaves of $\lambda$ are pointwise limits of leaves in $\lambda_n$.

So it suffices to show that for $N \in \Leaves \lambda$, $P \in N$, and $P_n \to P$, where $P_n \in N_n$ and $N_n \in \Leaves \lambda_n$, $\normal_{N_n}(P_n) \to \normal_N(P)$.
To do this, suppose that $P \in U_\alpha$; $F_\alpha^n$ is close in tangential $C^\infty$ to $F_\alpha$, and the label $k^n_\alpha$ of $N_n$ is close to the label $k_\alpha$ of $N$.
In particular, if we consider $N$ and $N_n$ as graphs of functions $u, u_n$ in the coordinates induced by $F_\alpha$, then $u_n \to u$ in $C^\infty$; however, in such coordinates, $u$ is a constant.
A bootstrapping argument based on (\ref{nabla as a normal}) then shows that, since $\dif u_n \to 0$ in $C^0$, $\normal_{N_n} \to \partial_y = \normal_N$ in $C^0$ near $P$.

\subsubsection{Convergence in the measure topology}
Suppose that $\mu_n$ is transverse to $\lambda_n$.
After possibly shrinking the $U_\alpha$ slightly for $\alpha \in A_E$, we may assume that they are precompact in $M$ and still form an open cover of $E$.
Then $K := \bigcup_{\alpha \in A_E} \overline{U_\alpha}$ is compact, so by Prohorov's theorem \cite[Theorem 13.29]{klenke2013probability}, there is a subsequence of $(T_{\mu_n})$ which converges to some $T_\mu|_K$ on $K$.
Moreover, by the portmanteau theorem,
$$\supp T_\mu|_K \subseteq \liminf_{n \to \infty} \supp T_{\mu_n}|_K \subseteq \liminf_{n \to \infty} \supp \lambda_n \cap K.$$
Here the $(\lambda_n)$ in the limit inferior refers to the subsequence which already converges in the Thurston topology (and has converging Ruelle-Sullivan currents).
In particular, the limit inferior is actually a limit and we conclude
$$\supp T_\mu|_K \subseteq \supp \lambda \cap K.$$
We may assume that $\mu_\alpha^n \to \mu_\alpha$ weakly for every $\alpha \in A_E$ and some positive Radon measures $\mu_\alpha$ (whose support is necessarily then contained in $K_\alpha$).
Taking the limit as $n \to \infty$ of the equation 
$$\int_{U_\alpha} T_{\mu_n} \wedge \varphi = \int_I \int_{\{k\} \times J} (F_\alpha^n)^* \varphi \dif \mu_\alpha^n(k),$$
we conclude that
$$\int_{U_\alpha} T_\mu|_K \wedge \varphi = \int_I \int_{\{k\} \times J} F_\alpha^* \varphi \dif \mu_\alpha(k).$$
In other words, $T_\mu|_K$ is Ruelle-Sullivan for $\lambda|_K$, possibly after shrinking $\lambda|_K$ so that their supports match.
By the measure-preserving condition in the definition of transverse measure, $T_\mu|_K$ extends uniquely to a Ruelle-Sullivan current $T_\mu$ on all of $M$, which then necessarily is a weak limit of $(T_{\mu_n})$.
This completes the proof of Theorem \ref{compactness theorem}.


%%%%%%%%%%%%%%%%%%%%%%%%%%%%%%%%%%%%%%
\subsection{Consequences of measured convergence}
We now prove Theorem \ref{implication theorem}. Suppose that $\dim M \leq 7$.
We use the following lemma:

\begin{lemma}\label{limits of measured geodesic lams are geodesic}
	The set of minimal measured laminations of bounded curvature is closed in the weak topology of measures.
\end{lemma}
\begin{proof}
Let $(\lambda, \mu)$ be a measured lamination and suppose that $(\lambda_i, \mu_i) \to (\lambda, \mu)$ in the weak topology of measures, where $(\lambda_i, \mu_i)$ are measured minimal and of bounded curvature.
Let $x \in \supp \lambda$ and $r > 0$ such that $B := B(x, r)$ is contractible.
In $B$, we can write $T_{\mu_i} = \dif u_i$ for some sequence of functions of least gradient $u_i \in BV(B)$.
Since $u_i$ is only defined up to a constant, we impose $\int_M \star u_i = 0$, so by Poincar\'e's inequality,
$$\|u_i\|_{L^1(B)} \lesssim r\mu_i(B) \leq 2r \mu(B) < \infty$$
for $i$ large.
So by the Miranda stability theorem, there exists a $1$-harmonic function $u$ such that along a subsequence, $\dif u_i \to \dif u$ in the weak topology of measures.
Then $T = \dif \mu$, so the leaves of $\lambda$ are level sets of $u$.
By Theorem \ref{main thm of old paper}, the leaves of $\lambda$ are minimal hypersurfaces, as desired.
\end{proof}

\subsubsection{Convergence in Thurston's geometric topology}
Let $(\lambda_n, \mu_n)$ be a sequence of measured minimal laminations converging to $(\lambda, \mu)$.
By (\ref{supports shrink in the limit}), for every $x \in \supp \lambda$, $\varepsilon > 0$, and large $n$, $\supp \lambda_n \cap B(x, \varepsilon)$ is nonempty, and by Lemma \ref{limits of measured geodesic lams are geodesic}, $\lambda$ is a minimal lamination.
By Theorem \ref{regularity theorem}, $\lambda, \lambda_n$ admit Lipschitz normal vectors, so by Lemma \ref{convergence of normals}, $\lambda_n \to \lambda$ in Thurston's geometric topology.

\subsubsection{Convergence in the \texorpdfstring{$C^{1-}$}{H\"older} flow box topology}
Let $(\lambda_n, \mu_n)$ be a sequence of measured minimal laminations converging to $(\lambda, \mu)$, such that $(\lambda_n)$ has bounded curvature.
Then by the above, $\lambda_n \to \lambda$ in Thurston's geometric topology.
After discarding some leaves of $\lambda_n$ we may assume that $\lambda$ is a maximal limit for the Thurston topology.
Moreover, every subsequence $(\lambda_{n_k})$ has a further subsequence $(\lambda_{n_{k_\ell}})$ which converges to some maximal limit $\tilde \lambda$ in the $C^{1-}$ flow box topology by Theorem \ref{compactness theorem}.
But convergence in the flow box topology implies convergence in Thurston's topology, so $\tilde \lambda = \lambda$.
Since $(\lambda_{n_k})$ was arbitrary, it follows that $\lambda_n \to \lambda$ in the $C^{1-}$ flow box topology.


%%%%%%%%%%%%%%%%%%%%
\subsection{Application to \texorpdfstring{$1$-harmonic}{one-harmonic} functions}
We finally prove Theorem \ref{main thm}. Suppose that $\dim M \leq 4$.

\subsubsection{$1$-harmonic function induces minimal lamination}
Let $u$ be a $1$-harmonic function on $M$.
By Theorem \ref{main thm of old paper}, the level sets of $u$ are closed embedded minimal hypersurfaces in $M$; let
$$Y = \{y \in \RR: \partial \{u > y\} \neq \emptyset\}$$
index the level sets of $u$.

Let $y, z \in Y$. If $y > z$, then $\{u > y\} \subseteq \{u > z\}$, so $\partial \{u > y\}$ lies on one side of $\partial \{u > z\}$.
By the maximum principle, it follows that either $\partial \{u > y\}$ and $\partial \{u > z\}$ are disjoint, or are equal.
Moreover, $\dif u$ is conormal to $\partial \{u > y\}$, so $\partial \{u > y\}$ has trivial normal bundle.
Therefore by the stable Bernstein theorem, after replacing $M$ with an element of a compact exhaustion $(X_m)$ of $M$, we may assume that for some $A > 0$,
\begin{equation}\label{curvature estimate on 1 harmonic}
	\sup_{y \in Y} \|\Two_{\partial \{u > y\}}\|_{C^0} \leq A.
\end{equation}
Then by Theorem \ref{regularity theorem}, $\supp \dif u \cap X_m$ is the support of a lamination $\lambda_m$ with leaves given by the level sets of $u$.
Diagonalizing against the compact exhaustion $(X_m)$ and using Theorem \ref{compactness theorem}, we obtain a lamination $\lambda = \lim_m \lambda_m$ on $M$ whose leaves are exactly the level sets of $u$.
It follows that $\bigcup_{y \in Y} \partial \{u > y\}$ is the support of $\lambda$, that $\lambda$ is minimal, and that $\dif u$ is conormal to $\lambda$.
In particular, we obtain an orientation on $\lambda$ from $\dif u$.

We now construct the transverse measure to $\lambda$.
In any oriented laminar coordinates $(k, x) \in K \times J$ for $\lambda$, $\partial_x u = 0$, so $\star |\dif u|$ defines a measure $\mu$ on $K$: given $\alpha < \beta$, let
$$\mu([\alpha, \beta] \cap K) := u(\beta, x) - u(\alpha, x)$$
for any (and hence every, since $\partial_x u = 0$) $x \in J$.
Since $(k, x)$ are oriented laminar coordinates, $u(\cdot, x)$ is nondecreasing, so $\mu$ is a positive measure.

If $(k', x') \in K' \times J$ is a different laminar coordinate system, and the transition map carries $\alpha, \beta$ to $\alpha', \beta'$, then
$$\mu'([\alpha', \beta'] \cap K') := u'(\beta', x') - u(\alpha', x') = u(\beta, x_1) - u(\alpha, x_2)$$
for some $x_1, x_2 \in J$. Since $\partial_x u = 0$,
$$u(\beta, x_1) - u(\alpha, x_2) = u(\beta, x_1) - u(\alpha, x_1) = \mu([\alpha, \beta] \cap K).$$
It follows that $\mu$ is transverse, and by construction $\mu$ lifts to $\star |\dif u|$ in $M$.
So by (\ref{polar ruelle sullivan}), $\dif u = \normal_\lambda |\dif u|$ is the Ruelle-Sullivan current for the measured oriented structure we just imposed on $\lambda$.

\subsubsection{Minimal lamination induces $1$-harmonic function}
Suppose that we are given a measured oriented minimal lamination $\lambda$, which then has a Ruelle-Sullivan current $T$.
Since $\dif T = 0$, we may assume, possibly after replacing $M$ with its universal cover, that $T$ is exact, say $T = \dif u$, and we just need to show that $u$ is $1$-harmonic.
If this is not true, then we can choose an open set $E \subseteq M$ with $C^\infty$ boundary and a function $v \in BV_\cpt(E)$ such that
$$\int_E \star |\dif u + \dif v| < \int_E \star |\dif u| < \infty.$$
Since $v$ has compact support, there exists a collar neighborhood $F \subseteq E$ of $\partial E$ such that for every $y \in \RR$,
\begin{equation}\label{collar condition}
	\partial \{u > y\} \cap F = \partial^* \{u + v > y\} \cap F.
\end{equation}
Since $\lambda$ is minimal, the coarea formula (see \cite[Proposition 2.5]{BackusFLG} for a proof at this regularity) implies that $1_{\{u > y\}}$ is a critical point of the total variation, hence is $1$-harmonic.
By (\ref{collar condition}), $1_{\{u + v > y\}} - 1_{\{u > y\}}$ has compact support in $E$, so by the coarea formula,
$$|\partial \{u > y\} \cap E| = \int_E \star |\dif 1_{\{u > y\}}| \leq \int_E \star |\dif 1_{\{u + v > y\}}| = |\partial^* \{u + v > y\} \cap E|.$$
Another application of the coarea formula gives
\begin{align*}
\int_E \star |\dif u| &= \int_{-\infty}^\infty |\partial \{u > y\} \cap E| \dif y \leq \int_{-\infty}^\infty |\partial^* \{u + v > y\} \cap E| \dif y \\
&= \int_E \star |\dif u + \dif v| < \int_E \star |\dif u|
\end{align*}
which is a contradiction.




%%%%%%%%%%%%%%%%%%%%%%%%%%%%%%%
% \appendix \section{Transverse cocycles}\label{transverse curves}
% In this appendix, we show that convergence in the weak topology of measures as we have stated it (that is, the weak topology on the space of Ruelle-Sullivan currents) is equivalent to the formulation of Thurston \cite[\S8.6]{thurston1979geometry} that is more familiar to geometric topologists.

% Let $\lambda$ be an oriented minimal lamination.
% By Theorem \ref{regularity theorem}, $\lambda$ has a global Lipschitz normal vector field $\normal_\lambda$ and is tangentially $C^\infty$.
% We shall assume that $\supp \lambda$ is a Lebesgue null set.
% This assumption is harmless, because in the application of interest to geometric topologists, $M$ is a closed hyperbolic surface, and then by the Gauss-Bonnet formula it is indeed true that $\supp \lambda$ is null \cite[\S8.5]{thurston1979geometry}.

% We now review preliminaries that were established in \cite[\S7.2]{daskalopoulos2020transverse}.
% A curve $\gamma: I \to M$ is said to be \dfn{positively transverse} to $\lambda$ if $\langle \gamma', \normal_\lambda \rangle > 0$ on $\supp \lambda$; by taking pushforwards by the inverse of a laminar flow box, this is equivalent to the condition of \cite[Definition 7.7]{daskalopoulos2020transverse}.
% We define negatively transverse curves similarly.
% The transverse curve $\gamma$ is \dfn{admissibly transverse} if, in addition, the endpoints of $\gamma$ lie in $M \setminus \supp \lambda$.

% % A sum of admissibly transverse curves is known as a \dfn{transverse $1$-chain}.
% % We write $C_1(M, \lambda)$ for the group of transverse $1$-chains, modulo transverse homotopies to $\lambda$.
% % A \dfn{transverse $1$-cocycle} is a representation $C_1(M, \lambda) \to \RR$ \cite[Definition 7.12]{daskalopoulos2020transverse}.

% % If $\mu$ is a transverse measure, then we can define a transverse $1$-cocycle, which we also denote by $\mu$, as follows.
% % By subdividing $\gamma \in C_1(M, \lambda)$, we may assume that $\gamma$ is a positively transverse curve in the image $U_\alpha$ of a flow box $F_\alpha$.
% % The projection of $(F_\alpha^{-1})_* \gamma$ to $I \subset K_\alpha$ is strictly increasing since $\gamma$ is positively transverse, so $\mu_\alpha$ induces a measure $\tilde \mu$ on $\gamma(I)$. We then define
% % \begin{equation}\label{definition of cocycle}
% % 	\mu(\gamma) := \int_{\gamma(I)} \langle \gamma', \normal_\lambda \rangle \dif \tilde \mu.
% % \end{equation}
% % This formula defines a positively transverse cocycle, and every positively transverse cocycle arises in this way; these facts are easy modifications of the discussion in \cite[\S7.2]{daskalopoulos2020transverse}.

% If $\gamma: I \to M$ is an admissibly transverse curve, and $\mu$ is a transverse measure, then we define a measure $\gamma^! \mu$, the \dfn{exceptional pullback}\footnote{One can only push forward measures in general.} of $\mu$, on $I$.
% First, we may assume that $\gamma$ is an alternating sum of admissibly positively transverse curves, and then by restricting to any such summand we may assume that $\gamma$ is admissibly positively transverse.
% Such alternating sums are known as \dfn{good subdivisions} and constructed in \cite[Lemma 7.9]{daskalopoulos2020transverse}.

% If $\gamma$ is positively admissibly transverse, we consider all decompositions $\gamma = \sum_i \gamma_i$ where $\gamma_i$ is positively admissibly transverse and has domain $I_i = [t_i, t_{i + 1}] \subseteq I$.
% Then, in a neighborhood of $\gamma_i$, $T_\mu$ is exact, say $T_\mu = \dif u$. Then we may set 
% $$\gamma^! \mu(I_i) := u(\gamma(t_{i + 1})) - u(\gamma(t_i)).$$
% Since $\supp \lambda$ is Lebesgue null, the set of $t$ such that $u(\gamma(t))$ is ill-defined is also Lebesgue null, and $u$ is a nondecreasing function since $\gamma$ is positively admissibly transverse.
% So $\gamma^! \mu$ is a well-defined Radon measure on $I$ whose (distributional) Radon-Nikod\'ym derivative is $\dif(\gamma^* u)$ (which itself is well-defined since $u$ is nondecreasing and hence $BV(I)$).
% We define $\gamma^! \mu$ for inadmissible transverse curves $\gamma$ by extending $\gamma$ slightly to a transverse curve and then taking intersections over extensions and applying continuity from above.

% Taking exceptional pullbacks, we see that we may view $\mu$ as the data of a Radon measure on every transverse curve satisfying a compatibility condition; this is the definition of transverse measure which appears in the definition of measure convergence in \cite[\S8.6]{thurston1979geometry}.


% \begin{proposition}\label{characterization of measure convergence}
% 	Let $(\lambda_n, \mu_n)$ and $(\lambda, \mu)$ be oriented measured minimal laminations with Lebesgue null supports. Then $(\lambda_n, \mu_n) \to (\lambda, \mu)$ iff for every positively transverse curve $\gamma$ to $\lambda$ defined in a small neighborhood of $\supp \lambda$:
% \begin{enumerate}
% \item $\gamma$ is eventually positively transverse to $\lambda_n$, and 
% \item $\gamma^! \mu_n \to \gamma^! \mu$ in the weak topology of measures.
% \end{enumerate}
% \end{proposition}
% \begin{proof}
% 	First assume that $(\lambda_n, \mu_n) \to (\lambda, \mu)$ and $\gamma$ is positively transverse to $\lambda$.
% 	By Lemma \ref{convergence of normals}, in a neighborhood of $\supp \lambda$, we have $\langle \normal_{\lambda_n}, \gamma' \rangle > 0$ for $n$ large, where $n$ can be chosen uniformly on $M$ since the image of $\gamma$ is compact. So $\gamma$ is eventually positively transverse to $\lambda_n$.

% 	By working locally we may assume that $T_\mu = \dif u$ and $T_{\mu_n} = \dif u_n$, and by possibly extending $\gamma$ if necessary we may assume that it is admissibly transverse.
% 	Since $\dif u_n \to \dif u$ in the weak topology of measures, the portmanteau theorem implies that for any curve $\rho$ from $x$ to $y$ such that $u$ is continuous near $x, y$, $u_n(y) - u_n(x) \to u(y) - u(x)$.
% 	Taking $\rho$ to range over admissible subcurves of $\gamma$, and applying the portmanteau theorem again, we conclude that $\dif(\gamma^* u_n) \to \dif(\gamma^* u)$ and hence $\gamma^! \mu_n \to \gamma^! \mu$.

% 	Conversely, if $\gamma^! \mu_n \to \gamma^! \mu$ for every positively transverse $\gamma$, then by the existence of flow box coordinates for $\lambda$ (given by Proposition \ref{regularity theorem}), we can actually foliate a neighborhood of any point $x$ of $\supp \lambda$ by admissibly positively transverse curves.
% 	These curves depend continuously on their intersection point with the leaf of $\supp \lambda$ containing $x$, so the fact that one of them is positively transverse to $\lambda_n$ for $n \geq N/2$ and some even integer $N$ implies that in a neighborhood of $x$, every curve in the foliation is positively transverse to $\lambda_n$ for $n \geq N$.
% 	Let $J$ be the set of such curves, equipped with the measure $\nu$ obtained by disintegrating the Riemannian measure into arc length measures $s_\gamma$ on each $\gamma$.
	
% 	If $\varphi$ is a continuous $d-1$-form with support near $x$, we may define the restriction $f_\gamma$ of $\varphi \wedge (\gamma')^\flat$ to $\gamma$.
% 	By the disintegration theorem,
% 	$$\int_M T_\mu \wedge \varphi = \int_J \int_I f_\gamma \frac{\dif(\gamma^! \mu)}{\dif s_\gamma} \dif s_\gamma \dif \nu(\gamma) = \int_J \int_I f_\gamma \dif(\gamma^! \mu) \dif \nu(\gamma).$$
% 	One can show using the dominated convergence theorem and the fact that $\gamma^! \mu_n \to \gamma^! \mu$ weakly that 
% 	$$\int_M T_\mu \wedge \varphi = \lim_{n \to \infty} \int_J \int_I f_\gamma \dif(\gamma^! \mu_n) \dif \nu(\gamma) = \lim_{n \to \infty} \int_M T_{\mu_n} \wedge \varphi$$
% 	but since $\varphi$ is arbitrary, it holds that $T_{\mu_n} \to T_\mu$ in the weak topology of measures.
% \end{proof}


\printbibliography

\end{document}
