\documentclass[reqno,11pt]{amsart}
\usepackage[letterpaper, margin=1in]{geometry}
\RequirePackage{amsmath,amssymb,amsthm,graphicx,mathrsfs,url,slashed,subcaption}
\RequirePackage[usenames,dvipsnames]{xcolor}
\RequirePackage[colorlinks=true,linkcolor=Red,citecolor=Green]{hyperref}
\RequirePackage{amsxtra}
\usepackage{cancel}
\usepackage{tikz-cd}

% \setlength{\textheight}{9.3in} \setlength{\oddsidemargin}{-0.25in}
% \setlength{\evensidemargin}{-0.25in} \setlength{\textwidth}{7in}
% \setlength{\topmargin}{-0.25in} \setlength{\headheight}{0.18in}
% \setlength{\marginparwidth}{1.0in}
% \setlength{\abovedisplayskip}{0.2in}
% \setlength{\belowdisplayskip}{0.2in}
% \setlength{\parskip}{0.05in}
%\renewcommand{\baselinestretch}{1.05}

\title{Spaces of minimal laminations and an application to 1-harmonic functions}
\author{Aidan Backus}
\date{October 2022}

\newcommand{\NN}{\mathbf{N}}
\newcommand{\ZZ}{\mathbf{Z}}
\newcommand{\QQ}{\mathbf{Q}}
\newcommand{\RR}{\mathbf{R}}
\newcommand{\CC}{\mathbf{C}}
\newcommand{\DD}{\mathbf{D}}
\newcommand{\PP}{\mathbf P}
\newcommand{\MM}{\mathbf M}
\newcommand{\II}{\mathbf I}
\newcommand{\Hyp}{\mathbf H}
\newcommand{\Sph}{\mathbf S}
\newcommand{\Group}{\mathbf G}
\newcommand{\GL}{\mathbf{GL}}
\newcommand{\Orth}{\mathbf{O}}
\newcommand{\SpOrth}{\mathbf{SO}}
\newcommand{\Ball}{\mathbf{B}}

\newcommand*\dif{\mathop{}\!\mathrm{d}}

\DeclareMathOperator{\dist}{dist}
\DeclareMathOperator{\MeasLam}{MeasLam}
\DeclareMathOperator{\MinLam}{MinLam}
\DeclareMathOperator{\Lam}{Lam}
\DeclareMathOperator{\supp}{supp}

\newcommand{\Leaves}{\mathscr L}
\newcommand{\Hypspace}{\mathscr H}

\newcommand{\Two}{\mathrm{I\!I}}


\newcommand{\Hilb}{\mathcal H}
\newcommand{\Homology}{\mathrm H}
\newcommand{\normal}{\mathbf n}
\newcommand{\radial}{\mathbf r}
\newcommand{\evect}{\mathbf e}
\newcommand{\vol}{\mathrm{vol}}

\newcommand{\diam}{\mathrm{diam}}
\newcommand{\inj}{\mathrm{inj}}
\newcommand{\Lip}{\mathrm{Lip}}
\newcommand{\Riem}{\mathrm{Riem}}

\newcommand{\Bmu}{\boldsymbol \mu}
\newcommand{\Bnu}{\boldsymbol \nu}
\newcommand{\Blambda}{\boldsymbol \lambda}

\newcommand{\pic}{\vspace{30mm}}
\newcommand{\dfn}[1]{\emph{#1}\index{#1}}

\renewcommand{\Re}{\operatorname{Re}}
\renewcommand{\Im}{\operatorname{Im}}

\newcommand{\loc}{\mathrm{loc}}
\newcommand{\cpt}{\mathrm{cpt}}

\def\Japan#1{\left \langle #1 \right \rangle}

\newtheorem{theorem}{Theorem}[section]
\newtheorem{badtheorem}[theorem]{``Theorem"}
\newtheorem{prop}[theorem]{Proposition}
\newtheorem{lemma}[theorem]{Lemma}
\newtheorem{sublemma}[theorem]{Sublemma}
\newtheorem{proposition}[theorem]{Proposition}
\newtheorem{corollary}[theorem]{Corollary}
\newtheorem{conjecture}[theorem]{Conjecture}
\newtheorem{axiom}[theorem]{Axiom}
\newtheorem{assumption}[theorem]{Assumption}

\newtheorem{mainthm}{Theorem}
\renewcommand{\themainthm}{\Alph{mainthm}}

% \newtheorem{claim}{Claim}[theorem]
% \renewcommand{\theclaim}{\thetheorem\Alph{claim}}
\newtheorem*{claim}{Claim}

\theoremstyle{definition}
\newtheorem{definition}[theorem]{Definition}
\newtheorem{remark}[theorem]{Remark}
\newtheorem{example}[theorem]{Example}
\newtheorem{notation}[theorem]{Notation}

\newtheorem{exercise}[theorem]{Discussion topic}
\newtheorem{homework}[theorem]{Homework}
\newtheorem{problem}[theorem]{Problem}

\makeatletter
\newcommand{\proofpart}[2]{%
  \par
  \addvspace{\medskipamount}%
  \noindent\emph{Part #1: #2.}
}
\makeatother



\numberwithin{equation}{section}


% Mean
\def\Xint#1{\mathchoice
{\XXint\displaystyle\textstyle{#1}}%
{\XXint\textstyle\scriptstyle{#1}}%
{\XXint\scriptstyle\scriptscriptstyle{#1}}%
{\XXint\scriptscriptstyle\scriptscriptstyle{#1}}%
\!\int}
\def\XXint#1#2#3{{\setbox0=\hbox{$#1{#2#3}{\int}$ }
\vcenter{\hbox{$#2#3$ }}\kern-.6\wd0}}
\def\ddashint{\Xint=}
\def\dashint{\Xint-}

\usepackage[backend=bibtex,style=alphabetic,giveninits=true]{biblatex}
\renewcommand*{\bibfont}{\normalfont\footnotesize}
\addbibresource{topics.bib}
\renewbibmacro{in:}{}
\DeclareFieldFormat{pages}{#1}

\newcommand\todo[1]{\textcolor{red}{TODO: #1}}


\begin{document}
\begin{abstract}
We collect several results governing the various modes of convergence for sequences of minimal laminations.
We then apply this theory to prove that a function is 1-harmonic iff it is the Ruelle-Sullivan current of a minimal lamination; this resolves an open problem of Daskalopoulos--Uhlenbeck.
\end{abstract}

\maketitle

%%%%%%%%%%%%%%%%%%%%%%%%%%%%%%%%%%%%%%%%%%%%%%%%%%%%%%%

% \tableofcontents

\section{Introduction}
The space of codimension-$1$ minimal laminations on a Riemannian manifold has been topologized in several different ways.
Thurston \cite[Chapter 8]{thurston1979geometry} introduced both his geometric topology as well as the weak topology of measures on the space of measured geodesic laminations.
Independently of Thurston, Colding--Minicozzi \cite[Appendix B]{ColdingMinicozziIV} introduced a topology that emphasized not the laminations themselves, but rather the coordinate charts which flatten them.
We shall explain how these three modes of convergence are related, as well as the regularity and compactness theorems associated to each such mode.

We then turn to the main goal of this series of papers, which also includes the prequel paper \cite{BackusFLG}.
We show that any $1$-harmonic function gives rise to a Ruelle-Sullivan current for a minimal lamination, and conversely.
This generalizes a theorem of Daskalopoulos--Uhlenbeck \cite[Theorem 6.1]{daskalopoulos2020transverse} and resolves the outstanding problems \cite[Problem 9.4]{daskalopoulos2020transverse} and \cite[Conjecture 9.5]{daskalopoulos2020transverse}.

%%%%%%%%%%%%%%%%%
\subsection{Minimal laminations}\label{Lams sections}
Let $I \subseteq \RR$ be an interval, and $M$ a Riemannian manifold of dimension $d \geq 2$.
A (codimension-$1$) \dfn{laminar flow box} is a $C^0$ coordinate chart $F: I \times \RR^{d - 1} \to M$ and a compact set $K \subseteq I$ such that each \dfn{leaf} $F(\{k\} \times \RR^{d - 1})$ is $C^2$.
A \dfn{laminar transition map} between two laminar flow boxes $(F_\alpha, K_\alpha), (F_\beta, K_\beta)$ is a $C^0$ map
$$\psi_{\alpha \beta}: I \times \RR^{d - 1} \to I \times \RR^{d - 1}$$
satisfying the usual transition relation
\begin{equation}\label{transition relation}
F_\alpha = F_\beta \circ \psi_{\alpha \beta},
\end{equation}
which maps each leaf $\{k\} \times \RR^{d - 1}$, $k \in K_\alpha$, to a leaf $\{\psi_{\alpha \beta}(k)\} \times \RR^{d - 1}$, so that $\psi_{\alpha \beta}$ is a homeomorphism $K_\alpha \to K_\beta$.
By a \dfn{laminar atlas} we shall mean an atlas for $M$, such that the coordinate charts are all laminar flow boxes and the transition maps are also laminar.

\begin{definition}
A \dfn{lamination} $\lambda$ consists of a nonempty closed set $S \subseteq M$, called its \dfn{support}, and a maximal laminar atlas $\{(F_\alpha, K_\alpha): \alpha \in A\}$ such that in the image $U_\alpha$ of each flow box $F_\alpha$,
$$S \cap U_\alpha = F_\alpha(K_\alpha \times \RR^{d - 1}).$$
If $\lambda$ is a lamination in the image of a flow box $F$, and $N := F(\{k\} \times \RR^{d - 1})$ is a leaf of $\lambda$, we call $k$ the \dfn{label} of $N$.
\end{definition}

Summarizing the above definitions, a lamination is a nonempty closed set $S$ with a $C^0$ local product structure which realizes it as $K \times \RR^{d - 1}$ for some compact set $K \subset \RR$.

We assume that the leaves are $C^2$ in order to ensure that the normal vectors to each leaves are well-defined in $C^1$, and in particular the second fundamental form and mean curvature of each leaf is well-defined.
Such laminations are sometimes called $C^2$ \dfn{along leaves} \cite{Morgan88}.
This is not the same thing as assuming that the lamination admits a $C^2$ atlas, as it may not be able to extend the normal vectors to each leaf to a $C^1$ vector field on $M$ even locally.
In any case, we will make more precise assertions about the existence of flow boxes of various regularities in \S\ref{Regularity}.

\begin{definition}
A lamination $\lambda$ is \dfn{minimal} if its leaves $F_\alpha(\{k\} \times \RR^{d - 1})$ have zero mean curvature, and is \dfn{geodesic} if, in addition, $d = 2$.
\end{definition}


%%%%%%%%%%%%%%%%%%
\subsection{Spaces of minimal laminations}\label{LamSpace section}
Let $M$ be a closed space form of dimension $2 \leq d \leq 7$.
In the literature, there are at least three different topologies on the space of laminations on $M$, which we now recall.

Thurston's geometric topology \cite[Chapter 8]{thurston1979geometry} says that a lamination $\lambda'$ is close to a lamination $\lambda$ if every leaf of $\lambda$ is close to a leaf of $\lambda'$ at least locally, and the same holds for their normal vectors $\normal$.

\begin{definition}
A sequence of laminations $\lambda_i$ converges to a lamination $\lambda$ in \dfn{Thurston's geometric topology} if, for every leaf $N$ of $\lambda$, every $x \in N$, and every $\varepsilon > 0$, there exists $i_\varepsilon \in \NN$ such that for every $i \geq i_\varepsilon$, $\supp \lambda_i$ intersects $B(x, \varepsilon)$, and for $x_i \in B(x, \varepsilon) \cap \supp \lambda_i$,
$$\dist_{SM}(\normal_{\lambda_i}(x_i), \normal_\lambda(x)) < 2\varepsilon.$$
\end{definition}

It is straightforward to show that Thurston's geometric topology does not depend on the choice of Riemannian metric on $M$, or the choice of extension of the distance function on $M$ to its sphere bundle $SM$, which are implicit in the statement thereof.
However, the limiting lamination is not unique, as if $\lambda_i \to \lambda$ and $\lambda'$ is a sublamination of $\lambda$, then $\lambda_i \to \lambda'$.
In particular, Thurston's topology is not Hausdorff, and we say that $\lambda$ is a \dfn{maximal limit} of a sequence $(\lambda_i)$ if $\lambda_i \to \lambda$ and for every $\lambda'$ such that $\lambda_i \to \lambda'$, $\lambda'$ is a sublamination of $\lambda$.

Independently of Thurston, Colding--Minicozzi \cite[Appendix B]{ColdingMinicozziIV} defined a sequence of laminations to converge ``if the corresponding coordinate maps converge;'' that is, if the laminar atlases converge.
This of course says nothing about the limiting set of leaves and in the sequel paper \cite{ColdingMinicozziV} they additionally impose that the sets of leaves converge ``as sets.''

In this paper we consider a similar condition to the one in \cite{ColdingMinicozziV}, which we believe to be more natural: that the laminar atlases converge and that the laminations themselves converge in Thurston's geometric topology.
To be more precise:

\begin{definition}
A sequence $(\lambda_i)$ of laminations \dfn{flow-box converges} in a function space $X$ to $\lambda$ if it converges in Thurston's geometric topology, and there exists a laminar atlas $(F_\alpha)$ for $\lambda$ such that for each $\alpha$, $F_\alpha$ and $(F_\alpha)^{-1}$ are limits in $X$ of flow boxes $F_\alpha^i$, $(F_\alpha^i)^{-1}$ in laminar atlases for $\lambda_i$.
\end{definition}

We now define convergence of laminations equipped with transverse measures.

\begin{definition}
Let $\lambda$ be a lamination with atlas $A$.
A \dfn{transverse measure} to $\lambda$ consists of Radon measures $\mu_\alpha$ with $\supp \mu_\alpha = K_\alpha$, $\alpha \in A$, such that each transition map $\psi_{\alpha \beta}$ is measure-preserving:
$$\mu_\alpha|_{K_\alpha \cap K_\beta} = \psi_{\alpha \beta}^* (\mu_\beta|_{K_\alpha \cap K_\beta}).$$
The pair $(\lambda, \mu)$ is called a \dfn{measured lamination}.
\end{definition}

Caveat lector: we assume that $\supp \mu_\alpha = K_\alpha$, but in \cite{daskalopoulos2020transverse}, it is only assumed that $\supp \mu_\alpha \subseteq K_\alpha$.
In particular, not every lamination admits a transverse measure.

The definition of transverse measure in terms of Radon measures on $K_\alpha$ is convenient because $K_\alpha$ is compact.
However, the definition is not intrinsic, and this causes problems when considering questions of convergence: the fact that the flow boxes of a convergent sequence of measured laminations converge should be a consequence of, not a part of, the definition!

To rectify this, we first observe that in the definition of a transverse measure, we cannot define a transverse measure to be one on the underlying manifold $M$ itself.
Indeed, Lebesgue measure is ``transverse'' to all foliations; thus such a definition forgets the ``direction'' the measure points in.
However, the notion of Ruelle-Sullivan current allows us to speak of a measure-theoretic object on $M$ which has a well-defined local product structure.

\begin{definition}
A lamination is \dfn{oriented} if one can choose its transition maps to all be orientation-preserving.
\end{definition}

\begin{definition}
Let $(\lambda, \mu)$ be a measured oriented lamination and let $(\chi_\alpha)_{\alpha \in A}$ be a subordinate partition of unity.
The \dfn{Ruelle-Sullivan current} $T_\mu$ associated to $(\lambda, \mu)$ is defined for all compactly supported $d-1$-forms $\varphi$ by
\begin{equation}\label{RS current}
\int_M T_\mu \wedge \varphi := \sum_{\alpha \in A} \int_{K_\alpha} \left[\int_{\{k\} \times \RR^{d - 1}} (F_\alpha^{-1})^* (\chi_\alpha \varphi) \right] \dif \mu_\alpha(k).
\end{equation}
\end{definition}

It is clear that any lamination is locally orientable, so the next definition makes sense.

\begin{definition}
A sequence of measured laminations $(\lambda_i, \mu_i)$ \dfn{converges} to $(\lambda, \mu)$ if locally, their Ruelle-Sullivan currents $T_{\mu_i} \to T_\mu$ converge in the weak topology of measures.
\end{definition}

The convergence of Ruelle-Sullivan currents, which is very convenient to work with analytically, is equivalent to a definition of measure convergence that may be more familiar to topologists, namely convergence of the transverse measure along each transverse curve, c.f. Appendix \ref{transverse curves}.

It is clear from the definitions that flow-box convergence implies Thurston convergence, and it is well-known that measure convergence implies Thurston convergence \cite[Proposition 8.10.3]{thurston1979geometry}.
We show that flow-box convergence actually sits in the middle of the chain of implications.

We write $C^{1-}$ for the Fr\'echet space $\bigcap_{\alpha < 1} C^\alpha$, where $C^\alpha$ are H\"older spaces.

\begin{definition}
A sequence $(\lambda_n)$ of laminations has \dfn{bounded curvature} if there exists $C > 0$ such that for any $n$ and any leaf $N$ of $\lambda_n$, the second fundamental form satisfies $\|\Two_N\|_{C^0} \leq C$.
\end{definition}

\begin{theorem}\label{implication theorem}
Let $M$ be a manifold of constant sectional curvature and dimension $2 \leq d \leq 7$.
Let $(\lambda_n, \mu_n)$ be measured minimal laminations in $M$, and $(\lambda_n, \mu_n) \to (\lambda, \mu)$.
Then:
\begin{enumerate}
	\item $\lambda_n \to \lambda$ in Thurston's geometric topology.
	\item If $(\lambda_n)$ has bounded curvature, then $\lambda_n \to \lambda$ in the $C^{1-}$ flow box topology.
\end{enumerate}
\end{theorem}

We also complement Theorem \ref{implication theorem} with some compactness results for the above modes of convergence.

\begin{theorem}\label{compactness theorem}
Let $M$ be a manifold of dimension $2 \leq d \leq 7$.
Let $(\lambda_n)$ be a sequence of minimal laminations of bounded curvature, and assume that for some compact set $E \Subset M$ and every leaf $N$ of $\lambda_n$, $N \cap E$ is nonempty. Then:
\begin{enumerate}
\item A subsequence converges as flow boxes in $C^{1-}$, and in particular in Thurston's topology, to a minimal lamination.
\item If $\mu_n$ is transverse to $\lambda_n$ and there exists $C > 0$ such that $\mu_n(M) \leq C$, then a further subsequence converges in the measure topology.
\end{enumerate}
\end{theorem}



%%%%%%%%%%%%%%%%%%
\subsection{Best Lipschitz and least gradient maps}\label{FLG section}
Geodesic laminations are of great interest to the Thurston school of geometric topology \cite[Chapter 8]{thurston1979geometry}.
Later Thurston introduced \dfn{best Lipschitz maps}, namely maps $v: M \to N$ between closed manifolds which minimize their Lipschitz constant $\Lip(v)$ subject to a constraint on their homotopy class.
These maps define a geodesic lamination whose support is the set of points $x$ so that the local Lipschitz constant of $v$ at $x$ is equal to $\Lip(v)$ \cite{thurston1998minimal}.
If $M, N$ are hyperbolic surfaces of the same genus $g$, then $\Lip(v)$ is the distance between $M$ and $N$ in \dfn{Thurston's asymmetric metric} on Teichm\"uller space.
This circle of ideas has been developed by the Thurston school \cite{papadopoulos:hal-00129729} but has recently also made contact with geometric PDE through the work of Daskalopoulos--Uhlenbeck \cite{daskalopoulos2020transverse,daskalopoulosPrep1,DaskalopoulosPrep2}.

To be more precise, if $M$ is a closed hyperbolic manifold, then the Euler-Lagrange equation for best Lipschitz maps $v: M \to \Sph^1$ is the $\infty$-Laplace equation \cite{daskalopoulos2020transverse}
\begin{equation}\label{infinity laplacian}
(\nabla^\mu \partial^\nu v) \partial_\mu v \partial_\nu v = 0.
\end{equation}
This equation is invariant under translations $v \mapsto v + y$, so by Noether's theorem, it has a conserved flux $\dif u$.
If $d = 2$, the associated conservation law is the $1$-Laplace equation.
We studied the $1$-Laplacian in the prequel paper \cite{BackusFLG}; here we recall the main result of that paper.

\begin{definition}
A function $u \in BV_\loc(M)$ has \dfn{least gradient}, or is \dfn{$1$-harmonic}, if for every $w \in BV_\cpt(M)$,
\begin{equation}\label{least gradient functional}
\int_M \star |\dif u| \leq \int_M \star |\dif u + \dif w|.
\end{equation}
\end{definition}

Here $\star |\dif u|$ is the total variation of the current $\dif u$; we refer to \S\ref{MeasurePrelims} for the precise definition.
The formal Euler-Lagrange equation for (\ref{least gradient functional}) is the $1$-Laplace equation
\begin{equation}\label{1Laplacian}
\dif^* \left(\frac{\dif u}{|\dif u|}\right) = 0.
\end{equation}
Formally, the $1$-Laplace equation (\ref{1Laplacian}) asserts that the level sets are indeed minimal, but since the derivation of the Euler-Lagrange equation for (\ref{least gradient functional}) is only formal, and the precise definition of weak solution \cite{Mazon14} does not directly imply that the level sets have zero mean curvature, this has to be checked separately.
This is the main result of the prequel paper \cite{BackusFLG}:

\begin{theorem}\label{main thm of old paper}
Let $M$ be a manifold of constant sectional curvature and $2 \leq d \leq 7$.
Then for every $1$-harmonic function $u: M \to \RR$ and $y \in \RR$, the level set $\partial \{u > y\}$ is an analytic embedded stable minimal hypersurface in $M$.
\end{theorem}
\begin{proof}[Proof sketch]
By a straightfoward modification of \cite[Theorem 1]{BOMBIERI1969}, the superlevel sets $\{u > y\}$ have least perimeter, that is their indicator functions have least gradient.
The regularity of boundaries of sets of least perimeter was established for $M = \RR^d$ by the classical work of de Giorgi and Miranda \cite{deGiorgi61, Miranda66} but their proof does not generalize nicely because it relies on the invariance of tangent vectors under parallel transport in order to define averages of normal vectors to sets of least perimeter.
In \cite[\S3]{BackusFLG} we establish a suitable method of taking averages of the normal vector provided that $M = \Hyp^d$ or $M = \Sph^d$.
This allows us to modify the relevant parts of \cite{Miranda66}.
See \cite[\S1]{BackusFLG} for a more detailed summary of \cite{BackusFLG}.
\end{proof}

We use Theorem \ref{main thm of old paper} as a regularity result at various points in this paper, which allows us to show that the limiting laminations have smooth leaves (rather than currents for leaves, say).
However, Theorem \ref{main thm of old paper} is of more interest to this paper rather than just as a regularity lemma; to motivate why, we recall the relationship between best Lipschitz maps, maps of least gradient, and geodesic laminations:

\begin{theorem}[Daskalopoulos--Uhlenbeck]\label{DU theorem}
Let $M$ be a closed hyperbolic surface and $v: M \to \Sph^1$ an $\infty$-harmonic function with maximal stretch geodesic lamination $\lambda$ and conserved flux $T$. Then:
\begin{enumerate}
\item $T$ is the Ruelle-Sullivan current for a measured oriented structure on $\lambda$.
\item There exists a $1$-harmonic, $\pi_1(M)$-equivariant function $u: \Hyp^2 \to \RR$ such that $\dif u$ is a lift of $T$ to the universal cover $\Hyp^2$.
\item Every level set of $u$ is the lift of a geodesic of $\lambda$ to $\Hyp^2$.
\end{enumerate}
\end{theorem}

We refer to the original paper \cite{daskalopoulos2020transverse} for a more precise statement.
Inspired by this theorem, Daskalopoulos--Uhlenbeck conjectured that for any $1$-harmonic function on $\Hyp^2$, $\dif u$ should be Ruelle-Sullivan for some (possibly not maximum-stretch) geodesic lamination \cite[Problem 9.4]{daskalopoulos2020transverse}, and conversely that if $T$ is a Ruelle-Sullivan current for some geodesic lamination, then local primitives of $T$ are $1$-harmonic \cite[Conjecture 9.5]{daskalopoulos2020transverse}.
Of course, if $d \geq 3$, then the level sets will be minimal hypersurfaces rather than geodesics.

Using Theorem \ref{compactness theorem}, the stable Bernstein theorem \cite{Schoen2016, Chodosh2021}, and Theorem \ref{main thm of old paper}, we prove the conjectures of Daskalopoulos--Uhlenbeck:

\begin{theorem}\label{main thm}
Let $M$ be a manifold of constant sectional curvature and dimension $2 \leq d \leq 4$.
\begin{enumerate}
\item Let $u$ be a $1$-harmonic function on $M$.
Then:
\begin{enumerate}
\item $\bigcup_{y \in \RR} \partial \{u > y\}$ is the support of a minimal lamination $\lambda$.
\item The leaves of $\lambda$ are the connected components of the level sets $\partial \{u > y\}$.
\item There is a measured oriented structure on $\lambda$ whose Ruelle-Sullivan current is $\dif u$.
\end{enumerate}
\item Conversely, if $\lambda$ is a minimal measured oriented lamination with Ruelle-Sullivan current $T$, and $\tilde M \to M$ is the universal cover of $M$, then there exists a $1$-harmonic, $\pi_1(M)$-equivariant function $u: \tilde M \to \RR$, such that $\dif u$ drops to $T$.
\end{enumerate}
\end{theorem}

However, Theorem \ref{main thm} leaves a key point -- the role of the $\infty$-Laplacian -- open, and we have not attempted to address this point here.
The Daskalopoulos--Uhlenbeck theorem was our main motivation for this series of papers; however, our work works in dimensions $d = 3, 4$ while Theorem \ref{DU theorem} is purely a statement about $d = 2$.
Indeed, the $\infty$-Laplacian gives geodesic laminations, but the $1$-Laplacian gives codimension-$1$ laminations, which are not geodesic if $d \geq 3$, so there is not an obvious generalization of Theorem \ref{DU theorem} for $d = 3, 4$.

It is tantalizing to think that a suitable system of coupled $\infty$-Laplacians will satisfy a generalized maximum-stretch condition on a codimension-$1$ minimal lamination $\lambda$, and that the conservation law for this conjectural system will be the $1$-Laplacian.
However, even if one was to derive such a system of $\infty$-Laplacians formally, the analysis for studying such a system would likely not be in place.
Indeed, the study of $\infty$-harmonic functions is almost entirely carried out in the language of viscosity solutions and comparison-with-cones, which only make sense when the target is $\RR$ rather than $\RR^m$ or a vector bundle.
Even so, we hope to return to this question in a later work: what is the correct generalization of the Daskalopoulos--Uhlenbeck theorem to $d = 3$?

%%%%%%%%%%%%%%%%%%%%%%%
\subsection{Outline of the paper}
In \S\ref{Prelims} we recall preliminaries.

In \S\ref{Regularity} we establish that every $C^0$ minimal lamination has a Lipschitz laminar atlas and a Lipschitz normal bundle.
We will use this regularity frequently through the remainder of the paper.

In \S\ref{CompactnessSec} we prove Theorem \ref{compactness theorem} and \ref{main thm}.

%%%%%%%%%%%%%%%%%%%%%%%%

\subsection{Acknowledgements}
I would like to thank Georgios Daskalopoulos for suggesting this project and for many helpful discussions.
I would also like to thank Chao Li for help understanding \cite{Chodosh2021}.

This research was supported by the National Science Foundation's Graduate Research Fellowship Program under Grant No. DGE-2040433.



%%%%%%%%%%%%%%%%%%%%%%%%%%%

\section{Preliminaries}\label{Prelims}
\subsection{Notation and conventions}
The operator $\star$ is the Hodge star, thus $\star 1$ is the Riemannian measure.
If $U$ is an open set, we write $|U| := \int_U \star 1$ for the volume of $U$, but if $U$ is a submanifold or rectifiable set of positive codimension, we instead write $|U|$ for its surface measure.

For a map $F: X \to Y$ between metric spaces, we write $\Lip(F)$ for its Lipschitz constant.
If $X, Y$ are connected Riemannian manifolds, one of which is $1$-dimensional, then we have $\Lip(F) = \|\dif F\|_{L^\infty}$.

We write $\normal_N$ for the normal vector (or conormal $1$-form) for a hypersurface $N$, $\nabla_N$ for the Levi-Civita connection, and $\Two_N := \nabla_N \normal_N$ for the second fundamental form.

We let $\Leaves \lambda$ denote the set of leaves of a lamination $\lambda$.

%%%%%%%%%%%%%%%%
\subsection{Hausdorff distance}
In order to measure when two leaves are ``close'', we shall consider the Hausdorff distance on the space of leaves, as defined in \cite[Chapter 4]{nadler2017continuum}.

\begin{definition}
Let $X$ be a compact metric space. The \dfn{Hausdorff distance} between two nonempty closed sets $A, B \subset X$ is
\begin{equation}\label{compact hausdorff distance}
	\dist(A, B) := \max\left(\max_{a \in A} \min_{b \in B} \dist(a, b), \max_{b \in B} \min_{a \in A} \dist(a, b)\right).
\end{equation}
The space of closed subsets of $X$ is the \dfn{hyperspace} $\Hypspace X$.
\end{definition}

\begin{definition}
If $X$ is a metrizable space and $(A_n)$ is a sequence of closed subsets of $X$, we define
\begin{equation}\label{Hausdorff is the limit set}
\lim_{n \to \infty} A_n := \left\{\lim_{n \to \infty} x_n: (x_n) \in \prod_{n=1}^\infty A_n\right\}.
\end{equation}
\end{definition}

\begin{proposition}\label{Hausdorff on a CMS}
Let $X$ be a compact metric space. Then:
\begin{enumerate}
\item The topology on $\Hypspace X$ is the topology for the convergence mode (\ref{Hausdorff is the limit set}).
\item The topology on $\Hypspace X$ is completely determined by the topology on $X$.
\item $\Hypspace X$ is a compact metric space.
\end{enumerate}
In particular, $\Hypspace$ is a self-map of the set of compact metrizable spaces.
\end{proposition}
\begin{proof}
By \cite[Theorem 4.11]{nadler2017continuum}, a sequence $(A_n)$ of closed subsets of $X$, the limit $A$ in $\Hypspace X$ of $A_n$ is the set of $x \in X$ such that for every $U \ni x$ open, there are $x_n \in A_n$ such that eventually $x_n \in U$.
But this is exactly the characterization of $\lim_n A_n$ given by (\ref{Hausdorff is the limit set}).
It is also independent of the metric on $X$, so as a metrizable space, $\Hypspace X$ is determined by the metrizable (rather than metric) structure on $X$.
The compactness now follows from \cite[Theorem 4.17]{nadler2017continuum}, and the fact that $\Hypspace$ is a self-map of the space of compact metrizable spaces follows.
\end{proof}

%%%%%%%%%%%%%%%%%%%%%
\subsection{Measure theory and functional analysis}\label{MeasurePrelims}
Let $X$ be a metrizable space, and let $C_c(X)$ be the space of compactly supported continuous functions $f: X \to \RR$.
Its dual $C_c(X)'$ is canonically isomorphic to the space of signed Radon measures on $X$, where the bilinear pairing is given by integration.
The weak topology on $C_c(X)'$ is known as the \dfn{weak topology of measures}.
Unpacking the definitions, a sequence $(\mu_n)$ of Radon measures converges to $\mu$ in the weak topology of measures iff for every continuous function $f: X \to \RR$,
$$\lim_{n \to \infty} \int_X f \dif \mu_n = \int_X f \dif \mu.$$
We shall frequently use the following characterization of weak convergence:

\begin{proposition}[portmanteau theorem]
	Let $(\mu_n)$ be a sequence of Radon measures on a compact metrizable space $X$ with $\mu_n(X) \lesssim 1$, and let $\mu$ be a Radon measure on $X$. The following are equivalent:
\begin{enumerate}
	\item $\mu_n \to \mu$ in the weak topology of measures.
	\item $\liminf_{n \to \infty} \mu_n(X) \geq \mu(X)$ and for every closed $Y \subseteq X$, $\limsup_{n \to \infty} \mu_n(Y) \leq \mu(Y)$.
	\item $\limsup_{n \to \infty} \mu_n(X) \leq \mu(X)$ and for every open $Z \subseteq X$, $\liminf_{n \to \infty} \mu_n(Z) \geq \mu(Z)$.
	\item For every $W \subseteq X$ with $\mu(\partial W) = 0$, $\lim_{n \to \infty} \mu_n(W) = \mu(W)$.
\end{enumerate}
	If $X$ is a manifold, these conditions are equivalent to:
\begin{enumerate}
	\setcounter{enumi}{4}
	\item For every $x \in X$ and almost every $0 < \varepsilon \ll 1$, $\lim_{n \to \infty} \mu_n(B(x, \varepsilon)) = \mu(B(x, \varepsilon))$.
\end{enumerate}
\end{proposition}
\begin{proof}
	See \cite[Theorem 13.16]{klenke2013probability} for the metrizable case.
	For the manifold case, we just observe that almost every $\varepsilon > 0$ satisfies $\mu(\partial B(x, \varepsilon)) = 0$. Indeed, if not, then we can find a set of real numbers $A$ such that $0$ is a condensation point of $A$, and for every $\varepsilon \in A$, $\mu(\partial B(x, \varepsilon)) > 0$.
	In particular, for every $\delta > 0$, $A \cap (0, \delta)$ is uncountable, so
	$$\mu(B(x, \delta)) \geq \sum_{\varepsilon \in A \cap (0, \delta)} \mu(\partial B(x, \delta)) = \infty,$$
	but since $X$ is a manifold, if $\delta$ is small enough then $B(x, \delta)$ is precompact.
	This contradicts that $\mu$ is a Radon measure.
\end{proof}

There are subtleties involved in the portmanteau theorem for noncompact $X$.
However, this will never be an issue, as we shall only use it locally, in small precompact balls.

If $X = M$ is a manifold, then we can consider instead the space $C_c(M, \Omega^\ell)$ of compactly supported continuous $\ell$-forms.
An $\ell$-\dfn{current} is an element of the dual space $C_c(M, \Omega^\ell)'$.\footnote{Strictly speaking, $C_c(M, \Omega_\ell)'$ is the space of $\ell$-currents of locally finite total variation. However, we will never need to consider $\ell$-currents that do not have locally finite total variation, so we suppress this technicality.}
We denote the pairing of an $\ell$-current $T$ and an $\ell$-form $\varphi$ by $\int_M T \wedge \varphi$.
Any $d-\ell$-form $\psi$ gives rise to an $\ell$-current $T$, the \dfn{Poincar\'e dual} of $\psi$, by $\int_M T \wedge \varphi = \int_M \psi \wedge \varphi$.
In particular, the Poincar\'e dual of any function is a $d$-current.
See \cite{simon1983GMT} for more on the theory of currents.

Again we have the weak topology on the space of $\ell$-currents; we also have the \dfn{exterior derivative}
$$\int_M \dif T \wedge \psi := -\int_M T \wedge \dif \psi$$
defined for any $\ell$-current $T$ such that $\psi \mapsto \int_M T \wedge \dif \psi$ extends from $C^1_c(M, \Omega^{\ell - 1})$ to an $\ell - 1$-current.

The following version of the Arzela-Ascoli theorem is well-known, but for convenience we recall it.

\begin{proposition}[Arzela-Ascoli theorem]\label{AA Holder}
Suppose that $(u_n)$ is a sequence of Lipschitz functions with bounded Lipschitz norms on a compact metric space. Then there is a subsequence of $(u_n)$ which converges in $C^{1-}$ to a Lipschitz function.
\end{proposition}
\begin{proof}
By the classical Arzela-Ascoli theorem, along a subsequence we have $u_n \to u$ in $C^0$, where
$$|u(x) - u(y)| \leq \liminf_{n \to \infty} |u_n(x) - u_n(y)| + 2 \|u_n - u\|_{C^0} \leq \liminf_{n \to \infty} \Lip(u_n)$$
and hence
$$\Lip(u) \leq \liminf_{n \to \infty} \Lip(u_n) < \infty.$$
Now let $0 < \alpha < 1$, and assume without loss of generality that $u = 0$, so that we must show that $\|u_n\|_{C^\alpha} \to 0$.
In fact,
$$\frac{|u_n(x) - u_n(y)|}{\dist(x, y)^\alpha} = \left|\frac{u_n(x) - u_n(y)}{\dist(x, y)}\right|^\alpha \cdot |u_n(x) - u_n(y)|^\alpha \leq \Lip(u_n)^\alpha \cdot \|u_n\|_{C^0}^{1 - \alpha}.$$
Since $u_n \to 0$ in $C^0$, and $(\Lip(u_n))$ is uniformly bounded, the claim follows.
\end{proof}

%%%%%%%%%%%%%%%%%%%%%%%%%
\subsection{Ruelle-Sullivan currents}
With the measure-theoretic machinery above in place, we recall the following facts about Ruelle-Sullivan currents, some of which already appeared in \cite[\S8]{daskalopoulos2020transverse}.

\begin{lemma}
Let $(\lambda, \mu)$ be a measured oriented lamination.
Then the Ruelle-Sullivan current $T_\mu$ is well-defined; it is honestly a $d-1$-current, and does not depend on the choice of partition of unity.
Moreover, $\dif T_\mu = 0$.
\end{lemma}
\begin{proof}
The various assertions of this lemma follow from \cite[Theorem 8.2]{daskalopoulos2020transverse} (at least when $d = 2$, but the general case is similar).
\todo{I also wrote up the proof of this in detail but it's kind of long. Include it?}
\end{proof}
% \begin{proof}
% We first claim that the right-hand side of (\ref{RS current}) is always finite, and is continuous in $\varphi$.
% In fact, possibly after refining $(\chi_\alpha)$, we may assume that it is a locally finite partition of unity.
% In particular, we just need to check the continuity in a single flow box:
% $$\left|\int_{K_\alpha} \left[\int_{\RR^{d - 1} \times \{k\}} (F_\alpha^{-1})^* (\chi_\alpha \varphi) \right] \dif \mu_\alpha(k)\right| \leq \int_{K_\alpha} \int_{\RR^{d - 1} \times \{k\}} |(F_\alpha^{-1})^* (\chi_\alpha \varphi)| \dif \mu_\alpha(k).$$
% The inner integral is controlled by $\|\varphi\|_{C^0(U_\alpha)} \cdot |U_\alpha|$ where $U_\alpha$ is the image of $F_\alpha$.
% The outer integral is then well-defined because it is against a Radon measure.

% We next observe that the choice of partition of unity is irrelevant, thus if $\varphi$ has compact support in $U_\alpha \cap U_\beta$, then
% \begin{equation}\label{well-defined of Ruelle-Sullivan}
% \int_{K_\alpha} \int_{\RR^{d - 1} \times \{k\}} (F_\alpha^{-1})^* \varphi \dif \mu_\alpha(k) = \int_{K_\beta} \int_{\RR^{d - 1} \times \{k\}} (F_\beta^{-1})^* \varphi \dif \mu_\beta(k).
% \end{equation}
% Indeed,
% \begin{align*}
% \int_{K_\alpha} \int_{\RR^{d - 1} \times \{k\}} (F_\alpha^{-1})^* \varphi \dif \mu_\alpha(k)
% &= \int_{K_\beta} (F_\alpha F_\beta^{-1})^* \left[\int_{\RR^{d - 1} \times \{k\}} (F_\alpha^{-1})^* \varphi \dif \mu_\alpha(k)\right] \\
% &= \int_{K_\beta} \left[\int_{\RR^{d - 1} \times \{k\}} (F_\beta^{-1})^* \varphi\right] (F_\alpha F_\beta^{-1})^* \dif \mu_\beta(k) \\
% &= \int_{K_\beta} \int_{\RR^{d - 1} \times \{k\}} (F_\beta^{-1})^* \varphi \dif \mu_\beta(k)
% \end{align*}
% where the last equation is because of the measure-preserving nature of the transition maps; this proves (\ref{well-defined of Ruelle-Sullivan}).

% Finally, if a $d-2$-form $\psi$ has compact support in a single flow box, then
% $$\int_{\RR^{d - 1} \times \{k\}} (F_\alpha^{-1})^* \dif \psi = \int_{\RR^{d - 1} \times \{k\}} \dif((F_\alpha^{-1})^* \psi) = 0$$
% by Stokes' theorem, so
% \begin{align*}
% \int_M \dif T_\mu \wedge \psi &= -\int_M T_\mu \wedge \dif \psi \\
% &= -\int_{K_\alpha} \int_{\RR^{d - 1} \times \{k\}} (F_\alpha^{-1})^* \dif \psi \dif \mu_\alpha(k) = 0. \qedhere
% \end{align*}
% \end{proof}

Though (\ref{RS current}) is the more traditional way of stating the definition of a Ruelle-Sullivan current, there is a more intrinsic way as well.
We first observe that if $\mu$ is a transverse measure, then $\mu$ defines a measure on $\supp \lambda$: in each flow box $F_\alpha$, an open set $U$ has measure
\begin{equation}\label{transverse measure of an open set}
\mu(U) := \int_{K_\alpha} |F_\alpha(\RR^{d - 1} \times \{k\}) \cap U| \dif \mu_\alpha(k).
\end{equation}

\begin{lemma}
For an oriented measured lamination $(\lambda, \mu)$, the polar decomposition of $T_\mu$ is
\begin{equation}\label{polar ruelle sullivan}
T_\mu = \normal_\lambda \mu.
\end{equation}
\end{lemma}
\begin{proof}
For an open set $U \subseteq M$ in a flow box $F_\alpha$, the total variation measure $|T_\mu|$ satisfies
$$|T_\mu|(U) = \sup_{\|\varphi\|_{C^0} \leq 1} \int_{K_\alpha} \int_{\RR^{d - 1} \times \{k\}} \varphi \dif \mu_\alpha(k)$$
where the supremum ranges over $d-1$-forms $\varphi$ with compact support in $U$.
However, $\star \normal_\lambda$ is the Riemannian measure on $F_\alpha(\RR^{d - 1} \times \{k\})$, so
$$\int_{\RR^{d - 1} \times \{k\}} \varphi \leq \int_{\RR^{d - 1} \times \{k\}} (F_\alpha^{-1})^*(\star \normal_\lambda).$$
Since $\|\normal^\lambda\|_{C^0} = 1$, it follows that a sequence of cutoffs of $\star \normal_\lambda$ to more and more of $U$ is a maximizing sequence.
Therefore $\normal_\lambda$ is the polar part of (\ref{polar ruelle sullivan}), and
$$|T_\mu|(U) = \int_{K_\alpha} \int_{\RR^{d - 1} \times \{k\}} (F_\alpha^{-1})^*(1_U \star \normal_\lambda) \dif \mu_\alpha(k).$$
The inner integral is the Riemannian measure of $F_\alpha(\RR^{d - 1} \times \{k\}) \cap U$, so by (\ref{transverse measure of an open set}), $|T_\mu| = \mu$.
\end{proof}

The above computation motivates the definition of Ruelle-Sullivan current of a \emph{nonorientable} lamination.
To be more precise, if $\lambda$ is a nonorientable lamination with normal vector field $\normal_\lambda$, then we can view $\normal_\lambda$ as a section of a (necessarily twisted) line bundle $L$ over $M$.
We can then define $T_\mu$ to be $\normal_\lambda \mu$, which makes sense as a distributional section of $L$, and can be tested against any $d-1$-form on $M$ whose support is contained in a contractible set.
In particular, we shall speak of the Ruelle-Sullivan current of any measured lamination, even if it is nonorientable.

\begin{lemma}\label{convergence of normals}
If $(\lambda_n, \mu_n) \to (\lambda, \mu)$, $x_n \in \supp \lambda_n$ converges to $x \in \supp \lambda$, and $(\lambda_n), \lambda$ have continuous normal vector fields $(\normal_n), \normal$, then $\normal_n(x_n) \to \normal(x)$ pointwise.
\end{lemma}
\begin{proof}
	Choose a continuous $d-1$-form $\varphi$ which extends $\star \normal$.
	Then for every $\varepsilon > 0$,
	$$\int_{B(x, \varepsilon)} T_\mu \wedge \varphi = \mu(B(x, \varepsilon))$$
	so by the portmanteau theorem, for almost every $\varepsilon > 0$,
	\begin{equation}\label{epsilon is a continuity set}
		\lim_{n \to \infty} \frac{\int_{B(x, \varepsilon)} T_{\mu_n} \wedge \varphi}{\mu_n(B(x, \varepsilon))} = \frac{\int_{B(x, \varepsilon)} T_\mu \wedge \varphi}{\mu(B(x, \varepsilon))} = 1.
	\end{equation}
	On the other hand, if we assume that there exists $\delta, \varepsilon > 0$ such that for every $y \in \supp \lambda_n \cap B(x, \varepsilon)$,
	$$|\sin \angle(\normal_n, \normal)| \geq \delta,$$
	then possibly after shrinking $\varepsilon$ we may assume that (\ref{epsilon is a continuity set}) holds, hence by (\ref{polar ruelle sullivan}),
	$$\int_{B(x, \varepsilon)} T_{\mu_n} \wedge \varphi = \int_{B(x, \varepsilon)} \normal_n \wedge \star \normal \dif \mu_i \leq (1 - O(\delta)) \mu_n(B(x, \varepsilon))$$
	and therefore $\delta = 0$, a contradiction.
\end{proof}


%%%%%%%%%%%%%%%%%%%%%%%%%%%%
\subsection{Calculus of variations}
We record various well-known facts about $1$-harmonic functions and minimal surfaces.

\begin{proposition}[Miranda stability theorem]
  Suppose that $M$ is a compact manifold, possibly with boundary.
	If a sequence of functions $(u_n)$ (not necessarily of the same trace) is bounded in $L^1(M)$ and satisfies
\begin{equation}\label{boundedness in Miranda}
	\limsup_{n \to \infty} \int_M \star |\dif u_n| \leq \liminf_{n \to \infty} \inf_{v|_{\partial M} = 0} \int_M \star |\dif(u_n + v)| < \infty,
\end{equation}
	then there exists a function $u$ of least gradient such that along a subsequence, $u_n \to u$ in $L^1(M)$ and $\dif u_n \to \dif u$ in the weak topology of measures.
\end{proposition}
\begin{proof}
The forgetful map $BV(M) \to L^1(M)$ is compact, so (\ref{boundedness in Miranda}) and the bounds in $L^1(M)$ imply that $(u_n)$ has a convergent subsequence.
The rest of the proof is similar to \cite[Teorema 3 and Osservazione 3]{Miranda67}; see \todo{\cite{BackusFLG}} for the straightforward modifications.
\end{proof}

\begin{proposition}[maximum principle]\label{max princip}
Let $u$ be a $1$-harmonic function.
If $u$ attains a local maximum, then $u$ is constant.
\end{proposition}
\begin{proof}
Let $y$ be a local maximum of $u$; then $\partial \{u \geq y\}$ is a minimal hypersurface by Theorem \ref{main thm of old paper}, or else it is empty. If it is empty, then $M = \{u \geq y\}$ and $u$ is constant, so we exclude this case.
In a neighborhood of any point of $\partial \{u \geq y\}$ we may decrease $\int \star |\dif u|$ by decreasing $y$ a small amount, which violates that $u$ has least gradient.
\end{proof}

\begin{theorem}[stable Bernstein theorem]
	Let $M$ be a manifold of bounded geometry, and dimension $d \in \{2, 3, 4\}$.
	Then there exists $A \geq 0$ such that for every complete stable minimal hypersurface $N$ in $M$ with trivial normal bundle,
	$$|\Two_N(P)| \leq \frac{A}{1 + \dist(P, \partial M)}.$$
\end{theorem}
\begin{proof}
	If $d = 2$ this clearly holds with $A = 0$.
	If $d = 3$ this was proven in \cite{Schoen2016}, and we refer to \cite[Theorem 2.10]{colding2011course} for a modern proof.
	If $d = 4$ this result can be derived from \cite[Theorem 1]{Chodosh2021} using the sort of techniques discussed in \cite[\S3]{White13}.
\end{proof}


%%%%%%%%%%%%%%%%%%%%%%%%%%%%%%%%%%%%%%%%%%
\section{Regularity of flow boxes}\label{Regularity}
The goal of this section is to prove the following regularity theorem for minimal laminations that we will use several times.
The proof is based on \cite[Theorem 1.1]{Solomon86}, which addresses the case that $\lambda$ is a minimal foliation, and does not explictly spell out the quantitative regularity of the laminar atlas.
Lipschitz regularity is optimal even for the nicest case of a geodesic foliation of $\Hyp^2$ \cite[\S1]{Solomon86}, so this result is sharp.

\begin{proposition}\label{regularity theorem}
Let $(M, g)$ be a Riemannian manifold of dimension $d \geq 2$ and bounded geometry.
Let $\lambda$ be a minimal lamination in $M$ such that for some $A > 0$ and every leaf $N \in \Leaves \lambda$,
\begin{equation}\label{curvature bound in regularity}
	\|\Two_N\|_{C^0} \leq A.
\end{equation}
Then:
\begin{enumerate}
\item There exists a Lipschitz subbundle of $TM$ which restricts to a normal vector field on each leaf of $\lambda$.
\item There exist $L = L(g, A) > 0$ and $r = r(g, A) > 0$, and a Lipschitz laminar atlas $(F_\alpha)$ for $\lambda$, such that for every $\alpha$,
\begin{equation}\label{conorm of flow box}
	\max(\Lip(F_\alpha), \Lip(F_\alpha^{-1})) \leq L,
\end{equation}
and the image of $F_\alpha$ contains a ball of size $r$.
\end{enumerate}
\end{proposition}

To begin the proof we first consider when we can represent the leaves of $\lambda$ as graphs in a uniform way.

\begin{lemma}\label{existence of tubes}
	Let $N$ be a connected hypersurface in $\RR^d = \RR^{d - 1}_x \times \RR_y$ which is tangent to $\{y = 0\}$ at the origin.
	If $\|\Two_N\|_{C^0} \leq \log(5/4)$, then for every $(x, y) \in N \cap B_1$,
	$$\max(|y|, 0.5 \cdot |\normal_N(x, y) - \partial_y|) \leq \|\Two_N\|_{C^0}.$$
\end{lemma}
\begin{proof}
	Near $0$, $N$ can be represented a graph $\{y = f(x)\}$, since it is tangent to $\{y = 0\}$.
	This representation is valid on the component of the set $\{|\nabla f(x)| < \infty\}$ containing $0$, and it is related to the unit normal by
\begin{equation}\label{nabla as a normal}
	\nabla f(x) = \frac{\partial_y - \normal(x, f(x))}{\sqrt{1 + |\nabla f(x)|^2}}.
\end{equation}
	Taking derivatives of both sides,
	$$-\nabla^2 f(x) = \frac{\nabla \normal(x, f(x)) \cdot (\partial_x \otimes \partial_x + \nabla f(x) \otimes \partial_y)}{\sqrt{1 + |\nabla f(x)|^2}} + \frac{\nabla^2 f(x) \cdot (\nabla f(x) \otimes (\partial_y - \normal(x, f(x))))}{(1 + |\nabla f|^2)^{3/2}}.$$
	Here $-\nabla^2$ denotes the negative Hessian, not the Laplacian.
	We can use (\ref{nabla as a normal}) to bound
	$$|\partial_y - \normal(x, f(x))| \leq |\nabla f(x)|\sqrt{1 + |\nabla f(x)|^2} \leq |\nabla f(x)| + |\nabla f(x)|^2.$$
	Since
	$$|\partial_x \otimes \partial_x + \nabla f(x) \otimes \partial_y| \leq \sqrt{1 + |\nabla f(x)|^2},$$
	and $\nabla \normal = \Two_N$, we conclude
\begin{equation}\label{bound Hessian by Two}
	|\nabla^2 f(x)| \leq |\Two_N(x, f(x))| + |\nabla^2 f(x)| (|\nabla f(x)|^2 + |\nabla f(x)|^3).
\end{equation}

	In order to control the error terms in (\ref{bound Hessian by Two}), we make the \dfn{bootstrap assumption}
\begin{equation}\label{bootstrap}
	|\nabla f(x)| \leq 1/4,
\end{equation}
	which is at least valid in some small neighborhood $B_R$ of $0$ since (\ref{nabla as a normal}) and the fact that $N$ is tangent to $\{y = 0\}$ at $0$ imply that $\nabla f(0) = 0$.
	One can show using Taylor's theorem that (\ref{bootstrap}) implies that 
	$$\frac{1}{1 - |\nabla f(x)|^2 - |\nabla f(x)|^3} \leq 1 + |\nabla f(x)|,$$
	so by (\ref{bound Hessian by Two}),
	$$|\nabla^2 f(x)| \leq \frac{\|\Two_N\|_{C^0}}{1 - |\nabla f(x)|^2 - |\nabla f(x)|^3} \leq \|\Two_N\|_{C^0} \cdot (1 + |\nabla f(x)|).$$
	If we write $x = (r, \theta)$, then by the mean value theorem, it follows that for $r \leq R$,
\begin{align*}
	|\nabla f(x)| &\leq \|\Two_N\|_{C^0} \int_0^r 1 + |\nabla f(s, \theta)| \dif s.
\end{align*}
	So by Gr\"onwall's inequality,
\begin{align*}
	|\nabla f(x)| &\leq r \|\Two_N\|_{C^0} + \|\Two_N\|_{C^0}^2 \int_0^r s \exp(\|\Two_N\|_{C^0} \cdot (r - s)) \dif s \\
	&= r \|\Two_N\|_{C^0} + \exp(r \|\Two_N\|_{C^0}) - r \|\Two_N\|_{C^0} - 1 \\
	&= \exp(r \|\Two_N\|_{C^0}) - 1.
\end{align*}
	Recalling that $\|\Two_N\|_{C^0} \leq \log(5/4)$, and $r \leq R < 1$, we conclude that on $B_R$, $|\nabla f| < 1/4$.
	That is, we recover the bootstrap assumption (\ref{bootstrap}) on some ball $B_{R'}$ for some $R' > R$, and hence on the entire ball $B_1$.

	By the mean value theorem again,
\begin{align*}
	|f(x)| &\leq \int_0^r |\nabla f(s, \theta)| \dif s \leq \int_0^r \exp(s \|\Two_N\|_{C^0}) - 1 \dif s \\
	&= \|\Two_N\|_{C^0}^{-1} (\exp(r \|\Two_N\|_{C^0}) - r\|\Two_N\| - 1) \\
	&\leq r^2 \|\Two_N\|_{C^0}
\end{align*}
	where we again used the bound $\|\Two_N\|_{C^0} \leq \log(5/4) < 0.25$ to control the higher-order terms in the Taylor expansion of $\exp(r \|\Two_N\|_{C^0})$. Since $r < 1$ was arbitrary we conclude $\|f\|_{C^0} \leq \|\Two_N\|_{C^0}$. Moreover, by (\ref{nabla as a normal}),
\begin{align*}
	|\normal(x, f(x)) - \partial_y| &\leq |\nabla f(x)| \leq 2r \|\Two_N\|_{C^0}. \qedhere
\end{align*}
\end{proof}

Our next lemma guarantees the existence of normal coordinates in which the leaves of $\lambda$ are close to hyperplanes $\{y = y_0\}$.
An analogous result was proven by \cite{Solomon86} (without the quantitative dependence) by a very different means, using the regularity theory for integral flat convergence of minimal currents \cite[Theorem 5.3.14]{federer2014geometric}.
We did not do this, both because \cite{Solomon86} requires that $\lambda$ be a foliation, and because it does not seem particularly easy to recover quantitative bounds from the highly general theory of \cite[Chapter 5]{federer2014geometric}.

\begin{lemma}\label{lams have C0 fields}
	For every sufficiently small $\delta > 0$ there exist $r = r(\delta, g, A) > 0$ such that for every minimal lamination $\lambda$ satisfying (\ref{curvature bound in regularity}) we can choose normal coordinates $(x, y) \in \RR^{d - 1} \times \RR$ on $B(P, r)$ so that
\begin{equation}\label{normal is basically dy}
	\|\normal_\lambda - \partial_y\|_{C^0(B(P, r))} \leq \delta.
\end{equation}
\end{lemma}
\begin{proof}
	Suppose not.
	Then, after rescaling to set $A = 1$, there exist minimal laminations $\lambda_n'$ in $B(P, 1/n)$ such that
	$$\sup_{N \in \Leaves \lambda_n'} \|\Two_N\|_{C^0} \leq 1,$$
	but no matter how we rotate normal coordinates based at $P$, (\ref{normal is basically dy}) fails for $\lambda_n'$.
	By taking normal coordinates and rescaling by $n$, we can find (possibly nonminimal) laminations $\lambda_n$ in $B_1 \subset \RR^d$ such that
% \begin{sublemma}
% 	For each $n \geq \inj(g)$ there exists a lamination $\lambda_n$ in $\RR^d$ so that every rotation of $B_1$ fails (\ref{normal is basically dy}), but satisfies
\begin{equation}\label{bounds on Two in representation}
	\sup_{N \in \Leaves \lambda_n} \|\Two_N\|_{C^0} \lesssim_g \frac{1}{n^2}
\end{equation}
	but for every rotation of $\RR^d$, (\ref{normal is basically dy}) fails.
% \end{sublemma}

% We remark carefully that the lamination in the above sublemma may not be minimal.
% However this fact will be irrelevant.

% \begin{proof}
% 	Since $n \geq \inj(g)$ we can pass to the tangent space and rescale $B(P, 1/n)$ to obtain $B_1$. The scaling makes the second fundamental form $\Two_N'$ with respect to $g$ satisfy
% $$\|\Two_N'\|_{C^0} \leq \frac{e^{-O(n^2)}}{n^2} \lesssim \frac{1}{n^2}.$$
% 	Here and always all implied constants may depend on $g$.
% 	In normal coordinates,
% 	$$g_{\mu \nu} = \delta_{\mu \nu} + \sum_{k=2}^\infty c_k \frac{|x|^k}{n^k},$$
% 	and differentiating this expression gives that the connection $1$-form $\Gamma$ for the Levi-Civita connection $\nabla_g$ satisfies $|\Gamma| \lesssim n^{-2}$.
% 	But $\|\Two_N - \Two_N'\|_{C^0} \lesssim \|\Gamma\|_{C^0}$, since $\Two_N - \Two_N'$ is the difference between the euclidean and $\nabla_g$ derivatives of the conormal.
% \end{proof}

	By Lemma \ref{existence of tubes}, every leaf of $\lambda_n$ is $O(n^{-2})$-close to its tangent spaces in $C^1$.
	In particular, if $n \gg \delta^{-1/2}$, and the normal vector at some point to $N \in \Leaves \lambda_n$ is $\partial_y$, then
	$$\|\normal_N - \partial_y\|_{C^0(B(P, r))} \ll \delta.$$
	We can always impose this for some $N$ by applying a rotation of $\RR^d$.
	But by our contradiction assumption, there exists some leaf $N'$ of $\lambda_n$ and some $P \in N'$ so that $|\normal_{N'}(P) - \partial_y| \geq \delta$ and hence if $n$ is large enough,
	$$\inf_{N'} |\normal_{N'} - \partial_y| \geq \frac{\delta}{2}.$$
	By the reverse triangle inequality it follows that
\begin{equation}\label{discrepancy in normals}
	\inf_{\substack{P \in N\\ P' \in N'}} \sin \angle(\normal_N(P), \normal_{N'}(P')) \gtrsim \delta
\end{equation}
	at least if $\delta$ is smaller than an absolute constant.
	
	In order to obtain a contradiction, we may assume that $r < 1$ is arbitrarily small and consider the restriction of $\lambda_n$ to $B_r$.
	In particular, we may assume that there are points $P \in N \cap B_r$ and $P' \in N' \cap B_r$ for some $r \ll \delta$.
	By (\ref{discrepancy in normals}), the angle $\theta$ between the tangent spaces $T_P N$ and $T_{P'} N'$ is $\gtrsim \delta$, but $|P - P'| \ll \delta$.
	This is only possible if $T_P N$ and $T_{P'} N'$ intersect in some ball $B_{r'}$ for some $r' \ll 1$, say $r' = 1/2$.
	But then, since $N, N'$ are $O(n^{-2})$-close to $T_P N$ and $T_{P'} N'$ in $C^0$, if $n$ is large, then $N$ and $N'$ intersect in $B_1$.
	However, $N, N'$ were assumed to be leaves of the same lamination $\lambda_n$ in $B_1$, so this is a contradiction.
\end{proof}

\begin{proof}[Proof of Proposition \ref{regularity theorem}]
Fix $\delta > 0$ to be chosen later, and $P \in M$.
By Lemma \ref{lams have C0 fields}, there exists $r = r(\delta, g, A) > 0$ such that $B(P, r)$ admits cylindrical coordinates $(x, y) \in 3\Ball^{d - 1} \times (-2, 2)$, and
\begin{equation}\label{normal is almost constant}
\|\normal - \partial_y\|_{C^0(B(P, r))} < \delta.
\end{equation}
If $\delta$ is chosen small enough depending on $g$, then we may assume that in $2\Ball^{d - 1} \times (-1, 1)$,
every leaf is the graph of a function, say $u_k: 3\Ball^{d - 1} \to (-2, 2)$ where $u_k(0) = k$.
Then $u_k$ solves a quasilinear elliptic PDE $Lu_k = 0$, so that if $r$ is chosen small enough depending on $g$ and $\sup_{N \in \Leaves \lambda} \|\Two_N\|_{C^0}$, then the ellipticity of $L$ and H\"older norms of the coefficients of $L$ only depend on $g$, but not on $k$ or $\delta$.
So by a straightforward modification of \cite[Corollary 16.7]{gilbarg2015elliptic}, for every $m \geq 0$ there exists $C_m = C_m(g) > 0$ such that
\begin{equation}\label{norms on uk}
\sup_{|k| < 1} \|u_k\|_{C^m} \leq C_m.
\end{equation}
Now for $k \in (-1, 1)$ fixed, let $k < \ell < 1$, and let $v_{\ell k} := u_\ell - u_k$.
Then $v_{\ell k}$ is the difference of two elements of the kernel of $L$, so $v_{\ell k}$ solves a linear elliptic PDE $Q_k v_{\ell k} = 0$, where the ellipticity of $Q_k$ and the H\"older norms of the coefficients of $Q_k$ only depend on $g$ and $C_m$ for some $m$, but not on $\ell, k, \delta$.
Moreover, $v_{\ell k} > 0$: clearly $v_{\ell k} \geq 0$, and if $v_{\ell k}(x) = 0$ for some $x$, then $v_{\ell k} = 0$ by the maximum principle, which implies $k = \ell$, a contradiction.
By the Schauder \cite[Theorem 6.2]{gilbarg2015elliptic} and Harnack \cite[Theorem 9.25]{gilbarg2015elliptic} inequalities, for every $x \in \Ball^{d - 1}$,
\begin{equation}\label{Schauder Harnack}
	\|\dif v_{\ell k}\|_{C^0(\Ball^{d - 1})} \lesssim_{C_m, g} \|v_{\ell k}\|_{C^0(2 \Ball^{d - 1})} \lesssim_{C_m, g} \inf_{C^0(\Ball^{d - 1})} v_{\ell k} \leq v_{\ell k}(x).
\end{equation}
In particular, for every $x$,
$$|\dif u_\ell(x) - \dif u_k(x)| \lesssim_{C_m, g} |u_\ell(x) - u_k(x)|$$
so there exists $C = C'(C_m, g)$ such that
\begin{equation}\label{vertical Lipschitz}
|\normal(x, u_\ell(x)) - \normal_k(x, u_k(x))| \leq C |u_\ell(x) - u_k(x)|.
\end{equation}

To extend (\ref{vertical Lipschitz}) to a Lipschitz bound on $\normal$, let $X_1, X_2 \in (\Ball^{d - 1} \times (-1, 1)) \cap \supp \lambda$.
Then there exist $x_1, x_2 \in \Ball^{d - 1}$ and $k_1, k_2 \in (-1, 1)$ such that $X_i = (x_i, u_{k_i}(x_i))$.
Setting $Y := (x_2, u_{k_1}(x_2))$,
$$|\normal(X_1) - \normal(X_2)| \leq |\normal(X_1) - \normal(Y)| + |\normal(Y) - \normal(X_2)|.$$
Then by (\ref{norms on uk}) and the mean value theorem,
$$|\normal(X_1) - \normal(Y)| \lesssim |\dif u_{k_1}(x_1) - \dif u_{k_1}(x_2)| \leq C_2 |X_1 - Y|.$$
Moreover, by (\ref{vertical Lipschitz}),
$$|\normal(Y) - \normal(X_2)| \leq C|u_{k_1}(x) - u_{k_2}(x)| = C|Y - X_2|.$$
If $\delta$ is chosen smaller than some absolute constant, then by (\ref{normal is almost constant}),
$$|\sin \angle(X_1 - Y, X_2 - Y)| > 1 - O(\delta)$$
and we conclude by the Pythagorean theorem that
$$|Y - X_2|^2 + |X_1 - Y|^2 \lesssim |X_1 - X_2|^2.$$
In conclusion,
$$|\normal(X_1) - \normal(X_2)| \lesssim_g |X_1 - X_2|$$
which implies that $\normal$ is Lipschitz on $V \cap \supp \lambda$, where $V$ is the neighborhood of $P$ which was mapped to $\Ball^{d - 1} \times (-1, 1)$ by the cylindrical coordinates $(x, y)$.
In particular, $V$ contains a ball of the form $B(P, s)$, where $s$ only depends on $r$.
Note that $r$ only depends on $g$ and $\sup_{N \in \Leaves \lambda} \|\Two_N\|_{C^0}$.
Moreover, $|X_1 - X_2| \sim r \dist(X_1, X_2)$ by definition of the coordinates $(x, y)$, so we obtain
\begin{equation}\label{lipschitz normal}
	|\Lip(\normal)| \lesssim r^{-1}.
\end{equation}
Taking a Lipschitz extension of $\normal$ to $V$ we obtain the desired Lipschitz normal subbundle.

Following \cite[Appendix B]{ColdingMinicozziIV}, we construct the laminar flow box
\begin{align*}
	F: \RR^{d - 1}_\xi \times \RR_\eta &\to V \subseteq \RR^{d - 1}_x \times \RR_y \\
	(\xi, \eta) &\mapsto (\xi, f(\xi, \eta))
\end{align*}
by setting
$$f(\xi, \eta) := u_\eta(\xi)$$
if $u_\eta$ exists, and if $k < \eta < \ell$ and there does not $k < \eta' < \ell$ such that $u_{\eta'}$ exists, then
$$f(\xi, \eta) := u_k(\xi) + \frac{\eta - k}{\ell - k} v_{\ell k}(\xi)$$
is the linear interpolant of $u_k$ and $u_\ell$.
Then it suffices to estimate $\|\dif f\|_{L^\infty}$ and $\|\dif f^{-1}\|_{L^\infty}$ in order to control $\Lip(F)$ and $\Lip(F^{-1})$.

First, in the interpolated region $k < \eta < \ell$,
$$\dif f = \dif u_k + \frac{\eta - k}{\ell - k} \dif v_{\ell k} + \frac{v_{\ell k}}{\ell - k} \dif \eta.$$
From (\ref{norms on uk}), $|\dif u_k| \leq C_1 \lesssim 1$, and from (\ref{Schauder Harnack}) with $x = 0$,
$$\max(|v_{\ell, k}|, |\dif v_{\ell k}|) \lesssim \ell - k;$$
we conclude that on the interpolated regions, $|\dif f| \lesssim 1$.
In the other direction, since $\dif u_k$ and $\dif v_{\ell k}$ are orthogonal to $\dif \eta$, (\ref{Schauder Harnack}) gives
$$|\dif f| \geq \inf \frac{v_{\ell k}}{\ell - k} \gtrsim \frac{v_{\ell k}(0)}{\ell - k} = 1.$$

We now turn to the foliated region. If $u_\eta$ exists, then
$$\dif f = \dif u_\eta + \frac{\partial u_\eta}{\partial \eta} \dif \eta.$$
We again have $|\dif u_\eta| \lesssim 1$, and from (\ref{Schauder Harnack}) the second term is $\lesssim 1$ as well, hence $|\dif f| \lesssim 1$.
Moreover,
$$|\dif f| \geq \left|\frac{\partial u_\eta}{\partial \eta}\right| \geq \inf_{k < \ell} \frac{v_{\ell k}(x)}{\ell - k}.$$
By (\ref{Schauder Harnack}),
\begin{align*}
	v_{\ell k}(\xi) &\geq v_{\ell k}(0) - \|\dif v_{\ell k}\|_{C^0} \cdot |\xi| \\
	&\geq \ell - k - O(v_{\ell k}(0)) |\xi| \\
	&\geq (1 - O(|\xi|)) (\ell - k).
\end{align*}
Thus for $|\xi|$ smaller than an absolute constant, $|\dif f| \geq 1/2$.
This implies that $F$ (or rather, the composition of $F$ with the change of coordinates at the start of this proof) is a laminar flow box in a small neighborhood of $(0, 0)$ whose image has radius $cr$ for some small $c > 0$, and whose Lipschitz constants are comparable to $O(r^{-1})$.
\end{proof}

%%%%%%%%%%%%%%%%%%%%%%%%%%%%%%%%%%%%%%%%%
\section{Proofs of main theorems}\label{CompactnessSec}
We shall now use Proposition \ref{regularity theorem} to prove Theorems \ref{compactness theorem}, \ref{implication theorem}, and \ref{main thm}.
Throughout, let $M$ be a manifold of constant sectional curvature and dimension $2 \leq d \leq 7$.

\subsection{Compactness}
We begin by proving Theorem \ref{compactness theorem}.

Let $P \in M$.
By Proposition \ref{regularity theorem}, there exist $r > 0$ and $L \geq 1$ such that for every large $n \in \NN$, $B(P, r)$ is contained in the image of a flow box $F_n$ for $\lambda_n$ with Lipschitz constant $L$, such that $F_n(0, 0) = P$.
By Proposition \ref{AA Holder}, along a subsequence $F_n \to F$ in $C^{1-}$ for some $C^{1-}$ map $F: I \times J \to B(P, r)$ and some $I \subseteq \RR$, $J \subseteq \RR^{d - 1}$, such that on the image $V$ of $F$, $F^{-1}$ is also $C^{1-}$.
Moreover, $F(0, 0) = P$, so that $F: I \times J \to V$ is a Lipschitz isomorphism onto a set which contains $P$.
Since $P$ was arbitrary, it follows that we can find laminar atlases $(F_\alpha^n. K_\alpha^n)$ for each large $n \in \NN$ such that $F_\alpha^n \to F_\alpha$, where the images of $F_\alpha$ are an open cover $(U_\alpha)$ of $M$, and $(F_\alpha)$ satisfies the usual transition relations.

We now construct the limiting lamination.
By Proposition \ref{Hausdorff on a CMS} and the fact that $I$ is a compact metric space, we may diagonalize so that for every $\alpha$, either $K^n_\alpha \to K_\alpha$ for some nonempty $K_\alpha$ in $\Hypspace I$, or for all $n \geq n^*(\alpha)$, $K_\alpha^n$ is empty (in which case we define $K_\alpha = \emptyset$).

\begin{lemma}\label{label sets are nonempty}
	There exists $\alpha$ such that $K_\alpha$ is nonempty.
\end{lemma}
\begin{proof}
	Suppose not, and let $E$ be the compact set such that every leaf of every lamination meets $E$.
	We can find a finite set $A_E$ such that $E \subseteq \bigcup_{\alpha \in A_E} U_\alpha$. Then for
	$$n \geq \max_{\alpha \in A_E} n^*(\alpha)$$
	and $\alpha \in A_E$, $K_\alpha^n = \emptyset$, so no leaves of $\lambda_n$ meet $U_\alpha$, and hence no leaves of $\lambda_n$ meet $E$.
	This is a contradiction since $\lambda_n$ has a leaf.
\end{proof}

Now let $\psi_{\alpha \beta}$ and $\psi_{\alpha \beta}^n$ be the transition maps, thus $\psi_{\alpha \beta}^n$ induces a map
$$\psi_{\alpha \beta}^n: K_\alpha^n \to K_\beta^n.$$
By convergence of $(F_\alpha^n)$, $\psi_{\alpha \beta}$ induces a map $K_\alpha \to K_\beta$.

By a \dfn{cocycle of labels} we shall mean the data of $k_\alpha \in K_\alpha$, for every $\alpha \in A'$ and some maximal $A' \subseteq A$, such that 
$$k_\beta = \psi_{\alpha \beta}(k_\alpha).$$
Given a cocycle of labels $(k_\alpha)$ we define
\begin{equation}\label{cocycle implies hypersurface}
	N \cap U_\alpha = F_\alpha(\{k_\alpha\} \times \RR^{d - 1}).
\end{equation}

\begin{lemma}
	Every cocycle of labels defines a complete minimal hypersurface by (\ref{cocycle implies hypersurface}).
\end{lemma}
\begin{proof}
Let $(k_\alpha)_{\alpha \in A'}$ be a cocycle of labels.
We have the cocycle condition
$$(N \cap U_\alpha) \cap U_\beta = (N \cap U_\beta) \cap U_\alpha$$
which follows from the fact that
\begin{align*}
F_\alpha(\{k_\alpha\} \times \RR^{d - 1}) \cap U_\beta
&= F_\beta(\psi_{\alpha \beta}(\{k_\beta\} \times \RR^{d - 1})) \cap U_\alpha \cap U_\beta \\
&= F_\beta(\psi_{\alpha \beta}(\{k_\beta\} \times \RR^{d - 1})) \cap U_\alpha.
\end{align*}
From the cocycle condition, it follows that $N$ honestly defines a Lipschitz hypersurface in $M$, which is complete in $\bigcup_{\alpha \in A'} U_\alpha$.
If $\overline N$ intersects $U_\alpha$ for some $\alpha \notin A'$, then $N$ intersects $U_\beta$ for some $\beta \in A'$ so that $U_\beta \cap U_\alpha \cap \overline N$ is nonempty.
But then $\psi_{\beta \alpha}(k_\beta)$ must be defined and so by maximality of $A'$, $\alpha \in A'$, a contradiction.
Therefore $N$ is complete in $M$.

To prove minimality, let
$$u_\alpha(k, x) = 1_{k > k_\alpha}$$
and similarly $u_\alpha^n(k, x) = 1_{k > k_\alpha^n}$ where $(k_\alpha^n) \in \prod_n K_\alpha^n$ converges to $k_\alpha$.
Then the pullback of $u_\alpha^n$ by $F_\alpha \circ (F_\alpha^n)^{-1}$ converges away from $F_\alpha^{-1}(N \cap U_\alpha)$, and hence almost everywhere to $u_\alpha$, since $F_\alpha \circ (F_\alpha^n)^{-1}$ converges in $C^{1-}$ to the identity.
But $u_\alpha^n$ has approximately least gradient in the sense of (\ref{boundedness in Miranda}) with respect to the pullback metric $F_\alpha^* g$, so its limit $u_\alpha$ must have least gradient by the Miranda stability theorem.
Thus by Theorem \ref{main thm of old paper}, $N \cap U_\alpha$ is minimal.
\end{proof}

Let $\lambda$ be the lamination with laminar atlas $(F_\alpha, K_\alpha)$.
The facts that every cocycle of labels defines a complete hypersurfaces, and that $K_\alpha$ is compact for every $\alpha$ and nonempty for some $\alpha$, ensure that $\lambda$ is actually a lamination.
The fact that every cocycle of labels defines a \emph{minimal} hypersurface implies that $\lambda$ is itself minimal.

\begin{lemma}
	$\lambda_n \to \lambda$ in Thurston's geometric topology.
\end{lemma}
\begin{proof}
Leaves of $\lambda$ are pointwise limits of leaves in $\lambda_n$, so it suffices to show that for $N \in \Leaves \lambda$, $P \in N$, and $P_n \to P$, where $P_n \in N_n$ and $N_n \in \Leaves \lambda_n$, $\normal_{N_n}(P_n) \to \normal_N(P)$.
To do this, suppose that $P \in U_\alpha$; $F_\alpha^n$ is close in $C^{1-}$ to $F_\alpha$, and the label $k^n_\alpha$ of $N_n$ is close to the label $k_\alpha$ of $N$.
So
$$F_\alpha^{-1}(N_n) = F_\alpha^{-1}(F_\alpha^n(\{k_\alpha^n\} \times \RR^{d - 1}))$$
is close in $C^{1-}_\loc$ to $F_\alpha^{-1}(N) = \{k_\alpha\} \times \RR^{d - 1}$, and the claim follows.
\end{proof}

Finally, if $\mu_n$ is transverse to $\lambda_n$, then we fix a compact set $E$ which meets every leaf of $\lambda_n$, and cover $E$ by finitely many flow boxes $\{U_\alpha: \alpha \in A_E\}$.
After possibly shrinking the $U_\alpha$ slightly, we may assume that they are precompact in $M$ and still form an open cover of $E$.
Then $K := \bigcup_{\alpha \in A_E} \overline{U_\alpha}$ is compact, so by Prohorov's theorem \cite[Theorem 13.29]{klenke2013probability}, there is a subsequence of $(T_{\mu_n})$ which converges to some $T_\mu|_K$ on $K$.
Moreover, by the portmanteau theorem,
$$\supp T_\mu|_K \subseteq \liminf_{n \to \infty} \supp T_{\mu_n}|_K \subseteq \liminf_{n \to \infty} \supp \lambda_n \cap K.$$
Here the $(\lambda_n)$ in the limit inferior refers to the subsequence which already converges in the Thurston topology (and has converging Ruelle-Sullivan currents).
In particular, the limit inferior is actually a limit and we conclude
$$\supp T_\mu|_K \subseteq \supp \lambda \cap K.$$
For each $\alpha \in A_E$, the definition of weak limit implies that
$$\int_{U_\alpha} T_\mu \wedge \varphi = \int_I \int_{\{k\} \times \RR^{d - 1}} F_\alpha^* \varphi \dif \mu(k)$$
where $\mu$ is a weak limit of $(\mu_n)$.
In particular, $T_\mu|_K$ is Ruelle-Sullivan for $\lambda|_K$, possibly after shrinking $\lambda|_K$ so that their supports match.
By the measure-preserving condition in the definition of transverse measure, $T_\mu|_K$ extends uniquely to a Ruelle-Sullivan current $T_\mu$ on all of $M$, which then necessarily is a weak limit of $(T_{\mu_n})$.
This completes the proof of Theorem \ref{compactness theorem}.


%%%%%%%%%%%%%%%%%%%%%%%%%%%%%%%%%%%%%%
\subsection{Measured convergence implies flow box convergence}
In order to show that convergence in the weak topology of measures implies convergence in the $C^{1-}$ flow box topology, we first show a weaker mode of convergence, namely Thurston's geometric topology.
Thurston claimed this fact \cite[Proposition 8.10.3]{thurston1979geometry} in case $d = 2$, but his proof left something to be desired as it did not justify why the limit is geodesic, or why the convergence respects the normal vectors.

\begin{lemma}\label{limits of measured geodesic lams are geodesic}
	The set of minimal measured laminations is closed in the weak topology of measures.
\end{lemma}
\begin{proof}
Let $(\lambda, \mu)$ be a measured lamination and suppose that $(\lambda_i, \mu_i) \to (\lambda, \mu)$ in the weak topology of measures, where $(\lambda_i, \mu_i)$ are measured minimal.
Let $x \in \supp \lambda$ and $r > 0$ such that $B := B(x, r)$ is contractible.
In $B$, we can write $T_{\mu_i} = \dif u_i$ for some sequence of functions of least gradient $u_i \in BV(B)$.
Since $u_i$ is only defined up to a constant, we impose $\int_M \star u_i = 0$, so by Poincar\'e's inequality,
$$\|u_i\|_{L^1(B)} \lesssim r\mu_i(B) \leq 2r \mu(B) < \infty$$
for $i$ large.
So by the Miranda stability theorem, there exists a $1$-harmonic function $u$ such that along a subsequence, $\dif u_i \to \dif u$ in the weak topology of measures.
But then we must have $T = \dif u$, so $\lambda$ is minimal by Theorem \ref{main thm of old paper}.
\end{proof}

We are now ready to prove Theorem \ref{implication theorem}.
So let $(\lambda_n, \mu_n)$ be a sequence of measured minimal laminations convering to $(\lambda, \mu)$.
Then
\begin{equation}\label{support is nonincreasing}
	\supp \lambda \subseteq \liminf_{n \to \infty} \supp \lambda_n.
\end{equation}
Indeed, if $x \in \supp \lambda$ then for all $\varepsilon > 0$, $\mu(B(x, \varepsilon)) > 0$.
Then by the portmanteau theorem, for all $n$ large, $\mu_n(B(x, \varepsilon)) > 0$; this proves (\ref{support is nonincreasing}).

By (\ref{support is nonincreasing}), for every $x \in \supp \lambda$, $\varepsilon > 0$, and large $n$, $\supp \lambda_n \cap B(x, \varepsilon)$ is nonempty, and by Lemma \ref{limits of measured geodesic lams are geodesic}, $\lambda$ is a minimal lamination.
By Proposition \ref{regularity theorem}, $\lambda, \lambda_n$ admit Lipschitz normal vectors, so by Lemma \ref{convergence of normals}, $\lambda_n \to \lambda$ in Thurston's geometric topology.

Now suppose that $(\lambda_n)$ has bounded curvature.
After discarding some leaves of $\lambda_n$ we may assume that $\lambda$ is a maximal limit.
Moreover, every subsequence $(\lambda_{n_k})$ has a further subsequence $(\lambda_{n_{k_\ell}})$ which converges to some maximal limit $\tilde \lambda$ in the $C^{1-}$ flow box topology by Theorem \ref{compactness theorem}.
But convergence in the flow box topology implies convergence in Thurston's topology, so $\tilde \lambda = \lambda$.
Since $(\lambda_{n_k})$ was arbitrary, it follows that $\lambda_n \to \lambda$ in the $C^{1-}$ flow box topology.


%%%%%%%%%%%%%%%%%%%%
\subsection{Application to 1-harmonic functions}
We finally prove Theorem \ref{main thm}.
We assume that $d \in \{2, 3, 4\}$.

Let $u$ be a $1$-harmonic function on $M$.
By Theorem \ref{main thm of old paper}, the level sets of $u$ are closed embedded minimal hypersurfaces in $M$; let
$$Y = \{y \in \RR: \partial \{u > y\} \neq \emptyset\}$$
index the level sets of $u$.

Let $y, z \in Y$. If $y > z$, then $\{u > y\} \subseteq \{u > z\}$, so $\partial \{u > y\}$ lies on one side of $\partial \{u > z\}$.
By the maximum principle for minimal surfaces \cite[Corollary 1.28]{colding2011course}, it follows that either $\partial \{u > y\}$ and $\partial \{u > z\}$ are disjoint, or are equal.
Moreover, $\dif u$ is conormal to $\partial \{u > y\}$, so $\partial \{u > y\}$ has trivial normal bundle.
Therefore by the stable Bernstein theorem, after replacing $M$ with an element of a compact exhaustion $(X_m)$ of $M$, we may assume that for some $A > 0$,
\begin{equation}\label{curvature estimate on 1 harmonic}
	\sup_{y \in Y} \|\Two_{\partial \{u > y\}}\|_{C^0} \leq A.
\end{equation}
We then choose a dense sequence $(y_n)$ in $Y$ and let $N_n = \partial \{u > y_n\}$.
Since $N_n$ is closed, $S_n := \bigcup_{k < n} N_k$ is also closed, so $S_n$ is the support of a minimal lamination $\lambda_n$.
By (\ref{curvature estimate on 1 harmonic}) and Theorem \ref{compactness theorem}, after passing to a subsequence we may find a limit $\lambda$ of $(\lambda_n)$ in the $C^{1-}$ flow box topology.
Diagonalizing against the compact exhaustion $(X_m)$, we can return to the original manifold $M$.

Every $N_n$ is a leaf of $\lambda$, and if $y \in Y$, then $\partial \{u > y\}$ is approximated by some subsequence of $(N_n)$ and so also is a leaf of $\lambda$.
Since this property characterizes leaves of $\lambda$, we conclude that the leaves of $\lambda$ are exactly the level sets of $u$.
It follows that $\bigcup_{y \in Y} \partial \{u > y\}$ is the support of $\lambda$, that $\lambda$ is minimal, and that $\dif u$ is conormal to $\lambda$.
In particular, we obtain an orientation on $\lambda$ from $\dif u$.

We now construct the transverse measure to $\lambda$.
In any oriented laminar coordinates $(k, x) \in K \times \RR^{d - 1}$ for $\lambda$, $\partial_x u = 0$, so $\star |\dif u|$ defines a measure $\mu$ on $K$: given $\alpha < \beta$, let
$$\mu([\alpha, \beta] \cap K) := u(\beta, x) - u(\alpha, x)$$
for any (and hence every, since $\partial_x u = 0$) $x \in \RR^{d - 1}$.
By Proposition \ref{max princip}, $u(\beta, x) > u(\alpha, x)$, so $\mu$ is a positive measure.
If $(k', x') \in K' \times \RR^{d - 1}$ is a different laminar coordinate system, and the transition map carries $\alpha, \beta$ to $\alpha', \beta'$, then
$$\mu'([\alpha', \beta'] \cap K') := u'(\beta', x') - u(\alpha', x') = u(\beta, x_1) - u(\alpha, x_2)$$
for some $x_1, x_2 \in \RR^{d - 1}$. Since $\partial_x u = 0$,
$$u(\beta, x_1) - u(\alpha, x_2) = u(\beta, x_1) - u(\alpha, x_1) = \mu([\alpha, \beta] \cap K).$$
It follows that $\mu$ is transverse, and by construction $\mu$ lifts to $\star |\dif u|$ in $M$.
Therefore $\dif u = \normal_\lambda |\dif u|$ is the Ruelle-Sullivan current for the measured oriented structure we just imposed on $\lambda$.

For the converse, we assume that we are given a measured oriented minimal lamination $\lambda$, which then has a Ruelle-Sullivan current $T$.
Since $\dif T = 0$, we may assume, possibly after replacing $M$ with its universal cover, that $T$ is exact, say $T = \dif u$, and we just need to show that $u$ is $1$-harmonic.
If this is not true, then we can choose an open set $E \subseteq M$ with $C^\infty$ boundary and a function $v \in BV_\cpt(E)$ such that
$$\int_E \star |\dif u + \dif v| < \int_E \star |\dif u| < \infty.$$
Since $v$ has compact support, there exists a collar neighborhood $F \subseteq E$ of $\partial E$ such that for every $y \in \RR$,
$$\partial \{u > y\} \cap F = \partial^* \{u + v > y\} \cap F.$$
But by Theorem \ref{main thm of old paper}, the level sets $\partial \{u > y\}$ are stable minimal, so it follows that
$$|\partial \{u > y\} \cap E| \leq |\partial^* \{u + v > y\} \cap E|.$$
So by the coarea formula (see \todo{\cite{BackusFLG}} for a proof at this regularity),
\begin{align*}
\int_E \star |\dif u| &= \int_{-\infty}^\infty |\partial \{u > y\} \cap E| \dif y \leq \int_{-\infty}^\infty |\partial^* \{u + v > y\} \cap E| \dif y \\
&= \int_E \star |\dif u + \dif v| < \int_E \star |\dif u|
\end{align*}
which is a contradiction.




%%%%%%%%%%%%%%%%%%%%%%%%%%%%%%%
\appendix \section{Convergence along transverse curves}\label{transverse curves}
In this appendix, we show that convergence in the weak topology of measures as we have stated it (that is, the weak topology on the Ruelle-Sullivan currents) is equivalent to the formulation that is more popular among topologists, in terms of transverse curves.

\begin{definition}
	Let $\lambda$ be an oriented Lipschitz lamination. A curve $\gamma$ is \dfn{positively transverse} to $\lambda$ if there exists an (oriented, Lipschitz) laminar atlas $(F_\alpha)$ for $\lambda$ such that if $\gamma(t) \in \supp \lambda$, then $(k \circ F_\alpha \circ \gamma)' > 0$.
\end{definition}

Under these hypotheses, a transition map $\psi_{\alpha \beta}$ between two co-oriented laminar flow boxes for $\lambda$ will not send a tangent vector to a fiber of a point of $K_\alpha$ to a vector which is not tangent to any fiber of $K_\beta$.
Therefore if $F_\alpha$ witnesses that $\gamma$ is positively transverse, then so does $F_\beta$, and hence the maximal oriented Lipschitz laminar atlas for $\lambda$ witnesses that $\gamma$ is positively transverse.

Now if $\gamma$ is a positively transverse curve with values in the image of $F_\alpha$, and $\mu$ is a transverse measure, we can set $\gamma^! \mu(E)$ for a Borel set $E \subseteq I$ to be
$$\gamma^! \mu(E) = \mu_\alpha(\gamma(E)).$$
By construction, $k \circ F_\alpha \circ \gamma|_{\gamma^{-1}(\supp \lambda)}$ is a homeomorphism onto its image $\subseteq K_\alpha$, and $\gamma^! \mu$ is supported on $\gamma^{-1}(\supp \lambda)$, so even though it is not in general possible to pull back measures, $\gamma^! \mu$ makes sense as a measure.
The measure-preserving property of the maximal measured oriented Lipschitz laminar atlas $(F_\alpha)$ implies that $\gamma^! \mu$ does not depend on the flow box $F_\alpha$, and so by countable additivity we may extend the definition of $\gamma^! \mu$ to every positively transverse curve $\gamma$, including those not taking values in a flow box.

\begin{proposition}\label{characterization of measure convergence}
	Let $(\lambda_n, \mu_n)$ and $(\lambda, \mu)$ be oriented measured minimal laminations. Then $(\lambda_n, \mu_n) \to (\lambda, \mu)$ iff for every positively transverse curve $\gamma$ to $\lambda$ defined in a small neighborhood of $\supp \lambda$, $\gamma$ is eventually transverse to $\lambda_n$ and $\gamma^! \mu_n \to \gamma^! \mu$ in the weak topology of measures.
\end{proposition}
\begin{proof}
	We first observe that for a transverse measure $\nu$ and $f \in C(I)$ with $f \circ \gamma^{-1}$ supported in the image of a Lipschitz flow box $F_\alpha$,
	$$\int_I f \dif(\gamma^! \nu) = \int_{K_\alpha} F_\alpha^{-1}(f(\gamma^{-1}(k))) \dif \nu_\alpha(k).$$
	(A Lipschitz atlas exists by Proposition \ref{regularity theorem}.)
	However, we may choose $f(\gamma^{-1}(k))$ to be equal to the integral over $\{k\} \times \RR^{d - 1}$ of a $d-1$-form $\varphi$, and conversely, given a $d-1$-form, we may define $f$ by
	$$f(\gamma^{-1}(k)) := \int_{\{k\} \times \RR^{d - 1}} \varphi.$$
	It follows that
\begin{equation}\label{transverse curves vs ruelle sullivan}
	\int_I f \dif(\gamma^! \nu) = \int_{K_\alpha} \int_{\{k\} \times \RR^{d - 1}} \varphi \dif \nu_\alpha(k).
\end{equation}

	We now prove that $(\lambda_n, \mu_n) \to (\lambda, \mu)$ and $\gamma$ positively transverse to $\lambda$ implies $\gamma$ eventually positively transverse to $\lambda_n$.
	If $\gamma$ is positively transverse to $\lambda$, then $g(\normal_\lambda, \gamma') > 0$ on $\supp \lambda$.
	We will later prove (in Proposition \ref{regularity theorem}) that $\normal_\lambda$ admits a Lipschitz extension, which can be chosen so that $g(\normal_\lambda, \gamma') > 0$ holds everywhere.
	Then Lemma \ref{convergence of normals} gives the desired result.

	So if $(\lambda_n, \mu_n) \to (\lambda, \mu)$, and $\gamma$ is positively transverse to $\lambda$, then it is also positively transverse to $\lambda_n$, and application of (\ref{transverse curves vs ruelle sullivan}) with $\nu = \mu_n$ and $\nu = \mu$ shows that $\gamma^! \mu_n \to \gamma^! \mu$.
	The converse is similar.
\end{proof}


\printbibliography

\end{document}
