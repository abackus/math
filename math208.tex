\documentclass[12pt]{report}
\usepackage[utf8]{inputenc}
\usepackage[margin=1in]{geometry} 
\usepackage{amsmath,amsthm,amssymb}
\usepackage{mathrsfs}

\usepackage{enumitem}
%\usepackage[shortlabels]{enumerate}
\usepackage{tikz-cd}
\usepackage{mathtools}
\usepackage{amsfonts}
\usepackage{amscd}
\usepackage{makeidx}
\usepackage{enumitem}
\title{$C^*$-algebras}
\author{Aidan Backus}
\date{December 2019}


\newcommand{\NN}{\mathbb{N}}
\newcommand{\ZZ}{\mathbb{Z}}
\newcommand{\QQ}{\mathbb{Q}}
\newcommand{\RR}{\mathbb{R}}
\newcommand{\CC}{\mathbb{C}}
\newcommand{\CP}{\mathbb{CP}}
\newcommand{\DD}{\mathbb{D}}

\newcommand{\AAA}{\mathcal A}
\newcommand{\BB}{\mathcal B}
\newcommand{\HH}{\mathcal H}

\newcommand{\CVect}{\mathbf{Vect}_\CC}
\newcommand{\Grp}{\mathbf{Grp}}

\newcommand{\Aut}{\operatorname{Aut}}
\newcommand{\Cantor}{\mathcal{C}}
\newcommand{\D}{\mathcal{D}}
\newcommand{\card}{\operatorname{card}}
\newcommand{\diam}{\operatorname{diam}}
\newcommand{\End}{\operatorname{End}}
\DeclareMathOperator*{\esssup}{ess\,sup}
\newcommand{\FF}{\mathcal{F}}
\newcommand{\GL}{\operatorname{GL}}
\newcommand{\Hom}{\operatorname{Hom}}
\newcommand{\id}{\operatorname{id}}
\newcommand{\Ind}{\operatorname{Ind}}
\newcommand{\interior}{\operatorname{int}}
\newcommand{\lcm}{\operatorname{lcm}}
\newcommand{\Lip}{\operatorname{Lip}}
\newcommand{\MM}{\mathcal M}
\newcommand{\OO}{\mathcal{O}}
\newcommand{\PGL}{\operatorname{PGL}}
\newcommand{\pic}{\vspace{30mm}}
\newcommand{\pset}{\mathcal{P}}
\newcommand{\Rep}{\operatorname{Rep}}
\newcommand{\Res}{\operatorname{Res}}
\newcommand{\Riem}{\mathcal{R}}
\newcommand{\RVect}{\RR\operatorname{-Vect}}
\newcommand{\Sch}{\mathcal{S}}
\newcommand{\SL}{\operatorname{SL}}
\newcommand{\spn}{\operatorname{span}}
\newcommand{\supp}{\operatorname{supp}}

\newcommand{\altrep}{\rho_{\text{alt}}}
\newcommand{\trivrep}{\rho_{\text{triv}}}
\newcommand{\regrep}{\rho_{\text{reg}}}
\newcommand{\stdrep}{\rho_{\text{std}}}

\def\Xint#1{\mathchoice
{\XXint\displaystyle\textstyle{#1}}%
{\XXint\textstyle\scriptstyle{#1}}%
{\XXint\scriptstyle\scriptscriptstyle{#1}}%
{\XXint\scriptscriptstyle\scriptscriptstyle{#1}}%
\!\int}
\def\XXint#1#2#3{{\setbox0=\hbox{$#1{#2#3}{\int}$ }
\vcenter{\hbox{$#2#3$ }}\kern-.6\wd0}}
\def\ddashint{\Xint=}
\def\dashint{\Xint-}

\renewcommand{\Re}{\operatorname{Re}}
\renewcommand{\Im}{\operatorname{Im}}
\newcommand{\dfn}[1]{\emph{#1}\index{#1}}
 
\newtheorem{theorem}{Theorem}[chapter]
\newtheorem{lemma}[theorem]{Lemma}
\newtheorem{proposition}[theorem]{Proposition}
\newtheorem{corollary}[theorem]{Corollary}
\newtheorem{axiom}[theorem]{Axiom}
\theoremstyle{definition}
\newtheorem{definition}[theorem]{Definition}
\newtheorem{remark}[theorem]{Remark}
\newtheorem{example}[theorem]{Example}
\newtheorem{exercise}[theorem]{Exercise}
\newtheorem{problem}[theorem]{Problem}

\makeindex
\begin{document}

\maketitle

\tableofcontents

\chapter{Preliminaries}
Here we treat functional analysis in a high level of abstraction.

Throughout these notes, we mean by $f \preceq g$ that there is a universal constant $C > 0$ such that $f \leq C g$.

\section{Locally convex spaces}
    Fix a vector space $V$.
\begin{definition}
    $V$ is said to be a \dfn{topological vector space} if it is equipped with a topology for which addition and multiplication are continuous.
\end{definition}
\begin{definition}
    $V$ is said to be \dfn{locally convex} if $V$ is equipped with a family of seminorms $P_\alpha$ and the initial topology with respect to the $P_\alpha$.
\end{definition}
    This is the smallest topology containing the open sets $P_\alpha([0, \varepsilon))$ for each $\alpha$ and each $\varepsilon > 0$ and which is translation-invariant.
    
    The most useful examples of locally convex spaces are Banach spaces.
\begin{definition}
    $V$ is said to be a \dfn{Banach space} if $V$ is equipped with the topology arising from a complete norm.
\end{definition}
\begin{definition}
    If $V$ is a topological vector space, then the \dfn{dual space} of $V^*$ is the space of \emph{continuous} linear maps $V \to \CC$.
\end{definition}
\begin{definition}
    Let $W$ be a Banach space and define a norm on $\Hom(V, W)$ by
    $$||T|| = \sup_{||v|| \leq 1} ||Tv||.$$
\end{definition}
    So $V^*$ is a normed space, $V^* \subseteq \Hom(V, \CC)$. In general it is very difficult to construct elements of $V^*$. In general we cannot guarantee constructively that $V^*$ is nontrivial. On the other hand, it is often impossible to construct linear functions which are discontinuous (for example, any linear functional on a Banach space will be continuous if it was constructed without the axiom of choice). 
\begin{definition}
    A function $f: V \to \CC$ is said to be \dfn{sublinear} if it obeys the triangle inequality and if for each $c > 0$ and $x \in V$, $f(cx) = cf(x)$. 
\end{definition}
    Obviously seminorms are sublinear. Minkowski gauges are another useful example.
\begin{definition}
    Let $K \subseteq V$. Then:
\begin{enumerate}
    \item $K$ is \dfn{convex} if for each $x, y\in K$, $c \in [0, 1]$, $cx + (1-c)y \in K$.
    \item $K$ is \dfn{balanced} if for each $c \in [0, 1]$, $cK \subseteq K$.
\end{enumerate}
    If $K$ is balanced and convex, then the \dfn{Minkowski gauge} of $K$ is the functional
    $$p_K(x) = \inf_{cK \ni x} c.$$
\end{definition}
    Notice that the balanced condition suggests that $K$ needs to be close to the origin. Moreover, Minkowski gauges are sublinear.
    
    Sublinear estimates allow us to construct functionals using the axiom of choice, while still guaranteeing that they are continuous.
\begin{theorem}[Hanh-Banach]
    \index{Hanh-Banach theorem}
    Assume that $p: V \to \CC$ is sublinear, $U \subset V$ a subspace, and $f: U \to \CC$ a linear functional. If $f$ is dominated by $p$, i.e. for each $x \in U$, $|f(x)| \leq |p(x)|$, then $f$ extends to $V$.
\end{theorem}
    In general the extension of $f$ will only be unique in case $U$ is dense. So we have to use the axiom of choice to construct $f$.
\begin{proof}
    The extension to the complex case is trivial so we replace $\CC$ with $\RR$. Assume that $f$ is defined on a space $W$, $U \subseteq W \subset V$. Choose $v \in V \setminus W$ and define $f(v)$ such that for each $w \in W$ and $s,t \geq 0$,
$$\frac{p(w-sv)}{s} \leq h(v) \leq \frac{p(w+tv) - f(w)}{t}.$$
    This is always possible because
    $$f((t+s)w) \leq p((t+s)w) = p((t+s)w + tsv - tsv) \leq p(sw + stv) + p(tw - stv)$$
    so
    $$\frac{f(w) - p(w - sv)}{s} \leq \frac{p(w+tv) - f(w)}{t}.$$
    Therefore for any $W$ and $v$ we can extend $f$ to $W + v$. If $\mathcal W$ is the family of subspaces of $V$ on which $f$ is defined and $\mathcal C \subset \mathcal W$ is a chain, then $\mathcal C$ therefore has an upper bound. Since $U \in \mathcal W$, Zorn's lemma implies that $\mathcal W$ has a maximal element, which is clearly $V$.
\end{proof}
    In case $p$ is the norm of $V$, this implies that $V^*$ is nontrivial. The Hanh-Banach theorem also has a useful geometric interpretation.
\begin{theorem}[Hanh-Banach separation theorem]
    \index{Hanh-Banach separation theorem}
    Let $\RR$ be the scalar field and $A, B \subset V$ be convex, nonempty, and disjoint. If $A$ is open then there is a $\varphi \in V^*$ and $t \in \RR$ such that for every $a \in A$ and $b \in B$,
    $$\varphi(a) < t \leq \varphi(b).$$
\end{theorem}
\begin{proof}
    Choose $a_0 \in A$ and $b_0 \in B$, and let $C = A - B + b_0 - a_0$. Then $0 \in C$, $C$ is convex, and $C = \bigcup_{b \in B} A - b + b_0 - a_0$, so $C$ is open. If $x = b_0 - a_0$, then $x \notin C$.
    
    By the Hanh-Banach theorem, choose a $\varphi \in V^*$ such that $\varphi(x_0) = 1$ and $\varphi < 1$ on $C$. Given $a \in A$ and $b \in B$ we have
    $$\varphi(a) < \varphi(b) + \varphi(a_0) - \varphi(b_0) + 1 = \varphi(b).$$
    If $t = \inf_B \varphi$, then this gives, for every $a$ and $b,$
    $$\varphi(a) \leq t \leq \varphi(b).$$
    Since $A$ is open, $\varphi(A)$ is open so the claim holds.
\end{proof}
    In particular, linear functionals separate points.

    Let's consider more properties of convexity. Let the scalar field be $\RR$ and let $S \subset V$ be nonempty, compact, and convex.
\begin{definition}
    A \dfn{face} of $S$ is a nonempty, compact, convex set $K \subseteq S$ such that for each $x \in K$, if there are $y_1, y_2 \in S$ and $c \in (0, 1)$ such that $x = cy_1 + (1-c)y_2$, then $y_1, y_2 \in K$. If $|K| = 1$, then $K$ is called an \dfn{extreme point}.
\end{definition}
\begin{definition}
    If $X \subset V$ is a set, the \dfn{convex hull} of $X$ is the smallest set containing $X$ which is closed and convex.
\end{definition}
    For example, if $S$ is a convex polygon, then the extreme points of $S$ are its vertices, and $S$ is the convex hull of its vertices. The Krein-Milman theorem says that this phenomenon happens even in infinite dimensions.
\begin{theorem}[Krein-Milman]
    \index{Krein-Milman theorem}
    $S$ is the convex hull of its extreme points.
\end{theorem}
\begin{proof}
    Assume that $S'$ is the convex hull of the extreme points of $S$. Then $S' \subseteq S$, so $S'$ is compact. If there is an $x_0 \in S \setminus S'$, then since $V^*$ separates points, there is a $\lambda \in V^*$ such that $\lambda(S') < \lambda(x_0)$. If $C = \max \lambda(x_0)$, then $\varphi^{-1}(C) \cap S$ contains no extreme points of $S$. We can contradict this by showing that every convex compact set has an extreme point.

    Let $\mathcal F$ be the set of all faces of $S$. Clearly $S \in \mathcal F$ so $\mathcal F$ is nonempty. If $\mathcal C \subset \mathcal F$ is a chain, then $\bigcap \mathcal C$ is a face, so by Zorn's lemma $\mathcal F$ has a minimal element $S_0$.
    
    Let $\varphi \in V^*$. Since $S_0$ is convex, it is connected, so $\varphi(S_0)$ is compact and connected. In particular, $\varphi(S_0) = [a, b]$ for some $a \leq b$. So $\varphi^{-1}(b) \cap S_0$ is nonempty, convex, and compact. If $x \in \varphi^{-1}(b) \cap S_0$,
    $$x = ty_1 + (1-t)y_2,$$
    then $y_1,y_2 \in S_0$. Therefore $\varphi(x) = b = c\varphi(y_1) + (1-c)\varphi(y_2)$, so $\varphi(y_1), \varphi(y_2) \geq c$. Therefore $y_1, y_2 \in \varphi^{-1}(b) \cap S_0$, so $\varphi^{-1}(b) \cap S_0$ is a face and by minimality, $\varphi^{-1}(b) \cap S_0 = S_0$. So $\varphi(S_0) = b$. Since $\varphi$ was arbitrary and $V^*$ separates points, $|S_0| = 1$. So $S$ has an extreme point. 
\end{proof}


\section{Hilbert spaces}
    \begin{definition}
    A \dfn{Hilbert space} $V$ is a Banach space whose norm arises from an inner product.
    \end{definition}
    The basic fact about Hilbert spaces $V$ is that $V^* = V$.
\begin{theorem}[Riesz representation theorem]
    \index{Riesz representation theorem for Hilbert spaces}
    The association
\begin{align*}
    V &\to V^*\\
    v &\mapsto (w \mapsto \langle v, w\rangle)
\end{align*}
    is a surjective isometry.
\end{theorem}
\begin{proof}
    Evaluating $w \mapsto \langle v, w\rangle$ at $v$, we see $||v|| = ||v||_{op}$. So we just have to check surjectivity. Let $\varphi \in V^*$, and $F = \ker \varphi$. If $F = 0$ we're done; otherwise $F^\perp$ is nonempty. Let $z \in F^\perp$ and $\alpha = \varphi(z)/||z||$. Then for any $x \in V$,
    \begin{align*}
        \langle x, \alpha z\rangle  &=  \left\langle x - \frac{\varphi(x)}{\varphi(z)} z, \alpha z\right\rangle + \left\langle \frac{\varphi(x)}{\varphi(z)}z, \alpha z\right\rangle \\&= \left\langle \frac{\varphi(x)}{\varphi(z)}z, \alpha z\right\rangle
            = \frac{\varphi(x)}{\varphi(z)} \varphi(z) = \varphi(x).
    \end{align*}
\end{proof}


\section{Bochner integration}
    Now we fix a Banach space $B$ and a measure space $(X, \Sigma, \mu)$. Recall that the \dfn{Caratheodory construction} is the standard way of building $(X, \Sigma, \mu)$: we define a semiring $\Sigma_0$ of sets (i.e. a family of sets closed under finite intersection and subsets of finite disjoint unions) and a countably additive function $\mu$ on $\Sigma_0$, which then extends to an outer measure $\mu^*$ on the power set $\pset(X)$. If $E \subseteq X$ satisfies the \dfn{Caratheodory criterion}, i.e. that for all $F \subseteq X$,
    $$\mu^*(F) = \mu^*(F \cap E) + \mu^*(F \setminus E),$$
    then we declare that $E$ is measurable. The measurable sets form a $\sigma$-algebra $\Sigma$ on which $\mu^*$ is outer measurable (note that $\mu^*$ did not have to be constructed from a semiring for this step to work; any outer measure will do) and we define the restriction $\mu$ of $\mu^*$ to $\Sigma$ to be the desired outer measure.
\begin{definition}
    A $B$-valued \dfn{integrable simple function} is a finite linear combination of functions \begin{align*}
        \chi_E^b: X &\to B\\
        E \ni x &\mapsto b\\
        E^c \ni x &\mapsto 0
    \end{align*} where $E$ is a measurable set with $||\mu(E)|| < \infty$, $b \in B$.
    
    The integral of a $B$-valued ISF $f = \sum_n \chi_{E_n}^{b_n}$ is
    $$\int_X f ~d\mu = \sum_n b_n \mu(E_n)$$
    and the $L^1$-norm is $||f||_{L^1} = \int_X |f| ~d\mu$.
\end{definition}
    Then $L^1$ is naturally the Cauchy completion of the ISF.
\begin{definition}
    A function $X \to B$ is a $B$-valued \dfn{integrable function} if it lies in $L^1$.
\end{definition}
\begin{definition}
    For $p \in (1, \infty)$, the $L^p$ norm of $f: X \to B$ is
    $$||f||_{L^p} = \left(\int_X ||f(x)||^p ~d\mu(x)\right)^{1/p}$$
    and the $L^\infty$ norm is $||f||_{L^\infty} = \lim_{p \to \infty} ||f||_{L^p} = \esssup ||f||$.
\end{definition}
    The usual Lebesgue convergence theorems hold:
\begin{theorem}[Lebesgue convergence theorems]
    Let $\{f_n\}$ be a pointwise convergent sequence of integrable functions. Then:
\begin{enumerate}
    \item If each $f_n \leq f_{n+1}$,
    $$\lim_n \int f_n = \int \lim_n f_n < \infty.$$
    \item If there is an integrable function $g > 0$ such that every $|f_n| \leq g$,
    $$\lim_n \int f_n = \int \lim_n f_n \leq g.$$
    \item $$\int \liminf_n f_n \leq \liminf_n \int f_n.$$
\end{enumerate}
\end{theorem}

    Now let's make some estimates which will actually prove that the $L^p$-norm is a norm, besides being useful later.
\begin{theorem}[Jensen's inequality]
    \index{Jensen's inequality}
    Let $f: \RR \to \RR$ be convex and $g$ an integrable function. Then
    $$f\left(\int g\right) \leq \int f \circ g.$$
\end{theorem}
\begin{theorem}[Holder's inequality]
    \index{Holder's inequality}
    Let
    $$\frac{1}{p} + \frac{1}{q} = 1.$$
    Then $||fg||_{L^1} \leq ||f||_{L^p} ||g||_{L^q}$.
\end{theorem}
\begin{proof}
    The mapping $x \mapsto x^p$ is convex so if $f, g \geq 0$,
\begin{align*}
    \int fg
        &= \left(\int g^q\right) \int fg^{1-q} \frac{g^q}{\int g^q}
        \leq \left(\int g^q\right) \left(\int f^p g^{p(1-q)}\frac{g^q}{\int g^q}\right)^{1/p}\\
        &= \left(\int g^q\right) \left(\left(\int g^q\right) \left(\int f^p\right)\right)^{1/p}
        \leq \left(\int f^p\right)^{1/p} \left(\int g^q\right)^{1/q}.
\end{align*}
\end{proof}
    Notice that Holder's inequality implies that $L^2$ is a Hilbert space with inner product
    $$\langle f, g\rangle = \int fg.$$
\begin{theorem}[Minkowski's inequality]
    \index{Minkowski's inequality}
    Let
    $$\frac{1}{p} + \frac{1}{q} = 1.$$
    Then
    $$||f + g||_{L^p} \leq ||f||_{L^p} + ||g||_{L^p}.$$
\end{theorem}
\begin{proof}
    By Holder's inequality,
    \begin{align*}
        \int |f+g|^p
            &= \int |f+g||f+g|^{p-1}
            \leq \int (|f| + |g|) |f+g|^{p-1}
            \\&\leq \left(\left(\int |f|^p \right)^{1/p} + \left(\int |g|^p\right)^{1/p}\right)\left(\int |f+g|^{(p-1)\left(\frac{p}{p-1}\right)} \right)^{1-\frac{1}{p}}\\
            &= (||f||_{L^p} + ||g||_{L^p}) \frac{||f+g||_{L^p}^p}{||f+g||_{L^p}}.
    \end{align*}
\end{proof}
    Now we discuss change of variables.
\begin{definition}
    Let $\nu$ be a measure. Then
\begin{enumerate}
    \item $\nu$ is \dfn{absolutely continuous} with respect to $\mu$ if for every measurable set $A$, $\mu(A) = 0$ implies $\nu(A)$.
    \item $\nu$ is \dfn{singular} with respect to $\mu$ if there are disjoint measurable sets $A, B$ such that $X = A \cap B$, $\nu(A) = 0$ and $\mu(A) = 0$.
    \item If there is a measurable function $f$ such that for every measurable set $A$,
    $$\nu(A) = \int_A f ~d\mu,$$
    then $f$ is the \dfn{Radon-Nikodym derivative} of $\nu$, written
    $$f = \frac{d\nu}{d\mu}.$$
\end{enumerate} 
\end{definition}
\begin{theorem}[Radon-Nikodym]
    \index{Radon-Nikodym theorem}
    Let $\mu$ be $\sigma$-finite and $\nu$ be a positive measure. Then there is a unique decomposition $\nu = \nu_a + \nu_s$ such that $\nu_a$ is absolutely continuous and $\nu_s$ is singular (with respect to $\mu$). Moreover, $\nu_a$ has a Radon-Nikodym derivative.
\end{theorem}
    In particular, if $\nu$ was already absolutely continuous, then $\nu_s = 0$ and $\nu$ has a Radon-Nikodym derivative.
\begin{proof}
    Uniqueness is obvious. First assume $\mu(X) < \infty$. Then $\mu + \nu$ is finite, so $L^\infty(\mu + \nu) \subseteq L^1(\mu + \nu)$. So by the Cauchy-Schwarz inequality, if $f$ is an ISF,
    $$\left|\int f ~d\nu\right| \leq ||f||_{L^1(\nu)} \leq ||f||_{L^1(\mu+\nu)} \preceq ||f||_{L^2(\mu)}.$$
    So $\int \cdot ~d\nu$ is $L^2$-continuous on ISF, hence on $L^2(\mu + \nu)$. So by the Riesz representation theorem, there is a nonnegative $h \in L^1(\mu + \nu)$ such that
    $$\int f ~d\nu = \int \int fh ~d(\mu + \nu)$$
    for any $f \in L^2$. In particular, if $A$ is measurable,
    $$\int_A h ~d(\mu + \nu) = \nu(A) \leq (\mu + \nu)(A).$$
    Without loss of generality we assume $h \leq 1$. If $g \in L^\infty(\nu)$,
    $$\int g ~d\nu = \int gh ~d\mu + \int gh ~d\nu.$$
    So if $Y$ is the set of all $y$ such that $0 \leq h(y) < 1$, it follows that $\mu(Y) = \mu(X)$. By induction,
    $$\int g ~d\nu = \int g(h + \dots + h^n) ~d\mu + \int gh^n ~d\nu.$$
    Since $h \leq 1$, the dominated convergence theorem implies
    $$\int gh^n ~d\nu \to \int_{X \setminus Y} g ~d\nu$$
    and if
    $$f = \frac{h\chi_Y}{1-h}$$
    we have
    $$\int g ~d\nu = \int_Y gf ~d\mu + \int_{X \setminus Y} g ~d\nu$$
    and take $\nu_s(A) = \nu(A \cap (X \setminus Y))$. Then we take
    $$\nu_a(A) = \int_A f ~d\mu$$
    so $f$ is the Radon-Nikodym derivative of $\nu_a$, $\nu_a + \nu_s = \nu$ by taking $g = \chi_A$.
    
    To extend to the $\sigma$-finite case, break up $X$ into countably many finite measure spaces and sum over them.
\end{proof}
    Next we discuss iterated integrals. Given measure spaces $(X, S, \mu)$ and $(Y, T, \nu)$, we need a $\sigma$-algebra on $X \times Y$ and a measure defined on that $\sigma$-algebra. To do this, we use the Caratheodory construction.
\begin{definition}
    If $E \in S$ and $F \in T$, then $E \times F$ is a \dfn{measurable rectangle}. Let $S \otimes T$ denote the smallest $\sigma$-algebra containing the measurable rectangles, and on for each measurable rectangle, define a countably additive function by
    $$\mu \otimes \nu(E \times F) = \mu(E) \nu(F).$$
\end{definition}
    By the monotone convergence theorem $d\nu$, $\mu \otimes \nu$ is countably additive. So the Caratheodory construction gives rise to a measure $\mu \otimes \nu$ whose measurable sets include $S \otimes T$ (in fact, is the completion of $S \otimes T$).
\begin{definition}
    The measure space $(X \times Y, S \otimes T, \mu \otimes \nu)$ is the \dfn{product measure space} of $(X, S, \mu)$ and $(Y, T, \nu)$.
\end{definition}
    Straight from the definitions, we know that for every measurable rectangle $E \times F$,
    $$\int \chi_{E \times F} ~d(\mu \otimes \nu) = \iint \chi_{E \times F} ~d\mu ~d\nu = \iint \chi_{E \times F} ~d\nu ~d\mu.$$
    
    For a function $f$ defined on $X \times Y$ we define $f^y(x) = f(x, y)$ and $f_x(y) = f(x, y)$. For a set $G \subseteq X \times Y$, we define $G^y = \{x \in X: (x, y) \in G\}$ and $G_x = \{y \in Y: (x, y) \in G\}$. 
\begin{theorem}[Fubini]
    \index{Fubini's theorem}
    Let $f \in L^1(\mu \otimes \nu)$ and assume $\mu \otimes \nu$ is $\sigma$-finite. Then for almost every $y$, $f^y \in L^1(\mu)$. Moreover, the function
    $$F(y) = \int f^y ~d\mu$$
    has $F \in L^1(\nu)$, and
    $$\int f ~d(\mu \otimes \nu) = \iint f^y ~d\mu ~d\nu = \iint f_x ~d\nu ~d\mu.$$
\end{theorem}
    The assumption of $\sigma$-finiteness is not optional here, and Fubini's theorem can fail for large cardinality measure spaces.
\begin{definition}
    Let $M$ be a family of subsets of $X$ such that for every countable chain of $A_n$ in $M$ and $\bigcup_n A_n = A$ or $\bigcap_n A_n = A$, $A \in M$. Then we say $M$ is a \dfn{monotone class}.
\end{definition}
    If $R$ is a ring of sets, then the smallest monotone class $M$ containing $R$ is also a ring, and it is not hard to see that $M$ is the smallest $\sigma$-algebra containing $R$.
\begin{lemma}
    Let $G \in S \otimes T$. Then:
\begin{enumerate}
    \item $G_x \in T$ and $G^y \in S$.
    \item $x \mapsto \nu(G_x)$ and $y \mapsto \mu(G^y)$ are measurable.
    \item One has
    $$\mu \otimes \nu(G) = \int (x \mapsto \nu(G_x)) ~d\mu(x) = \int (y \mapsto \mu(G^y)) ~d\nu(y) = \iint \chi_G ~d\mu ~d\nu.$$
\end{enumerate}
\end{lemma}
\begin{proof}
    This is obvious if $G$ is a measurable rectangle. We shall show that the algebra of sets on which this claim holds is a monotone class, hence a $\sigma$-algebra. Clearly if $\bigcup_n G_n = G$ then $G$ has the property. Given $x \in X$, $\bigcup_n (G_n)_x = G_x$, so $G_x \in T$. Therefore the chain of functions $x \mapsto \nu(G_n)_x$ converges to $x \mapsto \nu(G_x)$ which is therefore measurable. So by the monotone convergence theorem,
    $$\lim_n \mu \otimes \nu(G_n) = \lim_n \int (x \mapsto \nu((G_n)_x) ~d\mu(x) = \int (x \mapsto \nu(G_x) ~d\mu(x) = \mu \otimes \nu(G).$$
    So this algebra is closed under ascending chains. The proof in the other direction is similar but you have to start by assuming that $\mu \otimes \nu(G_1) < \infty$.
\end{proof}
\begin{lemma}
    Let $f \geq 0$ be $S \otimes T$-measurable. Then
    $$\int f ~d\mu \otimes \nu = \iint f ~d\mu ~d\nu.$$
\end{lemma}
\begin{proof}
    Let $\{f_n\}$ be a chain of ISFs. This claim is obvious for ISF, so the monotone convergence theorem on the $f_n^y$ for each $y \in Y$.
\end{proof}
\begin{theorem}[Tonelli]
    \index{Tonelli's theorem}
    If $f$ is $S \otimes T$-measurable, $g(x) = ||f(x)||$, $g^y \in L^1(\mu)$, and $(y \mapsto \int g^y ~d\mu) \in L^1(\nu)$, then $f \in L^1(\mu \otimes \nu)$. 
\end{theorem}
\begin{proof}
    Clear by the lemmata.
\end{proof}
\begin{proof}[Proof of Fubini's theorem]
    Let $g(x, y) = ||f(x, y)||$. Then if $\{f_n\}$ is a sequence of ISF converging to $f$, $g$ dominates the $f_n$. Apply the dominated convergence theorem twice, once for each integral.
\end{proof}

\section{Duality}
    Fix a normed space $V$. We consider properties of $V^*$. Since $\CC$ is complete, $V^*$ is a Banach space; in particular, $V^{**}$ is a Banach space. So we can always embed $V$ in a Banach space by the mapping
\begin{align*}
    V &\to V^{**}\\
    v &\mapsto (\varphi \mapsto \varphi(v)).
\end{align*}
    However, $V^{**}$ is rarely the completion of $V$ if $V$ is infinite-dimensional. Moreover, the topology of $V^*$ is a bit awkward to work with, since a convergence in operator norm is much stronger than convergence pointwise.
\begin{definition}
    The \dfn{weakstar topology} of $V^*$ is the initial topology such that every evaluation $\varphi \mapsto \varphi(v)$ is continuous.
\end{definition}
    In other words, the weakstar topology is the topology of pointwise convergence.
\begin{theorem}[Banach-Alaoglu]
    \index{Banach-Alaoglu theorem}
    Let $B$ be the closed unit ball of $V^*$. Then $B$ is weakstar compact.
\end{theorem}
    Like the Hanh-Banach and Krein-Milman theorems, the proof of Banach-Alaoglu uses the axiom of choice. However, the Banach-Alaoglu theorem is not really nonconstructive, since if $V$ is separable, we can use a diagonalization argument to prove it instead. Banach-Alaoglu generalizes to locally convex spaces.
\begin{proof}
    Let
    $$D_v = \{z \in \CC: |z| \leq ||v||\}$$
    and $D = \prod_{v \in V} D_v$. By Tychonoff's theorem, $D$ is compact. Moreover, there is a natural embedding
\begin{align*}
    \iota: V^* &\to D\\
    f &\mapsto \{f(v)\}_{v \in V}.
\end{align*}
    Since the product topology is the topology of pointwise convergence, $\iota$ is a homeomorphism $V^* \to \iota(V^*)$. So we just need to show that $\iota(V^*)$ is closed. So let $\{\{f_\alpha(v)\}_{v \in V}\}_{\alpha \in A}$ be a net in $D$, which converges to a $\{\varphi_v\}_{v \in V}$. Then $f(v) = \varphi_v$ is a linear functional and $f_\alpha \to f$ pointwise so $\{\varphi_v\}_{v \in V} \in \iota(V^*)$.
\end{proof}
    Now we compute the duals of the main examples of Banach spaces we have presented so far.
\begin{theorem}
    Let $p, q \in [1, \infty]$ and assume $\mu$ is $\sigma$-finite.
    $$\frac{1}{p} + \frac{1}{q} = 1.$$
    Then $(L^p(\mu))^* = L^q(\mu)$.
\end{theorem}
    Actually, this theorem is true without the $\sigma$-finiteness; however, it becomes much more difficult.
\begin{proof}
    For $g \in L^q$, one has $||g||_{p^*} \leq ||g||_q$ by Holder's inequality and by taking larger and larger measurable sets $E$ and considering $\int_E g$, we check $||g||_{p^*} \geq ||g||_q$. So we just need to show that the map $L^q \to L^{p^*}$ is surjective.
    
    If $h \in L^p$ and $X$ splits into finite measure spaces $X_k$ we put $h_k = \chi_{X_k}h$, so $\sum_k h_k = h$ in $L^p$ by the dominated convergence theorem. If $\varphi \in L^{p^*}$ then $\varphi(\sum_k h_k) = \sum_k \varphi(h_k)$ so we might as well assume $X = X_1$, viz. $\mu(X) < \infty$. Then $L^\infty \subseteq L^p$, so $\varphi \in (L^\infty)^*$. We can define an absolutely continuous measure $\nu$ by $\nu(A) = \varphi(\chi_A)$, and by the Radon-Nikodym theorem, there is a Radon-Nikodym derivative $f$ of $\nu$.
    
    Let $Y_n = \{x \in X: |f(x)| \leq n\}$ and let $g = f/|f|^{q-2}$, where $g(x) = 0$ if $f(x) = 0$, and $g_n = \chi_{Y_n}g_n$. Then $|g|^p = |f|^q$ and
    $$\int_{Y_n} |f|^q = \int g_nf = \varphi(g_n) \preceq ||g_n||_p \preceq ||f_n||_{L^p(Y_n)}.$$
    So $||f||_{L^q(Y_n)} < \infty$, and by the monotone convergence theorem, $f \in L^q$.
\end{proof}

\section{Vector lattices}
    We now consider the natural order structure of a space.
\begin{definition}
    A \dfn{vector lattice} is a vector space $V$ equipped with a partial order $\leq$ which is translation-invariant such that $(V, \leq)$ is a lattice, and such that for each $c \geq 0$ and $x \leq y$, $cx \leq cy$.
\end{definition}
    Recall that a lattice is just a poset which is closed under finite joins $\vee$ (suprema) and meets $\wedge$ (infima). Actually, we just need to check that $V$ is a semilattice, since multiplication by $-1$ implies that a semilattice is already a lattice.
    
    If $V$ is a vector lattice and $v \in V$, we define $f_\pm = \pm f \vee 0$. Then $f = f_+ - f_-$ and we define the absolute value (or valuation) $|f| = f_+ + f_-$.
\begin{definition}
    A \dfn{Banach lattice} is a vector lattice $V$ which is a Banach lattice, such that $|x| \leq |y|$ whenever $||x|| \leq ||y||$.
\end{definition}
\begin{example}
    A function space mapping into $\RR$ is usually a Banach lattice with the natural ordering, $f \leq g$ iff for every $x$, $f(x) \leq g(x)$. For example, $C(X)$ is a lattice. Spaces of operators are Banach lattices as well, whose positive elements are precisely the positive operators; as are spaces of signed measures, where the positive measures are the positive elements.
\end{example}
\begin{theorem}
    Let $V$ be a Banach lattice. There is a natural ordering on $V^*$, such that $f \in V^*$ is positive iff for each positive $v \in V$, $f(v) \geq 0$, and such that $f \leq g$ iff for every positive $v \in V$, $f(v) \leq g(v)$.
\end{theorem}
\begin{proof}
    Take the definition of positive functionals as in the statement of the theorem. If $f$ and $-f$ are both positive, each $v = v_+ - v_-$ has $f(v_+) \geq 0$ but $f(v_-) \leq 0$. So $f(v) = 0$. Since $v$ was arbitrary, $f = 0$.
    
    Given $f \in V^*$, define
    $$f^+(v) = \sup_{0 \leq x \leq v} f(x)$$
    for $v \geq 0$. Then $f^+ \geq f$, and $f^+$ is finite because if $x \leq v$, $|f(x)| \leq ||f|| ||v||$. Moreover, if $v, w \geq 0$, it is easy to check $f^+(v+w) = f^+(v) + f^+(w)$. So $f^+$ is positive-linear, so extends to all of $V$ and so $f^+ \in V^*$.
    
    Clearly $f^+ - f \geq 0$. We need to show this is optimal, i.e. $f^+ = f \vee 0$. Assume $g \geq f \vee 0$. Then for $0 \leq x \leq v$, $f(x) \leq g(x) \leq g(v)$, so taking the $\sup$ over $x$ we have $f^+(v) \leq g(v)$. The other direction is similar. So $f^+ = f \vee 0$.
\end{proof}
    Fix a compact Hausdorff space $X$, $|X| \geq 2$ (so in particular, every set which separates points is nonempty). Let us now study the behavior of sublattices of $C(X) = C(X \to \RR)$.
\begin{theorem}[Dini]
    \index{Dini's theorem}
    Let $L$ be a sublattice of $C(X)$, and define $g(x) = \inf_{f \in L} f(x)$. For each $\varepsilon > 0$, there exists a $h \in L$ such that $g \leq h \leq g + \varepsilon$.
\end{theorem}
\begin{proof}
    For each $f \in L$ let $U_f = \{x \in X: f(x) - g(x) \leq \varepsilon\}$. Then the $U_f$ are an open cover of $X$, which has a finite subcover by functions $f_1, \dots, f_k$. Take $h = \bigwedge_{j \leq k} f_j$.
\end{proof}
    When can a lattice be used to approximate any function in $C(X)$? A necessary condition is that the lattice strongly separates points. This turns out to be sufficient as well.
\begin{definition}
    A set $A \subseteq C(X)$ \dfn{separates points} if for each $x, y \in X$, there is an $f \in A$ such that $f(x) \neq f(y)$. If, in addition, the constant functions $\RR \subseteq A$, then $A$ \dfn{strongly separates points}.
\end{definition}
\begin{theorem}[Stone-Weierstrass]
    \index{Stone-Weierstrass theorem}
    If $L \subseteq C(X)$ is a sub-vector lattice or a subalgebra which strongly separates points, then $L$ is dense in $C(X)$.
\end{theorem}
    The lattice case is also called the \dfn{Kakutani-Krein theorem}.
\begin{lemma}
    \label{sw lem 1}
    Let $L$ be a sublattice of $C(X)$ which separates points and is closed under multiplication and addition by elements of $\RR$. Then if $B \subseteq X$ is compact, $p \in X \setminus B$, and $a, b \in \RR$, there is a $g \in L$ such that $g \geq a$, $g(p) = a$ and $g > b$ on $B$.
\end{lemma}
\begin{proof}
    For each $x \in B$ there exists $g_x \in L$ such that $g_x(p) = a$ and $g_x(x) = b+1$. Let $U_x = \{y\in X: g_x(y) > b\}$. Since $x \in U_x$, the $U_x$ are an open cover of $B$ with finite subcover $U_{x_1}, \dots, U_{x_k}$. Take $g = a \vee \bigvee_{j \leq k} g_{x_k}$.
\end{proof}
\begin{lemma}
    \label{sw lem 2}
    Assume that $L$ is a closed unital subalgebra of $C(X)$. Then $L$ is a lattice.
\end{lemma}
\begin{proof}
    Choose $\varepsilon > 0$ and apply the classical Weierstrass theorem to $[-1, 1]$ to find a polynomial $P_\varepsilon$ which approximates $|\cdot|$ in $L^\infty$-norm by $\varepsilon$. Then for each $f \in L$, we can approximate $|f|$ by $P_\varepsilon \circ f$. Since $L$ is unital, $P_\varepsilon \circ f \in L$. So $|f| \in L$, since $L$ is closed. The lattice operations $\vee$ and $\wedge$ can be expressed in terms of algebra operations $+$ and $\cdot$, and $|\cdot|$, so $L$ is closed under lattice operations.
\end{proof}
\begin{proof}[Proof of Stone-Weierstrass]
    First consider the case that $L$ is a lattice. Given $f \in C(X)$, define $L_f = \{g \in L: g \geq f\}$. Then $L_f$ is a sublattice of $L$. Given $x \in X$, $\delta > 0$, the set $B = \{y \in X: f(y) \geq f(x) + \delta\}$ is closed. Since $X$ is compact, there is an $M > 0$ such that $f < M$. Apply Lemma \ref{sw lem 1} with $a = f(x) + \delta$ and $b = M$, so there is a $g \in L$ such that $g \geq f(x) + \delta$, $g(x) = f(x) + \delta$ and $g > M$ on $B$. So $f \leq g \leq g + \delta$, so $f = \bigwedge L_f$. Therefore by Dini's theorem, there is an $h \in L$ with the desired properties.
    
    For the algebra case, since $L$ strongly separates points, $L$ is unital. Therefore $\overline L$ is a closed unital algebra, $\overline L$ is a closed lattice whose closure is $C(X)$, by Lemma \ref{sw lem 2} and the above case. So $\overline L = C(X)$.
\end{proof}
This even extends to decaying functions on locally compact Hausdorff spaces, by taking the one-point compactification.

\section{Positive Radon measures}
The usual construction of measures by ISF is somewhat unnatural when we have a nice topology, since then we can define integration in terms of continuous functions. Clearly ``nice" in this context implies locally compact Hausdorff; these conditions are also sufficient (though $\sigma$-compactness also helps). Throughout this section, we fix a locally compact Hausdorff space $X$ and consider the space $C_c(X)$ of compactly supported continuous functions $X \to \CC$.

We have not given a topology on $C_c(X)$, so a functional is just an element of $\Hom(C_c(X), \CC)$ for now.
\begin{definition}
    A \dfn{positive Radon measure} on $X$ is a functional on $C_c(X)$.
\end{definition}
Let us prove that a positive Radon measure is actually a measure in a natural way. First, we put a topology on $C_c(X)$. We start by putting the $L^\infty$-topology on $C_c(U)$ for each open set $U \subseteq X$ with compact closure.
\begin{definition}
    The \dfn{inductive limit topology} of $C_c(X)$ is the final (i.e. strongest) topology on $C_c(X)$ such that $\varphi: C_c(X) \to Y$ is continuous provided that for each open set $U \subseteq X$ with compact closure, $\varphi|_{C_c(U)}$ is continuous.
\end{definition}
In other words, the inductive limit topology is the final topology which makes the natural maps $C_c(U) \to C_c(X)$ continuous. A positive Radon measure is continuous for the inductive limit topology, as can be seen by taking an $h \in C_c(X)$ which is $1$ on $U$, so $||\varphi||_{C_c(U)} \leq \varphi(h)$.

Now we need some general facts about locally compact Hausdorff spaces.
\begin{definition}
    A (continuous) \dfn{partition of unity} on a subordinate to an open cover $U_1, \dots, U_n$ is a family of (continuous) functions $f_1, \dots, f_n$ which are compactly supported in $U_i$, such that $\sum_i f_i = 1$.
\end{definition}
\begin{theorem}
    \label{partitions of unity}
    For any finite open cover $\mathcal U$ of a compact set, there is a partition of unity subordinate to $\mathcal U$.
\end{theorem}
\begin{lemma}
    Let $K \subseteq X$ be compact. If $U_1, \dots, U_n$ is an open cover of $K$ there are compact sets $K_1, \dots, K_n$, $K_i \subseteq U_i$, which cover $K$.
\end{lemma}
\begin{proof}
    For each $x \in K$ choose a $j$ such that $U_j \ni x$ and an open set $V_x \ni x$ such that
    $$V_x \subset \overline V_x \subset U_j.$$
    Then the $V_x$ are an open cover of $K$ so they reduce to a finite subcover $V_{x_1}, \dots, V_{x_p}$. For each $k \leq p$ choose a $j_k \leq n$ such that $V_{x_k} \subseteq U_{j_k}$ and let $W_j = \bigcup_{j_k=j} V_k \subseteq U_j$. Then $\overline W_j \subseteq U_j$ and the $\overline W_j$ contain the $V_x$s, so are a compact cover of $K$.
\end{proof}
\begin{proof}[Proof of Theorem \ref{partitions of unity}]
    Fix a compact set $K$. By the lemma, we can choose $D_j \subseteq U_j$ a compact cover of $K$ and $g_j$ supported in $U_j$ with $g_j \geq 1$ on $D_j$, and $h = \sum_j g_j$. Then $h \geq 1$ on $C$ and put $k = h \vee 1 \geq 1$. So $1/k$ exists and we can put $f_j = g_j/k$, to force $\sum_j f_j = 1$.
\end{proof}

\begin{definition}
    A \dfn{content} is a function defined on sets into $[0, \infty]$ which is monotone, countably subadditive, and finitely additive, and which carries compact sets to $[0, \infty)$. A content $\mu$ is said to be \dfn{inner regular} if for every open set $U$,
    $$\mu(U) = \sup_{\substack{\overline V \subseteq U\\V \text{open}\\\overline V \text{compact}}} \mu(V).$$
\end{definition}
Fix a positive Radon measure $\varphi$, and define an inner-regular content $\mu$ on the topology $\mathcal T$ on $X$ by
\begin{align*}
    \mu: \mathcal T &\to [0, \infty]\\
    U &\mapsto \sup_{\substack{f \in C_c(U)\\0 \leq f \leq 1}} \varphi(f).
\end{align*}
Given a content $\nu$, we can extend $\nu$ to an outer measure $\nu^*$ on the power set $\pset$ by
$$\nu^*(A) = \inf_{\substack{U \subseteq A\\U \in \mathcal T}} \nu(U).$$
In turn, then, $\nu^*$ restricts to a measure, also called $\nu$, on its measurable $\sigma$-algebra, by the Caratheodory construction. So, in particular, $\mu$ gives rise to a measure.

\begin{definition}
    Let $\nu$ be a Borel measure. We say that $\nu$ is \dfn{outer regular} if for every Borel set $E$,
    $$\mu(E) = \inf_{\substack{E \subseteq U\\U \in \mathcal T}} \mu(U)$$
    and \dfn{inner regular} if for every \emph{open} set $U$,
    $$\mu(U) = \sup_{\substack{\overline V \subseteq U\\V \text{open}\\\overline V \text{compact}}} \mu(V).$$
\end{definition}
    We state the main result.
\begin{theorem}[Riesz-Markov representation theorem]
    \index{Riesz-Markov representation theorem}
    $\mu$ is a positive Borel measure which is both inner and outer regular, and $\varphi$ is the unique functional such that for every $f \in C_c(X)$,
    $$\varphi(f) = \int f~d\mu.$$
\end{theorem}
    The proof of the Riesz-Markov representation theorem is quite long, so we only sketch it.
\begin{proof}[Proof sketch]
    Let $\nu^*$ be an outer measure which is finitely additive and inner regular on the topology of $X$, and let $U$ be open. Then Caratheodory's criterion holds for $U$ and $\nu^*$ on open sets. Approximating any subset of $X$ by an open set, Caratheodory's criterion holds on the power set for $U$ and $\nu^*$. So $U$ is $\nu^*$-measurable, and $\nu^*$ restricts to a Borel measure $\nu$. In particular, $\mu$ is a Borel measure.
    
    If $f \in C_c(X)$, and $f \geq 1$ on an open set $U$, $\varphi(f) \geq \mu^*(U)$. Approximating any set $A$ by an open set, we see that $\varphi(f) \geq \mu^*(A)$ whenever $f \geq 1$ on $A$. On the other hand, if $f \leq 1$ on $A$, a monotone convergence argument shows that $\mu^*(A) \geq \varphi(f)$. Since $C_c(X)$ is a Banach lattice, we can replace $f$ with $f^+$ and by decomposing $X$ into a chain of sets $X_n \{x \in X: f(x) \geq n\varepsilon\}$ and summing the $f|_{X_n} \setminus f|_{X_{n-1}}$ we prove
    $$\varphi(f) = \int f~d\mu.$$
    
    Since $\mu$ was inner and outer regular as a content, approximation by open sets implies regularity on Borel sets. Moreover, if $\psi$ is a positive Radon measure, define a content $\nu$ by
    $$\nu(U) = \sup_{\substack{f \leq \chi_U\\f \in C_c(U)}} \int f ~d\nu.$$
    If $\nu = \mu$ it follows that $\psi = \varphi$.
\end{proof}
    Notice that on the other hand, a complex measure $\nu$ on $C_c(X)$ gives rise to a functional $\psi$ by
    $$\psi(f) = \int f ~d\nu.$$
    The positive part of $\psi$ is in fact the positive part of $\nu$.
    
\section{Baire categories}
    Now we look at a topological analogue of ``measure zero."
\begin{definition}
    Let $X$ be a topological space. A set $S \subseteq X$ is \dfn{nowhere dense} if for every open set $U$, $S \cap U$ is not dense in $U$. A set $T \subseteq X$ is \dfn{meager} or \dfn{of the first category} if $T$ is the countable union of nowhere dense sets. A set $W \subseteq X$ is \dfn{of the second category} if it is not of the first category, or \dfn{comeager} if it is the complement of a meager set.
\end{definition}
\begin{lemma}
    For a topological space, the following are equivalent:
\begin{enumerate}
    \item Every countable union of closed sets with empty interior has empty interior. 
    \item Every countable intersection of open dense sets is dense. 
    \item Every nonempty open set is of the second category.
\end{enumerate}
\end{lemma}
    This is basically obvious.
\begin{definition}
    A topological space is a \dfn{Baire space} if one (and all) of the above criteria hold.
\end{definition}
\begin{theorem}[Baire category theorem]
    \index{Baire category theorem}
    Every completely pseudometrizable or locally compact Hausdorff space is Baire.
\end{theorem}
    For example, a Banach space is Baire.
\begin{proof}
    Let $U_n$ be a sequence of open dense sets, and let $W$ be open, in the space $X$. Then $W \cap U_1$ is nonempty and open, say $x_1 \in W \cap U_1$. If $X$ is pseudometrizable, then there is a $\varepsilon_1 \in (0, 1)$ such that the open ball $V_1 = B(x_1, \varepsilon_1)$ satisfies $K_1 = \overline B(x_1, \varepsilon_1) \subseteq W \cap U_1$; if $X$ is locally compact Hausdorff, then there is a compact set with nonempty interior $V_1 \subseteq K_1 \subseteq W \cap U_1$. Iterate using the denseness of the $U_n$ and the axiom of choice to construct a sequence $x_n \in V_n \subseteq K_n \subseteq V_{n-1} \cap U_n$. If $X$ is pseudometrizable, then we can always choose $\varepsilon_n < 1/n$, so the $x_n$ are a Cauchy sequence. Otherwise, $\bigcap_n K_n$ is nonempty anyways by the finite intersection property. Either way, we can find an $x \in \bigcap_n K_n \subseteq \bigcap_n U_n$ such that $x \in W$. So $\bigcap_n U_n$ is dense.
\end{proof}
Actually, we didn't use the full axiom of choice. The Baire category theorem is equivalent over ZF to the following axiom.
\begin{axiom}[Axiom of dependent choice]
    \index{axiom of dependent choice}
    Let $X$ be a nonempty set and $R$ be a binary relation. If, for every $a \in X$, there is a $b \in X$ such that $aRx$, then there is a sequence of $x_n$ such that $x_nRx_{n+1}$.
\end{axiom}
The axiom of dependent choice is not strong enough to prove the existence of nonmeasurable sets, for example. Moreover, if $X$ is assumed to be separable, then the Baire category theorem just follows from induction, without even dependent choice.

\begin{theorem}[uniform boundedness principle]
    \index{uniform boundedness principle}
    Let $X$ be a Banach space and $Y$ a normed space, and $F$ be a set of linear mappings $X \to Y$. If for every $x \in X$,
    $$\sup_{T \in F} ||Tx|| < \infty,$$
    then
    $$\sup_{\substack{T \in F\\||x|| = 1}} ||Tx|| = \sup_{T \in F} ||T||.$$
\end{theorem}
The uniform boundedness principle is also called the Banach-Steinhaus theorem.\index{Banach-Steinhaus theorem} The proof is a standard application of the Baire category theorem: construct a chain of closed sets whose union is the entire space, which implies that one is not meager.
\begin{proof}
    Let
    $$X_n = \{x \in X: \sup_{T \in F} ||Tx|| \leq n\}.$$
    Then the $X_n$ are a closed chain whose union is $X$. So by the Baire category theorem, there is an $x \in X$, $m > 0$, and $\varepsilon > 0$ such that $B(x, \varepsilon) \subset X_m$. So if $||u|| < 1$ and $T \in F$,
\begin{align*}
    ||Tu|| &= \varepsilon^{-1} ||T(x + \varepsilon u) - Tx||
        \leq \varepsilon^{-1} ||T(x + \varepsilon u)|| + \varepsilon^{-1} ||Tx||
        \leq 2\frac{m}{\varepsilon}.
\end{align*}
    Taking the $\sup$ over $u$ of both sides,
    $$\sup_{T \in F} ||T|| \leq 2\frac{m}{\varepsilon} < \infty.$$
\end{proof}
\begin{theorem}[open mapping theorem]
    \index{open mapping theorem}
    If $A: X \to Y$ is a surjective continuous linear mapping between Banach spaces, then $A$ is open.
\end{theorem}
The open mapping theorem is also called the Banach-Schauder theorem.\index{Banach-Schauder theorem}
\begin{proof}
    We must show that if $U$ is the open unit ball of $X$, then $A(U)$ is open. Since $X = \bigcup_k kU$, $Y = \bigcup_k A(kU)$. By the Baire category theorem, there is a $k > 0$, $\varepsilon > 0$, and $y \in Y$ such that $B(y, \varepsilon) \subseteq \overline{A(kU)}$. If $V$ is the unit ball of $Y$, $v \in V$, $y+\varepsilon v \in \overline{A(kU)}$ so
    $$\varepsilon v \in \overline{A(kU)} + \overline{A(kU)} \subseteq \overline{A(2kU)}.$$
    So if $L = 2k/\varepsilon$, $V \subseteq \overline{A(LU)}$. 
    
    In other words, for every $y \in Y$ and $\varepsilon > 0$ there is an $x \in X$ such that $||x|| \leq L||y||$ and $||y-Ax|| < \varepsilon$. In particular, given $y \in V$ we can choose $x_1$ such that $||x_1|| \leq L$ and $||x-Ax_1|| < 1/2$. Choose $||x_{n+1}|| \leq L2^{-n}$ such that
    $$||y - A(x_1 + \dots + x_n) - Ax_{n+1}|| < 2^{-n-1},$$
    by induction and the axiom of (dependent) choice. The sequence of partial sums is therefore Cauchy, so we can put $x = \sum_n x_n$, and $Ax = y$ by the above estimates. Also
    $$||x|| = \lim_{n \to \infty} \left|\left| \sum_{k\leq n} x_k\right|\right| \leq \sum_{n=1}^\infty x_n < 2L.$$
    So $y \in A(2LU)$. Therefore $V \subseteq A(2LU)$ which was to be shown.
\end{proof}
\begin{theorem}[closed graph theorem]
    Let $A: X \to Y$ be a linear mapping between Banach spaces. If the graph of $A$ is closed in $X \oplus Y$, then $A$ is continuous.
\end{theorem}
    Notice that while there isn't a canonical norm for $X \oplus Y$, any $\ell^p$ norm will do; since $X \oplus Y$ is a finite direct sum, all $\ell^p$ norms are equivalent. In particular, $X \oplus Y$ is a Banach space.
\begin{proof}
    Let $\Gamma$ be the graph of $A$, which is equipped with a natural (linear, bijective) projection $\pi_X: \Gamma \to X$. Since
    $$||P(x, Ax)|| = ||x|| \leq ||(x, Ax)||,$$
    $||P|| \leq 1 < \infty$. So by the open mapping theorem,
    $$||Tx|| \preceq ||P^{-1}x|| + ||x|| \preceq ||x||.$$
\end{proof}

\chapter{Preliminaries on Banach algebras}
\begin{definition}
A \dfn{Banach algebra} is a Banach space equipped with a bilinear, associative multiplication such that
$$||xy|| \leq ||x|| ||y||.$$
If $*$ is a linear involution on $\mathcal A$ such that $(xy)^* = y^*x^*$ and $1^*= 1$ if $\mathcal A$ is unital. then we say that $\mathcal A$ is a \emph{$*$-algebra}.
\end{definition}

\begin{definition}
\index{$C^*$-algebra}
Let $\mathcal A$ be a $*$-algebra. If one has the \emph{$C^*$-identity}
$$||x^*x|| = ||x||^2,$$
then we say that $\mathcal A$ is a \emph{$C^*$-algebra}.
\end{definition}

For example, if $\mathcal H$ is a Hilbert space, then $\mathcal B(\mathcal H)$ is a $C^*$-algebra. Later we will learn that sub-$*$-algebras of $\mathcal B(\mathcal H)$ are the only examples of $C^*$-algebras.

Often the norm topology is too strong, so we introduce a new topology which is weaker on $\BB(\HH)$.
\begin{definition}
    The \dfn{strong operator topology} is the locally convex topology on $\BB(\HH)$ defined by the seminorms
    $$P_\xi(T) = ||T\xi||.$$
\end{definition}
In other words, a sequence converges in the strong operator topology $T_n \to T$ iff for each $\xi \in \HH$, $||(T_n - T)\xi|| \to 0$. So the strong operator topology is the topology of pointwise convergence.
\begin{definition}
    A \dfn{von Neumann algebra} $\mathcal A$ is a sub-$*$-algebra of $\BB(\HH)$ which is closed in the strong operator topology.
\end{definition}


\section{The spectrum}
Fix a Banach algebra $A$.
\begin{definition}
    Let $a \in A$. The \dfn{spectrum} $\sigma(a)$ is the set of $z \in \CC$ such that $\sigma(a) - z$ is not invertible. The \dfn{resolvent} $\rho(a)$ is the complement of $\sigma(a)$.
\end{definition}

\begin{lemma}
    Let $a \in A$. If $||a|| < 1$ then $1 - a$ is invertible with inverse
    $$(1 - a)^{-1} = \sum_{n=0}^\infty a^n.$$
\end{lemma}
\begin{proof}
    The partial sums converge since $||a|| < 1$. Therefore
    $$(1 - a) \sum_{n=0}^\infty a^n = (1 - a) \lim_{n \to \infty} \sum_{k=0}^n a^k = \lim_{n \to \infty} \sum_{k=0}^n a^k - a^{k-1} = 1$$
    since the summands telescope.
\end{proof}
In particular, if $||1 - a|| < 1$ then $a$ is invertible. 
\begin{definition}
    The \dfn{general linear group} of $A$ is $\GL(A)$, the group of invertible elements of $A$.
\end{definition}
By the above lemma, there is a ball $B$ around $1$ contained in $\GL(A)$. By continuity of translation, we can carry $B$ to be centered at any point of $\GL(A)$. Therefore $\GL(A)$ is an open set.

\begin{proposition}
    The function $z \mapsto (z - a)^{-1}$ is holomorphic on $\rho(a) \cup \infty$.
\end{proposition}
    In this case, holomorphy is indicated by local existence of a convergent power series.
\begin{proof}
    We have
    $$(a - z)^{-1} = \sum_{n=0}^\infty (a - z_0)^{-n-1}(z-z_0)^n$$
    for each $z_0 \in \rho(a)$ and $z$ close enough to $z_0$ that the power series converges. To see that the function is still holomorphic at $\infty$, notice that
    $$(a - z^{-1})^{-1} = z(1 - az)^{-1}$$
    which vanishes as $z \to 0$. Replacing $z$ by $z^{-1}$, we see that the function is bounded close to infinity, and continuous, so holomorphic there.
\end{proof}

We now observe that the usual proofs of Cauchy's integral formula and its friends such as Cauchy's estimate and Liouville's theorem go through even in case of holomorphic functions $U \to A$, $U \subseteq \CC$ open. 

We now come to the famous Gelfand-Mazur theorem, which can be thought of as a ``restatement of the fundamental theorem of algebra" for our purposes. For the notation, recall that the map $z \mapsto z1$ is an embedding of $\CC$ in any Banach algebra. 
\begin{theorem}[Gelfand-Mazur]
    \index{Gelfand-Mazur theorem}
    If $A = \GL(A) \cup 0$, then $A = \CC$.
\end{theorem}
\begin{proof}
    Let $a \in \GL(A)$ and assume towards contradiction that $a \notin \CC$. Then the resolvent $z \mapsto (a - z)^{-1}$ is a holomorphic function defined on the Riemann sphere, so constant. Taking $z = \infty$, the resolvent is identically $0$, but also identically $a^{-1}$ (taking $z = 0$). This is a contradiction.
\end{proof}
Notice that this fails over $\RR$, as witnessed by $\CC$ as well as the quaternions $\mathbb H$. This is why we study Banach algebras over $\CC$.

\section{Ideals}
Let $I$ be an ideal of $A$. It is immediate that the norm-closure $\overline I$ is an ideal. Moreover, since $\GL(A) \ni 1$ is open, if $I$ is a proper ideal, then $I$ does not meet $\GL(A)$ and so $\overline I$ does not contain $1$, so $\overline \cdot$ preserves propriety. Therefore maximal ideals are closed. Moreover, for continuous morphisms, kernels are closed, so we might as well only study closed ideals.

If $I$ is a (left, right) ideal then $A/I$ is a (left, right) module over $A$, equipped with the seminorm
$$||a|| = \inf_{d \in I} ||a - d||.$$
In case $I$ is closed, this seminorm is actually a norm, and complete since $A$ is complete. So we end up with a Banach space.

\begin{definition}
    A \dfn{Banach module} over $A$ is an $A$-module $M$ which is a Banach space, such that
    $$||am||_M \leq ||a||_A ||m||_M.$$
\end{definition}
It is not very hard to check that $M = A/I$ is a Banach module. In fact, for $b, c \in I$, we have
$$||am||_M \leq ||(a - c)(m - d)||_A \leq ||a - c||_A ||m - d||_A.$$
Taking the $\inf$ over $c, d$ of both sides, we have
$$||am||_M \leq ||a||_A ||m||_M.$$
In case $I$ is two-sided, $M$ is a Banach $(A, A)$-bimodule, or in other words, a Banach algebra.

In what follows we use $\Hom(A, B)$ to mean the $K$-algebra of morphisms of $K$-algebras $A \to B$ over some field $K$ (which is usually $\CC$).

If $I$ is a maximal ideal, therefore, $A/I$ is a field, and so $A/I = \CC$ by the Gelfand-Mazur theorem. But a maximal ideal gives a epimorphism $A \to \CC$, and conversely, the kernel of a such an epimorphism is a maximal ideal. This gives a bijection between the maximal spectrum of $A$ and $\Hom(A, \CC) \setminus 0$, which we call $\hat A$.

\begin{lemma}
    Let $K$ be a field and $A$ a unital $K$-algebra. Let $\varphi \in \Hom(A, K)$. Then if $a \in A$, $\varphi(a) \in \sigma(a)$.
\end{lemma}
\begin{proof}
    We have $\varphi(a - \varphi(a)) = 0$.
\end{proof}
\begin{lemma}
    If $\varphi: A \to \CC$ is a nonzero morphism, then $||\varphi|| \leq 1$.
\end{lemma}
\begin{proof}
    $\varphi(a) \in \sigma(a)$ so $||\varphi(a)|| \leq ||a||$.
\end{proof}

Therefore $\hat A$ is contained in the unit ball $A'_1$ of the dual $A'$. Since nets in $\hat A$ act continuously on $A$, their pointwise convergence preserves operations of $A$. So $\hat A$ is closed. In particular, the Banach-Alaoglu theorem implies that $\hat A$ is a weakstar compact Hausdorff space.

\begin{definition}
    Let $a \in A$. The \dfn{Gelfand transform} $\hat a$ is the function
    $$\hat a(\varphi) = \varphi(a),$$
    for $\varphi \in \hat A$.
\end{definition}
Notice that $||\hat a||_{L^\infty(\hat A)} \leq ||a||_A$ and $\hat a(\hat A) \subseteq \sigma_A(a)$. Conversely, let $\lambda \in \sigma_A(a)$. Then $a - \lambda$ is not invertible, so there is a maximal ideal $I \supseteq (a - \lambda)$ and an epimorphism $\varphi$ such that $\ker \varphi = I$. Thus $\lambda \in \hat a(\hat A)$. Therefore $\hat A = \sigma_A(a)$, but the proof of this is highly nonconstructive.

\begin{example}
    Recall that $c_0(\NN)$, the set of $x \in \ell^\infty(\NN)$ such that $x_n \to 0$ as $n \to \infty$, is a closed ideal of $A = \ell^\infty(\NN)$. Therefore there is a $\varphi$ such that $\varphi(c_0(\NN)) = 0$. But, in fact, $\hat A = \beta\NN$, where $\beta$ is the Stone-Cech functor. It follows that it is consistent with ZF without the axiom of choice that $\varphi$ does not exist.
\end{example}

\begin{definition}
    Let $a \in A$. The \dfn{spectral radius} of $a$ is
    $$r(a) = \max_{\lambda \in \sigma(a)} |\lambda|.$$
\end{definition}
Equivalently, $r(a) = ||\hat a||_{L^\infty(\hat A)}$. Therefore we have $r(ab) \leq r(a)r(b)$. 

\section{The holomorphic functional calculus}
    As usual, let $A$ be a commutative Banach algebra.
\begin{definition}
    Let $a \in A$ and let $f$ be a holomorphic function on $D(0, ||a|| + \varepsilon)$. Put
    $$f(z) = \sum_{n=0}^\infty \alpha_n z^n.$$
    The \dfn{holomorphic functional calculus} is the morphism $f \mapsto f(a)$ defined by
    $$f(a) = \sum_{n=0}^\infty \alpha_n a^n.$$
\end{definition}
    The Taylor series of $f$ converges uniformly absolutely on $D(0, ||a||)$, so the partial sums of $f(a)$ form a Cauchy sequence in $A$. Therefore $f(a)$ is a well-defined element of $A$, and we can think of $f$ as a mapping $U \to A$, where $U$ consists of elements of $A$ that are small enough. If $f$ is entire, then $f$ lifts to a function $A \to A$.
\begin{theorem}[spectral mapping theorem]
    \index{spectral mapping theorem}
    If $\lambda \in \sigma(a)$ then $f(\lambda) \in \sigma(f(a))$. 
\end{theorem}
\begin{proof}
    We have
    \begin{align*}
        f(a) - f(\lambda) &= \sum_{n=0}^\infty \alpha_n(a^n - \lambda^n)
            = \sum_{n=0}^\infty \alpha_n(a - \lambda)(a^{n-1} + a^{n-2}\lambda + \dots + \lambda^{n-1})\\
            &= (a - \lambda)b
    \end{align*}
    for some $b$, if we can show that the partial sums are a Cauchy sequence. In fact
    $$||a^{n-1} + \dots + \lambda^{n-1}|| \leq n||a||^{n-1}$$
    which is the right-hand side of $f'(||a||)$ (which clearly converges, so partial sums are Cauchy). Therefore $f(a) - f(\lambda) = (a-\lambda)b$. So if $f(a) - f(\lambda)$ is invertible, then so is $a - \lambda$.
\end{proof}





\chapter{Preliminaries on $*$-algebras}
\section{Weights}
\begin{definition}
\index{weight}
Let $\mathcal A$ be a sub-$*$-algebra of $\mathcal B(\mathcal H)$. A map $\omega: \mathcal A^+ \to [0, \infty]$ is a \emph{weight} if $\omega$ is additive and if $\omega(ta) = t\omega(a)$ whenever $t \geq 0$.
\end{definition}

Fix a weight $\omega$. By $m_\omega$ we mean the span of the set of positive $a$ such that $\omega(a) < \infty$, and by $m_\omega^{sa}$ we mean the closure of the set of positive $a$ such that $\omega(a) < \infty$ under subtraction. Clearly $\omega$ extends uniquely to $m_\omega^{sa}$ by $\omega(b-c) = \omega(b) - \omega(c)$. So $\omega$ extends to a positive linear functional on $m_\omega$ in the obvious way. On the other hand, if $\varphi$ is any positive linear functional on $\mathcal B(\mathcal H)$, then $\varphi$ is a weight such that $m_\varphi = \mathcal B(\mathcal H)$.

Now we define $n_\omega$ to be the set of $a \in \mathcal A$ such that $\omega(a^*a) < \infty$, which is clearly a subspace of $\mathcal A$.
\begin{lemma}
    $n_\omega$ is a left ideal of $\mathcal A$.
\begin{proof}
    If $T \geq 0$ then
    \begin{align*}
        \langle S^*TS\xi, \xi\rangle
            &= \langle TS\xi, S\xi\rangle
            \leq ||T|| ||S\xi||^2\\
            &= ||T|| \langle S\xi, S\xi\rangle
            = ||T|| \langle S^*S\xi, \xi\rangle.
    \end{align*}
    So if $d \in \mathcal A$ and $a \in n_\omega$ then
    \begin{align*}
        (da)^*da = a^*d^*da \leq ||d^*d|| a^*a = ||d||^2 a^*a
    \end{align*}
    whence
    $$||\omega((da)^*(da)) \leq ||d||^2 \omega(a^*a) < \infty.$$
\end{proof}
\end{lemma}
\begin{definition}
\index{tracial weight}
If $\omega(a^*a) = \omega(aa^*)$ then $\omega$ is \emph{tracial}.
\end{definition}
Clearly if $\omega$ is tracial then $n_\omega$ is a two-sided ideal. For example, if $\omega$ is actually the trace,
$$\omega(x) = \sum_j \langle x^*xe_j, e_j\rangle$$
for $\{e_j\}_j$ an orthonormal basis of the separable Hilbert space $\mathcal H$, then $\omega$ is tracial and $n_\omega$ is just the space of trace-class operators and $\omega$ is tracial.

\index{polarization identity}
Recall the polarization identity:
$$4b^*a = \sum_{k=0}^3 i^k(a+i^kb)^*(a+i^kb).$$
From this we are justified in defining, on $n_\omega$,
$$\langle a, b\rangle_\omega = \omega(b^*a).$$
This would be an inner product if $N_\omega = \{a \in \mathcal A: \omega(a^*a) = 0\}$ were trivial. Clearly $N_\omega$ is a subspace, so we can take the completion of $n_\omega/N_\omega$ and recover a Hilbert space.

\section{The GNS construction}
\begin{definition}
    The completion of $n_\omega/N_\omega$ is denoted $L^2(\mathcal A, \omega)$.
\end{definition}
If $b \in n_\omega$, then
$$\langle ab, ab\rangle_\omega
    = \omega(b^*a^*ab)
    \leq ||a^*a|| \omega(b^*b) = ||a||^2 ||b||_\omega^2.$$
So if $a \in \mathcal A$ then $\xi \mapsto a\xi$ is a well-defined, bounded operator on $n_\omega/N_\omega$ and so extends to $L^2(\omega)$. 

\begin{definition}
    \index{$*$-representation}
    A \emph{$*$-representation} is a morphism of $*$-algebras (i.e. a morphism of algebras preserving $*$) into $\BB(\HH)$.
\end{definition}
If $a \in \AAA$, $b,c \in n_\omega$, then
$$\langle ab, c\rangle_\omega = \omega(c^*ab) = \omega((a^*c)^*b) = \langle b, a^*c\rangle_\omega,$$
which descends to $L^2(\omega)$. So we can define a $*$-representation
\begin{align*}
    L: \AAA &\to \BB(L^2(\omega))\\
        a &\mapsto (\xi \mapsto a\xi).
\end{align*}
\begin{definition}
    \index{GNS construction}
    \index{left regular representation}
    The map $L$ is called the \emph{GNS construction} (for Gelfand-Neimark-Segal) of $\AAA$, or the \emph{left regular representation} of $\AAA$.
\end{definition}
The GNS construction allows us to assume that $\AAA$ is actually acting on a Hilbert space, namely $L^2(\omega)$. So a $C^*$-algebra is always an operator algebra.

We can also define a right regular representation\index{right regular representation},
\begin{align*}
    R: \AAA &\to \BB(L^2(\omega))\\
        a &\mapsto (\xi \mapsto \xi a).
\end{align*}
Notice that $R$ is an antihomomorphism.

\begin{lemma}
    Assume that for each positive $a \in \AAA$, $\sqrt a$ exists. Then $m_\omega \subseteq n_\omega$, and in particular $m_\omega$ is a sub-$*$-algebra of $\AAA$.
    \end{lemma}
\begin{proof}
If $a \in m_\omega$ is positive, 
    $$\omega(\sqrt a^2) = \omega(\sqrt a^* \sqrt a) = \omega(a)$$
so $\sqrt a \in n_\omega$. Since $n_\omega$ is a left ideal, $a \in n_\omega$.
\end{proof}


\begin{example}
    Let $X$ be a measure space and $K \in L^2(X \times X)$. Then the integral operator $T_K: L^2(X) \to L^2(X)$ has $||T_K||_{\BB^2} = ||K||_{L^2}$. Indeed, if $\{\xi_n\}_n$ is a Hilbert basis for $L^2(X)$ then
\begin{align*}
    \sum_n ||T_k\xi_n||^2
        &= \sum_{m,n} |\langle T_k\xi_m, \xi_n\rangle|^2
        = \sum_{m,n} \left|\iint_{X \times X} K(x, y) \xi_n(y) \xi_m(x) ~dx ~dy\right|^2\\
        &= \sum_{m,n} |\langle K, \xi_m \otimes \xi_n\rangle|^2
        = ||K||_{L^2}
\end{align*}
    since the $\xi_m \otimes \xi_n$ form a Hilbert basis for $L^2(X \times X) = L^2(X) \otimes L^2(X)$.
\end{example}
\begin{example}
    If $\AAA = C([0, 1])$ and
    $$\omega(f) = \int_0^1 f(t) ~dt$$
    then $\omega$ is a tracial weight on $\AAA$ such that $n_\omega = \AAA$. But of course $n_\omega$ is a Banach space when given the $\BB^2 = L^2$ norm. Its completion is $L^2([0, 1])$.
\end{example}
\begin{proposition}
    Let $\omega$ be a tracial weight and $\AAA$ be a sub-$*$-algebra of $\BB_0(\HH)$. If $b \geq 0$ and $b \in m_\omega$, and $a \in \BB(\HH)$ then
    $$|\omega(ab)| \leq ||a||_{op}|\omega(b)|.$$
\end{proposition}
\begin{proof}
    We have
\begin{align*}
    |\omega(ab)|^2
        &= |\omega(a \sqrt b \sqrt b)|^2
        = |\omega(\sqrt b a \sqrt b)|^2
        = |\langle a \sqrt b, \sqrt b\rangle|^2\\
        &\leq \langle a\sqrt b, a\sqrt b\rangle \langle \sqrt b, \sqrt b\rangle
        = \omega(\sqrt b a^* a\sqrt b)\omega(b)
        \leq ||a||^2 \omega(b)^2.
\end{align*}
\end{proof}

\section{The $\BB^p$ spaces}
\begin{definition}
    If $T \in \BB^2(\HH)$ then $T$ is called a \dfn{Hilbert-Schmidt operator}.
\end{definition}
\begin{example}
    Let $\HH$ be the separable Hilbert space. Take $\AAA = \BB_0(\HH)$ and $\omega$ to be the trace. Since $\BB_0(\HH)$ has square roots and $\omega$ is tracial, we can apply the above result to prove that $n_\omega$ and $m_\omega$ are two-sided ideals and hence sub-$*$-algebras.
    
    If we write $|T| = \sqrt T^2$, and let $\BB^p(\HH)$ be the space of $T \in \BB_0(\HH)$ such that $\omega(|T|^p) < \infty$, then $\BB^1(\HH) = m_\omega(\HH)$ and $\BB^2(\HH) = n_\omega(\HH)$.
    
    We think of $\BB^p(\HH)$ as the noncommutative analogue of $\ell^p$.
\end{example}
Let's check that that example actually makes sense.
\begin{theorem}
    $\BB^2(\HH)$ is a Banach space.
\end{theorem}
\begin{proof}
    First observe that $||T||_{op} \leq ||T||_2$. To do this, compute the trace of $T$ by using an orthonormal basis containing a $\xi$ such that $||T\xi|| \geq ||T||_{op} - \varepsilon$. As this is possible for any $\varepsilon > 0$ the claim holds.
    
    Now assume that $\{T_n\}_n$ is $2$-Cauchy, so in particular $op$-Cauchy. So there is a $T \in \BB_0(\HH)$ such that $T_n \to^{op} T$.
    
    If $P$ is a finite-rank projection then $(T-T_n)P$ is a finite-rank operator, hence $\in \BB^2(\HH)$. So
\begin{align*}
    ||(T-T_n)P||_2^2
        &= tr P(T-T_n)^*(T-T_n)P
        = tr (T-T_n)P(T-T_n)^*
        = \lim_{k \to \infty} tr (T_k-T_n)P(T_k -T_n)^*\\
        &\leq \limsup_{k \to \infty} (T_k-T_n)(T_k-T_n)^*
        = \limsup_{k \to \infty} ||T_k-T_n||_2^2.
\end{align*}
Let $C_n = \limsup_{k \to \infty} ||T_k-T_n||_2^2$. Then $C_n \to 0$ and
$$||(T-T_n)P||_2^2 \leq C_n$$
regardless of the choice of $n$ and $P$. Since $T-T_n$ is a compact operator, we can approximate it arbitrarily well by $(T-T_n)P$ by choosing $P$. So $||T-T_n||_2^2 \to 0$.
\end{proof}
Recall the \dfn{polar decomposition} of $T \in \BB_0(\HH)$ is the factorization
$$T = V|T|$$
where $|T| = \sqrt T^2$ and $V$ is a \dfn{partial isometry}, i.e. an isometry on its cokernel.
\begin{lemma}
    If $T \in \BB^1(\HH)$ and $A \in \BB(\HH)$ then
    $$|tr(AT)| \leq ||A||_{op} tr|T|.$$
\end{lemma}
\begin{proof}
    Write $T = V|T|$. Then
\begin{align*}
    |tr(AT)|
        &= |tr(AV|T|)|
        \leq ||AV||_{op} tr|T|
        \leq ||A||_{op} tr|T|.
\end{align*}
\end{proof}
\begin{lemma}
    $||\cdot||_1 = tr|\cdot|$ is a norm on $\BB^1(\HH)$.
\end{lemma}
\begin{proof}
    Let $S,T \in \BB^1(\HH)$ and $S+T = W|S+T|$. Then
\begin{align*}
    tr|S+T|
        &= tr W^*(S+T)
        = tr(W^*S) + tr(W^*T)
        \leq |trW^*S| + |trW^*T|
        \leq tr|S| + tr|T|.
\end{align*}
\end{proof}
\begin{theorem}
    $\BB^1(\HH)$ is a Banach algebra.
\end{theorem}
\begin{proof}
    Since $||T||_{op} \leq ||T||_1$ the proof is basically the same as for Hilbert-Schmidt operators.
\end{proof}
\begin{theorem}
    $\BB^1(\HH)^* = \BB(H)$.
\end{theorem}
\begin{proof}
    If $A \in \BB(\HH)$, let $\Psi_A(T) = tr(AT)$. Then
$$||\Psi_A(T)|| \leq ||A||||T||_1.$$
    So $A \mapsto \Psi_A$ is an isometry and so $\BB(\HH) \subseteq \BB^1(\HH)^*$.
    
    Let $\Psi \in \BB^1(\HH)^*$ and $\xi,\eta \in \HH$. Define a bounded operator $\langle \xi, \eta\rangle_O$ by
    $$\langle \xi, \eta\rangle_O\zeta = \xi \langle \eta, \zeta\rangle.$$
    (So $\langle\cdot,\cdot\rangle_O$ is an operator-valued pseudo-inner product (the pseudo- here means that it could be zero).
    Define a semilinear form
    $$B_\Psi(xi, \eta) = \Psi\langle \xi, \eta\rangle_O.$$
    So $|B_\Psi(\xi, \eta)| \leq ||\Psi||||\xi||||\eta||$. Therefore by the Riesz representation theorem, there is an operator $A$ such that $B_\Psi(\xi, \eta) = \langle A\xi, \eta\rangle$. Therefore $||A|| = ||\Psi||$ and $\Psi = \Psi_A$. So $\BB^1(\HH)^* \subseteq \BB(\HH)$.
\end{proof}

\section{Representation theory of groups}
Let $G$ be a group with a good topology (so $G$ admits a Haar measure).

\begin{definition}
A \dfn{unitary representation} of $G$ is a continuous morphism of groups $G \to U(H)$. It is \dfn{irreducible} if the only $G$-invariant subspaces are trivial. 
\end{definition}
For $\pi$ a unitary representation, we have $\pi(x)^* = \pi(x)^{-1}$.

\begin{example}
    Let $G = \SL(3, \ZZ)$. Then the ``obvious" map $G \to \SL(3, \CC)$ is not a unitary representation. In fact $G$ has very few finite-dimensional unitary representations, because $G$ is not compact.
\end{example}

\begin{definition}
    The \dfn{left regular representation} of $G$ is the map $G \to U(L^2(G))$ given by
    $$\pi(x)(\xi)(y) = \xi(x^{-1}y).$$
\end{definition}

It is natural to want to study the subalgebra of $B(H)$ generated by $\pi(G)$ for $\pi$ a unitary representation. This will be given by linear combinations of the $\pi(x)$s as $x \in G$, which we identify with the space $C_c(G)$ of compactly supported continuous functions on $G$. Namely, for $f \in G$ we define
$$\pi(f) = \int_G f(x) \pi(x) ~dx.$$
\begin{definition}
    The norm-closure of $\pi(C_c(G))$ is the \dfn{reduced C$^*$-algebra} of $G$.
\end{definition}
Now an easy computation shows
$$\pi(f)\pi(g) = \pi(f*g)$$
and of course $\pi(f)^* = \pi(f^*)$ where we define $f^*(x) = \overline f(x^{-1})$. Finally, we observe that
$$||\pi(f)|| \leq ||f||_{L^1(G)}$$
so $\pi$ is a $*$-Banach algebra morphism which extends to a map
$$\pi: L^1(G) \to B(L^2(G)).$$
This leads to the abstract theory of Fourier transform.



\chapter{Compact operators}
Let $B_0(H)$ denote the algebra of compact operators in $H$. This is a closed ideal of $H$, hence a $C^*$ algebra (proof: it is the closure of the ideal $B_f(H)$ of finite rank operators in $H$.) It will be one of our main examples of a noncommutative, nonunital $C^*$ algebra.

\section{Representation theory}
We now study the representation theory of $B_0(H)$.

Like any $C^*$ algebra, $B_0(H)$ has a normalized approximate identity, sequential if $H$ is separable. Decompose $H$ by transfinite recursion as
$$H = \bigoplus_{\alpha < \kappa} \CC$$
where $\kappa$ is some cardinal ($\kappa = \aleph_0$ if $H$ separable) and the biproduct is in the category of Hilbert spaces. For $\lambda < \kappa$, let $H_\lambda = \bigoplus_{\alpha < \lambda} H_\lambda$ so $H$ is the injective limit of the $H_\lambda$; then let $e_\lambda$ be the natural projection $H \to H_\lambda$. The $e_\lambda$ form a net with respect to the natural ordering on $\kappa$ and are obviously an approximate identity.

Recall that if we fix a representation $\pi: A \to B(H)$, we can view $H$ as a module over $A$ by defining $a\xi = \pi(a)(\xi)$. Recall also that a representation is said to be nondegenerate if $HA$ is dense in $A$.

In fact, any representation of $B_0(H)$ is faithful. Since representations are continuous, and $B_0(H)$ has no closed ideals (since $B_f(H)$ contains all proper ideals of $B_0(H)$, and is dense in $H$), any representation of $B_0(H)$ is faithful.

There is a natural $*$-representation $B_0(H) \to B(H)$ given by the inclusion map. Since $e_\lambda\xi \to \xi$, this representation is nondegenerate. In some sense this is the only such representation.
\begin{lemma}
    A nondegenerate $*$-representation of $B_0(H)$ is isomorphic to a direct sum to copies of the representation $B_0(H) \to B(H)$. In particular, the only irreducible such representation is the representation $B_0(H) \to B(H)$. 
\end{lemma}
\begin{proof}
    Let $\langle \xi, \eta\rangle_0$ be the $B_0(H)$-valued inner product
    $$\langle \xi, \eta\rangle_0 \zeta = \xi \langle \eta, \zeta\rangle.$$
    In fact such an inner product has values in rank-$1$ operators since $\xi \langle \eta, \zeta\rangle$ lies in the span of $\xi$.

    For $T \in B(H)$, $T\langle \xi, \eta\rangle_0\zeta = (T\xi)\langle \eta, \zeta\rangle$ so $T\langle \xi, \eta\rangle_0 = \langle T\xi, \eta\rangle_0$, and $\langle \xi, \eta\rangle_0T = \langle \xi, T^*\eta\rangle_0$.

    Let $\pi: B_0(H) \to B(V)$ be a nondegenerate $*$-representation, $\xi \in H$ a unit vector. Then $\langle \xi, \xi\rangle_0$ is a rank-$1$ projection. Since $\pi$ is faithful, $\pi(\langle \xi, \xi\rangle_0)$ is a nonzero projection. Let $v$ be a unit vector of $\langle \xi, \xi\rangle_0(V)$ and define $Q: H \to V$ by $Q\eta = \langle \eta, \xi\rangle_0 v$. Then by a tedious computation, $Q$ is an isometry.

    We now show that $Q$ commutes the representations. Let $T = \langle \omega, \zeta\rangle_0$. Any operator in $B_0(H)$ can be written as an infinite linear combination of rank-$1$ operators so it suffices to show that $QT = TQ$. In fact,
    $$Q(T\eta) = \langle T\eta, \zeta\rangle_0 v = TQ(\eta).$$
    Also, $Q(H)^\perp$ is $\pi$-invariant, so we repeat the argument on $Q(H)^\perp$ to see that we have
    $$V = Q(H) \oplus Q(H)^\perp$$
    as $B_0(H)$-modules. Now run Zorn's lemma to keep decomposing $Q(H)^\perp$ until we hit an irreducible representation.
\end{proof}

This is a very remarkable property of $B_0(H)$. To see why, we need something stronger than ZFC.
\begin{definition}
    A \dfn{$\Diamond$-sequence} is a net of sets $\alpha \mapsto A_\alpha$, for $\alpha < \aleph_1$, such that for any $A \subseteq \aleph_1$,
    $$\hat A = \{\alpha < \aleph_1: A \cap \alpha = A_\alpha\}$$
    is stationary in $\aleph_1$.
\end{definition}
In other words, for every closed and unbounded (``club") set $C \subseteq \aleph_1$, $C \cap \hat A$ is nonempty. The existence of a $\Diamond$-sequence implies that $V = L$, in particular implying GCH.

Naimark conjectured that if $A$ was a $C^*$-algebra with only one irreducible representation, then $A = B_0(H)$. This is true if $A$ is separable.
\begin{theorem}[Ackemann-Weaver]
    If there is a $\Diamond$-sequence, then there is a $C^*$-algebra $A$ which has only one irreducible representation such that $A \neq B_0(H)$.
\end{theorem}

In this lemma, you should read $A = B(H)$ and $I = B_0(H)$.
\begin{lemma}
    Let $A$ be a $*$-normed algebra and $I$ a $*$-ideal of $A$ with normalized approximate unit. Then every nondegenerate $*$-representation of $I$ extends uniquely to $A$.
\end{lemma}
\begin{proof}
    Let $\pi: I \to B(H)$ be such a representation. Define
    $$\tilde \pi(a) \sum_\alpha \pi(d_\alpha) \xi_\alpha = \sum_\alpha \pi(ad_\alpha) \xi_\alpha$$
    where the $\xi_\alpha$ are a Hilbert basis of $H$. Then $\tilde \pi$ is a well-defined function since if $\sum_\alpha \pi(d_\alpha)\xi_\alpha = 0$, then
    $$\sum_\alpha \pi(ad_\alpha)\xi_\alpha = \lim_\lambda \sum_\alpha \pi(ae_\lambda d_\alpha) \xi_\alpha = \lim_\lambda \pi(ae_\lambda) \sum_\alpha \pi(d_\alpha)\xi_\alpha = 0$$
    since the $e_\lambda$ are a normalized approximate unit. This is unique because
    $$\tilde \pi(a)\pi(d)\xi = \pi(ad)\xi$$
    and the $\pi(ad)\xi$ are dense in $H$.
\end{proof}

\begin{lemma}
    Let $A,I$ be as above and let $\pi$ be an irreducible representation of $A$. Then either $I \subseteq \ker\pi$ or $\pi$ is an irreducible representation of $I$.
\end{lemma}
\begin{proof}
    Assume $I$ is not contained in $\ker \pi$. Then $\overline{IH}$ is nonzero and $A$-invariant. Since $\pi$ is irreducible, $\overline{IH} =H$. Therefore $\pi$ is a nondegenerate representation of $I$.

    We have $e_\lambda\xi \to \xi$ for any $\xi$ since $\pi$ is nondegenerate. Let $K \subseteq H$ be nondegenerate and nonzero. Using $e_\lambda$, $\overline{IK} = K$ so $\overline{IK}$ is $A$-invariant. Therefore since $\pi$ is irreducible, $K = H$.
\end{proof}

\begin{lemma}
    Let $A,I$ be as above. Let $\pi: A \to B(H)$ and $\rho: A \to B(K)$ be irreducible representations. If $\pi \cong \rho$ as representations of $I$, then $\pi \cong \rho$ as representations of $A$.
\end{lemma}
\begin{proof}
    Let $U: H \to K$ be an isomorphism of $I$-modules. For $d \in I$,
    $$U(\pi(a)\pi(d)\xi) = U(\pi(ad)\xi) = \rho(ad)U\xi = \rho(a)U(\pi(d)\xi).$$
    So $U\pi = \rho U$.
\end{proof}

\begin{theorem}[Burnside]
    Assume $H \neq \CC$. Let $A \subseteq B_0(H)$ be a $C^*$-algebra. If $A$ acts on $H$ irreducibly, then $A = B_0(H)$.
\end{theorem}
\begin{proof}
    By assumption on $H$, $A \neq 0$. Let $T \neq 0$. Then $T^*T \in A$ is nonzero, so we can assume without loss of generality that $T$ is self-adjoint. Moreover, $C^*(T, 1) = C(\sigma(T))$ acts on $H$ as an abelian monoid.

    Let $\lambda \in C(\sigma(T))$. Taking bump functions centered on $\lambda$ we can find a sequence of $\xi_n \in H$ such that $(T - \lambda)\xi_n \to 0$. Since $T$ is comapct, $T$ sends the unit ball of $H$ to a precompact set. So the $T\xi_n$ have a weak limit $\eta$. Thus $T\eta = \lambda\eta$. Therefore $T$ has an eigenvector for $\lambda$. So the only limit point of $\sigma(T)$ is $0$, because the other eigenvectors are all orthogonal.

    Let $P$ be a projection of minimal rank in $A$. We claim that $P$ is a rank-$1$ projection. In fact, $PTP$ is a self-adjoint operator on the finite-dimensional space $PH$, so has spectral projections in $A$ whose rank is the same as that of $P$ by minimality. Thus there is a unique such spectral projection; i.e. there is an $s \in \RR$ such that $PTP = sP$. Moreover, if $\xi,\eta \in PH$ are orthonormal, $R \in A$, then $\langle R\xi, \eta\rangle = s\langle\xi, \eta\rangle = 0$. Therefore $\langle A\xi, \eta\rangle = 0$, yet $A$ acts irreducibly, which is a contradiction. Therefore $P$ is a rank-$1$ projection.

    Now $R,S \in A$ implies that $RPS$ is a rank-$1$ projection. Thus $\overline{APAH} = H$. So $\overline{APA}$ is the set of rank-$1$ operators. Any compact operator can be written as an infinite linear combination of those.
\end{proof}

\begin{corollary}
    Let $A$ be a $C^*$-algebra and $\pi: A \to B(H)$ an irreducible representation. If $\pi(A)$ contains a nonzero compact operator then $\pi(A)$ contains $B_0(H)$.
\end{corollary}
\begin{proof}
    Let $I = \pi(A) \cap B_0(H)$. Then $I$ acts irreducibly on $H$ so by Burnside's theorem, $\overline{\pi(I)} = B_0(H)$. But $\pi$ takes $I/\ker \pi$ to $B_0(H)$ isometrically (since it is injective), so $\pi(I)$ is closed since $I$ is complete. Thus $\pi(I) = B_0(H)$.
\end{proof}

\section{The type-I dichotomy}
Let $A$ be a $C^*$-algebra. We write $\hat A$ to denote the set of all isomorphism classes of irreducible representations of $A$.

\begin{definition}
    Let $\pi: A \to B(H)$ range over $\hat A$. We say that $A$ is \dfn{liminal} or \dfn{CCR} if for every $\pi$, $\pi(A) = B_0(H)$. We say $A$ is \dfn{postliminal} or \dfn{GCR} if for every $\pi$, $B_0(H) \subseteq \pi(A)$. We say $A$ is \dfn{antiliminal} or \dfn{NCR} if for every $\pi$, $B_0(H) \cap \pi(A) = 0$.

    If $A$ is a von Neumann algebra, we say that $A$ is \dfn{type-I} if $A$ is postliminal. We say that $A$ is \dfn{non-type-I} if $A$ is not postliminal.
\end{definition}




\newpage
\printindex

\end{document}