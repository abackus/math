\documentclass{ip-journal}
\usepackage{graphicx}
\usepackage{tikz-cd}
\usepackage{mathtools}
\usepackage{amsfonts}
\usepackage{verbatim}
\usepackage{textcomp}
\usepackage{amssymb}
\usepackage{cite}
%\usepackage{color}
\usepackage[all]{xy}
%\usepackage{showlabels}
\usepackage{hyperref}
\usepackage{soul}
% \hypersetup{
%     colorlinks,
%     citecolor=black,
%     filecolor=black,
%     linkcolor=black,
%     urlcolor=black
% }
\usepackage{tikz}
\usepackage[all]{xy}
\usetikzlibrary{backgrounds}
% 
%\setlength{\oddsidemargin}{.25in} 
%\setlength{\evensidemargin}{.25in}
%\setlength{\textwidth}{6in}
%\setlength{\topmargin}{-0.2in} \setlength{\textheight}{9in}
%\vfuzz2pt % Don't report over-full v-boxes if over-edge is small
%\hfuzz2pt % Don't report over-full h-boxes if over-edge is small
% THEOREMS -------------------------------------------------------
\newtheorem{theorem}{Theorem}[section]
\newtheorem{lemma}[theorem]{Lemma}
\newtheorem{proposition}[theorem]{Proposition}
\newtheorem{corollary}[theorem]{Corollary}
\newtheorem{problem}[theorem]{Problem}   
%\newtheorem{definition}[theorem]{Definition}
\newtheorem{assumption}[theorem]{Assumption}
\newtheorem{claim}[theorem]{Claim}
\allowdisplaybreaks
\newtheorem{notation}[theorem]{Notation}
\newtheorem{conjecture}[theorem]{Conjecture}
\newtheorem{question}[theorem]{Question}
\newtheorem{thmconj}[theorem]{Theorem-Conjecture}




\newtheorem{keytechnicallemmaformodelspace}[theorem]{The Key Technical Lemma for Model Space}
\newtheorem{keytechnicallemma}[theorem]{The Key Technical Lemma}
\theoremstyle{definition}
\newtheorem{definition}[theorem]{Definition}
\newtheorem{remark}[theorem]{Remark}
\newtheorem{example}[theorem]{Example}

% \newenvironment{proofET}%
%{{\sc Proof of Theorem~\ref{thm:mingrad}}}
%{{\sc q.e.d.} \\}
 \newenvironment{thm:mingrad}
 {{\sc Proof of Theorem~\ref{thm:mingrad}.}}
{{\sc q.e.d.} \\}
\newenvironment{transmeasure}
 {{\sc Proof of Theorem~\ref{transmeasure}.}}
{{\sc q.e.d.} \\}
 \newenvironment{thm:supptmeasure} 
{{\sc Proof of Theorem~\ref{thm:supptmeasure}.}}
{{\sc q.e.d.} \\}
\newenvironment{straightline} 
{{\sc Proof of Theorem~\ref{straightline}.}}
{{\sc q.e.d.} \\}
 \newenvironment{princregconv}
{{\sc Proof of Corollary~\ref{princregconv}.}}
{{\sc q.e.d.} \\}
\newenvironment{proofappendix2}%
{{\sc Proof of Lemma~\ref{federeroutermeasure}.}}%
{{\sc q.e.d.} \\}
\newenvironment{proofappendix3}%
{{\sc Proof of Lemma~\ref{cd2general}.}}%
{{\sc q.e.d.} \\}
\newenvironment{proofappendix4}%
{{\sc Proof of Lemma~\ref{usordv}.}}%
{{\sc q.e.d.} \\}
  \newenvironment{chika}%
{{\sc Proof of Corollary~\ref{corolgeodesic}.}}%
{{\sc q.e.d.} \\}
 \newenvironment{thmgeodesic}%
{{\sc Proof of Theorem~\ref{thmgeodesic}.}}%
{{\sc q.e.d.} \\}
 \newenvironment{bach2}%
{{\sc Proof of Theorem~\ref{hmco2}.}}%
{{\sc q.e.d.} \\}
  \newenvironment{rick}%
{{\sc Proof of claim~(\ref{claim:bdvar}).}}%
{{\sc q.e.d.} \\}
 \newenvironment{linebundle}%
{{\sc Proof of Theorem~\ref{conversetomeasure}.}}%
{{\sc q.e.d.} \\}
\numberwithin{equation}{section}
\newtheorem*{theorem*}{Theorem}
  \newenvironment{ne1}%
{{\sc Proof of Lemma~\ref{tri1}.}}%
{{\qed} \\}
\newenvironment{ne2}%
{{\sc Proof of Theorem~\ref{regularity}.}}%
{{\qed} \\}
\newenvironment{proof:main}%
{{\sc Proof of Theorem~\ref{theorem:mainexistence}.}}%
{{\qed} \\}
\newenvironment{proof:pluriharmonic}%
    {{\sc Proof of Theorem~\ref{theorem:pluriharmonic}.}}%
  {{\qed} \\}  
   \newenvironment{proofof(iv)}%
    {{\sc Proof of $(iv)$.}}%
  {{\qed} \\}  
% MATH -----------------------------------------------------------
\newcommand{\norm}[1]{\left\Vert#1\right\Vert}
\newcommand{\abs}[1]{\left\vert#1\right\vert}
\newcommand{\R}{\mathbb R}
\newcommand{\RR}{\mathcal R}
\newcommand{\Z}{\mathbb Z}
\newcommand{\T}{\mathbb T}
\newcommand{\PP}{\mathbb P}
\newcommand{\N}{\mathbb N}
\newcommand{\C}{\mathbb C}
\newcommand{\D}{\mathbb D}
\newcommand{\Q}{\mathbb Q}
\newcommand{\Sp}{\mathbb S}
\newcommand{\Ker}{\mathcal G}
\newcommand{\B}{\mathcal B}
\newcommand{\oT}{\mathcal T}
\newcommand{\ep}{\epsilon}
\newcommand{\eps}{\varepsilon}
\newcommand{\lam}{\lambda}
\newcommand{\ra}{\rangle}
\newcommand{\la}{\langle}
\newcommand{\ttau}{{\tilde \tau}}
\newcommand{\cV}{\frac{1}{2\pi}E^{\gamma_{j}}[\Sp^1]}
\newcommand{\cVV}{\frac{1}{2\pi}E^{\gamma_{j}}[\Sp^1]+\frac{1}{2\pi}E^{\gamma_{i}}[\Sp^1]}
% \voffset -.5in

% \voffset -.5in



\begin{document}

\title[Transverse measures]{Transverse Measures and Best Lipschitz and Least Gradient Maps}
\author[Daskalopoulos]{Georgios Daskalopoulos}
\address{Department of Mathematics 
                 Brown University, 
                 Providence, RI}%02912}
%\email{georgios_daskalopoulos@brown.edu}
\author[Uhlenbeck]{Karen Uhlenbeck}
\address{Department of Mathematics University of Texas, Austin, TX and Distinguished Visiting Professor Institute for Advanced Study, Princeton, NJ}
%\email{uhlen@math.utexas.edu }

\thanks{
GD supported in part by NSF DMS-2105226.}
\maketitle

\begin{abstract}
Motivated by  work of Thurston on defining a version of Teichm\"uller theory based on best Lipschitz maps between surfaces, we study infinity-harmonic  maps from a hyperbolic manifold to the circle. The best Lipschitz constant is taken on a geodesic lamination.  Moreover, in the surface case the dual problem leads to a function of least gradient  which defines a transverse measure on the lamination.  We also discuss  the construction of least gradient functions from transverse measures via primitives to Ruelle-Sullivan currents.
\end{abstract}

\section{Introduction}
%This is the first in a series of papers, where we attempt to connect three different concepts: Best Lipschitz (or $\infty$-harmonic functions), maps of least gradient and measured laminations. The original motivation is to try to understand the work of Thurston \cite{thurston} and eventually develop the analytic techniques 
%This paper fails short of this goal. The analytic subtleties in deal with infinity harmonic maps between surfaces 
%This paper is inspired by the work of Thurston \cite{thurston} on the connection between measured laminations and best Lipschitz maps.

Bill Thurston  in a 1986 preprint which was later revised in 1998 (cf. \cite{thurston})  introduced best Lipschitz maps as a tool of studying an $L^\infty$-version of Teichm\"uller theory. His motivation was to replace the Teichm\"uller distance between two conformal structures $\rho$ and $\sigma$  defined as the dilatation of the Teichm\"uller map by the best Lipchitz constant of a map in the homotopy class of the identity between the corresponding hyperbolic surfaces.  This is what is now known as {\it Thurston's asymmetric metric}. This theory has been further developed by the Thurston school, see for example \cite{papa} or \cite{kassel} and the references therein. We were drawn to  the subject of this paper in part after discussions with Athanase Papadopoulos by the possibility of developing some analytic understanding of this theory. However, even the most basic tools in partial differential equations for approaching this subject are lacking. 

As a starting point, we looked at the subject of this paper: Best Lipschitz maps $u: M^2 \rightarrow S^1$ (or more generally functions  $\tilde u: \tilde M^2 \rightarrow \R$ equivariant under a cohomology class $\rho: \pi_1(M) \rightarrow \R$). This problem does not seem to have been treated in the topological literature (however see Problem~\ref{toyprobm} in the last section on how it fits with Thurston), but our results fit in nicely with existing concepts. In analysis, the subject of $\infty$-harmonic maps, albeit for maps $u: \Omega \rightarrow \R$ for $\Omega$ a domain in $\R^2$ (or even $\R^n$), are in place. See for example: \cite{crandal}, \cite{evans-savin} and \cite{evans-smart}. 

On the opposite side of $\infty$-harmonic maps lies the theory of 1-harmonic maps, which are also known as maps of least gradient. Here, the flavor of the analysis is completely different, see for example \cite{ziemer}, \cite{ziemer2} and \cite{mazon} among other references. See also \cite{juutinen} for a variational construction of least gradient functions for Euclidean domains obtained as limits of $q$-harmonic functions where $q \rightarrow 1$. 

 Thurston  conjectured  that the duality between best Lipschitz maps and measures  should fit well in his theory of geodesic laminations, as stated in the introduction of his paper \cite{thurston}:

{\it I currently think that a characterization of minimal stretch maps should be possible in a considerably more general context (in particular, to include some version for all Riemannian surfaces), and it should be feasible with a simpler proof based more on general principles-in particular, the max flow min cut principle, convexity, and $L^0 \leftrightarrow L^\infty$ duality.}

The goal of this paper is to exhibit the duality proposed by Thurston between best Lipschitz maps ($L^\infty$) and Radon measures ($L^0$) explicitly in the simpler case of functions  and how it fits with the theory of measured laminations and transverse cocycles developed by Thurston and Bonahon. See for example \cite{thurston2}, \cite{bonahon1} and \cite{bonahon2}. In fact,  we will exhibit  the duality  explicitly between the $\infty$-harmonic map $u$ and its dual least gradient map $v \in BV$ inducing the Radon measure $dv$: {\it The map $u$ defines the geodesic lamination, whereas $dv$ the transverse measure on the geodesic lamination defined by $u$.} The duality is more or less given by Hodge duality, namely $*du=dv$ as it will be made precise in this paper.

%Throughout the paper, we use some basic constructions from differential topology, which we review here as not all readers will have them at their finger tips. The cohomology class of $u: M \rightarrow S^1$ is the cohomology class associated with the closed one form $du$ via Hodge theory. There is a pairing of this cohomology class with the homology class of closed curves $\gamma$ via 
%\[
%<du,\gamma> = \int_M du_{\gamma(t)}(\gamma'(t))dt. 
%\]
%By Stokes' theorem, this is invariant of the choice of $u$ and the curve $\gamma$ in a homotopy class. An arbitrary closed form (written $dv$ in the same misuse of notation) represents a dual cohomology class via the Hodge star operator, from which we can construct  a flat line bundle $L$. The function $v$ is a section of this bundle, and $dv$ pairs with $\gamma$ in the same way as $du$ did.
%
%However, we need a slightly more general construction. Given a closed one form $\beta$ on $M$,  $d + \beta$ can be seen as a connection on a principal bundle $E$ with fiber $S^1$.   If the cohomology class of $\beta$ is integral, then a special case is the map $u:M \rightarrow S^1$ where $\beta=du$. In any case, $u: M \rightarrow  E$ is equivalent (up to a constant) to the lift $\tilde u: \tilde M \rightarrow \R$, where $\tilde u$ is equivariant under the periods of $\beta$. So we can lift the circle bundle $E$ and talk instead about an affine bundle and $u$ becomes a section of the associated affine bundle.  The reason is that, when we talk about distributions which are sections of an affine line bundle, only the second formulation is available to us.  We remark as an aside to the analysts, that when we integrate by parts over $M$, boundary terms vanish because of this invariance.

Following other authors, we first study the limits of the critical points $u_p$ of the functional
\[
u \mapsto \int_M |du|^p *1.
\] 
The $L^\infty$ norms of $du_p$ are uniformly bounded, and we obtain a set of weak limits $u$ of the $\infty$-harmonic equation. The function $u$ is Lipschitz and, as proven by Evans, Savin and Smart in \cite{evans-savin} and \cite{evans-smart}, for Euclidean metrics $u$ is differentiable. However, if $n>2$ it is not known that $du$ is continuous,  and even in the case $n=2$,
$du$ is only H\"older continuous. Since the above results are not available  for non-flat metrics we 
bypass this problem by developing the bare minimum of the theory of comparison with cones for the hyperbolic metric. In particular, we show that, as in the case of Euclidean domains,  there exists a notion of  gradient flow in this setting without assuming any differentiability. 
  
Following Thurston, in order to connect  with topology, in (\ref{normlength}) we introduce the invariant
 \begin{equation}\label{defKref}
  K= \sup  \frac{|<du,\gamma>|}{length(\gamma)}
\end{equation} 
 where  the $\sup$ is over  the set of free homotopy classes of simple closed curves  $\gamma$ in $M$.
Then, we are able to show:
 \begin{theorem}\label{Theorem 1.1} For $M$ compact, hyperbolic in any dimension, and $u$ an $\infty$-harmonic map constructed as a limit of $u_p$,  the local Lipschitz constant $L_u$ of $u$ (see Section~\ref{lipconst}) takes on its maximum $L$ on a geodesic lamination $\lambda_u$. Moreover, $ K$ is equal $L$, the Lipschitz constant of $u$.
 \end{theorem}
 

This is a combination of Theorems~\ref{straightline} and~\ref{K=L}.
We then turn to the dual problem in dimension 2.  A  duality of this form has been noticed before by Aronson and Lindqvist as far back as 1988 (cf. \cite{aronsonlin}), but it may go even further  back to Werner Fenchel in 1949 (cf. \cite[p.81-82]{temam} and \cite{fenchel}).  From the Euler-Lagrange equations for $u_p$, we know $|du_p|^{p-2}*du_p$ is closed, and we normalize it and set it equal to $dv_q$, where $v_q$ lifts to a function with factors of automorphy varying with $q$. Here $1/p + 1/q = 1$. The function $v_q$ is a critical point of $\int_M |dv|^q*1$ where now $q \rightarrow 1$. This leads to one of our main theorems:
\begin{theorem}\label{Theorem 1.2}
 The set of weak limits $v_q \rightarrow v$ as $q \rightarrow 1$ is nonempty. A limit $v$ is  of least gradient  among maps defining the same homology class. The support of $dv$ is on the  lamination $\lambda_u$  on which $L_u = L$,  obtained from any dual $\infty$-harmonic map $u$.
\end{theorem}
This is a combination of Theorem~\ref{lemma:limmeasures2}, Theorem~\ref{thm:supptmeasure} and  Theorem~\ref{thmlegr}.

Note that so far we have restricted ourselves to the case of maps to $S^1$. This was done for the sake of simplicity only and everything can be generalized to arbitrary real cohomology classes $\rho \in H^1(M, \R)$. In other words, we can replace best Lipschitz maps $u: M \rightarrow S^1$ by functions $\tilde u$ defined on the universal cover that are equivariant under a homomorphism $\rho:\pi_1(M) \rightarrow \R$. In Section~\ref{sectevp} we describe this extension of our results to the equivariant case.


Finally, we turn to the concepts in the Thurston literature.  We assume $M = M^2$ is a hyperbolic surface and define transverse measures in Section~\ref{sect7}. The next is one of the main results of the paper:
\begin{theorem} A least gradient map $v$ as in Theorem~\ref{Theorem 1.2}  induces a transverse measure on the naturally oriented geodesic lamination on which $L_u$ takes on the best Lipschitz constant $L$.
\end{theorem}
This is Theorem~\ref{transmeasure}. We show that $v$ is constant on the connected components of the complement of the lamination $\lambda_u$ from which we can construct a transverse cocycle in the sense of \cite{bonahon2}. We use the approximation by $v_q$ to show that it is non-negative and thus defines a transverse measure.  In addition, we discuss the connection between measured laminations and functions of bounded variation.

 More precisely,  we construct a measure $\nu$ on admissible transversals $f:[c,d] \rightarrow M.$  In the universal cover, we have a function of bounded variation $v.$  We show that for an admissible transversal $f$, $g = f^*v$ is a function of bounded variation and we define the transverse measure $\nu(f)$ as the norm of $g = f^*v$ on the interval.  Every function of bounded variation $g$ on an interval can be written as the sum of a non-increasing function $g^+$ and  non-decreasing function $g^-$, and the norm  is simply $|g^+(d) - g^+(c)| + |g^-(d) - g^-(c)|.$  This norm is invariant under homotopy through admissible transversals.  The difficulty is to match the topological definition of transversal and transverse measure with the analytical definition of  bounded variation.

\begin{theorem}\label{bird} A function $\tilde v$ on the universal cover of a hyperbolic surface that is equivariant, locally bounded and constant on the plaques of an oriented lamination defines a transverse cocycle $\nu$. Moreover,  if  $\nu$  is a transverse measure, then $\tilde v$ is locally of bounded variation.
  \end{theorem}

This is proved in Theorem~\ref{thm:tranco} and Theorem~\ref{transcoismeasthm}. 
We then  prove a partial converse to Theorem~\ref{bird}. In a 1975 paper, Ruelle and Sullivan constructed in a very general setting  closed currents from transverse measures. We show that we have enough regularity to make this rigorous in the setting of transverse measures on geodesic laminations of surfaces.
 More precisely:

 \begin{theorem}The Ruelle-Sullivan current associated to an oriented geodesic lamination in a hyperbolic surface is well defined and closed. A primitive $v$ of the  Ruelle-Sullivan current exists and is locally of bounded variation.
 \end{theorem} 
 This is a combination of Theorem~\ref{welldefcur} and Theorem~\ref{conversetomeasure}. We expect to show that for an appropriate choice of orientation $v$ is always a least gradient in a future paper.   
 
 
 We also point out that the decomposition of measured laminations into minimal components  corresponds to the decomposition of functions of bounded variation
 \[ 
 v = v_{jump} + v_{cantor},
 \]
 where $dv_{jump}$ has support on closed geodesics in the lamination and $dv_{cantor}$ has support on the minimal components with leaves infinite  geodesics.
 We end the paper by giving a  long list of open problems. 
 \vspace{.1 in}
 
A brief outline of the paper is as follows:
\begin{itemize}
\item Section~\ref{sect:pharmmaps}: {\it$p$-harmonic maps (and their limits).}
This is a review of the properties of the $p$-harmonic equation and its limits as $p \rightarrow \infty$. We also prove a useful maximum estimate needed in Section~\ref{consmeasure}.
\item Section~\ref{sect:conjug}: {\it The conjugate equation for finite $q$.} We define the dual harmonic map for $1/p+1/q=1$ and introduce the adapted coordinate system. 
We also discuss the flat structure induced by the coordinate $(u_p,v_q)$.
\item Section~\ref{qgoesto1}: {\it The limit $q \rightarrow 1$.}
The limiting map of bounded variation is constructed.
\item Section~\ref{sect:crandal}: {\it Geodesic laminations associated to the $\infty$-harmonic map.}
In this section, $M$ is hyperbolic of any dimension. We study comparison with cones and provide a proof of Theorem~\ref{Theorem 1.1}.
\item Section~\ref{consmeasure}:  {\it The concentration of the measure.}
A straightforward but surprising application of the Euler-Lagrange equations for $u_p$ and $u$. The statement is roughly that  small $L^1$-norm implies that the dual measure $dv$ has support on the lamination and is in a weak sense  orthogonal to $du$. We are able to apply this to properties of $v$, for example to show that $v$ is of least gradient.
\item Section~\ref{sect7}: {\it Construction of the transverse measure from the least gradient map.}
This is a tricky section, as it necessitates forming a bridge between the concepts in analysis and the concepts in topology. To evaluate a measure $dv$ on a curve $\gamma$, analysis usually requires the derivative of $\gamma$ to exist, whereas it is important in topology to define the measure on continuous transversals.
\item Section~\ref{ruelles}: {\it From transverse measures to functions of bounded variation.}
 We construct the Ruelle-Sullivan current and show that we have enough regularity to make this rigorous in the setting of transverse measures on laminations. We also construct a primitive to the Ruelle-Sullivan current and  discuss the role of BV functions on  transverse measures.
\item Section~\ref{conjectures}: {\it Conjectures and open problems.}
We give a list of some problems we think we can solve given enough time. The last few problems are enticing. Where there is some analysis, there is little topology, and vice versa.
\end{itemize}
{\bf Acknowledgements.}
Many thanks to those who helped us untangle both the analysis and the topology. Special thanks goes to Craig Evans, Camillo de Lellis, Athanase Papadopoulos, Rafael Poitre and Ovidiu Savin for useful conversations. We have enjoyed working on this project and hope others will appreciate it as well.


\section{$p$-harmonic maps}\label{sect:pharmmaps}
In this section we collect basic facts about different types of harmonic functions. We review the notion of $p$-harmonic functions both for finite $p$ and $p=\infty$. Solutions for finite $p$ obey the theory of elliptic differential equations (cf. \cite{uhlen}), whereas  solutions  to the $\infty$-Laplacian are constructed as limits of  harmonic functions for finite $p$. For the Euclidean domain metric, the local theory of the $\infty$-Laplacian is well-known to analysts. See for example, \cite{arcrju}, \cite{crandal}, \cite{jensen}, \cite{evans-savin}, \cite{evans-smart} and \cite{lindqvist} and all the references therein. The complication in our situation comes from the fact that the maps we are considering take values in $S^1$ instead of $\R$ and also that the domain metric is non-Euclidean. 
%In fact, we will have to work out from scratch  the results we need about infinity harmonic functions from hyperbolic manifolds and avoiding to use the theory that so far has only been developed for Euclidean domains. 

\subsection{$p$-harmonic maps to the circle} Let $(M, g)$ be a closed  smooth Riemannian manifold  of dimension $n \geq 2$ and let $\tilde M$ denote its universal cover with the induced Riemannian metric. By a {\it {  fibration of $M$ over the circle}} we mean a non trivial homotopy class of maps   
\[
f: M \rightarrow S^1.
\]
Note that we are not assuming apriori that $f$ is a submersion.
Equivalently, and for
\[
\rho=f_*:\pi_1(M) \rightarrow \Z=\pi_1(S^1)
\]
we can consider instead the class of $\rho$-equivariant maps
\[
\tilde f: \tilde M \rightarrow \R,
\]
i.e maps satisfying
\begin{equation*}\label{equiccmd}
\tilde f(\gamma \tilde x)=  \tilde f( \tilde x)+ \rho(\gamma), \ \forall \gamma \in \pi_1(M)\  \mbox{and} \   \forall \tilde x \in \tilde M.
\end{equation*}
On the space 
 $W^{1,p}(M, S^1)$, $n<p< \infty$ of maps in a fixed  homotopy class 
% defined by $\rho$ as the space of $W^{1,p}_{loc}$-functions on the universal cover satisfying 
% (\ref{equiccmd}) almost everywhere. Since $|df|$ is invariant under $\pi_1(M)$
 we consider the functional
\begin{equation}\label{pharmfun}
J_p(f)=\int_M|df|^p *1.
\end{equation}
 A unique minimizer $u_p$ of the functional $J_p$ exists in the homotopy class and is 
 called a {\it{$p$-harmonic map.}} It satisfies the equation
\begin{equation}\label{pharm}
div(|\nabla u_p|^{p-2}\nabla u_p)=d^*(|du_p|^{p-2}du_p)=0.
\end{equation}

The existence of $u_p$ is  standard.  Consider a minimizing sequence $u^j$ of $J_p$ in a homotopy class. This makes sense since $p>n$ and hence the maps are continuous. By weak compactness and lower semicontinuity  $u^j$  converge weakly in $W^{1,p}$ to a minimizer
$u_p$ which is in the same homotopy class. 
The argument above can be modified also in the general case $ p>1  $ by minimizing $J_p$  on the space of Lipschitz maps in the given homotopy class.  Since we are only interested in large values of $p$ we  omit the details.

%We can view $u^j$ as equivariant functions $\tilde u^j$ on the universal cover satisfying (\ref{equiccmd}). Fix a smooth fundamental domain $\digamma \subset \tilde M$ for the action of $\pi_1(M)$.  Given $W \subset \subset \tilde M$, we can choose by compactness a finite set of $\gamma \in \pi_1(M)$ such that $W \subset \cup_\gamma \gamma \digamma$. We have a uniform bound of te $L^p$ norm of $du^j$ on $W$.

%By replacing   $\tilde u^j$ by $\tilde u^j-c^j$ where $c^j$ is the average value of $\tilde u^j$ on $W$ and applying the Poincare inequality, we may assume that $\tilde u^j$  are bounded in $W^{1,p}$. Hence there is subsequence converging weakly to a function on $W^{1,p}(W)$minimizer $\tilde u_p$. By a diagonalization argument we can prove that there exists a map $\tilde u_p$ defined on the universal cover such that 
%$\tilde u^j$  converges to $\tilde u_p$ in $W^{1,p}_{loc}$. The pointwise convergence of $\tilde u^j$ to $\tilde u_p$ implies that $u_p$ satisfies the same equivariance property (\ref{equiccmd}) almost everywhere. By restricting to the fundamental domain $\digamma$ and applying semicintinuity of energy we obtain that $\tilde u_p$ is a minimizer.


There is an abundance of literature on regularity of $p$-harmonic functions and $p$-harmonic maps.
%\begin{proposition}[cf. \cite{aless}, Proposition 3.3] Assume $n=2$. There exist constants $\alpha$, $k$, $0<\alpha \leq 1$, $0\leq k <1$
% depending only  on $p$, and $g$ such that for any coordinate neighborhood $\Omega$ there exist and $s, h \in  C^\alpha(\Omega)$ such that 
% \[
% (u_p)_z = e^sh, 
% \]
% where $h$ is a $k$-quasiregular mapping in $\Omega$.
%\end{proposition}
 
% By  \cite{aless}, Proposition 2.5 the real and imaginary part of $h$ have isolated $k$-prong singularities with $k>0$. By an Euler characteristic argument we conclude that there are finitely many singular points. Thus,

\begin{theorem} \label{nonsingular} Let
$u_p: (M, g) \rightarrow S^1$
denote the $p$-harmonic map in the homotopy class of $f: M \rightarrow S^1$. Then,  $u_p \in C^{1, \alpha}$ and if $\Omega \subset \subset \{|du_p| \neq 0\}$, then $u \in  C^\infty(\Omega)$ (cf. \cite{uhlen}). If $n=2$, then  the number of singular points $|du_p|= 0$ is finite and bounded by the Euler characteristic of $M$ (cf. \cite{manfredi} and \cite{aless}).  
\end{theorem}

\subsection{ Best Lipschitz maps and $\infty$-harmonic maps}\label{lipconst}
For K  a subset of a Riemannian manifold  $(M, g)$,  and $f: K \rightarrow S^1$,  its   {\it{Lipschitz constant}} in $K$  is defined by
\[
L_f(K):= \inf \{ L \in \R : d_{S^1}(f(x), f(y)) \leq Ld_g(x,y) \  \forall x, y \in K \} .
\]
In the above, $\inf \emptyset = +\infty$. We say $f$ is Lipschitz in $K$, if 
$L_f(K) < +\infty$. We write 
\[
L_f=L_f(M)
\]
for the {\it global Lipschitz constant.}
For $U$ be an open subset of $M$ and $x \in U$, we define the {\it{local Lipschitz constant }}
\[
L_f(x) := \lim_{r \rightarrow 0} L_f(B_r (x)).
\]
Clearly, if $f$ has Lipschitz constant $L$ in $U$, then
$L_f(x) \leq L.$


\begin{proposition}\label{crandal0}[\cite{crandal}, Lemma 4.3] For any function $f:U \rightarrow \R$, 
\begin{itemize}
\item $(i)$ the map $x \mapsto L_f(x)$ is upper semicontinuous.
\item $(ii)$  $ df \in  L^\infty(U)$ holds in the sense of distributions if and only if $L_f(x)$ is bounded
on $U$;  then
\[
\sup_{x \in U} L_f(x)= |df|_{L^\infty(U)}\ \mbox{and} \ L_f(x)=\lim_{r \rightarrow 0}|df|_{L^\infty(B_r(x))}.
\]
\item $(iii)$ If $U$ is convex, then
\[
 L_f(U)=|df|_{L^\infty(U)}.
\]
\end{itemize}
\end{proposition}

%In this subsection $(M, g)$ continues to denote a closed Riemannian manifold of dimension $n>1$  and  
%$\tilde M \simeq \R^n$ denotes its universal cover with the metric induced from $g$.
\begin{definition}\label{defbl} 
The infimum of the global Lipschitz constant $L_f$ for all $f:M \rightarrow S^1$ in a fixed homotopy class is called the {\it{best Lipschitz constant}}. A Lipschitz map
\[
u: M \rightarrow S^1
\]
is called {\it{a best Lipschitz  map}},  if for any  Lipschitz map $f: M \rightarrow S^1 $  homotopic to $u$ 
\begin{equation*}
L_u \leq L_f. 
\end{equation*}

\end{definition}

%\begin{definition}\label{infinharm} A Lipschitz map
%\[
%u: M \rightarrow S^1.
%\]
%is called $\infty$-harmonic (resp. $\infty$-subharmonic or $\infty$-superharmonic) if its lift to the universal cover
%\[
%\tilde u: \tilde M \rightarrow \R
%\]
%is  an $\infty$-harmonic (resp. $\infty$-subharmonic or $\infty$-superharmonic) function. By this we mean a viscosity solution (resp. subsolution or supersolution) of the 
%\emph{$\infty$-Laplace equation}
%\[
%\triangle^\infty \tilde u= \frac{1}{2} <grad(\tilde u),  grad(|grad(\tilde u)|^2)>_g= 0, 
%\]
%where $<,>_g$ denotes the inner product associated with $g$. For more details on the notion of viscosity solutions we refer to \cite{crandal} or \cite{lindqvist}.
%\end{definition}
%
%In this paper we will only deal with a special type of viscosity solutions that are also variational solutions to the $\infty$-Laplace equation in a fixed homotopy class. The basic construction  is summarized in the following

\begin{theorem}\label{thm:limminimizer}
Let $(M, g)$ be a closed  smooth Riemannian manifold  of dimension $n \geq 2$.  
For each $p>n$, let $u_p$ be the $p$-harmonic map homotopic to  
 a Lipschitz map $f: M \rightarrow S^1$. Given a sequence $p \rightarrow \infty$, there exists a subsequence (denoted again by $p$)  and a  Lipschitz map $u: M \rightarrow S^1$ such that:
\begin{itemize}
\item $(i)$ $u_p \rightarrow u$ uniformly.
\item $(ii)$ $u$ is  best Lipschitz with Lipschitz constant equal to the best Lipschitz constant. Furthermore, $u$ also minimizes the Lipschitz constant for the local  Dirichlet problem subject to its own boundary conditions.
\item $(iii)$ $du_p  \rightharpoonup du \ \mbox{and} \ *du_p \rightharpoonup *du \ \mbox{weakly in} \  L^s \ \forall s>n$.
\item  $(iv)$ $du$ is closed. 
\end{itemize}
\end{theorem}
\begin{proof} We follow directly the proof of the existence of $\infty$-harmonic functions for the Dirichlet problem (cf. \cite{lindqvist}, Chapter 3).  We only give a sketch:
Take a sequence $p\rightarrow \infty$ and $\epsilon >0$. By H\"older, and the fact that $u_p$ is a minimizer of $J_p$, we have for $n<s<p$ large,   
\begin{eqnarray*}
\frac{1}{vol(M)^{1/s}}J_s(u_p)^{1/s} &\leq& \frac{1}{vol(M)^{1/p}} J_p(u_p)^{1/p} \\
&\leq& \frac{1}{vol(M)^{1/p}} J_p(f)^{1/p} \\
&\leq&  |df|_{L^\infty}+\epsilon.
\end{eqnarray*}
 Hence, $|du_p|_{L^s}$ is uniformly bounded and thus, after passing to a subsequence 
(denoted again by $p$),
\[
du_p \rightharpoonup du \ \mbox{weakly in} \  L^s.
\]
By semicontinuity,
\[
 |du|_{L^s} \leq \liminf  |du_p |_{L^s} \leq |df|_{L^\infty}+2\epsilon.
\]
By a diagonalization argument, we can choose a single subsequence $p$ such that
\[
du_p \rightharpoonup du  \ \mbox{weakly in}  \ L^s,\ \forall s
\]
 and by taking $s \rightarrow \infty$,
\[
|du |_{L^{\infty}} \leq |df|_{L^\infty}.
\]
By going to the universal cover, the same inequality holds for $\tilde u$ and any $\tilde f$. Thus, by the convexity of $\tilde M$ and
 the mean value theorem (cf. Proposition~\ref{crandal0}, $(iii)$), it follows that $u$ is a best Lipschitz map with best Lipschitz constant $L_u=L$. 

Moreover, as in \cite{lindqvist}, Theorem 3.2, $u$ is also a local minimizer for the Dirichlet problem subject to its own boundary conditions.  
Properties $(iii)$-$(iv)$ follow immediately from the argument sketched above.
%To show that $du$ is unique, suppose $u$ and $u'$ are two best Lipschitz maps obtained as weak sequential limits of $u_p$. Lift to the universal cover, to $\tilde u$ and $\tilde u'$. In a fundamental domain
%\[
%\max \tilde u' - \min \tilde u = m
%\]
%and because $\tilde u$ and $ \tilde u'$ are equivariant with respect to the same representation, this is true on all of $\tilde M$. Hence
%\[ 
%\max \tilde u' \leq \tilde u +m \ \mbox{on} \ \tilde M.
%\]  
% By continuity there is some $m'\leq m$ such that $\tilde u' - \tilde u\leq m'$ and some non-empty set of points $x$ such that $\tilde u'(x) - \tilde u(x)=m'$. But $\tilde u'-\tilde u$ descends to an infinity harmonic function on $M$. This violates the maximum principle for infinity harmonic functions (cf. \cite{lindqvist}, Proposition 6.2), unless 
%$\tilde u'(x) = \tilde u(x) + m'$ for all $x$. Hence $du = du'$.

\end{proof}

\begin{definition}\label{infinharm}
We call $u$ as in the previous theorem {\it{$\infty$-harmonic}}. Notice that in this paper we are only concerned with solutions that are limits of $p$-harmonic maps to $S^!$. Sometimes these are called {\it{variational solutions of the $\infty$-Laplace equation}}. If the domain is Euclidean then variational solutions are also {\it{viscosity solutions}} of the $\infty$-Laplace equation. We will not attempt to develop such a notion for non-Euclidean metrics in the present paper. For more details on the notion of viscosity solutions in Euclidean space we refer to \cite{crandal} or \cite{lindqvist}. For open problems we ask the reader to go to the last section.

\end{definition}

\begin{remark}If the domain metric is Euclidean it has been shown that $u$ has the additional properties
\begin{itemize}
\item  If $n=2$, then $du$ and $*du$ are in $C^\alpha$ (cf. \cite{evans-savin}).
\item  If $n>2$, then $du$  exists everywhere but is not known to be continuous (cf. \cite{evans-smart}).
\end{itemize}
It is very likely that these results also hold for the hyperbolic metric but since they only have been written  down carefully for the Euclidean metric  we will not use them in this paper.
\end{remark}
 
 \begin{lemma}\label{pintconvto}
\[
\lim_{p \rightarrow \infty} \left( \int_M |du_p|^p*1 \right )^{1/p} =  L.
\]
\end{lemma}
\begin{proof}If $f$ denotes a  best Lipschitz map in the homotopy class, the fact that $u_p$ is a minimizer for $J_p$ implies
\[
\left( \int_M |du_p|^p*1 \right )^{1/p}\leq \left( \int_M |df|^p*1 \right )^{1/p} \leq (vol M)^{1/p} L.
\]
Hence the $\limsup$ is less than equal to $L$. On the other hand, if $\liminf=a<L$, then proceeding as in the proof of the previous theorem, there exists a Lipschitz map $u$ such that
\[
|du|_{L^\infty}\leq a<L
\]
which contradicts the best Lipschitz constant.
\end{proof}

\subsection{The Maximum estimate}

We know that the $p$-harmonic maps $u_p$ are smooth away from their critical points.  However, in Section~\ref{consmeasure} we will need the following result:

\begin{proposition}\label{maxest}  $\lim_{p \rightarrow \infty} \max |du_p| = L.$
\end{proposition}

Since $L$ is the best Lipschitz constant $\max |du_p| \geq L$, so we need prove an upper bound.
Let $s = u_p/L.$   This simplifies the normalizations.  The size of $S^1$ does not enter into the calculations. %and
%\[
% ||w||_q = \left(\int_M |w|^q *1 \right)^{1/q}.
%\]

\begin{lemma} Let 
\[
w = |ds|^p = (|du_p|/L)^p.
\] 
Let $W^{1,2} (M) \subset L^{2a}(M)$, where $a$ is arbitrary for $dim M = 2$ and $n/(n-2)$ when $dim M > 2$.  Then
\[
     \max w = \lim_{l \rightarrow \infty} |w|_{L^l}  \leq Cp^{a/a-1}. 
\]
The constant $C$ depends only on the norm of the embedding and the Ricci curvature of $M$ and not on $p.$
\end{lemma}

\begin{proof}
The Proposition follows easily from the lemma, as
\[ 
\max |du_p| = \max w^{1/p}L \leq (Cp^{a/a-1})^{1/p} L \rightarrow L.
\]
The proof of the lemma is standard, using the Bochner formula and Moser iteration and only needs to be included to keep track of $p.$ So we will be brief. 

In the usual way, we integrate the Euler-Lagrange equations $d^*|ds|^{p-2}du = 0$ against a term $d^*\phi ds$ where $\phi$ is a non-negative function on $M.$ We integrate by parts and complete the Laplacian to obtain
\begin{eqnarray*}
\lefteqn{\int_M<\triangle (|ds|^{p-2}ds), \phi ds> *1}\\
&=&\int_M<d |ds|^{p-2} \wedge ds, d\phi \wedge ds>*1\\
&=&\int_M<d |ds|^{p-2}, d\phi> |ds|^2-<d |ds|^{p-2}, ds> <d\phi, ds>*1.
\end{eqnarray*}
Next use the Bochner formula for 1-forms and integrate by parts to obtain
\begin{eqnarray*}\label{myst1}
\lefteqn{\int_M<\nabla(|ds|^{p-2}ds), \nabla(\phi ds)>*1}\\
&=&-\int_M Ricc(ds,ds) \phi |ds|^{p-2}*1+\int_M<d |ds|^{p-2}, d\phi>  |ds|^2 \nonumber\\
&-&<d |ds|^{p-2}, ds> <d\phi, ds>*1.
\end{eqnarray*}
\\
After expanding the left hand side and bringing the second term to the left hand side, we obtain an expression
%\begin{eqnarray}\label{myst1}
%-\int_M Ricc(ds,ds)|ds|^{p-2}\phi*1=\int_M <D_2(|ds|^{p-2}d_1s),D_1(\phi d_2s)>*1
% \end{eqnarray} 
%%-------------------------------------------
%%
%%The proof of the Lemma is standard, using the Bochner formula and Moser iteration and only needs to be included to keep track of $p.$ So we will be brief. 
%%In the usual way, we integrate the Euler-Lagrange equations $d^*|ds|^{p-2}du = 0$ against a term $d^*\phi ds$ where $\phi$ is a non-negative function on M. We integrate by parts, exchange $\{D,D\}$ for a term involving Ricci curvature, and integrate back. \\
%% -----------
%% 
%%Alternatively, we can 
%%%start with the Bochner formula for 1-forms $dd^*+d^*d=\nabla^*\nabla+ Ricc$ and integrate $|ds|^{p-2}ds $ against a term $\phi ds $. Note that in both cases 
%We use the fact that $|ds|^{p-2}ds  \in W^{1,2}$.
%
%
%
%
%When  worked out we obtain an expression
%\begin{eqnarray}\label{myst1}
%-\int_M Ricc(ds,ds)|ds|^{p-2}\phi*1=\int_M <D_2(|ds|^{p-2}d_1s),D_1(\phi d_2s)>*1
% \end{eqnarray}
% Here the numbers 1 and 2 refer to the pairing in the inner product.\\
% -----------
%Integrate by parts, using the fact that $|ds|^{p-2}ds  \in W^{1,2}$ and the Euler-Lagrange equations
%\[ 
%d*|ds|^{p-2}ds = 0,
%\]
%to obtain an expression
 \begin{eqnarray}\label{myst11}
 \int_M A*1 &=&- \int_M  Ricc(ds,ds)\phi |ds|^{p-2}*1.
 \end{eqnarray} 
% In order to assure that the integrals above converge we use the fact that by \cite[Proposition 2]{ivan} $|ds|^{(p-2)/2}ds \in W^{1,2}$. By taking $d$ of the above expression 
% $d(|ds|^{(p-2)/2})ds \in L^2$ and hence also $\nabla(|ds|^{(p-2)/2}ds)-d(|ds|^{(p-2)/2})ds=|ds|^{(p-2)/2}\nabla ds \in L^2$.
% From this and the fact that $ds \in C^\alpha$ we can deduce that $|ds|^rds \in W^{1,2}$ for all $r \geq (p-2)/2$. 
% This justifies that the integral in (\ref{myst11})  converges.
% 
% We now proceed with the calculation.
%The integrand on the left-hand side has four terms when worked out
%\begin{eqnarray}
%A &=&  \phi |ds|^{p-2}((|\nabla ds|^2 +(p-2)|d|ds|)^2 + |ds|^{p-1}(d|ds|,d\phi) \nonumber\\
%   &+& <d|ds|^{p-2}ds, d\phi ds>.
%\end{eqnarray}
%\begin{eqnarray*}
%\lefteqn{ \phi (|ds|^{p-2}|\nabla ds|^2 +<d |ds|^{p-2}ds, \nabla ds>)} \nonumber\\
%   &+& |ds|^{p-2}<\nabla ds,d\phi ds> + <d|ds|^{p-2}ds, d\phi ds>\nonumber\\
%   &=& \phi |ds|^{p-2}(|\nabla ds|^2 +(p-2)(d|ds|)^2)  \\
%   &+& |ds|^{p-2}<\nabla ds,d\phi ds>  + <d|ds|^{p-2}ds, d\phi ds>\nonumber.
%\end{eqnarray*}
%\begin{eqnarray*}
%\lefteqn{ \phi (|ds|^{p-2}|\nabla ds|^2 +<d |ds|^{p-2}ds, \nabla ds>)} \nonumber\\
%   &+& |ds|^{p-2}<\nabla ds,d\phi ds> + <d|ds|^{p-2}ds, d\phi ds>\nonumber\\
%   &=& \phi |ds|^{p-2}(|\nabla ds|^2 +(p-2)(d|ds|)^2)  \\
%   &+& |ds|^{p-2}<\nabla ds,d\phi ds>  + <d|ds|^{p-2}ds, d\phi ds>\nonumber.
%\end{eqnarray*}
%\\
%-------
The integrand on the left-hand side has four terms when worked out.
\begin{eqnarray*}
A &=& \phi |ds|^{p-2}(|Dds|^2 +(p-2)|d|ds||^2) \\
&+& |ds|^{p-1}<d|ds|,d\phi>+ <d|ds|^{p-2},ds><d\phi,ds>.
\end{eqnarray*}
Recall $w = |ds|^p.$ We insert $\phi = w^{2l-1}, l\geq1/2$ in the expression.  Note
\[
dw = p|ds|^{p-1}d|ds|, \ d\phi = p(2l-1)|ds|^{2pl-p-1}d|ds|.
\]
and that all four terms in  $A$ are non-negative. We ignore the last term.
Using  the inequality $|d|ds|| \leq |D ds|$ we obtain,
\begin{eqnarray}\label{expA}
A &\geq&  |ds|^{p(2l-1)} |ds|^{p-2}(|D ds|^2 +(p-2)|d|ds||^2)  \nonumber\\
&+& |ds|^{p-1}<d|ds|,p(2l-1)|ds|^{2pl-p-1}d|ds|> \\
   &\geq&(p-1+p(2l-1))|ds|^{2pl-2}|d|ds||^2.  \nonumber
   \end{eqnarray}
   Note
\begin{eqnarray}\label{expw}
1/pl |dw^l|^2 &=&1/pl|lw^{l-1}dw|^2  \nonumber\\
&=&1/pl (l |ds|^{p(l-1)} p|ds|^{p-1}d|ds|)^2\\
&=& pl |ds|^{2pl-2}|d|ds||^2.  \nonumber
   \end{eqnarray}
 Using (\ref{expA}) and (\ref{expw})
 \begin{eqnarray}\label{myst3}
 \frac{1}{pl} |dw^l|^2 \leq \frac{(p-1) +(2l-1)p}{(pl)^2} |dw^l|^2 \leq A.
\end{eqnarray}
Also the right-hand side of (\ref{myst11}) is bounded
by
 \begin{eqnarray}\label{myst345}
 - \int_M  Ricc(ds,ds)\phi |ds|^{p-2}*1 \leq R |w^l|_{L^2}^2
 \end{eqnarray}
 and $R$ is the maximum of the negative Ricci curvature.
Combining (\ref{myst1}), (\ref{expA}), (\ref{myst3}) and (\ref{myst345}) with the Sobolev embedding theorem we get
\begin{eqnarray*}\label{myst4}
|w|_{L^{2la}}^{2l} &\leq& \gamma (|dw^l|_{L^2}^2 + |w^l|_{L^2}^2  )\\
&\leq& \gamma (pl R + 1) |w^l|_{L^2}^2 .
\end{eqnarray*}
Here $\gamma$ refers comes from norm of the Sobolev embedding.  We next simple take $1/2l$-root of this inequality to get
\[
      |w|_{L^{2la}} \leq (Cpl)^{1/2l} |w|_{L^{2l}}.
\]
Now let $l_0 = 1/2$ and $l_{i+1} = a l_i,$ and iterate the inequality. It is an easy exercise to see that 
\[
   |w|_{L^{a^{j+1}}} \leq (Cp)^{\sum_0^j  \frac{1}{a^i}} a^{\sum_0^j   \frac{i}{a^i}} |w|_{L^1}.
\]
The result follows from this.
\end{proof}



\section{The conjugate equation for finite $q$}\label{sect:conjug}
In this section $dim (M)=n=2$. Let $1<q \leq p< \infty$ such that $1/p + 1/q = 1$.
 For each $p$-harmonic map $u_p$, we construct  dual harmonic functions $\tilde v_q$ defined on the universal cover of $M$ and equivariant with respect to representations $\alpha_q: \pi_1(M) \rightarrow \R$. For functions on the plane, this duality has already appeared in  \cite{aronsonlin}. Our main result  is to show that the functions $\tilde v_q$ are locally uniformly  bounded and the representations $\alpha_q$ are uniformly bounded. Together, away from the zeroes of $\tilde u_p$,  the two functions $\tilde u_p$ and $\tilde v_q$ define a convenient coordinate system on the universal cover, called the {\it{adapted coordinate system.}}
\subsection{The conjugate harmonic equation}
Fix $2\leq p < \infty$ and define $0< q \leq 2$ by
\begin{equation}\label{form:conjugate}
\frac{1}{p}+\frac{1}{q}=1.
\end{equation}
Let $u_p$ be a minimizer of the functional 
$J_p$ in the homotopy class of a Lipschitz map $f: M \rightarrow S^1$.
Let 
\[
\tilde u_p: \tilde M \rightarrow \R
\]
be the lift of $u_p$ to the universal cover, equivariant under 
$\rho: \pi_1(M) \rightarrow \Z$. We define the dual 1-form $ \tilde \Psi_q$
\begin{equation}\label{dualform}
\tilde \Psi_q=|d \tilde u_p|^{p-2}*d\tilde u_p.
\end{equation}
\begin{lemma}  $ \tilde \Psi_q$ is a closed, invariant form under the action of $\pi_1(M)$, hence there exists a unique primitive
\[
\tilde w_q: \tilde M \rightarrow \R, \ \ d\tilde w_q= \tilde \Psi_q
\]
equivariant under the period homomorphism
\[
 \beta_q: \pi_1(M) \rightarrow \R; \ \ \beta_q(\gamma)=\int_z^{\gamma z} \tilde \Psi_q
\]
and normalized as
\[
\int_\digamma \tilde w_q *1=0
\]
where $\digamma$ is a fixed fundamental  domain in $\tilde M$.
\end{lemma} 
\begin{proof} The condition $d \tilde \Psi_q=0$ is just the $p$-harmonic equation (\ref{pharm}). The invariance of $\tilde \Psi_q$ follows from the fact that 
$\tilde u_p$ is the pullback of $ u_p$ to the universal cover. For the equivariance under the period homomorphism, see for example \cite{foster}, Section 10.
\end{proof}


\begin{lemma}\label{conjugate} The map $\tilde w_q$  satisfies the $q$-harmonic map equation (\ref{pharm}), for $q$ as in (\ref{form:conjugate}).
\end{lemma}
\begin{proof} 
Notice that  equation (\ref{form:conjugate}) implies
\[
(p-1)(q-2)+p-2=0,
\]
hence
\begin{eqnarray*}
|d\tilde w_q|^{q-2}*d\tilde w_q &=&
 \left| |d\tilde u_p|^{p-2}*d\tilde u_p \right|^{q-2}|d\tilde u_p|^{p-2}*^2d\tilde u_p  \\
&=&  |d\tilde u_p|^{(p-1)(q-2)+p-2} *^2 d\tilde u_p  \\
&=&   -d\tilde u_p.  
\end{eqnarray*}
Thus
\[
d^*(|d\tilde w_q|^{q-2}d\tilde w_q)=0. 
\]
 \end{proof}
 
 \begin{remark}\label{dual-eqn}  Note the duality
\[
 d\tilde w_q=|d\tilde u_p|^{p-2}*d\tilde u_p, \ \ \ -d\tilde u_p=|d\tilde w_q|^{q-2}*d\tilde w_q.
\]
 This can also be explained by means of Fenchel's duality for convex variational integrals. See \cite{fenchel}, \cite{temam} and \cite{aronsonlin} for more details on this kind of analysis.
 Motivated by the case $p=2$, we call $\tilde w_q$ the {\it conjugate harmonic} to $\tilde u_p$. 
 \end{remark}
 

 
 \subsection{The normalization} For the rest of the paper we will make the following normalizations:
 
Choose a factor $k_p$ so that
\begin{equation}\label{normintv1}
  \int_M  | k_p d u_p|^p*1 =   k_p.
\end{equation}
Let 
\begin{equation}\label{normintv2}
 U_p = k_p d  u_p  \ \mbox{ and} \ \    V_q = |  U_p|^{p-2}*  U_p.
\end{equation}
 Let $\tilde U_p$ and $\tilde V_q$ denote the lifts to the universal cover and let $\tilde v_q: \tilde M \rightarrow \R$
such that
\begin{equation}\label{normintv3}
 d\tilde v_q=\tilde V_q; \ \ \int_\digamma\tilde v_q *1=0.
 \end{equation}
Notice that $\tilde v_q$ is a rescaling of the conjugate harmonic function $\tilde w_q$ defined in the previous section, $\tilde v_q=k_p^{p-1} \tilde w_q$.
Under the normalizations above, in a fundamental domain $\digamma \subset \tilde M$,
\begin{eqnarray} \label{kappavolform0}
\int_\digamma d\tilde u_p \wedge d \tilde v_q 
&=&
\int_\digamma k_p^{-1} \tilde U_p \wedge | \tilde U_p|^{p-2}* \tilde U_p \nonumber\\
&=&\int_\digamma k_p^{-1} | \tilde U_p|^p *1\\
  &=&1\nonumber.
\end{eqnarray}

\begin{lemma} \label{klemma1}Under the normalizations above, 
$\lim_{p \rightarrow \infty} k_p = L^{-1}. $
%(I conjecture that $k_p$ is about $p^{1/p}$ if we ever need it.) 
\end{lemma}
 \begin{proof} 
 By (\ref{normintv1}) and Lemma~\ref{pintconvto}  
\[ 
\lim_{p \rightarrow \infty} k_p^{(1/p)-1}=\lim_{p \rightarrow \infty}  \left( \int_M |du_p|^p*1 \right )^{1/p} = L.
\]
%  (1-p)/p \ln k_p
By taking logarithms, $ \lim_{p \rightarrow \infty} \ln k_p=-\ln L$,
which implies the Lemma.
\end{proof}

Also, 
\begin{eqnarray} \label{kappanorm0}
 \int_\digamma |d\tilde v_q|*1&=& \int_\digamma | \tilde U_p|^{p-1}*1\nonumber\\
 &\leq &(vol M)^{1/p}  \left( \int_M | \tilde U_p|^p*1\right)^{\frac{p-1}{p}} \ \mbox{(by H\"older)} \\
   &= &(vol M)^{1/p} k_p^{\frac{p-1}{p}} \nonumber\\
   &\leq &(vol M)^{1/p}(L^{-1}+\epsilon_p) \ (\mbox{where $\epsilon_p \rightarrow 0$, \ by Lemma~\ref{klemma1}}).\nonumber\\
   &\approx& L^{-1} \ (\mbox{for $p$ large}).\nonumber
\end{eqnarray}

 We denote by
\[
\alpha_q: \pi_1(M) \rightarrow \R; \ \  \alpha_q(\gamma)=\int_z^{\gamma z}  \tilde V_q
\]
the period homomorphism of the rescaled form $\tilde V_q=d \tilde v_q$. Notice that by definition, $\tilde v_q$ is equivariant under $\alpha$, i.e
\[
\tilde v_q(\gamma z)=\tilde v_q(z)+\alpha_q(\gamma).
\]
It follows that the closed 1-form $\tilde V_q=d \tilde v_q$ is invariant under the action of $\pi_1(M)$ and descends to a closed 1-form $ V_q$ on $M$. 
The representation $\alpha_q: \pi_1(M) \rightarrow \R$ acting on $\R$ via affine isometries, defines a flat  fiber bundle $\tilde M \times_{\alpha_q} \R \rightarrow M$ (specifically a flat affine bundle) and $\tilde v_q$ defines a  section $v_q : M  \rightarrow \tilde M \times_{\alpha_q} \R$. Sometimes it is common to call $v_q$ a {\it{twisted map}}. Under this notation, $V_q=dv_q$.

Note that (\ref{kappavolform0}) and (\ref{kappanorm0}) imply,
\begin{equation} \label{kappavolform}
\int_Md  u_p \wedge dv_q=1
\end{equation}
and
\begin{equation} \label{kappanorm}
 |dv_q|_{L^1(M)} \approx L^{-1}
\end{equation}
for $q$ close to 1. Furthermore,
\begin{equation} \label{alphagamma}
\alpha_q(\gamma)=\int_{\gamma}  dv_q=\int_M \omega_\gamma \wedge dv_q 
\end{equation}
where  
$\omega_\gamma \in \Omega^1(M)$ denotes the closed form Poincare dual to the homology class defined by $\gamma$.
Notice that
for any $\gamma \in \pi_1(M)$ and $0<c_\gamma :=|\omega_\gamma|_{L^\infty(M)}$, we have from (\ref{kappanorm}) for $q$ close to 1
\begin{equation}\label{boundtrlength}
|\alpha_q(\gamma)| \leq c_\gamma \int_M  \left| dv_q \right|*1 \approx c_\gamma L^{-1}.
\end{equation}


 \subsection{The adapted coordinate system} \label{adapted} The pair of functions $(\tilde u_p, \tilde v_q)$ can be used to define a convenient coordinate system on  $\tilde M \backslash \mathcal \{| d\tilde u_p|=0\}$ which we call {\it{the adapted coordinate system}}. More precisely, in the coordinate system $(\tilde u_p, \tilde v_q)$,
the metric $g$ is given by
\begin{equation}
g=\begin{pmatrix} \label{metric}
 \tau_1^2 & 0 \\
 0 & \tau_2^2
\end{pmatrix}
\end{equation}
with
\begin{equation}\label{metric2}
|d \tilde u_p|=\tau_1^{-1}; \ \ \tau_2=  (\tau_1/{k_p})^{p-1}.
\end{equation}
To prove the statement above, it is better to think in terms of the co-metric. Set
\[
\tau_1^{-1}=\left| d\tilde u_p \right |
\]
and note that
\[
\tau_2^{-1}=|d\tilde v_q|=| \tilde U_p|^{p-1}= (\tau_1/{k_p})^{1-p}
\]
and
\[
d\tilde u_p \wedge *d\tilde v_q=  {k_p}^{p-1}d\tilde u_p \wedge |d  \tilde u_p|^{p-2}d \tilde u_p=0.
\]
We have thus proven (\ref{metric}) and (\ref{metric2}). 


\subsection{The normalized flow of $u$}
Let    $\frac{\partial}{\partial u_p}$ denote the vector field dual to the 1-form $du_p$. By Theorem~\ref{nonsingular}, this vector field is globally $C^{\alpha}$ and smooth away from its zeroes. 
 The {\it{normalized gradient flow}} is the flow $\psi_t$  of the vector field $\frac{\partial}{\partial u_p}$. The flow $\psi_t$ lifts to the  universal cover and is given in the 
 local adapted coordinates $(\tilde u_p, \tilde v_q)$,  by
\[ 
(\tilde u_p, \tilde v_q) \mapsto \tilde \psi_t(\tilde u_p, \tilde v_q)=(\tilde u_p+t, \tilde v_q).
\]
Notice that  the  1-forms $d\tilde u_p$ and $d\tilde v_q$ are  invariant under the normalized gradient flow. 

The flow $\psi_t$ on $M$ has interesting dynamics. Choose a regular fiber $\Xi$ and for example assume that $\Xi$ is connected. The normalized gradient flow $\psi_1$ of $u_p$ at time 1 maps $\Xi \backslash$ (points which flow into critical points) to $\Xi \backslash$ (points which flow out of critical points). The map $\psi_1$ gives an interval exchange map of $\Xi$ to itself which is of interest in itself,  though we will exploit it more in this article. See Problem~\ref{Problem 8}.
We next prove:

%$(x,y) \mapsto \psi_t(x,y)=(x+t, y)$ of $u$  (the fibers of $u$ are not preserved by the gradient flow, because the solution $x(t)$ depends on $y$). Furthermore,
%\[
%dy=\frac{\sigma_1^{2-p}}{\kappa}*du=\frac{|du|^{p-2}}{\kappa}*du
%\]
%and
%\[
%dx \wedge dy=\frac{|du|^{p}}{\kappa}*1.
%\]
%\end{lemma}
%\begin{proof} The  statement about invariance follows immediately from Lemma~\ref{gradflow} since
%\[
%\phi_t^*dy=d (y \circ \phi_t)=dy. 
%\]
%For the second notice,
%\[
%dx \wedge *dx  = |dx|^2 \sigma_1\sigma_2 dx \wedge dy =  \kappa \sigma_1^{p-2} dx \wedge dy
%\]
%which implies
%\[
%dy=\frac{\sigma_1^{2-p}}{\kappa}*dx
%\]
%and 
%\[
%dx \wedge dy=\frac{\sigma_1^{2-p}}{\kappa}dx \wedge *dx=\frac{\sigma_1^p}{\kappa}*1.
%\]
%Notice that the statement that dy is closed is equivalent to the $p$-harmonic map equation (\ref{pharm}).
%\end{proof}
\begin{proposition}\label{locbd}The $ \tilde v_q$ are locally uniformly  bounded in $L^\infty$ for all $q$.
\end{proposition}
%\begin{proof} Choose a regular fiber $\Xi$ and assume that $\Xi$ is connected. Then $v_q$ can be chosen locally in the fiber $0\leq v_q \leq 1$ by the choice of normalization. The normalized gradient flow $\psi_1$ of $u_p$ at time 1 maps $\Xi \backslash$ (points which flow into critical points) to $\Xi \backslash$ (points which flow out of critical points). The function $v_q$ is constant on this normalized gradient flow. The map $\psi_1$ gives an interval exchange map of $\Xi$ to itself which is of interest in itself. But we have now given a fundamental domain for $\Xi$ in $\tilde M$ on which $0\leq v_q \leq 1$. Let $m$ be the length of the shortest fiber. If we calculate $v_q(\gamma(t))$ on an arbitrary curve, if $v_q(\gamma(1)) - v_q(\gamma(0) = D$, the curve will have to cross the fiber at least $D-1$ times, and therefore it must be of length at least $(D-1)/m$. This gives a bound on $v_q$ along curves of bounded length.
%The proof is easily modified if there is no connected fibers.
%It is important to note the lack of continuity implied by the bound $(D-1)/m$, not $D/m$.
%\end{proof}

\begin{proof} Choose a regular fiber $\Xi$ of $u_p$ and let $[\Xi] \in H_1(M, \Z)$ denote its homology class. Let $ \omega_\Xi$ be a  closed 1-form representing the Poincare dual of $[\Xi]$ 
%can be represented by the closed form $\omega_\Xi=adu_p$, where $a \in \R$. Indeed, let $\tilde \Xi$ denote the lift of $\Xi$ to the universal cover $\tilde M$ of $M$. 
%Since  $\Xi$ is a regular fiber, it does not contain any critical point of $du_p$, hence, with respect to the adapted coordinates,  we can write in a neighborhood of $\tilde \Xi$,
%\[
%\sigma^*(\omega_\Xi)=\tilde ad\tilde u_p+\tilde bd\tilde v_q
%\] 
%for some locally defined functions $\tilde  a$ and $\tilde  b$. Thus,
%\[
%\omega_\Xi=\tilde adu_p+\tilde bdv_q
%\] 
%for some locally defined smooth functions $  a$ and $b$ near $\Xi$.
%
%Since
%\[
%\int_\Xi \omega_\Xi=\Xi \cdot 
%\Xi=0
%\]
and let  $ \digamma$ denote a fundamental domain in $\tilde M$. Then,
\begin{eqnarray*}
\int_{\tilde \Xi \cap \digamma }|d \tilde v_q| &=& \left |\int_{\tilde \Xi \cap \digamma} d \tilde v_q \right| \ (\mbox{because $d \tilde v_q$ is nonzero on $\Xi$})\\
&=& \left |\int_{ \Xi } d  v_q \right| \ (\mbox{because $d \tilde v_q$ descents to $d v_q$ in $M$})\\
&=&\left | \int_{M } \omega_\Xi \wedge d  v_q \right| \ (\mbox{because $\omega_\Xi$ is  Poincare dual of $\Xi$})\\
&\leq &c_\Xi(L^{-1}+1) \ (\mbox{by} \ (\ref{kappanorm})),
\end{eqnarray*}
where, since all $u_p$ are homotopic, $c_\Xi$ is a topological constant.
This implies that $d\tilde v_q$ are uniformly bounded in $L^1(\Xi \cap \digamma)$ and hence  $\tilde v_q$ are uniformly bounded in $L^\infty(\Xi \cap \digamma)$.
Since $\tilde v_q$ is invariant under the normalized gradient flow, it follows that $\tilde v_q$ is uniformly bounded on the open dense set of the fundamental domain $\digamma$ consisting of all non-critical trajectories. Hence, by continuity, $\tilde v_q$  is uniformly bounded on the closure of $\digamma$. Since the representation $\alpha_q$ is also uniformly bounded by (\ref{boundtrlength}), the local boundedness of $\tilde v_q$ in $\tilde M$ follows.
\end{proof}
%Corollary 1: Existence of representation 
%Corollary 2: $v_q \rightarrow v$ in $L^s$ for all $s$.


\section{The limit $q \rightarrow 1$}\label{qgoesto1}
In this section we construct a DeRham 1-current $V=dv$ obtained as a limit as $q \rightarrow 1$ of the closed forms $dv_q$ associated to the normalized conjugate harmonic functions to $u_p$. We  show that there exists a limiting representation $\alpha: \pi_1(M) \rightarrow \R$ and an $\alpha$-equivariant function $\tilde v$ whose derivative induces the current $V$. We further show that the function $\tilde v$ is locally in $L^\infty$ and locally of bounded variation. In Section~\ref{consmeasure} we will show that $\tilde v$ is  of least gradient (1-harmonic).


\begin{proposition} \label{lemma:limmeasures0} Given a sequence $q \rightarrow 1$, there exists a subsequence (denoted again by $q$) such that:
\begin{itemize}
\item $(i)$ There exists a closed  1-current $ V \in \mathcal D_1( M)$ such that 
$dv_q \rightharpoonup  V.$\\
\item $(ii)$ There exists a closed  1-current $ \tilde V \in \mathcal D_1( \tilde M)$ such that 
$d\tilde v_q \rightharpoonup  \tilde V.$ Furthermore, if $\sigma: \tilde M \rightarrow M$ denotes the universal covering map, then
$\sigma^*(V)= \tilde V.$\\
\item $(iii)$ There exists a representation 
$\alpha: \pi_1(M) \rightarrow \R$
such that for any $\gamma \in \pi_1(M)$,
$\alpha(\gamma)=\lim_{q \rightarrow 1} \alpha_q(\gamma).$
Furthermore,
$\alpha(\gamma)=   V( \omega_\gamma)$
 where $\omega_\gamma$ is the Poincare dual to the homology class defined by $\gamma$.
 \item $(iv)$ The homology class $[V] \in H_1(M, \R)$ is dual to $\alpha$.
 \end{itemize}
 \end{proposition}
\begin{proof} 
For $\phi \in  \Omega^1(M)$ a test function and $|\phi|_{L^\infty} \leq 1$, we have by (\ref{kappanorm}) that
\[
  \left | \int_M \phi \wedge dv_q  \right |  \leq C.
\]
  By weak compactness (cf. \cite[Lemma 2.15]{simon}), there exists  $V \in  \mathcal D_1( M)$ such that (after passing to a subsequence)
  \[
  dv_q \rightharpoonup  V 
  \]
  and $V$ is closed, being the  distributional limit of closed forms. 

For $(ii)$, the proof of the convergence is exactly the same as the proof of $(i)$. In order to prove the statement about the pullback, 
consider an open cover of $M$ given by basic sets and let $ \{ V_i \}$ be the cover of $\tilde M$ obtained by the preimage of the sets in $M$. Let $\zeta_i$ be a partition of unity subordinate to $ \{ V_i \}$. By definition, after identifying $V_i \simeq \sigma (V_i)$ and $dv_q= d\tilde v_q$,
\begin{eqnarray*}
\sigma^*(V)(\phi)&=&\sum_i  V(\zeta_i \phi)
= \lim_{q \rightarrow 1} \int_{\tilde M}  \sum_i \zeta_i \phi \wedge dv_q\\
&= & \lim_{q \rightarrow 1} \int_{\tilde M}   \phi \wedge d \tilde v_q
=  \tilde V(\phi).
\end{eqnarray*}




To prove $(iii)$, note that by the weak convergence of $dv_q$,
\[
\alpha_q(\gamma)= \int_\gamma  dv_q =\int_M  \omega_\gamma \wedge dv_q  \rightarrow  V( \omega_\gamma) =\alpha(\gamma).
\]

For  $(iv)$ notice that $\alpha$ factors through the abelianization of $\pi_1(M)$ to define an element in $H_1(M, \R)^*=H^1(M, \R)$ which is dual to $[V]$ by $(iii)$.
\end{proof}

\begin{definition}\label{def:bdvar} Let $U \subset M$ an open set and $f \in L^1(U)$. We define
\[
||df||_U=\sup\{ \int_M d\phi  \wedge f: \phi \in \mathcal D^1(U), \ \max|\phi| \leq 1 \} 
\]
 and set
\[
|f|_{BV(U)}= |f|_{L^1(U)}+||df||_U.
\]
We say that $f$ is of {\it{bounded variation}} in $U$ if $|f|_{BV(U)} < \infty$. 
\end{definition}



\begin{theorem} \label{lemma:limmeasures2} There exists a sequence  $q \rightarrow 1$  and $ \tilde v:  \tilde M \rightarrow  \R$ such that $\tilde v_q$ converges to $\tilde v$ {\it weakly} in $BV_{loc}(\tilde M)$ and {\it strongly} in $L^s_{loc}(\tilde M)$ for all $s \geq 1$.
Furthermore, $\tilde v$ has the following properties:
\begin{itemize}
\item $(i)$ $\tilde v$ is locally in $L^\infty$ and locally of bounded variation
%\item $(ii)$ $\tilde v$ is normalized so that
%  \[
%  \int_\digamma \tilde v *1=0.
%  \]
\item $(ii)$ $\tilde v$ is equivariant under $\alpha$, i.e for every $ \gamma \in \pi_1(\tilde M)$ and a.e. $ z \in \tilde M$
  \[
  \tilde v(\gamma z)=\tilde v(z)+ \alpha(\gamma)
  \]
%  \item $(iii)$ $\tilde v$ is locally a map of least gradient.
\end{itemize} 

\end{theorem}
\begin{proof} Fix $W \subset  \subset \tilde M$, and choose a finite number $\gamma_1,...,\gamma_N \in \pi_1(M)$ such that
\[
W \subset \subset \bigcup_{i=1}^N \gamma_i(\digamma).
\]
Since, by Proposition~\ref{lemma:limmeasures0}(ii),  $| \alpha_q(\gamma_i)| \leq C$  for $i=1,...,N $ and $j=1,2,...$, we obtain by the equivariance of $\tilde v_q$, (\ref{normintv3})
 and the fact that 
$\gamma_i$ act as isometries on $\tilde M$ that
\begin{equation}\label{estW}
\left| \frac{1}{vol(W)} \int_W \tilde v_q(z)dz \right| \leq  \frac{NC |\digamma|}{vol(W)} \leq C'.
\end{equation}
%Hence, we can choose a further subsequence so that
%\[
%\int_W \tilde v_j \rightarrow \bar v_0.
%\]
Now set,
\[
w_q(x)=\tilde v_q( x)- \frac{1}{vol(W)} \int_W \tilde v_q( z)dz; \ \ dw_q=d\tilde v_q.
\]
Similarly, by the Poincare inequality and (\ref{kappanorm0})
\[
|w_q|_{L^1(W)} \leq c |dw_q|_{L^1(W)}= c|d\tilde v_q|_{L^1} \leq C,
\]
which combined with (\ref{estW}), implies
\[
|\tilde v_q|_{W^{1,1}(W)} \leq C.
\]
Hence, there exists a subsequence (denoted again by $\tilde v_q$) and $\tilde v^W \in BV(W)$
such that
\[
\tilde v_q \xrightharpoonup{BV(W)} \tilde v^W.
\]
By a diagonalization argument we can define $\tilde v \in BV_{loc}(\tilde M)$ such that
\[
\tilde v_q \xrightharpoonup{BV_{loc}(\tilde M)} \tilde v.
\]
By the Rellich Lemma and the fact that $\tilde v_q$ are locally uniformly bounded by Proposition~\ref{locbd}, 
\[
\tilde v_q  \rightarrow \tilde v \in L^s_{loc} \ \ \forall s \geq 1.
\]
%Furthermore, by definition of $BV_{loc}(\tilde M)$,
%\[
%d\tilde v_q \rightharpoonup d \tilde v.
%\]
To show that  $\tilde v$ is locally bounded, fix $W \subset \tilde M$ compact. Again, since   
$|\tilde v_q|_{L^\infty} \leq C$  by Proposition~\ref{locbd}, and  $\tilde v_q  \rightarrow \tilde v$  in $L^s(W)$ for all $s$, it follows that $|\tilde v|_{L^s(W)} \leq C$ uniformly in $s$ and thus $\tilde v \in L^\infty(W)$.

Statement $(ii)$ follows from the equivariance 
$\tilde v_q(\gamma z)=\tilde v_q(z)+ \alpha_q(\gamma)$ 
and the fact that $L^s_{loc}$ convergence implies a.e convergence. 
 Since we have already shown that the functions $\tilde v_q$ converge strongly to $\tilde v$ in $L^s_{loc}$ for $s>1$. 
\end{proof}

\begin{remark} We will see in Section~\ref{consmeasure} that $\tilde v$ is a locally a function of least gradient. For Euclidean domains this follows also from  \cite{juutinen}, Proposition 4.5. We will give a proof of this fact in Theorem~\ref{thmlegr}.
\end{remark}

\begin{definition} Let $ L=\tilde M \times_\alpha \R$ be the flat affine bundle associated to the representation $\alpha$ and $v$  the section of  $ L$ induced from $\tilde v$. For an $L^1$-section $\xi: M \rightarrow L$, set $||d\xi||=||d\xi||_M$ and $|\xi|_{BV}=|\xi|_{BV(M)}$ as in Definition~\ref{def:bdvar}.  
 With this definition, $v$ becomes a  {\it section (twisted map) of bounded variation. } 
In view of Theorem~\ref{lemma:limmeasures2}, $d \tilde v=\tilde V$ and we  will denote
\[
V=dv.
\]
\end{definition}
\begin{remark}For the rest of the paper we fix sequential limits $u=\lim_{p \rightarrow \infty} u_p$ and $v=\lim_{q \rightarrow 1} v_q$ in the appropriate function spaces described above. We conjecture that $u$ and $v$ are essentially unique, though we are unable to prove this. See Conjectures~\ref{Conjecture 2} and~\ref{Conjecture 3}.
\end{remark} 

\begin{remark}\label{radonms}
Recall that, by the Riesz representation theorem \cite[Chapter 6, (2.14)]{simon},   given a $p$-current $S \in \mathcal D_p (U)$ of finite mass,
we can write
\[
S(\phi)=\int_{ U} \phi \wedge \vec {S} \ |dS|; \ \ \phi \in \mathcal D^{n-p} ( U) 
\]
for a Radon measure $|d S|$ and a measurable section $ \vec {S}$ of $\Lambda^p(\tilde M)$  where
$| \vec {S}|=1$ $|d S|$-a.e. It is customary to write the $p$-form valued Radon measure $\vec {S} \ |dS|$ by $S$ and use the notation
\[
S(\phi)=\int_{ U} \phi \wedge S.
\]
We will use this notation throughout the rest of the paper.
\end{remark}

\section{The geodesic lamination associated to the $\infty$-harmonic map}\label{sect:crandal}
For this section we allow $(M, g)$ to be a closed hyperbolic manifold of any dimension $n \geq 2$. We show that the gradient lines of the $\infty$-harmonic map $u$ at the points of maximum stretch define a geodesic lamination. The major difficulty lies in defining the gradient lines of $u$, because $grad(u)$ is not even known to be continuous.  We overcome this issue by adapting to the hyperbolic metric an argument due to Crandall for Euclidean space. This is a hyperbolic version of what is known as {\it{comparison with cones}} and which for Euclidean metrics is  equivalent to the notion of viscosity solutions of the $\infty$-Laplace equation (cf. \cite{crandal} or \cite{lindqvist}). We will not attempt to develop such a theory in this paper and we only prove the bare minimum that we need for our topological applications. For more details on open problems see Section~\ref{conjectures}.

\subsection{Statement of the theorem} We start by recalling the notion of a geodesic lamination.
%We give a proof modeled on a proof shown to us by Craig Evans for Euclidean space. We first clarify our use of the expression $|du(x)|$ for a Lipschitz map $u$. Here $L$ is the Lipschitz constant of the $\infty$ harmonic map $u.$
%
% Let  
% \[
% L_(B,x)  = (max y in B : |u(x) - u(y)|/dist(x,y)).
% \]
%\begin{definition}   $|dux)| = \lim_{ r \rightarrow 0} L_{B_(x)}$ for $B_r(x) = (y in M: distance (x,y) <=r}.$
%\end{definition}
%\begin{definition}  the set $\lambda$ is a geodesic lamination of $M$ if
%     1) $\lambda$ is closed
%     2) for $x \in \lambda$, $\lambda(x)$ is an embedded geodesic containing $x$. 
%     3) $\lambda(x)$ intersect $\lambda(y)$ is either $ \lambda(x) = \lambda(y)$ or empty.
%\end{definition}

\begin{definition}A geodesic lamination $\lambda$ is a closed subset of  $(M, g)$   which is a disjoint union of simple, complete geodesics. 
\end{definition}

The next theorem is the main result of the section. Recall from Section~\ref{lipconst} that $L_u(x)$ denotes the  local Lipschitz constant at $x$.


\begin{theorem}\label{straightline} Let  $(M, g)$ be a closed hyperbolic manifold of dimension $n \geq 2$   and let $ u: M \rightarrow S^1$ be $\infty$-harmonic (i.e a limit of $p$ harmonic maps for $p \rightarrow \infty$) with  Lipschitz constant $L:=|du|_{L^\infty(M)}$. Then,
\[
 \lambda_u=\{ x \in M: L_u(x)= L \} 
\]
is a geodesic lamination in $M$.
\end{theorem}

First, note the following straightforward:
\begin{lemma} \label{realized} Let  $(M, g)$ be a closed Riemannian manifold  and $ f: M \rightarrow S^1$ a  Lipschitz map  with global Lipschitz constant $L=L_f(M)$. Then, the set
\[
 \lambda_f=\{ x \in M: L_f(x)= L \} 
\]
is non-empty and closed.
\end{lemma}
\begin{proof} It follows  from Proposition~\ref{crandal0} $(i)$, on the upper semicontinuity of the local Lipschitz constant. Here are the details: By Proposition~\ref{crandal0}, take a sequence $x_i$ such that  $L_f(x_i) \nearrow L$. By compactness, we may assume $ x_i \rightarrow x$ and by upper semicontinuity $L_f(x) \geq \lim_i L_f(x_i) = L$. Thus, $x \in  \lambda_f$ and hence $ \lambda_f \neq \emptyset$. By  upper semicontinuity 
\[
 \lambda_f=\{ x \in M: L_f(x)= L \}=\{ x \in M: L_f(x) \geq L \} 
\]
is closed.
\end{proof}

%Theorem 5.3:  Let $\lambda = (x in M such that |du(x)| = L)$ . If the infinity harmonic map $u$ is the limit of a sequence of p harmonic maps for $p \rightarrow \infty$, then $\lambda$  is a geodesic lamination.

%Proposition 5.4 (your 5.3). The set $\lambda$ is non-empty and closed.
%
%(Put in your proof).
\subsection{Comparison with cones}
%\begin{definition}\label{copco} Let $M$ be a compact, hyperbolic manifold, i.e $\tilde M =H^n$ and denote by
%$d(.,.)$ the distance function.
%A cone function with vertex $x_0 \in H^n$ is a function of the form
%\[
%c(x)=A + B d(x,x_0),
%\]
% where $a,b \in \R.$ 
% Let $\Omega \subset H^n$ an open, connected subset (possibly $\Omega = H^n$). 
% A function $u \in C(\Omega)$ satisfies
%{\it{comparisons with cones from above}} if it possesses the following property: 
%For every open set $V$ with 
% $\bar V \subset  \Omega$ compact and every cone function $c$ with vertex $x_0 \in H^n \backslash V$,
%\[
%u  \leq c \  \mbox{in} \ \partial V \Longrightarrow u \leq c \ \mbox{in} \ V.
%\]
%We say that $u$ satisfies {\it{comparisons with cones from below}} if $-u$ satisfies  comparisons with cones from above. Finally, $u$ satisfies {\it{comparisons with cones }} if it satisfies comparisons with cones from above and below.
%\end{definition}
In this section we prove that our minimizers satisfy comparison with cones. For Euclidean metrics this is known to be equivalent to the notion of viscosity solution of the $\infty$-Laplace equation (cf. \cite{crandal}). In the present article we deal primarily with hyperbolic metrics and we expect every local result known for the Euclidean metric to also hold for our case as well. Below we will only prove the bare minimum necessary to prove our theorem on geodesic laminations, leaving most analytic aspects for a future project.

We first note that the
 map $d(x,x_0)$ can be approximated by  cone $ p$-harmonic functions $c_p(x) = f_p(d(x_0,x))$. In Euclidean space $\R^n$,
 \[ 
 f_p(t) = t^{\frac{p-n}{p-1}}=t^{1-\frac{n-1}{p-1}}
 \] 
 and in hyperbolic space $H^n$, by a function $f_p(t)$ satisfying 
 \[
 \frac{df_p(t)}{dt} = (1/\sinh(t))^{\frac{n-1}{p-1}}. 
 \]
 
\begin{lemma}The function $f_p(t)$ is $p$-harmonic and $f_p(t) \rightarrow t$  uniformly on compact sets of $H^n$.  
\end{lemma}
\begin{proof} The metric on $H^n$ can be written in polar coordinates as
\[
g=dt^2+\sinh^2 t d\theta^2
\]
where $dt$ is hyperbolic length and $d \theta$ is the metric on $S^{n-1}$. From this, it follows immediately that $f_p(t)$ is $p$-harmonic.
To show the second statement, write
\[
f_p'(t)=h_p'(t)(t/\sinh(t))^{(n-1)/(p-1)}\left (1-(n-1)/(p-1)\right )^{-1} 
\]
where 
\[
\ h_p(t)=t^{1-(n-1)/(p-1)}.
\]
From this we obtain
\[
a_p h_p'(t) \leq f_p'(t) \leq h_p'(t) b_p 
\]
where $a_p$ and $b_p$ are constants converging to 1 as $p \rightarrow \infty$. Thus
\[
a_p t^{1-(n-1)/(p-1)} \leq \int_0^t f_p'(s)ds \leq b_p t^{1-(n-1)/(p-1)} 
\]
hence, by normalizing $f_p$ so that $f_p(0)=0$,
\[
a_p t^{1-(n-1)/(p-1)} \leq  f_p(t) \leq b_p t^{1-(n-1)/(p-1)} 
\]
from which the convergence follows.
\end{proof}

Since we can approximate both the $\infty$-harmonic function and the cone by $p$-harmonic maps, we get the proof of the following 
\begin{proposition}\label{comocones1}  If
\[ 
u(x) \leq A + B d(x,x_0) = c(x)
\]
for $ x \in \partial B_r(x_0)$  and at $x = x_0$, then 
\[
u(x) \leq c(x) \ \forall x \in B_r(x_0).
\] 
\end{proposition}

\begin{proof}  Both the function $u$ and the cone $c $ are  uniform limits in $C^0$ of $p$-harmonic functions $u_p$ and $c_p$ respectively. Hence, for $x \in \partial B_r(x_0)$ and also for $x = x_0$. 
\begin{eqnarray*}
 u_p(x) &<& \epsilon + u(x) \\
 &\leq& \epsilon + A + B d(x,x_0)\\
 &<& \epsilon(1 + Br) + A + B f_p(d(x_0,x)) \\
 &\leq& \epsilon(1 + Br)  +c_p(x)
  \end{eqnarray*}
 Here $\epsilon = \epsilon(p) \rightarrow 0$ as $p \rightarrow \infty$. However, both $u_p$ and the cone $c_p$ are $p$-harmonic functions. By the strong maximum principle for $p$-harmonic functions applied to the punctured disc $B_r^*(x_0)$, we obtain
 \begin{eqnarray*}
u(x) &\leq &u_p(x) + \epsilon \\
&\leq & c_p(x) + \epsilon(2 + Br)\\
&\leq & c(x) + 2\epsilon(1 + Br).
  \end{eqnarray*}
Since $\epsilon = \epsilon(p)\rightarrow 0$ as $p \rightarrow \infty$, this finishes the proof. 
\end{proof}



%\begin{proposition}\label{comocones1} Let $u_p \rightarrow u$ be a variational solution as in Definition~\ref{infinharm}. Then, 
%$u$ satisfies comparison with cones.
%\end{proposition}
%
%\begin{proof} Let $V  \subset H^n$ open with $\bar V$ compact, and $u$, $c$ as in Definition~\ref{copco}. Both $u$ and the cone $c $ are  local uniform limits  of $p$-harmonic functions $u_p$ and $c_p=A+Bf_p(d(x_0,.))$ respectively. Hence, for $x \in \partial V$ 
%\begin{eqnarray*}
% u_p(x) &<& \epsilon + u(x) \\
% &\leq& \epsilon + A + B d(x,x_0)\\
% &<& \epsilon(1 + B) + A + B f_p(d(x_0,x)) \\
% &=& \epsilon(1 + B)  +c_p(x)
%  \end{eqnarray*}
% Here $\epsilon = \epsilon(p) \rightarrow 0$ as $p \rightarrow \infty$. However, both $u_p$ and the cone $c_p$ are $p$-harmonic functions. By the strong maximum principle for $p$-harmonic functions, we obtain
% \begin{eqnarray*}
%u(x) &\leq &u_p(x) + \epsilon \\
%&\leq & c_p(x) + \epsilon(2 + B)\\
%&\leq & c(x) + \epsilon(3 + B)
%  \end{eqnarray*}
%on $V$. Since $\epsilon = \epsilon(p)\rightarrow 0$ as $p \rightarrow \infty$, this finishes the proof. 
%\end{proof}
%
%By applying Proposition~\ref{comocones1} for the punctured ball of radius $r$ we immediately obtain (see also \cite[formula (6.1)]{lindqvist})
%
\begin{corollary}\label{comocones2}
The ratio
\[
\max_{d(x,x_0)=r}\frac{u(x)-u(x_0)}{r}
\]
is increasing in $r$. The same holds with $u$ replaced by $-u$.
\end{corollary} 

\begin{proposition}\label{comocones2}  Let $x_0 \in \lambda_u$ be arbitrary and $B_r(x_0) \subset M$ (we can lift to the covering space if we choose). Assume that as $ x_i \rightarrow x_0$, 
$ \frac{u(x_i)-u(x_0)}{d(x_i,x_0)} \rightarrow +L \ (\mbox{resp.} -L).$  
Then 
\[
u(x)= u(x_0) + Lr \ (\mbox{resp.} -Lr)
\]
 for some point $x \in \partial B_r(x_0)$, and the geodesic between $x_0$ and $x$ lies in $\lambda_u.$
\end{proposition}

\begin{proof} Let $B = \max_{x \in \partial B_r} \frac{u(x)- u(x_0)}{r}.$  Let $x \in \partial B_r(x_0)$ on which $B$ is taken on.  Since $L$ is the  Lipschitz constant $B \leq L.$
Suppose $B < L$, then  by Proposition~\ref{comocones1}, $u (x) \leq u(x_0) +Bd(x,x_0)$. But 
\[
\lim_i \frac{u(x_i) - u(x_0)}{d(x_i,x_0)} = L > B,
\] 
and for some $x_i$,
\[ 
u(x_i) - u(x_0) > Bd(x_i,x_0).
\] 
This gives a contradiction to the statement that $u$ lies under the cone $c$. So $B = L.$

Let $\lambda_0$ be the geodesic parameterized by arc length between $x_0$ and $x$.  Then,  since $L$ is the best Lipschitz constant, for $0 \leq s < t \leq r$
\[
 u(\lambda_0(t)) -u(\lambda_0(s)) \leq L(t-s).
 \]
But 
\[
u(x)- u(x_0) = u(\lambda_0(r) ) - u(\lambda_0(0)) = Lr. 
\]   
This gives estimates above and below on $u(\lambda_0(t))$ that shows
\[  
u(\lambda_0(t)) = u(x_0) + Lt.
\]
For the case of $-L$, apply the same procedure to $-u$.
\end{proof}



\begin{straightline} We have shown that every point $x \in \lambda_u$ lies in a geodesic in $\lambda_u.$ We need only show that a) the entire geodesic lies in $\lambda_u$ and b) if the geodesics intersect, they form an angle of 0 or $\pi$.  We show b) first, as it is part of a). Suppose two geodesics $\lambda_1$ and $\lambda_2 \in \lambda_u$ meet at $x_0$, and that $x_0 = \lambda_1(0)$ is an interior point of $ \lambda_1.$ Assume also that $\lambda_2(0) = x_0$ and that the geodesics are parameterized by arc length.  Then
\[ 
u(\lambda_1(t)) = u(x_0) + Lt
\]
 for $t$ of both signs and 
 \[
 u(\lambda_2(s)) = u(x_0) + Ls
 \]
  for either $s > 0$ or $s < 0.$ Using the fact that $ L$ is the best Lipschitz constant, 
  \[ 
  Ld (\lambda_1 (t), \lambda_2(s)) \geq
 |u(\lambda_1(t)) - u(\lambda_2(s))| = |L(t - s)|. 
  \] 
It follows that  $|t-s|$ is the distance from $\lambda_1(t)$ to $\lambda_2(s)$ along the geodesics and must be greater than or equal to the actual distance. But we already know from the inequality that it is less than or equal to the distance between them on $M$. Equality follows; hence the geodesics meet at angle 0 if we parameterize them both in the direction of increasing $u.$

To show that the entire geodesic $\lambda_0$ lies in $\lambda_u$, we suppose not.  Then, choose $x_0$ near but not at the end the part of of geodesic $\lambda_0$ which lies in $\lambda_u.$  The Lipschitz constant at $x_0$ is taken on in both directions. More precisely, there are sequences 
$ x_i^\pm \rightarrow x_0$, 
$ \frac{u(x_i^\pm)-u(x_0)}{d(x_i^\pm,x_0)} \rightarrow \pm L .$ This follows by a straightforward argument using comparison with cones (cf. \cite[Lemma 4.6]{crandal}).
Hence by Proposition~\ref{comocones2}, there are two geodesic rays emanating from $x_0$ on which take on the best Lipschitz constant from above and below, both of  which are in $\lambda_u$ until they reach the boundary of $B_r(x_0).$  By the previous argument, these rays must make an angle of either 0 or $\pi $ with $\lambda_0.$ Hence $\lambda_0$ intersect $B_r(x_0)$ in $\lambda_u.$
\end{straightline}


\subsection{Another interpretation of the best Lipschitz constant} 
In this section we fix $(M,g)$ a closed hyperbolic manifold. Let $u: M \rightarrow S^1$ be an $\infty$-harmonic map in a given homotopy class  with best Lipchitz constant $L=L_u$. Let
$\tilde u : \tilde M \rightarrow \R$
denote the lift to the universal cover, equivariant  under the homomorphism $\rho:  \pi_1(M) \rightarrow \Z$. 
%i.e 
%\[
%\tilde u(\gamma z)=  \tilde u( z)+ \rho(\gamma), \ \forall \gamma \in \Gamma\  \mbox{and} \   \forall z \in \tilde M.
%\]
%For $\gamma \in \pi_1(M)$, let $l_g(\gamma)$ denote the length of the geodesic homotopic to $\gamma$. Equivalently, via the identification $\Gamma \simeq \pi_1(M)$,
%\[
%l_g(\gamma)=\inf_{\tilde x \in H^n}d_{H^n}(\tilde x, \gamma(\tilde x)).
%\]
Let $\mathcal S$ denote the set of free homotopy classes of simple closed curves in $M$.  Given $\gamma \in \mathcal S$,
let $l_g(\gamma)$ denote the length of the geodesic representative of $\gamma$ and
define the functional
\begin{equation}\label{normlength}
K: \mathcal S \rightarrow \R_{\geq 0}, \ \  K(\gamma)=  \frac{|\rho(\gamma)|}{l_g(\gamma)}.
\end{equation}
In the above, by a slight abuse of notation, we denote by $\gamma$ also the element in $\pi_1(M)$ corresponding to the free homotopy class $\gamma$.
Let 
\[
K= \sup_{\gamma \in  \mathcal S} K(\gamma)
\]
and note that 
\begin{equation}\label{ineq0}
K \leq L.
\end{equation} 
Indeed, given $\gamma \in \mathcal S$ denote by $\tilde \gamma: [0, T] \rightarrow \tilde M$ the lift of any loop in  $\gamma$ parametrized by arc length. Note that,
\begin{eqnarray}\label{water}
|\rho(\gamma)|&=&  |\tilde u(\tilde \gamma(0))-  \tilde u(\tilde \gamma(T))| 
 = \left |\int_0^T  \frac{d(\tilde u \circ \tilde \gamma)}{dt}  dt  \right |.
 \end{eqnarray}
By taking $\tilde \gamma$ a lift of the geodesic representative in the free homotopy class $\gamma$, and noting $T=l_g(\gamma)$,
\begin{eqnarray}\label{water2}
|\rho(\gamma)|
 &\leq& \int_0^T \left |d \tilde u_{ \tilde \gamma(t)} \right | dt  \leq LT. \nonumber
\end{eqnarray}
Hence
\[
\frac{|\rho(\gamma)|}{l_g(\gamma)} \leq L,
\]
which implies (\ref{ineq0}). 

The following theorem is a version of \cite[Theorem 8.5]{thurston}. 
  \begin{theorem}\label{K=L} $K=L$.
\end{theorem}
\begin{proof}  
Let $\beta$ be a leaf of the maximum stretch lamination $\lambda_u$  parameterized according to arc length. Because $M$ is compact, for any $n$, we can find $t_1 < t_2 - 1$ such that 
$d_g(\beta(t_2), \beta(t_1))< 1/n$. (If $\beta$ is closed this holds trivially for any $n$ by taking $\beta(t_2)=\beta(t_1))$. Choose the closed geodesic $\gamma_n$ to be the geodesic homotopic to the broken  geodesic $\beta_n$ made up by following $\beta$ from $t_1$ to $t_2$ and then connecting $\beta(t_2)$ to $\beta(t_1)$ by a short geodesic of length less than $1/n$.
Note that
 $l_g(\gamma_n) \leq l_g(\beta_n) < t_2 - t_1 + 1/n$.
By (\ref{water}) and noting that $\beta$ is a curve of stretch $L$ for $u$, 
\begin{eqnarray}\label{water2}
|\rho(\gamma_n)|=|\rho(\beta_n)| \geq L(t_2 - t_1 - 1/n).
\end{eqnarray}
Hence,
\[
K \geq K(\gamma_n) > \frac{L(t_2 - t_1 - 1/n)}{t_2 - t_1 + 1/n} \rightarrow L
\]
as $n \rightarrow \infty$.
\end{proof}

\section{The concentration of the measure}\label{consmeasure}
In this section, we will use the Euler-Lagrange equations to determine properties of the limiting measures on the maps which take on the best Lipschitz constants. 
The statements are actually statements about $L^1$ norms being small, which implies that the limiting measures are zero away from the set of maximum stretch $\{ L_u=L \}=\lambda_u$.  They make sense in the limit applications only when a continuous function is inserted in the integrals. However, the limits are still zero, since the sup norm of a test function is bounded by the modulus of continuity. Note that in this section we will not make use of the results of Section~\ref{sect:crandal} that $\lambda_u$ is a geodesic lamination. Also the results about the concentration of the measure work in any dimensions and any Riemannian metric.

In Section~\ref{LGRad} we will specialize to the case $n=2$ and show that the map $v$ obtained as a limit of the maps $v_q$ as $q \rightarrow 1$ is a map of least gradient. We will not explore this property further in this paper, however in Section~\ref{conjectures} we will indicate how this property can be used together with results about minimizing currents to give another proof that $\lambda_u$ is a geodesic lamination on the support of the measure $dv$. 

Finally in Section~\ref{sectevp} we will show how the results of the previous sections can be generalized to cover the equivariant problem for a general real valued homomorphism $\rho$.  There are no real changes. Our paper could have been written to include this more general situation from the start.  We did not do this, as many in our target audience would have found it a source of added confusion in a paper that already contains unfamiliar topics.

\subsection{The support of $V=dv$} Let  $(M,g)$ be a Riemannian manifold  of dimension $n \geq 2$ and 
let $u: M \rightarrow S^1$ be an $\infty$-harmonic map obtained as a sequential limit $u=\lim_{p \rightarrow \infty}u_p$ as in Theorem~\ref{nonsingular}. In order to simplify the notation, for this section only, we  renormalize the measure on $M$ so that the best Lipschitz constant $L=L_u= 1$.  Carrying factors of this constant around makes everything more difficult to write and read.

 As with the case of dimension 2, we continue  with the normalization $k_p$ as in  (\ref{normintv1}) 
 and by Lemma~\ref{klemma1}, 
\begin{equation}\label{klemma11}
\lim_{p \rightarrow \infty} k_p = 1.
\end{equation}
We define the 1-form $ U_p = k_p d  u_p$ and the closed $n-1$ form $V_q = |  U_p|^{p-2}*  U_p$. As in  (\ref{kappanorm0}) and (\ref{kappavolform}),
\begin{equation} \label{kappavolform21}
\int_M |V_q|*1=\int_M |  U_p|^{p-1}*1
 \approx 1 \ \ \mbox{(for $p$ large)}
\end{equation}
and
\begin{equation} \label{kappavolform22}
\int_M d  u_p \wedge V_q=1.
\end{equation}
As in Proposition~\ref{lemma:limmeasures0},
for $\phi \in  \Omega^1(M)$ a test function and $|\phi|_{L^\infty} \leq 1$,  (\ref{kappavolform21}) implies,
  $\left | \int_M \phi \wedge V_q  \right | $ is uniformy bounded,  
  hence 
  \[
  V_q \rightharpoonup  V, 
  \]
  where $V$ is a closed, $n-1$ current. In our previous notation, for $n=2$, $V=dv$.

The main result of this section is the following theorem:
\begin{theorem}\label{thm:supptmeasure} The support of the current $V$  is contained in the locus of maximum stretch $ \lambda_u$ of $u$.
\end{theorem}

\begin{lemma}  \label{klemma2}  Suppose $0 \leq e_p \leq e \leq 1$. Then 
\[
 e_p^{p-2}(e^2 - e_p^2) < 2/(p-2).
 \]
\end{lemma}

\begin{proof}  Let  $s_p = \frac{e_p}{e}$.    Then the expression we are trying to bound can be written as
\[
 e^p {s_p}^{p-2}(1 - s_p^2).
 \]
   But by calculus, the maximum of $s_p^{p-2}(1-s_p^2)$ is less than $2/(p-2)$. Since $e \leq 1$, we are done.
\end{proof}

\begin{lemma}  \label{kprop 3} Let  $U_p=k_pdu_p$, $U=du$ and
\[ 
G(p) =  2<U_p ,U_p -U> =|U_p|^2+ |U_p -U|^2 - |U|^2.
\]
  Define $Y_p$ as the set on which $G(p) \geq 0$.    Then
 \[   
 \lim_{p\rightarrow \infty}  \int_{Y_p}  |U_p|^{p-2}G(p)*1 = 0.
 \]
\end{lemma}
\begin{proof}   The difference $u_p - u$ is a function on $M$.  Hence, from the Euler-Lagrange equations for $u_p$ we have
\[
 \int_M  |du_p|^{p-2}<du_p,du_p - du>*1 = 0.
\]
Multiply by $k_p^p$ and substitute the expressions for $U_p$ and $U$ to get
\begin{equation}\label{kpU}
\int_M |U_p|^{p-2}<U_p, U_p - k_pU>*1 = 0.
\end{equation}
By (\ref{klemma11}) and (\ref{kappavolform21}),
\begin{eqnarray*}
\lefteqn{ \lim_{p\rightarrow \infty}  \int_M |U_p|^{p-2}(<U_p, U_p - k_pU>-1/2G(p))*1} \nonumber\\
&=&  \lim_{p\rightarrow \infty}  \int_M |U_p|^{p-2}<U_p, (U - k_pU)>*1 \nonumber\\
&\leq& \lim_{p\rightarrow \infty} (1-k_p) \int_M |U_p|^{p-1}|U|*1\\
&=& 0 \nonumber.
\end{eqnarray*}  
Combining with (\ref{kpU}),
\begin{equation}\label{kpU234}
 \lim_{p\rightarrow \infty} \int_M |U_p|^{p-2}G(p)*1 = 0
\end{equation}
and our proposition is proved if we can show that
\[
\lim_{p\rightarrow \infty} \int_{M \backslash Y_p} |U_p|^{p-2}G(p)*1 = 0.
\] 
Therefore, we need to bound the integral of
\[
 |U_p|^{p-2}(|U|^2 - |U_p|^2 - |U_p - U|^2)
 \]
  over the set where it is positive.
But this expression is bounded by 
 \[
 |U_p|^{p-2}(|U|^2 - |U_p|^2) < 2/p-2
 \]
on the larger set where $|U| \geq |U_p|$ by applying Lemma~\ref{klemma2} (for $e_p=|U_p|$ and $e=|U|$). This gives the desired bound.
\end{proof}



\begin{proposition} \label{kprop 4}  
\[
\lim_{p\rightarrow \infty}\int_M |U_p|^{p-2}|U_p - U|^2 *1 = 0.
\]
\end{proposition}

\begin{proof}   We have from Lemma~\ref{kprop 3} that
\[
\lim_{p\rightarrow \infty} \int_{Y_p} |U_p|^{p-2}(|U_p|^2 +|U_p -U|^2-|U|^2)*1  = 0.
\]

On the set ${|U|^2 \leq |U_p|^2 + |U_p -U|^2}$, this gives the desired estimate of the integral over that set.
On the compliment
 \[
 {|U|^2 > |U_p|^2 + |U_p-U|^2}
 \]  
 from Lemma~\ref{klemma2} (for $e_p=|U|_p$,  $e=|U|$)  we have that point-wise
\[
|U_p|^{p-2}|U_p - U|^2 < |U_p|^{p-2}(|U|^2 -|U_p|^2) < 2/(p-2).
\]
This bounds the integral on the entire manifold.
\end{proof}

%\begin{corollary} \label{uniquedu} For any two choices for $U = du$ and $U'=du'$,  we have
%\[ 
% \lim_{p\rightarrow \infty} \int_M |U_p|^{p-2}|U-U'|^2*1 = 0.
% \] 
%Hence  the derivatives of any two functions realizing the best Lipschitz constant must vanish on the support of $T$. 
%\end{corollary}
% 
   

%\item $(c)$ We can also show that 
%\[
%\lim_{p\rightarrow \infty} \int_M |<dv_p,*U>|*1 = 0
%\]
%since $<U_p,*U_p> = 0. $ This partly shows  that the measure $dv$ is supported along the fiber.
         



\begin{proposition} \label{prop:supptmeasure0}   If $\phi$ has support on the set where $|U| \leq \lambda < 1$, then
\[
\lim_{p\rightarrow \infty} \int_M |U_p|^p| \phi| *1 = 0.
 \]
\end{proposition}
\begin{proof} 
We go again to the estimate from Lemma~\ref{kprop 3} 
\[
\int_{Y_p} |U_p|^{p-2}\left((1-\lambda)|U_p|^2 +( \lambda|U_p|^2 +|U_p-U|^2 - |U|^2) \right )*1 \rightarrow 0.
\]
This provides a bound for the 
 integral of $(1 -\lambda)|U_p|^p$ over the set $|U|^2 \leq \lambda|U_p|^2 + |U-U_p|^2$.   In general, over the complimentary set, we do not have a bound.  However, if we are integrating over a set where  $|U| \leq \lambda$, just using the inequality that on that set 
 \[
 |U_p| \leq  \lambda^{-1/2} |U|
 \] 
 we have the pointwise bound
\[ 
|U_p|^p \leq |U|^p \lambda^{-p/2} \leq \lambda^{p/2}.
\]
Since $\lambda < 1$, the point-wise limit is 0. This bounds the integral over the entire manifold.
\end{proof}

\begin{thm:supptmeasure} 
As in Remark~\ref{radonms}, let $|V| $ denote the Radon measure associated to the distribution $V$.
By \cite[Chapter 6, (2.14)]{simon}, the weak convergence $V_q  \rightharpoonup V$
implies that for any open set $W \subset M \backslash  \lambda_u$
\begin{eqnarray*}
|V|(W) &\leq& \liminf_{q \rightarrow 1} | V_q|(W) \\
&=&\liminf_{p \rightarrow \infty} \int_W |U_{p}|^{p-1}*1\\
&\leq& \liminf_{p \rightarrow \infty} \left( \int_W |U_p|^p*1\right)^{\frac{p-1}{p}}\\
&=& 0
\end{eqnarray*}
The last equality is from Proposition~\ref{prop:supptmeasure0}.
\end{thm:supptmeasure} 

\begin{corollary} \label{currentsupportfiber} There is a sequence $p \rightarrow \infty$ (or equivalently $q \rightarrow 1$) such that 
\[
\lim_{q\rightarrow 1} \int_M |*du \wedge dv_q|*1 = 0.
\]
%Furthermore, if $n=2$,
%\[
%*du \wedge dv=0.
%\]
\end{corollary}
\begin{proof}
We have
\begin{eqnarray*}
\lefteqn{\lim_{q\rightarrow 1} \int_M |*du \wedge dv_q|*1}\\
&=&\lim_{p\rightarrow \infty}\int_M |U_p|^{p-2} |du \wedge U_p| *1\\
&\leq& \lim_{p\rightarrow \infty} \left(\int_M |U_p|^{p-2}| U_p \wedge U_p| *1
+\int_M |U_p|^{p-2}|<du-U_p, U_p>| *1\right)\\
&=& \lim_{p\rightarrow \infty}\int_M |U_p|^{p-2}|<du-U_p, U_p>| *1\\
&\leq&\lim_{p\rightarrow \infty} \int_M |U_p|^{p-1}|du-U_p| *1\\
&\leq&\lim_{p\rightarrow \infty} \int_M |U_p|^{(p-2)/2}|du-U_p||U_p|^{p/2} *1\\
&\leq&\lim_{p\rightarrow \infty} \left(\int_M |U_p|^{p-2}|du-U_p|^2*1\right)^{1/2} \left(\int_M |U_p|^p *1\right)^{1/2}\\
&=& 0 \ \ \mbox{(by Proposition~\ref{kprop 4}, (\ref{klemma11}) and  (\ref{normintv1}))}.
\end{eqnarray*}
%To prove the second equality, we have to assume that $n=2$ because in higher dimensions we don't necessarily know that $du$ is continuous. Let $f \in \mathcal D_1(M)$. Then,
%\begin{eqnarray*}
%|*du \wedge dv(f)|&=&
%\lim_{p \rightarrow \infty} \left| \int_M f*du \wedge dv_{q} \right| \\
%&\leq& |f|_{L^\infty}\lim_{p \rightarrow \infty}  \int_M |*du \wedge dv_{q}| \\
%&=& 0.
%\end{eqnarray*}
%We now prove the second equality. Fix $p_0$
%\begin{eqnarray*}
%\lefteqn{ \left| \int_M |du_{p_0} \wedge dv_q|*1-\int_M |du_p \wedge dv_q|*1 \right|}\\
%&\leq& \int_M |(du_{p_0}-du_p) \wedge dv_q|*1\\
%%&=&\lim_{p\rightarrow \infty}\int_M |U_p|^{p-2} |du \wedge U_p| *1\\
%%&\leq& \lim_{p\rightarrow \infty}\int_M |U_p|^{p-2}| U_p \wedge U_p| 
%%+\lim_{p\rightarrow \infty}\int_M |U_p|^{p-2}|<du-U_p, U_p>| *1\\
%%&=& \lim_{p\rightarrow \infty}\int_M |U_p|^{p-2}|<du-U_p, U_p>| *1\\
%&\leq&\int_M |U_p|^{p-1}|du_{p_0}-du_p| *1\\
%%&\leq&\lim_{p\rightarrow \infty} \int_M |U_p|^{p-1}|du-U_p| *1\\
%&\leq& \int_M |U_p|^{(p-2)/2}|du_{p_0}-du_p||U_p|^{p/2} *1\\
%&\leq& \left(\int_M |U_p|^{p-2}|du_{p_0}-du_p|^2*1\right)^{1/2} \left(\int_M |U_p|^p *1\right)^{1/2}\\
%&\approx& \left(\int_M |U_p|^{p-2}|du_{p_0}-du_p|^2*1\right)^{1/2} \\
%&\leq& \left(\int_M |U_p|^{p-2}|du_{p_0}-du|^2*1\right)^{1/2} \\
%&+& \left(\int_M |U_p|^{p-2}|du-U_p|^2*1\right)^{1/2}\\
%\end{eqnarray*}
%
%By taking $p \rightarrow \infty$ as before the second term goes to 0. Thus
%\begin{eqnarray*}
%\lefteqn{ \lim_{q \rightarrow 1}\int_M |*du_{p_0} \wedge dv_q|*1}\\
%&\leq& \left(\int_M |U_p|^{p-2}|du_{p_0}-du|^2*1\right)^{1/2} \left(\int_M |U_p|^p *1\right)^{1/2}\\
%\end{eqnarray*}
%%&=& 0 \ \ \mbox{(by Proposition~\ref{kprop 4}, \ (\ref{klemma11}) \ and \ (\ref{normintv1}))}
%
%which proves the first equality. 
\end{proof}

\subsection{Stronger version of the support argument} In this section we show that  Proposition~\ref{kprop 4} and Proposition~\ref{prop:supptmeasure0} can be modified to cover the case where we replace $U=du$ by the derivative of any Lipschitz map $u'$ in the same homotopy class of $u$.
More precisely, let
\[
u': \tilde M \rightarrow \R
\]
be a $\rho$-equivariant  Lipschitz map and let 
\[
c'=\max |du'|, \  \ U'=\frac{1}{c'} du'.
\]
We continue with the normalization of the best Lipschitz constant $L = 1$ and since $u'$ is in the same homotopy class of $u$ we have $c' \geq 1$.
\begin{proposition} \label{kprop 4*}  
\[
\lim_{p\rightarrow \infty}\int_M |U_p|^{p-2}|U_p - U'|^2 *1 \leq C (c'-1).
\]
\end{proposition}
\begin{proof} We have to adapt the proof of Lemma~\ref{kprop 3} and Proposition~\ref{kprop 4}. First, we set
\[ 
G(p) =  2<U_p ,U_p -U'> =|U_p|^2+ |U_p -U'|^2 - |U'|^2.
\]
Equation (\ref{kpU}) has to be modified to
\begin{equation*}
\int_M |U_p|^{p-2}<U_p, U_p - k_pc'U'>*1 = 0,
\end{equation*}
hence
\begin{eqnarray*}\label{kpUc}
\int_M |U_p|^{p-2}<U_p, U_p - k_pU'>*1 &\leq& \int_M |U_p|^{p-1}| k_pU' - k_pc'U'|*1 \\
&\leq& Ck_p (c'-1)
\end{eqnarray*}
and equation (\ref{kpU234}) to
\begin{eqnarray}\label{kpU234c}
\lim_{p\rightarrow \infty} \int_M |U_p|^{p-2}G(p)*1 \leq C (c'-1).
\end{eqnarray} 
This  error persists through the rest of the proof without any additional changes, from which the result follows.
\end{proof}

A  consequence is the following generalization of Theorem~\ref{thm:supptmeasure} about the stretch locus of any best Lipchitz map:
\begin{corollary}\label{strsuppp}
 The support of the current $V$  is contained in the locus of maximum stretch $ \lambda_{u'}$ for any best Lipschitz map $u'$.
\end{corollary}
\begin{proof}With the normalization $L=1$ the best Lipschitz map $u'$ has $c'=1$. We continue the proof of Theorem~\ref{thm:supptmeasure} by using 
Proposition~\ref{kprop 4*} instead of Proposition~\ref{kprop 4}. Since $c'=1$ both Propositions yield the same answer, so there is no difference in the argument.
 \end{proof}
 
 We can rephrase the corollary above as follows: 
  Following \cite[Definition 1.2]{kassel}, let $\mathcal F$ denote the collection of $\rho$-equivariant  best Lipschitz functions $u': \tilde M \rightarrow \R$ and define 
  \[
\lambda=\cap_{u' \in \mathcal F} \lambda_{u'}.
\]
By \cite[Lemma 5.2]{kassel}, $\lambda$ is a geodesic lamination which
 plays the role of Thurston's chain recurrent lamination (cf. \cite[Theorem 8.2]{thurston}).
 Corollary~\ref{strsuppp} can be restated by saying that the support of the current $V$  is contained in $\lambda$.

 
\subsection{Maps of least gradient}\label{LGRad} In this section we assume $n=2$ and write $V=dv$ where $v: M \rightarrow L$ is the section corresponding to $\tilde v: \tilde M \rightarrow \R$ equivariant under $\alpha$.
Least gradient is usually defined with respect to the Dirichlet problem in a domain.  When we have a section  $v$  we can, of course, define it with respect to the Dirichlet problem on domains in M, but we will give a more global definition. We will first prove the following consequences of Proposition~\ref{kprop 4}.

\begin{corollary}\label{withoutcont}  If $n = 2$ then,
\[
 \ \lim_{q\rightarrow 1} \int_M du \wedge dv_q = 1 \ \mbox{and} \ \lim_{p \rightarrow \infty} \int_M du_p \wedge dv =1.
\]
\end{corollary}
\begin{proof}  
The proof of the first equality is similar to Corollary~\ref{currentsupportfiber}:
\begin{eqnarray*}
\lefteqn{\lim_{q\rightarrow 1} \int_M du \wedge dv_q}\\
&=&\lim_{p\rightarrow \infty} \left( \int_M (du-du_p) \wedge dv_q+ \int_M du_p \wedge dv_q \right)\\
&=&\lim_{p\rightarrow \infty} \int_M (du-du_p) \wedge dv_q  +1 \ \ \mbox{(by (\ref{kappavolform22}))} \\
&=& 1 \ (\mbox{as in the proof of Corollary~\ref{currentsupportfiber}}).
\end{eqnarray*}


For the second, 
continue the normalization of $L = 1$.  We use Proposition~\ref{maxest}
\begin{equation}\label{normcp}
c_l := max|du_l| \rightarrow L=1 \ \mbox{as} \  l \rightarrow \infty.
\end{equation}
Let 
\[
{U_l}'=  \frac{1}{c_l}du_l. 
\]
By Proposition~\ref{kprop 4*},
\begin{equation}\label{variant64}
 \lim_ {p \rightarrow \infty} \int_M |U_p|^{p-2} |U_p - U'_l|^2 *1 \leq C(c_l-1).
\end{equation}
Note,
\begin{eqnarray}\label{cuo13}
\lefteqn{\left |\int_M (U_p - U'_l)  \wedge dv_q  \right |} \nonumber \\
&= & \left |\int_M |U_p|^{p-2} <U_p - U'_l, U_p> *1  \right | \nonumber \\
&\leq & \int_M |U_p|^{p-1}|U_p-U'_l| *1   \\
 &\leq &\left( \int_M |U_p|^p *1 \right)^{1/2}\left( \int_M |U_p|^{p-2} |U_p - U'_l|^2 *1 \right)^{1/2}   \nonumber\\
 &\leq & {k_p} ^{1/2}C (c_l-1)^{1/2}  \ (\mbox{by (\ref{variant64}) and (\ref{normintv1})}). \nonumber
\end{eqnarray}
Thus, (\ref{klemma1}) implies
\begin{equation}\label{cuo1}
 \lim_ {p \rightarrow \infty}  \left|\int_M (du_p - U'_l)  \wedge dv_q  \right | \leq C (c_l-1)^{1/2}.
\end{equation}
 By (\ref{kappanorm}),
\begin{equation}\label{cuo2}
\left| \int_M (U'_l - du_l )  \wedge dv_q \right | \leq \max |U'_l - du_l| \leq c_l-1.
\end{equation}
By combining (\ref{cuo1}) and (\ref{cuo2})
\begin{equation}\label{cuo3}
\left| \int_M (du_p - du_l )  \wedge dv_q \right |  \leq  c_l-1+C (c_l-1)^{1/2}.
\end{equation}
Take now $p \rightarrow \infty$ ($q \rightarrow 1$) and use $dv_q  \rightharpoonup dv$,
 and (\ref{kappavolform}) to obtain
 \begin{equation}\label{cuo4}
\left| 1 -\int_M du_l   \wedge dv \right |  \leq c_l-1+C (c_l-1)^{1/2}.
\end{equation}
By (\ref{normcp}) $c_l \rightarrow 1$, hence  
the result follows.
\end{proof}



\begin{theorem}\label{thmlegr} The section $v$ is a section of least gradient in the sense that for  all functions $\phi$ on $M$ of bounded variation
\[
||dv|| \leq ||d (v+\phi)||. 
\]
\end{theorem}

\begin{proof} First of all, we show that $||d(u + \phi)|| \geq 1/L$.
  We pick a sequence $p$ such that $v_q \rightarrow v$.  Let $U'_p = \frac{1}{c_p}du_p$, where $c_p$ as in (\ref{normcp})
is a normalizing factor which sets  $\max |U'_p|=1$.  Then
\begin{eqnarray*} 
\lim_{p \rightarrow \infty} \int_M U'_p \wedge (dv + d\phi)   &=& \lim_{p \rightarrow \infty} \int_M U'_p \wedge dv \\
&=& \lim_{p \rightarrow \infty} 1/{c_p} \int_M    du_p \wedge dv\\
&=& 1/L  \ (\mbox{by   Corollary~\ref{withoutcont} and (\ref{normcp})}).
\end{eqnarray*}
We are going to complete the proof by showing that $||dv|| = 1/L$. Indeed, for any $\Phi \in\Omega^1(M)$ with $\max | \Phi| \leq 1$,
\begin{eqnarray*}
\int_M   \Phi \wedge dv &=& 
\lim_{p \rightarrow \infty}    \int_M   \Phi \wedge dv_q  \\
    &\leq&   \lim_{p \rightarrow \infty}  \int_M |dv_q|*1\\
    &\leq& 1/L \ (\mbox{by} \  (\ref{kappanorm0})).
\end{eqnarray*}
\end{proof}


%\begin{corollary}\label{thmlegr1} $\tilde v$ is least gradient in any ball in $\tilde M=H^2$ in the sense that $||d\tilde v + d\xi||_B \geq || dv||_B$ for any $\xi$ in BV with compact support in the interior of $B$. Moreover, equality holds if and only if $\xi = 0$.
%\end{corollary}
%\begin{remark} This is weaker than allowing $\xi$ to be in $BV(B)$ with 0 boundary values. However, see Remark 2.2 of \cite{mazon} and Theorem 2.2 of \cite{ziemer2},
% the two notions are equivalent for domains in $R^n$ and it is likely this will also hold for domains in manifolds.
%\end{remark}
%\begin{proof} {\bf{CHANGE}}
% Following the proof of the theorem, 
% \begin{equation}\label{BVboundary}
%||d\tilde v||_B = \lim_{j \rightarrow \infty}   \int_B f_j d\tilde u^\flat \wedge d \tilde v
%\end{equation}
%where $f_j $ is a cut-off function which is 1 on $B_j =\{x \in B,  d(x, \partial B) >1/j \}$ and $f_j = 0$ on $B_{j+1}$. Given $\xi$, we can assume $\xi$ has support in $B_n$ for some $n$. Let $\Phi_j = f_j d\tilde u^\flat$.
%Then,
%\[
%\int_B  \Phi_j \wedge (d \tilde v + d\xi)  = \int_B f_j d \tilde u^\flat \wedge d\tilde v   + \int_B f_j d \tilde u^\flat \wedge d\xi.
%\]
%However $\xi$ has support in $B_n$ so for $j > n$,
%\[
%\int_B f_j d \tilde u^\flat \wedge d\xi= \int_{B_n} d\tilde u^\flat \wedge d\xi = \int_B d\tilde u^\flat \wedge d\xi =0.
%\]
%Combining with (\ref{BVboundary}),
%\[
%\lim_{j \rightarrow \infty}\int_B  \Phi_j \wedge (d \tilde v + d\xi)=||d\tilde v||_B
%\]
%and since $ \max |\Phi_j| \leq 1$,
%\[
%||d\tilde v +d \xi ||_B\geq ||d\tilde v||_B.
%\]
%Now suppose $\xi \neq 0$. Then the connected open sets $\hat S_j $ where $B \backslash \lambda_u = \cup \hat S_j$ all intersect $\partial B$, since they are otherwise bounded by geodesics. If $\xi \neq 0$, then $d\xi \neq 0$ on some $\hat S_i$, and  $\int_B   \phi \wedge d\xi= 1$ for some smooth $\phi$ with the support in $S$ in $\hat S_i $ with $\max_S |\tilde u^\flat| = \delta < 1$. Let $\epsilon = \frac{(1-\delta)}{\max|\phi|}$. If we let $\Phi = f_j d \tilde u^\flat+ \epsilon \phi$. Then $|\Phi(x)| \leq 1$ at all points of $B$ and
%\[
%\int_B \Phi \wedge (d\tilde v+ d\xi)  = \int_B f_j d \tilde u^\flat \wedge d\tilde v   + \epsilon.
%\]
%Here we use again the fact that $d\xi$ pairs with $f_j d \tilde u^\flat $ to give zero and that by Theorem~\ref{thm:supptmeasure}, the support of $d\tilde v$ does not intersect the support of $\phi$. The contradiction that $||\xi \wedge d\tilde v  ||_B > || d\tilde v  ||_B $ follows from (\ref{BVboundary}).
%\end{proof}
%



\subsection{The equivariant problem}\label{sectevp} The results of the previous sections generalize in a straightforward way if we replace the map $f: M \rightarrow S^1$ by a 
$\rho$-equivariant map. 
More precisely, let $\rho \in H^1(M, \R)$. We can view $\rho$ as a homomorphism $\rho: \pi_1(M) \rightarrow \R$ and consider maps 
\[
\tilde f: \tilde M \rightarrow \R
\] 
satisfying the equivariance relation
\begin{equation}\label{equivfr}
\tilde f(\gamma \tilde x)=  \tilde f( \tilde x)+ \rho(\gamma), \ \forall \gamma \in \pi_1(M)\  \mbox{and} \   \forall \tilde x \in \tilde M. 
\end{equation}
In the case when $\rho$ is integer valued the map $\tilde f$ descends to a map $f:M \rightarrow S^1$ with induced homomorphism $f_*=\rho$ on the fundamental groups as studied in the previous sections. We will denote by $f$ the induced section of the flat affine bundle
$\tilde M \times_\rho \R$.

Next note that because of (\ref{equivfr}), the 1-form $d\tilde f$ is invariant under $\rho$. Hence it descends to a closed 1-form on $M$ which we denote by $df$.  We can proceed as before with minimizing  integral (\ref{pharmfun}) to obtain a $\rho$-equivariant map $\tilde u_p: \tilde M \rightarrow \R$ satisfying the $p$-harmonic map equation (\ref{pharm}). Furthermore, by taking $p \rightarrow \infty$ we obtain an infinity harmonic $\rho$-equivariant map $\tilde u: \tilde M \rightarrow \R$.  The map $\tilde u$ is a best Lipschitz map in the sense that it minimizes the Lipschitz constant among all $\rho$-equivariant maps. Theorem~\ref{thm:limminimizer} generalizes to this case.

The definition of the dual harmonic function $\tilde v_q$ in Section~\ref{sect:conjug} goes unchanged since its definition is purely in terms of $du_p$. The same goes with the convergence results as $q \rightarrow 1$ in Section~\ref{qgoesto1}.

The definition of the maximum stretch set $\lambda_u$ and proof of Lemma \ref{straightline} only involves $L=|du|_{L^\infty}$ and thus makes sense for any equivariant map $\tilde u$. The theory on comparison with cones is local and thus it is not affected by going to equivariant maps.  The proof of Theorem~\ref{realized} remains unchanged. Finally, the results of this section on the support of the measure $V=dv$ and the least gradient property only involve the equivariant map $v$ and are not dependent on where $du$ came from. Thus there are no changes here as well. We state this in the form of the following theorem:

\begin{theorem}\label{thm:equivsit} Fix a homomorphism $\rho: \pi_1(M) \rightarrow \R$. There exists a $\rho$-equivariant infinity harmonic function $\tilde u: \tilde M \rightarrow \R$ and a least gradient function $\tilde v: \tilde M \rightarrow \R$ equivariant under a representation $\alpha: \pi_1(M) \rightarrow \R$. Furthermore, the support of the measure $dv$ is in the maximum stretch lamination defined by $\tilde u$.
\end{theorem}

 
\section{Construction of the transverse measure from the least gradient map}\label{sect7} 
In this section we assume $M$ is a closed hyperbolic surface, i.e $\tilde M=H^2$. We first review the concepts we need from topology to get the result about transverse measures.  These include Definition~\ref{flowbox}  flow boxes, Definition~\ref{orientflow} orientation, Definition~\ref{MDefinition5A}  transversals and Definition~\ref{deftransversecocycle} transverse cocycle. Following Bonahon, we connect the notion of  functions $\tilde v$ which are $\pi_1(M)$-equivariant and locally constant on $\tilde M \backslash \tilde \lambda$ with transverse cocycles.  In Theorem~\ref{transmeasure}, we use his theorem that a transverse cycle is a transverse measure if and only if it is non-negative to show that the least gradient map $v$ constructed in Theorem~\ref{lemma:limmeasures2} defines a transverse measure on the maximum stretch lamination $\lambda_u$ associated with the $\infty$-harmonic map $u$.   The definition of a transverse measure is equivalent to a function   on the universal cover with the right properties fits in well with our function of bounded variation $v$ (or $\tilde v$) which is constant on the components of $ M \backslash  \lambda$. See Theorem~\ref{transcoismeasthm}.

  \subsection{Flow boxes}
  
We start with the following elementary lemma from hyperbolic geometry
\begin{lemma}\label {lemma:flow} Let $\lambda$ be a lamination, and $f$ a geodesic orthogonal to a leaf $\lambda_0$. For $k \in \lambda \cap Im(f)$ and $\lambda_k$ be the leaf of $\lambda$ through $k$, let $n(k) =e^{ i \kappa(k)}$ be the unit tangent direction of $\lambda_k$ when it intersects the  geodesic $f$ at $k$ and the same for $k'$. Then, there is a constant $c>0$ such that $|\kappa(k) - \kappa(k')| \leq cd_{H^2}(k,k')$.
\end{lemma}
\begin{proof} We use the unit disk model of hyperbolic space, and place the geodesic formed by $f $ on the $x$ axis, and the point $k$ at the origin. In other words, write 
\[
f: (-1,1) \rightarrow H^2 \simeq D^2; \ f(t)=t, \ k=0
\]
and the geodesic $\lambda_k$ is the straight line
\[
\lambda_k = \{n(k)t: \ -1 < t < 1\}.
\]
Similarly, the geodesic $\lambda_{k'}$ through $k'=f(k')>0$ (if $k'<0$ reverse the role of $k$ and $k'$) is the geodesic
\[
\lambda_{k'}= \{n(k')\frac{t +w}{1 + \bar w t}: \   -1 < t < 1\},
\]
where $w =\overline{ n(k')} k'$.  The geodesics $n(k) t$ and $n(k') t$ intersect, and $\lambda_k$ and $\lambda_{k'}$ do not, so for some $w'' = \overline{ n(k')}k''$, $0 < k'' \leq k'$, the geodesics $n(k) t$ and $n(k')\frac{t + w''}{1 + \bar w'' t}$ intersect at the endpoints on the unit circle $t=1$ or $t= -1$. Then
\[
\pm n(k) = \pm n(k')\frac{1 \pm w''}{1 \pm \bar w''}
\]
or equivalently,
\[
\frac{n(k)}{n(k')} = 1 \pm \frac{w'' - \bar w'' }{1 \pm \bar w''}.
\]
Then $| 1 - e^{i(\kappa(k) - \kappa(k'))}| \leq 2 \frac{|w''|}{1 - |w''|} \leq  \frac{2k'}{1-k'}$.
Since $d_{H^2}(k,k')= \tanh^{-1}(k') $, this inequality converts to the inequality in the Lemma provided $d_{H^2}(k,k')$ is not too large.
\end{proof}

\begin{definition}\label{flowbox} By a {\it flow box} or a {\it chart} for a geodesic lamination $\lambda$ we mean a bi-Lipschitz homeomorphism 
\begin{equation}\label{eqn:flowbox}
F: R^\sharp=[a,b] \times [c,d]  \rightarrow  F(R^\sharp) =R\subset M; \ F=F(t,s)
\end{equation}
such that there exists a closed set $K \subset (c,d)$ of Hausdorff dimension 0 such that
\[
F^{-1}(\lambda)= [a,b] \times K.
\]
\end{definition}
%{\bf{Change 4: These are classical flow boxes. Should I include a proof they exist and are Lipschitz????}}


 \begin{proposition} \label{prop:flow} Any geodesic lamination  on a closed hyperbolic surface (M,g) has an open neighborhood covered by a finite number of flow boxes (\ref{eqn:flowbox}). Furthermore,
   $F$  can be chosen so that  $\frac{\partial F}{\partial t}$  is a Lipschitz vector field along $F$.  \end{proposition}
\begin{proof} Let $f: [c,d] \rightarrow M$ be a geodesic transversal to the lamination as in Lemma~\ref{lemma:flow}. Let
\begin{equation}\label{normal}
n: [c,d] \rightarrow \R
\end{equation}
 denote the Lipschitz function defined as follows. Let $K = \{ k \in [c,d]: f(k) \in \lambda \}$ and $n(k) \in T_{f(k)}(M)$ be the unit tangent vector to the leaf of $\lambda$ through $k$, $k \in K$. By Lemma~\ref{lemma:flow},  $|n(k)- n(k')| \leq cd_{H^2}(k,k')$.  Extend $n$ to a Lipschitz function  on the interval by linearly interpolating on each interval in the complement of $K$. 
Define 
\[
F(t,s) = exp_{f(s)}(t n(s)). 
\]
Then $F$ is Lipschitz and
\[
\frac{\partial F}{\partial t}=dexp_{f(s)}(t n(s))n(s),
\]
is also Lipschitz. In order to show that $F^{-1}$ is Lipschitz, first observe that it is clearly Lipschitz in the $t$-direction. In order to show Lipschitz in the $s$-direction choose $k<k'$ and place as before $k=0$. It suffices to show for the geodesics $\lambda_k$ and $\lambda_{k'}$ and $t$ small (independently of $k$ and $k'$),
\[
\left| n(k)t- n(k')\frac{t +w}{1 + \bar w t} \right| \geq ck'.
\]
This can be easily achieved since $|w|=k'$ and $|t|$ are both small, $|n(k)-n(k')|<Ck'$ by Lemma~\ref{lemma:flow} and the denominator is uniformly bounded away from zero.
 \end{proof}
 
 \begin{remark}\label{nonsmooth}
 Note that $\frac{\partial F}{\partial s}$ is in $L^\infty$ but not necessarily continuous unless $\frac{d n} {ds} $ is. At the moment we are unable to 
 obtain such regularity for $\frac{d n} {ds} $. 
 \end{remark}
 Note that by construction,
 \begin{equation}\label{norone} 
 \left  |\frac{\partial F}{\partial t} (s,t) \right |=1 \ \ \mbox{for} \ \ s \in K.
 \end{equation} 

 
 \begin{definition}\label{orientflow} A geodesic lamination $\lambda$ is called {\it orientable}, if there exists a Lipschitz unit vector field $n$ defined in a neighborhood of $\lambda$ and transverse to the leaves. 
  \end{definition}
 
 
 Note that by Lemma~\ref{lemma:flow}  a normal vector field exists locally, so the issue is existence of a global vector field. Also note that together with a choice of an ambient orientation for $M$, a choice of $n$ determines an orientation of the leaves. More precisely, the direction of the leaves followed $n$ must coincide with the orientation of $M$.
 
 Definition~\ref{orientflow} is clearly equivalent to any of the following conditions:\\
  $(i)$ There is a cover of a neighborhood of $\lambda$ with flow boxes $F$ as in Definition~\ref{flowbox} such that $F$ are orientation preserving
 with respect to  the ambient orientation of the manifold $M$ and the product orientation on $R^\sharp$. \\
 $(ii)$ Given $x \in \lambda$ and $\beta: (-\infty, \infty) \rightarrow M$ an orientation preserving parametrization of the leaf through $x$ with $\beta(0)=x$, there exists $\epsilon >0$ such that the map
 \begin{equation}\label{p1}
p_1\circ F^{-1}\circ  \beta: [-\epsilon, \epsilon] \rightarrow [a,b]
\end{equation}
 is orientation preserving, where $p_1$ denotes projection onto $[a,b]$.
 A cover of a neighborhood of $\lambda$ by flowboxes as above is called an {\it{oriented atlas}} of the lamination. An oriented atlas determines completely the orientation of $\lambda$. 

 
 Throughout the section we fix an oriented atlas for $\lambda$ consisting of flow boxes $\{F\}$.
  

\begin{definition}\label{MDefinition5A}
For a continuous path $f: [l, m] \rightarrow M$, we let
\begin{equation}\label{defK}
K=f^{-1}\left(f([l, m]) \cap \lambda \right)
\end{equation}
and call $f$  {\it{transverse to the lamination}} $\lambda$ if for every $k \in K$ there exists a flow box 
$F: R^\sharp=[a,b] \times [c,d]  \rightarrow  F(R^\sharp) =R\subset M $
at $f(k)$ and $\eta=\eta(k)>0$ such that 
\begin{equation}\label{p2}
 p_2 \circ F^{-1} \circ f: [k-\eta, k+\eta] \rightarrow [c,d]
\end{equation}
is a homeomorphism onto its image, where $p_2$ denotes projection onto $[c,d]$.
We call $f$  an {\it{admissible transversal}}, if in addition $f(l), f(m)\in M_0=M \backslash \lambda$.
\end{definition}

\begin{definition}\label{MDefinition5} Let $f: [l, m] \rightarrow M$ be an admissible transversal. We say that 
  $f$ is {\it{positively (resp. negatively) transverse}} to $ \lambda$ if for every $k \in K$ and every oriented flow box $F$ at $f(k)$ the map (\ref{p1}) is increasing (resp. decreasing) function of s.

Note that all our definitions are clearly seen to be independent of the parameterization. 
\end{definition}

%This follows immediately from the definition of $\nu$ in terms of $\beta$. \begin{definition}\label{MDefinition8} Let $F: R^\sharp=[a, b] \times [0,1]  \rightarrow M$ be a smooth diffeomorphism into its image $R=F(R^\sharp)$. We say that $F=F(t,s)$ is a {\it{smooth homotopy  positively transverse to $ \lambda_u$}} if
%\begin{itemize}
%\item $(i)$$ F([a,b] \times \{ 0,1\})  \subset M_0$
%\item $(ii)$ The family of transversals $s \mapsto f_t(s):=F(t,s)$ depending smoothly on $t$ are positively oriented for all $t$
%\item $(iii)$ $dt \wedge ds = J(F) * 1$, where the Jacobian $J(F) > 0$.
%\end{itemize}
%We define the notion of {\it{smooth homotopy  negatively transverse to $ \lambda_u$}} if we replace condition $(ii)$ with negatively oriented instead of positively oriented.
%\end{definition}
%\begin{remark}
%  It is important to note that we don't require $F$ to be a homotopy in the usual sense i.e $F([a,b] \times  \{s\})$ to be parallel to the leaves of the lamination (cf. Definition~\ref{flowbox}), since we are unable to obtain enough regularity in the $s$-direction for these more restrictive flow boxes (cf. Remark~\ref{nonsmooth}).
%\end{remark}

\begin{definition}\label{MDefinition12} Let $f:[l,m] \rightarrow M$ be an admissible transversal. If $l =l_0 <l_1 < ...< l_n = m$ is a division of $[l,m]$ into intervals on which $f(l_i) \in M_0$ and $f_i(s) = f(s)$, $l_{i-1}\leq s \leq l_i$ is alternatively positively and negatively transverse to $ \lambda$, we say $[l,m] = \bigcup_i [l_{i-1},l_i]$ is a good subdivision for $f$.
\end{definition}
\begin{lemma}\label{MLemma13}  Let $f: [l,m] \rightarrow M$ be an admissible transversal to an oriented lamination $ \lambda$. Then $[l,m]$ has a good subdivision for $f$.
\end{lemma}
\begin{proof} Since there are finitely  many flow boxes we may assume without loss of generality that the image of $f$ is contained in $R$ for some oriented flow box
$F: R^\sharp=[a,b] \times [c,d]  \rightarrow  F(R^\sharp) =R\subset M $. Consider the
continuous map
\[
g=p_2 \circ F^{-1} \circ f: [l, m] \rightarrow [c,d].
\]
Given a point $k \in K$, consider open interval $[k-\epsilon(k), k+\epsilon(k)]$ around $k$ such that $g$ is strictly monotone. By compactness, we can cover $K$ with finitely many such intervals and let $\epsilon=\min \epsilon(k)$.
  
We now construct the good subdivision $l =l_0 <l_1 < ...< l_n = m$. For each $k \in K$ assign  $+=sign(k)$ to the interval $[k-\epsilon(k), k+\epsilon(k)]$ if $f$ defines a positive transversal and $-=sign(k)$ if it defines a negative transversal.
 If $K =\emptyset$, $n = 1$. Let $k_1 = \min_{k \in K} k$ and assign the sign of $sign(k_1)$ to the first interval. Let $k_2 = \min_{k \in K} k$ such that $sign(k_2)$ has the opposite sign. If there is no such $k_2$, $n = 1$. If there is such a $k_2$, choose $l_1 < k_2$ as the largest point less than $k_2$ on for which $f$  is strictly monotone in the interval $[ l_1, k_2]$. Proceed inductively. The process is finite as there is a lower bound $\epsilon$ on the size of the intervals.
\end{proof}



\subsection{Transverse cocycles}\label{tranco} Let $\lambda$ be an oriented geodesic lamination and let  $M_0=M \backslash \lambda$. We write $M_0=\bigcup S$ for {\it finitely many} connected components $S$ called the {\it principal regions} or {\it open plaques} (cf. \cite[Lemma 4.3]{casson}). Lifting to the universal cover we denote $\tilde {M_0}= \tilde M \backslash \tilde \lambda=\bigcup \tilde S$ where  $\tilde S$ is the preimage of $S$. Each component $\tilde S_j$ of $\tilde S$  is also called an open plaque and the projection map $\tilde {S_j} \rightarrow S$ is the universal cover of $S$. Furthermore, the closure of $\tilde {S_j}$ in $H^2$ is a contractible surface with geodesic boundary (cf. \cite[Lemma 4.1]{casson}) and its boundary is contained in the preimage of the boundary leaves of $\lambda$ (cf. \cite[definition and remark on p.61]{casson}).   In this section we start with a map  
   \[
   \tilde v: \tilde M \rightarrow \R
   \]
   with the following properties:
   
   \begin{itemize}
   \item $(i)$ $\tilde v$ is equivariant under a representation $\alpha :\pi_1(M) \rightarrow \R$
   \item $(ii)$ $\tilde v \equiv a_j$ is constant on each plaque $\tilde S_j \subset \tilde M \backslash \tilde \lambda$
   \item $(iii)$ $ \tilde v$ is locally bounded.
   \end{itemize}
   
   
   
 
 For $S$ and $S'$ open plaques, we set
\begin{equation}\label{MDefinition3}
\beta (S, S') = \tilde v(S)- \tilde v(S').
\end{equation}
Note that since $\tilde v$ is equivariant under  $\alpha$, it follows that $\beta$ is invariant under the action of $\pi_1$. The goal of this section is to define a transverse cocycle $\nu$ induced by $\beta$. 



\begin{definition}\label{MDefinition6} For $f :[l,m] \rightarrow M $ an admissible transversal positively oriented, define
$\nu(f) = \beta(S_m, S_l)$ where $\tilde f(l) \in S_l$,  $\tilde f(m) \in S_m$ are the open
plaques containing the endpoints of a lift $\tilde f$. For $f $ an admissible transversal negatively oriented, define
$\nu(f) = -\beta(S_m, S_l)$. Since $\beta$ is invariant under $\pi_1$, this is independent of the lift. 
Finally for an admissible transversal $f$ and  a good subdivision, we define 
\[
\nu(f) = \sum_i  \nu(f_i).
\]  
\end{definition}
\begin{lemma}\label{Mproposition7} $\nu(f)$ does not depend on the choice of $l_i$ in a good subdivision. If $f$ is split into two sub-arcs $f(s) = f_1(s), \ l \leq s  \leq  p$ and $f(s) = f_2(s), \  p \leq s \leq m$ with $f(p) \in M_0$, then $\nu(f) = \nu (f_1) + \nu (f_2)$. Moreover, $\nu$ is invariant under homotopies of $f$  which preserve the lamination and are transverse to the lamination. Finally, $\nu(f) = \nu(f^-)$ where $f^-$ denotes $f$ with the reverse parametrization.
\end{lemma}
\begin{proof}
 We notice that if we choose a second set of $l_i'$, there is an arc between $l_i$ and $l_i'$ which lies in $M_0$. By the properties of $ \nu$, we may move the endpoints of a transversal in $M_0$ without changing $\nu$. Also, the definitions of $\nu$ do not depend on the choice of parameter. Hence the two definitions of $\nu$ agree. The additive property under subdivision of transversals and invariance under change of orientation are immediate  from the definition. 
 
 To see the  invariance under homotopies, consider a  homotopy
 \[
 F: R^\sharp=[a,b] \times [c,d]  \rightarrow  F(R^\sharp) =R\subset M; \ F=F(t,s)
 \]
 and set $f_t=F(t,.)$. 
 Now consider a good subdivision $l =l_0 <l_1 < ...< l_n = m$  of $[l,m]$
  and note that because the homotopy preserves the lamination, the end points
  $f_t(l_i)$ all lie in the same plaque for $t \in [a,b]$.  Therefore,
  \[ 
  \nu(f_t \big |_{[l_i l_{i+1}]})=\nu(f_b \big |_{[l_i l_{i+1}]}).
  \]
  The rest follows from the additive property of $\nu$ with respect to subdivisions.
  \end{proof}
 
 For the next definition, see \cite[page 120]{bonahon2}.
 \begin{definition}\label{deftransversecocycle}
 A {\it{transverse cocycle}} $c$ for an oriented lamination $\lambda$ is a map 
 \[
 c: \{ admissible \  transversals \} \rightarrow \R 
 \]
 which satisfies the following properties: 
 \begin{itemize}
\item $(i)$ $c(f) = c(f_1) + c(f_2)$ when $f$ is decomposed into two subarcs as in Lemma~\ref{Mproposition7}.
\item $(ii)$ $c(f) = c (f')$ when $f$ is carried into $f'$ by a homotopy which preserves $ \lambda$ and is transverse to the foliation.
\item $(iii)$ $c(f) = c(f^-)$ where $f^-$ denotes $f$ with the reverse parametrization.
\end{itemize}
\end{definition}

Lemma~\ref{Mproposition7} now implies immediately:

\begin{theorem}\label{thm:tranco}A function $\tilde v$ satisfying properties $(i)$-$(iii)$ defines a transverse cocycle $\nu$. 
\end{theorem}

The following is Thurston's definition of transverse measure, more or less in Thurston's own words. (See \cite[Section 8.6]{thurston2}.)
\begin{definition}A transverse measure $\nu$ for a geodesic lamination $\lambda$ means a measure  defined on each local leaf space $[c,d]$ of every flow box,
in such a way that the coordinate changes are measure preserving. Alternatively one
may think of $\nu$ as a measure defined on every admissible (unoriented) transversal 
to $\lambda$, supported on the intersection of the transversal with the lamination and invariant under local projections along leaves of $\lambda$. 

In this paper we use this definition except we allow the support of the measure to possibly be strictly contained in the intersection of the transversal with the lamination. It is straightforward that a transverse cocycle $c$ is  a transverse measure iff $c(f) \geq 0$ for every $f$ positively transverse to $\lambda$ (cf. \cite[Proposition 18]{bonahon2}.)
\end{definition}

 
 \subsection{The transverse measure on $\lambda_u$} We now go back to the sequence
$v_q $ of $q$-harmonic sections  converging  as in Theorem~\ref{lemma:limmeasures2} and Theorem~\ref{thmlegr} to a fixed least gradient section $v$ along a sequence $q \rightarrow 1$. Also,   
$\lambda_u$  is the geodesic lamination of maximum stretch of the $\infty$-harmonic map $u$ constructed in Section~\ref{sect:crandal}. 
The main theorem of the section is:

\begin{theorem}\label{transmeasure}The least gradient map $v$ induces a transverse measure $\nu$ on the geodesic lamination $\lambda_u$. 
\end{theorem}

Let   $\sigma: \tilde M \rightarrow M$ denote the universal cover, and denote the lift of $v_q$ by $\tilde v_q$, the lift of $v$ by $\tilde v$ and so forth.
 Let $M \backslash  \lambda_u = M_0$ and $\tilde M_0 = \sigma^{-1}(M_0)$. 
By Theorem~\ref{straightline},
\begin{equation}\label{MLemma1}
M_0 = |du|^{-1}([0,L)) \ \mbox{and} \ \tilde M_0 = |d\tilde u|^{-1}([0,L).
\end{equation}
The lamination $\lambda_u$ has in our context a natural orientation given by $grad\ u$.
Let $\tilde M_0 =\bigcup S_j$ where $S_j$ are the open connected components of $\tilde M_0$.
 \begin{lemma}\label{MLemma2} $\tilde v(x) = a_j$ is constant for $x$ in the open plaque $S_j$ and the constants $a_j$ are locally bounded in $\tilde M$.
Moreover, the sequence $\tilde v_{q_j}$ converges to the constant $a_j$ in 
$W^{1,1}_{loc}(S_j)$. 
\end{lemma}
\begin{proof}
Let $\tilde B$ in $S_j$ be a closed ball in $S_j$ and let $\chi_{\tilde B}$ denote its characteristic function. From Proposition~\ref{prop:supptmeasure0}, by (\ref{normintv1}), (\ref{normintv2}) and Lemma~\ref{klemma1}, 
\[
\lim_{q \rightarrow 1} \int_{\tilde B} |d \tilde v_q|^q *1 =
L^{-1} \lim_{q \rightarrow 1} \int |U_p|^p \chi_{\tilde B}*1 = 0.
\]
By combining with Theorem~\ref{lemma:limmeasures2}, the  $\tilde v_q $ converge to $\tilde v$ in $W^{1,1}_{loc}(\tilde B)$ and also in $L^s_{loc}(\tilde B)$ for all $s$ where $d\tilde v=0$. Thus,  $\tilde v=a_j$ in $S_j$.
\end{proof}

\begin{transmeasure}
It suffices to show $\nu$ is non-negative on positive transversals. Let 
  $F: R^\sharp=[a, b] \times [0,1]  \rightarrow M$ a smooth map such that $f_t (s) = F(t,s)$  are positively transverse to to $\lambda_u$ and $f=f_b$. Notice that we don't require $F$ to be a flow box as we cannot simultaneously assume that $F$ is smooth. See Remark~\ref{nonsmooth}.
  %  -----------------------------------
%  Then $\nu(f_t)$ is independent of $t$ and $\nu =
%\nu(f_t ) \geq 0$.
%
% The fact that $\nu(f_t)$ is independent of $t$ follows from Lemma~\ref{Mproposition7}. \\
%----------------------------------------- 
 We will use the fact that $v_q \rightarrow v$ as $q \rightarrow 1$ in $L^1_{loc}$. Also, the image of $F$ is simply connected and we may choose real valued representatives of $v_q$, $u_p$, $v$ and $u$  rather than working in the cover. 
%Choose $c$ so that:
%\begin{itemize}
%\item $(i)$ $F( [a,b] \times [0,c]) = R_0$ in $M \backslash  \lambda_u$
%\item$(ii)$ $v$ is identically $a_0$ on $R_0$
%\item$(iii)$ $F( [a,b] \times [1-c,1])=R_1$ in $M \backslash  \lambda_u$
%\item$(iv)$ $v$ is identically $a_1$ on $R_1$.
%\end{itemize}
%Our goal is to show that $\nu =a_1 - a_ 0 \geq 0$. 
%By the chain rule
%\[
%\frac{d}{ds}v_q(F(t,s)) = (dv_q)_{F(t,s)}\left(\frac{dF}{ds}\right).
%\]

Choose $R_0^\sharp = [a,b] \times [0,c]$, $R_1^\sharp=[a,b]  \times [1-c,1]$ so that $F(R_i^\sharp) \subset M_0$ and $v$ is constant equal to $a_i$ on $R_i=F(R_i^\sharp)$. We have to show $a_1-a_0 \geq 0$. It will be convenient in the computations to write  $ v_q\circ F= v^\sharp_q$ and similarly for any other function defined on a subset of $R$. Choose a non negative cut-off function $\xi^\sharp \in C^\infty_0(R^\sharp)$ such that $\xi^\sharp (t,s) = \xi^\sharp (t, 1-s)$ and
\begin{eqnarray}\label{Ksi1}
\mu:=\int_{R_i^\sharp}\xi^\sharp(t,s)dtds > 0.
\end{eqnarray}
By the chain rule, 
\[
\frac{dv^\sharp_q}{ds}  = (d v_q)\circ F \frac{dF}{ds},
\]
hence by integrating in  $s$, we get
\[
 v_q^\sharp(t,1-\tau) -  v_q^\sharp(t,\tau) = \int_\tau^{1 - \tau} (dv_q)\circ F \left( \frac{dF}{ds} \right)ds.
\]
%\[
%\int_a^b v_q^\sharp(t,1-\tau)dt - \int_a^b v_q^\sharp(t,\tau)dt = \int_a^b\int_\tau^{1 - \tau} (dv_q)\circ F \left( \frac{dF}{ds} \right)dsdt.
%\]
Now multiply by $\xi^\sharp$ and integrate in $t$ from $a$ to $b$ and in $\tau$ from 0 to $c$ to obtain
\begin{eqnarray}\label{Ksi2}
\lefteqn {\int_{R_1^\sharp} \xi^\sharp(t,s)v^\sharp_q(t,s) dtds - \int_{R_0^\sharp} \xi^\sharp(t,s)v^\sharp_q(t,s)   dtds} \nonumber \\
&=&\int_a^b\int_0^c\int_\tau^{1 - \tau} (dv^\sharp_q)_{F(t,s)}\left(\frac{dF(t,s)}{ds}\right)ds\xi^\sharp(t,\tau)d\tau dt\\
&=&\int_{R^\sharp} (dv^\sharp_q)_{F(t,s)}\left(\frac{dF(t,s)}{ds}\right) \Xi^\sharp(t,s) dt ds. \nonumber
\end{eqnarray}
Here the positive function $\Xi^\sharp (t,s)$ can be explicitly computed from interchanging integration in $s$ and $\tau$. For $s < 1/2$
\[ 
\Xi^\sharp(t,s) = \int_0^{\min(s,c)} \xi^\sharp(t,\tau) d\tau= \Xi^\sharp(t,1-s).
\]
We will not use the explicit formula, however note that $\Xi^\sharp$ has compact support in the interior of $R^\sharp$ and hence $\Xi= \Xi^\sharp \circ F^{-1}$ has compact support in the interior of $R$.
By using  (\ref{normintv2}), (\ref{Ksi2}) implies
\begin{eqnarray*}
\lefteqn {\int_{R_1^\sharp} \xi^\sharp(t,s)v^\sharp_q(t,s) dtds - \int_{R_0^\sharp} \xi^\sharp(t,s)v^\sharp_q(t,s)   dtds}\\
&=&\int_{R} dv_q\left(\frac{dF}{ds} \circ F^{-1} \right)  \Xi  J(F^{-1})*1\\
&= &  \int_{R} |U_p|^{p-2} *U_p \left(\frac{dF}{ds}\circ F^{-1} \right)\Xi  J(F^{-1})*1.
\end{eqnarray*}
By (\ref{Ksi1}) and the fact that $ v_q \rightarrow  v$ in $L^1_{loc}$, the left-hand side has the limit $\mu(a_1-a_2)$.  
%Equivalently,
%\begin{eqnarray*}
%\int_{R_1} \xi v_q*1 - \int_{R_0} \xi v_q*1 =\int_{R} d v_q\left(\frac{dF}{ds}\right) (\Xi \circ F^{-1}) J(F) * 1.
%\end{eqnarray*}
%To compute the right hand side, note 
%Next integrate this equation over $ [a,b] \times  [\tau, 1 - \tau] $ to get
%\[
%\int_a^b\left( v_q(F(t,1 - \tau)) - v_q(F(t,\tau )) \right) dt=   \int_a^b \int_\tau^{ 1-\tau}(dv_q)_{F(t,s)}\left(\frac{dF}{ds}\right) dsdt.
%\]
%After integrating  over $0 \leq \tau  \leq c$,
%\begin{eqnarray*}
%\int_0^c\int_a^b\left( v_q(F(t,1 - \tau)) - v_q(F(t,\tau)) \right) dtd\tau 
%&= & \int_0^c\int_a^b\int_\tau^{ 1-\tau} (dv_q)_{F(t,s)}\left(\frac{dF}{ds}\right) dsdtd\tau\\
% &= &
% \int_0^c \int_{Q_\tau}(dv_q)_{F(t,s)}\left(\frac{dF}{ds}\right) dsdtd\tau \\
%\end{eqnarray*}
%where $Q_\tau=[a,b] \times [\tau, 1-\tau].$
%
%Since $F([a,b] \times [0, c] ) \cup F( [a,b] \times [1-c, 1])$  is contained in $M_0$, $v_q$ converges to $v$ in $L^1_{loc}(M_0)$ and the convergence in $L^1$ is independent of the metric we use, the left hand side converges to
%$c(b-a)(a_1 - a_0)$. Again, all we need show is that this is non-negative. The size of $c(b-a)> 0$ is unimportant.
%Using the Jacobian of the diffeomorphism $F $ we can convert the right hand side to an integral over the image $R_\tau =F(Q_\tau)$.  
%{\bf{GEORGE: Have to iro out the details.}} 
%\begin{eqnarray*}
% \int_{R} d v_q\left(\frac{dF}{ds}\right) \Xi  J(F) * 1
%&= &  \int_{R} |U_p|^{p-2} *U_p \left(\frac{dF}{ds}\right)(\Xi \circ F^{-1}) J(F)*1.
%%&= & k_p\int_0^c \int_{R_\tau} |U_p|^{p-2} *du_p \left(\frac{dF}{ds}\right) J(F)*1d\tau
%\end{eqnarray*}
%\[
%\lim_{q\rightarrow 1} \int_Rdv_q \left(\frac{dF}{ds}\right)  J(F) *1
%\]
%We now use (\ref{normintv2}) and Lemma~\ref{klemma1}   to rewrite this as
%\[
%\lim_{p \rightarrow \infty} \int_R |U_p|^{p-2}*U_p \left(\frac{dF}{ds}\right)J(F)*1= L^{-1}\lim_{p \rightarrow \infty} \int_R|U_p|^{p-2}*du_p \left(\frac{dF}{ds}\right)J(F)*1
%\]
By Proposition~\ref{kprop 4}  and Lemma~\ref{klemma1}, 
\begin{eqnarray*}
&&\lim_{p \rightarrow \infty}\int_{R}|U_p|^{p-2}*U_p \left(\frac{dF}{ds}\right)(\Xi \circ F^{-1})J(F)*1\\
&=& L^{-1}\lim_{p \rightarrow \infty}\int_{R}|U_p|^{p-2}*du_p \left(\frac{dF}{ds}\right)(\Xi \circ F^{-1})J(F)*1.
\end{eqnarray*}
However, in the definition of positively transverse $*du_p \left(\frac{dF}{ds}\right) > 0$,  the Jacobian $J(F)>0$ and $\Xi \geq 0$ with $\Xi > 0$ on a set of positive measure. So the right hand side is the limit of positive numbers; hence the limit must be non-negative.
By comparing with the left hand side, we obtain that  $a_1 - a_0 \geq 0$. 
\end{transmeasure}
  
  
  \begin{remark}We do not claim that the limit is positive. There can be leaves of $ \lambda_u$ on which the transverse measure vanishes.
\end{remark}
  
  We end the section by proving a general theorem relating the notion of transverse cocycles with functions of bounded variation in the case when the cocycle is non-negative. More precisely, we show:
  
  
  \begin{theorem}\label{transcoismeasthm} Assume $\lambda$ is an oriented geodesic lamination and $\tilde v: \tilde M \rightarrow \R$ satisfies properties $(i)$-$(iii)$ as in Section~\ref{tranco}. If the transverse cocycle $\nu$ associated to $\tilde v$ via Theorem~\ref{thm:tranco} is a transverse measure, then $\tilde v$ is locally of bounded variation.
  \end{theorem}
  
  
%  The local boundedness of $\tilde v$ implies that $\tilde v \in L^1_{loc}(M)$. We are next going to show that the measure $d\tilde v$ has bounded variation.
\begin{proof} 
  Since the problem is local we will work locally in $M$ instead of $\tilde M$ and consider $v$ instead of $\tilde v$. Let
   $F=F(t,s): R^\sharp=[a,b] \times [c,d]  \rightarrow  R \subset M$ be
 a flow box as in (\ref{eqn:flowbox}), set $f_t(s) = F(t,s)$ and consider the fixed transversal $f = f_b$. By definition,  $f_b$ is  positively oriented with respect to the oriented lamination $\lambda$ and since $\nu$ is non-negative by assumption, the function 
\begin{eqnarray}\label{gsharp}
g^\sharp(s) &: =&\int_c^sf^*( d\nu)
 =\int_{\{b\} \times [c,s]} d\nu \nonumber \\
& =& \int_c^s d\nu \ \ \mbox{(by a slight abuse of notation)} 
\end{eqnarray}
is non-decreasing. Furthermore,
\begin{equation}\label{formvsharp}
v^\sharp(t,s)=v^\sharp(t,c)+g^\sharp(s).
\end{equation}
In order to show (\ref{formvsharp}), assume that $F(t,c)$ is in the plaque $S_0$ and  $v^\sharp(t,c)=v^\sharp(b,c)=a_0$ and $F(t,s)$ is in the plaque $S$ and  $v^\sharp(t,s)=v^\sharp(b,s)=a$. Since the transversal $f=f_b$ is positively oriented with respect to the lamination, we have by Definition~\ref{MDefinition6} that $g^\sharp(s)=\beta(S,S_0)=a-a_0$. Hence  (\ref{formvsharp}) follows.

Since the measure $\nu$ is positive, the function $g^\sharp$ is monotone and hence of bounded variation. Formula ~(\ref{formvsharp}) then implies that $v^\sharp$ is of bounded variation. Since $\tilde v=v^\sharp \circ F^{-1}$, \cite[Theorem 3.16]{ambrosio} implies that $\tilde v$ is locally of bounded variation with and  $|d \tilde v| \leq F_*|dv^\sharp|$ locally.
\end{proof}
%---------------------------------------------------
%
%{\bf{NOTE: CUT OUT THE REST- SOME OF IT WILL BE ADDED IN SECTION 8 Theorem~\ref{dense}}}
%
%
%---------------------------------------------------------
%
%
%
%
%
% Let $K \subset [c,d]$ be a closed set of Hausdorff dimension 0 such that $F^{-1}(\lambda)=[a,b] \times K$.
% We will start by considering the case when the set $K$ is finite. This is the case globally when the lamination consists only of closed leaves.  However, locally we can always approximate $K$ by a discrete set of points as it will be explained below. 
%  
%  Given $K$ finite and
% $k \in K$,  there exist $s < k<s'$ such that 
%\[
%g^\sharp(s') - g^\sharp(s) := jump_k \geq 0,
%\]
%thus the measure $\nu$ is the delta function equal to $jump_k$  supported at the point $k$.
%
%
%
%
%%{\bf{Change 6: It seems the notation is inconsistent with the next proposition. $I_k$ is an open interval around k not with end points k. Should I change this????}}
%\begin{proposition}\label{discrleav2}  If for a flow box, $K=\{k_1,...,k_n \}$ is finite,
%the total variation of the measure of $dv$, satisfies 
%\[
% ||dv||_R =(b-a)\sum_j jump_{ k_j}=(b-a)\nu(f).
%\]
%\end{proposition}
%\begin{proof} We write $R = \cup \overline {R_j}$ for finitely many rectangles where  $R_j$ is bound above and below by the geodesics $\lambda_j$ and $\lambda_{j-1}$ and by the transversals coresponding to $t=a$ and $t=b$ from left and right. By Stokes' theorem and the fact that $\phi $ vanishes on a neighborhood of the end transversals
%\[
%\int_{R_j} v_n d\phi =
%\int_a^b g^\sharp(k_j) \phi^\sharp \frac{d\lambda_j}{dt} dt - \int_a^bg^\sharp(k_{j-1})\phi^\sharp \frac{d\lambda_{j-1}}{dt}dt.
%\]
%Adding together, collecting the terms which integrate over the same leaf together, we obtain 
%\begin{equation}\label{sumjumps}
%\int_{R} v_n d\phi =\sum_j \int_a^b jump_{k_j}\phi^\sharp \frac{d\lambda_j}{dt} dt.
%\end{equation}
%Noting that $\left | \frac{d\lambda_j}{dt}\right | =\left | \frac{d\lambda_j}{dt} \big |_{t=0}\right | =|n(k_j)|=1$,
%we get
%\begin{eqnarray}\label{sumjumps2}
% \left |\sum_j \int_a^b jump_{k_j}\phi^\sharp \frac{d\lambda_j}{dt} dt \right | \nonumber
% &\leq& (b-a) \max |\phi| \sum_j |jump_{ k_j}| \nonumber \\
% &=& (b-a) \max |\phi| \sum_j jump_{ k_j} \nonumber \\
%&=&  (b-a) max|\phi| \sum_j \int_{k_{j-1}}^{k_j} d\nu \\
% &=&  (b-a) max|\phi| \nu(f) \nonumber
%\end{eqnarray}
%which proves an upper bound for the total variation. Notice that in the second equality we used the fact that $\nu$ is non-negative. By taking $\phi$ so that $\phi^\sharp \frac{d\lambda_j}{dt}$ approximates 1, we obtain equality.
% \end{proof}
%
%We now return to the case of arbitrary $K$, also fix $\phi \in C^\infty(\tilde M)$ with support contained in the interior of $R$. 
%%As before, it will be convenient in the computations to write  $ v\circ F= v^\sharp$ and similarly for any other function defined on a subset of $R$. 
%We write $[c,d]\backslash K =\bigcup I_m$, for open intervals $I_m =(k_m, k_{m+1})$ and let $R_m=F([a,b] \times I_m)$. 
%%be the rectangular shaped region bordered by the two end transversals and the geodesics $\gamma_j$ and $\gamma_{j+1}$ which pass through $k_j$ and $k_{j+1}$. 
%By assumption, $v \equiv a_m$ is constant on $R_m$ corresponding to an open plaque.
%Associate each $R_m$ with its left endpoint $k_m$  and let $k_0 = c$ and $k_1 = d$. Otherwise we have given them an arbitrary ordering. We are going to approximate $v$ with $v_n$, such that $v_n$ takes only finitely many values. We order the set $\{k_m: m \leq n \}$ by $k_{n,j} < k_{n, j+1}$ 
%and define
%\begin{equation}\label{vnsharp2} 
%v^\sharp_n(t,s) = v^\sharp(t,c)+g^\sharp(k_{n,j}) ; \ \  k_{n,j}<s< k_{n, j+1}.
%\end{equation}
% so that
%  $v_n $ is constant on rectangles that are suitable combinations of the $R_m$, $m\leq n$. The jumps at the geodesics  indexed by $k_{n,j}$ are just $g^\sharp(k_{n, j+1}) - g^\sharp(k_{n,j})$.
%
%Next we are going to show,
%\begin{equation} \label{convergebv2}
%\lim_{ m \rightarrow \infty} v_m = v
%\end{equation} 
%weakly in BV and pointwise in $R$.
%First note that for any $n$,
%\begin{equation}\label{vnsharp3}
%\sum_j jump_{ k_{n,j}}=\sum_j \int_{k_{j-1}}^{k_j} d\nu= \int_c^d d\nu=\nu(f), 
%\end{equation}
%hence Proposition~\ref{discrleav2} and the fact $v_n$ are locally bounded in $R$, implies  a uniform estimate on the BV norm of $v_n$. Also,
%\begin{equation*}
%v_n(x) = v(x) \ \ \mbox{on}\ R_m  \ \ \mbox{for} \ \ m \leq n.
%\end{equation*}
% That is, once we include the right endpoint in the list, $v_n = v$ on the rectangle to the left. So any weak limit in BV of $v_n$ must be $v$. This proves $v$ is a BV function.
%\end{proof}
%

% \begin{eqnarray*}
% \int_{\tilde M}\tilde v d\phi &=& \sum_j a_j\int_{ \Omega \cap S_j} d\phi\\
% &=&\sum_j a_j\int_{ \partial (\Omega \cap S_j)} \phi\\
% &=&\sum_j a_j \int_{  \Omega \cap \partial  S_j} \phi\\
% &=& C |\phi|_{L^\infty} ({\mathcal H}^1 \partial \Omega) +{\mathcal H}^1(\lambda \cap \Omega)) \\
% \end{eqnarray*}
% where $C$ depends on an upper bound for $|a_j|$ and ${\mathcal H}^1$ denotes 1-dimensional Hausdorff measure. The last is bounded by $|\phi|_{L^\infty}$ since a lamination has a finite 1--dimensional Hausdorff measure (cf. \cite[Section 10]{thurston} and \cite{birman}).

%\subsection{The non-orientable case} In this paper we are mainly concerned about oriented laminations because our geodesic lamination of maximum stretch $\lambda_u$ is naturally oriented. In this section we will explain how a function of bounded variation $\tilde v: \tilde M \rightarrow \R$ satisfying properties $(i)$-$(iii)$ as in Section~\ref{tranco} defines a transverse measure on a non necessarily oriented geodesic lamination $\lambda$ in $M$.
%
%First choose a neighborhood $U$ of $\lambda$ and  a double cover $\hat p: \hat U \rightarrow U$ such that the preimage $\hat \lambda$ of $\lambda$ is oriented. Indeed, the topological obstruction of trivializing a real line bundle can be lifted by going to an appropriate double cover. Note that this double cover doesn't necessarily come from a cover of the entire manifold if for example the plaques consist of an odd number of cusps.  Let $\tilde U \subset \tilde M$ denote the preimage of $U$ on the universal cover of $M$ and 
%$\hat {\tilde U} \simeq {\hat p}^*(\tilde U) \rightarrow \hat U$ the pullback of the fibration $\tilde U \rightarrow U$ under $\hat p$.
%\begin{equation*}
%\begin{tikzcd}
%  \hat {\tilde U} \simeq {\hat p}^*(\tilde U) \arrow[r, "{{\hat p}^*}"] \arrow[d]
%    &  \tilde U \arrow[d ] \\
%  \hat U  \arrow[r, "{\hat p}" ]
%& U \end{tikzcd}
%\end{equation*}

\section{From transverse measures to functions of bounded variation}\label{ruelles}
In the previous sections we showed that, given an oriented geodesic lamination $\lambda$ in a hyperbolic surface $M$ and  a locally bounded function $v$, which  is  constant on the plaques of $M_0=M \backslash \lambda$, we can construct a transverse cocycle $\nu$. Moreover, if $\nu$ is non-negative, then $\nu$ is a transverse measure and this forces $v$ to be of bounded variation. In this section we will start with a transverse measure $\nu$ on $\lambda$ and we will construct  $v$ as a primitive of BV to the Ruelle-Sullivan current. We continue to assume throughout the section that $M$ is a closed hyperbolic surface.


 \subsection{The Ruelle-Sullivan current} 
 
 In 1975 Ruelle-Sullivan \cite{sullivan} constructed a current for a transverse measure on a partial foliation. The next construction follows theirs (with less regularity for $F$), but we repeat it for completeness.

 \begin{definition}\label{integration} Let $\Lambda=(\lambda, \nu)$ be an oriented measured geodesic lamination and $F_i: R_i^\sharp=[a_i,b_i] \times [c_i,d_i] \rightarrow R_i= F_i(R_i^\sharp) \subset M$ be flow boxes as in Definition~\ref{flowbox} covering a neighborhood $\mathcal U$ of $\lambda$.
 Define an 1-current
 $T_\Lambda$  by setting
 \[
 T_\Lambda(\phi)=\sum_i \int_{(c_i,d_i)}\left(\int_{[a_i,b_i] \times \{ s\}}F_i^*(\phi_i) \right)d\nu(s); \ \ \phi=\sum_i \phi_i
 \]
 where $\phi \in \mathcal D^1(\mathcal U)$ and $\phi_i \in \mathcal D^1(R_i)$.
 \end{definition}
 
 \begin{theorem}\label{welldefcur} $T_\Lambda$ is a well defined 1-current. Furthermore, $T_{\Lambda} $ is closed and thus defines an element
 \[
 [T_{\Lambda}] \in H_1(M, \R).
 \]
 \end{theorem}
 \begin{proof}
 First, note that because $\frac{\partial F_i}{\partial t}$ is continuous, 
 \[
 s \mapsto \int_{[a_i,b_i] \times \{ s\}}F_i^*(\phi)= \int_{a_i}^{b_i} ({F_i}_s)^*\phi
 \]
 is a continuous function in $s$, so
 we can integrate against a Radon measure. To show it is independent of the choice of flow box, first consider the case where $ \phi$ is compactly supported in the intersection of two flow boxes $F$ and $F'$.
 Then,
 \begin{eqnarray*}
 \int_c^d\int_a^bF_s^*(\phi)d\nu(s)&=&\int_c^d\int_a^b({F'}^{-1}F)_s^*{F'}_s^*(\phi)d\nu(s)\\
 &=&\int_{a'}^{b'}\int_{c'}^{d'}{F'}_s^*(\phi)({F}^{-1}F')_s^*(d\nu(s))\\
 &=&\int_{a'}^{b'}\int_{c'}^{d'}{F'}_s^*(\phi)d\nu(s),\\
 \end{eqnarray*}
 the last equality because the transverse measure is invariant under the transition functions ${F}^{-1}F'$.
 We can reduce the general case to this case, as follows: Consider two atlases consisting of flow boxes $\{F_i\}$ and $\{F'_{i'}\}$ and let  $\{\xi_i\}$ and $\{\xi_{i'}\}$ be partitions of unity subordinate to the above covers. We can write 
 \[
 \phi=\sum_{i,i'}\xi_i\xi_{i'} \phi.
 \]
By the previous case,
 \[
 \int_c^d\int_a^bF_s^*(\xi_i\xi_{i'}\phi)d\nu(s)=\int_{a'}^{b'}\int_{c'}^{d'}{F'}_s^*(\xi_i\xi_{i'}\phi)d\nu(s),
 \]
 thus by summing over $i,i'$ we obtain the desired equality.
 To show it is closed note that if $f \in \mathcal D^0(R_i)$ is supported in one flow box, 
 \[
 \int_{[a_i,b_i] \times \{ s\}}F_i^*(df)=\int_{a_i}^{b_i} \frac{\partial}{\partial t} (f \circ F_i)(t,s)dt=0
 \]
 hence
 \[
 T_\Lambda (df)=0.
 \] 
 \end{proof}
 
% \begin{theorem} \label{equalcurrents} If $\Lambda_u= (\lambda_u, \nu)$ is the oriented measured lamination of Theorem~\ref{transmeasure}, then as currents
%\[
%T_{\Lambda_u}=dv.
%\]
%\end{theorem}
%\begin{proof} {\bf{To be added from the previous version}}.
%\end{proof}
 




\subsection{Constructing a primitive of the Ruelle-Sullivan current} 
In Theorem~\ref{transcoismeasthm}, we showed that a cocycle which defines a transverse measure is associated to a function of bounded variation.  For example, if the cocycle is non-negative, then by a theorem of Bonahon it is a transverse measure.  We now show directly that the cocycle of Bonahon and the corresponding function of bounded variation can be constructed directly from the Ruelle-Sullivan current.

\begin{theorem}\label{conversetomeasure} Given an oriented measured geodesic  lamination $\Lambda=(\lambda, \nu)$, there exists a flat real affine rank 1 bundle  $L$ and a section $v: M \rightarrow L$ of bounded variation  such that
\begin{equation}\label{T=dv}
T_\Lambda=dv.
\end{equation}
\end{theorem}

\begin{proof}

Let $F=F(t,s): R^\sharp=[a,b] \times [c,d]  \rightarrow  R \subset M$
be a flow-box as in (\ref{eqn:flowbox}). 
First we are going to consider the local problem 
and construct $v=v_F$ on the image  of $F$.
Let $f_t(s) = F(t,s)$ and consider the fixed transversal $f = f_b$.  
Consider the non-decreasing function $g^\sharp(s)$ defined as in (\ref{gsharp})
by
$g^\sharp(s) =\int_c^sf^*( d\nu)=
 \int_c^s d\nu $
%We also parameterize the $t$ variable so that $|\frac{d \lambda_0}{dt} | = 1$ for all leaves $\lambda_0$ in $\lambda$ which intersect $R$.
%Let 
%\[
%K = \{k \in [c,d]: f(k) \in \lambda \}.
%\]
% $K$ is a closed set of measure 0 and Haussdorff dimension 0, and $g^\sharp$ is constant on the connected components of $[c,d] \backslash K$. Denote the leaf of $\lambda$ which passes through $k \in K$ by $\lambda_k$.
%Throughout this section the maps $F$ and $f_t $ identify functions on $[a,b] \times [c,d]$ or a subset with functions on $R$. As in the previous sections, we use $g^\sharp = f^*g$, $v^\sharp = F^*v$
%and so forth consistently. Remember, however, that $F$ is only Lipschitz although we may assume $f $ is smooth.
%Let
and let
\begin{equation}\label{vsharp}
v^\sharp(t,s) = g^\sharp(s). 
\end{equation}
We define
$v_F:=v^\sharp \circ F^{-1}$ in the image of the flow box $F$ in terms of $g^\sharp$ as above. 
Note that by the invariance of $\nu$ under homotopies, $v_F$
is constant on the plaques in the image of $F$. Also
 $v_F$ is  bounded. Furthermore, $v^\sharp$ is of bounded variation, because $g^\sharp$ is monotone and  
 as in the proof of Theorem~\ref{transcoismeasthm}, we can conclude that $ v_F$ is of bounded variation.
 
 Note that for compactly supported $\phi^\sharp=\phi_1dt+\phi_2ds \in \mathcal D^1(R)$,  
 \begin{eqnarray*}\label{dvsharp1}
  \lefteqn{\int_{(c,d)}\left(\int_{(a,b) \times \{ s\}}\phi^\sharp \right)d\nu(s)} \nonumber\\
 &=&- \int_{(c,d)}\frac{d}{ds}\left(\int_{(a,b) \times \{ s\}}\phi^\sharp \right) g^\sharp ds  \nonumber\\
 &=& -\int_c^d\left(\int_a^b \frac{\partial}{\partial s}\phi_1(t,s)dt \right) g^\sharp(s) ds \nonumber\\
 &=& -\int_c^d\left(\int_a^b \frac{\partial}{\partial s}\phi_1(t,s)dt- \frac{\partial}{\partial t}\phi_2(t,s)dt \right) g^\sharp(s) ds \\
 &=& -\int_a^b \int_c^d \left( \frac{\partial}{\partial s}\phi_1(t,s)- \frac{\partial}{\partial t}\phi_2(t,s) \right)  v^\sharp(t,s)dsdt \nonumber\\
 &=& -\int \int_{(a,b) \times (c,d)} v^\sharp d \phi^\sharp  \nonumber\\
 &=& dv^\sharp(\phi^\sharp). \nonumber
  \end{eqnarray*}
 In the forth equality we used that $\phi_2$ is compactly supported.
 By Definition~\ref{integration}, this implies that in the interior of a flow box $F$ (\ref{T=dv}) holds.
 % ---------------------------------------------------
%
%{\bf{NOTE: CUT OUT THE REST- REPLACE WITH Theorem~\ref{dense}}}
%
%
%---------------------------------------------------------
% 
% 
%Parts of the proof are very similar to Theorem~\ref{transcoismeasthm}, so we will only sketch.
%  Let $K \subset [c,d]$ be as before  
%  and we approximate locally the lamination by laminations consisting of finitely many leaves corresponding to finite subsets  of $K$. 
%%  such that $F^{-1}(\lambda)=[a,b] \times K$. We will start by considering the case when the set $K$ is finite. This is the case globally when the lamination consists only of closed leaves.  However, locally we can always approximate $K$ by a discrete set of points and below we will study this case. 
%%  
%%  Given
%% $k \in K$,  there exist $s < k<s'$ such that 
%%\[
%%g^\sharp(s') - g^\sharp(s) := jump_k \geq 0,
%%\]
%%thus the measure is the delta function equal to $jump_k$  supported at the point $k$.
%%
%%
%%
%%
%%%{\bf{Change 6: It seems the notation is inconsistent with the next proposition. $I_k$ is an open interval around k not with end points k. Should I change this????}}
%%\begin{proposition}\label{jump}  If for a flow box, $K=\{k_1,...,k_n \}$ is finite,
%%then $v$ is a jump function on $R$.  Moreover,
%%\[  
%%||dv||_R = \sum k_j jump_{k_j} = (b-a)\nu(f).
%%\]
%%\end{proposition}
%%\begin{proof} We write $c=k_0< k_1 <... < k_{n+1} =d$. The function $g^\sharp=a_j$ is constant in the open intervals $I_j =(k_j, k_{j+1})$, $v^\sharp =a_j$ on $[a,b] \times I_j$ so
%% $v = a_j$ constant on the rectangular shaped region $R_j=F([a,b] \times I_j)$ bordered by the two end transversals and the geodesics $\lambda_j$ and $\lambda_{j+1}$ which pass through $k_j$ and $k_{j+1}$. Such a function is a jump function.
%% Finally, the formula on the total   variation of $dv$ follows from (\ref{sumjumps}) and (\ref{sumjumps2}).
%% \end{proof}
%%% If the set $K$ is finite,  we obtain as in Claim~\ref{discrleav2}, the uniform estimate
%%% \begin{equation*}
%%% \left |\int_R v d \phi \right | \leq C(b-a)\nu(f) \max |\phi|
%%%\end{equation*}
%%%where the constant $C>0$ depends only on $F$.
%%
%%In the case of a general lamination, we approximate $(\lambda, \nu)$ in the flow box $F: R^\sharp=[a,b] \times [c,d] \rightarrow R \subset M$ as in Theorem~\ref{transcoismeasthm} by  laminations $(\lambda_m, \nu_m)$ consisting of finitely many leaves. 
%More precisely,  we write $ K$ as a nested union of finite sets $K_n$ of  points $k_{n,j} \in [c,d]$,  ordered as 
% $k_{n,j}< k_{n,j+1}$, where  $k_{n,j}$ corresponds  to geodesics $\lambda_{n,j}$  and the measure $\nu_n$ is a delta function equal to 
% \[
% jump_{k_{n,j}}=g^\sharp(s')-g^\sharp(s); \ k_{n,j-1}<s<k_{n,j}<s'< k_{n,j+1}
% \] 
% supported in $K_n$. Let $v_n$ be the  function of bounded variation corresponding to $\nu_n$, i.e
% \begin{equation}\label{misseqn}
% v_n^\sharp(t,s)=\int_c^sd\nu_n=\sum_{k_{n,j}\leq s} jump_{k_{n,j}}.
% \end{equation}
%As in (\ref{convergebv2}),
%$\lim_{ n \rightarrow \infty} v_n = v$ weakly in BV and pointwise.
%This proves that $v$ is a BV function. 
% 
% We now turn to the proof on (\ref{T=dv}) on the image of the flow box $F$.
% % \begin{claim}\label{claim:bdvar} $v$ is of bounded variation.
%% \end{claim}
%%  We will start first by considering the case of closed leaves.  
%%If $k$ in $K$ with $f(k)$ on the closed geodesic $\lambda_k$, then there exist $s < k<s'$ such that 
%%\[
%%g^\sharp(s') - g^\sharp(s) = jump_k \geq 0.
%%\]
%%%{\bf{Change 6: It seems the notation is inconsistent with the next proposition. $I_k$ is an open interval around k not with end points k. Should I change this????}}
%%\begin{corollary}\label{discrleav} If $\nu$ is a transverse measure and $F$ a flow box which intersects a foliation $\lambda$ in a discrete set of leaves as above, then for every one-form $\phi$ with compact support in $R$, we have
%%\[
%% \left |\int_R v d \phi \right | \leq (b-a)\nu(f) \max |\phi|.
%%\]
%%\end{corollary}
%%\begin{proof} We write $R = \cup \overline {R_j}$ where $R_j$ as in the proof of Proposition~\ref{jump} and  the integral over $R$ is just the sum of the integrals over  $R_j$. But by Stokes theorem and the fact that $\phi $ vanishes on a neighborhood of the end transversals
%%\[
%%\int_{R_j} v d\phi =
%%\int_a^b \kappa_j \phi \frac{d\lambda_j}{dt} dt - \int_a^b\kappa_{j-1}\phi \frac{d\lambda_{j-1}}{dt}dt.
%%\]
%%Adding together, collecting the terms which integrate over the same leaf together, and noting that $\left | \frac{d\lambda_j}{dt}\right | \leq \max \left| \frac{dF}{dt}\circ F^{-1} \right|$,
%%we get
%%\begin{eqnarray*}
%% \left |\sum_j \int jump_{k_j}\phi\frac{d\lambda_j}{dt} dt \right | 
%% &\leq& (b-a) \max|\phi| \sum_j |jump k_j| \\ 
%% &=& (b-a) max|\phi| \nu(f).
%%\end{eqnarray*}
%%\end{proof}
%%\begin{rick}
%%We now consider the case where $\lambda_k$ does not come from a closed geodesic. We may discard isolated leaves, as they do not carry any measure.  We will  construct an approximating sequence $v_n$  to $v$ in $R$ which is a jump function. We will also assume that 
%%$R \cap \lambda$ contains no closed leaves. 
%%
%%If $K$ is finite, then $g$ is constant. Suppose $K$ is infinite. Let 
%%$R \backslash \lambda = \cup R_m$ where each $R_m$ is an open rectangle as
%%described in the proof of Proposition~\ref{jump}. Associate each $R_m$ with its left endpoint $k_m$ and let $k_0 = c$ and $k_1 = d$. Otherwise we have given them an arbitrary ordering. To construct $v_n$, we order the set $\{k_m: m \leq n \}$ by $k_{n,j} < k_{n, j+1}$. With $g^\sharp$ as in (\ref{gsharp}), we define
%%\begin{equation}\label{vnsharp} 
%%v^\sharp_n(t,s) = g^\sharp_n(s) = g^\sharp(k_{n,j}); \ \  k_{n,j}<s< k_{n, j+1},
%%\end{equation}
%% so that $v $ is constant on rectangles that are suitable combinations of the $R_m$, $m\leq n$. The jumps at the geodesics indexed by $k_{n,j}$ are just $g^\sharp(k_{n, j+1}) - g^\sharp(k_{n,j})$, which are well-defined because $g^\sharp$ is continuous.
%% 
%% We next claim,
%%\begin{equation} \label{convergebv}
%%\lim_{ m \rightarrow \infty} v_m = v
%%\end{equation} 
%%weakly in BV and pointwise.
%%To prove~\ref{convergebv} notice from Corollary~\ref{discrleav} that we have a uniform estimate on the BV norm of $v_n$. Note that for all $m \geq n$, $v_n(x) = v(x)$ on $R_j$ for $ j \leq n$. That is, once we include the right endpoint in the list, $v_n = v$ on the rectangle to the left. So any weak limit in BV of $v_n$ must be $v$. This proves $v$ is a BV function.
%% \end{rick}
%% 
%% We approximate again the measured lamination $(\lambda, \nu)$ by laminations $(\lambda_n, \nu_n)$ intersecting the image of the flow-box in a finite number of leaves corresponding to points  $k_{n,j} \in [c,d]$   and the measure $\nu_n$ is a delta function equal to $jump_{k_{n,j}}$ supported in the finite set of  points $k_{n,j}$. Let $v_n$ be the  function of bounded variation corresponding to $\nu_n$, i.e
%% \begin{equation}\label{misseqn}
%% v_n^\sharp(t,s)=\int_c^sd\nu_n=\sum_{k_{n,j}\leq s} jump_{k_{n,j}}.
%% \end{equation}
%% For $\phi \in \mathcal D^1(R)$, by (\ref{sumjumps}), 
%%  \begin{eqnarray*}
%% \int_{R} v_n d\phi &=&\sum_j \int_a^b jump_{k_{n,j}}\phi^\sharp \frac{d\lambda_{n,j}}{dt} dt \\
%% &=&\int_c^d\left(\int_{[a,b] \times \{ s\}}\phi^\sharp \right)d\nu_n(s)\\
%% &=&T_{\Lambda_n}(\phi).  
%%  \end{eqnarray*}
%  We next claim that 
%  \begin{equation}\label{squi}
%  \nu_n  \rightharpoonup \nu.
% \end{equation} 
%  Indeed, (\ref{gsharp}), (\ref{vsharp}), (\ref{misseqn})  and the convergence of $v_n$ to $v$ imply that for $s<d$,
%  \begin{eqnarray}\label{proh}
%  \lim_n\int_c^sd\nu_n= \lim_nv_n^\sharp(b,s)=v^\sharp(b,s)=\int_c^sd\nu.
%  \end{eqnarray}
%  Equation~(\ref{proh}) implies that if $I=(s,s']$, for $c<s<s'\leq d$, then 
%   \begin{eqnarray}\label{proh2}
%   \nu_n(I) \rightarrow \nu(I).
%  \end{eqnarray}
%  Using the formula $\nu(I\cup I')=\nu(I)+\nu( I')-\nu(I\cap I')$, and noting that $I\cap I'$ is either empty or of the form $(s,s']$, we have that $(\ref{proh2})$ holds also for finite unions of sets of the form $I=(s,s']$.
%  For an open set $U$, let $U=\cup_j I_j$. By continuity of measure, for each $\epsilon > 0$,
%  there exists $j_0$ such that $\nu(U) \leq \nu(\cup_{i=1}^{j_0} I_j)+\epsilon$. 
%  Since
%  \[
%  \nu(U)-\epsilon \leq \nu(\cup_{i=1}^{j_0} I_j)=\lim_n  \nu_n(\cup_{i=1}^{j_0} I_j)\leq \liminf_n  \nu_n(U)
%  \]
%  and $\epsilon$ was arbitrary, we get 
%  \[
%  \nu(U) \leq  \liminf_n  \nu_n(U)
%  \] 
%  for any open set $U$. 
%  By \cite[Theorem 1(ii), page 54]{evans}, this implies (\ref{squi}),
%  which in turn implies $dv_n=T_{\Lambda_n}  \rightharpoonup T_{\Lambda}$. Since also  $dv_n  \rightharpoonup dv$, the proof is complete.
% % It suffices to show that both currents agree on functions supported in the interior of  flow box $\{F: R=[a,b]\times [c,d] \rightarrow M\}$ where
%%$F(u,t)=(u, F_u(t))$. Let $\psi=f_1 du +f_2 *du$ be such a form.
%%First notice that, by Corollary~\ref{currentsupportfiber}, we have  $dv(f_2 *du)=f_2*du \wedge dv=0$ and similarly by Definition~\ref{integration},
%% \begin{eqnarray*}
%% T_{\Lambda_u}(f_2 *du)&=& \int_{(c,d)}\left(\int_{[a,b] \times \{ t\}}F^*(f_2) *du \right)dm(t)\\
%% &=& 0
%%  \end{eqnarray*}
%%  because $*du \left(\frac{\partial}{\partial u}\right)=0$.
%%Now,
%%  \begin{eqnarray*}
%% T_{\Lambda_u}(f_1 du)&=& \int_{(c,d)}\left(\int_{[a,b] \times \{ t\}} F^*(f_1) du \right)dm(t)\\
%% &=& \int_{F(R)}f_1 du \wedge dv  \ \mbox{(by Proposition~\ref{formdudm})}\\
%% &=& dv(f_1 du).
%%  \end{eqnarray*}
%%  This proves the theorem
%  
 If $F: R^\sharp \rightarrow M$ and $F': {R'}^\sharp \rightarrow M$ are two flow boxes  which intersect in a non-empty connected set which contains a ball, then on the overlap
 \begin{equation}\label{globaltheory}
 dv_F = dv_F'=T_\Lambda
 \end{equation}
 hence 
$v_F = v_F' + c.$


 
% \begin{corollary}\label{globaltheory} If $F: R^\sharp \rightarrow M$ and $F': {R'}^\sharp \rightarrow M$ are two flow boxes of an oriented geodesic lamination which intersect in a non-empty connected set which contains a ball, then on the overlap
%\[ 
%v_F = v_F' + c.
%\]
%\end{corollary}
%\begin{proof} Immediate, since on the overlap 
%\[
%dv_F = dv_F'=T_\Lambda.
%\]
%\end{proof}
%\begin{proof} {\bf Number 2: (Karen's proof) Can we skip?}Assume the flow boxes are oriented. In the general case if there is no prescribed orientation, we may have no orientation, and some flow boxes may need to be oppositely oriented. This is the source of the $\pm$. Also, if two flow boxes overlap, they contain a flow box in the intersection, so it is enough to assume  that the image of $F'$ in contained in the image of $F$.
%Let $R \cap  M_0 = R_0$. Assume that $F(0,b) \in R_0$, $F(0,b') \in R_1$. Then $u_F(x) = \beta(R_x) - \beta(R_0)$ and 
%$u_{F'}x =\beta(R_x) - \beta(R_1)$ for $x \in R_x$. Hence they differ by $\beta(R_1) - \beta(R_0)$.
%\end{proof}
 
% \begin{remark} In the case when the lamination is not oriented we have
%\[ 
%v_F = \pm v_F' + c.
%\]
%\end{remark}
%By linear Hodge theory applied to the closed current $T_\Lambda$, we have
%\[
%T_\Lambda = A + df.
%\]
%Here $f$ is a real valued distribution and alpha is a harmonic 1-form on $M$. By (\ref{globaltheory}), locally, and up to a constant,  $f=v_F$. Hence, $f$ is a locally bounded BV function on $M$.   We let $L$ be the line bundle associated to $A$, so that $A = dw$, with $w: M \rightarrow L$.  Then $T = d(w + f)$ globally, where $w + f$ is a section of $L$.
We now proceed with constructing a flat affine  line bundle $L$ over the surface $M$ and a global section $v: M \rightarrow L$ formed by piecing together the local primitives of the Ruelle-Sullivan current $T=T_\Lambda$.
 
Choose a smoothing $T^\epsilon$ of $T$,
 $d (T^\epsilon)=(d T)^\epsilon=0$
and let
 $\tilde T^\epsilon =\sigma^*(T^\epsilon)$ be the pullback to the universal cover. Let  $\tilde v^\epsilon$ be a primitive of  $\tilde T^\epsilon $ equivariant under representations
 $\alpha^\epsilon: \pi_1(M) \rightarrow \R.$
 By the Poincare Lemma, we can write
 $\tilde T^\epsilon=   d\tilde v^\epsilon$,
 where $v^\epsilon$ is a smooth real valued function equivariant under the representation 
 $\alpha^\epsilon$. As is Section~\ref{qgoesto1},
 $\alpha^\epsilon \rightarrow \alpha$ and
 the convergence as distributions $d\tilde v^\epsilon=\tilde T^\epsilon \rightarrow T$ and weak compactness in the space BV, implies that up to a constant $\tilde v^\epsilon \rightharpoonup \tilde v$  in $BV_{loc}(\tilde M)$ where $\tilde v$ is equivariant under $\alpha$.
  If $L$ denotes the flat affine line bundle associated to 
 $\alpha$, then $v$ is a section of $L$ of bounded variation.
\end{proof}

\begin{corollary}Let $\nu$ denote the measure on the lamination $\lambda_u$ constructed in Theorem~\ref{transmeasure} associated to the least gradient map $v$. If $\Lambda_u=(\lambda_u, \nu)$, then the Ruelle-Sullivan current $T_{\Lambda_u}=dv$.
\end{corollary}

\begin{proof}The measure $\nu$ is related with the least gradient map $v$ by formulas (\ref{gsharp}) and (\ref{formvsharp}). The Ruelle-Sullivan current associated to $\nu$ is given by the derivative of a new function $v$ given by (\ref{vsharp}) which only differs from (\ref{formvsharp}) by a constant. Thus $T_{\Lambda_u}=dv$. 
\end{proof}

%\begin{theorem}\label{dense}Let $\Lambda=(\lambda, \nu)$ be an oriented geodesic lamination with Ruelle-Sullivan curent
%$T_\Lambda=dv$. Let $F: R^\sharp \rightarrow R \subset M$ be a flow box. Then, there exists a sequence of geodesic laminations $\Lambda_n=(\lambda_n, \nu_n)$ defined on $R$ such that
%\begin{itemize}
%\item $\lambda_n$ consists of closed leaves.
%\item $\nu_n  \rightharpoonup \nu.$
%\item If $v_n$ denotes the primitive of the Ruelle-Sullivan current $T_{\Lambda_n}$ and $v$ denotes the primitive of the Ruelle-Sullivan current $T_{\Lambda}$, then $\lim_{ n \rightarrow \infty} v_n = v$ weakly in BV and pointwise.
%\end{itemize}
%\end{theorem}

%where $\omega_\gamma$ is the Poincare dual of $\gamma$.
%
%\begin{linebundle}
% We now proceed with constructing a flat line bundle $L$ over the surface $M$ and a global section $v: M \rightarrow L$ formed by piecing together the local primitives of the Ruelle-Sullivan current $T=T_\Lambda$.
% 
% First, choose a smoothing $T^\epsilon$ of $T$. Since $dT=0$, 
% \[
% d (T^\epsilon)=(d T)^\epsilon=0.
% \]
% Let $\tilde T^\epsilon =\sigma^*(T^\epsilon)$ be the pullback of $T^\epsilon$ on the universal cover and, by the ordinary Poincare Lemma, let   $\tilde v^\epsilon$ be a primitive of  $\tilde T^\epsilon $ equivariant under the representation
% \[
% \alpha^\epsilon: \pi_1(M) \rightarrow \R,
% \]
% i.e.
% \begin{equation}\label{epsequiv}
% \tilde  v^\epsilon(\gamma \tilde x)=\tilde  v^\epsilon( \tilde x)+ \alpha^\epsilon(\gamma); \ \ \forall \tilde x \in \tilde M, \  \forall \gamma \in \pi_1(M).
% \end{equation}
% This is defined as in (\ref{alphagamma}). 
% Given $\gamma \in  \pi_1(M)$ and $\omega_\gamma$  the Poincare dual of 
% $\gamma$,  
% \[
% \alpha^\epsilon(\gamma)=\int_M \omega_\gamma \wedge T^\epsilon.
% \]
% The weak convergence $T^\epsilon \rightharpoonup T$ implies
% that
% \[
% \alpha^\epsilon(\gamma)=T^\epsilon(\omega_\gamma) \rightarrow T(\omega_\gamma)=:\alpha(\gamma).
% \] 
% It follows that the representations  
% $\alpha^\epsilon \rightarrow \alpha$. 
% 
% The convergence of distributions $d\tilde v^\epsilon=\tilde T^\epsilon \rightharpoonup T$ and weak compactness in the space BV, implies that up to a constant $\tilde v^\epsilon \rightharpoonup \tilde v$  in $BV_{loc}(\tilde M)$.
% In particular,
% \[
% \tilde v^\epsilon \rightarrow \tilde v \ \ \mbox{in} \ \ L^1_{loc}(\tilde M)  \ \ \mbox{and a.e},
% \]
% thus, by (\ref{epsequiv})
% \begin{equation}\label{epsequiv2}
% \tilde v(\gamma \tilde x)= \tilde  v( \tilde x)+ \alpha(\gamma); \ \ \mbox{a.e} \ x \in \tilde M, \  \forall \gamma \in \pi_1(M).
% \end{equation} 
%  Hence, if $L$ denotes the flat line bundle associated to 
% $\alpha$, then $v$ is a section of $L$ of bounded variation.
%\end{linebundle}

\subsection{The decomposition in terms of functions of bounded variation}
In the previous sections we showed how transverse measures  correspond to  functions of bounded variation. In this section we will explore how different types of leaves of the lamination correspond to different types of functions of bounded variation. 

First
recall  that
a leaf  $\lambda_0$ of $\lambda$ is called {\it{isolated}}  if for each $x \in \lambda_0$ there exists a neighborhood $U$ of $x$ such that $(U, U \cap \lambda_0)$  is homeomorphic to (disc, diameter) \cite[Definition p.46]{casson}.  
 A  geodesic lamination $\lambda$ is called {\it{minimal}}, if it is minimal with respect to inclusion.

\begin{theorem}\label{cassonmin}\cite[Theorem 4.7 and Corollary 4.7.2]{casson}  A geodesic lamination is minimal iff each leaf is dense. Any geodesic lamination is the union of finitely many minimal sub-laminations and of finitely many infinite isolated leaves, whose ends spiral along the minimal sub-laminations.
\end{theorem}
 







%------------------------------
%There is a bit say about the analysis interpretation of transverse measures. 
%%Recall that $\lambda_u$ was an oriented geodesic lamination, and $\nu$  a transverse measure for $\lambda_u$.
%The next discussion applies to any oriented geodesic lamination $\lambda$ with a transverse measure $\nu$. 
%Let $\tilde \lambda$ denote the lift of $\lambda$ to the universal cover $\sigma: \tilde M \rightarrow M$. 
% From the definition of transverse measure, we see immediately that if $f$ is a transversal that intersects $\lambda$, in a point  on an isolated once,  any transverse measure on $f$ will put a non negative delta function at the point of intersection. By the homotopy invariance of a transverse measure, this is constant along $\lambda_0$. Define this to be the jump at $\lambda_0$.
In the presence  of a transverse measure $\nu$ one can easily characterize isolated leaves by using the homotopy invariance of the measure. 
If a leaf is closed, then $\nu$ is an atomic measure i.e a delta function supported at the point of intersection of the transversal and the corresponding function of bounded variation $v$ is a jump function. On the other hand, a spiraling isolated leaf $\lambda_0$ cannot support a non-zero measure because a transversal crossing $\lambda_0$  at a limit point  would have infinite measure, as it crosses $\lambda_0$ an infinite number of times.

In view of the above, from now on we will assume that the oriented lamination $\lambda$ is a disjoint union of finitely many minimal sub-laminations,  and
 we will  explore the dichotomy between closed leaves and infinite non-isolated leaves in terms of functions of bounded variation.

  

%  The lamination $\tilde \lambda$ is the closure of its discrete leaves. There are two kinds of discrete leaves:  those that are the lift of closed geodesics, and those that are the lift of  embedded, recurrent infinite geodesics on $M$.  From the definition, we see immediately that if $f$ is a transversal that intersects $\lambda$, in a point  on a discrete leaf $\lambda_0 = \sigma(\tilde \lambda_0)$ once,  any transverse measure on $f$ will put a non negative delta function at the point of intersection. By the homotopy invariance of a transverse measure, this is constant along $\lambda_0$. Define this to be the jump at $\lambda_0$.

%\begin{proof} If $\lambda_0$ were recurrent, not periodic, a transversal crossing $\lambda$ at a limit point of $\lambda_0$ would have infinite measure, as it crosses $\lambda_0$ an infinite number of times.
%\end{proof}
%George: Do you have a reference we can give?  Also is discrete the right word?  Is there another word in the literature?

%However, far more rigorous is the following construction.  We start by noting that, given any oriented transversal $f$, any transverse measure $\nu$ determines a function $\tilde v$  (technically a BV function) on a neighborhood of a lift $\tilde f$ by fixing its value on one end point and extending it using transverse flow boxes. Denote the pull back of this function by $\tilde f^*\tilde v$.
%
%
%\begin{proposition}\label{MProposition18} If $(\lambda, \nu)$ is a minimal (oriented) measured geodesic lamination with no closed leaves,  and $f: [c,d] \rightarrow M$ is an oriented transversal there is an open dense set of $[c,d]$ on which $\tilde f^*\tilde v$ is differentiable with derivative 0.  Moreover, $\tilde f^*\tilde v$ is continuous but not absolutely continuous.
%\end{proposition}
%
%Proof.  Note that $\tilde f^*\tilde v$ is constant on the dense open intervals   $ \tilde f [c,d]$ intersect $\tilde M_0$.  The properties of transverse measure imply $\tilde f^*\tilde v$ is continuous.  Such a function is  called a Cantor function after George Cantor, who defined such a function in 1883 [D-M-R-V].
%
%George:  Is there a reference in the Thurston literature?

%The measure $\nu $ does not immediately determine a global function $\tilde v$. However, by reversing the procedure by which we obtained $\nu$ from $\nu$, we can obtain  $\nu(f)$ which is defined on transversals $f$  to $\lambda$ and changes sign is under $s \mapsto -s$. This function on transversals  $\nu$ is countably additive under subdivision and under homotopy through transversals to f.  It is also clearly invariant under a generic homotopy which allows $f$ to cross discrete leaves of $\lambda$.  We conjecture it is invariant under all homotopy if the homotopy is arranged correctly with respect to flow boxes, but we leave that conjecture for the final section.  See Conjecture 6.

%George:  I did a literature search, and could find nothing on this anywhere.   The cohomology class of the line bundle that v maps to   would be an invariant of the pair $(\nu, orientations on \lambda)$ and the proof that this is well-defined would be the same proof that $\tilde v$ is well-defined. Do I have the wrong intuition about homotopy and laminations?


Recall that there are three types of functions of bounded variation defined on a ball in a Riemannian manifold:\\ 
$(i)$ Functions in the Sobolev class $W^{1,1}$.\\
$(ii)$ Jump functions across a 
(countably) rectifiable set of codimension 1.\\
 $(iii)$ Cantor functions, which are
continuous functions with derivative zero on a dense open set.
  A nice description of Cantor functions can be
found in \cite{cantorexp}. 



In fact, the derivative of any function of bounded variation can be decomposed into these three types, according to the following theorem
(cf. \cite[Theorem 3.78 and Proposition 3.92]{ambrosio}).


\begin{theorem}\label{decocantor1}
Let  $v: B \rightarrow \R$  be a function of bounded variation  defined in a ball $B$. Then, there is a canonical decomposition  
\[
dv = (dv)_0 + (dv)_{jump} + (dv)_{cantor}
\]
where: 
\begin{itemize}
 \item $(i)$ The measure $(dv)_0$ is absolutely continuous with respect the Lebesgue measure
and the measure $(dv)_{jump} + (dv)_{cantor}$ is singular with respect to Lebesgue measure.
\item $(ii)$ $(dv)_{cantor}$  vanishes on  every Borel set  with $\sigma$-finite $n-1$ Hausdorff measure.
\item $(iii)$   $(dv)_{jump}$ is computed as a measure of a jump discontinuity on a countably $n-1$ dimensional rectifiable set.
\end{itemize}
\end{theorem}


\begin{corollary}\label{decocantor2} Let $v$ be the  primitive of the Ruelle-Sullivan current  associated to an oriented measured geodesic lamination
$\Lambda=(\lambda, \nu)$. In the decomposition of the measure $dv$, 
\begin{itemize}
\item $(i)$ $(dv)_0 = 0$.
\item $(ii)$ $(dv)_{jump}$ is supported on closed isolated leaves. 
\item $(iii)$ $(dv)_{cantor}$ is supported on minimal laminations which are not closed isolated leaves.
\end{itemize}
\end{corollary}




%% BV functions have a standard decomposition.  Here we can work in a ball $B$ in $M$ and use a local definition of $v: B \rightarrow \R$, since it is not a global theorem.  A BV function defined in a neighborhood of $B$  has a canonical decomposition $v = v_0 + v_{jump} + v_{cantor}$, where $v_0$ in $W^{1,1}(B)$, $v_{jump}$ is locally constant with jumps along rectifiable sets of codimension 1 and $v_{cantor}$ is continuous with derivative zero on an open dense set.  See  \cite[Theorem 3.77 and Proposition 3.92]{ambrosio} and \cite{evans}  for precise definitions and details.
%%
%%George:  I can?t get those books off the web. Possibly available from the Princeton University library when I get back. I do not understand the u notes Camillo sent to me, but they are also unpublished (not on the web)  I will work at getting the statement of this theorem precise.  Perhaps we don?t have to define everything exactly.  By the way, the theorem is true for vector-valued functions and there is something about a rank 1 theorem.  Might turn out to be very useful mapping surfaces to surfaces.

%\begin{theorem}\label{Mtheorem19} The canonical division of $v$ obtained as in section 4 is $v = v_{jump} + v_{cantor}$, where the support of $dv_{jump}$ is on the closed geodesics $\lambda_ {periodic}$ and $dv_{cantor}$ is supported on $\lambda - \lambda_{0 periodic}$.
%\end{theorem}
\begin{proof}
The support of $dv$ lies on the lamination, which is of measure 0, so 
$(dv)_0= 0$.
  By the decomposition Theorem~\ref{cassonmin}, we can write  $\lambda$
 as a disjoint union of minimal sublaminations $\lambda_1$ which is the disjoint union of  closed leaves and $\lambda_2$ which is the disjoint union of all sublaminations consisting  of minimal non-isolated leaves. Given a flow box $F$, there is a finite number of points in $K=\{k_1,...,k_n \}$ corresponding to closed isolated leaves  and write $c=k_0< k_1 <... < k_{n+1} =d$. These coincide with the atoms of the measure $\nu$, hence by \cite[Corollary 3.33]{ambrosio} the discontinuity set of $dg^\sharp=d\nu$ must be equal to the set 
 $\{ k_1,..., k_n\}$. Since $v^\sharp(s,t)=g^\sharp(s)$, it follows that the discontinuity set of $dv$ is equal to $\lambda_1$. By 
 \cite[Definition 3.91]{ambrosio}, $(dv)_{jump}$ is supported on $\lambda_1$ and $(dv)_{cantor}=dv-(dv)_{jump}$ on $\lambda_2$.
 % let $g^\sharp(s)=\int_c^s d\nu $ and 
% note that the discontinuity set of $g^\sharp$ which is equal to the atoms of the measure $\nu$ the notation as on Proposition~\ref{discrleav2}, there is a finite number of points in $K=\{k_1,...,k_n \}$ corresponding to closed isolated leaves  and write $c=k_0< k_1 <... < k_{n+1} =d$. The function $g^\sharp=a_j$ is constant in the open intervals $I_j =(k_j, k_{j+1})$,  so
% $v = a_j$ constant on the rectangular shaped region $R_j=F([a,b] \times I_j)$ bordered by the two end transversals and the geodesics $\lambda_j$ and $\lambda_{j+1}$ which pass through $k_j$ and $k_{j+1}$. In other words, the 
% 
% Such a function is a jump function and $(dv)_{jump}$ is a the measure associated to the jump function.
%
% 
% 
% The lamination $\Lambda_1$ is a set which is 1-rectifiable, whereas  $\Lambda_2$ is purely unrectifiable. Hence 
% \[
% (dv)_{jump}=dv \Big |_{\Lambda_1} and \ (dv)_{cantor}=dv \Big |_{\Lambda_2}.
% \]
%
%
%
%
%By the decomposition Theorem~\ref{cassonmin}, the measured lamination $\lambda$ can be written as a disjoint union of minimal sublaminations of two types: The first type is  consisting of closed geodesics. With the notation as on Proposition~\ref{discrleav2}, there is a finite number of points in $K=\{k_1,...,k_n \}$ corresponding to closed isolated leaves  and write $c=k_0< k_1 <... < k_{n+1} =d$. The function $g^\sharp=a_j$ is constant in the open intervals $I_j =(k_j, k_{j+1})$,  so
% $v = a_j$ constant on the rectangular shaped region $R_j=F([a,b] \times I_j)$ bordered by the two end transversals and the geodesics $\lambda_j$ and $\lambda_{j+1}$ which pass through $k_j$ and $k_{j+1}$. Such a function is a jump function and $(dv)_{jump}$ is a the measure associated to the jump function.
%Note that $v$ is constant on the open plaques in $M_0$ corresponding to the minimal, non-isolated components of the lamination, hence it must be a Cantor function.
\end{proof}

Actually we can prove a stronger statement

\begin{proposition} If $v$ is the primitive for the Ruelle-Sullivan current associated to an oriented measured geodesic lamination $\Lambda$, then 
\[
v = v_1 + v_2
\]
 and $dv_1=(dv)_{jump}$, $dv_2=(dv)_{cantor}$. Here $v_1$ is a jump function and $v_2$ is a Cantor function.
\end{proposition}
\begin{proof}
% The support of $(dv)_{jump}$ is on closed geodesics and the support of $(dv)_{cantor}$ is on the union of minimal laminations in which every leaf is dense. By the decomposition Theorem~\ref{cassonmin}, these are disjoint measured laminations. Let $v_{jump}$ be a primitive for $(dv)_{jump}$ and $v_{cantor}$ for $(dv)_{cantor}$. Hence the proposition follows since the primitives are unique (up to an additive constant).
% \end{proof}
% 
% -----------------
% \begin{proof}The support of $dv$ lies on the lamination $\lambda$, which is of measure 0, so 
%$(dv)_0= 0$.
%  By the decomposition Theorem~\ref{cassonmin} we can write  $\Lambda$
% as a disjoint union of minimal sublaminations $\Lambda_1$ which is the union of sublaminations consisting of closed leaves and $\Lambda_2$ which is the union of all sublaminations consisting  of minimal non-isolated leaves. The lamination $\Lambda_1$ is a set which is 1-rectifiable, whereas  $\Lambda_2$ is purely unrectifiable. Hence 
% \[
% (dv)_{jump}=dv \Big |_{\Lambda_1} and \ (dv)_{cantor}=dv \Big |_{\Lambda_2}.
% \]
% 
Let  $\Lambda_i=(\lambda_i, \nu)$ where $\lambda_i$ as in the previous Corollary and $T_{\Lambda_i}$  the  Ruelle-Sullivan currents corresponding to $\Lambda_i$.  Then,
 $T_{\Lambda_i}$ is closed and let $v_i$ be their primitives, $dv_i=T_{\Lambda_i}$ as in Theorem~\ref{conversetomeasure}. Since
 $T_{\Lambda}=T_{\Lambda_1}+T_{\Lambda_2}$ and since the primitives are unique (up to an additive constant),
 $v=v_1+v_2$. By construction, $dv_i$ is supported on $\lambda_i$ and thus $dv_1=(dv)_{jump}$ and $dv_2=(dv)_{cantor}$.
 % With the notation as on Proposition~\ref{discrleav2}, in a given flow-box, there is a finite number of points in $K=\{k_1,...,k_n \}$ corresponding to closed isolated leaves  and write $c=k_0< k_1 <... < k_{n+1} =d$. The function $g^\sharp=a_j$ is constant in the open intervals $I_j =(k_j, k_{j+1})$,  so
% $v = a_j$ constant on the rectangular shaped region $R_j=F([a,b] \times I_j)$ bordered by the two end transversals and the geodesics $\lambda_j$ and $\lambda_{j+1}$ which pass through $k_j$ and $k_{j+1}$. Thus $v_1=:v_{jump}$  is a jump function and $dv_{jump}$ is a jump discontinuity measure. 
% 
% Notice that 
%  
% Note that $v_2$ is constant on the open plaques in $M_0$ corresponding to the minimal, non-isolated components of the lamination, hence $v_2=v_{cantor}$ must be a Cantor function.
 \end{proof}
% 
%  \begin{lemma} Suppose that the support of the Ruelle-Sullivan current $T$ is on a set $S = S_1 \cup S_2$ where $S_1 \cap S_2 = \emptyset$. Then $T=T_1+T_2$, where $T_j$ has support on $S_j$ and $v = v_1 + v_2$, where $dv_j = T_j$.
% \end{lemma}
%\begin{proof} Let $T_j = \phi_j T$ where $\phi_1$ is a smooth function such that $\phi_1(x) = 1$ for $x \in S_1$ and $\phi_1(x) = 0$ for $x \in S_2$ and vice versa for $\phi_2$. Then $T_j$ is closed and the primitive for $T_j$ is constructed as in Theorem~\ref{conversetomeasure}. Then $v = v_1 + v_2$ has the requisite properties since $v$ is unique up to a constant.
%\end{proof}

 
 This also applies to the transverse least gradient measures obtained from best Lipschitz maps. Note that the cohomology classes associated with the transverse measures and laminations add.


 
\begin{corollary} Let $\Lambda=(\lambda, \nu)$ be an oriented lamination without isolated leaves. Then the primitive $v$ of the Ruelle-Sullivan current $T_\Lambda$ defines a continuous but not absolutely continuous section $v: M \rightarrow L$ whose derivative is zero almost everywhere.
\end{corollary}

%\subsection{The role of orientation}
%So far throughout the paper, the laminations which we construct on which the best Lipschitz constant is achieved are naturally oriented. To construct the Ruelle-Sullivan current and a BV map $v$ associated to the current, it is necessary to introduce an orientation. Since the infinite isolated leaves can carry no measure, we obtain a consistent orientation on a lamination by assigning an orientation to each minimal sub-lamination as in Theorem~\ref{cassonmin}.  If the lamination has $N$ minimal sub laminations in which every leaf is dense, it appears as if there might be $2^N$ choices of orientation, where $N$ is the number of minimal components.  These, of course, leads to inconsistencies if there are isolated connecting leaves, but this inconsistency does not effect the transverse measure.
%
%In the Thurston approach, one goal is to obtain a maximal transverse measure and lamination, which divides the manifold into ideal triangles.  We can see from the results of the previous sections,  that the lamination obtained from an $\infty$-harmonic map into a circle (or a line bundle) does divide  the manifold up into ideal cells.  The critical points of $u_p$ are saddle points, and we expect this to be true of $u$. A model for saddle singularities for $\infty$-harmonic maps is known \cite{aronson1} and \cite{aronson2}. In addition the lamination comes with a natural orientation; there are never ideal triangles. The maximal oriented lamination which might be attained is ideal quadrilaterals, which come naturally oriented.  At first one might think that different orientations on these minimal components might come from different choices of a line bundle for $u$.
%
%However, we give an example which shows this approach is too naive. While we have at this time no final answer as to how to choose an orientation, some choices of orientation are clearly better than others when starting from the transverse measure. See Figure above.
%
%
%\begin{figure}[t]\label{figureOR}
%  \includegraphics[width=3.5in]{g1.jpg}
%  %\caption{}
%  \label{fig-orientation}
%\end{figure}
%
% \begin{example} We form a surface of genus 3 from 2 copies of a handle attached to rectangles. In the sketch, we identify $A$ with $A'$, $B$ with $B'$ and the closed curve $CD $ to the closed curve $C'D'$. If this is done with a reflexive symmetry through $A = A'$ and $B=B'$, and an additional one through the dotted lines, after uniformization, the closed curves $A = A'$ and $B=B'$ will be geodesics of equal length. We put a transverse measure of a delta function on 
% $A = A'$ and $B=B'$.
%If we orient the closed geodesic $A = A'$ opposite to $B=B'$ as in the sketch, the Ruelle-Sullivan current will be $dv$, where $v$ is identically 1 on the right and 0 on the left. This is a poor choice of orientation, as the curves $A$ and $B$ meet the circle $CD$ with opposite orientation. The correct orientation is to reverse either $A$ or $B$. Another explanation is the homology class of 
%$A \cup B$ with the orientation as drawn is 0, whereas with the correct orientation, it is $2A$. With the correct orientation, one conjectures that the two closed geodesics $A$ and $B$ take on the best Lipschitz constant for the cohomology class of the map which collapses the two handles to the rectangles and projects both rectangles on to the circle formed by $A$.
%\end{example}
%
% 
%
%
%%The support of $dv$ lies on the lamination, which is of measure 0, so
%% $v_0= 0$. Propositions 17 and 18 identify the support of the other two function in the decomposition.
%%We return to the case where $\nu$ is an arbitrary transverse measure on a geodesic lamination $\lambda$. The existence of a closed one form $dv$ is attributed to Ruelle and Sullivan \cite{sullivan}. We could not find a proof of the existence of sufficiently smooth flow boxes which preserve the lamination, so we give an alternative proof. The measure $\nu$ does not immediately determine a global function $\tilde v$ but it is not difficult to construct one locally. Let $f: [c,d] \rightarrow M$ be an immersed, positively oriented transversal $c< 0 <d$, and let $\tilde f: [c,d] \rightarrow H^2$ a lift. Recall $\tilde M_0 = \cup_j S_j$. Let 
%%\[
%%n(\tilde f) = \{j : \tilde f(a) \in S_j \ \mbox{for some} \ a \in [c,d] \}.
%%\]
%% Assume $\tilde f(0) \in S_0$.
%% 
%%\begin{definition}\label{MDefinition 20} For some $\tilde f(a) \in S_j$, $j \in n(\tilde f)$, let $f_a(s) = f(s)$ for $s \in [0,a]$. Now let $v^{\tilde f}$ to be the constant sign a $\nu(f_a)$ on all of $S_j$.
%%\end{definition}
%%\begin{proposition}\label{MProposition21} $v^{\tilde f}$ has the following properties.
%%\begin{itemize}
%%\item $(i)$ $v^{\tilde f}$ is independent of the choice of $a$, $\tilde f(a) \in S_j$
%%\item $(ii)$ Interior closure $\cup_{j \in n(\tilde f)} S_j = S_{\tilde f}$ is a contractible open set
%%in $H^2$
%%\item $(iii)$ $v^{\tilde f}$ extends to a locally bounded BV function on $S_{\tilde f}$ which has
%%derivative 0 on a dense open set.
%%\item $(iv)$ If $g: [0,1] \rightarrow M$ is an oriented transversal to $\lambda$ with a lift $\tilde g \in S_f$,
%%then
%%\[
%% \nu(g) = \tilde f(\tilde g(1)) - v^{\tilde f}(\tilde g(0))
%% \]
%%\item $(iv)$ Let $f$ and $g$ be two oriented transversals with lifts $\tilde f$ and $\tilde g$.
%%If $U = S_{\tilde f} \cap S_{\tilde g} \neq \emptyset$,  then $v^{\tilde f} =v^{\tilde g} + C.$
%%\end{itemize} 
%%\end{proposition}
%%Proof: $(i)$ follows trivially since in this case $\nu(f_a) = \nu(f_a')$ if $f(a)$ and $f(a')$ are in the same plaque.\\
%%$(ii)$ The plaques themselves are contractible onto the image of $\tilde f$.\\
%%To see $(iii)$, note that by definition, $v^{\tilde f}$ is constant on  $S_j$, which by definition is a dense open set, and is locally bounded since $\nu(f_a)$ has that property. The singularities are jump singularities along lifts of the
%%closed geodesics. On $\tilde \lambda \backslash \tilde \lambda_{closed}$, $v^{\tilde f}$ is continuous, since $\nu(f_a)$ extends to a continuous function.\\
%%$(iv)$ follows from the fact that, after reparameterizarion, $g: [c_1,d_1] \rightarrow M$ is homotopic to $f: [c_1,d_1] \rightarrow M$ through transversals to $\lambda$.
%
%%\subsection{The role of orientation}\label{orient}
%%{\bf Karen: Just wrote what you sent me. Need to work on this}
%%The laminations which come from $\infty$-harmonic maps $u: M \rightarrow S^1$ have the property that every leaf has an orientation which is consistent with the orientation on a finite cover of oriented flow boxes. Hence the dual least gradient map $v: M  \rightarrow L $ provides us with the cohomology class. (We expect the same to be true in the more general case if $u$ is a section of an oriented flat line bundle $E$.) However, the $\pm dv$ we obtain from the Ruelle-Sullivan theorem cannot always have this property. It might be true that the ambiguity of $\pm$ could be resolved by going to a finite cover, but this does not seem to be the case. As already mentioned in \cite{bonahon4} if $\lambda$ is a maximal lamination (i.e all its plaques are ideal triangles) then $\lambda$ cannot have an orientation.
%%%Observation: Suppose \lambda is a geodesic foliation of M such that one of the plaques in the universal cover is a geodesic triangle connection points (A,B,C) on the boundary of H^2 by great circles AB,BC,CA. Then \lambda cannot have an orientation. 
%%This follows from the fact that two geodesics in $H^2$ with the same endpoint at infinity will ultimately appear in the lift of one of the flow boxes, and hence must both be oriented towards or away from the limit point at infinity. 
%%One cannot arrange that at all three points of an ideal triangle.
%%
%%We expect geodesic quadrilaterals in the cover if u is a section of a flat line bundle. {\bf{CHECK}}In this case, $V = dv \otimes du$ is a singular quadratic differential which factors. Note that since $dv$ is transverse and singular, it is enough to know $d/dt \lambda = \sharp du$ on the leaves, where we use $\sharp$ to take vector fields to dual forms and vice versa. In general we expect 
%%\[
%%\Xi = dv  \otimes  \sharp d/dt \lambda
%%\]
%% to represent a quadratic differential when symmetrized.
%%
%% 
%%The recipe for computing $\Xi$ on vector fields $(V,W)$ is
%% \[
%%\int \Xi(V,W) *1 = \int  dv(V) \sharp \Lambda(W)*1 = \int dv \wedge Q.
%%\]
%%Here 
%%$Q = (\Lambda,W)*V$, which is continuous. The integral only depends on the definition of 
%%$Q$ on the support of $dv$, so it is independent of how we made the extension of $n$ off $K$. However, we had to know it had a continuous extension for the calculation to be valid.
%%
%%\begin{theorem} $\Xi = dv \otimes \sharp d/dt\lambda$ is a well defined Radon measure with values in $T^*(M) \otimes T^*(M)$. It does not depend on orientation. Then $dv = \Xi( ,\Lambda)$ is a well-defined closed Radon measure whose sign depends on the orientation of the flow box 
%%$F.$
%%\end{theorem}
%%
%%
%%\begin{remark} It should be true that locally we can replace $\sharp \Lambda$ locally by $du$. I don?t know how to frame the PDE which says that. If $\Xi$ is a smooth rank one section of $T*(M) \otimes T*(M)$, how do I say it is
%%$dv \otimes du$?
%%\end{remark}
%%
%%
%%
%%
%

\section{Conjectures and open problems}\label{conjectures}
As mentioned already the authors' motivation for this paper is understanding Thurston's work of best Lipschitz maps between surfaces. As such, the results of this paper only serve as a toy problem in understanding the more difficult problem of best Lipschitz maps between surfaces.
This paper is by no means complete and is only meant to be the preliminary part of a more thorough study.   
This section contains some  suggestions for new directions for research. 

The main topic of study in this paper is $\infty$-harmonic maps from hyperbolic manifolds to $S^1$ and their maximum stretch laminations. The theory of $\infty$-harmonic functions has been thoroughly worked out for Euclidean metrics but so far no work has been done for variable metrics. For example, there is no reference of viscosity solutions for other than flat metrics and there is no reference for the equivalence with the notion of comparison with cones. In Section~\ref{sect:crandal} (cf. Proposition~\ref{comocones1}), we worked out the bare minimum of what we needed from the theory of comparison with cones in order to obtain our results on  geodesic laminations. However, the theory  is far from complete. In particular, for the sake of simplicity, we only considered the hyperbolic metric, thus leaving the theory for general metrics as a conjecture: 

\begin{conjecture} Develop the theory of $\infty$-harmonic functions for  general  Riemannian metrics. Most of the known results about $\infty$-harmonic functions (including the theorems of Crandall on gradients \cite{crandal} and the regularity results of  Evans-Savin \cite{evans-savin} and Evans-Smart \cite{evans-smart}) should carry over to this case.
 In particular, show that
 Theorem~\ref{straightline} holds for any Riemannian manifold $(M,g)$.
\end{conjecture}





%First of all, we use two local theorems which are proved and carefully written down for domains in $\R^2$ and $\R^n$. We need them in $H^2$ or more generally in $M^n$ for arbitrary Riemannian metrics. This seems to be similar to the case of the monotonicity theorems. The proof in a curved space follows that in $\R^n$ with some small perturbations which disappear in the limits as in the small, a manifold looks like $\R^n$. Papers often give the proof in $\R^n$ with the comment that it extends to  Riemannian manifolds. The experts we have consulted agree that is true in these cases. 

%In all cases, $B$ is a small open neighborhood of a point in Riemannian manifold $(M,g)$, $B' \subset  B$ a smaller relative compact neighborhood.
%In all cases, we assume that $u: B \rightarrow \R$ is an $\infty$-harmonic function obtained as a limit of $p$-harmonic functions as $p\rightarrow \infty$.

%\begin{conjecture}\label{Conjecture 1a} Let $B$ is an open set in a Riemannian surface $(M,g)$ and
%$u: B \rightarrow \R$  an $\infty$-harmonic function. Then, for $B' \subset  B$ a  relative compact open subset, $u \big|_{B'} \in C^{1,\alpha}(B')$ for some $\alpha > 0$. 
%\end{conjecture}

%\begin{thmconj}\label{Conjecture 1b}\cite{evans-smart}For $dim B > 2$, if $u: B \rightarrow \R$ is an $\infty$-harmonic function, then $u$ is differentiable at every point in $B$.
%\end{thmconj}
%\begin{thmconj}\label{Conjecture 1c}\cite[Proposition 6.2]{crandal}   If $u: B \rightarrow \R$ is an $\infty$-harmonic function, $x \in B$. Then, there is a $T > 0$ and a Lipschitz continuous curve $\gamma : [0, T ) \rightarrow B$ with the following properties:
%  \begin{itemize}
%\item (i) $\gamma(0) = x$
%\item (ii) $|\gamma'(t)| \leq 1$ a.e. on $[0,T)$
%\item (iii) $L_u (\gamma(t)) \geq L_u(x)$ on $[0,T)$
%\item (iv) $u(\gamma(t)) \geq u(x) + tL_u(x)$ on $[0,T)$
%%\item (v) $t \mapsto u(\gamma(t))$ is convex on $[0,T)$.
%\end{itemize}
%\end{thmconj}


%Theorem-Conjecture 1c: [see C, Proposition 6.2]. If u: B?-> R is L^\infinity harmonic then there exists a curve \gamma:[0,a]?> B with the following properties: (Here L_u(y) = |du|(y)|)
%\gamma(0) = x,
%|\gamma?(t)| <= 1
%L_u(\gamma(t) >= L_u(x) u(\gamma(t)>= u(x) + t L_u(x) u(\gamma(t)) is a convex function of t.
% George: You can copy from the interior of the paper. But I wanted to keep it a local theorem.
%Theorem-Conjecture 1d: (see Ju). If v_q is a sequence of W^(1,q) functions v_q: B?>R with a uniform bound on ||v_q||_(W^(1,q) as q?>1, then any weak limit v in L^\s, s < n/(n-2) of v_q is a least gradient BV function in B?.

%Theorem~\ref{Conjecture 1c} as stated by Crandall has an additional global conclusion which we do not need, since our curves are geodesics and automatically are unique. The local implies the global.

There are two uniqueness theorems which we believe to be true, but cannot prove.
\begin{conjecture}\label{Conjecture 2} The $\infty$-harmonic map $u: M \rightarrow S^1$ in a homotopy class is unique up to rotation in $S^1$. Same conjecture for the equivariant problem.
\end{conjecture}

%Given two such maps $u$ and $u'$, one can consider their difference in the universal cover $\tilde u'- \tilde u : \tilde M 
%\rightarrow \R$. One can try to use maximum principle arguments, but the problem is that the maximum might well be obtained on a closed set which is not homotopically trivial and has no neighborhood diffeomorphic to an open domain in the plane. This conjecture turns out not to be a serious deficiency, as Theorem~\ref{K=L} gives another characterization of the lamination on which $|du| = L$. This set is geometrically determined and must be the same for any two $\infty$-harmonic maps $u$ and $u'$  (see Corollary~\ref{strsuppp}). Moreover
%$du = du'$ on the support of any limiting measure $dv$.

The uniqueness proofs do not carry over for maps into $S^1$.  They are based on constructions which involve taking the maximum of $u$. This has nothing to do with the hyperbolic metric. One would meet the same problems in the following problem in Euclidean space.
Let $\Omega$ be an annular region in $\R^2$, choose a map $u_0: \Omega \rightarrow S^1$, let $b =u_0 \big |_{\partial \Omega}$.  Find the $\infty$-harmonic map $u$ with $u \big |_{\partial \Omega}=b$.   Existence and regularity are straightforward. Is $u$ unique?


\begin{conjecture}\label{Conjecture 3}  The  BV section $v: M \rightarrow L$ in Theorem~\ref{transmeasure}  is unique.
\end{conjecture}

 This is two problems. The first part is to show the cohomology class $\alpha$ of $L$ is unique, and the second is the analogous of Conjecture~\ref{Conjecture 2} for $v$ instead of $u$. Recall from Theorem~\ref{thm:equivsit}  that,  a-priori, $\alpha$ depends on the sequence  $q \rightarrow 1$. Is $\alpha$ independent of the sequence? The second part is that, given $\alpha$ (or equivalently $L$), the least gradient map $v: M \rightarrow L$ is unique. This is a serious deficiency, since $v$ determines a transverse measure and a geodesic lamination may support different transverse measures (not uniquely ergodic).

\begin{problem}\label{Problem 5}
 We know by Theorem~\ref{thmlegr} that any weak limit $v$ of the $q$-harmonic functions $v_q$ is of least gradient. Use the map $v$ to prove directly  that the support of $dv$ is a geodesic lamination, bypassing the need to use any properties of $\infty$-harmonic functions and our proof that the best Lipschitz constant is achieved on a geodesic lamination. 
  \end{problem}
  
%  Below we will present a  sketch for a solution to Problem~\ref{Problem 5}: 
%  and thus sketching another proof of the following:
%  
% \begin{theorem}\label{spptlam}The closed set $suppt(dv) \subset \lambda_u$ is a geodesic lamination.
%\end{theorem}

%We now briefly sketch how to proceed with the answer to Problem~\ref{Problem 5}. We  first state the following hyperbolic analogue of \cite[Proposition 3.5]{gorny}. The proof is carefully written down for the Euclidean metric and one expects the extension to the hyperbolic case  more or less along the same lines.
%
%\begin{thmconj}\label{gornylemma} Let $\tilde v$ denote the least gradient map of Corollary~\ref{thmlegr1} and let $B$ be an open ball in $\tilde M$. For each $t \in \R$ consider the set $E_t :=B \cap \{\tilde v \geq t \} $. Then:
%\begin{itemize}
%\item  $(i)$  $E_t$ is a minimal set in $B$. 
%\item   $(ii)$ $\partial E_t$ is a disjoint union of geodesic segments. 
%\item $(iii)$ The ends of the geodesic segments lie in $\partial B$.
%\item $(iv)$ If  $ \partial E_t \cap \partial E_{t'} \neq \emptyset$ then $\partial E_t=\partial E_{t'}$.
%\end{itemize}
%\end{thmconj}

%\begin{proof}
%Since, by Corollary~\ref{thmlegr1}, $\tilde v$ is of least gradient in $B$, \cite[Theorem 1]{bombieri}  implies that  the characteristic function $\chi_t$ of $E_t$ is of least gradient and therefore  $E_t$ is a minimal set in $B$. 
%
%By the regularity theorem for minimizing currents \cite[Theorem 10.11]{giusti}, $\partial E_t$ is locally equal to a union of analytic curves which by minimality must be geodesics.
%
%By maximum principle both the ends of the geodesic segments must be in $\partial B$. {\bf{????}}
%
%If $t<t'$, then $E_{t'} \subset E_t$ and if $x \in \partial E_t \cap \partial E_{t'}$, then \cite[Theorem 2.2]{ziemer3} implies that $\partial E_t$ agrees with $\partial E_{t'}$ in a neighborhood of $x$.
%\end{proof}
%  
  
%  From here it is relatively easy to show that the support of $dv$ is a geodesic lamination without recourse to results about $\infty$-harmonic maps used in Theorem~\ref{straightline}.  However, this does not directly show that the entire closed set $\{x : |du(x)| = L\}$ is a geodesic lamination, but only that it contains a geodesic lamination. 
  
  A partial converse to the statement of Problem~\ref{Problem 5} should also hold: 
  
  
  \begin{conjecture} Suppose that $\lambda$ is an arbitrary oriented lamination on $M$ with a transverse measure.  Let $\tilde v: \tilde M \rightarrow \R$ be a primitive for the Ruelle Sullivan current associated with the transverse measure. For any ball $B \subset \tilde M$, 
  $\tilde v \big|_B$ is of least gradient.
  \end{conjecture}
  
  The main point in the above statement is that the boundary of the sets $\tilde v \geq t$ in $B$ are geodesics. From this, one should be able to deduce like in the Euclidean case that $\tilde v$ is a locally a map of least gradient. 
%The connection between least gradient functions on surfaces and geodesic laminations still seems unexplored. A natural question would be the following:
%
%\begin{problem} Replace the BV section $v$ associated with an arbitrary geodesic lamination $(\lambda, \nu)$ as in Theorem~\ref{conversetomeasure} with a least gradient section $v$. This would imply by results of \cite{gorny} (written down carefully for the Euclidean metric and very likely generalizable for the hyperbolic metric) that the decomposition of measures given in Theorem~\ref{decocantor1} and  Corollary~\ref{decocantor2} reflecting the decomposition of the minimal components of the lamination into different types, would be induced by a decomposition on the level of functions. More precisely, there is a unique (up to additive constants) decomposition
%\[
%v=v_{jump} + v_{cantor}
%\]
%where $v_{jump}$ is a jump function and $v_{cantor}$ is a Cantor (continuous) function. Furthermore, both functions are of least gradient.
%\end{problem}
%Due to the length of this paper we will return to this question in a future publication.
%   We now continue with different questions.




%\begin{thmconj}\label{Theorem-Conjecture 4}  The results of this paper extend to  arbitrary representations $\rho:\pi_1(M) \rightarrow \R$. In other words, given $\rho$,  there exists an $\infty$-harmonic function  $ \tilde u: \tilde M \rightarrow \R$ equivariant under $\rho$ and a dual least gradient function $ \tilde v: \tilde M \rightarrow \R$ equivariant under another representation $\alpha:\pi_1(M) \rightarrow \R$.
%\end{thmconj}
%%This conjecture bears connection with Thurston's work \cite{thurston} and we will come back later.
%% Prescribing the line bundle $E$ for $u: M\rightarrow E$ is an analogue of the Dirichlet problem for $u$. Prescribing the bundle $v: M \rightarrow L$ is an analogue of a Neuman problem for $u$. For finite $1< q \leq p < \infty$, both Dirichlet and Neumann problems have solutions (unique up to equivalence). We are asking the question for $p = \infty$, $q = 1$.
% 
%We say this is a 
%Theorem-Conjecture because we believe our paper could have been written to include this case.  We did not do this, as many in our target audience would have found it a source of added confusion in a paper that already contains unfamiliar topics. Theorem-Conjecture~\ref{Theorem-Conjecture 4}  is also a very important intermediate step leading to Problem~\ref{Conjecture 10} where we replace $S^1$, or rather its universal cover $\R$ by an arbitrary $\R$-tree.

%\begin{thmconj}\label{Theorem-Conjecture 5}  Given a representation $\rho$, show that the functional K of (\ref{defKref}) is well-defined on the space of all geodesic laminations with transverse measures.
%\end{thmconj}
%
%We leave this to a subsequent paper to prove. 
 
\begin{thmconj}\label{Theorem-Conjecturepun} The results of this paper extend to surfaces with punctures. 
\end{thmconj}
Throughout the paper we restricted ourselves to the case of closed manifolds. However, Thurston's theory works also for laminations on surfaces with punctures. Most of the results in this paper are local and  carry through also for punctured surfaces without significant change.

The next problem is the analogue of Thurston's construction \cite{thurston} adapted to our situation. Before we state the problem we need some notation. Given a cohomology class $\rho \in H^1(M,\R)$ and  a hyperbolic metric $g$ on $M$ we can consider $K$ as in (\ref{normlength}) defined on the space of oriented measured laminations. Equivalently, let $\mathcal M \mathcal L_\rho^{or}$ denote the space of oriented measured geodesic laminations whose homology class is dual to $\rho$, i.e oriented measured laminations $\lambda$ subject to the topological constraint $\rho(\lambda)=1$. On this space we consider the length functional
\[
l_g: \mathcal M \mathcal L_\rho^{or} \rightarrow \R
\] 
associating to a measured lamination $\lambda$ its length $l_g(\lambda)$ with respect to $g$.
\begin{problem}\label{toyprobm} Compute critical points of the  function $l_g$ and study the connection with $\rho$-equivariant best Lipschitz functions and their maximum stretch laminations. Carry through Thurston's construction for this case. 
\end{problem}

%This should be a very doable problem.
%\begin{problem}\label{Conjecture 6} Given a line bundle $E$ over $ M$,  there exists an $\infty$-harmonic section $u: M\rightarrow E$ and a dual least gradient section $v: M \rightarrow L$ for a line bundle $L$.
%\end{problem}
%%This conjecture bears connection with Thurston's work \cite{thurston} and we will come back later.
% Prescribing the line bundle $E$ for $u: M\rightarrow E$ is an analogue of the Dirichlet problem for $u$. Prescribing the bundle $v: M \rightarrow L$ is an analogue of a Neuman problem for $u$. For finite $1< q \leq p < \infty$, both Dirichlet and Neumann problems have solutions (unique up to equivalence). We are asking the question for $p = \infty$, $q = 1$.
%\begin{conjecture}\label{Conjecture 7}
% A transverse measure to a geodesic lamination determines a singular quadratic differential. Formulate this properly and describe a least gradient quadratic differential.
%\end{conjecture}

\begin{problem}\label{Problem 8} The gradient field for a p-harmonic function $u_p$ determines an interval exchange map on a regular fiber $u_p^{-1}(t)$. Study the invariants of these as $p \rightarrow \infty$.
 \end{problem}
 We refer the reader to the very readable paper of Masur \cite{masur}.

\begin{problem}\label{Conjecture 10} Investigate the theory of $p$-harmonic maps, $\infty$-harmonic maps and least gradient maps into trees and their duality. 
  \end{problem}
%  Some aspects of the theory of  \cite{gromov-schoen}, \cite{korevaar-schoen1} and \cite{korevaar-schoen2} go through without much trouble. By utilizing the notion of energy of maps to trees developed in \cite{gromov-schoen}, \cite{korevaar-schoen1} and \cite{korevaar-schoen2}   the analysis  should carry  over to this case.
For a  combinatorial approach to best Lipschitz maps to trees, see \cite{naor-sheffield}. 

\begin{problem}\label{Problem 12}
 Develop a theory of $\infty$-harmonic maps  $u: M^3 \rightarrow S^1$ where $M^3$ is a hyperbolic 3-manifold.
 \end{problem}

From Section~\ref{sect:crandal} we have shown that the set of maximum stretch $L_u = L$ is a geodesic lamination. However, the geometry of the dual problems or the two form
$dv_q =|du_p|^{p-2}*du_p$ and the limit $q \rightarrow 1$ is unexplored territory. The dual 2-form $*du$ is a transverse area measurement which is far less rigid than length. Purely geometric descriptions of hyperbolic 3-manifolds which fiber over a circle are sorely lacking, so it is worth exploring any possibility. 


\begin{problem}\label{Problem 15} Let $M = M^3$ be a hyperbolic manifold which fibers over a circle. Study least gradient maps $v: M \rightarrow S^1$, or more generally equivariant  least gradient maps to $\R$ or trees.
 \end{problem}
 This is a promising problem, since a lot is known about least gradient maps in three dimensions. For the Dirichlet problem for domains in $\R^3$, the level sets of a least gradient function are minimal surfaces; hence we expect this least gradient map to tie into the theory of minimal surfaces in $M$. 
 
 As mentioned already the motivation for this paper was in understanding Thurston's work of best Lipschitz maps between surfaces. As such, the results of this paper only serve as a toy problem.  We conclude by stating the motivating problem and a quick preview of our approach in the forthcoming papers \cite{daskal-uhlen2} and \cite{daskal-uhlen3}:
 
 \begin{problem}\label{Problem 11} Is there an analogous analytical theory of $\infty$-harmonic maps $u: M \rightarrow N$ between hyperbolic surfaces which ties into Thurston's results on the asymmetric metric on Teichm\"uller space using best Lipschitz maps?
\end{problem}
 
 The analysis is entirely lacking for this problem, although there is a  topological theory due to Thurston and his school (cf. \cite{thurston}, \cite{papa} and \cite{kassel}). 
  The main problem in the analysis is the lack of a good notion  of viscosity solutions for systems and this seems out of reach at this point. In \cite{daskal-uhlen2} and \cite{daskal-uhlen3} we will  bypass this issue and develop a theory analogous to this paper that ties in with Thurston.
  
  Like in this paper, the first step is to define a good notion of $p$-approximations of best Lipschitz maps.  In the case when the target has dimension greater than 1, $p$-harmonic maps is not the right notion since they do not converge to best Lipschitz maps.  We consider  maps minimizing the $p$-Schatten-von Neumann norm of the gradient instead of the $L^p$-norm.  This version of $p$-harmonic maps have even weaker regularity properties and don't satisfy maximum principle. This makes it hard to prove  comparison with cones. We have to
  rely on the result of Gueritaud-Kassel \cite{kassel} in order to show that the maximum stretch set of the infinity harmonic map contains  Thurston's canonical lamination. 
  
  Another difference  with the scalar case is the construction of the dual functions and the limiting measures. When the target is a hyperbolic surface,  $v$ has values in the Lie algebra of $SO(2,1)$ instead of $\R$. We construct these measures by analyzing the conservation laws coming from the symmetries of the target and extend the support argument  to show that $dv$ has support on the canonical lamination.
  
  There are two points that we entirely missed in this paper which we will explore in \cite{daskal-uhlen2} and \cite{daskal-uhlen3}. The first is the role of symmetries of the domain manifold. Much like $dv$, there exist  Radon measures associated to  best Lipschitz maps corresponding to the symmetries of the domain. Again the support of this measure is on the canonical lamination. The second point is the interpretation of the cohomology class of $dv$ as well as its counterpart coming from the symmetries of the domain in terms of Thurston's earthquakes. 






%\section{An Extremal problem on the space of measured laminations} 
%In this section we specialize to the case where $(M, g)$ is a closed hyperbolic surface.
%Following \cite{thurston}, we define a functional $K$ associated to the hyperbolic structure $g$ and the representation $\rho$ defined  on the space of measured geodesic laminations on $(M,g)$. Our functional plays the analogous role of Thurston's functional $K$ associated to a pair of hyperbolic metrics $g, h$ on $M$ (cf. \cite{thurston}  middle of page 3; note that we have omitted the $\log$ from our definition of $K$, since we are not interested in defining a distance function in this paper). The main result of the section is Theorem~\ref{thmKL} where we show that our functional $K$  attains its maximum value  on the measured geodesic lamination $\Lambda_u=(\lambda_u, m)$ of maximum stretch of $u$ and is equal to the best Lipschitz constant $L$ of the homotopy class of maps determined by $\rho$. 
%
%\subsection{The functional K on simple closed curves}
%Fix $(M, g)$ a closed hyperbolic surface and  let  $u: M \rightarrow S^1$ be a best Lipschitz map with best Lipschitz constant
% $L=L_u$. Let $\tilde u: \tilde M \rightarrow \R$ 
%denote the lift to the universal cover equivariant under the representation
%$\rho: \Gamma \rightarrow \Z. $ Given $\gamma$ a simple closed curve in $M$ denote by $l_g(\gamma)$ the length of its geodesic representative. Equivalently $l_g(\gamma)$ is equal to the translation length of the element $\gamma \in \pi_1(M)$, i.e
%\[
%l_g(\gamma)= \inf_{z \in H^2}d_{H^2}(z, \gamma z).
%\]
%Let $\mathcal S$ denote the set of simple closed curves of $M$ and define the functional
%\begin{equation}\label{normlength}
%K: \mathcal S \rightarrow \R_{\geq 0}, \ \  K(\gamma)=  \frac{|\rho(\gamma)|}{l_g(\gamma)}.
%\end{equation}
%Set 
%\[
%K= \sup_{\gamma \in  \mathcal S} K(\gamma)
%\]
%and note that 
%\begin{equation}\label{ineq0}
%K \leq L_u.
%\end{equation} 
%Indeed, $\forall z \in H^2$, 
%\begin{eqnarray*}
%|\rho(\gamma)| 
%=   |u(z)- u(\gamma z)| 
%\leq  L_u d_{H^2}(z, \gamma z).
%\end{eqnarray*}
%Hence
%\[
%|\rho(\gamma)| \leq L_u l_g(\gamma), \  \ \forall \gamma \in \mathcal S
%\]
%which implies (\ref{ineq0}).  The functional $K$ extends in the obvious way on the set of multi-curves $\{ \gamma_i, t_i\}$ (i.e $\gamma_i$ is a simple closed curve and $t_i \in \R$) by setting
%\begin{equation}\label{Kmulticurve}
%K(\{ \gamma_i, t_i\})=  \frac{\sum_i |t_i | |\rho(\gamma_i)|}{\sum_i  |t_i |l_g(\gamma_i)}
%\end{equation}
%and clearly by (\ref{ineq0})
%\begin{equation}\label{ineq}
%K(\{ \gamma_i, t_i\}) \leq L_u.
%\end{equation}
%
%
%%\begin{itemize}
%%\item $\lambda \subset \cup_{i=1}^N F_i(R_i)$
%%\item $int(F_i(R_i)) \cap int(F_j(R_j))=\emptyset$ if $i \neq j$.
%%\end{itemize}
%%We define the length of the lamination 
%%\begin{equation}\label{length1}
%%l_g(\Lambda)=\sum_{i=1}^N l_g(\Lambda \cap F_i(R_i))
%%\end{equation}
%%where for $F_i: R_i=[a_i,b_i]\times [c_i,d_i] \rightarrow M$ we set
%%\begin{eqnarray}\label{length2}
%%l_g(\Lambda \cap F_i(R_i))&=&\int_{R_i}\left|\frac{\partial F_i(s,t)}{\partial s}\right|_gdsdm(t) \nonumber \\
%%&=&\int_{K_i}l_g(L_t)dm(t)
%%\end{eqnarray}
%%where $K_i \subset  [c_i,d_i] $ is the local leaf space of $\lambda$ and $L_t$ denotes the local leaf of $\lambda$ corresponding to $t \in K_i$.
%\subsection{DeRham currents associated to oriented measured laminations}  
%
%Following \cite{sullivan}, we first note  that an oriented geodesic lamination defines a closed current as follows
% \begin{definition}\label{integration}Let $\Lambda=(\lambda, m)$ be an oriented geodesic lamination. Define an 1-current
% $T_\Lambda$  by
% \[
% T_\Lambda(\omega)=\sum_i \int_{(c_i,d_i)}\left(\int_{[a_i,b_i] \times \{ t\}}F_i^*(\zeta_i \omega) \right)dm(t)
% \]
% where $\zeta_i$ is a partition of unity subordinate to  a cover of $M$ by flow boxes
% $F_i: R_i=[a_i,b_i] \times [c_i,d_i] \rightarrow M$.
% \end{definition}
% The current $T_{\Lambda} $ is closed (cf. \cite{sullivan}) and it defines an 
% element
%%$[T_{\Lambda}] \in H_1( U, \R)$,
%%which we can push via the sequence of maps
%% $ H_1(U, \R)\rightarrow H_1(M, \R)$ 
%% to define an element
% \[
% [\Lambda]:=[T_{\Lambda}] \in H_1(M, \R).
% \]
%% Notice that the homology class $[\Lambda]$ depends on a choice of  orientation of $\Lambda_o$ hence it is only well defined up to sign.
%
%
%
%
%
%%Given a minimal measured lamination $\Lambda=(\lambda, m)$ (and a choice of an orientation of
%%$\lambda$ or its orientation double cover),  let 
%% $[\Lambda] \in H_1(M, \R)$ denote its associated homology class.
% Given a  representation $\rho:\pi_1(M) \rightarrow \Z$, 
%since $\Z$ is abelian,  $\rho$ factors as
% \[\begin{tikzcd}
%\pi_1(M)  \arrow{d} \arrow{rd}{\rho}  \\
%H_1(M, \Z) \arrow{r}{ \rho} & \Z
%\end{tikzcd}
%\]
%and extends uniquely to
%\[
% \rho_\R: H_1(M, \R) \rightarrow \R.
%\]
% Thus given an oriented geodesic lamination $\Lambda$ we can associate a well defined number
% \[
%\rho_\R( [\Lambda]) \in \R.
% \]
%%  Notice that the absolute value is needed, because we want the definition to be independent of the orientation.
%% The construction can be repeated for every minimal component of a lamination, and define
%% \[
%% |\rho_\R( [\Lambda])|=\sum_i |\rho_\R( [\Lambda_i])|
%% \] 
%% where $\Lambda_i$ are the minimal components of $\Lambda$.
% 
%Next,  recall that given a measured lamination $\Lambda=(\lambda, m)$ on $(M,g)$ we define its length as follows.
% First, choose a finite number of flow boxes $\{F_i: R_i=[a_i,b_i]\times [c_i,d_i] \rightarrow M\}$ whose interiors cover  $\lambda$
%and for a partition of unity $\zeta_i$ subordinate to $\{F_i(R_i)\}$,  define
%\begin{eqnarray}\label{length2}
%l_g(\Lambda )=\sum_i \int_{(c_i,d_i)}\left(\int_{[a_i,b_i] \times \{ t\}} F_i^*(\zeta_i) \left|\frac{\partial F_i(s,t)}{\partial s}\right|_gds\right)dm(t). 
%\end{eqnarray}
%
%
%%  \[
%% l_g(m)=\int_\lambda dl_gdm
%% \]
%%meaning that, locally, we first integrate the length measure along the leaves of $\lambda$ and then integrate the corresponding local function on the space of leaves of $\lambda$ with respect to the transverse measure $m$. 
%%We have 
%\subsection{Extension of K to oriented measured laminations}  
%
%We first recall the notion of minimal laminations
%\begin{definition} \label{def: minlam} A  geodesic lamination $\lambda$ is called {\it{minimal}}, if it is minimal with respect to inclusion.\end{definition}
%
%\begin{theorem}[\cite{casson} Theorem 4.7 and Corollary 4.7.2] A geodesic lamination is minimal iff each leaf is dense. Any geodesic lamination is the union of finitely many minimal sub-laminations and of finitely many infinite isolated leaves, whose ends spiral along the minimal sub-laminations.
%\end{theorem}
%
%\begin{definition} \label{def: Klam} Let  $\Lambda$ be an oriented geodesic lamination and $\Lambda_i$ its minimal components. We define
%\[
%K(\Lambda)=\frac{\sum_i|\rho_\R([\Lambda_i])|}{\sum_il_g(\Lambda_i)}
%\]
%Notice that the definition is consistent with (\ref{Kmulticurve}).
%\end{definition}
%
%By going to the orientation double cover, we can extend $K$ to all laminations, not necessarily oriented ones. Indeed, let $\Lambda=(\lambda, m)$ be a geodesic measured lamination which we may assume, without loss of generality, that it is a union of finitely many minimal sublaminations. There exists a double cover $M_o \rightarrow M$  
%and a geodesic measured lamination $\Lambda_o=(\lambda_o, m_o)$ on $M_o$ that is the orientation double cover of $\Lambda=(\lambda, m)$. By composing with the map $M_o \rightarrow M$, we can also extend
% $\rho_\R: H_1(M, \R) \rightarrow \R$ to $\rho_\R: H_1(M_o, \R) \rightarrow \R$ and make the following
%
%%We can thus push the homology class $[\Lambda_o] \in H_1(M_o, \R)$ to 
%%$H_1(M, \R)$, to define a homology class $[\Lambda] \in H_1(M, \R)$.
%% By going to the orientation double cover, we can extend $K$ to all laminations, not necessarily oriented laminations. Indeed, let $\Lambda=(\lambda, m)$ be a lamination and let $\Lambda_i=(\lambda_i, m_i)$
%%be a non-orientable minimal component. Let  $\lambda_{i,o} \rightarrow \lambda_i$ be the orientation double cover of $\Lambda_i=(\lambda_i, m_i)$.  There exists an open neighborhood $U_i$ of $\lambda_i$ and a double cover $  U_{i,o} \rightarrow U_i$ that extends the covering $  \lambda_{i,o} \rightarrow \lambda_i$ and  pulling back the measure $m_i$ we obtain  an orientable  measured geodesic lamination  $ \Lambda_{i,o}=(\lambda_{i,o},  m_{i,o})$ in $U_o$.   We can take the image of the homology class $1/2[\Lambda_{i,o}] \in H_1(U_o, \R)$ to $H_1(M, \R)$ via the sequence of maps $U_{i,o} \rightarrow U_i \subset M$ to define a homology class $[\Lambda_i] \in H_1(M, \R)$.
%%If $\Lambda$ is orientable, we set $\Lambda_o=\Lambda$.
%%If $ \Lambda$ is not minimal, then we define $ \Lambda_o$ by replacing every We can now use the same formula to define $K$, by replacing each $\Lambda_i$ with $\Lambda_{i,o}$. 
%\begin{definition}\label{nonoriented}For a general measured lamination $\Lambda$ (not necessarily orientable) we define 
%\[
%K(\Lambda)=K(\Lambda_o)
%\]
%where $\Lambda_o$ is the orientation double cover of $\Lambda$.
%\end{definition}
%
%Recall that the space $\mathcal M \mathcal L$ is equipped with the weak$^*$ topology of measures defined as follows (cf. \cite{thurston2}, Section 8.10)
%\begin{definition} \label{weakconvmeas} A sequence of geodesic laminations $\Lambda^n=(\lambda^n, m^n)$ converges to a geodesic lamination $\Lambda=(\lambda, m)$ in  the space $\mathcal M \mathcal L$, if
%\begin{itemize}
%\item The supports $\lambda^n \rightarrow \lambda$ geometrically. This means that, given any neighborhood of a point in $\lambda$, the direction fields of the geodesics in $\lambda^n $ converge to the corresponding direction fields of the geodesics of $\lambda$ (cf. \cite{thurston2}, Section 8.5).
%\item Given $F: R=[a,b] \times [c,d] \rightarrow M$  a flow box of $\Lambda$, there exist  flow boxes $F^n: R \rightarrow M$ for $\Lambda^n$ that converge uniformly to the flow box $F: R \rightarrow M$ (this is possible by the convergence of $\lambda^n \rightarrow \lambda$ geometrically) such that the measures $m^n$ on $\{a\} \times (c,d)$ converge in the weak$^*$ topology to the measure $m$.
%\end{itemize}
%\end{definition}
%%Let
%%$\Lambda=(\lambda, m)$ be a measured geodesic lamination. For $\epsilon>0$,  a finite collection of transversals  $ f_i: J_i \rightarrow M $ as in Definition~\ref{kkdef1} and  a finite number of smooth functions  $ g \in \mathcal F_i \subset C^\infty (J_i)$, we define
%%$\mathcal N_{\epsilon, \{J_i\}, \{\mathcal F_i \}}(\Lambda)$
%%to be the set of measured geodesic laminations $\Lambda'=(\lambda', m')$ such that $\lambda'$ is transverse to $\{J_i \}$ and
%%\[
%%\left| \int_{J_i}g(t)dm(t)-  \int_{J_i}g(t)dm'(t)\right| <\epsilon
%%\]
%%for all $J_i $ and all $g \in \mathcal F_i$. In the above formula we denote by $m$ also the pullback of the transverse measure on $J_i$ as in Definition~\ref{transdef}. The weak$^*$ topology on $ \mathcal M \mathcal L$ is defined as the topology generated by the sets $\mathcal N_{\epsilon, \{J_i\}, \{\mathcal F_i \}}(\Lambda)$. (cf. \cite{thurston2}, Section 8.10).
%\begin{proposition}\label{functionalcont}Fix a representation $\rho: \pi_1(M) \rightarrow \Z$. Then, 
%\[
%K: \mathcal M \mathcal L \rightarrow \R_{\geq 0}
%\]
%is continuous with respect to the weak$^*$ topology on $ \mathcal M \mathcal L$.
%Furthermore,
%\[
%K(\Lambda)\leq L
%\]
%where $L=L_u$ is the best Lipschitz constant in the homotopy class of maps defined by $\rho$.
%\end{proposition}
%%\begin{proposition}\label{functionalcont2} The functional 
%%\[
%%K: \mathcal M \mathcal L \rightarrow \R_{\geq 0}; \  \ \Lambda \mapsto K(\Lambda).
%%\]
%%is continuous. 
%%\end{proposition}
%%{\bf{We don't need the next, but if we want we can show that for the piecewise smooth structure on $\mathcal M \mathcal L$, $K$ is piecewise smooth. We can even compute its derivative and its critical points maybe even the second variation.}}
%\begin{proof}
%To show that $K$ is continuous, let $\Lambda^n=(\lambda^n, m^n) \rightarrow  \Lambda=(\lambda, m)$ in $\mathcal M \mathcal L$. We first are going to show that $[\Lambda^n] = [T_{\Lambda^n}]\rightarrow [T_\Lambda]=[\Lambda]$ in $H_1(M,\R)$, hence also $\rho_\R([\Lambda^n]) \rightarrow \rho_\R([\Lambda])$. By going to a cover, if necessary, we can assume  that $\Lambda^n$ and $\Lambda$ are orientable. It suffices to show that, given $F: R=[a,b] \times [c,d] \rightarrow M$  a flow box of $\Lambda$,  $T_{\Lambda^n} \rightarrow T_\Lambda$ on 1-forms supported in $F(R)$. 
%Let $F^n: R \rightarrow M$ be flow boxes for $\Lambda^n$ as in Definition~\ref{weakconvmeas} that converge uniformly to the flow box $F: R \rightarrow M$. Recalling also Definition~\ref{integration}, for  an 1-form $\omega$ supported in $F(R)$,   
%\begin{eqnarray*}
% T_{\Lambda^n}(\omega)&= &\int_{(c,d)}\left(\int_{[a,b] \times \{ t\}}{F^n}^*( \omega) \right)dm^n(t)\\
% &\rightarrow& \int_{(c,d)}\left(\int_{[a,b] \times \{ t \}}F^*( \omega) \right)dm(t)
%\end{eqnarray*}
%as claimed. Similarly,
%by (\ref{length2}),  $l_g(\Lambda_n) \rightarrow  l_g(\Lambda)$. Hence $K$ is continuous. Since multicurves are dense in $\mathcal M \mathcal L$ (cf. \cite{thurston2}, Proposition 8.10.7), the desired inequality follows from (\ref{ineq}).
%\end{proof}
%
%
% 
% \subsection{The lamination $\Lambda_u=(\lambda_u, m)$ maximizes K}  
%Recall the closed current $T_{\Lambda_u}$ associated to the geodesic lamination $\Lambda_u$ described in the previous section. We first are going to show
%\begin{theorem} \label{equalcurrents}As currents
%\[
%T_{\Lambda_u}=dv.
%\]
%\end{theorem}
%\begin{proof}It suffices to show that both currents agree on functions supported in the interior of  flow box $\{F: R=[a,b]\times [c,d] \rightarrow M\}$ where
%$F(u,t)=(u, F_u(t))$. Let $\psi=f_1 du +f_2 *du$ be such a form.
%First notice that, by Corollary~\ref{currentsupportfiber}, we have  $dv(f_2 *du)=f_2*du \wedge dv=0$ and similarly by Definition~\ref{integration},
% \begin{eqnarray*}
% T_{\Lambda_u}(f_2 *du)&=& \int_{(c,d)}\left(\int_{[a,b] \times \{ t\}}F^*(f_2) *du \right)dm(t)\\
% &=& 0
%  \end{eqnarray*}
%  because $*du \left(\frac{\partial}{\partial u}\right)=0$.
%Now,
%  \begin{eqnarray*}
% T_{\Lambda_u}(f_1 du)&=& \int_{(c,d)}\left(\int_{[a,b] \times \{ t\}} F^*(f_1) du \right)dm(t)\\
% &=& \int_{F(R)}f_1 du \wedge dv  \ \mbox{(by Proposition~\ref{formdudm})}\\
% &=& dv(f_1 du).
%  \end{eqnarray*}
%  This proves the theorem.
%\end{proof}
%
%We next compute the length of $\Lambda_u$.
% \begin{lemma} \label{lamma:lengthlam}The length of the maximum stretch lamination $\Lambda_u=(\lambda_u, m)$ is given by
% \[
%  l_g(\Lambda_u)=L^{-1}
%  \]
%  where $L$ is the best Lipschitz constant. 
% \end{lemma}
% \begin{proof}
% By (\ref{length2}), if 
% $F: R=[a,b]\times [c,d] \rightarrow M$ is a flow box such that $F(u,t)=(u, F_u(t))$, then
%\begin{eqnarray*}
%l_g(\Lambda_u \cap F(R))&=&\int_{R}\left|\frac{\partial F(u,t)}{\partial u}\right|_gdudm(t)\\
%&=&\int_{R}\left|\frac{\partial }{\partial u}\right|_gdudm\\
%&=&L^{-1}\int_{R}dudm\\
%&=&L^{-1}\int_{R}dudv \ (\mbox{by Proposition~\ref{formdudm})}.
%\end{eqnarray*} 
%Hence, for a partition of unity $\zeta_i$ subordinate to a cover $\{F_i(R_i)\}$ as in (\ref{length2}), we obtain by  (\ref{normdudv1}),
%\[
%l_g(\Lambda_u)=L^{-1}\int_{M}(\sum_i \zeta_i)dudv=L^{-1}.
%\]
%\end{proof}
%
%
%
%We are now ready to prove the  main result of the section
%\begin{theorem}\label{thmKL} Fix a representation $\rho: \pi_1(M) \rightarrow \Z$. Then, 
%\[
%\sup_{\Lambda \in \mathcal M \mathcal L}K(\Lambda)=L
%\]
%where $L$ is the best Lipschitz constant in the homotopy class of maps defined by $\rho$. Moreover, the supremum is attained  by the measured lamination $\Lambda_u=(\lambda_u,m)$. 
%\end{theorem}
%
%\begin{proof}
%We will first compute explicitly $\rho_\R([\Lambda_u])$ in terms of a basis of $H_1(M, \R)$ and the representation $\alpha$ as follows:
%Given $A \in H_1(M,Z)$ let $\omega_A=PD(A) \in \Omega^1(M)$ denote the Poincare dual closed form satisfying (cf. \cite{gh} p.56 and p.59)
%  \[
%  \int_B \omega_A= \sharp(A, B)=\int_M \omega_A \wedge \omega_B.
%  \]
%  Since $dv$ is the weak limit of $dv_{q_j}$
%  \begin{eqnarray}
%  < dv, \omega_A >&=& \lim_j \int_M dv_{q_j} \wedge \omega_A \nonumber \\
% &=& \lim_j \int_A dv_{q_j}  \nonumber\\
% &=& \lim_j \alpha_{q_j}(A) \\
% &=&  \alpha(A).\nonumber
%   \end{eqnarray}
% Fix a basis $\{A_i \}_{i=1}^{2g}$  of $H_1(M,Z)$, define 
% \[
% c_i= \alpha(A_i)
% \]
% and rearrange   the basis so that $c_i \geq 0$. We thus have that
% \[
% [dv]=\sum_{i=1}^{2g}c_i A_i
% \]
% and hence
% \[
% \rho_\R([\Lambda_u])=  \rho_\R( [dv])=\sum_{i=1}^{2g}c_i \rho(A_i).
% \]
% We now claim
% \begin{equation}\label{prodrep}
% \rho_\R([\Lambda_u])=  \sum_{i=1}^{2g}c_i \rho(A_i)=\int_M dudv=1.
% \end{equation}
%  Let $\digamma \subset \tilde M$ denote a fundamental domain. We can view $\digamma$ as a  $4g$-gon with boundary components $\{\partial \digamma_i^{\pm} \}_{i=1}^{2g}$. We assume that opposite sides 
% $\partial \digamma_i^+$ and $\partial \digamma_i^-$ are identified via the deck transformation corresponding to $A_i$.
% First note, that if $z^+\in \partial \digamma_i^+$ is identified with $z^-\in \partial \digamma_i^-$ via $A_i$, then
%  \[
%  u_{p_j}(z^+)-u_{p_j}(z^-)=\rho(A_i).
%  \]
% Thus,
%  \begin{eqnarray*}
%1&=& \int_M du_{p_j}\wedge dv_{q_j}=\sum_{i=1}^{2g} \int_{\digamma_i^+}u_{p_j} dv_{q_j}-\int_{\digamma_i^-}u_{p_j}dv_{q_j} \\
%&=& \sum_{i=1}^{2g}\rho(A_i)\int_{\digamma_i}dv_{q_j} \\
%&=& \sum_{i=1}^{2g} \rho(A_i)  \alpha_{q_j}(A_i).
%\end{eqnarray*}
%By taking $q_j \rightarrow 1$, (\ref{prodrep}) follows.
%
%To complete the proof of the theorem, first notice that by Proposition~\ref{functionalcont}, the supremum  is less or equal to $L$.
%The rest follows from
%Lemma~\ref{lamma:lengthlam} and (\ref{prodrep}).
% \end{proof}
% 
%%\section{Maps of least gradient}
%%Recall the notion of equivariant functions of bounded variation  from Section~\ref{qgoesto1}.
%%Let $\tilde s: \tilde M \rightarrow \R$ be a function  in $BV_{loc}(\tilde M)$ and $\alpha$-equivariant, i.e for a.e $x \in \tilde M$ and every $\gamma \in \pi_1(M)$
%%\[
%%\tilde s(\gamma x)=\tilde s( x)+ \alpha(\gamma).
%%\]
%%Let $E_\alpha=\tilde M \times_\alpha \R$ denote  the associated flat line bundle and $s$ the  $L^1$ section  induced by $\tilde s$.  
%%We will call $s$ a {\it{section of bounded variation}} and denote it by $s \in \Gamma_{BV}(E)$. If $U \subset M$ is an open set contained in a coordinate chart $E_\alpha$, we define 
%%%for By choosing an open cover $\{U_i\} $ of $M$ by local trivializations of $E_\alpha$ and a partition of unity $\{ \zeta_i \}$ subordinate to the cover, we define 
%%% the norm of $s$ 
%%\[
%%||ds||_U=\sup_{\phi \in  \mathcal D^1(U), |\phi|_{L^\infty} \leq 1}  \int_U  s d\phi.
%%\]
%%\begin{definition}$u \in \Gamma_{BV}(E)$ is called {\it{a section of least gradient}} or equivalently $\tilde u$ {\it{a function of least gradient among $\alpha$-equivariant maps of locally bounded variation, }} if 
%%\[
%%||du|| \leq ||ds|| \ \ \forall s \in \Gamma_{BV}(E).
%%\]
%%\end{definition}
%%
%%The main theorem of the section is
%%\begin{theorem} \label{thm:mingrad} The section $v \in \Gamma_{BV}(E)$ associated to the map $\tilde v \in  BV_{loc}(\tilde M)$  is a section of least gradient.  
%%\end{theorem}
%%
%%Before we proceed we will make the following  
%%\begin{definition} Fix a representation $\alpha: \pi_1(M) \rightarrow \R$, a sequence  $\epsilon_j>0$  such that $ \epsilon_j \rightarrow 0$ and a sequence  $q_j>1$  such that $ q_j \rightarrow 1$. 
%%%and $\alpha_j:\pi_1(M) \rightarrow \R $ a sequence of representations such that $\alpha_j \rightarrow \alpha$. 
%%A sequence of sections $ w_{q_j} \in W^{1,q_j}(E_{\alpha})$  (or equivalently $\alpha$-equivariant functions $\tilde w_{q_j}$) such that 
%%\[
%%|d w_{q_j}|_{L^{q_j}} \leq (1+\epsilon_j)| d \hat w_{q_j}|_{L^{q_j}}
%%\]
%%where $\hat w_{q_j}$ is a $q_j$-harmonic section of $E_{\alpha}$ is called an {\it{$\epsilon_j$-sequence of almost  $q_j$-harmonic sections (or equivalently an $\epsilon_j$-sequence of almost  $q_j$-harmonic $\alpha$-equivariant functions).}}
%%\end{definition}
%%
%%\begin{lemma}\label{lemma:prepare1} Let $\tilde w_{q_j}: \tilde M \rightarrow \R$ be an $\epsilon_j$-sequence  of almost  $q_j$-harmonic $\alpha$-equivariant functions such that  $\tilde w_{q_j}  \xrightharpoonup{BV_{loc}(\tilde M)} \tilde w$ as $q_j \rightarrow 1$. Then  $\tilde w$ is a map of least gradient. 
%%\end{lemma}
%%\begin{proof}
%%Let $w_{q_j}$ be the  section of the flat line bundle $E_\alpha=\tilde M \times_\alpha \R$  associated with $\tilde w_{q_j}$.  Let $u$ be another section of $E_\alpha$  of bounded variation and let
%%$u_\delta$ denote its mollification. This is defined as follows: With respect to the (finite) trivializing cover $\{U_i \simeq B_1(0) \}$ for $E_\alpha$, a partition of unity $\zeta_i$ subordinate to the cover and $f \in C^\infty_c(B_1(0))$, define
%%\[
%%u_\delta=\sum_i (\zeta_i u)*f_\delta 
%%\] 
%%Then
%%\begin{eqnarray*}
%%||dw||&\leq &\liminf_{q_j \rightarrow 1}||dw_{q_j}|| \ \mbox{(by semicontinuity)}\\
%%&= &\liminf_{q_j \rightarrow 1}|dw_{q_j}|_{L^1}\\
%%&\leq &\liminf_{q_j \rightarrow 1}|dw_{q_j}|_{L^{q_j}} \ \mbox{(by H\"older and the fact $p_j \rightarrow \infty$)}\\
%%&\leq &\liminf_{q_j \rightarrow 1} (1+\epsilon_j) |du_\delta|_{L^{q_j}} \ \mbox{(since  $w_{q_j}$ is an almost $\epsilon_j$-minimizer)}\\
%%&= &|du_\delta|_{L^1} \ \mbox{(since   $u_\delta$ is smooth and $\epsilon_j \rightarrow 0$.)}\\
%%\end{eqnarray*}
%%By taking $\delta \rightarrow 0$,
%%we obtain $||dw||\leq ||du||$ as desired.
%%\end{proof}
%%
%%\begin{thm:mingrad}
%%Recall from Theorem~\ref{lemma:limmeasures2} that there exists a sequence of representations $\alpha_j \rightarrow \alpha$ and a sequence $ \{v_{q_j} \}$ of   $q_j$-harmonic sections of $E_{\alpha_j}$ such that  such that $ v_{q_j} \rightarrow  v$ in $BV( M)$ as $q_j \rightarrow 1$.
%%The  representations $\alpha_j \rightarrow \alpha$ define flat line bundles $E_{\alpha_j}= \tilde M \times_{\alpha_j} \R$ over the closed surface $M$ that are all topologically trivial, together with smooth  bundle isomorphisms
%%\[
%%\kappa_j: E_{\alpha_j}= \tilde M \times_{\alpha_j} \R \rightarrow E_\alpha= \tilde M \times_{\alpha} \R.
%%\]
%%%defined by sending the equivalence class $[\tilde x,t]_{\alpha_j} \in \tilde M \times_{\alpha_j} \R \rightarrow E$ to $[\tilde x,t]_{\alpha} \in \tilde M \times_{\alpha} \R$. 
%%%An equivalent way of defining the isomorphism (\ref{defkappa}) is as follows. The section of the trivial line bundle on $\tilde M$ which is identically equal to one, descends to a flat section $s_{\alpha_j}$ of the line bundle 
%%%$E_{\alpha_j}$ and similarly to a flat section $s_\alpha$ of $E_\alpha$. The isomorphism $\kappa_j$ has the property 
%%%\begin{equation}\label{defkappa2}
%%%\kappa_j \circ s_{\alpha_j}=s_\alpha.
%%%\end{equation}  
%%%Define the section of $E_\alpha$ given by
%%%\[
%%% w_{q_j}=\kappa_j \circ v_{q_j}.
%%%\]
%%By endowing $E_{\alpha_j}$ and $E_{\alpha}$ with the natural Riemannian metrics induced from the product metric on $\tilde M \times \R$, the convergence $\alpha_j \rightarrow \alpha$
%%implies that 
%%\begin{equation}\label{estimatekappa}
%%|\kappa_j|_{C^1}, |\kappa_j^{-1}|_{C^1}= 1+\epsilon_j; \ \epsilon_j \rightarrow 0.
%%\end{equation}
%%
%%We claim that $w_{q_j}$ is an $\epsilon_j$-sequence of almost  $q_j$-harmonic sections.
%%Let $ u_{q_j}$ be a $q_j$-harmonic  section of $E_\alpha$ and let 
%%\[
%%\hat u_{q_j}= \kappa_j^{-1} \circ u_{q_j}.
%%\]
%%Then
%%\begin{eqnarray}\label{estwjj}
%% |dw_{q_j}|_{L^{q_j}} &\leq& (1+\epsilon_j)|dv_{q_j}|_{L^{q_j}} \ \mbox{(by (\ref{estimatekappa}))} \nonumber\\
%% &\leq& (1+\epsilon_j)|d \hat u_{q_j}|_{L^{q_j}} \ \mbox{(since $v_{q_j}$ is a minimizer)}\\
%% &\leq& (1+\epsilon_j)^2|du_{q_j}|_{L^{q_j}} \ \mbox{(by (\ref{estimatekappa}))}  \nonumber 
%%\end{eqnarray} 
%%Thus, $w_{q_j}$ is an $\epsilon_j$-sequence  of almost  $q_j$-harmonic sections of $E_\alpha$ as claimed. Estimate (\ref{estwjj}) implies as before that after passing to a subsequence,
%%\[
%%w_{q_j} \rightarrow w \ \mbox{in} \ BV(M)
%%\]
%% and Lemma~\ref{lemma:prepare1}
%%implies that $w$ is a section of least gradient of $E_\alpha$. 
%%
%%We are left to show $v=w$. With respect to the trivializations $s_{\alpha_j}$,  $s_\alpha$, we write
%%\[
%%v_{q_j}=\hat v_{q_j}s_{\alpha_j}, \ v=\hat v s_\alpha, \ w=\hat w s_\alpha.
%%\] 
%%By definition,
%%\[
%%w_j=\kappa_j \circ v_{q_j}=\hat v_{q_j}s_\alpha.
%%\]
%%The convergence in $L^1$ of $v_{q_j}$ to $v$, is equivalent to the convergence of $\hat v_{q_j}$ to $\hat v$ and 
%%by the formula above, it implies the convergence of $w_{q_j}$ to $v$. Hence $v=w$.
%%%Fix a $x \in M$ and let $\tilde x \in  int(F) \subset\tilde M$ be a lift, where $int(F)$ denotes the interior of a fundamental domain for the action of $\pi_1(M)$. Since $\alpha_j \rightarrow \alpha$
%%%there exists an open set $W \subset  int(F)$ such that $v_j(W)$ is contained in the intersection of the fundamental domains for the action of $\alpha_j(\pi_1(M))$ and $\alpha(\pi_1(M))$ for $j$ sufficiently large.
%%%Thus for $\tilde z \in W$ we can uniquely lift $v_j(z)=[\tilde z, \tilde v_j(\tilde z)]_{\alpha_j} \rightarrow [\tilde z, \tilde w(\tilde z)]_{\alpha_j}$ to $ \tilde v_j(\tilde z)  \rightarrow \tilde v(\tilde z)$ and $w_j(z)=[\tilde z, \tilde v_j(\tilde z)]_\alpha \rightarrow [\tilde z, \tilde w(\tilde z)]_\alpha$ to $ \tilde v_j(\tilde z)  \rightarrow \tilde w(\tilde z)$.
%%%Thus $\tilde v=\tilde w$ in $W$, and since $W$ is a neighborhood of an arbitrary point $\tilde x \in \tilde M$ it must be $v=w$.
%%\end{thm:mingrad}
%%
%%
%%
%%
%%\begin{theorem} \label{prop:mingrad} The map $v \in  BV_{loc}(\tilde M)$  is a minimizer of the gradient among all maps in $BV_{loc}(\tilde M)$ which are equivariant under $\alpha$. 
%%\end{theorem}
%%\begin{proof} 
%%By Lemma~\ref{prepare} we may assume that we the we have the sequence of $q_j$-harmonic maps $v_j$ is equivariant with respect to the same representation $\alpha$.
%%  \end{proof}
%%
%%
%%
%%
%%
%%\begin{lemma}\label{lemma:step2} Let $\alpha_j :\pi_1(M) \rightarrow \R$ be a sequence or representations that converges to a representation $\alpha :\pi_1(M) \rightarrow \R$. Let $\tilde v_j$ and  $\tilde w_j$ be as in Lemma~\ref{lemma:step1}. Then, there exists a subsequence (call it again $\{j\}$) such that
%%\[
%%\tilde w_j  \xrightharpoonup{BV_{loc}(\tilde M)} \tilde v.
%%\]
%%\end{lemma}
%%
%%
%%
%%
%%
%%
%%
%%
%%
%%
%%
%%
%%
%%Let  $q_j \rightarrow 1$ and $v_j=v_{q_j}$ be the $q_j$-harmonic map equivariant under the representation $\alpha_j$. Let 
%%$\bar v_j: M \rightarrow S^1 \subset \C$ be the map induced by $v_j$, clearly $\bar v_j$ are in $L^\infty$. The closed 1-forms $dv_j=d\bar v_j$ are H\"older continuous and uniformly in $L^1$ by (\ref{kappanorm}). Thus, as complex valued functions $\bar v_j \in W^{1,1}(M, \C)$ are uniformly bounded. Since $W^{1,1}(M, \C)$ is compactly contained in $L^s$ for $s<2$ we obtain a function $\bar v: M\rightarrow \C$  such that, after passing to a subsequence, 
%%\begin{equation}\label{ls}
%%\bar v_j \xrightarrow{L^s} \bar v.
%%\end{equation}
%%In particular $\bar v_j \rightarrow \bar v$ almost everywhere and thus 
%%\[
%%\bar v: M \rightarrow S^1.
%%\]
%%Furthermore, for $\phi \in  \Omega^1(M)$ a test function and $|\phi|_{L^\infty} \leq 1$, we have by (\ref{kappanorm}) that
%%\[
%% < d\bar v_j,  \phi >=  \int_M d\bar v_j \wedge \phi   \leq C.
%%\]
%%  By weak compactness (cf. \cite{simon} Lemma 2.15) there exists  $T \in  \mathcal D^1( M)$ such that
%%  \[
%%  d\bar v_j \rightharpoonup  T 
%%  \]
%%  and $T$ is closed, being the  distributional limit of closed forms. Finally, in a simply connected coordinate patch $U \subset M$ 
%%  \begin{eqnarray}\label{locexact}
%%  T \big|_U=d\bar v\big|_U.
%% \end{eqnarray}
%%  Indeed, for $\phi \in \mathcal D^1(U)$, 
%%  \begin{eqnarray}
%% < T,  \phi >&=& \lim_j < d\bar v_j,  \phi > \nonumber\\
%% &=& \lim_j \int_M d\bar v_j \wedge \phi \nonumber \\
%% &=& \lim_j \int_M \bar v_j \wedge d\phi \\
%% &=& \int_M \bar v  d\phi \ \ (\mbox{by} (\ref{ls}))\nonumber 
%%\end{eqnarray}
%% proving (\ref{locexact}).
%% 
%%
%%
%%
%%
%%
%%
%%  
%% 
%%
%% 
%%We first define 
%%
%%
%%
%%\begin{proposition} \label{cor:limmeasures2}
%%There exists a current $ d\tilde u \wedge dv \in \mathcal D_2(\tilde M)$
%%such that
%%\begin{equation}\label{conv2dist}
%%d\tilde u_j \wedge dv_j \rightharpoonup d\tilde u \wedge dv.
%%\end{equation}
%%\end{proposition}
%%\begin{proof}
%%We first  define $d\tilde u \wedge dv \in \mathcal D_2 (\tilde M)$. For $\omega \in D^1 (\tilde M)$, write
%%\[
%%dv(\omega)=\int_{\tilde M} <\omega, \vec{dv}>\mu_{dv}
%%\]
%%for a measurable function $\vec{dv}$ with values in $\Lambda^1(M)$ and a Radon measure $\mu_{dv}$ where
%%$|\vec{dv}|=1$ $\mu_{dv}$-a.e (cf. \cite{simon} Chapter 6, (2.7)).
%%For $f \in  \mathcal D^0 (\tilde M)$, define
%%\[
%%d \tilde u  \wedge dv(f)=\int_{\tilde M} <fd\tilde u, \vec{dv}>\mu_{dv}.
%%\]
%%Since $d\tilde u$ is continuous the above integral is finite and defines a continuous functional on $D^0 (\tilde M)$.
%%
%%We next claim:
%%\[
%%v_jd \tilde u_j \rightharpoonup vd  \tilde  u \ \ \mathcal D_1 (\tilde M).
%%\]
%%First note that $v_jd \tilde u_j$ and $vd  \tilde  u$ are distributions defined by functions in $L^1_{loc}$. For  $\xi \in \mathcal D^1(\tilde M)$ with support in a compact set $K \subset \tilde M$, $ s<2$ and  $ \frac{1}{s}+\frac{1}{t}=1$,
%%\begin{eqnarray*}
%%\int_{\tilde M} v_jd \tilde u_j \wedge \xi -vd  \tilde  u \wedge \xi &=& \int_K (v_jd  \tilde u_j-vd  \tilde u_j) \wedge \xi + \int_K (vd \tilde u_j-vd  \tilde u) \wedge \xi \\
%%&\leq& | v_j-v|_{L^s(K)}|d \tilde u_j|_{L^t(K)} | \xi |_{L^\infty(K)}\\ 
%%&+& \int_K (d \tilde u_j-d  \tilde u) \wedge v\xi \\
%%&\rightarrow& 0.
%%\end{eqnarray*}
%%We next claim
%%\[
%%d(v_jd \tilde u_j) \rightharpoonup d(vd  \tilde  u) \ \ \mathcal D_2 (\tilde M).
%%\]
%%Indeed, for $f \in  \mathcal D^0 (\tilde M)$,
%%\[
%%d (v_jd\tilde u_j) (f) =\int_{\tilde M} df \wedge (v_j d\tilde u_j) \rightarrow  \int_{\tilde M} df \wedge  v d\tilde u=d (vd\tilde u)(f)
%%\]
%%To prove (\ref{conv2dist}), chose  $f \in  \mathcal D^0 (\tilde M)$ and note
%% \begin{eqnarray*}
%%\int_{\tilde M} fd \tilde u_j\wedge dv_j  =\int_{\tilde M} df \wedge (v_j d\tilde u_j) \rightarrow  \int_{\tilde M} df \wedge  v d\tilde u 
%%\end{eqnarray*}
%%Thus,
%%\[
%%\omega=d(ydu)=dy \wedge du.
%%\]
%%\end{proof}
%%\begin{conjecture} As currents
%%\[
%%\omega=dx \wedge dy.
%%\]
%%\end{conjecture}
%%\begin{proof} 
%%We first claim:
%%\[
%%y_jdu_j \rightharpoonup ydu \ \mbox{weakly in} \ L^1.
%%\]
%%Indeed, for a smooth test function $\xi \in \Omega^1(M)$, $ s<2$ and  $ \frac{1}{s}+\frac{1}{t}=1$,
%%\begin{eqnarray*}
%%\int y_jdu_j \wedge \xi -ydu \wedge \xi &=& \int (y_jdu_j-ydu_j) \wedge \xi + \int (ydu_j-ydu) \wedge \xi \\
%%&\leq& | y_j-y|_{L^s}|du_j|_{L^t} | \xi |_{L^\infty}\\ 
%%&+& \int (du_j-du) \wedge y\xi \rightarrow 0.
%%\end{eqnarray*}
%%Moreover, for any smooth test function $f$,
%%\begin{eqnarray*}
%%\int fdy_j \wedge du_j  =\int fd(y_jdu_j) = \int   y_jdu_j \wedge df \rightarrow  \int ydu \wedge df 
%%\end{eqnarray*}
%%Thus,
%%\[
%%\omega=d(ydu)=dy \wedge du.
%%\]
%%\end{proof}
%%
%%
%\section{further exploring}
%\begin{itemize}
%\item Construction of measure $m$ as limit of geodesics in the spirit of \cite{gh}.
%\item Relation between $\alpha$ and H\"older distributions.
%\item Duality of $u_p$ and $v_q$ in the spirit of Fenchel and Bers.
%\end{itemize}

\begin{thebibliography}{ABC}
%%\bibitem[BJW]{barron} E. Barron, R. Jensen and C. Wang. {\it{The Euler Equation and Absolute Minimizers of $L^\infty$ Functionals}}.  Arch. Rational Mech. Anal. 157,  255-283 (2001).
%%\bibitem[Bo]{bonahon}  F. Bonahon. 
%%{\it Geodesic laminations on surfaces.}  Contemporary Mathematics Volume 269 (2001).
%\bibitem[Ar1]{aron1}G. Aronsson. {\it  Extension of functions satisfying Lipschitz conditions}. Ark. Mat. 6,  551-561, (1967).
%\bibitem[Ar2]{aron2}G. Aronsson. {\it On the partial differential equation $u^2_xu_{xx} +2u_xu_yu_{xy} +u^2_yu_{yy} =0$}. Ark. Mat. 7,  395-425, (1968).
%%\bibitem[BPT]{bpapatro}  A. Belkhirat, A. Papadopoulos and M. Troyanov. {\it Thurston's weak metric on the Teichm�ller space of the torus}. Trans. Amer. Math. Soc. 357, 3311-3324, (2005).
%%\bibitem [Fe]{federer} H. Federer. {\it Geometric Measure Theory.} Die Grundlagen der Math. Wissenschaften, Band 153, Springer-Verlag, Berlin and New York, (1969).
%\bibitem [NS]{naor} A. Naor and S. Sheffield. {\it Absolutely minimal Lipschitz extensions of tree-valued mappings.} Math. Ann. 354, 3 , 1049-1078, (2012).
%%\bibitem[Sch]{schoen} R. Schoen. {\it Analytic Aspects of the Harmonic Map Problem.}   Seminar on Nonlinear Partial Differential Equations, Mathematical Sciences Research Institute Publications, vol 2. Springer, New York, NY, 321-358 (1984).
%\bibitem[SS]{smart} S. Sheffield and C. Smart.
%{\it Vector-valued optimal Lipschitz extensions.} Communications on Pure and Applied Math. vol LXV 128-154 (2012). 

\bibitem[Al-S]{aless} G. Alessandrini and M. Sigalotti.
{\it Geometric properties of solutions to the anisotropic p-Laplace equation in dimension 2.} Annales Academiae Scientiarum Fennicae Mathematica Vol. 21, 249-266 (2001).
\bibitem[A-F-P]{ambrosio} L. Ambrosio, N. Fusco, D. Pallara. {\it{Functions of bounded variation and free discontinuity problems.}} Oxford Mathematical Monographs (2000).
\bibitem[Ar1]{aronson1} G. Aronsson. {\it On certain singular solutions of the partial differential equation
$u_x^2u_{xx}+2u_xu_yu_{xy}+u_y^2u_{yy}=0$.}
Manuscripta math. 47, 133-151 (1984).
 \bibitem[Ar2]{aronson2} G. Aronsson. {\it Constructon of  singular solutions to the $p$-harmonic  equation and its limit equation for $p=\infty$}.
 Manuscripta math. 56, 135-158 (1986).
 \bibitem[Ar-C-J]{arcrju} G. Aronsson, M. Crandal and P. Juutinen. {\it {A tour of the theory of absolutely minimizing functions.}}
Bulletin of the AMS., Vol. 41, No 4, 439-505 (2004).
 \bibitem[Ar-L]{aronsonlin} G. Aronsson and P. Lindqvist. {\it{On p-Harmonic Functions in the Plane and Their Stream Functions.}} Journal of Differential Equations, Vol. 74, Issue 1, 157-178 (1988).
%  \bibitem[B-B-S]{ballmann}W. Ballmann, M. Brin and R. Spatzier. {\it{ Structure of manifolds of nonpositive curvature II.}} Ann. of Math. 122, 205-235 (1985).
%\bibitem[B-S]{birman}J. Birman and  C. Series. {\it{Geodesics with bounded intersection numbers on surfaces are sparsely distributed.}} Topology 24, 217-225 (1985).
%\bibitem[B-dG-G]{bombieri}E. Bombieri, E. deGiorgi and E. Giusti. {it{ Minimal cones and the Bernstein problem.}} Invent.Math 7, 243-268 (1969).
\bibitem[Bo1]{bonahon1} F. Bonahon. {\it Geodesic laminations on surfaces.} Contemporary Mathematics Volume 269 (2001).
\bibitem[Bo2]{bonahon2} F. Bonahon. {\it Transverse H\"older distributions for geodesic laminations.} Topology Vol. 36, No. 1, 103-122  (1997).
%\bibitem[Bo3]{bonahon} F. Bonahon. {\it Shearing hyperbolic surfaces, bending pleated surfaces and Thurston's symplectic form.} 
%Annales de la faculte des sciences de Toulouse, tome 5, no 2, 233-297 (1996).
%\bibitem[Ca]{calegari} D. Calegari. {\it Foliations and the Geometry of 3-Manifolds.} 
%Oxford Mathematical Monographs. (2007)
\bibitem[Ca-Bl]{casson} A. Casson and S. Bleiler. {\it Automorphisms of surfaces after Nielsen and Thurston.} LMS students texts 9 (1988).
\bibitem[C]{crandal} M. Crandal.
{\it A Visit with the $\infty$-Laplace Equation.} Calculus of Variations and Nonlinear Partial Differential Equations, 75-122 (2008).
\bibitem[DU1]{daskal-uhlen2} G. Daskalopoulos and K. Uhlenbeck. {\it Analytic properties of Minimal Stretch maps and geodesic laminations.} Preprint.
\bibitem[DU2]{daskal-uhlen3} G. Daskalopoulos and K. Uhlenbeck. {\it Lie Algebra valued Transverse Measures and Earthquakes.} In preparation.
 \bibitem[D-M-R-V]{cantorexp}  O. Dovgosheya, O. Martiob, V. Ryazanovaand M. Vuorinen. {\it {The Cantor function.}} Expo. Math. 24,  1-37 (2006).
%\bibitem[doC]{docarmo} M. do Carmo.  {\it {Differential Geometry of Curves and Surfaces.}} Prentice-Hall (1976). 
\bibitem[Ek-Te]{temam}I. Ekeland and  R. Temam. {\it Convex Analysis and Variational Problems.} North-Holland, Amsterdam, (1976).
\bibitem[E-G]{evans} L. Evans and R. Gariepy. {\it{Measure theory and fine properties of functions.}} Studies in Advanced Mathematics, CRC Press (1992).
\bibitem[E-Sv]{evans-savin} L. Evans and O. Savin. {\it $C^{1, \alpha}$ regularity for infinity harmonic functions in two dimensions.}
Calculus of Variations 32(3), 325-347 (2008).
\bibitem[E-Sm]{evans-smart} C. Evans and C. Smart. 
{\it Everywhere differentiability of infinity harmonic functions.} Calculus of Variations and Partial Differential Equations
Volume 42, Issue 12, 289-299 (2011).
\bibitem[Fe]{fenchel} W. Fenchel. {\it On Conjugate Convex Functions.} Canad. J. Math. 1, 73-77, (1949).
\bibitem[Fo]{foster} O. Foster. {\it Lectures on Riemann Surfaces.} Graduate texts in mathematics, Springer-Verlag (1981).
%\bibitem[Giu]{giusti}  E. Giusti {\it Minimal Surfaces and Functions of Bounded Variation.}  Monographs in Mathematics, Vol. 80, Springer Science (1984).
%\bibitem[G-H]{gh} P. Griffiths and J. Harris. {\it Principles of Algebraic Geometry.}  John Willey (1978).
%\bibitem[GS]{gromov-schoen} M. Gromov and R. Schoen. {\it Harmonic maps into singular
%spaces and $p$-adic superrigidity for lattices in groups of rank
%one.} Publ. Math. IHES 76 (1992),  165-246.
%\bibitem[Gor]{gorny} W. Gorny. {\it Planar least gradient problem: existence, regularity and anisotropic case.} Calc. Var.  57:98,  (2018).
\bibitem[Gu-K]{kassel} F. Gueritaud and F. Kassel.
{\it Maximally stretched laminations on geometrically finite hyperbolic manifolds.} Geom. Topol. Volume 21, Number 2, 693-840 (2017).
\bibitem[Je]{jensen} R. Jensen. {\it Uniqueness of Lipschitz extension: minimizing the sup norm of the gradient.} Archive for Rational Mechanics and Analysis 123, 51-74 (1993).
\bibitem[Ju]{juutinen} P. Juutinen. {\it{p-harmonic approximation of functions of least gradient.}} Indiana  Math. Jour. Vol 54, 1015-1030 (2005).
%\bibitem[KS1]{korevaar-schoen1}  N. Korevaar and R. Schoen.  {\it
%Sobolev spaces and harmonic maps into metric space targets. }  Comm. Anal. Geom. 1 (1993) 561-659.
%\bibitem[KS2]{korevaar-schoen2}  N. Korevaar and R. Schoen.  {\it
%Global existence theorem for harmonic maps to non-locally compact spaces.}  Comm.  Anal. Geom. 5 (1997) 333-387.
\bibitem[L]{lindqvist} P. Lindqvist. {\it Notes on the Infinity Laplace Equation.} Springer Briefs in Mathematics (2016).
\bibitem[Man]{manfredi} J. Manfredi. {\it p-harmonic functions in the plane.} Proc. Amer. Math. Soc. 103(2), 473-479 (1988).
\bibitem[Mas]{masur} H. Masur. {\it Interval Exchange Transformations and Measured Foliations.} Ann. of Math. Vol. 115, No. 1,  169-200 (1982).
\bibitem[M-R-L]{mazon} J. Mazon, J. Rossi and S. Segura de Leon. {\it Functions of least gradient and 1-harmonic functions}. Indiana Math J. vol 63, no 4, 1067-1084 (2014).
\bibitem[N-S]{naor-sheffield} A. Naor and S. Sheffield. {\it Absolutely minimal Lipschitz extension of tree-valued mappings.} Math. Ann. vol. 354, 1049-1078 (2012).
\bibitem[Pa-Th]{papa} A. Papadopoulos and G. Theret. {\it On Teichm\"uller's metric and Thurston's asymmetric metric on Teichm\"uller space.}  Handbook of Teichm\"uller Theory, Volume 1, 11, European Math. Soc. Publishing House (2007).
\bibitem[Ru-S]{sullivan} D. Ruelle and D. Sullivan. {\it Currents, flows and diffeomorphisms}. Topology 14, 319-327 (1975).
\bibitem[Si]{simon} L. Simon. {\it Introduction to Geometric Measure Theory.} Tsinghua Lectures (2014).
\bibitem[S-W-Z]{ziemer} P. Sternberg, G. Williams and W. Ziemer. {\it Existence, uniqueness, and regularity for functions of least gradient.}  Jour. Reine Angew. Math. 430, 35-60 (1992).
\bibitem[S-Z]{ziemer2} P. Sternberg and W. Ziemer. {\it The Dirichlet problem for functions of least gradient.} Ni Wei-Ming et al (ed.) Degenerate diffusions. Proceedings of the IMA workshop held at the University of Minnesota, Springer-Verlag IMA Vol Math. Appl. 47.  197-214 (1993).
%\bibitem[S-W-Z]{ziemer3} P. Sternberg, G. Williams and W. Ziemer. {\it Existence, uniqueness and regularity for functions of least
%gradient.} J. Reine Angew. Math. 430, 35-60 (1992).
\bibitem[Thu1]{thurston} W. Thurston. {\it Minimal stretch maps between hyperbolic surfaces.} Preprint arXiv: math/9801039.
\bibitem[Thu2]{thurston2} W. Thurston. {\it The Geometry and Topology of Three-Manifolds.}  Electronic version  http://www.msri.org/publications/books/gt3m/ (2002).
\bibitem[U]{uhlen} K. Uhlenbeck. {\it {Regularity for a class of nonlinear elliptic systems.}} Acta Mathematica 138, 219-240 (1977).
%\bibitem[Wo]{wolfStr} M. Wolf. {\it On The Existence Of Jenkins--Strebel Differentials Using Harmonic Maps From Surfaces To Graphs.}
%Ann. Acad. Scient. Fen. Series A,Vol. 20,  269-278 (1995)
\end{thebibliography}

\end{document}