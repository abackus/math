\documentclass[reqno,11pt]{amsart}
\usepackage[letterpaper, margin=1in]{geometry}
\RequirePackage{amsmath,amssymb,amsthm,graphicx,mathrsfs,url,slashed,subcaption}
\RequirePackage[usenames,dvipsnames]{xcolor}
\RequirePackage[colorlinks=true,linkcolor=Red,citecolor=Green]{hyperref}
\RequirePackage{amsxtra}
\usepackage{cancel}
\usepackage{tikz-cd}
%\usepackage[T1]{fontenc}

% \setlength{\textheight}{9.3in} \setlength{\oddsidemargin}{-0.25in}
% \setlength{\evensidemargin}{-0.25in} \setlength{\textwidth}{7in}
% \setlength{\topmargin}{-0.25in} \setlength{\headheight}{0.18in}
% \setlength{\marginparwidth}{1.0in}
% \setlength{\abovedisplayskip}{0.2in}
% \setlength{\belowdisplayskip}{0.2in}
% \setlength{\parskip}{0.05in}
%\renewcommand{\baselinestretch}{1.05}

\title{Dupin's theorem in general dimension}
\author{Aidan Backus}
\address{Department of Mathematics, Brown University}
\email{aidan\_backus@brown.edu}
\date{\today}

\newcommand{\NN}{\mathbf{N}}
\newcommand{\ZZ}{\mathbf{Z}}
\newcommand{\QQ}{\mathbf{Q}}
\newcommand{\RR}{\mathbf{R}}
\newcommand{\CC}{\mathbf{C}}
\newcommand{\DD}{\mathbf{D}}
\newcommand{\PP}{\mathbf P}
\newcommand{\MM}{\mathbf M}
\newcommand{\II}{\mathbf I}
\newcommand{\Hyp}{\mathbf H}
\newcommand{\Sph}{\mathbf S}
\newcommand{\Group}{\mathbf G}
\newcommand{\GL}{\mathbf{GL}}
\newcommand{\Orth}{\mathbf{O}}
\newcommand{\SpOrth}{\mathbf{SO}}
\newcommand{\Ball}{\mathbf{B}}

\newcommand*\dif{\mathop{}\!\mathrm{d}}

\DeclareMathOperator{\card}{card}
\DeclareMathOperator{\dist}{dist}
\DeclareMathOperator{\id}{id}
\DeclareMathOperator{\supp}{supp}
\DeclareMathOperator{\Teich}{Teich}
\DeclareMathOperator{\tr}{tr}

\newcommand{\Leaves}{\mathscr L}
\newcommand{\Lagrange}{\mathcal L}
\newcommand{\Hypspace}{\mathscr H}

\newcommand{\Chain}{\underline C}

\newcommand{\Two}{\mathrm{I\!I}}

\newcommand{\normal}{\mathbf n}
\newcommand{\radial}{\mathbf r}
\newcommand{\evect}{\mathbf e}
\newcommand{\vol}{\mathrm{vol}}

\newcommand{\diam}{\mathrm{diam}}
\newcommand{\Ell}{\mathrm{Ell}}
\newcommand{\inj}{\mathrm{inj}}
\newcommand{\Lip}{\mathrm{Lip}}
\newcommand{\Riem}{\mathrm{Riem}}

\newcommand{\Min}{\mathrm{Min}}
\newcommand{\Max}{\mathrm{Max}}

\newcommand{\dfn}[1]{\emph{#1}\index{#1}}

\renewcommand{\Re}{\operatorname{Re}}
\renewcommand{\Im}{\operatorname{Im}}

\newcommand{\loc}{\mathrm{loc}}
\newcommand{\cpt}{\mathrm{cpt}}

\def\Japan#1{\left \langle #1 \right \rangle}

\newtheorem{theorem}{Theorem}
\newtheorem{badtheorem}[theorem]{``Theorem"}
\newtheorem{prop}[theorem]{Proposition}
\newtheorem{lemma}[theorem]{Lemma}
\newtheorem{sublemma}[theorem]{Sublemma}
\newtheorem{proposition}[theorem]{Proposition}
\newtheorem{corollary}[theorem]{Corollary}
\newtheorem{conjecture}[theorem]{Conjecture}
\newtheorem{axiom}[theorem]{Axiom}
\newtheorem{assumption}[theorem]{Assumption}

\newtheorem{mainthm}{Theorem}
\renewcommand{\themainthm}{\Alph{mainthm}}

\newtheorem{claim}{Claim}[theorem]
\renewcommand{\theclaim}{\thetheorem\Alph{claim}}
% \newtheorem*{claim}{Claim}

\theoremstyle{definition}
\newtheorem{definition}[theorem]{Definition}
\newtheorem{remark}[theorem]{Remark}
\newtheorem{example}[theorem]{Example}
\newtheorem{notation}[theorem]{Notation}

\newtheorem{exercise}[theorem]{Discussion topic}
\newtheorem{homework}[theorem]{Homework}
\newtheorem{problem}[theorem]{Problem}

\makeatletter
\newcommand{\proofpart}[2]{%
  \par
  \addvspace{\medskipamount}%
  \noindent\emph{Part #1: #2.}
}
\makeatother



% \numberwithin{equation}{section}


% Mean
\def\Xint#1{\mathchoice
{\XXint\displaystyle\textstyle{#1}}%
{\XXint\textstyle\scriptstyle{#1}}%
{\XXint\scriptstyle\scriptscriptstyle{#1}}%
{\XXint\scriptscriptstyle\scriptscriptstyle{#1}}%
\!\int}
\def\XXint#1#2#3{{\setbox0=\hbox{$#1{#2#3}{\int}$ }
\vcenter{\hbox{$#2#3$ }}\kern-.6\wd0}}
\def\ddashint{\Xint=}
\def\dashint{\Xint-}

\usepackage[backend=bibtex,style=alphabetic,giveninits=true]{biblatex}
\renewcommand*{\bibfont}{\normalfont\footnotesize}
\renewbibmacro{in:}{}
\DeclareFieldFormat{pages}{#1}

\newcommand\todo[1]{\textcolor{red}{TODO: #1}}


\begin{document}

\maketitle

Let $d \geq 2$.
We say that two foliations $\mathscr F, \mathscr G$ of codimension $1$ in $\RR^d$ are \dfn{orthogonal} if for any $p \in \RR^d$, the leaves $X, Y$ of $\mathscr F, \mathscr G$ passing through $p$ are orthogonal at $p$.
We only consider $C^\infty$ foliations in the sequel.

\begin{theorem}[Dupin theorem in general dimension]
Let $\vec{\mathscr F} = (\mathscr F_1, \dots, \mathscr F_d)$ be a $d$-tuple of mutually orthogonal foliations of codimension $1$ in $\mathbf R^d$.
Let $p \in \RR^d$, let $X_i$ denote the leaf of $\mathscr F_i$ passing through $p$, and let 
$$Y_i := \bigcap_{j \neq i} X_j.$$
Then $Y_i$ is tangent to a principal curvature vector of $X_j$ at $p$ whenever $i \neq j$.
\end{theorem}
\begin{proof}
We first check that the conclusion of the theorem makes sense.
In fact, $Y_i$ is the intersection of $d - 1$ smooth hypersurfaces, each of which are mutually orthogonal and therefore are transverse.
So $Y_i$ is a smooth curve, and its tangent space at $p$ is a line.

Near $p$, we can find a coordinate system $\vec \xi = (\xi^1, \dots, \xi^d)$ such that the level sets of $\xi^i$ are the leaves of $\mathscr F_i$.
Let $g$ be the euclidean metric tensor written in the coordinates $\vec \xi$.
Orthogonality of $\mathscr F_i$ and $\mathscr F_j$ means that $\langle \dif \xi^i, \dif \xi^j\rangle = 0$. 
Therefore the cometric
$$g^{ij} = \frac{\partial \xi^i}{\partial x^\mu} \frac{\partial \xi^j}{\partial x^\nu} \delta^{\mu \nu}$$
is diagonal; therefore the same holds for the metric
\begin{equation}\label{metric formula}
g_{ij} = \frac{\partial x^\mu}{\partial \xi^i} \frac{\partial x^\nu}{\partial \xi^j} \delta_{\mu \nu}.
\end{equation}

Let $(i, j, k)$ be a distinct triple.
Differentiating the equation $g_{ij} = 0$, we obtain
\begin{equation}\label{differentiated metric}
0 = \frac{\partial^2 x^\mu}{\partial \xi^i \partial \xi^k}  \frac{\partial x^\nu}{\partial \xi^j} \delta_{\mu \nu} + \frac{\partial x^\mu}{\partial \xi^i} \frac{\partial^2 x^\nu}{\partial \xi^j \partial \xi^k} \delta_{\mu \nu}.
\end{equation}
Thus if we set
$$a_{ijk} := \frac{\partial^2 x^\mu}{\partial \xi^i \partial \xi^j}  \frac{\partial x^\nu}{\partial \xi^k} \delta_{\mu \nu}$$
then we have the system of equations 
\begin{align*}
a_{ijk} + a_{ikj} &= 0, \\
a_{jik} + a_{jki} &= 0, \\
a_{kij} + a_{kji} &= 0, \\
a_{ijk} - a_{jik} &= 0, \\
a_{ikj} - a_{kij} &= 0, \\
a_{jki} - a_{kji} &= 0,
\end{align*}
where the first three equations arise from (\ref{differentiated metric}) and the last three originate from symmetry of second partial derivatives.
The only solution of this system of this equation is $a_{ijk} = 0$ for every distinct triple $(i, j, k)$, or in other words 
$$0 = \frac{\partial^2 x^\mu}{\partial \xi^i \partial \xi^j}  \frac{\partial x^\nu}{\partial \xi^k} \delta_{\mu \nu}.$$
Combining this equation with (\ref{metric formula}) and the equation $g_{ij} = 0$, we see that every vector in the list 
$$L := \left(\frac{\partial^2 \vec x}{\partial \xi^i \partial \xi^j}, \frac{\partial \vec x}{\partial \xi^1}, \dots, \widehat{\frac{\partial \vec x}{\partial \xi^k}}, \dots, \frac{\partial \vec x}{\partial \xi^d}\right)$$
(where the roof means to remove that entry from the list) is orthogonal to $\partial \vec x/\partial \xi^k$.
Therefore every vector in $L$ is contained in a vector space of dimension $d - 1$, but the length of $L$ is $d$, so $L$ is linearly dependent.

Let $\Two_{(k)}$ be the second fundamental form of $X_k$, written in the coordinate system
$$\vec \xi_k := (\xi^1, \dots, \widehat{\xi^k}, \dots, \xi^d).$$
Then $\Two_{(k)ij}$ is equal to a scalar times the Hodge star of the wedge product of all of the vectors in $L$.
Since the wedge product of $L$ vanishes, so does $\Two_{(k)ij}$, so $\Two_{(k)}$ is diagonal.
Since the metric $h_{(k)}$ on $X_k$ is given by $h_{(k)ij} = g_{ij}$, $h_{(k)}$ is also diagonal in the coordinate system $\vec \xi_k$, it follows that $\partial_{\xi_i}$ is a principal curvature vector of $X_k$ whenever $i \neq k$.
But $\partial_{\xi_i}$ is normal to $X_i$, so it is tangent to the curve $Y_i$, as desired.
\end{proof}


\printbibliography

\end{document}
