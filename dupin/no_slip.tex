\documentclass[reqno,11pt]{amsart}
\usepackage[letterpaper, margin=1in]{geometry}
\RequirePackage{amsmath,amssymb,amsthm,graphicx,mathrsfs,url,slashed,subcaption}
\RequirePackage[usenames,dvipsnames]{xcolor}
\RequirePackage[colorlinks=true,linkcolor=Red,citecolor=Green]{hyperref}
\RequirePackage{amsxtra}
\usepackage{cancel}
\usepackage{tikz-cd}
%\usepackage[T1]{fontenc}

% \setlength{\textheight}{9.3in} \setlength{\oddsidemargin}{-0.25in}
% \setlength{\evensidemargin}{-0.25in} \setlength{\textwidth}{7in}
% \setlength{\topmargin}{-0.25in} \setlength{\headheight}{0.18in}
% \setlength{\marginparwidth}{1.0in}
% \setlength{\abovedisplayskip}{0.2in}
% \setlength{\belowdisplayskip}{0.2in}
% \setlength{\parskip}{0.05in}
%\renewcommand{\baselinestretch}{1.05}

\title{No-slip boundary condition}
\author{Aidan Backus}
\address{Department of Mathematics, Brown University}
\email{aidan\_backus@brown.edu}
\date{\today}

\newcommand{\NN}{\mathbf{N}}
\newcommand{\ZZ}{\mathbf{Z}}
\newcommand{\QQ}{\mathbf{Q}}
\newcommand{\RR}{\mathbf{R}}
\newcommand{\CC}{\mathbf{C}}
\newcommand{\DD}{\mathbf{D}}
\newcommand{\PP}{\mathbf P}
\newcommand{\MM}{\mathbf M}
\newcommand{\II}{\mathbf I}
\newcommand{\Hyp}{\mathbf H}
\newcommand{\Sph}{\mathbf S}
\newcommand{\Group}{\mathbf G}
\newcommand{\GL}{\mathbf{GL}}
\newcommand{\Orth}{\mathbf{O}}
\newcommand{\SpOrth}{\mathbf{SO}}
\newcommand{\Ball}{\mathbf{B}}

\newcommand*\dif{\mathop{}\!\mathrm{d}}

\DeclareMathOperator{\card}{card}
\DeclareMathOperator{\dist}{dist}
\DeclareMathOperator{\id}{id}
\DeclareMathOperator{\supp}{supp}
\DeclareMathOperator{\Teich}{Teich}
\DeclareMathOperator{\tr}{tr}

\newcommand{\Leaves}{\mathscr L}
\newcommand{\Lagrange}{\mathcal L}
\newcommand{\Hypspace}{\mathscr H}

\newcommand{\Chain}{\underline C}

\newcommand{\Two}{\mathrm{I\!I}}

\newcommand{\normal}{\mathbf n}
\newcommand{\radial}{\mathbf r}
\newcommand{\evect}{\mathbf e}
\newcommand{\vol}{\mathrm{vol}}

\newcommand{\diam}{\mathrm{diam}}
\newcommand{\Ell}{\mathrm{Ell}}
\newcommand{\inj}{\mathrm{inj}}
\newcommand{\Lip}{\mathrm{Lip}}
\newcommand{\Riem}{\mathrm{Riem}}

\newcommand{\Min}{\mathrm{Min}}
\newcommand{\Max}{\mathrm{Max}}

\newcommand{\dfn}[1]{\emph{#1}\index{#1}}

\renewcommand{\Re}{\operatorname{Re}}
\renewcommand{\Im}{\operatorname{Im}}

\newcommand{\loc}{\mathrm{loc}}
\newcommand{\cpt}{\mathrm{cpt}}

\def\Japan#1{\left \langle #1 \right \rangle}

\newtheorem{theorem}{Theorem}
\newtheorem{badtheorem}[theorem]{``Theorem"}
\newtheorem{prop}[theorem]{Proposition}
\newtheorem{lemma}[theorem]{Lemma}
\newtheorem{sublemma}[theorem]{Sublemma}
\newtheorem{proposition}[theorem]{Proposition}
\newtheorem{corollary}[theorem]{Corollary}
\newtheorem{conjecture}[theorem]{Conjecture}
\newtheorem{axiom}[theorem]{Axiom}
\newtheorem{assumption}[theorem]{Assumption}

\newtheorem{mainthm}{Theorem}
\renewcommand{\themainthm}{\Alph{mainthm}}

\newtheorem{claim}{Claim}[theorem]
\renewcommand{\theclaim}{\thetheorem\Alph{claim}}
% \newtheorem*{claim}{Claim}

\theoremstyle{definition}
\newtheorem{definition}[theorem]{Definition}
\newtheorem{remark}[theorem]{Remark}
\newtheorem{example}[theorem]{Example}
\newtheorem{notation}[theorem]{Notation}

\newtheorem{exercise}[theorem]{Discussion topic}
\newtheorem{homework}[theorem]{Homework}
\newtheorem{problem}[theorem]{Problem}

\makeatletter
\newcommand{\proofpart}[2]{%
  \par
  \addvspace{\medskipamount}%
  \noindent\emph{Part #1: #2.}
}
\makeatother



% \numberwithin{equation}{section}


% Mean
\def\Xint#1{\mathchoice
{\XXint\displaystyle\textstyle{#1}}%
{\XXint\textstyle\scriptstyle{#1}}%
{\XXint\scriptstyle\scriptscriptstyle{#1}}%
{\XXint\scriptscriptstyle\scriptscriptstyle{#1}}%
\!\int}
\def\XXint#1#2#3{{\setbox0=\hbox{$#1{#2#3}{\int}$ }
\vcenter{\hbox{$#2#3$ }}\kern-.6\wd0}}
\def\ddashint{\Xint=}
\def\dashint{\Xint-}

\usepackage[backend=bibtex,style=alphabetic,giveninits=true]{biblatex}
\renewcommand*{\bibfont}{\normalfont\footnotesize}
\renewbibmacro{in:}{}
\DeclareFieldFormat{pages}{#1}

\newcommand\todo[1]{\textcolor{red}{TODO: #1}}


\begin{document}

\maketitle

We sum Latin indices over $1, \dots, d$ and Greek indices over $1, \dots, d - 1$.

If $\tau$ is a vector in $\RR^d$ we let $D_\tau$ denote the directional derivative with respect to $\tau$.
If $\tau$ is tangent to $\partial \Omega$, we let $\nabla_\tau$ denote the \emph{covariant} directional derivative with respect to $\tau$, computed using the Riemannian metric on $\partial \Omega$.

The \dfn{Gauss-Codazzi theorem} says that there exists a symmetric $2$-tensor field $\Two$ on $\partial \Omega$, such that for every vector field $v$ tangent to $\partial \Omega$, and every unit tangent vector $\tau$ to $\partial \Omega$,
\begin{equation}\label{Gauss Codazzi}
(D_\tau v)_i = (\nabla_\tau v)_i + (\tau \cdot (\Two v)) n_i.
\end{equation}
The tensor $\Two$ is called the \dfn{second fundamental form} of $\partial \Omega$.
(The metric is the \dfn{first fundamental form}, except nobody calls it that.)

One can check that at the origin, $\Two$ is your matrix $D^2 \phi(0)$.
Indeed, the \dfn{principal curvatures} of $\partial \Omega$ are the eigenvalues of $\Two$.
Unfortunately, the map that sends a matrix to its eigenvalues is not smooth, so the principal curvatures are only Lipschitz.

The regularity of $\Two$ is easy to compute, but most references just take this as obvious, so I prove it here:

\begin{lemma}
Assume that $\partial \Omega$ is $C^{k + 2}$.
Then $\Two \in C^k$.
\end{lemma}
\begin{proof}
We can cover $\partial \Omega$ by balls in which, after rotating,
$$\partial \Omega = \{(x_1, \dots, x_{d - 1}, f(x_1, \dots, x_{d - 1})): x_1, \dots, x_{d - 1} \in U\}$$
where $U \subseteq \RR^{d - 1}$ is an open set and $f \in C^{k + 2}(U)$.
The function $f$ induces a coordinate system on $\partial \Omega$, by the $C^{k + 2}$-diffeomorphism 
$$\Psi(x_1, \dots, x_{d - 1}) := (x_1, \dots, x_{d - 1}, f(x_1, \dots, x_{d - 1})).$$
In this coordinate system, the metric on $\partial \Omega$ is 
$$g_{\alpha \beta} = \delta_{\alpha \beta} + D_\alpha f D_\beta f.$$
So $g_{\alpha \beta} \in C^{k + 1}$.
The Christoffel symbols of the coordinate system $\Psi$ is
$$2{\Gamma_{\alpha \beta}}^\gamma = g^{\gamma \delta} (D_\alpha g_{\beta \delta} + D_\beta g_{\alpha \delta} - D_\delta g_{\alpha \beta})$$
so ${\Gamma_{\alpha \beta}}^\gamma \in C^k$.
In particular, if $\tau$ is any tangent vector to $\partial \Omega$, $\nabla_\tau$ is a first-order differential operator with $C^k$ coefficients, since in the coordinate system given by $\Psi$,
$$(\nabla_\tau v)_\alpha = \tau^\beta D_\beta v_\alpha + {\Gamma_{\alpha \beta}}^\gamma \tau^\beta v_\gamma.$$
Moreover, the normal vector is 
$$n = \left(\frac{D_1 f}{\sqrt{1 + |Df|^2}}, \dots, \frac{D_{d - 1} f}{\sqrt{1 + |Df|^2}}, \frac{1}{\sqrt{1 + |Df|^2}}\right)$$
which is $C^{k + 1}$.
Taking $v$ to be the pushforward of the $\alpha$th coordinate vector by $\Psi$, and $\tau$ to be the pushforward of the $\beta$th coordinate vector by $\Psi$ (so $v, \tau$ are both $C^{k + 1}$), we see that
$$\Two_{\alpha \beta} = (D_\tau v - \nabla_\tau v) \cdot n.$$
The right-hand side is clearly $C^k$.
\end{proof}

Assume that $u \cdot n = 0$, thus $u$ is tangent to $\partial \Omega$.
Let $\omega_{ij} := D_i u_j - D_j u_i$ be the vorticity $2$-form, and $2\mathbb D_{ij} := D_i u_j + D_j u_i$ be the deformation $2$-tensor.
Then for any vector field $\tau$ which is tangent to $\partial \Omega$,
$$2\tau \cdot (\mathbb Dn) = -\omega_{ij} \tau^i n^j + 2(D_\tau u) \cdot n.$$
Next we compute using (\ref{Gauss Codazzi}) and the fact that $\nabla_\tau u$ is tangent to $\partial \Omega$ (since it is the covariant derivative of a tangent vector field)
$$(D_\tau u) \cdot n = (\nabla_\tau u) \cdot n + \tau \cdot (\Two u) = \tau \cdot (\Two u).$$
Thus we have 
\begin{equation}\label{appendix equation}
2\tau \cdot (\mathbb Dn) + \alpha u \cdot \tau = (\omega \tau) \cdot n + ((\Two + \alpha)\tau) \cdot u.
\end{equation}
If $\tau$ is a vector field of principal curvature $\lambda_j$, then this is exactly the equation you derived.
But in general we can only choose such vector fields to be Lipschitz, while if $\tau$ is at least $C^k$ and $\partial \Omega$ is $C^{k + 2}$ then all the terms in (\ref{appendix equation}) are at least $C^k$.


\printbibliography

\end{document}
