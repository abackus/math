\documentclass[reqno,12pt,letterpaper]{amsart}
\RequirePackage{amsmath,amssymb,amsthm,graphicx,mathrsfs,url}
\RequirePackage[usenames,dvipsnames]{color}
\RequirePackage[colorlinks=true,linkcolor=Red,citecolor=Green]{hyperref}
\RequirePackage{amsxtra}
\usepackage{tikz-cd}

\setlength{\textheight}{8.50in} \setlength{\oddsidemargin}{0.00in}
\setlength{\evensidemargin}{0.00in} \setlength{\textwidth}{6.08in}
\setlength{\topmargin}{0.00in} \setlength{\headheight}{0.18in}
\setlength{\marginparwidth}{1.0in}
\setlength{\abovedisplayskip}{0.2in}
\setlength{\belowdisplayskip}{0.2in}
\setlength{\parskip}{0.05in}
\renewcommand{\baselinestretch}{1.10}

\title[Statement of interest for UT Austin summer school]{UTA statement of interest}
\author{Aidan Backus}
\date{March 2021}

\newcommand{\NN}{\mathbf{N}}
\newcommand{\ZZ}{\mathbf{Z}}
\newcommand{\QQ}{\mathbf{Q}}
\newcommand{\RR}{\mathbf{R}}
\newcommand{\CC}{\mathbf{C}}
\newcommand{\DD}{\mathbf{D}}
\newcommand{\PP}{\mathbf P}

\DeclareMathOperator{\card}{card}
\DeclareMathOperator{\ch}{ch}
\DeclareMathOperator{\diag}{diag}
\DeclareMathOperator{\dom}{dom}
\DeclareMathOperator{\Gal}{Gal}
\DeclareMathOperator{\id}{id}
\DeclareMathOperator{\rank}{rank}
\DeclareMathOperator*{\Res}{Res}
\DeclareMathOperator{\sgn}{sgn}
\DeclareMathOperator{\singsupp}{sing~supp}
\DeclareMathOperator{\Spec}{Spec}
\DeclareMathOperator{\supp}{supp}
\newcommand{\tr}{\operatorname{tr}}

\newcommand{\dbar}{\overline \partial}

\DeclareMathOperator{\atanh}{atanh}
\DeclareMathOperator{\csch}{csch}
\DeclareMathOperator{\sech}{sech}

\DeclareMathOperator{\Ell}{Ell}
\DeclareMathOperator{\WF}{WF}

\newcommand{\pic}{\vspace{30mm}}
\newcommand{\dfn}[1]{\emph{#1}\index{#1}}

\renewcommand{\Re}{\operatorname{Re}}
\renewcommand{\Im}{\operatorname{Im}}


\newtheorem{theorem}{Theorem}[section]
\newtheorem{badtheorem}[theorem]{``Theorem"}
\newtheorem{prop}[theorem]{Proposition}
\newtheorem{lemma}[theorem]{Lemma}
\newtheorem{proposition}[theorem]{Proposition}
\newtheorem{corollary}[theorem]{Corollary}
\newtheorem{conjecture}[theorem]{Conjecture}
\newtheorem{axiom}[theorem]{Axiom}

\theoremstyle{definition}
\newtheorem{definition}[theorem]{Definition}
\newtheorem{remark}[theorem]{Remark}
\newtheorem{example}[theorem]{Example}

\newtheorem{exercise}[theorem]{Discussion topic}
\newtheorem{homework}[theorem]{Homework}
\newtheorem{problem}[theorem]{Problem}

\newtheorem*{ack}{Acknowledgements}
\newtheorem*{notate}{Notation}

%\usepackage{color}
%\hypersetup{%
%    colorlinks=true, % make the links colored%
%    linkcolor=blue, % color TOC links in blue
%    urlcolor=red, % color URLs in red
%    linktoc=all % 'all' will create links for everything in the TOC
%Ning added hyperlinks to the table of contents 6/17/19
%}

\usepackage[backend=bibtex,style=alphabetic,maxcitenames=50,maxnames=50]{biblatex}
%\addbibresource{zworski_paper.bib}
\renewbibmacro{in:}{}
\DeclareFieldFormat{pages}{#1}

\begin{document}
%\begin{abstract}
%\end{abstract}

\maketitle

%\tableofcontents

%\section{hi}

I am writing to express interest in attending the UT Austin summer school in the analysis of PDE.

I am currently a first-year graduate student at Brown University, and previously was an undergraduate at UC Berkeley.
At Berkeley, I took a general course on the analysis of PDE, as well as courses on microlocal analysis from Maciej Zworski and on the Cauchy problem in general relativity from Sung-Jin Oh.
At Brown, I have taken a course on PDE and spectral geometry, and am currently taking a course on fluid dynamics and a course on Riemann surfaces -- just today, in fact, we proved the Riemann-Roch theorem and discussed how index theory can also be used to prove the Gauss-Bonnet theorem.

I am fortunate to be advised by Prof. Zworski and Justin Holmer in research on resonances for the nonlinear Schr\"odinger equation.
This work has taken a rather geometric direction, both because the resonances themselves live on a certain Riemann surface and because it is technically convenient to view the linearization of the cubic NLS as a pseudodifferential equation on a certain asymptotically hyperbolic manifold.

I have also become the adviser of a reading course for an undergraduate interest in mathematical physics.
We have followed Evans' book, focusing on the Hamilton-Jacobi system and how it can be used to reduce physical problems to variational problems.
While it seems that my student is learning a lot of geometry and analysis from the reading course, I have also benefitted: my knowledge of the calculus of variations is somewhat lacking (and my knowledge of physics is \emph{very} lacking!), so I have had to always stay ``one week ahead" of her.
In future semesters I would like to mentor her or other undergraduate students in similar topics.

Owing to the Riemann surfaces course, my work on the cubic NLS, and my undergraduate mentorship, I have recently become quite interested in geometric PDE and their physical applications.
As such, when Benoit Pausader suggested that I sign up for the UT Austin summer school, and I saw that Robert Hasholfer is speaking on the mean curvature flow, I jumped at the opportunity.
Since then, I've talked to Prof. Oh about the school, and he suggested that my experience with pseudodifferential operators would be invaluable in understanding his course on Cauchy problems for dispersive PDE.
Prof. Oh also sent me his paper on the Cauchy problem for the Hall-MHD system, which I look forward to reading.

Summing up, my current research and mentorship have led me to both the geometric topics that will be discussed by Prof. Hasholfer, as well as the more physical topics that Tarek Elgindi and Prof. Oh will be speaking on.
Because my knowledge of variational methods is so shaky, I would also benefit from Franceso Maggi's course.

Thank you for considering my application.
I hope to meet you in May!






\end{document}
