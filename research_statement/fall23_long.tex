\documentclass[reqno,11pt]{amsart}
\usepackage[letterpaper, margin=1in]{geometry}
\RequirePackage{amsmath,amssymb,amsthm,graphicx,mathrsfs,url,slashed,subcaption}
\RequirePackage[usenames,dvipsnames]{xcolor}
\RequirePackage[colorlinks=true,linkcolor=Red,citecolor=Green]{hyperref}
\RequirePackage{amsxtra}
\usepackage{cancel}
\usepackage{tikz, wrapfig}

% Add the 2020 MSC
\makeatletter
\@namedef{subjclassname@2020}{\textup{2020} Mathematics Subject Classification}
\makeatother

%\usepackage[T1]{fontenc}

% \setlength{\textheight}{9.3in} \setlength{\oddsidemargin}{-0.25in}
% \setlength{\evensidemargin}{-0.25in} \setlength{\textwidth}{7in}
% \setlength{\topmargin}{-0.25in} \setlength{\headheight}{0.18in}
% \setlength{\marginparwidth}{1.0in}
% \setlength{\abovedisplayskip}{0.2in}
% \setlength{\belowdisplayskip}{0.2in}
% \setlength{\parskip}{0.05in}
%\renewcommand{\baselinestretch}{1.05}

\title{Research statement}
\author{Aidan Backus}
\address{Department of Mathematics, Brown University}
\email{aidan\_backus@brown.edu}
\date{\today}

\newcommand{\NN}{\mathbf{N}}
\newcommand{\ZZ}{\mathbf{Z}}
\newcommand{\QQ}{\mathbf{Q}}
\newcommand{\RR}{\mathbf{R}}
\newcommand{\CC}{\mathbf{C}}
\newcommand{\DD}{\mathbf{D}}
\newcommand{\PP}{\mathbf P}
\newcommand{\MM}{\mathbf M}
\newcommand{\II}{\mathbf I}
\newcommand{\Hyp}{\mathbf H}
\newcommand{\Sph}{\mathbf S}
\newcommand{\Torus}{\mathbf T}
\newcommand{\Group}{\mathbf G}
\newcommand{\GL}{\mathbf{GL}}
\newcommand{\Orth}{\mathbf{O}}
\newcommand{\SpOrth}{\mathbf{SO}}
\newcommand{\Ball}{\mathbf{B}}

\newcommand*\dif{\mathop{}\!\mathrm{d}}

\DeclareMathOperator{\card}{card}
\DeclareMathOperator{\dist}{dist}
\DeclareMathOperator{\id}{id}
\DeclareMathOperator{\Hom}{Hom}
\DeclareMathOperator{\mesh}{mesh}
\DeclareMathOperator{\coker}{coker}
\DeclareMathOperator{\supp}{supp}
\DeclareMathOperator{\tr}{tr}

\newcommand{\Two}{\mathrm{I\!I}}
\newcommand{\weakto}{\rightharpoonup}

\newcommand{\normal}{\mathbf n}
\newcommand{\vol}{\mathrm{vol}}

\newcommand{\diam}{\mathrm{diam}}
\DeclareMathOperator{\sech}{sech}
\newcommand{\inj}{\mathrm{inj}}
\newcommand{\Lip}{\mathrm{Lip}}
\newcommand{\Riem}{\mathrm{Riem}}

\DeclareMathOperator*{\essinf}{ess\,inf}
\DeclareMathOperator*{\esssup}{ess\,sup}

\newcommand{\dfn}[1]{\emph{#1}\index{#1}}

\renewcommand{\Re}{\operatorname{Re}}
\renewcommand{\Im}{\operatorname{Im}}

\newcommand{\loc}{\mathrm{loc}}
\newcommand{\cpt}{\mathrm{cpt}}

\def\Japan#1{\left \langle #1 \right \rangle}

\newtheorem{theorem}{Theorem}[section]
\newtheorem{badtheorem}[theorem]{``Theorem"}
\newtheorem{prop}[theorem]{Proposition}
\newtheorem{lemma}[theorem]{Lemma}
\newtheorem{sublemma}[theorem]{Sublemma}
\newtheorem{proposition}[theorem]{Proposition}
\newtheorem{corollary}[theorem]{Corollary}
\newtheorem{conjecture}[theorem]{Conjecture}
\newtheorem{axiom}[theorem]{Axiom}
\newtheorem{assumption}[theorem]{Assumption}

\newtheorem{mainthm}{Theorem}
\renewcommand{\themainthm}{\Alph{mainthm}}

\newtheorem{claim}{Claim}[theorem]
\renewcommand{\theclaim}{\thetheorem\Alph{claim}}
% \newtheorem*{claim}{Claim}

\theoremstyle{definition}
\newtheorem{definition}[theorem]{Definition}
\newtheorem{remark}[theorem]{Remark}
\newtheorem{example}[theorem]{Example}
\newtheorem{notation}[theorem]{Notation}

\newtheorem{exercise}[theorem]{Discussion topic}
\newtheorem{homework}[theorem]{Homework}
\newtheorem{problem}[theorem]{Problem}

\makeatletter
\newcommand{\proofpart}[2]{%
  \par
  \addvspace{\medskipamount}%
  \noindent\emph{Part #1: #2.}
}
\makeatother



\numberwithin{equation}{section}


% Mean
\def\Xint#1{\mathchoice
{\XXint\displaystyle\textstyle{#1}}%
{\XXint\textstyle\scriptstyle{#1}}%
{\XXint\scriptstyle\scriptscriptstyle{#1}}%
{\XXint\scriptscriptstyle\scriptscriptstyle{#1}}%
\!\int}
\def\XXint#1#2#3{{\setbox0=\hbox{$#1{#2#3}{\int}$ }
\vcenter{\hbox{$#2#3$ }}\kern-.6\wd0}}
\def\ddashint{\Xint=}
\def\dashint{\Xint-}

\usepackage[backend=bibtex,style=alphabetic,giveninits=true]{biblatex}
\renewcommand*{\bibfont}{\normalfont\footnotesize}
\addbibresource{research_statement.bib}
\renewbibmacro{in:}{}
\DeclareFieldFormat{pages}{#1}

\newcommand\todo[1]{\textcolor{red}{TODO: #1}}


\begin{document}

\maketitle

%%%%%%%%%%%%%%%%%%%%%%%%%%%%%%%%%%%%%%%%%%%%%%%%%%%%%%%
My research area is \dfn{geometric measure theory} (GMT), which is the study of geometric objects in the setting of low regularity.
Applications of analysis to differential geometry typically must pass through geometric measure theory: this is because the typical method of constructing solutions of partial differential equations (PDE) is to first show that a solution $u$ exists but possibly has low regularity, and only then prove that $u$ is smooth.
The most famous example of this procedure is the resolution of Plateau's problem.
In one form, this famous theorem asserts that for every $2 \leq d \leq 7$ and every closed oriented smooth $d - 2$-dimensional submanifold $N \subset \Sph^{d - 1}$, $N$ is the boundary of an analytic $d - 1$-dimensional submanifold $M \subset \RR^d$ which minimizes its surface area among all such hypersurfaces.
To prove this, one must construct a ``candidate'' solution $M$, which a priori is just locally a distributional solution of the minimal surface PDE; one must then use GMT to show that $M$ is $C^{1 + \alpha}$, and then Schauder estimates take over and imply that $M$ is analytic \cite{Giusti77}.

My dissertation focuses on applications of geometric measure theory to geometric topology and $L^\infty$ variational problems, but I am also interested in connections with harmonic analysis.
To briefly summarize my dissertation, I have shown that solutions of a certain variational problem, the \dfn{least gradient problem}, can be identified with laminations of area-minimizing hypersurfaces.
These Cantor set-like structures have been heavily studied on hyperbolic surfaces, where they satisfy a ``max flow-min cut'' theorem which relates them to Thurston's geometry on Teichm\"uller space \cite{thurston1998minimal}.
By exploiting the duality between the least gradient problem and a generalization of the $\infty$-Laplace equation, I showed that this continuous max-flow min-cut theorem can be established on a wide class of Riemannian manifolds $M$.
Since every homology class in $H_{d - 1}(M, \RR)$ corresponds to a solution of the least gradient problem, I obtained certain results about the structure of $H_{d - 1}(M, \RR)$.
More recently, I am investigating the $\infty$-Laplacian for maps between manifolds.
\todo{Cut this part?}

%%%%%%%%%%%%%%%%%%%%%%%%%%%
\section{Least gradient functions, laminations, and geometric topology}

%%%%%%%%%%%%%%%%%%%%%%%%%%%
\section{\texorpdfstring{$L^\infty$}{L-infinity} calculus of variations}
The classical \dfn{calculus of variations} is concerned with PDE whose solutions are minimizers of certain integral energies.
The prototypical example is that solutions $u$ of Laplace's equation minimize the Dirichlet energy $\int |\nabla u|^2$.
The \dfn{$L^\infty$ calculus of variations} studies PDE whose solutions instead minimize certain quantities which are defined pointwise; the motivating example is the \dfn{$\infty$-Laplace equation}
$$\langle \nabla^2 u, \nabla u \otimes \nabla u\rangle = 0,$$
whose solutions $u: U \to \RR$ are characterized by the property for every open ball $V \subseteq U$, $u|_V$ minimizes its \dfn{Lipschitz constant}
$$\Lip(u, V) := \sup_{x, y \in V} \frac{|u(x) - u(y)|}{|x - y|}$$
among all functions $v$ with $v|_{\partial V} = u|_{\partial V}$ \cite{Crandall2008}.

The $\infty$-Laplacian is totally nonlinear and admits solutions which are not $C^2$; these features are typical of $L^\infty$ variational PDE, and imply that the expression $\langle \nabla^2 u, \nabla u \otimes \nabla u\rangle$ does not make sense.
Since the $\infty$-Laplace equation does satisfy the maximum principle, one defines its solutions to be the \dfn{viscosity solutions}, namely those functions which formally satisfy the maximum principle for the $\infty$-Laplacian \cite{Crandall2008}.

However, maximum principles and viscosity solutions do not make sense for maps valued in vector spaces or manifolds.
If we are interested in the problem of finding an optimal Lipschitz map between Riemannian manifolds, we need a new solution concept.
Daskalopolous and Uhlenbeck proposed that the solutions should be the H\"older limits of solutions of approximating $L^p$ variational problems \cite{daskalopoulos2022analytic}, while Sheffield and Smart, and Jensen, proposed solution concepts motivated by combinatorics and measure theory respectively \cite{Sheffield12}.
It is easy to establish existence of the Daskalopolous--Uhlenbeck solutions, but it is not clear in which sense they are ``optimal'' Lipschitz maps.
Meanwhile, the Sheffield--Smart and Jensen solutions minimize various pointwise quantities related to the Lipschitz constant, but it is unclear whether they exist.

Therefore, I would like to resolve:

\begin{conjecture}\label{infinity laplacian conjecture}
Every Daskalopolous--Uhlenbeck solution of the vector-valued $\infty$-Laplacian is a Sheffield--Smart solution.
For smooth maps $u$ with a principal singular vector field, $u$ is a Daskalopolous--Uhlenbeck solution iff $u$ is a classical solution.
\end{conjecture}

This result is known if the codomain is $\RR$.
It implies that for any bounded Lipschitz domain $U \subseteq \RR^d$ and Lipschitz map $f: \partial U \to \RR^D$, there exists a Sheffield--Smart solution $u: U \to \RR^D$ of the vector-valued $\infty$-Laplacian such that $u|_{\partial U} = f$.

I established a negative result in this direction.
Tight forms are defined analogously to the Daskalopolous--Uhlenbeck solutions, as they are H\"older limits of $L^p$ variational systems as $p \to \infty$.
We recall that if $F$ is tight, then $\|F\|_{L^\infty} \leq \|G\|_{L^\infty}$ whenever $F - G$ is exact.
One can show that if $d = 3$ and the $p$-tight forms are bounded in $C^{1 + \alpha}$, then the limiting tight $2$-form satisfies 
\begin{equation}\label{tight PDE}
\begin{cases}
\dif F = 0 \\
(\nabla_i F_{jk}) F^{jk} {F^i}_\ell = 0.
\end{cases}
\end{equation}
In general the PDE for the tight form is more complicated; below we simply refer to it as the \dfn{tight PDE}.
A theorem of Barron, Jensen, and Wang implies that if $F$ is smooth and minimizes its $L^\infty$ norm in every small ball, then $F$ is a classical solution of the tight PDE \cite{Barron2001}.

\begin{theorem}[{\cite{BackusCML1}}]
Let $F$ be the flux $2$-form of the unit vertical vector field to the Hopf fibration on $\Sph^3$.
Then $F$ is a classical solution of (\ref{tight PDE}).
\end{theorem}

In particular, since $F$ is nonzero but exact, $F$ does not minimize its $L^\infty$ norm globally, and is not a tight $2$-form.
The proof uses the fact that $\star F$ is a contact $1$-form.
This gives us some hope that Conjecture \ref{infinity laplacian conjecture}, still holds, as there is no clear analogue of contactness for maps, as opposed to differential forms.
In fact, I am able to show that, when a classical solution $F$ to (\ref{tight PDE}) fails to satisfy any sort of optimality, it must be because $\star F$ looks like a contact $1$-form somewhere:

\begin{theorem}[{\cite{BackusCML1}}]
Let $F$ be a classical solution to the tight PDE.
If $\ker(\star F|_{|F| > 0})$ is integrable, then $F$ minimizes its $L^\infty$ norm in every small ball.
\end{theorem}


%%%%%%%%%%%%%%%%%%%%%%%%%%%
\section{Connections with harmonic analysis}
An \dfn{uncertainty principle} is an informal assertion that a function $f$ and its Fourier transform $\mathscr Ff$ cannot both be ``localized'' in some sense.
An example is the \dfn{fractal uncertainty principle} (FUP), which asserts that $f$ and $\mathscr Ff$ cannot both ``resemble fractals''.
FUP has various applications to chaotic dynamical systems; a typical proof in this direction proceeds by showing that eigenfunctions of the Laplacian on a hyperbolic manifold cannot ``interact in a sufficiently complicated way'' without violating FUP \cite{Dyatlov2016, DyatlovSemiclassical18}.

To make this more precise, recall that a compact set $X \subset \RR^d$ is \dfn{Ahlfors-David regular} if there exists $\delta \in [0, d]$ and a Radon measure $\mu$ with support $X$ such that for every $x \in X$ and $0 < r < 1$,
$$\mu(B(x, r)) \sim r^\delta.$$
In that case $X$ has Minkowski and Hausdorff dimension $\delta$ with uniform constants at all scales.
We write $X_h := \{x \in \RR^d: \dist(x, X) < h\}$ and introduce the \dfn{semiclassical Fourier transform}
$$\mathscr F_h f(\xi) := \frac{1}{(2\pi h)^{d/2}} \int_{\RR^d} e^{i x \cdot \xi/h} f(x) \dif x.$$
It is straightforward to show that for any pair $X, Y \subset \RR^d$ of AD-regular sets, there exists $\beta \geq \max(0, (d - \delta_X - \delta_Y)/2)$ such that 
$$\|1_{X_h} \mathscr F_h 1_{Y_h}\|_{L^2 \to L^2} \lesssim h^\beta.$$
If $0 < \delta_X, \delta_Y < d$ and $Y$ satisfies a slightly stronger condition than AD-regular, it is known that we can take $\beta > 0$ \cite{Bourgain18, cohen2023fractal}.
If $d = 1$ and $0 < \delta_X, \delta_Y < 1$, then Dyatlov and Jin showed that we can take $\beta > (d - \delta_X - \delta_Y)/2$ \cite{Dyatlov18}.

The Dyatlov--Jin theorem cannot hold in full generality when $d \geq 2$, because it fails for Cantor sets which are contained in dual lines.
In joint work with James Leng and Zhongkai Tao, we generalized the Dyatlov--Jin theorem to higher dimensions under the assumption the assumption that $X, Y$ are \dfn{nonorthogonal} at all scales.
To be more precise, we assume that for every $0 < r < 1$ and every $x_0 \in X$, $y_0 \in Y$, there exist $x_1, x_2 \in B(x_0, r) \cap X$ and $y_1, y_2 \in B(y_0, r) \cap Y$ such that $|(x_1 - y_1) \cdot (x_2 - y_2)| \gtrsim r^2$.
This is essentially equivalent to being AD-regular from below when $d = 1$, but if $d \geq 2$, it also implies that $X, Y$ cannot concentrate in orthogonal subspaces at any scale.
We recall that a Borel measure $\mu$ with support $X$ is \dfn{doubling} if for every $x \in X$ and $0 < r < 1$, $\mu(B(x, r)) \sim \mu(B(x, 2r))$; this is slightly weaker than requiring that $X$ be AD-regular from above.

\begin{theorem}[{\cite{backus2023fractal}}]
Assume that the compact sets $X, Y \subset \RR^d$ have Minkowski dimensions $0 < \delta_X, \delta_Y < d$, are the supports of doubling measures, and are nonorthogonal.
Then there exists $\beta > d - \delta_X - \delta_Y/2$ such that 
$$\|1_{X_h} \mathscr F_h 1_{Y_h}\|_{L^2 \to L^2} \lesssim h^\beta.$$
\end{theorem}

In addition to the nonorthogonality hypothesis, the key new tool is a discretization of a compact set $X$ into a tree at all scales, which ``respects the structure of doubling measures on $X$" in the sense that if $\mu$ is a doubling measure with support $X$, then the $\mu$-measure of each node $I$ of the tree is comparable to the $\mu$-measures of all of the children of $I$.
Dyatlov and Jin were able to construct a similar tree by showing that if $X$ is AD-regular and $\delta_X < 1$, then $X$ is totally disconnected in a quantitative sense; we needed a completely different approach, since if $\delta_X \geq 1$, then $X$ can be connected. 

More generally, I am interested in how the integral transforms of harmonic analysis interact with the fractalline objects studied in GMT. 
Solomon introduced an X-ray transform for differential forms on $\RR^d$, and was able to use this transform to show that, modulo the below conjecture, every compactly supported $p$-current on $\RR^d$ can be recovered from its projections onto $k$-planes, provided $p < k \leq d$ \cite{Solomon11}.
I would like to establish this conjecture.

\begin{conjecture}[inversion formula for Solomon's X-ray transform]
For $1 < k < d$, let $\mathcal R_k$ be Solomon's X-ray transform on $k$-planes, and let $\mathscr F$ be the Fourier transform.
Then for every $p$-form $\alpha$ on $\RR^d$, $1 \leq p < k$, with $|\alpha| \lesssim |x|^{-d-\varepsilon}$ as $x \to \infty$,
$$\mathscr F(\alpha) = |x|^{d - k} \left(\frac{\Pi}{k - p} + \frac{\Phi}{d - p}\right) \mathscr F(\mathcal R_k^* \mathcal R_k \alpha)$$
where $\Pi(x)$ projects a $p$-form to the hyperplane perpendicular to $x$, and $\Phi(x)$ is the adjoint of $\Pi(x)$.
\end{conjecture}

\printbibliography

\end{document}
