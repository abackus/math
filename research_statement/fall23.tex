
\documentclass[11pt]{article}
\usepackage[letterpaper, margin=1in]{geometry}
\usepackage{amsmath}
\usepackage{amssymb}
\usepackage{amsthm}
\usepackage{amscd}
\usepackage{amsfonts}
\usepackage{graphicx}%
\usepackage{fancyhdr}


\theoremstyle{plain} \numberwithin{equation}{section}
\newtheorem{theorem}{Theorem}%[section]
\newtheorem{corollary}[theorem]{Corollary}
\newtheorem{conjecture}{Conjecture}
\newtheorem{lemma}[theorem]{Lemma}
\newtheorem{proposition}[theorem]{Proposition}
\theoremstyle{definition}
\newtheorem{definition}[theorem]{Definition}
\newtheorem{finalremark}[theorem]{Final Remark}
\newtheorem{remark}[theorem]{Remark}
\newtheorem{example}[theorem]{Example}
\newtheorem{question}{Question} \topmargin-2cm

% \textwidth6in

% \setlength{\topmargin}{0in} \addtolength{\topmargin}{-\headheight}
% \addtolength{\topmargin}{-\headsep}

% \setlength{\oddsidemargin}{0in}

% \oddsidemargin  0.0in \evensidemargin 0.0in %\parindent0em

\pagestyle{fancy}\lhead{Research Statement} \rhead{September 2023}
\chead{{\large{\bf Aidan Backus}}} \lfoot{} \rfoot{\bf \thepage} \cfoot{}

\newcounter{list}


\begin{document}

% \raisebox{1cm}

My research is in geometric measure theory (GMT), which is the generalization of differential geometry to the low-regularity setting.
My dissertation specifically deals with certain variational problems on a Riemannian manifold $M$, which encode topological data -- in particular, the structure of laminations -- about $M$.

Our basic object of interest is \emph{functions of least gradient}, namely those functions $u$ which minimize their $BV$ seminorm subject to a suitable boundary condition.
Classical work in GMT shows that the level sets of a function $u$ of least gradient on a manifold of dimension $d \leq 7$ are area-minimizing hypersurfaces.
Later, Auer and Bangert observed that the level sets form a \emph{lamination}, or in other words a foliation of a closed subset of $M$.
I established the Lipschitz regularity of this lamination.

\begin{theorem}\label{least gradient means lamination}
Let $u$ be a function of locally least gradient on an open domain in a Riemannian manifold $M$ of dimension $d \leq 7$.
Then there is a Lipschitz oriented measured lamination $\lambda_u$, whose leaves are the level sets of $u$.
\end{theorem}

We establish Theorem \ref{least gradient means lamination} by first proving a more general result: given a disjoint collection $\mathcal C$ of minimal hypersurfaces with a uniform bound on their curvatures, we can find a Lipschitz lamination with leaf set $\mathcal C$.
As a corollary, we also establish the local compactness of certain topologies on the space of laminations with minimal leaves.

We then turn to the convex dual problem to functions of least gradient.
We show that this is the problem of finding a $d - 1$-form $F$ which is \emph{tight} in the sense that $F$ minimizes its $L^\infty$ norm subject to a suitable boundary condition, and is a solution of a certain PDE generalizing the $\infty$-Laplace equation in a suitable variational sense.
Geometrically, the tight form $F$ is a \emph{calibration} of the leaves of $\lambda_u$: in other words, $F$ is the area form on the leaves, and $dF = 0$.

\begin{theorem}
For each cohomology class $\rho \in H^{d - 1}(M, \mathbb R)$, there exists a tight form $F$ representing $\rho$.
Moreover, there is an equivariant function $u$ of least gradient on $\tilde M$, such that $du \wedge F = \|F\|_{L^\infty}$, and $F/\|F\|_{L^\infty}$ is a calibration of the leaves of $\lambda_u$.
\end{theorem}

There is a ``canonical'' geodesic lamination on a closed hyperbolic surface; as shown by Thurston, the canonical lamination encodes properties of the duality between the natural norm on the tangent space $T_g\mathcal T$ to Teichm\"uller space $\mathcal T$, and its dual norm on $T'_g \mathcal T$.
We use the tight form to construct a lamination with similar properties to the canonical geodesic lamination, which we call the \emph{canonical lamination} induced by $\rho$, and which encodes certain properties of the duality between the natural norm on homology $H_{d - 1}(M, \mathbb R)$, called the \emph{stable norm}, and its dual \emph{costable norm} on $H^{d - 1}(M, \mathbb R)$.
The connection between these norms and the variational problems we are considering is that the $BV$ seminorm of a function of least gradient equals the stable norm, while the $L^\infty$ norm of the tight form is the costable norm.
Here is a sample application:

\begin{theorem}
Suppose that $\wedge: H^1(M, \mathbb R) \otimes H^1(M, \mathbb R) \to H^2(M, \mathbb R)$ is injective modulo symmetric tensors.
Then the stable unit ball of $H_{d - 1}(M, \mathbb R)$ is strictly convex.
\end{theorem}

In future work, I hope to address not just functions of least gradient, but maps of least gradient $u: M \to \mathbb R^c$.
Such maps have appeared in the literature on image denoising and Teichm\"uller theory, but are untreated in the PDE and GMT literature.
A first step is to construct the Euler-Lagrange equations, and determine in what weak sense do maps of least gradient solve them.
Here we can hope to take some inspiration from work of Maz\'on, Rossi, and Segura de L\'eon on functions of least gradient and the $1$-Laplacian.
We know from that work that there can be no strong uniqueness theory for maps of least gradient, but one can at least ask for an existence theory, as well as a geometric interpretation analogous to the interpretation of functions of least gradient as laminations.


\end{document}
