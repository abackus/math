% Created 2021-06-10 Thu 19:42
% Intended LaTeX compiler: pdflatex
\documentclass[reqno,12pt,letterpaper]{amsart}
               \RequirePackage{amsthm,mathrsfs,url}
\RequirePackage[colorlinks=true,linkcolor=Red,citecolor=Green]{hyperref}
               \RequirePackage[usenames,dvipsnames]{color}
               \RequirePackage{amsxtra}
               \usepackage{tikz-cd}
\newtheorem{theorem}{Theorem}[section]
\newtheorem{lemma}[theorem]{Lemma}
\theoremstyle{definition}
\newtheorem{definition}[theorem]{Definition}
\DeclareMathOperator{\sgn}{sgn}
               \setlength{\textheight}{8.50in} \setlength{\oddsidemargin}{0.00in}
               \setlength{\evensidemargin}{0.00in} \setlength{\textwidth}{6.08in}
               \setlength{\topmargin}{0.00in} \setlength{\headheight}{0.18in}
               \setlength{\marginparwidth}{1.0in}
               \setlength{\abovedisplayskip}{0.2in}
               \setlength{\belowdisplayskip}{0.2in}
               \setlength{\parskip}{0.05in}
               \renewcommand{\baselinestretch}{1.10}
\usepackage[utf8]{inputenc}
\usepackage[T1]{fontenc}
\usepackage{graphicx}
\usepackage{grffile}
\usepackage{longtable}
\usepackage{wrapfig}
\usepackage{rotating}
\usepackage[normalem]{ulem}
\usepackage{amsmath}
\usepackage{textcomp}
\usepackage{amssymb}
\usepackage{capt-of}
\usepackage{hyperref}
\newcommand{\F}{\mathbb F}
\author{Aidan Backus}
\date{10 June, 2021}
\title{The Riemann zeta function and the charged ergodic theorem}
\hypersetup{
 pdfauthor={Aidan Backus},
 pdftitle={The Riemann zeta function and the charged ergodic theorem},
 pdfkeywords={},
 pdfsubject={},
 pdfcreator={Emacs 27.2 (Org mode 9.4.6)}, 
 pdflang={English}}
\begin{document}

\maketitle
\tableofcontents


\section{Preliminaries}
\label{sec:org120d474}
In a \href{https://arxiv.org/pdf/2106.04644.pdf}{recent paper}, Peter Burton of UT Austin announced a proof of the following theorem:

\begin{theorem}
There exists a constant \(h > 0\) such that every zero of the Riemann zeta function has real part \(< 1 - 2h\).
\end{theorem}

In fact, this follows from a result that Burton calls the \emph{charged mean ergodic theorem} concerning the representation theory of the free group \(\F\) on two generators, which itself is a consequence of the negation of the Connes' embedding conjecture (or, equivalently, the equality of the complexity classes \(\mathsf{RE}\) and \(\mathsf{MIP^*}\).

This sounds a bit like crackpottery, and the bizarre acknowledgements section of Burton's paper, which alludes to ``numerous artificial neural networks'' that helped Burton write the paper, doesn't do him any favors. However, I was rather curious, as the proof that \(\mathsf{RE} = \mathsf{MIP^*}\) implies the charged mean ergodic theorem apparently only uses elementary facts about C*-algebras.

Anyways, progress on the Riemann hypothesis would be a great 23rd birthday gift, so I hope this pans out. Below are my attempts to understand the proof that \(\mathsf{RE} = \mathsf{MIP^*}\) implies the charged mean ergodic theorem.

\subsection{Representation theory}
\label{sec:org6823e3f}

Let \(\F\) be the free group on generators \(a,b\).
We write \(\iota\) for the empty word.

If \(w\) is a word we write \(|w|\) for its length and \(|w|_a\) for the number of times that \(a\) or \(a^{-1}\) appears in \(w\).
We also write \(\delta_w(x) = 1\) if \(w = x\) and \(\delta_w(x) = 0\) otherwise.
We write \(\chi_0\) for the character of \(\mathbb F\) defined by \(\chi_0(w) = (-1)^{|w|}\), and \(\chi_1\) for the character \(\chi_1(w) = (-1)^{|w|_a}\).

\begin{definition}
The \emph{fundamental mass distribution} on \(\F\) is the element
$$\mu = \frac{1}{4} \sum_{|w| = 1} \delta_w$$
of \(\ell^1(\F)\).

The \emph{fundamental charge distribution} on \(\F\) is the element
$$\lambda = \delta_\iota + \frac{1}{16} \sum_{|w| = 2} \chi_1(w) \delta_w$$
of \(\ell^1(\F)\).
\end{definition}

\subsection{Connes' embedding conjecture}
\label{sec:org8735eef}

Let \(A, B\) be C*-algebras. Let us recall the notion of extremal tensor products of C*-algebras.
Since \(A, B\) are rings, they have \emph{algebraic} tensor products \(A \otimes B\), taken in the category of complex algebras.
However, this does not give them a C*-norm.
Let \(A \otimes_{max} B\) be \(A \otimes B\) equipped with the largest possible norm such that \(A \otimes_{max} B\) is a C*-algebra.
Similarly define \(A \otimes_{min} B\).

If \(\Gamma\) is a countable discrete group, we have the C*-algebra \(C^*(\Gamma)\), the \emph{full group C*-algebra}, of \(\Gamma\), defined by the universal property that every unitary representation \(\rho: \Gamma \to B(H)\) of \(\Gamma\) extends uniquely to a representation \(\overline \rho: C^*(\Gamma) \to B(H)\).
Equivalently it is the completion of \(\ell^1(\Gamma)\) with respect to the C*-norm
$$||x|| = \sup_\rho ||\rho(x)||$$
where \(\rho\) ranges over unitary representations.

If \(\Delta\) is another discrete group, then we have canonical maps
$$\ell^1(\Gamma \times \Delta) \to C^*(\Gamma) \otimes C^*(\Delta)$$
arising from the above. So, if \(\phi \in \ell^1(\Gamma \times \Delta)\), we can define \(||\phi||_{min}\) and \(||\phi||_{max}\).

\begin{theorem}[negation of Connes' embedding conjecture]
There exists \(\phi \in \ell^1(\F^2)\) such that \(||\phi||_{max} > ||\phi||_{min}\).
\end{theorem}

We now set
$$h = 1 - \frac{||\phi||_{min}}{||\phi||_{max}} > 0.$$
Thus we are ready to state our main theorem.

\begin{theorem}[charged mean ergodic theorem]
For every unitary representation \(\rho\) of \(\F\), one has
$$||\rho(\lambda)||^2 \leq 1 - h.$$
\end{theorem}

\section{Proof of the charged ergodic theorem}
\label{sec:orgcd1e76f}

\subsection{Properties of \(\phi\)}
\label{sec:orgaf9d325}

Let \(\Gamma = \F^2\).
Suppose that \(||\phi||_{max} > ||\phi||_{min}\). By a rescaling we can assume \(||\phi||_{max} = 1\).
By the C*-identity,
$$||\phi^*\phi||_{max} = ||\phi||_{max}^2 = 1 > ||\phi||_{min}^2 = ||\phi^*\phi||_{min}$$
so it is no loss to assume that \(\phi\) is positive.

Since the space of functions of finite support on \(\Gamma\) is dense in \(\ell^1(\Gamma)\), we may also assume, after incurring an arbitrarily small loss, that \(\phi\) is supported on a set of pairs \((v, w)\) with \(\max(|v|, |w|) \leq r\).
Let \(R\) be the number of words of length \(\leq r\); then \(\phi\) is supported on a set of cardinality \(\leq R^2\).
Let \(M = ||\phi||_{\ell^1(\Gamma)}\).

Since \(||\phi||_{max} = 1\), there is a representation \(\rho\) for which \(||\phi(\rho)|| = 1\), i.e. there are unit vectors \(\psi_n\) such that
$$\lim_{n \to \infty} ||\psi_n - \rho(\phi)\psi_n|| = 0.$$

We write \(\phi^k\) for the convolution power of \(\phi\).
Since convolution increases the size of the support, there exists \(k\) such that \(\phi^k(v, w) = 1\) whenever \(|v| = |w| = 1\).
Moreover, \(\rho(\phi^k) = \rho(\phi)^k\), since \(\rho\) is a representation of \(\ell^1(\Gamma)\).
In particular, it follows that
$$\lim_{n \to \infty} ||\psi_n - \rho(\phi^k)\psi_n|| = 0.$$
So for every \(\varepsilon > 0\) we can find \(n\) so large that
$$\max(||\psi_n - \rho(\phi^k)\psi_n||, ||\psi_n - \rho(\phi^{2k)}\psi_n||) \leq \frac{\varepsilon}{2M^{2k}R^{2k}}.$$
It's a fact from high school trigonometry that this implies
$$|\langle \rho(\phi^k)\psi_n, \psi_n\rangle| \geq \cos\left(\frac{\varepsilon}{2M^{2k}R^{2k}}\right).$$

\subsection{The sign of \(\phi\)}
\label{sec:org9cd1023}

Let \(S = \{a, b, a^{-1}, b^{-1}\}\).
Let us now prove the following lemma:

\begin{lemma}
There exists \(\alpha: S^2 \to \{-1, 1\}\) such that for every \((v, w) \in \Gamma\) and \((u, t) \in S^2\),
$$\sgn \phi^k(uv, tw) = \alpha(u, t) \sgn \phi^k(v, w).$$
\end{lemma}




\subsection{Computation of \(\alpha\)}
\label{sec:orgffd9c44}

\subsection{Proof of the ergodic theorem}
\label{sec:org2558108}
\end{document}