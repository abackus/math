\documentclass[reqno,11pt]{amsart}
\usepackage[letterpaper, margin=1in]{geometry}
\RequirePackage{amsmath,amssymb,amsthm,graphicx,mathrsfs,url,slashed,subcaption}
\RequirePackage[usenames,dvipsnames]{xcolor}
\RequirePackage[colorlinks=true,linkcolor=Red,citecolor=Green]{hyperref}
\RequirePackage{amsxtra}
\usepackage{cancel}
\usepackage{tikz-cd}
%\usepackage[T1]{fontenc}

% \setlength{\textheight}{9.3in} \setlength{\oddsidemargin}{-0.25in}
% \setlength{\evensidemargin}{-0.25in} \setlength{\textwidth}{7in}
% \setlength{\topmargin}{-0.25in} \setlength{\headheight}{0.18in}
% \setlength{\marginparwidth}{1.0in}
% \setlength{\abovedisplayskip}{0.2in}
% \setlength{\belowdisplayskip}{0.2in}
% \setlength{\parskip}{0.05in}
%\renewcommand{\baselinestretch}{1.05}

\title{Vector-valued maps of least gradient}
\author{Aidan Backus}
\address{Department of Mathematics, Brown University}
\email{aidan\_backus@brown.edu}
\date{\today}

\newcommand{\NN}{\mathbf{N}}
\newcommand{\ZZ}{\mathbf{Z}}
\newcommand{\QQ}{\mathbf{Q}}
\newcommand{\RR}{\mathbf{R}}
\newcommand{\CC}{\mathbf{C}}
\newcommand{\DD}{\mathbf{D}}
\newcommand{\PP}{\mathbf P}
\newcommand{\MM}{\mathbf M}
\newcommand{\II}{\mathbf I}
\newcommand{\Torus}{\mathbf T}
\newcommand{\Hyp}{\mathbf H}
\newcommand{\Sph}{\mathbf S}
\newcommand{\Group}{\mathbf G}
\newcommand{\GL}{\mathbf{GL}}
\newcommand{\Orth}{\mathbf{O}}
\newcommand{\SpOrth}{\mathbf{SO}}
\newcommand{\Ball}{\mathbf{B}}

\newcommand*\dif{\mathop{}\!\mathrm{d}}
\newcommand*\Dif{\mathop{}\!\mathrm{D}}

\DeclareMathOperator{\card}{card}
\DeclareMathOperator{\dist}{dist}
\DeclareMathOperator{\End}{End}
\DeclareMathOperator{\id}{id}
\DeclareMathOperator{\Hom}{Hom}
\DeclareMathOperator{\coker}{coker}
\DeclareMathOperator{\supp}{supp}
\DeclareMathOperator{\Vect}{Vect}
\DeclareMathOperator{\tr}{tr}


\DeclareMathOperator{\svd}{svd}
\DeclareMathOperator{\SVD}{SVD}

\newcommand{\Leaves}{\mathscr L}
\newcommand{\Lagrange}{\mathscr L}
\newcommand{\Hypspace}{\mathscr H}

\newcommand{\Chain}{\underline C}

\newcommand{\Two}{\mathrm{I\!I}}
\newcommand{\Ric}{\mathrm{Ric}}

\newcommand{\normal}{\mathbf n}
\newcommand{\radial}{\mathbf r}
\newcommand{\evect}{\mathbf e}
\newcommand{\vol}{\mathrm{vol}}

\newcommand{\diam}{\mathrm{diam}}
\newcommand{\Ell}{\mathrm{Ell}}
\newcommand{\inj}{\mathrm{inj}}
\newcommand{\Lip}{\mathrm{Lip}}
\newcommand{\sgn}{\operatorname{sgn}}
\newcommand{\MCL}{\mathrm{MCL}}
\newcommand{\Riem}{\mathrm{Riem}}

\newcommand{\frkg}{\mathfrak g}

\newcommand{\Mass}{\mathbf M}
\newcommand{\Comass}{\mathbf L}

\newcommand{\Min}{\mathrm{Min}}
\newcommand{\Max}{\mathrm{Max}}

\newcommand{\dfn}[1]{\emph{#1}\index{#1}}

\renewcommand{\Re}{\operatorname{Re}}
\renewcommand{\Im}{\operatorname{Im}}

\newcommand{\loc}{\mathrm{loc}}
\newcommand{\cpt}{\mathrm{cpt}}

\def\Japan#1{\left \langle #1 \right \rangle}

\newtheorem{theorem}{Theorem}[section]
\newtheorem{badtheorem}[theorem]{``Theorem"}
\newtheorem{prop}[theorem]{Proposition}
\newtheorem{lemma}[theorem]{Lemma}
\newtheorem{sublemma}[theorem]{Sublemma}
\newtheorem{proposition}[theorem]{Proposition}
\newtheorem{corollary}[theorem]{Corollary}
\newtheorem{conjecture}[theorem]{Conjecture}
\newtheorem{axiom}[theorem]{Axiom}
\newtheorem{assumption}[theorem]{Assumption}

\newtheorem{mainthm}{Theorem}
\renewcommand{\themainthm}{\Alph{mainthm}}

\newtheorem{claim}{Claim}[theorem]
\renewcommand{\theclaim}{\thetheorem\Alph{claim}}
% \newtheorem*{claim}{Claim}

\theoremstyle{definition}
\newtheorem{definition}[theorem]{Definition}
\newtheorem{remark}[theorem]{Remark}
\newtheorem{example}[theorem]{Example}
\newtheorem{notation}[theorem]{Notation}

\newtheorem{exercise}[theorem]{Discussion topic}
\newtheorem{homework}[theorem]{Homework}
\newtheorem{problem}[theorem]{Problem}

\makeatletter
\newcommand{\proofpart}[2]{%
  \par
  \addvspace{\medskipamount}%
  \noindent\emph{Part #1: #2.}
}
\makeatother



\numberwithin{equation}{section}


% Mean
\def\Xint#1{\mathchoice
{\XXint\displaystyle\textstyle{#1}}%
{\XXint\textstyle\scriptstyle{#1}}%
{\XXint\scriptstyle\scriptscriptstyle{#1}}%
{\XXint\scriptscriptstyle\scriptscriptstyle{#1}}%
\!\int}
\def\XXint#1#2#3{{\setbox0=\hbox{$#1{#2#3}{\int}$ }
\vcenter{\hbox{$#2#3$ }}\kern-.6\wd0}}
\def\ddashint{\Xint=}
\def\dashint{\Xint-}

\usepackage[backend=bibtex,style=alphabetic,giveninits=true]{biblatex}
\renewcommand*{\bibfont}{\normalfont\footnotesize}
\addbibresource{least_gradient_maps.bib}
\renewbibmacro{in:}{}
\DeclareFieldFormat{pages}{#1}

\newcommand\todo[1]{\textcolor{red}{TODO: #1}}


\begin{document}
% \begin{abstract}

% \end{abstract}

\maketitle

%%%%%%%%%%%%%%%%%%%%%%%%%%%%%%%%%%%%%%%%%%%%%%%%%%%%%%%
\section{Preliminaries on matrix norms}
In this paper we focus on the trace norm.
But it will be convenient to study all the Schatten-von Neumann norms at once.
To be precise, if $A \in \Hom(\RR^d, \RR^c)$ is a matrix, consider the positive factor
$$Q(A) := (AA^\dagger)^{1/2} \in \End(\RR^c).$$
Then introduce the \dfn{Schatten-von Neumann norm}
$$|A|_{sv^p} := \tr(Q(A)^p)^{1/p}$$
for $p \in [1, \infty)$ and extend to $p = \infty$ by taking limits.
For $p = 1$ this is the trace norm, for $p = 2$ it is the Frobenius norm, and for $p = \infty$ it is the spectral norm.

\section{Calibration tensors}
\begin{definition}
A matrix $R \in \Hom(\RR^d, \RR^c)$ is a \dfn{polar factor} of a matrix $A \in \Hom(\RR^d, \RR^c)$ if $|R|_{sv^\infty} \leq 1$ and
$$\tr(R^\dagger \dif u) = |\dif u|_{sv^1}.$$
\end{definition}

Recall Anzellotti theory \cite{Anzellotti1983}.
In my previous paper I showed that Anzellotti theory holds even on Riemannian manifolds.

\begin{definition}
We say that a measurable section $R: M \to \Hom(TM, \RR^c)$ \dfn{calibrates} a function $u \in BV(M \to \RR^c)$ if:
\begin{enumerate}
\item $\dif^* R = 0$.
\item $\||R|_{sv^\infty}\|_{L^\infty} \leq 1$.
\item In the sense of Anzellotti theory, $R$ is a polar factor of $\dif u$.
\end{enumerate}
\end{definition}

\subsection{Solving the Dirichlet problem}
\begin{proposition}\label{existence}
Let $h \in L^1(\partial M \to \RR^c)$.
Then there exists a map $u \in BV(M \to \RR^c)$ and a calibration $R$ of $u$, such that on $\partial M$,
\begin{equation}\label{Dirichlet condition}
R\normal_M \in \sgn(h - u).
\end{equation}
\end{proposition}

The proof is similar to \cite[Theorem 2.4]{Mazon14} (see also \cite[Proposition 5.9]{Andreu-Vaillo2004}), and the main novelty is to apply those techniques to the PDE 
\begin{equation}\label{Dask pLap}
\dif^* (Q(\dif u_p)^{p - 2} \dif u_p) = 0
\end{equation}
studied by \cite{daskalopoulos2022analytic} instead of the scalar $p$-Laplacian.
In doing so, we shall have to use the various Schatten-von Neumann norms, and so it is convenient to introduce the \dfn{relaxed functionals}, defined for $p \in [1, \infty]$, $h \in L^1(\partial M \to \RR^c)$, and $u \in BV(M \to \RR^c)$, as
$$\Phi_{h, p}(u) := \int_M |\dif u|_{sv^p} + \int_{\partial M} |h - u|.$$

\subsubsection{Proof for \texorpdfstring{$h \in W^{1/2, 2}$}{regular h}}
By the inverse trace inequality, there exists $v \in W^{1, 2}_h$, the space of $W^{1, 2}$ extensions of $h$, with 
$$\|v\|_{W^{1, 2}} \lesssim \|h\|_{W^{1/2, 2}}.$$
Then by H\"older's inequality, $v \in W^{1, p}_h$ for any $p \in [1, 2]$, so by a slight modification of \cite[Theorem 2.12]{daskalopoulos2022analytic}, we can solve (\ref{Dask pLap}) for $u_p \in W^{1, p}_h$.
Let $Q_p := Q(\dif u_p)$.

\begin{lemma}
There exists $C > 0$ independent of $p$ such that 
$$\||\dif u_p|_{sv^p}\|_{L^p} \leq C.$$
\end{lemma}
\begin{proof}
Integrating by parts, we see that 
$$\int_M \tr(Q_p^{p - 2} \dif u_p^\dagger \dif(v - u_p)) = 0$$
and hence
\begin{align*}
\int_M |\dif u_p|_{sv^p}^p
&= \int_M \tr(Q_p^{p - 2} \dif u_p^\dagger \dif u_p)
= \int_M \tr(Q_p^{p - 2} \dif u_p^\dagger \dif v).
\end{align*}
By von Neumann's trace inequality,
$$\tr(Q_p^{p - 2} \dif u_p^\dagger \dif v) \leq |Q_p^{p - 2} \dif u_p^\dagger|_{sv^q} |\dif v|_{sv^p} = |\dif u_p|_{sv^p}^q |\dif v|_{sv^p}$$
so by H\"older's inequality,
$$\int_M |\dif u_p|_{sv^p}^p \leq \int_M |\dif u_p|_{sv^p}^q |\dif v|_{sv^p} \leq \left(\int_M |\dif u_p|_{sv^p}^p\right)^{1/q} \||\dif v|_{sv^p}\|_{L^p}.$$
In particular, if $p \leq 2$, then by H\"older's inequality again,
$$\||\dif u_p|_{sv^p}\|_{L^p} \leq \||\dif v|_{sv^p}\|_{L^p} \lesssim_M \||\dif v|_{sv^2}\|_{L^2},$$
which does not depend on $p$ but only on $\|h\|_{W^{1/2, 2}}$.
\end{proof}

By H\"older's inequality,
$$\||\dif u_p|_{sv^1}\|_{L^1} \leq d \||\dif u_p|_{sv^p}\|_{L^1} \leq d C^{1/q} |M|^{1/p}$$
which is bounded as $p \to 1$ (that is, as $q \to \infty$).
Since the trace of $u_p$ along $\partial M$ does not depend on $p$, it follows from Poincar\'e's inequality that $(u_p)$ is uniformly bounded in $BV$.
By the isoperimetric inequality, if $r \in [1, r^*)$ where $r^* := \frac{d}{d - 1}$, then $BV$ embeds compactly in $L^r$.
So there exists $u \in BV$ such that along a subsequence, $u_p \to u$ in $L^r$ for any $r \in [1, r^*)$, and almost everywhere.

If $s < q$ then 
$$\int_M |Q_p^{p - 2} \dif u_p|_{sv^\infty}^s \leq \int_M |\dif u_p|_{sv^p}^{(p - 1)s} \leq \left(\int_M |\dif u_p|_{sv^p}^p\right)^{s/q} |M|^{1 - s/q} \leq C^{s/q} |M|^{1 - s/q}.$$
Taking $q \to \infty$ we see that for $s \in [1, \infty)$, $Q_p^{p - 2} \dif u_p$ is bounded in $L^s$, so there exists $R$ such that along a subsequence,
$$Q_p^{p - 2} \dif u_p \rightharpoonup R$$
in $L^s$.
If $\psi \in C^1(M \to \RR^d)$ then, by the weak convergence and some integrations by parts,
$$\int_M \psi^\dagger \dif^* R = -\int_M \dif \psi^\dagger R = -\lim_{p \to 1} \int_M \dif \psi^\dagger Q_p^{p - 2} \dif u_p = 0.$$
Therefore $\dif^* R = 0$.

\begin{lemma}
We have
$$\||R|_{sv^\infty}\|_{L^\infty} \leq 1.$$
\end{lemma}
\begin{proof}
Let 
$$Z_{p, \varepsilon} := \left\{|\dif u_p|_{sv^\infty} > \frac{1}{\varepsilon}\right\}$$
so that, by Chebyshev's inequality,
$$|Z_{p, \varepsilon}| \leq \varepsilon^p \int_M |\dif u_p|_{sv^\infty}^p \leq \varepsilon^p \int_M |\dif u_p|_{sv^p}^p \leq C^p \varepsilon^p.$$
Arguing as for $R$, we can find a measurable field of matrices $S_\varepsilon$ such that $Q_p^{p - 2} \dif u_p 1_{Z_{p, \varepsilon}} \rightharpoonup S_\varepsilon$ in $L^1$.
Moreover, by H\"older's inequality,
$$\int_{Z_{p, \varepsilon}} |\dif u_p|_{sv^\infty}^{p - 1} \leq d^{p - 1} \int_{Z_{p, \varepsilon}} |\dif u_p|_{sv^p}^{p - 1} \leq d^{p - 1} \left(\int_M |\dif u_p|^p\right)^{1/q} |Z_{p, \varepsilon}|^{1/p} \leq C^{1 + p/q} d^{p - 1} \varepsilon.$$
Taking $p \to 1$ we see that 
\begin{equation}\label{estimate on error S}
\int_M |S_\varepsilon|_{sv^\infty} \leq C\varepsilon.
\end{equation}
If we set $R_\varepsilon := R - S_\varepsilon$, then $Q_p^{p - 2} \dif u_p (1 - 1_{Z_{p, \varepsilon}}) \rightharpoonup R_\varepsilon$ in $L^1$.
Since 
$$\||\dif u_p|_{sv^\infty}^{p - 1} (1 - 1_{Z_{p, \varepsilon}})\|_{L^\infty} \leq \varepsilon^{1 - p},$$
it follows that $\||R_\varepsilon|_{sv^\infty}\|_{L^\infty} \leq 1$.
Combining this estimate with (\ref{estimate on error S}) and taking $\varepsilon \to 0$ we conclude the lemma.
\end{proof}

\begin{lemma}
$R$ is a polar factor of $\dif u$.
\end{lemma}
\begin{proof}
Let $\varphi \in C^1_\cpt(M)$ and $\psi := \varphi u_p$.
After integrating by parts,
$$0 = \int_M \tr(\dif \psi^\dagger Q_p^{p - 2} \dif u_p) = \int_M \varphi \tr(\dif u_p^\dagger Q_p^{p - 2} \dif u_p) + \int_M (\dif \varphi \otimes u_p)^\dagger Q_p^{p - 2} \dif u_p.$$
To control the first term, we use the cyclicity of trace:
$$\tr(\dif u_p^\dagger Q_p^{p - 2} \dif u_p) = \tr(Q_p^{p - 2} \dif u_p \dif u_p^\dagger) = \tr(Q_p^p) = |\dif u_p|_{sv^p}^p.$$
So by semicontinuity of total variation,
$$\lim_{p \to 1} \int_M \varphi \dif u_p^\dagger Q_p^{p - 2} \dif u_p = \lim_{p \to 1} \int_M \varphi |\dif u_p|_{sv^p}^p \geq \int_M \varphi |\dif u|_{sv^1}.$$
To control the second term, choose a H\"older pair $(r, s)$ with $r \in (1, r^*)$, so $s < \infty$.
Thus $Q_p^{p - 2} \dif u_p \rightharpoonup R$ in $L^s$ and $\dif \varphi \otimes u_p \to \dif \varphi \otimes u$ in $L^r$.
Integrating by parts using $\dif^* R = 0$, and applying the cyclicity of trace,
$$\lim_{p \to 1} \int_M \tr((\dif \varphi \otimes u_p)^\dagger Q_p^{p - 2} \dif u_p) = \int_M \tr((\dif \varphi \otimes u)^\dagger R) = -\int_M \varphi \tr(\dif u^\dagger R) = -\int_M \varphi \tr(R^\dagger \dif u),$$
hence
$$0 \geq \int_M \varphi |\dif u|_{sv^1} - \int_M \varphi \tr(R^\dagger \dif u).$$
Since $\varphi$ was arbitrary, and $\||R|_{sv^\infty}\|_{L^\infty} \leq 1$,
\begin{align*}
|\dif u|_{sv^1} &\leq \tr(R^\dagger \dif u) \leq |\dif u|_{sv^1}. \qedhere
\end{align*}
\end{proof}

In particular, $R$ is a calibration of $u$.
We now must check the boundary condition (\ref{Dirichlet condition}).
To do this, we estimate using Young's inequality
\begin{align*}
\Phi_{h, p}(u_p)
&= \int_M |\dif u_p|_{sv^p} 
\leq \frac{1}{p} \int_M |\dif u_p|^p + \frac{|M|}{q}.
\end{align*}
The first term is
\begin{align*}
\frac{1}{p} \int_M |\dif u_p|^p
&= \frac{1}{p} \int_M \tr(\dif u_p^\dagger Q_p^{p - 2} \dif u_p) + \frac{|M|}{q} 
= \frac{1}{p} \int_M \tr(\dif v^\dagger Q_p^{p - 2} \dif u_p) + \frac{|M|}{q}.
\end{align*}
Taking the limit as $p \to 1$ and using the lower semicontinuity of the relaxed functionals $\Phi_{h, p}$, we conclude 
$$\Phi_{h, 1}(u) \leq \int_M \tr(\dif v^\dagger R) = \int_{\partial M} \tr(v^\dagger R\normal_M) = \int_{\partial M} \tr(h^\dagger R\normal_M).$$
Therefore 
$$\Phi_{h, 1}(u) \leq \int_{\partial M} \tr((h - u)^\dagger R\normal_M) + \int_{\partial M} \tr(u^\dagger R\normal_M)$$
and, since $R$ is a polar factor of $\dif u$,
$$\int_{\partial M} \tr(u^\dagger R\normal_M) = \int_M \tr(R^\dagger \dif u) = \int_M |\dif u|_{sv^1}.$$
In summary,
$$\int_{\partial M} |u - h| \leq \int_{\partial M} \tr((h - u)^\dagger R\normal_M) \leq \int_{\partial M} |u - h|,$$
implying the boundary condition.

\subsubsection{The general case}
Let $h_n \in W^{1/2, 2}(\partial M)$ converge to $h$ in $L^1$, and let $v_n, u_n, R_n$ be as in the $W^{1/2, 2}$ case.
After taking a subsequence, there exists $R$ such that $\dif^* R = 0$ and $R_n \rightharpoonup^* R$.
\todo{This part should be basically the same, I'll write it out later}

\subsection{The main theorem}
\begin{theorem}
The following are equivalent for $u \in BV(M \to \RR^c)$:
\begin{enumerate}
\item $u$ has least gradient.
\item There exists a calibration of $u$.
\end{enumerate}
\end{theorem}
\begin{proof}
Let $R$ be a calibration of $u$, and let $v$ be a competitor. Then
$$0 = \int_M (v - u)^\dagger \dif^* R = \int_M \tr(R^\dagger \dif u) - \int_M \tr(R^\dagger \dif v).$$
Since $R$ is a polar factor of $\dif u$,
$$\int_M |\dif u|_{sv^1} = \int_M \tr(R^\dagger \dif u) = \int_M \tr(R^\dagger \dif v) \leq \int_M |\dif v|_{sv^1},$$
so $u$ has least gradient. \todo{Add in the relaxed functionals}

Conversely, let $u$ be a map of least gradient, let $h := u|_{\partial M}$.
By Lemma \ref{existence} there exists a map $v$ and a calibration $R$ of $v$ satisfying the boundary condition (\ref{Dirichlet condition}), and \todo{since $u$ has least gradient,
$$\Phi_{h, 1}(u) \leq \Phi_{h, 1}(v)$$
which implies $|\dif u|_{sv^1} = \tr(R^\dagger \dif u)$ and the Dirichlet condition}.
\end{proof}

\subsection{Examples}
\begin{enumerate}
\item Functions of least gradient: This is obvious.
\item The identity map: the identity matrix is a calibration.
\item If $u_j: M \to \RR^{c_j}$ are maps of least gradient, then their direct sum
\begin{align*}
u: M^n &\to \RR^{c_1} \times \cdots \times \RR^{c_n} \\
(x_1, \dots, x_n) &\mapsto (u_1(x_1), \dots, u_n(x_n))
\end{align*}
is a map of least gradient.
Indeed, if $R_j$ is a calibration of $u_j$, then the matrix 
$$\begin{bmatrix}R_1 \\ & \ddots \\ && R_n\end{bmatrix}$$
is a calibration of $u$. 
\item Increasing corotational maps: If $\overline u: \RR_+ \to \RR_+$ is a nondecreasing function with $\overline u(r) \to 0$ as $r \to 0$, then in polar coordinates, 
$$u(r, \theta) := (\overline u(r), \theta)$$
has least gradient.
In fact, the identity matrix is a calibration.
To see this, for each $x \in \RR^d \setminus \{0\}$ we observe that if we take $\partial_r$ to be the unit radial vector, and $\partial_{\theta_1}, \dots, \partial_{\theta_{d - 1}}$ to be an orthonormal basis for $T_x S$ where $S$ is the sphere centered on $0$ with $x \in S$, then $\partial_r, \partial_{\theta_1}, \dots, \partial_{\theta_{d - 1}}$ is an orthonormal basis of $T_x \RR^d$. 
In this basis, 
$$\dif u = \begin{bmatrix}\overline u' \\ & \id_{\RR^{d - 1}}\end{bmatrix}.$$
If $\overline u'$ is nonnegative then it is obvious that $\id_{\RR^d}$ is a polar factor of this matrix. 
At $0$, we use the continuity of $\overline u$ at $0$ to see that $u$ does not jump near $0$, so $0$ doesn't contribute to $\int_M |\dif u|_{sv^1}$ and we can excise it. \todo{Write this out more carefully...}
\item Conserved currents for $\infty$-harmonic maps from surfaces: see below.
\end{enumerate}

\subsection{Non-examples}
\begin{enumerate}
\item Rotation-invariant maps. Let $\overline u: \RR_+ \to \RR_+$ and $u(r, \theta) := (\overline u(r), 0)$.
Then $u$ solves the Dirichlet problem for least gradient maps on a ball $\{r < r^*\}$ and $u$ is constant on $\{r = r^*\}$.
So $u$ is a competitor with the constant map on $\{r \leq r^*\}$ whose total variation is $0$.
\end{enumerate}

\section{The conservation law for the Schatten-von Neumann \texorpdfstring{$\infty$-Laplacian}{infinity-Laplacian}}
\begin{theorem}
Let $\tilde M$ be a simply connected surface, $N$ a Riemannian symmetric space, and let $f: \tilde M \to N$ be a $sv^\infty$ $\infty$-harmonic map with distinct singular values everywhere.
Let $\mathfrak g$ be the Lie algebra of translations of $N$, and let $u: \tilde M \to \mathfrak g$ be the potential for the conserved current corresponding to the action of $\mathfrak g$ on $N$. Then:
\begin{enumerate}
\item $u$ has least gradient.
\item $\star \dif f/\Lip(f)$ is a calibration of $u$.
\item $\supp \dif u \subseteq \MCL(f)$, the maximum Lipschitz locus of $f$.
\item $u$ has rank one.
\end{enumerate}
\end{theorem}

\todo{In order to formulate this, we can try to formulate Noether's theorem for the $L^\infty$ calculus of variations...}

\begin{example}
The $\infty$-harmonic map in the identity homotopy class from $[0, 1]^2 \to [0, 2]^2$ (edges glued) is $2$.
There's no geodesic lamination!
However, there is a geodesic lamination corresponding to $[0, 1]^2 \to [0, 1] \times [0, 2]$, namely the vertical foliation.
\end{example}

\begin{conjecture}
$\MCL(f)$ contains a geodesic lamination, $f$ is tangent to the geodesics, and $f$ maximally stretches the geodesics.
\end{conjecture}

This probably follows by combining the rank-one theorem with Topoganov comparison.
This conjecture is pretty ambitious though.

%%%%%%%%%%%%%%%%%%%
\section{The structure of a least gradient map at a jump discontinuity}
Intuitively the jump discontinuities shouldn't ``talk to'' the rest of the map, so it should like a calibrated hypersurface which gets crushed to a point.
By the rank-one theorem, $\dif^* R$ is a PDE for the level set versus the image set.

%%%%%%%%%%%%%%%%%%%
\section{Monotonicity formula}
\begin{theorem}
Let $u$ be a map of least gradient, with $sv^p$ where $p \leq 2$.
Then for $x \in M$ there exists $A \geq 0$ depending continuously on $x$ such that for every $0 < r \lesssim 1$,
$$\frac{\dif}{\dif r} \left[e^{Ar^2} r^{1 - d} \int_{B(x, r)} |\dif u|_{sv^p}\right] \geq 0.$$
\end{theorem}

It follows from the argument in \cite{Miranda66} where we use the estimate 
$$\left|\begin{bmatrix}A & B\end{bmatrix}\right|_{sv^p}^2 \leq |A|_{sv^p}^2 + |B|_{sv^p}^2$$
which holds for $p \leq 2$. (Or maybe it points the other way? Anyways, the case $p \leq 2$ is the one that works)

%%%%%%%%%%%%%%%%%%%

\section{Total variation flow}
\begin{conjecture}
Well-posedness of the Cauchy-Dirichlet problem for the total variation flow for any matrix norm.
\end{conjecture}

This has applications to image denoising if it's true \cite{GórnyMazón+2022}.
This is interesting enough that it could be its own paper, maybe.
It might also be a trivial consequence of the Dirichlet problem.

\printbibliography

\end{document}
