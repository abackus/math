\documentclass[reqno,11pt]{amsart}
\usepackage[letterpaper, margin=1in]{geometry}
\RequirePackage{amsmath,amssymb,amsthm,graphicx,mathrsfs,url,slashed,subcaption}
\RequirePackage[usenames,dvipsnames]{xcolor}
\RequirePackage[colorlinks=true,linkcolor=Red,citecolor=Green]{hyperref}
\RequirePackage{amsxtra}
\usepackage{cancel}
\usepackage{tikz-cd}
%\usepackage[T1]{fontenc}

% \setlength{\textheight}{9.3in} \setlength{\oddsidemargin}{-0.25in}
% \setlength{\evensidemargin}{-0.25in} \setlength{\textwidth}{7in}
% \setlength{\topmargin}{-0.25in} \setlength{\headheight}{0.18in}
% \setlength{\marginparwidth}{1.0in}
% \setlength{\abovedisplayskip}{0.2in}
% \setlength{\belowdisplayskip}{0.2in}
% \setlength{\parskip}{0.05in}
%\renewcommand{\baselinestretch}{1.05}

\title{Vector-valued maps of least gradient}
\author{Aidan Backus}
\address{Department of Mathematics, Brown University}
\email{aidan\_backus@brown.edu}
\date{\today}

\newcommand{\NN}{\mathbf{N}}
\newcommand{\ZZ}{\mathbf{Z}}
\newcommand{\QQ}{\mathbf{Q}}
\newcommand{\RR}{\mathbf{R}}
\newcommand{\CC}{\mathbf{C}}
\newcommand{\DD}{\mathbf{D}}
\newcommand{\PP}{\mathbf P}
\newcommand{\MM}{\mathbf M}
\newcommand{\II}{\mathbf I}
\newcommand{\Torus}{\mathbf T}
\newcommand{\Hyp}{\mathbf H}
\newcommand{\Sph}{\mathbf S}
\newcommand{\Group}{\mathbf G}
\newcommand{\GL}{\mathbf{GL}}
\newcommand{\Orth}{\mathbf{O}}
\newcommand{\SpOrth}{\mathbf{SO}}
\newcommand{\Ball}{\mathbf{B}}

\newcommand*\dif{\mathop{}\!\mathrm{d}}
\newcommand*\Dif{\mathop{}\!\mathrm{D}}

\DeclareMathOperator{\card}{card}
\DeclareMathOperator{\dist}{dist}
\DeclareMathOperator{\End}{End}
\DeclareMathOperator{\id}{id}
\DeclareMathOperator{\Hom}{Hom}
\DeclareMathOperator{\coker}{coker}
\DeclareMathOperator{\supp}{supp}
\DeclareMathOperator{\Vect}{Vect}
\DeclareMathOperator{\tr}{tr}


\DeclareMathOperator{\svd}{svd}
\DeclareMathOperator{\SVD}{SVD}

\newcommand{\Leaves}{\mathscr L}
\newcommand{\Lagrange}{\mathscr L}
\newcommand{\Hypspace}{\mathscr H}

\newcommand{\Chain}{\underline C}

\newcommand{\Two}{\mathrm{I\!I}}
\newcommand{\Ric}{\mathrm{Ric}}

\newcommand{\normal}{\mathbf n}
\newcommand{\radial}{\mathbf r}
\newcommand{\evect}{\mathbf e}
\newcommand{\vol}{\mathrm{vol}}

\newcommand{\diam}{\mathrm{diam}}
\newcommand{\Ell}{\mathrm{Ell}}
\newcommand{\inj}{\mathrm{inj}}
\newcommand{\Lip}{\mathrm{Lip}}
\newcommand{\sgn}{\operatorname{sgn}}
\newcommand{\MCL}{\mathrm{MCL}}
\newcommand{\Riem}{\mathrm{Riem}}

\newcommand{\frkg}{\mathfrak g}

\newcommand{\Mass}{\mathbf M}
\newcommand{\Comass}{\mathbf L}

\newcommand{\Min}{\mathrm{Min}}
\newcommand{\Max}{\mathrm{Max}}

\newcommand{\dfn}[1]{\emph{#1}\index{#1}}

\renewcommand{\Re}{\operatorname{Re}}
\renewcommand{\Im}{\operatorname{Im}}

\newcommand{\loc}{\mathrm{loc}}
\newcommand{\cpt}{\mathrm{cpt}}

\def\Japan#1{\left \langle #1 \right \rangle}

\newtheorem{theorem}{Theorem}[section]
\newtheorem{badtheorem}[theorem]{``Theorem"}
\newtheorem{prop}[theorem]{Proposition}
\newtheorem{lemma}[theorem]{Lemma}
\newtheorem{sublemma}[theorem]{Sublemma}
\newtheorem{proposition}[theorem]{Proposition}
\newtheorem{corollary}[theorem]{Corollary}
\newtheorem{conjecture}[theorem]{Conjecture}
\newtheorem{axiom}[theorem]{Axiom}
\newtheorem{assumption}[theorem]{Assumption}

\newtheorem{mainthm}{Theorem}
\renewcommand{\themainthm}{\Alph{mainthm}}

\newtheorem{claim}{Claim}[theorem]
\renewcommand{\theclaim}{\thetheorem\Alph{claim}}
% \newtheorem*{claim}{Claim}

\theoremstyle{definition}
\newtheorem{definition}[theorem]{Definition}
\newtheorem{remark}[theorem]{Remark}
\newtheorem{example}[theorem]{Example}
\newtheorem{notation}[theorem]{Notation}

\newtheorem{exercise}[theorem]{Discussion topic}
\newtheorem{homework}[theorem]{Homework}
\newtheorem{problem}[theorem]{Problem}

\makeatletter
\newcommand{\proofpart}[2]{%
  \par
  \addvspace{\medskipamount}%
  \noindent\emph{Part #1: #2.}
}
\makeatother



\numberwithin{equation}{section}


% Mean
\def\Xint#1{\mathchoice
{\XXint\displaystyle\textstyle{#1}}%
{\XXint\textstyle\scriptstyle{#1}}%
{\XXint\scriptstyle\scriptscriptstyle{#1}}%
{\XXint\scriptscriptstyle\scriptscriptstyle{#1}}%
\!\int}
\def\XXint#1#2#3{{\setbox0=\hbox{$#1{#2#3}{\int}$ }
\vcenter{\hbox{$#2#3$ }}\kern-.6\wd0}}
\def\ddashint{\Xint=}
\def\dashint{\Xint-}

\usepackage[backend=bibtex,style=alphabetic,giveninits=true]{biblatex}
\renewcommand*{\bibfont}{\normalfont\footnotesize}
\addbibresource{least_gradient_maps.bib}
\renewbibmacro{in:}{}
\DeclareFieldFormat{pages}{#1}

\newcommand\todo[1]{\textcolor{red}{TODO: #1}}


\begin{document}
\begin{abstract}

\end{abstract}

\maketitle

%%%%%%%%%%%%%%%%%%%%%%%%%%%%%%%%%%%%%%%%%%%%%%%%%%%%%%%
\section{Calibrated solutions}
\begin{definition}
A matrix $R$ is \dfn{suborthogonal} if $R^\dagger R \leq 1$.
The suborthogonal matrix $R$ is a \dfn{polar factor} of a matrix $A$ if
$$\tr(R^\dagger \dif u) = |\dif u|_{sv^1}.$$
\end{definition}

\begin{lemma}
Let $R$ be a suborthogonal matrix. Then
\begin{equation}\label{suborthogonal matrices are sub1}
|R|_{sv^\infty} \leq 1.
\end{equation}
\end{lemma}
\begin{proof}
This follows because $|R|_{sv^\infty}^2$ is the top eigenvalue of the positive matrix $R^\dagger R \leq 1$.
\end{proof}

\begin{definition}
Let $R$ be a Borel measurable field of suborthogonal matrices.
We say that $R$ \dfn{calibrates} a function $u \in BV(M \to \RR^d)$ if:
\begin{enumerate}
\item $R$ is divergence-free.
\item For $|\dif u|$-almost every $x \in M$, $R(x)$ is a polar factor of $\dif u(x)$.
\end{enumerate}
\end{definition}

\begin{lemma}\label{existence}
Let $h \in L^1(\partial M \to \RR^d)$.
Then there exists a map $u \in BV(M \to \RR^d)$ and a calibration $R$ of $u$, such that for any $X \in C^0(\partial M \to \RR^d)$,
\begin{equation}\label{Dirichlet condition}
X^\dagger R\normal_M \in \sgn(X^\dagger(h - u))
\end{equation}
in the sense of Anzellotti theory.
\end{lemma}
\begin{proof}
Let's do this for $h \in W^{1/2, 2}$.
By the inverse trace inequality, there exists $v \in W^{1, 2}_h$, the space of $W^{1, 2}$ extensions of $h$, with 
$$\|v\|_{W^{1, 2}} \lesssim \|h\|_{W^{1/2, 2}}.$$
Solve
$$\begin{cases}
-\dif^\dagger (Q(\dif u_p)^{p - 2} \dif u_p) = 0, \\
u_p|_{\partial M} = h 
\end{cases}.$$
Let $Q_p := Q(\dif u_p)$.
Integrating by parts, we see that 
$$\int_M \tr(Q_p^{p - 2} \dif u_p^\dagger \dif(v - u_p)) = 0$$
and hence
\begin{align*}
\int_M |\dif u_p|_{sv^p}^p
&= \int_M \tr(Q_p^{p - 2} \dif u_p^\dagger \dif u_p)
= \int_M \tr(Q_p^{p - 2} \dif u_p^\dagger \dif v).
\end{align*}
By von Neumann's trace inequality ($(p, q)$ is a H\"older pair)
$$\tr(Q_p^{p - 2} \dif u_p^\dagger \dif v) \leq |Q_p^{p - 2} \dif u_p^\dagger|_{sv^q} |\dif v|_{sv^p} = |\dif u_p|_{sv^p}^q |\dif v|_{sv^p}$$
so by H\"older's inequality,
$$\int_M |\dif u_p|_{sv^p}^p \leq \int_M |\dif u_p|_{sv^p}^q |\dif v|_{sv^p} \leq \left(\int_M |\dif u_p|_{sv^p}^p\right)^{1/q} \||\dif v|_{sv^p}\|_{L^p}.$$
In particular, if $p \leq 2$, then by H\"older's inequality again,
$$\||\dif u_p|_{sv^p}\|_{L^p} \leq \||\dif v|_{sv^p}\|_{L^p} \lesssim_M \||\dif v|_{sv^2}\|_{L^2},$$
which does not depend on $p$ but only on $\|h\|_{W^{1/2, 2}}$.
Another application of H\"older's inequality shows that $\||\dif u_p|_{sv^p}\|_{L^1}$ is similarly bounded independently of $p$.
Since the trace of $u_p$ along $\partial M$ does not depend on $p$, it follows from Poincar\'e's inequality that $(u_p)$ is uniformly bounded in $BV$.
By the isoperimetric inequality, if $r \in [1, r^*)$ where $r^* := \frac{d}{d - 1}$, then $BV$ embeds compactly in $L^r$.
So there exists $u \in BV$ such that along a subsequence, $u_p \to u$ in $L^r$ for any $r \in [1, r^*)$, and almost everywhere.

The hard part: studying the limiting behavior of $Q_p^{p - 2}$..
\end{proof}

\begin{theorem}
The following are equivalent for $u \in BV(M \to \RR^d)$:
\begin{enumerate}
\item $u$ has least gradient.
\item There exists a calibration of $u$.
\end{enumerate}
\end{theorem}
\begin{proof}
Let $R$ be a calibration of $u$, and let $v$ be a competitor. Then
$$0 = \int_M (v - u) \nabla \cdot R = \int_M \tr(R^\dagger \dif u) - \int_M \tr(R^\dagger \dif v).$$
Since $R$ is a polar factor of $\dif u$,
$$\int_M |\dif u|_{sv^1} = \int_M \tr(R^\dagger \dif u) = \int_M \tr(R^\dagger \dif v).$$
By von Neumann's trace inequality and (\ref{suborthogonal matrices are sub1}),
$$\int_M \tr(R^\dagger \dif v) \leq \int_M |R^\dagger|_{sv^\infty} |\dif v|_{sv^1} \leq \int_M |\dif v|_{sv^1},$$
so $u$ has least gradient.

Conversely, let $u$ be a map of least gradient, let $h := u|_{\partial M}$.
By Lemma \ref{existence} there exists a map $v$ and a calibration $R$ of $v$ satisfying the boundary condition (\ref{Dirichlet condition})...
\end{proof}


\printbibliography

\end{document}
