\documentclass[reqno,11pt]{amsart}
\usepackage[letterpaper, margin=1in]{geometry}
\RequirePackage{amsmath,amssymb,amsthm,graphicx,mathrsfs,url,slashed,subcaption}
\RequirePackage[usenames,dvipsnames]{xcolor}
\RequirePackage[colorlinks=true,linkcolor=Red,citecolor=Green]{hyperref}
\RequirePackage{amsxtra}
\usepackage{cancel}
\usepackage{tikz-cd}
%\usepackage[T1]{fontenc}

% \setlength{\textheight}{9.3in} \setlength{\oddsidemargin}{-0.25in}
% \setlength{\evensidemargin}{-0.25in} \setlength{\textwidth}{7in}
% \setlength{\topmargin}{-0.25in} \setlength{\headheight}{0.18in}
% \setlength{\marginparwidth}{1.0in}
% \setlength{\abovedisplayskip}{0.2in}
% \setlength{\belowdisplayskip}{0.2in}
% \setlength{\parskip}{0.05in}
%\renewcommand{\baselinestretch}{1.05}

\title{Vector-valued maps of least gradient}
\author{Aidan Backus}
\address{Department of Mathematics, Brown University}
\email{aidan\_backus@brown.edu}
\date{\today}

\newcommand{\NN}{\mathbf{N}}
\newcommand{\ZZ}{\mathbf{Z}}
\newcommand{\QQ}{\mathbf{Q}}
\newcommand{\RR}{\mathbf{R}}
\newcommand{\CC}{\mathbf{C}}
\newcommand{\DD}{\mathbf{D}}
\newcommand{\PP}{\mathbf P}
\newcommand{\MM}{\mathbf M}
\newcommand{\II}{\mathbf I}
\newcommand{\Hyp}{\mathbf H}
\newcommand{\Sph}{\mathbf S}
\newcommand{\Group}{\mathbf G}
\newcommand{\GL}{\mathbf{GL}}
\newcommand{\Orth}{\mathbf{O}}
\newcommand{\SpOrth}{\mathbf{SO}}
\newcommand{\Ball}{\mathbf{B}}

\newcommand*\dif{\mathop{}\!\mathrm{d}}
\newcommand*\Dif{\mathop{}\!\mathrm{D}}

\DeclareMathOperator{\card}{card}
\DeclareMathOperator{\dist}{dist}
\DeclareMathOperator{\End}{End}
\DeclareMathOperator{\id}{id}
\DeclareMathOperator{\Hom}{Hom}
\DeclareMathOperator{\coker}{coker}
\DeclareMathOperator{\supp}{supp}
\DeclareMathOperator{\Vect}{Vect}
\DeclareMathOperator{\tr}{tr}

\newcommand{\Leaves}{\mathscr L}
\newcommand{\Lagrange}{\mathscr L}
\newcommand{\Hypspace}{\mathscr H}

\newcommand{\Chain}{\underline C}

\newcommand{\Two}{\mathrm{I\!I}}
\newcommand{\Ric}{\mathrm{Ric}}

\newcommand{\normal}{\mathbf n}
\newcommand{\radial}{\mathbf r}
\newcommand{\evect}{\mathbf e}
\newcommand{\vol}{\mathrm{vol}}

\newcommand{\diam}{\mathrm{diam}}
\newcommand{\Ell}{\mathrm{Ell}}
\newcommand{\inj}{\mathrm{inj}}
\newcommand{\Lip}{\mathrm{Lip}}
\newcommand{\MCL}{\mathrm{MCL}}
\newcommand{\Riem}{\mathrm{Riem}}

\newcommand{\frkg}{\mathfrak g}

\newcommand{\Mass}{\mathbf M}
\newcommand{\Comass}{\mathbf L}

\newcommand{\Min}{\mathrm{Min}}
\newcommand{\Max}{\mathrm{Max}}

\newcommand{\dfn}[1]{\emph{#1}\index{#1}}

\renewcommand{\Re}{\operatorname{Re}}
\renewcommand{\Im}{\operatorname{Im}}

\newcommand{\loc}{\mathrm{loc}}
\newcommand{\cpt}{\mathrm{cpt}}

\def\Japan#1{\left \langle #1 \right \rangle}

\newtheorem{theorem}{Theorem}[section]
\newtheorem{badtheorem}[theorem]{``Theorem"}
\newtheorem{prop}[theorem]{Proposition}
\newtheorem{lemma}[theorem]{Lemma}
\newtheorem{sublemma}[theorem]{Sublemma}
\newtheorem{proposition}[theorem]{Proposition}
\newtheorem{corollary}[theorem]{Corollary}
\newtheorem{conjecture}[theorem]{Conjecture}
\newtheorem{axiom}[theorem]{Axiom}
\newtheorem{assumption}[theorem]{Assumption}

\newtheorem{mainthm}{Theorem}
\renewcommand{\themainthm}{\Alph{mainthm}}

\newtheorem{claim}{Claim}[theorem]
\renewcommand{\theclaim}{\thetheorem\Alph{claim}}
% \newtheorem*{claim}{Claim}

\theoremstyle{definition}
\newtheorem{definition}[theorem]{Definition}
\newtheorem{remark}[theorem]{Remark}
\newtheorem{example}[theorem]{Example}
\newtheorem{notation}[theorem]{Notation}

\newtheorem{exercise}[theorem]{Discussion topic}
\newtheorem{homework}[theorem]{Homework}
\newtheorem{problem}[theorem]{Problem}

\makeatletter
\newcommand{\proofpart}[2]{%
  \par
  \addvspace{\medskipamount}%
  \noindent\emph{Part #1: #2.}
}
\makeatother



\numberwithin{equation}{section}


% Mean
\def\Xint#1{\mathchoice
{\XXint\displaystyle\textstyle{#1}}%
{\XXint\textstyle\scriptstyle{#1}}%
{\XXint\scriptstyle\scriptscriptstyle{#1}}%
{\XXint\scriptscriptstyle\scriptscriptstyle{#1}}%
\!\int}
\def\XXint#1#2#3{{\setbox0=\hbox{$#1{#2#3}{\int}$ }
\vcenter{\hbox{$#2#3$ }}\kern-.6\wd0}}
\def\ddashint{\Xint=}
\def\dashint{\Xint-}

\usepackage[backend=bibtex,style=alphabetic,giveninits=true]{biblatex}
\renewcommand*{\bibfont}{\normalfont\footnotesize}
\addbibresource{least_gradient_maps.bib}
\renewbibmacro{in:}{}
\DeclareFieldFormat{pages}{#1}

\newcommand\todo[1]{\textcolor{red}{TODO: #1}}


\begin{document}
\begin{abstract}

\end{abstract}

\maketitle

%%%%%%%%%%%%%%%%%%%%%%%%%%%%%%%%%%%%%%%%%%%%%%%%%%%%%%%
\section{Introduction}
Applications to image processing \cite{Goldluecke10} and Teichm\"uller theory \cite{daskalopoulos2022analytic}.

Unlike in \cite{BackusCML}, least gradient maps DO NOT correspond to minimal laminations, since the relevant coarea formula is 
$$\int_M f \star \sqrt{|\det(\dif u \otimes \dif u)|} = \int_V \int_{\{u = \lambda\}} f \dif S_\lambda \dif \lambda$$
but $\sqrt{|\det(\dif u \otimes \dif u)|}$ is obviously not a norm on $\Hom(TM, V)$, and it's not even coercive.
Maybe this means there's some kind of illposedness theorem for minimal laminations in higher codimension, because it's not coercive?

\section{Least gradient maps}
\subsection{The definition of least gradient}
Let $M$ be a Riemannian manifold and $V \to M$ a metric Lipschitz vector bundle, with $L^\infty$ metric connection $\Dif$.

Let $|\cdot|$ be a norm on the bundle $T'M \otimes V$ of color-twisted $1$-forms.
We call the norm \dfn{color-isotropic} if for every $R \in \SpOrth(V)$, $x \in M$, and every $v \in \Hom(T_x M, V)$, $|v| = |Rv|$.

\begin{definition}
The \dfn{total variation} of a map $u: M \to V$ is $\int_M \star |\Dif u|$.
A map $u: M \to V$ has \dfn{least gradient} if $u$ is a minimizer of its total variation (for the Dirichlet or Neumann problems).
\end{definition}

\subsection{Derivatives of operator norms}
\begin{theorem}
Let $E$ be a Banach space, $f(x) := |x|$, and $S$ the set of $x \in E$ such that $f$ is Frech\'et differentiable at $x$.
Then the Frech\'et derivative $\dif f: S \to E'$ has the following properties:
\begin{enumerate}
\item $\langle \dif f(x), x\rangle = |x|$. \label{operator norm props formula}
\item $S$ is a cone, and $\dif f$ is $0$-homogeneous. \label{operator norm props homo}
\item Let $|\cdot|'$ be the dual norm on $E'$. Then $|\dif f(x)|' = 1$. \label{operator norm props duality}
\end{enumerate}
\end{theorem}
\begin{proof}
We first recall that
$$\langle \dif f(x), y\rangle = \lim_{\varepsilon \to 0} \frac{|x + \varepsilon y| - |x|}{\varepsilon}.$$
If $y = x$, then
$$\frac{|x + \varepsilon x| - |x|}{\varepsilon} = \frac{(1 + \varepsilon)|x| - |x|}{\varepsilon} = \frac{\varepsilon |x|}{\varepsilon} = |x|,$$
proving (\ref{operator norm props formula}).
If $\lambda > 0$ is scalar, then
\begin{align*}
\langle \dif f(\lambda x), y\rangle
&= \lim_{\varepsilon \to 0} \frac{|\lambda x + \varepsilon y| - |\lambda x|}{\varepsilon} \\
&= \lim_{\varepsilon \to 0} \frac{\lambda}{\varepsilon} \left(\left|x + \frac{\varepsilon}{\lambda} y\right| - |x|\right) \\
&= \lambda \left\langle \dif f(x), \frac{y}{\lambda}\right\rangle = \langle \dif f(x), y\rangle,
\end{align*}
proving (\ref{operator norm props homo}).
Finally, we have 
$$\frac{|x + \varepsilon y| - |x|}{\varepsilon} \leq \frac{|x| + \varepsilon |y| - |x|}{\varepsilon} = |y|.$$
Taking $\varepsilon \to 0$ we conclude (\ref{operator norm props duality}).
\end{proof}


\subsection{Formal Euler-Lagrange equation}
\begin{definition}
Let $f$ be a $0$-homogeneous map in the commutative diagram 
$$\begin{tikzcd}
\Hom(TM, V) \setminus \{0\} \arrow["f"]{rr} \arrow[two heads]{dr} && \Hom(TM, V) \setminus \{0\} \arrow[two heads]{dl} \\ & M 
\end{tikzcd}$$
such that $|f| \leq 1$ almost everywhere.
We say that $u \in BV_\loc(U, V)$ is a \dfn{calibrated solution} of 
\begin{equation}\label{EL}
\Dif^* f(\Dif u) = 0
\end{equation}
if there exists $X \in L^\infty(U, \Hom(TM, V)) \cap L^\infty(U, \Hom(TM, V), \dif u)$ such that $\Dif^* X = 0$, $|X| \leq 1$, and for $\Dif u$-almost every $x \in \supp \Dif u$, 
$$X(x) = f(\Dif u(x)).$$
\end{definition}

This needs to be justified with some serious measure theory...

The calibrated solutions form a presheaf.
Thus we can take the sheafification.

\begin{definition}
A section of the sheafification of the presheaf of calibrated solutions of the PDE (\ref{EL}) is a \dfn{locally calibrated solution} of (\ref{EL}).
\end{definition}


What is the Euler-Lagrange equation for least gradient maps?

Do least gradient maps exist? What about if $V$ is a Lipschitz vector bundle with (necessarily $L^\infty$) connection?

\begin{definition}
A \dfn{calibrated solution} of (EL) is like in \cite{Mazon14}.
A \dfn{locally calibrated solution} is a section of the sheafification of the presheaf of calibrated solutions.
\end{definition}

Minimizers are calibrated. The correct notion of solution is locally calibrated solutions.
Show this using $p$-harmonic approximation like in \cite{Mazon14}.

\section{\texorpdfstring{$\infty$-harmonic}{Infinity-harmonic} maps from surfaces to symmetric spaces}
Let $f: M \to N$ be a (Schatten-von Neumann) $\infty$-harmonic map.
Since only the Schatten-von Neumann $\infty$-harmonic maps are best Lipschitz, by an $\infty$-harmonic map we tacitly mean a Schatten-von Neumann $\infty$-harmonic map.

Since $f$ is best Lipschitz, the pullback bundle $f^*(TN)$ is a Lipschitz Riemannian bundle, with $L^\infty$ connection $\Dif(f)$.
See \cite{Oh_2018} for more about connections in low regularity.
Note carefully that the curvature of such a bundle is NOT DEFINED, except possibly in some sense of twisted Schwartz distributions.

If $N$ is hyperbolic and $M$ is a surface, WTS the object in \cite[Theorem 3.13]{daskalopoulos2022analytic} is the Schatten-von Neumann least gradient section $u$ of $f^*(TN)$.
Also WTS that if $N$ is a symmetric space and $M$ is a surface, then $\dif u$ is the Noetherian flux of the $\infty$-harmonic map.
(Actually it's not very clear where \cite{daskalopoulos2022analytic} actually uses the hypothesis that $M$ is a surface, but this seems important as otherwise you should get tight forms??)

\subsection{The conservation law for \texorpdfstring{$p$-harmonic}{p-harmonic} maps}
Let $u: M \to N$ be a (Schatten-von Neumann) $p$-harmonic map where $2 < p < \infty$.
The Lagrangian is 
$$\Lagrange(\xi) = \frac{|\xi|_{sv^p}^p}{p}$$
whose derivative is 
$$\frac{\partial \Lagrange}{\partial \xi} = Q(\xi, \xi)^{\frac{p - 2}{2}} \xi.$$

Let $\frkg$ be the Lie algebra of Killing fields on $N$. Strictly speaking, this is a bundle, but let's ignore this point for now.
The Noether current arising from $\frkg$ is 
$$(J, X) := \frac{\partial \Lagrange}{\partial \xi} X = Q(\nabla u, \nabla u)^{\frac{p - 2}{2}} (\nabla u, X) = Q(\nabla u, \nabla u)^{\frac{p - 2}{2}} \nabla (u \cdot X).$$
The point is that parallel transporting $|\dif u|^p$ along $X$ doesn't do anything, since $X$ is a Killing field, so we don't pick up the $\Lambda^X$ term (which would otherwise be a correction).
If $N$ is a hyperbolic manifold, then $\frkg$ is a Lorentz algebra, hence semisimple, and we can think of $J$ as a map into $\frkg$ using the Killing form.

The Noether current satisfies $\nabla \cdot J = 0$.
We can think of it as a $d - 1$-form by writing 
$$\dif\left(Q(\dif u)^{p - 2} \star \dif(u \cdot X)\right) = 0.$$
\todo{$\frkg$ might not have a positive inner product so what does $Q(\dif u)$ mean?}
Therefore we are entitled to define the \dfn{Noetherian potential}
$$\dif v = Q(\dif u)^{p - 2} \star \dif u,$$
so that $v$ is a $d - 2$-form valued in $\frkg'$.
If $M$ is a surface, then $v$ is a $\frkg'$-valued scalar field.

\begin{proposition}
Let $u$ be a $p$-harmonic map, and let $v$ be the Noetherian potential.
Then $v$ is $q$-harmonic:
$$\dif^* (Q(\dif v)^{q - 2} \dif v) = 0.$$
\end{proposition}
\begin{proof}
Just compute 
$$Q(\dif v)^{q - 2} \star \dif v = Q(\dif u)^{\frac{(q - 2)(p - 1)}{2}} Q(\dif u)^{p - 2} \star^2 \dif u = \pm \dif u.$$
Taking $\dif$ of both sides gives $0$.
\end{proof}

In particular $v$ is a minimizer of the Schatten-von Neumann $q$-Dirichlet energy.
So it should converge as $p \to \infty$ to a least gradient map, at least after normalization. 
This suggets that maybe we don't care about the vector bundle formalism, and the Lipschitzness doesn't actually matter.

\todo{This seems too good to be true, check everything carefully to make sure that this isn't some crazy Lie algebra bundle with $L^\infty$ connection}

\appendix 
\section{Conservation laws valued in a Lie algebra}
I am not aware of any expositions of Noether's theorem that make it explicit that the conservation law is valued in the dual of the Lie algebra.
So here's a proof from which that result is obvious.
It would probably be more natural to view $M \times N$ as a bundle over $M$ but whatever.

Let
$$J^1(M, N) := \{(x, y, \xi): x \in M, y \in N, \xi \in \Hom(T_x M, T_y N)\}$$
be the $1$-jet bundle with values in $N$.
Let $\Vect(M)$ denote the Lie algebra of smooth vector fields on $M$.
We say that $\frkg$ acts on $J^1(M, N)$ if we are given a Lie algebra homomorphism $\frkg \to \Vect(M)$, and a map $M \times \frkg \to \Vect(N)$ such that each fiber is a homomorphism.
We write $X_1$ for the vector field on $M$ and $X_2$ for the family of vector fields on $N$.
In that case we define the flow map
$$\varphi(X)(x, y, \xi) := \left(\varphi(X_1, x)x, \varphi(X_2, x, y)y, \varphi(\nabla X_2, x, y)\xi\right).$$
A Lagrangian $\Lagrange: J^1(M, N) \to \RR$ is pulled back by $X \in \frkg$ as 
$$\Lagrange^X(x, y, \xi) := \Lagrange\left(x, \varphi(X_2, x)y, \varphi(\nabla X_2, x, y)\xi\right) + \nabla_x \cdot \left[\Lagrange(x, y, \xi) X_1(x)\right].$$
We say that the Lagrangian $\Lagrange$ is \dfn{preserved} by $\frkg$ if for every critical point $u$ of the action 
$$I(u) := \int_M \Lagrange(x, u(x), \nabla u(x)) \dif V(x)$$
there exists a linear map $\frkg \to \Vect(M)$, $X \mapsto \Lambda^X$, such that 
$$\Lagrange^X(x, u(x), \nabla u(x)) - \Lagrange(x, u(x), \nabla u(x)) = \nabla \cdot \Lambda^X(x) + O(|X|^2).$$
In that case we define the \dfn{Noether current} $J: M \to \Hom(\frkg, TM)$, 
$$(J(x), X) := \Lagrange(x, u(x), \nabla u(x)) X_1(x) + \frac{\partial \Lagrange}{\partial \xi}(x, u(x), \nabla u(x)) \cdot X_2(x, u(x)) - \Lambda^X(x).$$
We define the divergence $\nabla \cdot J: M \to \frkg'$ of a map $J: M \to \Hom(\frkg, TM)$ entrywise:
$$(\nabla \cdot J, X) := \nabla \cdot (J, X).$$
Recall that if $\frkg$ is semisimple (for example, $\frkg$ a Lorentz algebra), then the Killing form induces an isomorphism $\frkg \to \frkg'$ of vector spaces, so we can think of $J$ as a $\frkg$-valued vector field.

\begin{theorem}[Noether's theorem]
Suppose that $\Lagrange$ is preserved by $\frkg$ and $u$ is a critical point of the action $I$.
Then the Noether current $J$ associated to $u$ satisfies $\nabla \cdot J = 0$.
\end{theorem}
\begin{proof}
Given $X \in \frkg$, let $\delta_X u(x) := X_2(x, u(x))$ and $\delta_X (\nabla u)(x) := \nabla X_2(x, u(x)) \cdot \nabla u(x)$ be the variations of $u$, $\nabla u$ induced by flowing along $X_2$.
Then by the chain rule, 
$$\nabla(\delta_X u)(x) = \nabla X_2(x, u(x)) \cdot \nabla u(x) = \delta_X (\nabla u)(x).$$
By the chain rule, 
$$\Lagrange\left(\varphi(X)u, \varphi(\nabla X, u)\nabla u\right) - \Lagrange(u, \dif u) = \frac{\partial \Lagrange}{\partial y}(u, \dif u) \cdot \delta_X u + \frac{\partial \Lagrange}{\partial \xi}(u, \nabla u) \cdot \delta_X(\nabla u) + O(|X|^2).$$
By the Euler-Lagrange equation 
$$\nabla \cdot \frac{\partial \Lagrange}{\partial \xi}(u, \nabla u) = \frac{\partial \Lagrange}{\partial y}(u, \nabla u),$$
we obtain 
\begin{align*}
&\Lagrange\left(\varphi(X_1)u, \varphi(\nabla X_2, u)\nabla u\right) - \Lagrange(u, \nabla u) \\
&\qquad = \nabla_x \cdot \left[\frac{\partial \Lagrange}{\partial \xi}(u, \nabla u)\right] \cdot \delta_X u + \frac{\partial \Lagrange}{\partial \xi}(u, \nabla u) \cdot \delta_X (\nabla u) + O(|X|^2) \\
&\qquad = \nabla_x \cdot \left[\frac{\partial \Lagrange}{\partial \xi}(u, \nabla u)\right] \cdot \delta_X u + \frac{\partial \Lagrange}{\partial \xi}(u, \nabla u) \cdot \nabla (\delta_X u) + O(|X|^2) \\
&\qquad = \nabla \cdot \left[\frac{\partial \Lagrange}{\partial \xi}(u, \nabla u) \cdot \delta_X u\right] + O(|X|^2).
\end{align*}
By the fact that $\Lagrange$ is preserved by $\frkg$, there is a vector field $\Lambda_X$ which depends linearly on $X$ such that
\begin{align*}
  \nabla \cdot \Lambda^X 
  &= \Lagrange^X(u, \nabla u) - \Lagrange(u, \nabla u) + O(|X|^2) \\
  &= \nabla_x \cdot \left[\Lagrange(u, \nabla u) X_1 + \frac{\partial \Lagrange}{\partial \xi}(u, \nabla u) \cdot X_2(u)\right] + O(|X|^2).
\end{align*}
Comparing linear parts, we get 
$$\nabla \cdot \left[\Lambda^X - \Lagrange(u, \nabla u) X_1 + \frac{\partial \Lagrange}{\partial \xi}(u, \nabla u) \cdot X_2(u) \right] = 0$$
and since $X$ was arbitrary we conclude.
\end{proof}

\section{\texorpdfstring{$L^\infty$}{L-infinity} gauge fields}
Let $V$ be a topological vector bundle over a Lipschitz (or even smooth) manifold $M$, which we shall call the \dfn{color bundle}.
We say that $V$ is \dfn{Lipschitz} if its transition maps are locally Lipschitz.
We are going to establish a similar theory to the theory of $W^{2, d/2}$ vector bundles in \cite{Oh_2018}.

The composition of Lipschitz maps is Lipschitz, so Lipschitz vector bundles pull back by Lipschitz maps.
A Lipschitz vector bundle admits locally Lipschitz sections and a locally Lipschitz metric.
Since the transpose of a matrix-valued Lipschitz map is Lipschitz, the dual bundle of a Lipschitz bundle is Lipschitz; since the product of matrix-valued Lipschitz maps is Lipschitz, the tensor product of Lipschitz bundles is Lipschitz.
In particular, if $V, W$ are Lipschitz bundles, then $\Hom(V, W) = V' \otimes W$ is Lipschitz.

Let $V$ be a Lipschitz vector bundle of rank $r$, with trivializations $(U_\alpha)$ and transition maps $(\psi_{\alpha \beta})$.
A \dfn{$L^\infty_\loc$ gauge field} $A$ consists of local sections $A_\alpha \in L^\infty_\loc(U_\alpha, T'M \otimes \RR^{r \times r})$ satisfying the usual transition conditions
$$A_\alpha = [\psi_{\alpha \beta}, A_\beta] - \dif \psi_{\alpha \beta}.$$
The \dfn{$L^\infty$ connection} $\Dif$ induced by an $L^\infty_\loc$ gauge field $A$ satisfies, for each local section $s \in W^{1, \infty}_\loc(U_\alpha, V)$,
$$\Dif s = \dif s + A_\alpha s$$
(where as usual we used the trivialization to identify $W^{1, \infty}_\loc(U_\alpha, V)$ with $W^{1, \infty}_\loc(U_\alpha, \RR^r)$).
Note that the curvature of an $L^\infty$ connection must be understood as a $2$-form of Lie algebra-valued distributions of order $1$.
The pullback of an $L^\infty$ connection by a Lipschitz map is $L^\infty$.

Let $s \in BV_\loc(M, V)$. Then $\dif s$ is well-defined as a $V$-valued Radon measure and $A_\alpha s$ is well-defined in $L^\infty_\loc$; since $L^\infty_\loc$ sections can be understood as (Radon-Nikod\'ym derivatives of) $V$-valued Radon measures, $\Dif s$ makes sense as a $T'M \otimes V$-valued Radon measure, or as a $V$-valued normal $d - 1$-current.
Such currents are so common that we shall call them \dfn{color-twisted}.

Let $N$ be a Riemannian manifold, and $f: M \to N$ be a locally Lipschitz map.
Then $f^* (TN)$ is a Lipschitz metric vector bundle, and $f^* \nabla_N$ is an $L^\infty$ metric connection.
This is the setting in which we shall need the above machinery: $f$ is best Lipschitz, and we are interested in $BV_\loc$ sections of the Lipschitz bundle $f^* (TN)$.



\printbibliography

\end{document}
