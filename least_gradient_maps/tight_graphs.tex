\documentclass[reqno,11pt]{amsart}
\usepackage[letterpaper, margin=1in]{geometry}
\RequirePackage{amsmath,amssymb,amsthm,graphicx,mathrsfs,url,slashed,subcaption}
\RequirePackage[usenames,dvipsnames]{xcolor}
\RequirePackage[colorlinks=true,linkcolor=Red,citecolor=Green]{hyperref}
\RequirePackage{amsxtra}
\usepackage{cancel}
\usepackage{tikz-cd}
%\usepackage[T1]{fontenc}

% \setlength{\textheight}{9.3in} \setlength{\oddsidemargin}{-0.25in}
% \setlength{\evensidemargin}{-0.25in} \setlength{\textwidth}{7in}
% \setlength{\topmargin}{-0.25in} \setlength{\headheight}{0.18in}
% \setlength{\marginparwidth}{1.0in}
% \setlength{\abovedisplayskip}{0.2in}
% \setlength{\belowdisplayskip}{0.2in}
% \setlength{\parskip}{0.05in}
%\renewcommand{\baselinestretch}{1.05}

\title{Optimal Lipschitz maps between Riemannian manifolds}
\author{Aidan Backus}
\address{Department of Mathematics, Brown University}
\email{aidan\_backus@brown.edu}
\date{\today}

\newcommand{\NN}{\mathbf{N}}
\newcommand{\ZZ}{\mathbf{Z}}
\newcommand{\QQ}{\mathbf{Q}}


\newcommand{\RR}{\mathbf{R}}
\newcommand{\CC}{\mathbf{C}}
\newcommand{\DD}{\mathbf{D}}
\newcommand{\PP}{\mathbf P}
\newcommand{\MM}{\mathbf M}
\newcommand{\II}{\mathbf I}
\newcommand{\Torus}{\mathbf T}
\newcommand{\Hyp}{\mathbf H}
\newcommand{\Sph}{\mathbf S}
\newcommand{\Group}{\mathbf G}
\newcommand{\GL}{\mathbf{GL}}
\newcommand{\Orth}{\mathbf{O}}
\newcommand{\SpOrth}{\mathbf{SO}}
\newcommand{\Ball}{\mathbf{B}}

\newcommand*\dif{\mathop{}\!\mathrm{d}}
\newcommand*\Dif{\mathop{}\!\mathrm{D}}

\DeclareMathOperator{\card}{card}
\DeclareMathOperator{\dist}{dist}
\DeclareMathOperator{\End}{End}
\DeclareMathOperator{\id}{id}
\DeclareMathOperator{\Hom}{Hom}
\DeclareMathOperator{\coker}{coker}
\DeclareMathOperator{\supp}{supp}
\DeclareMathOperator{\Vect}{Vect}
\DeclareMathOperator{\tr}{tr}

\DeclareMathOperator{\asinh}{asinh}
\DeclareMathOperator{\sech}{sech}

\DeclareMathOperator{\svd}{svd}
\DeclareMathOperator{\SVD}{SVD}

\newcommand{\Leaves}{\mathscr L}
\newcommand{\Lagrange}{\mathscr L}
\newcommand{\Hypspace}{\mathscr H}

\newcommand{\Chain}{\underline C}

\newcommand{\Two}{\mathrm{I\!I}}
\newcommand{\Ric}{\mathrm{Ric}}

\newcommand{\normal}{\mathbf n}
\newcommand{\radial}{\mathbf r}
\newcommand{\evect}{\mathbf e}
\newcommand{\vol}{\mathrm{vol}}
\newcommand{\Gr}{\mathrm{Gr}}
\newcommand{\Sec}{\mathrm{sec}}

\newcommand{\diam}{\operatorname{diam}}
\newcommand{\dom}{\mathrm{dom}}
\newcommand{\Ell}{\mathrm{Ell}}
\newcommand{\inj}{\mathrm{inj}}
\newcommand{\Lip}{\mathrm{Lip}}
\newcommand{\ncf}{\operatorname{ncf}}
\newcommand{\sgn}{\operatorname{sgn}}
\newcommand{\MCL}{\mathrm{MCL}}
\newcommand{\Riem}{\mathrm{Riem}}

\DeclareMathOperator*{\essinf}{ess\,inf}
\DeclareMathOperator*{\esssup}{ess\,sup}

\newcommand{\frkg}{\mathfrak g}

\newcommand{\Mass}{\mathbf M}
\newcommand{\Comass}{\mathbf L}

\newcommand{\Min}{\mathrm{Min}}
\newcommand{\Max}{\mathrm{Max}}

\newcommand{\dfn}[1]{\emph{#1}\index{#1}}

\renewcommand{\Re}{\operatorname{Re}}
\renewcommand{\Im}{\operatorname{Im}}

\newcommand{\loc}{\mathrm{loc}}
\newcommand{\cpt}{\mathrm{cpt}}

\def\Japan#1{\left \langle #1 \right \rangle}

\newtheorem{theorem}{Theorem}[section]
\newtheorem{badtheorem}[theorem]{``Theorem"}
\newtheorem{prop}[theorem]{Proposition}
\newtheorem{lemma}[theorem]{Lemma}
\newtheorem{sublemma}[theorem]{Sublemma}
\newtheorem{proposition}[theorem]{Proposition}
\newtheorem{corollary}[theorem]{Corollary}
\newtheorem{conjecture}[theorem]{Conjecture}
\newtheorem{axiom}[theorem]{Axiom}
\newtheorem{assumption}[theorem]{Assumption}

\newtheorem{mainthm}{Theorem}
\renewcommand{\themainthm}{\Alph{mainthm}}

\newcommand{\weakto}{\rightharpoonup}

\newtheorem{claim}{Claim}[theorem]
\renewcommand{\theclaim}{\thetheorem\Alph{claim}}
% \newtheorem*{claim}{Claim}

\theoremstyle{definition}
\newtheorem{definition}[theorem]{Definition}
\newtheorem{remark}[theorem]{Remark}
\newtheorem{example}[theorem]{Example}
\newtheorem{notation}[theorem]{Notation}

\newtheorem{exercise}[theorem]{Discussion topic}
\newtheorem{homework}[theorem]{Homework}
\newtheorem{problem}[theorem]{Problem}

\makeatletter
\newcommand{\proofpart}[2]{%
  \par
  \addvspace{\medskipamount}%
  \noindent\emph{Part #1: #2.}
}
\makeatother



\numberwithin{equation}{section}


% Mean
\def\Xint#1{\mathchoice
{\XXint\displaystyle\textstyle{#1}}%
{\XXint\textstyle\scriptstyle{#1}}%
{\XXint\scriptstyle\scriptscriptstyle{#1}}%
{\XXint\scriptscriptstyle\scriptscriptstyle{#1}}%
\!\int}
\def\XXint#1#2#3{{\setbox0=\hbox{$#1{#2#3}{\int}$ }
\vcenter{\hbox{$#2#3$ }}\kern-.6\wd0}}
\def\ddashint{\Xint=}
\def\dashint{\Xint-}

\usepackage[backend=bibtex,style=alphabetic,giveninits=true]{biblatex}
\renewcommand*{\bibfont}{\normalfont\footnotesize}
\addbibresource{least_gradient_maps.bib}
\renewbibmacro{in:}{}
\DeclareFieldFormat{pages}{#1}

\newcommand\todo[1]{\textcolor{red}{TODO: #1}}


\begin{document}
\begin{abstract}

\end{abstract}

\maketitle
%%%%%%%%%%%%%%%%%%%%%%%%%%%%%%%%%%%%%%%%%%%%%%%%%%%%%%%
\section{Introduction}

\section{Discretization}
\subsection{Discretizing the Riemannian manifold}
Throughout we assume that there is a fixed continuous increasing function $\omega: [0, \infty) \to [0, \infty)$ such that $\omega(0) = 0$, called a \dfn{regulator}.
Intuitively, $\omega$ measures the oscillation of the Lipschitz moduli.

\begin{definition}
If $x \in M$, define for each $r > 0$,
$$\Lip_c(u, x, r) := \sup_{y \in \partial B(x, r)} \frac{|u(x) - u(y)|}{\dist(x, y)}$$
and introduce the \dfn{centered Lipschitz modulus}
$$\Lip_c(u, x) := \lim_{r \to 0} \Lip(u, x, r).$$
We say that a Lipschitz map $u: M \to \RR$ is \dfn{$\omega$-regulated} if for every $t > 0$ and $r < \omega(t)$,
$$|\Lip_c(u, x) - \Lip_c(u, x, r)| < t.$$
We also introduce the (uncentered) \dfn{Lipschitz modulus}
$$\Lip(u, x) := \lim_{r \to 0} \sup_{y, z \in B(x, r)} \frac{|u(y) - u(z)|}{\dist(y, z)}.$$
\end{definition}

\begin{conjecture}
Suppose that $M$ is compact and $u: M \to \RR$ is Lipschitz.
Then there exists a regulator $\omega$ such that $u$ is $\omega$-regulated.
\end{conjecture}

\begin{conjecture}
For any Lipschitz map $u$,
$\Lip(u, x) = \Lip_c(u, x)$.
\end{conjecture}

The following discretization will be suitable for studying $\omega$-regulated maps:

\begin{definition}
Given a Riemannian manifold $(M, g)$ and a regulator $\omega$, fix discretization parameters:
\begin{enumerate}
\item $\varepsilon_2$ is the scale that $M$ is discretized at.
Let $V$ be the set of all center points of a maximal packing of $M$ by $\varepsilon_2/2$-balls (so that $\{x \in V: B(x, \varepsilon_2)\}$ is an open cover of $M$).
It must be chosen small enough depending on $k, \varepsilon_0, \omega, \varepsilon_1$ and an absolute constant.
\item $\varepsilon_1$ is the scale that Lipschitz moduli are discretized at. Let
$$E = \{(x, y) \in V: \varepsilon_1 \leq \dist(x, y) < (1 + \varepsilon_0) \varepsilon_1\}.$$
It must be chosen small enough depending on $k, \varepsilon_0, \omega$.
\item $\varepsilon_0$ appears whenever we need a small parameter.
It must be chosen small enough depending on $k$, and smaller than an absolute contant.
Ultimately we are going to take $\varepsilon_0 \to 0$.
\item $k$ is the curvature scale of $g$, so that 
$$|\Riem_g| \leq \frac{1}{k^2}.$$
\end{enumerate}
We call the undirected graph $(V, E)$ the \dfn{Freedman-Headrick discretization} of $(M, g, \omega)$ at scale $\varepsilon_0$.
\end{definition}

This is basically the discretization of Freedman and Headrick \cite{Freedman2017}.
For the time being let's completely ignore the curvature of $g$ and take $k = \infty$.
We also assume that $M$ is compact (possibly with boundary).


\subsection{Discretizing the local Lipschitz constants}
\begin{definition}
Let $u: M \to \RR^m$ be a Lipschitz map.
If $x \in V$, introduce the \dfn{discrete Lipschitz modulus}
$$\Lip_d(u, x) := \frac{1}{\varepsilon_1} \sup_{y \sim x} |u(x) - u(y)|.$$
\end{definition}

\begin{lemma}
Suppose that $u$ is $\omega$-regulated, $\varepsilon_1 < \omega(\varepsilon_0)$, and $\varepsilon_2 < \varepsilon_0 \varepsilon_1$.
Then 
$$|\Lip_d(u, x) - \Lip_c(u, x)| < (1 + 2 \Lip(u)) \varepsilon_0.$$
\end{lemma}
\begin{proof}
Given $y \in \partial B(x, \varepsilon_1)$ there exists $y' \in B(y, \varepsilon_2)$ such that $y' \sim x$.
We then estimate 
\begin{align*}
|u(x) - u(y)|
&\leq |u(x) - u(y') + |u(y) - u(y')| \\
&\leq \Lip_d(u, x) \varepsilon_1 + \Lip(u) \varepsilon_2 \\
&= \Lip_d(u, x) |x - y| + \Lip(u) \varepsilon_2.
\end{align*}
Thus 
$$\Lip_c(u, x, \varepsilon_1) \leq \Lip_d(u, x) + \Lip(u) \frac{\varepsilon_2}{\varepsilon_1}.$$
Therefore
$$\Lip_c(u, x) < \Lip_d(u, x) + (1 + \Lip(u)) \varepsilon_0.$$
Conversely, given $y \sim x$ there exists $y' \in \partial B(x, \varepsilon_1)$ such that
$$\dist(y, y') < 2\varepsilon_0\varepsilon_1.$$
Then 
\begin{align*}
|u(x) - u(y)| 
&\leq |u(x) - u(y')| + |u(y) - u(y')| \\
&< \Lip_c(u, x, \varepsilon_1) \varepsilon_1 + 2\Lip(u) \varepsilon_0 \varepsilon_1.
\end{align*}
It follows that 
\begin{align*}
\Lip_d(u, x) 
&< \Lip_c(u, x, \varepsilon_1) + 2\Lip(u) \varepsilon_0 \\
&\leq \Lip_c(u, x) + (1 + 2 \Lip(u)) \varepsilon_0. \qedhere 
\end{align*}
\end{proof}

\begin{lemma}
Suppose that $u: M \to \RR$ is tight and $\omega$-regulated.
Then for any $\omega$-regulated $v$,
$$\sup_{\Lip_d v < \Lip_d u} \Lip_d u < \sup_{\Lip_d u < \Lip_d v} \Lip_d v + ...$$
\end{lemma}
\begin{proof}
Suppose that $\Lip_d(v, x) < \Lip_d(u, x)$.?
\end{proof}



\printbibliography

\end{document}
