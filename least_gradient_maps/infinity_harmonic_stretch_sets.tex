\documentclass[reqno,11pt]{amsart}
\usepackage[letterpaper, margin=1in]{geometry}
\RequirePackage{amsmath,amssymb,amsthm,graphicx,mathrsfs,url,slashed,subcaption}
\RequirePackage[usenames,dvipsnames]{xcolor}
\RequirePackage[colorlinks=true,linkcolor=Red,citecolor=Green]{hyperref}
\RequirePackage{amsxtra}
\usepackage{cancel}
\usepackage{tikz-cd}
%\usepackage[T1]{fontenc}

% \setlength{\textheight}{9.3in} \setlength{\oddsidemargin}{-0.25in}
% \setlength{\evensidemargin}{-0.25in} \setlength{\textwidth}{7in}
% \setlength{\topmargin}{-0.25in} \setlength{\headheight}{0.18in}
% \setlength{\marginparwidth}{1.0in}
% \setlength{\abovedisplayskip}{0.2in}
% \setlength{\belowdisplayskip}{0.2in}
% \setlength{\parskip}{0.05in}
%\renewcommand{\baselinestretch}{1.05}

\title[$\infty$-harmonic maps absolutely minimize]{$\infty$-harmonic maps from surfaces are absolute minimizers}
\author{Aidan Backus}
\address{Department of Mathematics, Brown University}
\email{aidan\_backus@brown.edu}
\date{\today}

\newcommand{\NN}{\mathbf{N}}
\newcommand{\ZZ}{\mathbf{Z}}
\newcommand{\QQ}{\mathbf{Q}}
\newcommand{\RR}{\mathbf{R}}
\newcommand{\CC}{\mathbf{C}}
\newcommand{\DD}{\mathbf{D}}
\newcommand{\PP}{\mathbf P}
\newcommand{\MM}{\mathbf M}
\newcommand{\II}{\mathbf I}
\newcommand{\Torus}{\mathbf T}
\newcommand{\Hyp}{\mathbf H}
\newcommand{\Sph}{\mathbf S}
\newcommand{\Group}{\mathbf G}
\newcommand{\GL}{\mathbf{GL}}
\newcommand{\Orth}{\mathbf{O}}
\newcommand{\SpOrth}{\mathbf{SO}}
\newcommand{\Ball}{\mathbf{B}}

\newcommand*\dif{\mathop{}\!\mathrm{d}}
\newcommand*\Dif{\mathop{}\!\mathrm{D}}

\DeclareMathOperator{\card}{card}
\DeclareMathOperator{\dist}{dist}
\DeclareMathOperator{\End}{End}
\DeclareMathOperator{\id}{id}
\DeclareMathOperator{\Hom}{Hom}
\DeclareMathOperator{\coker}{coker}
\DeclareMathOperator{\supp}{supp}
\DeclareMathOperator{\Vect}{Vect}
\DeclareMathOperator{\tr}{tr}


\DeclareMathOperator{\svd}{svd}
\DeclareMathOperator{\SVD}{SVD}

\newcommand{\Leaves}{\mathscr L}
\newcommand{\Lagrange}{\mathscr L}
\newcommand{\Hypspace}{\mathscr H}

\newcommand{\Chain}{\underline C}

\newcommand{\Two}{\mathrm{I\!I}}
\newcommand{\Ric}{\mathrm{Ric}}

\newcommand{\normal}{\mathbf n}
\newcommand{\radial}{\mathbf r}
\newcommand{\evect}{\mathbf e}
\newcommand{\vol}{\mathrm{vol}}

\newcommand{\diam}{\mathrm{diam}}
\newcommand{\dom}{\mathrm{dom}}
\newcommand{\Ell}{\mathrm{Ell}}
\newcommand{\inj}{\mathrm{inj}}
\newcommand{\Lip}{\mathrm{Lip}}
\newcommand{\sgn}{\operatorname{sgn}}
\newcommand{\MCL}{\mathrm{MCL}}
\newcommand{\Riem}{\mathrm{Riem}}

\DeclareMathOperator*{\esssup}{ess\,sup}

\newcommand{\frkg}{\mathfrak g}

\newcommand{\Mass}{\mathbf M}
\newcommand{\Comass}{\mathbf L}

\newcommand{\Min}{\mathrm{Min}}
\newcommand{\Max}{\mathrm{Max}}

\newcommand{\dfn}[1]{\emph{#1}\index{#1}}

\renewcommand{\Re}{\operatorname{Re}}
\renewcommand{\Im}{\operatorname{Im}}

\newcommand{\loc}{\mathrm{loc}}
\newcommand{\cpt}{\mathrm{cpt}}

\def\Japan#1{\left \langle #1 \right \rangle}

\newtheorem{theorem}{Theorem}[section]
\newtheorem{badtheorem}[theorem]{``Theorem"}
\newtheorem{prop}[theorem]{Proposition}
\newtheorem{lemma}[theorem]{Lemma}
\newtheorem{sublemma}[theorem]{Sublemma}
\newtheorem{proposition}[theorem]{Proposition}
\newtheorem{corollary}[theorem]{Corollary}
\newtheorem{conjecture}[theorem]{Conjecture}
\newtheorem{axiom}[theorem]{Axiom}
\newtheorem{assumption}[theorem]{Assumption}

\newtheorem{mainthm}{Theorem}
\renewcommand{\themainthm}{\Alph{mainthm}}

\newcommand{\weakto}{\rightharpoonup}

\newtheorem{claim}{Claim}[theorem]
\renewcommand{\theclaim}{\thetheorem\Alph{claim}}
% \newtheorem*{claim}{Claim}

\theoremstyle{definition}
\newtheorem{definition}[theorem]{Definition}
\newtheorem{remark}[theorem]{Remark}
\newtheorem{example}[theorem]{Example}
\newtheorem{notation}[theorem]{Notation}

\newtheorem{exercise}[theorem]{Discussion topic}
\newtheorem{homework}[theorem]{Homework}
\newtheorem{problem}[theorem]{Problem}

\makeatletter
\newcommand{\proofpart}[2]{%
  \par
  \addvspace{\medskipamount}%
  \noindent\emph{Part #1: #2.}
}
\makeatother



\numberwithin{equation}{section}


% Mean
\def\Xint#1{\mathchoice
{\XXint\displaystyle\textstyle{#1}}%
{\XXint\textstyle\scriptstyle{#1}}%
{\XXint\scriptstyle\scriptscriptstyle{#1}}%
{\XXint\scriptscriptstyle\scriptscriptstyle{#1}}%
\!\int}
\def\XXint#1#2#3{{\setbox0=\hbox{$#1{#2#3}{\int}$ }
\vcenter{\hbox{$#2#3$ }}\kern-.6\wd0}}
\def\ddashint{\Xint=}
\def\dashint{\Xint-}

\usepackage[backend=bibtex,style=alphabetic,giveninits=true]{biblatex}
\renewcommand*{\bibfont}{\normalfont\footnotesize}
\addbibresource{least_gradient_maps.bib}
\renewbibmacro{in:}{}
\DeclareFieldFormat{pages}{#1}

\newcommand\todo[1]{\textcolor{red}{TODO: #1}}


\begin{document}
\begin{abstract}

\end{abstract}

\maketitle
%%%%%%%%%%%%%%%%%%%%%%%%%%%%%%%%%%%%%%%%%%%%%%%%%%%%%%%
\section{Introduction}
The prototypical PDE in the $L^\infty$ calculus of variations is the $\infty$-Laplacian 
$$\langle \nabla^2 u, \dif u \otimes \dif u\rangle = 0$$
whose viscosity solutions $u$ are characterized as absolutely minimizing Lipschitz functions:

\begin{definition}
Let $M$ be a Riemannian manifold.
A Lipschitz map $u: M \to \RR^d$ is \dfn{absolutely minimizing Lipschitz} if for every Lipschitz domain $U \subseteq M$, 
$$\Lip(u|_U) = \Lip(u|_{\partial U}).$$
\end{definition}

In fact, the equivalence between absolutely minimizing Lipschitz functions and $\infty$-harmonic functions is usually established by showing that both conditions are equivalent to comparison with cones \cite[\S2]{Crandall2008}.

To find absolutely minimizing Lipschitz maps $u: M \to N$ between manifolds, it is natural to try to solve the \dfn{spectral $\infty$-Laplacian}
$$\langle \nabla^2 u, v_1(\dif u) \otimes v_1(\dif u)\rangle = 0$$
where $v_1(\dif u)$ is the principal singular vector of $\dif u$.
Since $N$ does not have an order structure, we do not have the maximum principle; therefore the notions of ``viscosity solution'' and ``comparison with cones'' make no sense.
Sheffield and Smart addressed these issues by assuming higher regularity of $u$ and nondegeneracy of $v_1(\dif u)$ \cite{Sheffield2010VectorvaluedOL}, while Daskalopoulos and Uhlenbeck \emph{defined} the spectral $\infty$-Laplacian by a $p$-regularization scheme but were not able to prove that the solutions were absolute minimizers \cite{daskalopoulos2022analytic}.

In this paper we prove that spectral $\infty$-harmonic maps from surfaces to euclidean spaces are $\infty$-harmonic.
In this case the convex dual problem has a particularly simple structure, and so we are able to use the duality to establish the result.

It seems that our approach should be feasible for dealing with myriad $L^\infty$ variational problems with well-behaved dual problems.
For example, it is natural to expect that the \dfn{Frobenius $\infty$-Laplacian}
$$\langle \dif u, \dif (|\dif u|^2)\rangle = 0$$
is characterized as giving absolute minimizers of $\||\dif u|_{sv^2}\|_{L^\infty}$, where $|\cdot|_{sv^2}$ denotes the Frobenius norm.
We have not pursued this angle here because we are not aware of any geometric significance of the Frobenius norm of the derivative, though the Frobenius $\infty$-Laplacian has been studied in the geometry literature; see for example \cite{Ou2012}.

\subsection{Main theorem}
For a linear map $A: X \to Y$ between finite-dimensional Hilbert spaces, we write $A^\dagger: Y \to X$ for its adjoint, and $Q(A) := (AA^\dagger)^{1/2}$, which is a positive semidefinite endomorphism of $Y$.
We write $\sigma_1(A) \geq \cdots \geq \sigma_k(A)$ for the singular values of $A$, written in decreasing order.
For $p \in [1, \infty)$, we introduce the \dfn{Schatten-von Neumann norm}
$$|A|_{sv^p} := (\tr Q(A)^p)^{1/p},$$
and in the case $p = \infty$, we write 
$$|A|_{sv^\infty} = \sigma_1(A).$$
For $p = 1, 2, \infty$, we recall that $|A|_{sv^p}$ is the tracial, Frobenius, and spectral norm, respectively.

Fix a Riemannian surface $M$, possibly with boundary, and a euclidean space $\RR^d$.
We say that a map $u_p \in W^{1, p}(M, \RR^d)$ is \dfn{Schatten-von Neumann $p$-harmonic}, if it solves the equation
$$\dif^* (Q(\dif u_p)^{p - 2} \dif u_p) = 0.$$
Since $\RR^d$ is nonpositively curved, this is equivalent to minimizing the integral $\int_M |\dif u_p|_{sv^p}^p $
among all compactly supported perturbations of $u_p$.

\begin{definition}
We say that $u \in W^{1, \infty}(M, \RR^d)$ is a \dfn{variational solution} of the spectral $\infty$-Laplacian, or simply that $u$ is \dfn{spectral $\infty$-harmonic}, if there is a sequence of $p \to \infty$ such that there exist $u_p \in W^{1, p}(M, \RR^d)$ such that, for every $r \in [1, \infty)$, $u_p \weakto u$ in $W^{1, r}(M, \RR^d)$.
We call $u_p$ a \dfn{$p$-regularization} of $u$.
\end{definition}

\begin{mainthm}
Let $u$ be spectral $\infty$-harmonic.
Then $u$ is absolutely minimizing Lipschitz.
\end{mainthm}

%%%%%%%%%%%%%%%%%%%%%%%%%%%
\subsection{Acknowledgements}
I would like to thank Georgios Daskalopoulos for helpful comments.

This research was supported by the National Science Foundation's Graduate Research Fellowship Program under Grant No. DGE-2040433.

% %%%%%%%%%%%%%%%
% \section{Formulating the problem}
% \subsection{Maps into Riemannian symmetric spaces}
% Throughout this paper, we fix a Riemannian manifold $M$, an open domain $U \subseteq M$ with smooth boundary, and a Riemannian symmetric space $N$.
% The Riemannian structure on $M$ will only have a minor role, however, and the reader may take $M = \RR^d$, $U = \Ball^d$.

% By the Nash embedding theorem, it is no loss to assume that $N$ is isometrically embedded into some euclidean space $\RR^D$.
% In particular, a choice of Nash embedding defines the Sobolev spaces $W^{s, p}(U, N)$, which are the Banach manifolds of maps $u \in W^{s, p}(U, \RR^D)$ such that for almost every $x \in U$, $u(x) \in N$.

% Given a map $u: M \to N$, we can consider $u^{-1}(TN)$-valued $k$-forms on $U$.
% These are sections of the bundle $\Omega^k \otimes u^{-1}(TN)$.
% In particular, $\dif u$ is an $u^{-1}(TN)$-valued $1$-form.

% %%%%%%%%%%%%%%
% \subsection{Calculus of variations in \texorpdfstring{$L^\infty$}{L-infinity}}
% Fix some $q \in (1, \infty)$.
% For a linear map $\xi: T_x M \to T_y N$, let $Q(\xi) := (\xi \xi^\dagger)^{1/2}$, and introduce the \dfn{Schatten-von Neumann norm}
% $$|\xi|_{sv^q} := (\tr Q(\xi)^q)^{1/q}.$$
% \todo{Add in the edge cases, which have to be regularized twice}
% Given a Lipschitz map $u: \overline U \to N$, and $x \in \overline U$, we introduce the \dfn{local maximum stretch}
% $$\Comass(u, x) := \inf_{\varepsilon > 0} \esssup_{x' \in B(x, \varepsilon)} |\dif u(x')|_{sv^q},$$
% which is well-defined by Rademacher's theorem.
% In fact, if $u$ is $C^1$, then $\Comass(u, x)$ is nothing more than $|\dif u(x)|_{sv^q}$. 

% \begin{lemma}
% For any Lipschitz map $u: \overline U \to N$, the local maximum stretch $x \mapsto \Comass(u, x)$ is upper semicontinuous.
% \end{lemma}
% \begin{proof}
% \todo{Prove me}
% \end{proof}

% Given a set $\Gamma \subseteq U$, we introduce the \dfn{maximum stretch}
% $$\Comass(u, \Gamma) := \sup_{x \in \Gamma} \Comass(u, x).$$
% By upper semicontinuity, if $V$ is an open set with smooth boundary,
% $$\Comass(u, \partial V) \leq \Comass(u, V).$$

% \begin{definition}
% A Lipschitz map $u: U \to N$ is an \dfn{absolute minimizer} with respect to the norm $|\cdot|_{sv^q}$ if for every open set $V \subseteq U$ with smooth boundary $\partial V$,
% $$\Comass(u, \partial V) = \Comass(u, V).$$
% \end{definition}

% The goal of this paper, and indeed of the $L^\infty$ calculus of variations, is to compute absolute minimizers.

% Let $\nabla_u$ denote the pullback of the Levi-Civita connection to $u^{-1}(TN)$.
% A computation with \cite[Theorem 5.2]{Barron2001} shows that \todo{check it} smooth absolute minimizer $u$ solves the \dfn{$sv^q$ $\infty$-Laplacian}
% \begin{equation}\label{EulerLagrangeAronsson}\tag{inf-Laplace}
% \langle \nabla_u \dif u, Q(\dif u)^{q - 2} \dif u \otimes Q(\dif u)^{q - 2} \dif u\rangle = 0,
% \end{equation}
% but on the other hand most absolute minimizers are not smooth.

% %%%%%%%%%%%%
% \subsection{\texorpdfstring{$p$-regularization}{p-regularization} of the PDE}
% At the time of writing, there is no suitably fleshed out notion of viscosity solution for totally nonlinear strongly coupled systems of PDE, such as (\ref{EulerLagrangeAronsson}).
% Thus, following Daskalopoulos and Uhlenbeck \cite{daskalopoulos2022analytic}, we regularize (\ref{EulerLagrangeAronsson}) by replacing it by a sequence of variational problems in $L^p$ as $p \to \infty$.

% To formally derive these variational problems, suppose that we are interested in an absolute minimizer $u$ of $\||\dif u|_{sv^q}\|_{L^\infty}$.
% We thus seek $u_p \in W^{1, p}(U, N)$ which minimizes 
% $$I_p(u_p) := \frac{1}{p} \int_U |\dif u_p|_{sv^q}^p \dif V.$$
% Let $u_p$ be a minimizer of $I_p$, and let $v$ be a section of $u_p^{-1}(TN)$.
% We have
% \begin{align*}
% 0 
% &= \frac{\dif}{\dif t} I_p(u_p + t\dif v)|_{t = 0}\\
% &= \frac{1}{p} \int_U \frac{\partial}{\partial t} |\dif u_p + t\dif v|_{sv^q}^{p/2} \dif V\bigg|_{t = 0} \\
% &= \int_U |\dif u_p|_{sv^q}^{p - q} Q(\dif u_p)^{q - 2} \langle \dif u_p, \nabla_{u_p} v\rangle \dif V.
% \end{align*}
% This is the weak formulation of the \dfn{$sv^q$ $p$-Laplacian}
% \begin{equation}\label{pReg}\tag{p-Laplace}
% \nabla^*_{u_p} \left[|\dif u_p|_{sv^q}^{p - q} Q(\dif u_p)^{q - 2} \dif u_p\right] = 0.
% \end{equation}

% \begin{definition}
% A Lipschitz map $u: \overline U \to N$ is a \dfn{variational solution} of (\ref{EulerLagrangeAronsson}) if there exists a sequence of $p \to \infty$, and $u_p \in W^{1, p}(U, N)$, such that:
% \begin{enumerate}
% \item $u_p|_{\partial U} = u|_{\partial U}$.
% \item $u_p$ is a weak solution of (\ref{pReg}).
% \item For every $r \in [1, \infty)$, $u_p \rightharpoonup u$ in $W^{1, r}$.
% \end{enumerate}
% We call $u_p$ a \dfn{$p$-regularization} of $u$.
% \end{definition}

% \begin{proposition}
% Let $h: \overline U \to N$ be a Lipschitz map.
% Then there exists a variational solution $u$ of (\ref{EulerLagrangeAronsson}) such that $u|_{\partial U} = h$.
% \end{proposition}
% \begin{proof}
% \todo{Use the coercivity. By Nash embedding, we're studying a constrained optimization problem in $W^{1, p}(\overline U, \RR^D)$.}
% \end{proof}

% If $N = \RR$, then it is more natural to study viscosity solutions of (\ref{EulerLagrangeAronsson}).
% However, in that case we are merely studying the $\infty$-Laplacian.
% By \cite[Theorem 12]{lindqvist2016notes}, every variational solution of the $\infty$-Laplacian is a viscosity solution; the converse follows from the uniqueness of viscosity solutions \cite[Theorem 27]{lindqvist2016notes}.

% \subsection{Differential forms}
% \todo{Do the problem formulation for differential forms too!}

% %%%%%%%%%%%%%%%%%%%%%

% \subsection{Statement of the main theorems}
% In both of these theorems, let $U$ be a smooth domain in a Riemannian manifold $M$, let $N$ be a Riemannian symmetric space, and let $|\cdot|_{sv^q}$ be a Schatten-von Neumann norm.

% \begin{mainthm}
% Every variational solution $u$ of (\ref{EulerLagrangeAronsson}) is an absolute minimizer of $\||\dif u|_{sv^q}\|_{L^\infty}$.
% \end{mainthm}

% \begin{mainthm}
% Every variational solution of \todo{the $sv^q$ Hodge system} is an absolute minimizer of $\||\dif u|_{sv^q}\|_{L^\infty}$.
% \end{mainthm}


% %%%%%%%%%%%%%%
% \section{The dual problems}
% \subsection{Conservation laws}
% We consider (\ref{EulerLagrangeAronsson}).
% We more or less follow \cite[\S3.5]{daskalopoulos2022analytic}.
% Since $N$ is a symmetric space, $N$ admits a product decomposition
% $$N = \prod_j N_j = \prod_j G_j / K_j$$
% where we write $o = (o_j)$, $o_j \in N_j$, and
% \begin{enumerate}
% \item $G_j$ is a Lie group, whose Lie algebra $\frkg_j$ is the Lie algebra of local isometries of $N_j$.
% \item The type of $\frkg_j$ is either euclidean, compact, or noncompact.
% \item $K_j \subset G_j$ is the isotropy group of $o_j$.
% \end{enumerate}
% Let $B_j$ be the Killing form on the Lie algebra $\frkg_j$, and let $g_j$ be the Riemannian metric on $N_j$.
% We introduce a nondegenerate indefinite inner product $\hat B_j$ on $\frkg_j$, as follows:
% \begin{enumerate}
% \item If $\frkg_j$ has euclidean type, then $N_j$ is flat, so there is a maximal abelian subalgebra $\mathfrak m_j \subset \frkg_j$ such that every tangent space $T_{y_j} N_j$ is identified with $\mathfrak m_j$. Therefore $g_j$ is a positive inner product on $\mathfrak m_j$. We choose any positive-definite extension $\hat B_j$ of $g_j$ to $\frkg_j$.
% \item If $\frkg_j$ has compact type, then $B_j$ is negative-definite and there exists $\lambda > 0$ such that on each tangent space $T_{y_j} N_j \subset \frkg_j$, we have $g_j = -\lambda B_j$. We set $\hat B_j := -\lambda^{-1} B_j$.
% \item If $\frkg_j$ has noncompact type, then $B_j$ is nondegenerate and there exists $\lambda > 0$ such that on each tangent space $T_{y_j} N_j \subset \frkg_j$, we have $g_j = \lambda B_j$. We set $\hat B_j := \lambda^{-1} B_j$.
% \end{enumerate}
% Given $v \in \frkg_j$ and $w \in \frkg_k$, let
% $$\hat B(v, w) := \begin{cases}
% \hat B_j(v, w),& j = k \\
% 0,& j \neq k.
% \end{cases}$$
% Thus $\hat B$ is a nondegenerate indefinite inner product on $\frkg := \oplus_j \frkg_j$.
% We say that $X \in \frkg$ is \dfn{spacelike} if $\hat B(X, X) > 0$ or $X = 0$, and write $P$ for the cone of spacelike vectors.
% Thus for every $y \in N$ we have an isometric inclusion $(T_y N, g) \subseteq (P, \hat B)$.

% We say that a $\frkg$-valued $k$-form $\xi$ is \dfn{spacelike} if every contraction $\langle \xi, v\rangle$ with a $k$-blade $v$ is spacelike.
% We write $\Omega^k \otimes P$ for the sheaf whose setions are spacelike $k$-forms.
% Crucially, if $u: U \to N$ is a smooth map, then $\dif u$ can be viewed as a $\frkg$-valued $1$-form, and $\dif u$ is spacelike.

% \begin{lemma}
% The Euler-Lagrange equation (\ref{pReg}) can be rewritten as 
% $$\int_U \langle |\dif u_p|_{sv^q}^{p - q} Q(\dif u_p)^{q - 2} \dif u_p, \dif \varphi\rangle \dif V = 0$$
% for every $\varphi \in W^{1, p}(U, \frkg)$ such that $\varphi(x) \in T_{u_p(x)} N$.
% \end{lemma}
% \begin{proof}
% Since $\nabla_{u_p}$ is the projection of the trivial connection $\dif$ to $u_p^{-1}(TN)$, and $\varphi$ is a section of $u_p^{-1}(TN)$, $\nabla_{u_p} \varphi = \dif \varphi$. \todo{Explain it better}
% \end{proof}

% Let $u_p$ solve (\ref{pReg}).
% Introduce the dual $d - 1$-form
% $$\dif v_p := |\dif u_p|_{sv^q}^{p - q} Q(\dif u_p)^{q - 2} \star \dif u_p.$$
% Since $\dif u_p$ is spacelike, so is $\dif v_p$.
% According to (\ref{pReg}), $\dif v_p = 0$, so at least locally we can view it as the exterior derivative of a $\frkg$-valued $d - 2$-form $v_p$.

% \begin{lemma}
% One has 
% $$\dif^*(|\dif v_p|_{sv^{q'}}^{p' - q'} Q(\dif v_p)^{q' - 2} \dif v_p) = 0.$$
% \end{lemma}
% \begin{proof}
% We just check 
% $$|\dif v_p|_{sv^{q'}}^{p' - q'} Q(\dif v_p)^{q' - 2} \dif v_p = \star \dif u_p$$
% which is clearly coclosed.
% \end{proof}

% \todo{Do all this connections stuff much more carefully. Do we really need $\frkg$, since $\dif v$ is a $u^{-1}(TN)$-valued $1$-form anyways?}

% \subsection{Differential forms}

% \subsection{Convergence to a calibration}

\todo{The structure of this paper should be: Introduction, The dual problem, Proof of main theorem}

%%%%%%%%%%%%%%
\section{Setting up the proof}
\subsection{Maps of tracial least gradient}
\todo{Explain what the wedge products mean}
Fix an open subset $U \subseteq M$ of the surface $M$, such that $U$ has a smooth boundary.
We will be interested in the Dirichlet problem for the spectral $\infty$-Laplacian on $U$.
Here, we introduce the convex dual to that problem.

\begin{definition}
Let $v \in BV(U, \RR^d)$. We say that $v$ has \dfn{tracial least gradient} if $v$ minimizes $\int_U |\dif v|_{sv^1} $ among all compactly supported variations of $v$.
\end{definition}

\begin{definition}
Let $v \in BV(U, \RR^d)$. We say that an $\RR^d$-valued $1$-form $F$ is a \dfn{calibration tensor} for $v$, if $\dif F = 0$, $\||F|_{sv^\infty}\|_{L^\infty} = 1$, and 
$$\dif v \wedge F = |\dif v|_{sv^1} .$$
\end{definition}

By Anzellotti theory \cite{Anzellotti1983}, $\dif v \wedge F$ is well-defined as a Radon measure if $v \in BV$, $F \in L^\infty$, and $\dif F \in L^d_\loc$.
Therefore the notion of calibration tensor makes sense for any closed $F \in L^\infty$.
I wrote an appendix to Paper 3 which justifies Anzellotti theory for Riemannian metrics, and could move it here if we wanted to release this paper first.

Since, for $u \in W^{1, \infty}(U, \RR^d)$, we have 
$$\Lip(u) = \||\dif u|_{sv^\infty}\|_{L^\infty},$$
it follows that if $\Lip(u) = 1$ then $\dif u$ meets the hypotheses of a calibration tensor.

\begin{proposition}\label{calibrated implies least gradient}
Suppose that $v \in BV(U, \RR^d)$, and $v$ has a calibration tensor. Then $v$ has tracial least gradient.
\end{proposition}
\begin{proof}
We follow \cite[Theorem 2.5]{Mazon14} which is the case when $d = 1$.
Let $w \in BV(U, \RR^d)$ be a competitor to $v$. Then by Stokes' theorem,
\begin{align*}
\int_U |\dif v|_{sv^1} 
&= \int_U \dif v \wedge F = \int_U \dif w \wedge F
\leq \int_U |\dif w|_{sv^1} . \qedhere
\end{align*}
\end{proof}

\subsection{Convex duality for \texorpdfstring{$p$-harmonic}{p-harmonic} maps}
We will be interested in the convex dual to the Dirichlet problem for the Schatten-von Neumann $p$-Laplacian, where $p \in (1, \infty)$.
In fact, this problem is the Schatten-von Neumann $q$-Laplacian, where $(p, q)$ is a H\"older pair.

\begin{lemma}\label{dual maps are harmonic}
Let $u_p: U \to \RR^d$ be Schatten-von Neumann $p$-harmonic, suppose that $H^1(U, \RR) = 0$, and introduce the dual $1$-form
$$\dif v_q = \star Q(\dif u_p)^{p - 2} \dif u_p.$$
Then $\dif v_q$ is the exterior derivative of a Schatten-von Neumann $q$-harmonic map $v_q$.
\end{lemma}
\begin{proof}
Since $(p, q)$ is a H\"older pair,
$$Q(\dif v_q)^{q - 2} \dif v_q = Q(\dif u_p)^{(q - 2)(p - 1)/2} Q(\dif u_p)^{p - 2} \star \dif u_p = \star \dif u_p,$$
which is coexact.
\end{proof}

In order to use the dual $q$-harmonic maps, it is crucial that the following estimate holds uniformly as $q \to 1$.

\begin{proposition}[Caccioppoli's inequality]
Let $V \Subset U$, $q \in (1, \infty)$, and suppose that $v_q$ is Schatten-von Neumann $q$-harmonic on $U$.
Then
$$\int_V |\dif v_q|_{sv^q}^q \lesssim q^q \int_{U \setminus V} |v_q|^q.$$
\end{proposition}
\begin{proof}
We follow \cite[Theorem 11.20]{kinnunen2021maximal} which gives the proof when $d = 1$.
Choose $\chi \in C^\infty_\cpt(U \to [0, 1])$ such that $\chi = 1$ on $V$ and $\|\dif \chi\|_{C^0} \lesssim 1$.
An integration by parts gives
$$0 = \int_U \tr(\dif \psi^\dagger Q(\dif v_q)^{p - 2} \dif v_q) = \int_U q\chi^{q - 1} \tr((v \otimes \dif \chi)^\dagger Q(\dif v_q)^{q - 2} \dif v_q) + \int_U \chi^q |\dif v_q|_{sv^q}^q.$$
By H\"older's inequality,
\begin{align*}
\int_U \chi^q |\dif v_q|_{sv^q}^q 
&\leq q \int_U \chi^{q - 1} |v_q \otimes \dif \chi|_{sv^\infty} |\dif v_q|_{sv^q}^{q - 1}
\leq q \left(\int_U \chi^q |\dif v_q|_{sv^q}^q\right)^{1/p} \left(\int_U |\dif \chi|^q |v_q|^q\right)^{1/q}.
\end{align*}
After rearranging terms we get 
$$\int_U \chi^q |\dif v_q|_{sv^q}^q \leq q^q \int_U |\dif \chi|^q |v_q|^q$$
so we get 
\begin{align*}
\int_V |\dif v_q|_{sv^q}^q &\leq \int_U \chi^q |\dif v_q|_{sv^q}^q \leq q^q \int_U |\dif \chi|^q |v_q|^q
\leq q^q \|\dif \chi\|_{C^0}^q \int_{U \setminus V} |v_q|^q. \qedhere 
\end{align*}
\end{proof}

In fact, the only obstruction to removing the assumption that $M$ is a surface is:

\begin{conjecture}[gauge-invariant Caccioppoli inequality]
Let $V \Subset U$, $q \in (1, \infty)$, and suppose that $F_q$ is a $d - 1$-form on $U$ such that 
$$\begin{cases}
  \dif F_q = 0, \\
  \dif^*(Q(F_q)^{q - 2} F_q) = 0.
\end{cases}$$
Then 
$$\int_V |F_q|_{sv^q}^q \lesssim q^q \int_{U \setminus V} |F_q|_{sv^q}^q.$$
\end{conjecture}

%%%%%%%%%%%%%
\subsection{A renormalization condition}
Suppose that we have Schatten-von Neumann $p$-harmonic maps $u_p \in W^{1, p}(M, \RR)$ converging to a spectral $\infty$-harmonic map $u$ weakly in $W^{1, r}$, $r \in [1, \infty)$.
Suppose also that we fix a precompact open set $U \subseteq M$, and normalize $u$ so that $\Lip(u|_{\overline U}) = 1$.
We then define $\kappa_p$ by 
\begin{equation}\label{normalization condition}
\kappa_p^{1 - p} := \int_U |\dif u_p|_{sv^p}^p.
\end{equation}
These $\kappa_p$ will be used the renormalize the dual $q$-harmonic maps $v_q$ so that they converge on $U$.
in fact, by an argument identical to \cite[Lemma 6.1]{daskalopoulos2022analytic},
\begin{equation}\label{convergence of normalizations}
\lim_{p \to \infty} \kappa_p = 1.
\end{equation}

%%%%%%%%%%%%%%
\subsection{The stretch set}
Let $X$ be a metric space.
For a Lipschitz map $u: X \to \RR^d$ and $x \in X$, we write 
$$\Lip(u, x) := \limsup_{\varepsilon \to 0} \Lip(u|_{B(x, \varepsilon)}).$$
Since the quantity $\Lip(u|_{B(x, \varepsilon)})$ is monotone, the limit superior defining $\Lip(u, x)$ is actually a limit and an infimum.
Therefore $x \mapsto \Lip(u, x)$ is upper semicontinuous \cite[Lemma 4.2]{Crandall2008}.

\begin{definition}
The \dfn{stretch set} of a Lipschitz map $u: X \to \RR^d$ is the set 
$$\lambda_u := \{x \in X: \Lip(u, x) = \Lip(u)\}.$$
\end{definition}

\begin{lemma}\label{stretch set is compact}
If $X$ is a compact metric space and $u: X \to \RR^d$ is a Lipschitz map, then the stretch set $\lambda_u$ is a nonempty compact subset of $X$.
\end{lemma}
\begin{proof}
The stretch set is the set of maxima of the upper semicontinuous function $x \mapsto \Lip(u, x)$ on the compact set $X$.
\end{proof}

We now arrive at the key estimate of this paper which generalizes \cite[Proposition 6.5]{daskalopoulos2022transverse}.
We show that if $u$ is spectral $\infty$-harmonic, then most of the mass of the $p$-regularizations of $u$ accumulates on $\lambda_u$.

\begin{proposition}\label{main estimate}
Let $U \subseteq M$ a precompact open set of smooth boundary.
Let $u: M \to \RR^d$ be spectral $\infty$-harmonic, such that $\Lip(u|_{\overline U}) = 1$.
Let $\lambda \subseteq \overline U$ be the stretch set of $u|_{\overline U}$, choose $p$-regularizations $u_p$ of $u$, and let $\kappa_p$ satisfy (\ref{normalization condition}).
Then
$$\lim_{p \to \infty} \kappa_p^p \int_{U \setminus \lambda} |\dif u_p|_{sv^p}^p  = 0.$$
\end{proposition}
\begin{proof}
\todo{The main estimate is here}
\end{proof}

\begin{example}
The stretch set does not need to be a geodesic lamination in any canonical way.
Take $M = (0, 1)^2$, and let $u: M \to \RR^2$ be a dilation of the inclusion map.
Then on any compact subset of $M$, $\lambda_u$ is the entire space, and there is no canonical way to assign it the structure of a geodesic lamination.
\end{example}

%%%%%%%%%%%%%%%%
\section{Proof of the main theorem}
\begin{theorem}
Let $u: M \to \RR^d$ be a spectral $\infty$-harmonic map, and let $U \subseteq M$ be a precompact open set with smooth boundary and $H^1(U, \RR) = 0$.
Let $\lambda$ be the stretch set of $u$.
Then $\lambda \cap \partial U$ is nonempty.
\end{theorem}
\begin{proof}
\todo{Existence of the tracial dual should be a separate section}
Without loss of generality, we may assume that $\Lip(u|_U) = 1$.
The stretch set $\lambda$ is a nonempty compact subset of $\overline U$ by Lemma \ref{stretch set is compact}.
We assume that the theorem fails, so there exists an open set $V \Subset U$ such that $\lambda \subseteq \overline V$.
Choose $p$-regularizations $u_p$, let $\kappa_p$ be as in (\ref{normalization condition}), and introduce the dual functions $v_q$, where $(p, q)$ is a H\"older pair,
$$\dif v_q = \kappa_p^{p - 1} |\dif u_p|_{sv^p}^{p - 2} \star \dif u_p.$$
We assume on the annulus $W := U \setminus \overline V$ that we have the gauge-fixing condition
\begin{equation}\label{annulus average is 0}
\int_W v_q = 0.
\end{equation}
The maps $v_q$ are well-defined, since $H^1(U, \RR) = 0$.
By Lemma \ref{dual maps are harmonic}, $v_q$ is Schatten-von Neumann $q$-harmonic.
From the definition of $\dif v_q$, we have
\begin{equation}\label{estimate on Lq}
\int_U |\dif v_q|_{sv^q}^q = \kappa_p^p \int_U |\dif u_p|_{sv^p}^p = \kappa_p.
\end{equation}

Fix a H\"older pair $(r, s)$ such that $r \in (1, \frac{d}{d - 1})$, and observe that, by H\"older's inequality and (\ref{estimate on Lq}),
$$\int_U |\dif v_q|_{sv^1} \lesssim \left(\int_U |\dif v_q|_{sv^q}^q\right)^{1/q} = \kappa_p^{1/q} \lesssim 1.$$
So by the isoperimetric inequality, after taking a subsequence, $v_q \to v$ in $L^r$ for some $v \in BV(U)$, as $q \to 1$.

\begin{lemma}
$v$ has tracial least gradient.
\end{lemma}
\begin{proof}
\todo{Check the signs in this proof}
Let $\varphi \in C^\infty_\cpt(U)$ and $\psi := v\varphi$.
Integrating by parts,
$$0 = \int_U \tr(\dif \psi^\dagger Q(\dif v_q)^{q - 2} \dif v_q) = \int_U \varphi |\dif v_q|_{sv^q}^q + \int_U \tr((v_q \otimes \dif \varphi)^\dagger Q(\dif v_q)^{q - 2} \dif v_q).$$
By semicontinuity of total variation,
\begin{align*}
\int_U \varphi |\dif v|_{sv^1}
&\leq \lim_{q \to 1} \int_U \varphi |\dif v_q|_{sv^q}^q 
= -\lim_{q \to 1} \int_U \tr((v_q \otimes \dif \varphi)^\dagger Q(\dif v_q)^{q - 2} \dif v_q).
\end{align*}
Owing to the convergences $v_q \dif \varphi \to v \dif \varphi$ in $L^s$ and $|\dif v_q|^{q - 2} \dif v_q \weakto -\star \dif u$ in $L^r$,
$$
-\lim_{q \to 1} \int_U \tr((v_q \otimes \dif \varphi)^\dagger Q(\dif v_q)^{q - 2} \dif v_q) = \int_U v\dif \varphi \wedge \dif u.
$$
An integration by parts and the fact that $\Lip(u) \leq 1$ now gives
$$\int_U \varphi |\dif v| \leq \int_U \varphi \dif u \wedge \dif v \leq \int_U \varphi |\dif v|_{sv^1}.$$
Since $\varphi$ was arbitrary, it holds that $\dif u$ calibrates $v$, and the result follows from Proposition \ref{calibrated implies least gradient}.
\end{proof}

By Propositoin \ref{main estimate},
\begin{equation}\label{convergence to stretch set}
\lim_{q \to 1} \int_A |\dif v_q|_{sv^q}^q = \lim_{p \to \infty} \kappa_p^p \int_A |\dif u_p|_{sv^p}^p = 0.
\end{equation}
So by H\"older's inequality, $1_A \dif v_q \to 0$ in $L^1$, so $1_A \dif v = 0$.
From the normalization (\ref{annulus average is 0}) and the convergence in $L^r$, we see that $\int_A v = 0$, hence $1_A v = 0$.
But $v$ has least gradient, so it follows that $v = 0$.
In particular, $v_q \to 0$ in $L^r$.

To derive a contradiction, we estimate
$$k_p = \int_U |\dif v_q|_{sv^q}^q = \int_V |\dif v_q|_{sv^q}^q + \int_A |\dif v_q|_{sv^q}^q =: \mathbf{I} + \mathbf{II}.$$
By the Caccioppoli and H\"older inequalities, and the fact that $v_q \to 0$ in $L^r$,
$$\mathbf{I} \lesssim \int_A |v_q|^q \lesssim \left(\int_A |v_q|^r\right)^{q/r} \to 0.$$
By (\ref{convergence to stretch set}), $\mathbf{II} \to 0$.
Therefore $\kappa_p \to 0$, a contradiction to (\ref{convergence of normalizations}).
\end{proof}

\begin{corollary}
Suppose that $u$ is spectral $\infty$-harmonic.
Then $u$ is absolutely minimizing Lipschitz on $U$.
\end{corollary}
\begin{proof}
Suppose not, so there exists an open set $V \subseteq U$ such that 
$$\Lip(u|_V) > \Lip(u|_{\partial V}).$$
So the stretch set of $u|_V$ is contained in $V$, a contradiction.
\end{proof}

\section{Conjectures}
\begin{conjecture}
Same result, where $M$ is any Riemannian manifold, and $\RR^d$ is an NPC symmetric space.
\end{conjecture}

\begin{conjecture}
Same result, where the spectral $\infty$-harmonic maps are replaced by tight forms and the Lipschitz constant is replaced by the comass.
\end{conjecture}

\begin{conjecture}
Converses of the above conjectures.
\end{conjecture}

\begin{conjecture}
Something about the dichotomy between totally nonholomorphic functions and holomorphic functions \cite{Sheffield2010VectorvaluedOL}.
\end{conjecture}

\begin{conjecture}
If $u$ is spectral $\infty$-harmonic and $\sigma_1(\dif u) > \sigma_2(\dif u)$, then $\lambda_u$ is ``comprised of geodesics.''
\end{conjecture}


\printbibliography

\end{document}
