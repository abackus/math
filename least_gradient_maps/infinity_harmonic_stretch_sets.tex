\documentclass[reqno,11pt]{amsart}
\usepackage[letterpaper, margin=1in]{geometry}
\RequirePackage{amsmath,amssymb,amsthm,graphicx,mathrsfs,url,slashed,subcaption}
\RequirePackage[usenames,dvipsnames]{xcolor}
\RequirePackage[colorlinks=true,linkcolor=Red,citecolor=Green]{hyperref}
\RequirePackage{amsxtra}
\usepackage{cancel}
\usepackage{tikz-cd}
%\usepackage[T1]{fontenc}

% \setlength{\textheight}{9.3in} \setlength{\oddsidemargin}{-0.25in}
% \setlength{\evensidemargin}{-0.25in} \setlength{\textwidth}{7in}
% \setlength{\topmargin}{-0.25in} \setlength{\headheight}{0.18in}
% \setlength{\marginparwidth}{1.0in}
% \setlength{\abovedisplayskip}{0.2in}
% \setlength{\belowdisplayskip}{0.2in}
% \setlength{\parskip}{0.05in}
%\renewcommand{\baselinestretch}{1.05}

\title[$\infty$-harmonic maps absolutely minimize]{$\infty$-harmonic maps from surfaces are absolute minimizers}
\author{Aidan Backus}
\address{Department of Mathematics, Brown University}
\email{aidan\_backus@brown.edu}
\date{\today}

\newcommand{\NN}{\mathbf{N}}
\newcommand{\ZZ}{\mathbf{Z}}
\newcommand{\QQ}{\mathbf{Q}}
\newcommand{\RR}{\mathbf{R}}
\newcommand{\CC}{\mathbf{C}}
\newcommand{\DD}{\mathbf{D}}
\newcommand{\PP}{\mathbf P}
\newcommand{\MM}{\mathbf M}
\newcommand{\II}{\mathbf I}
\newcommand{\Torus}{\mathbf T}
\newcommand{\Hyp}{\mathbf H}
\newcommand{\Sph}{\mathbf S}
\newcommand{\Group}{\mathbf G}
\newcommand{\GL}{\mathbf{GL}}
\newcommand{\Orth}{\mathbf{O}}
\newcommand{\SpOrth}{\mathbf{SO}}
\newcommand{\Ball}{\mathbf{B}}

\newcommand*\dif{\mathop{}\!\mathrm{d}}
\newcommand*\Dif{\mathop{}\!\mathrm{D}}

\DeclareMathOperator{\card}{card}
\DeclareMathOperator{\dist}{dist}
\DeclareMathOperator{\End}{End}
\DeclareMathOperator{\id}{id}
\DeclareMathOperator{\Hom}{Hom}
\DeclareMathOperator{\coker}{coker}
\DeclareMathOperator{\supp}{supp}
\DeclareMathOperator{\Vect}{Vect}
\DeclareMathOperator{\tr}{tr}

\DeclareMathOperator{\sech}{sech}

\DeclareMathOperator{\svd}{svd}
\DeclareMathOperator{\SVD}{SVD}

\newcommand{\Leaves}{\mathscr L}
\newcommand{\Lagrange}{\mathscr L}
\newcommand{\Hypspace}{\mathscr H}

\newcommand{\Chain}{\underline C}

\newcommand{\Two}{\mathrm{I\!I}}
\newcommand{\Ric}{\mathrm{Ric}}

\newcommand{\normal}{\mathbf n}
\newcommand{\radial}{\mathbf r}
\newcommand{\evect}{\mathbf e}
\newcommand{\vol}{\mathrm{vol}}

\newcommand{\diam}{\mathrm{diam}}
\newcommand{\dom}{\mathrm{dom}}
\newcommand{\Ell}{\mathrm{Ell}}
\newcommand{\inj}{\mathrm{inj}}
\newcommand{\Lip}{\mathrm{Lip}}
\newcommand{\sgn}{\operatorname{sgn}}
\newcommand{\MCL}{\mathrm{MCL}}
\newcommand{\Riem}{\mathrm{Riem}}

\DeclareMathOperator*{\essinf}{ess\,inf}
\DeclareMathOperator*{\esssup}{ess\,sup}

\newcommand{\frkg}{\mathfrak g}

\newcommand{\Mass}{\mathbf M}
\newcommand{\Comass}{\mathbf L}

\newcommand{\Min}{\mathrm{Min}}
\newcommand{\Max}{\mathrm{Max}}

\newcommand{\dfn}[1]{\emph{#1}\index{#1}}

\renewcommand{\Re}{\operatorname{Re}}
\renewcommand{\Im}{\operatorname{Im}}

\newcommand{\loc}{\mathrm{loc}}
\newcommand{\cpt}{\mathrm{cpt}}

\def\Japan#1{\left \langle #1 \right \rangle}

\newtheorem{theorem}{Theorem}[section]
\newtheorem{badtheorem}[theorem]{``Theorem"}
\newtheorem{prop}[theorem]{Proposition}
\newtheorem{lemma}[theorem]{Lemma}
\newtheorem{sublemma}[theorem]{Sublemma}
\newtheorem{proposition}[theorem]{Proposition}
\newtheorem{corollary}[theorem]{Corollary}
\newtheorem{conjecture}[theorem]{Conjecture}
\newtheorem{axiom}[theorem]{Axiom}
\newtheorem{assumption}[theorem]{Assumption}

\newtheorem{mainthm}{Theorem}
\renewcommand{\themainthm}{\Alph{mainthm}}

\newcommand{\weakto}{\rightharpoonup}

\newtheorem{claim}{Claim}[theorem]
\renewcommand{\theclaim}{\thetheorem\Alph{claim}}
% \newtheorem*{claim}{Claim}

\theoremstyle{definition}
\newtheorem{definition}[theorem]{Definition}
\newtheorem{remark}[theorem]{Remark}
\newtheorem{example}[theorem]{Example}
\newtheorem{notation}[theorem]{Notation}

\newtheorem{exercise}[theorem]{Discussion topic}
\newtheorem{homework}[theorem]{Homework}
\newtheorem{problem}[theorem]{Problem}

\makeatletter
\newcommand{\proofpart}[2]{%
  \par
  \addvspace{\medskipamount}%
  \noindent\emph{Part #1: #2.}
}
\makeatother



\numberwithin{equation}{section}


% Mean
\def\Xint#1{\mathchoice
{\XXint\displaystyle\textstyle{#1}}%
{\XXint\textstyle\scriptstyle{#1}}%
{\XXint\scriptstyle\scriptscriptstyle{#1}}%
{\XXint\scriptscriptstyle\scriptscriptstyle{#1}}%
\!\int}
\def\XXint#1#2#3{{\setbox0=\hbox{$#1{#2#3}{\int}$ }
\vcenter{\hbox{$#2#3$ }}\kern-.6\wd0}}
\def\ddashint{\Xint=}
\def\dashint{\Xint-}

\usepackage[backend=bibtex,style=alphabetic,giveninits=true]{biblatex}
\renewcommand*{\bibfont}{\normalfont\footnotesize}
\addbibresource{least_gradient_maps.bib}
\renewbibmacro{in:}{}
\DeclareFieldFormat{pages}{#1}

\newcommand\todo[1]{\textcolor{red}{TODO: #1}}


\begin{document}
\begin{abstract}

\end{abstract}

\maketitle
%%%%%%%%%%%%%%%%%%%%%%%%%%%%%%%%%%%%%%%%%%%%%%%%%%%%%%%
\section{Introduction}
% The prototypical PDE in the $L^\infty$ calculus of variations is the $\infty$-Laplacian 
% $$\langle \nabla^2 u, \dif u \otimes \dif u\rangle = 0$$
% whose viscosity solutions $u$ are characterized as absolutely minimizing Lipschitz functions:

\begin{definition}
Let $M$ be a Riemannian manifold.
A Lipschitz map $u: M \to \RR^d$ is \dfn{absolutely minimizing Lipschitz} if for every precompact open set $U \subseteq M$ with smooth boundary and $H^1(U, \RR) = 0$,
$$\Lip(u, \overline U) = \Lip(u, \partial U).$$
\end{definition}

% In fact, the equivalence between absolutely minimizing Lipschitz functions and $\infty$-harmonic functions is usually established by showing that both conditions are equivalent to comparison with cones \cite[\S2]{Crandall2008}.

To find absolutely minimizing Lipschitz maps $u: M \to N$ between manifolds, it is natural to try to solve the \dfn{spectral $\infty$-Laplacian}
$$\langle \dif \langle \dif u, \xi_1(\dif u)\rangle, \xi_1(\dif u)\rangle = 0$$
where $\xi_1(\dif u)$ is the principal singular vector of $\dif u$ \cite{Sheffield2010VectorvaluedOL}.
% Since $N$ does not have an order structure, we do not have the maximum principle; therefore the notions of ``viscosity solution'' and ``comparison with cones'' make no sense.
% Previous attempts to study this equation have either assumed higher regularity of $u$ and nondegeneracy of the principal singular value $\sigma_1(\dif u)$ \cite{Sheffield2010VectorvaluedOL}, or \emph{define} the spectral $\infty$-Laplacian in terms of a $p$-regularization scheme but were not able to establish absolute minimality of solutions \cite{daskalopoulos2022analytic}.
In this paper we prove that spectral $\infty$-harmonic maps from surfaces to euclidean spaces are $\infty$-harmonic.
% In this case the convex dual problem has a particularly simple structure, and so we are able to use the duality to establish the result.

% It seems that our approach should be feasible for dealing with myriad $L^\infty$ variational problems with well-behaved dual problems.
% For example, it is natural to expect that the \dfn{Frobenius $\infty$-Laplacian}
% $$\langle \nabla(\dif u), \dif u \otimes \dif u\rangle = 0$$
% is characterized as giving absolute minimizers of $\||\dif u|_{2}\|_{L^\infty}$, where $|\cdot|_{2}$ denotes the Frobenius norm.
% We have not pursued this angle here because we are not aware of any geometric significance of the Frobenius norm of the derivative, though the Frobenius $\infty$-Laplacian has been studied in the geometry literature; see for example \cite{Ou2012}.

For a linear map $A: X \to Y$ between finite-dimensional Hilbert spaces, we write $A^\dagger: Y \to X$ for its adjoint, and $Q(A) := (AA^\dagger)^{1/2}$, which is a positive semidefinite endomorphism of $Y$.
We write $\sigma_1(A) \geq \cdots \geq \sigma_k(A)$ for the singular values of $A$, written in decreasing order, and $\xi_i(A)$ for the corresponding singular vectors.
For $p \in [1, \infty)$, we introduce the \dfn{Schatten-von Neumann norm}
$$|A|_{p} := (\tr Q(A)^p)^{1/p},$$
and in the case $p = \infty$, we write 
$$|A|_{\infty} := \sigma_1(A).$$
For $p = 1, 2, \infty$, we recall that $|A|_{p}$ is the tracial, Frobenius, and spectral norm, respectively.

Fix a Riemannian surface $M$, possibly with boundary, and a euclidean space $\RR^d$.
We say that a map $u_p \in W^{1, p}(M, \RR^d)$ is \dfn{Schatten-von Neumann $p$-harmonic}, if it solves the equation
$$\dif^* (Q(\dif u_p)^{p - 2} \dif u_p) = 0.$$
Since $\RR^d$ is nonpositively curved, this is equivalent to minimizing the integral $\int_M |\dif u_p|_{p}^p $
among all compactly supported perturbations of $u_p$.

\begin{definition}
We say that $u \in W^{1, \infty}(M, \RR^d)$ is a \dfn{variational solution} of the spectral $\infty$-Laplacian, or simply that $u$ is \dfn{spectral $\infty$-harmonic}, if there is a sequence of $p \to \infty$ such that there exist $u_p \in W^{1, p}(M, \RR^d)$ such that, for every $s \in [1, \infty)$, $u_p \weakto u$ in $W^{1, s}_\loc(M, \RR^d)$.
We call $u_p$ a \dfn{$p$-regularization} of $u$.
\end{definition}

\begin{mainthm}\label{main thm}
Let $M$ be a Riemannian surface.
Let $u: M \to \RR^d$ be spectral $\infty$-harmonic.
Then $u$ is absolutely minimizing Lipschitz.
\end{mainthm}

%%%%%%%%%%%%%%%%%%%%%%%%%
\subsection{Counterexamples}
\begin{example}
Let us see the importance of the hypothesis $H^1(U, \RR) = 0$, even in the scalar-valued case.
Consider the hyperbolic cylinder $M := \Sph^1_\theta \times \RR_x$ with the metric $\dif x^2 + \cosh^2 x \dif \theta^2$.
Then $M$ has a single closed geodesic $\gamma$, which winds around $\{x = 0\}$, and has circumference $2\pi$.
Let $N$ be a circle of circumference $4\pi$, and consider the map 
$$u(x, \theta) := 2\theta.$$
Now $\dif u = 2 \dif \theta$, so
$$\Lip(u, x) = 2|\dif \theta|(x) = 2 \sech |x|.$$
In particular, $\lambda_u = \gamma$, so $u$ is not globally AML.
On the other hand, we can compute the Christoffel symbols
\begin{align*}
\Gamma^\theta_{\theta \theta} &= \Gamma^\theta_{x x} = 0, \\
\Gamma^\theta_{x \theta} &= \Gamma^\theta_{\theta x} = \tanh x.
\end{align*}
It follows that 
$$\nabla^2 u = 2 \tanh x \dif x \dif \theta,$$
and hence $\Delta_\infty u = 0$.
\end{example}

\begin{example}
The stretch set of a spectral $\infty$-harmonic map does not need to be a geodesic lamination in any canonical way.
Take $M = \Torus^2$, and $N$ a dilation of $M$ by a factor of $2$, and $u: M \to N$ the dilation map.
Then $u$ is spectral $\infty$-harmonic since it is affine, $\lambda_u = M$, and there is no canonical way to assign it the structure of a geodesic lamination.
\end{example}

%%%%%%%%%%%%%%%%%%%%%%%%%%%
\subsection{Acknowledgements}
I would like to thank Georgios Daskalopoulos for helpful comments.

This research was supported by the National Science Foundation's Graduate Research Fellowship Program under Grant No. DGE-2040433.

% %%%%%%%%%%%%%%%
% \section{Formulating the problem}
% \subsection{Maps into Riemannian symmetric spaces}
% Throughout this paper, we fix a Riemannian manifold $M$, an open domain $U \subseteq M$ with smooth boundary, and a Riemannian symmetric space $N$.
% The Riemannian structure on $M$ will only have a minor role, however, and the reader may take $M = \RR^d$, $U = \Ball^d$.

% By the Nash embedding theorem, it is no loss to assume that $N$ is isometrically embedded into some euclidean space $\RR^D$.
% In particular, a choice of Nash embedding defines the Sobolev spaces $W^{s, p}(U, N)$, which are the Banach manifolds of maps $u \in W^{s, p}(U, \RR^D)$ such that for almost every $x \in U$, $u(x) \in N$.

% Given a map $u: M \to N$, we can consider $u^{-1}(TN)$-valued $k$-forms on $U$.
% These are sections of the bundle $\Omega^k \otimes u^{-1}(TN)$.
% In particular, $\dif u$ is an $u^{-1}(TN)$-valued $1$-form.

% %%%%%%%%%%%%%%
% \subsection{Calculus of variations in \texorpdfstring{$L^\infty$}{L-infinity}}
% Fix some $q \in (1, \infty)$.
% For a linear map $\xi: T_x M \to T_y N$, let $Q(\xi) := (\xi \xi^\dagger)^{1/2}$, and introduce the \dfn{Schatten-von Neumann norm}
% $$|\xi|_{q} := (\tr Q(\xi)^q)^{1/q}.$$
% \todo{Add in the edge cases, which have to be regularized twice}
% Given a Lipschitz map $u: \overline U \to N$, and $x \in \overline U$, we introduce the \dfn{local maximum stretch}
% $$\Comass(u, x) := \inf_{\varepsilon > 0} \esssup_{x' \in B(x, \varepsilon)} |\dif u(x')|_{q},$$
% which is well-defined by Rademacher's theorem.
% In fact, if $u$ is $C^1$, then $\Comass(u, x)$ is nothing more than $|\dif u(x)|_{q}$. 

% \begin{lemma}
% For any Lipschitz map $u: \overline U \to N$, the local maximum stretch $x \mapsto \Comass(u, x)$ is upper semicontinuous.
% \end{lemma}
% \begin{proof}
% \todo{Prove me}
% \end{proof}

% Given a set $\Gamma \subseteq U$, we introduce the \dfn{maximum stretch}
% $$\Comass(u, \Gamma) := \sup_{x \in \Gamma} \Comass(u, x).$$
% By upper semicontinuity, if $V$ is an open set with smooth boundary,
% $$\Comass(u, \partial V) \leq \Comass(u, V).$$

% \begin{definition}
% A Lipschitz map $u: U \to N$ is an \dfn{absolute minimizer} with respect to the norm $|\cdot|_{q}$ if for every open set $V \subseteq U$ with smooth boundary $\partial V$,
% $$\Comass(u, \partial V) = \Comass(u, V).$$
% \end{definition}

% The goal of this paper, and indeed of the $L^\infty$ calculus of variations, is to compute absolute minimizers.

% Let $\nabla_u$ denote the pullback of the Levi-Civita connection to $u^{-1}(TN)$.
% A computation with \cite[Theorem 5.2]{Barron2001} shows that \todo{check it} smooth absolute minimizer $u$ solves the \dfn{$q$ $\infty$-Laplacian}
% \begin{equation}\label{EulerLagrangeAronsson}\tag{inf-Laplace}
% \langle \nabla_u \dif u, Q(\dif u)^{q - 2} \dif u \otimes Q(\dif u)^{q - 2} \dif u\rangle = 0,
% \end{equation}
% but on the other hand most absolute minimizers are not smooth.

% %%%%%%%%%%%%
% \subsection{\texorpdfstring{$p$-regularization}{p-regularization} of the PDE}
% At the time of writing, there is no suitably fleshed out notion of viscosity solution for totally nonlinear strongly coupled systems of PDE, such as (\ref{EulerLagrangeAronsson}).
% Thus, following Daskalopoulos and Uhlenbeck \cite{daskalopoulos2022analytic}, we regularize (\ref{EulerLagrangeAronsson}) by replacing it by a sequence of variational problems in $L^p$ as $p \to \infty$.

% To formally derive these variational problems, suppose that we are interested in an absolute minimizer $u$ of $\||\dif u|_{q}\|_{L^\infty}$.
% We thus seek $u_p \in W^{1, p}(U, N)$ which minimizes 
% $$I_p(u_p) := \frac{1}{p} \int_U |\dif u_p|_{q}^p \dif V.$$
% Let $u_p$ be a minimizer of $I_p$, and let $v$ be a section of $u_p^{-1}(TN)$.
% We have
% \begin{align*}
% 0 
% &= \frac{\dif}{\dif t} I_p(u_p + t\dif v)|_{t = 0}\\
% &= \frac{1}{p} \int_U \frac{\partial}{\partial t} |\dif u_p + t\dif v|_{q}^{p/2} \dif V\bigg|_{t = 0} \\
% &= \int_U |\dif u_p|_{q}^{p - q} Q(\dif u_p)^{q - 2} \langle \dif u_p, \nabla_{u_p} v\rangle \dif V.
% \end{align*}
% This is the weak formulation of the \dfn{$q$ $p$-Laplacian}
% \begin{equation}\label{pReg}\tag{p-Laplace}
% \nabla^*_{u_p} \left[|\dif u_p|_{q}^{p - q} Q(\dif u_p)^{q - 2} \dif u_p\right] = 0.
% \end{equation}

% \begin{definition}
% A Lipschitz map $u: \overline U \to N$ is a \dfn{variational solution} of (\ref{EulerLagrangeAronsson}) if there exists a sequence of $p \to \infty$, and $u_p \in W^{1, p}(U, N)$, such that:
% \begin{enumerate}
% \item $u_p|_{\partial U} = u|_{\partial U}$.
% \item $u_p$ is a weak solution of (\ref{pReg}).
% \item For every $r \in [1, \infty)$, $u_p \rightharpoonup u$ in $W^{1, r}$.
% \end{enumerate}
% We call $u_p$ a \dfn{$p$-regularization} of $u$.
% \end{definition}

% \begin{proposition}
% Let $h: \overline U \to N$ be a Lipschitz map.
% Then there exists a variational solution $u$ of (\ref{EulerLagrangeAronsson}) such that $u|_{\partial U} = h$.
% \end{proposition}
% \begin{proof}
% \todo{Use the coercivity. By Nash embedding, we're studying a constrained optimization problem in $W^{1, p}(\overline U, \RR^D)$.}
% \end{proof}

% If $N = \RR$, then it is more natural to study viscosity solutions of (\ref{EulerLagrangeAronsson}).
% However, in that case we are merely studying the $\infty$-Laplacian.
% By \cite[Theorem 12]{lindqvist2016notes}, every variational solution of the $\infty$-Laplacian is a viscosity solution; the converse follows from the uniqueness of viscosity solutions \cite[Theorem 27]{lindqvist2016notes}.

% \subsection{Differential forms}
% \todo{Do the problem formulation for differential forms too!}

% %%%%%%%%%%%%%%%%%%%%%

% \subsection{Statement of the main theorems}
% In both of these theorems, let $U$ be a smooth domain in a Riemannian manifold $M$, let $N$ be a Riemannian symmetric space, and let $|\cdot|_{q}$ be a Schatten-von Neumann norm.

% \begin{mainthm}
% Every variational solution $u$ of (\ref{EulerLagrangeAronsson}) is an absolute minimizer of $\||\dif u|_{q}\|_{L^\infty}$.
% \end{mainthm}

% \begin{mainthm}
% Every variational solution of \todo{the $q$ Hodge system} is an absolute minimizer of $\||\dif u|_{q}\|_{L^\infty}$.
% \end{mainthm}


% %%%%%%%%%%%%%%
% \section{The dual problems}
% \subsection{Conservation laws}
% We consider (\ref{EulerLagrangeAronsson}).
% We more or less follow \cite[\S3.5]{daskalopoulos2022analytic}.
% Since $N$ is a symmetric space, $N$ admits a product decomposition
% $$N = \prod_j N_j = \prod_j G_j / K_j$$
% where we write $o = (o_j)$, $o_j \in N_j$, and
% \begin{enumerate}
% \item $G_j$ is a Lie group, whose Lie algebra $\frkg_j$ is the Lie algebra of local isometries of $N_j$.
% \item The type of $\frkg_j$ is either euclidean, compact, or noncompact.
% \item $K_j \subset G_j$ is the isotropy group of $o_j$.
% \end{enumerate}
% Let $B_j$ be the Killing form on the Lie algebra $\frkg_j$, and let $g_j$ be the Riemannian metric on $N_j$.
% We introduce a nondegenerate indefinite inner product $\hat B_j$ on $\frkg_j$, as follows:
% \begin{enumerate}
% \item If $\frkg_j$ has euclidean type, then $N_j$ is flat, so there is a maximal abelian subalgebra $\mathfrak m_j \subset \frkg_j$ such that every tangent space $T_{y_j} N_j$ is identified with $\mathfrak m_j$. Therefore $g_j$ is a positive inner product on $\mathfrak m_j$. We choose any positive-definite extension $\hat B_j$ of $g_j$ to $\frkg_j$.
% \item If $\frkg_j$ has compact type, then $B_j$ is negative-definite and there exists $\lambda > 0$ such that on each tangent space $T_{y_j} N_j \subset \frkg_j$, we have $g_j = -\lambda B_j$. We set $\hat B_j := -\lambda^{-1} B_j$.
% \item If $\frkg_j$ has noncompact type, then $B_j$ is nondegenerate and there exists $\lambda > 0$ such that on each tangent space $T_{y_j} N_j \subset \frkg_j$, we have $g_j = \lambda B_j$. We set $\hat B_j := \lambda^{-1} B_j$.
% \end{enumerate}
% Given $v \in \frkg_j$ and $w \in \frkg_k$, let
% $$\hat B(v, w) := \begin{cases}
% \hat B_j(v, w),& j = k \\
% 0,& j \neq k.
% \end{cases}$$
% Thus $\hat B$ is a nondegenerate indefinite inner product on $\frkg := \oplus_j \frkg_j$.
% We say that $X \in \frkg$ is \dfn{spacelike} if $\hat B(X, X) > 0$ or $X = 0$, and write $P$ for the cone of spacelike vectors.
% Thus for every $y \in N$ we have an isometric inclusion $(T_y N, g) \subseteq (P, \hat B)$.

% We say that a $\frkg$-valued $k$-form $\xi$ is \dfn{spacelike} if every contraction $\langle \xi, v\rangle$ with a $k$-blade $v$ is spacelike.
% We write $\Omega^k \otimes P$ for the sheaf whose setions are spacelike $k$-forms.
% Crucially, if $u: U \to N$ is a smooth map, then $\dif u$ can be viewed as a $\frkg$-valued $1$-form, and $\dif u$ is spacelike.

% \begin{lemma}
% The Euler-Lagrange equation (\ref{pReg}) can be rewritten as 
% $$\int_U \langle |\dif u_p|_{q}^{p - q} Q(\dif u_p)^{q - 2} \dif u_p, \dif \varphi\rangle \dif V = 0$$
% for every $\varphi \in W^{1, p}(U, \frkg)$ such that $\varphi(x) \in T_{u_p(x)} N$.
% \end{lemma}
% \begin{proof}
% Since $\nabla_{u_p}$ is the projection of the trivial connection $\dif$ to $u_p^{-1}(TN)$, and $\varphi$ is a section of $u_p^{-1}(TN)$, $\nabla_{u_p} \varphi = \dif \varphi$. \todo{Explain it better}
% \end{proof}

% Let $u_p$ solve (\ref{pReg}).
% Introduce the dual $d - 1$-form
% $$\dif v_p := |\dif u_p|_{q}^{p - q} Q(\dif u_p)^{q - 2} \star \dif u_p.$$
% Since $\dif u_p$ is spacelike, so is $\dif v_p$.
% According to (\ref{pReg}), $\dif v_p = 0$, so at least locally we can view it as the exterior derivative of a $\frkg$-valued $d - 2$-form $v_p$.

% \begin{lemma}
% One has 
% $$\dif^*(|\dif v_p|_{{q'}}^{p' - q'} Q(\dif v_p)^{q' - 2} \dif v_p) = 0.$$
% \end{lemma}
% \begin{proof}
% We just check 
% $$|\dif v_p|_{{q'}}^{p' - q'} Q(\dif v_p)^{q' - 2} \dif v_p = \star \dif u_p$$
% which is clearly coclosed.
% \end{proof}

% \todo{Do all this connections stuff much more carefully. Do we really need $\frkg$, since $\dif v$ is a $u^{-1}(TN)$-valued $1$-form anyways?}

% \subsection{Differential forms}

% \subsection{Convergence to a calibration}


%%%%%%%%%%%%%%
\section{The dual problem}
\subsection{Preliminaries}
We start by establishing notation.
Given $a, b \geq 0$, we write $a \lesssim b$ to mean that $a \leq Cb$ for some implied constant $C \geq 1$.
We write $a \ll b$ to mean that $a/b \to 0$.
We write $x = o(1)$ to mean that $|x| \ll 1$.
Given vector spaces $X, Y$, let $\Hom(X, Y)$ denote the space of linear maps $X \to Y$, and $\End(X) := \Hom(X, X)$.
We call $(p, q)$ a \dfn{H\"older pair} if $p, q \in [1, \infty]$ and $p^{-1} + q^{-1} = 1$.

Given finite-dimensional Hilbert spaces $X, Y$, linear maps $A \in \Hom(X, Y)$ and $B \in \Hom(Y, X)$, and a H\"older pair $(p, q)$ we have \dfn{von Neumann's trace inequality}
$$|\tr(AB)| \leq \sum_i \sigma_i(A) \sigma_i(B) \leq |A|_p |B|_q.$$
In particular,
$$|AB|_1 \leq |A|_p |B|_q$$
which is the form of the von Neumann trace inequality that we will most often use \cite[Chapter IV]{bhatia1997matrix}.

Given a vector space $X$, let $\Omega^k \otimes X$ denote the sheaf of $X$-valued $k$-forms.
Since $\Omega^1 \otimes X = \Hom(TM, X)$, it makes sense to take Schatten-von Neumann norms of sections of $\Omega^1 \otimes X$.
The wedge product acts on vector-valued forms as 
$$\begin{tikzcd}
(\Omega^k \otimes X) \times (\Omega^\ell \otimes Y) \arrow[r, "\wedge"] &\Omega^{k + \ell} \otimes (X \otimes Y).
\end{tikzcd}$$
This construction is especially helpful when $X = Y$ is a Hilbert space and we use the inner product, because then we can identify $X \otimes X$ with $\End(X)$ and get a scalar-valued $k + \ell$-form as follows:
$$\begin{tikzcd}
(\Omega^k \otimes X) \times (\Omega^\ell \otimes X) \arrow[r, "\wedge"] &\Omega^{k + \ell} \otimes \End(X) \arrow[r, "\tr"] &\Omega^{k + \ell}.
\end{tikzcd}$$

Given a function space $\mathcal X$ and a presheaf $\mathscr F$, let $\mathcal X(\cdot, \mathscr F)$ be the functor of local sections of $\mathscr F$ with components in $\mathcal X$. 

Let $\Omega^k_{\rm cl}$ be the sheaf of closed $k$-forms.
Let $\mathcal M$ be the space of Radon measures.
From Anzellotti theory \cite{Anzellotti1983} and von Neumann's trace inequality, we have:

\begin{lemma}\label{Anzellotti}
Suppose that $M$ is a Riemannian surface. Then the map
\begin{align*}
L^\infty(M, \Omega^1_{\rm cl} \otimes \RR^d) \times \mathcal M(M, \Omega^1_{\rm cl} \otimes \RR^d) \ni (F, \beta) \mapsto \tr(F \wedge \beta) \in \mathcal M(M, \Omega^2_{\rm cl})
\end{align*}
is a well-defined, continuous bilinear pairing which satisfies the inequality of Radon measures
$$|\tr(F \wedge \beta)| \leq |F|_{\infty} |\beta|_{1}.$$
\end{lemma}

I wrote an appendix to Paper 3 which justifies Anzellotti theory for Riemannian metrics, and could move it here if we wanted to release this paper first.

%%%%%%%%%%%%%%%%%%
\subsection{Maps of tracial least gradient}
Fix an open subset $U \subseteq M$ of the surface $M$, such that $U$ has a smooth boundary.
We will be interested in the Dirichlet problem for the spectral $\infty$-Laplacian on $U$.
Here, we introduce the convex dual to that problem.

\begin{definition}
Let $v \in BV(U, \RR^d)$. We say that $v$ has \dfn{tracial least gradient} if $v$ minimizes $\int_U |\dif v|_{1} \star 1$ among all compactly supported variations of $v$.
\end{definition}

Here, $|\dif v|_{1} \star 1$ is the positive Radon measure defined by setting, for every open set $V \subseteq U$,
$$\int_V |\dif v|_{1} \star 1 = \sup \left\{\int_V v \dif \varphi: \varphi \in C^1_\cpt(V, \Omega^1 \otimes \RR^d), \||\varphi|_{\infty}\|_{C^0} \leq 1\right\}.$$

\begin{definition}
Let $v \in BV(U, \RR^d)$. We say that $F \in L^\infty(U, \Omega^1_{\rm cl} \otimes \RR^d)$ is a \dfn{calibration tensor} for $v$ if $\||F|_{\infty}\|_{L^\infty} = 1$ and
\begin{align*}
\tr(F \wedge \dif v) &= |\dif v|_{1} \star 1.
\end{align*}
\end{definition}

By Lemma \ref{Anzellotti}, the definition of calibration tensor makes sense.
If $u \in W^{1, \infty}(U, \RR^d)$ satisfies $\Lip(u) = 1$, then $F := \dif u$ is a closed $1$-form with $\||F|_{\infty}\|_{L^\infty} = 1$.
Conversely, every calibration tensor is locally the derivative of a map $u$ with $\Lip(u) = 1$.

\begin{lemma}\label{calibrated implies least gradient}
Suppose that $v \in BV(U, \RR^d)$, and $v$ has a calibration tensor. Then $v$ has tracial least gradient.
\end{lemma}
\begin{proof}
Let $w \in BV(U, \RR^d)$ be a competitor to $v$, and let $F$ be a calibration tensor. Then by Stokes' theorem,
\begin{align*}
\int_U |\dif v|_{1} \star 1
&= \int_U \tr(F \wedge \dif v) = \int_U \tr(F \wedge \dif w)
\leq \int_U |\dif w|_{1} \star 1. \qedhere
\end{align*}
\end{proof}

In the case $d = 1$, \cite[Theorem 2.5]{Mazon14} asserts that the converse also holds.
It is very likely that the converse holds in our setting, and indeed that every tracial least gradient map is calibrated by the derivative of a spectral $\infty$-harmonic map.
But we shall never need this fact, so we do not prove it.

%%%%%%%%%%%%%%
\subsection{Construction of the dual maps}
Throughout this section we fix an open set $U \subseteq M$ with smooth boundary and $H^1(U, \RR) = 0$.
Let $(p, q)$ be a H\"older pair; in other words, $p^{-1} + q^{-1} = 1$.
One can show that the convex dual of the Schatten-von Neumann $p$-Laplacian is the Schatten-von Neumann $q$-Laplacian.
We have an explicit formula for the dual $q$-harmonic map to a $p$-harmonic map.

\begin{lemma}\label{dual maps are harmonic}
Let $u_p: U \to \RR^d$ be Schatten-von Neumann $p$-harmonic, and introduce the dual $1$-form
$$\dif v_q = \star Q(\dif u_p)^{p - 2} \dif u_p.$$
Then $\dif v_q$ is the exterior derivative of a Schatten-von Neumann $q$-harmonic map $v_q$.
Moreover,
\begin{equation}\label{duality equation}
  Q(\dif v_q)^{q - 2} \dif v_q = \star \dif u_p.
\end{equation}
\end{lemma}
\begin{proof}
We first observe that, since $u_p$ is $p$-harmonic,
$$\dif(\star Q(\dif u_p)^{p - 2} \dif u_p) = 0,$$
implying that $\dif v_q$ is closed.
It is then exact since $H^1(U, \RR) = 0$, so the map $v_q$ exists.

Since $(p, q)$ is a H\"older pair,
$$Q(\dif v_q)^{q - 2} \dif v_q = Q(\dif u_p)^{(q - 2)(p - 1)/2} Q(\dif u_p)^{p - 2} \star \dif u_p = \star \dif u_p,$$
proving (\ref{duality equation}).
Since $\star \dif u_p$ is coclosed, $v_q$ is $q$-harmonic.
\end{proof}

We now want to go to the limit $p \to \infty$.
Let $u$ be spectral $\infty$-harmonic and let $u_p \weakto u$ be a $p$-regularization.
We normalize $u$ so that $\Lip(u, \overline U) = 1$.
We then define $\kappa_p$ by 
\begin{equation}\label{normalization condition}
\kappa_p^{1 - p} := \int_U |\dif u_p|_{p}^p \star 1.
\end{equation}
These $\kappa_p$ will be used the renormalize the dual $q$-harmonic maps $v_q$ so that they converge on $U$.
By an argument identical to \cite[Lemma 6.1]{daskalopoulos2022analytic},
\begin{equation}\label{convergence of normalizations}
\lim_{p \to \infty} \kappa_p = 1.
\end{equation}

Since $H^1(U, \RR) = 0$, there exist maps $v_q: U \to \RR^d$ such that
\begin{equation}\label{normalized dual map}
\dif v_q = \star \kappa_p^{p - 1} Q(\dif u_p)^{p - 2} \dif u_p.
\end{equation}
We call $v_q$ the \dfn{dual map} to $u_p$.
By Lemma \ref{dual maps are harmonic}, $v_q$ is Schatten-von Neumann $q$-harmonic; moreover,
\begin{equation}\label{estimate on Lq}
\int_U |\dif v_q|_{q}^q \star 1 = \kappa_p^p \int_U |\dif u_p|_{p}^p \star 1 = \kappa_p.
\end{equation}
Observe that, by H\"older's inequality, (\ref{estimate on Lq}), and (\ref{convergence of normalizations}),
$$\int_U |\dif v_q|_{1} \star 1 \lesssim \left(\int_U |\dif v_q|_{q}^q \star 1\right)^{1/q} = \kappa_p^{1/q} \lesssim 1.$$
Therefore there exists $v \in BV(U)$ such that along a subsequence, $v_q \weakto v$ in $BV$ as $q \to 1$.
In particular, by the isoperimetric inequality, $v_q \to v$ in $L^{3/2}$.

\begin{proposition}\label{constructing the dual map}
Let $u_p: M \to \RR^d$ be Schatten-von Neumann $p$-harmonic maps converging to a spectral $\infty$-harmonic map $u$, with $\Lip(u, \overline U) = 1$.
Let $v_q: U \to \RR^d$ be the dual maps, and suppose that $v_q \to v$ in $L^{3/2}$ and $v_q \weakto v$ in $BV$.
Then $v$ has tracial least gradient, and $\dif u$ is a calibration tensor for $v$.
\end{proposition}
\begin{proof}
Let $\chi \in C^\infty_\cpt(U)$.
Integrating by parts,
\begin{align*}
0 &= \int_U \tr(Q(\dif v_q)^{q - 2} \dif v_q \wedge \star \dif (\chi v_q)) \\
&= \int_U \chi|\dif v_q|_{q}^q \star 1 + \int_U \tr(Q(\dif v_q)^{q - 2} \dif v_q \wedge \star (v_q \otimes \dif \chi)).
\end{align*}
From the weak convergence in $BV$ and H\"older's inequality
\begin{align*}
\int_U \chi|\dif v|_{1} \star 1
&\leq \lim_{q \to 1} \int_U \chi |\dif v|_1 \star 1 
\leq \lim_{q \to 1} \int_U \chi|\dif v_q|_q^q \star 1 \\
&= -\lim_{q \to 1} \int_U \tr(Q(\dif v_q)^{q - 2} \dif v_q \wedge \star (v_q \otimes \dif \chi)).
\end{align*}
Multiplication by a test function does not affect convergence, so $v_q \otimes \dif \chi \to v \otimes \dif \chi$ in $L^{3/2}$.
Meanwhile, multiplying both sides of (\ref{duality equation}), which applies to the unnormalized dual maps, by $\kappa_p^{(p - 1)(q - 1)}$, and using the fact that $(p - 1)(q - 1) = 1$, we see that 
$$Q(\dif v_q)^{q - 2} \dif v_q = \kappa_p \star \dif u_p.$$
Using $u_p \weakto u$ in $W^{1, 3}$ and (\ref{normalization condition}), we deduce that in $L^3$,
$$Q(\dif v_q)^{q - 2} \dif v_q \weakto \star \dif u.$$
Using the weak convergence in $L^3$ and the strong convergence in $L^{3/2}$, plus the fact that $(3, \frac{3}{2})$ is a H\"older pair, we conclude
$$-\lim_{q \to 1} \int_U \tr(Q(\dif v_q)^{q - 2} \dif v_q \wedge \star (v_q \otimes \dif \chi)) = -\int_U \tr(\star \dif u \wedge \star (v \otimes \dif \chi)).$$
We then cancel the Hodge stars and integrate by parts:
$$-\int_U \tr(\star \dif u \wedge \star (v \otimes \dif \chi)) = -\int_U \tr(\dif u \wedge (v \otimes \dif \chi)) = \int_U \chi \tr(\dif u \wedge \dif v).$$
The above computation and the fact that $\Lip(u) \leq 1$ give
$$\int_U \chi|\dif v|_{1} \star 1 \leq \int_U \chi\dif u \wedge \dif v \leq \int_U \chi|\dif v|_{1} \star 1.$$
Since $\chi$ was arbitrary, it holds that $\dif u$ calibrates $v$, and the result follows from Lemma \ref{calibrated implies least gradient}.
\end{proof}


%%%%%%%%%%%%%%%
\subsection{An energy estimate on the dual maps}
We shall need the fact that the following estimate holds uniformly as $q \to 1$.

\begin{proposition}[Caccioppoli's inequality]\label{Caccioppoli}
Let $V \Subset U$, $q \in (1, \infty)$, and suppose that $v_q$ is Schatten-von Neumann $q$-harmonic on $U$.
Then
$$\int_V |\dif v_q|_{q}^q \star 1 \lesssim q^q \int_{U \setminus V} |v_q|^q \star 1.$$
\end{proposition}
\begin{proof}
We follow \cite[Theorem 11.20]{kinnunen2021maximal}.
Choose $\chi \in C^\infty_\cpt(U, [0, 1])$ such that $\chi = 1$ on $V$ and $\|\dif \chi\|_{C^0} \lesssim 1$.
An integration by parts gives
\begin{align*}
0 &= \int_U \tr(Q(\dif v_q)^{p - 2} \dif v_q \wedge \star \dif (\chi^q v_q)) \\
&= \int_U q\chi^{q - 1} \tr(Q(\dif v_q)^{q - 2} \dif v_q \wedge \star (v_q \otimes \dif \chi)) + \int_U \chi^q |\dif v_q|_{q}^q \star 1.
\end{align*}
By H\"older's inequality,
\begin{align*}
\int_U \chi^q |\dif v_q|_{q}^q \star 1
&\leq q \int_U \chi^{q - 1} |v_q \otimes \dif \chi|_{\infty} |\dif v_q|_{q}^{q - 1} \\
&\leq q \left(\int_U \chi^q |\dif v_q|_{q}^q \star 1\right)^{1/p} \left(\int_U |\dif \chi|^q |v_q|^q \star 1\right)^{1/q}.
\end{align*}
After rearranging terms we get 
$$\int_U \chi^q |\dif v_q|_{q}^q \star 1 \leq q^q \int_U |\dif \chi|^q |v_q|^q \star 1$$
so we get 
\begin{align*}
\int_V |\dif v_q|_{q}^q \star 1 &\leq \int_U \chi^q |\dif v_q|_{q}^q \star 1 \leq q^q \int_U |\dif \chi|^q |v_q|^q \star 1
\leq q^q \|\dif \chi\|_{C^0}^q \int_{U \setminus V} |v_q|^q \star 1. \qedhere 
\end{align*}
\end{proof}

%%%%%%%%%%%%%%
\section{Application to the primal problem}
\subsection{The stretch set}
Let $X$ be a metric space.
For a Lipschitz map $u: X \to \RR^d$ and $x \in X$, we write 
$$\Lip(u, x) := \limsup_{\varepsilon \to 0} \Lip(u, B(x, \varepsilon)).$$
Since the quantity $\Lip(u, B(x, \varepsilon))$ is monotone, the limit superior defining $\Lip(u, x)$ is actually a limit and an infimum.
Therefore $x \mapsto \Lip(u, x)$ is upper semicontinuous \cite[Lemma 4.2]{Crandall2008}.

\begin{definition}
The \dfn{stretch set} of a Lipschitz map $u: X \to \RR^d$ is the set 
$$\lambda_u := \{x \in X: \Lip(u, x) = \Lip(u)\}.$$
\end{definition}

\begin{lemma}\label{stretch set is compact}
If $X$ is a compact metric space and $u: X \to \RR^d$ is a Lipschitz map, then the stretch set $\lambda_u$ is a nonempty compact subset of $X$.
\end{lemma}
\begin{proof}
The stretch set is the set of maxima of the upper semicontinuous function $x \mapsto \Lip(u, x)$ on the compact set $X$.
\end{proof}

We now arrive at the key estimate of this paper which generalizes \cite[Proposition 6.5]{daskalopoulos2022transverse}.
We show that if $u$ is spectral $\infty$-harmonic, then most of the energy of the $p$-regularizations of $u$ accumulates on $\lambda_u$.

\begin{proposition}\label{main estimate}
Let $U \subseteq M$ a precompact open set of smooth boundary.
Let $u: M \to \RR^d$ be spectral $\infty$-harmonic, such that $\Lip(u, U) = 1$.
Let $\lambda \subseteq \overline U$ be the stretch set of $u|_{\overline U}$, choose $p$-regularizations $u_p$ of $u$, and let $\kappa_p$ satisfy (\ref{normalization condition}).
Then as $p \to \infty$,
\begin{equation}\label{decay}
\int_{U \setminus \lambda} |\dif u_p|_{p}^p \star 1 \ll \kappa_p^{-p}.
\end{equation}
\end{proposition}

In the proof of Proposition \ref{main estimate}, we denote $F_p := \kappa_p \dif u_p$, $F := \dif u$, $Q_p := Q(F_p)$,
$$H_p := \tr(Q_p^{p - 2} F_p \wedge \star(F_p - F)).$$
We write $G_p := U \cap \{\star H_p \geq 0\}$.
Possibly up to a Hodge star, the quantities $F, G_p, H_p$ are the same as in \cite[\S7]{daskalopoulos2022analytic}.
Note that $F_p$ corresponds to $U_p$ in that paper, since we are already using $U$ to mean an open set.
By von Neumann's trace inequality,
$$\star \tr(Q_p^{p - 2} F_p \wedge \star F) \leq |Q_p^{p - 2} F_p|_q |F|_p = |F_p|_p^{\frac{p}{q}} |F|_p$$
hence we have the bound
\begin{equation}\label{bound on half of Hp}
  \star \tr(Q_p^{p - 2} F_p \wedge \star F) \leq |F_p|_p^{p - 1} |F|_p
\end{equation}
which will be repeatedly useful.

The argument is not quite the same as in \cite[\S7]{daskalopoulos2022analytic} because I found it a bit tedious to try to untangle the terms arising from the negative curvature of the target from the other terms, so there were a few parts that I reproved from scratch.

\begin{lemma}\label{decay of tau}
Let $\chi \in C^1_\cpt(U, [0, 1])$.
Then as $p \to \infty$, $\chi H_p \to 0$ in $L^1$.
\end{lemma}
\begin{proof}
We first split up the integral using the support property of $\chi$:
\begin{align*}
\int_M \chi H_p
&= \int_U \chi \tr(Q_p^{p - 2} F_p \wedge \star(F_p - F)) \\
&= (\kappa_p - 1) \int_U \chi \tr(Q_p^{p - 2} F_p \wedge \star F) + \int_U \chi \tr(Q_p^{p - 2} (F_p \wedge \star (F_p - \kappa_p F))) \\
&=: \mathbf I + \mathbf{II}.
\end{align*}
To estimate $\mathbf I$, we use (\ref{bound on half of Hp}), H\"older's inequality, the fact that $\||F|_\infty\|_{L^\infty} \leq 1$, (\ref{normalization condition}), and (\ref{convergence of normalizations}):
\begin{align*}
|\mathbf I| &\leq |\kappa_p - 1| \int_U |F_p|_p^{p - 1} |F|_p \star 1 \ll \int_U |F_p|_p^{p - 1} \star 1 \lesssim \left(\kappa_p^p \int_U |\dif u_p|_p^p \star 1\right)^{\frac{1}{q}} \lesssim 1.
\end{align*}
To estimate $\mathbf{II}$, we integrate by parts:
\begin{align*}
0 &= \int_U \chi \tr(\dif^*(Q(\dif u_p)^{p - 2} \dif u_p) \otimes (u_p - u)) \star 1 \\
&= -\int_U \dif \chi \wedge \tr(Q(\dif u_p)^{p - 2} \dif u_p \otimes (u_p - u)) - \int_U \chi \tr(Q(\dif u_p)^{p - 2} \dif u_p \wedge \star \dif (u_p - u)).
\end{align*}
After we multiply through by $\kappa_p^p$, it follows that 
$$\mathbf{II} = - \kappa_p^p \int_U \dif \chi \wedge \tr(Q(\dif u_p)^{p - 2} \dif u_p \otimes (u_p - u)).$$
We estimate using von Neumann's trace inequality and H\"older's inequality:
\begin{align*}
|\mathbf{II}| &= \kappa_p^p \left|\int_U \dif \chi \wedge \tr(Q(\dif u_p)^{p - 2} \dif u_p \otimes (u_p - u))\right| \\
&\lesssim \kappa_p^p \|\dif \chi\|_{C^0} \|u_p - u\|_{C^0} \left(\int_U |\dif u_p|_p^p \star 1\right)^{\frac{p - 1}{p}}.
\end{align*}
By Sobolev embedding, since $u_p \weakto u$ in $W^{1, 3}$, $u_p \to u$ in $C^0$.
Therefore, by (\ref{convergence of normalizations}),
\begin{align*}
|\mathbf{II}| &\ll \|\dif \chi\|_{C^0} \kappa_p^p \kappa_p^{-\frac{(p - 1)^2}{p}} = \|\dif \chi\|_{C^0} \kappa_p^{\frac{1 - 2p}{p}} \lesssim \|\dif \chi\|_{C^0}. 
\end{align*}
Therefore
$$\lim_{p \to \infty} \int_M \chi H_p = 0.$$

For $p \geq 2$, we have \cite[Proposition 2.3]{daskalopoulos2022analytic}
$$-(\star H_p) = \star \tr(Q_p^{p - 2} F_p \wedge \star (F_p - F)) \leq \frac{|F_p|_p^p - |F|_p^p}{p} \leq \frac{|F_p|_p^p}{p}.$$
Since $\chi \geq 0$, we conclude 
\begin{equation}\label{convexity of Schatten norms}
-(\chi \star H_p) \leq \frac{\chi |F_p|_p^p}{p}.
\end{equation}
The inequality (\ref{convexity of Schatten norms}) is vacuous on $G_p$, but if we integrate over $M \setminus G_p$, which is the set on which $\star H_p \leq 0$, we obtain
\begin{align*}
0 \leq -\int_{M \setminus G_p} \chi H_p \leq \frac{1}{p} \int_U \chi |F_p|_p^p \star 1 \lesssim \frac{1}{p}.
\end{align*}
It follows that $1_{M \setminus G_p} \chi H_p \to 0$ in $L^1$, and hence $\chi H_p \to 0$ in $L^1$.
\end{proof}

\begin{proof}[Proof of Proposition \ref{main estimate}]
Let $0 < \varepsilon < 1$ and
$$\chi \in C^1_\cpt(U \cap \{|F|_\infty \leq 1 - \varepsilon\}, [0, 1]).$$
Applying (\ref{bound on half of Hp}),
$$\star \tr(Q_p^{p - 2} F_p \wedge \star F) \leq |F_p|_p^{p - 1} |F|_p \leq 2^{\frac{1}{p}} |F_p|_p^{p - 1} |F|_\infty.$$
Combining this with the support property of $\chi$, we deduce
\begin{equation}\label{source of decay}
\chi \tr(Q_p^{p - 2} F_p \wedge \star F) \leq 2^{\frac{1}{p}} (1 - \varepsilon) |F_p|_p^{p - 1} \star 1;
\end{equation}
the factor of $1 - \varepsilon$ here will be the source of decay needed to prove (\ref{decay}).
To apply (\ref{source of decay}), we need to estimate $\chi |F_p|_p^p$ in $L^1$.
We shall do this on each of $G_p$ and its complement.

For the integral on $G_p$, we apply (\ref{source of decay}) to bound
\begin{align*}
\int_{G_p} \chi |F_p|_p^p \star 1
&= \int_{G_p} \chi \tr(Q_p^{p - 2} F_p \wedge \star (F_p + F - F)) \\
&= \int_{G_p} \chi H_p + \int_{G_p} \chi \tr(Q_p^{p - 2} F_p \wedge \star F) \\
&\leq \int_{G_p} \chi H_p + 2^{\frac{1}{p}} (1 - \varepsilon) \int_{G_p} \chi |F_p|_p^{p - 1} \star 1 \\
&=: \mathbf I + \mathbf{II}.
\end{align*}
We now split into cases based on the relative rates of decay of $\mathbf I$ and $\mathbf{II}$.
After passing to a subsequence of $p \to \infty$, we may assume that either there exists $c > 0$ such that for every $p \gg 1$,
\begin{equation}\label{one dominates}
\mathbf I \geq c \mathbf{II},
\end{equation}
or as $p \to \infty$,
\begin{equation}\label{two dominates}
\mathbf I \ll \mathbf{II}.
\end{equation}
If (\ref{one dominates}) holds, then we use Lemma \ref{decay of tau} to show that $\mathbf I \ll 1$.
Therefore, by (\ref{one dominates}),
$$\int_{G_p} \chi |F_p|_p^p \star 1 \leq (1 + O(c^{-1})) \mathbf I \ll 1.$$
If instead (\ref{two dominates}) holds, then we estimate by H\"older's inequality:
\begin{equation}\label{if two dominates}
\mathbf{II}
\leq (2|G_p|)^{\frac{1}{p}} (1 - \varepsilon) \left(\int_{G_p} \chi^q |F_p|_p^p \star 1\right)^{\frac{1}{q}}
\leq (2|U|)^{\frac{1}{p}} (1 - \varepsilon) \left(\int_{G_p} \chi |F_p|^p \star 1\right)^{\frac{1}{q}}
\end{equation}
Here we used $\chi \leq 1$ and $G_p \subseteq U$ to replace $\chi^q$ and $|G_p|$ with $\chi$ and $|U|$, respectively.
If we multiply both sides of the inequality
$$\int_{G_p} \chi |F_p|^p \star 1 \leq \mathbf I + \mathbf{II}$$
by $(\int_{G_p} \chi |F_p|^p \star 1)^{-1/q}$ and apply (\ref{if two dominates}), then we have
\begin{align*}
\left(\int_{G_p} \chi |F_p|_p^p \star 1\right)^{\frac{1}{p}} 
&\leq (1 - \varepsilon) (2|U|)^{\frac{1}{p}} + \left(\int_{G_p} \chi |F_p|_p^p \star 1\right)^{-\frac{1}{q}} \mathbf I \\
&=: (1 - \varepsilon) (2|U|)^{\frac{1}{p}} + \delta(p).
\end{align*}
Here $\delta(p)$ can be bounded using (\ref{two dominates}) and (\ref{if two dominates}) as $p \to \infty$:
$$\delta(p) \ll \left(\int_{G_p} \chi |F_p|_p^p \star 1\right)^{-\frac{1}{q}} \mathbf{II} \lesssim 1.$$
Taking $p$ so large that $\delta(p) < \frac{\varepsilon}{4}$ and $(2|U|)^{\frac{1}{p}} < 1 + \frac{\varepsilon}{4}$, we obtain
$$\int_{G_p} \chi |F_p|_p^p \star 1 < \left(1 - \frac{\varepsilon}{2}\right)^p \ll 1.$$
In particular, regardless of whether (\ref{one dominates}) or (\ref{two dominates}) holds,
\begin{equation}\label{decay on positive}
\lim_{p \to \infty} \int_{G_p} \chi |F_p|_p^p \star 1 = 0.
\end{equation}

We now turn to the integral over $M \setminus G_p$.
In that case, we can use (\ref{source of decay}) to estimate $\chi |F_p|^p$:
\begin{align*}
\chi |F_p|_p^p &= \chi \star \tr(Q_p^{p - 2} F_p \wedge \star F_p) < \chi \star \tr(Q_p^{p - 2} F_p \wedge \star F) \leq 2^{\frac{1}{p}} (1 - \varepsilon) |F_p|_p^{p - 1}.
\end{align*}
Thus after rearranging terms and integrating,
$$\int_{M \setminus G_p} \chi |F_p|_p^p \star 1 \leq 2(1 - \varepsilon)^p \ll 1.$$
Combining this fact with (\ref{decay on positive}), we obtain 
$$\lim_{p \to \infty} \int_M \chi |F_p|_p^p \star 1 = 0.$$
But $\chi$ and $\varepsilon$ were arbitrary.
We can find a compact exhaustion $(W_\varepsilon)$ of $U \setminus \lambda$ such that $\Lip(u, W_\varepsilon) \leq 1 - \varepsilon$.
Therefore there is an approximation $\chi_\varepsilon \in C^1_\cpt(W_\varepsilon)$ to $1_{U \setminus \lambda}$ in $L^1$.
Taking $\varepsilon \to 0$ and applying $F_p = \kappa_p \dif u_p$, we deduce (\ref{decay}).
\end{proof}

%%%%%%%%%%%%%%%%
\subsection{The main theorem}
\begin{proposition}\label{main prop}
Let $u: M \to \RR^d$ be a spectral $\infty$-harmonic map, and let $U \subseteq M$ be a precompact open set with smooth boundary and $H^1(U, \RR) = 0$.
Let $\lambda$ be the stretch set of $u|_{\overline U}$.
Then $\lambda \cap \partial U$ is nonempty.
\end{proposition}
\begin{proof}
Without loss of generality, we may assume that $\Lip(u, \overline U) = 1$.
The stretch set $\lambda$ is a nonempty compact subset of $\overline U$ by Lemma \ref{stretch set is compact}.
We assume that the theorem fails, so there exists an open set $V \Subset U$ such that $\lambda \subseteq \overline V$.
Let $W := U \setminus \overline V$.

Choose $p$-regularizations $u_p$, let $\kappa_p$ be as in (\ref{normalization condition}), and introduce the dual $q$-harmonic maps $v_q$ which satisfy (\ref{normalized dual map}) and 
\begin{equation}\label{annulus average is 0}
\int_W v_q \star 1 = 0.
\end{equation}
By Proposition \ref{constructing the dual map}, there is a tracial least gradient map $v$ with $v_q \to v$ in $L^{3/2}$ and $v_q \weakto v$ in $BV$.

By (\ref{estimate on Lq}) and Proposition \ref{main estimate},
\begin{equation}\label{convergence to stretch set}
\lim_{q \to 1} \int_W |\dif v_q|_{q}^q \star 1 = \lim_{p \to \infty} \kappa_p^p \int_W |\dif u_p|_{p}^p \star 1 = 0.
\end{equation}
In particular, by the weak convergence in $BV$ and H\"older's inequality, 
$$\int_W |\dif v|_1 \star 1 \leq \lim_{q \to 1} \int_W |\dif v_q|_1 \star 1 \leq \lim_{q \to 1} \int_W |\dif v_q|_q^q \star 1 = 0.$$
Therefore $1_W \dif v = 0$.
From the normalization (\ref{annulus average is 0}) and the convergence in $L^{3/2}$, we see that $\int_W v \star 1 = 0$.
From these considerations, it follows that $1_W v = 0$.
But $v$ has tracial least gradient on $U$, and since $v = 0$ on the collar neighborhood $W$ of $\partial U$, $v$ is competing with $0$; hence $v = 0$.
In particular, $v_q \to 0$ in $L^{3/2}$.

To derive a contradiction, we estimate using (\ref{convergence to stretch set})
$$\kappa_p = \int_U |\dif v_q|_{q}^q \star 1 = \int_V |\dif v_q|_{q}^q \star 1 + \int_W |\dif v_q|_{q}^q \star 1 = \int_W |\dif v_q|_q^q \star 1 + o(1).$$
By Caccioppoli's inequality (Proposition \ref{Caccioppoli}), H\"older's inequality, and the fact that $v_q \to 0$ in $L^{3/2}$,
$$\int_W |\dif v_q|_q^q \star 1 \lesssim q^q \int_W |v_q|^q \star 1 \lesssim \left(\int_W |v_q|^{3/2} \star 1\right)^{\frac{2q}{3}} \ll 1.$$
Therefore $\kappa_p \to 0$, a contradiction to (\ref{convergence of normalizations}).
\end{proof}

\begin{proof}[Proof of Theorem \ref{main thm}]
Suppose not, so there exists a precompact open set $U \subseteq M$ with smooth boundary and $H^1(U, \RR) = 0$ such that 
$$\Lip(u, \overline U) > \Lip(u, \partial U).$$
So the stretch set of $u|_{\overline U}$ is contained in $U$, a contradiction to Proposition \ref{main prop}.
\end{proof}

\begin{conjecture}
Same result, where $M$ is any Riemannian manifold, and $\RR^d$ is an NPC symmetric space.
To get rid of the assumption that $M$ is a surface, we need to show that under appropriate topological conditions, if $\beta_q$ is a $d - 1$-form with 
$$\begin{cases}
\dif \beta_q = 0 \\
\dif^*(|\beta_q|^{q - 2} \beta_q) = 0
\end{cases}$$
and $U$ is contractible then 
$$\int_V |\beta_q|^q \lesssim \int_{U \setminus V} |\beta_q|^q.$$
\end{conjecture}

\begin{conjecture}
Same result, where the spectral $\infty$-harmonic maps are replaced by tight forms and the Lipschitz constant is replaced by the comass.
\end{conjecture}

\begin{conjecture}
Converses of the above conjectures.
\end{conjecture}

\section{Geodesic laminations}
\begin{definition}
The \dfn{nonconformality} of a spectral $\infty$-harmonic map $u$ is 
$$\Sigma_u(x) := \liminf_{\delta \to 0} \essinf_{y \in B(x, \delta)} \sigma_1(\dif u)(y) - \sigma_2(\dif u)(y).$$
If $\Sigma_u > 0$ then we say that $u$ is \dfn{nonconformal}.
\end{definition}

Since the liminf in the definition is actually a sup (since we are taking essinf's over smaller and smaller sets), $\Sigma_u$ is lower semicontinuous.
Therefore $u$ is nonconformal on an open set $U$ and we want to show that $\lambda_u \cap U$ is a geodesic lamination where $\xi_1(\dif u)$ is tangent to the leaves.
  
\begin{conjecture}
Compatibility of variational solutions with the nonconformal $C^2$ solutions of \cite{Sheffield2010VectorvaluedOL}.
The idea would be to foliate $M$ by geodesics obtained by integrating $\xi_1(\dif u)$.
\end{conjecture}

\begin{conjecture}
Let $M$ be a Riemann surface and $u: M \to \CC$ (or $u: M \to E$ where $E$ is an elliptic curve) be spectral $\infty$-harmonic.
If $u$ is nonconformal, then $\lambda_u$ is a geodesic lamination.
This is closely related to the rank-$1$ condition on $v$.
More generally, $\lambda_u$ splits into a geodesic lamination and a set on which $u$ or $\overline u$ is holomorphic.
\end{conjecture}


\printbibliography

\end{document}
