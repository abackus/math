\documentclass[reqno,11pt]{amsart}
\usepackage[letterpaper, margin=1in]{geometry}
\RequirePackage{amsmath,amssymb,amsthm,graphicx,mathrsfs,url,slashed,subcaption}
\RequirePackage[usenames,dvipsnames]{xcolor}
\RequirePackage[colorlinks=true,linkcolor=Red,citecolor=Green]{hyperref}
\RequirePackage{amsxtra}
\usepackage{cancel}
\usepackage{tikz-cd}
%\usepackage[T1]{fontenc}

% \setlength{\textheight}{9.3in} \setlength{\oddsidemargin}{-0.25in}
% \setlength{\evensidemargin}{-0.25in} \setlength{\textwidth}{7in}
% \setlength{\topmargin}{-0.25in} \setlength{\headheight}{0.18in}
% \setlength{\marginparwidth}{1.0in}
% \setlength{\abovedisplayskip}{0.2in}
% \setlength{\belowdisplayskip}{0.2in}
% \setlength{\parskip}{0.05in}
%\renewcommand{\baselinestretch}{1.05}

\title[$\infty$-harmonic maps absolutely minimize]{$\infty$-harmonic maps into symmetric spaces are absolute minimizers}
\author{Aidan Backus}
\address{Department of Mathematics, Brown University}
\email{aidan\_backus@brown.edu}
\date{\today}

\newcommand{\NN}{\mathbf{N}}
\newcommand{\ZZ}{\mathbf{Z}}
\newcommand{\QQ}{\mathbf{Q}}
\newcommand{\RR}{\mathbf{R}}
\newcommand{\CC}{\mathbf{C}}
\newcommand{\DD}{\mathbf{D}}
\newcommand{\PP}{\mathbf P}
\newcommand{\MM}{\mathbf M}
\newcommand{\II}{\mathbf I}
\newcommand{\Torus}{\mathbf T}
\newcommand{\Hyp}{\mathbf H}
\newcommand{\Sph}{\mathbf S}
\newcommand{\Group}{\mathbf G}
\newcommand{\GL}{\mathbf{GL}}
\newcommand{\Orth}{\mathbf{O}}
\newcommand{\SpOrth}{\mathbf{SO}}
\newcommand{\Ball}{\mathbf{B}}

\newcommand*\dif{\mathop{}\!\mathrm{d}}
\newcommand*\Dif{\mathop{}\!\mathrm{D}}

\DeclareMathOperator{\card}{card}
\DeclareMathOperator{\dist}{dist}
\DeclareMathOperator{\End}{End}
\DeclareMathOperator{\id}{id}
\DeclareMathOperator{\Hom}{Hom}
\DeclareMathOperator{\coker}{coker}
\DeclareMathOperator{\supp}{supp}
\DeclareMathOperator{\Vect}{Vect}
\DeclareMathOperator{\tr}{tr}


\DeclareMathOperator{\svd}{svd}
\DeclareMathOperator{\SVD}{SVD}

\newcommand{\Leaves}{\mathscr L}
\newcommand{\Lagrange}{\mathscr L}
\newcommand{\Hypspace}{\mathscr H}

\newcommand{\Chain}{\underline C}

\newcommand{\Two}{\mathrm{I\!I}}
\newcommand{\Ric}{\mathrm{Ric}}

\newcommand{\normal}{\mathbf n}
\newcommand{\radial}{\mathbf r}
\newcommand{\evect}{\mathbf e}
\newcommand{\vol}{\mathrm{vol}}

\newcommand{\diam}{\mathrm{diam}}
\newcommand{\dom}{\mathrm{dom}}
\newcommand{\Ell}{\mathrm{Ell}}
\newcommand{\inj}{\mathrm{inj}}
\newcommand{\Lip}{\mathrm{Lip}}
\newcommand{\sgn}{\operatorname{sgn}}
\newcommand{\MCL}{\mathrm{MCL}}
\newcommand{\Riem}{\mathrm{Riem}}

\DeclareMathOperator*{\esssup}{ess\,sup}

\newcommand{\frkg}{\mathfrak g}

\newcommand{\Mass}{\mathbf M}
\newcommand{\Comass}{\mathbf L}

\newcommand{\Min}{\mathrm{Min}}
\newcommand{\Max}{\mathrm{Max}}

\newcommand{\dfn}[1]{\emph{#1}\index{#1}}

\renewcommand{\Re}{\operatorname{Re}}
\renewcommand{\Im}{\operatorname{Im}}

\newcommand{\loc}{\mathrm{loc}}
\newcommand{\cpt}{\mathrm{cpt}}

\def\Japan#1{\left \langle #1 \right \rangle}

\newtheorem{theorem}{Theorem}[section]
\newtheorem{badtheorem}[theorem]{``Theorem"}
\newtheorem{prop}[theorem]{Proposition}
\newtheorem{lemma}[theorem]{Lemma}
\newtheorem{sublemma}[theorem]{Sublemma}
\newtheorem{proposition}[theorem]{Proposition}
\newtheorem{corollary}[theorem]{Corollary}
\newtheorem{conjecture}[theorem]{Conjecture}
\newtheorem{axiom}[theorem]{Axiom}
\newtheorem{assumption}[theorem]{Assumption}

\newtheorem{mainthm}{Theorem}
\renewcommand{\themainthm}{\Alph{mainthm}}

\newtheorem{claim}{Claim}[theorem]
\renewcommand{\theclaim}{\thetheorem\Alph{claim}}
% \newtheorem*{claim}{Claim}

\theoremstyle{definition}
\newtheorem{definition}[theorem]{Definition}
\newtheorem{remark}[theorem]{Remark}
\newtheorem{example}[theorem]{Example}
\newtheorem{notation}[theorem]{Notation}

\newtheorem{exercise}[theorem]{Discussion topic}
\newtheorem{homework}[theorem]{Homework}
\newtheorem{problem}[theorem]{Problem}

\makeatletter
\newcommand{\proofpart}[2]{%
  \par
  \addvspace{\medskipamount}%
  \noindent\emph{Part #1: #2.}
}
\makeatother



\numberwithin{equation}{section}


% Mean
\def\Xint#1{\mathchoice
{\XXint\displaystyle\textstyle{#1}}%
{\XXint\textstyle\scriptstyle{#1}}%
{\XXint\scriptstyle\scriptscriptstyle{#1}}%
{\XXint\scriptscriptstyle\scriptscriptstyle{#1}}%
\!\int}
\def\XXint#1#2#3{{\setbox0=\hbox{$#1{#2#3}{\int}$ }
\vcenter{\hbox{$#2#3$ }}\kern-.6\wd0}}
\def\ddashint{\Xint=}
\def\dashint{\Xint-}

\usepackage[backend=bibtex,style=alphabetic,giveninits=true]{biblatex}
\renewcommand*{\bibfont}{\normalfont\footnotesize}
\addbibresource{least_gradient_maps.bib}
\renewbibmacro{in:}{}
\DeclareFieldFormat{pages}{#1}

\newcommand\todo[1]{\textcolor{red}{TODO: #1}}


\begin{document}
% \begin{abstract}

% \end{abstract}

\maketitle
%%%%%%%%%%%%%%%%%%%%%%%%%%%%%%%%%%%%%%%%%%%%%%%%%%%%%%%
\section{Introduction}
The prototypical PDE in the $L^\infty$ calculus of variations is the $\infty$-Laplacian 
$$\langle \nabla^2 u, \dif u \otimes \dif u\rangle = 0$$
whose viscosity solutions $u$ are characterized as \dfn{absolutely minimizing Lipschitz functions}: for any open set $U \subseteq \dom(u)$,
$$\Lip(u|_U) = \Lip(u|_{\partial U}).$$
This result is usually proven using, and is equivalent to, comparison with cones \cite[\S2]{Crandall2008}.

To find absolutely minimizing Lipschitz maps $u: M \to N$ between manifolds, it is natural to try to solve the \dfn{spectral $\infty$-Laplacian}
$$\langle \nabla^2 u, v_1(\dif u) \otimes v_1(\dif u)\rangle = 0$$
where $v_1(\dif u)$ is the principal singular vector of $\dif u$.
Since $N$ does not have an order structure, we do not have the maximum principle; therefore the notions of ``viscosity solution'' and ``comparison with cones'' make no sense.
Sheffield and Smart addressed these issues by assuming higher regularity of $u$ and nondegeneracy of $v_1(\dif u)$ \cite{Sheffield2010VectorvaluedOL}, while Daskalopoulos and Uhlenbeck \emph{defined} the spectral $\infty$-Laplacian by a $p$-regularization scheme but were not able to prove that the solutions were absolute minimizers \cite{daskalopoulos2022analytic}.

In this paper we prove that the solutions $u$ of $L^\infty$ variational problems arising from Schatten-von Neumann integrals with a target symmetry are absolute minimizers of their Lagrangians.
This class of variational problems includes the $\infty$-Laplacian, the spectral $\infty$-Laplacian for a symmetric space $N$, and the PDE for a tight calibration.
Since we are considering systems of PDE, we eschew viscosity solutions and define solutions using $p$-regularization.
Using the symmetry of the target space $N$, and an $L^\infty$ version of Noether's theorem, we find a conserved flux $\dif v$.
We show that $\dif v$ is concentrated on a neighborhood of the set $\lambda$ where $u$ attains the $L^\infty$ norm of $\mathscr L(\dif u)$.
If $\lambda$ does not meet the boundary of the domain, then $\dif v$ has compact support, and we deduce that $u = 0$, a contradiction.

%%%%%%%%%%%%%%%%%%%%%%%%%%%
\subsection{Acknowledgements}
I would like to thank Georgios Daskalopoulos for helpful comments.

This research was supported by the National Science Foundation's Graduate Research Fellowship Program under Grant No. DGE-2040433.

%%%%%%%%%%%%%%%
\section{Formulating the problem}
\subsection{Maps into Riemannian symmetric spaces}
Throughout this paper, we fix a Riemannian manifold $M$, an open domain $U \subseteq M$ with smooth boundary, and a Riemannian symmetric space $N$.
The Riemannian structure on $M$ will only have a minor role, however, and the reader may take $M = \RR^d$, $U = \Ball^d$.

By the Nash embedding theorem, it is no loss to assume that $N$ is isometrically embedded into some euclidean space $\RR^D$.
In particular, a choice of Nash embedding defines the Sobolev spaces $W^{s, p}(U, N)$, which are the Banach manifolds of maps $u \in W^{s, p}(U, \RR^D)$ such that for almost every $x \in U$, $u(x) \in N$.

Given a map $u: M \to N$, we can consider $u^{-1}(TN)$-valued $k$-forms on $U$.
These are sections of the bundle $\Omega^k \otimes u^{-1}(TN)$.
In particular, $\dif u$ is an $u^{-1}(TN)$-valued $1$-form.

%%%%%%%%%%%%%%
\subsection{Calculus of variations in \texorpdfstring{$L^\infty$}{L-infinity}}
Fix some $q \in (1, \infty)$.
For a linear map $\xi: T_x M \to T_y N$, let $Q(\xi) := (\xi \xi^\dagger)^{1/2}$, and introduce the \dfn{Schatten-von Neumann norm}
$$|\xi|_{sv^q} := (\tr Q(\xi)^q)^{1/q}.$$
\todo{Add in the edge cases, which have to be regularized twice}
Given a Lipschitz map $u: \overline U \to N$, and $x \in \overline U$, we introduce the \dfn{local maximum stretch}
$$\Comass(u, x) := \inf_{\varepsilon > 0} \esssup_{x' \in B(x, \varepsilon)} |\dif u(x')|_{sv^q},$$
which is well-defined by Rademacher's theorem.
In fact, if $u$ is $C^1$, then $\Comass(u, x)$ is nothing more than $|\dif u(x)|_{sv^q}$. 

\begin{lemma}
For any Lipschitz map $u: \overline U \to N$, the local maximum stretch $x \mapsto \Comass(u, x)$ is upper semicontinuous.
\end{lemma}
\begin{proof}
\todo{Prove me}
\end{proof}

Given a set $\Gamma \subseteq U$, we introduce the \dfn{maximum stretch}
$$\Comass(u, \Gamma) := \sup_{x \in \Gamma} \Comass(u, x).$$
By upper semicontinuity, if $V$ is an open set with smooth boundary,
$$\Comass(u, \partial V) \leq \Comass(u, V).$$

\begin{definition}
A Lipschitz map $u: U \to N$ is an \dfn{absolute minimizer} with respect to the norm $|\cdot|_{sv^q}$ if for every open set $V \subseteq U$ with smooth boundary $\partial V$,
$$\Comass(u, \partial V) = \Comass(u, V).$$
\end{definition}

The goal of this paper, and indeed of the $L^\infty$ calculus of variations, is to compute absolute minimizers.

Let $\nabla_u$ denote the pullback of the Levi-Civita connection to $u^{-1}(TN)$.
A computation with \cite[Theorem 5.2]{Barron2001} shows that \todo{check it} smooth absolute minimizer $u$ solves the \dfn{$sv^q$ $\infty$-Laplacian}
\begin{equation}\label{EulerLagrangeAronsson}\tag{inf-Laplace}
\langle \nabla_u \dif u, Q(\dif u)^{q - 2} \dif u \otimes Q(\dif u)^{q - 2} \dif u\rangle = 0,
\end{equation}
but on the other hand most absolute minimizers are not smooth.

%%%%%%%%%%%%
\subsection{\texorpdfstring{$p$-regularization}{p-regularization} of the PDE}
At the time of writing, there is no suitably fleshed out notion of viscosity solution for totally nonlinear strongly coupled systems of PDE, such as (\ref{EulerLagrangeAronsson}).
Thus, following Daskalopoulos and Uhlenbeck \cite{daskalopoulos2022analytic}, we regularize (\ref{EulerLagrangeAronsson}) by replacing it by a sequence of variational problems in $L^p$ as $p \to \infty$.

To formally derive these variational problems, suppose that we are interested in an absolute minimizer $u$ of $\||\dif u|_{sv^q}\|_{L^\infty}$.
We thus seek $u_p \in W^{1, p}(U, N)$ which minimizes 
$$I_p(u_p) := \frac{1}{p} \int_U |\dif u_p|_{sv^q}^p \dif V.$$
Let $u_p$ be a minimizer of $I_p$, and let $v$ be a section of $u_p^{-1}(TN)$.
We have
\begin{align*}
0 
&= \frac{\dif}{\dif t} I_p(u_p + t\dif v)|_{t = 0}\\
&= \frac{1}{p} \int_U \frac{\partial}{\partial t} |\dif u_p + t\dif v|_{sv^q}^{p/2} \dif V\bigg|_{t = 0} \\
&= \int_U |\dif u_p|_{sv^q}^{p - q} Q(\dif u_p)^{q - 2} \langle \dif u_p, \nabla_{u_p} v\rangle \dif V.
\end{align*}
This is the weak formulation of the \dfn{$sv^q$ $p$-Laplacian}
\begin{equation}\label{pReg}\tag{p-Laplace}
\nabla^*_{u_p} \left[|\dif u_p|_{sv^q}^{p - q} Q(\dif u_p)^{q - 2} \dif u_p\right] = 0.
\end{equation}

\begin{definition}
A Lipschitz map $u: \overline U \to N$ is a \dfn{variational solution} of (\ref{EulerLagrangeAronsson}) if there exists a sequence of $p \to \infty$, and $u_p \in W^{1, p}(U, N)$, such that:
\begin{enumerate}
\item $u_p|_{\partial U} = u|_{\partial U}$.
\item $u_p$ is a weak solution of (\ref{pReg}).
\item For every $r \in [1, \infty)$, $u_p \rightharpoonup u$ in $W^{1, r}$.
\end{enumerate}
We call $u_p$ a \dfn{$p$-regularization} of $u$.
\end{definition}

\begin{proposition}
Let $h: \overline U \to N$ be a Lipschitz map.
Then there exists a variational solution $u$ of (\ref{EulerLagrangeAronsson}) such that $u|_{\partial U} = h$.
\end{proposition}
\begin{proof}
\todo{Use the coercivity. By Nash embedding, we're studying a constrained optimization problem in $W^{1, p}(\overline U, \RR^D)$.}
\end{proof}

If $N = \RR$, then it is more natural to study viscosity solutions of (\ref{EulerLagrangeAronsson}).
However, in that case we are merely studying the $\infty$-Laplacian.
By \cite[Theorem 12]{lindqvist2016notes}, every variational solution of the $\infty$-Laplacian is a viscosity solution; the converse follows from the uniqueness of viscosity solutions \cite[Theorem 27]{lindqvist2016notes}.

\subsection{Differential forms}
\todo{Do the problem formulation for differential forms too!}

%%%%%%%%%%%%%%%%%%%%%

\subsection{Statement of the main theorems}
In both of these theorems, let $U$ be a smooth domain in a Riemannian manifold $M$, let $N$ be a Riemannian symmetric space, and let $|\cdot|_{sv^q}$ be a Schatten-von Neumann norm.

\begin{mainthm}
Every variational solution $u$ of (\ref{EulerLagrangeAronsson}) is an absolute minimizer of $\||\dif u|_{sv^q}\|_{L^\infty}$.
\end{mainthm}

\begin{mainthm}
Every variational solution of \todo{the $sv^q$ Hodge system} is an absolute minimizer of $\||\dif u|_{sv^q}\|_{L^\infty}$.
\end{mainthm}


%%%%%%%%%%%%%%
\section{The dual problems}
\subsection{Conservation laws}
We consider (\ref{EulerLagrangeAronsson}).
We more or less follow \cite[\S3.5]{daskalopoulos2022analytic}.
Since $N$ is a symmetric space, $N$ admits a product decomposition
$$N = \prod_j N_j = \prod_j G_j / K_j$$
where we write $o = (o_j)$, $o_j \in N_j$, and
\begin{enumerate}
\item $G_j$ is a Lie group, whose Lie algebra $\frkg_j$ is the Lie algebra of local isometries of $N_j$.
\item The type of $\frkg_j$ is either euclidean, compact, or noncompact.
\item $K_j \subset G_j$ is the isotropy group of $o_j$.
\end{enumerate}
Let $B_j$ be the Killing form on the Lie algebra $\frkg_j$, and let $g_j$ be the Riemannian metric on $N_j$.
We introduce a nondegenerate indefinite inner product $\hat B_j$ on $\frkg_j$, as follows:
\begin{enumerate}
\item If $\frkg_j$ has euclidean type, then $N_j$ is flat, so there is a maximal abelian subalgebra $\mathfrak m_j \subset \frkg_j$ such that every tangent space $T_{y_j} N_j$ is identified with $\mathfrak m_j$. Therefore $g_j$ is a positive inner product on $\mathfrak m_j$. We choose any positive-definite extension $\hat B_j$ of $g_j$ to $\frkg_j$.
\item If $\frkg_j$ has compact type, then $B_j$ is negative-definite and there exists $\lambda > 0$ such that on each tangent space $T_{y_j} N_j \subset \frkg_j$, we have $g_j = -\lambda B_j$. We set $\hat B_j := -\lambda^{-1} B_j$.
\item If $\frkg_j$ has noncompact type, then $B_j$ is nondegenerate and there exists $\lambda > 0$ such that on each tangent space $T_{y_j} N_j \subset \frkg_j$, we have $g_j = \lambda B_j$. We set $\hat B_j := \lambda^{-1} B_j$.
\end{enumerate}
Given $v \in \frkg_j$ and $w \in \frkg_k$, let
$$\hat B(v, w) := \begin{cases}
\hat B_j(v, w),& j = k \\
0,& j \neq k.
\end{cases}$$
Thus $\hat B$ is a nondegenerate indefinite inner product on $\frkg := \oplus_j \frkg_j$.
We say that $X \in \frkg$ is \dfn{spacelike} if $\hat B(X, X) > 0$ or $X = 0$, and write $P$ for the cone of spacelike vectors.
Thus for every $y \in N$ we have an isometric inclusion $(T_y N, g) \subseteq (P, \hat B)$.

We say that a $\frkg$-valued $k$-form $\xi$ is \dfn{spacelike} if every contraction $\langle \xi, v\rangle$ with a $k$-blade $v$ is spacelike.
We write $\Omega^k \otimes P$ for the sheaf whose setions are spacelike $k$-forms.
Crucially, if $u: U \to N$ is a smooth map, then $\dif u$ can be viewed as a $\frkg$-valued $1$-form, and $\dif u$ is spacelike.

\begin{lemma}
The Euler-Lagrange equation (\ref{pReg}) can be rewritten as 
$$\int_U \langle |\dif u_p|_{sv^q}^{p - q} Q(\dif u_p)^{q - 2} \dif u_p, \dif \varphi\rangle \dif V = 0$$
for every $\varphi \in W^{1, p}(U, \frkg)$ such that $\varphi(x) \in T_{u_p(x)} N$.
\end{lemma}
\begin{proof}
Since $\nabla_{u_p}$ is the projection of the trivial connection $\dif$ to $u_p^{-1}(TN)$, and $\varphi$ is a section of $u_p^{-1}(TN)$, $\nabla_{u_p} \varphi = \dif \varphi$. \todo{Explain it better}
\end{proof}

Let $u_p$ solve (\ref{pReg}).
Introduce the dual $d - 1$-form
$$\dif v_p := |\dif u_p|_{sv^q}^{p - q} Q(\dif u_p)^{q - 2} \star \dif u_p.$$
Since $\dif u_p$ is spacelike, so is $\dif v_p$.
According to (\ref{pReg}), $\dif v_p = 0$, so at least locally we can view it as the exterior derivative of a $\frkg$-valued $d - 2$-form $v_p$.

\begin{lemma}
One has 
$$\dif^*(|\dif v_p|_{sv^{q'}}^{p' - q'} Q(\dif v_p)^{q' - 2} \dif v_p) = 0.$$
\end{lemma}
\begin{proof}
We just check 
$$|\dif v_p|_{sv^{q'}}^{p' - q'} Q(\dif v_p)^{q' - 2} \dif v_p = \star \dif u_p$$
which is clearly coclosed.
\end{proof}

\todo{Do all this connections stuff much more carefully. Do we really need $\frkg$, since $\dif v$ is a $u^{-1}(TN)$-valued $1$-form anyways?}

\subsection{Differential forms}

\subsection{Convergence to a calibration}

%%%%%%%%%%%%%%
\section{Infinity harmonic maps are absolutely best Lipschitz}
\begin{definition}
We say that $\Delta_\infty u = 0$ in the \dfn{variational sense} on a ball $B$ if $u \in W^{1, \infty}(B)$ has for each $p \in (1, \infty)$ a \dfn{$p$-regularization} $u_p$, that is a solution in the weak sense of 
$$\begin{cases}
    \Delta_p u_p = 0 \\
    u_p|_{\partial B} = u|_{\partial B}
\end{cases}$$
such that along a subsequence, $u_p \to u$ in $C^0$ as $p \to \infty$.
\end{definition}

\begin{definition}
Suppose that $\Delta_\infty u = 0$ in the variational sense.
The \dfn{stretch set} is the set of points $x$ such that $\Lip(u, x) = \Lip(u)$.
\end{definition}

\begin{proposition}[Caccioppoli's inequality]
Let $V \Subset U \subseteq \RR^d$ and $u \in W^{1, p}_\loc(U)$.
If $\Delta_p u = 0$ on $U$, and $\rho := \dist(\partial U, \partial V)$, then
$$\int_V |\dif u|^p \leq \left(\frac{p}{\rho}\right)^p \int_{U \setminus V} |u|^p.$$
\end{proposition}
\begin{proof}
Choose $\varepsilon > 0$ and $\chi \in C^\infty_\cpt(U \to [0, 1])$ such that $\chi = 1$ on $V$ and
$$|\dif \chi| \leq \frac{1 + \varepsilon}{\rho}.$$
Plugging $\psi := u\chi^p$ into the $p$-Laplacian and integrating by parts,
$$0 = \int_U |\dif u|^{p - 2} \dif u \cdot \dif \psi = \int_U p\chi^{p - 1} u|\dif u|^{p - 2} \dif u \cdot \dif \chi + \int_U \chi^p |\dif u|^p.$$
By H\"older's inequality,
\begin{align*}
\int_U \chi^p |\dif u|^p 
&\leq p\int_U \chi^{p - 1} |u| |\dif u|^{p - 1} |\dif \chi| 
\leq p\left(\int_U \chi^p |\dif u|^p\right)^{1/q} \left(\int_U |\dif \chi|^p |u|^p\right)^{1/p}.
\end{align*}
After rearranging terms we get 
$$\int_U \chi^p |\dif u|^p \leq p^p \int_U |\dif \chi|^p |u|^p$$
so we get 
$$\int_V |\dif u|^p \leq \int_U \chi^p |\dif u|^p \leq p^p \int_U |\dif \chi|^p |u|^p \leq \left(\frac{(1 + \varepsilon)p}{\rho}\right)^p \int_{U \setminus V} |u|^p$$
which implies Caccioppoli's inequality when we take $\varepsilon \to 0$.
\end{proof}

The important fact here is that Caccioppoli's inequality is \emph{uniform} as $p \to 1$, and it also works on domains which are not concentric balls.
The proof however is the same as the usual one found in \cite[Chapter 11]{kinnunen2021maximal}.

\begin{theorem}
Let $U \Subset \RR^2$, $u \in W^{1, \infty}(U)$, and $\Delta_\infty u = 0$ in the variational sense.
Let $\lambda$ be the stretch set of $u$.
Then $\lambda \cap \partial U$ is nonempty.
\end{theorem}
\begin{proof}
Without loss of generality, we may assume that $\Lip(u) = 1$.
The stretch set $\lambda$ is a nonempty compact subset of $\overline U$ \cite[Lemma 4.2]{Crandall2008}, so if the theorem fails, then there exists an open set $V \Subset U$ such that $\lambda \Subset V$.
Choose a $p$-regularization $u_p$, let
$$k_p^{1 - p} := \int_B |\dif u_p|^p,$$
and introduce the dual functions $v_q$, where $(p, q)$ is a H\"older pair,
$$\dif v_q = k_p^{p - 1} |\dif u_p|^{p - 2} \star \dif u_p,$$
and we have the gauge-fixing condition on the annulus $A := U \setminus V$
\begin{equation}\label{annulus average is 0}
\int_A v_q = 0.
\end{equation}
Then $\Delta_q v_q = 0$ and $k_p \to 1$ \cite[\S3]{daskalopoulos2022transverse}.
We in particular have 
\begin{equation}\label{estimate on Lq}
\int_U |\dif v_q|^q = k_p^p \int_U |\dif u_p|^p = k_p.
\end{equation}

Fix a H\"older pair $(r, s)$ such that $r \in (1, \frac{d}{d - 1})$, and observe that, by H\"older's inequality and (\ref{estimate on Lq}),
$$\int_U |\dif v_q| \lesssim \left(\int_U |\dif v_q|^q\right)^{1/q} = k_p^{1/q} \lesssim 1.$$
So by the isoperimetric inequality, after taking a subsequence, $v_q \to v$ in $L^r$ for some $v \in BV(U)$, as $q \to 1$.

\begin{lemma}
There exists a calibration $\star R$ such that along a subsequence, $|\dif v_q|^{q - 2} \dif v_q \rightharpoonup R$ in $L^s$ as $q \to \infty$.
\end{lemma}
\begin{proof}
We follow the construction of the calibration in \cite[Proposition 5.9]{Andreu-Vaillo2004}.
If $p > s$ then by H\"older's inequality and (\ref{estimate on Lq}),
$$\int_U ||\dif v_q|^{q - 2} \dif v_q|^s = \int_U |\dif v_q|^{(q - 1)s} \lesssim \left(\int_U |\dif v_q|^q\right)^{s/p} = k_p^{s/p} \lesssim 1,$$
so by Alaoglu's theorem, there exists $R$ such that $|\dif v_q|^{q - 2} \dif v_q \rightharpoonup R$ in $L^s$.
If $\psi \in C^\infty_\cpt(U)$ then, integrating by parts,
$$\int_U \dif^* R \psi = \int_U R \wedge \dif \psi = \lim_{q \to 1} \int_U |\dif v_q|^{q - 2} \dif v_q \wedge \dif \psi = \lim_{q \to 1} \int_U \psi \Delta_q v_q = 0,$$
so $\dif^* R = 0$.

To estimate $\|R\|_{L^\infty}$, let
$$Z_{q, \varepsilon} := \left\{|\dif v_q| > \frac{1}{\varepsilon}\right\},$$
so by Chebyshev's inequality,
$$|Z_{q, \varepsilon}| \leq \varepsilon^q \int_U |\dif v_q|^q = \varepsilon^q k_p \int_U |\dif u_p|^p = \varepsilon^q k_p \lesssim \varepsilon^q.$$
Applying Alaoglu's theorem again, there exists $S_\varepsilon$ such that $|\dif v_q|^{q - 2} \dif v_q 1_{Z_{q, \varepsilon}} \rightharpoonup S_\varepsilon$ in $L^1$.
We estimate using H\"older's inequality and (\ref{estimate on Lq})
\begin{align*}
\int_{Z_{q, \varepsilon}} |\dif v_q|^{q - 1} \leq |Z_{q, \varepsilon}|^{1/q} \left(\int_{Z_{q, \varepsilon}} |\dif v_q|^q\right)^{1/p} \lesssim \varepsilon^{1/q} k_p^{1/p}.
\end{align*}
Taking the limit $q \to 1$ we obtain 
\begin{equation}\label{bad set is small}
\int_U |S_\varepsilon| \lesssim \varepsilon.
\end{equation}

Let $R_\varepsilon := R - S_\varepsilon$, so that if we set
$$R_{\varepsilon, q} := |\dif v_q|^{q - 2} \dif v_q (1 - 1_{Z_{q, \varepsilon}}),$$
then $R_{\varepsilon, q} \rightharpoonup R_\varepsilon$ in $L^1$.
Since
$$\|R_{\varepsilon, q}\|_{L^\infty} \leq \varepsilon^{1 - q} \leq 1 + o(1),$$
we have $\|R_\varepsilon\|_{L^\infty} \leq 1$.
Combining this fact with (\ref{bad set is small}) implies $\|R\|_{L^\infty} \leq 1$, as needed.
\end{proof}

\begin{lemma}
$v$ has least gradient.
\end{lemma}
\begin{proof}
By the calibration criterion for functions of least gradient \cite[Theorem 1.1]{Mazon14}, it suffices to show that $\star R$ calibrates $v$.
Let $\varphi \in C^\infty_\cpt(U)$ and $\psi := v\varphi$.
Integrating by parts,
$$0 = \int_U |\dif v_q|^{q - 2} \dif v_q \cdot \dif \psi = \int_U \psi |\dif v_q|^q + \int_U v_q |\dif v_q|^{q - 2} \dif v_q \cdot \dif \psi.$$
By semicontinuity of total variation,
\begin{align*}
\int_U \psi |\dif v|
&\leq \lim_{q \to 1} \int_U \psi |\dif v_q|^q 
= -\lim_{q \to 1} \int_U v_q |\dif v_q|^{q - 2} \dif v_q \cdot \dif \psi.
\end{align*}
Owing to the convergences $v_q \dif \psi \to v \dif \psi$ in $L^s$ and $|\dif v_q|^{q - 2} \dif v_q \rightharpoonup R$ in $L^r$,
$$
-\lim_{q \to 1} \int_U v_q |\dif v_q|^{q - 2} \dif v_q \cdot \dif \psi = -\int_U v\dif \psi \wedge \star R.
$$
Since $\dif^* R = 0$, an integration by parts and the fact that $\|R\|_{L^\infty} \leq 1$ now gives
$$\int_U \psi |\dif v| \leq \int_U \psi \dif v \wedge \star R \leq \int_U \psi |\dif v|.$$
Since $\varphi$ was arbitrary, it holds that $\star R$ calibrates $v$.
\end{proof}

By \cite[Proposition 6.5]{daskalopoulos2022transverse},
\begin{equation}\label{convergence to stretch set}
\lim_{q \to 1} \int_A |\dif v_q|^q = \lim_{p \to \infty} k_p^p \int_A |\dif u_p|^p = 0.
\end{equation}
So by H\"older's inequality, $1_A \dif v_q \to 0$ in $L^1$, so $1_A \dif v = 0$.
From the normalization (\ref{annulus average is 0}) and the convergence in $L^r$, we see that $\int_A v = 0$, hence $1_A v = 0$.
But $v$ has least gradient, so it follows that $v = 0$.
In particular, $v_q \to 0$ in $L^r$.

To derive a contradiction, we estimate
$$k_p = \int_U |\dif v_q|^q = \int_V |\dif v_q|^q + \int_A |\dif v_q|^q =: \mathbf{I} + \mathbf{II}.$$
By the Caccioppoli and H\"older inequalities, and the fact that $v_q \to 0$ in $L^r$,
$$\mathbf{I} \lesssim \int_A |v_q|^q \lesssim \left(\int_A |v_q|^r\right)^{q/r} \to 0.$$
By (\ref{convergence to stretch set}), $\mathbf{II} \to 0$.
Therefore $k_p \to 0$, a contradiction.
\end{proof}

\begin{corollary}
Let $U \Subset \RR^2$, and $\Delta_\infty u = 0$ on $U$.
Then $u$ is absolutely minimizing Lipschitz on $U$.
\end{corollary}
\begin{proof}
Suppose not, so there exists an open set $V \subseteq U$ such that 
$$\Lip(u|_V) > \Lip(u|_{\partial V}).$$
So the stretch set of $u|_V$ is contained in $V$, a contradiction.
\end{proof}

\section{Things I want to prove}
\begin{conjecture}
The same theorem holds for $\Delta_\infty$ replaced by any $\infty$-elliptic system with a conservation law for the target, and $\RR^2$ replaced by any $\RR^d$, where the ``stretch set'' $\lambda$ is the set of points where the $L^\infty$ Lagrangian attains its maximum.
\end{conjecture}

It is almost certain that the theorem holds:
\begin{enumerate}
\item If $u$ is the gauge potential of a tight $d - 1$-form.
\item If $u$ is an $sv^\infty$ $\infty$-harmonic map into a symmetric space. We have to replace the least gradient function with an $sv^1$ least gradient map into the (spacelike part of) the Lie algebra $\mathfrak g$, and the calibration with a $\mathfrak g$-valued calibration.
\end{enumerate}
In general, it probably works after replacing the least gradient map with a solution of some ``$1$-elliptic system'' and $R$ with some kind of ``calibration tensor.''
This result might be important enough that I may want to put the least gradient maps stuff on hold and write a short paper about this...

\printbibliography

\end{document}
