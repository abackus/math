\documentclass[reqno,11pt]{amsart}
\usepackage[letterpaper, margin=1in]{geometry}
\RequirePackage{amsmath,amssymb,amsthm,graphicx,mathrsfs,url,slashed,subcaption}
\RequirePackage[usenames,dvipsnames]{xcolor}
\RequirePackage[colorlinks=true,linkcolor=Red,citecolor=Green]{hyperref}
\RequirePackage{amsxtra}
\usepackage{cancel}
\usepackage{tikz-cd}
%\usepackage[T1]{fontenc}

% \setlength{\textheight}{9.3in} \setlength{\oddsidemargin}{-0.25in}
% \setlength{\evensidemargin}{-0.25in} \setlength{\textwidth}{7in}
% \setlength{\topmargin}{-0.25in} \setlength{\headheight}{0.18in}
% \setlength{\marginparwidth}{1.0in}
% \setlength{\abovedisplayskip}{0.2in}
% \setlength{\belowdisplayskip}{0.2in}
% \setlength{\parskip}{0.05in}
%\renewcommand{\baselinestretch}{1.05}

\title{Optimal Lipschitz maps between Riemannian manifolds}
\author{Aidan Backus}
\address{Department of Mathematics, Brown University}
\email{aidan\_backus@brown.edu}
\date{\today}

\newcommand{\NN}{\mathbf{N}}
\newcommand{\ZZ}{\mathbf{Z}}
\newcommand{\QQ}{\mathbf{Q}}
\newcommand{\RR}{\mathbf{R}}
\newcommand{\CC}{\mathbf{C}}
\newcommand{\DD}{\mathbf{D}}
\newcommand{\PP}{\mathbf P}
\newcommand{\MM}{\mathbf M}
\newcommand{\II}{\mathbf I}
\newcommand{\Torus}{\mathbf T}
\newcommand{\Hyp}{\mathbf H}
\newcommand{\Sph}{\mathbf S}
\newcommand{\Group}{\mathbf G}
\newcommand{\GL}{\mathbf{GL}}
\newcommand{\Orth}{\mathbf{O}}
\newcommand{\SpOrth}{\mathbf{SO}}
\newcommand{\Ball}{\mathbf{B}}

\newcommand*\dif{\mathop{}\!\mathrm{d}}
\newcommand*\Dif{\mathop{}\!\mathrm{D}}

\DeclareMathOperator{\card}{card}
\DeclareMathOperator{\dist}{dist}
\DeclareMathOperator{\End}{End}
\DeclareMathOperator{\id}{id}
\DeclareMathOperator{\Hom}{Hom}
\DeclareMathOperator{\coker}{coker}
\DeclareMathOperator{\supp}{supp}
\DeclareMathOperator{\Vect}{Vect}
\DeclareMathOperator{\tr}{tr}

\DeclareMathOperator{\asinh}{asinh}
\DeclareMathOperator{\sech}{sech}

\DeclareMathOperator{\svd}{svd}
\DeclareMathOperator{\SVD}{SVD}

\newcommand{\Leaves}{\mathscr L}
\newcommand{\Lagrange}{\mathscr L}
\newcommand{\Hypspace}{\mathscr H}

\newcommand{\Chain}{\underline C}

\newcommand{\Two}{\mathrm{I\!I}}
\newcommand{\Ric}{\mathrm{Ric}}

\newcommand{\normal}{\mathbf n}
\newcommand{\radial}{\mathbf r}
\newcommand{\evect}{\mathbf e}
\newcommand{\vol}{\mathrm{vol}}
\newcommand{\Gr}{\mathrm{Gr}}
\newcommand{\Sec}{\mathrm{sec}}

\newcommand{\diam}{\operatorname{diam}}
\newcommand{\dom}{\mathrm{dom}}
\newcommand{\Ell}{\mathrm{Ell}}
\newcommand{\inj}{\mathrm{inj}}
\newcommand{\Lip}{\mathrm{Lip}}
\newcommand{\ncf}{\operatorname{ncf}}
\newcommand{\sgn}{\operatorname{sgn}}
\newcommand{\MCL}{\mathrm{MCL}}
\newcommand{\Riem}{\mathrm{Riem}}

\DeclareMathOperator*{\essinf}{ess\,inf}
\DeclareMathOperator*{\esssup}{ess\,sup}

\newcommand{\frkg}{\mathfrak g}

\newcommand{\Mass}{\mathbf M}
\newcommand{\Comass}{\mathbf L}

\newcommand{\Min}{\mathrm{Min}}
\newcommand{\Max}{\mathrm{Max}}

\newcommand{\dfn}[1]{\emph{#1}\index{#1}}

\renewcommand{\Re}{\operatorname{Re}}
\renewcommand{\Im}{\operatorname{Im}}

\newcommand{\loc}{\mathrm{loc}}
\newcommand{\cpt}{\mathrm{cpt}}

\def\Japan#1{\left \langle #1 \right \rangle}

\newtheorem{theorem}{Theorem}[section]
\newtheorem{badtheorem}[theorem]{``Theorem"}
\newtheorem{prop}[theorem]{Proposition}
\newtheorem{lemma}[theorem]{Lemma}
\newtheorem{sublemma}[theorem]{Sublemma}
\newtheorem{proposition}[theorem]{Proposition}
\newtheorem{corollary}[theorem]{Corollary}
\newtheorem{conjecture}[theorem]{Conjecture}
\newtheorem{axiom}[theorem]{Axiom}
\newtheorem{assumption}[theorem]{Assumption}

\newtheorem{mainthm}{Theorem}
\renewcommand{\themainthm}{\Alph{mainthm}}

\newcommand{\weakto}{\rightharpoonup}

\newtheorem{claim}{Claim}[theorem]
\renewcommand{\theclaim}{\thetheorem\Alph{claim}}
% \newtheorem*{claim}{Claim}

\theoremstyle{definition}
\newtheorem{definition}[theorem]{Definition}
\newtheorem{remark}[theorem]{Remark}
\newtheorem{example}[theorem]{Example}
\newtheorem{notation}[theorem]{Notation}

\newtheorem{exercise}[theorem]{Discussion topic}
\newtheorem{homework}[theorem]{Homework}
\newtheorem{problem}[theorem]{Problem}

\makeatletter
\newcommand{\proofpart}[2]{%
  \par
  \addvspace{\medskipamount}%
  \noindent\emph{Part #1: #2.}
}
\makeatother



\numberwithin{equation}{section}


% Mean
\def\Xint#1{\mathchoice
{\XXint\displaystyle\textstyle{#1}}%
{\XXint\textstyle\scriptstyle{#1}}%
{\XXint\scriptstyle\scriptscriptstyle{#1}}%
{\XXint\scriptscriptstyle\scriptscriptstyle{#1}}%
\!\int}
\def\XXint#1#2#3{{\setbox0=\hbox{$#1{#2#3}{\int}$ }
\vcenter{\hbox{$#2#3$ }}\kern-.6\wd0}}
\def\ddashint{\Xint=}
\def\dashint{\Xint-}

\usepackage[backend=bibtex,style=alphabetic,giveninits=true]{biblatex}
\renewcommand*{\bibfont}{\normalfont\footnotesize}
\addbibresource{least_gradient_maps.bib}
\renewbibmacro{in:}{}
\DeclareFieldFormat{pages}{#1}

\newcommand\todo[1]{\textcolor{red}{TODO: #1}}


\begin{document}
\begin{abstract}

\end{abstract}

\maketitle
%%%%%%%%%%%%%%%%%%%%%%%%%%%%%%%%%%%%%%%%%%%%%%%%%%%%%%%
\section{Introduction}
Let $u: M \to N$ be a Lipschitz map between Riemannian manifolds.
If $M$ is a euclidean domain and $N = \RR$, then it is a consequence of the theory of viscosity solutions that there is essentially a unique notion of what it means for $u$ to be optimally Lipschitz, namely that $u$ is \dfn{$\infty$-harmonic}, in the sense that $u$ is a viscosity solution of 
$$(\nabla^i \partial^j u) \partial_i u \partial_j u = 0.$$

In the general case, however, the story drastically changes.
In addition to the classical notions of absolutely minimizing Lipschitz (AML) maps and strong AML maps \cite{Juutinen06}, we have tight, measure-tight, and (classically) $\infty$-harmonic maps \cite{Sheffield2010VectorvaluedOL} and variational $\infty$-harmonic maps \cite{daskalopoulos2022analytic}.\footnote{There is another notion of $\infty$-harmonic map in the literature \cite{Ou2012} which appears to be completely unrelated to the problem at hand.}
One would hope that, after developing a substitute for viscosity solutions for manifold-valued maps, we could show that most of these notions are equivalent.
We shall show that this dream is too optimistic, and give counterexamples to several natural implications.
However, we shall also show that under suitable hypotheses, some of the desired implications \emph{do} hold.

%%%%%%%%%%%%%%%%%%%%%%%%%%%
\subsection{Preliminaries}
We write
$$\Lip(u, S) := \sup_{\substack{x, y \in S \\ x \neq y}} \frac{\dist(u(x), u(y))}{\dist(x, y)}$$
for the Lipschitz constant of a map $u: S \to M$.
In particular if $\card S \leq 1$ then $\Lip(u, S) = -\infty$.
We set
$$\Lip(u, x) := \inf_{r > 0} \Lip(u, B(x, r))$$
which is upper semicontinuous \cite[Lemma 4.2(a)]{Crandall2008}.
If $U$ is a convex open set, then \cite[Lemma 4.2(d)]{Crandall2008}
$$\Lip(u, U) = \sup_{x \in U} \Lip(u, x).$$

%%%%%%%%%%%%%%%%%%%%%%%%%%%
\subsection{Acknowledgements}
I would like to thank Georgios Daskalopoulos for helpful comments.

This research was supported by the National Science Foundation's Graduate Research Fellowship Program under Grant No. DGE-2040433.

\section{AML and strong AML}
\begin{definition}
An \dfn{absolutely minimizing Lipschitz} (or \dfn{AML}) map $u: M \to N$ is a map such that for every open $U \subseteq M$,
$$\Lip(u, U) = \Lip(u, \partial U).$$
The map $u$ is \dfn{strong AML} if, instead, for every open $U \subseteq M$,
$$\sup_{x \in U} \Lip(u, x) = \sup_{x \in \partial U} \Lip(u, x).$$
\end{definition}

If $M$ is convex and $N = \RR$, then strong AML and AML are equivalent \cite[\S6]{Crandall2008}.
In general, one does not expect a relation between strong AML and AML without suitable convexity assumptions, since we have $\Lip(u, x) \leq \Lip(u, U)$ whenever $U$ is an open neighborhood of $x$, but the converse
$$\Lip(u, U) = \sup_{x \in U} \Lip(u, x)$$
only holds when $U$ is convex.

However, convexity is not enough to establish the equivalence of AML and strong AML.
The obstruction comes in the proof of the Kirszbraun-Valentine theorem \cite[Theorem A]{Lang1997}, which requires the use of comparison of geodesic triangles with those in a model geometry.
To state the curvature hypotheses necessary to legitimize such comparisons, we establish some notation:
\begin{enumerate}
\item Let $\Gr_2(P)$ denote the Grassmanian bundle of $2$-planes tangent to a Riemannian manifold $P$.
\item Let $\Sec_P: \Gr_2(P) \to \RR$ denote the sectional curvature of $P$.
\item For $\kappa \in \RR$, let $D_\kappa$ denote the diameter of the model geometry of curvature $\kappa$.
\end{enumerate}

\begin{theorem}[Kirszbraun-Valentine]
Suppose that $M, N$ are convex Riemannian manifolds, $S \subseteq M$, $L \geq 0$, $u: S \to N$ is a Lipschitz map with $\Lip(u, S) \leq L$, and one of the following curvature conditions holds:
\begin{enumerate}
\item $L \geq 1$, and there exists $\kappa \in \RR$ such that
\begin{align}
\Sec_N \leq \kappa \leq \Sec_M, \label{KV 1} \\
\diam u(S) \leq \frac{D_\kappa}{2}. \label{KV 2}
\end{align}
\item $\Sec_M \geq 0$ and $\Sec_N = 0$.
\end{enumerate}
Then there is a Lipschitz map $\overline u: M \to N$ such that $\overline u|_S = u$ and $\Lip(\overline u, M) \leq L$.
\end{theorem}

\begin{proposition}
Suppose that $M, N$ are convex Riemannian manifolds, $u: M \to N$ is a strong AML map, and one of the following curvature conditions holds:
\begin{enumerate}
\item For every set $S \subseteq M$ with $\card S \geq 2$, $\Lip(u, S) \geq 1$, and there exists $\kappa \in \RR$ such that (\ref{KV 1}) and (\ref{KV 2}) hold.
\item $\Sec_M \geq 0$ and $\Sec_N = 0$.
\end{enumerate}
Then $u$ is AML.
\end{proposition}
\begin{proof}
Let $U \subseteq M$ be open, such that $\Lip(u, \partial U) < \infty$, and let $x, y \in U$.
By the Kirszbraun-Valentine theorem, there is a Lipschitz map $v: \overline U \to N$ with $v|_{\partial U} = u|_{\partial U}$ and $\Lip(v, U) \leq \Lip(u, \partial U)$.
Since $v$ is a competitor on $U$ and $U$ is open,
\begin{equation}\label{stAML bounded by AML}
\sup_{x \in U} \Lip(u, x) \leq \sup_{x \in U} \Lip(v, x) \leq \Lip(v, U) \leq \Lip(u, \partial U).
\end{equation}

Let $x, y \in U$.
Since $M$ is convex, there is a minimizing geodesic $\gamma: [0, T] \to M$ from $x$ to $y$, parametrized by arc length.
We define a function $w$ on a subset of $[0, T]$ by setting $w(t) := u(\gamma(t))$ whenever $\gamma(t) \in \overline U$.
Then $w$ extends to a function on all of $[0, T]$ by linear interpolation.
Since $M \setminus \overline U$ is open, its preimage $V \subseteq (0, T)$ under $\gamma$ is also open, hence can be written $V = \bigcup_j (\alpha_j, \beta_j)$.
Then $\alpha_j, \beta_j \in \partial U$, so
$$\Lip(w, (\alpha_j, \beta_j)) = \frac{|w(\beta_j) - w(\alpha_j)|}{\beta_j - \alpha_j} = \frac{|u(\gamma(\beta_j)) - u(\gamma(\alpha_j))|}{\dist(\beta_j, \alpha_j)} \leq \Lip(u, \partial U).$$
By (\ref{stAML bounded by AML}), for $t \notin V$,
$$\Lip(w, t) \leq \Lip(u, \gamma(t)) \leq \Lip(u, \partial U)$$
and we conclude $\Lip(w, [0, T]) \leq \Lip(u, \partial U)$ since $[0, T]$ is convex.
Therefore 
\begin{align*}
|u(y) - u(x)| &= |w(T) - w(0)| \leq T \Lip(w, [0, T]) \leq \dist(y, x) \Lip(u, \partial U). \qedhere 
\end{align*}
\end{proof}

Gueritaud and Kassel \cite[Example 9.6]{Gueritaud17} constructed a counterexample to the generalization of the Kirszbraun-Valentine theorem to the case $M = N = \Hyp^2$, $L < 1$.
Note that in this case, the model geometry is $\Hyp^2$ itself, which has infinite diameter.
We show that this counterexample also shows that, if the target is negatively curved, then strong AML need not imply AML:

\begin{proposition}\label{strong AML does not imply AML}
Every homothetic contraction $\Hyp^2 \to \Hyp^2$ is strong AML but not AML.
\end{proposition}
\begin{proof}
Fix an origin $0 \in \Hyp^2$ and let $u$ be the homothety centered on $0$ which rescales by a factor of $0 < L < 1$.
We first observe that $\Lip(u, x) = L$ is constant, so $u$ is strong AML.
To see this, we write the hyperbolic metric in polar coordinates centered at $0$:
$$g = \dif r^2 + \sinh^2 r \dif \theta^2.$$
Fix $x = (r, \theta)$ and a tangent vector $\xi = (\rho, \phi)$ at $x$.
Then $\dif u \cdot \xi = (L\rho, \phi)$ is a tangent vector at $u(x) = (L\rho, \theta)$.
We have 
$$\frac{|\dif u \cdot \xi|^2}{|\xi|^2} = \frac{L^2 \rho^2 + \phi^2 \sinh^2(Lr)}{\rho^2 + \phi^2 \sinh^2 r}.$$
Since $\sinh^2 (Lr) < L^2 (\sinh^2 r)$, this ratio is maximized when $\phi = 0$, in which case it is $L^2$.
Therefore $\Lip(u, x) = |\dif u|_\infty = L$.
\todo{Compute Christoffel symbols of the pullback connection in the appendix and show that $u$ is classical $\infty$-harmonic and tight.}

Let $S$ be the set of vertices of an equilateral triangle $\mathcal T$ with barycenter $0$, and let $R$ be the distance of one (and hence any vertex) of $\mathcal T$ to $0$.
It was computed in \cite[Example 9.6]{Gueritaud17} that, with $c := \sqrt 3/2$,
$$\Lip(u, S) = \frac{\asinh(c \sinh (Lt))}{\asinh(c \sinh t)} < L.$$
Since $u$ is smooth and $\Hyp^2$ is convex,
$$\Lip(u, \Hyp^2 \setminus S) = \Lip(u, \Hyp^2) = \sup_{x \in \Hyp^2} \Lip(u, x) = L.$$
Since $\partial(\Hyp^2 \setminus S) = S$, it follows that
$$\Lip(u, \Hyp^2 \setminus S) > \Lip(u, \partial(\Hyp^2 \setminus S))$$
and $u$ cannot be AML.
\end{proof}

\todo{A similar example shows that maps $\Hyp^2 \to \RR^2$ can be strong AML but not AML.}

\todo{What about AML implies strong AML?}

\section{Tight and classically infinity-harmonic maps}
\begin{proposition}
Let $\infty$ denote the south pole of $\Sph^2$ and $0$ its north pole. 
A homothetic contraction $u: \Sph^2 \setminus \{\infty\} \to \Sph^2$ centered on $0$ is strong AML but is not tight.
Moreover, $u$ is smooth and nonconformal on $\Sph^2 \setminus \{0, \infty\}$ but $u$ is not classically $\infty$-harmonic on $\Sph^2 \setminus \{0, \infty\}$.
\end{proposition}
\begin{proof}
We begin as in the proof of Proposition \ref{strong AML does not imply AML}.
The metric, written in polar coordinates centered on $0$, is 
$$g = \dif r^2 + \sin^2 r \dif \theta^2,$$
so we have 
$$\frac{|\dif u \cdot \xi|^2}{|\xi|^2} = \frac{L^2 \rho^2 + \phi^2 \sin^2(Lr)}{\rho^2 + \phi^2 \sin^2 r}.$$
However, $\sin^2 (Lr) > L^2 \sin^2 r$, so this ratio is maximized when $\rho = 0$.
Thus the principal singular vector of $\dif u$ is $\csc r \cdot \partial_\theta$ and
$$\Lip(u, x) = \frac{L^2 \sin^2 (Lr)}{\sin^2 r}.$$
If $U$ is an open subset of $\Sph^2 \setminus \{\infty\}$, then $r|_U$ attains a maximum on $\partial U$, so it follows that $u$ is strong AML.
\todo{Compute the Christoffel symbols of the pullback connection in the appendix and conclude that }
% https://math.stackexchange.com/questions/2458136/pulling-back-affine-or-not-connections
$$\langle \nabla_u \langle \dif u, \csc r \cdot \partial_\theta\rangle, \csc r \cdot \partial_\theta\rangle \neq 0.$$
Therefore $u$ is not classically $\infty$-harmonic on $\Sph^2 \setminus \{0\}$, hence is not tight there.
Since $\dif u$ is continuous, $u$ is not tight near $0$ either.
\end{proof}

\section{Variational infinity-harmonic maps}
\begin{proposition}
Suppose that $M$ is a convex compact Riemannian manifold with nonempty boundary, $N$ is a convex complete Riemannian manifold, and $h: \partial M \to N$ is a Lipschitz map satisfying one of the following curvature conditions:
\begin{enumerate}
\item $\Lip(h, \partial M) \geq 1$, and there exists $\kappa \leq 0$ such that 
$$\sec_N \leq \kappa \leq \sec_M.$$
\item $\sec_M \geq 0$ and $\sec_N = 0$.
\end{enumerate}
Then there is a variational $\infty$-harmonic map $u: M \to N$ such that $u|_{\partial M} = h$ and $\Lip(u, M) = \Lip(h, \partial M)$.
\end{proposition}
\begin{proof}
By a straightforward modification of \cite[Theorem 2.12]{daskalopoulos2022analytic}, there are $p$-harmonic maps $u_p: M \to N$ extending $h$, which minimize the $p$-energy.
By the Kirszbraun-Valentine theorem, there is moreover a Lipschitz map $v: M \to N$ extending $h$ with $\Lip(v, \partial M) = \Lip(h, \partial M)$.
By H\"older's inequality and minimality of $u_p$, if $\dim M < r \leq p < \infty$,
\begin{align*}
\frac{1}{(|M| \dim M)^{1/r}} \left[\int_M |\dif u_p|_r^r \star 1\right]^{1/r}
&\leq \frac{1}{(|M| \dim M)^{1/p}} \left[\int_M |\dif u_p|_p^p \star 1\right]^{1/p} \\
&\leq \frac{1}{(|M| \dim M)^{1/p}} \left[\int_M |\dif v|_p^p \star 1\right]^{1/p}.
\end{align*}
But
$$|\dif v|_p(x) \leq (\dim M)^{1/p} \Lip(v, x) \leq (\dim M)^{1/p} \Lip(h, \partial M).$$
Therefore for each $\varepsilon > 0$, if $r$ is large enough, then
$$\left[\int_M |\dif u_p|_r^r \star 1\right]^{1/r} \leq \Lip(h, \partial M) + \varepsilon.$$
Hence $u_p \weakto u$ in $W^{1, r}$; by Sobolev embedding we have $u|_{\partial M} = h$.
By lower semicontinuity, 
$$\sup_{x \in M} \Lip(u, x) = \||\dif u|_\infty\|_{L^\infty} \leq \Lip(h, \partial M) + \varepsilon,$$
and since $M$ is convex it follows that $\Lip(u, x) \leq \Lip(h, \partial M)$.
\end{proof}

\begin{theorem}
Suppose that $M$ is a convex surface, $N = \RR^d$, and $u: M \to N$ is a variational $\infty$-harmonic map which does not suffer energy evacuation.
Then $u$ is strong AML. 
\end{theorem}

\begin{conjecture}
Every variational $\infty$-harmonic map does not have energy evacuation.
Therefore variational $\infty$-harmonic implies strong AML.
\end{conjecture}

\begin{conjecture}
Let $M, N$ be closed hyperbolic surfaces of genus $2$ such that the canonical lamination of $(M, N)$ does not admit a transverse measure.
Then the variational $\infty$-harmonic map homotopic to $\id: M \to N$ suffers energy evacuation, and is not strong AML.
\end{conjecture}

In the above conjectures it is necessary to make some sort of convexity or topological triviality assumption.
Daskalopoulos and Uhlenbeck showed that for generic pair of points $(g, h)$ in Teichm\"uller space, the induced (variational) $\infty$-harmonic map maximizes its local Lipschitz constant on the Thurston canonical lamination of $(g, h)$ \cite[Theorem 8.11]{daskalopoulos2022analytic}.
The proof uses an $\infty$-harmonic map to the circle which, as we shall now see, is neither AML nor strong AML.

\begin{proposition}
There is a surface $M$ and a (viscosity, hence classical and variational) $\infty$-harmonic map $u: M \to \Sph^1$ such that $u$ is not AML or strong AML.
\end{proposition}
\begin{proof}
Consider the cylinder $M := \Sph^1_\theta \times \RR_x$ with the metric
$$g := \dif x^2 + \frac{\cosh^2 x}{4} \dif \theta^2.$$
Then $M$ has a single closed geodesic $\gamma$, which winds around $\{x = 0\}$, and has circumference $\pi$.
Let $u: M \to \Sph^1$ be the projection map.
Now $\dif u = \dif \theta$, so
$$\Lip(u, (x, \theta)) = |\dif \theta|(x, \theta) = 2 \sech x.$$
This attains its maximum on $\gamma$, and if $U$ is any neighborhood of $\gamma$,
$$\sup_{(x, \theta) \in U} \Lip(u, (x, \theta)) > \sup_{(x, \theta) \in \partial U} \Lip(u, (x, \theta)),$$
hence $u$ is not strong AML.
A similar argument shows that $u$ is not AML.
Using the Christoffel symbols (\ref{hyperbolic cylinder 1})-(\ref{hyperbolic cylinder 2}), we see that
$$\nabla^2 u = \tanh x \dif x \dif \theta,$$
and hence $\Delta_\infty u = 0$.
\end{proof}

\todo{By taking $M \setminus U$ to be small disks around an equilateral triangle in $\Hyp^2$, show that variational $\infty$-harmonic maps need not be AML.}

\appendix 
\section{Geometric computations}
\subsection{Christoffel symbols}
\begin{enumerate}
\item The metric $\dif r^2 + \sinh^2 r \dif \theta^2$ has Christoffel symbols ... hence the pullback ... 
\item The metric $\dif r^2 + \sin^2 r \dif \theta^2$ has Christoffel symbols ... hence the pullback ... 
\item The metric $\dif x^2 + \cosh^2 x \dif \theta^2$ has Christoffel symbols 
\begin{align}
{\Gamma^\theta}_{\theta \theta} &= {\Gamma^\theta}_{x x} = 0, \label{hyperbolic cylinder 1} \\
{\Gamma^\theta}_{x \theta} &= {\Gamma^\theta}_{\theta x} = \tanh x. \label{hyperbolic cylinder 2}
\end{align}
\end{enumerate}


\printbibliography

\end{document}
