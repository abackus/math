\documentclass[reqno,11pt]{amsart}
\usepackage[letterpaper, margin=1in]{geometry}
\RequirePackage{amsmath,amssymb,amsthm,graphicx,mathrsfs,url,slashed,subcaption}
\RequirePackage[usenames,dvipsnames]{xcolor}
\RequirePackage[colorlinks=true,linkcolor=Red,citecolor=Green]{hyperref}
\RequirePackage{amsxtra}
\usepackage{cancel}
\usepackage{tikz, wrapfig}
%\usepackage[T1]{fontenc}

% \setlength{\textheight}{9.3in} \setlength{\oddsidemargin}{-0.25in}
% \setlength{\evensidemargin}{-0.25in} \setlength{\textwidth}{7in}
% \setlength{\topmargin}{-0.25in} \setlength{\headheight}{0.18in}
% \setlength{\marginparwidth}{1.0in}
% \setlength{\abovedisplayskip}{0.2in}
% \setlength{\belowdisplayskip}{0.2in}
% \setlength{\parskip}{0.05in}
%\renewcommand{\baselinestretch}{1.05}

\title{Convex duality for $BV$ variational systems, geodesic laminations, and optimal transport}
\author{Aidan Backus}
\address{Department of Mathematics, Brown University}
\email{aidan\_backus@brown.edu}
\date{\today}

\newcommand{\NN}{\mathbf{N}}
\newcommand{\ZZ}{\mathbf{Z}}
\newcommand{\QQ}{\mathbf{Q}}
\newcommand{\RR}{\mathbf{R}}
\newcommand{\CC}{\mathbf{C}}
\newcommand{\DD}{\mathbf{D}}
\newcommand{\PP}{\mathbf P}
\newcommand{\MM}{\mathbf M}
\newcommand{\II}{\mathbf I}
\newcommand{\Hyp}{\mathbf H}
\newcommand{\Sph}{\mathbf S}
\newcommand{\Group}{\mathbf G}
\newcommand{\GL}{\mathbf{GL}}
\newcommand{\Orth}{\mathbf{O}}
\newcommand{\SpOrth}{\mathbf{SO}}
\newcommand{\Ball}{\mathbf{B}}

\newcommand*\dif{\mathop{}\!\mathrm{d}}

\DeclareMathOperator{\card}{card}
\DeclareMathOperator{\dist}{dist}
\DeclareMathOperator{\id}{id}
\DeclareMathOperator{\Hom}{Hom}
\DeclareMathOperator{\PD}{PD}
\DeclareMathOperator{\coker}{coker}
\DeclareMathOperator{\supp}{supp}
\DeclareMathOperator{\sech}{sech}
\DeclareMathOperator{\Teich}{Teich}
\DeclareMathOperator{\tr}{tr}

\newcommand{\Leaves}{\mathscr L}
\newcommand{\Lagrange}{\mathscr L}
\newcommand{\Hypspace}{\mathscr H}

\newcommand{\Chain}{\underline C}


\newcommand{\weakto}{\rightharpoonup}

\newcommand{\Two}{\mathrm{I\!I}}
\newcommand{\Ric}{\mathrm{Ric}}

\newcommand{\normal}{\mathbf n}
\newcommand{\radial}{\mathbf r}
\newcommand{\evect}{\mathbf e}
\newcommand{\vol}{\mathrm{vol}}

\newcommand{\diam}{\mathrm{diam}}
\newcommand{\Ell}{\mathrm{Ell}}
\newcommand{\inj}{\mathrm{inj}}
\newcommand{\Lip}{\mathrm{Lip}}
\newcommand{\MCL}{\mathrm{MCL}}
\newcommand{\sgn}{\mathrm{sgn}}
\newcommand{\Riem}{\mathrm{Riem}}

\newcommand{\Mass}{\mathbf M}
\newcommand{\Comass}{\mathbf L}

\newcommand{\Min}{\mathrm{Min}}
\newcommand{\Max}{\mathrm{Max}}

\newcommand{\dfn}[1]{\emph{#1}\index{#1}}

\renewcommand{\Re}{\operatorname{Re}}
\renewcommand{\Im}{\operatorname{Im}}

\newcommand{\loc}{\mathrm{loc}}
\newcommand{\cpt}{\mathrm{cpt}}

\DeclareMathOperator*{\essinf}{ess\,inf}
\DeclareMathOperator*{\esssup}{ess\,sup}

\def\Japan#1{\left \langle #1 \right \rangle}

\newtheorem{theorem}{Theorem}[section]
\newtheorem{badtheorem}[theorem]{``Theorem"}
\newtheorem{prop}[theorem]{Proposition}
\newtheorem{lemma}[theorem]{Lemma}
\newtheorem{sublemma}[theorem]{Sublemma}
\newtheorem{proposition}[theorem]{Proposition}
\newtheorem{corollary}[theorem]{Corollary}
\newtheorem{conjecture}[theorem]{Conjecture}
\newtheorem{axiom}[theorem]{Axiom}
\newtheorem{assumption}[theorem]{Assumption}

\newtheorem{mainthm}{Theorem}
\renewcommand{\themainthm}{\Alph{mainthm}}

\newtheorem{claim}{Claim}[theorem]
\renewcommand{\theclaim}{\thetheorem\Alph{claim}}
% \newtheorem*{claim}{Claim}

\theoremstyle{definition}
\newtheorem{definition}[theorem]{Definition}
\newtheorem{remark}[theorem]{Remark}
\newtheorem{warning}[theorem]{Warning}
\newtheorem{example}[theorem]{Example}
\newtheorem{notation}[theorem]{Notation}

\newtheorem{exercise}[theorem]{Discussion topic}
\newtheorem{homework}[theorem]{Homework}
\newtheorem{problem}[theorem]{Problem}

\makeatletter
\newcommand{\proofpart}[2]{%
  \par
  \addvspace{\medskipamount}%
  \noindent\emph{Part #1: #2.}
}
\makeatother



\numberwithin{equation}{section}


% Mean
\def\Xint#1{\mathchoice
{\XXint\displaystyle\textstyle{#1}}%
{\XXint\textstyle\scriptstyle{#1}}%
{\XXint\scriptstyle\scriptscriptstyle{#1}}%
{\XXint\scriptscriptstyle\scriptscriptstyle{#1}}%
\!\int}
\def\XXint#1#2#3{{\setbox0=\hbox{$#1{#2#3}{\int}$ }
\vcenter{\hbox{$#2#3$ }}\kern-.6\wd0}}
\def\ddashint{\Xint=}
\def\dashint{\Xint-}

\usepackage[backend=bibtex,style=alphabetic,giveninits=true]{biblatex}
\renewcommand*{\bibfont}{\normalfont\footnotesize}
\addbibresource{minimal_calib.bib}
\renewbibmacro{in:}{}
\DeclareFieldFormat{pages}{#1}

\newcommand\todo[1]{\textcolor{red}{TODO: #1}}


\begin{document}
\begin{abstract}
    We use the $p$-Laplacian.
\end{abstract}

\maketitle

%%%%%%%%%%%%%%%%%%%%%%%%%%%%%%%%%%%%%%%%%%%%%%%%%%%%%%%
\section{Introduction}
We're going to use the ideas from \cite{górny2022dualitybased}.
In geometric applications we want to do $BV$ variational problems on currents so restricting to scalar fields like in the reference is a little silly.

Throughout, we assume that $M$ is a compact oriented Riemannian manifold with boundary and dimension $d$.
We assume that $1 \leq \ell \leq d - 1$.

\begin{definition}
A \dfn{$BV$ Lagrangian} is a continuous function $\Lagrange(x, \xi)$ on the $\ell$th exterior power $\Omega^\ell$ of the cotangent bundle of $M$, such that:
\begin{enumerate}
\item For each $x \in M$, $\Lagrange(x, \cdot)$ is convex.
\item One has 
$$|\Lagrange(x, \xi)| \lesssim \Japan \xi.$$
\item The \dfn{blowup at zero}
$$\Lagrange_0(x, \xi) := \lim_{t \to 0} t \Lagrange(x, \xi/t)$$
exists and is continuous. 
\end{enumerate}
\end{definition}

\section{The minimizing problem}
If $T$ is a closed $\ell$-current on $M$, we can decompose
$$T = \alpha + \dif u$$
where $\alpha$ is a harmonic $\ell$-form, and $u \in BV(M, \Omega^{\ell - 1})$ vanishes at $\partial M$.
Note that we absorb all the boundary data into $\alpha$, and in particular we can't see the torsion of the homology.
We know this is sharp since $H^1(SO(3), \ZZ) = \ZZ/2$ and the torsion geodesic is length-minimizing but clearly not calibrable.
We specify the boundary data of $T$, by specifying the cohomology class $\alpha$, and the trace $h$ of $u$.

\begin{definition}
A closed $\ell$-current $T$ is \dfn{$\Lagrange$-minimizing} if for every $v \in BV(M, \Omega^{\ell - 1})$ with $\iota^* v = 0$,
$$\int_M \Lagrange(T) \leq \int_M \Lagrange(T + \dif v).$$
\end{definition}

We probably need to relax this formulation.

Let $T^{\rm sing}$ be the singular part of $\dif u$.
In particular it doesn't see homology.
$M/\partial M$ isn't quite right in the below. I want whatever Chapter 5.9 of Taylor's PDE 1 says I want.

\begin{definition}
Let $\alpha$ be a harmonic $\ell$-form with $\iota^* \alpha \in L^1$.
A closed $\ell$-current $T$ is \dfn{$\Lagrange$-calibrable} cohomologous to $\alpha$ if there exists $F \in L^\infty(M, \Omega^{d - \ell})$ and $u \in BV(M, \RR)$ such that in the sense of subdifferentials,
\begin{align}
F &\in \star \partial_\xi \Lagrange(T) \\
\dif F &= 0 \\
F \wedge T^{\rm sing} &= \Lagrange_0(T^{\rm sing}) \\
\alpha + \dif u &= T \\
\iota^* \dif u &= 0 \\
\sgn(\iota^* F) &= \sgn(\star_{\partial M} \iota^* \alpha).
\end{align}
\end{definition}

By the Fenchel-Rockafellar theorem, $T$ is $\Lagrange$-minimizing iff $T$ is $\Lagrange$-calibrable.
In particular a solution exists where the boundary condition is interpreted in the relaxed sense, not in the trace sense.
The boundary condition in the definition of ``calibrable" doesn't quite make sense and I need to work out what it actually is.

Now observe that if $T$ is $\Lagrange(\xi) = |\xi|$ then we can solve for a tight form.

\begin{corollary}
A mass-minimizing submanifold (for real homology) is calibrable.
\end{corollary}

Need to explain how this corollary is compatible with \cite{liu2023homologically} which constrains when a manifold is calibrable.

\section{The eigenvalue problem}
We now do the same thing but instead minimize $\int |T|$ subject to $\int |u| = \lambda$.
We might be able to incorporate this into the previous problem, by adding $|\int|u| - \lambda|$ to the relaxed $BV$ functional.
What does this mean for $\dif F$?

\begin{corollary}
Every Cheeger submanifold is calibrable in the CMC sense.
\end{corollary}

So we can estimate the isoperimetric constant using calibrations.

\section{Geodesic laminations}
\begin{corollary}
    Every closed mass-minimizing $d - 1$-current is Ruelle-Sullivan for a geodesic lamination.
\end{corollary}
\begin{proof}
If $T$ is as given, let $F$ be the dual calibration $1$-form.
Then $F = \dif v$ for a $\infty$-harmonic function $v$.
Now use comparison with cones like in \cite{daskalopoulos2020transverse}.
We then need to prove the Lipschitz regularity but that's the same too.
\end{proof}

For the eigenvalue problem, we instead get curves of constant geodesic curvature. 
Can they intersect? What does that look like geometrically?

\begin{corollary}
Existence of $1$-dimensional canonical laminations.
\end{corollary}

\begin{corollary}
Suppose that $\alpha, \beta$ are in the in the same flat of the stable norm sphere for $H_1$.
Then $\alpha \cdot \beta = 0$.
\end{corollary}

This seems to have an application to optimal transport(!!) since the reason why G\'orny needs to assume $d = 2$ in \cite{górny2021applications} is so he can interpret $\dif u$ as a mass-minimizing $1$-current.
Maybe I should email Gorny and ask him if he wants to talk about this.

\printbibliography

\end{document}
