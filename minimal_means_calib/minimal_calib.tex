\documentclass[reqno,11pt]{amsart}
\usepackage[letterpaper, margin=1in]{geometry}
\RequirePackage{amsmath,amssymb,amsthm,graphicx,mathrsfs,url,slashed,subcaption}
\RequirePackage[usenames,dvipsnames]{xcolor}
\RequirePackage[colorlinks=true,linkcolor=Red,citecolor=Green]{hyperref}
\RequirePackage{amsxtra}
\usepackage{cancel}
\usepackage{tikz, wrapfig}
%\usepackage[T1]{fontenc}

% \setlength{\textheight}{9.3in} \setlength{\oddsidemargin}{-0.25in}
% \setlength{\evensidemargin}{-0.25in} \setlength{\textwidth}{7in}
% \setlength{\topmargin}{-0.25in} \setlength{\headheight}{0.18in}
% \setlength{\marginparwidth}{1.0in}
% \setlength{\abovedisplayskip}{0.2in}
% \setlength{\belowdisplayskip}{0.2in}
% \setlength{\parskip}{0.05in}
%\renewcommand{\baselinestretch}{1.05}

\title[$BV$ variational systems and geodesic laminations]{Convex duality for $BV$ variational systems, and geodesic laminations as optimal transporters}
\author{Aidan Backus}
\address{Department of Mathematics, Brown University}
\email{aidan\_backus@brown.edu}
\date{\today}

\newcommand{\NN}{\mathbf{N}}
\newcommand{\ZZ}{\mathbf{Z}}
\newcommand{\QQ}{\mathbf{Q}}
\newcommand{\RR}{\mathbf{R}}
\newcommand{\CC}{\mathbf{C}}
\newcommand{\DD}{\mathbf{D}}
\newcommand{\PP}{\mathbf P}
\newcommand{\MM}{\mathbf M}
\newcommand{\II}{\mathbf I}
\newcommand{\Hyp}{\mathbf H}
\newcommand{\Sph}{\mathbf S}
\newcommand{\Group}{\mathbf G}
\newcommand{\GL}{\mathbf{GL}}
\newcommand{\Orth}{\mathbf{O}}
\newcommand{\SpOrth}{\mathbf{SO}}
\newcommand{\Ball}{\mathbf{B}}

\newcommand*\dif{\mathop{}\!\mathrm{d}}

\DeclareMathOperator{\card}{card}
\DeclareMathOperator{\dist}{dist}
\DeclareMathOperator{\id}{id}
\DeclareMathOperator{\Hom}{Hom}
\DeclareMathOperator{\PD}{PD}
\DeclareMathOperator{\coker}{coker}
\DeclareMathOperator{\supp}{supp}
\DeclareMathOperator{\sech}{sech}
\DeclareMathOperator{\Teich}{Teich}
\DeclareMathOperator{\tr}{tr}

\newcommand{\Leaves}{\mathscr L}
\newcommand{\Lagrange}{\mathscr L}
\newcommand{\Hypspace}{\mathscr H}

\newcommand{\Chain}{\underline C}


\newcommand{\weakto}{\rightharpoonup}

\newcommand{\Two}{\mathrm{I\!I}}
\newcommand{\Ric}{\mathrm{Ric}}

\newcommand{\normal}{\mathbf n}
\newcommand{\radial}{\mathbf r}
\newcommand{\evect}{\mathbf e}
\newcommand{\vol}{\mathrm{vol}}

\newcommand{\diam}{\mathrm{diam}}
\newcommand{\Ell}{\mathrm{Ell}}
\newcommand{\inj}{\mathrm{inj}}
\newcommand{\Lip}{\mathrm{Lip}}
\newcommand{\MCL}{\mathrm{MCL}}
\newcommand{\sgn}{\mathrm{sgn}}
\newcommand{\Riem}{\mathrm{Riem}}

\newcommand{\Mass}{\mathbf M}
\newcommand{\Comass}{\mathbf L}

\newcommand{\Min}{\mathrm{Min}}
\newcommand{\Max}{\mathrm{Max}}

\newcommand{\dfn}[1]{\emph{#1}\index{#1}}

\renewcommand{\Re}{\operatorname{Re}}
\renewcommand{\Im}{\operatorname{Im}}

\newcommand{\loc}{\mathrm{loc}}
\newcommand{\ac}{\mathrm{ac}}
\newcommand{\cl}{\mathrm{cl}}
\newcommand{\cpt}{\mathrm{cpt}}
\newcommand{\sing}{\mathrm{sing}}

\DeclareMathOperator*{\essinf}{ess\,inf}
\DeclareMathOperator*{\esssup}{ess\,sup}

\def\Japan#1{\left \langle #1 \right \rangle}

\newtheorem{theorem}{Theorem}[section]
\newtheorem{badtheorem}[theorem]{``Theorem"}
\newtheorem{prop}[theorem]{Proposition}
\newtheorem{lemma}[theorem]{Lemma}
\newtheorem{sublemma}[theorem]{Sublemma}
\newtheorem{proposition}[theorem]{Proposition}
\newtheorem{corollary}[theorem]{Corollary}
\newtheorem{conjecture}[theorem]{Conjecture}
\newtheorem{axiom}[theorem]{Axiom}
\newtheorem{assumption}[theorem]{Assumption}

\newtheorem{mainthm}{Theorem}
\renewcommand{\themainthm}{\Alph{mainthm}}

\newtheorem{claim}{Claim}[theorem]
\renewcommand{\theclaim}{\thetheorem\Alph{claim}}
% \newtheorem*{claim}{Claim}

\theoremstyle{definition}
\newtheorem{definition}[theorem]{Definition}
\newtheorem{remark}[theorem]{Remark}
\newtheorem{warning}[theorem]{Warning}
\newtheorem{example}[theorem]{Example}
\newtheorem{notation}[theorem]{Notation}

\newtheorem{exercise}[theorem]{Discussion topic}
\newtheorem{homework}[theorem]{Homework}
\newtheorem{problem}[theorem]{Problem}

\makeatletter
\newcommand{\proofpart}[2]{%
  \par
  \addvspace{\medskipamount}%
  \noindent\emph{Part #1: #2.}
}
\makeatother



\numberwithin{equation}{section}


% Mean
\def\Xint#1{\mathchoice
{\XXint\displaystyle\textstyle{#1}}%
{\XXint\textstyle\scriptstyle{#1}}%
{\XXint\scriptstyle\scriptscriptstyle{#1}}%
{\XXint\scriptscriptstyle\scriptscriptstyle{#1}}%
\!\int}
\def\XXint#1#2#3{{\setbox0=\hbox{$#1{#2#3}{\int}$ }
\vcenter{\hbox{$#2#3$ }}\kern-.6\wd0}}
\def\ddashint{\Xint=}
\def\dashint{\Xint-}

\usepackage[backend=bibtex,style=alphabetic,giveninits=true]{biblatex}
\renewcommand*{\bibfont}{\normalfont\footnotesize}
\addbibresource{minimal_calib.bib}
\renewbibmacro{in:}{}
\DeclareFieldFormat{pages}{#1}

\newcommand\todo[1]{\textcolor{red}{TODO: #1}}


\begin{document}
\begin{abstract}
    We use the $p$-Laplacian.
\end{abstract}

\maketitle

%%%%%%%%%%%%%%%%%%%%%%%%%%%%%%%%%%%%%%%%%%%%%%%%%%%%%%%
\section{Introduction}
G\'orny and Maz\'on \cite{górny2021applications} studied variational problems whose Lagrangians have linear growth, using convex duality.
The natural Banach space for such problems is $BV$.
In geometric applications we want to do $BV$ variational problems on currents so restricting to scalar fields like in the reference is a little silly.
On the other hand, in optimal control and geometric analysis it turns out that currents cover all the applications.

Throughout, we assume that $M$ is a compact oriented Riemannian manifold with Lipschitz boundary $\partial M$ and dimension $d$.
We assume that $1 \leq \ell \leq d$, and let $\Omega^\ell$ be the $\ell$th exterior power of the cotangent bundle of $M$.
Let $\Omega^\ell_\cl$ be the subsheaf of \emph{closed} $\ell$-forms.

\begin{definition}
A \dfn{Lagrangian of linear growth} is a continuous function $\Lagrange$ on $\Omega^\ell$ such that:
\begin{enumerate}
\item For each $x \in M$, $\Lagrange(x, \cdot)$ is convex.
\item One has 
$$|\Lagrange(x, \xi)| \lesssim \Japan \xi.$$
\item The \dfn{blowdown}
$$\Lagrange_\infty(x, \xi) := \lim_{\varepsilon \to 0} \varepsilon \Lagrange(x, \xi/\varepsilon)$$
exists and is continuous. 
\end{enumerate}
\end{definition}

The existence of blowdowns is a particularly natural assumption, because we are interested in plugging in a singular measure to $\xi$.
Clearly this is only possible if $\Lagrange$ is well-behaved at infinity.

%%%%%%%%%%%%%%%%%%%%%%%%%%%%%
\section{The minimizing problem}
\subsection{Boundary data}
To specify a boundary condition, let
$$j: \partial M \to \overline M$$
be the inclusion map, and let $N := \overline M/\partial M$, which is a closed oriented manifold.
\todo{Is this really true?}

\begin{lemma}[Hodge theorem with Dirichlet constraints]
The cohomology group $H^\ell(N, \RR)$ is naturally isomorphic to the space of $\alpha \in C^\infty(M, \Omega^\ell)$ such that 
\begin{equation}\label{Dirichlet Hodge Laplacian}
\begin{cases}
  (\dif + \dif^*)^2 \alpha = 0 \\
  j^* \alpha = 0\\
  j^* (\dif^* \alpha) = 0.
\end{cases}
\end{equation}
\end{lemma}
\begin{proof}
By a long exact sequence in Hatcher, $H^\ell(N, \RR)$ is identified with the relative cohomology $H^\ell((M, \partial M), \RR)$, which is then identified with the above space of harmonic $\ell$-forms by \cite[Chapter 5, Proposition 9.9]{taylor2010partial}.
\end{proof}

Henceforth we think of $H^\ell(N, \RR)$ as the space of solutions of (\ref{Dirichlet Hodge Laplacian}) without comment.

\begin{definition}
The space of \dfn{Dirichlet boundary data} is $H^\ell(N, \RR) \times L^1(\partial M, \Omega^{d - 1})$.
\end{definition}

By the inverse trace theorem, there exists $u \in W^{1, 1}(M, \Omega^\ell)$ with $j^* u = h$.
However, $W^{1, 1}$ does not have a weakly compact unit ball, so it is necessary to pass to the larger space $BV$.
This is possible, because the pullback is bounded
$$j^*: BV(M, \Omega^{\ell - 1}) \to L^1(\partial M, \Omega^{\ell - 1}).$$

\begin{definition}
Let
$$(\alpha, h) \in H^\ell(N, \RR) \times L^1(\partial M, \Omega^{d - 1})$$
be Dirichlet boundary data.
The \dfn{competition class} $\mathscr A$ with data $(\alpha, h)$ is the space of all $\ell$-currents $T$ such that there exists $u \in BV(M, \Omega^{\ell - 1})$ such that $j^* u = h$ and
$$T = \alpha + \dif u.$$
\end{definition}

Our boundary condition cannot see torsion classes in $H^\ell(N, \ZZ)$.
But we know that this is necessary: if $M = N = SO(3)$, then $M$ is orientable with a torsion class $\alpha \in H^2(M, \ZZ)$.
The geodesic representing the Poincar\'e dual of $\alpha$ cannot be calibrated, so the dual formulation of the minimization problem does not make much sense in this context.

\todo{In \cite[Chapter 5, \S9]{taylor2010partial} we learn how to interchange the Dirichlet and Neumann data using the Hodge star.}
Henceforth we will only consider the Dirichlet problem.

%%%%%%%%%%%%%%%
\subsection{Minimizers}
\todo{Define the current $\Lagrange(T)$ using the blowdown.}

Let $(\alpha, h)$ be Dirichlet boundary data, and let $\mathscr A$ be the induced competition class.
We are interested in minimizing $\int_M \Lagrange(T)$ subject to $T \in \mathscr A$.
Unfortunately this is not actually possible.
So, we are going to minimize the relaxed functional
$$\mathcal F_{\alpha, h}(u) := \int_M \Lagrange(\alpha + \dif u) + \int_{\partial M} \Lagrange_\infty((h - u) \wedge \normal^\flat)$$
subject to the constraint $u \in L^2(M, \Omega^{\ell - 1})$.

%%%%%%%%%%%%%%
\subsection{Calibration of solutions}
Let $(\alpha, h)$ be Dirichlet boundary data for the Lagrangian of linear growth $\Lagrange$.

\begin{definition}
A form $u \in L^2(M, \Omega^{\ell - 1})$ is \dfn{$\Lagrange$-calibrable} subject to the boundary condition $(\alpha, h)$ if there exists $F \in L^\infty(M, \Omega^{d - \ell}_{\rm cl})$ such that:
\begin{align}
&F \in \partial_\xi \Lagrange(\alpha + \dif u^\ac), \label{calibration is a subgradient} \\
&F \wedge \dif u^{\rm sing} = \Lagrange_\infty(\dif u^\sing), \label{calibration of singular part} \\
&\Lagrange_\infty((h - u) \wedge \normal^\flat) = (h - u) \wedge F.\label{calibration of boundary condition}
\end{align}
We call $F$ a \dfn{$\Lagrange$-calibration} of $u$.
\end{definition}

\todo{Show that 
$$j^*(u \wedge F) \leq \Lagrange_\infty(u \wedge \normal^\flat)$$
}

\todo{Redo all this to allow for subdifferentials, so we can do the parabolic and eigenvalue problems easily.}

\begin{lemma}
If $u$ is $\Lagrange$-calibrable subject to the boundary condition $(\alpha, h)$, then $u$ minimizes $\mathcal F_{\alpha, h}$.
\end{lemma}
\begin{proof}
Let $F$ be a $\Lagrange$-calibration of $u$, and let $w \in L^2(M, \Omega^{\ell - 1})$ be a competitor.
By an approximation argument, we may assume that $w$ is smooth.
We use $\dif F = 0$ to integrate by parts:
\begin{align*}
0 
&= (-1)^\ell \int_M (w - u) \wedge \dif F \\
&= \int_M \dif(w - u) \wedge F - \int_{\partial M} (w - u) \wedge F \\
&= \int_M (\dif w - \dif u^\ac) \wedge F - \int_M \dif u^\sing \wedge F + \int_{\partial M} (h - w) \wedge F - \int_{\partial M} (h - u) \wedge F \\
&= \mathbf I - \mathbf{II} + \mathbf{III} - \mathbf{IV}.
\end{align*}
We then use (\ref{calibration is a subgradient}), (\ref{calibration of singular part}), ???, and (\ref{calibration of boundary condition}) respectively to compute:
\begin{align*}
\mathbf I &\leq \int_M \Lagrange(\alpha + \dif w) - \Lagrange(\alpha + \dif u^\ac), \\
-\mathbf{II} &= -\int_M \Lagrange_\infty(\dif u^\sing), \\ 
\mathbf{III} &\leq \int_{\partial M} \Lagrange_\infty((h - w) \wedge \normal^\flat), \\
-\mathbf{IV} &= -\int_{\partial M} \Lagrange_\infty((h - u) \wedge \normal^\flat).
\end{align*}
Adding up these terms and grouping in terms involving $w$ and terms involving $u$, we conclude 
\begin{align*}
0 
&\leq \int_M \Lagrange(\alpha + \dif w) + \int_{\partial M} \Lagrange_\infty((h - w) \wedge \normal^\flat) \\
&\qquad - \int_M \Lagrange(\alpha + \dif u^\ac) - \int_M \Lagrange_\infty(\dif u^\sing) - \int_{\partial M} \Lagrange_\infty((h - u) \wedge \normal^\flat) \\
&= \mathcal F_{\alpha, h}(w) - \mathcal F_{\alpha, h}(u). \qedhere
\end{align*}
\end{proof}

By the Fenchel-Rockafellar theorem, $T$ is $\Lagrange$-minimizing iff $T$ is $\Lagrange$-calibrable.
In particular a solution exists where the boundary condition is interpreted in the relaxed sense, not in the trace sense.
The boundary condition in the definition of ``calibrable" doesn't quite make sense and I need to work out what it actually is.

Now observe that if $T$ is $\Lagrange(\xi) = |\xi|$ then we can solve for a tight form.

\begin{corollary}
A mass-minimizing submanifold (for real homology) is calibrable.
\end{corollary}

Need to explain how this corollary is compatible with \cite{liu2023homologically} which constrains when a manifold is calibrable.

\section{The eigenvalue problem}
We now do the same thing but instead minimize $\int |T|$ subject to $\int |u| = \lambda$.
We might be able to incorporate this into the previous problem, by adding $|\int|u| - \lambda|$ to the relaxed $BV$ functional.
What does this mean for $\dif F$?

\begin{corollary}
Every Cheeger submanifold is calibrable in the CMC sense.
\end{corollary}

So we can estimate the isoperimetric constant using calibrations.

\section{Geodesic laminations}
\begin{corollary}
    Every closed mass-minimizing $d - 1$-current is Ruelle-Sullivan for a geodesic lamination.
\end{corollary}
\begin{proof}
If $T$ is as given, let $F$ be the dual calibration $1$-form.
Then $F = \dif v$ for a $\infty$-harmonic function $v$.
Now use comparison with cones like in \cite{daskalopoulos2020transverse}.
We then need to prove the Lipschitz regularity but that's the same too.
\end{proof}

For the eigenvalue problem, we instead get curves of constant geodesic curvature. 
Can they intersect? What does that look like geometrically?

\begin{corollary}
Existence of $1$-dimensional canonical laminations.
\end{corollary}

\begin{corollary}
Suppose that $\alpha, \beta$ are in the in the same flat of the stable norm sphere for $H_1$.
Then $\alpha \cdot \beta = 0$.
\end{corollary}

Unfortunately this last corollary isn't very interesting, because if $d = 2$ it's known and if $d \geq 3$ then you should always be able to perturb the geodesics so they don't intersect.
On the other hand, if $d \geq 3$ then maybe we can conclude that the canonical laminations are all disjoint, and so we get a master geodesic lamination of sorts containing ALL of the calibrated laminations.

This seems to have an application to optimal transport(!!) since the reason why G\'orny needs to assume $d = 2$ in \cite{górny2021applications} is so he can interpret $\dif u$ as a mass-minimizing $1$-current.
Maybe I should email Gorny and ask him if he wants to talk about this.

Can we use the same ideas as the proof of regularity of optimal transport in higher dimensions as well, to get regularity theorems for lots of $BV$ problems?

Can we use the dual problem in higher dimensions to show that a minimizing $k$-current has minimizing level sets?

Sometimes, like in my work with Aaron Kennon, you can show that there are smooth solutions of the (global, not local) dual problem using a heat flow.
When can we do this?
When can we, as in Auer--Bangert '06, use this to get regularity for the $BV$ minimizer?

\printbibliography

\end{document}
