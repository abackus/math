\documentclass[reqno,12pt,letterpaper]{amsart}
\RequirePackage{amsmath,amssymb,amsthm,graphicx,mathrsfs,url}
\RequirePackage[usenames,dvipsnames]{color}
\RequirePackage[colorlinks=true,linkcolor=Red,citecolor=Green]{hyperref}
\RequirePackage{amsxtra}
\usepackage{cancel}
\usepackage{tikz-cd}

\setlength{\textheight}{8.50in} \setlength{\oddsidemargin}{0.00in}
\setlength{\evensidemargin}{0.00in} \setlength{\textwidth}{6.08in}
\setlength{\topmargin}{0.00in} \setlength{\headheight}{0.18in}
\setlength{\marginparwidth}{1.0in}
\setlength{\abovedisplayskip}{0.2in}
\setlength{\belowdisplayskip}{0.2in}
\setlength{\parskip}{0.05in}
\renewcommand{\baselinestretch}{1.10}

\title{Ricci flow}
\author{Aidan Backus}
\date{December 2021}

\newcommand{\NN}{\mathbf{N}}
\newcommand{\ZZ}{\mathbf{Z}}
\newcommand{\QQ}{\mathbf{Q}}
\newcommand{\RR}{\mathbf{R}}
\newcommand{\CC}{\mathbf{C}}
\newcommand{\DD}{\mathbf{D}}
\newcommand{\PP}{\mathbf P}
\newcommand{\MM}{\mathbf M}
\newcommand{\II}{\mathbf I}
\newcommand{\Hyp}{\mathbf H}

\DeclareMathOperator{\card}{card}
\DeclareMathOperator{\cent}{center}
\DeclareMathOperator{\ch}{ch}
\DeclareMathOperator{\codim}{codim}
\DeclareMathOperator{\diag}{diag}
\DeclareMathOperator{\diam}{diam}
\DeclareMathOperator{\dom}{dom}
\DeclareMathOperator{\Gal}{Gal}
\DeclareMathOperator{\Hom}{Hom}
\DeclareMathOperator{\Jac}{Jac}
\DeclareMathOperator{\Lip}{Lip}
\DeclareMathOperator{\Met}{Met}
\DeclareMathOperator{\id}{id}
\DeclareMathOperator{\rad}{rad}
\DeclareMathOperator{\rank}{rank}
\DeclareMathOperator{\Radon}{Radon}
\DeclareMathOperator*{\Res}{Res}
\DeclareMathOperator{\sgn}{sgn}
\DeclareMathOperator{\singsupp}{sing~supp}
\DeclareMathOperator{\Spec}{Spec}
\DeclareMathOperator{\supp}{supp}
\DeclareMathOperator{\Tan}{Tan}
\newcommand{\tr}{\operatorname{tr}}

\newcommand{\Ric}{\mathrm{Ric}}
\newcommand{\Riem}{\mathrm{Riem}}

\newcommand{\dbar}{\overline \partial}

\DeclareMathOperator{\atanh}{atanh}
\DeclareMathOperator{\csch}{csch}
\DeclareMathOperator{\sech}{sech}

\DeclareMathOperator{\Div}{div}
\DeclareMathOperator{\grad}{grad}
\DeclareMathOperator{\Ell}{Ell}
\DeclareMathOperator{\WF}{WF}

\newcommand{\Hilb}{\mathcal H}
\newcommand{\normal}{\mathbf n}
\newcommand{\vol}{\mathrm{vol}}

\newcommand{\pic}{\vspace{30mm}}
\newcommand{\dfn}[1]{\emph{#1}\index{#1}}

\renewcommand{\Re}{\operatorname{Re}}
\renewcommand{\Im}{\operatorname{Im}}


\newtheorem{theorem}{Theorem}[section]
\newtheorem{badtheorem}[theorem]{``Theorem"}
\newtheorem{prop}[theorem]{Proposition}
\newtheorem{lemma}[theorem]{Lemma}
\newtheorem{claim}[theorem]{Claim}
\newtheorem{proposition}[theorem]{Proposition}
\newtheorem{corollary}[theorem]{Corollary}
\newtheorem{conjecture}[theorem]{Conjecture}
\newtheorem{axiom}[theorem]{Axiom}
\newtheorem{assumption}[theorem]{Assumption}

\theoremstyle{definition}
\newtheorem{definition}[theorem]{Definition}
\newtheorem{remark}[theorem]{Remark}
\newtheorem{example}[theorem]{Example}
\newtheorem{notation}[theorem]{Notation}

\newtheorem{exercise}[theorem]{Discussion topic}
\newtheorem{homework}[theorem]{Homework}
\newtheorem{problem}[theorem]{Problem}

\newtheorem{ack}{Acknowledgements}

\numberwithin{equation}{section}


% Mean
\def\Xint#1{\mathchoice
{\XXint\displaystyle\textstyle{#1}}%
{\XXint\textstyle\scriptstyle{#1}}%
{\XXint\scriptstyle\scriptscriptstyle{#1}}%
{\XXint\scriptscriptstyle\scriptscriptstyle{#1}}%
\!\int}
\def\XXint#1#2#3{{\setbox0=\hbox{$#1{#2#3}{\int}$ }
\vcenter{\hbox{$#2#3$ }}\kern-.6\wd0}}
\def\ddashint{\Xint=}
\def\dashint{\Xint-}

%\usepackage{color}
%\hypersetup{%
%    colorlinks=true, % make the links colored%
%    linkcolor=blue, % color TOC links in blue
%    urlcolor=red, % color URLs in red
%    linktoc=all % 'all' will create links for everything in the TOC
%Ning added hyperlinks to the table of contents 6/17/19
%}

% style=alphabetic
\usepackage[backend=bibtex,maxcitenames=50,maxnames=50]{biblatex}
\addbibresource{ricci_flow.bib}
\renewbibmacro{in:}{}
\DeclareFieldFormat{pages}{#1}

\begin{document}
\begin{abstract}
Notes on Ricci flow, based on Terry Tao's notes on the Poincar\'e conjecture
\end{abstract}

\maketitle

%%%%%%%%%%%%%%%%%%%%%%%%%%%%%%%%%%%%%%%%%%%%%%%%%%%%%%%

\tableofcontents

%%%%%%%%%%%%%%%%%%%%%%%%%%%%%%%%%%%%%%%%%%%%%%%%%%%%%%%%%%%%%%%%%%%%

\section{Connections}
Fix a smooth manifold $M$.

\begin{definition}
Let $V$ be a vector bundle. A \dfn{connection} on $V$ is the data of a \dfn{covariant derivative} $\nabla_X f$, a section of $V$,
for every vector field $X$ on $M$ and every section $f$ of $V$, such that
\begin{enumerate}
\item $(X, f) \mapsto \nabla_X f$ is bilinear.
\item If $\lambda$ is a scalar field then we have the \dfn{Leibniz rule}
$$\nabla_X(\lambda f) = \lambda \nabla_X f + (X\lambda)f.$$
\item If $\lambda$ is a scalar field then
$$\nabla_{\lambda X + Y} f = \lambda \nabla_X f + \nabla_Y f.$$
\end{enumerate}
An \dfn{affine connection} is a connection on $TM$.
\end{definition}

Note that every connection on $V$ also acts on all tensor powers of $V$, simply by declaring that the Leibniz rule be true and observing that we have a natural multiplication map
$$V^{\otimes k} times V^{\otimes \ell} \to V^{\otimes (k + \ell)}.$$
In particular, since the trivial bundle is $V^{\otimes 0}$, every connection acts on scalar fields.

If we have a fixed (possibly abstract) frame $\partial$, we write $\nabla_\beta = \nabla_{\partial_\beta}$.

\begin{definition}
Let $\nabla$ be an affine connection in a fixed frame $\partial$.
The \dfn{Christoffel symbols} of $\nabla$ are defined by
$$\Gamma^\alpha_{\beta\gamma} = \partial^\alpha \nabla_\beta \partial_\gamma.$$
\end{definition}

Since the Christoffel symbols depend on a choice of frame, they do not define a tensor field on $M$.
On the other hand, the Christoffel symbols determine $\nabla$.
It is clear that they are given by $d^3$ scalar fields, where $d$ is the dimension of $M$.

If $\nabla$ is an affine connection, we have
$$(\nabla_X Y)^\alpha = \nabla_X (Y^\alpha) + \Gamma^\alpha_{\beta\gamma} X^\beta Y^\gamma.$$

\begin{definition}
A connection $\nabla$ is \dfn{torsion-free} if
$$[X, Y] = \nabla_X Y - \nabla_Y X.$$
\end{definition}

\begin{definition}
Let $\nabla$ be a connection on $V$.
If for every $X$, $\nabla_X f = 0$, then $f$ is \dfn{parallel} to $\nabla$.
\end{definition}

It follows from the definitions that $\nabla$ is torsion-free iff every scalar field is parallel to $[\nabla_\alpha, \nabla_\beta]$.
However, it does not follow (thanks to the Leibniz rule) that this holds for tensor fields, so our next definition makes sense.

\begin{definition}
If $\nabla$ is a torsion-free affine connection, its \dfn{curvature} is the tensor field
$$R(X, Y) = [\nabla_X, \nabla_Y] - \nabla_{[X, Y]}$$
where $R(X, Y)$ is viewed as acting on vector fields; equivalently,
$$R^\delta_{\alpha\beta\gamma} X^\gamma = [\nabla_\alpha, \nabla_\beta] X^\delta.$$
The connection $\nabla$ is \dfn{flat} if its curvature vanishes.
\end{definition}

\begin{theorem}[Levi-Civita]
For every Riemannian metric $g$ there exists a unique torsion-free affine connection $\nabla$ such that $g$ is parallel to $\nabla$.
\end{theorem}
\begin{proof}
It suffices to specify the Christoffel symbols
$$\nabla_i \partial_j = \Gamma^{\ell}_{ij} \partial_\ell$$
and show they are uniquely defined.
The torsion-free condition imposes $\nabla_{\partial_i} \partial_j = \nabla_{\partial_j} \partial_i$, and this imposes the symmetry
$$\Gamma_{ij}^\ell = \Gamma_{ji}^\ell.$$
Writing out the condition that $g$ is parallel gives
$$\partial_k g_{ij} = \Gamma^\ell_{ki} g_{\ell j} + \Gamma^m_{kj} g_{im}.$$
Solving these equations gives the unique solution
$$2\Gamma^\ell_{ij} = (g_{jk,i} - g_{ij,k} + g_{ik,j})g^{k\ell}$$
which was desired.
\end{proof}

If $\nabla$ is the above connection, we can define $\nabla^\alpha = g^{\alpha\beta} \nabla_\beta$, which causes no trouble since $\nabla$ commutes with $g$.

\begin{definition}
The \dfn{Levi-Civita connection} $\nabla$ on a Riemannian manifold $(M, g)$ is the unique torsion-free affine connection to which $g$ is parallel.
The \dfn{Riemann curvature tensor} $\Riem$ of $g$ is the curvature of $\nabla$.
The \dfn{Laplace-Beltrami operator} is the differential operator
$$\Delta = \nabla_\alpha \nabla^\alpha.$$
The metric $g$ is \dfn{flat} if $\nabla$ is flat.
\end{definition}

Since the Levi-Civita connection is the canonical connection to use on a Riemannian manifold we will just implicitly refer to it.
In fact we write
$$f_{;\alpha} = \nabla_\alpha f$$
whenever $f$ is a tensor field.

\begin{theorem}[Bianchi identities]
One has
$$\Riem^\gamma_{\alpha \beta \delta} + \Riem^\gamma_{\beta\delta\alpha} + \Riem^\gamma_{\delta\alpha\beta} = 0$$
and
$$\Riem^\mu_{\alpha\beta\delta;\mu} + \Riem^\gamma_{\mu\alpha\delta;\beta} + \Riem^\gamma_{\beta\mu\delta;\alpha} = 0.$$
\end{theorem}




\printbibliography


\end{document}
