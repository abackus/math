\documentclass[final,12pt, leqno]{brownthesis}
\usepackage{lmodern}
\usepackage{setspace}
\usepackage{colonequals}
\usepackage[all,cmtip]{xy}
%\usepackage[alphabetic]{amsrefs}
\usepackage[T1]{fontenc} 
\usepackage{enumitem,kantlipsum}
\usepackage{textcomp} 
\usepackage{bm}
\usepackage{dsfont}
\usepackage{titlesec,float}
\usepackage{subfig}
\usepackage{mathtools,mathrsfs,epigraph,hyperref}
\usepackage{array}

\titleformat{\chapter}[display]
{\bfseries\LARGE}
{\filleft\MakeUppercase{\chaptertitlename} \Huge\thechapter}
{4ex}
{\titlerule
\vspace{2ex}%
\filright}
[\vspace{2ex}%
\titlerule]

\usepackage{amsmath,amssymb,graphicx,amsfonts,amsthm}
\usepackage{tikz,tikz-cd}

\makeatletter
\g@addto@macro\normalsize{%
  \setlength\abovedisplayskip{6pt}
  \setlength\belowdisplayskip{6pt}
  \setlength\abovedisplayshortskip{6pt}
  \setlength\belowdisplayshortskip{6pt}
}
\makeatother

\usepackage{amssymb}

\let\oldemptyset\emptyset
\let\emptyset\varnothing
%Questions

\newcommand{\todo}[1]{\textcolor{blue}{TODO: #1}}
\newcommand{\note}[1]{\textcolor{green}{Note: #1}}

\allowdisplaybreaks









\newcommand{\NN}{\mathbf{N}}
\newcommand{\ZZ}{\mathbf{Z}}
\newcommand{\QQ}{\mathbf{Q}}
\newcommand{\RR}{\mathbf{R}}
\newcommand{\CC}{\mathbf{C}}
\newcommand{\DD}{\mathbf{D}}
\newcommand{\PP}{\mathbf P}
\newcommand{\MM}{\mathbf M}
\newcommand{\II}{\mathbf I}
\newcommand{\Hyp}{\mathbf H}
\newcommand{\Sph}{\mathbf S}
\newcommand{\Group}{\mathbf G}
\newcommand{\GL}{\mathbf{GL}}
\newcommand{\Orth}{\mathbf{O}}
\newcommand{\SpOrth}{\mathbf{SO}}
\newcommand{\Ball}{\mathbf{B}}

\DeclareMathOperator*{\Expect}{\mathbf E}

\DeclareMathOperator{\avg}{avg}
\DeclareMathOperator{\card}{card}
\DeclareMathOperator{\codim}{codim}
\DeclareMathOperator{\diag}{diag}
\DeclareMathOperator{\diam}{diam}
\DeclareMathOperator{\dom}{dom}
\DeclareMathOperator{\Exc}{Exc}
\DeclareMathOperator{\Lip}{Lip}
\DeclareMathOperator{\Hom}{Hom}
\DeclareMathOperator{\id}{id}
\DeclareMathOperator{\rad}{rad}
\DeclareMathOperator{\rank}{rank}
\DeclareMathOperator{\Rm}{Rm}
\DeclareMathOperator{\Hess}{Hess}
\DeclareMathOperator{\sgn}{sgn}
\DeclareMathOperator{\supp}{supp}
\newcommand{\tr}{\operatorname{tr}}

\newcommand{\Mink}{\mathbf m}
\newcommand{\Ric}{\mathrm{Ric}}
\newcommand{\Riem}{\mathrm{Riem}}
\newcommand*\dif{\mathop{}\!\mathrm{d}}
\newcommand*\Dif{\mathop{}\!\mathrm{D}}
\newcommand{\LapQL}{\Delta^{\mathrm{ql}}}

\newcommand{\dbar}{\overline \partial}

\DeclareMathOperator{\atanh}{atanh}
\DeclareMathOperator{\csch}{csch}
\DeclareMathOperator{\sech}{sech}

\DeclareMathOperator{\Div}{div}
\DeclareMathOperator{\Gram}{Gram}
\DeclareMathOperator{\grad}{grad}
\DeclareMathOperator{\dist}{dist}
\DeclareMathOperator{\spn}{span}
\DeclareMathOperator{\Ell}{Ell}
\DeclareMathOperator{\WF}{WF}

\newcommand{\Two}{\mathrm{I\!I}}

\newcommand{\Lagrange}{\mathscr L}
\newcommand{\DirQL}{\mathscr D^{\mathrm{ql}}}
\newcommand{\DirL}{\mathscr D}

\newcommand{\Leaves}{\mathcal L}
\newcommand{\Hypspace}{\mathscr H}

\newcommand{\normal}{\mathbf n}
\newcommand{\vol}{\mathrm{vol}}
\newcommand{\dfn}[1]{\emph{#1}\index{#1}}

\renewcommand{\Re}{\operatorname{Re}}
\renewcommand{\Im}{\operatorname{Im}}

\newcommand{\loc}{\mathrm{loc}}
\newcommand{\cpt}{\mathrm{cpt}}

\def\Japan#1{\left \langle #1 \right \rangle}

\newtheorem{theorem}{Theorem}[section]
\newtheorem{badtheorem}[theorem]{``Theorem"}
\newtheorem{prop}[theorem]{Proposition}
\newtheorem{lemma}[theorem]{Lemma}
\newtheorem{sublemma}[theorem]{Sublemma}
\newtheorem{proposition}[theorem]{Proposition}
\newtheorem{corollary}[theorem]{Corollary}
\newtheorem{conjecture}[theorem]{Conjecture}
\newtheorem{axiom}[theorem]{Axiom}
\newtheorem{assumption}[theorem]{Assumption}

% \newtheorem{claim}{Claim}[theorem]
% \renewcommand{\theclaim}{\thetheorem\Alph{claim}}
\newtheorem*{claim}{Claim}

\theoremstyle{definition}
\newtheorem{definition}[theorem]{Definition}
\newtheorem{remark}[theorem]{Remark}
\newtheorem{example}[theorem]{Example}
\newtheorem{notation}[theorem]{Notation}

\newtheorem{exercise}[theorem]{Discussion topic}
\newtheorem{homework}[theorem]{Homework}
\newtheorem{problem}[theorem]{Problem}

\makeatletter
\newcommand{\proofpart}[2]{%
  \par
  \addvspace{\medskipamount}%
  \noindent\emph{Part #1: #2.}
}
\makeatother



\numberwithin{equation}{section}


% Mean
\def\Xint#1{\mathchoice
{\XXint\displaystyle\textstyle{#1}}%
{\XXint\textstyle\scriptstyle{#1}}%
{\XXint\scriptstyle\scriptscriptstyle{#1}}%
{\XXint\scriptscriptstyle\scriptscriptstyle{#1}}%
\!\int}
\def\XXint#1#2#3{{\setbox0=\hbox{$#1{#2#3}{\int}$ }
\vcenter{\hbox{$#2#3$ }}\kern-.6\wd0}}
\def\ddashint{\Xint=}
\def\dashint{\Xint-}


\usepackage[backend=bibtex,style=alphabetic,giveninits=true,hyperref,backref,backrefstyle=none]{biblatex}
\addbibresource{reference.bib}
\renewbibmacro{in:}{}
\DeclareFieldFormat{pages}{#1}


% \usepackage[footnotesize,bf]{caption}  % Reduces caption font sizes
\begin{document}
\doublespacing
\pagenumbering{alph} % hack to suppress hyperref errors
\submitdate{May 2025}
\title{\huge \textbf{Limiting behavior of convex duality for the $p$-Laplacian}}
\author{Aidan Backus}
%\degree=2
%\university{Brown University}
\dept{Mathematics}
%\faculty{Graduate School}
\degrees{
	} 
	
\principaladvisor{Georgios Daskalopolous}
 \reader{}
\reader{}
 \dean{Thomas A. Lewis}
  %\abstract{}

      %\abstractpage
      %\abstractpage
     \beforepreface
       \prefacesection{Vita}

\newpage


\prefacesection{Acknowledgments}
I would like to first and foremost thank and acknowledge my intellectual debt to my advisor, Georgios Daskalopolous, who suggested much of this work and always had helpful comments and suggestions.
I would also like to thank Georgios' advisor, Karen Uhlenbeck, whose influence is also apparent throughout this work.
Indeed, much of this thesis can be viewed as an expansion of the seminal work \cite{daskalopoulos2020transverse} of Georgios and Karen on best Lipschitz functions as solutions of the $\infty$-Laplace equation.

Since this work is rather indisciplinary, I have often consulted with other researchers in various fields.
The dramatis personae here includes but is not limited to Tai Borges, Kaya Ferendo, Tom Goodwillie, Haram Ko, Jeremy Kahn, JiaHua Zou, Stephen Obinna, and various pseudonymous members of the Analysts' Lair chat room on Discord.

Victor Bangert allowed me to view an early draft of \todo{his unfinished paper with Auer} which was a heavy influence on \todo{the stable norm parts}.

Wojciech Górny and I have had several helpful discussions about functions of least gradient and his influence can be felt throughout.

Much of this work was originally formulated in dimensions $d = 2, 3$.
Chao Li suggested \todo{his work} which can be used to extend this work to the case $d = 4$, and William Minicozzi suggested \todo{Schoen} which brings this work to its natural hypotheses of $2 \leq d \leq 7$.

My topics committee, consisting of Georgios, Christine Breiner, and Benoit Pausader, and my thesis committee, consisting of Georgios \todo{and the others}, both provided vital feedback which has influenced this work and my intellectual development more broadly.

My Ph.D. was supported financially by the National Science Foundation's Graduate Research Fellowship Program under Grant No. DGE-2040433.

Finally, as this Ph.D. was written at a time of crisis in the world (namely the COVID-19 pandemic) and in my personal life (as it was finished in spite of a car accident and legal and medical challenges), I would like to express gratitude for the immense emotional support I have received during my Ph.D., especially from my wife, Java Darleen Villano, but also my parents, Antoinette and Garrett Backus, and the Brown University math department Ph.D. coclasses of 2020 and 2021.
It is reasonable to assume that this work would not have been finished without it.



\tableofcontents
%\listoftables
\listoffigures
\afterpreface
\doublespacing




%\end{preliminaries}
%\newpage
%\endofprelim

\pagestyle{myheadings}


%------------------------ CONTENT ------------------------%
 %\ChNameUpperCase
 
 
 
\chapter{Introduction}
Let $p \in (1, \infty)$ and $M$ a Riemannian manifold of dimension $\geq 2$.
The $p$-Laplacian is the Euler-Lagrange equation 
$$\dif^*(|\dif u|^{p - 2} \dif u) = 0$$
associated to the $p$-Dirichlet energy
$$E_p(u) := \frac{1}{p} \int_M |\dif u|^p \star 1.$$
(Henceforth we shall simply call this quantity the \dfn{energy} of $u$, if $p$ is understood).

If $p = 2$, this equation reduces to the Laplacian, and on Riemann surfaces, the Laplacian is self-dual with respect to convex duality.
To be more precise, if $u$ is harmonic, then $u + iv$ is holomorphic iff $v$ is harmonic, and the Cauchy-Riemann equation
$$\begin{cases}
	\partial_x u - \partial_y v = 0 \\
	\partial_y u + \partial_x v = 0 
\end{cases}$$
is exactly the strong duality relation arising from the fact that the Fenchel transform of $E_2$ is itself.
In higher dimensions, the convex dual to the Laplacian is an elliptic system that we shall not elaborate on more here.

An analogue of the Cauchy-Riemann equations relating the $p$-Laplacian and $q$-Laplacian, where $p^{-1} + q^{-1} = 1$, was studied on hyperbolic Riemann surfaces \todo{by George and Karen}.
Nearly every result in this thesis is an outgrowth of an idea which appeared in loc.\ cit., and as such we rapidly review that series of papers in Chapter \ref{DaskUhlen}.
The goal of this thesis is to study the convex dual of the $p$-Laplacian in the limits $p \to 1$ and $p \to \infty$, even if $M$ is not a Riemann surface but possibly is a Riemannian manifold of dimension $\leq 7$.

Minimizers of the energy $E_1$ are said to have \dfn{least gradient}; by the coarea formula, we have 
$$E_1(u) = \int_{-\infty}^\infty |\partial \{u > y\}| \dif y.$$
This functional does not punish steep ascents or even jumps in $u$, and it forces that functions of least gradient must have level sets of minimal perimeter.
These properties have made functions of least gradient useful in \todo{applications} as well as in Teichm\"uller theory \todo{cite George and Karen}.
This last point shall be explained in Chapter \ref{convlams}, we see that the level sets of functions of least gradient form a lamination $\lambda$ by area-minimizing hypersurfaces, and this property completely describes functions of least gradient.
Along the way, we establish certain regularity and compactness theorems for laminations.

As a warm-up for Chapter \ref{convlams}, we establish, in Chapter \ref{deGiorgi}, de Giorgi's lemma on the regularity of minimal perimeter via Miranda's iteration technique \todo{cite these papers}.
This was known to be possible, and in any case the regularity of minimal perimeters in Riemannian manifolds had been established by other means, but to my knowledge nobody had written down the details of Miranda iteration in the presence of a Riemannian metric.

We turn to the convex dual of $E_1$ in Chapter \ref{bestcurl}.
Assuming that $M$ is closed, we construct dual closed $d - 1$-forms called \dfn{tight calibrations}.
The tight calibration is a maximal flow in the sense of the max-flow min-cut theorem and it is bottlenecked by the minimal lamination given by the map of least gradient.
On a formal level, the tight calibration $F$ solves the PDE 
$$
\begin{cases}
(\nabla_i F_{j_1 \cdots j_{d - 1}}) F^{j_1 \cdots j_{d - 1}} {F^i}_{k_1 \cdots k_{d - 2}} = 0, \\
\dif F = 0,
\end{cases}$$
though the lack of a theory of viscosity solutions for systems of PDE makes this assertion surprisingly difficult to make precise.
If $d = 2$, this equation reduces to the $\infty$-Laplacian whenever $F = \dif v$.

In Teichm\"uller theory, one associates to each homotopy class $\rho$ of maps $M \to N$ between two closed Riemann surfaces of the same genus $g \geq 2$ a geodesic lamination $\lambda$ of $M$, known as the \dfn{canonical lamination} of $\rho$ \todo{cite Thurston}.
The geometry of the canonical laminations encode certain facts about the duality between the Thurston norm on the tangent bundle to Teichm\"uller space $\widetilde{\mathcal M_g}$ and its dual norm on the cotangent bundle.
In our setting, we are able to construct a canonical lamination associated to each cohomology class $\rho \in H^{d - 1}(M, \RR)$ on the closed Riemannian manifold $M$.
The canonical laminations encode the analogous facts about the duality between Federer's stable norm on $H_{d - 1}(M, \RR)$ and its dual norm, which we call the \dfn{costable norm}, on $H^{d - 1}(M, \RR)$.
We have myriad topological applications of the structure theory of the canonical lamination.

\chapter{Preliminaries}
\section{Notation}
We write $\Japan \xi := \sqrt{1 + |\xi|^2}$ for the Japanese norm of a vector $\xi$.
We write $\Ball^d$ for the unit ball in euclidean space, and $\Sph^{d - 1}$ for the unit sphere.

The operator $\star$ is the Hodge star on $M$, thus $\star 1$ is the Riemannian measure of $M$.
We denote the musical isomorphisms by $\sharp, \flat$.
If $U$ is an open set, we write $|U| := \int_U \star 1$ for the volume of $U$, but if $U$ is a submanifold or rectifiable set of positive codimension, we instead write $|U|$ for its surface measure.
The $\delta$-dimensional Hausdorff measure is $\mathcal H^\delta$.

If we specify that $M$ is a manifold with boundary, we specifically mean that $M$ has a smooth boundary.
If $E$ is a set of locally finite perimeter, we assume that the boundaries of $E$ in the sense of point-set topology and measure theory agree; see Appendix \ref{boundary conventions}.

By a \dfn{hypersurface} we mean a $C^1$ submanifold of codimension $1$.
We write $\normal_N$ for the normal vector (or conormal $1$-form) for a hypersurface $N$, $\nabla_N$ for the Levi-Civita connection, and $\Two_N := \nabla_N \normal_N$ for the second fundamental form if it is defined.

For a map $F: X \to Y$ between metric spaces, we write $\Lip(F)$ for its Lipschitz constant.
If $X, Y$ are connected Riemannian manifolds, one of which is $1$-dimensional, then we have $\Lip(F) = \|\dif F\|_{L^\infty}$.

Let $X$ be a topological space, and $(Y_n)$ a sequence of closed subsets of $X$.
The \dfn{limit inferior} $\liminf_{n \to \infty} Y_n$ is the set of all $x \in X$ such that for every open neighborhood $U \ni x$, $U \cap Y_n$ is eventually nonempty.
The \dfn{limit superior} $\limsup_{n \to \infty} Y_n$ is the set of all $x \in X$ such that for every open neighborhood $U \ni x$, $U \cap Y_n$ is nonempty for infinitely many $n$.
If $\liminf_{n \to \infty} Y_n = \limsup_{n \to \infty} Y_n$, we call that set the \dfn{Hausdorff limit} $\lim_{n \to \infty} Y_n$.

\section{Radon measures}\label{portmanteau appendix}
Let $X$ be a metrizable space, and let $C_\cpt(X)$ be the space of compactly supported continuous functions $f: X \to \RR$.
Its dual $C_\cpt(X)'$ is canonically isomorphic to the space of signed Radon measures on $X$, where the bilinear pairing is given by integration.
The weak-star topology on $C_\cpt(X)'$ is known as the \dfn{weak topology of measures}.
Unpacking the definitions, a sequence $(\mu_n)$ of Radon measures converges to $\mu$ in the weak topology of measures iff for every continuous function $f: X \to \RR$,
$$\lim_{n \to \infty} \int_X f \dif \mu_n = \int_X f \dif \mu.$$

\begin{proposition}[portmanteau theorem]\label{portmanteau}
	Let $(\mu_n)$ be a sequence of positive Radon measures on a compact metrizable space $X$ with $\mu_n(X) \lesssim 1$, and let $\mu$ be a Radon measure on $X$. The following are equivalent:
\begin{enumerate}
	\item $\mu_n \to \mu$ in the weak topology of measures.
	\item $\liminf_{n \to \infty} \mu_n(X) \geq \mu(X)$ and for every closed $Y \subseteq X$, $\limsup_{n \to \infty} \mu_n(Y) \leq \mu(Y)$.
	\item $\limsup_{n \to \infty} \mu_n(X) \leq \mu(X)$ and for every open $Z \subseteq X$, $\liminf_{n \to \infty} \mu_n(Z) \geq \mu(Z)$.
	\item For every $W \subseteq X$ with $\mu(\partial W) = 0$, $\lim_{n \to \infty} \mu_n(W) = \mu(W)$.
\end{enumerate}
	If we choose a metric on $X$, then the above conditions imply:
\begin{enumerate}
	\setcounter{enumi}{4}
	\item For every $x \in X$ and all but countably many $\varepsilon > 0$, $\lim_{n \to \infty} \mu_n(B(x, \varepsilon)) = \mu(B(x, \varepsilon))$.
\end{enumerate}
\end{proposition}

\begin{lemma}\label{cardinality appendix}
Let $S$ be an uncountable set and $f: S \to (0, \infty)$. Then there exists a countable set $S' \subset S$ such that
$$\sum_{x \in S'} f(x) = \infty.$$
\end{lemma}
\begin{proof}
Define $S_n := f^{-1}([\frac{1}{n + 1}, \frac{1}{n}))$ for $n \in \NN$ (where we take the convention $1/n = \infty$).
Since $S$ is uncountable but $\NN$ is countable, it follows from the infinite pigeonhole principle that there exists $n \in \NN$ such that $S_n$ is infinite.
In particular there exists an infinite countable set $S' \subseteq S_n$, which then satisfies
\begin{align*}
\sum_{x \in S'} f(x) &\geq \sum_{x \in S'} \frac{1}{n + 1} = \infty. \qedhere 
\end{align*}
\end{proof}

\begin{proof}[Proof of Proposition \ref{portmanteau}]
	See \cite[Theorem 13.16]{klenke2013probability} for the equivalence of (1)--(4); \cite{klenke2013probability} deals with subprobability measures, but this is equivalent to measures of bounded total mass by a rescaling.

	We then must show that (4) implies (5); to do so, it suffices to show that for all but countably many $\varepsilon$, $\mu(\partial B(x, \varepsilon)) = 0$.
	Let
	$$A := \{\varepsilon > 0: \mu(\partial B(x, \varepsilon)) > 0\}.$$
	Since the sets $\partial B(x, \varepsilon)$ are disjoint, for every countable $A' \subseteq A$,
	$$\sum_{\varepsilon \in A'} \mu(\partial B(x, \varepsilon)) \leq \mu(X) < \infty,$$
	where $\mu(X) < \infty$ since $X$ is compact.
	It follows from (the contrapositive of) Lemma \ref{cardinality appendix} that $A$ is countable.
\end{proof}

There are subtleties involved in the portmanteau theorem for noncompact $X$.
However, this will never be an issue, as we shall only use it locally, in small precompact balls.

Now let $X = M$ be a manifold, and consider instead the space $C_\cpt(M, \Omega^\ell)$ of compactly supported continuous $\ell$-forms.
An $\ell$-\dfn{current of locally finite mass} (which we will simply call an \dfn{$\ell$-current}) is an element of the dual space $C_\cpt(M, \Omega^\ell)'$ \cite{simon1983GMT}.
We denote the pairing of an $\ell$-current $T$ and an $\ell$-form $\varphi$ by $\int_M T \wedge \varphi$; this defines the weak topology of measures on the space of currents.
To any $\ell$-current $T$ we may associate a positive Radon measure, its \dfn{mass measure} $\star |T|$, which satisfies for any function $f$,
$$\int_M f \star |T| := \sup_{|\varphi| \leq |f|} \int_M T \wedge \varphi,$$
and a $|T|$-measurable $d - \ell$-form $\psi$, the \dfn{polar part} \cite[Theorem 4.14]{simon1983GMT}, which satisfies $T = \psi |T|$, $|T|$-almost everywhere.

A function $u$ has \dfn{locally bounded variation} (denoted $u \in BV_\loc(M)$) if its distributional derivative $\dif u$ is a $d - 1$-current of locally finite mass, and a measurable set $E \subseteq M$ has \dfn{locally finite perimeter} if $1_E \in BV_\loc(M)$.



\section{Functions of bounded variation}\label{boundary conventions}
Given a function $u \in L^1_\loc(M)$, we can define its \dfn{total variation measure} in an open set $U$ by
$$\|\dif u\|_{TV(U)} := \sup_{\|\psi\|_{C^0} \leq 1} \int_M u \dif \psi$$
where $\psi$ ranges over all $d-1$-forms with compact support in $U$.
We sometimes abuse notation and write $\star |\dif u|$ for the total variation measure, even if $|\dif u|$ is a distribution but not a function.
Similarly we sometimes write $\star \partial_i u$, if we have chosen a coordinate system $(\partial_i)$.

If the total variation measure is a Radon measure, we say that $u \in BV_\loc(M)$, or that $u$ has \dfn{locally bounded variation}.
This is a diffeomorphism-invariant condition.
If $u$ has locally bounded variation and $U \subseteq M$ is open with Lipschitz boundary, then by \cite[Theorem 2.10]{Miranda67}, the trace $u|_{\partial U} \in L^1_\loc(\partial U)$ is well-defined.
Moreover, by \cite[Theorem 4.14]{simon1983GMT}, there exists a $\star|\dif u|$-measurable section $f$, the \dfn{polar part} of $\dif u$, of the cosphere bundle $S'M$ such that $\partial_i u = f_i |\dif u|$.
As in \cite{Miranda66, Giusti77}, most of the technical work in this paper amounts to controlling the oscillation of a polar section $f$ at fine scales.
In order to make this precise, we shall need to take averages of $f$ and apply a version of the Lebesgue differentiation theorem for curved vector bundles.

\begin{proposition}[Lebesgue differentiation theorem for a vector bundle]\label{LebesgueDiff}
Let $E \to M$ be a vector bundle over an oriented smooth manifold $M$, $\omega$ a Radon measure, and $f \in L^1_\loc(M, E, \omega)$.
Then there exists an $\omega$-null set $Z \subset M$ such that for every Riemannian metric on $M$, every trivialization $(F_1, \dots, F_\ell)$ of $E$ with dual trivialization $(F'_1, \dots, F'_\ell)$ of $E'$, and every $P \in M \setminus Z$,
$$f(P) = \lim_{r \to 0} \sum_{i=1}^\ell \left[\frac{\int_{B(P, r)} (F'_i, f) \dif \omega}{\omega(B(P, r))}\right] F_i(P).$$
\end{proposition}

We shall apply this proposition with $E := T'M$, $F_i = \dif x^i$.
Note carefully that the terms inside the limit \emph{are} dependent on the metric and the choice of trivialization; the content of this result is that \emph{the set of Lebesgue points is diffeomorphism-invariant}.

\begin{proof}
Choose a flat Riemannian metric, let $\mathcal F = ((F_i), (F_i'))$ be a pair of paralellizations of $E, E'$ such that $(F_i', F_j) = \delta_{ij}$, and $\ell$ the rank of $E$.
Then for every $\delta > 0$ there exists $\tilde f \in C_c(M, E)$ such that $\|f - \tilde f\|_{L^1(\omega)} < \delta$, thus
\begin{align*}
&\left|\sum_{i=1}^\ell \left[(F_i'(x), f(x)) - \dashint_{B(x, r)} (F_i', f) \dif \omega\right] F_i(x)\right| \\
&\qquad \leq \left|\sum_{i=1}^\ell (F_i'(x), f(x) - \tilde f(x)) F_i(x)\right| + \dashint_{B(x, r)} \left|\sum_{i=1}^\ell (F_i', f - \tilde f)F_i(x) \dif \omega \right| \\
&\qquad \qquad + \left|\sum_{i=1}^\ell \left[(F_i'(x), \tilde f(x)) - \dashint_{B(x, r)} (F_i, \tilde f) \dif \omega\right] F_i(x)\right| \\
&\qquad =: I_1(x) + I_{2, r}(x) + I_{3, r}(x).
\end{align*}
Here the integral defining $I_{2, r}(x)$ is valued in the fiber $E_x$ and $I_{3, r} \to 0$, $\omega$-almost everywhere as $x \to 0$.

By the proof of the Lebesgue differentiation theorem \cite[Chapter 3, Theorem 1.3]{stein2009real},
\begin{align*}
&\left\{x \in M: \limsup_{r \to 0} \left|\sum_{i=1}^\ell \left[(F_i'(x), f(x)) - \dashint_{B(x, r)} (F_i', f) \dif \omega\right] F_i(x)\right| > 2\varepsilon\right\} \\
&\qquad \subseteq \{I_1 > \varepsilon\} \cup \bigcap_{r > 0} \bigcup_{s <r} \{I_{2, s} > \varepsilon\} \\
&\qquad \subseteq \{f - \tilde f| > \varepsilon\} \cup \bigcap_{r > 0} \bigcup_{s < r} \left\{\dashint_{B(x, r)} |f - \tilde f| \dif \omega > \varepsilon\right\}
\end{align*}
The right-hand side is independent of $\mathcal F$ and $\delta$, but has $\omega$-measure $\lesssim \delta/\varepsilon$.
We choose $\delta \ll \varepsilon$ to conclude that it is $\omega$-null, and remains as such if we take a union over all possible $\mathcal F$ and $\varepsilon$.
Moreover, if $h$ is any Riemannian metric, then its balls have bounded eccentricity with respect to $g$, so we may replace $g$-balls with $h$-balls in the above set inclusion \cite[Chapter 3, Corollary 1.7]{stein2009real}.
\end{proof}

\begin{corollary}
The polar section $f: M \to S'M$ of a $BV$ function $u$ satisfies
\begin{equation}\label{Lebesgue point definition}
    f_\mu(P) = \left[\lim_{r \to 0} \frac{\int_{B(P, r)} \partial_\mu u \dif V}{\|\dif u\|_{TV(B(P, r))}}\right]
\end{equation}
for any coordinate system $(x^\mu)$ and any Riemannian metric $g$, and $\star|\dif u|$-almost every $P$.
The exceptional set does not depend on $(x^\mu)$ or $g$.
\end{corollary}

It follows from the above corollary that the following definitions, largely taken from \cite[Definition 3.3]{Giusti77}, which a priori refer to the metric or to a choice of coordinate system, are actually completely determined by the smooth structure on $M$.

\begin{definition}
Let $U \subseteq M$. We say that $U$ has \dfn{locally finite perimeter} if $1_U \in BV_\loc(M)$.
In that case we make the following definitions:
\begin{enumerate}
\item The \dfn{measure-theoretic boundary} $\partial U$ is the set of points whose Lebesgue density with respect to $M$ is $\in (0, 1)$.
\item The polar section of $1_U$ is called the \dfn{conormal $1$-form} $\normal_U$ to $\partial U$.
\item The set of points $P$ for which $\normal_U(P)$ satisfies (\ref{Lebesgue point definition}) is the \dfn{reduced boundary} $\partial^* U$.
\item The \dfn{perimeter} $|\partial^* U \cap E|$ in a Borel set $E$ is $\|d1_U\|_{TV(E)}$.
\end{enumerate}
\end{definition}

Choosing a coordinate system on $M$ in which the volume form is $\dif x^0 \wedge \cdots \wedge \dif x^{d - 1}$ (so we may assume that $M$ is a domain in $\RR^d$), we see from \cite[Chapter 4]{Giusti77} that the following properties of the reduced boundary hold:

\begin{proposition}\label{locality of Caccioppoli}
    Let $U$ be a set of locally finite perimeter.
    Then:
    \begin{enumerate}
    \item $\partial^* U$ is either empty or $d-1$-dimensional in the Hausdorff sense, and is $d-1$-rectifiable.
    \item $\partial^* U$ is a dense subset of $\partial U$.
    \item If $\normal_U$ extends to a continuous $1$-form on $\partial U$, then $\partial^* U = \partial U$ is a $C^1$ embedded hypersurface.
    \item If $\partial^* U = \partial U$ is a $C^1$ hypersurface, then $\normal_U$ is the conormal $1$-form on $\partial U$ as defined in differential topology, and the total variation measure of $1_U$ is the surface measure on $\partial U$.
\end{enumerate}
\end{proposition}

% As a first application of Proposition \ref{locality of Caccioppoli} we recover the following formulation of the coarea formula.

\begin{proposition}[coarea formula]\label{Coarea2}
Let $u \in BV_\loc(M)$ and $E$ an open set. Then
\begin{equation}\label{coarea formula}
\|\dif u\|_{TV(E)} = \int_{-\infty}^\infty |E \cap \partial^* \{u > y\}| \dif y.
\end{equation}
\end{proposition}
\begin{proof}
Reasoning identically to \cite[Theorem 1.23]{Giusti77}, we may assume that $u \in C^\infty(M)$.
If this is true and also $u$ has no critical points, then (\ref{coarea formula}) follows from Fubini's theorem, the fact that $|E \cap \partial \{u > y\}|$ is the surface area of $E \cap \{u = y\}$ (by Proposition \ref{locality of Caccioppoli}), and the change-of-variables formula.
However the left-hand side of (\ref{coarea formula}) is unaffected by critical points of $u$, and the right-hand side of (\ref{coarea formula}) is unaffected by critical values of $u$ by Sard's theorem, so (\ref{coarea formula}) holds even if $u \in C^\infty(M)$ has critical points.
\end{proof}

We write
$$\eta(u, U) := \inf_{v|_{\partial U} = 0} \|\dif(u + v)\|_{TV(U)}$$
for $u \in BV_\loc(M)$ and $U \subseteq M$ open with Lipschitz boundary, thus $u$ has least gradient iff $\eta(u, M) = \|\dif u\|_{TV(M)}$.
If $u, v \in BV(U)$, then using the coarea formula and reasoning analogously to \cite[Lemma 5.6]{Giusti77}, we obtain the a priori estimates
\begin{align}
|\eta(u, U) - \eta(v, U)| &\leq \|u - v\|_{L^1(\partial U)} \label{a priori estimate 1} \\
\eta(u, U) &\leq \|u\|_{L^1(\partial U)} \leq |\partial U| \cdot \|u\|_{L^\infty(M)}. \label{a priori estimate 2}
\end{align}

If $X = M$ is a Riemannian manifold with its Riemannian measure $\mu$, then there exists a representative $\tilde E$ of the equivalence class $E$ such that $\partial^{\rm top} \tilde E = \partial^{\rm meas} E$ \cite[Theorem 3.1]{Giusti77}; it is straightforward to check that $\partial^{\rm meas} E$ and $\tilde E$ do not depend on the choice of Riemannian metric.
\emph{We adopt the convention throughout this paper that we are working with representatives $\tilde E$ such that $\partial^{\rm top} \tilde E = \partial^{\rm meas} E$, and simply write $\partial E$ for the boundary.}
Since $\partial E = \partial^{\rm top} \tilde E$, $\partial E$ is a closed set. 


%%%%%%%%%%%%%%%
\section{Minimal submanifolds}\label{Leaf estimates}
We recall some well-known estimates on minimal surfaces in normal coordinates.
Let $g$ be a metric on $\RR^{d - 1}_x \times \RR_y$ satisfying the normal coordinates condition $g - I = O(K_0(|x|^2 + y^2))$, and a curvature bound $\|\Riem_g\|_{C^0} \leq K_0$, where $I$ is the identity matrix.
For a function $u \in C^1(4\Ball^{d - 1})$, let
$$Pu(x) = F(x, u(x), \nabla u(x), \nabla^2 u(x)) = 0$$
be the minimal surface equation.
Then by \cite[(7.21)]{colding2011course}, the coefficient $F$ has the form
$$F(x, y, \xi, H) = \tr H + O(K_0(|x| + |y|) + |\xi|)(1 + |H|)$$
at least for $|x| + |y| + |\xi| \lesssim 1$.
Thus for $\|u\|_{C^1(4\Ball^{d - 1})} \lesssim 1$, the minimal surface equation is uniformly elliptic, so that by Schauder estimates \cite[Theorem 6.2]{gilbarg2015elliptic}, for any $r \geq 0$,
\begin{equation}\label{norms on uk}
	\|u\|_{C^r(3\Ball^{d - 1})} \lesssim_r 1.
\end{equation}

\begin{lemma}
Suppose that $u_2 \geq u_1$ satisfy $Pu_1 = Pu_2 = 0$ on $4\Ball^{d - 1}$ and $v := u_2 - u_1$.
Then for $K_0 \ll 1$, $\|u_i\|_{C^1} \lesssim 1$, 
\begin{equation}\label{Schauder Harnack}
	\|\dif v\|_{C^0(\Ball^{d - 1})} \lesssim \sup_{2\Ball^{d - 1}} v \lesssim \inf_{\Ball^{d - 1}} v.
\end{equation}
\end{lemma}
\begin{proof}
By the proof of \cite[Theorem 7.3]{colding2011course}, there exists a linear partial differential operator $Q$ such that $Qv = 0$, and if $K_0$ is small enough and $u_1, u_2$ are bounded in $C^1$, then on $3\Ball^{d - 1}$, $Q$ is uniformly elliptic, and the coefficients are bounded in $C^1$.
The claim now follows from Schauder estimates and the Harnack inequality \cite[Corollary 9.25]{gilbarg2015elliptic}.
\end{proof}

The regularity of codimension-$1$ hypersurfaces is well-known.
For the reader's convenience, we record here a short argument which reduces the regularity theorem to Nash's embedding theorem and a weak form of Allard's $\varepsilon$-regularity theorem on $\RR^d$ which is easily established in lecture notes of de Lellis \cite[Theorem 3.2]{DeLellis18}.

\begin{theorem}\label{regularity}
Let $d \leq 7$. Let $E$ be a set of locally finite perimeter in the Riemannian manifold $M$, such that $1_E$ has least gradient.
Then $\partial E$ is a smooth area-minimizing minimal hypersurface.
\end{theorem}

To set up the proof, recall that the mean curvature $H_N$ of a rectifiable set $N \subset \RR^D$ satisfies, for every $C^\infty_\cpt(\RR^D)$ vector field $X$,
$$\delta N(X) = -\int_N X \cdot H \dif \mu_N$$
where $\delta N(X)$ is the first variation of area in the direction $X$, and $\dif \mu_N$ is the surface measure on $N$.
By \cite[Proposition 1.5]{DeLellis18}, 
$$\delta N(X) = \int_N \Div_{TN} X \dif \mu_N$$
where, given a measurable subbundle $E$ of the tangent bundle $T\RR^D$ with an orthonormal frame $v_1, \dots, v_\delta$, the divergence is 
$$\Div_E X := \sum_{i = 1}^\delta \langle v_i, \nabla X\rangle.$$

\begin{theorem}[Allard's $\varepsilon$-regularity theorem]
For any integers $1 \leq \delta < D$ there exist $\varepsilon, \gamma, \alpha > 0$ such that the following holds:

Let $N \subset \RR^D$ be a $\delta$-rectifiable set, $x \in N$, and $0 < r < 1$.
Suppose that
\begin{align}
\mu_N(B(x, r)) &\leq (|\Ball^\delta| + \varepsilon) r^\delta, \label{measure excess}\\
\|H_N\|_{L^\infty(N \cap B(x, r))} &\leq \varepsilon/r. \label{mean curvature excess}
\end{align}
Then $N \cap B(x, \gamma r)$ is a $C^{1 + \alpha}$ submanifold of $B(x, \gamma r)$ without boundary.
\end{theorem}

\begin{proof}[Proof of Theorem \ref{regularity}]
By the Nash embedding theorem, we may assume that $M$ is embedded in $\RR^D$ for some $D \geq d + 1$.
Let $N := \partial^* E$. 
By \cite[Theorem 4.11]{Giusti77} and the diffeomorphism invariance of $\partial^*$, $N$ is a $d - 1$-rectifiable subset of $\RR^D$.

Let $x \in N$. By a blowup argument similar to the euclidean case treated in \cite[Chapters 9-10]{Giusti77}, the appropriate rescaling of $N \cap B(x, r)$ can be approximated in the weak topology of measures by $\Ball^{d - 1}$.
% For $0 < r \ll 1$, the dilation $F_r$ by $r^{-1}$ of $(\exp_x)^{-1}(N \cap B(x, r))$ is an approximately mass-minimizing $d - 1$-rectifiable subset of the unit ball in $T_x M$ (since on small scales the exponential map is close to an isometry).
% By a suitable modification of \cite[Chapters 9-10]{Giusti77}, $d \leq 7$ implies that $[F_r]$ converges in the weak topology of measures on $T_x M$ to a hyperplane $T_x N$.
In particular, $r^{1 - d} \mu_N(B(x, r))$ is approximated arbitrarily well by $|\Ball^{d - 1}|$ as $r \to 0$, so for $r$ small, (\ref{measure excess}) holds.

Let $X$ be a $C^\infty_\cpt(B(x, r))$ vector field on $\RR^D$.
If $X$ is tangent to $M$, then $\delta N(X) = 0$ by minimality of $N$.
So we assume that $X$ is normal to $M$.
By the Gauss-Codazzi theorem, $\nabla_M X$ is given by a contraction of the second fundamental form $\Two_M$ with $X$.
Moreover, $\nabla_N X$ is a minor of $\nabla_M X$, so $\Div_{TN} X$ is a partial trace of $\Two_M \otimes X$.
In particular,
$$\left|\int_N \Div_{TN} X \dif \mu_N\right| \lesssim_{D, d} \int_N |\Two_M| |X| \dif \mu_N \leq \|\Two_M\|_{C^0} \|X\|_{L^1(N)}.$$
Therefore 
$$\|H_{N \to \RR^D}\|_{L^\infty} \lesssim_{D, d} \|\Two_M\|_{C^0}.$$
So for $r$ small, (\ref{mean curvature excess}) holds.

So by Allard's theorem, $N \cap B(x, \gamma r)$ is $C^{1 + \alpha}$ and a closed set, implying that the normal vector extends continuously to $\partial E \cap B(x, \gamma r)$.
By elliptic bootstrapping, it follows that $\partial E \cap B(x, \gamma r)$ is $C^\infty$.
\end{proof}

\begin{proposition}\label{minimal implies locally minimizing}
Let $d \leq 7$, let $i, A > 0$, and let $g$ be a Riemannian metric on $M$ with $\|g\|_{C^{2 + \alpha}} < \infty$ and injectivity radius $\geq i$.
Then there exists $\delta_* > 0$ which only depends on $\|g\|_{C^{2 + \alpha}}, i, A$, such that the following holds:

For every ball $B(p, \delta) \subset M$ with $\delta \leq \delta_*$ and every oriented minimal hypersurface $N \subset B(p, \delta)$ with $p \in N$, $\partial N \subset \partial B(p, \delta)$, $\|\Two_N\|_{C^0} \leq A$, and trivial normal bundle, $N$ is the unique absolute area-minimizer among all hypersurfaces with boundary $\partial N$.
\end{proposition}

This proposition (or at least, a qualitative version of it) seems to be well-known, but we are not aware of a suitable reference.
We give a version of the argument sketched by Otis Chodosh in a MathOverflow post \cite{MathOverflowMinimalLocal}.

\begin{lemma}\label{existence of absolute minimizers}
Let $d \leq 7$, let $U \subseteq M$ be a ball of finite radius and $H^1(U, \RR) = 0$, and let $N \subset M$ be a hypersurface of trivial normal bundle and $\Two_N$ bounded.
Then there exists a smooth hypersurface $N'$ with $N \cap \partial U = N' \cap \partial U$, which is absolutely area-minimizing.
\end{lemma}
\begin{proof}
Since $N$ has trivial normal bundle and $H^1(U, \RR) = 0$, integration along $N$ defines an exact $d - 1$-current $\dif u$, and we can normalize $u$ to be $0$ somewhere.
Since $\Two_N$ is bounded and $U$ has finite radius, the surface area of $N$ is finite, so $u \in BV(U)$, hence $u \in L^1(\partial U)$.
So by an easy modification of \cite[Theorem 1.20]{Giusti77} there exists a $\{0, 1\}$-valued function $u'$ of least gradient with $u'|_{\partial U} = u|_{\partial U}$, and since $d \leq 7$ it follows from Theorem \ref{main thm of old paper} that $\dif u'$ is the current associated to an absolutely area-minimizing smooth hypersurface $N'$ which is a competitor to $N$.
\end{proof}

\begin{lemma}
Let $Q_n$ be a sequence of second order linear elliptic operators on $\Ball^d$, whose coefficients (in nondivergence form) converge in $C^0$ to those of a second order linear elliptic operator $Q$.
Let $\lambda_1^{(n)}, \lambda_1 > 0$ be the first Dirichlet eigenvalues of $Q_n, Q$.
Then 
\begin{equation}\label{spectral gap}
\liminf_{n \to \infty} \lambda_1^{(n)} \geq \lambda_1.
\end{equation}
\end{lemma}
\begin{proof}
Let $\varphi$ be a test function with $\|\varphi\|_{L^2} = 1$, and let $u_n$ be the first Dirichlet eigenfunction of $Q_n$ with $\|u_n\|_{L^2} = 1$.
By the Rayleigh-Ritz formula, \cite[Theorem 8.13]{gilbarg2015elliptic}, and the boundedness of the coefficients of $Q_n$,
$$\|u_n\|_{W^{3, 2}} \lesssim \lambda_1^{(n)} \leq \langle \varphi, Q_n \varphi\rangle_{L^2} \lesssim 1.$$
In particular, $u_n$ converges strongly in $W^{2, 2}$ to a function $u$ along a subsequence.
Now by the Rayleigh-Ritz formula,
\begin{align*}
\lambda_1 &\leq \langle u, Qu\rangle \\
&= \langle u - u_n, Qu\rangle + \langle u, (Q - Q_n) u\rangle + \langle u_n, Q_n(u - u_n)\rangle + \langle u_n, Q_n u_n\rangle \\
&= o(1) + o(1) + o(1) + \lambda_1^{(n)}. \qedhere
\end{align*}
\end{proof}

Given a hypersurface $\Sigma \subset \Ball^d$ and a metric $h$ on $\Ball^d$, let
\begin{equation}\label{stability formula}
Q_{\Sigma, h} = -\Delta_{\Sigma, h} - |\Two_{\Sigma, h}|^2 - \Ric_h(\normal_{\Sigma, h}, \normal_{\Sigma, h})
\end{equation}
be the stability operator for $\Sigma$, and let $H_{\Sigma, h}$ be the mean curvature.
Then the second variation of area in the direction of a normal variation $s$ is \cite[Chapter 1, \S8.1]{colding2011course}
\begin{equation}\label{second variation formula}
\delta^2_s |\Sigma|_h = \langle s, Q_{\Sigma, h} s\rangle_{L^2(\Sigma, h)} + \|sH_{\Sigma, h}\|_{L^2(\Sigma, h)}^2.
\end{equation}

\begin{lemma}\label{uniform continuity of stability}
Suppose that $s \in W^{1, 2}(\Ball^{d - 1})$ satisfy $s|_{\partial \Ball^{d - 1}} = 0$.
Let $\mathscr N$ be the set of graphs in $\Ball^d$ of functions lying in some compact subset of $C^2(\Ball^{d - 1})$ (so $s$ induces a normal variation on each member of $\mathscr N$).
Let $\mathscr G$ be a compact set of Riemannian metrics on $\Ball^d$ for the $C^2$ topology.
Then
\begin{align*}
\mathscr N \times \mathscr G &\to \RR, \\
(\Sigma, h) &\mapsto \delta^2_s |\Sigma|_h
\end{align*}
is uniformly continuous on $\mathscr N$, with modulus of continuity only depending on $\|s\|_{W^{1, 2}}, \mathscr N, \mathscr G$.
\end{lemma}
\begin{proof}
From (\ref{second variation formula}) and the fact that $s \mapsto \langle s, Q_{\Sigma, h} s\rangle_{L^2(\Sigma, h)}$ is a continuous quadratic form on $W^{1, 2}$, it will suffice to show that
\begin{equation}\label{stability curvature map}
(\Sigma, h) \mapsto (Q_{\Sigma, h}, H_{\Sigma, h})
\end{equation}
is uniformly equicontinuous, where the topology on the space of differential operators $Q$ is the $C^0$ topology on their nondivergence form coefficients.
Moreover, since $\mathscr N \times \mathscr G$ is compact, it suffices to show that (\ref{stability curvature map}) is continuous.
This is clear from (\ref{stability formula}) and the fact that $H_{\Sigma, h}$ only depends on second derivatives of $(\Sigma, h)$.
\end{proof}

\begin{proof}[Proof of Proposition \ref{minimal implies locally minimizing}]
We proceed by compactness and contradiction.
Let $(g_j)$ be a sequence of metrics bounded in $C^{2 + \alpha}(\Ball^d)$, which are increasingly severe violations of the proposition, in the sense that for some sequence $\delta_j \to 0$, there exist oriented $g_j$-minimal hypersurfaces $N_j \subset B_j := B_{g_j}(0, \delta_j)$ which contain $0$ and are not uniquely area-minimizing, but such that $\|\Two_{N_j}\|_{C^0(B_j)} \leq A$.
Since $(g_j)$ is bounded in $C^{2 + \alpha}$, $(g_j)$ is contained in a compact subset of $C^2$ and their curvatures are bounded in $C^0$.

% and after rescaling and shrinking $\delta_j$ if necessary, we may assume that $\|\Riem_{g_j}\|_{C^0} \leq K_0$ as in \S\ref{Regularity}.
% After applying isometries as necessary, we may assume that $g_j$ is written in normal coordinates centered on $0$, hence $|g_j - I| \lesssim K_0 |x|^2$.
% By compactness of the forgetful map $C^{2 + \alpha} \to C^2$, we may further assume that $(g_j)$ converges in $C^2$ to a $2$-tensor $g$ on $\Ball^d$ such that $|g - I| \lesssim K_0 |x|^2$.
% Taking $K_0$ small enough, it follows that $g$ is positive-definite and hence is a Riemannian metric.

By Lemma \ref{existence of absolute minimizers}, there exist $g_j$-absolutely minimizing competitors $N_j'$ to $N_j$ in $B_j$.
Let $L_j, L_j' \subset \Ball^d$ be the rescalings of $N_j, N_j'$ by $\delta_j^{-1}$.
Since $A < \infty$, $(\Riem_{g_j})$ is bounded in $C^0$, and $\delta_j \to 0$, $L_j$ is an oriented hypersurface in $\Ball^d$ such that $0 \in L_j$ and $\Two_{L_j} \to 0$ uniformly, where $\Two_{L_j}$ is taken in the flat metric on $\Ball^d$.
By Lemma \ref{existence of tubes}, for $j$ large enough, $L_j$ is a graph, possibly after rotation, of some function $f_j$ such that $\|f_j\|_{C^2} \to 0$.
Let $L$ be the graph of $0$, so $L_j \to L$ as $C^2$ graphs.
Since $L$ is a disk, $L$ is the unique absolutely minimizer among its competitors for the flat metric.

Let $\eta_j$ be the area of an absolutely minimizing competitor of $L_j$ (for the flat metric).
Since $N_j'$ is absolutely minimizing for $g_j$, and $L_j \to L$ in $C^0$, the normal coordinates condition on $g_j$ implies that
$$|L| - o(1) \leq |L_j'| \leq \eta_j + o(1) \leq |L| + o(1)$$
where areas are taken in the flat metric.
So along a subsequence, by Miranda compactness (Proposition \ref{MirandaStability}), the current defined by $L_j'$ converges in the weak topology of measures to the current of an absolutely minimizing competitor $L'$ of $L$.
By uniqueness of $L$, $L = L'$; moreover, $|L_j'| \to |L|$, so by (\ref{supports shrink in the limit}), $L_j' \to L$ in the Hausdorff topology.
Therefore, by Lemma \ref{convergence of normals}, any sequence of points $p_j \in L_j'$ converges to some $p \in L$, with $\normal_{L_j'}(p_j) \to \normal_L(p)$.
So by Lemma \ref{existence of tubes}, near $p$, $L_j'$ is a graph over $L$, of a function $f_j'$ such that $\|f_j'\|_{C^1} \to 0$.

The size of the ball around $p$ in which $L_j'$ is a graph is bounded from below by a quantity depending on $\|f_j'\|_{C^2}$.
But by (\ref{norms on uk}), $\|f_j'\|_{C^2} \to 0$.
So by a bootstrapping argument and Lemma \ref{existence of tubes}, $L_j'$ is globally a graph over $L_j'$ for $j$ large enough.
Since $L_j$ is close to $L$ in $C^2$, if $j$ is large, then $L_j'$ is a graph over $L_j$.

Thus, for $j \gg 1$, we may view $L_j'$ as a normal variation of $L_j$ in the rescaled metric $g_j' := \delta_j^{1/2} g_j$, associated to a function $s_j$ on $L_j$.
But $\|f_j - f_j'\|_{C^2} \to 0$, so $s_j \to 0$ in $C^2 \subset W^{1, 2}$.
Let $Q_j$ be the stability operator $Q_{L_j, g_j'}$, and observe that since $L_j, L_j'$ are converging to $L$ in $C^2$, they lie in a compact subset of $C^2$. 
Moreover, $L_j$ has zero $g_j'$-mean curvature, so the $g_j'$-area of $L_j'$ satisfies
\begin{equation}\label{stability area bound}
|L_j'|_{g_j'} \geq |L_j|_{g_j'} + \langle s_j, Q_j s_j\rangle_{L^2(L_j, g_j')} - o(\|s_j\|_{W^{1, 2}(L_j, g_j')}^2)
\end{equation}
where the rate of convergence in the error term is independent of $j$ by Lemma \ref{uniform continuity of stability}.
By (\ref{norms on uk}), $\|s_j\|_{W^{1, 2}} \sim \|s_j\|_{L^2}$, which we use to replace the error term in (\ref{stability area bound}).

Let $Q = -\Delta$ be the stability operator of $L$, so that the nondivergence form coefficients of $Q_j$ converge in $C^0$ to coefficients of $Q$.
Let $\lambda_1 > 0$ be the first Dirichlet eigenvalue of $Q$.
By (\ref{spectral gap}), if $j$ is large enough, then $Q_j$ has first Dirichlet eigenvalue $\geq \lambda_1/2$.
Since $L_j'$ is an absolute minimizer, we obtain from (\ref{stability area bound}) that for $j$ large enough,
$$|L_j'|_{g_j'} \geq |L_j|_{g_j'} + \frac{\lambda_1}{2} \|s_j\|_{L^2}^2 - o(\|s_j\|_{L^2}^2) \geq |L_j'|_{g_j'} + \frac{\lambda_1}{3} \|s_j\|_{L^2}^2.$$
Rearranging terms and exploiting $\lambda_1 > 0$, we see that $s_j = 0$.
Therefore $L_j = L_j'$ is both an absolute minimizer and not an absolute minimizer, a contradiction.
\end{proof}

\section{Functions of least gradient}
\begin{definition}
Let $u \in BV_\loc(M)$.
\begin{enumerate}
\item $u$ has \dfn{least gradient} in $M$ if for every $v \in BV_\cpt(M)$,
$$\int_{\supp v} \star |\dif u| \leq \int_{\supp v} \star |\dif (u + v)|.$$
\item $u$ has \dfn{locally least gradient} if there exists a cover $\mathcal U$ of $M$ by open sets with smooth boundary such that for any $U \in \mathcal U$, $u|_U$ has least gradient.
\end{enumerate}
\end{definition}

Functions of (locally) least gradient are weak solutions, in a suitable sense, of the \dfn{$1$-Laplace equation}
\begin{equation}\label{1Laplacian}
	\nabla \cdot \left(\frac{\nabla u}{|\nabla u|}\right) = 0
\end{equation}
and so we also call functions of locally least gradient \dfn{$1$-harmonic functions} \cite{Mazon14}.
Observe that formally, (\ref{1Laplacian}) implies that the level sets of $u$ are minimal hypersurfaces.

The definition of ``least gradient'', even for functions on noncompact domains, is standard \cite{Miranda67}, but the definition of ``locally least gradient'' is not.
Our main theorem is local, however, so having a local condition is more natural for our purposes.
As the following examples show, ``least gradient'' is not a local condition.
\begin{enumerate}
\item Let $M = \Ball^3$ and $u$ be the indicator function of the region bounded by a catenoid $C$ which meets $\partial \Ball^3$ on two circles $\partial D_1, \partial D_2$, which bound disks $D_i$ of the same area $A$.
Then $\star |\dif u|$ is the surface measure on $C$ by the coarea formula.
Since $C$ is a minimal surface, $u$ has locally least gradient.
But we may choose $C$ so that $2A < |C|$.
Then $u$ competes with\footnote{If $M$ is a compact manifold with boundary, then we can replace the competition class $BV_\cpt$ with ``$BV$ with zero trace'' in the definition of functions of least gradient \cite[\S9]{Korte19}.} the indicator function $v$ of the region bounded by $D_2 - D_1$, and $\int_M \star |\dif v| = 2A$, so $u$ does not have least gradient.\footnote{The case that $C$ is a critical catenoid, thus $|C| = 2A$, is a useful counterexample in its own right \cite[Example 4.3]{górny2018}.}
\item Let $M = \Sph^2$ and $u$ be the indicator function of a hemisphere, so $\star |\dif u|$ is the length measure of a geodesic and hence $u$ has locally least gradient.
But there are no functions of least gradient on $\Sph^2$ except constants, since $\Sph^2$ is a closed manifold and so any function must compete with the constant functions.
\end{enumerate}



\begin{theorem}\label{main thm of old paper}
Let $u \in BV(M)$ have least gradient in $M$ and $y \in \RR$. Then $1_{\{u > y\}}$ has least gradient in $M$.
In particular, if $d \leq 7$, then $\partial \{u > y\}$ is the sum of complete disjoint embedded oriented minimal hypersurfaces.
\end{theorem}
\begin{proof}
Let $v := 1_{\{u > y\}}$.
Then $v$ has least gradient \cite[Theorem 1]{BOMBIERI1969}, so by Theorem \ref{regularity}, $\{u > y\}$ is bounded by disjoint embedded minimal hypersurfaces.
These hypersurfaces inherit an orientation from the current $\dif v$.
\end{proof}

\begin{proposition}[{\cite[Osservazione 3]{Miranda67}}]\label{MirandaStability}
  Suppose that $M$ is compact with boundary.
	If a sequence of functions $(u_n)$ (not necessarily of the same trace) is bounded in $L^1(M)$ and satisfies
\begin{equation}\label{boundedness in Miranda}
	\limsup_{n \to \infty} \int_M \star |\dif u_n| \leq \liminf_{n \to \infty} \inf_{v|_{\partial M} = 0} \int_M \star |\dif(u_n + v)| < \infty,
\end{equation}
	then there exists a function $u$ of least gradient such that along a subsequence, $u_n \to u$ in $L^1(M)$ and $\dif u_n \to \dif u$ in the weak topology of measures.
\end{proposition}

If $M$ is compact with boundary, and $h \in L^1(\partial M)$ is Dirichlet data, we introduce the \dfn{relaxed functional} for the Dirichlet problem,
$$\Phi_h(v) := \int_M \star |\dif v| + \int_{\partial M} |v - h| \dif \mathcal H^{d - 1}.$$
A \dfn{calibrating vector field} is a vector field $X$ such that $\|X\|_{L^\infty} \leq 1$ and $\nabla \cdot X = 0$.

\begin{theorem}[{\cite[\S2]{Mazon14}}]\label{relaxed formulation}
Let $M$ be compact with boundary, $u \in BV(M)$, and $h := u|_{\partial M}$.
The following are equivalent:
\begin{enumerate}
\item $u$ has least gradient.
\item $u$ minimizes $\Phi_h(v)$ among all $v \in BV(M)$.
\item $(\dif u)^\sharp/|\dif u|$ extends off of $\supp \dif u$ to a calibrating vector field $X$.
\end{enumerate}
\end{theorem}

\section{Laminations}\label{RS prelims}
Fix an interval $I \subset \RR$, a box $J \subset \RR^{d - 1}$, and a smooth Riemannian manifold $M = (M, g)$ of dimension $d \geq 2$.

\begin{definition}
A (codimension-$1$) \dfn{laminar flow box} is a $C^0$ coordinate chart $F: I \times J \to M$ and a compact set $K \subseteq I$, such that for each $k \in K$, $F|_{\{k\} \times J}$ is a $C^1$ embedding, and the \dfn{leaf} $F(\{k\} \times J)$ is a $C^1$ complete hypersurface in $F(I \times J)$.

Two laminar flow boxes $(F_\alpha, K_\alpha)$ and $(F_\beta, K_\beta)$ belong to the same \dfn{laminar atlas} if the transition map $\psi_{\alpha \beta}$ between $F_\alpha$ and $F_\beta$ maps each leaf $\{k\} \times J$, $k \in K_\alpha$, to a leaf $\{\psi_{\alpha \beta}(k)\} \times J$, so that $\psi_{\alpha \beta}$ is a homeomorphism $K_\alpha \to K_\beta$.
\end{definition}

\begin{definition}
A \dfn{lamination} $\lambda$ consists of a nonempty closed set $S \subseteq M$, called its \dfn{support}, and a maximal laminar atlas $\{(F_\alpha, K_\alpha): \alpha \in A\}$ such that in the image $U_\alpha$ of each flow box $F_\alpha$,
$$S \cap U_\alpha = F_\alpha(K_\alpha \times J).$$
If $\lambda$ is a lamination in the image of a flow box $F$, and $N := F(\{k\} \times J)$ is a leaf of $\lambda$, we call $k$ the \dfn{label} of $N$.
A \dfn{foliation} is a lamination with support $S = M$.
\end{definition}

Summarizing the above definitions, a lamination is a nonempty closed set $S$ with a $C^0$ local product structure which locally realizes it as $K \times J$ for some compact set $K \subset \RR$.
In some sources, including \cite{Auer01}, laminations are not required to have a $C^0$ local product structure, but are only required to have disjoint leaves.

\begin{definition}
We call a lamination $C^r$ (resp. \dfn{Lipschitz}) if its flow boxes are $C^r$ (resp. Lipschitz) coordinate charts, and say that it is \dfn{tangentially $C^r$} if for each flow box $(F, K)$, $F|_{\{k\} \times J}$ is a $C^r$ embedding for $k \in K$.\footnote{Such laminations are also known as $C^r$ \dfn{along leaves} \cite{Morgan88}.}
\end{definition}

In particular, we assume that laminations are $C^0$ and tangentially $C^1$; the latter assertion implies that the flow box can push forward the normal vector to each leaf, and in particular that the mean curvature to each leaf is well-defined in a distributional sense.
The Lipschitz regularity is particularly natural in light of the previous results of \cite{Solomon86, Zeghib04}.

In this paper we shall focus on laminations with minimal leaves\footnote{The word ``minimal'' is overloaded. In \cite{daskalopoulos2020transverse}, a \dfn{minimal lamination} is a lamination $\lambda$ in which every leaf is dense in $\supp \lambda$.
We adopt the terminology of \cite{ColdingMinicozziIV}.} and transverse measures.

\begin{definition}
A lamination $\lambda$ is \dfn{minimal} if its leaves $F_\alpha(\{k\} \times J)$ have zero mean curvature, and is \dfn{geodesic} if, in addition, $d = 2$.
\end{definition}

\begin{definition}
Let $\lambda$ be a lamination with atlas $A$.
A \dfn{transverse measure} to $\lambda$ consists of Radon measures $\mu_\alpha$ with $\supp \mu_\alpha = K_\alpha$, $\alpha \in A$, such that each transition map $\psi_{\alpha \beta}$ is measure-preserving:
$$\mu_\alpha|_{K_\alpha \cap K_\beta} = \psi_{\alpha \beta}^* (\mu_\beta|_{K_\alpha \cap K_\beta}).$$
The pair $(\lambda, \mu)$ is called a \dfn{measured lamination}.
\end{definition}

We assume that every transverse measure has full support, $\supp \mu_\alpha = K_\alpha$.
Therefore not every lamination $\lambda$ admits a transverse measure; for example, this happens if $\lambda$ has an isolated leaf which meets a compact transverse curve on a countable but infinite set \cite[Theorem 3.2]{Morgan88}.

\begin{definition}
Let $(\lambda, \mu)$ be a measured oriented lamination, with atlas $A$ and a subordinate partition of unity $(\chi_\alpha)$.
The \dfn{Ruelle-Sullivan current} $T_\mu$ associated to $(\lambda, \mu)$ is defined for all compactly supported $d-1$-forms $\varphi$ by
\begin{equation}\label{RS current}
\int_M T_\mu \wedge \varphi := \sum_{\alpha \in A} \int_{K_\alpha} \left[\int_{\{k\} \times J} (F_\alpha^{-1})^* (\chi_\alpha \varphi) \right] \dif \mu_\alpha(k).
\end{equation}
\end{definition}

Let $(\lambda, \mu)$ be a measured oriented lamination.
Then the Ruelle-Sullivan current $T_\mu$ is a well-defined closed $d-1$-current \cite[Theorem 8.2]{daskalopoulos2020transverse}. 
In particular, we may lift $T_\mu$ to the universal cover $\tilde M$, where it is exact \cite[Theorem 8.3]{daskalopoulos2020transverse}.
Moreover, $T_\mu$ has an intrinsic definition as the unique $d-1$-current with a certain polar decomposition.
To be more precise, recall that $\mu$ defines a measure on $\supp \lambda$: in each flow box $F_\alpha$, an open set $U$ has measure
\begin{equation}\label{transverse measure of an open set}
\mu(U) := \int_{K_\alpha} |F_\alpha(\{k\} \times J) \cap U| \dif \mu_\alpha(k).
\end{equation}

\begin{lemma}
For a measured oriented lamination $(\lambda, \mu)$, with Lipschitz normal vector $\normal_\lambda$, the polar decomposition of $T_\mu$ is
\begin{equation}\label{polar ruelle sullivan}
T_\mu = \normal_\lambda \mu.
\end{equation}
\end{lemma}
\begin{proof}
For an open set $U \subseteq M$ in a flow box $F_\alpha$, the total variation satisfies
$$\int_U \star |T_\mu| = \sup_{\|\varphi\|_{C^0} \leq 1} \int_{K_\alpha} \int_{\{k\} \times J} \varphi \dif \mu_\alpha(k)$$
where the supremum ranges over $d-1$-forms $\varphi$ with compact support in $U$.
However, $\star \normal_\lambda^\flat$ is the Riemannian measure on $F_\alpha(\{k\} \times J)$, so
$$\int_{\{k\} \times J} \varphi \leq \int_{\{k\} \times J} (F_\alpha^{-1})^*(\star \normal_\lambda^\flat).$$
Since $\|\normal^\lambda\|_{C^0} = 1$, it follows that a sequence of cutoffs of $\star \normal_\lambda^\flat$ to more and more of $U$ is a maximizing sequence.
Therefore $\normal_\lambda$ is the polar part of (\ref{polar ruelle sullivan}), and
$$\int_U \star |T_\mu| = \int_{K_\alpha} \int_{\{k\} \times J} (F_\alpha^{-1})^*(1_U \star \normal_\lambda^\flat) \dif \mu_\alpha(k).$$
The inner integral is the Riemannian measure of $F_\alpha(\{k\} \times J) \cap U$, so by (\ref{transverse measure of an open set}), $|T_\mu| = \mu$.
\end{proof}

The above computation motivates the definition of Ruelle-Sullivan current of a \emph{nonorientable} lamination.
To be more precise, if $\lambda$ is a nonorientable lamination with normal vector field $\normal_\lambda$, then we can view $\normal_\lambda$ as a section of a line bundle $L$ over $M$ of structure group $\ZZ/2$.
We can then define $T_\mu$ to be $\normal_\lambda \mu$, which makes sense as a distributional section of $L$, and can be tested against any continuous $d-1$-form on $M$ whose support is contained in a trivializing chart of $L$.
In particular, we shall speak of the Ruelle-Sullivan current of any measured lamination, even if it is nonorientable.

\begin{lemma}\label{convergence of normals}
If $(\lambda_n, \mu_n) \to (\lambda, \mu)$, $x_n \in \supp \lambda_n$ converges to $x \in \supp \lambda$, and $(\lambda_n), \lambda$ have continuous normal vector fields $(\normal_n), \normal$, then $\normal_n(x_n) \to \normal(x)$ pointwise.
\end{lemma}
\begin{proof}
	Choose a continuous $d-1$-form $\varphi$ which extends $\star \normal^\flat$.
	Then for every $\varepsilon > 0$,
	$$\int_{B(x, \varepsilon)} T_\mu \wedge \varphi = \mu(B(x, \varepsilon))$$
	so by Proposition \ref{portmanteau}, for almost every $\varepsilon > 0$,
	\begin{equation}\label{epsilon is a continuity set}
		\lim_{n \to \infty} \frac{\int_{B(x, \varepsilon)} T_{\mu_n} \wedge \varphi}{\mu_n(B(x, \varepsilon))} = \frac{\int_{B(x, \varepsilon)} T_\mu \wedge \varphi}{\mu(B(x, \varepsilon))} = 1.
	\end{equation}
	On the other hand, if we assume that there exist $\delta, \varepsilon > 0$ and a coordinate system such that for every $y \in \supp \lambda_n \cap B(x, \varepsilon)$,
	$$|\normal_n - \normal| \geq \delta,$$
	then possibly after shrinking $\varepsilon$ we may assume that (\ref{epsilon is a continuity set}) holds, hence by (\ref{polar ruelle sullivan}),
	$$\int_{B(x, \varepsilon)} T_{\mu_n} \wedge \varphi = \int_{B(x, \varepsilon)} \normal_n^\flat \wedge \star \normal^\flat \dif \mu_i \leq (1 - O(\delta)) \mu_n(B(x, \varepsilon))$$
	and therefore $\delta = 0$, a contradiction.
\end{proof}



%%%%%%%%%%
\chapter{Daskalopolous--Uhlenbeck duality}\label{DaskUhlen}

%%%%%%%%%%%%%%%%%%%%%%%%
\chapter{Regularity of minimal perimeters in Riemannian manifolds} \label{deGiorgi}

\section{Introduction}
An open subset $U$ of a Riemannian manifold-with-boundary $M$ is said to have \dfn{least perimeter} if its indicator function $1_U$ has least gradient -- that is, for every $\varphi \in BV(M)$ with $\varphi|_{\partial M} = 0$, 
\begin{equation}\label{least perimeter dfn}
\|\dif 1_U\|_{TV} \leq \|\dif 1_U + \dif \varphi\|_{TV},
\end{equation}
where $\|\cdot\|_{TV}$ denotes the total variation norm of a Radon measure.
Functions of least gradient satisfy the following classical interior regularity theorem:

\begin{theorem}\label{main thm 2}
Let $M$ be a Riemannian manifold-with-boundary of dimension $\leq 7$, and let $U \subset M$ have least perimeter.
Then $U$ is bounded by stable minimal hypersurfaces, smooth on the interior.
\end{theorem}

Here, a minimal hypersurface $N$ is \dfn{stable} if for every normal variation $(N_t)$ which has compact support on the interior of $M$, $\partial_t^2|_{t = 0} |N_t| \geq 0$.
Note that this implies Theorem \ref{Regularity} without going through Allard's $\varepsilon$-regularity theorem.

The proof of Theorem \ref{main thm 2} due to Federer \cite{Federer70} can be modified in a straightforward manner to hold on Riemannian manifolds.
Another proof, based on Allard's $\varepsilon$-regularity theorem, can also be modified \cite{DeLellis18}.
However, Theorem \ref{main thm 2} also follows from the classical work of Miranda \cite{Miranda66} on functions of least gradient, if we take as given the nonexistence of Simons cones.
See Morgan's book \cite[Chapter 8]{morgan2016geometric} for an account of the history of the regularity problem in codimension $1$.

One expects a direct generalization of Miranda's argument, and that is our goal in this present paper.
Miranda's key insight was that a $\varepsilon$-regularity theorem based on the oscillation of the normal vector, known as \dfn{de Giorgi's lemma}, can be deduced for sets of least perimeter using a monotonicity formula for functions of least gradient.
This argument is specific to codimension $1$ and completely avoids the geometric measure theory of varifolds (or currents) but uses the older and simpler theory of $BV$ functions.

%%%%%%%%%%%%%%%%

\subsection{Idea of the proof}
For an exposition of Miranda's proof of Theorem \ref{main thm 2} in the flat case, we refer the reader to \cite[Chapters 5-9]{Giusti77} as well as Miranda's original work \cite{Miranda66}.
Consider the \dfn{excess} $\Exc_\Omega U$ -- that is, the oscillation of the conormal $1$-form $\normal_U$ to the reduced boundary $\partial^* U$ of $U$, measured in an open set $\Omega$.
By definition,
\begin{equation}\label{intro excess}
\Exc_\Omega U := |\partial^* U \cap \Omega| - \left|\int_\Omega \normal_U \dif \mu\right|
\end{equation}
where $\mu$ is the surface measure on $\partial^* U$ and the integral is a vector-valued integral.
By a mollification argument using a monotonicity formula for functions of least gradient, it suffices to estimate this quantity when $\partial^* U$ is the graph of a $C^1$ solution $u$ of the minimal surface equation\footnote{Strictly speaking, $u$ need only be an approximate solution of the minimal surface equation, since the relevant mollification operator does not commute with the minimal surface operator.}
\begin{equation}\label{euclidean MSE}
\nabla \cdot \frac{\nabla u}{\sqrt{1 + |\nabla u|^2}} = 0,
\end{equation}
such that $\nabla u$ is small in $C^0$. 
Linearizing (\ref{euclidean MSE}) around the trivial solution, we obtain the Laplace equation; from this, we deduce that we may approximate $u$ by a sum of orthogonal harmonic polynomials, which then implies the estimate 
\begin{equation}\label{intro DGL}
\Exc_{B(P, r/2)} U \leq 2^{-d} \Exc_{B(P, r)} U.
\end{equation}
The estimate (\ref{intro DGL}) is called the \dfn{de Giorgi lemma}, and by the aforementioned mollification argument, holds even if $\partial^* U$ is not assumed $C^1$.
It is easy to see that the de Giorgi lemma implies that $\partial^* U$ is actually $C^1$; the theorem then follows from (\ref{least perimeter dfn}).

While the above scheme roughly is also the plan of this paper, it does not quite work as stated above.
There are a handful of technicalities caused by the presence of curvature; here we just address the most important one.

The definition (\ref{intro excess}) assumes that $\normal_U$ is a map into a fixed vector space, and makes no sense if $\normal_U$ must be understood as a section of a vector \emph{bundle}; in other words, it makes no sense if $M$ has curvature.
One can rectify this issue by introducing the parallel propagator
$$K(P, Q): T_Q'M \to T_P'M,$$
which acts by parallel transport along a geodesic $\gamma$ from $Q$ to $P$ whenever $Q$ is contained in the cut locus of $P$.
Then, if $\Omega \ni P$ is contained in the cut locus of $P$, it is natural to generalize (\ref{intro excess}) to
\begin{equation}\label{intro excess 2}
\Exc_\Omega(U, P) := |\partial^* U \cap \Omega| - \left|\int_\Omega K(P, Q) \normal_U(Q) \dif \mu(Q)\right|.
\end{equation}
In order for (\ref{intro excess 2}) to be useful, however, it should be approximately independent of the basepoint $P$: mollification may move $\partial^* U$, so even if we choose a basepoint on $\partial^* U$, we may have to move it later if we wish to impose that $P \in \partial^* U$.

In order to prove that (\ref{intro excess 2}) is approximately translation-invariant, we must estimate $K(P, Q)$.
This is accomplished using the formula \cite[Chapter II, \S2]{baez1994gauge}
$$K(P, Q) = \mathcal Pe^{-\int_Q^P \Gamma}$$
where $\mathcal P$ is the path-ordering symbol, and $\Gamma$ is the Christoffel symbol of $M$, viewed as a matrix-valued $1$-form.
The Christoffel symbol $\Gamma$ is suitably small when measured in normal coordinates based at a point of $\Omega$.
This last point is cruical: the proof of the de Giorgi lemma does not allow us to use an arbitrary coordinate system on $M$, but only coordinate systems in which the linearization of the minimal surface equation around the trivial solution gives the Laplace equation.
However, we can choose normal coordinates in which this happens, and hence conclude the de Giorgi lemma.

% If $M$ is a hyperbolic manifold, then there is another way to define the excess, which was inspired by recent work of Daskalopoulos and Uhlenbeck on $\infty$-harmonic maps between hyperbolic surfaces \cite{daskalopoulosPrep1}.
% Working locally, we can assume that $M$ is an open subset of hyperbolic space $\Hyp^d$, which we then identify with the future unit hyperboloid in the Minkowski spacetime $\RR^{1, d}$.
% One then obtains an embedding $TM \to T\RR^{1, d}$, so we may replace $TM$ with the flat vector bundle $M \times \RR^{1, d}$, in which parallel transport is trivial.
% However, we shall not take this approach here.


%%%%%%%%%%%%%%%%%%%%%%%%
\section{Preliminaries}\label{Prelims}
\subsection{Notation and conventions}
We denote the Riemannian metric by $g$.
When using the Einstein convention, Greek indices range over $0, 1, \dots$ while Latin indices range over $1, \dots$.
We write $y := x^0$.

%%%%%%%%%%%%%%%%%%%%%%%%%%
\subsection{The parallel propagator}
The key new ingredient for the proof of the regularity theorem on Riemannian manifolds is the parallel propagator
$$K(P, Q): T_Q'M \to T_P'M$$
which is defined whenever there exists a unique geodesic $\gamma$ from $Q$ to $P$.
It sends a cotangent vector to $M$ at $Q$ to its parallel transport along $\gamma$, which is a cotangent vector to $M$ at $P$.

To express the parallel propagator in coordinates, we recall from \cite[Chapter II, \S2]{baez1994gauge} that if $\Gamma$ is a square matrix of $1$-forms, its \dfn{path-ordered exponential} along a curve $\gamma: [0, 1] \to M$ is given by 
$$\mathcal Pe^{-\int_\gamma \Gamma} := \sum_{n=0}^\infty (-1)^n \int_{\Delta_n} \prod_{m=1}^n (\Gamma(\gamma(t_i)), \gamma'(t_i)) \dif t$$
where 
$$\Delta_n := \{t \in [0, 1]^n: t_1 \leq t_2 \leq \cdots \leq t_n\}$$
is the standard $n$-simplex.
The path-ordered exponential is then a square matrix; it is defined by a convergent series if $\Gamma$ is continuous \cite[Chapter II, \S2]{baez1994gauge}, and to first order, the path-ordered exponential is
\begin{equation}\label{path ordered exponential taylor series}
\mathcal Pe^{-\int_\gamma \Gamma} = I + O(|\gamma| \cdot \|\Gamma\|_{C^0}).
\end{equation}

\begin{proposition}
Suppose that there is a unique geodesic $\gamma$ from $Q$ to $P$.
Let $\Gamma$ be the Christoffel symbols of the Levi-Civita connection of $M$ acting on the cotangent bundle in some coordinate system, viewed as a $d \times d$-matrix of $1$-forms.
Then
\begin{equation}\label{path ordered exponential is propagator}
K(P, Q) = \mathcal Pe^{-\int_\gamma \Gamma}.
\end{equation}
\end{proposition}

To be more precise, if we use our coordinate system to identify $T_P'M$ and $T_Q'M$ with $\RR^d$, then the $d\times d$-matrix obtained from the path-ordered exponential can be viewed as a linear map
$$\mathcal Pe^{-\int_\gamma \Gamma}: T_Q'M \to T_P'M.$$
The assertion is that this map is exactly equal to $K(P, Q)$.
For a proof, see \cite[Chapter II, \S2]{baez1994gauge}.

We now use the parallel propagator to define an intrinsic averaging operator on the cotangent bundle.
Such an operator will necessarily depend on the choice of basepoint, since there is no canonical way to identify all of the cotangent spaces to $M$.

\begin{definition}
Let $\mu$ be a Radon measure, $U \subseteq M$ open with finite $\mu$-measure, $P \in U$, and $\xi$ a $1$-form on $U$.
Suppose that for every $Q \in U$ there exists a unique geodesic from $Q$ to $P$.
Then the \dfn{average} of $\xi$ in $U$ with respect to $\mu, P$ is 
$$\avg_{U, P, \mu} \xi := \frac{1}{\mu(U)} \int_U K(P, Q)\xi(Q) \dif \mu(Q).$$
\end{definition}

To be more explicit, $K(P, Q)\xi(Q)$ is a cotangent vector to $P$, so we have a vector-valued map 
\begin{align*}
U &\to T_PM \\
Q &\mapsto K(P, Q)\xi(Q).
\end{align*}
This vector-valued map has target a single vector space, rather than a vector bundle, so it makes sense to take its vector-valued integral.

\begin{proposition}\label{translation invariance}
Let $\mu$ be a Radon measure, $U \subseteq M$ open with finite $\mu$-measure, $P, Q \in U$, and $\xi$ a $1$-form on $U$.
Suppose also that $U$ has diameter $\leq \rho$ and for any $R \in U$ there exist unique geodesics from $R$ to $P$ and $Q$.
Then 
$$||\avg_{U, P, \mu} \xi| - |\avg_{U, Q, \mu} \xi|| \lesssim \rho^2 \|\xi\|_{L^\infty(\mu)}.$$
\end{proposition}
\begin{proof}
Choose normal coordinates at $P$, so that $\|\Gamma\|_{C^0(U)} \lesssim \rho$.
Moreover, the geodesics from $R$ to $P, Q$ have length $\leq \rho$.
Therefore by the reverse triangle inequality,
\begin{align*}
||\avg_{U, P, \mu} \xi| - |\avg_{U, Q, \mu} \xi||
&= \frac{1}{\mu(U)} \left|\left|\int_U \mathcal P e^{-\int_R^P \Gamma} \xi(R) \dif \mu(R)\right| - \left|\int_U \mathcal P e^{-\int_R^Q \Gamma} \xi(R) \dif \mu(R)\right|\right| \\
&\leq \frac{\|\xi\|_{L^\infty(\mu)}}{\mu(U)} \int_U \left|\mathcal P e^{-\int_R^P \Gamma} - \mathcal Pe^{-\int_R^Q \Gamma}\right| \dif \mu(R) \\
&\leq \|\xi\|_{L^\infty(\mu)} \sup_{R \in U} \left|\mathcal Pe^{-\int_R^P \Gamma} - \mathcal Pe^{-\int_R^Q \Gamma}\right|.
\end{align*}
By (\ref{path ordered exponential taylor series}), this quantity is $\lesssim \rho^2 \|\xi\|_{L^\infty(\mu)}$.
\end{proof}

%%%%%%%%%%%%%%%%%%%%%%%%%%%%

\section{Monotonicity formula}\label{MollifierSection}
Functions of least gradient satisfy a monotonicity formula (see \cite[Theorem 5.12]{Giusti77} for a proof on euclidean space) which is essential to the proof of the de Giorgi--Miranda theorem in several ways.
Most prominently, the monotonicity formula implies the existence of tangent cones to sets of least perimeter (allowing us to apply the hypothesis $d \leq 7$), implies certain precise estimates on the perimeter of a set of least perimeter, and governs the mollification process used in the proof of the de Giorgi lemma.

To formulate the monotonicity formula, define for a function $u \in BV(M)$ and $P \in M$ the quantity
\begin{equation}\label{integral of du}
I(u, P, r) := \int_{B(P, r)} K(P, Q) \frac{\dif u}{|\dif u|}(Q) \dif \mu(Q)
\end{equation}
where $\dif u/|\dif u|$ is the polar section and $\dif \mu := \star|\dif u|$ is the total variation measure.
For the proof and corollaries of the monotonicity formula it will be useful to establish some notation: $\dif \sigma$ is the area form on the unit sphere $\Sph^{d - 1}$ and $B_r := B(P, r)$.
In normal polar coordinates $(r, \theta) \in \RR_+ \times \Sph^{d - 1}$, the area form on geodesic spheres takes the form 
\begin{equation}\label{geodesic spheres area form}
\dif S_{\partial B_\rho} = (\rho^{d - 1} + O(\rho^{d + 1})) \dif \sigma.
\end{equation}

\begin{proposition}[monotonicity formula]\label{Monotone}
There exist $A \geq 0$ and $r_* > 0$ such that for any function $u$ of least gradient on $M$, with total variation measure $\mu$, and any $0 < r \leq r_*$,
\begin{equation}\label{weak monotonicity}
\frac{\dif}{\dif r}\left[e^{Ar^2}r^{1 - d} \mu(B(P, r))\right] \geq 0.
\end{equation}
Moreover, for any $0 < r_1 < r_2 \leq r_*$,
\begin{align*}
&|r_2^{1 - d} I(u, P, r_2) - r_1^{1 - d} I(u, P, r_1)|^2 \\
&\qquad \lesssim \left(1 + (d - 1) \log \frac{r_2}{r_1}\right) \left(r_2^{1 - d}\mu(B(P, r_2)) \right)
\left(\int_{r_1}^{r_2} \partial_r \left[e^{Ar^2} r^{1 - d} \mu(B(P, r))\right] \dif r\right)\\
&\qquad \qquad + r_2^{6-2d} \left(\mu(B(P, r_2))\right)^2.
\end{align*}
\end{proposition}

\begin{lemma}\label{monotonicity lemma}
There exists $A$ such that for every $u \in C^1(B_R)$, $0 < r_1 < r_2 < R$, if we let
$$E(r) := \mu(B_r) - \eta(u, r),$$
so that $E(R) = 0$ iff $u$ has least gradient, then there exists $A \geq 0$ such that for $R > 0$ small,
\begin{equation}\label{monotonicity lemma eqn}
0 \leq \int_{B_{r_2} \setminus B_{r_1}} \star r^{1 - d}\frac{(\partial_ru)^2}{|\dif u|} \leq 2\int_{r_1}^{r_2} \partial_r \left[e^{Ar^2} r^{1-d}\int_{B_r} \star |\dif u|\right] + \frac{O(E(r))}{r^d} \dif r.
\end{equation}
\end{lemma}
\begin{proof}
We follow the proof of \cite[Lemma 5.8]{Giusti77}.
This result is coordinate-invariant, so we may use whichever coordinates are convenient: we in fact use normal polar coordinates $(r, \theta)$.
By (\ref{geodesic spheres area form}), there exists $A \geq 0$ such that $e^{A\rho^2} \sqrt{\det g|_{\partial B_\rho}}$ is increasing.

Fix $s \in [r_1, r_2]$ and introduce a competitor $v(r, \theta) = u(s, \theta)$.
From the definition of $\eta$,
$$\eta(u, s) \leq \int_{B_r} \star |\dif v| = \int_0^s \int_{\partial B_r} \star_r |\dif v| \dif r.$$
Since $\partial_r v = 0$,
$$\int_{\partial B_r} \star_r |\dif v| \leq e^{As^2} \frac{r^{d - 1}}{s^{d - 1}} \int_{\partial B_s} \star_s |\dif v|.$$
So by Fubini's theorem
\begin{align*}
\eta(u, s) &\leq e^{As^2} \int_0^s \frac{r^{d - 1}}{s^{d - 1}} \dif r \cdot \int_{\partial B_s} \star_s |\dif v| = \frac{s e^{As^2}}{d} \int_{\partial B_s} \star_s |\dif v|\\
&\leq \frac{s e^{As^2}}{d - 1} \int_{\partial B_s} \star_s |\dif v|.
\end{align*}
By Gauss' lemma, $\partial_r$ is orthogonal to $\partial B_s$, so $\dif v$ is the orthogonal projection of $\dif u$ onto $T' \partial B_s$, and its orthocomplement is $\partial_r u$. Therefore by Taylor's theorem,
$$\int_{\partial B_s} \star_s |\dif v| \leq \int_{\partial B_s} \star_s |\dif u| \sqrt{1 - \frac{(\partial_r u)^2}{|\dif u|^2}} \leq \int_{\partial B_s} \star_s \left[|\dif u| - \frac{(\partial_r u)^2}{2 |\dif u|}\right]$$
or in other words
\begin{align*}
\int_{\partial B_s} \star_s \frac{(\partial_r u)^2}{2|\dif u|} &\leq \int_{\partial B_s} \star_s |\dif u| - \frac{d - 1}{s} e^{-As^2} \eta(u, s)\\
&\leq \int_{\partial B_s} \star_s |\dif u| - \frac{d - 1}{s} e^{-As^2} \int_{B_s} \star |\dif u| - O(s^{-1}E(s)).
\end{align*}
We moreover have for $\tilde A \geq 0$ that
$$e^{-\tilde As^2} \partial_s \left[e^{\tilde As^2} s^{1 - d} \int_{B_s} \star |\dif u|\right] = \left[2\tilde As^{2 - d} - \frac{d - 1}{s^d}\right]\int_{B_s} \star |\dif u| + s^{1 - d} \int_{\partial B_s} \star_s |\dif u|$$
so if we choose $\tilde A$ so that
$$-\frac{d - 1}{s} e^{-As^2} = 2\tilde As - \frac{d - 1}{s}$$
then
$$s^{1 - d} \int_{\partial B_s} \star_s |\dif u| - (d - 1)\frac{e^{-\tilde As^2}}{s^d} \int_{B_s} \star|\dif u| \leq e^{-\tilde As^2} \partial_s\left(e^{\tilde As^2} s^{1 - d} \int_{B_s} \star|\dif u|\right).$$
We moreover have $e^{-\tilde As^2} \leq 1$, so we can now integrate with respect to $\dif s$ and rename $\tilde A$ to $A$ to conclude.
\end{proof}

\begin{proof}[Proof of Proposition \ref{Monotone}]
Working in normal coordinates $(x^\alpha)$ based at $P$, we first computing using (\ref{path ordered exponential taylor series})
$$I_r := r^{1 - d} I(u, P, r) = r^{1 - d} \left[\int_{B(P, r)} \star \partial_\alpha u\right] \dif x^\alpha(P) + O(r^{3 - d}) \|\dif u\|_{TV(B(P, r))}.$$
Let $(\iota_0, \dots, \iota_{d - 1}): \Sph^{d - 1} \to \RR^d$ be the embedding of the unit sphere.
Applying the fundamental theorem of calculus,
\begin{align*}
(I_r)_\alpha &= \left[\int_{\Sph^{d - 1}} u(r, \theta) \iota_\alpha(\theta) \dif \sigma(\theta)\right] + O(r^{3 - d}) \|\dif u\|_{TV(B(P, r))}
\end{align*}
where we identified $\partial B_r$, the domain of $u(r, \cdot)$, with $\Sph^{d - 1}$ using normal coordinates.
Therefore
\begin{equation}\label{monotone dump the metric}
|I_{r_2} - I_{r_1}| \leq \int_{\Sph^{d - 1}} |u(r_2, \theta) - u(r_1, \theta)| \dif \sigma(\theta) + O(r^{3 - d}) \|\dif u\|_{TV(B(P, r))}.
\end{equation}
The metric $g$ plays no role in the dominant term of (\ref{monotone dump the metric}), so we may use \cite[Lemma 5.3]{Giusti77} to bound
$$0 \leq \int_{\Sph^{d - 1}} |u(r_2, \theta) - u(r_1, \theta)| \dif \sigma(\theta) \leq \int_{\Sph^{d - 1}} \int_{r_1}^{r_2} r^{1 - d}|\partial_r u(r, \theta)| \dif r \dif\sigma(\theta).$$
To reintroduce the metric we posit that $r_2$ is small enough that $\dif r \dif \sigma(\theta) \leq \star 2$.
We therefore have
\begin{equation}\label{monotone before cs}
\int_{\Sph^{d - 1}} \int_{r_1}^{r_2} r^{1 - d}|\partial_r u(r, \theta)| \dif r \dif\sigma(\theta) \leq 2 \int_{B_{r_2} \setminus B_{r_1}} \star r^{1 - d}|\partial_r u|
\end{equation}
and if we apply the Cauchy-Schwarz inequality and approximate $u$ by $C^1$ functions as on \cite[pg68]{Giusti77}, it follows from Lemma \ref{monotonicity lemma} that the right-hand side of (\ref{monotone before cs}) is
$$\lesssim \sqrt{\int_{B_{r_2} \setminus B_{r_1}} \star r^{1 - d} |\dif u|} \sqrt{\int_{r_1}^{r_2} \partial_r \left[e^{Ar^2} r^{1-d}\int_{B_r} \star |\dif u|\right] \dif r}.$$
The monotonicity (\ref{weak monotonicity}) follows at once.

Integrating by parts,
\begin{align*}
\int_{B_{r_2} \setminus B_{r_1}} \star r^{1 - d} |\dif u| &= \int_{r_1}^{r_2} r^{1 - d} \partial_r \int_{B_r} \star |\dif u| \dif r \\
&\leq r^{1 - d} \int_{B_r} \star |\dif u| + (d - 1) \int_{r_1}^{r_2} r^{-d} \int_{B_r} \star |\dif u| \dif r.
\end{align*}
Using (\ref{weak monotonicity}) we bound this second integral as
\begin{align*}
\int_{r_1}^{r_2} r^{-d} \int_{B_r} \star |\dif u| \dif r &\leq r^{1 - d} \log \frac{r_2}{r_1} \int_{B_{r_2}} \star |\dif u|.
\end{align*}
If we set
$$J_r := r^{1 - d} \int_{B_r} \star |\dif u|$$
then we can sum up our progress so far as
$$|I_{r_2} - I_{r_1}| \lesssim \sqrt{\left[1 + \log \frac{r_2}{r_1}\right] J_{r_2}} \sqrt{e^{Ar_2^2} J_{r_2} - e^{Ar_1^2} J_{r_1}} + r_2^2 J_{r_2}.$$
The claim now follows by squaring both sides and applying Cauchy-Schwarz.
\end{proof}

For a function $u$ on $M$, $P \in M$, we define the \dfn{blowup} of $u$ at $P$ to be the net of functions $u_t: T_PM \to \RR$, given by
$$u_t(v) := u\left(\exp_P(tv)\right).$$
If $u$ has least gradient and $M$ is curved, then (unlike in the flat case) its blowup $u_t$ need not have least gradient, but at least satisfies 
\begin{equation}\label{approximately least gradient}
\limsup_{t \to 0} \|\dif u_t\|_{TV(U)} \leq \limsup_{t \to 0} \eta(u_t, U) < \infty 
\end{equation}
where $U$ is any open subset of $T_PM$ with Lipschitz boundary.
This can be easily seen by Taylor expanding the metric in normal coordinates based at $P$; as the scale parameter $t \to 0$, the corrections from the metric vanish.
The Miranda stability theorem \cite[Osservazione 3]{Miranda66}, which traditionally has been stated for sequences of functions of least gradient, can be easily modified to hold in this setting.
We omit the details.

\begin{lemma}[Miranda stability theorem]\label{Miranda convergence}
If a net of functions $(u_t)$ (not necessarily of the same trace) satisfies (\ref{approximately least gradient}) for every open $U \Subset M$ with Lipschitz boundary,
then a subsequence converges weakly in $BV_\loc(M)$ to some function $u$ of least gradient.
\end{lemma}

\begin{corollary}\label{blowup theorem}
Suppose that $U$ is an open set with least perimeter in $B(P, r)$, $P \in \partial^* U$, and $u = 1_U$.
Then the blowup $(u_t)$ of $u$ converges as $t \to 0$ along a subsequence weakly in $BV$ to the indicator function of a set $C \subset T_PM$ such that $\partial C$ is a minimal cone containing $0$.
If $d \leq 7$, then $\partial C$ is a hyperplane.
\end{corollary}
\begin{proof}
By the Miranda stability theorem applied on the manifold $T_PM$, $u_t \to 1_C$ weakly in $BV$, where $C$ has least perimeter.
It is a standard consequence of (\ref{weak monotonicity}) that, since $(u_t)$ is a blowup, $\partial C$ is a minimal cone containing $0$.
See \cite[Theorem 9.3]{Giusti77} for the euclidean case; the proof is identical here.
If $d \leq 7$, it follows that $\partial C$ is a hyperplane \cite[Theorem 9.10 and Theorem 10.10]{Giusti77}.
\end{proof}

We now estimate the perimeter of a set of least perimeter, generalizing \cite[Remark 5.13]{Giusti77}.
As a consequence, if $u = 1_U$ where $U$ has least perimeter, then the error term in the monotonicity formula is of size $O(r_2^{d + 1})$.

\begin{corollary}\label{doubling dimension}
There exists $A \geq 0$ such that for every set $U$ of least perimeter in a ball $B_r = B(P, r)$, with $P \in \partial^* U$, and $r > 0$ small,
$$|\Ball^{d - 1}|e^{-Ar^2}r^{d - 1} \leq |\partial^*U \cap B_r| \leq |\Sph^{d - 1}|e^{Ar^2} r^{d - 1}.$$
\end{corollary}
\begin{proof}
The upper bound on $|\partial^* U \cap B_r|$ is immediate from (\ref{a priori estimate 2}) and (\ref{geodesic spheres area form}).
The lower bound is obtained from (\ref{weak monotonicity}), which implies that
$$\limsup_{\rho \to 0} e^{-A\rho^2} \rho^{1 - d} |\partial^* U \cap B_\rho| \leq |\partial^* U \cap B_r|.$$
To control the left-hand side we take a blowup $(u_\rho)$ of $1_U$.
By Corollary \ref{blowup theorem} we can pass to a subsequence so that $u_\rho \to 1_C$ for $C$ a half-space, which in particular is transverse to $B'_1$, where the prime denotes the euclidean metric on the tangent space.
Then
\begin{align*}
\limsup_{\rho \to 0} e^{-A\rho^2} \rho^{1 - d} |\partial^* U \cap B_\rho| &= \lim_{\rho \to 0} e^{O(\rho^2)} \int_{B'_1} \star'|\dif u_\rho|' = \int_{B'_1} \star'|\dif 1_C|.
\end{align*}
This last term is $|\partial C \cap B'_1|$, the measure of the intersection of the euclidean unit ball with a minimal cone through its origin.
By \cite[(5.16)]{Giusti77}, any minimal cone in $B'_1$ has measure at least $|\Ball^{d - 1}|$, the area of the unit ball of $\RR^{d - 1}$.
\end{proof}


%%%%%%%%%%%%%%%%%%%%%%%%%%%%
\section{De Giorgi lemma}\label{DeGiorgiSec}
We now introduce the quantity which governs the rate of convergence of the Lebesgue differentiation theorem for $\normal_U$, whenever $U$ is a set of locally finite perimeter.
More precisely, let $\mu$ be the surface measure of $\partial^* U$, which is by definition the total variation measure of $1_U$.
We study the convergence of the approximation
$$\normal_U(P, r) := \avg_{B(P, r), P, \mu} \normal_U.$$

\begin{definition}
The \dfn{excess} of a set $U \subset M$ of locally finite perimeter at $P \in \partial U$, such that $\partial^* U$ has surface measure $\mu$, in an open set $A \ni P$ with Lipschitz boundary is
$$\Exc_A(U, P) := \mu(U)\left(1 - \left|\avg_{A, P, \mu} \normal_U\right|\right).$$
For $\rho > 0$ we write $\Exc_\rho(U, P) := \Exc_{B(P, \rho)}(U, P)$.
\end{definition}

Since parallel transport preserves length, we have
$$|K(P, Q) \normal_U(Q)| = 1$$
and, averaging $K(P, Q) \normal_U(Q)$ in $Q$, we conclude
\begin{equation}\label{normal isnt too big}
|\normal_U(P, r)| \leq 1.
\end{equation}
From this we obtain the following monotonicity property: if $P \in A'$ and $A' \subseteq A$, then
\begin{equation}\label{approximate monotone}
0 \leq \Exc_{A'}(U, P) \leq \Exc_A(U, P).
\end{equation}
Moreover, by Proposition \ref{translation invariance}, if $P, Q \in A$ then 
\begin{equation}\label{translation invariant excess}
|\Exc_A(U, P) - \Exc_A(U, Q)| \lesssim (\diam A)^2 |\partial^* U \cap A|.
\end{equation}

Following \cite{Miranda66,Giusti77,deGiorgi61}, we proceed by controlling the excess using the following de Giorgi lemma, which is the Riemannian generalization of \cite[Theorem 8.1]{Giusti77}:

\begin{proposition}[de Giorgi lemma]\label{de Giorgi}
There exist constants $C, c, \rho_* > 0$ which only depend on $M$, such that for every $P \in M$, every $\rho$ such that $0 < \rho < \rho_*$, and every set $U \subset M$ of least perimeter such that
\begin{equation}\label{base case}
\Exc_\rho(U, P) \leq c\rho^{d - 1},
\end{equation}
we have
\begin{equation}\label{dGL concl}
\Exc_{\rho/2}(U, P) \leq 2^{-d} \Exc_\rho(U, P) + C\rho^{d + 1}.
\end{equation}
\end{proposition}

\begin{corollary}\label{DGL base case}
Let $d \leq 7$.
Assume the de Giorgi lemma, and let $U \subset M$ have least perimeter.
Then there exists $\rho = \rho(P) < \rho_*$, which is locally uniformly positive, such that (\ref{base case}) holds.
\end{corollary}
\begin{proof}
We follow \cite[pg109]{Giusti77}.
Let $Q \in \partial^* U$; we shall choose $\rho$ uniformly in a small neighborhood of $Q$, which is enough since $\partial^* U$ is dense in $\partial U$.
Since $\partial U$ has a tangent space at $Q$ by Corollary \ref{blowup theorem} and the fact that $d \leq 7$, $\Exc_r(U, Q) \ll r^{d - 1}$, thus we can choose $r \in (0, \rho_*)$ such that $\Exc_r(U, Q) \leq c(r/2)^{d - 1}$.
For $P \in B(Q, r/4)$ we have $B(P, r/4) \subseteq B(Q, r/2)$ and hence by the de Giorgi lemma, for $\rho := r/4$, (\ref{base case}) holds.
\end{proof}

\begin{proof}[Proof of Theorem \ref{main thm 2}, assuming the de Giorgi lemma]
We shall use the de Giorgi lemma inductively as in \cite[Theorem 8.2]{Giusti77}.
Let $P \in M$, $\rho$ be given by Corollary \ref{DGL base case}, $r := \rho/2^n$ for some $n \in \NN$, and $s \in (r/2, r)$.
Let
$\xi := \normal_U(P, r)$, $\eta = \normal_U(P, s)$, $m := |\partial^* U \cap B(P, s)|$, $M := |\partial^* U \cap B(P, r)|$, and $\gamma_n := \Exc_{\rho/2^n}(U, P)$.
Then $|\xi|^2 \leq 1$ and $|\eta|^2 \leq 1$, so
$$|\xi - \eta|^2 = |\xi|^2 + |\eta|^2 - 2 g^{-1}(\xi, \eta) \leq 2(1 - g^{-1}(\xi, \eta)).$$
To estimate the right-hand side, let $\mu$ be the surface measure on $\partial^* U$. Then
$$m(1 - g^{-1}(\xi, \eta)) = \int_{B(P, s)} (1 - g^{-1}(\xi, K(P, Q) \normal_U(Q))) \dif \mu(Q) =: I.$$
By the Cauchy-Schwarz inequality, (\ref{normal isnt too big}), and the fact that $K(P, Q)$ preserves length,
$$g^{-1}(\xi, K(P, Q) \normal_U(Q)) \leq 1,$$
which implies that, since $s \leq r$,
\begin{align*}
I
&\leq \int_{B(P, r)} (1 - g^{-1}(\xi, K(P, Q) \normal_U(Q))) \dif \mu(Q) 
\leq M(1 - |\xi|^2) \leq 2M(1 - |\xi|).
\end{align*}
But $M(1 - |\xi|)$ is exactly the definition of $\Exc_r(U, P)$, so putting everything together and using Corollary \ref{doubling dimension} to bound $m \gtrsim s^{d - 1} \gtrsim r^{d - 1}$, we have the bound
\begin{equation}\label{just need the excess}
|\xi - \eta|^2 \lesssim r^{1 - d} \Exc_r(U, P) = r^{1 - d} \gamma_n.
\end{equation}

By the de Giorgi lemma, there exists $C > 0$ such that
\begin{equation}\label{induction on gamma}
\gamma_n \leq \frac{\gamma_0}{2^{nd}} + \sum_{k=0}^n \frac{C\rho^{d + 1}}{2^{k(d + 1) + (n - k)d}} \leq \frac{\gamma_0 + C\rho^{d + 1}}{2^{nd}}.
\end{equation}
Indeed, by induction,
\begin{align*}
\gamma_{n + 1}
&\leq \frac{\gamma_0}{2^{(n + 1)d}} + 2^{-d} \sum_{k=0}^n \frac{C\rho^{d + 1}}{2^{k(d + 1) + (n - k)d}} + \frac{C\rho^{d + 1}}{2^{(n + 1)(d + 1)}} \\
&= \frac{\gamma_0}{2^{(n + 1)d}} + \sum_{k=0}^{n + 1} \frac{C\rho^{d + 1}}{2^{k(d + 1) + (n + 1 - k)d}}
\end{align*}
and (\ref{induction on gamma}) follows by summing the geometric series.
By (\ref{just need the excess}), (\ref{induction on gamma}), and the fact that $\rho$ is locally uniformly positive, it follows that along a subsequence, $\normal_U(\cdot, r)$ converges locally uniformly to $\normal_U$.
Therefore $\normal_U$ is continuous, so by Proposition \ref{locality of Caccioppoli}, $\partial U$ is a $C^1$ embedded hypersurface.
By elliptic regularity, $\partial U$ is smooth.

To prove the stability, let $(N_t)$ be a normal variation of $\partial U$ with compact support in the interior.
Then for $t$ small, $N_t$ bounds an open set $U_t$, and by the convexity (\ref{least perimeter dfn}) of the total variation,
$$|N_t| = \|\dif 1_{U_t}\|_{TV} \geq \|\dif 1_U\|_{TV} = |N_0|.$$
The stability then follows from the Taylor expansion of $|N_t|$ about $t = 0$.
\end{proof}

We break the proof of the the de Giorgi lemma into two steps, which correspond to \cite[Lemma 6.4]{Giusti77} (the $C^1$ case) and \cite[Lemma 7.5]{Giusti77} (reduction to the $C^1$ case).
Let $c_0 = c_0(M) > 0$ be a small constant to be determined later, and let $\alpha := 1/(2(1 - c_0))$.
We say that a vector field $X$ is $P$-\dfn{aligned} if there exists a normal coordinate frame $(\partial_\mu)$ based at $P$ such that $X = \partial_0$.

\begin{lemma}[de Giorgi lemma, $C^1$ case]\label{Miranda44}
Suppose that $c_0$ is small enough depending on $M$.
Then there exists a constant $c_1 = c_1(c_0, M)$, such that the following holds.
For every $\rho, \beta > 0$ small enough depending on $c_0$, and every set $U$ with $C^1$ boundary in $B(P, \rho)$, suppose that there exists a $P$-aligned vector field $X$ such that on $B(P, \rho)$,
\begin{align}
\Exc_\rho(U, P) &\leq \beta, \label{Miranda44 induction hyp}\\
|\partial^* U \cap B(P, \rho)| &\leq \eta(U, B(P, \rho)) + c_1 \beta, \label{Miranda44 minimality hyp} \\
(\normal_U, X) &\geq e^{-o(c_1^2)}. \label{Miranda44 normal hyp}
\end{align}
Then
\begin{equation}\label{Miranda44 concl}
\Exc_{\alpha \rho} (U, P) \leq (\alpha^{d + 1} + c_0) \beta + C\rho^{d + 1}.
\end{equation}
\end{lemma}

\begin{lemma}[reduction to the $C^1$ case]\label{single mollify}
There exists $c_2(c_1, M) > 0$ such that for any ball $B(P, \rho)$, with $\rho$ sufficiently small depending on $c_1$, and any set $U$ of least perimeter in $B(P, \rho)$ such that
\begin{equation}\label{single mollify hyp}
\Exc_\rho (U, P) \leq c_2 \rho^{d - 1},
\end{equation}
there exists an open set $V \subseteq B(P, (1 - c_1)\rho)$ and a $P$-aligned vector field $X$ such that $\partial V$ is $C^1$ and on $B(P, \rho)$, and such that for any $0 < \varpi \leq (1 - c_1)\rho$,
\begin{align}
|\Exc_\varpi (U, P) - \Exc_\varpi (V, P)| &\leq c_1 \Exc_\rho (U, P) + C\rho^{d + 1}, \label{single mollify excess} \\
|\partial V \cap B(P, (1 - c_1)\rho)| &\leq \eta(V, B(P, (1 - c_1)\rho)) + c_1 \Exc_\rho (U, P), \label{single mollify minimality} \\
(\normal_V, X) &\geq e^{-o(c_1^2)}. \label{single mollify normal}
\end{align}
\end{lemma}

In these lemmata, (\ref{Miranda44 induction hyp}), should be viewed as a bootstrap hypothesis, which is then improved by (\ref{Miranda44 concl}), while (\ref{Miranda44 minimality hyp}), (\ref{single mollify minimality}) asserts that the set in question is close to a set of least perimeter.
The technical condition (\ref{Miranda44 normal hyp}), (\ref{single mollify normal}) is used to find coordinates in which $\partial U$ can be represented by a graph; unlike (\ref{Miranda44 induction hyp}) and (\ref{Miranda44 minimality hyp}), which are conditions on the average behavior of $\partial U$, (\ref{Miranda44 normal hyp}) is a pointwise condition.

\begin{proof}[Proof of the de Giorgi lemma, assuming Lemmata \ref{Miranda44} and \ref{single mollify}]
Let $c_0 \leq 2^{-(d + 2)}$ and $c_1 \leq c_0/4$ be as in Lemma \ref{Miranda44}, and let $c := c_2$ be as in Lemma \ref{single mollify}.
Let $U$ be a set of least perimeter satisfying (\ref{single mollify hyp}), and let $V, X$ be as in Lemma \ref{single mollify}.
By (\ref{approximate monotone}) and (\ref{single mollify excess}),
\begin{align*}
\Exc_{(1 - c_1) \rho} (V, P) &\leq \Exc_{(1 - c_1) \rho} (U, P) + c_1 \Exc_\rho (U, P) + O(\rho^{d + 1}) \\
&\leq (1 + c_1) \Exc_\rho (U, P) + O(\rho^{d + 1}).
\end{align*}
Since $1/2 = \alpha (1 - c_0)$, by Lemma \ref{Miranda44}, it follows that
\begin{align*}
    \Exc_{\rho/2} (V, P) &\leq (2^{-(d + 1)} + c_0) (1 + c_1) \Exc_\rho (U, P) + O(\rho^{d + 1}) \\
    &\leq (2^{-(d + 1)} + 2^{-(d + 2)}) \Exc_\rho (U, P) + O(\rho^{d + 1})
\end{align*}
Finally, by (\ref{approximate monotone}) and (\ref{single mollify excess}),
\begin{align*}
    \Exc_{\rho/2} (U, P)
    &\leq \Exc_{\rho/2} (V, P) + c_1 \Exc_\rho (U, P) + O(\rho^{d + 1}) \\
    &\leq (2^{-(d + 1)} + 2^{-(d + 2)} + 2^{-(d + 3)}) \Exc_\rho (U, P) + O(\rho^{d + 1})\\
    &\leq 2^{-d} \Exc_\rho (U, P) + O(\rho^{d + 1}). \qedhere
\end{align*}
\end{proof}

%%%%%%%%%%%%%%%%%%%%%%%%%%%
\section{The smooth case}\label{smoothcase}
\subsection{Minimal graphs}
We now prove Lemma \ref{Miranda44}, following \cite[Chapter 6]{Giusti77}.
To set up the proof, let $(x^0, \dots, x^{d - 1})$ be normal coordinates based at $P$ such that $X = \partial_0$, and let $\Omega \subseteq \{x^0 = 0\}$ be a relatively open set.
We say that a $C^1$ hypersurface $N \subset M$ is \dfn{graphical} over $\Omega$ if for every integral curve $\gamma$ of $X$ passing through $\Omega$, $N$ intersects exactly once, and transversely.\footnote{Elsewhere in the literature this condition is called \dfn{strongly graphical}, and \dfn{graphical} means that the intersection may fail to be transverse.}
In coordinates, this means that $N = \{x^0 = u(x_1, \dots, x_{d - 1})\}$ for some $C^1$ \dfn{graphical function} $u: \Omega \to \RR$.
We equip $\Omega$ with the flat metric arising from the coordinates $(x^1, \dots, x^{d - 1})$, and let $\dif x$ denote the area form on $\Omega$ arising from its flat geometry.
The condition of being graphical only depends on $\Omega, X$, but not on the choice of coordinates.

\begin{lemma}
There is a map $\Lagrange$ taking $1$-jets on $\Omega$ to area forms on $\Omega$, with the following property: for any graphical hypersurface $N$ over $\Omega$, with graphical function $u$, and any $\Omega' \subseteq \Omega$, the area of the subset $N'$ of $N$ which is graphical over $\Omega'$ is
$$|N'| = \int_{\Omega'} \Lagrange(u, \dif u).$$
Moreover, if $\|\dif u\|_{C^0} \leq 1$, then
$$\Lagrange(u, \dif u) = \sqrt{1 + |\dif u|^2 + O(|x|^2 + \|u\|_{C^0}^2)} \dif x.$$
\end{lemma}
\begin{proof}
Let $\Psi: \Omega \to N$ be the diffeomorphism $x \mapsto (x, u(x))$.
For any $v, w \in T_x \Omega$, let
\begin{align*}
h(v, w) &:= g(x, u(x))(v, w), \\
h(X, v) &:= g(x, u(x))(X(x, u(x)), v), \\
h(X, X) &:= g(x, u(x))(X(x, u(x)), X(x, u(x))).
\end{align*}
These quantities are defined in coordinates, because we can use the coordinates to identify $T_{(x, 0)}M$ with $T_{(x, u(x))} M$, so that the quadratic form $g(x, u(x))$ on $T_{(x, u(x))} M$ acts on $T_{(x, 0)}M$.
In these coordinates we have $\Psi_* v = v + (\partial_v u) X(x, u(x))$, so that
\begin{align*}
\Psi^* g(v, w)
&= g(\Psi_* v, \Psi_* w) \\
&= h(v, w) + h(X, v) \partial_w u + h(X, w) \partial_v u + h(X, X) \partial_v u \partial_w u.
\end{align*}
If we set $v = \partial_i$ and $w = \partial_j$, $i, j = 1, \dots, d - 1$, then from the fact that the coordinates are normal we obtain $h(v, w) = \delta_{ij} + O(|x|^2 + u(x)^2)$, $h(X, v) = O(|x|^2 + u(x)^2)$, and $h(X, X) = 1 + O(|x|^2 + u(x)^2)$.
Moreover, $\partial_i u \partial_j u$ are the components of $\dif u \otimes \dif u$.
In conclusion,
$$\Psi^* g(\partial_i, \partial_j) = I + \dif u \otimes \dif u + O((|x|^2 + u(x)^2)(1 + |\dif u(x)|)).$$
However, $1 + |\dif u(x)| \leq 2$, which can be absorbed.
Furthermore, by \cite[(24)]{Petersen2008}, the determinant of $I + \dif u \otimes \dif u$ is $1 + (|\dif u|')^2$, which up to an error of size $|x|^2 + u(x)^2$ is equal to $|\dif u|^2$, since we are in normal coordinates.
The pullback of the area form of $N$ by $\Psi$ is $\sqrt{\det((\Psi^* g(\partial_i, \partial_j))_{ij})}$, and $\Psi$ identifies $N'$ with $\Omega'$, so we get 
\begin{align*}
|N'| &= \int_{\Omega'} \sqrt{1 + |\dif u(x)|^2 + O(|x|^2 + \|u\|_{C^0}^2)} \dif x. \qedhere
\end{align*}
\end{proof}

Let us write $\mathscr B_\rho$ for a ball in $\Omega$, of radius $\rho$ centered on the coordinate origin $P$.
If the graphical hypersurface $N$ intersects $\Omega$ at $P$, then on $\mathscr B_\rho$, we have from the Taylor expansion of $g$ that
\begin{equation}\label{approximate by euclidean lagrangian}
\Lagrange(u, \dif u) - \Lagrange(u, \avg_\rho \dif u) = \left(\Japan{\dif u} - \Japan{\avg_\rho \dif u}\right) \dif x + O(\rho^2)
\end{equation}
where $\avg_\rho := \avg_{\mathscr B_\rho, P, \dif x}$, taken using the flat metric on $\Omega$.
Here $\Japan{\dif u} := \sqrt{1 + |\dif u|^2}$ is the Japanese norm of $\dif u$, so that $\Japan{\dif u} \dif x$ is exactly the Lagrangian density for the euclidean minimal surface equation.
Thus, we reduce the problem of estimating $\Lagrange(u, \dif u) - \Lagrange(u, \avg_\rho \dif u)$ to the euclidean case, and hence can show the following analogue of \cite[Lemma 6.3]{Giusti77}.

\begin{lemma}[de Giorgi lemma, minimal graphs]\label{Miranda43}
There exists $c_1 = c_1(c_0, M) > 0$ such that for every $\beta, \rho > 0$, the following holds.
Let $N$ be a $C^1$ graphical hypersurface over $\mathscr B_\rho$, with graphical function $u$, which intersects $\Omega$ at $P$.
Let $I$ be the union of all integral curves of $X$ through $\mathscr B_\rho$.
Suppose that $\|\dif u\|_{C^1} \leq c_1$, and that
\begin{align}
\int_{\mathscr B_\rho} \Lagrange(u, \dif u) - \Lagrange(u, \avg_\rho \dif u) &\leq \beta \label{Miranda43 oscillation}, \\
\int_{\mathscr B_\rho} \Lagrange(u, \dif u) &\leq \eta(N, I) + c_1 \beta \label{Miranda43 minimality}.
\end{align}
Then for any $\tilde \alpha = \frac{1}{2} + O(c_0)$,
\begin{equation}\label{Miranda43 concl}
\int_{\mathscr B_{\tilde \alpha \rho}} \Lagrange(u, \dif u) - \Lagrange(u, \avg_{\tilde \alpha \rho} \dif u) \leq \left(\tilde \alpha^{d + 1} + \frac{c_0}{2}\right) \beta + C\rho^{d + 1}.
\end{equation}
\end{lemma}
\begin{proof}
Let $v$ be the harmonic function on $\mathscr B_\rho$ (with its euclidean metric) which equals $u$ on $\partial \mathscr B_\rho$.
By elliptic regularity, the maximum principle for harmonic functions, and (\ref{Miranda43 minimality}),
$$\|v\|_{C^1} \lesssim \|v\|_{C^0} \leq \|u\|_{C^0} \leq \rho \|\dif u\|_{C^1} \leq c_1.$$
In particular, $\Japan{\dif v} \lesssim 1$ and $v(x)^2 \lesssim \rho^2$, so by (\ref{approximate by euclidean lagrangian}) and the fact that $|\mathscr B_\rho| \sim \rho^{d - 1}$,
\begin{align*}
&\left|\int_{\mathscr B_\rho} \Lagrange(u, \dif u) - \Lagrange(v, \dif v) - \Japan{\dif u} \dif x + \Japan{\dif v} \dif x\right| \\
&\qquad \lesssim \int_{\mathscr B_\rho} (|x|^2 + u(x)^2) \Japan{\dif u} \dif x + (|x|^2 + v(x)^2) \Japan{\dif v} \dif x
\lesssim \rho^{d + 1}.
\end{align*}
Let $K$ be the graph of $v$, viewed as a submanifold of $M$.
Since $u$ and $v$ have the same trace, $|K| \geq \eta(K, I) = \eta(N, I)$, so
\begin{align*}
\int_{\mathscr B_\rho} \Lagrange(u, \dif u) - \Lagrange(v, \dif v) &\leq \int_{\mathscr B_\rho} \Lagrange(u, \dif u) - \eta(\Psi_w(\Omega), I) \leq c_1 \beta.
\end{align*}
We can then replace $\beta$ with $\beta + O(\rho^{d + 1})$ so that $u, v$ meet the hypotheses of \cite[Lemma 6.2]{Giusti77} which gives the result if $c_1$ is small enough.
\end{proof}

\subsection{Reduction to minimal graphs}
We have thus established the $C^1$ de Giorgi lemma, under the additional assumptions that $P \in N$, and that $N$ is graphical with a graphical function $u$ such that $\|\dif u\|_{C^0} \leq c_1$.
We shall deal with the assumption that $P \in N$ later, but first we remove the assumption that $N$ is graphical with small derivative.

\begin{lemma}\label{hypersurfaces are graphical}
Suppose that $\rho$ is small enough depending on $c_1$.
Let $U$ be a set with $C^1$ boundary in $B(P, \rho)$.
Suppose that $X$ is an aligned vector field for some normal coordinates $(x^\mu)$ based at $Q \in B(P, \rho)$, such that on $B(P, \rho)$,
\begin{align}
(\normal_U, X) &\geq e^{-o(c_1^2)}. \label{rep as a good graph hyp}
\end{align}
Then there exists a relatively open set $\Omega \subseteq \{x^0 = 0\}$ such that $\partial U$ is $X$-graphical over $\Omega$, and its graphical function $u$ satisfies $\|\dif u\|_{C^0} \leq c_1$.

Moreover, there exists a ball $\mathscr B \subseteq \Omega$ with the following property.
Let $\Omega^\alpha$ be the set of points in $\Omega$ which are initial data for $X$-integral curves which intersect $\partial U \cap B(P, \alpha \rho)$.
Then 
\begin{equation}\label{rep as a good graph set nests}
    \Omega^\alpha \subseteq \left(\alpha + \frac{c_0}{2}\right) \mathscr B \subset \mathscr B \subseteq \Omega.
\end{equation}
\end{lemma}
\begin{proof}
We first observe that if $\rho$ is small enough depending on $M$, then $B(P, \rho)$ appears convex in the coordinates $(x^\mu)$.
Indeed, $(x^\mu)$ is $O(\rho)$-close in the $C^2$ topology to a normal coordinate system $(y^\mu)$ based at $P$, so it suffices to show that $B(P, \rho)$ appears uniformly strictly convex as $\rho \to 0$ in the coordinates $(y^\mu)$.
However, in those coordinates $(y^\mu)$, $\partial B(P, \rho)$ appears to be a sphere of radius $\sim \rho$, so its principal curvatures appear to be $\gtrsim \rho^{-1}$, implying the uniformly strict convexity.

By \cite[Theorem 4.8]{Giusti77}, the convexity in coordinates of $B(P, \rho)$ and (\ref{rep as a good graph hyp}), there exists $\Omega \subseteq \{x^0 = 0\}$ and $u \in C^1(\Omega)$ such that $\partial U$ is the graph of $u$ and 
$$\|\dif u\|_{C^0} \leq \sup_{x_1, x_2 \in \Omega} \frac{|u(x_1) - u(x_2)|}{|x_1 - x_2|} \leq e^{o(c_1^2)}\sqrt{1 - e^{-o(c_1^2)}} \leq c_1.$$

It remains to construct $\mathscr B$ satisfying (\ref{rep as a good graph set nests}); we adapt the proof of \cite[(6.25)]{Giusti77} to this setting.
We may assume that $\Omega^\alpha$ is nonempty; otherwise just choose any ball $\mathscr B$ in the open set $\Omega$.
Let $y := x^0$ and $x := (x^1, \dots, x^{d - 1})$ denote the coordinate functions.
Since $\Omega^\alpha$ is nonempty, there exists $(x_*, y_*) \in \partial U \cap B(P, \alpha \rho)$, and then for $(x, y) \in \partial U$, we obtain from the bound on $\dif u$ that $|y - y_*| \lesssim c_1 \rho$.
Writing $(x_\natural, y_\natural)$ for the coordinates of $P$, we see from the Pythagorean theorem and the approximate euclideanness of the metric that for $(x, y) \in \partial B(P, \rho)$,
$$|x - x_\natural|^2 + (y - y_\natural)^2 = \rho^2 + O(\rho^3).$$
If we additionally impose $(x, y) \in \partial U$, then it follows that 
$$|x - x_\natural|^2 + (y_* - y_\natural)^2 = \rho^2 + O(c_1 \rho^2)$$
at least if $\rho$ is small depending on $c_1$.
But $(x, y) \in \partial U \cap \partial B(P, \rho)$ iff $x \in \partial \Omega$ and $y = u(x)$.
So $\partial \Omega$ lies in an annulus:
$$\partial \Omega \subset \{x: (1 - Cc_1)\rho^2 - (y_* - y_\natural)^2 \leq |x - x_\natural|^2 \leq (1 + Cc_1)\rho^2 - (y_* - y_\natural)^2\}$$
for some constant $C$ depending on $M$ only.
A similar argument with $(x, y) \in \partial U \cap \partial B(P, \alpha \rho)$ (instead of $\partial B(P, \rho)$) shows that
$$\partial \Omega^\alpha \subset \{x: (1 - Cc_1) \alpha^2 \rho^2 - (y_* - y_\natural)^2 \leq |x - x_\natural|^2 \leq (1 + Cc_1)\alpha^2 \rho^2 - (y_* - y_\natural)^2\}.$$
If we take $c_1$ small enough depending on $C, c_0$, then we may choose
$$\mathscr B := \mathscr B\left(x_\natural, \sqrt{(1 - Cc_1)\rho^2 - (y_* - y_\natural)^2}\right).$$
Then $\mathscr B \subseteq \Omega$ and since 
\begin{align*}
\left(\alpha + \frac{c_0}{2}\right) \sqrt{(1 - Cc_1)\rho^2 - (y_* - y_\natural)^2} &\geq \sqrt{(1 + Cc_1)\alpha^2 \rho^2 - (y_* - y_\natural)^2},
\end{align*}
we also have $(\alpha + c_0/2) \mathscr B \supseteq \Omega^\alpha$.
\end{proof}

\begin{proof}[Proof of Lemma \ref{Miranda44}]
Throughout this proof we assume that there exists some $Q \in \partial U \cap B(P, \alpha \rho)$.
If not, then (\ref{Miranda44 concl}) is vacuous since then $\Exc_{\alpha \rho} (U, P) = 0$.

Consider normal coordinates $(x^\mu)$ based at $Q$, such that at $Q$, $Y := \partial_0$ points in the same direction as $X$.
If $\rho$ is small enough depending on $c_1$, then by (\ref{Miranda44 normal hyp}), we have 
$$(\normal_U, Y) \geq e^{-o(c_1^2)}.$$
By Lemma \ref{hypersurfaces are graphical}, it follows that $\partial U$ is $Y$-graphical over some set $\Omega \subseteq \{x^0 = 0\}$, with graphical function $u$ satisfying $\|\dif u\|_{C^0} \leq c_1$.
Moreover, there exists a ball $\mathscr B \subseteq \Omega$ satisfying (\ref{rep as a good graph set nests}).
Let $J_t := \{(x, y) \in B(P, \rho): x \in t\mathscr B\}$ be the cylinder over $t\mathscr B$.

For $t \leq 1$, we can use (\ref{path ordered exponential taylor series}) and Corollary \ref{doubling dimension} to compute in the coordinates $(x^\mu)$ that
\begin{align*}
    \Exc_{J_t}(U, Q) &= |\partial U \cap J_t| - \int_{J_t} \mathcal Pe^{-\int_R^Q \Gamma} \normal_U(R) \dif \mu(R) \\
    &= |\partial U \cap J_t| - \left[\int_{J_t} (\normal_U)_\alpha \dif \mu\right] \dif x^\alpha + O(\rho^{d + 1}).
\end{align*}
Here $\Gamma$ are Christoffel symbols and $\mu$ is the surface measure.

Let $\nu$ denote the surface measure with respect to the euclidean metric on $M$ obtained from coordinates, and let $\normal_U'$ denote the conormal $1$-form with respect to the euclidean metric.
On \cite[pg83]{Giusti77} it is shown that
\begin{equation}\label{Excess versus Lagrangian}
\left|\left[\int_{J_t} (\normal_U')_\alpha \dif \nu\right] \dif x^\alpha\right| = \Japan{\avg_{t\rho} \dif u} |t\mathscr B|.
\end{equation}
By Corollary \ref{doubling dimension} and the approximate euclideanness of the metric, we can replace $\dif \nu$ with $\dif \mu$, and $\normal_U'$ with $\normal_U$, in (\ref{Excess versus Lagrangian}), at the price of an error of size $O(\rho^{d + 1})$.
So from (\ref{approximate by euclidean lagrangian}), we deduce 
\begin{align*}
\Exc_{J_t}(U, P)
&= \Exc_{J_t}(U, Q) + O(\rho^{d + 1}) \\
&= |\partial U \cap J_t| - \int_{t\mathscr B} \Japan{\avg_{t\rho} \dif u} \dif x + O(\rho^{d + 1}) \\
&= \int_{t\mathscr B} \Lagrange(u, \dif u) - \Lagrange(u, \avg_{t\rho} \dif u) + O(\rho^{d + 1}).
\end{align*}
Applying (\ref{rep as a good graph set nests}) and the properties (\ref{approximate monotone}) and (\ref{translation invariant excess}) of the excess, we obtain from Lemma \ref{Miranda43} with $\tilde \alpha := \alpha + c_0/2$ that
\begin{align*}
\Exc_{\alpha \rho}(U, P) 
&\leq \Exc_{J_{\alpha + c_0/2}}(U, P) + O(\rho^{d + 1}) \\
&\leq \int_{(\alpha + c_0/2) \mathscr B} \Lagrange(u, \dif u) - \Lagrange(u, \avg_{(\alpha + c_0)\rho} \dif u) + O(\rho^{d + 1}) \\
&\leq \left(\left(\alpha + \frac{c_0}{2}\right)^{d + 1} + \frac{c_0}{2}\right) \int_{\mathscr B} \Lagrange(u, \dif u) - \Lagrange(u, \avg_\rho \dif u) + O(\rho^{d + 1}).
\end{align*}
Since $\alpha < 1$, if $c_0$ is smaller than an absolute constant, $((\alpha + c_0/2)^{d + 1} + c_0/2) \leq \alpha^{d + 1} + c_0$.
We can then reverse the above computation to obtain 
\begin{align*}(\alpha^{d + 1} + c_0) \int_{\mathscr B} \Lagrange(u, \dif u) - \Lagrange(u, \avg_\rho \dif u) &\leq (\alpha^{d + 1} + c_0) \Exc_\rho(U, P) + O(\rho^{d + 1}). \qedhere 
\end{align*}
\end{proof}

%%%%%%%%%%%%%%%%%%%%%%%%%%%%%%%%%%%%%%%%%%%
\section{Mollification} \label{Mollifiers}
In this section we prove Lemma \ref{single mollify} by following \cite[Chapter 7]{Giusti77}.
The main point is to estimate a certain convolution operator which mollifies the set of least perimeter $U$ in such a way that, if $\normal_U$ is close to a given vector field $X$ on average, then its mollification is close to $X$ in $C^0$.
More precisely, given normal coordinates $(x^\mu)$ based at some point $P$, and a distribution $f$, we define 
\begin{equation}\label{convolution operator}
f_\varepsilon(x) := \frac{d + 1}{|\Ball^d| \varepsilon^d} \int_{\mathcal B(x, \varepsilon)} f(y) \left(1 - \frac{|x - y|}{\varepsilon}\right) \dif y.
\end{equation}
Here and always in this section, $\mathcal B(x, \varepsilon)$ is the \emph{euclidean} ball taken in the coordinates $(x^\mu)$.

It is possible to define an intrinsic operator, by replacing $|x - y|$ with $\dist(x, y)$, $\mathcal B$ with $B$, and $\dif y$ with $\dif V(y)$.
However, such an operator turns out to be rather annoying to work with, because it does not commute with $\dif$, so we shall not take this approach here.\footnote{Miranda \cite{Miranda66} uses a different convolution operator, which has an invariant interpretation as the pseudodifferential operator $(1 - \varepsilon^{-2} \Delta)^{-1}$, but sadly seems tricky to estimate in the presence of curvature.}

\subsection{Pointwise estimate on the normal vector}
When applied to sets of least perimeter with small excess, the convolution operator (\ref{convolution operator}) satisfies an analogue of \cite[Theorem 7.3]{Giusti77}, which in the flat case asserted that the normal vector of a convolved set of least perimeter is pointwise close to an aligned vector field; we now prove the same result.

\begin{lemma}\label{main mollifier lemma}
Let $\rho, \gamma > 0$ be small enough depending on $M, P$, and let $U$ be a set of least perimeter such that
\begin{equation}\label{hypothesis on main mollifier lemma}
\Exc_\rho(U, P) \leq \gamma \rho^{d - 1}.
\end{equation}
Furthermore, suppose that $\rho^2 \leq \gamma$.
Let $\varepsilon := \gamma^4\rho$, $\sigma := \gamma^{1/(2d)} \rho$, and $\varphi := (1_U)_\varepsilon$, taken in normal coordinates $(x^\mu)$ based at $P$.
Then there exists an aligned vector field $X$ such that for every $y \in (O(\gamma^2), 1 - O(\gamma^2))$, the level set $N_y := \partial \{\varphi > y\} \cap B(P, \rho - 2\sigma)$ is a $C^1$ hypersurface, and on $N_y$,
\begin{equation}\label{claim on main mollifier lemma}
    (\normal_{N_y}, X) \geq e^{-O(\rho^2 + \sigma)}.
\end{equation}
\end{lemma}

We begin the proof of Lemma \ref{main mollifier lemma} by letting $\delta := \gamma^d$ and $u := 1_U$.
We greedily construct a cover $\mathcal V = \{V_n: 1 \leq n \leq N\}$ of $\partial^* U \cap B(P, \varepsilon(1 - \delta))$, where $V_n := B(Q_n, 2\delta\varepsilon)$.
Since it was constructed greedily, $\mathcal V$ is \dfn{efficient} in the sense that there exists $C > 0$ only depending on $M$, such that any $Q \in B(P, \varepsilon(1 - \delta))$ is contained in $\leq C$ elements of $\mathcal V$.
Let $V_0$ be the annulus $B(P, \varepsilon) \setminus B(P, \varepsilon(1 - 2\delta))$.

To construct the aligned vector field $X$, let $v$ be the unit vector pointing in the same direction as $(\avg_{B(P, \rho), P, \mu} \normal_U)^\sharp$ where $\mu$ is the surface measure on $\partial^* U$; let $X$ be the aligned vector field extending $v$.
It then follows from (\ref{path ordered exponential taylor series}), Corollary \ref{doubling dimension}, and the fact that $\rho^2 \leq \gamma$ that 
\begin{equation}\label{hypothesis on mollifier sublemma}
\int_{B(P, \rho)} (|\dif u| - Xu)(z) \dif z = \Exc_\rho(U, P) + O(\rho^{d + 1}) \lesssim \gamma \rho^{d - 1}.
\end{equation}
We use this fact to estimate the convolution of $u$ in each ball $Q_n$.
However, $X$ is technically inconvenient as $|\dif u| - Xu$ can be negative. 
So we introduce $\tilde X := X/|X|$.

\begin{lemma}\label{mollifier sublemma}
Let $\tilde X := X/|X|$.
For every $Q \in \partial^* U \cap B(P, \varepsilon)$ and $x \in B(P, \rho - 2\sigma)$,
$$(1_{B(Q, 2\delta\varepsilon)}(|\dif u| - \tilde Xu))_\varepsilon(x) \lesssim \gamma^{1/(2d)} (1_{B(Q, \delta\varepsilon)} |\dif u|)_\varepsilon(x).$$
\end{lemma}
\begin{proof}
Let $\chi_\varepsilon$ be the convolution kernel.
Throughout this proof we may assume that $\gamma < 1/10$, so that $\sigma > \varepsilon$ and $\delta < 1/100$.
Then every point considered in this proof will be contained in the ball $B(P, 2\sigma)$, in which the metric takes the form $g_{\mu \nu} = \delta_{\mu \nu} + O(\sigma^2)$.
In particular, the volume form is $e^{O(\sigma^2)} \dif x$.

Let $V := B(Q, 2\delta\varepsilon)$; then we want to estimate 
$$(1_V(|\dif u| - \tilde Xu))_\varepsilon(x) = \int_{\mathcal B(x, \varepsilon) \cap V} \chi_\varepsilon(x - y) (|\dif u| - \tilde Xu)(y) \dif y.$$
Since $|\dif u| - \tilde Xu \geq 0$,
\begin{equation}\label{begin breaking up the mollifier}
\int_{\mathcal B(x, \varepsilon) \cap V} \chi_\varepsilon(x - y) (|\dif u| - \tilde Xu)(y) \dif y \leq \sup_{y \in V} \chi_\varepsilon(x - y) \int_V (|\dif u| - \tilde Xu)(z) \dif z.
\end{equation}
Moreover, by \cite[91]{Giusti77},
$$\sup_{y \in V} \chi_\varepsilon(x - y) \lesssim \inf_{y \in B(Q, \delta\varepsilon)} \chi_\varepsilon(x - y).$$

Let $W := B(Q, \sigma)$; by (\ref{weak monotonicity}), we obtain 
$$(2\delta\varepsilon)^{1 - d} \int_V |\dif u|(z) \dif z \leq e^{O(\sigma^2)} \sigma^{1 - d} \int_W |\dif u|(z) \dif z.$$
We use this to bound
\begin{align*}
(2\delta\varepsilon)^{1 - d} \int_V (|\dif u| - \tilde X u)(z) \dif z
&\leq \sigma^{1 - d} \int_W (|\dif u| - X u)(z) \dif z + O(\sigma^{3 - d}) \int_W |\dif u|(z) \dif z \\
&\qquad + \sigma^{1 - d} \int_W X u(z) \dif z - (2\delta\varepsilon)^{1 - d} \int_V Xu(z) \dif z \\
&\qquad + (2\delta\varepsilon)^{1 - d} \int_V (\tilde X - X)u(z) \dif z\\
&=: I_1 + I_2 + I_3 - I_4 + I_5.
\end{align*}
We then use (\ref{hypothesis on mollifier sublemma}) to bound
$$I_1 \leq \sigma^{1 - d} \int_{B(P, \rho)} (|\dif u| - X u)(z) \dif z \lesssim \gamma^{\frac{1 - d}{2d} + 1} \leq \gamma^{1/(2d)}.$$
By Corollary \ref{doubling dimension} we have $I_2 \lesssim \gamma^{1/(d - 1)}$, and since $X - \tilde X = O(\varepsilon^2)$ on $V \subseteq B(P, O(\varepsilon))$, $I_5 \lesssim \varepsilon^2 \lesssim \gamma^8$.

To estimate $I_3 - I_4$, we recall the notation (\ref{integral of du}) for vector-valued integrals.
For any $r > 0$, we can use (\ref{path ordered exponential taylor series}), the Taylor expansion of the metric, and Corollary \ref{doubling dimension} to estimate
$$r^{1 - d} \int_{B(Q, r)} Xu(z) \dif z = r^{1 - d} \left[\int_{B(Q, r)} K(Q, z) \dif u(z) \dif z\right]_0 + O(r^2) = I(u, Q, r)_0 + O(r^2).$$
Therefore we can apply the monotonicity formula, Proposition \ref{Monotone}, to compute
\begin{align*}
    I_3 - I_4 & \leq |\sigma^{1 - d} I(u, Q, \sigma) - (2 \delta \varepsilon)^{1 - d} I(u, Q, 2 \delta \varepsilon)| + O(\sigma^2) \\
    &\lesssim \sqrt{1 + (d - 1) \log \frac{\sigma}{2\delta\varepsilon}} \sqrt{\sigma^{1 - d} \int_W \star |\dif u|} \sqrt{\int_{2\delta\varepsilon}^\sigma \partial_r \left[e^{O(r^2)} r^{1 - d} \int_{B(Q, r)} \star |\dif u|\right] \dif r}\\
&\qquad + \sigma^{3 - d} \int_W \star |\dif u| + \sigma^2 \\
&=: J_1 J_2 J_3 + J_4 + J_5.
\end{align*}
Then (using Corollary \ref{doubling dimension} as necessary), we bound $J_1 \lesssim -\log \gamma$, $J_2 \lesssim 1$, and $J_4 \lesssim J_5 = \gamma^{1/(2d)}$.

In order to estimate $J_3$, we introduce a normal coordinate system $(\tilde x^\mu)$ based at $Q$, with $Y = \tilde \partial_0$ chosen so $Y(Q)$ and $X(Q)$ have the same span.
Then $Y(Q) = X(Q)/|X(Q)| = X(Q) + O(\varepsilon^2)$.
Taylor expanding $Y - X$ around $Q$, we obtain $\|Y - X\|_{C^0(W)} \lesssim \sigma$.
Then
\begin{align*}
J_3^2 &\leq \sigma^{1 - d} \int_W |\dif u|(z) \dif z - (2 \delta \varepsilon)^{1 - d} \int_V |\dif u|(z) \dif z + O(\sigma^{3 - d}) \int_W \star |\dif u| \\
&= \sigma^{1 - d} \int_W (|\dif u| - Xu)(z) \dif z + \sigma^{1 - d} \int_W (X - Y)u(z) \dif z + \sigma^{1 - d} \int_W Yu(z) \dif z \\
  &\qquad - (2 \delta\varepsilon)^{1 - d} \int_V Y u(z) \dif z + O(\sigma^{3 - d}) \int_W \star |\dif u| \\
&=: K_1 + K_2 + K_3 - K_4 + K_5.
\end{align*}
Then $K_1 = I_1 \leq \gamma^{1/2}$, $K_2 \lesssim \sigma = \gamma^{1/(2d)}$, and $K_5 = J_4 \lesssim \gamma^{1/(2d)}$.

To estimate $K_3 - K_4$, introduce the closed $d-1$-form $\psi := \dif \tilde x^1 \wedge \cdots \wedge \dif \tilde x^{d - 1}$.
Then
\begin{equation}\label{K3 calculus}
K_3 = \sigma^{1 - d} \int_W \dif u \wedge \psi = \sigma^{1 - d} \int_{U \cap \partial W} \psi.
\end{equation}
Since $W$ is a ball centered on the $(\tilde x^\mu)$-coordinate origin $P$, we can decompose
$$\partial W = \Gamma_+ \cup \Gamma_0 \cup \Gamma_-$$
where $\pm \tilde x^0 > 0$ on the hemispheres $\Gamma_\pm$ and $\Gamma_0$ is the equator.
Then all positive contributions to the integral in the right-hand side of (\ref{K3 calculus}) come from $\Gamma_+$.
However, if we set $W_0 := W \cap \{\tilde x^0 = 0\}$, then $\Gamma_+$ and $W_0$ are homologous relative to their common boundary $\Gamma_0$.
By Stokes' theorem,
$$K_3 \leq \sigma^{1 - d} \int_{\Gamma_+ \cap U} \psi \leq \sigma^{1 - d} \int_{\Gamma_+} \psi = \sigma^{1 - d} \int_{W_0} \psi.$$
Since $\psi$ is the euclidean volume form on $W_0$, and $W_0$ is a $d-1$-ball whose euclidean radius is $\leq \sigma + O(\sigma^3)$, it follows that
\begin{equation}\label{K3 calculus 2}
K_3 \leq |\Ball^{d - 1}| + O(\sigma^2).
\end{equation}
By Corollary \ref{doubling dimension}, the right-hand side of (\ref{K3 calculus 2}) is $\leq K_4 + O(\gamma^{1/d})$.

Adding up all the $K_i$, we finally conclude that $J_3 \lesssim \gamma^{1/d}$ and hence $I_1 + I_2 + I_3 - I_4 \lesssim \gamma^{1/(2d)}$.
Plugging this back into (\ref{begin breaking up the mollifier}),
$$(1_V(|\dif u| - X u))_\varepsilon(x) \lesssim (\delta\varepsilon)^{d - 1} \gamma^{1/(2d)} \inf_{y \in B(Q, \delta\varepsilon)} \chi_\varepsilon(x - y).$$
We finally apply Corollary \ref{doubling dimension} to prove
\begin{align*}
(\delta\varepsilon)^{d - 1} \inf_{y \in B(Q, \delta\varepsilon)} \chi_\varepsilon(x - y)
&\lesssim \int_{B(Q, \delta \varepsilon)} \chi_\varepsilon(x - y) |\dif u|(y) \dif y = |\dif u|_\varepsilon(x). \qedhere
\end{align*}
\end{proof}

\begin{proof}[Proof of Lemma \ref{main mollifier lemma}]
We begin by replacing $X$ with $\tilde X$.
Since $X = \tilde X + O(\varepsilon^2)$ on $B(P, 2\varepsilon)$, we obtain
$$((X - \tilde X)u)_\varepsilon(x) = \int_{\mathcal B(x, \varepsilon)} (X - \tilde X)u(y) \chi_\varepsilon(x - y) \dif y \lesssim \gamma^8 \int_{\mathcal B(x, \varepsilon)} \chi_\varepsilon(x - y) |\dif u|(y) \dif y.$$
In particular,  
$$(|\dif u| - Xu)_\varepsilon(x) = (|\dif u| - \tilde Xu)_\varepsilon(x) + \gamma^8 |\dif u|_\varepsilon(x).$$
By construction, $\supp \dif u \subseteq \bigcup_n V_n$, so
\begin{equation}\label{sum over cover}
(|\dif u| - \tilde Xu)_\varepsilon \leq \sum_{n = 0}^N (1_{V_n} (|\dif u| - \tilde Xu))_\varepsilon.
\end{equation}

It remains to estimate $|\dif u| - \tilde Xu$ on $V_0$.
As on \cite[92]{Giusti77}, $\chi_\varepsilon(x - y) \lesssim \varepsilon^{-d} \delta$ for $y \in V_0$; moreover, $V_0 \subseteq B(P, \varepsilon)$, so by Corollary \ref{doubling dimension},
$$\int_{V_0} (|\dif u| - \tilde Xu)(y) \chi_\varepsilon(x - y) \dif y \lesssim \varepsilon^{-d} \delta \int_{B(x, \varepsilon)} |\dif u|(y) \dif y \lesssim \frac{\delta}{\varepsilon}.$$
If $u(x) \in (O(\gamma^2), 1 - O(\gamma^2))$, then by \cite[Lemma 7.1]{Giusti77}, there exists $z \in \partial^* U \cap B(x, (1 - \gamma)\varepsilon)$.
In particular, $B(z, \gamma\varepsilon/2) \subseteq B(x, (1 - \gamma/2)\varepsilon)$ so on $B(z, \gamma\varepsilon/2)$, $\chi_\varepsilon(x - y) \gtrsim \varepsilon^{-d} \gamma$.
So 
$$\frac{\delta}{\varepsilon} \lesssim \gamma \inf_{y \in B(x, \varepsilon)} \chi_\varepsilon(x - y) \int_{B(x, \varepsilon)} |\dif u(y)| \dif y \leq \gamma |\dif u|_\varepsilon(x).$$
Plugging this estimate and Lemma \ref{mollifier sublemma} into (\ref{sum over cover}), we can apply the efficiency of $\mathcal V$ to see that on
$$\Omega := B(P, \rho - 2\sigma) \cap \{O(\gamma^2) < \varphi < 1 - O(\gamma^2)\}$$
we have
$$(|\dif u| - Xu)_\varepsilon \lesssim \gamma^{1/(2d)} |\dif u|_\varepsilon$$
which implies (\ref{claim on main mollifier lemma}).
In particular, near $\Omega$, one has $|\dif \varphi| > 0$.
Therefore $\varphi$ is a $C^1$ submersion by \cite[Lemma 7.1]{Giusti77}, which completes the proof.
\end{proof}

%%%%%%%%%%%%%%%%%%
\subsection{Application to the de Giorgi lemma}
We now can complete the proof of the de Giorgi lemma.
Aside from the fact that we use compactness-and-contradiction to obtain the result for a set of least perimeter, rather than a sequence of sets of least perimeter as in \cite{Giusti77, Miranda66}, the proof is almost identical to \cite[Lemma 7.5]{Giusti77}, so we just sketch it.

\begin{proof}[Proof of Lemma \ref{single mollify}]
Suppose not.
Then there exist balls $B_n := B(P_n, \rho_n)$ and sets $U_n$ of least perimeter in $B_n$ such that
\begin{equation}\label{single mollify Excess assumption}
\gamma_n := \rho_n^{1 - d} \Exc_{\rho_n} (U_n, P_n) \leq n^{-2},
\end{equation}
and $\rho_n \leq 1/n$, but such that for every open set $V_n \subseteq B(P_n, (1 - \varepsilon) \rho_n)$ with $C^1$ boundary, and every $P_n$-aligned vector field $X_n$, at least one of (\ref{single mollify normal}), (\ref{single mollify minimality}), or (\ref{single mollify excess}) fail.

Now let $w_n := (u_n)_{\gamma_n^4 \rho_n}$, where the convolution was taken with respect to normal coordinates based at $P_n$.
Draw $t \in [1 - c_1, 1]$ uniformly at random.
Applying Lemma \ref{main mollifier lemma} with $\gamma := n^{-2}$ and $\rho := \rho_n$ (so $\gamma \geq \rho^2$), we find $a_n = O(\gamma_n^2)$, $b_n = 1 - O(\gamma_n^2)$, and a $P_n$-aligned vector field $X_n$ such that for $y \in (a_n, b_n)$, the level sets $\{w_n = y\}$ have normal vector $C^0$-close to $X_n$.
In particular, by the coarea formula,
$$\int_{tB_n} \star |\dif w_n| = \int_0^1 |\partial^* \{w_n > y\} \cap tB_n| \dif y \geq \int_{a_n}^{b_n} |\partial^* \{w_n > y\} \cap tB_n| \dif y,$$
so by the mean value theorem, there exists $y_n \in (a_n, b_n)$ such that
$$|\partial^* \{w_n > y_n\} \cap tB_n| \leq \frac{1}{b_n - a_n} \int_{tB_n} \star |\dif w_n|.$$
We then set $V_n := \{w_n > y_n\}$, $v_n := 1_{V_n}$, so that $V_n$ has $C^1$ boundary in $tB_n$ and
\begin{equation}\label{MVT mollifier}
|V_n \cap tB_n| \leq \frac{1}{b_n - a_n} \int_{tB_n} \star |\dif w_n|.
\end{equation}
By (\ref{claim on main mollifier lemma}), $(\normal_{V_n}, X_n) \geq 1 - O(n^{-1/d})$, so for $n$ large enough depending on $c_1$, $V_n$ satisfies (\ref{single mollify normal}).

Let $\Gamma_n := \partial(tB_n)$ and draw $s \in [0, t]$ at random.
Arguing completely analogously to the proofs of \cite[(7.23), (7.22)]{Giusti77}, respectively, we see that almost surely,
\begin{align}
\|u_n - v_n\|_{L^1(\Gamma_n)} &\ll \gamma_n \label{trace of vn} \\
|\partial V_n \cap sB_n| &\leq |\partial^* U_n \cap sB_n| + \gamma_n. \label{difference of surface area}
\end{align}
The conjunction of (\ref{trace of vn}), (\ref{difference of surface area}), and (\ref{a priori estimate 1}) implies
\begin{equation}
||\partial^* U_n \cap tB_n| - |\partial V_n \cap tB_n|| \ll \gamma_n, \label{mollifier quant2}
\end{equation}
and the conjunction of (\ref{mollifier quant2}), (\ref{a priori estimate 1}), the fact that $U_n$ has least perimeter, (\ref{single mollify Excess assumption}) and (\ref{trace of vn}) implies (\ref{single mollify minimality}) if $n$ is large enough depending on $c_1$.

To derive a contradiction, we must show that $V_n$ satisfies (\ref{single mollify excess}) for $n$ large enough depending on $c_1$.
Let $\varpi \leq t\rho$, and estimate
\begin{align*}
    |\Exc_\varpi(U_n, P_n) - \Exc_\varpi(V_n, P_n)|
    &\leq ||\partial^* U_n \cap t B_n| - |\partial V_n \cap t B_n||\\
    &\qquad + \left|\left[\int_{B(P_n, \varpi)} \star \partial_\mu(u_n - v_n) \right] \dif x^\mu(P_n)\right| + O(\rho_n^{d + 1}) \\
    &=: I_1 + I_2 + I_3.
\end{align*}
By (\ref{mollifier quant2}), $I_1 \leq c_1 \Exc_\varpi(U_n, P_n)/3$ if $n$ is large, and $I_3$ is irrelevant.
By Stokes' theorem and (\ref{trace of vn}), if $n$ is large then
\begin{align*}
    I_2 &\leq \left|\left[\int_{\partial B(P, \varpi)} (\normal_{B_\varpi})_\mu (u_n - v_n) \dif S_{\partial B(P, \varpi)}\right] \dif x^\mu(P_n)\right| \leq \frac{c_1}{3} \Exc_\varpi(U_n, P_n) + O(\rho_n^{d + 1}).
\end{align*}
This implies (\ref{single mollify excess}) for $n$ large and so contradicts our assumptions.
\end{proof}

%%%%%%%%%
\chapter{Level sets of functions of least gradient and minimal laminations}\label{convlams}
\section{Introduction}
The space of codimension-$1$ minimal laminations on a Riemannian manifold has been topologized in several different ways.
Thurston \cite[Chapter 8]{thurston1979geometry} introduced both his geometric topology as well as the weak topology of measures on the space of measured geodesic laminations.
Independently of Thurston, Colding and Minicozzi \cite[Appendix B]{ColdingMinicozziIV} introduced a topology that emphasized not the laminations themselves, but rather the coordinate charts which flatten them.

We establish a regularity theorem for minimal laminations, which implies compactness properties for the aforementioned topologies.
We also show that a current is Ruelle-Sullivan with respect to a minimal lamination if and only if it is locally the exterior derivative of a function of least gradient, generalizing a theorem of Daskalopoulos and Uhlenbeck \cite[Theorem 6.1]{daskalopoulos2020transverse} and strengthing an unpublished result of Auer and Bangert \cite{Auer01, Auer12}.

%%%%%%%%%%%%%%%%%%
\subsection{Regularity of minimal laminations}
The definition of a minimal lamination are tedious to work with, both because one has to prove the existence of flow boxes which flatten sets which may be extremely rough, and because one has no quantitative control on said flow boxes.
However, if we have curvature bounds on the leaves and on the underlying manifold $M$, our first main theorem drastically changes the story: it shows that the lamination $\lambda$ can be reconstructed from its set of leaves, in such a way that the flow boxes for $\lambda$ are under control in the Lipschitz and tangentially $C^\infty$ sense.
Here, \dfn{tangential $C^\infty$} is the topology defined by seminorms $f \mapsto \|\nabla_N^m f\|_{C^0}$, where $N$ ranges over leaves of the given lamination $\lambda$, $\nabla_N$ is the Levi-Civita connection on $N$, $m$ ranges over $\NN$ (including $0$); thus we ignore derivatives normal to $N$.

\begin{theorem}\label{regularity theorem}
Let $K := \|\Riem_M\|_{C^0}$ and let $i$ be the injectivity radius of $M$, and suppose that $K < \infty$, $i > 0$.
Let $\mathcal S$ be a set of disjoint minimal hypersurfaces in $M$, such that for every $N \in \mathcal S$,
\begin{equation}\label{curvature bound in regularity}
	\|\Two_N\|_{C^0} \leq A,
\end{equation}
and that $\bigcup_{N \in \mathcal S} N$ is a closed subset of $M$. Then:
\begin{enumerate}
\item There exists a Lipschitz minimal lamination $\lambda$ whose leaves are exactly the elements of $\mathcal S$.
\item There exists a Lipschitz line bundle on $M$ which is normal to every leaf of $\lambda$.
\item There exist constants $L = L(A, K, i) > 0$ and $r = r(A, K, i) > 0$, and a Lipschitz laminar atlas $(F_\alpha)$ for $\lambda$, such that for every $\alpha$,
\begin{equation}\label{conorm of flow box}
	\max(\Lip(F_\alpha), \Lip(F_\alpha^{-1})) \leq L,
\end{equation}
and the image of $F_\alpha$ contains a ball of radius $r$.
\item $F_\alpha$ and $F_\alpha^{-1}$ are tangentially $C^\infty$, with seminorms only depending on $A, K, i$.
\end{enumerate}
\end{theorem}

In the remainder of this paper we prove two consequences of Theorem \ref{regularity theorem}: a characterization of minimal laminations (Theorem \ref{main thm}) and a compactness theorem (Theorem \ref{compactness theorem}), which we state below.
In both theorems, a bound on the curvature will be necessary in order to invoke Theorem \ref{regularity theorem}.

\begin{definition}
A sequence $(\lambda_n)$ of laminations has \dfn{bounded curvature} if there exists $C > 0$ such that for any $n$ and any leaf $N$ of $\lambda_n$, the second fundamental form satisfies $\|\Two_N\|_{C^0} \leq C$.
\end{definition}

Several similar results to Theorem \ref{regularity theorem} have appeared in the literature already, but Theorem \ref{regularity theorem} strengthens and clarifies them.
To our knowledge, the first related result is due to Solomon \cite[Theorem 1.1]{Solomon86}, which we improve on in several ways:
\begin{enumerate}
\item \label{foliation to lamination} Solomon's proof is for minimal foliations in $\RR^d$.
\item We obtain estimates which only depend on the curvatures of the leaves and $M$, and on the injectivity radius $i$; they do not depend on the regularity of a given $C^0$ laminar atlas.
\item In fact, we do not even assume the existence of a $C^0$ laminar atlas.
\end{enumerate}
As Solomon notes, it is easy to extend his proof to minimal foliations in a Riemannian manifold $M$; the key point of (\ref{foliation to lamination}) is that we would like Theorem \ref{regularity theorem} to be true for minimal \emph{laminations}.

Our work is closest to a compactness theorem due to Colding and Minicozzi for minimal laminations of a Riemannian manifold \cite[Appendix B]{ColdingMinicozziIV}.
Their new idea is to fill in the gaps between the leaves in Solomon's constructions by linear interpolation.
However, Colding and Minicozzi assume that the laminations have finitely many leaves, and that the curvature bound (\ref{curvature bound in regularity}) implies that all of the leaves can be represented as graphs at once.
Indeed, \emph {a priori}, the leaves could fail to be close to parallel, and then it would not be possible to construct a coordinate chart in which they are all graphs.

We eliminate such assumptions by showing that members of $\mathcal S$ must be ``close to parallel on small scales'', where the scale is governed by $A, K$.
Otherwise, since the scale is small, we may replace the elements of $\mathcal S$ by their tangent spaces, which would then intersect, contradicting the disjointness of $\mathcal S$.
This approach was already suggested by Thurston \cite[\S8.5]{thurston1979geometry} in the case of geodesic laminations, though he omitted the details. 

Using completely different techniques, Daskalopoulos and Uhlenbeck \cite[Proposition 7.3]{daskalopoulos2020transverse} obtained a version of Theorem \ref{regularity theorem} without any $C^0$ dependence, under the assumption that $M$ is a closed hyperbolic surface.
The key point of their argument is that the exponential map sends lines to geodesics, so it provides a much shorter proof of Theorem \ref{regularity theorem}, at the price of only working in dimension $2$.


%%%%%%%%%%%%%%%%%%
\subsection{Applications to \texorpdfstring{$1$-harmonic}{one-harmonic} functions}\label{FLG section}
The main result of this paper realizes the Ruelle-Sullivan current of a minimal lamination as the exterior derivative of a function of locally least gradient, and vice versa.


\begin{theorem}\label{main thm}
Suppose that $2 \leq d \leq 7$.
\begin{enumerate}
\item Let $u$ be a function of locally least gradient on $M$.
Then:
\begin{enumerate}
\item $\bigcup_{y \in \RR} \partial \{u > y\} \cup \partial \{u < y\}$ is the support of a Lipschitz minimal lamination $\lambda$.
\item The leaves of $\lambda$ are exactly the connected components of $\partial \{u > y\}$ or $\partial \{u < y\}$, with $y$ ranging over $\RR$.
\item There exists a measured oriented structure on $\lambda$ whose Ruelle-Sullivan current is $\dif u$.
\end{enumerate}
\item Conversely, if $H^1(M, \RR) = 0$ and $\lambda$ is a minimal measured oriented minimal lamination, then:
\begin{enumerate}
\item If $\lambda$ has bounded curvature, then any primitive $u$ of the Ruelle-Sullivan current of $\lambda$ has locally least gradient.
\item If the leaves of $\lambda$ are absolutely area-minimizing, then $u$ has least gradient.
\end{enumerate}
\end{enumerate}
\end{theorem}

In general, we must use both sublevel sets and superlevel sets in the statement of Theorem \ref{main thm}.
For example, the function
$$u(x, y) := 1_{x \leq 0} x$$
has least gradient on $\Ball^2$.
Then $\{x = 0\}$ is a leaf of the lamination and bounds $\{u < 0\}$ but does not bound a superlevel set.
However, the set of leaves arising from sublevel sets but not superlevel sets is countable, and we can do away with sublevel sets entirely if we assume that $\dif u$ has full support \cite[Lemma 2.11]{górny2018}.

The main ingredients in the proof of Theorem \ref{main thm} are Theorem \ref{regularity theorem}, the regularity theory of minimal hypersurfaces, and curvature estimates on stable minimal hypersurfaces due to Schoen, Simon, and Yau \cite{Schoen75,Schoen81}.
With these ingredients in place, it remains to show that the stability radii of the level sets of a function of locally least gradient are bounded from below, and locally the area of the level sets is bounded from above; this gives uniform curvature estimates on the level sets.

A similar result to Theorem \ref{main thm}, proven with somewhat different methods, was announced but never published by Auer and Bangert \cite{Auer01, Auer12}, who claimed to establish that a locally minimal $d - 1$-current is Ruelle-Sullivan for a lamination in a weaker sense than ours.
In particular, it does not seem that one can extract Lipschitz regularity directly from their methods.

Our motivation for Theorem \ref{main thm} is to generalize the work of Daskalopoulos and Uhlenbeck on $\infty$-harmonic maps from a closed hyperbolic surface to $\Sph^1$ \cite{daskalopoulos2020transverse}, which associates to each such map a geodesic lamination $\lambda$ and function $v$ of locally least gradient on the universal cover such that $\dif v$ drops to a Ruelle-Sullivan current for a sublamination of $\lambda$.
Inspired by this theorem, Daskalopoulos and Uhlenbeck conjectured that for any function of locally least gradient on the hyperbolic plane $\Hyp^2$, $\dif u$ should be Ruelle-Sullivan for some (possibly not maximum-stretch) geodesic lamination \cite[Problem 9.4]{daskalopoulos2020transverse}, and conversely that if $T$ is a Ruelle-Sullivan current for some geodesic lamination, then local primitives of $T$ have locally least gradient \cite[Conjecture 9.5]{daskalopoulos2020transverse}.
Of course such results are special cases of Theorem \ref{main thm}.

We stress that Theorem \ref{main thm} is an interior result.
We allow $M$ to have a boundary, infinite ends, or punctures, but do not study the limiting behavior of the lamination near those points.
The behavior of functions of least gradient near an infinite end is heavily constrained by the global behavior of minimal hypersurfaces \cite[\S4.4]{górny2021}, which is outside the scope of this paper.

In \S\ref{1harmonic apps} we use Theorem \ref{main thm} prove a generalization of G\'orny's decomposition of functions of least gradient \cite[Theorem 1.2]{górny2017planar} to our setting.
A simplified version of the statement is as follows:

\begin{corollary}
Let $u$ be a function of locally least gradient.
Then we can locally write $u$ as the sum of an absolutely continuous function of least gradient, a Cantor function of least gradient, and a jump function of least gradient.
\end{corollary}

% We would also like to highlight a possible further direction of study which we shall not address here.
% The associated parabolic flow to (\ref{1Laplacian}),
% \begin{equation}\label{level set flow}
% \partial_t u = |\nabla u| \nabla \cdot \left(\frac{\nabla u}{|\nabla u|}\right),
% \end{equation}
% is known as \dfn{level set flow} and acts on the level sets of $u$ by mean curvature flow.
% As such, it arises as a model of interfaces with minimal area, as a means of continuing mean curvature flow past its singular times, and in the \dfn{level set method} of computing minimal surfaces \cite{Chen89,Thomas05}.
% Previous work on the level set flow has been especially concerned with the evolution of the hypersurface $N(t) := \{u(t) = 0\}$ under the assumption that $N(0)$ is mean convex and $u$ is a (necessarily continuous) viscosity solution of (\ref{level set flow}) \cite{Evans91,Colding2016RegularityOT,sun2022generic}.
% It would be very interesting to prove a parabolic version of Theorem \ref{main thm} which asserts that under appropriate hypotheses on a measured oriented lamination $\lambda$ with Ruelle-Sullivan current $\dif u$, the level set flow of $u$ corresponds to a flow of $\lambda$ by mean curvature flow, even if $u$ is discontinuous.

%%%%%%%%%%%%%%%%%%
\subsection{Spaces of minimal laminations}\label{LamSpace section}
In the literature, there are at least three different topologies on the space of laminations on a Riemannian manifold $M$, which we now recall.

Thurston's geometric topology \cite[Chapter 8]{thurston1979geometry} says that a lamination $\lambda'$ is close to a lamination $\lambda$ if every leaf of $\lambda$ is close to a leaf of $\lambda'$ at least locally, and the same holds for their normal vectors $\normal$.

\begin{definition}
We define the basic open sets in \dfn{Thurston's geometric topology} to be defined by a lamination $\lambda$, $x \in M$, and $\varepsilon > 0$: the basic open set $\mathscr N(\lambda, x, \varepsilon)$ is the set of all laminations $\kappa$ such that there exists $y \in \supp \kappa \cap B(x, \varepsilon)$ such that the normal vectors are close: $\dist(\normal_\lambda(x), \normal_\kappa(y)) < \varepsilon$.
\end{definition}

A sequence of laminations $(\lambda_i)$ converges to a lamination $\lambda$ in Thurston's geometric topology iff, for every leaf $N$ of $\lambda$, every $x \in N$, and every $\varepsilon > 0$, there exists $i_{\varepsilon, x} \in \NN$ such that for every $i \geq i_{\varepsilon, x}$, $\supp \lambda_i$ intersects $B(x, \varepsilon)$, and for $x_i \in B(x, \varepsilon) \cap \supp \lambda_i$,
$$\dist_{SM}(\normal_{\lambda_i}(x_i), \normal_\lambda(x)) < 2\varepsilon.$$
It is straightforward to show that Thurston's geometric topology does not depend on the choice of Riemannian metric on $M$, or the choice of extension of the distance function on $M$ to its sphere bundle $SM$, which are implicit in the statement thereof.
However, the limiting lamination is not unique, as if $\lambda_i \to \lambda$ and $\lambda'$ is a sublamination of $\lambda$, then $\lambda_i \to \lambda'$.
In particular, Thurston's topology is not Hausdorff, and we say that $\lambda$ is a \dfn{maximal limit} of a sequence $(\lambda_i)$ if $\lambda_i \to \lambda$ and for every $\lambda'$ such that $\lambda_i \to \lambda'$, $\lambda'$ is a sublamination of $\lambda$.

Independently of Thurston, Colding and Minicozzi \cite[Appendix B]{ColdingMinicozziIV} defined a sequence of laminations to converge ``if the corresponding coordinate maps converge;'' that is, if the laminar atlases converge.
This of course says nothing about the limiting set of leaves and in the sequel paper \cite{ColdingMinicozziV} they additionally impose that the sets of leaves converge ``as sets.''

In this paper we consider a similar condition to the one in \cite{ColdingMinicozziV}, which we believe to be more natural: that the laminar atlases converge and that the laminations themselves converge in Thurston's geometric topology.
To be more precise:

\begin{definition}
A sequence $(\lambda_i)$ of laminations \dfn{flow-box converges} in a function space $X$ to $\lambda$ if it converges in Thurston's geometric topology, and there exists a laminar atlas $(F_\alpha)$ for $\lambda$ such that for each $\alpha$, $F_\alpha$ and $(F_\alpha)^{-1}$ are limits in $X$ of flow boxes $F_\alpha^i$, $(F_\alpha^i)^{-1}$ in laminar atlases for $\lambda_i$.
\end{definition}

The notion of flow-box convergence is mainly useful for tangential $C^\infty$ and the Fr\'echet space $C^{1-} := \bigcap_{0 \leq \theta < 1} C^\theta$, where $C^\theta$ are H\"older spaces.

Next we recall convergence of laminations equipped with transverse measures.
We remind the reader that in \S\ref{RS prelims} we define the Ruelle-Sullivan current of a possibly nonorientable measured lamination.

\begin{definition}
A sequence of measured laminations $(\lambda_i, \mu_i)$ \dfn{converges} to $(\lambda, \mu)$ if their Ruelle-Sullivan currents $T_{\mu_i} \to T_\mu$ converge in the weak topology of measures.
\end{definition}

Filling in some of the details of the argument of Colding and Minicozzi \cite[Appendix B]{ColdingMinicozziIV}, it follows from the regularity theorem, Theorem \ref{regularity theorem}, that once we have a bound on the curvatures of the leaves, every sequence of laminations has convergent subsequences in each of the above modes of convergence.

\begin{theorem}\label{compactness theorem}
Let $(\lambda_n)$ be a sequence of minimal laminations of bounded curvature, and let $E \subseteq M$ be a compact set. Then:
\begin{enumerate}
\item Suppose that for every $n$ and every leaf $N$ of $\lambda_n$, $N \cap E$ is nonempty. Then a subsequence of $(\lambda_n)$ converges in the $C^{1-}$ and tangentially $C^\infty$ flow box topology, and in particular in Thurston's geometric topology, to a minimal lamination.
\item Suppose that, in addition, $\mu_n$ is transverse to $\lambda_n$, there exists $\varepsilon > 0$ such that $\mu_n(E) > \varepsilon$, and there exists $C > 0$ such that $\mu_n(M) \leq C$, then a further subsequence converges in the measure topology.
\end{enumerate}
\end{theorem}

In \S\ref{relationships between modes}, we use Theorem \ref{compactness theorem} to explain how the above modes of convergence are related.
Here is a (not quite sharp) statement of these results:

\begin{corollary}
Let $(\lambda_n, \mu_n)$ be a sequence of measured minimal laminations of bounded curvature and $(\lambda, \mu)$ a measured minimal lamination.
If $d \leq 7$ and $(\lambda_n, \mu_n) \to (\lambda, \mu)$, then $\lambda_n \to \lambda$ in the $C^{1-}$ and tangential $C^\infty$ flow box topologies, hence in Thurston's geometric topology.
\end{corollary}

%%%%%%%%%%%%%%%%%%%%%%%
\subsection{Outline of the paper}
The rest of the paper is organized as follows:
\begin{itemize}
\item In \S\ref{Regularity}, we prove the regularity theorem, Theorem \ref{regularity theorem}.
\item In \S\ref{1harmonic sec}, we prove the equivalence of $1$-harmonic functions and measured oriented minimal laminations, Theorem \ref{main thm}, and apply it to study $1$-harmonic functions. This section relies on \S\ref{Regularity}, \S\ref{Prelims}, and Appendices \ref{boundary appendix} and \ref{locally minimizing appendix}.
\item In \S\ref{CompactnessSec}, we prove the compactness theorem, Theorem \ref{compactness theorem}, and explore the consequences for how the different modes of convergence are related to each other. This section applies \S\ref{Regularity}, \S\ref{Prelims}, and Appendix \ref{boundary appendix} for the proof of Theorem \ref{compactness theorem}, but the consequences of it also apply \S\ref{1harmonic sec}.
\item In Appendix \ref{boundary appendix}, we recall various technical results of geometric measure theory that we shall need.
\item In Appendix \ref{locally minimizing appendix}, we give a short proof that the radius of a ball in which a minimal hypersurface is absolutely area-minimizing is controlled from below by the curvature. The proof applies both \S\ref{Regularity} and \S\ref{Prelims}.
\end{itemize}



%%%%%%%%%%%%%%%%%%%%%%%%%%%%%%%%%%%%%%%%%%
\section{Regularity of laminations}\label{Regularity}

\subsection{A preliminary choice of coordinates}
We now construct normal coordinates in which the leaves of $\lambda$ are $C^1$-close to hyperplanes $\{y = y_0\}$.
The utility of this fact is that, if $f: \RR^{d - 1}_x \to \RR_y$, and its graph has normal vector $\normal$, then
\begin{equation}\label{nabla as a normal}
	\normal = \frac{\partial_y f - \nabla f}{\sqrt{1 + |\nabla f|^2}}.
\end{equation}
So if $Pf = 0$, then the leaves of $\lambda$ are minimal graphs which are small in $C^1$ and so we may apply (\ref{Schauder Harnack}) uniformly among all of the leaves at once.

A similar result was proven by \cite{Solomon86} (without the quantitative dependence) using the regularity theory for integral flat convergence of minimal currents \cite[Theorem 5.3.14]{federer2014geometric}.
We did not do this because it does not seem particularly easy to recover quantitative bounds from the highly general theory of \cite[Chapter 5]{federer2014geometric}.

\begin{lemma}\label{existence of tubes}
	Let $N$ be an embedded $C^2$ hypersurface in $\RR^d = \RR^{d - 1}_x \times \RR_y$ which is tangent to $\{y = 0\}$ at the origin.
	If $\|\Two_N\|_{C^0} \leq \frac{1}{8}$, then the connected component of $N \cap B(0, 1)$ containing $0$ is the graph over $\{y = 0\}$ of a function $f$ with
	$$|f(x)| \leq \|\Two_N\|_{C^0} |x|^2.$$
\end{lemma}
\begin{proof}
	Near $0$, $N$ can be represented a graph $\{y = f(x)\}$, since it is tangent to $\{y = 0\}$.
	This representation is valid on the component of the set $\{|\nabla f(x)| < \infty\}$ containing $0$, and it is related to the unit normal by (\ref{nabla as a normal}).
	Rearranging (\ref{nabla as a normal}) and taking derivatives,
	$$-\nabla^2 f(x) = \frac{\nabla \normal(x, f(x)) \cdot (\partial_x \otimes \partial_x + \nabla f(x) \otimes \partial_y)}{\sqrt{1 + |\nabla f(x)|^2}} - \frac{\nabla^2 f(x) \cdot (\nabla f(x) \otimes \normal(x, f(x)))}{(1 + |\nabla f|^2)^{3/2}}.$$
	Here $-\nabla^2$ denotes the negative Hessian, not the Laplacian.
	Since
	$$|\partial_x \otimes \partial_x + \nabla f(x) \otimes \partial_y| \leq \sqrt{1 + |\nabla f(x)|^2},$$
	and $\nabla \normal = \Two_N$, we conclude
\begin{equation}\label{bound Hessian by Two}
	|\nabla^2 f(x)| \leq |\Two_N(x, f(x))| + |\nabla^2 f(x)| |\nabla f(x)|.
\end{equation}
	In order to control the error terms in (\ref{bound Hessian by Two}), we make the \dfn{bootstrap assumption}
\begin{equation}\label{bootstrap}
	|\nabla f(x)| \leq \frac{1}{2},
\end{equation}
	which is at least valid in some small neighborhood $B_R$ of $0$ since (\ref{nabla as a normal}) and the fact that $N$ is tangent to $\{y = 0\}$ at $0$ imply that $\nabla f(0) = 0$.
	By (\ref{bound Hessian by Two}),
$$|\nabla^2 f(x)| \leq 2|\Two_N(x, f(x))|,$$
	and integrating this inequality one obtains for $|x| \leq R$ that
\begin{equation}\label{closed bootstrap}
	|\nabla f(x)| \leq 2|\Two_N(x, f(x))| |x| \leq \frac{1}{4}.
\end{equation}
	In particular, since $\nabla f \in C^1$, either $R \geq 1$ or there exists $R' > R$ such that the bootstrap assumption (\ref{bootstrap}) is valid on $B_{R'}$.
	Therefore (\ref{bootstrap}) is valid with $R = 1$.
	Integrating (\ref{closed bootstrap}), we obtain the desired conclusion.
\end{proof}


\begin{lemma}\label{lams have C0 fields}
	Suppose that $\delta > 0$ is small enough depending on $K$.
	Then there exists $r = r(\delta, K, i, A) > 0$ such that for every disjoint family of hypersurfaces $\mathcal S$ satisfying the curvature bound (\ref{curvature bound in regularity}) and every $p \in \bigcup_{N \in \mathcal S} N$, we can choose normal coordinates $(x, y) \in \RR^{d - 1} \times \RR$ based at $p$ so that
\begin{equation}\label{normal is basically dy}
	\sup_{N \in \mathcal S} \|\normal_\lambda - \partial_y\|_{C^0(B(p, r))} \leq \delta.
\end{equation}
\end{lemma}
\begin{proof}
Consider normal coordinates $(x, y)$ based at $p$, and write $\Two_N'$ for the second fundamental form of $N \in \mathcal S$ taken with respect to the euclidean metric from those coordinates, $\normal_N'$ the euclidean normal, $\nabla'$ the euclidean Levi-Civita connection, and $\Gamma$ the Christoffel symbols.
In particular, since $\normal_N^\flat$ is the conormal and satisfies $\normal_N^\flat = (\normal_N')^\flat/|\normal_N'|$, 
$$\Two_N' = \nabla' (\normal_N')^\flat = (\nabla - \Gamma) |\normal_N'| \normal_N^\flat = |\normal_N'| (\Two_N - |\normal_N'| \Gamma \otimes \normal_N^\flat) + \nabla' \normal_N' \otimes \normal_N^\flat.$$
Using estimates on normal coordinates we conclude that for every $0 < s < i$ and some absolute $C > 0$,
$$\|\Two_N'\|_{B(p, s)} \leq A + CKs.$$
After rescaling we may assume that $A \leq 1/16$, $K \leq 1/(32C)$, and $i \geq 2$, so $\|\Two_N'\|_{C^0(B(p, 2))} \leq 1/8$.
Then we apply Lemma \ref{existence of tubes}: for $q \in N \cap B(p, 1)$, $B(q, 1) \subseteq B(p, 2)$, and $\tilde x$ the euclidean coordinate on $T_q N$ induced by the normal coordinates $(x, y)$, $N \cap B(q, 1)$ is the graph of a function $f$ on $T_q N$ satisfying
\begin{equation}\label{living in a tube}
|f(\tilde x)| \leq A|\tilde x|^2.
\end{equation}
Here, and for the remainder of this proof, we use $|\cdot|$ to mean the euclidean metric only.

Let $0 < r < s\delta^2$ for some small absolute $s > 0$ to be chosen later, and suppose that for every choice of normal coordinates $(x, y)$ at $p$, (\ref{normal is basically dy}) fails.
Then, since every coordinate system fails to have the desired properties, we might as well choose one such that for some $N \in \mathcal S$ and some $q \in B(p, r) \cap N$, $\normal_N(q)$ is a scalar multiple of $\partial_y$.
By the contradiction assumption, we can choose $N' \in \mathcal S$ and $q \in B(p, r) \cap N'$ such that
$$|\normal_{N'}(q') - \partial_y| > \delta.$$
In particular, since $|\normal_{N'}(q')| = 1 + O(r^2)$ and $|\partial_y| = 1$, the angle $\theta$ between these two vectors is given by the law of cosines as 
$$1^2 + (1 + O(r^2))^2 - 1(1 + O(r^2)) \cos \theta = |\normal_{N'}(q') - \partial_y|^2$$
which can be neatly estimated for $s$ small enough as
$$\cos \theta < 1 - \frac{\delta^2}{2} + O(r^2) \leq 1 - \frac{\delta^2}{4}.$$
But $\theta$ is the angle between the tangent planes $T_q N$ and $T_{q'} N'$.
We consider the triangle $\Delta(q, q', r)$ where $r$ is a point of intersection of $P := T_q N$ and $P' := T_{q'} N'$, so again by the law of cosines, if $\alpha := |q - r|$ and $\beta := |q' - r|$,
$$\alpha^2 + \beta^2 - 2\alpha\beta \cos \theta = |q - q'|^2 \leq r^2.$$
By Young's inequality, it follows that 
$$r^2 \geq (\alpha^2 + \beta^2)(1 - \cos \theta) > (\alpha^2 + \beta^2) \frac{\delta^2}{4}$$
or in other words 
$$\alpha^2 + \beta^2 < \frac{4r^2}{\delta^2} < 4s^2 \delta^2$$
which means for $\delta$ small that $\max(\alpha, \beta) < 2c\delta < s/4$.
Hence $P, P'$ intersect in $B(p, s/4 + r) \subseteq B(p, s/2)$.

Now consider the tubes $\mathcal T, \mathcal T'$ of all points which are within $s^2/16$ of $P, P'$.
Since $P, P'$ intersect in $B(p, s/2)$, if $s$ is small, any graphs over $P, P'$ in $\mathcal T, \mathcal T'$ must intersect in $B(p, s)$.
In particular we can take $s < 1$ and conclude from (\ref{living in a tube}) that $N, N'$ are not disjoint, contradicting the definition of $\mathcal S$.
\end{proof}

\subsection{Proof of Theorem \ref{regularity theorem}}
Fix $\delta > 0$ to be chosen later, and $P \in M$.
Let $\normal$ be the normal vector to the hypersurfaces in $\mathcal S$.
By Lemma \ref{lams have C0 fields}, if $\delta \leq \delta_*$ for some $\delta_* = \delta_*(i, K) > 0$, there exists $r = r(\delta, i, K, A) > 0$ such that $B(P, r)$ admits rescaled normal coordinates $(x, y) \in 5\Ball^{d - 1} \times (-2, 2)$ in which the curvature of the rescaled metric has a $C^0$ norm $\leq K_0$ and
\begin{equation}\label{normal is almost constant}
\|\normal - \partial_y\|_{C^0(B(P, r))} \leq \delta.
\end{equation}
Moreover,
$$|\normal \cdot \partial_y| \geq 1 - |\normal - \partial_y| \geq 1 - \delta,$$
so if we select $\delta := \min(\delta_*, \frac{1}{4})$, then in $5\Ball^{d - 1} \times (-1, 1)$, then every leaf is the graph of a function, say $u_k: 5\Ball^{d - 1} \to (-2, 2)$ where $u_k(0) = k$, and
$$\|\dif u_k\|_{C^0} \leq \frac{1 - (1 - \delta)^2}{1 - \delta} \leq 1.$$
If $r$ is chosen small enough depending on $g$, then the metric $\tilde g$ induced by $g$ on $5\Ball^{d - 1} \times (-2, 2)$ satisfies $\|\Riem_{\tilde g}\|_{C^0} \leq K_0$.
Moreover, $\|u_k\|_{C^0} \leq 2$, and $u_k$ has a minimal graph, so the elliptic estimates stated in \S\ref{Leaf estimates} apply to $u_k$ uniformly in $k$.

Now let $-1 < k < \ell < 1$, and let $v_{\ell k} := u_\ell - u_k$.
By (\ref{Schauder Harnack}) with $v := v_{\ell k}$, for every $x \in \Ball^{d - 1}$,
\begin{equation}\label{bound on du}
|\dif u_\ell(x) - \dif u_k(x)| \lesssim |u_\ell(x) - u_k(x)|
\end{equation}
and it follows that
\begin{equation}\label{vertical Lipschitz}
|\normal(x, u_\ell(x)) - \normal(x, u_k(x))| \lesssim |u_\ell(x) - u_k(x)|.
\end{equation}

To extend (\ref{vertical Lipschitz}) to a Lipschitz bound on $\normal$, let $X_1, X_2 \in (\Ball^{d - 1} \times (-1, 1)) \cap \supp \lambda$.
Then there exist $x_1, x_2 \in \Ball^{d - 1}$ and $k_1, k_2 \in (-1, 1)$ such that $X_i = (x_i, u_{k_i}(x_i))$.
Setting $Y := (x_2, u_{k_1}(x_2))$,
$$|\normal(X_1) - \normal(X_2)| \leq |\normal(X_1) - \normal(Y)| + |\normal(Y) - \normal(X_2)|.$$
Then by (\ref{norms on uk}) and the mean value theorem,
$$|\normal(X_1) - \normal(Y)| \lesssim |\dif u_{k_1}(x_1) - \dif u_{k_1}(x_2)| \lesssim |X_1 - Y|.$$
Moreover, by (\ref{vertical Lipschitz}),
$$|\normal(Y) - \normal(X_2)| \lesssim |u_{k_1}(x) - u_{k_2}(x)| = |Y - X_2|.$$
Since $\delta \leq \frac{1}{4}$, by (\ref{normal is almost constant}),
$$|\sin \angle(X_1 - Y, X_2 - Y)| > 1 - O(\delta)$$
and we conclude by the Pythagorean theorem that
$$|Y - X_2|^2 + |X_1 - Y|^2 \lesssim |X_1 - X_2|^2.$$
In conclusion,
$$|\normal(X_1) - \normal(X_2)| \lesssim |X_1 - X_2|$$
which implies that $\normal$ is Lipschitz on $V \cap \supp \lambda$, where $V$ is the neighborhood of $P$ which was mapped to $\Ball^{d - 1} \times (-1, 1)$ by the cylindrical coordinates $(x, y)$.
In particular, $V$ contains a ball of the form $B(P, s)$, where $s$ only depends on $r$ (and $r$ only depends on $g$ and $A$).
Taking a Lipschitz extension of $\normal$ to $V$ we obtain the desired Lipschitz normal subbundle.

Following \cite[Appendix B]{ColdingMinicozziIV}, we construct the laminar flow box
\begin{align*}
	F: \RR^{d - 1}_\xi \times \RR_\eta &\to V \subseteq \RR^{d - 1}_x \times \RR_y \\
	(\xi, \eta) &\mapsto (\xi, f(\xi, \eta))
\end{align*}
by setting
$$f(\xi, \eta) := u_\eta(\xi)$$
if $u_\eta$ exists, and if $k < \eta < \ell$ and there does not $k < \eta' < \ell$ such that $u_{\eta'}$ exists, then
$$f(\xi, \eta) := u_k(\xi) + \frac{\eta - k}{\ell - k} v_{\ell k}(\xi)$$
is the linear interpolant of $u_k$ and $u_\ell$.

By (\ref{norms on uk}), $F$ is bounded in tangential $C^\infty$.
In particular, if $V$ is a vector field tangent to $\{\eta = k\}$, then the pushforward
$$F_* V = V^i \partial_{x^i} + (Vf) \partial_y$$
is well-defined, and pushforwards of such vector fields span the tangent bundle of the graph of $u_k$. 
The bound
\begin{equation}\label{xiLip of f}
	\|\partial_\xi f\|_{C^0} \lesssim \sup_k \|u_k\|_{C^1} \lesssim 1,
\end{equation}
a consequence of (\ref{norms on uk}), establishes that $\|F_* V\|_{C^0} \sim \|V\|_{C^0}$, and then 
$$\|(F_* V) F^{-1}\|_{C^0} \lesssim \|V(F \circ F^{-1})\|_{C^0} \leq \|V\|_{C^0} \sim \|F_* V\|_{C^0}.$$
Since $V$ was arbitrary we conclude that $F^{-1}$ is bounded in tangential $C^1$, hence in tangential $C^\infty$ by the inverse function theorem. 

It remains to show that $F$ is a Lipschitz isomorphism.
To do this, we first claim that $\Lip(f) \sim 1$.
In the $\xi$ direction, we use (\ref{xiLip of f}).
If $-1 < k < \ell < 1$, then by (\ref{bound on du}) and (\ref{Schauder Harnack}),
\begin{equation}\label{f lip}
	|f(\xi, k) - f(\xi, \ell)| \lesssim |u_k(\xi) - u_\ell(\xi)| \lesssim \ell - k.
\end{equation}
This shows that $f$ is Lipschitz in the $\eta$ direction on the leaves with constant comparable to $1$, and hence on its entire domain by linear interpolation, proving the claim.
We can then estimate using (\ref{f lip})
$$|F(\xi_1, \eta_1) - F(\xi_2, \eta_2)| \lesssim |\xi_1 - \xi_2| + \Lip(f)(|\xi_1 - \xi_2| + |\eta_1 + \eta_2|)$$
so that $\Lip(F) \lesssim 1 + \Lip(f) \lesssim 1$.

To obtain a bound on $\Lip(F^{-1})$, we observe that
\begin{equation}\label{F is coLip in xi}
|\xi_1 - \xi_2|^2
\leq |\xi_1 - \xi_2|^2 + |f(\xi_1, \eta_1) - f(\xi_2, \eta_1)|^2 
= |F(\xi_1, \eta) - F(\xi_2, \eta)|^2.
\end{equation}
By Harnack's inequality with $\eta_1 = k$ and $\eta_2 = \ell$, or $k \leq \eta_1 < \eta_2 \leq \ell$ if $\eta_1, \eta_2$ lie in the plaque between leaves $k, \ell$,
$$\frac{|f(\xi_1, \eta_1) - f(\xi_1, \eta_2)|}{|\eta_1 - \eta_2|} \gtrsim \frac{v_{\ell k}(\xi_1)}{\ell - k} \gtrsim \frac{v_{\ell k}(0)}{\ell - k} = 1$$
whence by the mean value theorem and (\ref{F is coLip in xi}),
\begin{align*}
	|\eta_1 - \eta_2| 
	&\lesssim |f(\xi_1, \eta_1) - f(\xi_1, \eta_2)| \\
	&\leq |f(\xi_1, \eta_1) - f(\xi_2, \eta_2)| + \|\partial_\xi f\|_{C^0} |\xi_1 - \xi_2| \\
	&\leq (1 + \|\partial_\xi f\|_{C^0}) |F(\xi_1, \eta_1) - F(\xi_2, \eta_2)|.
\end{align*}
By (\ref{norms on uk}) and the fact that either $\partial_\xi f = \partial_\xi u_\eta$, or there are $k,\ell$ such that $\partial_\xi f$ is the linear interpolation of $\partial_\xi u_k$ and $\partial_\xi u_\ell$, $\|\partial_\xi f\|_{C^0} \lesssim 1$.
Thus
$$|F(\xi_1, \eta_1) - F(\xi_2, \eta_2)| \gtrsim |\xi_1 - \xi_2|^2 + |\eta_1 - \eta_2|^2.$$
It follows that $\Lip(F^{-1}) \lesssim 1$, so $F$ is a Lipschitz isomorphism with constants comparable to $1$.

Finally, we compose $F$ with the change of coordinates at the start of this proof to obtain a laminar flow box in a small neighborhood of $(0, 0)$ whose image has radius $O(r)$, and whose Lipschitz constants are comparable to $O(r^{-1})$.

%%%%%%%%%%%%%%%%%%%%%%%%%%%%%%%%%%%%%%%%%



%%%%%%%%%%%%%%%%%%%%%%%

\section{Application to \texorpdfstring{$1$-harmonic}{one-harmonic} functions}\label{1harmonic sec}
The purpose of this section is to prove Theorem \ref{main thm}, and explore some of its consequences.
Throughout, we shall assume that the dimension of $M$ is $2 \leq d \leq 7$.

%%%%%%%%%%%%%
\subsection{Estimates on the level sets}
\begin{lemma}
There exists a continuous function $R: M \to \RR_+$ such that the following holds.
Let $p \in M$, $y \in \RR$, $0 < r \leq R(p)$, $u$ a function of least gradient on $B(p, r)$, and $N := \partial \{u > y\} \cap B(p, r)$.
Then 
\begin{equation}\label{least gradient area bound}
|N| \leq \frac{4\pi^{d/2}}{\Gamma(d/2)} r^{d - 1}.
\end{equation}
\end{lemma}
\begin{proof}
By Theorem \ref{main thm of old paper}, $v := 1_{\{u > y\}}|_{B(p, r)}$ is a function of least gradient, and the components of $\supp \dif v$ are minimal hypersurfaces whose surface area in $B(p, r)$ sums to $\int_{B(p, r)} \star |\dif v|$, so it suffices to estimate the total variation of a function of least gradient.

Let $h$ be the trace of $v$ along $\partial B(p, r)$; then $\|h\|_{L^\infty} \leq 1$ \cite[Theorem 2.10]{Giusti77}.
By Theorem \ref{relaxed formulation},
$$\int_{B(p, r)} \star |\dif v| = \Phi_h(v) \leq \Phi_h(0) = \int_{\partial B(p, r)} |h| \dif \mathcal H^{d - 1}.$$
If $r$ is small enough depending on the curvature of $M$ near $p$, we can approximate $\partial B(p, r)$ by a round $d - 1$-sphere of radius $r$, and conclude
\begin{align*}
	\int_{\partial B(p, r)} |h| \dif \mathcal H^{d - 1} &\leq |\partial B(p, r)| \leq 2|\Sph^{d - 1}| r^{d - 1} = \frac{4\pi^{d/2}}{\Gamma(d/2)} r^{d - 1}. \qedhere
\end{align*}
\end{proof}

\begin{lemma}\label{choose balls for main thm}
There exist continuous functions $R, K: M \to \RR_+$ such that the following holds.
Let $p \in M$, $y \in \RR$, $0 < r \leq R(p)$, $u$ a function of least gradient on $B(p, 2r)$, and $N := \partial \{u > y\} \cap B(p, 2r)$.
Then
\begin{equation}\label{least gradient curvature bound}
\|\Two_N\|_{C^0(B(p, r))} \leq \frac{K(p)}{r}.
\end{equation}
\end{lemma}
\begin{proof}
Let $q \in N \cap B(p, r)$ (so $B(q, r) \subseteq B(p, 2r)$).
By (\ref{least gradient area bound}), $|N \cap B(p, r)| \lesssim r^{d - 1}$.
Since $N$ bounds a set $\{u > y\}$ such that $v := 1_{\{u > y\}}|_{B(p, 2r)}$ has least gradient (by Theorem \ref{main thm of old paper}), the components of $N$ are stable.
Indeed, if $(N_t)$ is a normal variation of $N$ with compact support in $B(p, 2r)$, then for $t$ small, $N_t$ bounds an open set whose indicator function $v_t$ is a competitor to $v$, so
$$|N_t \cap B(p, 2r)| = \int_{B(p, 2r)} \star |\dif v_t| \geq \int_{B(p, 2r)} \star |\dif v| = |N \cap B(p, 2r)|.$$
So by \cite[pg785, Corollary 1]{Schoen81}\footnote{See also \cite[Theorem 3]{Schoen75} for an easier proof when $M$ has nonpositive curvature and dimension $d \leq 6$, or \cite[Chapter 2, \S\S4-5]{colding2011course} for a textbook treatment of a similar estimate.}
\begin{align*}
|\Two_N(q)| &\leq \|\Two_N\|_{C^0(B(q, r/2))} \lesssim \frac{1}{r}
\end{align*}
where the constant only depends on the curvature of $M$ near $p$.
\end{proof}

%%%%%%%%%%%%%%
\subsection{Proof of Theorem \texorpdfstring{\ref{main thm}}{B}: Locally least gradient implies minimal lamination}
\subsubsection{Structure of level sets}
Let $u$ be a function of locally least gradient.
The theorem is local, so we may replace $M$ by the balls $B(p, R(p))$ defined by Lemma \ref{choose balls for main thm}. 

By a \dfn{level set} we mean a connected component of $\partial \{u > y\}$ for some $y \in \RR$.
By Theorem \ref{main thm of old paper}, the level sets of $u$ are complete embedded minimal hypersurfaces in $M$.

Let $y, z \in \RR$. If $y > z$, then $\{u > y\} \subseteq \{u > z\}$, so $\partial \{u > y\}$ lies on one side of $\partial \{u > z\}$.
By the maximum principle for minimal hypersurfaces, it follows that either $\partial \{u > y\}$ and $\partial \{u > z\}$ are disjoint, or are equal.
If we shrank $M$ enough, then by (\ref{least gradient curvature bound}), there is $A \geq 0$ such that for any level set $N$,
\begin{equation}\label{curvature bound on level sets}
\|\Two_N\|_{C^0} \leq A.
\end{equation}
By (\ref{level sets define support}), $S := \bigcup_{y \in \RR} \partial \{u > y\}$ is dense in $\overline S = \supp \dif u$.

\subsubsection{Existence of flow boxes}
We say that a minimal hypersurface $N$ is a \dfn{generalized level set} if $N$ is locally the $C^2$ limit of level sets.
When working with generalized level sets we shall assume that $u \in L^\infty$.
Since we are working locally, this is no loss of generality \cite[Theorem 4.3]{Gorny20}.\footnote{The proof of this theorem does not use the euclidean metric anywhere.}

\begin{lemma}
The closed set $\overline S$ is covered by generalized level sets $N$ satisfying (\ref{curvature bound on level sets}).
\end{lemma}
\begin{proof}
Let $p \in \overline S$, and choose $p_n \in \partial \{u > y_n\}$ converging to $p$.
Since $u \in L^\infty$, $(y_n)$ must be a bounded sequence.
So after passing to a subsequence, we may assume that $(y_n)$ converges monotonically to some $y \in \RR$.

After passing to a smaller ball and rescaling, we may use (\ref{curvature bound on level sets}) to assume that the the minimal hypersurfaces $N_n := \partial \{u > y_n\}$ all have small curvature in $C^0$.
Thus by Lemmata \ref{existence of tubes} and \ref{lams have C0 fields}, each minimal hypersurface $N_n$ can locally be viewed as a graph of a function $f_n$ which is small in $C^2$, in normal coordinates centered on $p$.
By (\ref{norms on uk}), $\|f_n\|_{C^3} \lesssim 1$ and so along a subsequence, $f_n \to f$ in $C^2$, for some $f$ whose graph is a minimal hypersurface $N \ni p$ satisfying the same curvature bound (\ref{curvature bound on level sets}).
\end{proof}

% Moreover, since $(y_n)$ is a monotone sequence and the $N_n$ are disjoint, $(f_n)$ is also a monotone sequence.
% Therefore, after taking a reflection of the coordinate system if necessary, we may assume that $(f_n)$ is an increasing sequence.

% Thus $f \geq f_n$ for every $n$, and $f_n(0) < f(0) = 0$, since $x = (0, 0)$ is contained in $N$ but not any of the $N_n$.
% So by the maximum principle for minimal hypersurfaces, $N$ is disjoint from all the $N_n$.
% This argument works for any sequence $(x_n)$ which approximates $x$ with $(u(x_n))$ monotone, so $N \cap S$ is empty.

\begin{lemma}
Let $N, N'$ be generalized level sets.
Then $N = N'$, or $N \cap N'$ is empty.
\end{lemma}
\begin{proof}
If $N, N'$ are distinct generalized level sets which intersect at some point $p$, then after passing to a small neighborhood of $p$, we may assume that $N, N'$ are the graphs of functions $f, f'$ which are approximated in $C^2$ by sequences $f_n, f_n'$ whose graphs are level sets.
Since $p \in N \cap N'$ there exists $x$ such that $f(x) = f'(x)$.
By the maximum principle for minimal hypersurfaces and the fact that $f, f'$ are distinct, $x$ is a saddle point of $f - f'$, so there exist $x_+, x_-$ close to $x$ with $f(x_+) > f'(x_+)$ and $f(x_-) < f'(x_-)$.
Therefore for $n$ large enough, $f_n(x_+) > f_n'(x_+)$ and $f_n(x_-) < f'_n(x_-)$, so by the intermediate value theorem there exist $x_n$ with $f_n(x_n) = f_n(x_n')$.
But the level sets of $u$ are disjoint, so this is a contradiction.
\end{proof}

So by Theorem \ref{regularity theorem}, the generalized level sets are the leaves of a Lipschitz minimal lamination $\lambda$, which by (\ref{level sets define support}) has support equal to $\overline S = \supp \dif u$.

\subsubsection{Discarding the exceptional level sets}
Our next task is to show that every generalized level set, which is not a level set, is a component of $\partial \{u < y\}$ for some $y \in \RR$.
We work in flow box coordinates $(k, x) \in I \times J$ near $p$, and write $u(k, x) = \tilde u(k)$.

\begin{lemma}
For each $k \in I$ there is a neighborhood $I' \subseteq I$ of $k$ such that $\tilde u|_{I'}$ is monotone.
\end{lemma}
\begin{proof}
Suppose not.
Then, for each $\varepsilon > 0$, possibly after replacing $\tilde u$ with $-\tilde u$, we may find $k_1 < k_2$ in $[k - \varepsilon, k + \varepsilon]$ such that $\tilde u$ is strictly decreasing at $k_1$ and strictly increasing at $k_2$.
Moreover, since $u$ is constant away from its generalized level sets, we may assume that the $\{k_i\} \times J$ are generalized level sets, and hence are minimal hypersurfaces.

Let $R := [k_1, k_2] \times J$.
By Theorem \ref{relaxed formulation}, there is a vector field $X$ defined near $p$ such that $\nabla \cdot X = 0$, $X$ is the unit normal of $\{k = k_1\} \times J$ and $\{k = k_2\} \times J$ oriented in the direction that $u$ is increasing, and $\|X\|_{L^\infty} = 1$.
By our contradiction hypothesis, we may choose the orientation on $R$ such that for $i = 1, 2$,
$$\int_{\{k = k_i\} \times J} X \cdot \normal_{\partial R} \dif \mathcal H^{d - 1} = |\{k = k_i\} \times J|,$$
where $|\cdot|$ and $\mathcal H^{d - 1}$ both refer to the Hausdorff measure obtained from the Riemannian metric on $M$, rather than the euclidean metric from coordinates.
By the Lipschitz nature of the flow box coordinates, there is $\delta > 0$ independent of $\varepsilon$ such that $\{k = k_i\} \times J$ contains a ball of radius $\delta$.
Therefore by the monotonicity formula for minimal hypersurfaces, there is $c > 0$ independent of $\varepsilon$ such that
$$|\{k = k_i\} \times J| \geq c\delta^{d - 1}.$$
On the other hand, if $\sigma$ is a $d - 1$-face of $R$ which is not a generalized level set, then $[k_1, k_2]$ is an edge of $\sigma$, so by the Lipschitz regularity, there is $C > 0$ independent of $\varepsilon$ such that $|\sigma| \leq C\varepsilon$.
So by the divergence theorem, 
$$0 = \int_{\partial R} X \cdot \normal_{\partial R} \dif \mathcal H^{d - 1} \geq 2c\delta^{d - 1} - 2(d - 1)C \varepsilon.$$
If $\varepsilon$ is taken small enough, this is a contradiction.
\end{proof}

Possibly after shrinking and reorienting the flow box, we may assume that $\tilde u$ is nondecreasing.
We call such a flow box an \dfn{oriented flow box}.
Under this assumption, we may partition $I$ into open intervals $I_i$ on which $\tilde u$ is constant, and a compact set $K$ on which $\tilde u$ is strictly increasing.

Let $N$ be a generalized level set, so the restriction of $N$ to the flow box is $\{k\} \times J$ for some $k \in K$.
Let $C$ be the connected component of $K$ containing $k$ (noting carefully that $K$ may be totally disconnected).
Since $C$ is a connected closed subset of $\RR$, $C$ is a closed interval.
If $k$ is not the right endpoint of $C$, then there are $k_n > k$ decreasing to $k$ with, for any $x, x' \in J$,
$$u(k_n, x) = \tilde u(k_n) > \tilde u(k) = u(k, x').$$
Therefore $N$ is a component of $\partial \{u > \tilde u(k)\}$, and in particular $N$ is a level set.

So if $N$ is not a level set, $k$ is a right endpoint of $C$.
So there are $k_n < k$ increasing to $k$ with $\tilde u(k_n) < \tilde u(k)$, so reasoning as above, $N$ is a component of $\partial \{u < \tilde u(k)\}$.

\subsubsection{Construction of Ruelle-Sullivan current}
We work in oriented flow box coordinates $(k, x) \in I \times J$, with the function $\tilde u$ defined by $\tilde u(k) = u(k, x)$ as above, and let $K := \supp \dif \tilde u$.
We obtain a measure $\mu$ on $K$ by setting, for $k_1 < k_2$,
$$\mu([k_1, k_2] \cap K) := \tilde u(k_2) - \tilde u(k_1).$$
Since the coordinates are oriented, $\mu$ is a positive Radon measure.

Suppose that $(k', x') \in I' \times J$ is a different oriented flow box coordinate system, with $\tilde u'$ and $K'$, and the transition map carries $k_1, k_2$ to $k_1', k_2'$.
If $(k, x)$ and $(k', x')$ both are coordinate representations of $p \in M$, then $\tilde u(k, x) = u(p) = \tilde u'(k', x')$, hence
$$\mu'([k_1', k_2'] \cap K') := \tilde u'(k_2') - \tilde u'(k_1') = \tilde u(k_2) - \tilde u(k_1) = \mu([k_1, k_2] \cap K).$$
So $\mu$ is transverse, and arises from the disintegration of $\star |\dif u|$ in coordinates.
It follows from (\ref{polar ruelle sullivan}) that $\dif u = \normal_\lambda |\dif u|$ is the Ruelle-Sullivan current for the given measured oriented structure on $\lambda$.

%%%%%%%%%%%%%%%%

\subsection{Proof of Theorem \texorpdfstring{\ref{main thm}}{B}: Minimal lamination implies locally least gradient}
Suppose that $H^1(M, \RR) = 0$ and we are given a measured oriented minimal lamination $\lambda$, which then has a Ruelle-Sullivan current $\dif u$.
If $u$ does not have locally least gradient, then there exists $p \in M$ such that for any open set $E \ni p$, $u|_E$ does not have least gradient.
Since $\supp \dif u = \supp \lambda$, we may assume that $p \in \supp \lambda$ (for if $p \notin \supp \dif u$, then $u$ is actually constant in a neighborhood of $p$, hence trivially has least gradient there).

By Proposition \ref{minimal implies locally minimizing}, if $\lambda$ has bounded curvature, then there exists $\delta > 0$ depending on $p$, $g$, and the curvature of $\lambda$, such that for any $q \in B(p, \delta)$ and any leaf $N$ of $\lambda$, $N \cap B(q, 2\delta)$ is absolutely area-minimizing in $B(q, 2\delta)$.
In particular, $N \cap B(p, \delta)$ is absolutely area-minimizing in $B(p, \delta) \subseteq B(q, 2\delta)$, and we can let $E := B(p, \delta)$.
On the other hand, if the leaves of $\lambda$ are absolutely area-minimizing, then we can take $E := M$.

By construction, $u|_E$ does not have least gradient so there exists $v \in BV_\cpt(E)$ such that
\begin{equation}\label{not least gradient compact support}
\int_E \star |\dif u| > \int_E \star |\dif u + \dif v|.
\end{equation}
In particular, there exists a collar neighborhood $F$ of the boundary such that $v|_F = 0$.
Then for each $y \in \RR$,
$$\partial \{u > y\} \cap F = \partial^* \{u > y\} \cap F = \partial^* \{u + v > y\} \cap F,$$
so that $1_{\{u + v > y\}} - 1_{\{u > y\}}$ has compact support in $E$.
Since $\partial \{u > y\}$ is absolutely area-minimizing in $E$, $1_{\{u > y\}}$ has least gradient in $E$.
So we may estimate using the coarea formula (see \cite[Proposition 2.5]{BackusFLG} for a proof at this regularity)
\begin{align*}
\int_E \star |\dif u| &= \int_{-\infty}^\infty \int_E \star |\dif 1_{\{u > y\}}| \dif y \leq \int_{-\infty}^\infty \int_E \star |\dif 1_{\{u + v > y\}}| \dif y = \int_E \star |\dif u + \dif v|
\end{align*}
which is a contradiction of (\ref{not least gradient compact support}).
This completes the proof of Theorem \ref{main thm}.


%%%%%%%%%%%%%%%%%%%%%%%%%%%%
\subsection{The G\'orny decomposition}\label{1harmonic apps}
% \subsubsection{The Dirichlet problem}
% Let $\overline M$ be a compact Riemannian manifold with nonempty Lipschitz boundary $\partial M$, and interior $M$.
% A consequence of our formulation of the $1$-Laplacian, and the nonuniqueness and nonstability of minimal hypersurfaces, is that there are \emph{two} reasonable formulations for the Dirichlet problem for the $1$-Laplacian on $M$:
% \begin{enumerate}
% \item The Dirichlet problem for functions of least gradient. This needs to be formulated carefully for discontinuous boundary data because otherwise solutions will not exist \cite{spradlin2013traces}, but the relaxation formulation of \cite{Mazon14} suffices for boundary data in $L^1$.
% \item The Dirichlet problem for functions of locally least gradient. Given a function $h \in L^1(\partial M)$, find a function $u \in BV(M)$, so that we can cover $\overline M$ by open sets $U_\alpha$ equipped with data $h_\alpha \in L^1(\partial U_\alpha)$, such that $u$ solves the Dirichlet problem for functions of least gradient on $U_\alpha$ with data $h_\alpha$, and $h_\alpha|_{\partial M} = h$.
% \end{enumerate}
% Under certain convexity hypotheses on $\partial M$, the Dirichlet problem for functions of least gradient with $C^0$ data has a unique solution \cite{Sternberg1992}.
% However, the Dirichlet problem for functions of least gradient is highly nonlocal, unlike the Dirichlet problem for solutions of a PDE; moreover, restricting to functions of least gradient causes Theorem \ref{main thm} to fail.

% For discontinuous data, the Dirichlet problem for functions of least gradient does not have a unique solution.
% However, it is likely that the \emph{unmeasured} lamination \cite[Remark 2.8]{Mazon14} is uniquely determined by the Dirichlet data; see \cite{}

We now consider an analogue of the G\'orny decomposition \cite[Theorem 1.2]{górny2017planar} of a function of least gradient.
Recall that a \dfn{Cantor function} is a continuous function whose exterior derivative is mutually singular with Lebesgue measure.
A \dfn{jump function} is a function $u \in BV_\loc$ such that $\dif u$ equals its own jump part as in \cite[Definition 3.91]{Ambrosio2000FunctionsOB}.
In general, it is not possible to decompose a function $u$ of bounded variation into an absolutely continuous (that is, $W^{1, 1}_\loc$) part, a Cantor part, and a jump part \cite[Example 4.1]{Ambrosio2000FunctionsOB}.
G\'orny showed that for a function of least gradient on euclidean space, such a decomposition exists.

We give a new proof using Theorem \ref{main thm} which applies on curved domains.
As a byproduct, we obtain a new proof of the continuity of jump-free functions of least gradient \cite[Theorem 4.1]{HakkarainenKorteLahtiShanmugalingam+2015}, though their result holds in the higher generality that $M$ is a metric measure space.

\begin{proposition}
Let $u$ be a function of locally least gradient, and suppose that $H^1(M, \RR) = 0$. Then there exists a decomposition of $u$ into functions of locally least gradient
$$u = u_{ac} + u_C + u_j,$$
with mutually singular exterior derivatives, such that $u_{ac} \in W^{1, 1}_\loc(M) \cap C^0(M)$, $u_C$ is a Cantor function, and $u_j$ is a jump function.
Up to addition of additive constants, this decomposition is unique.
\end{proposition}
\begin{proof}
We work in flow box coordinates $(k, x) \in I \times J$ on some open set $V_\alpha \subseteq M$.
We may assume that $V_\alpha$ is so small that $u|_{V_\alpha}$ has least gradient.
In the flow box coordinates, $u(k, x) = \tilde u^\alpha(k)$ for some $\tilde u_\alpha: I \to \RR$. 
Since $\dif \tilde u^\alpha$ is the transverse measure, it is a Radon measure, so $\tilde u^\alpha$ has bounded variation and hence we have the Lebesgue decomposition on an interval \cite[Corollary 3.33]{Ambrosio2000FunctionsOB} 
$$\tilde u^\alpha = \tilde u^\alpha_{ac} + \tilde u^\alpha_C + \tilde u^\alpha_j$$
where $\tilde u^\alpha_{ac} \in W^{1, 1}_\loc(I)$, $\tilde u^\alpha_C$ is a Cantor function, and $\tilde u_j^\alpha$ is a jump function.
This decomposition is unique up to an addition of constants, and induces a decomposition of $\dif \tilde u^\alpha$ into mutually singular measures.
We then write $u^\alpha_\sigma(k, x) = \tilde u^\alpha_\sigma(k)$ to obtain a function on $V_\alpha \subseteq M$, where $\sigma \in \{ac, C, j\}$.
The functions $u^\alpha_{ac}, u^\alpha_C, u^\alpha_j$ are $W^{1,1}_\loc$, Cantor, and jump respectively, since Lipschitz isomorphisms preserve these conditions.
Moreover, since $\tilde u^\alpha_{ac}$ and $\tilde u^\alpha_C$ are jump-free functions of bounded variation on an interval, they are continuous; hence $u^\alpha_{ac}$ and $u^\alpha_C$ are continuous as well.

We next claim that $\dif u^\alpha_{ac}, \dif u^\alpha_C, \dif u^\alpha_j$ are mutually singular.
Since $\dif \tilde u^\alpha_{ac}, \dif \tilde u^\alpha_C, \dif \tilde u^\alpha_j$ are mutually singular, we have a decomposition
$$I = I_{ac} \sqcup I_C \sqcup I_j$$
such that for $\sigma \neq \tau$, $I_\tau$ is a $\dif \tilde u^\alpha_\sigma$-null set.
Applying Fubini's theorem, the same decomposition holds for $\dif u^\alpha_\sigma$ and $I \times J \cong V_\alpha$, implying mutual singularity of the $\dif u^\alpha_\sigma$.

We now claim that $u^\alpha_\sigma$ have least gradient on $V_\alpha$; this step is identical to the analogous step in \cite{górny2017planar}.
To ease notation we do this for $\sigma = j$; the other cases are similar.
If the claim fails, then there is some $v \in BV_\cpt(V_\alpha)$ such that $\int \star |\dif u^\alpha_j| > \int \star |\dif (u^\alpha_j + v)|$.
But if so, then by mutual singularity,
\begin{align*}
\int_{V_\alpha} \star |\dif u| &= \int_{V_\alpha} \star (|\dif u^\alpha_j| + |\dif u^\alpha_C| + |\dif u^\alpha_{ac}|) 
>  \int_{V_\alpha} \star (|\dif u^\alpha_j + \dif v| + |\dif u^\alpha_C| + |\dif u^\alpha_{ac}|) \\
&\geq \int_{V_\alpha} \star |\dif u + \dif v|,
\end{align*}
which contradicts that $u$ has least gradient.

Finally we glue the local decompositions together.
The measure-preserving property of transition maps and the uniqueness of the Lebesgue decomposition implies that
$$\dif u^\alpha_\sigma|_{V_\alpha \cap V_\beta} = \dif u^\beta_\sigma|_{V_\alpha \cap V_\beta}.$$
As closed currents form a sheaf, it follows that there exist unique closed currents $\dif u_\sigma$ on all of $M$ such that $\dif u_\sigma|_{V_\alpha} = \dif u^\alpha_\sigma$.
Since $H^1(M, \RR) = 0$, $\dif u_\sigma$ has an antiderivative $u_\sigma$, which has locally least gradient, since $u_\sigma|_{V_\alpha} = u_\sigma^\alpha$ has least gradient.
\end{proof}

% %%%%%%%%%%%%%%%%%%%%%%%%

% \subsection{Structure of the generalized level sets}\label{genSets}
% Suppose that $u$ has locally least gradient, and let $\lambda$ be the associated lamination.
% A byproduct of the proof of Theorem \ref{main thm} is that the leaves of $\lambda$ are generalized level sets of $u$, thus they are $C^2$ limits of components of level sets $\partial \{u > y\}$.
% Specifically, we must account for the following:

% \begin{example}
% Let $u(x, y) := |x|$ on $\Ball^2$; then $u$ has locally least gradient, since its level sets are lines.
% Moreover, $u$ attains a minimum on $\{x = 0\}$, which is a generalized level set of $u$.
% But $\{x = 0\}$ is not the boundary of any superlevel set $\{u > z\}$, since for $z > 0$, $\partial \{u > z\} = \{|x| = z\}$, and for $z \leq 0$, $\{u > z\} = \Ball^2$ has no boundary in $\Ball^2$.
% \end{example}

% It turns out that local extrema, as in the above example, is the only obstruction to a generalized level set being a level set.
% To be more precise, introduce the \dfn{lower semicontinuous envelope}
% $$u_{\rm lsc}(x) := \sup \{f(x): \text{$f$ is a lower semicontinuous function and $f \geq u$ almost everywhere}\}.$$
% Since the supremum of lower semicontinuous functions is lower semicontinuous, so is $u_{\rm lsc}$.
% By the G\'orny decomposition, $u = u_{\rm lsc}$ almost everywhere away from the hypersurfaces along which $u$ jumps, and if $u$ jumps along a hypersurface $N$, say from $y_-$ to $y_+$, where $y_- < y_+$, then 
% $$u_{\rm lsc}|_N = y_-.$$
% The set on which a lower semicontinuous function attains its minimum is a well-defined closed set, so we say that $u$ has a \dfn{local minimum} at $x \in M$ if $u_{\rm lsc}$ has a local minimum at $x$.
% We dually define the \dfn{upper semicontinuous envelope} $u_{\rm usc}$ and \dfn{local maxima} of $u$.

% \todo{Cite who came up with semicontinuous envelopes in this setting. Do we need the sub AND super level sets?}

% Let $N$ be a generalized level set of $u$.
% The restriction of $N$ to a flow box takes the form $\{k = k_*\}$ for some $k_*$ in the local leaf space $K$.
% By writing $\tilde u_{\rm lsc}(k) = u_{\rm lsc}(k, z)$ and $\tilde u_{\rm usc}(k) = u_{\rm usc}(k, z)$, for $k \in K$, we see that $u_{\rm usc}|_N, u_{\rm lsc}|_N$ are constant, say $u_{\rm usc}|_N = y_+$ and $u_{\rm lsc}|_N = y_-$.
% In particular $y_+ \geq y_-$, with strict inequality iff $u$ jumps along $N$.

% \begin{proposition}
% Let $u$ be a function of locally least gradient, and let $N \subset M$ be a generalized level set of $u$.
% Then either there exists $y \in \RR$ such that $N$ is a component of $\partial \{u > y\}$ or $\partial \{u < y\}$, or $u$ has a local extremum along $N$.
% \end{proposition}
% \begin{proof}
% Suppose that $N_n \subseteq \partial \{u > y_n\}$ converge (in $C^2$, hence in the Hausdorff sense) to $N$.
% By passing to a flow box and suppressing the $d - 1$ directions on which $u$ is constant, we may replace $u, u_{\rm usc}, u_{\rm lsc}$ by functions $\tilde u, \tilde u_{\rm usc}, \tilde u_{\rm lsc}$ on $\RR$, and $N_n$ by a point $k_n \in [0, 1]$, such $\tilde u_{\rm usc}, \tilde u_{\rm lsc}$ are semicontinuous envelopes of $\tilde u$.
% After taking a subsequence we can take $k_n \to k_*$ to see that $N$ corresponds to the point $k_*$.

% Suppose that $N$ is not a component of $\partial \{u > y\}$ or $\partial \{u < y\}$ for any $y$, so $k_*$ is not a point of $\partial \{\tilde u > y\}$ or $\partial \{\tilde u < y\}$.
% In particular if we take $y_-, y_+$ to be the values of $\tilde u_{\rm lsc}, \tilde u_{\rm usc}$ at $k_*$, then $k_*$ is not a point of $\partial \{\tilde u > y_-\}$ or $\partial \{\tilde u < y_+\}$.

% Suppose that $k_*$ is an interior point of $\{\tilde u > y_-\}$.
% Then $k_*$ is a local minimum, for on a punctured neighborhood $0 < |k - k_*| < \delta$ of $k_*$,
% $$\tilde u_{\rm lsc}(k) > y_- = \tilde u_{\rm lsc}(k_*).$$
% Similarly, if $k_*$ is an interior point of $\{\tilde u < y_+\}$, then $k_*$ is a local maximum.

% The alternative is that $k_*$ is interior to $\{\tilde u \leq y_-\} \cap \{\tilde u \geq y_+\}$.
% Therefore $y := y_- = y_+$, and $\tilde u$ is identically $y$ in a neighborhood of $k_*$.
% This is impossible, since $k_n \to k_*$ and $\tilde u$ is not constant along the sequence of $k_n$.
% \end{proof}

%%%%%%%%%%%%%%%%%%%%%%%%%%%%%%%%%%%%%%%%%
\section{Compactness of the space of laminations}\label{CompactnessSec}
In this section we prove Theorem \ref{compactness theorem}, the compactness theorem.
We then apply it to explore the implications between the different modes of convergence.

\subsection{Lemmata on modes of convergence}
\begin{lemma}
Suppose that $\lambda_n \to \lambda$ in the weak topology of measures or Thurston's geometric topology.
Then 
\begin{equation}\label{supports shrink in the limit}
\supp \lambda \subseteq \liminf_{n \to \infty} \supp \lambda_n.
\end{equation}
\end{lemma}
\begin{proof}
Let $x \in \supp \lambda$.
If the convergence is in the weak topology of measures, let $\mu, \mu_n$ be the transverse measures.
Then by Proposition \ref{portmanteau}, for any $\varepsilon > 0$,
$$\liminf_{n \to \infty} \mu_n(B(x, \varepsilon)) \geq \mu(B(x, \varepsilon)) > 0$$
so $\mu_n(B(x, \varepsilon)) \gtrsim 1$.
So for any $\varepsilon$ we can find $n$ and $x_n \in \supp \lambda_n \cap B(x, \varepsilon)$.
If instead the convergence is in the Thurston topology, we pass to a subsequence which realizes the limit inferior in (\ref{supports shrink in the limit}).
Then by definition of a basic open set, for every $\varepsilon > 0$ we can find $n$ and $x_n$ such that $x_n \in \supp \lambda_n \cap B(x, \varepsilon)$.
Either way, we conclude (\ref{supports shrink in the limit}).
\end{proof}

Given a connected oriented submanifold $N$, we let $[N]$ denote the integral current defined by integration along $N$, or equivalently the Ruelle-Sullivan current of the lamination whose only leaf is $N$.

\begin{lemma}\label{C1 close implies measure close}
Let $N_1, N_2$ be connected oriented submanifolds which are $C^1$-close.
Then $[N_1], [N_2]$ are close in the weak topology of measures.
\end{lemma}
\begin{proof}
Since $N_1, N_2$ are $C^1$, they admit triangulations, so we can reduce to the case that $N_i = (F_i)_* \sigma$, where $\sigma$ is the standard simplex and $F_i$ are $C^1$ maps.
Then the pullback maps $F_1^*$ and $F_2^*$ are close in $C^0$, and for any form $\psi$,
$$\int_M [N_i] \wedge \psi = \int_\sigma F^*_i \psi.$$
Therefore $[N_1], [N_2]$ are close in the weak topology of measures.
\end{proof}

\begin{lemma}\label{measured convergence is smooth convergence}
Let $C > 0$, let $(N_n)$ be a sequence of minimal hypersurfaces with $\|\Two_{N_n}\|_{C^0} \leq C$, and let $N$ be a $C^1$ hypersurface.
If $[N_n] \to [N]$ in the weak topology of measures, then $N$ is a minimal hypersurface.
\end{lemma}
\begin{proof}
Let $p \in N = \supp [N]$; by (\ref{supports shrink in the limit}), there exist $p_n \in N_n$ with $p_n \to p$.
Then by Lemma \ref{convergence of normals}, $\normal_{N_n}(p_n) \to \normal_N(p)$.
By assumption, $\|\nabla \normal_{N_n}\|_{C^0} \leq C$; moreover, we can choose normal coordinates $(x, y) \in \RR^{d - 1} \times \RR$ at $p$ with $\partial_y|_p = \normal_N(p)$.
So in a neighborhood of $p$, for every $n$ large, $\normal_{N_n}$ is $C^0$ close to $\partial_y$.
In particular, $N_n$ are the graphs of functions $u_n: \RR^{d - 1}_x \to \RR_y$ which are bounded in $C^1$ and solve $Pu_n = 0$.
By (\ref{norms on uk}), $(u_n)$ is precompact in $C^\infty$.
Similarly, $N$ is the graph of some $u$, and along a subsequence $u_{n_k} \to \tilde u$ in $C^\infty$ for some $\tilde u$, which then has a graph $\tilde N$.
By Lemma \ref{C1 close implies measure close}, $[N_{n_k}] \to [\tilde N]$ in the weak topology of measures; it follows that $\tilde N = N$, so $\tilde u = u$.
Therefore $u_{n_k} \to u$ in $C^\infty$, hence $Pu = 0$.
\end{proof}

%%%%%%%%%%%%%%%%%%

\subsection{Proof of Theorem \texorpdfstring{\ref{compactness theorem}}{C}}
\subsubsection{Construction of the limiting flow box}
Let $P \in M$, and let $(\lambda_n)$ be a sequence of minimal laminations of bounded curvature, such that every leaf of every lamination meets a compact set.

By Theorem \ref{regularity theorem}, there exist $r > 0$ and $L \geq 1$ such that for every large $n \in \NN$, $B(P, r)$ is contained in the image of a flow box $F_n$ for $\lambda_n$ with Lipschitz constant $L$, such that $F_n(0, 0) = P$.
By the Arzela-Ascoli theorem, along a subsequence $F_n \to F$ in $C^0$ for some map $F: I \times J \to B(P, r)$ and some $I \subseteq \RR$, $J \subseteq \RR^{d - 1}$, such that on the image $V$ of $F$, we also have the convergence $F_n^{-1} \to F^{-1}$.
Moreover, $F(0, 0) = P$, so that $F: I \times J \to V$ is a homeomorphism onto a set which contains $P$.
Since
$$\max(\Lip(F), \Lip(F^{-1})) \leq \limsup_{n \to \infty} \max(\Lip(F_n), \Lip(F_n^{-1})) \leq L,$$
it follows that $\max(\Lip(F), \Lip(F^{-1})) \leq L$, and for any $\theta \in (0, 1)$,
\begin{align*}
	\|F - F_n\|_{C^\theta}
	&\leq \Lip(F - F_n)^\theta \|F - F_n\|_{C^0}^{1 - \theta} \leq (2L)^\theta \|F - F_n\|_{C^0}^{1 - \theta}.
\end{align*}
It follows that $F_n \to F$ in $C^\theta$, hence in $C^{1-}$, and similarly for $F^{-1}$.
Since $(F_n)$ and $(F_n^{-1})$ are bounded in tangential $C^\infty$, a similar compactness argument to the above shows that $F_n \to F$ and $F_n^{-1} \to F^{-1}$ in tangential $C^\infty$.

Since $P$ was arbitrary, it follows that we can find laminar atlases $(F_\alpha^n, K_\alpha^n)$ for each large $n \in \NN$ such that $F_\alpha^n \to F_\alpha$ in $C^{1-}$, where the images of $F_\alpha$ and $F_\alpha^n$ are an open cover $(U_\alpha)$ of $M$ independent of $n$, and $(F_\alpha)$ satisfies the usual transition relations, and $F_\alpha$ is a Lipschitz isomorphism.

%%%%%%%%%%%%%%%%%%%%%%%

\subsubsection{Construction of the limiting lamination}
We now construct the limiting lamination.
We employ the Hausdorff hyperspace $\Hypspace I$ of closed subsets of $I$ to accomplish this.
Since $I$ is a compact metric space, so is $\Hypspace I$ \cite[Theorem 4.17]{nadler2017continuum}, so we may diagonalize so that for every $\alpha$, either $K^n_\alpha \to K_\alpha$ for some nonempty $K_\alpha$ in the Hausdorff distance on $I$, or for all $n \geq n^*(\alpha)$, $K_\alpha^n$ is empty (in which case we define $K_\alpha = \emptyset$).

In order to ensure that the laminations $\lambda_n$ do not escape to infinity, fix a compact set $E \subseteq M$ such that every leaf of every $\lambda_n$ meets $E$.
Then there exists a finite set $A_E \subseteq A$ such that $E \subseteq \bigcup_{\alpha \in A_E} U_\alpha$.

\begin{lemma}\label{label sets are nonempty}
	There exists $\alpha$ such that $K_\alpha$ is nonempty.
\end{lemma}
\begin{proof}
	Suppose not; then for
	$$n \geq \max_{\alpha \in A_E} n^*(\alpha)$$
	and $\alpha \in A_E$, $K_\alpha^n = \emptyset$, so no leaves of $\lambda_n$ meet $U_\alpha$, and hence no leaves of $\lambda_n$ meet $E$.
	This is a contradiction since $\lambda_n$ has a leaf.
\end{proof}

In each flow box $F_\alpha$ with $K_\alpha$ nonempty, we thus have the leaves of a lamination, namely $K_\alpha \times J$.
We now check the transition relations to ensure that they glue to a global lamination; this is straightforward but we include it for completeness.

Thus let $\psi_{\alpha \beta}$ and $\psi_{\alpha \beta}^n$ be the transition maps, thus $\psi_{\alpha \beta}^n$ induces a map
$$\psi_{\alpha \beta}^n: K_\alpha^n \to K_\beta^n.$$
By convergence of $(F_\alpha^n)$, $\psi_{\alpha \beta}$ induces a map $K_\alpha \to K_\beta$.

\begin{definition}
	A \dfn{cocycle of labels} $(k_\alpha)_{\alpha \in A'}$ is a set $A' \subseteq A$ and an element of $\prod_{\alpha \in A'} K_\alpha$, such that:
\begin{enumerate}
	\item The cocycle condition: $k_\beta = \psi_{\alpha \beta}(k_\alpha)$ for $\alpha, \beta \in A'$.
	\item For every $\alpha \in A'$, if $\psi_{\alpha \beta}(k_\alpha)$ is well-defined, then $\beta \in A'$.
\end{enumerate}
\end{definition}

\begin{lemma}
	Every cocycle of labels $(k_\alpha)_{\alpha \in A'}$ defines a complete minimal hypersurface $N$ such that
	$$N \cap U_\alpha = F_\alpha(\{k_\alpha\} \times J).$$
\end{lemma}
\begin{proof}
We have the cocycle condition
$$(N \cap U_\alpha) \cap U_\beta = (N \cap U_\beta) \cap U_\alpha$$
which follows from the fact that
\begin{align*}
F_\alpha(\{k_\alpha\} \times J) \cap U_\beta
&= F_\beta(\psi_{\alpha \beta}(\{k_\beta\} \times J)) \cap U_\alpha \cap U_\beta \\
&= F_\beta(\psi_{\alpha \beta}(\{k_\beta\} \times J)) \cap U_\alpha.
\end{align*}
From the cocycle condition, it follows that $N$ honestly defines a Lipschitz hypersurface in $M$, which is complete in $\bigcup_{\alpha \in A'} U_\alpha$.
If $\overline N$ intersects $U_\alpha$ for some $\alpha \notin A'$, then $N$ intersects $U_\beta$ for some $\beta \in A'$ so that $U_\beta \cap U_\alpha \cap \overline N$ is nonempty.
But then $\psi_{\beta \alpha}(k_\beta)$ must be defined, so $\alpha \in A'$, a contradiction.
Therefore $N$ is complete in $M$.

To prove minimality, let
$$u_\alpha(k, x) = (F_\alpha)_* 1_{k > k_\alpha}$$
and similarly $u_\alpha^n(k, x) = (F_\alpha^n)_* 1_{k > k_\alpha^n}$ where $(k_\alpha^n) \in \prod_n K_\alpha^n$ converges to $k_\alpha$.
Since $F_\alpha \circ (F_\alpha^n)^{-1}$ converges to the identity map in $C^{1-}$, and $F_\alpha^{-1}(N \cap U_\alpha)$ has zero measure, it follows that $u_\alpha^n \to u_\alpha$ almost everywhere, and hence in $L^1(I \times J)$ by the dominated convergence theorem.
But $u_\alpha^n$ has least gradient, so by Miranda compactness (Proposition \ref{MirandaStability}), $\dif u_\alpha^n \to \dif u_\alpha$ in the weak topology of measures.
Clearly $\dif u_\alpha = [N \cap U_\alpha]$ and similarly for $u_\alpha^n$, so by Lemma \ref{measured convergence is smooth convergence}, $N \cap U_\alpha$ is minimal.
\end{proof}

\begin{lemma}
	Let $\lambda$ be the lamination with laminar atlas $(F_\alpha, K_\alpha)$.
	Then $\lambda$ is well-defined and minimal.
\end{lemma}
\begin{proof}
Since 
$$\supp \lambda \cap U_\alpha = K_\alpha \times J$$
and $K_\alpha$ is compact, $\supp \lambda$ is closed.
Now if we choose $\alpha$ such that $K_\alpha$ is nonempty, every element of $K_\alpha$ uniquely determines a cocycle of labels, and hence a leaf of $\lambda$.
So $\supp \lambda$ is nonempty, and since all of its leaves are complete minimal, $\lambda$ is minimal.
\end{proof}

\subsubsection{Convergence in Thurston's geometric topology}
At this stage of the argument we have constructed a limiting lamination with limiting flow boxes; we now check that the sequence of laminations actually converges to the limiting lamination.

If $K_\alpha$ is nonempty, then any $k_\alpha \in K_\alpha$ is the limit of some sequence $(k_\alpha^n)_n \in \prod_n K_\alpha^n$ \cite[Theorem 4.11]{nadler2017continuum}.
Thus $\{k_\alpha\} \times J$ can be written as the set of limits of sequences $(k_\alpha^n, x)_n \in \prod_n K_\alpha^n \times J$, and so any leaf $N$ of $\lambda$ takes the form $N = \lim_{n \to \infty} N_n$ for some sequence $(N_n) \in \prod_n \Leaves \lambda_n$, where $\Leaves \lambda_n$ is the set of leaves of $\lambda_n$.
In other words, leaves of $\lambda$ are pointwise limits of leaves in $\lambda_n$.

So it suffices to show that for $N \in \Leaves \lambda$, $P \in N$, and $P_n \to P$, where $P_n \in N_n$ and $N_n \in \Leaves \lambda_n$, $\normal_{N_n}(P_n) \to \normal_N(P)$.
To do this, suppose that $P \in U_\alpha$; $F_\alpha^n$ is close in tangential $C^\infty$ to $F_\alpha$, and the label $k^n_\alpha$ of $N_n$ is close to the label $k_\alpha$ of $N$.
In particular, if we consider $N$ and $N_n$ as graphs of functions $u, u_n$ in the coordinates induced by $F_\alpha$, then $u_n \to u$ in $C^\infty$; however, in such coordinates, $u$ is a constant.
A bootstrapping argument based on (\ref{nabla as a normal}) then shows that, since $\dif u_n \to 0$ in $C^0$, $\normal_{N_n} \to \partial_y = \normal_N$ in $C^0$ near $P$.

\subsubsection{Convergence in the measure topology}
Suppose that $\mu_n$ is transverse to $\lambda_n$.
After possibly shrinking the $U_\alpha$ slightly for $\alpha \in A_E$, we may assume that they are precompact in $M$ and still form an open cover of $E$.
Then $K := \bigcup_{\alpha \in A_E} \overline{U_\alpha}$ is compact, so by Prohorov's theorem \cite[Theorem 13.29]{klenke2013probability}, there is a subsequence of $(T_{\mu_n})$ which converges to some $T_\mu|_K$ on $K$.
Since $\mu_n(E) \gtrsim 1$, $T_\mu$ is nonzero.
Moreover, by Proposition \ref{portmanteau}
$$\supp T_\mu|_K \subseteq \liminf_{n \to \infty} \supp T_{\mu_n}|_K \subseteq \liminf_{n \to \infty} \supp \lambda_n \cap K.$$
Here the $(\lambda_n)$ in the limit inferior refers to the subsequence which already converges in the Thurston topology (and has converging Ruelle-Sullivan currents).
In particular, the limit inferior is actually a limit and we conclude
$$\supp T_\mu|_K \subseteq \supp \lambda \cap K.$$
We may assume that $\mu_\alpha^n \to \mu_\alpha$ weakly for every $\alpha \in A_E$ and some positive Radon measures $\mu_\alpha$ (whose support is necessarily then contained in $K_\alpha$).
Taking the limit as $n \to \infty$ of the equation 
$$\int_{U_\alpha} T_{\mu_n} \wedge \varphi = \int_I \int_{\{k\} \times J} (F_\alpha^n)^* \varphi \dif \mu_\alpha^n(k),$$
we conclude that
$$\int_{U_\alpha} T_\mu|_K \wedge \varphi = \int_I \int_{\{k\} \times J} F_\alpha^* \varphi \dif \mu_\alpha(k).$$
In other words, $T_\mu|_K$ is Ruelle-Sullivan for $\lambda|_K$, possibly after shrinking $\lambda|_K$ so that their supports match.
By the measure-preserving condition in the definition of transverse measure, $T_\mu|_K$ extends uniquely to a Ruelle-Sullivan current $T_\mu$ on all of $M$, which then necessarily is a weak limit of $(T_{\mu_n})$.
This completes the proof of Theorem \ref{compactness theorem}.


%%%%%%%%%%%%%%%%%%%%%%%%%%%%%%%%%%%%%%
\subsection{Consequences of measured convergence}\label{relationships between modes}
We now apply Theorems \ref{main thm} and \ref{compactness theorem} to explain how the different modes of convergence are related to each other.
It is clear from the definitions that flow-box convergence implies Thurston convergence.
Moreover, for $d = 2$, Thurston claimed that that measure convergence implies Thurston convergence \cite[Proposition 8.10.3]{thurston1979geometry}, though he did not explicitly justify why the limit was geodesic, or why the convergence preserves the normal vectors.
We complete the proof that measure convergence implies Thurston convergence, and show that flow-box convergence sits in the middle of the chain of implications.

\begin{lemma}\label{limits of measured geodesic lams are geodesic}
Suppose that $d \leq 7$. The set of minimal measured laminations of bounded curvature is closed in the weak topology of measures.
\end{lemma}
\begin{proof}
Let $(\lambda, \mu)$ be a measured lamination and suppose that $(\lambda_i, \mu_i) \to (\lambda, \mu)$ in the weak topology of measures, where $(\lambda_i, \mu_i)$ are measured minimal and of bounded curvature.
By Proposition \ref{minimal implies locally minimizing} and the curvature bound, for every $x \in M$ there exists $r > 0$ such that every leaf of every lamination $\lambda_i$ is absolutely area-minimizing in $B(x, r)$.
After shrinking $r$ if necessary, we may assume that $H^1(B(x, r), \RR) = 0$.
Then, by Theorem \ref{main thm}, the Ruelle-Sullivan currents $T_{\mu_i}$ on $B(x, r)$ are the exterior derivatives of functions $u_i$ of least gradient.
Since $u_i$ is only defined up to a constant, we impose $\int_M \star u_i = 0$, so by Poincar\'e's inequality,
$$\|u_i\|_{L^1(B)} \lesssim r\mu_i(B(x, r)) \lesssim r^d < \infty$$
for $i$ large.
So by Miranda compactness (Proposition \ref{MirandaStability}), there exists a function $u$ of least gradient such that along a subsequence, $\dif u_i \to \dif u$ in the weak topology of measures.
Then $T = \dif \mu$, so the leaves of $\lambda$ are level sets of $u$.
By Theorem \ref{main thm of old paper}, the leaves of $\lambda$ are minimal hypersurfaces, as desired.
\end{proof}

\begin{proposition}
Suppose that $d \leq 7$.
Let $(\lambda_n, \mu_n)$ be measured minimal laminations in $M$, and $(\lambda_n, \mu_n) \to (\lambda, \mu)$.
Then $\lambda_n \to \lambda$ in Thurston's geometric topology.
\end{proposition}
\begin{proof}
By (\ref{supports shrink in the limit}), for every $x \in \supp \lambda$, $\varepsilon > 0$, and large $n$, $\supp \lambda_n \cap B(x, \varepsilon)$ is nonempty, and by Lemma \ref{limits of measured geodesic lams are geodesic}, $\lambda$ is a minimal lamination.
By Theorem \ref{regularity theorem}, $\lambda, \lambda_n$ admit Lipschitz normal vectors, so by Lemma \ref{convergence of normals}, $\lambda_n \to \lambda$ in Thurston's geometric topology.
\end{proof}

\begin{proposition}\label{convergence of traansverse measures means flow box convergence}
Suppose that $d \leq 7$.
Let $(\lambda_n, \mu_n)$ be measured minimal laminations in $M$ of bounded curvature, and $(\lambda_n, \mu_n) \to (\lambda, \mu)$.
Then $\lambda_n \to \lambda$ in the $C^{1-}$ and tangentially $C^\infty$ flow box topology.
\end{proposition}
\begin{proof}
We first observe that $\lambda_n \to \lambda$ in Thurston's geometric topology.
After discarding some leaves of $\lambda_n$ we may assume that $\lambda$ is a maximal limit for the Thurston topology.
Moreover, every subsequence $(\lambda_{n_k})$ has a further subsequence $(\lambda_{n_{k_\ell}})$ which converges to some maximal limit $\tilde \lambda$ in the $C^{1-}$ flow box topology by Theorem \ref{compactness theorem}.
But convergence in the flow box topology implies convergence in Thurston's topology, so $\tilde \lambda = \lambda$.
Since $(\lambda_{n_k})$ was arbitrary, it follows that $\lambda_n \to \lambda$ in the $C^{1-}$ flow box topology.
\end{proof}



%%%%%%%%%%%%%%%%%%%%%%

\chapter{Convex duality between minimal laminations and tight calibrations}\label{bestcurl}

\chapter{\texorpdfstring{$\infty$}{Infinity}-harmonic maps from surface are absolute minimizers}


\printbibliography

\end{document}
