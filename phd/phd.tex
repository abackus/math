\documentclass[final,12pt, leqno]{brownthesis}
\usepackage{lmodern}
\usepackage{setspace}
\usepackage{colonequals}
\usepackage[all,cmtip]{xy}
%\usepackage[alphabetic]{amsrefs}
\usepackage[T1]{fontenc} 
\usepackage{enumitem,kantlipsum}
\usepackage{textcomp} 
\usepackage{bm}
\usepackage{dsfont}
\usepackage{titlesec,float}
\usepackage{subfig}
\usepackage{mathtools,mathrsfs,epigraph,hyperref}
\usepackage{array}

\titleformat{\chapter}[display]
{\bfseries\LARGE}
{\filleft\MakeUppercase{\chaptertitlename} \Huge\thechapter}
{4ex}
{\titlerule
\vspace{2ex}%
\filright}
[\vspace{2ex}%
\titlerule]

\usepackage{amsmath,amssymb,graphicx,amsfonts,amsthm}
\usepackage{tikz,tikz-cd}

\makeatletter
\g@addto@macro\normalsize{%
  \setlength\abovedisplayskip{6pt}
  \setlength\belowdisplayskip{6pt}
  \setlength\abovedisplayshortskip{6pt}
  \setlength\belowdisplayshortskip{6pt}
}
\makeatother

\usepackage{amssymb}

\let\oldemptyset\emptyset
\let\emptyset\varnothing
%Questions

\newcommand{\todo}[1]{\textcolor{blue}{TODO: #1}}
\newcommand{\note}[1]{\textcolor{green}{Note: #1}}

\allowdisplaybreaks









\newcommand{\NN}{\mathbf{N}}
\newcommand{\ZZ}{\mathbf{Z}}
\newcommand{\QQ}{\mathbf{Q}}
\newcommand{\RR}{\mathbf{R}}
\newcommand{\CC}{\mathbf{C}}
\newcommand{\DD}{\mathbf{D}}
\newcommand{\PP}{\mathbf P}
\newcommand{\MM}{\mathbf M}
\newcommand{\II}{\mathbf I}
\newcommand{\Hyp}{\mathbf H}
\newcommand{\Sph}{\mathbf S}
\newcommand{\Group}{\mathbf G}
\newcommand{\GL}{\mathbf{GL}}
\newcommand{\Orth}{\mathbf{O}}
\newcommand{\SpOrth}{\mathbf{SO}}
\newcommand{\Ball}{\mathbf{B}}

\DeclareMathOperator*{\Expect}{\mathbf E}

\DeclareMathOperator{\avg}{avg}
\DeclareMathOperator{\card}{card}
\DeclareMathOperator{\codim}{codim}
\DeclareMathOperator{\diag}{diag}
\DeclareMathOperator{\diam}{diam}
\DeclareMathOperator{\dom}{dom}
\DeclareMathOperator{\Exc}{Exc}
\DeclareMathOperator{\Lip}{Lip}
\DeclareMathOperator{\Hom}{Hom}
\DeclareMathOperator{\id}{id}
\DeclareMathOperator{\rad}{rad}
\DeclareMathOperator{\rank}{rank}
\DeclareMathOperator{\Rm}{Rm}
\DeclareMathOperator{\Hess}{Hess}
\DeclareMathOperator{\sgn}{sgn}
\DeclareMathOperator{\supp}{supp}
\newcommand{\tr}{\operatorname{tr}}

\newcommand{\Mink}{\mathbf m}
\newcommand{\Ric}{\mathrm{Ric}}
\newcommand{\Riem}{\mathrm{Riem}}
\newcommand*\dif{\mathop{}\!\mathrm{d}}
\newcommand*\Dif{\mathop{}\!\mathrm{D}}
\newcommand{\LapQL}{\Delta^{\mathrm{ql}}}

\newcommand{\dbar}{\overline \partial}

\DeclareMathOperator{\atanh}{atanh}
\DeclareMathOperator{\csch}{csch}
\DeclareMathOperator{\sech}{sech}

\DeclareMathOperator{\Div}{div}
\DeclareMathOperator{\Gram}{Gram}
\DeclareMathOperator{\grad}{grad}
\DeclareMathOperator{\dist}{dist}
\DeclareMathOperator{\spn}{span}
\DeclareMathOperator{\Ell}{Ell}
\DeclareMathOperator{\WF}{WF}

\newcommand{\Two}{\mathrm{I\!I}}

\newcommand{\Lagrange}{\mathscr L}
\newcommand{\DirQL}{\mathscr D^{\mathrm{ql}}}
\newcommand{\DirL}{\mathscr D}

\newcommand{\Leaves}{\mathcal L}
\newcommand{\Hypspace}{\mathscr H}

\newcommand{\normal}{\mathbf n}
\newcommand{\vol}{\mathrm{vol}}
\newcommand{\dfn}[1]{\emph{#1}\index{#1}}

\renewcommand{\Re}{\operatorname{Re}}
\renewcommand{\Im}{\operatorname{Im}}

\newcommand{\loc}{\mathrm{loc}}
\newcommand{\cpt}{\mathrm{cpt}}

\def\Japan#1{\left \langle #1 \right \rangle}

\newtheorem{theorem}{Theorem}[section]
\newtheorem{badtheorem}[theorem]{``Theorem"}
\newtheorem{prop}[theorem]{Proposition}
\newtheorem{lemma}[theorem]{Lemma}
\newtheorem{sublemma}[theorem]{Sublemma}
\newtheorem{proposition}[theorem]{Proposition}
\newtheorem{corollary}[theorem]{Corollary}
\newtheorem{conjecture}[theorem]{Conjecture}
\newtheorem{axiom}[theorem]{Axiom}
\newtheorem{assumption}[theorem]{Assumption}

\newtheorem{mainthm}{Theorem}
\renewcommand{\themainthm}{\Alph{mainthm}}

% \newtheorem{claim}{Claim}[theorem]
% \renewcommand{\theclaim}{\thetheorem\Alph{claim}}
\newtheorem*{claim}{Claim}

\theoremstyle{definition}
\newtheorem{definition}[theorem]{Definition}
\newtheorem{remark}[theorem]{Remark}
\newtheorem{example}[theorem]{Example}
\newtheorem{notation}[theorem]{Notation}

\newtheorem{exercise}[theorem]{Discussion topic}
\newtheorem{homework}[theorem]{Homework}
\newtheorem{problem}[theorem]{Problem}

\makeatletter
\newcommand{\proofpart}[2]{%
  \par
  \addvspace{\medskipamount}%
  \noindent\emph{Part #1: #2.}
}
\makeatother



\numberwithin{equation}{section}


% Mean
\def\Xint#1{\mathchoice
{\XXint\displaystyle\textstyle{#1}}%
{\XXint\textstyle\scriptstyle{#1}}%
{\XXint\scriptstyle\scriptscriptstyle{#1}}%
{\XXint\scriptscriptstyle\scriptscriptstyle{#1}}%
\!\int}
\def\XXint#1#2#3{{\setbox0=\hbox{$#1{#2#3}{\int}$ }
\vcenter{\hbox{$#2#3$ }}\kern-.6\wd0}}
\def\ddashint{\Xint=}
\def\dashint{\Xint-}


\usepackage[backend=bibtex,style=alphabetic,giveninits=true,hyperref,backref,backrefstyle=none]{biblatex}
\addbibresource{reference.bib}
\renewbibmacro{in:}{}
\DeclareFieldFormat{pages}{#1}


% \usepackage[footnotesize,bf]{caption}  % Reduces caption font sizes
\begin{document}
\doublespacing
\pagenumbering{alph} % hack to suppress hyperref errors
\submitdate{May 2025}
\title{\huge \textbf{The geometry of p-elliptic equations}}
\author{Aidan Backus}
%\degree=2
%\university{Brown University}
\dept{Mathematics}
%\faculty{Graduate School}
\degrees{
	} 
	
\principaladvisor{Georgios Daskalopolous}
 \reader{}
\reader{}
 \dean{Thomas A. Lewis}
  %\abstract{}

      %\abstractpage
      %\abstractpage
     \beforepreface
       \prefacesection{Vita}

\newpage


\prefacesection{Acknowledgments}
I would like to first and foremost thank and acknowledge my intellectual debt to my advisor, Georgios Daskalopolous, who suggested much of this work and always had helpful comments and suggestions.
I would also like to thank Georgios' advisor, Karen Uhlenbeck, whose influence is also apparent throughout this work.
Indeed, much of this thesis can be viewed as an expansion of the seminal work \cite{daskalopoulos2020transverse} of Georgios and Karen on best Lipschitz functions as solutions of the $\infty$-Laplace equation.

Since this work is rather indisciplinary, I have often consulted with other researchers in various fields.
The dramatis personae here includes but is not limited to Christine Breiner, Tai Borges, Kaya Ferendo, Tom Goodwillie, Haram Ko, Jeremy Kahn, JiaHua Zou, and various pseudonymous members of the Analysts' Lair chat room on Discord.

The proof of \todo{the set theoretic lemma} was shown to me by Stephen Obinna.
Victor Bangert allowed me to view an early draft of \todo{his unfinished paper with Auer} which was a heavy influence on \todo{the stable norm parts}.
Chao Li graciously helped me understand his work \cite{Chodosh2021} which is a key ingredient in the theory developed in this thesis in the special case that the dimension $d = 4$.

My topics committee, consisting of Georgios, Christine, and Benoit Pausader, and my thesis committee, consisting of Georgios \todo{and the others}, both provided vital feedback which has influenced this work and my intellectual development more broadly.

My Ph.D. was supported financially by the National Science Foundation's Graduate Research Fellowship Program under Grant No. DGE-2040433.

Finally, as this Ph.D. was written at a time of crisis in the world (namely the COVID-19 pandemic) and in my personal life (as it was finished in spite of a car accident, legal troubles, and neuropsychological challenges), I would like to express gratitude for the immense emotional support I have received during my Ph.D., especially from my wife, Java Darleen Villano, but also my parents, Antoinette and Garrett Backus, and the Brown University math department Ph.D. coclasses of 2020 and 2021.
It is reasonable to assume that this work would not have been finished without it.



\tableofcontents
%\listoftables
\listoffigures
\afterpreface
\doublespacing




%\end{preliminaries}
%\newpage
%\endofprelim

\pagestyle{myheadings}


%------------------------ CONTENT ------------------------%
 %\ChNameUpperCase
 
 
 
  %%%%%%INTRO CHAPTER
\chapter{Introduction}

\section{Preliminaries and notation}

\chapter{A review of Daskalopolous--Uhlenbeck duality}


\chapter{Level sets of 1-harmonic functions}
\epigraph{Lifeblood and rubies red, precious coins cast aglow \\ 
Four hands our fourtunes mend, that want we may not know \\
And when we reach our ends, to scales we gladly go \\
Our fates are yours to bend, to hells and heavens flow
}{Kathryn Cwynar, ``In the Balance''}
\section{Introduction}
Throughout this paper, let $M$ be an oriented Riemannian manifold of metric $g$ and dimension $d$.
For a function $u \in BV_\loc(M)$, we write $\star |\dif u|$ for the total variation of the derivative, c.f. (\ref{total variation}).

\begin{definition}\label{main definitions}
A function $u \in BV_\loc(M)$ has \dfn{least gradient}, or is \dfn{$1$-harmonic}, if for every $v \in BV_\cpt(M)$,
\begin{equation}\label{least gradient functional}
\int_M \star |\dif u| \leq \int_M \star |\dif u + \dif v|.
\end{equation}
A set $U$ of locally finite perimeter has \dfn{least perimeter} if $1_U$ has least gradient.
\end{definition}

Aside from being of intrinsic interest, functions of least gradient arise naturally in applied mathematics via magnetic resonance imaging \cite{Tamasan2019, Joy09} and the continuum-time limit of certain combinatorial games \cite{Kohn06}.

The Euler-Lagrange equation for (\ref{least gradient functional}) is the $1$-Laplace equation\footnote{The precise definition for a weak solution of (\ref{1Laplacian}) is given by \cite{Mazon14}, but we will only work with the functional (\ref{least gradient functional}) and not the equation itself, so we will not state the definition of weak solution here.}
\begin{equation}\label{1Laplacian}
\dif^* \left(\frac{\dif u}{|\dif u|}\right) = 0.
\end{equation}
At least formally, (\ref{1Laplacian}) implies that the level sets of $u$ have zero mean curvature, and so one expects them to be minimal hypersurfaces.
It is a classical result that the level sets of a $1$-harmonic function have least perimeter \cite{BOMBIERI1969}, and this proof works in high generality.
The de Giorgi--Miranda regularity theorem \cite{deGiorgi61, Miranda66} shows that for $d \leq 7$ the level sets are then smooth minimal hypersurfaces, as one would formally expect from the $1$-Laplace equation (\ref{1Laplacian}).
We refer to the monograph of Giusti \cite[Part 1]{Giusti77} for an exposition of the proof of the de Giorgi--Miranda theorem.

The proof of the de Giorgi--Miranda regularity theorem strongly uses the flatness of the domain to take averages of $1$-forms in a covariant way.
In this paper, we show how one can take averages of $1$-forms in an approximately covariant way as long as the domain has appropriate symmetries.
Thus we are able to weaken the hypothesis of flatness in the de Giorgi--Miranda theorem and prove:

\begin{theorem}[de Giorgi--Miranda regularity theorem]\label{main lma}
Let $2 \leq d \leq 7$ and suppose that $M$ has constant sectional curvature.
Then every set of least perimeter in $M$ is bounded by embedded analytic stable minimal hypersurfaces.
\end{theorem}

%%%%%%%%%%%%%%%%%%%
\subsection{Applications of the main theorem}
To motivate the study of the level sets of $1$-harmonic functions, we turn to the other side of H\"older duality -- that is, to the $\infty$-Laplacian
$$\Delta_\infty v := (\nabla^\mu \partial^\nu v) \partial_\mu v \partial_\nu v.$$
The equation $\Delta_\infty v = 0$ is invariant under target translations $v \mapsto v + y$ and so Noether's theorem associates a conserved flux\footnote{By a \dfn{conserved flux}, we mean a closed $d-1$-form $\psi$; the equation $\dif \psi = 0$ can be viewed as a conservation law.} $\dif u$ to each $\infty$-harmonic function $v$.
Daskalopolous--Uhlenbeck \cite{daskalopoulos2020transverse,daskalopoulosPrep1} show that if $M = \Hyp^2$, then $u$ is a $1$-harmonic function, and the level sets of $u$ are geodesics contained in the \dfn{maximum stretch locus} $\{|\dif v| = \|\dif v\|_{L^\infty}\}$.
Maximum stretch loci of $\infty$-harmonic (or more generally best-Lipschitz) functions on surfaces were studied by Thurston, who showed them to be \dfn{geodesic laminations} -- that is, closed sets which are foliated by geodesics \cite{thurston1979geometry,thurston1998minimal}.
Thus the level sets of a $1$-harmonic function which arises as the potential of a Noetherian flux for the $\infty$-Laplacian form a geodesic lamination as well.

It is natural to expect that the same ideas hold in higher generality.
In general dimension one does not obtain a geodesic lamination but rather a \dfn{minimal lamination} -- that is, a closed set which is foliated by hypersurfaces of zero mean curvature; these hypersurfaces are called \dfn{leaves}.
In the sequel paper \cite{BackusCML}, we will apply Theorem \ref{main lma} to carefully treat the compactness theory of minimal laminations, reconciling the two competing approaches of Thurston \cite{thurston1979geometry} and Colding--Minicozzi \cite{ColdingMinicozziIV}.
We will then apply Theorem \ref{main lma} and the compactness theory to prove the following conjecture of Daskalopoulos--Uhlenbeck \cite[Problem 9.4, Conjecture 9.5]{daskalopoulos2020transverse}:

\begin{theorem}\label{main thm}
Let $M$ be a manifold of constant sectional curvature and dimension $2 \leq d \leq 4$.
\begin{enumerate}
\item Let $u$ be a $1$-harmonic function on $M$.
Then:
\begin{enumerate}
\item $\bigcup_{y \in \RR} \partial \{u > y\}$ is the support of a minimal lamination $\lambda$.
\item The leaves of $\lambda$ are the connected components of the level sets $\partial \{u > y\}$.
\item There is a measured oriented structure on $\lambda$ whose Ruelle-Sullivan current is $\dif u$.
\end{enumerate}
\item Conversely, if $\lambda$ is a minimal measured oriented lamination with Ruelle-Sullivan current $T$, and $\tilde M \to M$ is the universal cover of $M$, then there exists a $1$-harmonic, $\pi_1(M)$-equivariant function $u: \tilde M \to \RR$, such that $\dif u$ drops to $T$.
\end{enumerate}
\end{theorem}

Some definitions are in order.
A \dfn{Ruelle-Sullivan current} for $\lambda$ is a $d-1$-current $T$ such that for any compactly supported $d-1$-form $\varphi$,
$$\int_M T \wedge \varphi := \int_K \int_{\{k\} \times \RR^{d - 1}} \varphi \dif \mu(k)$$
where $\lambda$ is $K \times \RR^{d - 1}$, thus $\{k\} \times \RR^{d - 1}$ is a leaf of $\lambda$, and $\mu$ is a Radon measure on the space of leaves $K$.
The choice of orientation of the leaves, and the Radon measure $\mu$, are the \dfn{measured oriented structure} that we imposed on $\lambda$.

In Theorem \ref{main thm}, $T = \dif u$ has the role of the Noetherian flux for the $\infty$-Laplacian in the case $d = 2$.
However, it is not clear how, or even if, the $\infty$-Laplacian itself should enter the picture if $d \geq 3$.
We hope to return to this point in a later paper.

We believe that Theorem \ref{main thm} will be of interest to topologists, who may take the present paper as a black box, and thus only read \cite{BackusCML}.
Conversely, analysts will find no topology in this paper, besides the elementary theory of currents that we review in \S\ref{Prelims}.

Let us remark on one further application of Theorem \ref{main lma}.
G\'orny \cite[Theorem 1.2]{górny2017planar} uses the euclidean de Giorgi--Miranda theorem to prove a regularity theorem for functions of least gradient.
However, this proof does not elsewhere use the flatness of the domain, so we have:

\begin{theorem}[G\'orny's regularity theorem]
Let $2 \leq d \leq 7$ and suppose that $M$ is a simply connected manifold of constant sectional curvature.
Then any function $u: M \to \RR$ of least gradient can be written $u = u_j + u_c$ where $u_c$ is a continuous function of least gradient and $u_j$ is a jump function of least gradient.
\end{theorem}

This shows that $1$-harmonic functions are somewhat more regular than a general $BV$ function, as $z \mapsto \sqrt z$ on $\CC \setminus \RR_-$ does not admit such a decomposition \cite[Example 4.1]{Ambrosio2000FunctionsOB}.

%%%%%%%%%%%%%%

\subsection{Overview of the proof}
As in \cite{Miranda66, Giusti77}, we prove Theorem \ref{main lma} using the \dfn{de Giorgi lemma}, which controls the oscillation of the conormal $1$-form, or \dfn{excess}, to the reduced boundary to a set of least perimeter.
In this summary, we treat the case $M = \Hyp^d$ for simplicity.

In the euclidean case, the de Giorgi excess of a set $U$ of least perimeter in an open set $A$ is defined by
$$\Exc_A(U) := |\partial^* U \cap A| - \left|\int_A \partial_\mu 1_U \dif x^\mu \star 1\right|.$$
Here the first term is the surface measure of the reduced boundary of $U$ in $A$, and the integral in the second term is $\RR^d$-valued, using the identification $T_x'\RR^d \cong \RR^d$.
This identification is valid exactly because $\RR^d$ is flat; in particular, the excess is preserved by isometries of $\RR^d$, a key fact in the proof of the de Giorgi lemma.

In \S\ref{excess section} we resolve this conundrum.
We fix the coordinate frame $(\partial_\mu)$ obtained from the Poincar\'e ball model of hyperbolic geometry; then we push forward $(\partial_\mu)$ by isometries of $\Hyp^d$ to construct coordinate frames $(\partial_\mu^P)$ centered at each point $P \in \Hyp^d$.
We then define the excess
\begin{equation}\label{excess definition prelim}
\Exc_A(U, P) := |\partial^* U \cap A| - \left|\left[\int_A \partial_\mu^P 1_U \star 1\right] \dif x^\mu_P(P)\right|
\end{equation}
which is an element of $T_P' \Hyp^d$.
One can show that $\Exc_A(U, P)$ does not depend on the choice of isometries, but only on the basepoint $P$.

It remains to show that the excess respects translation along geodesics, possibly up to a perturbative term.
Following the strategy of \cite{daskalopoulosPrep1}, we embed $\Hyp^d$ in the Minkowski spacetime $\RR^{1, d}$ as the future unit hyperboloid.
One then obtains the following key estimate (Proposition \ref{translation invariance excess}): for $P, Q \in A$, $\rho := \diam A$,
\begin{equation}\label{almost translation invariance intro}
|\Exc_A(U, P) - \Exc_A(U, Q)| \lesssim \rho^{d + 1}.
\end{equation}
One can crudely predict this estimate as follows.
Scrutinizing (\ref{excess definition prelim}) we observe that, since we are integrating over the $d-1$-dimensional set $\partial^* U \cap A$, we must estimate the difference of integrands to be $O(\rho^2)$.
On the other hand, the scaling limit $\rho \to 0$ can also be viewed as the nonrelativistic limit, in which $\Hyp^d$ converges to the classical future unit slice $\{t = 1\}$.
So the non-translation-invariant tensor fields $\partial_\mu^P$, $\dif x^\mu_P$ converge to coordinate fields on flat space quadratically fast and we conclude the claim.

In \S\ref{MollifierSection} we recall Miranda's monotonicity formula \cite[Teorema 3.2]{Miranda66} for functions of least gradient (Proposition \ref{Monotone}).
It is used in three ways: to control the surface area of a minimal perimeter, to control the excess of a mollified minimal perimeter, and to show that a minimal perimeter has a tangent cone.

In \S\ref{Plateau section} we prove the de Giorgi lemma, Proposition \ref{de Giorgi}.
We seek to obtain the inductive bound
\begin{equation}\label{de Giorgi lemma intro}
\Exc_{B(P, r/2)}(U, P) \leq \frac{\Exc_{B(P, r)}(U, P)}{2^d} + O(r^{d + 1}),
\end{equation}
for a set $U$ of least perimeter such that $P \in \partial U$, as a standard argument shows that Theorem \ref{main lma} reduces to (\ref{de Giorgi lemma intro}).

We follow \cite[Chapters 6-7]{Giusti77} and begin by considering the case that $U$ is only an approximate minimizer of the area (\ref{least gradient functional}) and with $P$ merely close to $\partial U$, but such that $\partial U$ is $C^1$ and $|\nabla \normal_U|$ is small (Proposition \ref{Miranda44}).
By the approximate translation-invariance (\ref{almost translation invariance intro}), we may actually assume that $P \in \partial U$, and then a Taylor expansion of the metric in the coordinates $(x^\mu_P)$ reduces the problem to the euclidean case.
As in the euclidean case, one can use the monotonicity formula to show that mollification preserves the excess (Proposition \ref{main mollifier lemma}) which completes the proof.


%%%%%%%%%%%%%%%%%%%%%%%%%%%%%%%%%%%%%%%%%%%%%%%%

\subsection{Acknowledgements}
I would like to thank Georgios Daskalopoulos for suggesting this project and for many helpful discussions.
I would also like to thank Karen Uhlenbeck and Christine Breiner for helpful comments.

This research was supported by the National Science Foundation's Graduate Research Fellowship Program under Grant No. DGE-2040433.



%%%%%%%%%%%%%%%%%%%%%%%%%%%%%%%%%%%%%%%%%%%%%%%%%
\section{Preliminaries}\label{Prelims}
\subsection{Notation and conventions}
The operator $\star$ is the Hodge star, thus $\star 1$ is the Riemannian measure.
On a submanifold $\Sigma$ of codimension $\geq 1$, $\vol_\Sigma$ denotes the induced measure and $\star_\Sigma$ denotes the induced Hodge star. We also write $\star_\rho := \star_{\partial B(P, \rho)}$ if $P \in M$ is fixed.

When using the Einstein convention, Greek indices range over $0, 1, \dots$ while Latin indices range over $1, \dots$.
We write $y := x^0$.
Note carefully: we will never sum over indices on a Lorentzian manifold, but only on Riemannian manifolds.
Thus, if we have a timelike vector field $\partial_t$ we do not write $t = x^0$; $\partial_y = \partial_0$ is always \emph{spacelike}.
We use $\sharp, \flat$ for the musical isomorphisms: $(\varphi^\sharp)^\mu := g^{\mu \nu} \varphi_\nu$ and $X^\flat_\mu := g_{\mu \nu} X^\nu$.

We write $\Japan \xi := \sqrt{1 + |\xi|^2}$ for the Japanese norm of a vector $\xi$.

We consider the following manifolds: $\Ball^d$ is the unit ball in $\RR^d$, $\Sph^d$ the unit sphere in $\RR^{d + 1}$, $\Hyp^d$ is the hyperbolic space, and $\RR^{1, d}$ is the Minkowski spacetime.

%%%%%%%%%%%%%%%%%%%%%%%%%%%%%%%%%%%%%%%%%%%%%%%
\subsection{Functions of bounded variation}
An $\ell$-\dfn{current} on an open set $U$ is a continuous linear functional on the space of $C^\infty_\cpt$ differential $\ell$-forms on $U$.
We refer the reader to \cite{simon1983GMT} for a careful exposition of the theory of currents.
We write $\int_U \omega \wedge \psi$ for the pairing of an $\ell$-current $\omega$ with an $\ell$-form $\psi$ with compact support in $U$.
In particular, if $\varphi$ is an $d-\ell$-form, we identify it with its \dfn{Poincar\'e dual}, the $\ell$-current $\psi \mapsto \int_M \varphi \wedge \psi$.

We identify the derivative of a function $u$ with the $d-1$-current
$$\int_M \dif u \wedge \psi := -\int_M u \dif \psi,$$
which is well-defined as long as $u \in L^1_\loc(M)$.
For a vector field $X$, we write $\star (Xu) := \dif u \wedge \star (X^\flat)$.
A function $u$ has \dfn{bounded variation} if its total variation seminorm
\begin{equation}\label{total variation}
\int_M \star |\dif u| := \sup_{\substack{\|\psi\|_{C^0} \leq 1\\\supp \psi \Subset M}} \int_M \dif u \wedge \psi
\end{equation}
is finite. We write $BV(M)$ for the space of functions of bounded variation.
The local finiteness of $\int \star |\dif u|$, is diffeomorphism-invariant, and hence so is membership in $BV_\loc(M)$.

\begin{proposition}[trace theorem and Stokes formula]
Let $U \subseteq M$ be an open set with nonempty Lipschitz boundary, and $u \in BV(U)$.
Then the trace $v \in L^1(\partial U)$ is well-defined,
%and is characterized by the conditions that for $\vol_{\partial U}$-almost every $x$,
%\begin{equation}\label{convergence of trace}
%\int_{U \cap B(x, \varepsilon)} \star |v(x) - u| \ll \varepsilon^d,
%\end{equation}
and satisfies for every $d - 1$-form $\psi$,
\begin{equation}\label{Miranda IBP}
\int_U \dif u \wedge \psi + \int_U u \dif \psi = \int_{\partial U} v\psi.
\end{equation}
\end{proposition}
\begin{proof}
By a partition of unity argument we can reduce these results to \cite[Teorema 1]{Miranda67}.
\end{proof}

\begin{proposition}[polar decomposition]
For every $u \in BV_\loc(M)$ there exists a $\star |\dif u|$-measurable section $f$ of the cosphere bundle $S'M$ such that for every compactly supported $d-1$-form $\psi$,
\begin{equation}\label{RNy formula}
\int_M \dif u \wedge \psi = \int_M f|\dif u| \wedge \psi.
\end{equation}
\end{proposition}
\begin{proof}
This follows from \cite[Theorem 4.14]{simon1983GMT}.
\end{proof}

Let $f: M \to S'M$ be given by (\ref{RNy formula}).
As in \cite{Miranda66, Giusti77}, most of the technical work in this paper amounts to controlling the oscillation of $f$ at fine scales.
In order to make this precise, we shall need to take ``averages'' of $f$, but $f$ is a section of a curved vector bundle and so averaging is not well-defined.
We show that it is at least well-defined in the fine-scale limit, by proving a form of the Lebesgue differentiation theorem which is manifestly covariant.

To state our Lebesgue differentiation theorem, observe that if $\omega$ is a current with locally finite total variation $|\omega|$, then for any Riemannian metric, $\star|\omega|$ is a Radon measure, and the sheaves $L^p_\loc(\cdot, \star |\omega|)$, $p \in [1, \infty]$, are independent of the metric.
So we write $L^p_\loc(M, \omega)$ for such a sheaf.
We similarly refer to $\omega$-null sets and $\omega$-measurable sets and functions.

\begin{proposition}[Lebesgue differentiation theorem for a vector bundle]\label{LebesgueDiff}
Let $E \to M$ be a vector bundle over an oriented smooth manifold $M$, $\omega$ a current on $M$ with locally finite total variation $|\omega|$, and $f \in L^1_\loc(M, E, \omega)$.
Then there exists an $\omega$-null set $Z \subset M$ such that for every Riemannian metric on $M$, every trivialization $(F_1, \dots, F_\ell)$ of $E$ with dual trivialization $(F'_1, \dots, F'_\ell)$ of $E'$, and every $P \in M \setminus Z$,
$$f(P) = \lim_{r \to 0} \sum_{i=1}^\ell \left[\frac{\int_{B(P, r)} (F'_i, f) \star |\omega|}{\int_{B(P, r)} \star |\omega|}\right] F_i(P).$$
\end{proposition}

We shall apply this proposition with $E := T'M$, $F_\mu = \dif x^\mu$.
Note carefully that the terms inside the limit \emph{are} dependent on the metric and the choice of trivialization, thus the assertion is that the dependence goes away in the limit, and that the set on which the limit converges is independent.
Indeed, the idea is to scrutinize the proof of the Lebesgue differentiation theorem \cite[Chapter 3, Theorem 1.3]{stein2009real} and observe that the null sets constructed in the proof can be covered by null sets which do not depend on the Riemannian metric or the trivialization.

\begin{proof}
Choose a flat Riemannian metric, let $\dif \mu := \star |\omega|$, $\mathcal F = ((F_i), (F_i'))$ a pair of paralellizations of $E, E'$ such that $(F_i', F_j) = \delta_{ij}$, and $\ell$ the rank of $E$.
Then for every $\delta > 0$ there exists $\tilde f \in C_c(M, E)$ such that $\|f - \tilde f\|_{L^1(\mu)} < \delta$, thus
\begin{align*}
&\left|\sum_{i=1}^\ell \left[(F_i'(x), f(x)) - \dashint_{B(x, r)} (F_i', f) \dif \mu\right] F_i(x)\right| \\
&\qquad \leq \left|\sum_{i=1}^\ell (F_i'(x), f(x) - \tilde f(x)) F_i(x)\right| + \dashint_{B(x, r)} \left|\sum_{i=1}^\ell (F_i', f - \tilde f)F_i(x) \dif \mu \right| \\
&\qquad \qquad + \left|\sum_{i=1}^\ell \left[(F_i'(x), \tilde f(x)) - \dashint_{B(x, r)} (F_i, \tilde f) \dif \mu\right] F_i(x)\right| \\
&\qquad =: I_1(x) + I_{2, r}(x) + I_{3, r}(x).
\end{align*}
Here the integral defining $I_{2, r}(x)$ is valued in the fiber $E_x$.

By the proof of the Lebesgue differentiation theorem, $\{I_1 > \varepsilon\} \subseteq \{|f - \tilde f| > \varepsilon\}$ and $\{I_{2, r} > \varepsilon\} \subseteq \{\dashint_{B(x, r)} |f - \tilde f|\dif \mu\}$, which are $\mathcal F$-independent sets of measure $\lesssim \delta/\varepsilon$.
Meanwhile $I_{3, r} \to 0$ pointwise as $r \to 0$, so for
\begin{equation}\label{definition of null set}
Z_{\varepsilon, \mathcal F} := \left\{x \in M: \limsup_{r \to 0} \left|\sum_{i=1}^\ell \left[(F_i'(x), f(x)) - \dashint_{B(x, r)} (F_i', f) \dif \mu\right] F_i(x)\right| > 2\varepsilon\right\},
\end{equation}
one has
$$Z_{\varepsilon, \mathcal F} \subseteq \bigcap_{r > 0}\bigcup_{s <r} \{I_{1, s} > \varepsilon\} \cup \{I_{2, s} > \varepsilon\}.$$
The right-hand side is independent of $\mathcal F$ and $\delta$, but has $\mu$-measure $\lesssim \delta/\varepsilon$, so it is $\omega$-null.
Thus the union taken over all possible $\mathcal F$ and $\varepsilon$ is also $\omega$-null.

Now let $g$ be a Riemannian metric and $h$ our flat reference metric.
Then $\varphi := \sqrt{\det g/\det h}$ satisfies $\star_g|\omega| = \varphi \dif \mu$, and as $\varphi$ is continuous it does not contribute in the limit superior in (\ref{definition of null set}).
Moreover, the balls $B_g(x, r)$ have bounded eccentricity with respect to $h$, so we can replace $B(x, r)$ with $B_g(x, r)$ in (\ref{definition of null set}) without affecting $Z_{\varepsilon, \mathcal F}$ \cite[Chapter 3, Corollary 1.7]{stein2009real}.
\end{proof}

\begin{corollary}
The section $f: M \to S'M$ in the polar decomposition (\ref{RNy formula}) satisfies
\begin{equation}\label{Lebesgue point definition}
    f(P) = \left[\lim_{r \to 0} \frac{\int_{B(x, r)} \star \partial_\mu u}{\int_{B(x, r)} \star |\dif u|}\right] ~\dif x^\mu(P)
\end{equation}
for any coordinate system $(x^\mu)$ and any Riemannian metric $g$, and $\star|\dif u|$-almost every $P$.
The exceptional set does not depend on $(x^\mu)$ or $g$.
\end{corollary}

It follows from the above corollary that the following definitions, which a priori refer to the metric or to a choice of coordinate system, are actually completely determined by the smooth structure on $M$.

\begin{definition}
Let $U \subseteq M$. We say that $U$ has \dfn{locally finite perimeter} if $1_U \in BV_\loc(M)$.
In that case we make the following definitions:
\begin{enumerate}
\item The \dfn{measure-theoretic boundary} $\partial U$ is the set of points whose Lebesgue density with respect to $M$ is $\in (0, 1)$.
\item The polar section of $1_U$ is called the \dfn{conormal $1$-form} $\normal_U$ to $\partial U$.
\item The set of points $P$ for which $\normal_U(P)$ exists is the \dfn{reduced boundary} $\partial^* U$.
\item The \dfn{perimeter} $|\partial^* U \cap E|$ in a Borel set $E$ is $\int_E \star |\dif u|$.
\end{enumerate}
\end{definition}

Our definition of reduced boundary and conormal $1$-form follows \cite[Definition 3.3]{Giusti77} and is due to \cite{deGiorgi55}.
See \cite[Chapter 6]{Pugh02} for the definition of Lebesgue density.
Choosing a coordinate system on $M$ in which the volume form is $\dif x^0 \wedge \cdots \wedge \dif x^{d - 1}$, we see from \cite[Chapters 1-4]{Giusti77} that the following properties of the reduced boundary hold:

\begin{proposition}\label{locality of Caccioppoli}
    Let $U$ be a set of locally finite perimeter.
    Then:
    \begin{enumerate}
    % \item $\partial^* U$ is either empty or $d-1$-dimensional in the Hausdorff sense, and is $d-1$-rectifiable.
    \item $\partial^* U$ is a dense subset of $\partial U$.
    \item If $\normal_U$ extends to a continuous $1$-form on $\partial U$, then $\partial^* U = \partial U$ is a $C^1$ embedded hypersurface.
    \item If $\partial^* U = \partial U$ is a $C^1$ hypersurface, then $\normal_U$ is the conormal $1$-form on $\partial U$ as defined in differential topology, and $\star |\dif 1_U|$ is the induced measure on $\partial U$.
\end{enumerate}
\end{proposition}

% As a first application of Proposition \ref{locality of Caccioppoli} we recover the following formulation of the coarea formula.

\begin{proposition}[coarea formula]\label{Coarea2}
Let $u \in BV_\loc(M)$ and $E$ an open set. Then
\begin{equation}\label{coarea formula}
\int_E \star |\dif u| = \int_{-\infty}^\infty |E \cap \partial^* \{u > y\}| \dif y.
\end{equation}
\end{proposition}
\begin{proof}
Reasoning identically to \cite[Theorem 1.23]{Giusti77}, we may assume that $u \in C^\infty(M)$.
If this is true and also $u$ has no critical points, then (\ref{coarea formula}) follows from Fubini's theorem, the fact that $|E \cap \partial \{u > y\}|$ is the surface area of $E \cap \{u = y\}$ (by Proposition \ref{locality of Caccioppoli}), and the change-of-variables formula.
However the left-hand side of (\ref{coarea formula}) is unaffected by critical points of $u$, and the right-hand side of (\ref{coarea formula}) is unaffected by critical values of $u$ by Sard's theorem, so (\ref{coarea formula}) holds even if $u \in C^\infty(M)$ has critical points.
\end{proof}

%%%%%%%%%%%%%%%%%%%%%%%
\subsection{Functions of least gradient}
We write
$$\eta(u, U) := \inf_{v \in BV_\cpt(U)} \int_U \star |\dif(u + v)|$$
for $u \in BV_\loc(M)$ and $U \subseteq M$ open with Lipschitz boundary, thus $u$ has least gradient iff $\eta(u, U) = \int_U \star |\dif u|$ for every $U$.
% The Dirichlet problem for functions of least gradient does not depend on whether one optimizes over compactly supported perturbations, or the more general trace-free perturbations \cite{Sternberg93}, so
% $$\eta(u, U) = \inf_{v|_{\partial U} = 0} \int_U \star |\dif(u + v)|.$$
If $u, v \in BV(U)$, then using the coarea formula and reasoning analogously to \cite[Lemma 5.6]{Giusti77}, we obtain the a priori estimates
\begin{align}
|\eta(u, U) - \eta(v, U)| &\leq \|u - v\|_{L^1(\partial U)} \label{a priori estimate 1} \\
\eta(u, U) &\leq \|u\|_{L^1(\partial U)} \leq |\partial U| \cdot \|u\|_{L^\infty(M)}. \label{a priori estimate 2}
\end{align}

\begin{proposition}[Miranda stability theorem]\label{Miranda convergence}
If a sequence of functions $(u_n)$ (not necessarily of the same trace) satisfies for every open $U \Subset M$ with Lipschitz boundary
$$\limsup_{n \to \infty} \int_U \star |\dif u_n| \leq \liminf_{n \to \infty} \eta(u_n, U) < \infty,$$
and $u_n \to u$ in $L^1_\loc(M)$, then $u$ has least gradient, and $\dif u_n \to \dif u$ in the weak topology of measures.
\end{proposition}
\begin{proof}
The proof is similar to \cite[Teorema 3 and Osservazione 3]{Miranda67}, if we observe that we are allowed to add a term of size $o(1)$ to the right-hand side of the inequalities \cite[(2.8), (2.9), (2.13), and (2.14)]{Miranda67}.
The fact that the convergence in the weak topology of measures is equivalent to the convergence in the sense of \cite[Osservazione 3]{Miranda67} follows from the portmanteau theorem \cite[Theorem 13.16]{klenke2013probability}.
\end{proof}

%%%%%%%%%%%%%%%%%%%%%%%%%%%%%%%%%%%%%%%%%%%%%%
\section{Averages of differential forms}\label{excess section}
\subsection{Construction of averages}
The classical de Giorgi lemma is concerned with the rate of convergence in the Lebesgue differentiation theorem of the conormal $1$-form $\normal_U$ to a set $U$ of least perimeter.
As with most tools of harmonic analysis, the Lebesgue differentiation theorem breaks diffeomorphism symmetry.
This is true even when stated carefully as in Proposition \ref{LebesgueDiff}, which asserts that the limiting behavior is diffeomorphism-invariant, but \emph{not} that the rate of convergence of the averages is.

If the Levi-Civita connection $\nabla$ is flat, then $\nabla$ induces canonical isomorphisms $T_P'M \to T_Q'M$ for each $Q$ close enough to $P$ that we can ignore the effects of monodromy.
Hence $\nabla$ identifies $1$-forms $\xi$ defined near $P$ with $T_P'M$-valued functions $\tilde \xi$.
One can then define the average of $\xi$ with respect to a measure $\omega$,
\begin{equation}\label{averages and flat connections}
\avg_U \xi := \dashint_U \tilde \xi \dif \omega
\end{equation}
where the right-hand side is a vector-valued integral and $U \ni P$, and so $\avg_U \xi \in T_P'M$ (but could also be viewed as a $1$-form by parallel translation).
In particular, the choice of $P$ does not matter; in other words this notion of averaging respects translation and rotation symmetries.
This fails, however, if $\nabla$ has curvature, even if it arises from a metric $g$ with translation and rotation symmetries, since the holonomy group of $\nabla$ obstructs the naturality of the isomorphisms $T_P'M \to T_Q'M$.

Now suppose that $\nabla$ is the Levi-Civita connection of a metric $g$ with constant sectional curvature, thus $g$ has translation and rotation symmetries.
Then $\nabla$ still has holonomy, but we can define averages as follows.

Since $g$ has constant sectional curvature $K \in \RR$, then we can cover $M$ by charts $(x^\mu)$ in which $g$ takes the form
\begin{equation}\label{constant sectional curvature metric}
g_{\mu\nu} = \frac{\delta_{\mu\nu}}{(1 + K|x|^2/4)^2}.
\end{equation}
Without loss of generality, $M$ is equal to such a chart.
We fix the origin $O$, where $x = 0$, and introduce a smooth family $(\Phi^P)_{P \in M}$ of oriented isometries $M \to M$, such that $\Phi^O$ is a rotation and $\Phi^P(O) = P$.
We write
$$\partial^P_\mu := \Phi^P_* \partial_{x^\mu}$$
and $x^\mu_P := x^\mu \circ \Phi^P$.
The choices of $(x^\mu)$ and $(\Phi^P)$ amount to selecting an oriented coordinate frame based at $P$ in which the metric takes the form (\ref{constant sectional curvature metric}).
The family of frames $(\partial^P_\mu)$ is uniquely determined up to a \dfn{gauge transformation} -- that is, a section $\chi$ of the bundle $\SpOrth(TM) \to M$.

\begin{definition}
The \dfn{average} of a $1$-form $\xi$ over an open set $U$ based at $P$ with respect to a measure $\omega$ to be
$$\avg_{U, P, \omega} \xi := \left[\dashint_U (\xi, \partial^P_\mu) \dif \omega\right] \dif x^\mu(P).$$
\end{definition}

We sometimes suppress the subscripts $U, P, \omega$ if they are clear or irrelevant.
By definition, $\avg_{U, P} \xi$ is again an element of $T_P'M$, and one can check that if $K = 0$ then it equals (\ref{averages and flat connections}).
Proposition \ref{LebesgueDiff} implies that $\avg_{B(P, r), P} \xi$ converges to $\xi(P)$ as $r \to 0$ unless $P$ is an element of an $\omega$-null set.
It is also gauge invariant (or equivalently rotation invariant): if we obtain $\widetilde{\avg_{U, P}} \xi$ from a coordinate system obtained by rotating $\Phi^P$ by $\chi(P) \in \SpOrth(T_PM)$, then
$$\widetilde{\avg_{U, P, \omega}} \xi = \left[\dashint_{B(P, r)} \chi^\nu_\mu \xi_\nu \dif \omega\right] (\chi(P)^{-1})_\lambda^\mu \dif x^\lambda_P(P) = \avg_{U, P} \xi.$$
The average also satisfies the following sort of weak translation-invariance:

\begin{proposition}\label{translation invariance}
The average in an open set $A$ satisfies
$$||\avg_{A, P} \xi| - |\avg_{A, Q} \xi|| \lesssim (\diam A)^2 \|\xi\|_{C^0}.$$
\end{proposition}

%%%%%%%%%%%%%%%%%%
\subsection{Proof of weak translation-invariance}
In the proof of Proposition \ref{translation invariance}, we will fill in the details in the case $K < 0$, as the case $K = 0$ is trivial and the case $K > 0$ is similar and strictly easier; we shall briefly remark on that later.

We first rapidly review the facts about the hyperboloid model that we will need.
For a more thorough discussion which uses many of the same ideas as the proof of Proposition \ref{translation invariance}, see \cite[\S3.1, \S4.1]{daskalopoulosPrep1}.
Let $\RR^{1, d} = \RR_t \times \RR_y \times \RR_x^{d - 1}$ be the Minkowski spacetime with its metric $-\dif t^2 + \dif y^2 + |\dif x|^2$.
Expressing $\Hyp^d$ in the Poincar\'e ball model, we obtain an embedding
\begin{align*}
\Psi: \Hyp^d &\to \RR^{1, d} \\
x &\mapsto \frac{1}{1 - |x|^2/4 - y^2/4} \begin{bmatrix}1 + |x|^2/4 + y^2/4\\y \\ x\end{bmatrix}
\end{align*}
whose image is the unit future hyperboloid, as in the proof of \cite[Proposition 3.5]{lee1997riemannian}, and $\Psi(O) = (1, 0, 0)$.
This embedding induces a split exact sequence of $\SpOrth^+(\RR^{1, d})$-bundles over $\Hyp^d$
\begin{equation}\label{splitting of tangent bundle}
0 \to T\Hyp^d \to \Hyp^d \times \RR^{1, d} \to N\Hyp^d \to 0
\end{equation}
where $N\Hyp^d$ is the normal bundle of the embedding $\Psi$ \cite[(3.4)]{daskalopoulosPrep1}.
Here $\SpOrth^+(\RR^{1, d})$ is the properly orthochronous Lorentz group.

\begin{lemma}
The splitting (\ref{splitting of tangent bundle}) induces canonical isomorphisms of $\SpOrth^+(\RR^{1, d})$-representations
\begin{equation}\label{SES}
\RR^{1, d} = N'_P \Hyp^d \oplus T'_P \Hyp^d.
\end{equation}
for each $P \in \Hyp^d$.
Writing $\xi_O$ for the orthogonal projection of $\xi \in T'_P \Hyp^d \subseteq \RR^{1, d}$ to $T'_O \Hyp^d$,
\begin{equation}\label{Dask estimate}
|\xi| \leq |\xi_O| \leq e^{\dist(O, P)^2/2} |\xi|.
\end{equation}
\end{lemma}
\begin{proof}
We consider the adjoint sequence
$$0 \to N' \Hyp^d \to \Hyp^d \times \RR^{1, d} \to T' \Hyp^d \to 0$$
to (\ref{splitting of tangent bundle}).
Since (\ref{splitting of tangent bundle}) is split exact, we have a direct sum of $\SpOrth^+(\RR^{1, d})$-bundles $\Hyp^d \times \RR^{1, d} = T' \Hyp^d \oplus N' \Hyp^d$.
Here $\Hyp^d \times \RR^{1, d}$ is equipped with its trivial connection $\dif$ and has trivial monodromy group, so the parallel transport maps induce canonical isomorphisms between each fiber of $\Hyp^d \times \RR^{1, d}$ and $\RR^{1, d}$.
Therefore we have canonical isomorphisms (\ref{SES}).

Now (\ref{Dask estimate}) easily follows from \cite[\S4.1]{daskalopoulosPrep1}:\footnote{One may object that \cite[\S4.1]{daskalopoulosPrep1} refers to the splitting of the tangent bundle (\ref{splitting of tangent bundle}) rather than the cotangent bundle, but after conjugating by a musical isomorphism we see that the same estimates hold for the splitting of the cotangent bundle.}
By (\cite[(4.1)]{daskalopoulosPrep1}), $\xi_O$ is just $\xi$ with its timelike part $\xi_t$ set to $0$, thus
$$|\xi_O|^2 = |\xi_x|^2 + |\xi_y|^2 \geq |\xi_x|^2 + |\xi_y|^2 - |\xi_t|^2 = |\xi|^2.$$
On the other hand, by \cite[Lemma 4.2, Lemma 4.1]{daskalopoulosPrep1},
\begin{align*}
|\xi_O| &\leq (1 + |P - O|^2/2) |\xi| \leq (1 + \rho^2 \cosh \rho/2) |\xi| \leq e^{\rho^2/2} |\xi|. \qedhere
\end{align*}
\end{proof}

\begin{lemma}
The pushforward morphism $\Psi_*: T_{(x, y)} \Hyp^d \to T_{\Psi(x, y)} \RR^{1, d}$ is
\begin{equation}\label{pushforward estimates}
\Psi_* = \begin{bmatrix}y & x \\ 1 \\ & \id \end{bmatrix} + O(|x|^2 + y^2).
\end{equation}
\end{lemma}
\begin{proof}
We compute for $f(y, x) := |x|^2/4 + y^2/4$ that
\begin{align*}
\Psi_* &= \frac{1}{1 - f} \begin{bmatrix} \partial_y f & \partial_x f \\ 1 \\ & \id \end{bmatrix} + \frac{1}{(1 - f)^2} \begin{bmatrix}(1 + f) \partial_y f & (1 + f) \partial_x f \\
y \partial_y f & y \partial_x f \\
(\partial_y f) x & x \otimes \partial_x f
\end{bmatrix} \\
&= \begin{bmatrix}y & x \\ 1 \\ & \id \end{bmatrix} + O(|x|^2 + y^2)
\end{align*}
since $f(y, x) = O(|x|^2 + y^2)$ and $2\dif f(y, x) = (y, x)$.
\end{proof}

\begin{lemma}
Let $P \in \Hyp^d$ be given by $x^i = 0$, $y = \rho$. Consider the Lorentz boost
$$\Lambda := \begin{bmatrix}\cosh \rho & \sinh \rho \\ \sinh \rho & \cosh \rho \\ &&\id\end{bmatrix}$$
Up to a gauge transformation, it holds that for each $X \in \Hyp^d$, $Y := (\Phi^P)^{-1}(X)$, the below diagram commutes:
\begin{equation}\label{Lorentz boost diagram}
\begin{tikzcd}[column sep=50pt, row sep=30pt]
\RR^{1, d} \arrow[r, "\Lambda"] & \RR^{1, d} \\
T_Y \Hyp^d \arrow[u, "\dif \Psi(Y)"] \arrow[r, "\dif \Phi^P(Y)"] & T_X \Hyp^d \arrow[u, "\dif \Psi(X)"]
\end{tikzcd}
\end{equation}
\end{lemma}
\begin{proof}
Up to a gauge transformation, we may assume that $\Phi^P$ acts by hyperbolic translation along the $y$-axis.
On the other hand, $\Lambda \in \SpOrth^+(\RR^{1, d})$, so it preserves the unit future hyperboloid $\Psi(\Hyp^d)$ and acts by isometry.
Since $\Lambda$ clearly also preserves $\{(t, y, 0) \in \RR^{1, d}\}$, $\Lambda|\Psi(\Hyp^d)$ must act by hyperbolic translation on the image of the $y$-axis and hence the diagram
$$
\begin{tikzcd}
\RR^{1, d} \arrow[r, "\Lambda"] & \RR^{1, d} \\
\Hyp^d \arrow[u, "\Psi"] \arrow[r, "\Phi^P"] & \Hyp^d \arrow[u, "\Psi"]
\end{tikzcd}
$$
commutes. Linearizing this diagram at $Y$, we obtain (\ref{Lorentz boost diagram}).
\end{proof}

\begin{lemma}\label{DoVF lemma}
Let $A \subseteq \Hyp^d$ be an open set containing $O, P$.
Then up to a gauge transformation,
\begin{equation}\label{difference of vector fields}
\|\partial^P_\mu - \partial_\mu\|_{C^0(A)} \lesssim (\diam A)^2.
\end{equation}
\end{lemma}
\begin{proof}
We may assume by applying an isometry that $x^i(P) = 0$.
Let $X \in A$; we prove $|\partial^P_\mu(X) - \partial_\mu(X)| \lesssim \varepsilon^2$ for $\varepsilon := \diam A$.
Let $Y = (\Phi^P)^{-1}(X)$, so that $\partial^P_\mu(X) = \dif \Phi^P(Y) \partial_\mu(Y)$.
For $Z \in \Hyp^d$, we assign $T_Z \Hyp^d$ the basis $(\partial_\mu(Z))$.
Then the operator $I: T_Y \Hyp^d \to T_X \Hyp^d$ given by $I \partial_\mu(Y) = \partial_\mu(X)$ is represented by the identity matrix, and
$$|\partial^P_\mu(X) - \partial_\mu(X)| \leq |\dif \Phi^P(Y) - I| = |\dif \Psi(X) \circ \dif \Phi^P(Y) - \dif \Psi(X) \circ I|.$$
Up to a gauge transformation, (\ref{Lorentz boost diagram}) commutes, so
$$|\partial^P_\mu(X) - \partial_\mu(X)| \leq |\Lambda - \dif \Psi(Y) - \dif \Psi(X) \circ I|.$$
If $Y = (x^*, y^*)$, then $X = (x^*, y^* + \rho) + O(\varepsilon^2)$ \cite[(4.5.5)]{ratcliffe2006foundations} since $X \in A$, so
\begin{align*}
\Lambda \circ \dif \Psi(Y) &= \begin{bmatrix}1 & \rho \\ \rho & 1 \\ && \id \end{bmatrix} \begin{bmatrix}y^* & x^* \\ 1 \\ & \id \end{bmatrix} + O(\varepsilon^2) = \begin{bmatrix}y^* + \rho & x^* \\ 1 \\ & \id\end{bmatrix} \begin{bmatrix}1 \\ & \id\end{bmatrix} + O(\varepsilon^2) \\
&= \dif \Psi(X) \circ I + O(\varepsilon^2). \qedhere
\end{align*}
\end{proof}

\begin{proof}[Proof of Proposition \ref{translation invariance} for $K < 0$]
We may assume by rescaling that $K = -1$, by locality that $M = \Hyp^d$, by applying an isometry that $Q = O$, and by applying a gauge transformation that (\ref{difference of vector fields}) holds.
Writing
$$v := \avg_{A, P} \xi, \quad w := \avg_{A, Q} \xi$$
and $\varepsilon := \diam A$, we seek to show $\|v| - |w\| \lesssim \varepsilon^2$
Using the splitting of (\ref{SES}) we can view both covectors $v \in T_P' \Hyp^d, w \in T_O' \Hyp^d$ as elements of $\RR^{1, d}$.

We first estimate
$$\|v| - |v_O\| \leq \varepsilon^2 |v| \leq \varepsilon^2 \|\xi\|_{C^0}$$
using (\ref{Dask estimate}) and the fact that $e^{\varepsilon^2/2} - 1 \leq \varepsilon^2$ for $\varepsilon < 1$.
Moreover, the reverse triangle inequality $||v_O| - |w|| \leq |v_O - w|$ is valid because $v_O, w$ are elements of the spacelike subspace $T'_O \Hyp^d$ of $\RR^{1, d}$, thus
$$||v| - |w|| \leq ||v| - |v_O|| + ||v_O| - |w|| \leq \varepsilon^2 \|\xi\|_{C^0} + |v_O - w|.$$
Moreover,
\begin{align*}
|v_O - w| &= \left|\left[\dashint_A (\xi, \partial^P_\mu) \dif \omega\right] (\dif x_P^\mu(P))_O - \left[\dashint_A (\xi, \partial^O_\mu) \dif \omega\right] \dif x^\mu(O)\right| \\
&\leq \left[\dashint_A |\partial^P_\mu - \partial_\mu| \dif \omega\right] \cdot |\dif x^\mu(O)| + \left|\dashint_A (\xi, \partial_\mu^P) \dif \omega\right| \cdot |(\dif x^\mu_P(P))_O - \dif x^\mu(O)|\\
&=: I + J
\end{align*}
It is clear that $I \leq \sum_\mu \|\partial^P_\mu - \partial_\mu\|_{C^0(A)} \cdot \|\xi\|_{C^0}$, which is $\lesssim \varepsilon^2 \|\xi\|_{C^0}$ by (\ref{difference of vector fields}).
By the triangle inequality,
$$J \leq \sum_\mu \left[|(\dif x^\mu_P(P) - \dif x^\mu(P))_O| + |\dif x^\mu(P)_O - \dif x^\mu(O)|\right] \|\xi\|_{C^0}.$$
From (\ref{Dask estimate}), the fact that $e^{\varepsilon^2/2} \leq 2$, a musical isomorphism, and (\ref{difference of vector fields}), we dispose of the first term as
$$|(\dif x^\mu_P(P) - \dif x^\mu(P))_O| \leq 2 |\dif x^\mu_P(P) - \dif x^\mu(P)| \leq 2 \|\partial^P_\mu - \partial_\mu\|_{C^0(A)} \lesssim \varepsilon^2.$$
Finally, we recall that $\dif x^\mu(P)_O$ is exactly the spacelike part of $\dif \Psi^\mu(P)$.
So as an element of $\RR^{1, d}$, $\dif x^\mu(P)_O$ is the $\mu$th column of the matrix in (\ref{pushforward estimates}) with the first (that is, timelike) row set to $0$, plus $O(\varepsilon^2)$.
Therefore $\dif x^\mu(P)_O = \dif x^\mu(O) + O(\varepsilon^2)$.
\end{proof}

Now we sketch the case $K > 0$.
We identify $\Sph^d \setminus \{\infty\}$ with $\RR^d$ via stereographic projection, giving an embedding $\Psi: \RR^d \to \RR^{d + 1}$ as the unit sphere (minus its south pole).
It is easy to show that $\RR^{d + 1} = T'_P \Sph^d \oplus N'_P \Sph^d$ and $||\xi| - |\xi_O|| \lesssim \dist(O, P)^2 |\xi|$ whenever $\xi \in T_P \Sph^d$.
Here $O$ and $\infty$ are mutual antipodes, and there are no technicalities caused by an indefinite metric.
The pushforward formula (\ref{pushforward estimates}) still holds, and (\ref{Lorentz boost diagram}) holds with the Lorentz boost replaced by a rotation matrix.
We still have
$$\Phi^P(x^*, y^*) = (x^*, y^* + \rho) + O(|x^*|^2 + |y^*|^2),$$
since this just expresses the fact that translation in stereographic projection and spherical translation agree to second order.
Therefore (\ref{difference of vector fields}) holds and the rest of the proof is essentially identical.

%%%%%%%%%%%%%%%%%%%%%%%%%%%%%%%%%%%%%%%%%%%%%%%

\subsection{The excess}
We now introduce the quantity which governs the rate of convergence of the Lebesgue differentiation theorem for $\normal_U$, whenever $U$ is a set of locally finite perimeter.
More precisely, we study the convergence of the approximation
$$\normal_U(P, r) := \avg_{B(P, r), P, \star |\dif 1_U|} \normal_U.$$
Here $\star |\dif 1_U|$ can be viewed as the surface measure of $\partial^* U$, so $\normal_U(\cdot, r)$ is an averaged version of $\normal_U$.
It could be written more explicitly as
$$\normal_U(P, r) = \left[\int_{B(P, r)} \partial_\mu^P 1_U \star 1\right] \dif x_P^\mu(P).$$

\begin{lemma}\label{gauge invariance of the normal}
Let $U$ be a set of locally finite perimeter. Then
$$|\normal_U(P, r)| \leq e^{C|K|r^2}.$$
\end{lemma}
\begin{proof}
We compute for $M := |\partial^* U \cap B(P, r)|$ and $V := (\Phi^P)^{-1}(U)$ that
\begin{align*}
|\normal_U(P, r)|^2 &= |\normal_V(O, r)|^2 = M^{-2} \left|\sum_\mu \int_{B(O, r)} \partial_\mu 1_V \star 1\right|^2 \\
&\leq M^{-2} \max_\mu \|\partial_\mu\|_{C^0(B(0, r))}^2 |\partial^* V \cap B(O, r)|^2 \\
&\leq \max_\mu \|g_{\mu\mu}\|_{C^0(B(O, r))}.
\end{align*}
The claim now easily follows from the formula (\ref{constant sectional curvature metric}) for the metric.
\end{proof}

\begin{definition}
The \dfn{excess} of a set $U \subset M$ of locally finite perimeter at $P \in \partial U$ and in the open set $A \ni P$ with Lipschitz boundary is
$$\Exc_A(U, P) := |\partial^* U \cap A| - \left|\avg_{A, P, \star |\dif 1_U|} \normal_U\right|.$$
For $\rho > 0$ we write $\Exc_\rho(U, P) := \Exc_{B(P, \rho)}(U, P)$.
\end{definition}

We will control the rate of growth of the excess with the de Giorgi lemma, Proposition \ref{de Giorgi}.
For now, we just check that it has suitable symmetry properties:

\begin{lemma}
Let $U$ be a set of locally finite perimeter, let $A' \subseteq A$ be open sets with Lipschitz boundary, and let $P \in A'$. Then
\begin{equation}\label{approximate monotone}
-C |K| (\diam A')^2 |\partial^* U \cap A'| \leq \Exc_{A'}(U, P) \leq \Exc_A(U, P) + C |K|(\diam A)^2 |\partial^* U \cap A|.
\end{equation}
\end{lemma}
\begin{proof}
To deduce the lower bound on $\Exc_{A'}(U, P)$, we compute for $\rho := \diam A$ that
\begin{align*}
    \left|\int_{A'} \partial^P_\mu 1_U \star 1 \dif x_P^\mu(P)\right|
 & \leq \max_\mu \|\partial^P_\mu\|_{C^0(A')} \cdot |\partial^* U \cap A'| \leq e^{CK\rho^2} |\partial^* U \cap A'| \\
 & \leq |\partial^* U \cap A'| + C|K|\rho^2 |\partial^* U \cap A'|.
\end{align*}
The proof of the upper bound is similar.
\end{proof}

\begin{proposition}\label{translation invariance excess}
The excess satisfies for $P, Q \in A$
$$|\Exc_A(U, P) - \Exc_A(U, Q)| \lesssim (\diam A)^2 |\partial^* U \cap A|.$$
\end{proposition}
\begin{proof}
Immediate from Proposition \ref{translation invariance}.
\end{proof}

%%%%%%%%%%%%%%%%%%%%%%%%%%%%%%%%%%%%%

\section{Monotonicity formula}\label{MollifierSection}
Let $u$ be a function of least gradient, at first on $\RR^d$.
Then $u$ satisfies a monotonicity formula \cite[Theorem 5.12]{Giusti77}, and the main idea of the proof is to bound the growth of $\int \star |\dif u|$ using the vector-valued integral of $\dif u$ -- that is,
\begin{equation}\label{integral of du}
I(u, P, r) := \avg_{B(P, r), P, \star |\dif u|} \dif u \cdot \int_{B(P, r)} \star |\dif u|.
\end{equation}
We now apply the above averaging techniques to extend the monotonicity formula to the manifold case.

\begin{proposition}[monotonicity formula]\label{Monotone}
Let $u$ be a function of least gradient on a manifold $M$ of constant sectional curvature $K$.
Then there exists $0 \leq A \lesssim |K|$ such that for $0 < r_1 < r_2 \ll 1$,
\begin{equation}\label{weak monotonicity}
\frac{\dif}{\dif r}\left[e^{Ar^2}r^{1 - d} \int_{B(P, r)} \star |\dif u|\right] \geq 0.
\end{equation}
and
\begin{align*}
&|r_2^{1 - d} I(u, P, r_2) - r_1^{1 - d} I(u, P, r_1)|^2 \\
&\qquad \lesssim \left(1 + (d - 1) \log \frac{r_2}{r_1}\right) \left(r_2^{1 - d}\int_{B(P, r_2)} \star |\dif u| \right)
\left(\int_{r_1}^{r_2} \partial_r \left[e^{Ar^2} r^{1 - d} \int_{B(P, r)} \star |\dif u|\right] \dif r\right)\\
&\qquad \qquad + |K|^2 r_2^{6-2d} \left(\int_{B(P, r_2)} \star |\dif u|\right)^2.
\end{align*}
\end{proposition}

%%%%%%%%%%%%%%%%%%%%%
\subsection{Proof of the monotonicity formula}
Throughout the proof of Proposition \ref{Monotone} we write $\dif \sigma$ for the usual volume form on $\Sph^{d - 1}$.
We begin with an estimate for a smoothed out version of functions of least gradient.
Its euclidean case can be isolated from the proof of \cite[Lemma 5.8]{Giusti77}.

\begin{lemma}\label{monotonicity lemma}
There exists $A$ such that for every $u \in C^1(B_R)$, $0 < r_1 < r_2 < R$, if we let
$$E(r) = \int_{B_r} \star |\dif u| - \eta(u, r),$$
so that $E(R) = 0$ iff $u$ has least gradient, then there exists $A \geq 0$ such that for $R > 0$ small,
\begin{equation}\label{monotonicity lemma eqn}
0 \leq \int_{B_{r_2} \setminus B_{r_1}} \star r^{1 - d}\frac{(\partial_ru)^2}{|\dif u|} \leq 2\int_{r_1}^{r_2} \partial_r \left[e^{Ar^2} r^{1-d}\int_{B_r} \star |\dif u|\right] + \frac{O(E(r))}{r^d} \dif r.
\end{equation}
\end{lemma}
\begin{proof}
This result is coordinate-invariant, so we may use whichever coordinates are convenient: we in fact use normal polar coordinates $(r, \theta)$.
We fix $s \in [r_1, r_2]$ and introduce a competitor $v(r, \theta) = u(s, \theta)$.
From the definition of $\eta$,
\begin{equation}\label{consequence of least gradient monotone}
    \eta(u, s) \leq \int_U \star |\dif v| = \int_0^s \int_{\partial B_r} \star_r |\dif v| \dif r.
\end{equation}
We now recall that
$$\vol_\rho(\theta) = \left[\rho^{d - 1} - \frac{\rho^d}{3} \Ric_P(\theta, \theta) + O(\rho^{d + 1})\right] \dif \sigma(\theta)$$
where the implied constant depends on the curvature of $M$.
Thus we can find $A > 0$ such that for all $\rho$ small enough that $e^{A\rho^2} \sqrt{\det g|_{\partial B_\rho}}$ is monotone in $\rho$ for some $A > 0$, as long as $\rho$ is small enough.
Applying $\partial_r v = 0$ it follows that
\begin{equation}\label{introduce the ricci tensor}
\int_{\partial B_r} \star_r |\dif v| \leq e^{As^2} \frac{\tilde r^{d - 1}}{s^{d - 1}} \int_{\partial B_s} \star_s |\dif v|.
\end{equation}
Applying (\ref{consequence of least gradient monotone}) and Fubini's theorem,
\begin{align*}
\eta(u, s) &\leq e^{As^2} \int_0^s \frac{r^{d - 1}}{s^{d - 1}} \dif r \cdot \int_{\partial B_s} \star_s |\dif v| = \frac{s e^{As^2}}{d} \int_{\partial B_s} \star_s |\dif v|\\
&\leq \frac{s e^{As^2}}{d - 1} \int_{\partial B_s} \star_s |\dif v|.
\end{align*}
By Gauss' lemma, $\dif v$ is the orthogonal projection of $\dif u$ onto $T' \partial B_s$, and its orthocomplement is $\partial_r u$. Therefore by Taylor's theorem,
$$\int_{\partial B_s} \star_s |\dif v| \leq \int_{\partial B_s} \star_s |\dif u| \sqrt{1 - \frac{(\partial_r u)^2}{|\dif u|^2}} \leq \int_{\partial B_s} \star_s \left[|\dif u| - \frac{(\partial_r u)^2}{2 |\dif u|}\right]$$
or in other words
\begin{align*}
\int_{\partial B_s} \star_s \frac{(\partial_r u)^2}{2|\dif u|} &\leq \int_{\partial B_s} \star_s |\dif u| - \frac{d - 1}{s} e^{-As^2} \eta(u, s)\\
&\leq \int_{\partial B_s} \star_s |\dif u| - \frac{d - 1}{s} e^{-As^2} \int_{B_s} \star |\dif u| - O(s^{-1}E(s)).
\end{align*}
We moreover have for $\tilde A \geq 0$ that
$$e^{-\tilde As^2} \partial_s \left[e^{\tilde As^2} s^{1 - d} \int_{B_s} \star |\dif u|\right] = \left[2\tilde As^{2 - d} - \frac{d - 1}{s^d}\right]\int_{B_s} \star |\dif u| + s^{1 - d} \int_{\partial B_s} \star_s |\dif u|$$
so if we choose $\tilde A$ so that
$$-\frac{d - 1}{s} e^{-As^2} = 2\tilde As - \frac{d - 1}{s}$$
then
$$s^{1 - d} \int_{\partial B_s} \star_s |\dif u| - (d - 1)\frac{e^{-\tilde As^2}}{s^d} \int_{B_s} \star|\dif u| \leq e^{-\tilde As^2} \partial_s\left(e^{\tilde As^2} s^{1 - d} \int_{B_s} \star|\dif u|\right).$$
We moreover have $e^{-\tilde As^2} \leq 1$, so we can now integrate with respect to $\dif s$ and rename $\tilde A$ to $A$ to conclude.
\end{proof}

\begin{proof}[Proof of Proposition \ref{Monotone}]
Let
$$I_r := r^{1 - d} I(u, P, r) = r^{1 - d} \left[\int_{B(P, r)} \partial_\mu^P u \star 1\right] \dif x^\mu_P(P).$$
Applying the fundamental theorem of calculus and the fact that $\sqrt{\det g} = e^{-O(Kr^2)}$,
\begin{align*}
I_r &= r^{1 - d} \left[\int_{\partial B(P, r)} u \cdot (\normal, \partial_\mu^P) \star 1\right] \dif x^\mu_P(P) \\
&= \left[\int_{\Sph^{d - 1}} u(r, \theta) \dif \sigma(\theta)\right] \dif x^\mu_P(P) + O(|K|r^{3 - d}) \int_{B(P, r)} \star |\dif u|.
\end{align*}
and hence
\begin{equation}\label{monotone dump the metric}
|I_{r_2} - I_{r_1}| \leq \int_{\Sph^{d - 1}} |u(r_2, \theta) - u(r_1, \theta)| \dif \sigma(\theta) + O(|K|r^2) \int_{B(P, r_2)} \star |\dif u|.
\end{equation}
Henceforth we write $B_r := B(P, r)$.
The metric $g$ plays no role in the dominant term of (\ref{monotone dump the metric}), so we may use \cite[Lemma 5.3]{Giusti77} to bound
$$0 \leq \int_{\Sph^{d - 1}} |u(r_2, \theta) - u(r_1, \theta)| \dif \sigma(\theta) \leq \int_{\Sph^{d - 1}} \int_{r_1}^{r_2} r^{1 - d}|\partial_r u(r, \theta)| \dif r \dif\sigma(\theta).$$
To reintroduce the metric we posit that $r_2$ is small enough that $\dif r \dif \sigma(\theta) \leq \star 2$.
We therefore have
\begin{equation}\label{monotone before cs}
\int_{\Sph^{d - 1}} \int_{r_1}^{r_2} r^{1 - d}|\partial_r u(r, \theta)| \dif r \dif\sigma(\theta) \leq 2 \int_{B_{r_2} \setminus B_{r_1}} \star r^{1 - d}|\partial_r u|
\end{equation}
and if we apply the Cauchy-Schwarz inequality and approximate $u$ by $C^1$ functions as on \cite[pg68]{Giusti77}, it follows from Lemma \ref{monotonicity lemma} that the right-hand side of (\ref{monotone before cs}) is
$$\lesssim \sqrt{\int_{B_{r_2} \setminus B_{r_1}} \star r^{1 - d} |\dif u|} \sqrt{\int_{r_1}^{r_2} \partial_r \left[e^{Ar^2} r^{1-d}\int_{B_r} \star |\dif u|\right] \dif r}.$$
The monotonicity (\ref{weak monotonicity}) follows at once.

Integrating by parts,
\begin{align*}
\int_{B_{r_2} \setminus B_{r_1}} \star r^{1 - d} |\dif u| &= \int_{r_1}^{r_2} r^{1 - d} \partial_r \int_{B_r} \star |\dif u| \dif r \\
&\leq r^{1 - d} \int_{B_r} \star |\dif u| + (d - 1) \int_{r_1}^{r_2} r^{-d} \int_{B_r} \star |\dif u| \dif r.
\end{align*}
Using (\ref{weak monotonicity}) we bound this second integral as
\begin{align*}
\int_{r_1}^{r_2} r^{-d} \int_{B_r} \star |\dif u| \dif r &\leq r^{1 - d} \log \frac{r_2}{r_1} \int_{B_{r_2}} \star |\dif u|.
\end{align*}
If we set
$$J_r := r^{1 - d} \int_{B_r} \star |\dif u|$$
then we can sum up our progress so far as
$$|I_{r_2} - I_{r_1}| \lesssim \sqrt{\left[1 + \log \frac{r_2}{r_1}\right] J_{r_2}} \sqrt{e^{Ar_2^2} J_{r_2} - e^{Ar_1^2} J_{r_1}} + |K|r_2^2 J_{r_2}.$$
The claim now follows by squaring both sides and applying Cauchy-Schwarz.
\end{proof}

%%%%%%%%%%%%%%%%%%%%%%%%%%%%%%%%%%
\subsection{Minimal tangent cones}
Since we are considering the regularity of minimal surfaces, we need to use the $d \leq 7$ hypothesis to show the regularity of tangent cones to minimal surfaces.

\begin{definition}
    For a function $u$ on $M$, $P \in M$ we define the \dfn{blowup} of $u$ at $P$ to be the net of functions $u_t: T_PM \to \RR$, given by
    $$u_t(v) = u\left(\exp_P(tv)\right).$$
\end{definition}

\begin{proposition}\label{blowup theorem}
Suppose that $U$ is an open set with least perimeter in $B(P, r)$, $P \in \partial^* U$, and $u = 1_U$.
Furhermore, suppose that $d \leq 7$.
Then the blowup $(u_t)$ of $u$ converges as $t \to 0$ along a subsequence (that we also denote $t \to 0$) in $L^1_\loc$ and almost everywhere, to the indicator function $v$ of a half-space $C \subset T_PM$ such that $0 \in \partial C$.
Moreover, $\dif u_t \to \dif v$ in the weak topology of measures.
\end{proposition}
\begin{proof}
If we only require that $\partial C$ is a minimal cone rather than a hyperplane, then this is a standard consequence of the Miranda stability theorem (Proposition \ref{Miranda convergence}) and the monotonicity formula (\ref{weak monotonicity}), see for example \cite[Theorem 9.3]{Giusti77} for the euclidean case.
However, a minimal cone of dimension $\leq 6$ is necessarily a hyperplane \cite[Theorem 9.10 and Theorem 10.10]{Giusti77}.
\end{proof}

We now generalize the surface area estimates of \cite[Remark 5.13]{Giusti77}.

\begin{corollary}\label{doubling dimension}
If $d \leq 7$ then there exists $A \geq 0$ such that for every set $U$ of least perimeter in a ball $B_r = B(P, r)$, with $P \in \partial^* U$, and $r > 0$ small,
$$|\Ball^{d - 1}|e^{-Ar^2}r^{d - 1} \leq |\partial^*U \cap B_r| \leq |\Sph^{d - 1}|e^{Ar^2} r^{d - 1}.$$
\end{corollary}
\begin{proof}
The upper bound on $|\partial^* U \cap B_r|$ is obtained by using (\ref{a priori estimate 2}) and the fact that the surface area of $\partial B_r$ is $|\Sph^{d - 1}|(1 + O(r^2))r^{d - 1}$.
This can be seen by integrating $\star_{\partial B_r} 1$ along $\partial B_r$ in normal coordinates and applying the Taylor expansion of the Riemannian measure.
The lower bound is obtained from the monotonicity formula, which implies that
$$\limsup_{\rho \to 0} e^{-A\rho^2} \rho^{1 - d} |\partial^* U \cap B_\rho| \leq |\partial^* U \cap B_r|.$$
To control the left-hand side we take a blowup $(u_\rho)$ of $1_U$.
By Proposition \ref{blowup theorem} we can pass to a subsequence so that $u_\rho \to 1_C$ for $C$ a half-space, which in particular is transverse to $B'_1$, where the prime denotes the euclidean metric on the tangent space.
Then
\begin{align*}
\limsup_{\rho \to 0} e^{-A\rho^2} \rho^{1 - d} |\partial^* U \cap B_\rho| &= \lim_{\rho \to 0} e^{O(\rho^2)} \int_{B'_1} \star'|\dif u_\rho|' = \int_{B'_1} \star'|\dif 1_C|.
\end{align*}
This last term is $|\partial C \cap B'_1|$, the measure of the intersection of the euclidean unit ball with a hyperplane through its origin.
In other words it is the measure $|\Ball^{d - 1}|$ of the unit ball in $\RR^{d - 1}$.
\end{proof}

We will frequently use Corollary \ref{doubling dimension} to bound error terms.
For example, if $u = 1_U$ where $U$ has least perimeter, then the error term in the monotonicity formula is of size $O(|K|r_2^{d + 1})$.



%%%%%%%%%%%%%%%%%%%%%%%%%%%%%%%%%%%%%%%%%%%%%%
\section{Regularity of sets of least perimeter}\label{Plateau section}
\subsection{Induction on scale}
Fix a manifold $M$ of constant sectional curvature and dimension $2 \leq d \leq 7$.
We prove our main theorem, Theorem \ref{main lma}, for $M$.
Following \cite{Miranda66,Giusti77,deGiorgi61}, we proceed by controlling the excess using the following de Giorgi lemma \cite[Theorem 8.1]{Giusti77}:

\begin{proposition}[de Giorgi lemma]\label{de Giorgi}
There exist $C, c, \rho_* > 0$, such that for every $P \in M$, $\rho$ such that $0 < \rho < \rho_*$, and set $U \subset M$ of least perimeter such that
\begin{equation}\label{base case}
\Exc_\rho(U, P) \leq c\rho^{d - 1},
\end{equation}
we have
\begin{equation}\label{dGL concl}
\Exc_{\rho/2}(U, P) \leq 2^{-d} \Exc_\rho(U, P) + C|K|\rho^{d + 1}.
\end{equation}
The constants $C, c, \rho_*$ only depend on $d$ for $|K| \lesssim 1$ (but may explode as $|K| \to \infty$).
\end{proposition}

In order to apply the de Giorgi lemma we recall its consequences: the base case \cite[pg109]{Giusti77} and the inductive case \cite[Corollary 8.3]{Giusti77}.
There are not new ideas in these corollaries, but we do have a new error term in (\ref{dGL concl}) that was not in \cite[Theorem 8.1]{Giusti77}, so for completeness we reprove them.

\begin{corollary}[base case]
Assume the de Giorgi lemma, and let $U \subset M$ have least perimeter.
Then there exists $\rho = \rho(P) < \rho_*$, which is locally uniform in $P \in \partial U$, such that (\ref{base case}) holds.
\end{corollary}
\begin{proof}
Let $Q \in \partial^* U$; we shall choose $\rho$ uniformly in a small neighborhood of $Q$, which is enough since $\partial^* U$ is dense in $\partial U$.
Since $\partial U$ has a tangent space at $Q$ by Proposition \ref{blowup theorem}, $\Exc_r(U, Q) \ll r^{d - 1}$, thus we can choose $r \in (0, \rho_*)$ such that $\Exc_r(U, Q) \leq c(r/2)^{d - 1}$.
For $P \in B(Q, r/4)$ we have $B(P, r/4) \subseteq B(Q, r/2)$ and hence by the de Giorgi lemma, for $\rho := r/4$, (\ref{base case}) holds.
\end{proof}

\begin{corollary}[inductive case]
Assume the de Giorgi lemma, and let $U \subset M$ have least perimeter.
Then $\normal_U$ extends to a continuous $1$-form on $\partial U$, where $\partial U$ is endowed with the subspace topology induced by $M$.
\end{corollary}
\begin{proof}
We use the de Giorgi lemma inductively as in \cite[Theorem 8.2]{Giusti77}.
Suppose that $r/2 < s < r = \rho/2^n$ for some $n$ and some $\rho$ satisfying (\ref{base case}).
To fix notation, let
$\xi := \normal_U(P, r)$, $\eta = \normal_U(P, s)$, $m := |\partial^* U \cap B(P, s)|$, $M := |\partial^* U \cap B(P, r)|$, and $\gamma_n := \Exc_{\rho/2^n}(U, P)$.
We first estimate
$$|\xi - \eta|^2 = |\xi|^2 + |\eta|^2 - 2 g^{-1}(\xi, \eta) \leq 2(1 - g^{-1}(\xi, \eta)) + C|K|r^2.$$
To estimate the right-hand side we write
$$m(1 - g^{-1}(\xi, \eta)) = \int_{B(P, s)} \star(|\dif 1_U| - g^{-1}(\xi, \dif x^\mu_P(P)) \partial^P_\mu 1_U).$$
Now we bound
$$ g^{-1}(\xi, \dif x^\mu_P(P)) \partial^P_\mu 1_U \leq |\xi| \cdot |\dif 1_U| \cdot \max_\mu |\partial^P_\mu| \leq e^{C|K|r^2} |\dif 1_U|,$$
which implies that, since $s \leq r$,
$$\int_{B(P, s)} \star(|\dif 1_U| - g^{-1}(\xi, \dif x^\mu_P(P)) \partial^P_\mu 1_U) \leq \int_{B(P, r)} \star(e^{C|K|r^2} |\dif 1_U| - g^{-1}(\xi, \dif x^\mu_P(P)) \partial^P_\mu 1_U).$$
By Corollary \ref{doubling dimension}, the first term integrates to $\leq M + C|K|r^{d + 1}$, and so by definition of $\xi$,
\begin{align*}
\int_{B(P, r)} \star(e^{CKr^2} |\dif 1_U| - g^{-1}(\xi, \dif x^\mu_P(P)) \partial^P_\mu 1_U) &\leq M(1 - |\xi|^2) + C|K|r^{d + 1}\\
&\leq 2M(1 - |\xi|) + C|K|r^{d + 1}.
\end{align*}
But $M(1 - |\xi|)$ is exactly the definition of $\Exc_r(U, P)$, so putting everything together and using Corollary \ref{doubling dimension} to bound $m \gtrsim s^{d - 1} \gtrsim r^{d - 1}$, we have the bound
\begin{equation}\label{just need the excess}
|\xi - \eta|^2 \lesssim r^{1 - d} \Exc_r(U, P) + |K|r^2.
\end{equation}
Then, since $r = \rho(V)/2^n$, $\Exc_r(U, P) = \gamma_n$, and for $C' := C|K|\rho^{d + 1}$, we have
\begin{equation}\label{induction on gamma}
\gamma_n \leq \frac{\gamma_0}{2^{nd}} + \sum_{k=0}^n \frac{C'}{2^{k(d + 1) + (n - k)d}} \leq \frac{\gamma_0 + C'}{2^{nd}}.
\end{equation}
Indeed, by induction,
\begin{align*}
\gamma_{n + 1}
&\leq \frac{\gamma_0}{2^{(n + 1)d}} + 2^{-d} \sum_{k=0}^n \frac{C'}{2^{k(d + 1) + (n - k)d}} + \frac{C'}{2^{(n + 1)(d + 1)}} \\
&= \frac{\gamma_0}{2^{(n + 1)d}} + \sum_{k=0}^{n + 1} \frac{C'}{2^{k(d + 1) + (n + 1 - k)d}}
\end{align*}
and (\ref{induction on gamma}) follows by summing the geometric series.
Reasoning as on \cite[pg100]{Giusti77} we conclude that for \emph{any} $s < r < \rho$, not just those of the form $r/2 < s < r = \rho/2^n$,
$$|\normal_U(P, r) - \normal_U(P, s)| \lesssim \sqrt{\frac{r}{\rho}}$$
and so $\normal_U(\cdot, r) \to \normal_U$ locally uniformly.
Since $\normal_U(\cdot, r)$ is continuous, the claim follows.
\end{proof}

\begin{proof}[Proof of Theorem \ref{main lma}, assuming the de Giorgi lemma]
If $U$ is a set of least perimeter, then $\normal_U$ is continuous for the subspace topology on $\partial U$, so by Proposition \ref{locality of Caccioppoli} it is a $C^1$ embedded hypersurface.
Since $\partial U$ is locally area minimizing, it must have zero mean curvature and hence, by elliptic bootstrapping, is analytic.
Actually, since $U$ is an absolute minimizer of (\ref{least gradient functional}), the Hessian of the area functional must be positive-definite at $\partial U$, thus $\partial U$ is stable.
\end{proof}

%%%%%%%%%%%%%%%%%%%%%%%%%%%%%%%%%%%%%

\subsection{The smooth case}
We begin the proof of the de Giorgi lemma with the following special case which is analogous to \cite[Lemma 6.4]{Giusti77}.
To state it we fix some constants: $c_0 = c_0(d, K) > 0$ is a small given constant, $\alpha = 1/2 + O(c_0)$ is also given, and $c_1 = c_1(c_0, d)$ will be chosen in the proof of Proposition \ref{Miranda44}.
Constants called $C$, and implied constants, are assumed to only depend on $d$ for $|K| \lesssim 1$ but possibly blow up as $K \to \infty$.
We shall always assume that we are working in a gauge so that
\begin{equation}\label{oscillation of isometries}
\|\dif(P \mapsto \Phi^P)\|_{C^0} \lesssim 1.
\end{equation}
This can always be ensured by not introducing unnecessary oscillation in $P \mapsto \Phi^P$ when we change coordinates, as we will only change coordinates boundedly many times in each stage of the induction.

\begin{proposition}[de Giorgi lemma, $C^1$ case]\label{Miranda44}
Assume the gauge condition (\ref{oscillation of isometries}).
For every $\rho, \beta$ satisfying $0 < \rho \ll_{c_1} |K|^{-1/2}$ and $0 < \beta \lesssim_{c_0, d} 1$, and every set $U$ with $C^1$ boundary in $B(P, \rho)$, if
$$\Exc_\rho(U, P) \leq \beta$$
and on $B(O, \rho)$, we have
\begin{align}
(\normal_U, \partial_0^P) &\geq e^{-o(c_1^2)}, \label{Miranda44 normal hyp} \\
|\partial^* U \cap B(P, \rho)| &\leq \eta(U, B(P, \rho)) + c_1 \beta, \label{Miranda44 minimality hyp}
\end{align}
where $c_1 = c_1(c_0, d) \ll 1$ is to be chosen later, then
\begin{equation}\label{Miranda44 concl}
\Exc_{\alpha \rho} (U, P) \leq (\alpha^{d + 1} + O(c_0)) \beta + C|K|\rho^{d + 1}.
\end{equation}
\end{proposition}

We introduce the Lagrangian
$$\Lagrange(y, \xi) := \frac{\Japan{\xi}}{(1 + K(|x|^2 + y^2)/4)^{d - 1}} \dif x,$$
for $x \in \RR^{d - 1}$, $y \in \RR$, and $\xi \in T'_x \RR^{d - 1}$.
Here, as always, $\Japan \xi$ denotes the Japanese norm $\sqrt{1 + |\xi|^2}$.
For $K = 0$ this Lagrangian reduces to the usual minimal surface equation, as studied in \cite[Chapter 6]{Giusti77}.

We first show that critical points of $w \mapsto \Lagrange(w, \dif w)$ define minimal surfaces in $M$.
To this end we define for $w \in C^1(\Omega)$, $\Omega \subseteq \RR^{d - 1}$, the map $\Psi_w: \Omega \to M$ given by
$$\Psi_w(x)^i := x^i, \quad \Psi_w(x)^0 := w(x).$$

\begin{lemma}\label{Plateau setup lemma}
Let $w \in C^1(\Omega)$. Then $(\Psi_w^{-1})^* \Lagrange(w, \dif w)$ is the Riemannian measure on $\Psi_w(\Omega)$.
\end{lemma}
\begin{proof}
For $(\partial_i)$ the standard frame on $\Omega$, the metric on the image of $\Psi_w$ satisfies
\begin{align*}
\Psi_w^*(g|\Psi_w(\Omega))(\partial_i, \partial_j) &= g_{\mu\nu} \partial_i \Psi_w^\mu \partial_j \Psi_w^\nu \\
&= \frac{\delta_{\mu\nu}}{(1 + K(|x|^2 + w^2)/4)^2} (\partial_i \Psi_w^\mu \partial_j \Psi_w^\nu) \\
&= \frac{\delta_{ij} + \partial_i w \partial_j w}{(1 + K(|x|^2 + w^2)/4)^2}.
\end{align*}
By \cite[(24)]{Petersen2008} we have $\det((\delta_{ij} + \partial_i w \partial_j w)_{ij}) = 1 + |\dif w|^2$.
\end{proof}

We can view the derivative of a function $w: \Omega \to \RR$ as a map $\dif w: \Omega \to \RR^{d - 1}$.
In particular, it is well-defined to take the average $\avg_\rho \dif w$ of $\dif w$ over the ball $\mathscr B_\rho := \{x \in \RR^{d - 1}: |x| < \rho\}$, c.f. (\ref{averages and flat connections}).
Here and henceforth we use $\mathscr B$ to refer to balls in $\RR^{d - 1}$, and reserve $B$ for balls in $M$.

We now show that $\Lagrange(w, \dif w) - \Lagrange(w, \avg_\rho \dif w)$ is close to the analogous quantity for the euclidean case if $\rho$ is small.

\begin{lemma}
Let $\beta, \rho > 0$, $w \in C^1(B_\rho)$, $w(0) = 0$, $\|\dif w\|_{C^0} \ll_{c_0, K} 1$, and assume that $P \in M$ satisfies $\Psi_w(0) \in B(P, \rho)$. Then
\begin{align}
\int_{\mathscr B_\rho} \Lagrange(w, \dif w) - \Lagrange(w, \avg_\rho \dif w)
&= \int_{\mathscr B_\rho} \Japan{\dif w} \dif x - \Japan{\avg_\rho \dif w} \dif x + O(|K| \rho^{d + 1}) \label{compare Lagrangians}.
\end{align}
\end{lemma}
\begin{proof}
The left-hand side of (\ref{compare Lagrangians}) is
\begin{align*}
&\left|\int_{\mathscr B_\rho} ((1 + K(|x|^2 + w(x)^2)/4)^{1 - d} - 1)(\Japan{\dif w} - \Japan{\avg_\rho \dif w}) \dif x\right| \\
&\qquad \lesssim_d |K| \int_{\mathscr B_\rho} (|x|^2 + w(x)^2) \cdot \left|\Japan{\dif w} - \Japan{\avg_\rho \dif w}\right| \dif x.
\end{align*}
But $|\mathscr B_\rho| \sim \rho^{d - 1}$, and
$$|\Japan{\dif w} - \Japan{\avg_\rho \dif w}| \lesssim \exp(\|\dif w\|_{C^0}) \lesssim 1,$$
so
\begin{align*}
(|x|^2 + w(x)^2) &\leq (1 + \|\dif w\|_{C^0}^2) \rho^2 \lesssim \rho^2.
\end{align*}
This proves (\ref{compare Lagrangians}).
\end{proof}

We now prove an analogue of \cite[Lemma 6.3]{Giusti77}, which is the de Giorgi lemma for a minimal graph near $O$.
Thanks to (\ref{compare Lagrangians}), the proof is essentially the same as the euclidean case.

\begin{lemma}[de Giorgi lemma, minimal graphs]\label{Miranda43}
If $c_0 \ll_d 1$ then there exists $c_1 = c_1(c_0) > 0$ such that for every $\beta > 0$, $0 < \rho < 1$, and $w \in C^1(\mathscr B_\rho)$, if we denote by
$I_w$ the cylinder in $M$, $I := \{|x| < \rho\}$, and assume that $w(0) = 0$, $\|\dif w\|_{C^0} \leq c_1$, and
\begin{align}
\int_{\mathscr B_\rho} \Lagrange(w, \dif w) - \Lagrange(w, \avg_\rho \dif w) &\leq \beta \label{Miranda43 oscillation}, \\
\int_{\mathscr B_\rho} \Lagrange(w, \dif w) &\leq \eta(\Psi_w(\Omega), I) + c_1 \beta \label{Miranda43 minimality},
\end{align}
then
\begin{equation}\label{Miranda43 concl}
\int_{\mathscr B_{\alpha \rho}} \Lagrange(w, \dif w) - \Lagrange(w, \avg_{\alpha \rho} \dif w) \leq (\alpha^{d + 1} + c_0) \beta + C|K|\rho^{d + 1}.
\end{equation}
\end{lemma}
\begin{proof}
Let $u$ be the harmonic function on $\mathscr B_\rho$ which equals $w$ on $\partial \mathscr B_\rho$.
By elliptic regularity, the maximum principle for harmonic functions, and (\ref{Miranda43 minimality}),
$$\|u\|_{C^1} \lesssim \|u\|_{C^0} \leq \|w\|_{C^0} \leq \rho \|\dif w\|_{C^1} \leq c_1.$$
In particular, $\Japan{\dif u} \lesssim 1$ and $u(x)^2 \lesssim \rho^2$, so
\begin{align*}
&\left|\int_{\mathscr B_\rho} \Lagrange(w, \dif w) - \Lagrange(u, \dif u) - \Japan{\dif w} \dif x + \Japan{\dif u} \dif x\right| \\
&\qquad \leq |K| \int_{\mathscr B_\rho} (|x|^2 + w(x)^2) \Japan{\dif w} \dif x + (|x|^2 + u(x)^2) \Japan{\dif u} \dif x
\lesssim_d |K| \rho^{d + 1}.
\end{align*}
Since $u$ and $w$ have the same trace, $|N_u \cap I| \leq \eta(\Psi_w(\Omega), I)$, thus by Lemma \ref{Plateau setup lemma},
\begin{align*}
\int_{\mathscr B_\rho} \Lagrange(w, \dif w) - \Lagrange(u, \dif u) &\leq \int_{\mathscr B_\rho} \Lagrange(w, \dif w) - \eta(\Psi_w(\Omega), I) \leq c_1 \beta.
\end{align*}
Moreover, we meet the hypotheses of (\ref{compare Lagrangians}).
We can moreover replace $\beta$ with some $\beta' \in [\beta, \beta + C|K|\rho^{d + 1}]$ so that $u, w$ meet the hypotheses of \cite[Lemma 6.2]{Giusti77} which gives the result for $c_0 \ll_d 1$.
\end{proof}

In order to apply Lemma \ref{Miranda43} we must show that we can, after applying a suitable isometry, bound a set $U$ of locally finite perimeter by a set $\Psi_w(\Omega)$ for some suitable $w, \Omega$.
However, $\Psi_w(\Omega)$ was defined with reference to a choice of coordinates, and one could imagine that the ball $B$ that $\partial U$ embeds into fails to be convex in such coordinates, thus the domain $\Omega$ could fail to be contractible.
Thus in the Riemannian case we must check that the choice of coordinates does not mangle the convexity of $B$, or equivalently the second fundamental form of $\partial B$.

\begin{lemma}\label{convex balls}
Every ball of radius $\lesssim |K|^{-1/2}$ which contains $O$ appears convex in the Poincar\'e ball model (for $K < 0$) or stereographic projection (for $K > 0$).
\end{lemma}
\begin{proof}
We begin by recalling how the second fundamental form transforms under a conformal change of variables.
Suppose that $g = e^{2\varphi} \tilde g$ for some metric $\tilde g$ and scalar field $\varphi$.
Then by \cite[(12)]{Mondino18}, the second fundamental form of a hypersurface $S$ is
\begin{equation}\label{conformal second form}
\Two^S = \tilde \Two^S - (\tilde \normal_S^\sharp \varphi) \cdot \tilde g.
\end{equation}
In the special case that $g$ has constant sectional curvature $K$, we can take $\tilde g_{ij} = \delta_{ij}$ and
\begin{equation}\label{conformal factor}
\varphi(x) = -\log (1 + K|x|^2/4).
\end{equation}

We now compute $\Two^S$ for $S = \partial B(O, r)$, as follows.
The distance to $\{|x| = \tilde r\}$ from $O$ is
$$r = \int_0^{\tilde r} \sqrt{g_{00}(t, 0, \dots, 0)} \dif t = \int_0^{\tilde r} \frac{\dif t}{1 + \tilde Kt^2} = \frac{1}{\sqrt{\tilde K}} \arctan\left(\sqrt{\tilde K} r\right)$$
where $\tilde K := K/4$ and we have taken suitable holomorphic extensions of $\sqrt \cdot$ and $\arctan(\cdot)$.
Inverting this equation we see that on $S$,
%$$|x| = \frac{1}{\sqrt{\tilde K}} \tan \left(\sqrt{\tilde K} r\right) = -\frac{1}{\sqrt{-\tilde K}} \tanh \left(\sqrt{-\tilde K} r\right).$$
%Regardless of the sign of $K$ we conclude
$|x| = r - O(|K|r^3)$.
From (\ref{conformal second form}) and (\ref{conformal factor}), and the fact that $\tilde \normal_S^\sharp(x) = x$, we conclude
\begin{equation}\label{Two of a sphere}
\Two^S_{ij} = \frac{1}{|x|} \delta_{ij} + \frac{2K|x|}{1 + K|x|^2/2} \delta_{ij} = \frac{\delta_{ij}}{r + O(|K|r^3)} = \frac{g_{ij}}{r + O(|K|r^3)}.
\end{equation}

Now we return to the situation of $S = \partial B$ where $B = B(P, r)$ is not centered on $O$, but only contains $O$.
Since (\ref{Two of a sphere}) is tensorial, it remains true that $\Two^S = g/(r + O(|K|r^3))$.
We let $\tilde \Two$ denote the second fundamental form of $S$ in coordinates, so by (\ref{conformal second form}),
$$|\tilde \Two - \Two| \lesssim |\dif \varphi| \lesssim |K|r.$$
Thus for $r < \min(O(|K|^{-1/2}), 1/2)$, $\tilde \Two \geq r^{-1} I$ is positive-definite, so in coordinates $B$ is locally convex (up to a rotation, $S$ is locally the graph of a convex function).
By the Tietze convexity theorem, it follows that $B$ is convex in coordinates.
\end{proof}

We are now ready to represent a set in $M$ as a function.
Thanks to Lemma \ref{convex balls}, the proof of the next lemma is essentially the same as the construction of the sequence of functions $(\omega_j)$ in the proof of \cite[Lemma 6.4]{Giusti77}, but without the wholly unnecessary appeal to proof by contradiction.

\begin{lemma}\label{rep as a good graph}
For every $\rho \ll_{c_0, c_1, d} |K|^{-1/2}$, every set $U$ of $C^1$ perimeter in $B(P, \rho)$, and every $Q \in B(P, \rho)$, such that
\begin{align}
(\normal_U, \partial_0^Q) &\geq e^{-o(c_1^2)} \label{rep as a good graph hyp}
\end{align}
there exists an open set $\Omega \subset \RR^{d - 1}$, $w \in C^1(\Omega)$, and a ball $\mathscr B \subset \RR^{d - 1}$, such that $\partial U = \Phi^Q(\Psi_w(\Omega))$,
\begin{equation}\label{rep as a good graph small derivative}
\|\dif w\|_{C^0} \leq c_1,
\end{equation}
and the pullback $\Omega^\alpha$ of $\partial U \cap B(P, \alpha \rho)$ to $\RR^{d - 1}$ satisfies
\begin{equation}\label{rep as a good graph set nests}
    \Omega^\alpha \subseteq (\alpha + c_0) \mathscr B \subset \mathscr B \subseteq \Omega.
\end{equation}
\end{lemma}
\begin{proof}
Without loss of generality we may assume $Q = O$ and $\Phi^O$ is the identity. Then (\ref{rep as a good graph hyp}) simply asserts that $\normal := \normal_U$ satisfies $\normal_0 \geq e^{-o(c_1^2)}$.
By Lemma \ref{convex balls}, $B(P, \rho)$ appears convex in coordinates, so by \cite[Theorem 4.8]{Giusti77} there exists an open set $\Omega \subset \RR^{d - 1}$ and $w \in C^1(\Omega)$ such that $U$ is bounded by $\{y = w\}$ and
$$\|\dif w\|_{C^0} \leq \sup_{x_1, x_2 \in \Omega} \frac{|w_n(x_1) - w(x_2)|}{|x_1 - x_2|} \leq e^{o(c_1^2)}\sqrt{1 - e^{-o(c_1^2)}} \leq c_1.$$
Therefore (\ref{rep as a good graph small derivative}) holds.

We begin the proof of (\ref{rep as a good graph set nests}) by letting $\Psi_w(x_0, y_0) := P$.

\begin{sublemma}
Let $r \leq \rho$, $\Omega(r) := \Psi_w^{-1}(\partial U \cap B(P, r))$, and $S(r)$ the set of all $x \in \Omega$ such that there exists $y$ such that $(x, y) \in \partial B(P, r)$.
If $\Omega(r)$ is nonempty, then $\sigma^+(r) := \max_{x \in S(r)} |x - x_0|$ and $\sigma^- := \min_{x \in S(r)} |x - x_0|$ satisfy
$$\sigma^\pm(r)^2 = r^2 - (\inf_{\Omega'} w - y_0)^2 + O(c_1 \rho^2).$$
\end{sublemma}
\begin{proof}
Since $\Omega(r)$ is nonempty, $\partial U$ meets $B(P, r)$ and hence $\partial B(P, r)$ (since $\partial U$ also meets $\partial B(P, \rho)$ and is connected).
In particular, there is an element of $\partial U \cap \partial B(P, r)$ which witnesses that $S(r)$ is nonempty.

Writing $w^+ := \sup_{\Omega(r)} w$ and $w^- := \inf_{\Omega(r)} w$, and using (\ref{rep as a good graph small derivative}) and the fact that $\diam \Omega(r) \leq \diam \Omega \lesssim \rho$,
\begin{equation}\label{w is almost constant}
w^+ \leq w^- + \|\dif w\|_{C^0} \diam \Omega(r) \leq w^- + O(c_1 \rho).
\end{equation}
After discarding terms of size $O(|K| \rho^2)$ arising from the metric, we may assume that $B(P, r)$ is a euclidean ball and that $\sigma^\pm$ are the radii of disks in $B(P, r)$ centered on a point $(x_0, y)$ where $y \in [w^-, w^+]$, thus $y = w^- + O(c_1 \rho)$ by (\ref{w is almost constant}) and hence
$$(y - y_0)^2 = (w^- - y_0)^2 + (y - w^-)^2 + 2(y - w^-)(w^- - y_0) = (w^- - y_0)^2 + O(c_1 \rho^2).$$
So by the Pythagorean theorem applied to the triangle with legs $[y, y_0]$ and $[x_0, x_0 + \sigma^\pm]$,
\begin{align*}
r^2 &= (\sigma^\pm)^2 + (y - y_0)^2 + O(|K| \rho^2) = (\sigma^\pm)^2 + (w^- - y_0)^2 + O(c_1 \rho^2). \qedhere
\end{align*}
\end{proof}

From the definitions, $\mathscr B(x_0, \sigma^-(\rho)) \subseteq \Omega$ and $\mathscr B(x_0, \sigma^+(\alpha \rho)) \subseteq \Omega^\alpha$.
Since
$$\inf_{\Omega^\alpha} w \leq \sup_\Omega w \leq \inf_\Omega w + \diam \Omega \cdot \|\dif w\|_{C^0} \leq \inf_\Omega w + O(c_1 \rho^2)$$
and $|K| \rho^2 \leq c_1 \rho$ for $\rho \ll |K|^{-1}$, we have for $v := \inf_\Omega w - y_0$ that
$$\sigma^+(\alpha \rho)^2 = \alpha^2\left(\rho^2 - \frac{v^2}{\alpha^2}\right) + O(c_1 \rho^2) \leq \alpha^2(\rho^2 - v^2) + O(c_1 \rho^2) \leq \alpha^2 \sigma^-(\rho)^2 + O(c_1 \rho^2).$$
Moreover, since $\partial U \cap B(P, \alpha \rho)$ is nonempty,
$$\sigma^-(\rho)^2 \geq \rho^2 - (\alpha \rho - |K|\rho^3)^2 - O(c_1 \rho^2) \gtrsim \rho^2$$
which implies that
$$\sigma(\alpha \rho) \leq \sqrt{\alpha^2 \sigma^-(\rho)^2 + O(c_1 \rho^2)} \leq \alpha \sigma^-(\rho) + O(c_1) \frac{\rho^2}{\alpha \sigma^-(\rho)} \leq \alpha \sigma^-(\rho) + O(c_1 \rho).$$
In particular,
$$\Omega^\alpha \subseteq \mathscr B(x_0, \alpha \sigma^-(\rho) + O(c_1 \rho)) \subseteq \mathscr B(x_0, \sigma^-(\rho)) \subseteq \Omega$$
which, along with the fact that $O(c_1) \leq c_0$, gives the claim with $\mathscr B := \mathscr B(x_0, \sigma^-(\rho))$.
\end{proof}

The above setup allows us to show the $C^1$ case of the de Giorgi lemma, an analogue of \cite[Lemma 6.4]{Giusti77}.
We cannot quite proceed as in \cite{Giusti77}, however, as we need to carefully apply the approximate translation symmetry.

\begin{proof}[Proof of Proposition \ref{Miranda44}]
Throughout this proof we assume that there exists some $Q \in \partial U \cap B(P, \alpha \rho)$.
If not, then (\ref{Miranda44 concl}) is vacuous since then $\Exc_{\alpha \rho} (U, P) = 0$.

Let $I: T_YM \to T_XM$ be the identity matrix with respect to the coordinate frame $(\partial_\mu)$, where $(\Phi^Q)^{-1}(Y) = (\Phi^P)^{-1}(X)$ and $X \in B(P, \rho)$ is given.
By our gauge assumption (\ref{oscillation of isometries}),
$$|\Phi^Q \circ (\Phi^P)^{-1} - I| \lesssim \rho^2 + \|\dif(Z \mapsto \Phi^Z)\|_{C^0} \rho \lesssim \rho.$$
If $\rho$ is small enough depending on $c_1$ and the implicit constant in our gauge assumption, then $\rho \ll c_1^2$.
So by (\ref{Miranda44 normal hyp}), it follows that on $B(Q, \rho)$,
$$(\normal_U, \partial^Q_0) \geq O(\rho) + (\normal_U, \partial^P_0) \geq O(\rho) + e^{-o(c_1^2)} \geq e^{-o(c_1^2)}.$$
Therefore, by Lemma \ref{rep as a good graph},
there exist $\mathscr B, \Omega \subset \RR^{d - 1}$ satisfying (\ref{rep as a good graph set nests}), and $w: \Omega \to \RR$ such that $\|\dif w\|_{C^0} \leq c_1$, and such that the image of $\Phi^Q \circ \Psi_w$ is
$$\Gamma := \partial U \cap \{(x_Q^1, \dots, x_Q^{d - 1}) \in \Omega\}.$$
Since $Q \in \Gamma$, $w(0) = 0$.

Let $\mathscr B' := (\alpha + c_0) \mathscr B$, and for $W \subseteq \RR^{d - 1}$ open, introduce the cylinder
$$I_W := B(P, \rho) \cap \{(x^1_Q, \dots, x^{d - 1}_Q) \in W\}.$$
We write $\star'$ and $|\cdot|'$ for the euclidean Hodge star and absolute value with respect to the coordinates $(x^\mu_Q)$.
If we put
$$\Exc_A'(U) := \int_A \star' |\dif 1_U|' - \left|\int_A \star' (\normal_U, \partial^Q_\mu) \dif x^\mu_Q(Q)\right|',$$
then a Taylor expansion of the metric (\ref{constant sectional curvature metric}) gives for $A \subseteq B(Q, r)$ that
$$\Exc_A(U) = \Exc_A'(U) + O(r^{d + 1}).$$
On the other hand, by the proof of \cite[Lemma 6.4]{Giusti77}, for any ball $\mathscr D$ there is a dilate of $\mathscr D$ by a dilation of size $e^{O(c_0)}$ such that
\begin{align*}
\Exc_{I_{\mathscr D}}'(U) &= \int_{e^{O(c_0)} \mathscr D} \langle \dif w\rangle \dif x - \langle \avg_{e^{O(c_0)} \mathscr D} \dif w\rangle \dif x.
\end{align*}
Applying (\ref{compare Lagrangians}) we arrive at
\begin{equation}\label{excess versus lagrangian}
\Exc_{I_{\mathscr D}}(U) = \int_{e^{O(c_0)} \mathscr D} \Lagrange(w, \dif w) - \Lagrange(w, \avg_{e^{O(c_0)} \mathscr D} \dif w) + O(|K| \rho^{d + 1}).
\end{equation}

We are ready to prove (\ref{Miranda44 concl}).
By (\ref{rep as a good graph set nests}) and (\ref{excess versus lagrangian}),
\begin{align*}
\int_{e^{-O(c_0)} \mathscr B} \Lagrange(w, \dif w) - \Lagrange(w, \avg_{e^{-O(c_0)} \mathscr B} \dif w)
&\leq \Exc_{I_{\mathscr B}}(U) + O(|K| \rho^{d + 1}) \\
&\leq \Exc_\rho(U, P) + O(|K| \rho^{d + 1}) \\
&\leq \beta + O(|K| \rho^{d + 1}).
\end{align*}
By Lemma \ref{Miranda43}, it follows that
$$\int_{e^{O(c_0)} \mathscr B'} \Lagrange(w, \dif w) - \Lagrange(w, \avg_{e^{O(c_0)} \mathscr B'} \dif w) \leq (\alpha^d + O(c_0)) \beta + O(|K| \rho^{d + 1}).$$
Now applying (\ref{rep as a good graph set nests}) and (\ref{excess versus lagrangian}) again,
\begin{align*}
\Exc_{\alpha \rho}(U, P)
&\leq \Exc_{I_{\mathscr B'}}(U) + O(|K| \rho^{d + 1}) \\
&\leq \int_{e^{O(c_0)} \mathscr B'} \Lagrange(w, \dif w) - \Lagrange(w, \avg_{e^{O(c_0)} \mathscr B'} \dif w) + O(|K| \rho^{d + 1}) \\
&\leq (\alpha^d + O(c_0)) \beta + O(|K| \rho^{d + 1}). \qedhere
\end{align*}
\end{proof}

%%%%%%%%%%%%%%%%%%%%%%%%%%%%%%%%%%%%%%%%%%%
\subsection{Mollification}
We now reduce the de Giorgi lemma to its $C^1$ case.
To do so we need to introduce a suitable convolution kernel, but in order to carry out convolution we must work in coordinates.
To be more precise, the convolution $f * g$ of two functions defined near $O$, and the subtraction $x - y$ of two points near $O$, are defined in terms of the affine structure induced by the coordinates $(x^\mu)$.
In particular, neither of these operations are even approximately translation-invariant, and unlike in \cite{Giusti77} we must take care to ensure that this will not create unacceptable error terms.

Following \cite[Chapter 7]{Giusti77} we define the convolution kernel
$$\chi_\varepsilon(x) := \frac{d + 1}{|\Ball^d|} \varepsilon^{-d}1_{|x| < \varepsilon} \left(1 - \frac{|x|}{\varepsilon}\right)$$
We write $B_\varepsilon := B(O, \varepsilon)$ and $u_\varepsilon := u * \chi_\varepsilon$ whenever $u \in BV(B_{2\varepsilon})$.
When applied to sets of least perimeter with small excess, this convolution operator satisfies the following analogue of \cite[Theorem 7.3]{Giusti77}, which asserts that the convolution has a normal vector which is $C^0$ close to a constant.

\begin{proposition}\label{main mollifier lemma}
Let $q := \min(1/4, 1/(2(d - 1))$.
There exists $c > 0$ such that for every $0 < \rho \lesssim 1$, every $0 < \gamma \lesssim 1$, and every set $U$ of least perimeter such that
\begin{equation}\label{hypothesis on main mollifier lemma}
\Exc_\rho(U, O) \leq \gamma \rho^{d - 1},
\end{equation}
if we let $\varepsilon := \gamma^4\rho$, $\sigma := \gamma^{1/(2(d - 1))}\rho$, and $\varphi := (1_U)_\varepsilon$, then:
\begin{enumerate}
\item For every $y \in (c\gamma^2, 1 - c\gamma^2)$, the level set $\partial \{\varphi > y\} \cap B_{\rho - 2\sigma}$ is a $C^1$ hypersurface.
\item Possibly after applying a rotation around $O$, for every $x \in B_{\rho - 2\sigma}$ such that $c\gamma^2 < \varphi(x) < 1 - c\gamma^2$,
\begin{equation}\label{claim on main mollifier lemma}
\frac{\dif \varphi}{|\dif \varphi|} \geq 1 - O(\gamma^q) - O(\rho^2).
\end{equation}
\end{enumerate}
\end{proposition}

The estimate (\ref{claim on main mollifier lemma}) says that the normal vector to level sets $\{\varphi = y\}$, where $y$ is far from $\{0, 1\}$, is approximately constant in the coordinates $(x^\mu)$.
In particular, if $\gamma, \rho$ is small depending on $c_1$, then the level sets will satisfy (\ref{Miranda44 normal hyp}).
To deal with the somewhat large number of parameters involved here, it may helpful to think of the application to the de Giorgi lemma, in which case $c_1$ is frozen and $\gamma \sim \rho^2$, hence
$$1 \gg \gamma^q \gg \sigma \gg \rho \gg \gamma \gg \varepsilon \gg \varepsilon \delta > 0.$$
Since we can take losses of size $O(\rho^2)$, we shall mean by $|\cdot|$ the euclidean norm in the coordinates, and we shall work with the euclidean measure rather than $\star 1$.

The idea behind the proof of Proposition \ref{main mollifier lemma}, as in \cite{Giusti77}, is to cover $\partial^* U$ by small balls and apply the monotonicity formula in each ball.
Our next lemma thus estimates the behavior of $u$ in a single ball.

\begin{lemma}\label{mollifier sublemma}
Let $\delta := \gamma^d$, and let $u := 1_U$ satisfy
\begin{equation}\label{hypothesis on mollifier sublemma}
\int_{B_\rho} (|\dif u| - \partial_0 u)(z) \dif z \leq \rho^{d - 1} \gamma.
\end{equation}
Then for every $P \in \partial^* U \cap B_\varepsilon$ and $x \in B_{\rho - 2\sigma}$,
$$(1_{B(P, 2\delta\varepsilon)}(|\dif u| - \partial_0 u))_\varepsilon(x) \lesssim \gamma^q (1_{B(P, \delta\varepsilon)} |\dif u|)_\varepsilon(x).$$
\end{lemma}
\begin{proof}
We first claim that for $r > 0$ so small that $B(P, 2r) \subseteq B_\varepsilon$,
\begin{equation}\label{bound the kernel}
\sup_{y \in B(P, 2r)} \chi_\varepsilon(x - y) \lesssim \inf_{y \in B(P, r)} \chi_\varepsilon(x - y).
\end{equation}
In the euclidean case, this result can be isolated from the proof of \cite[Theorem 7.3]{Giusti77}.
Otherwise, we can use the smallness of $\varepsilon$ and the Taylor expansion of the metric to approximate $g$-balls by euclidean balls.
This suffices to prove (\ref{bound the kernel}), since $\chi_\varepsilon$ is uniformly continuous.

Now let $V := B(P, 2\delta\varepsilon)$.
Integrating (\ref{bound the kernel}) against $1_V(|\dif u| - \partial_0)$,
\begin{equation}\label{kernel bounded}
    (1_V(|\dif u| - \partial_0))_\varepsilon(x) \lesssim \inf_{y \in B(P, \delta\varepsilon)} \chi_\varepsilon(x - y) \int_V (|\dif u| - \partial_0 u)(z) \dif z.
\end{equation}
To estimate the right-hand side of (\ref{kernel bounded}), we assume that $\gamma$ is chosen so small that $\sigma > 2\delta\varepsilon$.
We then set $W := B(P, \sigma)$ and apply the weak monotonicity formula (\ref{weak monotonicity}) to get
$$(2\delta\varepsilon)^{1 - d} \int_V |\dif u|(z) \dif z \leq e^{O(|K| \sigma^2)} \sigma^{1 - d} \int_W |\dif u|(z) \dif z$$
and hence
\begin{align*}
(2\delta\varepsilon)^{1 - d} \int_V (|\dif u| - \partial_0 u)(z) \dif z
&\leq \sigma^{1 - d} \int_W (|\dif u| - \partial_0 u)(z) \dif z + O(|K|) \sigma^{3 - d} \int_W |\dif u|(z) \dif z \\
&\qquad + \sigma^{1 - d} \int_W \partial_0 u(z) \dif z - (2\delta\varepsilon)^{1 - d} \int_V \partial_0 u(z) \dif z\\
&=: I_1 + I_2 + I_3 - I_4.
\end{align*}

Before estimating the $I_i$, we next claim that we may assume that
\begin{equation}\label{submollifier gauge assumption}
|\partial_0 - \partial_0^P| \lesssim \gamma^{1/(d - 1)}
\end{equation}
on $W$. Indeed, the statement of Lemma \ref{mollifier sublemma} is independent of the choice of gauge $\Phi^P$, so we can make a gauge transformation, apply Lemma \ref{DoVF lemma}, and recall $O \in W$ (since $\sigma > \varepsilon$) to get $|\partial_0 - \partial_0^P| \lesssim \sigma^2$
which gives (\ref{submollifier gauge assumption}).

We then use (\ref{hypothesis on mollifier sublemma}) to bound
$$I_1 \leq \sigma^{1 - d} \int_{B_\rho} (|\dif u| - \partial_0 u)(z) \dif z \leq \gamma^{\frac{1 - d}{2(d - 1)}} \gamma = \gamma^{1/2}.$$
Also by Corollary \ref{doubling dimension} we have $I_2 \lesssim |K| \gamma^{1/(d - 1)}$.
We can ignore the $K$ since we are allowing constants to blow up as $K \to \infty$.

To estimate $I_3 - I_4$, we recall the notation (\ref{integral of du}) for vector-valued integrals, and apply the monotonicity formula, Proposition \ref{Monotone}, to compute
\begin{align*}
    I_3 - I_4 &= \sigma^{1 - d} I(u, Q, \sigma)_0 - (2 \delta \varepsilon)^{1 - d} I(u, Q, 2\delta\varepsilon)_0 + O(|K| \sigma^2) \\
    &\leq |\sigma^{1 - d} I(u, Q, \sigma) - (2 \delta \varepsilon)^{1 - d} I(u, Q, 2 \delta \varepsilon)| + O(|K| \sigma^2)) \\
    &\lesssim \sqrt{1 + (d - 1) \log \frac{\sigma}{2\delta\varepsilon}} \sqrt{\sigma^{1 - d} \int_W \star |\dif u|} \sqrt{\int_{2\delta\varepsilon}^\sigma \partial_r \left[e^{Ar^2} r^{1 - d} \int_{B(Q, r)} \star |\dif u|\right] \dif r}\\
&\qquad + |K| \sigma^{3 - d} \int_W \star |\dif u| + |K| \sigma^2 \\
&=: J_1 J_2 J_3 + J_4 + J_5.
\end{align*}
Here $0 \leq A \lesssim |K|$ is as in Proposition \ref{Monotone}.
By definition we have $J_1 \lesssim -\log \gamma$, $J_4 = I_1 \lesssim \gamma^{1/(d - 1)}$, and $J_5 \lesssim \gamma^{1/(d - 1)}$.
From Corollary \ref{doubling dimension} we have $J_2 \lesssim 1$.
We then need to bound $J_3$:

\begin{sublemma}
    We have $J_3 \lesssim \gamma^q$.
\end{sublemma}
\begin{proof}
We bound
\begin{align*}
J_3^2 &\leq \sigma^{1 - d} \int_W |\dif u|(z) \dif z - (2 \delta \varepsilon)^{1 - d} \int_V |\dif u|(z) \dif z + O(|K|) \sigma^{3 - d} \int_W \star |\dif u| \\
&= \sigma^{1 - d} \int_W (|\dif u| - \partial_0 u)(z) \dif z + \sigma^{1 - d} \int_W (\partial_0 - \partial_0^P)u(z) \dif z + \sigma^{1 - d} \int_W \partial_0^Pu(z) \dif z \\
  &\qquad - (2 \delta\varepsilon)^{1 - d} \int_V \partial_0^P u(z) \dif z + O(|K|) \sigma^{3 - d} \int_W \star |\dif u| \\
&=: K_1 + K_2 + K_3 - K_4 + K_5.
\end{align*}
Then $K_1 = I_1 \leq \gamma^{1/2}$, $K_2 = I_2 \lesssim \gamma^{1/(d - 1)}$, and $K_5 = J_4 \lesssim \gamma^{1/(d - 1)}$.

To estimate $K_3 - K_4$, we introduce the $d - 1$-form
$$\psi := \dif x^1_P \wedge \cdots \wedge \dif x^{d - 1}_P.$$
Then
\begin{equation}\label{K3 calculus}
K_3 = \sigma^{1 - d} \int_W \dif u \wedge \psi = \sigma^{1 - d} \int_{U \cap \partial W} \psi.
\end{equation}
We decompose
$$\partial W = \Gamma_+ \cup \Gamma_0 \cup \Gamma_-$$
where $\pm x^0_P > 0$ on the hemispheres $\Gamma_\pm$ and $\Gamma_0$ is the equator.
Then all positive contributions to the integral in the right-hand side of (\ref{K3 calculus}) come from $\Gamma_+$.
Moreover, as $d-1$-chains in $M$, $\partial \Gamma_+ = \Gamma_0$.
However, if we set $N := \{x^0_P = 0\}$ and $W_0 := W \cap N$, then $\Gamma_0 = \partial W_0$, and $\Gamma_+$ and $W_0$ have the same homology class relative to $\Gamma_0$.
But $\dif \psi = 0$, so by Stokes' theorem,
$$K_3 \leq \sigma^{1 - d} \int_{\Gamma_+ \cap U} \psi \leq \sigma^{1 - d} \int_{\Gamma_+} \psi = \sigma^{1 - d} \int_{W_0} \psi.$$
Since $\psi$ is the euclidean volume form on $W_0$, and $W_0$ is a $d-1$-ball whose euclidean radius is $\leq \sigma + O(|K| \sigma^3)$, it follows that
\begin{equation}\label{K3 calculus 2}
K_3 \leq |\Ball^{d - 1}| + O(|K| \sigma^2).
\end{equation}
By Corollary \ref{doubling dimension}, the right-hand side of (\ref{K3 calculus 2}) is $\leq K_4 + O(\gamma^{1/(d - 1)})$.
Adding up all the $K_i$ we complete the proof.
\end{proof}

By (\ref{kernel bounded}),
$$(1_V(|\dif u| - \partial_0 u))_\varepsilon(x) \lesssim (\delta\varepsilon)^{d - 1} \gamma^q \inf_{y \in B(P, \delta\varepsilon)} \chi_\varepsilon(x - y).$$
We finally apply Corollary \ref{doubling dimension} to prove
\begin{align*}
(\delta\varepsilon)^{d - 1} \inf_{y \in B(P, \delta\varepsilon)} \chi_\varepsilon(x - y)
&\lesssim \int_{B(P, \delta \varepsilon)} \chi_\varepsilon(x - y) |\dif u|(y) \dif y = |\dif u|_\varepsilon(x). \qedhere
\end{align*}
\end{proof}

\begin{proof}[Proof of Proposition \ref{main mollifier lemma}]
Let $\delta := \gamma^d > 0$ and $u := 1_U$.
We greedily construct a cover $\mathcal V = \{V_n: 1 \leq n \leq N\}$ of $\partial^* U \cap B_{\varepsilon(1 - 2\delta)}$ by balls of radius $2\delta\varepsilon$, centered on points $Q_n \in \partial^* U \cap B_{\varepsilon(1 - \delta)}$, which is \dfn{efficient} in the sense that each $Q \in \partial^*U \cap B_{\varepsilon(1 - \delta)}$ only lies in $O_d(1)$ balls $V_n$.
We set $V_0 := B_\varepsilon \setminus B_{\varepsilon(1 - 2\delta)}$.
Then $\supp \dif u \subseteq \bigcup_n V_n$, so
\begin{equation}\label{sum over cover}
(|\dif u| - \partial_0 u)_\varepsilon \leq \sum_{n = 0}^N (1_{V_n} (|\dif u| - \partial_0 u))_\varepsilon.
\end{equation}
After rotating the coordinate frame we may assume that $\dif u = \partial_0 u \dif x^0$. Then
\begin{equation}\label{excess is d0u}
    \int_{B(P, \rho)} (|\dif u| - \partial_0 u) \dif x = \Exc_\rho(U, P) + O(\rho^{d + 1}) \lesssim \rho^{d - 1}(\gamma + \rho^2).
\end{equation}
Applying (\ref{excess is d0u}) and reasoning similarly to \cite[pg92]{Giusti77}, one can show that there exists $c > 0$ such that
\begin{equation}\label{V0 case}
    (1_{V_0} (|\dif u| - \partial_0 u))_\varepsilon \lesssim (\gamma + \rho^2) |\dif u|_\varepsilon
\end{equation}
on $B_\sigma \cap \{c\gamma^2 < \varphi < 1 - c\gamma^2\}$.
% The proof of (\ref{V0 case}) is essentially given by \cite[pg92]{Giusti77}, so we just sketch it.
% For $y \in V_0$, $\chi_\varepsilon(x - y) \lesssim \delta/\varepsilon^d$, so using Corollary \ref{doubling dimension}, one can show
% $$\int_{V_0} \chi_\varepsilon(x - y)(|\psi| \cdot |\dif u| - \star(\dif u \wedge \psi))(y) \dif y \lesssim \frac{\gamma^d}{\varepsilon}.$$
% Here we used $\|\psi\|_{L^\infty} \lesssim 1$.
% One can use \cite[Lemma 7.1]{Giusti77}, the assumption $c\gamma^2 < \varphi < 1 - c\gamma^2$, and the fact that $g$ is a perturbation of the euclidean metric to obtain
% $$\int_{B_\varepsilon} \chi_\varepsilon(x - y) |\dif u|(y) \dif y \gtrsim \frac{\gamma^{d - 1}}{\varepsilon}$$
% which then implies (\ref{V0 case}).
Since $\mathcal V$ is efficient, we can sum (\ref{V0 case}) and Lemma \ref{mollifier sublemma} (with $\gamma$ replaced by $\gamma + \rho^2$) over $n$ in (\ref{sum over cover}) to show that
$$(|\dif u| - \partial_0 u)_\varepsilon \lesssim (\gamma^q + \rho^2) |\dif u|_\varepsilon$$
which implies (\ref{claim on main mollifier lemma}).
In particular near $\varphi^{-1}(y) \cap B_{\rho - 2\sigma}$, where $y \in (c\gamma^2, 1 - c\gamma^2)$, one has $|\dif u| > 0$.
Therefore $u$ is a $C^1$ submersion by \cite[Lemma 7.1]{Giusti77}, which completes the proof.
\end{proof}

%%%%%%%%%%%%%%%%%%%%%%%%%%%
\subsection{Reducing to the smooth case}
In order to combine the above results we use the following analogue of \cite[Lemma 7.5]{Giusti77}.
We do not assert that the $C^1$ manifold $\partial V$ is close to $\partial U$ in $C^0$, and in general it may only be close to a rotation of $\partial U$. The relevant fact is the estimate (\ref{single mollify excess}), that their excesses are close.

\begin{lemma}\label{single mollify}
For every $\varepsilon > 0$ there exists $\delta = \delta(d, K, \varepsilon) > 0$ and $r = r(d, K, \varepsilon) > 0$ such that for any ball $B(P, \rho)$ such that $\rho < r$ and any set $U$ of least perimeter in $B(P, \rho)$ such that
$$\Exc_\rho (U, P) \leq \delta \rho^{d - 1},$$
there exists a set $V$ of $C^1$ perimeter in $B(P, (1 - \varepsilon)\rho)$ such that
\begin{align}
((\Phi^P)^* \normal_V)_0 &\geq e^{-o(\varepsilon^2)}, \label{single mollify normal}\\
|\partial V \cap B(P, (1 - \varepsilon)\rho)| &\leq \eta(V, B(P, (1 - \varepsilon)\rho)) + \varepsilon \Exc_\rho (U, P), \label{single mollify minimality}
\end{align}
and for $\rho/10 \leq \varpi \leq (1 - \varepsilon)\rho$,
\begin{equation}
|\Exc_\varpi (U, P) - \Exc_\varpi (V, P)| \leq \varepsilon \Exc_\rho (U, P) + C|K| \rho^{d + 1}. \label{single mollify excess}
\end{equation}
\end{lemma}
\begin{proof}
If not, then there exist balls $B_n := B(P_n, \rho_n)$ and sets $U_n$ of least perimeter in $B_n$ such that
\begin{equation}\label{single mollify Excess assumption}
\gamma_n := \rho_n^{1 - d} \Exc_{\rho_n} (U_n, P_n) \leq n^{-2},
\end{equation}
and $\rho_n \leq 1/n$, but such that for every set $V_n$ of $C^1$ perimeter in $B(P_n, (1 - \varepsilon) \rho_n)$, at least one of (\ref{single mollify normal}), (\ref{single mollify minimality}), or (\ref{single mollify excess}) fail.
Applying an isometry, we may assume that $P_n = O$.

Now let $w_n := (u_n)_{\gamma_n^4 \rho_n}$ be the mollification of $u_n$, draw $t \in [0, 1]$ uniformly at random, and let $c, q > 0$ be as in Proposition \ref{main mollifier lemma} (which we apply with $\gamma := n^{-2}$ and $\rho := \rho_n$), $a_n := c\gamma_n^2$, $b_n := 1 - c\gamma_n^2$.
By the coarea formula, Proposition \ref{Coarea2},
$$\int_{tB_n} \star |\dif w_n| = \int_0^1 |\partial^* \{w_n > y\} \cap tB_n| \dif y \geq \int_{a_n}^{b_n} |\partial^* \{w_n > y\} \cap tB_n| \dif y,$$
so by the mean value theorem, there exists $y_n \in (a_n, b_n)$ such that
$$|\partial^* \{w_n > y_n\} \cap tB_n| \leq \frac{1}{b_n - a_n} \int_{tB_n} \star |\dif w_n|.$$
We then set $V_n := \{w_n > y_n\}$, $v_n := 1_{V_n}$, so that $V_n$ has $C^1$ boundary in $tB_n$ and
\begin{equation}\label{MVT mollifier}
|V_n \cap tB_n| \leq \frac{1}{b_n - a_n} \int_{tB_n} \star |\dif w_n|.
\end{equation}
Since $\grad w_n$ is normal to the level sets of $w_n$, $\normal_{V_n} = \dif w_n/|\dif w_n|$.
So by (\ref{claim on main mollifier lemma}),
$$(\normal_{V_n})_0 \geq 1 - O(n^{-2q} + n^{-2}) = 1 - O(n^{-2q}).$$
Thus for $n$ large, $V_n$ satisfies (\ref{single mollify normal}).

Let $\Gamma_n := \partial(tB_n)$.
Arguing completely analogously to the proofs of \cite[(7.23), (7.22)]{Giusti77}, respectively, we see that almost surely,
\begin{align}
\|u_n - v_n\|_{L^1(\Gamma_n)} &\ll \gamma_n \label{trace of vn} \\
|\partial V_n \cap sB_n| &\leq |\partial^* U_n \cap sB_n| + \gamma_n. \label{difference of surface area}
\end{align}
Here we are using the fact that $\sum_n n^{-2} = \pi^2/6$ is finite.
The conjunction of (\ref{trace of vn}), (\ref{difference of surface area}), and (\ref{a priori estimate 1}) implies
\begin{equation}
||\partial^* U_n \cap tB_n| - |\partial V_n \cap tB_n|| \ll \gamma_n, \label{mollifier quant2}
\end{equation}
and the conjunction of (\ref{mollifier quant2}), (\ref{a priori estimate 1}), the fact that $U_n$ has least perimeter, (\ref{single mollify Excess assumption}) and (\ref{trace of vn}) implies (\ref{single mollify minimality}).

To derive a contradiction, we must show that $V_n$ satisfies (\ref{single mollify excess}) for $n$ large.
To this end we fix $\varepsilon > 0$ and $\varpi \in [\rho/10, t\rho]$ where $t > 1 - \varepsilon$ can be chosen at random almost surely.
Then
\begin{align*}
    |\Exc_\varpi(U, O) - \Exc_\varpi(V, O)|
    &\leq ||\partial^* U_n \cap t B_n| - |\partial V_n \cap t B_n||\\
    &\qquad + \left|\left[\int_{B_\varpi} \star \partial_\mu(u_n - v_n) \right] \dif x^\mu(P)\right| + O(|K| \rho_n^{d + 1}) \\
    &=: I_1 + I_2 + I_3.
\end{align*}
By (\ref{mollifier quant2}), $I_1 < \varepsilon \Exc_\varpi(U_n, P_n)/2$ if $n$ is large, and $I_3$ is irrelevant.
Since $\dif \psi_\mu = 0$, by Stokes' theorem and (\ref{trace of vn}), if $n$ is large then
\begin{align*}
    I_2 &\leq \left|\left[\int_{\partial B_\varpi} \star_\varpi (\normal_{B_\varpi})_\mu (u_n - v_n)\right] \dif x^\mu(O)\right| \leq \frac{\varepsilon}{2} \Exc_\varpi(U_n, P_n) + O(|K| \rho_n^{d + 1}).
\end{align*}
This implies (\ref{single mollify excess}) for $n$ large and so contradicts our assumptions.
\end{proof}

\begin{proof}[Proof of the de Giorgi lemma]
Choose $A = A(d) \geq 1$ to be larger than the implied constant in the $O(c_0)$ in (\ref{Miranda44 concl}).
Choose $c_0 = c_0(A, d, K) \leq 2^{-(d + 3)}/A$ so small that $c_0$ meets the hypotheses of Lemma \ref{Miranda43}, and let $\alpha := 1/(2(1 - c_0))$.
Choose $c_1 = c_1(c_0, d, K) \leq c_0$ so small that $c_1$ meets the hypotheses of Proposition \ref{Miranda44}.

Given a set $U$ of least perimeter, let $V$ be the set obtained from Lemma \ref{single mollify} with $\varepsilon := c_1$, and let $\beta := \Exc_{B(P, \rho)} U$.
Then $(\normal_V, \partial_0^P) \geq e^{-o(c_1^2)}$ on $B(P, (1 - c_0)\rho)$, and
$$|\partial V \cap B(P, (1 - c_0)\rho)| \leq \eta(V, B(P, (1 - c_0)\rho)) + c_1 \beta.$$
By (\ref{single mollify excess}) and (\ref{approximate monotone}),
\begin{align*}
\Exc_{(1 - c_0) \rho} (V, P) &\leq \Exc_{(1 - c_0) \rho} (U, P) + c_0 \Exc_\rho (U, P) + C|K| \rho^{d + 1} \\
&\leq (1 + c_0) \beta + C |K| \rho^{d + 1}.
\end{align*}
Since $1/2 = \alpha (1 - c_0)$, by Proposition \ref{Miranda44}, it follows that
\begin{align*}
    \Exc_{\rho/2} (V, P) &\leq (2^{-(d + 1)} + Ac_0) (1 + c_0) \beta + C |K| \rho^{d + 1} \\
    &\leq (2^{-(d + 1)} + 2^{-(d + 2)}) \beta + C |K| \rho^{d + 1}.
\end{align*}
Finally, by (\ref{single mollify excess}) and (\ref{approximate monotone}),
\begin{align*}
    \Exc_{\rho/2} (U, P)
    &\leq \Exc_{\rho/2} (V, P) + c_0 \Exc_\rho (U, P) + C|K| \rho^{d + 1} \\
    &\leq (2^{-(d + 1)} + 2^{-(d + 2)} + 2^{-(d + 3)}) \beta + C |K| \rho^{d + 1}\\
    &\leq 2^{-d} \beta + C |K| \rho^{d + 1}. \qedhere
\end{align*}
\end{proof}


\chapter{Compactness for minimal laminations}
\section{Introduction}
The space of codimension-$1$ minimal laminations on a Riemannian manifold has been topologized in several different ways.
Thurston \cite[Chapter 8]{thurston1979geometry} introduced both his geometric topology as well as the weak topology of measures on the space of measured geodesic laminations.
Independently of Thurston, Colding--Minicozzi \cite[Appendix B]{ColdingMinicozziIV} introduced a topology that emphasized not the laminations themselves, but rather the coordinate charts which flatten them.
We shall explain how these three modes of convergence are related, as well as the regularity and compactness theorems associated to each such mode.

We then turn to the main goal of this series of papers, which also includes the prequel paper \cite{BackusFLG}.
We show that any $1$-harmonic function gives rise to a Ruelle-Sullivan current for a minimal lamination, and conversely.
This generalizes a theorem of Daskalopoulos--Uhlenbeck \cite[Theorem 6.1]{daskalopoulos2020transverse} and resolves the outstanding problems \cite[Problem 9.4]{daskalopoulos2020transverse} and \cite[Conjecture 9.5]{daskalopoulos2020transverse}.

%%%%%%%%%%%%%%%%%
\subsection{Minimal laminations}\label{Lams sections}
Let $I \subset \RR$ be an interval, $J \subset \RR^{d - 1}$ a box, and $M$ a Riemannian manifold of dimension $d \geq 2$.
A (codimension-$1$) \dfn{laminar flow box} is a $C^0$ coordinate chart $F: I \times J \to M$ and a compact set $K \subseteq I$ such that each \dfn{leaf} $F(\{k\} \times J)$ is a $C^2$ complete hypersurface of the image of $F$.

Now let $(F_\alpha, K_\alpha)$ and $(F_\beta, K_\beta)$ be laminar flow boxes.
We say that they belong to the same \dfn{laminar atlas} if the transition map $\psi_{\alpha \beta}$ between $F_\alpha$ and $F_\beta$ maps each leaf $\{k\} \times J$, $k \in K_\alpha$, to a leaf $\{\psi_{\alpha \beta}(k)\} \times J$, so that $\psi_{\alpha \beta}$ is a homeomorphism $K_\alpha \to K_\beta$.

\begin{definition}
A \dfn{lamination} $\lambda$ consists of a nonempty closed set $S \subseteq M$, called its \dfn{support}, and a maximal laminar atlas $\{(F_\alpha, K_\alpha): \alpha \in A\}$ such that in the image $U_\alpha$ of each flow box $F_\alpha$,
$$S \cap U_\alpha = F_\alpha(K_\alpha \times J).$$
If $\lambda$ is a lamination in the image of a flow box $F$, and $N := F(\{k\} \times J)$ is a leaf of $\lambda$, we call $k$ the \dfn{label} of $N$.
\end{definition}

Summarizing the above definitions, a lamination is a nonempty closed set $S$ with a $C^0$ local product structure which realizes it as $K \times J$ for some compact set $K \subset \RR$.

We assume that the leaves of a lamination $\lambda$ are $C^2$ in order to ensure that the second fundamental form and mean curvature of each leaf is well-defined.
Such laminations are sometimes called $C^2$ \dfn{along leaves} \cite{Morgan88}.
A related notion, which we shall find useful, is that a lamination $\lambda$ is \dfn{tangentially} $C^r$, in the sense that the flow boxes $F_\alpha$ restrict to $C^r$ embeddings along each leaf $\{k\} \times J$, $k \in K_\alpha$.
In any case, we will make more precise assertions about regularity in \S\ref{Regularity}.

In this paper we shall focus on laminations with minimal leaves.\footnote{Caveat lector: The word ``minimal'' is overloaded. In \cite{casson_bleiler_1988,daskalopoulos2020transverse}, a \dfn{minimal lamination} is a lamination $\lambda$ in which every leaf is dense in $\supp \lambda$.
We adopt the terminology of \cite{Ohshika90}, which unfortunately clashes with the rest of the literature.}

\begin{definition}
A lamination $\lambda$ is \dfn{minimal} if its leaves $F_\alpha(\{k\} \times J)$ have zero mean curvature, and is \dfn{geodesic} if, in addition, $d = 2$.
\end{definition}


%%%%%%%%%%%%%%%%%%
\subsection{Spaces of minimal laminations}\label{LamSpace section}
Let $M$ be a manifold of constant sectional curvature and dimension $2 \leq d \leq 7$.
In the literature, there are at least three different topologies on the space of laminations on $M$, which we now recall.

Thurston's geometric topology \cite[Chapter 8]{thurston1979geometry} says that a lamination $\lambda'$ is close to a lamination $\lambda$ if every leaf of $\lambda$ is close to a leaf of $\lambda'$ at least locally, and the same holds for their normal vectors $\normal$.

\begin{definition}
A sequence of laminations $(\lambda_i)$ converges to a lamination $\lambda$ in \dfn{Thurston's geometric topology} if, for every leaf $N$ of $\lambda$, every $x \in N$, and every $\varepsilon > 0$, there exists $i_\varepsilon \in \NN$ such that for every $i \geq i_\varepsilon$, $\supp \lambda_i$ intersects $B(x, \varepsilon)$, and for $x_i \in B(x, \varepsilon) \cap \supp \lambda_i$,
$$\dist_{SM}(\normal_{\lambda_i}(x_i), \normal_\lambda(x)) < 2\varepsilon.$$
\end{definition}

It is straightforward to show that Thurston's geometric topology does not depend on the choice of Riemannian metric on $M$, or the choice of extension of the distance function on $M$ to its sphere bundle $SM$, which are implicit in the statement thereof.
However, the limiting lamination is not unique, as if $\lambda_i \to \lambda$ and $\lambda'$ is a sublamination of $\lambda$, then $\lambda_i \to \lambda'$.
In particular, Thurston's topology is not Hausdorff, and we say that $\lambda$ is a \dfn{maximal limit} of a sequence $(\lambda_i)$ if $\lambda_i \to \lambda$ and for every $\lambda'$ such that $\lambda_i \to \lambda'$, $\lambda'$ is a sublamination of $\lambda$.

Independently of Thurston, Colding--Minicozzi \cite[Appendix B]{ColdingMinicozziIV} defined a sequence of laminations to converge ``if the corresponding coordinate maps converge;'' that is, if the laminar atlases converge.
This of course says nothing about the limiting set of leaves and in the sequel paper \cite{ColdingMinicozziV} they additionally impose that the sets of leaves converge ``as sets.''

In this paper we consider a similar condition to the one in \cite{ColdingMinicozziV}, which we believe to be more natural: that the laminar atlases converge and that the laminations themselves converge in Thurston's geometric topology.
To be more precise:

\begin{definition}
A sequence $(\lambda_i)$ of laminations \dfn{flow-box converges} in a function space $X$ to $\lambda$ if it converges in Thurston's geometric topology, and there exists a laminar atlas $(F_\alpha)$ for $\lambda$ such that for each $\alpha$, $F_\alpha$ and $(F_\alpha)^{-1}$ are limits in $X$ of flow boxes $F_\alpha^i$, $(F_\alpha^i)^{-1}$ in laminar atlases for $\lambda_i$.
\end{definition}

The notion of flow-box convergence is mainly useful in the case that $X = C^{1-} := \bigcup_{0 \leq \alpha < 1} C^\alpha$, where $C^\alpha$ are H\"older spaces.

We now define convergence of laminations equipped with transverse measures.

\begin{definition}
Let $\lambda$ be a lamination with atlas $A$.
A \dfn{transverse measure} to $\lambda$ consists of Radon measures $\mu_\alpha$ with $\supp \mu_\alpha = K_\alpha$, $\alpha \in A$, such that each transition map $\psi_{\alpha \beta}$ is measure-preserving:
$$\mu_\alpha|_{K_\alpha \cap K_\beta} = \psi_{\alpha \beta}^* (\mu_\beta|_{K_\alpha \cap K_\beta}).$$
The pair $(\lambda, \mu)$ is called a \dfn{measured lamination}.
\end{definition}

Caveat lector: we assume that $\supp \mu_\alpha = K_\alpha$, but in \cite{daskalopoulos2020transverse}, it is only assumed that $\supp \mu_\alpha \subseteq K_\alpha$.
In particular, not every lamination admits a transverse measure.

% The definition of transverse measure in terms of Radon measures on $K_\alpha$ is convenient because $K_\alpha$ is compact.
% However, the definition is not intrinsic, and this causes problems when considering questions of convergence: the fact that the flow boxes of a convergent sequence of measured laminations converge should be a consequence of, not a part of, the definition.

% To rectify this, we first observe that in the definition of a transverse measure, we cannot define a transverse measure to be a positive measure on the support of $\supp \lambda$.
% For such a notion of transversality, Lebesgue measure would be transverse to every foliation on $\RR^d$!

Now suppose that $\lambda$ is \dfn{oriented} -- that is, all its transition maps are orientation-preserving.
Then we can define measure convergence in terms of its Ruelle-Sullivan current \cite{Ruelle75}.

\begin{definition}
Let $(\lambda, \mu)$ be a measured, oriented, tangentially $C^1$ lamination, and let $(\chi_\alpha)_{\alpha \in A}$ be a subordinate partition of unity.
The \dfn{Ruelle-Sullivan current} $T_\mu$ associated to $(\lambda, \mu)$ is defined for all compactly supported $d-1$-forms $\varphi$ by
\begin{equation}\label{RS current}
\int_M T_\mu \wedge \varphi := \sum_{\alpha \in A} \int_{K_\alpha} \left[\int_{\{k\} \times J} (F_\alpha^{-1})^* (\chi_\alpha \varphi) \right] \dif \mu_\alpha(k).
\end{equation}
\end{definition}

\begin{definition}
A sequence of measured, tangentially $C^1$ laminations $(\lambda_i, \mu_i)$ \dfn{converges} to $(\lambda, \mu)$ if their local Ruelle-Sullivan currents $T_{\mu_i} \to T_\mu$ converge in the weak topology of measures.
\end{definition}

The convergence of Ruelle-Sullivan currents, which is very convenient to work with analytically, is equivalent to a definition of measure convergence that may be more familiar to topologists, namely convergence of the transverse measure along each transverse curve.
We recall this fact in Appendix \ref{transverse curves}.

How are the above modes of convergence related?
It is clear from the definitions that flow-box convergence implies Thurston convergence, and it is well-known that measure convergence implies Thurston convergence \cite[Proposition 8.10.3]{thurston1979geometry}.
We show that flow-box convergence actually sits in the middle of the chain of implications.

\begin{definition}
A sequence $(\lambda_n)$ of laminations has \dfn{bounded curvature} if there exists $C > 0$ such that for any $n$ and any leaf $N$ of $\lambda_n$, the second fundamental form satisfies $\|\Two_N\|_{C^0} \leq C$.
\end{definition}

\begin{theorem}\label{implication theorem}
Let $M$ be a manifold of constant sectional curvature and dimension $2 \leq d \leq 7$.
Let $(\lambda_n, \mu_n)$ be measured minimal laminations in $M$, and $(\lambda_n, \mu_n) \to (\lambda, \mu)$.
Then:
\begin{enumerate}
	\item $\lambda_n \to \lambda$ in Thurston's geometric topology.
	\item If $(\lambda_n)$ has bounded curvature, then $\lambda_n \to \lambda$ in the $C^{1-}$ flow box topology.
\end{enumerate}
\end{theorem}

We also complement Theorem \ref{implication theorem} with some compactness results for the above modes of convergence.

\begin{theorem}\label{compactness theorem}
Let $M$ be a manifold of dimension $2 \leq d \leq 7$.
Let $(\lambda_n)$ be a sequence of minimal laminations of bounded curvature, and assume that for some compact set $E \Subset M$ and every leaf $N$ of $\lambda_n$, $N \cap E$ is nonempty. Then:
\begin{enumerate}
\item A subsequence converges in the $C^{1-}$ flow box topology, and in particular in Thurston's geometric topology, to a minimal lamination.
\item If $\mu_n$ is transverse to $\lambda_n$ and there exists $C > 0$ such that $\mu_n(M) \leq C$, then a further subsequence converges in the measure topology.
\end{enumerate}
\end{theorem}



%%%%%%%%%%%%%%%%%%
\subsection{Best Lipschitz and least gradient maps}\label{FLG section}
Geodesic laminations are of great interest to the Thurston school of geometric topology \cite[Chapter 8]{thurston1979geometry}.
Later Thurston introduced \dfn{best Lipschitz maps}, namely maps $v: M \to N$ between closed manifolds which minimize their Lipschitz constant $\Lip(v)$ subject to a constraint on their homotopy class.
These maps define a geodesic lamination whose support is the set of points $x$ so that the local Lipschitz constant of $v$ at $x$ is equal to $\Lip(v)$ \cite{thurston1998minimal}.
If $M, N$ are hyperbolic surfaces of the same genus $g$, then $\Lip(v)$ is the distance between $M$ and $N$ in \dfn{Thurston's asymmetric metric} on Teichm\"uller space.
This circle of ideas has been developed by the Thurston school \cite{papadopoulos:hal-00129729} but has recently also made contact with geometric PDE through the work of Daskalopoulos--Uhlenbeck \cite{daskalopoulos2020transverse,daskalopoulosPrep1}.

To be more precise, if $M$ is a closed hyperbolic manifold, then the Euler-Lagrange equation for best Lipschitz maps $v: M \to \Sph^1$ is the $\infty$-Laplace equation
\begin{equation}\label{infinity laplacian}
	\langle \nabla^2 v, \nabla v \otimes \nabla v\rangle = 0.
\end{equation}
Thus, every \dfn{$\infty$-harmonic function} -- that is, a viscosity solution of (\ref{infinity laplacian}) -- is best Lipschitz \cite{daskalopoulos2020transverse}.
The $\infty$-Laplace equation is invariant under translations $v \mapsto v + y$, so by Noether's theorem, it has a conserved flux $\dif u$.
If $d = 2$, the associated conservation law is the $1$-Laplace equation
\begin{equation}\label{1Laplacian}
\dif^* \left(\frac{\dif u}{|\dif u|}\right) = 0.
\end{equation}
A \dfn{$1$-harmonic function}, or variational solution of (\ref{1Laplacian}), is a minimizer $u$ of the total variation energy
\begin{equation}\label{least gradient functional}
	\int_M \star |\dif u| \leq \int_M \star |\dif u + \dif w|
\end{equation}
subject to the Dirichlet condition $\supp w \Subset M$.

Formally, the $1$-Laplace equation (\ref{1Laplacian}) asserts that the level sets are minimal surfaces, but one has to verify this rigorously for variational solutions.
This is the main result of the prequel paper \cite{BackusFLG}:

\begin{theorem}\label{main thm of old paper}
Let $M$ be a manifold of constant sectional curvature and $2 \leq d \leq 7$.
Then for every $1$-harmonic function $u: M \to \RR$ and $y \in \RR$, the level set $\partial \{u > y\}$ is an analytic embedded stable minimal hypersurface in $M$.
\end{theorem}
\begin{proof}[Proof sketch]
By a straightfoward modification of \cite[Theorem 1]{BOMBIERI1969}, the superlevel sets $\{u > y\}$ have least perimeter, that is their indicator functions have least gradient.
The regularity of boundaries of sets of least perimeter was established for $M = \RR^d$ by de Giorgi's lemma \cite{Miranda66} but the proof does not generalize nicely, because it relies on the invariance of tangent vectors under parallel transport in order to define averages of normal vectors to sets of least perimeter.
In \cite[\S3]{BackusFLG} we establish a suitable method of taking averages of the normal vector provided that $M = \Hyp^d$ or $M = \Sph^d$.
This allows us to modify the relevant parts of \cite{Miranda66}.
See \cite[\S1]{BackusFLG} for a more detailed summary of \cite{BackusFLG}.
\end{proof}

We use Theorem \ref{main thm of old paper} as a regularity result at various points in this paper, which allows us to show that the limiting laminations have smooth leaves (rather than currents for leaves, say).
However, Theorem \ref{main thm of old paper} is of more interest to this paper rather than just as a regularity lemma; to motivate why, we recall the relationship between best Lipschitz maps, maps of least gradient, and geodesic laminations:

\begin{theorem}[Daskalopoulos--Uhlenbeck]\label{DU theorem}
Let $M$ be a closed hyperbolic surface and $v: M \to \Sph^1$ an $\infty$-harmonic function with maximal stretch geodesic lamination $\lambda$ and conserved flux $T$. Then:
\begin{enumerate}
\item $T$ is the Ruelle-Sullivan current for a measured oriented structure on $\lambda$.
\item There exists a $1$-harmonic, $\pi_1(M)$-equivariant function $u: \Hyp^2 \to \RR$ such that $\dif u$ is a lift of $T$ to the universal cover $\Hyp^2$.
\item Every level set of $u$ is the lift of a geodesic of $\lambda$ to $\Hyp^2$.
\end{enumerate}
\end{theorem}

We refer to the original paper \cite{daskalopoulos2020transverse} for a more precise statement.
Inspired by this theorem, Daskalopoulos--Uhlenbeck conjectured that for any $1$-harmonic function on $\Hyp^2$, $\dif u$ should be Ruelle-Sullivan for some (possibly not maximum-stretch) geodesic lamination \cite[Problem 9.4]{daskalopoulos2020transverse}, and conversely that if $T$ is a Ruelle-Sullivan current for some geodesic lamination, then local primitives of $T$ are $1$-harmonic \cite[Conjecture 9.5]{daskalopoulos2020transverse}.
Of course, if $d \geq 3$, then the level sets will be minimal hypersurfaces rather than geodesics.

Using Theorem \ref{compactness theorem}, the stable Bernstein theorem \cite{Schoen2016, Chodosh2021}, and Theorem \ref{main thm of old paper}, we prove the conjectures of Daskalopoulos--Uhlenbeck:

\begin{theorem}\label{main thm}
Let $M$ be a manifold of constant sectional curvature and dimension $2 \leq d \leq 4$.
\begin{enumerate}
\item Let $u$ be a $1$-harmonic function on $M$.
Then:
\begin{enumerate}
\item $\bigcup_{y \in \RR} \partial \{u > y\}$ is the support of a minimal lamination $\lambda$.
\item The leaves of $\lambda$ are the connected components of the level sets $\partial \{u > y\}$.
\item There is a measured oriented structure on $\lambda$ whose Ruelle-Sullivan current is $\dif u$.
\end{enumerate}
\item Conversely, if $\lambda$ is a minimal measured oriented lamination with Ruelle-Sullivan current $T$, and $\tilde M \to M$ is the universal cover of $M$, then there exists a $1$-harmonic, $\pi_1(M)$-equivariant function $u: \tilde M \to \RR$, such that $\dif u$ drops to $T$.
\end{enumerate}
\end{theorem}

However, Theorem \ref{main thm} leaves a key point -- the role of the $\infty$-Laplacian -- open, and we have not attempted to address this point here.
The Daskalopoulos--Uhlenbeck theorem was our main motivation for this series of papers; however, our work works in dimensions $d = 3, 4$ while Theorem \ref{DU theorem} is purely a statement about $d = 2$.
Indeed, the $\infty$-Laplacian gives geodesic laminations, but the $1$-Laplacian gives codimension-$1$ laminations, which are not geodesic if $d \geq 3$, so there is not an obvious generalization of Theorem \ref{DU theorem} for $d = 3, 4$.

It is tantalizing to think that a suitable system of coupled $\infty$-Laplacians will satisfy a generalized maximum-stretch condition on a codimension-$1$ minimal lamination $\lambda$, and that the conservation law for this conjectural system will be the $1$-Laplacian.
However, even if one was to derive such a system of $\infty$-Laplacians formally, the analysis for studying such a system would likely not be in place.
Indeed, the study of $\infty$-harmonic functions is almost entirely carried out in the language of viscosity solutions and comparison-with-cones, which only make sense when the target is $\RR$ rather than $\RR^m$ or a vector bundle.
Even so, we hope to return to this question in a later work: what is the correct generalization of the Daskalopoulos--Uhlenbeck theorem to closed hyperbolic threefolds?

%%%%%%%%%%%%%%%%%%%%%%%
\subsection{Outline of the paper}
In \S\ref{Prelims} we recall preliminaries.

In \S\ref{Regularity} we establish that every $C^0$ minimal lamination has a Lipschitz laminar atlas and a Lipschitz normal bundle, and is tangentially $C^\infty$.
We will use this regularity frequently through the remainder of the paper.

In \S\ref{CompactnessSec} we prove Theorems \ref{implication theorem}, \ref{compactness theorem}, and \ref{main thm}.

We also include Appendix \ref{transverse curves}, where we explain why our definition of convergence in the weak topology of measures is equivalent to a more standard definition.

%%%%%%%%%%%%%%%%%%%%%%%%

\subsection{Acknowledgements}
I would like to thank Georgios Daskalopoulos for suggesting this project and for many helpful discussions.
I would also like to thank Chao Li for help understanding \cite{Chodosh2021}.

This research was supported by the National Science Foundation's Graduate Research Fellowship Program under Grant No. DGE-2040433.



%%%%%%%%%%%%%%%%%%%%%%%%%%%

\section{Preliminaries}\label{Prelims}
\subsection{Notation and conventions}
The operator $\star$ is the Hodge star, thus $\star 1$ is the Riemannian measure.
We denote the musical isomorphisms by $\sharp, \flat$.
If $U$ is an open set, we write $|U| := \int_U \star 1$ for the volume of $U$, but if $U$ is a submanifold or rectifiable set of positive codimension, we instead write $|U|$ for its surface measure.
We write $\normal_N$ for the normal vector (or conormal $1$-form) for a hypersurface $N$, $\nabla_N$ for the Levi-Civita connection, and $\Two_N := \nabla_N \normal_N$ for the second fundamental form.

We consider the following manifolds: $\Ball^d$ is the unit ball in $\RR^d$, $\Sph^d$ the unit sphere in $\RR^{d + 1}$, and $\Hyp^d$ is the hyperbolic space.

For a map $F: X \to Y$ between metric spaces, we write $\Lip(F)$ for its Lipschitz constant.
If $X, Y$ are connected Riemannian manifolds, one of which is $1$-dimensional, then we have $\Lip(F) = \|\dif F\|_{L^\infty}$.

We let $\Leaves \lambda$ denote the set of leaves of a lamination $\lambda$.

%%%%%%%%%%%%%%%%
% \subsection{Hausdorff distance}
% In order to measure when two leaves are ``close'', we shall consider the Hausdorff distance on the space of leaves, as defined in \cite[Chapter 4]{nadler2017continuum}.

% \begin{definition}
% Let $X$ be a compact metric space. The \dfn{Hausdorff distance} between two nonempty closed sets $A, B \subset X$ is
% \begin{equation}\label{compact hausdorff distance}
% 	\dist(A, B) := \max\left(\max_{a \in A} \min_{b \in B} \dist(a, b), \max_{b \in B} \min_{a \in A} \dist(a, b)\right).
% \end{equation}
% The space of closed subsets of $X$ is the \dfn{hyperspace} $\Hypspace X$.
% \end{definition}

% \begin{definition}
% If $X$ is a metrizable space and $(A_n)$ is a sequence of closed subsets of $X$, we define
% \begin{equation}\label{Hausdorff is the limit set}
% \lim_{n \to \infty} A_n := \left\{\lim_{n \to \infty} x_n: (x_n) \in \prod_{n=1}^\infty A_n\right\}.
% \end{equation}
% \end{definition}

% \begin{proposition}\label{Hausdorff on a CMS}
% Let $X$ be a compact metric space. Then:
% \begin{enumerate}
% \item The topology on $\Hypspace X$ is the topology for the convergence mode (\ref{Hausdorff is the limit set}).
% \item The topology on $\Hypspace X$ is completely determined by the topology on $X$.
% \item $\Hypspace X$ is a compact metric space.
% \end{enumerate}
% In particular, $\Hypspace$ is a self-map of the set of compact metrizable spaces.
% \end{proposition}
% \begin{proof}
% By \cite[Theorem 4.11]{nadler2017continuum}, a sequence $(A_n)$ of closed subsets of $X$, the limit $A$ in $\Hypspace X$ of $A_n$ is the set of $x \in X$ such that for every $U \ni x$ open, there are $x_n \in A_n$ such that eventually $x_n \in U$.
% But this is exactly the characterization of $\lim_n A_n$ given by (\ref{Hausdorff is the limit set}).
% It is also independent of the metric on $X$, so as a metrizable space, $\Hypspace X$ is determined by the metrizable (rather than metric) structure on $X$.
% The compactness now follows from \cite[Theorem 4.17]{nadler2017continuum}, and the fact that $\Hypspace$ is a self-map of the space of compact metrizable spaces follows.
% \end{proof}

%%%%%%%%%%%%%%%%%%%%%
\subsection{Measure theory}\label{MeasurePrelims}
Let $X$ be a metrizable space, and let $C_\cpt(X)$ be the space of compactly supported continuous functions $f: X \to \RR$.
Its dual $C_\cpt(X)'$ is canonically isomorphic to the space of signed Radon measures on $X$, where the bilinear pairing is given by integration.
The weak topology on $C_\cpt(X)'$ is known as the \dfn{weak topology of measures}.
Unpacking the definitions, a sequence $(\mu_n)$ of Radon measures converges to $\mu$ in the weak topology of measures iff for every continuous function $f: X \to \RR$,
$$\lim_{n \to \infty} \int_X f \dif \mu_n = \int_X f \dif \mu.$$
We shall frequently use the following characterization of weak convergence:

\begin{proposition}[portmanteau theorem]
	Let $(\mu_n)$ be a sequence of positive Radon measures on a compact metrizable space $X$ with $\mu_n(X) \lesssim 1$, and let $\mu$ be a Radon measure on $X$. The following are equivalent:
\begin{enumerate}
	\item $\mu_n \to \mu$ in the weak topology of measures.
	\item $\liminf_{n \to \infty} \mu_n(X) \geq \mu(X)$ and for every closed $Y \subseteq X$, $\limsup_{n \to \infty} \mu_n(Y) \leq \mu(Y)$.
	\item $\limsup_{n \to \infty} \mu_n(X) \leq \mu(X)$ and for every open $Z \subseteq X$, $\liminf_{n \to \infty} \mu_n(Z) \geq \mu(Z)$.
	\item For every $W \subseteq X$ with $\mu(\partial W) = 0$, $\lim_{n \to \infty} \mu_n(W) = \mu(W)$.
\end{enumerate}
	If $X$ is a manifold, these conditions are equivalent to:
\begin{enumerate}
	\setcounter{enumi}{4}
	\item For every $x \in X$ and almost every $0 < \varepsilon \ll 1$, $\lim_{n \to \infty} \mu_n(B(x, \varepsilon)) = \mu(B(x, \varepsilon))$.
\end{enumerate}
\end{proposition}
\begin{proof}
	See \cite[Theorem 13.16]{klenke2013probability} for the metrizable case.
	For the manifold case, we just observe that almost every $\varepsilon > 0$ satisfies $\mu(\partial B(x, \varepsilon)) = 0$. Indeed, if not, then we can find a set of real numbers $A$ such that $0$ is a condensation point of $A$, and for every $\varepsilon \in A$, $\mu(\partial B(x, \varepsilon)) > 0$.
	In particular, for every $\delta > 0$, $A \cap (0, \delta)$ is uncountable, so
	$$\mu(B(x, \delta)) \geq \sum_{\varepsilon \in A \cap (0, \delta)} \mu(\partial B(x, \delta)) = \infty,$$
	but since $X$ is a manifold, if $\delta$ is small enough then $B(x, \delta)$ is precompact.
	This contradicts that $\mu$ is a Radon measure.
\end{proof}

There are subtleties involved in the portmanteau theorem for noncompact $X$.
However, this will never be an issue, as we shall only use it locally, in small precompact balls.

If $X = M$ is a manifold, then we can consider instead the space $C_\cpt(M, \Omega^\ell)$ of compactly supported continuous $\ell$-forms.
An $\ell$-\dfn{current} is an element of the dual space $C_\cpt(M, \Omega^\ell)'$.\footnote{Strictly speaking, $C_\cpt(M, \Omega_\ell)'$ is the space of $\ell$-currents of locally finite total variation. However, we will never need to consider $\ell$-currents that do not have locally finite total variation, so we suppress this technicality.}
We denote the pairing of an $\ell$-current $T$ and an $\ell$-form $\varphi$ by $\int_M T \wedge \varphi$.
Any $d-\ell$-form $\psi$ gives rise to an $\ell$-current $T$, the \dfn{Poincar\'e dual} of $\psi$, by $\int_M T \wedge \varphi = \int_M \psi \wedge \varphi$.
In particular, the Poincar\'e dual of any function is a $d$-current.
Again we have the \dfn{weak topology of measures} on the space of $\ell$-currents.

\begin{definition}
Let $T$ be an $\ell$-current.
\begin{enumerate}
\item If $\psi \mapsto \int_M T \wedge \dif \psi$ extends from $C^1_\cpt(M, \Omega^{\ell - 1})$ to an $\ell - 1$-current, we define the \dfn{exterior derivative}
$$\int_M \dif T \wedge \psi := -\int_M T \wedge \dif \psi.$$
\item We define a positive Radon measure $\star |T|$, the \dfn{total variation}, by
$$\int_M f \star |T| := \sup_{|\varphi| \leq |f|} \int_M T \wedge \varphi.$$
\item We define the \dfn{polar part} $\psi$ of $T$ by 
$$\int_M T \wedge \varphi = \int_M (\psi \wedge \varphi)|T|.$$
\end{enumerate}
\end{definition}

See \cite{simon1983GMT} for more on the theory of currents.
The polar part is well-defined by \cite[Theorem 4.14]{simon1983GMT}.

% The following version of the Arzela-Ascoli theorem is well-known, but for convenience we recall it.

% \begin{proposition}[Arzela-Ascoli theorem]\label{AA Holder}
% Suppose that $(u_n)$ is a sequence of Lipschitz functions with bounded Lipschitz norms on a compact metric space. Then there is a subsequence of $(u_n)$ which converges in $C^{1-}$ to a Lipschitz function.
% \end{proposition}
% \begin{proof}
% By the classical Arzela-Ascoli theorem, along a subsequence we have $u_n \to u$ in $C^0$, where
% $$|u(x) - u(y)| \leq \liminf_{n \to \infty} |u_n(x) - u_n(y)| + 2 \|u_n - u\|_{C^0} \leq \liminf_{n \to \infty} \Lip(u_n) \cdot \dist(x, y)$$
% and hence
% $$\Lip(u) \leq \liminf_{n \to \infty} \Lip(u_n) < \infty.$$
% Now let $0 < \alpha < 1$, and assume without loss of generality that $u = 0$, so that we must show that $\|u_n\|_{C^\alpha} \to 0$.
% In fact,
% $$\frac{|u_n(x) - u_n(y)|}{\dist(x, y)^\alpha} = \left|\frac{u_n(x) - u_n(y)}{\dist(x, y)}\right|^\alpha \cdot |u_n(x) - u_n(y)|^\alpha \leq \Lip(u_n)^\alpha \cdot \|u_n\|_{C^0}^{1 - \alpha}.$$
% Since $u_n \to 0$ in $C^0$, and $(\Lip(u_n))$ is uniformly bounded, the claim follows.
% \end{proof}

With the measure-theoretic machinery above in place, we recall some facts about Ruelle-Sullivan currents.
Let $(\lambda, \mu)$ be a measured, oriented, tangentially $C^1$ lamination.
Then the Ruelle-Sullivan current $T_\mu$ is a well-defined closed $d-1$-current \cite[Theorem 8.2]{daskalopoulos2020transverse}. 
In particular, we may lift $T_\mu$ to the universal cover $\tilde M$, where it is exact \cite[Theorem 8.3]{daskalopoulos2020transverse}.
Moreover, $T_\mu$ has an intrinsic definition as the unique $d-1$-current with a certain polar decomposition.
To be more precise, recall that $\mu$ defines a measure on $\supp \lambda$: in each flow box $F_\alpha$, an open set $U$ has measure
\begin{equation}\label{transverse measure of an open set}
\mu(U) := \int_{K_\alpha} |F_\alpha(\{k\} \times J) \cap U| \dif \mu_\alpha(k).
\end{equation}

% \begin{proof}
% We first claim that the right-hand side of (\ref{RS current}) is always finite, and is continuous in $\varphi$.
% In fact, possibly after refining $(\chi_\alpha)$, we may assume that it is a locally finite partition of unity.
% In particular, we just need to check the continuity in a single flow box:
% $$\left|\int_{K_\alpha} \left[\int_{\RR^{d - 1} \times \{k\}} (F_\alpha^{-1})^* (\chi_\alpha \varphi) \right] \dif \mu_\alpha(k)\right| \leq \int_{K_\alpha} \int_{\RR^{d - 1} \times \{k\}} |(F_\alpha^{-1})^* (\chi_\alpha \varphi)| \dif \mu_\alpha(k).$$
% The inner integral is controlled by $\|\varphi\|_{C^0(U_\alpha)} \cdot |U_\alpha|$ where $U_\alpha$ is the image of $F_\alpha$.
% The outer integral is then well-defined because it is against a Radon measure.

% We next observe that the choice of partition of unity is irrelevant, thus if $\varphi$ has compact support in $U_\alpha \cap U_\beta$, then
% \begin{equation}\label{well-defined of Ruelle-Sullivan}
% \int_{K_\alpha} \int_{\RR^{d - 1} \times \{k\}} (F_\alpha^{-1})^* \varphi \dif \mu_\alpha(k) = \int_{K_\beta} \int_{\RR^{d - 1} \times \{k\}} (F_\beta^{-1})^* \varphi \dif \mu_\beta(k).
% \end{equation}
% Indeed,
% \begin{align*}
% \int_{K_\alpha} \int_{\RR^{d - 1} \times \{k\}} (F_\alpha^{-1})^* \varphi \dif \mu_\alpha(k)
% &= \int_{K_\beta} (F_\alpha F_\beta^{-1})^* \left[\int_{\RR^{d - 1} \times \{k\}} (F_\alpha^{-1})^* \varphi \dif \mu_\alpha(k)\right] \\
% &= \int_{K_\beta} \left[\int_{\RR^{d - 1} \times \{k\}} (F_\beta^{-1})^* \varphi\right] (F_\alpha F_\beta^{-1})^* \dif \mu_\beta(k) \\
% &= \int_{K_\beta} \int_{\RR^{d - 1} \times \{k\}} (F_\beta^{-1})^* \varphi \dif \mu_\beta(k)
% \end{align*}
% where the last equation is because of the measure-preserving nature of the transition maps; this proves (\ref{well-defined of Ruelle-Sullivan}).

% Finally, if a $d-2$-form $\psi$ has compact support in a single flow box, then
% $$\int_{\RR^{d - 1} \times \{k\}} (F_\alpha^{-1})^* \dif \psi = \int_{\RR^{d - 1} \times \{k\}} \dif((F_\alpha^{-1})^* \psi) = 0$$
% by Stokes' theorem, so
% \begin{align*}
% \int_M \dif T_\mu \wedge \psi &= -\int_M T_\mu \wedge \dif \psi \\
% &= -\int_{K_\alpha} \int_{\RR^{d - 1} \times \{k\}} (F_\alpha^{-1})^* \dif \psi \dif \mu_\alpha(k) = 0. \qedhere
% \end{align*}
% \end{proof}

\begin{lemma}
For a measured, oriented, tangentially $C^1$ lamination $(\lambda, \mu)$, with Lipschitz normal vector $\normal_\lambda$, the polar decomposition of $T_\mu$ is
\begin{equation}\label{polar ruelle sullivan}
T_\mu = \normal_\lambda \mu.
\end{equation}
\end{lemma}
\begin{proof}
For an open set $U \subseteq M$ in a flow box $F_\alpha$, the total variation satisfies
$$\int_U \star |T_\mu| = \sup_{\|\varphi\|_{C^0} \leq 1} \int_{K_\alpha} \int_{\{k\} \times J} \varphi \dif \mu_\alpha(k)$$
where the supremum ranges over $d-1$-forms $\varphi$ with compact support in $U$.
However, $\star \normal_\lambda^\flat$ is the Riemannian measure on $F_\alpha(\{k\} \times J)$, so
$$\int_{\{k\} \times J} \varphi \leq \int_{\{k\} \times J} (F_\alpha^{-1})^*(\star \normal_\lambda^\flat).$$
Since $\|\normal^\lambda\|_{C^0} = 1$, it follows that a sequence of cutoffs of $\star \normal_\lambda^\flat$ to more and more of $U$ is a maximizing sequence.
Therefore $\normal_\lambda$ is the polar part of (\ref{polar ruelle sullivan}), and
$$\int_U \star |T_\mu| = \int_{K_\alpha} \int_{\{k\} \times J} (F_\alpha^{-1})^*(1_U \star \normal_\lambda^\flat) \dif \mu_\alpha(k).$$
The inner integral is the Riemannian measure of $F_\alpha(\{k\} \times J) \cap U$, so by (\ref{transverse measure of an open set}), $|T_\mu| = \mu$.
\end{proof}

The above computation motivates the definition of Ruelle-Sullivan current of a \emph{nonorientable} lamination.
To be more precise, if $\lambda$ is a nonorientable lamination with normal vector field $\normal_\lambda$, then we can view $\normal_\lambda$ as a section of a (necessarily twisted) line bundle $L$ over $M$.
We can then define $T_\mu$ to be $\normal_\lambda \mu$, which makes sense as a distributional section of $L$, and can be tested against any continuous $d-1$-form on $M$ whose support is contained in a contractible set.
In particular, we shall speak of the Ruelle-Sullivan current of any measured lamination, even if it is nonorientable.

\begin{lemma}\label{convergence of normals}
If $(\lambda_n, \mu_n) \to (\lambda, \mu)$, $x_n \in \supp \lambda_n$ converges to $x \in \supp \lambda$, and $(\lambda_n), \lambda$ have continuous normal vector fields $(\normal_n), \normal$, then $\normal_n(x_n) \to \normal(x)$ pointwise.
\end{lemma}
\begin{proof}
	Choose a continuous $d-1$-form $\varphi$ which extends $\star \normal^\flat$.
	Then for every $\varepsilon > 0$,
	$$\int_{B(x, \varepsilon)} T_\mu \wedge \varphi = \mu(B(x, \varepsilon))$$
	so by the portmanteau theorem, for almost every $\varepsilon > 0$,
	\begin{equation}\label{epsilon is a continuity set}
		\lim_{n \to \infty} \frac{\int_{B(x, \varepsilon)} T_{\mu_n} \wedge \varphi}{\mu_n(B(x, \varepsilon))} = \frac{\int_{B(x, \varepsilon)} T_\mu \wedge \varphi}{\mu(B(x, \varepsilon))} = 1.
	\end{equation}
	On the other hand, if we assume that there exist $\delta, \varepsilon > 0$ and a coordinate system such that for every $y \in \supp \lambda_n \cap B(x, \varepsilon)$,
	$$|\normal_n - \normal| \geq \delta,$$
	then possibly after shrinking $\varepsilon$ we may assume that (\ref{epsilon is a continuity set}) holds, hence by (\ref{polar ruelle sullivan}),
	$$\int_{B(x, \varepsilon)} T_{\mu_n} \wedge \varphi = \int_{B(x, \varepsilon)} \normal_n^\flat \wedge \star \normal^\flat \dif \mu_i \leq (1 - O(\delta)) \mu_n(B(x, \varepsilon))$$
	and therefore $\delta = 0$, a contradiction.
\end{proof}


%%%%%%%%%%%%%%%%%%%%%%%%%%%%
\subsection{Calculus of variations}
We record various well-known facts about $1$-harmonic functions and minimal surfaces.

\begin{proposition}[Miranda stability theorem]
  Suppose that $M$ is a compact manifold, possibly with boundary.
	If a sequence of functions $(u_n)$ (not necessarily of the same trace) is bounded in $L^1(M)$ and satisfies
\begin{equation}\label{boundedness in Miranda}
	\limsup_{n \to \infty} \int_M \star |\dif u_n| \leq \liminf_{n \to \infty} \inf_{v|_{\partial M} = 0} \int_M \star |\dif(u_n + v)| < \infty,
\end{equation}
	then there exists a function $u$ of least gradient such that along a subsequence, $u_n \to u$ in $L^1(M)$ and $\dif u_n \to \dif u$ in the weak topology of measures.
\end{proposition}
\begin{proof}
The forgetful map $BV(M) \to L^1(M)$ is compact, so (\ref{boundedness in Miranda}) and the bounds in $L^1(M)$ imply that $(u_n)$ has a convergent subsequence.
The rest of the proof is similar to \cite[Teorema 3 and Osservazione 3]{Miranda67}; see \todo{\cite{BackusFLG}} for the straightforward modifications.
\end{proof}

% \begin{proposition}[maximum principle]\label{max princip}
% Let $u$ be a $1$-harmonic function.
% If $u$ attains a local maximum, then $u$ is constant.
% \end{proposition}
% \begin{proof}
% Let $y$ be a local maximum of $u$; then $\partial \{u \geq y\}$ is a minimal hypersurface by Theorem \ref{main thm of old paper}, or else it is empty. If it is empty, then $M = \{u \geq y\}$ and $u$ is constant, so we exclude this case.
% In a neighborhood of any point of $\partial \{u \geq y\}$ we may decrease $\int \star |\dif u|$ by decreasing $y$ a small amount, which violates that $u$ has least gradient.
% \end{proof}

\begin{theorem}[stable Bernstein theorem]
	Let $M$ be a manifold of bounded geometry, and dimension $d \in \{2, 3, 4\}$.
	Then there exists $A \geq 0$ such that for every complete stable minimal hypersurface $N$ in $M$ with trivial normal bundle,
	$$|\Two_N(P)| \leq \frac{A}{1 + \dist(P, \partial M)}.$$
\end{theorem}
\begin{proof}
	If $d = 2$ this clearly holds with $A = 0$.
	If $d = 3$ this was proven in \cite{Schoen2016}, and we refer to \cite[Theorem 2.10]{colding2011course} for a modern proof.
	If $d = 4$ this result can be derived from \cite[Theorem 1]{Chodosh2021} using the sort of techniques discussed in \cite[\S3]{White13}.
\end{proof}


%%%%%%%%%%%%%%%%%%%%%%%%%%%%%%%%%%%%%%%%%%
\section{Regularity of flow boxes}\label{Regularity}
The goal of this section is to prove the following regularity theorem for minimal laminations that we will use several times.
The proof is based on \cite[Theorem 1.1]{Solomon86}, which addresses the case that $\lambda$ is a minimal foliation, and does not explictly spell out the quantitative regularity of the laminar atlas.
Lipschitz regularity is optimal even for the nicest case of a geodesic foliation of $\Hyp^2$ \cite[\S1]{Solomon86}, so this result is sharp.

\begin{proposition}\label{regularity theorem}
Let $(M, g)$ be a Riemannian manifold of dimension $d \geq 2$ and bounded geometry.
Let $\lambda$ be a minimal lamination in $M$ such that for some $A > 0$ and every leaf $N \in \Leaves \lambda$,
\begin{equation}\label{curvature bound in regularity}
	\|\Two_N\|_{C^0} \leq A.
\end{equation}
Then:
\begin{enumerate}
\item There exists a Lipschitz line bundle on $M$ which is normal to every leaf of $\lambda$.
\item There exist constants $L = L(g, A) > 0$ and $r = r(g, A) > 0$, and a Lipschitz laminar atlas $(F_\alpha)$ for $\lambda$, such that for every $\alpha$,
\begin{equation}\label{conorm of flow box}
	\max(\Lip(F_\alpha), \Lip(F_\alpha^{-1})) \leq L,
\end{equation}
and the image of $F_\alpha$ contains a ball of radius $r$.
\item $F_\alpha$ is tangentially $C^\infty$, with seminorms only depending on $A, g$.
\end{enumerate}
\end{proposition}

To begin the proof we first consider when we can represent the leaves of $\lambda$ as graphs in a uniform way.

\begin{lemma}\label{existence of tubes}
	Let $N$ be a connected hypersurface in $\RR^d = \RR^{d - 1}_x \times \RR_y$ which is tangent to $\{y = 0\}$ at the origin.
	If $\|\Two_N\|_{C^0} \leq \log(5/4)$, then for every $(x, y) \in N \cap B_1$,
	$$\max(|y|, 0.5 \cdot |\normal_N(x, y) - \partial_y|) \leq \|\Two_N\|_{C^0}.$$
\end{lemma}
\begin{proof}
	Near $0$, $N$ can be represented a graph $\{y = f(x)\}$, since it is tangent to $\{y = 0\}$.
	This representation is valid on the component of the set $\{|\nabla f(x)| < \infty\}$ containing $0$, and it is related to the unit normal by
\begin{equation}\label{nabla as a normal}
	\nabla f(x) = \frac{\partial_y - \normal(x, f(x))}{\sqrt{1 + |\nabla f(x)|^2}}.
\end{equation}
	Taking derivatives of both sides,
	$$-\nabla^2 f(x) = \frac{\nabla \normal(x, f(x)) \cdot (\partial_x \otimes \partial_x + \nabla f(x) \otimes \partial_y)}{\sqrt{1 + |\nabla f(x)|^2}} + \frac{\nabla^2 f(x) \cdot (\nabla f(x) \otimes (\partial_y - \normal(x, f(x))))}{(1 + |\nabla f|^2)^{3/2}}.$$
	Here $-\nabla^2$ denotes the negative Hessian, not the Laplacian.
	We can use (\ref{nabla as a normal}) to bound
	$$|\partial_y - \normal(x, f(x))| \leq |\nabla f(x)|\sqrt{1 + |\nabla f(x)|^2} \leq |\nabla f(x)| + |\nabla f(x)|^2.$$
	Since
	$$|\partial_x \otimes \partial_x + \nabla f(x) \otimes \partial_y| \leq \sqrt{1 + |\nabla f(x)|^2},$$
	and $\nabla \normal = \Two_N$, we conclude
\begin{equation}\label{bound Hessian by Two}
	|\nabla^2 f(x)| \leq |\Two_N(x, f(x))| + |\nabla^2 f(x)| (|\nabla f(x)|^2 + |\nabla f(x)|^3).
\end{equation}

	In order to control the error terms in (\ref{bound Hessian by Two}), we make the \dfn{bootstrap assumption}
\begin{equation}\label{bootstrap}
	|\nabla f(x)| \leq 1/4,
\end{equation}
	which is at least valid in some small neighborhood $B_R$ of $0$ since (\ref{nabla as a normal}) and the fact that $N$ is tangent to $\{y = 0\}$ at $0$ imply that $\nabla f(0) = 0$.
	One can show using Taylor's theorem that (\ref{bootstrap}) implies that 
	$$\frac{1}{1 - |\nabla f(x)|^2 - |\nabla f(x)|^3} \leq 1 + |\nabla f(x)|,$$
	so by (\ref{bound Hessian by Two}),
	$$|\nabla^2 f(x)| \leq \frac{\|\Two_N\|_{C^0}}{1 - |\nabla f(x)|^2 - |\nabla f(x)|^3} \leq \|\Two_N\|_{C^0} \cdot (1 + |\nabla f(x)|).$$
	If we write $x = (r, \theta)$, then by the mean value theorem, it follows that for $r \leq R$,
\begin{align*}
	|\nabla f(x)| &\leq \|\Two_N\|_{C^0} \int_0^r 1 + |\nabla f(s, \theta)| \dif s.
\end{align*}
	So by Gr\"onwall's inequality,
\begin{align*}
	|\nabla f(x)| &\leq r \|\Two_N\|_{C^0} + \|\Two_N\|_{C^0}^2 \int_0^r s \exp(\|\Two_N\|_{C^0} \cdot (r - s)) \dif s \\
	&= r \|\Two_N\|_{C^0} + \exp(r \|\Two_N\|_{C^0}) - r \|\Two_N\|_{C^0} - 1 \\
	&= \exp(r \|\Two_N\|_{C^0}) - 1.
\end{align*}
	Recalling that $\|\Two_N\|_{C^0} \leq \log(5/4)$, and $r \leq R < 1$, we conclude that on $B_R$, $|\nabla f| < 1/4$.
	That is, we recover the bootstrap assumption (\ref{bootstrap}) on some ball $B_{R'}$ for some $R' > R$, and hence on the entire ball $B_1$.

	By the mean value theorem again,
\begin{align*}
	|f(x)| &\leq \int_0^r |\nabla f(s, \theta)| \dif s \leq \int_0^r \exp(s \|\Two_N\|_{C^0}) - 1 \dif s \\
	&= \|\Two_N\|_{C^0}^{-1} (\exp(r \|\Two_N\|_{C^0}) - r\|\Two_N\| - 1) \\
	&\leq r^2 \|\Two_N\|_{C^0}
\end{align*}
	where we again used the bound $\|\Two_N\|_{C^0} \leq \log(5/4) < 0.25$ to control the higher-order terms in the Taylor expansion of $\exp(r \|\Two_N\|_{C^0})$. Since $r < 1$ was arbitrary we conclude $\|f\|_{C^0} \leq \|\Two_N\|_{C^0}$. Moreover, by (\ref{nabla as a normal}),
\begin{align*}
	|\normal(x, f(x)) - \partial_y| &\leq |\nabla f(x)| \leq 2r \|\Two_N\|_{C^0}. \qedhere
\end{align*}
\end{proof}

Our next lemma guarantees the existence of normal coordinates in which the leaves of $\lambda$ are close to hyperplanes $\{y = y_0\}$.
An analogous result was proven by \cite{Solomon86} (without the quantitative dependence) by a very different means, using the regularity theory for integral flat convergence of minimal currents \cite[Theorem 5.3.14]{federer2014geometric}.
We did not do this, both because \cite{Solomon86} requires that $\lambda$ be a foliation, and because it does not seem particularly easy to recover quantitative bounds from the highly general theory of \cite[Chapter 5]{federer2014geometric}.

\begin{lemma}\label{lams have C0 fields}
	For every sufficiently small $\delta > 0$ there exist $r = r(\delta, g, A) > 0$ such that for every minimal lamination $\lambda$ satisfying (\ref{curvature bound in regularity}) we can choose normal coordinates $(x, y) \in \RR^{d - 1} \times \RR$ on $B(P, r)$ so that
\begin{equation}\label{normal is basically dy}
	\|\normal_\lambda - \partial_y\|_{C^0(B(P, r))} \leq \delta.
\end{equation}
\end{lemma}
\begin{proof}
	Suppose not.
	Then, after rescaling to set $A = 1$, there exist minimal laminations $\lambda_n'$ in $B(P, 1/n)$ such that
	$$\sup_{N \in \Leaves \lambda_n'} \|\Two_N\|_{C^0} \leq 1,$$
	but no matter how we rotate normal coordinates based at $P$, (\ref{normal is basically dy}) fails for $\lambda_n'$.
	By taking normal coordinates and rescaling by $n$, we can find (possibly nonminimal) laminations $\lambda_n$ in $B_1 \subset \RR^d$ such that
% \begin{sublemma}
% 	For each $n \geq \inj(g)$ there exists a lamination $\lambda_n$ in $\RR^d$ so that every rotation of $B_1$ fails (\ref{normal is basically dy}), but satisfies
\begin{equation}\label{bounds on Two in representation}
	\sup_{N \in \Leaves \lambda_n} \|\Two_N\|_{C^0} \lesssim_g \frac{1}{n^2}
\end{equation}
	but for every rotation of $\RR^d$, (\ref{normal is basically dy}) fails.
% \end{sublemma}

% We remark carefully that the lamination in the above sublemma may not be minimal.
% However this fact will be irrelevant.

% \begin{proof}
% 	Since $n \geq \inj(g)$ we can pass to the tangent space and rescale $B(P, 1/n)$ to obtain $B_1$. The scaling makes the second fundamental form $\Two_N'$ with respect to $g$ satisfy
% $$\|\Two_N'\|_{C^0} \leq \frac{e^{-O(n^2)}}{n^2} \lesssim \frac{1}{n^2}.$$
% 	Here and always all implied constants may depend on $g$.
% 	In normal coordinates,
% 	$$g_{\mu \nu} = \delta_{\mu \nu} + \sum_{k=2}^\infty c_k \frac{|x|^k}{n^k},$$
% 	and differentiating this expression gives that the connection $1$-form $\Gamma$ for the Levi-Civita connection $\nabla_g$ satisfies $|\Gamma| \lesssim n^{-2}$.
% 	But $\|\Two_N - \Two_N'\|_{C^0} \lesssim \|\Gamma\|_{C^0}$, since $\Two_N - \Two_N'$ is the difference between the euclidean and $\nabla_g$ derivatives of the conormal.
% \end{proof}

	By Lemma \ref{existence of tubes}, every leaf of $\lambda_n$ is $O(n^{-2})$-close to its tangent spaces in $C^1$.
	In particular, if $n \gg \delta^{-1/2}$, and the normal vector at some point to $N \in \Leaves \lambda_n$ is $\partial_y$, then
	$$\|\normal_N - \partial_y\|_{C^0(B(P, r))} \ll \delta.$$
	We can always impose this for some $N$ by applying a rotation of $\RR^d$.
	But by our contradiction assumption, there exists some leaf $N'$ of $\lambda_n$ and some $P \in N'$ so that $|\normal_{N'}(P) - \partial_y| \geq \delta$ and hence if $n$ is large enough,
	$$\inf_{N'} |\normal_{N'} - \partial_y| \geq \frac{\delta}{2}.$$
	By the reverse triangle inequality it follows that
\begin{equation}\label{discrepancy in normals}
	\inf_{\substack{P \in N\\ P' \in N'}} \sin \angle(\normal_N(P), \normal_{N'}(P')) \gtrsim \delta
\end{equation}
	at least if $\delta$ is smaller than an absolute constant.
	
	In order to obtain a contradiction, we may assume that $r$ is small depending on $\delta$, and consider the restriction of $\lambda_n$ to $B_r$.
	In particular, we may assume that there are points $P \in N \cap B_r$ and $P' \in N' \cap B_r$ for some $r \ll \delta$.
	By (\ref{discrepancy in normals}), the angle $\theta$ between the tangent spaces $T_P N$ and $T_{P'} N'$ is $\gtrsim \delta$, but $|P - P'| \ll \delta$.
	This is only possible if $T_P N$ and $T_{P'} N'$ intersect in some ball $B_{r'}$ for some $r' \ll 1$, say $r' = 1/2$.
	But then, since $N, N'$ are $O(n^{-2})$-close to $T_P N$ and $T_{P'} N'$ in $C^0$, then $N$ and $N'$ intersect in $B_1$ if $n$ is large enough.
	However, $N, N'$ were assumed to be leaves of the same lamination $\lambda_n$ in $B_1$, so this is a contradiction.
\end{proof}

\begin{proof}[Proof of Proposition \ref{regularity theorem}]
Fix $\delta > 0$ to be chosen later, and $P \in M$.
By Lemma \ref{lams have C0 fields}, there exists $r = r(\delta, g, A) > 0$ such that $B(P, r)$ admits cylindrical, normal coordinates $(x, y) \in 3\Ball^{d - 1} \times (-2, 2)$, and
\begin{equation}\label{normal is almost constant}
\|\normal - \partial_y\|_{C^0(B(P, r))} < \delta.
\end{equation}
If $\delta$ is chosen small enough depending on $g$, then we may assume that in $2\Ball^{d - 1} \times (-1, 1)$,
every leaf is the graph of a function, say $u_k: 3\Ball^{d - 1} \to (-2, 2)$ where $u_k(0) = k$.

If $r$ is chosen small enough depending on $g$ and some $\varepsilon > 0$, then the metric $h$ induced by $g$ on $3\Ball^{d - 1} \times (-2, 2)$ has curvature $\Riem_h$ satisfying $\|\Riem_h\|_{C^0} \leq \varepsilon$.
Writing $Lu = 0$ for the minimal surface equation on $(2\Ball^{d - 1} \times (-1, 1), h)$, it follows that the ellipticity of $L$ and the scale-invariant H\"older norms of $L$ are bounded independently of $\delta$ if $\varepsilon$ is smaller than some absolute constant.
Moreover, $Lu_k = 0$ since the graph of $u_k$ is a minimal surface, and we have $\|u_k\|_{C^0} \leq 2$.
So by a straightforward modification of \cite[Corollary 16.7]{gilbarg2015elliptic}, we have uniformly in $k \in (-1, 1)$ that
\begin{equation}\label{norms on uk}
\|u_k\|_{C^2} \lesssim 1.
\end{equation}

Now let $-1 < k < \ell < 1$, and let $v_{\ell k} := u_\ell - u_k$.
Then $v_{\ell k}$ is the difference of two elements of the kernel of $L$, so $v_{\ell k}$ solves a linear elliptic PDE $Q_{\ell k} v_{\ell k} = 0$.
The ellipticity of $Q_{\ell k}$ and the scale-invariant H\"older norms of $Q_{\ell k}$ only depend on $g$ and $C$, but not on $\ell, k, \delta$.
These assertions about $Q_{\ell k}$ follow from (\ref{norms on uk}) and a routine modification of \cite[Lemma 1.26]{colding2011course}.

Since the graphs of $u_\ell, u_k$ are leaves of a lamination, $v_{\ell k} > 0$.
By the Schauder \cite[Theorem 6.2]{gilbarg2015elliptic} and Harnack \cite[Theorem 9.25]{gilbarg2015elliptic} inequalities applied to the elliptic operator $Q_{\ell k}$, it follows that for every $x \in \Ball^{d - 1}$,
\begin{equation}\label{Schauder Harnack}
	\|\dif v_{\ell k}\|_{C^0(\Ball^{d - 1})} \lesssim \|v_{\ell k}\|_{C^0(2 \Ball^{d - 1})} \lesssim \inf_{\Ball^{d - 1}} v_{\ell k} \leq v_{\ell k}(x).
\end{equation}
In particular, for every $x$,
\begin{equation}\label{bound on du}
|\dif u_\ell(x) - \dif u_k(x)| \lesssim |u_\ell(x) - u_k(x)|
\end{equation}
and it follows that
\begin{equation}\label{vertical Lipschitz}
|\normal(x, u_\ell(x)) - \normal(x, u_k(x))| \lesssim |u_\ell(x) - u_k(x)|.
\end{equation}

To extend (\ref{vertical Lipschitz}) to a Lipschitz bound on $\normal$, let $X_1, X_2 \in (\Ball^{d - 1} \times (-1, 1)) \cap \supp \lambda$.
Then there exist $x_1, x_2 \in \Ball^{d - 1}$ and $k_1, k_2 \in (-1, 1)$ such that $X_i = (x_i, u_{k_i}(x_i))$.
Setting $Y := (x_2, u_{k_1}(x_2))$,
$$|\normal(X_1) - \normal(X_2)| \leq |\normal(X_1) - \normal(Y)| + |\normal(Y) - \normal(X_2)|.$$
Then by (\ref{norms on uk}) and the mean value theorem,
$$|\normal(X_1) - \normal(Y)| \lesssim |\dif u_{k_1}(x_1) - \dif u_{k_1}(x_2)| \lesssim |X_1 - Y|.$$
Moreover, by (\ref{vertical Lipschitz}),
$$|\normal(Y) - \normal(X_2)| \lesssim |u_{k_1}(x) - u_{k_2}(x)| = |Y - X_2|.$$
If $\delta$ is chosen smaller than some absolute constant, then by (\ref{normal is almost constant}),
$$|\sin \angle(X_1 - Y, X_2 - Y)| > 1 - O(\delta)$$
and we conclude by the Pythagorean theorem that
$$|Y - X_2|^2 + |X_1 - Y|^2 \lesssim |X_1 - X_2|^2.$$
In conclusion,
$$|\normal(X_1) - \normal(X_2)| \lesssim |X_1 - X_2|$$
which implies that $\normal$ is Lipschitz on $V \cap \supp \lambda$, where $V$ is the neighborhood of $P$ which was mapped to $\Ball^{d - 1} \times (-1, 1)$ by the cylindrical coordinates $(x, y)$.
In particular, $V$ contains a ball of the form $B(P, s)$, where $s$ only depends on $r$ (and $r$ only depends on $g$ and $A$).
% Moreover, $|X_1 - X_2| \sim r \dist(X_1, X_2)$ by definition of the coordinates $(x, y)$, so we obtain
% \begin{equation}\label{lipschitz normal}
% 	|\Lip(\normal)| \lesssim r^{-1}.
% \end{equation}
Taking a Lipschitz extension of $\normal$ to $V$ we obtain the desired Lipschitz normal subbundle.

Following \cite[Appendix B]{ColdingMinicozziIV}, we construct the laminar flow box
\begin{align*}
	F: \RR^{d - 1}_\xi \times \RR_\eta &\to V \subseteq \RR^{d - 1}_x \times \RR_y \\
	(\xi, \eta) &\mapsto (\xi, f(\xi, \eta))
\end{align*}
by setting
$$f(\xi, \eta) := u_\eta(\xi)$$
if $u_\eta$ exists, and if $k < \eta < \ell$ and there does not $k < \eta' < \ell$ such that $u_{\eta'}$ exists, then
$$f(\xi, \eta) := u_k(\xi) + \frac{\eta - k}{\ell - k} v_{\ell k}(\xi)$$
is the linear interpolant of $u_k$ and $u_\ell$.
It is clear that $F$ is uniformly tangentially $C^\infty$, since $u_k$ and $v_{\ell k}$ are bounded in $C^\infty$ by elliptic bootstrapping.

It remains to show that $F$ is a Lipschitz isomorphism.
To do this, we first claim that $\Lip(f) \sim 1$.
It is clear that $f$ is Lipschitz (in fact, smooth) in the $\xi$ direction, with constant comparable to $\sup_k \|u\|_{C^1} \lesssim 1$ by (\ref{norms on uk}).
If $-1 < k < \ell < 1$, then by (\ref{bound on du}) and (\ref{Schauder Harnack}),
\begin{equation}\label{f lip}
	|f(\xi, k) - f(\xi, \ell)| \lesssim |u_k(\xi) - u_\ell(\xi)| \lesssim \ell - k.
\end{equation}
This shows that $f$ is Lipschitz in the $\eta$ direction on the leaves with constant comparable to $1$, and hence on its entire domain by linear interpolation, proving the claim.

We also need to estimate the co-Lipschitz norm, using (\ref{Schauder Harnack}),
\begin{equation}\label{f colip}
	\frac{|f(\xi, \eta_1) - f(\xi, \eta_2)|}{|\eta_1 - \eta_2|} \geq \inf_{k < \ell} \frac{v_{\ell k}(\xi)}{\ell - k} \gtrsim \inf_{k < \ell} \frac{v_{\ell k}(0)}{\ell - k} = 1.
\end{equation}

We can then estimate using (\ref{f lip})
$$|F(\xi_1, \eta_1) - F(\xi_2, \eta_2)| \lesssim |\xi_1 - \xi_2| + \Lip(f)(|\xi_1 - \xi_2| + |\eta_1 + \eta_2|)$$
so that $\Lip(F) \lesssim 1 + \Lip(f) \lesssim 1$.
One can also use the inverse function theorem and (\ref{f lip}) and (\ref{f colip}) to bound 
$$\Lip(F^{-1}) \lesssim \Lip(f)\left(1 + \sup_\xi \sup_{\eta_1 < \eta_2} \frac{|\eta_1 - \eta_2|}{|f(\xi, \eta_1) - f(\xi, \eta_2)|}\right) \lesssim 1.$$
This implies that $F$ is a Lipschitz isomorphism with Lipschitz constants comparable to $1$.
So the composition of $F$ with the change of coordinates at the start of this proof is a laminar flow box in a small neighborhood of $(0, 0)$ whose image has radius $cr$ for some small $c > 0$, and whose Lipschitz constants are comparable to $O(r^{-1})$.
\end{proof}

%%%%%%%%%%%%%%%%%%%%%%%%%%%%%%%%%%%%%%%%%
\section{Proofs of main theorems}\label{CompactnessSec}
We shall now use Proposition \ref{regularity theorem} to prove Theorems \ref{compactness theorem}, \ref{implication theorem}, and \ref{main thm}.
Throughout, let $M$ be a manifold of constant sectional curvature and dimension $2 \leq d \leq 7$.

\subsection{Compactness}
We begin by proving Theorem \ref{compactness theorem}.

Let $P \in M$.
By Proposition \ref{regularity theorem}, there exist $r > 0$ and $L \geq 1$ such that for every large $n \in \NN$, $B(P, r)$ is contained in the image of a flow box $F_n$ for $\lambda_n$ with Lipschitz constant $L$, such that $F_n(0, 0) = P$.
By the Arzela-Ascoli theorem and the interpolation theorem for H\"older spaces, along a subsequence $F_n \to F$ in $C^{1-}$ for some $C^{1-}$ map $F: I \times J \to B(P, r)$ and some $I \subseteq \RR$, $J \subseteq \RR^{d - 1}$, such that on the image $V$ of $F$, $F^{-1}$ is also $C^{1-}$.
Moreover, $F(0, 0) = P$, so that $F: I \times J \to V$ is a Lipschitz isomorphism onto a set which contains $P$.
Since $P$ was arbitrary, it follows that we can find laminar atlases $(F_\alpha^n, K_\alpha^n)$ for each large $n \in \NN$ such that $F_\alpha^n \to F_\alpha$, where the images of $F_\alpha$ and $F_\alpha^n$ are an open cover $(U_\alpha)$ of $M$ independent of $n$, and $(F_\alpha)$ satisfies the usual transition relations.

We now construct the limiting lamination.
Let $\Hypspace I$ be the space of closed subsets of $I$ with its Hausdorff distance; since $I$ is a compact metric space, so is $\Hypspace I$ \cite[Theorem 4.17]{nadler2017continuum}, so we may diagonalize so that for every $\alpha$, either $K^n_\alpha \to K_\alpha$ for some nonempty $K_\alpha$ in the Hausdorff distance on $I$, or for all $n \geq n^*(\alpha)$, $K_\alpha^n$ is empty (in which case we define $K_\alpha = \emptyset$).

In order to ensure that the laminations $\lambda_n$ do not escape to infinity, fix a compact set $E \subseteq M$ such that every leaf of every $\lambda_n$ meets $E$.
Then there exists a finite set $A_E \subseteq A$ such that $E \subseteq \bigcup_{\alpha \in A_E} U_\alpha$.

\begin{lemma}\label{label sets are nonempty}
	There exists $\alpha$ such that $K_\alpha$ is nonempty.
\end{lemma}
\begin{proof}
	Suppose not; then for
	$$n \geq \max_{\alpha \in A_E} n^*(\alpha)$$
	and $\alpha \in A_E$, $K_\alpha^n = \emptyset$, so no leaves of $\lambda_n$ meet $U_\alpha$, and hence no leaves of $\lambda_n$ meet $E$.
	This is a contradiction since $\lambda_n$ has a leaf.
\end{proof}

Now let $\psi_{\alpha \beta}$ and $\psi_{\alpha \beta}^n$ be the transition maps, thus $\psi_{\alpha \beta}^n$ induces a map
$$\psi_{\alpha \beta}^n: K_\alpha^n \to K_\beta^n.$$
By convergence of $(F_\alpha^n)$, $\psi_{\alpha \beta}$ induces a map $K_\alpha \to K_\beta$.

\begin{definition}
	A \dfn{cocycle of labels} $(k_\alpha)_{\alpha \in A'}$ is a set $A' \subseteq A$ and an element of $\prod_{\alpha \in A'} K_\alpha$, such that:
\begin{enumerate}
	\item The cocycle condition: $k_\beta = \psi_{\alpha \beta}(k_\alpha)$ for $\alpha, \beta \in A'$.
	\item For every $\alpha \in A'$, if $\psi_{\alpha \beta}(k_\alpha)$ is well-defined, then $\beta \in A'$.
\end{enumerate}
\end{definition}

\begin{lemma}
	Every cocycle of labels $(k_\alpha)_{\alpha \in A'}$ defines a complete minimal hypersurface $N$ such that
	$$N \cap U_\alpha = F_\alpha(\{k_\alpha\} \times J).$$
\end{lemma}
\begin{proof}
We have the cocycle condition
$$(N \cap U_\alpha) \cap U_\beta = (N \cap U_\beta) \cap U_\alpha$$
which follows from the fact that
\begin{align*}
F_\alpha(\{k_\alpha\} \times J) \cap U_\beta
&= F_\beta(\psi_{\alpha \beta}(\{k_\beta\} \times J)) \cap U_\alpha \cap U_\beta \\
&= F_\beta(\psi_{\alpha \beta}(\{k_\beta\} \times J)) \cap U_\alpha.
\end{align*}
From the cocycle condition, it follows that $N$ honestly defines a Lipschitz hypersurface in $M$, which is complete in $\bigcup_{\alpha \in A'} U_\alpha$.
If $\overline N$ intersects $U_\alpha$ for some $\alpha \notin A'$, then $N$ intersects $U_\beta$ for some $\beta \in A'$ so that $U_\beta \cap U_\alpha \cap \overline N$ is nonempty.
But then $\psi_{\beta \alpha}(k_\beta)$ must be defined, so $\alpha \in A'$, a contradiction.
Therefore $N$ is complete in $M$.

To prove minimality, let
$$u_\alpha(k, x) = 1_{k > k_\alpha}$$
and similarly $u_\alpha^n(k, x) = 1_{k > k_\alpha^n}$ where $(k_\alpha^n) \in \prod_n K_\alpha^n$ converges to $k_\alpha$.
Then the pullback of $u_\alpha^n$ by $F_\alpha \circ (F_\alpha^n)^{-1}$ converges pointwise away from $F_\alpha^{-1}(N \cap U_\alpha)$ to $u_\alpha$, since $k_\alpha^n \to k_\alpha$.
Since $F_\alpha \circ (F_\alpha^n)^{-1}$ converges to the identity map in $C^{1-}$, and $F_\alpha^{-1}(N \cap U_\alpha)$ has zero measure, it follows that $u_\alpha^n \to u_\alpha$ almost everywhere, and hence in $L^1(I \times J)$ by the dominated convergence theorem.
But $u_\alpha^n$ has approximately least gradient in the sense of (\ref{boundedness in Miranda}) with respect to the pullback metric $F_\alpha^* g$, so its limit $u_\alpha$ must have least gradient by the Miranda stability theorem.
Thus by Theorem \ref{main thm of old paper}, $N \cap U_\alpha$ is minimal.
\end{proof}

\begin{lemma}
	Let $\lambda$ be the lamination with laminar atlas $(F_\alpha, K_\alpha)$.
	Then $\lambda$ is well-defined and minimal.
\end{lemma}
\begin{proof}
Since 
$$\supp \lambda \cap U_\alpha = K_\alpha \times J$$
and $K_\alpha$ is compact, $\supp \lambda$ is closed.
Now if we choose $\alpha$ such that $K_\alpha$ is nonempty, every element of $K_\alpha$ uniquely determines a cocycle of labels, and hence a leaf of $\lambda$.
So $\supp \lambda$ is nonempty, and since all of its leaves are complete minimal, $\lambda$ is minimal.
\end{proof}

\begin{lemma}
	$\lambda_n \to \lambda$ in Thurston's geometric topology.
\end{lemma}
\begin{proof}
If $K_\alpha$ is nonempty, then any $k_\alpha \in K_\alpha$ is the limit of some sequence $(k_\alpha^n)_n \in \prod_n K_\alpha^n$ \cite[Theorem 4.11]{nadler2017continuum}.
Thus $\{k_\alpha\} \times J$ can be written as the set of limits of sequences $(k_\alpha^n, x)_n \in \prod_n K_\alpha^n \times J$, and so any leaf $N$ of $\lambda$ can be written 
$$N = \left\{\lim_{n \to \infty} P_n: P_n \in N_n\right\}$$
for some sequence $(N_n) \in \prod_n \Leaves \lambda_n$.
In other words, leaves of $\lambda$ are pointwise limits of leaves in $\lambda_n$.

So it suffices to show that for $N \in \Leaves \lambda$, $P \in N$, and $P_n \to P$, where $P_n \in N_n$ and $N_n \in \Leaves \lambda_n$, $\normal_{N_n}(P_n) \to \normal_N(P)$.
To do this, suppose that $P \in U_\alpha$; $F_\alpha^n$ is close in $C^{1-}$ to $F_\alpha$, and the label $k^n_\alpha$ of $N_n$ is close to the label $k_\alpha$ of $N$.
So in some neighborhood $B$ of $P$, $N_n \cap B$ is close to $N \cap B$ in $C^{1-}$.
By assumption, $\|\Two_{N_n \cap B}\|_{C^0} \leq A$ and so $N_n \cap B$ is bounded in $C^2$.
The interpolation theorem for H\"older spaces then implies that $N_n \cap B$ is close to $N \cap B$ in $C^1$, so $\normal_{N_n}(P_n)$ is close to $\normal_N(P)$.
\end{proof}

Finally, suppose that $\mu_n$ is transverse to $\lambda_n$.
After possibly shrinking the $U_\alpha$ slightly for $\alpha \in A_E$, we may assume that they are precompact in $M$ and still form an open cover of $E$.
Then $K := \bigcup_{\alpha \in A_E} \overline{U_\alpha}$ is compact, so by Prohorov's theorem \cite[Theorem 13.29]{klenke2013probability}, there is a subsequence of $(T_{\mu_n})$ which converges to some $T_\mu|_K$ on $K$.
Moreover, by the portmanteau theorem,
$$\supp T_\mu|_K \subseteq \liminf_{n \to \infty} \supp T_{\mu_n}|_K \subseteq \liminf_{n \to \infty} \supp \lambda_n \cap K.$$
Here the $(\lambda_n)$ in the limit inferior refers to the subsequence which already converges in the Thurston topology (and has converging Ruelle-Sullivan currents).
In particular, the limit inferior is actually a limit and we conclude
$$\supp T_\mu|_K \subseteq \supp \lambda \cap K.$$
We may assume that $\mu_\alpha^n \to \mu_\alpha$ weakly for every $\alpha \in A_E$ and some positive Radon measures $\mu_\alpha$ (whose support is necessarily then contained in $K_\alpha$).
Moreover, $F^n_\alpha \to F_\alpha$ in $C^{1-}$ and $F^n_\alpha$ is bounded in tangential $C^\infty$, so $F_\alpha$ is tangentially $C^\infty$ by interpolation of H\"older spaces, and hence we have the convergence of pullbacks $(F^n_\alpha)^* \to F_\alpha^*$ in $C^0$ for any compactly supported $d-1$-form $\varphi$ on $U_\alpha$.
Taking the limit as $n \to \infty$ of the equation 
$$\int_{U_\alpha} T_{\mu_n} \wedge \varphi = \int_I \int_{\{k\} \times J} (F_\alpha^n)^* \varphi \dif \mu_\alpha^n(k),$$
we conclude that
$$\int_{U_\alpha} T_\mu|_K \wedge \varphi = \int_I \int_{\{k\} \times J} F_\alpha^* \varphi \dif \mu_\alpha(k).$$
In other words, $T_\mu|_K$ is Ruelle-Sullivan for $\lambda|_K$, possibly after shrinking $\lambda|_K$ so that their supports match.
By the measure-preserving condition in the definition of transverse measure, $T_\mu|_K$ extends uniquely to a Ruelle-Sullivan current $T_\mu$ on all of $M$, which then necessarily is a weak limit of $(T_{\mu_n})$.
This completes the proof of Theorem \ref{compactness theorem}.


%%%%%%%%%%%%%%%%%%%%%%%%%%%%%%%%%%%%%%
\subsection{Measured convergence implies flow box convergence}
In order to show that convergence in the weak topology of measures implies convergence in the $C^{1-}$ flow box topology, we first show a weaker mode of convergence, namely Thurston's geometric topology.
Thurston claimed this fact \cite[Proposition 8.10.3]{thurston1979geometry} in case $d = 2$, but he did not justify why the limit is geodesic, or why the convergence respects the normal vectors.

\begin{lemma}\label{limits of measured geodesic lams are geodesic}
	The set of minimal measured laminations is closed in the weak topology of measures.
\end{lemma}
\begin{proof}
Let $(\lambda, \mu)$ be a measured lamination and suppose that $(\lambda_i, \mu_i) \to (\lambda, \mu)$ in the weak topology of measures, where $(\lambda_i, \mu_i)$ are measured minimal.
Let $x \in \supp \lambda$ and $r > 0$ such that $B := B(x, r)$ is contractible.
In $B$, we can write $T_{\mu_i} = \dif u_i$ for some sequence of functions of least gradient $u_i \in BV(B)$.
Since $u_i$ is only defined up to a constant, we impose $\int_M \star u_i = 0$, so by Poincar\'e's inequality,
$$\|u_i\|_{L^1(B)} \lesssim r\mu_i(B) \leq 2r \mu(B) < \infty$$
for $i$ large.
So by the Miranda stability theorem, there exists a $1$-harmonic function $u$ such that along a subsequence, $\dif u_i \to \dif u$ in the weak topology of measures.
But then we must have $T = \dif u$, so $\lambda$ is minimal by Theorem \ref{main thm of old paper}.
\end{proof}

We are now ready to prove Theorem \ref{implication theorem}.
So let $(\lambda_n, \mu_n)$ be a sequence of measured minimal laminations convering to $(\lambda, \mu)$.
Then
\begin{equation}\label{support is nonincreasing}
	\supp \lambda \subseteq \liminf_{n \to \infty} \supp \lambda_n.
\end{equation}
Indeed, if $x \in \supp \lambda$ then for all $\varepsilon > 0$, $\mu(B(x, \varepsilon)) > 0$.
Then by the portmanteau theorem, for all $n$ large, $\mu_n(B(x, \varepsilon)) > 0$; this proves (\ref{support is nonincreasing}).

By (\ref{support is nonincreasing}), for every $x \in \supp \lambda$, $\varepsilon > 0$, and large $n$, $\supp \lambda_n \cap B(x, \varepsilon)$ is nonempty, and by Lemma \ref{limits of measured geodesic lams are geodesic}, $\lambda$ is a minimal lamination.
By Proposition \ref{regularity theorem}, $\lambda, \lambda_n$ admit Lipschitz normal vectors, so by Lemma \ref{convergence of normals}, $\lambda_n \to \lambda$ in Thurston's geometric topology.

Now suppose that $(\lambda_n)$ has bounded curvature.
After discarding some leaves of $\lambda_n$ we may assume that $\lambda$ is a maximal limit.
Moreover, every subsequence $(\lambda_{n_k})$ has a further subsequence $(\lambda_{n_{k_\ell}})$ which converges to some maximal limit $\tilde \lambda$ in the $C^{1-}$ flow box topology by Theorem \ref{compactness theorem}.
But convergence in the flow box topology implies convergence in Thurston's topology, so $\tilde \lambda = \lambda$.
Since $(\lambda_{n_k})$ was arbitrary, it follows that $\lambda_n \to \lambda$ in the $C^{1-}$ flow box topology.


%%%%%%%%%%%%%%%%%%%%
\subsection{Application to 1-harmonic functions}
We finally prove Theorem \ref{main thm}.
We assume that $d \in \{2, 3, 4\}$.

Let $u$ be a $1$-harmonic function on $M$.
By Theorem \ref{main thm of old paper}, the level sets of $u$ are closed embedded minimal hypersurfaces in $M$; let
$$Y = \{y \in \RR: \partial \{u > y\} \neq \emptyset\}$$
index the level sets of $u$.

Let $y, z \in Y$. If $y > z$, then $\{u > y\} \subseteq \{u > z\}$, so $\partial \{u > y\}$ lies on one side of $\partial \{u > z\}$.
By the maximum principle for minimal surfaces \cite[Corollary 1.28]{colding2011course}, it follows that either $\partial \{u > y\}$ and $\partial \{u > z\}$ are disjoint, or are equal.
Moreover, $\dif u$ is conormal to $\partial \{u > y\}$, so $\partial \{u > y\}$ has trivial normal bundle.
Therefore by the stable Bernstein theorem, after replacing $M$ with an element of a compact exhaustion $(X_m)$ of $M$, we may assume that for some $A > 0$,
\begin{equation}\label{curvature estimate on 1 harmonic}
	\sup_{y \in Y} \|\Two_{\partial \{u > y\}}\|_{C^0} \leq A.
\end{equation}
We then choose a dense sequence $(y_n)$ in $Y$ and let $N_n = \partial \{u > y_n\}$.
Since $N_n$ is closed, $S_n := \bigcup_{k < n} N_k$ is also closed, so $S_n$ is the support of a minimal lamination $\lambda_n$.
By (\ref{curvature estimate on 1 harmonic}) and Theorem \ref{compactness theorem}, after passing to a subsequence we may find a limit $\lambda$ of $(\lambda_n)$ in the $C^{1-}$ flow box topology.
Diagonalizing against the compact exhaustion $(X_m)$, we can return to the original manifold $M$.

Every $N_n$ is a leaf of $\lambda$, and if $y \in Y$, then $\partial \{u > y\}$ is approximated by some subsequence of $(N_n)$ and so also is a leaf of $\lambda$.
Since this property characterizes leaves of $\lambda$, we conclude that the leaves of $\lambda$ are exactly the level sets of $u$.
It follows that $\bigcup_{y \in Y} \partial \{u > y\}$ is the support of $\lambda$, that $\lambda$ is minimal, and that $\dif u$ is conormal to $\lambda$.
In particular, we obtain an orientation on $\lambda$ from $\dif u$.

We now construct the transverse measure to $\lambda$.
In any oriented laminar coordinates $(k, x) \in K \times J$ for $\lambda$, $\partial_x u = 0$, so $\star |\dif u|$ defines a measure $\mu$ on $K$: given $\alpha < \beta$, let
$$\mu([\alpha, \beta] \cap K) := u(\beta, x) - u(\alpha, x)$$
for any (and hence every, since $\partial_x u = 0$) $x \in J$.
Since $(k, x)$ are oriented laminar coordinates, $u(\cdot, x)$ is nondecreasing, so $\mu$ is a positive measure.

If $(k', x') \in K' \times J$ is a different laminar coordinate system, and the transition map carries $\alpha, \beta$ to $\alpha', \beta'$, then
$$\mu'([\alpha', \beta'] \cap K') := u'(\beta', x') - u(\alpha', x') = u(\beta, x_1) - u(\alpha, x_2)$$
for some $x_1, x_2 \in J$. Since $\partial_x u = 0$,
$$u(\beta, x_1) - u(\alpha, x_2) = u(\beta, x_1) - u(\alpha, x_1) = \mu([\alpha, \beta] \cap K).$$
It follows that $\mu$ is transverse, and by construction $\mu$ lifts to $\star |\dif u|$ in $M$.
Therefore $\dif u = \normal_\lambda |\dif u|$ is the Ruelle-Sullivan current for the measured oriented structure we just imposed on $\lambda$.

For the converse, we assume that we are given a measured oriented minimal lamination $\lambda$, which then has a Ruelle-Sullivan current $T$.
Since $\dif T = 0$, we may assume, possibly after replacing $M$ with its universal cover, that $T$ is exact, say $T = \dif u$, and we just need to show that $u$ is $1$-harmonic.
If this is not true, then we can choose an open set $E \subseteq M$ with $C^\infty$ boundary and a function $v \in BV_\cpt(E)$ such that
$$\int_E \star |\dif u + \dif v| < \int_E \star |\dif u| < \infty.$$
Since $v$ has compact support, there exists a collar neighborhood $F \subseteq E$ of $\partial E$ such that for every $y \in \RR$,
$$\partial \{u > y\} \cap F = \partial^* \{u + v > y\} \cap F.$$
But by Theorem \ref{main thm of old paper}, the level sets $\partial \{u > y\}$ are stable minimal, so it follows that
$$|\partial \{u > y\} \cap E| \leq |\partial^* \{u + v > y\} \cap E|.$$
So by the coarea formula (see \todo{\cite{BackusFLG}} for a proof at this regularity),
\begin{align*}
\int_E \star |\dif u| &= \int_{-\infty}^\infty |\partial \{u > y\} \cap E| \dif y \leq \int_{-\infty}^\infty |\partial^* \{u + v > y\} \cap E| \dif y \\
&= \int_E \star |\dif u + \dif v| < \int_E \star |\dif u|
\end{align*}
which is a contradiction.




%%%%%%%%%%%%%%%%%%%%%%%%%%%%%%%
\section{Transverse cocycles}\label{transverse curves}
In this appendix, we show that convergence in the weak topology of measures as we have stated it (that is, the weak topology on the space of Ruelle-Sullivan currents) is equivalent to the formulation of Thurston \cite[\S8.6]{thurston1979geometry} that is more familiar to geometric topologists.

We shall loosely follow \cite[\S7.2]{daskalopoulos2020transverse}.
Let $\lambda$ be an oriented minimal lamination.
By Proposition \ref{regularity theorem}, $\lambda$ has a global Lipschitz normal vector field $\normal_\lambda$ and is tangentially $C^\infty$.
We shall assume that $\supp \lambda$ is a Lebesgue null set.
This assumption is harmless, because in the application of interest to geometric topologists, $M$ is a closed hyperbolic surface, and then by the Gauss-Bonnet formula it is indeed true that $\supp \lambda$ is null \cite[\S8.5]{thurston1998minimal}.

A curve $\gamma: I \to M$ is said to be \dfn{positively transverse} to $\lambda$ if $\langle \gamma', \normal_\lambda \rangle > 0$ on $\supp \lambda$; by taking pushforwards by the inverse of a laminar flow box, this is equivalent to the condition of \cite[Definition 7.7]{daskalopoulos2020transverse}.
We define negatively transverse curves similarly.
The transverse curve $\gamma$ is \dfn{admissibly transverse} if, in addition, the endpoints of $\gamma$ lie in $M \setminus \supp \lambda$.

% A sum of admissibly transverse curves is known as a \dfn{transverse $1$-chain}.
% We write $C_1(M, \lambda)$ for the group of transverse $1$-chains, modulo transverse homotopies to $\lambda$.
% A \dfn{transverse $1$-cocycle} is a representation $C_1(M, \lambda) \to \RR$ \cite[Definition 7.12]{daskalopoulos2020transverse}.

% If $\mu$ is a transverse measure, then we can define a transverse $1$-cocycle, which we also denote by $\mu$, as follows.
% By subdividing $\gamma \in C_1(M, \lambda)$, we may assume that $\gamma$ is a positively transverse curve in the image $U_\alpha$ of a flow box $F_\alpha$.
% The projection of $(F_\alpha^{-1})_* \gamma$ to $I \subset K_\alpha$ is strictly increasing since $\gamma$ is positively transverse, so $\mu_\alpha$ induces a measure $\tilde \mu$ on $\gamma(I)$. We then define
% \begin{equation}\label{definition of cocycle}
% 	\mu(\gamma) := \int_{\gamma(I)} \langle \gamma', \normal_\lambda \rangle \dif \tilde \mu.
% \end{equation}
% This formula defines a positively transverse cocycle, and every positively transverse cocycle arises in this way; these facts are easy modifications of the discussion in \cite[\S7.2]{daskalopoulos2020transverse}.

If $\gamma: I \to M$ is an admissibly transverse curve, and $\mu$ is a transverse measure, then we define a measure $\gamma^! \mu$, the \dfn{exceptional pullback}\footnote{One can only push forward measures in general.} of $\mu$, on $I$.
First, we may assume that $\gamma$ is an alternating sum of admissibly positively transverse curves, and then by restricting to any such summand we may assume that $\gamma$ is admissibly positively transverse.
Such alternating sums are known as \dfn{good subdivisions} and constructed in \cite[Lemma 7.9]{daskalopoulos2020transverse}.

If $\gamma$ is positively admissibly transverse, we consider all decompositions $\gamma = \sum_i \gamma_i$ where $\gamma_i$ is positively admissibly transverse and has domain $I_i = [t_i, t_{i + 1}] \subseteq I$.
Then, in a neighborhood of $\gamma_i$, $T_\mu$ is exact, say $T_\mu = \dif u$. Then we may set 
$$\gamma^! \mu(I_i) := u(\gamma(t_{i + 1})) - u(\gamma(t_i)).$$
Since $\supp \lambda$ is Lebesgue null, the set of $t$ such that $u(\gamma(t))$ is ill-defined is also Lebesgue null, and $u$ is a nondecreasing function since $\gamma$ is positively admissibly transverse.
So $\gamma^! \mu$ is a well-defined Radon measure on $I$ whose (distributional) Radon-Nikod\'ym derivative is $\dif(\gamma^* u)$ (which itself is well-defined since $u$ is nondecreasing and hence $BV(I)$).
We define $\gamma^! \mu$ for inadmissible transverse curves $\gamma$ by extending $\gamma$ slightly to a transverse curve and then taking intersections over extensions and applying continuity from above.

Taking exceptional pullbacks, we see that we may view $\mu$ as the data of a Radon measure on every transverse curve satisfying a compatibility condition; this is the definition of transverse measure which appears in the definition of measure convergence in \cite[\S8.6]{thurston1979geometry}.


\begin{proposition}\label{characterization of measure convergence}
	Let $(\lambda_n, \mu_n)$ and $(\lambda, \mu)$ be oriented measured minimal laminations. Then $(\lambda_n, \mu_n) \to (\lambda, \mu)$ iff for every positively transverse curve $\gamma$ to $\lambda$ defined in a small neighborhood of $\supp \lambda$:
\begin{enumerate}
\item $\gamma$ is eventually positively transverse to $\lambda_n$, and 
\item $\gamma^! \mu_n \to \gamma^! \mu$ in the weak topology of measures.
\end{enumerate}
\end{proposition}
\begin{proof}
	First assume that $(\lambda_n, \mu_n) \to (\lambda, \mu)$ and $\gamma$ is positively transverse to $\lambda$.
	By Lemma \ref{convergence of normals}, in a neighborhood of $\supp \lambda$, we have $\langle \normal_{\lambda_n}, \gamma' \rangle > 0$ for $n$ large, where $n$ can be chosen uniformly on $M$ since the image of $\gamma$ is compact. So $\gamma$ is eventually positively transverse to $\lambda_n$.

	By working locally we may assume that $T_\mu = \dif u$ and $T_{\mu_n} = \dif u_n$, and by possibly extending $\gamma$ if necessary we may assume that it is admissibly transverse.
	Since $\dif u_n \to \dif u$ in the weak topology of measures, the portmanteau theorem implies that for any curve $\rho$ from $x$ to $y$ such that $u$ is continuous near $x, y$, $u_n(y) - u_n(x) \to u(y) - u(x)$.
	Taking $\rho$ to range over admissible subcurves of $\gamma$, and applying the portmanteau theorem again, we conclude that $\dif(\gamma^* u_n) \to \dif(\gamma^* u)$ and hence $\gamma^! \mu_n \to \gamma^! \mu$.

	Conversely, if $\gamma^! \mu_n \to \gamma^! \mu$ for every positively transverse $\gamma$, then by the existence of flow box coordinates for $\lambda$ (given by Proposition \ref{regularity theorem}), we can actually foliate a neighborhood of any point $x$ of $\supp \lambda$ by admissibly positively transverse curves.
	These curves depend continuously on their intersection point with the leaf of $\supp \lambda$ containing $x$, so the fact that one of them is positively transverse to $\lambda_n$ for $n \geq N/2$ and some even integer $N$ implies that in a neighborhood of $x$, every curve in the foliation is positively transverse to $\lambda_n$ for $n \geq N$.
	Let $J$ be the set of such curves, equipped with the measure $\nu$ obtained by disintegrating the Riemannian measure into arc length measures $s_\gamma$ on each $\gamma$.
	
	If $\varphi$ is a continuous $d-1$-form with support near $x$, we may define the restriction $f_\gamma$ of $\varphi \wedge (\gamma')^\flat$ to $\gamma$.
	By the disintegration theorem,
	$$\int_M T_\mu \wedge \varphi = \int_J \int_I f_\gamma \frac{\dif(\gamma^! \mu)}{\dif s_\gamma} \dif s_\gamma \dif \nu(\gamma) = \int_J \int_I f_\gamma \dif(\gamma^! \mu) \dif \nu(\gamma).$$
	One can show using the dominated convergence theorem and the fact that $\gamma^! \mu_n \to \gamma^! \mu$ weakly that 
	$$\int_M T_\mu \wedge \varphi = \lim_{n \to \infty} \int_J \int_I f_\gamma \dif(\gamma^! \mu_n) \dif \nu(\gamma) = \lim_{n \to \infty} \int_M T_{\mu_n} \wedge \varphi$$
	but since $\varphi$ is arbitrary, it holds that $T_{\mu_n} \to T_\mu$ in the weak topology of measures.
\end{proof}

\chapter{The harmonic infinity-Yang-Mills equation}

\chapter{Applications to computational geometry}

\printbibliography

\end{document}
