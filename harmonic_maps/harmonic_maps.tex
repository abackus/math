\documentclass[reqno,12pt,letterpaper]{amsart}
\RequirePackage{amsmath,amssymb,amsthm,graphicx,mathrsfs,url,slashed}
\RequirePackage[usenames,dvipsnames]{xcolor}
\RequirePackage[colorlinks=true,linkcolor=Red,citecolor=Green]{hyperref}
\RequirePackage{amsxtra}
\usepackage{cancel}
\usepackage{tikz-cd}

\setlength{\textheight}{8.50in} \setlength{\oddsidemargin}{0.00in}
\setlength{\evensidemargin}{0.00in} \setlength{\textwidth}{6.08in}
\setlength{\topmargin}{0.00in} \setlength{\headheight}{0.18in}
\setlength{\marginparwidth}{1.0in}
\setlength{\abovedisplayskip}{0.2in}
\setlength{\belowdisplayskip}{0.2in}
\setlength{\parskip}{0.05in}
\renewcommand{\baselinestretch}{1.10}

\title{Harmonic maps}
\author{Georgios Daskalopoulos}
\date{May 2022}

\newcommand{\NN}{\mathbf{N}}
\newcommand{\ZZ}{\mathbf{Z}}
\newcommand{\QQ}{\mathbf{Q}}
\newcommand{\RR}{\mathbf{R}}
\newcommand{\CC}{\mathbf{C}}
\newcommand{\DD}{\mathbf{D}}
\newcommand{\PP}{\mathbf P}
\newcommand{\MM}{\mathbf M}
\newcommand{\II}{\mathbf I}
\newcommand{\Hyp}{\mathbf H}
\newcommand{\Sph}{\mathbf S}
\newcommand{\Group}{\mathbf G}
\newcommand{\GL}{\mathbf{GL}}
\newcommand{\Orth}{\mathbf{O}}
\newcommand{\SpOrth}{\mathbf{SO}}
\newcommand{\Ball}{\mathbf{B}}

\DeclareMathOperator*{\Expect}{\mathbf E}

\DeclareMathOperator{\avg}{avg}
\DeclareMathOperator{\card}{card}
\DeclareMathOperator{\cent}{center}
\DeclareMathOperator{\ch}{ch}
\DeclareMathOperator{\codim}{codim}
\DeclareMathOperator{\diag}{diag}
\DeclareMathOperator{\diam}{diam}
\DeclareMathOperator{\dom}{dom}
\DeclareMathOperator{\Exc}{Exc}
\newcommand{\ext}{\mathrm{ext}}
\DeclareMathOperator{\Gal}{Gal}
\DeclareMathOperator{\Hom}{Hom}
\DeclareMathOperator{\Iso}{Iso}
\DeclareMathOperator{\Jac}{Jac}
\DeclareMathOperator{\Lip}{Lip}
\DeclareMathOperator{\Met}{Met}
\DeclareMathOperator{\id}{id}
\DeclareMathOperator{\rad}{rad}
\DeclareMathOperator{\rank}{rank}
\DeclareMathOperator{\Rm}{Rm}
\DeclareMathOperator{\Hess}{Hess}
\DeclareMathOperator{\Hol}{Hol}
\DeclareMathOperator{\Radon}{Radon}
\DeclareMathOperator*{\Res}{Res}
\DeclareMathOperator{\sgn}{sgn}
\DeclareMathOperator{\singsupp}{sing~supp}
\DeclareMathOperator{\Spec}{Spec}
\DeclareMathOperator{\supp}{supp}
\DeclareMathOperator{\Tan}{Tan}
\newcommand{\tr}{\operatorname{tr}}

\newcommand{\Mink}{\mathbf m}
\newcommand{\Ric}{\mathrm{Ric}}
\newcommand{\Riem}{\mathrm{Riem}}
\newcommand*\dif{\mathop{}\!\mathrm{d}}
\newcommand*\Dif{\mathop{}\!\mathrm{D}}
\newcommand{\LapQL}{\Delta^{\mathrm{ql}}}

\newcommand{\dbar}{\overline \partial}

\DeclareMathOperator{\atanh}{atanh}
\DeclareMathOperator{\csch}{csch}
\DeclareMathOperator{\sech}{sech}

\DeclareMathOperator{\Div}{div}
\DeclareMathOperator{\Gram}{Gram}
\DeclareMathOperator{\grad}{grad}
\DeclareMathOperator{\dist}{dist}
\DeclareMathOperator{\spn}{span}
\DeclareMathOperator{\Ell}{Ell}
\DeclareMathOperator{\WF}{WF}

\newcommand{\Lagrange}{\mathscr L}
\newcommand{\DirQL}{\mathscr D^{\mathrm{ql}}}
\newcommand{\DirL}{\mathscr D}

\newcommand{\Hilb}{\mathcal H}
\newcommand{\Homology}{\mathrm H}
\newcommand{\normal}{\mathbf n}
\newcommand{\radial}{\mathbf r}
\newcommand{\evect}{\mathbf e}
\newcommand{\vol}{\mathrm{vol}}

\newcommand{\pic}{\vspace{30mm}}
\newcommand{\dfn}[1]{\emph{#1}\index{#1}}

\renewcommand{\Re}{\operatorname{Re}}
\renewcommand{\Im}{\operatorname{Im}}

\newcommand{\loc}{\mathrm{loc}}
\newcommand{\cpt}{\mathrm{cpt}}

\def\Japan#1{\left \langle #1 \right \rangle}
\newcommand{\parl}{\left(}
\newcommand{\parr}{\right)}
\newcommand{\bral}{\left\langle}
\newcommand{\brar}{\right\rangle}

\newtheorem{theorem}{Theorem}[section]
\newtheorem{badtheorem}[theorem]{``Theorem"}
\newtheorem{prop}[theorem]{Proposition}
\newtheorem{lemma}[theorem]{Lemma}
\newtheorem{sublemma}[theorem]{Sublemma}
\newtheorem{claim}[theorem]{Claim}
\newtheorem{proposition}[theorem]{Proposition}
\newtheorem{corollary}[theorem]{Corollary}
\newtheorem{conjecture}[theorem]{Conjecture}
\newtheorem{axiom}[theorem]{Axiom}
\newtheorem{assumption}[theorem]{Assumption}

\theoremstyle{definition}
\newtheorem{definition}[theorem]{Definition}
\newtheorem{remark}[theorem]{Remark}
\newtheorem{example}[theorem]{Example}
\newtheorem{notation}[theorem]{Notation}

\newtheorem{exercise}[theorem]{Discussion topic}
\newtheorem{homework}[theorem]{Homework}
\newtheorem{problem}[theorem]{Problem}

\makeatletter
\newcommand{\proofpart}[2]{%
  \par
  \addvspace{\medskipamount}%
  \noindent\emph{Part #1: #2.}
}
\makeatother

\newtheorem{ack}{Acknowledgements}

\numberwithin{equation}{section}


% Mean
\def\Xint#1{\mathchoice
{\XXint\displaystyle\textstyle{#1}}%
{\XXint\textstyle\scriptstyle{#1}}%
{\XXint\scriptstyle\scriptscriptstyle{#1}}%
{\XXint\scriptscriptstyle\scriptscriptstyle{#1}}%
\!\int}
\def\XXint#1#2#3{{\setbox0=\hbox{$#1{#2#3}{\int}$ }
\vcenter{\hbox{$#2#3$ }}\kern-.6\wd0}}
\def\ddashint{\Xint=}
\def\dashint{\Xint-}

\usepackage[backend=bibtex,style=alphabetic]{biblatex}
\renewcommand*{\bibfont}{\normalfont\footnotesize}
\addbibresource{topics.bib}
\renewbibmacro{in:}{}
\DeclareFieldFormat{pages}{#1}


\begin{document}

\maketitle

%%%%%%%%%%%%%%%%%%%%%%%%%%%%%%%%%%%%%%%%%%%%%%%%%%%%%%%

% \tableofcontents

\section{Harmonic maps}
Let $f: M \to N$ be a map, let $(x_\alpha)$ be coordinates on $M$, $(y_i)$ coordinates on $N$, $(\gamma_{\alpha\beta})$ a metric on $M$, and $(g_{ij})$ a metric on $N$.
We can view $\dif f$ as a section 
$$\dif f \in \Omega^1(f^* TN) := C^\infty(T^* M \otimes f^* TN)$$
and the codifferential induced by the Levi-Civita connection on $f^* TN$ by
$$\nabla^*: \Omega^1(f^* TN) \to \Omega^0(f^* TN) := C^\infty(f^* TN).$$

\begin{definition}
Let 
$$\tau(f) := -\nabla^* \dif f.$$
Then $f$ is a \dfn{harmonic map} if it satisfies the \dfn{harmonic maps equation} $\tau(f) = 0$.
\end{definition}

\subsection{Expression in local coordinates}
We write the harmonic maps equation in local coordinates by setting 
$$\frac{\partial}{\partial f_i} := \frac{\partial}{\partial y_i} \circ f \in C^\infty_\loc(f^* TN)$$
and defining
$$\omega\left(\frac{\partial}{\partial y_i}\right) := \Gamma_{jk}^i \dif y_k \otimes \frac{\partial}{\partial y_j},$$
where $\Gamma_{jk}^i$ are Christoffel symbols on $N$, and $\tilde \omega := f^* \omega$.
Then 
\begin{align*}
\nabla^* \dif f &= -\star \nabla \star \dif f \\
&= -\star (\dif + \tilde \omega)\parl\star \dif f_i \otimes \frac{\partial}{\partial f_i}\parr \\
&= - \star \dif \star \dif f_i \otimes \frac{\partial}{\partial f_i} - (-1)^{m - 1} \star\parl\star \dif f_i \otimes \tilde \omega\parl\frac{\partial}{\partial f_i}\parr\parr\\
&= \Delta f_i \otimes \frac{\partial}{\partial f_i} - (-1)^{m - 1} \star \parl \frac{\partial f_i}{\partial x_\alpha} \star \dif x_\alpha \wedge f^*(\Gamma_{ijk}^k \dif y_k) \frac{\partial}{\partial f_j}\parr.
\end{align*}
Moreover,
\begin{align*}
\frac{\partial f_i}{\partial x_\alpha} \star \dif x_\alpha \wedge f^*(\Gamma_{ijk}^k \dif y_k) \frac{\partial}{\partial f_j}
&= \frac{\partial f_i}{\partial x_\alpha} \star \dif x_\alpha \wedge \Gamma_{ik}^j \circ f \cdot \frac{\partial f_k}{\partial x_\beta} \dif x_\beta \frac{\partial}{\partial f_j} \\
&= \frac{\partial f_i}{\partial x_\alpha} \frac{\partial f_k}{\partial x_\beta} \Gamma_{ik}^j \circ f \frac{\partial}{\partial f_j} \star (\star \dif x_\alpha \wedge \dif x_\beta).
\end{align*}
In conclusion,
\begin{align*}
\nabla^* \dif f &= \Delta f_i \otimes \frac{\partial}{\partial f_i} - (-1)^{m - 1} \parl \frac{\partial f_i}{\partial x_\alpha} \frac{\partial f_k}{\partial x_\beta} \Gamma_{ik}^j \circ f \frac{\partial}{\partial f_j} \star (\star \dif x_\alpha \wedge \dif x_\beta) \parr \\
&= \Delta f_k \otimes \frac{\partial}{\partial f_k} - \gamma^{\alpha \beta} \frac{\partial f_i}{\partial x_\alpha} \frac{\partial f_j}{\partial x_\beta} \Gamma_{ij}^k \circ f \frac{\partial}{\partial f_k}.
\end{align*}
Thus the harmonic maps equation is 
\begin{equation}
\Delta^- f_k + \gamma^{\alpha \beta} \frac{\partial f_i}{\partial x_\alpha} \frac{\partial f_j}{\partial x_\beta} \Gamma_{ij}^k \circ f = 0.
\end{equation}
Expanding out the definition of $\Delta^-$, this in fact reads
\begin{equation}
\gamma^{\alpha \beta} \parl \frac{\partial^2 f_k}{\partial x_\alpha \partial x_\beta} - {}^M \Gamma_{\alpha \beta}^\gamma \frac{\partial f_k}{\partial x_\gamma} + {}^N \Gamma_{ij}^k \circ f \frac{\partial f_i}{\partial x^\alpha} \frac{\partial f_j}{\partial x^\beta} \parr = 0.
\end{equation}

\subsection{The energy of maps}
Recall 
$$\dif f \parl \frac{\partial}{\partial x_\alpha} \parr \bigg|_p = \frac{\partial f_i}{\partial x_\alpha} \frac{\partial}{\partial y_i} \bigg|_{f(p)}$$
hence 
$$\dif f_p = \frac{\partial f_i}{\partial x_\alpha} \otimes \frac{\partial}{\partial y_i} \bigg|_{f(p)}$$
or using our notation $\partial/\partial f_i = \partial/\partial y_i \circ f$ from the previous section,
$$\dif f = \frac{\partial f_i}{\partial x_\alpha} \dif x_\alpha \otimes \frac{\partial}{\partial f_i}.$$

\begin{definition}
Set 
$$e(f) := \frac{1}{2} ||\dif f||^2 = \frac{1}{2} \frac{\partial f_i}{\partial x_\alpha} \frac{\partial f_j}{\partial x_\beta} \gamma^{\alpha \beta} g_{ij} \circ f.$$
The \dfn{energy} of $f$ is 
$$E(f) := \int_M |e(f)|^2 \star 1 = \frac{1}{2} \int_M \gamma^{\alpha \beta}(x) g_{ij}(f(x)) \frac{\partial f_i}{\partial x_\alpha} \frac{\partial f_j}{\partial x_\beta} \sqrt{\gamma(x)} \dif x_1 \cdots \dif x_n.$$
\end{definition}

\begin{lemma}
Let $(f_t)$ be a one-parameter family of $C^\infty$ maps.
Then for the induced connection 
$$\nabla \frac{\partial f}{\partial t} = \nabla_{\partial/\partial t} \dif f$$
where $f = f_t$, and 
$$\frac{\partial f}{\partial t} = \frac{\partial f_i}{\partial t} \frac{\partial}{\partial f} \in \Gamma(f^* TN).$$
\end{lemma}
\begin{proof}
First,
\begin{align*}
\nabla_{\partial/\partial t} \dif f &= \frac{\partial^2 f_i}{\partial t \partial x_\alpha} \dif x_\alpha \otimes \frac{\partial}{\partial f_i} + \frac{\partial f_i}{\partial x_\alpha} \dif x_\alpha \otimes \nabla_{\partial/\partial t} \frac{\partial}{\partial f_i}.
\end{align*}
We compute 
\begin{align*}
    \frac{\partial f_i}{\partial x_\alpha} \dif x_\alpha \otimes \nabla_{\partial/\partial t} \frac{\partial}{\partial f_i} &= \frac{\partial f_i}{\partial x_\alpha} \dif x_\alpha \otimes \parl \nabla_{f_*(\partial/\partial t)} \frac{\partial}{\partial y_i} \parr \circ f \\
    &= \frac{\partial f_i}{\partial x_\alpha} \dif x_\alpha \otimes \frac{\partial f_j}{\partial t} \nabla_{\partial/\partial y_j} \parl \frac{\partial}{\partial y_i} \parr \circ f \\
    &= \frac{\partial f_i}{\partial x_\alpha} \dif x_\alpha \otimes \frac{\partial f_j}{\partial t} \nabla_{\partial/\partial y_i} \parl \frac{\partial}{\partial y_j} \parr \circ f \\
    &= \frac{\partial f_j}{\partial t} \dif x_\alpha \otimes \nabla_{\frac{\partial f_i}{\partial x_\alpha} \frac{\partial}{\partial y_i}} \parl \frac{\partial}{\partial y_j} \parr \circ f.
\end{align*}
Therefore 
\begin{align*}
\nabla_{\partial/\partial t} \dif f &= \frac{\partial^2 f_i}{\partial t \partial x_\alpha} \dif x_\alpha \otimes \frac{\partial}{\partial f_i} + \frac{\partial f_j}{\partial t} \dif x_\alpha \otimes \nabla_{\frac{\partial f_i}{\partial x_\alpha} \frac{\partial}{\partial y_i}} \parl \frac{\partial}{\partial y_j} \parr \circ f \\
&= \frac{\partial}{\partial x_\alpha} \parl \frac{\partial f_j}{\partial t} \parr \frac{\partial}{\partial f_j} \otimes \dif x_\alpha + \frac{\partial f_j}{\partial t} \nabla_{f_*(\partial/\partial x_\alpha)} \parl \frac{\partial}{\partial y_j} \parr \circ f \dif x_\alpha \\
&= \frac{\partial}{\partial x_\alpha} \parl \frac{\partial f_j}{\partial t} \parr \frac{\partial}{\partial f_j} \otimes \dif x_\alpha + \frac{\partial f_j}{\partial t} \nabla_{\partial/\partial x_\alpha} \parl \frac{\partial}{\partial f_j} \parr \dif x_\alpha \\
&= \nabla_{\partial/\partial x_\alpha} \parl \frac{\partial f}{\partial t} \parr \dif x_\alpha = \nabla \frac{\partial f}{\partial t}. \qedhere
\end{align*}
\end{proof}

\begin{corollary}
One has 
$$\frac{\dif}{\dif t} E(f_t) = \int_M \bral \nabla \frac{\partial f}{\partial t}, \dif f\brar \star 1 = \int_M \bral \frac{\partial f}{\partial t}, -\tau f\brar \star 1.$$
\end{corollary}
\begin{proof}
We compute 
\begin{align*}
\frac{\dif}{\dif t} E(f_t) &= \frac{1}{2} \int_M \frac{\dif}{\dif t} \bral \dif f, \dif f \brar \star 1\\
&= \int_M \bral \nabla_{\partial/\partial t} \dif f, \dif f \brar \star 1\\
&= \int_M \bral \nabla \frac{\partial f}{\partial t}, \dif f \brar \star 1\\
&= \int_M \bral \frac{\partial f}{\partial t}, \nabla^* \dif f \brar \star 1\\
&= \int_M \bral \frac{\partial f}{\partial t}, -\tau f \brar \star 1. \qedhere
\end{align*}
\end{proof}

\begin{corollary}
Critical points of the energy functional are the harmonic maps.
\end{corollary}
\begin{proof}
Let $f_t$ be a family of maps with 
$$\frac{\dif}{\dif t}\bigg|_{t = 0} f_t = \psi \in C^\infty(f^* TM).$$
Then 
$$\frac{\dif}{\dif t}\bigg|_{t = 0} E(f_t) = \int_M \bral \psi, -\tau f \brar \star 1 = 0$$
which holds for every $\psi \in C^\infty(f^* TM)$ iff $\tau f = 0$.
\end{proof}

\subsection{The Dirichlet and Neumann problems}
We extend the Riemann curvature 
$$R^N: TN \times TN \times TN \to TN$$
to a curvature
$$R^N: f^* TN \times f^* TN \times f^* TN \to f^* TN$$
in the natural way.

\begin{lemma}
Let $f_t: M \to N$, and let $V$ be a vector field along $f_t$. Then 
$$\nabla_{\partial/\partial t} \nabla V = \nabla \nabla_{\partial/\partial t} V - R^N \parl \dif f, \frac{\partial f}{\partial t} \parr V.$$
\end{lemma}
\begin{proof}
Set $V = \partial/\partial f_i$, then 
\begin{align*}
\nabla_{\partial/\partial t} \parl \nabla \frac{\partial}{\partial f_i} \parr &= \nabla_{\partial/\partial t} \parl \nabla_{f_*(\partial/\partial x_\alpha)} \frac{\partial}{\partial y_i} \parr \circ f \otimes \dif x_\alpha \\
&= \nabla_{\partial/\partial t} \parl \nabla_{\frac{\partial f_j}{\partial x_\alpha} \frac{\partial}{\partial y_j}} \frac{\partial}{\partial y_i} \parr \circ f \otimes \dif x_\alpha \\
&= \nabla_{\partial/\partial t} \parl \frac{\partial f_j}{\partial x_\alpha} \nabla_{\partial/\partial y_j} \frac{\partial}{\partial y_j} \parr \circ f \otimes \dif x_\alpha \\
&= \frac{\partial^2}{\partial x_\alpha \partial t} \parl \nabla_{\partial/\partial y_j} \frac{\partial}{\partial y_i} \parr \circ f \otimes \dif x_\alpha + \frac{\partial f_j}{\partial x_\alpha} \nabla_{\partial/\partial t} \parl \nabla_{\partial/\partial y_j} \frac{\partial}{\partial y_i} \parr \circ f \otimes \dif x_\alpha.
\end{align*}
The second term is 
\begin{align*}
\frac{\partial f_j}{\partial x_\alpha} \nabla_{\partial/\partial t} \parl \nabla_{\partial/\partial y_j} \frac{\partial}{\partial y_i} \parr \circ f \otimes \dif x_\alpha &= \frac{\partial f_j}{\partial x_\alpha} \parl \nabla_{f_*(\partial/\partial t)} \nabla_{\partial/\partial y_j} \frac{\partial}{\partial y_i} \parr \circ f \otimes \dif x_\alpha \\
&= \frac{\partial f_j}{\partial x_\alpha} \frac{\partial f_p}{\partial t} \parl \nabla_{\partial/\partial y_p} \nabla_{\partial/\partial y_j} \frac{\partial}{\partial y_i} \parr \circ f \otimes \dif x_\alpha \\
&= \frac{\partial f_j}{\partial x_\alpha} \frac{\partial f_p}{\partial t} \parl \nabla_{\partial/\partial y_j} \nabla_{\partial/\partial y_p} \frac{\partial}{\partial y_i} \parr \circ f \otimes \dif x_\alpha \\
&\qquad - \frac{\partial f_j}{\partial x_\alpha} \frac{\partial f_p}{\partial t} R^N \parl \frac{\partial}{\partial y_j}, \frac{\partial}{\partial y_p} \parr \frac{\partial}{\partial y_i} \circ f \otimes \dif x_\alpha.
\end{align*}
Undoing the above computation we see that 
$$\frac{\partial^2}{\partial x_\alpha \partial t} \parl \nabla_{\partial/\partial y_j} \frac{\partial}{\partial y_i} \parr \circ f \otimes \dif x_\alpha + \frac{\partial f_j}{\partial x_\alpha} \frac{\partial f_p}{\partial t} \parl \nabla_{\partial/\partial y_j} \nabla_{\partial/\partial y_p} \frac{\partial}{\partial y_i} \parr \circ f \otimes \dif x_\alpha = \nabla \nabla_{\partial/\partial t} V.$$
Moreover,
\begin{align*}
\frac{\partial f_j}{\partial x_\alpha} \frac{\partial f_p}{\partial t} R^N \parl \frac{\partial}{\partial y_j}, \frac{\partial}{\partial y_p} \parr \frac{\partial}{\partial y_i} \circ f \otimes \dif x_\alpha &= R^N \parl \dif f, \frac{\partial f}{\partial t} \parr V. \qedhere
\end{align*}
\end{proof}

\begin{corollary}
One has 
$$\frac{\dif^2}{\dif t^2} E(f_t) = ||\nabla \frac{\partial f}{\partial t}||^2 - \int_M \bral R^N \parl \dif f, \frac{\partial f}{\partial t} \parr \frac{\partial f}{\partial t}, \dif f \brar \star 1 + \int_M \bral \nabla_{\partial/\partial t} \frac{\partial f}{\partial t}, \tau f \brar \star 1.$$
\end{corollary}
\begin{proof}
We compute 
\begin{align*}
\frac{\dif^2}{\dif t^2} E(f_t) &= \int_M \frac{\dif}{\dif t} \bral \nabla \frac{\partial f}{\partial t}, \dif f \brar \star 1 \\
&= \int_M \parl \bral \nabla_{\partial/\partial t} \nabla \frac{\partial f}{\partial t}, \dif f \brar + \bral \nabla \frac{\partial f}{\partial t}, \nabla_{\partial/\partial t} \dif f \brar \parr \star 1 \\
&= \int_M \parl \bral \nabla \nabla_{\partial/\partial t} \frac{\partial f}{\partial t}, \dif f \brar - \bral R^N \parl \dif f, \frac{\partial f}{\partial t} \parr \frac{\partial f}{\partial t}, \dif f \brar + ||\nabla \frac{\partial f}{\partial t}||^2 \parr \star 1. \qedhere
\end{align*}
\end{proof}

\begin{corollary}
If $N$ has $\leq 0$ sectional curvature and $f_t$ is harmonic then $E(f_t)$ is convex.
\end{corollary}

\begin{corollary}
Let $u, \phi: M \to N$ be homotopic with $u|\partial M = \phi|\partial M$.
If $N$ has $\leq 0$ sectional curvature and $u$ is harmonic, then
$$E(u) \leq E(\phi).$$
\end{corollary}
\begin{proof}
Let $f_t$ be a geodesic homotopy between $u, \phi$, thus $f_0 = u$, $f_1 = \phi$. Then $E(t) = E(f_t)$ is convex, and $E'(0) = 0$. So $E(1) \geq E(0)$, hence $E(\phi) \geq E(u)$.
\end{proof}

\begin{corollary}
Let $f_0, f_1$ be homotopic harmonic maps with $f_0|\partial M = f_1|\partial M$.
If $N$ has $\leq 0$ sectional curvature, then:
\begin{enumerate}
\item If $\partial M$ is nonempty, then $f_0 = f_1$.
\item If $\partial M$ is empty and $N$ has $< 0$ sectional curvature, then either $f_0 = f_1$ or the rank of $f_0$ is $\leq 1$ and $N$ has $< 0$ sectional curvature.
\end{enumerate}
\end{corollary}
\begin{proof}
Let $F$ be a geodesic homotopy between $f_0, f_1$, $E(t) = F(\cdot, t)$. Then $E$ is convex and $E'(0) = E'(1) = 0$. So $E' = 0$, so $E'' = 0$, so 
$$\nabla \frac{\partial F}{\partial t} = 0$$
and 
$$\bral R^N \parl \dif f, \frac{\partial f}{\partial t} \parr \frac{\partial f}{\partial t}, \dif f \parr = 0.$$
Then 
$$\frac{\partial}{\partial x_\alpha} ||\frac{\partial F}{\partial t}||^2 = 2 \bral \nabla_{\partial/\partial x_\alpha} \frac{\partial F}{\partial t}, \frac{\partial F}{\partial t} \brar = 0$$
which implies that $||\partial F/\partial t||$ is constant.
But $\partial F/\partial t = 0$ on $\partial M$, so $\partial F/\partial t = 0$ everywhere if $\partial M$ is nonempty and hence $f_0 = f_1$.

If $\partial M$ is empty, then if $||\partial F/\partial t|| = 0$, then $f_0 = f_1$.
Otherwise, $\partial F/\partial t \neq 0$ for every $x, t$. So $\dif f$ is parallel to $\partial F/\partial t$, which implies that the image of $\dif f$ has dimension $\leq 1$.
\end{proof}

\subsection{Examples of harmonic maps}
First suppose $M = \Sph^1$.
Then 
$$E(f) = \frac{1}{2} \int_0^{2\pi} |\dot f(t)| \dif t$$
and the critical points of $E(f)$ are geodesics.
We can also see this from the harmonic maps equation.
Since $\Sph^1$ is $1$-dimensional we can take $\gamma^{\alpha \beta} = \delta^{\alpha \beta}$.
Then 
$$\frac{\partial^2 f_k}{\partial t^2} + \Gamma_{ij}^k \frac{\partial f_i}{\partial x_\alpha} \frac{\partial f_j}{\partial x_\beta} = 0.$$

Another example are when $M, N$ are K\"ahler manifolds, and $f$ is a holomorphic map. Then $f$ is harmonic.

\section{Properties of harmonic maps}
Recall that 
$$\Delta^- f = - \dif^* \dif f = \Div \grad f = \tr(x \mapsto \nabla_x (\dif f)^v)$$
where 
$$v: T^*M \to TM.$$
In coordinates we have 
$$\Delta^- f = \frac{1}{\sqrt \gamma} \frac{\partial}{\partial x_\beta} \parl \sqrt \gamma \gamma^{\alpha \beta} \frac{\partial f}{\partial x_\alpha} \parr = g^{\alpha \beta} \frac{\partial^2 f}{\partial x_\alpha \partial x_\beta} - \Gamma_{\alpha \beta}^\gamma \gamma^{\alpha \beta} \frac{\partial f}{\partial x_\gamma}.$$
As for $\tau$,
$$\tau(f) = -\nabla^* \dif f = \tr(x \mapsto \nabla_x (\dif f)^v)$$
where we extended
$$v: T^* M \otimes f^* TN \to TM \otimes f^* TN.$$
In coordinates,
$$\tau(f)_k = \nabla^- f_k + \Gamma_{ij}^k \frac{\partial f_i}{\partial x_\alpha} \frac{\partial f_j}{\partial x_\beta} \gamma^{\alpha \beta}.$$

\begin{corollary}
If $(e_\alpha)$ is an orthonormal basis of $TM$ then 
$$\tau(f) = \nabla_{e_\alpha} (\dif f \cdot e_\alpha).$$
\end{corollary}
\begin{proof}
We have 
$$\dif f = (\dif f \cdot e_\alpha) e_\alpha^*, \qquad \dif f^v = (\dif f \cdot e_\alpha) e_\alpha.$$
Since 
$$\nabla_{e_\beta}(\dif f(e_\alpha) \cdot e_\alpha) = \nabla_{e_\beta} (\dif f \cdot e_\alpha) e_\alpha + \dif f (e_\alpha) \nabla_{e_\beta} e_\alpha,$$
\begin{align*}
\tau(f) &= \bral \nabla_{e_\beta}(\dif f(e_\alpha)) e_\alpha, e_\beta \brar + \dif f(e_\alpha) \bral \nabla_{e_\beta} e_\alpha, e_\beta \brar \\
&= \nabla_{e_\alpha}(\dif f(e_\alpha)) + \dif f(e_\alpha) \bral \nabla_{e_\alpha} e_\beta, e_\beta \brar \\
&= \nabla_{e_\alpha}(\dif f \cdot e_\alpha). \qedhere 
\end{align*}
\end{proof}

\begin{lemma}
Let $u \in C^2(M, N)$ and $h \in C^2(N, \RR)$.
Let $\Dif^2h: TN \times TN \to RR$ be the Hessian of $h$, thus 
$$\Dif^2 h(X, Y) = \bral \nabla_X \grad h, Y\brar,$$
and $(e_\alpha)$ an orthonormal frame of $M$ with $u_{e_\alpha} = \dif u(e_\alpha)$. Then
$$\Delta^-(h \circ u) = \Dif^2 h(u_{e_\alpha}, u_{e_\alpha}) + \bral \grad h \circ u, \tau(u) \brar.$$
\end{lemma}
\begin{proof}
First,
\begin{align*}
\grad(h \circ u) &= \bral \grad(h \circ u), e_\alpha \brar e_\alpha = \dif(h \circ u)(e_\alpha) e_\alpha \\
&= \dif h \circ u u_{e\alpha} e_\alpha = \bral \grad h \circ u, u_{e_\alpha} \brar e_\alpha.
\end{align*}
Therefore 
\begin{align*}
\Delta^- (h \circ u) &= \Div \grad (h \circ u) \\
&= \tr(e_\beta \mapsto \nabla_{e_\beta} \bral \grad h \circ u, u_{e_\alpha} \brar e_\alpha)\\
&= \tr(e_\beta \mapsto \bral \nabla_{e_\beta} \grad(h \circ u), u_{e_\alpha} \brar e_\alpha + \bral \grad(h \circ u), \nabla_{e_\beta} u_{e_\beta} \brar e_\alpha \\
&\qquad + \bral \grad(h \circ u), u_{e_\alpha} \brar \nabla_{e_\beta} e_\alpha) \\
&= \Dif^2 h(u_{e_\beta}, u_{e_\alpha}) \bral e_\beta, e_\alpha \brar + \bral \grad(h \circ u), \nabla_{e_\beta} u_\alpha \brar \bral e_\beta, e_\alpha \brar \\
&\qquad + \bral \grad (h \circ u), u_{e_\alpha} \brar \bral \nabla_{e_\beta} e_\alpha, e_\alpha \brar \\
&= \Dif^2 h(u_{e_\alpha}, u_{e_\alpha}) + \bral \grad(h \circ u), \nabla_{e_\alpha} u_{e_\alpha} \brar + \frac{1}{2} e_\beta \bral e_\alpha, e_\alpha \brar \bral \grad(h \circ u), u_{e_\alpha} \brar \\
&= \Dif^2 h(u_{e_\alpha}, u_{e_\alpha}) + \bral \grad(h \circ u), \tau(u) \brar + 0. \qedhere
\end{align*}
\end{proof}

\begin{corollary}
If $u$ is harmonic and $h$ is convex then $h \circ u$ is subharmonic.
\end{corollary}
\begin{proof}
One has 
\begin{align*}
\Delta^- (h \circ u) &= \Dif^2 h(u_{e_\alpha}, u_{e_\alpha}) \geq 0. \qedhere
\end{align*}
\end{proof}

\begin{theorem}[Weitzenb\"ock formula]
Let $f: M \to N$ be a harmonic map and $(e_\alpha)$ an orthonormal frame for $TM$. Then 
\begin{align*}
\Delta^- e(f) &= |\nabla \dif f|^2 + \frac{1}{2} \bral \dif f \Ric^M(e_\alpha), \dif f e_\alpha \brar \\
&\qquad - \frac{1}{2} \bral R^N(\dif f e_\alpha, \dif f e_\beta) \dif f e_\beta, \dif f e_\alpha \brar.
\end{align*}
\end{theorem}
\begin{proof}
Since $f$ is harmonic,
$$\gamma^{\alpha \beta} f_{i/\alpha\beta} - \gamma^{\alpha\beta} {}^M \Gamma_{\alpha \beta}^\eta f_{i/\eta} + \gamma^{\alpha\beta} {}^N \Gamma_{k\ell}^i \circ f f_{k/\alpha} f_{\ell/\beta}.$$
Let $x \in M$, $f(x) = y$, and normal coordinates around them. Differentiating,
\begin{align*}
f_{i/\alpha \alpha \varepsilon} &= {}^M \Gamma_{\alpha\alpha/\varepsilon}^\eta f_{i/\eta} - {}^N \Gamma_{k\ell/m}^i f_{m/\varepsilon} f_{k\alpha} f_{\ell/\alpha} \\
&= \frac{1}{2} (\gamma_{\alpha \eta/\alpha \varepsilon} + \gamma_{\alpha \eta/\alpha \varepsilon} - \gamma_{\alpha \alpha/\eta \varepsilon}) f_{i/\eta} \\
&\qquad - \frac{1}{2} (g_{ki/\ell m} + g_{\ell i/km} - g_{k\ell/im}) f_{m/\varepsilon} f_{k/\alpha} f_{\ell/\alpha}.
\end{align*}
So,
\begin{align*}
\Delta^- \parl \frac{1}{2} \gamma^{\alpha \beta} g_{ij} \circ f f_{i/\alpha} f_{j/\beta} \parr &= \frac{1}{\sqrt \gamma} \frac{\partial}{\partial x_\sigma} \parl \sqrt \gamma \gamma^{\sigma \tau} \frac{\partial}{\partial \tau} \parl \frac{1}{2} \gamma^{\alpha \beta} g_{ij} \circ f f_{i/\alpha} f_{j/\beta} \parr \parr \\
&= f_{i/\alpha \sigma} f_{i/\alpha \sigma} - \frac{1}{2} (\gamma_{\alpha \beta/\sigma \sigma} + \gamma_{\sigma\sigma/\alpha \beta} - \gamma_{\sigma\alpha/\sigma \beta} - \gamma_{\sigma \alpha/\sigma \beta}) f_{i/\alpha} f_{i/\beta} \\
&\qquad + \frac{1}{2} (g_{ij/k\ell} + g_{k\ell/ji} - g_{ik/j\ell} - g_{j\ell/ik}) f_\alpha^i f_\alpha^j f_\sigma^k f_\sigma^\ell \\
&= f_{i/\alpha \sigma} f_{i/\alpha \sigma} + \frac{1}{2} \Ric^M_{\alpha \beta} f_{i/\alpha} f_{j/\beta} - \frac{1}{2} R^N_{ikj\ell} f_{i/\alpha} f_{j/\alpha} f_{k/\sigma} f_{\ell/\sigma}. \qedhere
\end{align*}
\end{proof}

In particular, if
$$\frac{\partial f}{\partial t} = \tau(f),$$
then 
\begin{align*}
\Delta^- e(f) - \frac{\partial e(t)}{\partial t} &= |\nabla \dif f|^2 + \frac{1}{2} \bral \dif f \Ric^M(e_\alpha), \dif f e_\alpha \brar \\
&\qquad - \frac{1}{2} \bral R^N (\dif f e_\alpha, \dif f e_\beta) \dif f e_\beta, \dif f e_\alpha \brar.
\end{align*}

For a harmonic map, we of course have 
\begin{equation}\label{schauder estimate prep}
\Delta^- f_i = -\gamma^{\alpha \beta} \Gamma_{jk}^i \frac{\partial f_j}{\partial x_\alpha} \frac{\partial f_k}{\partial x_\beta}.
\end{equation}

\begin{theorem}
If $f: M \to N$ is harmonic, and $N$ has $\leq 0$ sectional curvature, then 
$$|f|_{C^{2 + \alpha}} \leq c$$
where $c > 0$ depends on $E(f)$ and the geometries of $M, N$.
\end{theorem}
\begin{proof}
Since $\Delta^- e(f) \geq -C e(f)$, the maximum principle gives 
$$\sup_M e(f) \leq \int_M \star e(f) = E(f).$$
Now the right-hand side of (\ref{schauder estimate prep}) is $C^0$-bounded. So by elliptic regularity, $f_i$ is $C^{1 + \alpha}$-bounded.
But then the right-hand side of (\ref{schauder estimate prep}) is $C^\alpha$-bounded, so $f_i$ is $C^{2 + \alpha}$-bounded.
\end{proof}

\begin{corollary}
If $f: M \to N$ is harmonic, and $N$ has $\leq 0$ sectional curvature, then $f \in C^\infty(M, N)$.
\end{corollary}
\begin{proof}
Keep bootstrapping with (\ref{schauder estimate prep}).
\end{proof}

\printbibliography

\end{document}
