\documentclass[12pt]{report}
\usepackage[utf8]{inputenc}
\usepackage[margin=1in]{geometry}
\usepackage{amsmath,amsthm,amssymb}
\usepackage{mathrsfs}

\usepackage{enumitem}
%\usepackage[shortlabels]{enumerate}
\usepackage{tikz-cd}
\usepackage{mathtools}
\usepackage{amsfonts}
\usepackage{amscd}
\usepackage{makeidx}
\usepackage{enumitem}
\title{Analysis notes}
\author{Aidan Backus}
\date{December 2019}


\newcommand{\NN}{\mathbb{N}}
\newcommand{\ZZ}{\mathbb{Z}}
\newcommand{\QQ}{\mathbb{Q}}
\newcommand{\RR}{\mathbb{R}}
\newcommand{\CC}{\mathbb{C}}
\newcommand{\PP}{\mathbb{P}}
\newcommand{\DD}{\mathbb{D}}

\newcommand{\AAA}{\mathcal A}
\newcommand{\BB}{\mathcal B}
\newcommand{\HH}{\mathcal H}

\newcommand{\CVect}{\mathbf{Vect}_\CC}
\newcommand{\Grp}{\mathbf{Grp}}
\newcommand{\Open}{\mathbf{Open}}
\newcommand{\Set}{\mathbf{Set}}

\newcommand{\Aut}{\operatorname{Aut}}
\newcommand{\Cantor}{\mathcal{C}}
\newcommand{\D}{\mathcal{D}}
\newcommand{\card}{\operatorname{card}}
\newcommand{\ch}{\operatorname{ch}}
\newcommand{\diam}{\operatorname{diam}}
\newcommand{\End}{\operatorname{End}}
\DeclareMathOperator*{\esssup}{ess\,sup}
\newcommand{\FF}{\mathcal{F}}
\newcommand{\GL}{\operatorname{GL}}
\newcommand{\Hom}{\operatorname{Hom}}
\newcommand{\id}{\operatorname{id}}
\newcommand{\Ind}{\operatorname{Ind}}
\DeclareMathOperator*{\Inv}{Inv}
\newcommand{\interior}{\operatorname{int}}
\newcommand{\lcm}{\operatorname{lcm}}
\newcommand{\Lie}{\operatorname{Lie}}
\newcommand{\Lip}{\operatorname{Lip}}
\newcommand{\MM}{\mathcal M}
\newcommand{\OO}{\mathcal{O}}
\newcommand{\PGL}{\operatorname{PGL}}
\newcommand{\pic}{\vspace{30mm}}
\newcommand{\Prim}{\operatorname{Prim}}
\newcommand{\pset}{\mathcal{P}}
\newcommand{\Ric}{\operatorname{Ric}}
\newcommand{\Rep}{\operatorname{Rep}}
\newcommand{\Res}{\operatorname{Res}}
\newcommand{\Riem}{\mathcal{R}}
\newcommand{\Q}{\mathcal Q}
\newcommand{\RVect}{\RR\operatorname{-Vect}}
\newcommand{\Sch}{\mathcal{S}}
\newcommand{\SL}{\operatorname{SL}}
\newcommand{\Spec}{\operatorname{Spec}}
\newcommand{\spn}{\operatorname{span}}
\newcommand{\supp}{\operatorname{supp}}

\newcommand{\altrep}{\rho_{\text{alt}}}
\newcommand{\trivrep}{\rho_{\text{triv}}}
\newcommand{\regrep}{\rho_{\text{reg}}}
\newcommand{\stdrep}{\rho_{\text{std}}}

\newcommand{\dbar}{\overline\partial}

\def\Xint#1{\mathchoice
{\XXint\displaystyle\textstyle{#1}}%
{\XXint\textstyle\scriptstyle{#1}}%
{\XXint\scriptstyle\scriptscriptstyle{#1}}%
{\XXint\scriptscriptstyle\scriptscriptstyle{#1}}%
\!\int}
\def\XXint#1#2#3{{\setbox0=\hbox{$#1{#2#3}{\int}$ }
\vcenter{\hbox{$#2#3$ }}\kern-.6\wd0}}
\def\ddashint{\Xint=}
\def\dashint{\Xint-}

\renewcommand{\Re}{\operatorname{Re}}
\renewcommand{\Im}{\operatorname{Im}}
\newcommand{\dfn}[1]{\emph{#1}\index{#1}}

\theoremstyle{definition}
\newtheorem{theorem}{Theorem}[chapter]
\newtheorem{lemma}[theorem]{Lemma}
\newtheorem{proposition}[theorem]{Proposition}
\newtheorem{corollary}[theorem]{Corollary}
\newtheorem{axiom}[theorem]{Axiom}
\newtheorem{conjecture}[theorem]{Conjecture}
\newtheorem{definition}[theorem]{Definition}
\newtheorem{remark}[theorem]{Remark}
\newtheorem{example}[theorem]{Example}
\newtheorem{exercise}[theorem]{Exercise}
\newtheorem{problem}[theorem]{Problem}

\makeindex

\begin{document}

\maketitle

\tableofcontents

\chapter{Functional analysis}
Here we treat functional analysis in a high level of abstraction.

Throughout these notes, we mean by $f \preceq g$ that there is a universal constant $C > 0$ such that $f \leq C g$.

\section{Locally convex spaces}
    Fix a vector space $V$.
\begin{definition}
    $V$ is said to be a \dfn{topological vector space} if it is equipped with a topology for which addition and multiplication are continuous.
\end{definition}
\begin{definition}
    $V$ is said to be \dfn{locally convex} if $V$ is equipped with a family of seminorms $P_\alpha$ and the initial topology with respect to the $P_\alpha$.
\end{definition}
    This is the smallest topology containing the open sets $P_\alpha([0, \varepsilon))$ for each $\alpha$ and each $\varepsilon > 0$ and which is translation-invariant.

    The most useful examples of locally convex spaces are Banach spaces.
\begin{definition}
    $V$ is said to be a \dfn{Banach space} if $V$ is equipped with the topology arising from a complete norm.
\end{definition}
\begin{definition}
    If $V$ is a topological vector space, then the \dfn{dual space} of $V^*$ is the space of \emph{continuous} linear maps $V \to \CC$.
\end{definition}
\begin{definition}
    Let $W$ be a Banach space and define a norm on $\Hom(V, W)$ by
    $$||T|| = \sup_{||v|| \leq 1} ||Tv||.$$
\end{definition}
    So $V^*$ is a normed space, $V^* \subseteq \Hom(V, \CC)$. In general it is very difficult to construct elements of $V^*$. In general we cannot guarantee constructively that $V^*$ is nontrivial. On the other hand, it is often impossible to construct linear functions which are discontinuous (for example, any linear functional on a Banach space will be continuous if it was constructed without the axiom of choice).
\begin{definition}
    A function $f: V \to \CC$ is said to be \dfn{sublinear} if it obeys the triangle inequality and if for each $c > 0$ and $x \in V$, $f(cx) = cf(x)$.
\end{definition}
    Obviously seminorms are sublinear. Minkowski gauges are another useful example.
\begin{definition}
    Let $K \subseteq V$. Then:
\begin{enumerate}
    \item $K$ is \dfn{convex} if for each $x, y\in K$, $c \in [0, 1]$, $cx + (1-c)y \in K$.
    \item $K$ is \dfn{balanced} if for each $c \in [0, 1]$, $cK \subseteq K$.
\end{enumerate}
    If $K$ is balanced and convex, then the \dfn{Minkowski gauge} of $K$ is the functional
    $$p_K(x) = \inf_{cK \ni x} c.$$
\end{definition}
    Notice that the balanced condition suggests that $K$ needs to be close to the origin. Moreover, Minkowski gauges are sublinear.

    Sublinear estimates allow us to construct functionals using the axiom of choice, while still guaranteeing that they are continuous.
\begin{theorem}[Hanh-Banach]
    \index{Hanh-Banach theorem}
    Assume that $p: V \to \CC$ is sublinear, $U \subset V$ a subspace, and $f: U \to \CC$ a linear functional. If $f$ is dominated by $p$, i.e. for each $x \in U$, $|f(x)| \leq |p(x)|$, then $f$ extends to $V$.
\end{theorem}
    In general the extension of $f$ will only be unique in case $U$ is dense. So we have to use the axiom of choice to construct $f$.
\begin{proof}
    The extension to the complex case is trivial so we replace $\CC$ with $\RR$. Assume that $f$ is defined on a space $W$, $U \subseteq W \subset V$. Choose $v \in V \setminus W$ and define $f(v)$ such that for each $w \in W$ and $s,t \geq 0$,
$$\frac{p(w-sv)}{s} \leq h(v) \leq \frac{p(w+tv) - f(w)}{t}.$$
    This is always possible because
    $$f((t+s)w) \leq p((t+s)w) = p((t+s)w + tsv - tsv) \leq p(sw + stv) + p(tw - stv)$$
    so
    $$\frac{f(w) - p(w - sv)}{s} \leq \frac{p(w+tv) - f(w)}{t}.$$
    Therefore for any $W$ and $v$ we can extend $f$ to $W + v$. If $\mathcal W$ is the family of subspaces of $V$ on which $f$ is defined and $\mathcal C \subset \mathcal W$ is a chain, then $\mathcal C$ therefore has an upper bound. Since $U \in \mathcal W$, Zorn's lemma implies that $\mathcal W$ has a maximal element, which is clearly $V$.
\end{proof}
    In case $p$ is the norm of $V$, this implies that $V^*$ is nontrivial. The Hanh-Banach theorem also has a useful geometric interpretation.
\begin{theorem}[Hanh-Banach separation theorem]
    \index{Hanh-Banach separation theorem}
    Let $\RR$ be the scalar field and $A, B \subset V$ be convex, nonempty, and disjoint. If $A$ is open then there is a $\varphi \in V^*$ and $t \in \RR$ such that for every $a \in A$ and $b \in B$,
    $$\varphi(a) < t \leq \varphi(b).$$
\end{theorem}
\begin{proof}
    Choose $a_0 \in A$ and $b_0 \in B$, and let $C = A - B + b_0 - a_0$. Then $0 \in C$, $C$ is convex, and $C = \bigcup_{b \in B} A - b + b_0 - a_0$, so $C$ is open. If $x = b_0 - a_0$, then $x \notin C$.

    By the Hanh-Banach theorem, choose a $\varphi \in V^*$ such that $\varphi(x_0) = 1$ and $\varphi < 1$ on $C$. Given $a \in A$ and $b \in B$ we have
    $$\varphi(a) < \varphi(b) + \varphi(a_0) - \varphi(b_0) + 1 = \varphi(b).$$
    If $t = \inf_B \varphi$, then this gives, for every $a$ and $b,$
    $$\varphi(a) \leq t \leq \varphi(b).$$
    Since $A$ is open, $\varphi(A)$ is open so the claim holds.
\end{proof}
    In particular, linear functionals separate points.

    Let's consider more properties of convexity. Let the scalar field be $\RR$ and let $S \subset V$ be nonempty, compact, and convex.
\begin{definition}
    A \dfn{face} of $S$ is a nonempty, compact, convex set $K \subseteq S$ such that for each $x \in K$, if there are $y_1, y_2 \in S$ and $c \in (0, 1)$ such that $x = cy_1 + (1-c)y_2$, then $y_1, y_2 \in K$. If $|K| = 1$, then $K$ is called an \dfn{extreme point}.
\end{definition}
\begin{definition}
    If $X \subset V$ is a set, the \dfn{convex hull} of $X$ is the smallest set containing $X$ which is closed and convex.
\end{definition}
    For example, if $S$ is a convex polygon, then the extreme points of $S$ are its vertices, and $S$ is the convex hull of its vertices. The Krein-Milman theorem says that this phenomenon happens even in infinite dimensions.
\begin{theorem}[Krein-Milman]
    \index{Krein-Milman theorem}
    $S$ is the convex hull of its extreme points.
\end{theorem}
\begin{proof}
    Assume that $S'$ is the convex hull of the extreme points of $S$. Then $S' \subseteq S$, so $S'$ is compact. If there is an $x_0 \in S \setminus S'$, then since $V^*$ separates points, there is a $\lambda \in V^*$ such that $\lambda(S') < \lambda(x_0)$. If $C = \max \lambda(x_0)$, then $\varphi^{-1}(C) \cap S$ contains no extreme points of $S$. We can contradict this by showing that every convex compact set has an extreme point.

    Let $\mathcal F$ be the set of all faces of $S$. Clearly $S \in \mathcal F$ so $\mathcal F$ is nonempty. If $\mathcal C \subset \mathcal F$ is a chain, then $\bigcap \mathcal C$ is a face, so by Zorn's lemma $\mathcal F$ has a minimal element $S_0$.

    Let $\varphi \in V^*$. Since $S_0$ is convex, it is connected, so $\varphi(S_0)$ is compact and connected. In particular, $\varphi(S_0) = [a, b]$ for some $a \leq b$. So $\varphi^{-1}(b) \cap S_0$ is nonempty, convex, and compact. If $x \in \varphi^{-1}(b) \cap S_0$,
    $$x = ty_1 + (1-t)y_2,$$
    then $y_1,y_2 \in S_0$. Therefore $\varphi(x) = b = c\varphi(y_1) + (1-c)\varphi(y_2)$, so $\varphi(y_1), \varphi(y_2) \geq c$. Therefore $y_1, y_2 \in \varphi^{-1}(b) \cap S_0$, so $\varphi^{-1}(b) \cap S_0$ is a face and by minimality, $\varphi^{-1}(b) \cap S_0 = S_0$. So $\varphi(S_0) = b$. Since $\varphi$ was arbitrary and $V^*$ separates points, $|S_0| = 1$. So $S$ has an extreme point.
\end{proof}


\section{Hilbert spaces}
    \begin{definition}
    A \dfn{Hilbert space} $V$ is a Banach space whose norm arises from an inner product.
    \end{definition}
    The basic fact about Hilbert spaces $V$ is that $V^* = V$.
\begin{theorem}[Riesz representation theorem]
    \index{Riesz representation theorem for Hilbert spaces}
    The association
\begin{align*}
    V &\to V^*\\
    v &\mapsto (w \mapsto \langle v, w\rangle)
\end{align*}
    is a surjective isometry.
\end{theorem}
\begin{proof}
    Evaluating $w \mapsto \langle v, w\rangle$ at $v$, we see $||v|| = ||v||_{op}$. So we just have to check surjectivity. Let $\varphi \in V^*$, and $F = \ker \varphi$. If $F = 0$ we're done; otherwise $F^\perp$ is nonempty. Let $z \in F^\perp$ and $\alpha = \varphi(z)/||z||$. Then for any $x \in V$,
    \begin{align*}
        \langle x, \alpha z\rangle  &=  \left\langle x - \frac{\varphi(x)}{\varphi(z)} z, \alpha z\right\rangle + \left\langle \frac{\varphi(x)}{\varphi(z)}z, \alpha z\right\rangle \\&= \left\langle \frac{\varphi(x)}{\varphi(z)}z, \alpha z\right\rangle
            = \frac{\varphi(x)}{\varphi(z)} \varphi(z) = \varphi(x).
    \end{align*}
\end{proof}


\section{Bochner integration}
    Now we fix a Banach space $B$ and a measure space $(X, \Sigma, \mu)$. Recall that the \dfn{Caratheodory construction} is the standard way of building $(X, \Sigma, \mu)$: we define a semiring $\Sigma_0$ of sets (i.e. a family of sets closed under finite intersection and subsets of finite disjoint unions) and a countably additive function $\mu$ on $\Sigma_0$, which then extends to an outer measure $\mu^*$ on the power set $\pset(X)$. If $E \subseteq X$ satisfies the \dfn{Caratheodory criterion}, i.e. that for all $F \subseteq X$,
    $$\mu^*(F) = \mu^*(F \cap E) + \mu^*(F \setminus E),$$
    then we declare that $E$ is measurable. The measurable sets form a $\sigma$-algebra $\Sigma$ on which $\mu^*$ is outer measurable (note that $\mu^*$ did not have to be constructed from a semiring for this step to work; any outer measure will do) and we define the restriction $\mu$ of $\mu^*$ to $\Sigma$ to be the desired outer measure.
\begin{definition}
    A $B$-valued \dfn{integrable simple function} is a finite linear combination of functions \begin{align*}
        \chi_E^b: X &\to B\\
        E \ni x &\mapsto b\\
        E^c \ni x &\mapsto 0
    \end{align*} where $E$ is a measurable set with $||\mu(E)|| < \infty$, $b \in B$.

    The integral of a $B$-valued ISF $f = \sum_n \chi_{E_n}^{b_n}$ is
    $$\int_X f ~d\mu = \sum_n b_n \mu(E_n)$$
    and the $L^1$-norm is $||f||_{L^1} = \int_X |f| ~d\mu$.
\end{definition}
    Then $L^1$ is naturally the Cauchy completion of the ISF.
\begin{definition}
    A function $X \to B$ is a $B$-valued \dfn{integrable function} if it lies in $L^1$.
\end{definition}
\begin{definition}
    For $p \in (1, \infty)$, the $L^p$ norm of $f: X \to B$ is
    $$||f||_{L^p} = \left(\int_X ||f(x)||^p ~d\mu(x)\right)^{1/p}$$
    and the $L^\infty$ norm is $||f||_{L^\infty} = \lim_{p \to \infty} ||f||_{L^p} = \esssup ||f||$.
\end{definition}
    The usual Lebesgue convergence theorems hold:
\begin{theorem}[Lebesgue convergence theorems]
    Let $\{f_n\}$ be a pointwise convergent sequence of integrable functions. Then:
\begin{enumerate}
    \item If each $f_n \leq f_{n+1}$,
    $$\lim_n \int f_n = \int \lim_n f_n < \infty.$$
    \item If there is an integrable function $g > 0$ such that every $|f_n| \leq g$,
    $$\lim_n \int f_n = \int \lim_n f_n \leq g.$$
    \item $$\int \liminf_n f_n \leq \liminf_n \int f_n.$$
\end{enumerate}
\end{theorem}

    Now let's make some estimates which will actually prove that the $L^p$-norm is a norm, besides being useful later.
\begin{theorem}[Jensen's inequality]
    \index{Jensen's inequality}
    Let $f: \RR \to \RR$ be convex and $g$ an integrable function. Then
    $$f\left(\int g\right) \leq \int f \circ g.$$
\end{theorem}
\begin{theorem}[Holder's inequality]
    \index{Holder's inequality}
    Let
    $$\frac{1}{p} + \frac{1}{q} = 1.$$
    Then $||fg||_{L^1} \leq ||f||_{L^p} ||g||_{L^q}$.
\end{theorem}
\begin{proof}
    The mapping $x \mapsto x^p$ is convex so if $f, g \geq 0$,
\begin{align*}
    \int fg
        &= \left(\int g^q\right) \int fg^{1-q} \frac{g^q}{\int g^q}
        \leq \left(\int g^q\right) \left(\int f^p g^{p(1-q)}\frac{g^q}{\int g^q}\right)^{1/p}\\
        &= \left(\int g^q\right) \left(\left(\int g^q\right) \left(\int f^p\right)\right)^{1/p}
        \leq \left(\int f^p\right)^{1/p} \left(\int g^q\right)^{1/q}.
\end{align*}
\end{proof}
    Notice that Holder's inequality implies that $L^2$ is a Hilbert space with inner product
    $$\langle f, g\rangle = \int fg.$$
\begin{theorem}[Minkowski's inequality]
    \index{Minkowski's inequality}
    Let
    $$\frac{1}{p} + \frac{1}{q} = 1.$$
    Then
    $$||f + g||_{L^p} \leq ||f||_{L^p} + ||g||_{L^p}.$$
\end{theorem}
\begin{proof}
    By Holder's inequality,
    \begin{align*}
        \int |f+g|^p
            &= \int |f+g||f+g|^{p-1}
            \leq \int (|f| + |g|) |f+g|^{p-1}
            \\&\leq \left(\left(\int |f|^p \right)^{1/p} + \left(\int |g|^p\right)^{1/p}\right)\left(\int |f+g|^{(p-1)\left(\frac{p}{p-1}\right)} \right)^{1-\frac{1}{p}}\\
            &= (||f||_{L^p} + ||g||_{L^p}) \frac{||f+g||_{L^p}^p}{||f+g||_{L^p}}.
    \end{align*}
\end{proof}
    Now we discuss change of variables.
\begin{definition}
    Let $\nu$ be a measure. Then
\begin{enumerate}
    \item $\nu$ is \dfn{absolutely continuous} with respect to $\mu$ if for every measurable set $A$, $\mu(A) = 0$ implies $\nu(A)$.
    \item $\nu$ is \dfn{singular} with respect to $\mu$ if there are disjoint measurable sets $A, B$ such that $X = A \cap B$, $\nu(A) = 0$ and $\mu(A) = 0$.
    \item If there is a measurable function $f$ such that for every measurable set $A$,
    $$\nu(A) = \int_A f ~d\mu,$$
    then $f$ is the \dfn{Radon-Nikodym derivative} of $\nu$, written
    $$f = \frac{d\nu}{d\mu}.$$
\end{enumerate}
\end{definition}
\begin{theorem}[Radon-Nikodym]
    \index{Radon-Nikodym theorem}
    Let $\mu$ be $\sigma$-finite and $\nu$ be a positive measure. Then there is a unique decomposition $\nu = \nu_a + \nu_s$ such that $\nu_a$ is absolutely continuous and $\nu_s$ is singular (with respect to $\mu$). Moreover, $\nu_a$ has a Radon-Nikodym derivative.
\end{theorem}
    In particular, if $\nu$ was already absolutely continuous, then $\nu_s = 0$ and $\nu$ has a Radon-Nikodym derivative.
\begin{proof}
    Uniqueness is obvious. First assume $\mu(X) < \infty$. Then $\mu + \nu$ is finite, so $L^\infty(\mu + \nu) \subseteq L^1(\mu + \nu)$. So by the Cauchy-Schwarz inequality, if $f$ is an ISF,
    $$\left|\int f ~d\nu\right| \leq ||f||_{L^1(\nu)} \leq ||f||_{L^1(\mu+\nu)} \preceq ||f||_{L^2(\mu)}.$$
    So $\int \cdot ~d\nu$ is $L^2$-continuous on ISF, hence on $L^2(\mu + \nu)$. So by the Riesz representation theorem, there is a nonnegative $h \in L^1(\mu + \nu)$ such that
    $$\int f ~d\nu = \int \int fh ~d(\mu + \nu)$$
    for any $f \in L^2$. In particular, if $A$ is measurable,
    $$\int_A h ~d(\mu + \nu) = \nu(A) \leq (\mu + \nu)(A).$$
    Without loss of generality we assume $h \leq 1$. If $g \in L^\infty(\nu)$,
    $$\int g ~d\nu = \int gh ~d\mu + \int gh ~d\nu.$$
    So if $Y$ is the set of all $y$ such that $0 \leq h(y) < 1$, it follows that $\mu(Y) = \mu(X)$. By induction,
    $$\int g ~d\nu = \int g(h + \dots + h^n) ~d\mu + \int gh^n ~d\nu.$$
    Since $h \leq 1$, the dominated convergence theorem implies
    $$\int gh^n ~d\nu \to \int_{X \setminus Y} g ~d\nu$$
    and if
    $$f = \frac{h\chi_Y}{1-h}$$
    we have
    $$\int g ~d\nu = \int_Y gf ~d\mu + \int_{X \setminus Y} g ~d\nu$$
    and take $\nu_s(A) = \nu(A \cap (X \setminus Y))$. Then we take
    $$\nu_a(A) = \int_A f ~d\mu$$
    so $f$ is the Radon-Nikodym derivative of $\nu_a$, $\nu_a + \nu_s = \nu$ by taking $g = \chi_A$.

    To extend to the $\sigma$-finite case, break up $X$ into countably many finite measure spaces and sum over them.
\end{proof}
    Next we discuss iterated integrals. Given measure spaces $(X, S, \mu)$ and $(Y, T, \nu)$, we need a $\sigma$-algebra on $X \times Y$ and a measure defined on that $\sigma$-algebra. To do this, we use the Caratheodory construction.
\begin{definition}
    If $E \in S$ and $F \in T$, then $E \times F$ is a \dfn{measurable rectangle}. Let $S \otimes T$ denote the smallest $\sigma$-algebra containing the measurable rectangles, and on for each measurable rectangle, define a countably additive function by
    $$\mu \otimes \nu(E \times F) = \mu(E) \nu(F).$$
\end{definition}
    By the monotone convergence theorem $d\nu$, $\mu \otimes \nu$ is countably additive. So the Caratheodory construction gives rise to a measure $\mu \otimes \nu$ whose measurable sets include $S \otimes T$ (in fact, is the completion of $S \otimes T$).
\begin{definition}
    The measure space $(X \times Y, S \otimes T, \mu \otimes \nu)$ is the \dfn{product measure space} of $(X, S, \mu)$ and $(Y, T, \nu)$.
\end{definition}
    Straight from the definitions, we know that for every measurable rectangle $E \times F$,
    $$\int \chi_{E \times F} ~d(\mu \otimes \nu) = \iint \chi_{E \times F} ~d\mu ~d\nu = \iint \chi_{E \times F} ~d\nu ~d\mu.$$

    For a function $f$ defined on $X \times Y$ we define $f^y(x) = f(x, y)$ and $f_x(y) = f(x, y)$. For a set $G \subseteq X \times Y$, we define $G^y = \{x \in X: (x, y) \in G\}$ and $G_x = \{y \in Y: (x, y) \in G\}$.
\begin{theorem}[Fubini]
    \index{Fubini's theorem}
    Let $f \in L^1(\mu \otimes \nu)$ and assume $\mu \otimes \nu$ is $\sigma$-finite. Then for almost every $y$, $f^y \in L^1(\mu)$. Moreover, the function
    $$F(y) = \int f^y ~d\mu$$
    has $F \in L^1(\nu)$, and
    $$\int f ~d(\mu \otimes \nu) = \iint f^y ~d\mu ~d\nu = \iint f_x ~d\nu ~d\mu.$$
\end{theorem}
    The assumption of $\sigma$-finiteness is not optional here, and Fubini's theorem can fail for large cardinality measure spaces.
\begin{definition}
    Let $M$ be a family of subsets of $X$ such that for every countable chain of $A_n$ in $M$ and $\bigcup_n A_n = A$ or $\bigcap_n A_n = A$, $A \in M$. Then we say $M$ is a \dfn{monotone class}.
\end{definition}
    If $R$ is a ring of sets, then the smallest monotone class $M$ containing $R$ is also a ring, and it is not hard to see that $M$ is the smallest $\sigma$-algebra containing $R$.
\begin{lemma}
    Let $G \in S \otimes T$. Then:
\begin{enumerate}
    \item $G_x \in T$ and $G^y \in S$.
    \item $x \mapsto \nu(G_x)$ and $y \mapsto \mu(G^y)$ are measurable.
    \item One has
    $$\mu \otimes \nu(G) = \int (x \mapsto \nu(G_x)) ~d\mu(x) = \int (y \mapsto \mu(G^y)) ~d\nu(y) = \iint \chi_G ~d\mu ~d\nu.$$
\end{enumerate}
\end{lemma}
\begin{proof}
    This is obvious if $G$ is a measurable rectangle. We shall show that the algebra of sets on which this claim holds is a monotone class, hence a $\sigma$-algebra. Clearly if $\bigcup_n G_n = G$ then $G$ has the property. Given $x \in X$, $\bigcup_n (G_n)_x = G_x$, so $G_x \in T$. Therefore the chain of functions $x \mapsto \nu(G_n)_x$ converges to $x \mapsto \nu(G_x)$ which is therefore measurable. So by the monotone convergence theorem,
    $$\lim_n \mu \otimes \nu(G_n) = \lim_n \int (x \mapsto \nu((G_n)_x) ~d\mu(x) = \int (x \mapsto \nu(G_x) ~d\mu(x) = \mu \otimes \nu(G).$$
    So this algebra is closed under ascending chains. The proof in the other direction is similar but you have to start by assuming that $\mu \otimes \nu(G_1) < \infty$.
\end{proof}
\begin{lemma}
    Let $f \geq 0$ be $S \otimes T$-measurable. Then
    $$\int f ~d\mu \otimes \nu = \iint f ~d\mu ~d\nu.$$
\end{lemma}
\begin{proof}
    Let $\{f_n\}$ be a chain of ISFs. This claim is obvious for ISF, so the monotone convergence theorem on the $f_n^y$ for each $y \in Y$.
\end{proof}
\begin{theorem}[Tonelli]
    \index{Tonelli's theorem}
    If $f$ is $S \otimes T$-measurable, $g(x) = ||f(x)||$, $g^y \in L^1(\mu)$, and $(y \mapsto \int g^y ~d\mu) \in L^1(\nu)$, then $f \in L^1(\mu \otimes \nu)$.
\end{theorem}
\begin{proof}
    Clear by the lemmata.
\end{proof}
\begin{proof}[Proof of Fubini's theorem]
    Let $g(x, y) = ||f(x, y)||$. Then if $\{f_n\}$ is a sequence of ISF converging to $f$, $g$ dominates the $f_n$. Apply the dominated convergence theorem twice, once for each integral.
\end{proof}

\section{Duality}
    Fix a normed space $V$. We consider properties of $V^*$. Since $\CC$ is complete, $V^*$ is a Banach space; in particular, $V^{**}$ is a Banach space. So we can always embed $V$ in a Banach space by the mapping
\begin{align*}
    V &\to V^{**}\\
    v &\mapsto (\varphi \mapsto \varphi(v)).
\end{align*}
    However, $V^{**}$ is rarely the completion of $V$ if $V$ is infinite-dimensional. Moreover, the topology of $V^*$ is a bit awkward to work with, since a convergence in operator norm is much stronger than convergence pointwise.
\begin{definition}
    The \dfn{weakstar topology} of $V^*$ is the initial topology such that every evaluation $\varphi \mapsto \varphi(v)$ is continuous.
\end{definition}
    In other words, the weakstar topology is the topology of pointwise convergence.
\begin{theorem}[Banach-Alaoglu]
    \index{Banach-Alaoglu theorem}
    Let $B$ be the closed unit ball of $V^*$. Then $B$ is weakstar compact.
\end{theorem}
    Like the Hanh-Banach and Krein-Milman theorems, the proof of Banach-Alaoglu uses the axiom of choice. However, the Banach-Alaoglu theorem is not really nonconstructive, since if $V$ is separable, we can use a diagonalization argument to prove it instead. Banach-Alaoglu generalizes to locally convex spaces.
\begin{proof}
    Let
    $$D_v = \{z \in \CC: |z| \leq ||v||\}$$
    and $D = \prod_{v \in V} D_v$. By Tychonoff's theorem, $D$ is compact. Moreover, there is a natural embedding
\begin{align*}
    \iota: V^* &\to D\\
    f &\mapsto \{f(v)\}_{v \in V}.
\end{align*}
    Since the product topology is the topology of pointwise convergence, $\iota$ is a homeomorphism $V^* \to \iota(V^*)$. So we just need to show that $\iota(V^*)$ is closed. So let $\{\{f_\alpha(v)\}_{v \in V}\}_{\alpha \in A}$ be a net in $D$, which converges to a $\{\varphi_v\}_{v \in V}$. Then $f(v) = \varphi_v$ is a linear functional and $f_\alpha \to f$ pointwise so $\{\varphi_v\}_{v \in V} \in \iota(V^*)$.
\end{proof}
    Now we compute the duals of the main examples of Banach spaces we have presented so far.
\begin{theorem}
    Let $p, q \in [1, \infty]$ and assume $\mu$ is $\sigma$-finite.
    $$\frac{1}{p} + \frac{1}{q} = 1.$$
    Then $(L^p(\mu))^* = L^q(\mu)$.
\end{theorem}
    Actually, this theorem is true without the $\sigma$-finiteness; however, it becomes much more difficult.
\begin{proof}
    For $g \in L^q$, one has $||g||_{p^*} \leq ||g||_q$ by Holder's inequality and by taking larger and larger measurable sets $E$ and considering $\int_E g$, we check $||g||_{p^*} \geq ||g||_q$. So we just need to show that the map $L^q \to L^{p^*}$ is surjective.

    If $h \in L^p$ and $X$ splits into finite measure spaces $X_k$ we put $h_k = \chi_{X_k}h$, so $\sum_k h_k = h$ in $L^p$ by the dominated convergence theorem. If $\varphi \in L^{p^*}$ then $\varphi(\sum_k h_k) = \sum_k \varphi(h_k)$ so we might as well assume $X = X_1$, viz. $\mu(X) < \infty$. Then $L^\infty \subseteq L^p$, so $\varphi \in (L^\infty)^*$. We can define an absolutely continuous measure $\nu$ by $\nu(A) = \varphi(\chi_A)$, and by the Radon-Nikodym theorem, there is a Radon-Nikodym derivative $f$ of $\nu$.

    Let $Y_n = \{x \in X: |f(x)| \leq n\}$ and let $g = f/|f|^{q-2}$, where $g(x) = 0$ if $f(x) = 0$, and $g_n = \chi_{Y_n}g_n$. Then $|g|^p = |f|^q$ and
    $$\int_{Y_n} |f|^q = \int g_nf = \varphi(g_n) \preceq ||g_n||_p \preceq ||f_n||_{L^p(Y_n)}.$$
    So $||f||_{L^q(Y_n)} < \infty$, and by the monotone convergence theorem, $f \in L^q$.
\end{proof}

\section{Vector lattices}
    We now consider the natural order structure of a space.
\begin{definition}
    A \dfn{vector lattice} is a vector space $V$ equipped with a partial order $\leq$ which is translation-invariant such that $(V, \leq)$ is a lattice, and such that for each $c \geq 0$ and $x \leq y$, $cx \leq cy$.
\end{definition}
    Recall that a lattice is just a poset which is closed under finite joins $\vee$ (suprema) and meets $\wedge$ (infima). Actually, we just need to check that $V$ is a semilattice, since multiplication by $-1$ implies that a semilattice is already a lattice.

    If $V$ is a vector lattice and $v \in V$, we define $f_\pm = \pm f \vee 0$. Then $f = f_+ - f_-$ and we define the absolute value (or valuation) $|f| = f_+ + f_-$.
\begin{definition}
    A \dfn{Banach lattice} is a vector lattice $V$ which is a Banach lattice, such that $|x| \leq |y|$ whenever $||x|| \leq ||y||$.
\end{definition}
\begin{example}
    A function space mapping into $\RR$ is usually a Banach lattice with the natural ordering, $f \leq g$ iff for every $x$, $f(x) \leq g(x)$. For example, $C(X)$ is a lattice. Spaces of operators are Banach lattices as well, whose positive elements are precisely the positive operators; as are spaces of signed measures, where the positive measures are the positive elements.
\end{example}
\begin{theorem}
    Let $V$ be a Banach lattice. There is a natural ordering on $V^*$, such that $f \in V^*$ is positive iff for each positive $v \in V$, $f(v) \geq 0$, and such that $f \leq g$ iff for every positive $v \in V$, $f(v) \leq g(v)$.
\end{theorem}
\begin{proof}
    Take the definition of positive functionals as in the statement of the theorem. If $f$ and $-f$ are both positive, each $v = v_+ - v_-$ has $f(v_+) \geq 0$ but $f(v_-) \leq 0$. So $f(v) = 0$. Since $v$ was arbitrary, $f = 0$.

    Given $f \in V^*$, define
    $$f^+(v) = \sup_{0 \leq x \leq v} f(x)$$
    for $v \geq 0$. Then $f^+ \geq f$, and $f^+$ is finite because if $x \leq v$, $|f(x)| \leq ||f|| ||v||$. Moreover, if $v, w \geq 0$, it is easy to check $f^+(v+w) = f^+(v) + f^+(w)$. So $f^+$ is positive-linear, so extends to all of $V$ and so $f^+ \in V^*$.

    Clearly $f^+ - f \geq 0$. We need to show this is optimal, i.e. $f^+ = f \vee 0$. Assume $g \geq f \vee 0$. Then for $0 \leq x \leq v$, $f(x) \leq g(x) \leq g(v)$, so taking the $\sup$ over $x$ we have $f^+(v) \leq g(v)$. The other direction is similar. So $f^+ = f \vee 0$.
\end{proof}
    Fix a compact Hausdorff space $X$, $|X| \geq 2$ (so in particular, every set which separates points is nonempty). Let us now study the behavior of sublattices of $C(X) = C(X \to \RR)$.
\begin{theorem}[Dini]
    \index{Dini's theorem}
    Let $L$ be a sublattice of $C(X)$, and define $g(x) = \inf_{f \in L} f(x)$. For each $\varepsilon > 0$, there exists a $h \in L$ such that $g \leq h \leq g + \varepsilon$.
\end{theorem}
\begin{proof}
    For each $f \in L$ let $U_f = \{x \in X: f(x) - g(x) \leq \varepsilon\}$. Then the $U_f$ are an open cover of $X$, which has a finite subcover by functions $f_1, \dots, f_k$. Take $h = \bigwedge_{j \leq k} f_j$.
\end{proof}
    When can a lattice be used to approximate any function in $C(X)$? A necessary condition is that the lattice strongly separates points. This turns out to be sufficient as well.
\begin{definition}
    A set $A \subseteq C(X)$ \dfn{separates points} if for each $x, y \in X$, there is an $f \in A$ such that $f(x) \neq f(y)$. If, in addition, the constant functions $\RR \subseteq A$, then $A$ \dfn{strongly separates points}.
\end{definition}
\begin{theorem}[Stone-Weierstrass]
    \index{Stone-Weierstrass theorem}
    If $L \subseteq C(X)$ is a sub-vector lattice or a subalgebra which strongly separates points, then $L$ is dense in $C(X)$.
\end{theorem}
    The lattice case is also called the \dfn{Kakutani-Krein theorem}.
\begin{lemma}
    \label{sw lem 1}
    Let $L$ be a sublattice of $C(X)$ which separates points and is closed under multiplication and addition by elements of $\RR$. Then if $B \subseteq X$ is compact, $p \in X \setminus B$, and $a, b \in \RR$, there is a $g \in L$ such that $g \geq a$, $g(p) = a$ and $g > b$ on $B$.
\end{lemma}
\begin{proof}
    For each $x \in B$ there exists $g_x \in L$ such that $g_x(p) = a$ and $g_x(x) = b+1$. Let $U_x = \{y\in X: g_x(y) > b\}$. Since $x \in U_x$, the $U_x$ are an open cover of $B$ with finite subcover $U_{x_1}, \dots, U_{x_k}$. Take $g = a \vee \bigvee_{j \leq k} g_{x_k}$.
\end{proof}
\begin{lemma}
    \label{sw lem 2}
    Assume that $L$ is a closed unital subalgebra of $C(X)$. Then $L$ is a lattice.
\end{lemma}
\begin{proof}
    Choose $\varepsilon > 0$ and apply the classical Weierstrass theorem to $[-1, 1]$ to find a polynomial $P_\varepsilon$ which approximates $|\cdot|$ in $L^\infty$-norm by $\varepsilon$. Then for each $f \in L$, we can approximate $|f|$ by $P_\varepsilon \circ f$. Since $L$ is unital, $P_\varepsilon \circ f \in L$. So $|f| \in L$, since $L$ is closed. The lattice operations $\vee$ and $\wedge$ can be expressed in terms of algebra operations $+$ and $\cdot$, and $|\cdot|$, so $L$ is closed under lattice operations.
\end{proof}
\begin{proof}[Proof of Stone-Weierstrass]
    First consider the case that $L$ is a lattice. Given $f \in C(X)$, define $L_f = \{g \in L: g \geq f\}$. Then $L_f$ is a sublattice of $L$. Given $x \in X$, $\delta > 0$, the set $B = \{y \in X: f(y) \geq f(x) + \delta\}$ is closed. Since $X$ is compact, there is an $M > 0$ such that $f < M$. Apply Lemma \ref{sw lem 1} with $a = f(x) + \delta$ and $b = M$, so there is a $g \in L$ such that $g \geq f(x) + \delta$, $g(x) = f(x) + \delta$ and $g > M$ on $B$. So $f \leq g \leq g + \delta$, so $f = \bigwedge L_f$. Therefore by Dini's theorem, there is an $h \in L$ with the desired properties.

    For the algebra case, since $L$ strongly separates points, $L$ is unital. Therefore $\overline L$ is a closed unital algebra, $\overline L$ is a closed lattice whose closure is $C(X)$, by Lemma \ref{sw lem 2} and the above case. So $\overline L = C(X)$.
\end{proof}
This even extends to decaying functions on locally compact Hausdorff spaces, by taking the one-point compactification.

\section{Positive Radon measures}
The usual construction of measures by ISF is somewhat unnatural when we have a nice topology, since then we can define integration in terms of continuous functions. Clearly ``nice" in this context implies locally compact Hausdorff; these conditions are also sufficient (though $\sigma$-compactness also helps). Throughout this section, we fix a locally compact Hausdorff space $X$ and consider the space $C_c(X)$ of compactly supported continuous functions $X \to \CC$.

We have not given a topology on $C_c(X)$, so a functional is just an element of $\Hom(C_c(X), \CC)$ for now.
\begin{definition}
    A \dfn{positive Radon measure} on $X$ is a functional on $C_c(X)$.
\end{definition}
Let us prove that a positive Radon measure is actually a measure in a natural way. First, we put a topology on $C_c(X)$. We start by putting the $L^\infty$-topology on $C_c(U)$ for each open set $U \subseteq X$ with compact closure.
\begin{definition}
    The \dfn{inductive limit topology} of $C_c(X)$ is the final (i.e. strongest) topology on $C_c(X)$ such that $\varphi: C_c(X) \to Y$ is continuous provided that for each open set $U \subseteq X$ with compact closure, $\varphi|_{C_c(U)}$ is continuous.
\end{definition}
In other words, the inductive limit topology is the final topology which makes the natural maps $C_c(U) \to C_c(X)$ continuous. A positive Radon measure is continuous for the inductive limit topology, as can be seen by taking an $h \in C_c(X)$ which is $1$ on $U$, so $||\varphi||_{C_c(U)} \leq \varphi(h)$.

Now we need some general facts about locally compact Hausdorff spaces.
\begin{definition}
    A (continuous) \dfn{partition of unity} on a subordinate to an open cover $U_1, \dots, U_n$ is a family of (continuous) functions $f_1, \dots, f_n$ which are compactly supported in $U_i$, such that $\sum_i f_i = 1$.
\end{definition}
\begin{theorem}
    \label{partitions of unity}
    For any finite open cover $\mathcal U$ of a compact set, there is a partition of unity subordinate to $\mathcal U$.
\end{theorem}
\begin{lemma}
    Let $K \subseteq X$ be compact. If $U_1, \dots, U_n$ is an open cover of $K$ there are compact sets $K_1, \dots, K_n$, $K_i \subseteq U_i$, which cover $K$.
\end{lemma}
\begin{proof}
    For each $x \in K$ choose a $j$ such that $U_j \ni x$ and an open set $V_x \ni x$ such that
    $$V_x \subset \overline V_x \subset U_j.$$
    Then the $V_x$ are an open cover of $K$ so they reduce to a finite subcover $V_{x_1}, \dots, V_{x_p}$. For each $k \leq p$ choose a $j_k \leq n$ such that $V_{x_k} \subseteq U_{j_k}$ and let $W_j = \bigcup_{j_k=j} V_k \subseteq U_j$. Then $\overline W_j \subseteq U_j$ and the $\overline W_j$ contain the $V_x$s, so are a compact cover of $K$.
\end{proof}
\begin{proof}[Proof of Theorem \ref{partitions of unity}]
    Fix a compact set $K$. By the lemma, we can choose $D_j \subseteq U_j$ a compact cover of $K$ and $g_j$ supported in $U_j$ with $g_j \geq 1$ on $D_j$, and $h = \sum_j g_j$. Then $h \geq 1$ on $C$ and put $k = h \vee 1 \geq 1$. So $1/k$ exists and we can put $f_j = g_j/k$, to force $\sum_j f_j = 1$.
\end{proof}

\begin{definition}
    A \dfn{content} is a function defined on sets into $[0, \infty]$ which is monotone, countably subadditive, and finitely additive, and which carries compact sets to $[0, \infty)$. A content $\mu$ is said to be \dfn{inner regular} if for every open set $U$,
    $$\mu(U) = \sup_{\substack{\overline V \subseteq U\\V \text{open}\\\overline V \text{compact}}} \mu(V).$$
\end{definition}
Fix a positive Radon measure $\varphi$, and define an inner-regular content $\mu$ on the topology $\mathcal T$ on $X$ by
\begin{align*}
    \mu: \mathcal T &\to [0, \infty]\\
    U &\mapsto \sup_{\substack{f \in C_c(U)\\0 \leq f \leq 1}} \varphi(f).
\end{align*}
Given a content $\nu$, we can extend $\nu$ to an outer measure $\nu^*$ on the power set $\pset$ by
$$\nu^*(A) = \inf_{\substack{U \subseteq A\\U \in \mathcal T}} \nu(U).$$
In turn, then, $\nu^*$ restricts to a measure, also called $\nu$, on its measurable $\sigma$-algebra, by the Caratheodory construction. So, in particular, $\mu$ gives rise to a measure.

\begin{definition}
    Let $\nu$ be a Borel measure. We say that $\nu$ is \dfn{outer regular} if for every Borel set $E$,
    $$\mu(E) = \inf_{\substack{E \subseteq U\\U \in \mathcal T}} \mu(U)$$
    and \dfn{inner regular} if for every \emph{open} set $U$,
    $$\mu(U) = \sup_{\substack{\overline V \subseteq U\\V \text{open}\\\overline V \text{compact}}} \mu(V).$$
\end{definition}
    We state the main result.
\begin{theorem}[Riesz-Markov representation theorem]
    \index{Riesz-Markov representation theorem}
    $\mu$ is a positive Borel measure which is both inner and outer regular, and $\varphi$ is the unique functional such that for every $f \in C_c(X)$,
    $$\varphi(f) = \int f~d\mu.$$
\end{theorem}
    The proof of the Riesz-Markov representation theorem is quite long, so we only sketch it.
\begin{proof}[Proof sketch]
    Let $\nu^*$ be an outer measure which is finitely additive and inner regular on the topology of $X$, and let $U$ be open. Then Caratheodory's criterion holds for $U$ and $\nu^*$ on open sets. Approximating any subset of $X$ by an open set, Caratheodory's criterion holds on the power set for $U$ and $\nu^*$. So $U$ is $\nu^*$-measurable, and $\nu^*$ restricts to a Borel measure $\nu$. In particular, $\mu$ is a Borel measure.

    If $f \in C_c(X)$, and $f \geq 1$ on an open set $U$, $\varphi(f) \geq \mu^*(U)$. Approximating any set $A$ by an open set, we see that $\varphi(f) \geq \mu^*(A)$ whenever $f \geq 1$ on $A$. On the other hand, if $f \leq 1$ on $A$, a monotone convergence argument shows that $\mu^*(A) \geq \varphi(f)$. Since $C_c(X)$ is a Banach lattice, we can replace $f$ with $f^+$ and by decomposing $X$ into a chain of sets $X_n \{x \in X: f(x) \geq n\varepsilon\}$ and summing the $f|_{X_n} \setminus f|_{X_{n-1}}$ we prove
    $$\varphi(f) = \int f~d\mu.$$

    Since $\mu$ was inner and outer regular as a content, approximation by open sets implies regularity on Borel sets. Moreover, if $\psi$ is a positive Radon measure, define a content $\nu$ by
    $$\nu(U) = \sup_{\substack{f \leq \chi_U\\f \in C_c(U)}} \int f ~d\nu.$$
    If $\nu = \mu$ it follows that $\psi = \varphi$.
\end{proof}
    Notice that on the other hand, a complex measure $\nu$ on $C_c(X)$ gives rise to a functional $\psi$ by
    $$\psi(f) = \int f ~d\nu.$$
    The positive part of $\psi$ is in fact the positive part of $\nu$.

    Now if $S$ is a locally compact semigroup, we let $M(S)$ be the set of all finite Radon measures on $S$. This is a convolution algebra, with
    $$\mu*\nu(f) = \iint_S f(xy) ~d\mu(x) ~d\nu(y).$$

\section{Baire categories}
    Now we look at a topological analogue of ``measure zero."
\begin{definition}
    Let $X$ be a topological space. A set $S \subseteq X$ is \dfn{nowhere dense} if for every open set $U$, $S \cap U$ is not dense in $U$. A set $T \subseteq X$ is \dfn{meager} or \dfn{of the first category} if $T$ is the countable union of nowhere dense sets. A set $W \subseteq X$ is \dfn{of the second category} if it is not of the first category, or \dfn{comeager} if it is the complement of a meager set.
\end{definition}
\begin{lemma}
    For a topological space, the following are equivalent:
\begin{enumerate}
    \item Every countable union of closed sets with empty interior has empty interior.
    \item Every countable intersection of open dense sets is dense.
    \item Every nonempty open set is of the second category.
\end{enumerate}
\end{lemma}
    This is basically obvious.
\begin{definition}
    A topological space is a \dfn{Baire space} if one (and all) of the above criteria hold.
\end{definition}
\begin{theorem}[Baire category theorem]
    \index{Baire category theorem}
    Every completely pseudometrizable or locally compact Hausdorff space is Baire.
\end{theorem}
    For example, a Banach space is Baire.
\begin{proof}
    Let $U_n$ be a sequence of open dense sets, and let $W$ be open, in the space $X$. Then $W \cap U_1$ is nonempty and open, say $x_1 \in W \cap U_1$. If $X$ is pseudometrizable, then there is a $\varepsilon_1 \in (0, 1)$ such that the open ball $V_1 = B(x_1, \varepsilon_1)$ satisfies $K_1 = \overline B(x_1, \varepsilon_1) \subseteq W \cap U_1$; if $X$ is locally compact Hausdorff, then there is a compact set with nonempty interior $V_1 \subseteq K_1 \subseteq W \cap U_1$. Iterate using the denseness of the $U_n$ and the axiom of choice to construct a sequence $x_n \in V_n \subseteq K_n \subseteq V_{n-1} \cap U_n$. If $X$ is pseudometrizable, then we can always choose $\varepsilon_n < 1/n$, so the $x_n$ are a Cauchy sequence. Otherwise, $\bigcap_n K_n$ is nonempty anyways by the finite intersection property. Either way, we can find an $x \in \bigcap_n K_n \subseteq \bigcap_n U_n$ such that $x \in W$. So $\bigcap_n U_n$ is dense.
\end{proof}
Actually, we didn't use the full axiom of choice. The Baire category theorem is equivalent over ZF to the following axiom.
\begin{axiom}[Axiom of dependent choice]
    \index{axiom of dependent choice}
    Let $X$ be a nonempty set and $R$ be a binary relation. If, for every $a \in X$, there is a $b \in X$ such that $aRx$, then there is a sequence of $x_n$ such that $x_nRx_{n+1}$.
\end{axiom}
The axiom of dependent choice is not strong enough to prove the existence of nonmeasurable sets, for example. Moreover, if $X$ is assumed to be separable, then the Baire category theorem just follows from induction, without even dependent choice.

\begin{theorem}[uniform boundedness principle]
    \index{uniform boundedness principle}
    Let $X$ be a Banach space and $Y$ a normed space, and $F$ be a set of linear mappings $X \to Y$. If for every $x \in X$,
    $$\sup_{T \in F} ||Tx|| < \infty,$$
    then
    $$\sup_{\substack{T \in F\\||x|| = 1}} ||Tx|| = \sup_{T \in F} ||T||.$$
\end{theorem}
The uniform boundedness principle is also called the Banach-Steinhaus theorem.\index{Banach-Steinhaus theorem} The proof is a standard application of the Baire category theorem: construct a chain of closed sets whose union is the entire space, which implies that one is not meager.
\begin{proof}
    Let
    $$X_n = \{x \in X: \sup_{T \in F} ||Tx|| \leq n\}.$$
    Then the $X_n$ are a closed chain whose union is $X$. So by the Baire category theorem, there is an $x \in X$, $m > 0$, and $\varepsilon > 0$ such that $B(x, \varepsilon) \subset X_m$. So if $||u|| < 1$ and $T \in F$,
\begin{align*}
    ||Tu|| &= \varepsilon^{-1} ||T(x + \varepsilon u) - Tx||
        \leq \varepsilon^{-1} ||T(x + \varepsilon u)|| + \varepsilon^{-1} ||Tx||
        \leq 2\frac{m}{\varepsilon}.
\end{align*}
    Taking the $\sup$ over $u$ of both sides,
    $$\sup_{T \in F} ||T|| \leq 2\frac{m}{\varepsilon} < \infty.$$
\end{proof}
\begin{theorem}[open mapping theorem]
    \index{open mapping theorem}
    If $A: X \to Y$ is a surjective continuous linear mapping between Banach spaces, then $A$ is open.
\end{theorem}
The open mapping theorem is also called the Banach-Schauder theorem.\index{Banach-Schauder theorem}
\begin{proof}
    We must show that if $U$ is the open unit ball of $X$, then $A(U)$ is open. Since $X = \bigcup_k kU$, $Y = \bigcup_k A(kU)$. By the Baire category theorem, there is a $k > 0$, $\varepsilon > 0$, and $y \in Y$ such that $B(y, \varepsilon) \subseteq \overline{A(kU)}$. If $V$ is the unit ball of $Y$, $v \in V$, $y+\varepsilon v \in \overline{A(kU)}$ so
    $$\varepsilon v \in \overline{A(kU)} + \overline{A(kU)} \subseteq \overline{A(2kU)}.$$
    So if $L = 2k/\varepsilon$, $V \subseteq \overline{A(LU)}$.

    In other words, for every $y \in Y$ and $\varepsilon > 0$ there is an $x \in X$ such that $||x|| \leq L||y||$ and $||y-Ax|| < \varepsilon$. In particular, given $y \in V$ we can choose $x_1$ such that $||x_1|| \leq L$ and $||x-Ax_1|| < 1/2$. Choose $||x_{n+1}|| \leq L2^{-n}$ such that
    $$||y - A(x_1 + \dots + x_n) - Ax_{n+1}|| < 2^{-n-1},$$
    by induction and the axiom of (dependent) choice. The sequence of partial sums is therefore Cauchy, so we can put $x = \sum_n x_n$, and $Ax = y$ by the above estimates. Also
    $$||x|| = \lim_{n \to \infty} \left|\left| \sum_{k\leq n} x_k\right|\right| \leq \sum_{n=1}^\infty x_n < 2L.$$
    So $y \in A(2LU)$. Therefore $V \subseteq A(2LU)$ which was to be shown.
\end{proof}
\begin{theorem}[closed graph theorem]
    Let $A: X \to Y$ be a linear mapping between Banach spaces. If the graph of $A$ is closed in $X \oplus Y$, then $A$ is continuous.
\end{theorem}
    Notice that while there isn't a canonical norm for $X \oplus Y$, any $\ell^p$ norm will do; since $X \oplus Y$ is a finite direct sum, all $\ell^p$ norms are equivalent. In particular, $X \oplus Y$ is a Banach space.
\begin{proof}
    Let $\Gamma$ be the graph of $A$, which is equipped with a natural (linear, bijective) projection $\pi_X: \Gamma \to X$. Since
    $$||P(x, Ax)|| = ||x|| \leq ||(x, Ax)||,$$
    $||P|| \leq 1 < \infty$. So by the open mapping theorem,
    $$||Tx|| \preceq ||P^{-1}x|| + ||x|| \preceq ||x||.$$
\end{proof}


\chapter{Complex analysis}
Throughout, we identify $\RR^2 = \CC$ and write $z = x + iy$, $dz = dx + idy$, so that $d\overline z = dx - i dy$. Then
$$dz \wedge d\overline z = 2i dx \wedge dy = 2i dA.$$
We thus write $2 \partial f = \partial_x f - i\partial_y f$ and $2\overline f = \partial_x f + i\partial_y f$, so that
$$df = \partial f ~dz + \overline \partial f ~d\overline z.$$
We let $K$ be a compact set in an $\Omega$-precompact open set $\omega$, where $\Omega$ is open in $\CC$. So we have inclusions
$$K \subset \omega \subset \overline \omega \subset \Omega \subseteq \CC.$$
We will always assume that $\partial \omega$ is a positively oriented, piecewise-$C^1$ Jordan curve.

\section{Cauchy-Green formula}
Making the change of variable $dA \mapsto dz \wedge d\overline z$ in Green's formula, we arrive at the following generalization of the Cauchy integal formula.
\begin{theorem}[Cauchy-Green]
    \index{Cauchy-Green formula}
    Let $f \in C^1(\omega)$. For each $\zeta \in \omega$,
    $$f(\zeta) = \frac{1}{2\pi i}\left(\int_{\partial\omega} \frac{f(z)}{z - \zeta}dz + \iint_\omega \frac{\overline \partial f(z)}{z - \zeta} ~dz \wedge d\overline z\right).$$
\end{theorem}
\begin{definition}
    The \dfn{Cauchy-Riemann equation} is the equation
    $$\overline \partial f = 0.$$
    If $f \in C^1(\omega)$ solves the Cauchy-Riemann equation, we say that $f$ is a \dfn{holomorphic function}, written $f \in A(\omega)$.
\end{definition}
So in case $f$ is holomorphic, we recover the classical Cauchy integral formula from the Cauchy-Green theorem.
\begin{theorem}
    Let $\mu$ be a finite Borel measure on $\CC$ with compact support $K$. Let
    $$u(\zeta) = \int_\CC \frac{d\mu(z)}{z - \zeta}.$$
    Then $u$ is holomorphic on $K^c$. If $\varphi \in C^k(\omega)$ and $2\pi i \mu = \varphi ~dz\wedge d\overline z$, then $u \in C^k(\omega)$ and $\overline u = \varphi$ on $\omega$. In particular, if $\mu$ solves the Cauchy-Riemann equation in the distributional sense, then $\mu$ is holomorphic.
\end{theorem}
\begin{corollary}[Cauchy]
    \index{Cauchy's inequality}
    If $u \in A(\Omega)$ then
    $$||\partial^j u||_{L^\infty(K)} \preceq_{K,j} ||u||_{L^1(\omega)}.$$
\end{corollary}
\begin{proof}
    Differentiate under the integral sign in the Cauchy-Green formula, and then use the triangle inequality.
\end{proof}
\begin{corollary}
    If we are given a sequence of $u_n \in A(\Omega)$ which converges locally uniformly to $u$, then $u \in A(\Omega)$.
\end{corollary}
\begin{proof}
    This is obvious for uniform convergence, but holomorphy is a local property.
\end{proof}
\begin{corollary}[Montel]
    \index{Montel's theorem}
    If we are given a sequence in $A(\Omega)$ which is locally bounded, then there is a locally uniformly convergent subsequence, whose limit is in particular holomorphic.
\end{corollary}
\begin{proof}
    Use Arzela-Ascoli on Cauchy's inequality. Then use locally uniform convergence.
\end{proof}
\begin{corollary}[root test]
    \index{root test}
    If $u(z) = \sum_n a_nz^n$, then $u$ is analytic on $D(0, \limsup_n |a_n|^{1/n})$.
\end{corollary}
\begin{corollary}[Taylor]
    \index{Taylor's theorem}
    If $u \in A(D(0, R))$, then $u \in C^\infty(D(0, R))$ and for $z \in D(0, R)$ we have
    $$u(z) = \sum_{n=0}^\infty \frac{\partial^n u(0)}{n!} z^n.$$
\end{corollary}
\begin{proof}
    Differentiate under the integral sign in the Cauchy-Green formula, then use the root test.
\end{proof}
\begin{corollary}
    If $\Omega$ is connected and $u \in A(\Omega)$, and there is a $z \in \Omega$ such that for all $j$, $\partial^ju(z) = 0$, then $u = 0$.
\end{corollary}
\begin{proof}
    Taylor series propagate to connected components.
\end{proof}
\begin{corollary}[Weierstrass preparation theorem]
    \index{Weierstrass preparation theorem}
    If $u \in A(\Omega)$, $0 \in \Omega$, $u \neq 0$, and $u(0) = 0$ with order $k$, then there is a $v \in A(\Omega)$ so that $u(z) = z^k v(z)$.
\end{corollary}
\begin{proof}
    Factor the $z^k$ out of the Taylor series.
\end{proof}
\begin{corollary}
    If $u \in A(\omega) \cap C(\overline \omega)$ then
    $$||u||_{L^\infty(\omega)} = ||u||_{L^\infty(\partial \omega)}.$$
\end{corollary}

\section{Conformal mappings}
\begin{definition}
\index{function!biholomorphic}
\index{function!conformal}
\index{function!angle-preserving}
\index{set!conformal}
\index{set!conformally equivalent}
\index{complex diffeomorphism}
A function between open sets $f: U \to V$ is \emph{biholomorphic}, \emph{conformal}, \emph{angle-preserving}, or a \emph{complex diffeomorphism} if $f$ is a bijection and $f$ and $f^{-1}$ are both holomorphic.

If such a conformal map exists then $U$ and $V$ are \emph{conformally equivalent} or \emph{conformal}.
\end{definition}


\begin{lemma}
\label{rotation of the disk}
Let $F: \DD \to \DD$ be conformal and $F(0) = 0$. Then $\exists \omega \in S^1$ such that for each $z \in \DD$,
$$F(z) = \omega z.$$
\begin{proof}
Apply Schwarz to $F$ and $F^{-1}$ so that
$$|z| \leq |F(z)| \leq |z|.$$
\end{proof}
\end{lemma}

\begin{theorem}
\label{automorphism of the disk}
\index{automorphism of the disk}
$F: \DD \to \DD$ is conformal if and only if there exist unique $a \in \DD$ and $\omega \in S^1$ such that
$$F(z) = \omega \frac{z - a}{1 - \overline a z}.$$
Moreover, $a$ can be chosen so that $F(a) = 0$, and
$$F^{-1}(z) = \omega^{-1} \frac{z + a}{1 + \overline a z}.$$
\begin{proof}
Suppose that $F$ is conformal and let $G: \DD \to \DD$ be given by
$$G(z) = \frac{z - a}{1 - \overline a z}.$$
Then since $|\overline a z| < 1$, $G \circ F^{-1}$ is conformal and $G(a) = 0$, thus $G \circ F^{-1}(0) = 0$. So by \ref{rotation of the disk}, $G \circ F^{-1}(z) = \omega z$ for some $\omega \in S^1$. Existence of $\omega$ is immediate (how could one rotate the disk at two different speeds?) and since $a$ is determined by the preimage of $0$, $a$ is unique as well.

On the other hand, if
$$F(z) = \omega \frac{z - a}{1 - \overline a z},$$
then since $|\overline a z| < 1$, $F$ is holomorphic, as is its inverse (that it actually is an inverse is immediate).
\end{proof}
\end{theorem}

\begin{theorem}[Riemann mapping theorem]
\index{Riemann mapping theorem}
If $U \subseteq \CC$ is open and simply connected, then $U$ is conformal with $\DD$ or $\CC$.
\end{theorem}

\begin{lemma}
\label{riem 1}
\index{Riemann mapping theorem!proof}
Suppose that $U \neq \CC$ is simply connected and $z_0 \in U$. Then the space
$$\FF = \{f: U \to \DD \, | \, f \text{ is holomorphic and injective, and } f(z_0) = 0\}$$
is a nonempty normal family.
\begin{proof}
For convenience, let $V$ be the strip
$$V = \{z \in \CC: |\Im z| < \pi\}.$$
Let $a \notin U$ and observe that $z - a \neq 0$. Since $\exp$ has a period of $2\pi i$, $\exp$ is a conformal map $V \to \CC \setminus \{0\}$. In particular, its inverse $\log$ is conformal $\CC \setminus \{0\} \to V$. Now let $\ell(z) = \log(z - a)$; $\ell$ is holomorphic and injective.

Suppose that there does not exist $\delta > 0$ such that for each $z \in U$,
$$|\ell(z) - \ell(z_0) - 2\pi i| < \delta.$$
Then there is a sequence $z_n$ in $U$ such that $\ell(z_n) \to \ell(z_0) + 2\pi i$. By continuity of $\exp$, $z_n \to z_0 e^{2\pi i} = z_0$. So $\ell(z_n) \to \ell(z_0) \neq \ell(z_0) + 2\pi i$, which is a contradiction, so $\delta$ exists.

Let
$$g(z) = \frac{1}{\ell(z) - \ell(z_0) - 2\pi i}.$$
Then $g$ is clearly injective and holomorphic and $|g| < \delta^{-1}$, so $g$ is bounded, and
$$f(z) = \frac{g(z) - g(z_0)}{\delta^{-1} + |g(z_0)|}$$
satisfies $f \in \FF$. Thus $\FF$ is nonempty.

Moreover, each $f \in \FF$ satisfies $|f| < 1$, so $\FF$ is uniformly bounded by $1$; therefore $\FF$ is normal by Montel's theorem.
\end{proof}
\end{lemma}

\begin{lemma}
\label{riem 2}
With $\FF$ and $z_0$ as in \ref{riem 1}, there exists a conformal $F \in \FF$ and if
$$\lambda = \sup_{f \in \FF} |f'(z_0)|$$
then $|F'(z_0)| = \lambda$.
\end{lemma}

$\FF$ is precompact in $\OO$, but this doesn't mean that $\FF$ is compact! So, while we want to use compactness to prove the existence of a function $F$ that satisfies the desired hypotheses, we'll have to prove that in fact $F \in \FF$ rather than $F \in \partial \FF$.

\begin{proof}[Proof of \ref{riem 2}]
Since elements of $\FF$ are injective, their derivatives are nonzero by the argument principle, so $\lambda > 0$, and there is a sequence $f_n$ such that $|f_n'(z_0)| \to \lambda$. By Montel's theorem, it has a convergent subsequence with a limit $F \in \OO$, such that $F(z_0) = 0$ and $F'(z_0) = \lambda$ by Weierstrass' theorem. Moreover, $|F| \leq 1$. $F$ is nonconstant since $\lambda > 0$, so by the open mapping theorem, $|F| < 1$, and by Hurwitz' theorem, $F$ is injective. Therefore $F \in \FF$.

Moreover, if $F$ is not surjective, then there exists $w \in \DD \setminus F(U)$. Let
$$\psi(z) = \frac{w - z}{1 - \overline wz}.$$
By \ref{automorphism of the disk}, $\psi$ is an automorphism of $\DD$, $\psi(w) = 0$ and $|\psi \circ F| > 0$. Now define
$$g(z) = \exp\left(\frac{\log \psi(F(z))}{2}\right).$$
Then $g(z_0) = \sqrt w$.
By \ref{automorphism of the disk} again,
$$\tilde \psi(z) = \frac{\sqrt w - z}{1 - \overline{\sqrt w} z}$$
is also an automorphism of $\DD$ and $\tilde \psi(\sqrt w) = 0$. Let $G = \tilde \psi \circ g$. Then $G \in \FF$.

Now let $s(z) = z^2$ and $\varphi = \psi^{-1} \circ s \circ \tilde \psi^{-1}$. Then $\varphi(0) = 0$ so by Schwarz $|\varphi(z)| \leq |z|$. Moreover, $\varphi$ is not injective anywhere, so $|\varphi'| < 1$. But $F = \varphi \circ G$ so
$$\lambda = |F'(z_0)| = |\varphi'(0) G'(z_0) | < |G'(z_0)| \leq \lambda$$
which is a contradiction.

So $F$ is surjective. Therefore the inverse function theorem implies that $F$ is conformal.
\end{proof}
Clearly this lemma implies the Riemann mapping theorem.




\section{Approximation by polynomials}
Let $K \subset \Omega \subseteq \CC$ be compact. If $K$ is the compactification of a disc, then it is easy to uniformly approximate holomorphic functions on $K$ by polynomials, just by truncating the Taylor series.

\begin{example}
    Let $K$ be the compactification of an annulus and assume $u$ has a pole in the center of $K$. If $p_j \to u$ on $K$ and the $p_j$ are entire functions, then in particular $p_j \to u$ on $\partial K$, so they are a Cauchy sequence in $A(K)$, so on a disk containing $K$, and in particular the pole of $u$. Therefore $p_j \to p$, an entire function, even though $u$ has a pole. Notice that we can still approximate $u$ by meromorphic functions, though.
\end{example}
\begin{theorem}[Runge approximation theorem]
    \index{Runge approximation theorem}
    The following are equivalent.
\begin{enumerate}
    \item If $u$ is holomorphic near $K$, then there are functions $u_j \in A(\Omega)$ such that $u_j \to u$ uniformly on $K$.
    \item The complement $\Omega \setminus K$ has no $\Omega$-precompact connected components.
    \item For each $z \in \Omega \setminus K$ there is an $f \in A(\Omega)$ such that
    $$|f(z)| > ||f||_{L^\infty(K)}.$$
\end{enumerate}
\end{theorem}
\begin{proof}
    Let us first prove that not-2 implies not-3 and not-1.

    If not-2, then there is a $K$-precompact component $O$ of $\Omega \setminus K$, so $\partial O \subseteq K$. If $f \in A(\Omega)$, then
    $$||f||_{L^\infty(O)} = ||f||_{L^\infty(\partial O)} \leq ||f||_{L^\infty(K)}$$
    by the maximum principle, implying not-3. Moreover, if 1 were true, then for every $f$ holomorphic near $K$, we could approximate $f$ by $f_j \in A(\Omega)$ uniformly. We have
    $$||f_j - f_k||_{L^\infty(\overline O)} \leq ||f_j - f_k||_{L^\infty(K)},$$
    so the $f_j$ form a Cauchy sequence in $L^\infty(K)$, which converge to a holomorphic function $F \in A(O)$. But if $f$ has a pole in $O$, then $f \neq O$, a contradiction, so we have not-1.

    Now we show 1 and 2 imply 3. Let $L = K \cup \overline{D(z, \varepsilon)}$ where $\epsilon < d(z, K)$. Then every component of $L$ is either a component of $K$ or else $\overline{D(z, \varepsilon)}$, and $L$ satisfies the hypotheses of 2, so in particular satisfies the hypotheses of 1. Let $u \in A(L)$ be given by $u = 1$ near $z$ and $u$ on $K$. By 1, $u$ is uniformly approximable in $A(\Omega)$. Assume $f \in A(\Omega)$ is such that $||f - u||_{L^\infty(L)} < \delta$ for $\delta$ small enough; then $f$ witnesses 3.

    Finally we show 2 implies 1. Let $X_2$ be the set of restrictions of holomorphic functions to $K$, and let $X_1$ be the set of restrictions of holomorphic functions on $\Omega$ to $K$. Then $X_1 \subset X_2 \subset C(K)$, and 1 holds iff $\overline{X_1} = \overline{X_2}$. So we must show $\overline{X_2} \subseteq \overline{X_1}$. By the Hanh-Banach and Riesz-Markov theorems, this is equivalent to showing that for every finite Borel measure $\mu$ with support in $K$ and every $f \in A(\Omega)$, $\int f ~d\mu = 0$. In particular, we can prove this with the addition assumption that $f$ is only holomorphic near $K$. So fix such a $f, \mu$.

    Let $\varphi$ be the holomorphic function given by $\mu$. Since $\varphi = 0$ on $\CC \setminus \Omega$, $\varphi = 0$ on any component of $\CC \setminus K$. Moreover,
    $$\frac{1}{z - \zeta} = -\sum_{j=0}^\infty \zeta^{-j-1}z^j$$
    whenever the sum converges, i.e. for $|\zeta|$ large enough. Therefore
    $$\varphi(\zeta) = -\sum_{j=0}^\infty \zeta^{-j-1} \int z^j ~d\mu(z) = 0$$
    for $|\zeta|$ large enough. By 2, $\Omega \setminus K$ has no $\Omega$-precompact components, so every component of $\Omega \setminus K$ touches $\partial \Omega$ or is unbounded. Therefore $\varphi = 0$ on $\CC \setminus K$.

    Let $\psi$ be a cutoff which is $1$ whenever $f$ is holomorphic. Then, taking $\omega = \CC$, we have
    $$\psi(z) = \frac{1}{2\pi i}\iint_\CC \frac{\overline \partial \psi(\zeta)}{\zeta - z} d\overline \zeta \wedge d\zeta.$$
    For each $z \in K$, the function
    $$\zeta \mapsto \frac{\overline \partial \psi(\zeta) f(\zeta)}{\zeta - z}$$
    is smooth since $\zeta \notin K$. So we are entitled to use Fubini's theorem to prove
\begin{align*}
    \int_\CC f ~d\mu &= \int_\CC \psi f ~d\mu
        = \frac{1}{2\pi i} \iiint_{\CC^2} \frac{\overline \partial \psi(\zeta) f(\zeta)}{\zeta - z} ~d\overline \zeta \wedge d\zeta ~d\mu(z)\\
        &= \frac{1}{2\pi i} \iint_\CC f(\zeta) \overline \partial \psi(\zeta) \int_\CC \frac{d\mu(z)}{\zeta - z} ~d\overline \zeta \wedge d\zeta
        = 0.
\end{align*}
    This proves 1.
\end{proof}
\begin{corollary}
    Let $K \subset \CC$ be compact, such that $\CC \setminus K$ is connected. Every function which is holomorphic near $K$ can be approximated by polynomials uniformly on $K$.
\end{corollary}
\begin{definition}
    The \dfn{holomorphically convex hull} of $K$ in $\Omega$, written $\hat K$, is the set of $z \in \Omega$ such that for every $f \in A(\Omega)$,
    $$|f(z)| \leq ||f||_{L^\infty(K)}.$$
    If $K = \hat K$, we say that $K$ is \dfn{holomorphically convex}.
\end{definition}
\begin{example}
    Let $K$ be as in the previous example; then $\hat K$ is the outer disc in the definition of $K$ as an annulus.
\end{example}
It is easy to check that
$$d(K, \CC \setminus \Omega) = d(\hat K, \CC \setminus \Omega).$$

Recall that $\ch K$ denotes the convex hull of $K$. If $K$ is convex, then $K$ is topologically $\overline{D(0, 1)}$, which is clearly holomorphically convex by Runge's approximation theorem. But the connection between convexity and holomorphic convexity is stronger than that.
\begin{proposition}
    For any $K$, $\hat K \subseteq \ch K$.
\end{proposition}
\begin{proof}
    It is easy to check that $\ch K$ is the intersection of half-planes
    $$H_{a,c} = \{z \in \CC: \Re(az) \leq c\}.$$
    Fix $a \in \CC$, $c \in \RR$; we will show $\hat K \subset H_{a,c}$. Let $z \in \hat K$. Then $|e^{az}| \leq \max_{w \in K} |e^{aw}|$. So $\Re e^{az} \leq \max_{w \in W} \Re |e^{aw}|$, implying
    $$\Re az \leq \max_{w \in K} aw \leq c.$$
    Therefore $z \in \hat K$.
\end{proof}
Moreover, $\hat K$ is the union of $K$ with all $\overline O$, for each $\CC$-precompact connected component $O$ of $\Omega \setminus K$.

\begin{definition}
    The \dfn{polynomially convex hull} of a compact set $K$ is the set
    $$\hat K = \{z \in \CC: |p(z)| \leq \max_{w \in K} |p(w)| \text{for every polynomial} p\}.$$
    If $K = \hat K$, then $K$ is \dfn{polynomially convex}.
\end{definition}

\section{Sheaves}
Let $X$ be a topological space. By $\Open(X)$ we will denote the posetal category of open sets in $X$; that is, objects are open sets in $X$ and morphisms are inclusions.

\begin{definition}
    A \dfn{presheaf} on $X$ is a functor $\Open(X)^{op} \to C$ for some concrete category $C$.

    If $U \subseteq X$ is an open set, and $\mathcal F: \Open(X)^{op} \to C$ a presheaf on $X$, then elements of $\mathcal F(U)$ are called \dfn{sections} of $\mathcal F$ at $U$. Sections of $\mathcal F$ at $X$ are called \dfn{global sections}.
\end{definition}
To motivate this terminology, let us assume that we are given a fiber bundle $\pi: E \to X$ (the most trivial case of this is when $E = X \times Y$ for some space $Y$, and $\pi$ is projection onto the first factor; as with any notion of bundle, the point is that $E$ is locally a product space.) Then the sections of $\pi$, i.e. the sets $\pi^{-1}(U)$, are exactly a sheaf (to be defined later), and thus a presheaf, $\Open(X)^{op} \to \Set$.

Now notice that every morphism in $\Open(X)^{op}$ is epic, so it makes sense to define the restriction maps $f \mapsto f|_V$ for sections $f \in \mathcal F(U)$ and $V \subseteq U$.

\begin{definition}
    Let $\mathcal F: \Open(X)^{op} \to C$ be a presheaf on $X$. Assume that for every open set $U \in \Open(X)$ and every open cover $\{U_j\}$ of $U$ we have the following conditions:
\begin{enumerate}
    \item For every pair of sections $f,g \in \mathcal F(U)$, if we have $f|_{U_i} = g|_{U_i}$ for every $i$, then $f = g$.
    \item If for every $i$ we have a section $f_i \in \mathcal F(U_i)$ such that on intersections, $f_i|_{U_i \cap U_j} = f_j|_{U_i \cap U_j}$, then there is a section $f \in \mathcal F(U)$ such that for every $i$, $f|_{U_i} = f_i$.
\end{enumerate}
    Then we say that $\mathcal F$ is a \dfn{sheaf} on $X$.
\end{definition}
\begin{proposition}
    Let $\mathcal F$ be a sheaf which is only defined on an open base, but otherwise satisfying all the conditions. Then $\mathcal F$ uniquely determines a sheaf on the entire topology.
\end{proposition}

    We now put sheaves into a category.
\begin{definition}
    Let $\mathcal F, \mathcal G: \Open(X)^{op} \to C$ be presheaves. A \dfn{morphism of presheaves} over $X$ is a natural transformation $\psi: \mathcal F \to \mathcal G$. If $\mathcal F$ and $\mathcal G$ are sheaves, then $\psi$ is a \dfn{morphism of sheaves}.
\end{definition}
That is, a morphism of sheaves $\psi: \mathcal F \to \mathcal G$ consists of, for each open set $U \subseteq X$, a morphism $\psi(U) \in \Hom(\mathcal F(U), \mathcal G(U))$ such that if $U \subseteq V$ for some open set $V \subseteq X$, then the diagram
$$\begin{tikzcd}
\mathcal F(V) \arrow[r,"\psi(V)"] \arrow[d] &\mathcal G(V) \arrow[d]\\
\mathcal F(U) \arrow[r,"\psi(U)"] &\mathcal G(U)
\end{tikzcd}$$
commutes.

In several complex variables, we are interested in holomorphic germs. The following family of definitions allows us to talk about germs algebraically.
\begin{definition}
    Let $C$ be a category such that for every directed set $\mathcal D$ in $C$, a colimit exists at $\mathcal D$, and let $x \in X$. Let $\mathcal F: \Open(X)^{op} \to C$ be a sheaf. The \dfn{stalk} of $\mathcal F$ at $x$ is the colimit
    $$\mathcal F_x = \varinjlim_{U \in \mathcal D_x} \mathcal F(U)$$
    where $\mathcal D_x$ is the directed set of all open sets $U \ni x$. An element of $\mathcal F_x$ is called a \dfn{germ} of $\mathcal F$ at $x$.
\end{definition}




\section{Subharmonicity}
\begin{definition}
    Let $X$ be a topological space. An \dfn{upper-semicontinuous function} on $X$ is a function $u: X \to [-\infty, \infty)$ such that for each $s \in \RR$, the preimage of the ray $[-\infty, s)$ is open in $X$.
\end{definition}
Notice that there is a dual notion of lower-semicontinuity, by considering the rays $(s, \infty]$. A function which is both upper- and lower-semicontinuous is just continuous, since open rays form a base of the topology of $\RR$.

\begin{definition}
    Let $\Omega \subseteq \RR^n$ be an open set. A \dfn{subharmonic function} on $\Omega$ is a upper-semicontinuous function $u: \Omega \to [-\infty, \infty)$ such that for every compact set $K \subseteq \Omega$ and every continuous function $h: K \to \RR$ which is harmonic in $K$, if $h \leq u$ on $\partial K$, then $h \leq u$ on $K$.
\end{definition}
This definition makes just as much sense in $\CC^n$, or even on a Riemannian manifold; one just needs a Laplace-Beltrami operator $\Delta_g$, so that we have a notion of harmonicity $\Delta_g h = 0$. By the maximum modulus principle, a harmonic function is already subharmonic (since it cannot attain its maximum on the boundary of a compact set).

Let $u$ be subharmonic. Then if $c > 0$, $cu$ is subharmonic (simply by replacing $h$ with $ch$ for each $h$ in the definition of subharmonicity). Moreover, if $A$ is a set of subharmonic functions and $u = \sup A$, then $u$ is subharmonic provided that $u$ is upper-semicontinuous (simply by considering the $h$ such that $v \leq h$ for every $v \in A$, which exist since $u$ is upper-semicontinuous and so finite).

\begin{proposition}
Let $u_1, \dots$ be a decreasing sequence of subharmonic functions. Then $u = \lim_j u_j$ is subharmonic.
\end{proposition}
\begin{proof}
    Note that
    $$\{z \in \Omega: u(z) < s\} = \bigcup_j \{z \in \Omega: u_j(s)\}$$
    is open so $u$ is upper-semicontinuous. If $h, K$ are as in the definition of subharmonicity and $\varepsilon > 0$ then the set
    $$\{z \in \partial K: u_j(z) \geq h(z) + \varepsilon\}$$
    is compact and decreasing as $j \to \infty$. The intersection of nonempty compact sets is nonempty, but the intersection is empty by definition of $H$, so there the sequence is eventually empty. Therefore $u_j \leq h + \varepsilon$ for $j$ large enough. So $u \leq h$.
\end{proof}

Now we consider equivalent definitions of subharmonicity.
\begin{proposition}
Let $u$ be an upper-semicontinuous function on $\Omega \subseteq \CC$. Let $\delta > 0$ and let $\Omega_\delta = \{z \in \Omega: d(z, \Omega^c) > \delta\}$. The following are equivalent:
\begin{enumerate}
    \item $u$ is subharmonic.
    \item If $D \subseteq \Omega$ is a compact disk, and $f$ is a polynomial such that $u \leq f$ on $\partial D$, then $u \leq f$ in $D$.
    \item For each $z \in \Omega_\delta$,
    $$2\pi u(z) \leq \int_0^{2\pi} u(z+re^{i\theta}) ~d\theta.$$
    \item For each positive measure $\mu$ on $[0, \delta]$ and each $z \in \Omega_\delta$,
    \begin{equation}
        \label{subharmonic equation in C}
        2\pi \mu([0, \delta]) u(z) \leq \int_0^\delta \int_0^{2\pi} u(z + re^{i\theta}) ~d\theta ~d\mu(r).
    \end{equation}
    \item For each $z \in \Omega_\delta$ there is a positive measure $\mu$ on $[0, \delta]$, such that Equation \ref{subharmonic equation in C} holds and such that $\mu((0, \delta]) > 0$.
\end{enumerate}
\end{proposition}
\begin{proof}
    Obviously 1 implies 2 and 3 implies 4 implies 5.

    Assume 2. To prove 3, let $z \in \Omega_\delta$ and $r \leq \delta$. Let $D$ be the disk of all $\zeta$ such that $|\zeta - z| \leq r$. If $\varphi(\theta) = \sum_k a_k e^{ik\theta}$ is a trigonometric polynomial such that $u(z + re^{i\theta}) \leq \varphi(\theta)$ for every $\theta \in [0, 2\pi]$, then $f(\zeta) = a_0 + 2\sum_{k\geq 1} a_k(\zeta - z)^k/r^k$ has $u \leq \Re f$ on $\partial D$, so on $D$ since $f$ is a polynomial. Plugging in $\theta = 0$, we have
\begin{equation}
    \label{subharmonic in C proof}
    u(z) \leq a_0 = \frac{1}{2\pi} \int_0^{2\pi} \varphi(\theta) ~d\theta.
\end{equation}
    Since the trigonometric polynomials are an algebra, they are dense in the space of continuous functions. Therefore Equation \ref{subharmonic in C proof} holds for any continuous $\varphi$. This proves 3.

    Assume 5 and let $h, K$ be as in the definition of subharmonicity. If $M = \sup u - h > 0$ then $u - h = M$ on some nonempty compact set $K_0$ by semicontinuity of $u - h$. Let $z_0 \in K_0$. Then
    $$\int_0^{2\pi} \int_0^\delta (u-h)(z_0 + re^{i\theta}) ~d\mu(r) ~d\theta < 2\pi (u-h)(z_0) \mu([0, \delta)).$$
    This is a contradiction of 5, so $u$ is subharmonic.
\end{proof}
    It follows that the class of subharmonic functions is closed under addition, and subharmonicity is a local property. Moreover, if $f$ is holomorphic on $\Omega$, it follows that $\log|f|$ is subharmonic: by the maximum modulus principle, $|f|$ does not attain its maximum on a compact disk $D$.

\begin{theorem}
    Assume $\Omega \subseteq \CC$ is open and connected and $u$ is subharmonic on $\Omega$ is not identically $-\infty$. Then $u \in L^1_{loc}(\Omega)$ and for any $v \in C^2_{comp}(\Omega)$, $v \geq 0$, $\langle u, \Delta v\rangle \geq 0$. If $u \in C^2(\Omega)$, then $\Delta u \geq 0$.
\end{theorem}
\begin{theorem}
    Suppose $u \in L^1_{loc}(\Omega)$ and for every $v \in C^2_{comp}(\Omega)$, $v \geq 0$, $\langle u, \Delta v \rangle \geq 0$. Then, up to a null set, $u$ is subharmonic. In particular, the mollification of $u$ is subharmonic.
\end{theorem}
From the above theorem we see that for $u \in L^1_{loc}$, then $u$ is subharmonic exactly when $\Delta u \geq 0$ in the weak sense.



\section{Operator theory}
\begin{theorem}[Schur]
\label{Schur}
Let $X$ be a $\sigma$-finite measure space and let $K$ be an integral operator. If there is a $p: X \to [0, \infty)$ and a $\lambda > 0$ such that
$$\int_X |K(x, y)|p(y) ~dy \leq \lambda p(x)$$
and
$$\int_X |K(x,y)| p(x) ~dx \leq \lambda p(y)$$
then
$$||K||_{L^2 \to L^2} \leq \lambda.$$
\end{theorem}
\begin{proof}
Let $u$ be an integrable simple function. By the Cauchy-Schwarz inequality and Fubini's theorem,
\begin{align*}
    ||Ku||^2_{L^2} &= \int_X \left|\int_X K(x, y)u(y) ~dy\right|^2 ~dx &\leq \left|\iint_{X^2} |K(x, y)|^{1/2}p(y)^{1/2} \frac{|u(y)|}{p(y)^{1/2}} |K(x, y)|^{1/2} ~dx ~dy\right|\\
    &\leq \int_X\left(\int_X |K(x, y)| p(y) ~dy\int_X \frac{|u(y)|^2}{p(y)}|K(x, y)| ~dy\right)~dx\\
    &\leq \lambda \int_X \frac{p(x)}{p(y)}|K(x, y)| |u(y)|^2 ~dy ~dx\\
    &\leq \lambda^2 |u(y)|^2 ~dy = \lambda^2 ||u||_{L^2}^2.
\end{align*}
Since ISF is dense in $L^2$ we're done.
\end{proof}
As a corollary we can easily compute the $\ell^2$ norm of a matrix, since every matrix is an integral operator for counting measure on $\{1, 2, \dots, n\}$.

\begin{theorem}
\label{norm of the resolvent}
Let $A$ be a self-adjoint operator and
$$R_A(\lambda) = (A - \lambda)^{-1}$$
its resolvent. Then we have
$$||R_A(\lambda)|| = \frac{1}{d(\lambda, \sigma(A))}.$$
\end{theorem}
Notice the utility of this theorem: we do not assume that $A$ is a bounded linear operator, so we cannot use the spectral radius theorem.
\begin{proof}
Since $\lambda$ is in the resolvent set, $d(\lambda, \sigma(A)) > 0$. We can assume $A$ is unbounded; therefore $R_A$ is bounded, and its spectrum consists of $0$ and the set of all $1/(\mu - \lambda)$, for $\mu \in \sigma(A)$. By the spectral radius theorem,
$$||(A - \lambda)^{-1}|| = \sup_{\mu \in \sigma(A)} |\lambda - \mu|^{-1}$$
which proves the claim.
\end{proof}

\chapter{Elementary dynamical systems}
\begin{definition}
By a \dfn{discrete dynamical system} we mean a transformation $T: X \to X$.
\end{definition}
We think of $T^n(X)$ as the state of $X$ at time $n$.
\begin{definition}
By a \dfn{continuous dynamical system} we mean a family of transformations $\varphi_t: X \to X$ satisfying the homomorphism assumption $\varphi_{t+s} = \varphi_t\varphi_s$.
\end{definition}
We can always build a continuous system from a discrete one and vice versa. Because discrete systems are much easier to study, we usually try to reduce the study of dynamical systems to the study of discrete dynamical systems.


\section{Types of dynamical systems}
\begin{definition}
A \dfn{periodic point} of a dynamical system $T$ is a $x$ such that there is a $t$ with $T^t(x) = x$.
\end{definition}
In dynamical systems, we want to know how many periodic points are there in a dynamical system. This is too specific so it is also common to look for recurrence: if a trajectory $x$ starts at $x_0$, how often does $x_n$ approximate $x_0$?

\begin{definition}
An \dfn{invariant set} $Y \subseteq X$ of a dynamical system $T: X \to X$ is a set such that $T^{-1}(A) = A$.
\end{definition}
\begin{definition}
An \dfn{invariant measure} $\mu$ (defined on a $\sigma$-algebra $\Sigma$) is one such that for every measurable set $A \in \Sigma$, $T^{-1}(A) \in \Sigma$ and $\mu(T^{-1}(A)) = \mu(A)$.
\end{definition}
In dynamical systems, we want to study invariant sets and invariant measures. The reason why we study the pullback in the definition of invariant measure is that $A \mapsto \mu(T^{-1}(A))$ is always a measure, but if $X = \{0, 1\}$, $T(0) = T(1) = 0$, $\mu$ counting measure, then $A \mapsto \mu(T(A))$ is not a measure.

We also will study structural stability, i.e. when a small perturbation of $T$ preserves the properties of $T$.
\begin{example}
KAM theory implies that the solar system is at least approximately structurally stable. Therefore the solar system will not collapse in our lifetimes.
\end{example}

\begin{definition}
Let $T: X \to X$ and $S: Y \to Y$ be dynamical systems. Then $S$ is \dfn{semiconjugate} to $T$, or that $T$ is a \dfn{factor} of $S$) if there is a surjective $\pi: Y \to X$ such that $T \circ \pi = \pi \circ S$. If $\pi$ is actually invertible, then $S$ and $T$ are \dfn{conjugate}.
\end{definition}
As far as set theory is concerned, conjugacy means that $S$ and $T$ are identical. If we may assume that $\pi$ is measure-preserving, smooth, etc., then it will follow that $S$ and $T$ are identical in the appropriate categories.
\begin{example}
Let $T: S^1 \to S^1$ be the dynamical system which rotates the circle by $2\pi\alpha$ for some $\alpha \in \RR$. Let $S: \RR^2 \setminus 0 \to \RR^2 \setminus 0$ to be the rotation of the punctured plane by $2\pi\alpha$. Then $T$ is a factor of $S$, witnessed by the transformation $\pi: \RR^2 \setminus 0 \to S^1$ which sends a point to its projection onto the circle.

We are mainly interested in $T$ when $\alpha$ is irrational (in which case $T$ is called the \dfn{irrational rotation}), in which case $T$ has no periodic points. We will show that there are no interesting examples of invariant sets for $T$, and in fact if we restrict to Borel measures, there is exactly one invariant measure of $T$, namely the Lebesgue measure. There is no structural stability because the rational numbers are dense in $\RR$.
\end{example}

We branch off into different subfields. Let $T: X \to X$ be a transformation. If $X$ is a metric space and $T$ is continuous, then we are studying \dfn{topological dynamics}. If $X$ is a smooth manifold and $T$ is smooth, then we are studying \dfn{smooth dynamics}. In this case, we have an auxiliary dynamical system defined by the differential form $df: TX \to TX$ which sends $T_xX \to T_{f(x)}X$. Finally, if $X$ is a measure space and $\mu$ is an invariant measure, then we are studying \dfn{ergodic theory}. Ergodic theory will be one of the main themes of this course.

One can also study \dfn{holomorphic dynamics}, where $T: \CC \to \CC$ is a holomorphic function. One can ``generalize" this to the study of rational functions from a variety to itself.

We now introduce Hamiltonian systems.
\begin{definition}
Let $X$ be a smooth manifold and let $\omega$ be a nondegenerate $2$-form such that $d\omega = 0$. Then we say that $(X, \omega)$ is a \dfn{symplectic manifold}.
\end{definition}
\begin{definition}
Let $X$ be a symplectic manifold. For a function $f: X \to \RR$, let $H_f$ be the vector field defined by the relation $\omega(\cdot, H_f) = df$. Then we define the \dfn{Hamiltonian dynamical system} on $X$ by the ordinal differential equation
$$\dot \rho(t) = H_f(\rho(t))$$
where $\rho(0)$ is given.
\end{definition}
In this case, the measure $\omega^n/n!$ is an invariant measure of $\omega$.
\begin{example}
Let $X$ be the cotangent space of $\RR^n$ and let $\omega = \sum_j d\xi_j \wedge dx_j$. Then $(X, \omega)$ is a symplectic manifold and
$$H_f = \sum_j \frac{\partial f}{\partial \xi_j} \partial x_j - \frac{\partial f}{\partial x_j} \partial_{\xi_j}.$$
\end{example}
\begin{example}
We show that the irrational rotation is a Hamiltonian system. Let $(\RR^2, dx\wedge dy)$ be our symplectic manifold, so
$$H_f = \frac{\partial f}{\partial x}\partial_y - \frac{\partial f}{\partial y}\partial_x.$$
Now let
$$f(x, y) = \frac{x^2 + y^2 - 1}{2}$$
so $\dot x = -\partial_yf$ and $\dot y = \partial_xf$. Then $S = \varphi_{-2\pi\alpha}$.

The irrational rotation is an especially useful example because it is simultaneously a topological, smooth, holomorphic, ergodic-theoretic and Hamiltonian system.
\end{example}

\section{Properties of the irrational rotation}
Let $T$ be the irrational rotation. For every $f: S^1 \to \RR$, let
$$S_Nf(\theta) = \frac{1}{N} \sum_{j=0}^{N-1} f(T^j(\theta)).$$
Let
$$\overline f = \frac{1}{2\pi} \int_0^{2\pi} f(\varphi) ~d\varphi.$$
\begin{theorem}
One has
$$\lim_{N \to \infty} S_Nf = \overline f,$$
pointwise if $f \in C(S^1)$ and in $L^2$ if $f \in L^2(S^1)$.
\end{theorem}
\begin{proof}
By the Stone-Weierstrass theorem, trigonometric polynomials are dense in $C(S^1)$ (in the $L^\infty$ topology), so to prove pointwise convergence we just need to check on trigonometric polynomials. It then suffices to check for the trigonometric basis $f(\theta) = e^{i\ell\theta}$. If $\ell = 0$ then this is obvious. If $\ell \neq 0$,
$$S_Nf(\theta) = \frac{1}{N} \sum_{j=0}^{N-1} e^{i\ell(\theta + 2\pi j\alpha)} = \frac{e^{i\ell\theta}}{N} \frac{1 - e^{i\ell2\pi N\alpha}}{1 - e^{i\ell2\pi\alpha}}.$$
Since $\alpha$ is irrational, the denominator is never $0$, hence is bounded from below. The numerator is clearly bounded from above, so as $N \to \infty$, $S_Nf \to 0$ uniformly. Meanwhile, $\overline f = 0$ since $f$ is periodic of period $2\pi\ell$. This proves the pointwise claim.

One can prove $L^2$-convergence by taking the Fourier series
$$f = \sum_\ell f_\ell e^\ell$$
where $e^\ell(\theta) = e^{i\ell\theta}$.
Then $f_0 = \overline f$. We have $\langle S_N(e^\ell), S_N(e^k)\rangle = 0$ whenever $k \neq \ell$ by orthogonality.
\begin{align*}\left|\left|\frac{1}{N} S_N\left(\sum_\ell f_\ell e^\ell\right) - f_0\right|\right|_{L^2}^2 &\leq \left|\left|\sum_{|\ell| \leq k} \frac{f_\ell}{N} S_N(e^\ell)\right|\right|_{L^2}^2\\&\quad + \left|\left|\sum_{|\ell| > k} \frac{f_\ell}{N} S_N(e^\ell) - f_0\right|\right|_{L^2}^2
\end{align*}
and
$$ \left|\left|\sum_{|\ell| > k} \frac{f_\ell}{N} S_N(e^\ell)\right|\right|_{L^2}^2 = \sum_{|\ell| > k} \left|\left|\frac{f_\ell}{N} S_N(e^\ell)\right|\right|^2_{L^2} \leq \sum_{|\ell| > k} |f_\ell|^2$$
and for any $\varepsilon > 0$ we may choose $k$ so that the sum over $|\ell| > k$ is at most $\varepsilon$. The sum over $|\ell| \leq k$ can also be chosen less than $\varepsilon$ by taking $N$ big enough so we're done.
\end{proof}
\begin{corollary}
$(2\pi n\alpha)_{n \in \NN}$ is dense in $S^1$.
\end{corollary}
\begin{proof}
We have for every $f \in C(S^1)$,
$$\frac{1}{N} \sum_{n=0}^{N-1} f(2\pi n\alpha) = \frac{1}{2\pi} \int_0^{2\pi} f(\varphi) ~d\varphi.$$
Suppose the claim fails. Then there is a open $U \subseteq S^1$ such that for every $n \in \NN$, $n\alpha \notin U$. Let $f$ be zero outside of $U$, such that $\int f \neq 0$. Then the left-hand side is $0$ while the right-hand side is nonzero.
\end{proof}
\begin{corollary}
The only invariant Borel probability measure of the irrational rotation is Lebesgue measure.
\end{corollary}
\begin{proof}
Suppose that $\mu$ is an invariant measure. Then
$$\int_{S^1} f(x + 2\pi n\alpha) ~d\mu(x) = \int_{S^1} f(x) ~d\mu(x)$$
but the $2\pi n\alpha$ are dense, so in fact we have
$$\int_{S^1} f(x + y) ~d\mu(x) = \int_{S^1} f(x) ~d\mu(x)$$
whence $\mu$ is rotation-invariance, hence the Lebesgue measure.
\end{proof}
\begin{definition}
A dynamical system $T$ with fixed $\sigma$-algebra $\Sigma$ is \dfn{uniquely ergodic} if there is a unique $T$-invariant probability measure on $\Sigma$.
\end{definition}
So we have just proven that the irrational rotation is uniquely ergodic for the Borel $\sigma$-algebra. This is a very unusual property.
\begin{corollary}
Modulo null sets, there are no invariant proper subsets of the irrational rotation.
\end{corollary}
\begin{proof}
Let $A$ be an invariant set and let $f = 1_A$, $g = 1 - f$. Then
$$\langle S_Nf, g\rangle = \int_0^{2\pi} S_Nf(\theta)\overline{g(\theta)} ~d\theta = \int_0^{2\pi} f(\theta)\overline{g(\theta)} ~d\theta = 0$$
but also
$$\langle f, g\rangle = |A|(2\pi - |A|)$$
so by the equality we have $|A|(2\pi - |A|) = 0$.
\end{proof}

\section{The ergodic theorems}
We now extend the above results to general dynamical systems.

Let $(X, \mu)$ be a probability space, and for a measurable function $f$ and measure-preserving transformation $T$, let
$$S_nf(x) = \sum_{j=0}^{n-1} f(T^j(x)).$$
be $n$ times the the time average of $f$. For example if $f$ is an indicator function for a set $A$ then $S_nf$ counts the number of times that we visit $f$. Let
$$\overline f = \int_X f ~d\mu$$
be the space average of $f$. So if $f$ is an indicator function then $\overline f$ is the probability of $A$. For the irrational rotation we proved that $S_n/nf \to \overline f$ pointwise and in $L^2$. This is a special case of the ergodic theorems.

\begin{lemma}
If $g \geq 0$ is an integrable function then
$$\int_X g\circ T ~d\mu = \int_X g~d\mu.$$
\end{lemma}
\begin{proof}
For indicator functions this is obvious. By taking sums we extend to simple functions and then use monotone convergence.
\end{proof}

\begin{definition}
Let $Uf = f \circ T$, the \dfn{Koopman operator} of $T$.
\end{definition}
By the lemma, $U$ is an isometry on $L^2$ and we have
$$S_nf(x) = \sum_{j=0}^{n-1} U^jf(x).$$
\begin{theorem}
Let $H$ be a Hilbert space and $U \in B(H)$ is an operator such that $||U|| \leq 1$. Let
$$\Inv = \{f \in H: Uf = f\}.$$
Let $P: H \to \Inv$ be the orthogonal projection. Then for every $f \in H$,
$$\sum_{1}{n} \sum_{j=0}^{n-1} U^jf \to Pf$$
in $L^2$.
\end{theorem}
The proof of this theorem uses the Banach-Alaoglu theorem and the following lemma:
\begin{lemma}
$Ug = g$ iff $U^*g = g$.
\end{lemma}
\begin{proof}
Since the adjoint is an involution we just need to check one direction. Suppose $Ug = g$. Then
$$0 = ||U^*g - g||^2 = ||U^*g||^2 + ||g||^2 - 2 \Re \langle U^*g, g\rangle = ||U^*g||^2 - ||g||^2 \leq ||Ug||^2 - ||g||^2 = 0.$$
\end{proof}
\begin{proof}[Proof of theorem]
It suffices to prove the claim when $f \in \Inv^\perp$, in which case $Pf = 0$. We expand
$$S_nf = \sum_{j=0}^{n-1} U^jf$$
as
$$||S_nf/n||^2 = \langle f, S_n^*S_n f/n^2\rangle$$
and
$$||S_nf/n|| \leq \frac{1}{n} \sum_{j=0}^{n-1} ||U^jf|| \leq ||f||$$
which is bounded. Now $f \in \Inv^\perp$ and $g_n = S_n^*S_nf/n^2$ is a bounded sequence, so $g_n$ has a weak limit. Suppose that $g$ is a weak limit; we claim that $g \in \Inv$, so $\langle f, g\rangle = 0$. But
$$(1 - U^*)(S_n^*/n) = \frac{1}{n} \sum_{j=0}^{n-1} (1 - U^*)(U^*)^{j-1} = \frac{1 - (U^*)^n}{n}$$
whose operator norm is bounded by $2/n$, so
$$(1 - U^*)g_n \to 0.$$
This implies that $(1 - U^*)g = 0$, so by the lemma $Ug = g$ and $g \in \Inv$.
\end{proof}
\begin{corollary}[von Neumann mean ergodic theorem]
\index{von Neumann mean ergodic theorem}
Let $(X, \mu, T)$ be a measure-preserving system and $f \in L^2(\mu)$. Let $\Inv(T)$ be the space of Koopman-invariant functions for $\mu$ and let $P: L^2(\mu) \to \Inv(T)$ be the orthogonal projection. Then
$$\frac{1}{n} \sum_{j=0}^{n-1} f(T^j(x)) \to Pf$$
in $L^2(\mu)$.
\end{corollary}
Note that the mean ergodic theorem does not assume that $\mu$ is a probability measure. However, when one applies the mean ergodic theorem he usually wants to assume that indicator functions are in $L^2$, which is only possible when $\mu$ is a finite measure.

\begin{lemma}
For every $g \in \Inv$ and $f \in L^2$,
$$\int_X (Pf)g ~d\mu = \int_X fg ~d\mu.$$
\end{lemma}
\begin{proof}
$$\langle Pf, g\rangle = \langle f, P^*g \rangle = \langle f, Pg \rangle = \langle f, g\rangle.$$
\end{proof}
\begin{corollary}
If $A$ is an invariant set and $\mu(A) < \infty$ then for every $f \in L^2$,
$$\int_A Pf = \int_A f.$$
\end{corollary}
\begin{corollary}
$P$ is a positive-semidefinite operator.
\end{corollary}
\begin{proof}
Suppose $f \geq 0$. Then $Pf(x)$ is the average of the Koopman iterates $f(T^j(x)) \geq 0$.
\end{proof}
\begin{corollary}
If $\mu$ is a probability measure then
$$\int_X Pf~d\mu = \int_X f ~d\mu.$$
\end{corollary}
\begin{corollary}
If $\mu$ is a probability measure and $f > 0$ a.e. then $Pf > 0$ a.e.
\end{corollary}
\begin{proof}
Note that $(Pf)^{-1}(0)$ is invariant since $Pf \in \Inv$, and has finite measure, so
$$\int_{(Pf)^{-1}(0)} f = \int_{(Pf)^{-1}(0)} Pf = 0.$$
Since $f$ is nonnegative this implies that $Pf = 0$ implies $f = 0$.
\end{proof}
\begin{example}
Applying the mean ergodic theorem to the irrational rotation we have
$$Pf = \overline f.$$
\end{example}
\begin{example}
Let $\Sigma_m^+$ be the Cantor space of infinite sequences in the alphabet $\{0, \dots, m-1\}$. Define the \dfn{shift map}
$$T(x_1x_2x_3\cdots) = x_2x_3x_4\cdots.$$
The dynamical system $(\Sigma_m^+, T)$ has many invariant measures. If $y$ is a word in $\{0, \dots, m-1\}$, let $C_y$ denote the cylinder of all sequences which begin with $y$. A Borel measure is defined uniquely (assuming it is well-defined at all) by assigning measures to each of the $C_y$.
In fact choose a probability vector $p$, i.e. $p = (p_0, \dots, p_{m-1})$ with $p_j \in [0, 1]$ such that $\sum_j p_j = 1$, and let
$$\mu_p(C_y) = \prod_j p_{y_j}.$$
Then $\mu_p$ is a Borel probability measure on $\Sigma_m^+$, and
$$\mu_p(T^{-1}(C_y)) = \mu_p\left(\bigcup_j C_{j \cdot y}\right) = \sum_j \mu_p(C_{j \cdot y}) = \mu_p(C_y).$$
Therefore $\mu_p$ is an invariant measure. As we will prove, this gives an ergodic system that is not uniquely ergodic. In case $p = (1/10, \dots, 1/10)$, and $m = 10$, this gives a construction of Lebesgue measure, which can be used to prove that almost every number is normal.
\end{example}

We now prove a recurrence theorem of Caratheodory, which confusingly is not named after Caratheodory.
\begin{theorem}[Poincare recurrence]
\index{Poincare recurrence theorem}
Suppose that $(X, \mu)$ is a probability space, $T: X \to X$ measure-preserving, and $B$ is measurable. Then for $\mu$-a.e. $x \in B$, there are infinitely many $n \in \NN$ such that $T^nx \in B$.
\end{theorem}
\begin{proof}
Let $f = 1_B$. By the mean ergodic theorem,
$$\frac{1}{n}\sum_{j=0}^{n-1} f(T^j(x)) \to_{L^2} Pf(x) > 0$$
a.e. in $B$. By the Riesz-Weyl theorem, there is a subsequence $(n_k)_k \in \NN$ such that
$$\frac{1}{n_k} \sum_{j=0}^{n_k-1} f(T^j(x)) \to_{a.e.} Pf(x).$$
The claimed property is true a.e. for the $n_k$.
\end{proof}
\begin{example}
Suppose we have two chambers connected, and a gas only in one chamber. By the second law of thermodynamics, the gas will spread throughout the two chambers. But by Poincare recurrence, the gas will almost surely return to the first chamber. However, the recurrence time of such a phenomenon may exceed the lifespan of the universe.
\end{example}
There is also a combinatorial proof of Poincare recurrence.
\begin{proof}[Proof of Poincare recurrence]
If there were only finitely many such $n$ we would be able to assume wlog that there were only zero such $n$, by time-translation. Then for every $x \in B$ and $n \in \NN$, $T^n(x) \notin B$. So if $n \neq m$, and $x \in T^{-n}(B) \cap T^{-m}(B)$, then $T^{n-m}(T^m(x)) = T^n(x) \in B$, a contradiction. Therefore the $T^{-n}(B)$ are disjoint and have positive measure, yet there are infinitely many of them and $\mu$ is a probability measure, a contradiction.
\end{proof}

To introduce the pointwise ergodic theorem, we need to define conditional expectation.
\begin{definition}
Let $(X, \Sigma, \mu)$ be a probability space and $J \subseteq \Sigma$ a $\sigma$-algebra. Given $f$ a $\Sigma$-measurable function, $A \in \Sigma$, let
$$\mu_f(A) = \int_A f~d\mu.$$
Then define the \dfn{conditional expectation}
$$E(f|J) = \frac{d(\mu_f|_J)}{d(\mu|_J)},$$
the Radon-Nikodym derivative of $\mu_f$ with respect to $\mu$ when restricted to the $\sigma$-algebra $J$.
\end{definition}
\begin{example}
If $J$ is the trivial $\sigma$-algebra then $E(f|J)$ is the constant function given by the mean $E(f)$.
\end{example}
\begin{example}
Let $(X, \Sigma, \mu)$ be $[0, 1]$ and $J$ be the $\sigma$-algebra generated by $[0, 1/2]$. Then $E(f|J)(x)$ is the average of $f$ on $[0, 1/2]$ if $x \leq 1/2$ or is the average of $f$ on $(1/2, 1]$ for $x > 1/2$.
\end{example}
\begin{theorem}[Birkhoff pointwise ergodic theorem]
\index{Birkhoff pointwise ergodic theorem}
Suppose that $(X, \Sigma, \mu)$ is a probability space, $T: X \to X$ measure-preserving, and $f \in L^1(\mu)$. Then for a.e. $x\in X$,
$$\lim_{n \to \infty} \frac{1}{n} \sum_{j=0}^{n-1} f(T^j(x)) = E(f|J)(x)$$
where $J$ is the $\sigma$-algebra of invariant sets,
$$J = \{A \in \Sigma: T^{-1}(A) = A\}.$$
\end{theorem}
\begin{corollary}
Let $J,\mu$ be as in the pointwise ergodic theorem, and $P$ as in the mean ergodic theorem. If $f \in L^2(\mu)$ then $E(f|J) = Pf$.
\end{corollary}
Note that for any open set of reals $U$, $E(f|J)^{-1}(U) \in J$ iff $E(f|J) \circ T = E(f|J)$. In fact, for any invariant set $A$,
$$\int_A E(f|J) ~d\mu = \int_A f~d\mu.$$
\begin{lemma}
The limit in the pointwise ergodic theorem exists $\mu$-a.e.
\end{lemma}
\begin{proof}
Let
$$\overline f(x) = \limsup_{n \to \infty} \frac{1}{n} \sum_{j=0}^{n-1} f(T^j(x))$$
and similarly for $\underline f$ and $\liminf$. Clearly $\overline f \geq \underline f$ and it suffices to show that $\overline f = \underline f$, $\mu$-a.e. In fact it suffices to show that
$$\int_X \overline f ~d\mu \leq \int_X f~d\mu \leq \int_X \underline f ~d\mu.$$
Replacing $f$ with $-f$ we see that in fact the seemingly weaker statement
$$\int_X \overline f ~d\mu \leq \int_X f~d\mu$$
is sufficient.

Fix $M > 0$, and let $\overline f_M = \min(\overline f, M)$. Then $\overline f_M \leq M$.
\begin{lemma}
$\overline f_M > -\infty$, $\mu$-a.e.
\end{lemma}
\begin{proof}[Proof of sublemma]
For any function $g$, $g \geq -|g|$. So
$$\int_X \overline f_M ~d\mu \geq \int_X \limsup_{n \to \infty} \frac{1}{n} \sum_{j=0}^{n-1} -|f \circ T^j| ~d\mu.$$
By Fatou's lemma,
$$\int_X \overline f_M ~d\mu \geq \limsup_{n \to \infty} \frac{1}{n} \int_X \sum_{j=0}^{n-1} -|f \circ T^j| ~d\mu = -\int_X |f| ~d\mu > -\infty$$
since $T$ is measure-preserving.
\end{proof}

Fix $\varepsilon > 0$. If $\overline f(x) < \infty$ then there is an $n$ such that
$$\frac{1}{n} \sum_{j=0}^{n-1} f(T^j(x)) \geq \overline f(x) - \varepsilon.$$
Let $n(x)$ be the smallest such $n$ witnessing this. If $\overline f(x) = \infty$ then there is a $n$ such that
$$\frac{1}{n} \sum_{j=0}^{n-1} f(T^j(x)) \geq M$$
so we may let $n(x)$ be the smallest such witness. Then
$$\overline f_M(x) \leq \frac{1}{n(x)} \sum_{j=0}^{n-1} f(T^j(x)) + \varepsilon.$$

For each $R > 0$, let
$$A_R = \{x \in X: n(x) > R\}.$$
Then the $A_R$ form a chain: if $R' > R$ then $A_{R'} \subseteq A_R$. Moreover, $\bigcap_R A_R = \emptyset$ and $\mu$ is a probability measure, so $\mu(A_R) \to 0$ as $R \to \infty$. Since $\mu$ is a probability measure, the constants are in $L^1(\mu)$ so
$$\lim_{R \to \infty} \int_{A_R} (|f| + M) ~d\mu = 0.$$

Fix $R > 1$. Let $n_i(x)$ be defined inductively. Let $n_0(x) = 0$. If $T^{n_i(x)}(x) \notin A_R$, let
$$n_{i+1}(x) = n_i(x) + n(T^{n_i(x)}(x)).$$
Otherwise, $T^{n_i(x)}(x) \in A_R$ and let $n_{i+1}(x) = n_i(x) + 1$, and we have
$$1 \leq n_{i+1} - n_i \leq R.$$

Suppose $T^{n_i(x)}(x) \notin A_R$. Then
$$n(T^{n_i(x)}(x))\overline f_M(T^{n_i(x)}(x)) \leq \sum_{j=0}^{n(T^{n_i(x)}(x))} f(T^{k + n_i(x)})(x) + n(T^{n_i(x)}(x))\varepsilon.$$
Clearly the constants are invariant functions and $\overline f$ is invariant under $T$. So
$$(n_{i+1} - n_i)(x) \overline f_M(x) \leq \sum_{j=n_i(x)}^{n_i(x) + 1} f(T^j(x)) + (n_{i+1} - n_i)(x) \varepsilon.$$

On the other hand, if $T^{n_i(x)}(x) \in A_R$, then we use the estimate $\overline f_M(x) \leq M$ and $n_{i+1}(x) - n(x) = 1$ to see that
$$(n_{i+1} - n_i)(x) \overline f_M(x) \leq M.$$

Let
$$\tilde f_M = f + (|f| + M)1_{A_R}.$$
Then in both cases,
$$(n_{i+1} - n_i)(x) \overline f_M(x) \leq \sum_{j=n_i(x)}^{n_{i+1}(x)} \tilde f_M(T^j(x)) + (n_{i+1} - n_i)(x)\varepsilon.$$

Fix some large $N$; then for any $x$ there is $k$ such that $n_k(x) \leq N \leq n_{k+1}(x)$. Then the series in question telescope and
\begin{align*}N \overline f_M(x) &\leq \sum_{j=0}^{n_k(x) - 1} \tilde f_M(T^j(x)) + (N - n_k(x))\overline f_M(x) + n_k(x)\varepsilon
\\& \leq \sum_{j=0}^{N-1} \tilde f_M(T^j(x)) - \sum_{j=n_k(x)}^{N-1} \tilde f_M(T^j(x)) + RM + N\varepsilon
\\& \leq \sum_{j=0}^{N-1} \tilde f_M(T^j(x)) - \sum_{j = N-R}^{N-1} |f(T^j(x))| + RM + N\varepsilon
\end{align*}
since $-|f| \leq |\tilde f|$. Dividing both sides by $N$,
$$\overline f_M(x) \leq \frac{1}{N} \sum_{j=0}^{N-1} \tilde f_M(T^j(x)) + \frac{1}{N} \sum_{j=N-R}^N |f(T^j(x))| + \varepsilon + \frac{RM}{N}.$$
We integrate both sides $d\mu(x)$. Then since $\mu(A) \leq \mu(X) = 1$,
\begin{align*}\int_X \overline f_M ~d\mu &\leq \int_X \tilde f_M + \frac{R}{N} \int_X |f| + \varepsilon + \frac{RM}{N} ~d\mu
\\ &\leq \int_X f~d\mu + \int_{A_R} (|f| + M)~d\mu + \frac{R}{N} ||f||_{L^1(\mu)} + \varepsilon + \frac{RM}{N}
\end{align*}
\end{proof}
We are finally ready to prove the pointwise ergodic theorem.
\begin{proof}[Proof of pointwise ergodic theorem]

\end{proof}
\begin{example}
We apply the pointwise ergodic theorem to the shift map. Fix a probability vector $p$. As a black box, we will assume that the shift map is ergodic; i.e. if $A \subseteq \Sigma_m^+$ is a Borel set and $T^{-1}(A) = A$ then $\mu_p(A)(1 - \mu_p(A)) = 0$. Let $\ell$ be a letter and $f = 1_{C_\ell}$ the indicator function of its cylinder. Then the Birkhoff ergodic average for $n$ iterates on a sequence $x$ is $1/n$ times the number of $j$ such that $x_j = \ell$, for $j \leq n$, since
$$T^j(x) = x_{j+1}x_{j+2}\cdots.$$
By the pointwise ergodic theorem, the Birkhoff average converges to $p_\ell$, $\mu_p$-a.e. As a corollary, we have proven the \dfn{law of large numbers}.
\end{example}








\chapter{Banach algebras}
\begin{definition}
A \dfn{Banach algebra} is a Banach space equipped with a bilinear, associative multiplication such that
$$||xy|| \leq ||x|| ||y||.$$
If $*$ is a linear involution on $\mathcal A$ such that $(xy)^* = y^*x^*$ and $1^*= 1$ if $\mathcal A$ is unital. then we say that $\mathcal A$ is a \emph{$*$-algebra}.
\end{definition}

\begin{definition}
\index{$C^*$-algebra}
Let $\mathcal A$ be a $*$-algebra. If one has the \emph{$C^*$-identity}
$$||x^*x|| = ||x||^2,$$
then we say that $\mathcal A$ is a \emph{$C^*$-algebra}.
\end{definition}

For example, if $\mathcal H$ is a Hilbert space, then $\mathcal B(\mathcal H)$ is a $C^*$-algebra. Later we will learn that sub-$*$-algebras of $\mathcal B(\mathcal H)$ are the only examples of $C^*$-algebras.

Often the norm topology is too strong, so we introduce a new topology which is weaker on $\BB(\HH)$.
\begin{definition}
    The \dfn{strong operator topology} is the locally convex topology on $\BB(\HH)$ defined by the seminorms
    $$P_\xi(T) = ||T\xi||.$$
\end{definition}
In other words, a sequence converges in the strong operator topology $T_n \to T$ iff for each $\xi \in \HH$, $||(T_n - T)\xi|| \to 0$. So the strong operator topology is the topology of pointwise convergence.
\begin{definition}
    A \dfn{von Neumann algebra} $\mathcal A$ is a sub-$*$-algebra of $\BB(\HH)$ which is closed in the strong operator topology.
\end{definition}


\section{The spectrum}
Fix a Banach algebra $A$.
\begin{definition}
    Let $a \in A$. The \dfn{spectrum} $\sigma(a)$ is the set of $z \in \CC$ such that $\sigma(a) - z$ is not invertible. The \dfn{resolvent} $\rho(a)$ is the complement of $\sigma(a)$.
\end{definition}

\begin{lemma}
    Let $a \in A$. If $||a|| < 1$ then $1 - a$ is invertible with inverse
    $$(1 - a)^{-1} = \sum_{n=0}^\infty a^n.$$
\end{lemma}
\begin{proof}
    The partial sums converge since $||a|| < 1$. Therefore
    $$(1 - a) \sum_{n=0}^\infty a^n = (1 - a) \lim_{n \to \infty} \sum_{k=0}^n a^k = \lim_{n \to \infty} \sum_{k=0}^n a^k - a^{k-1} = 1$$
    since the summands telescope.
\end{proof}
In particular, if $||1 - a|| < 1$ then $a$ is invertible.
\begin{definition}
    The \dfn{general linear group} of $A$ is $\GL(A)$, the group of invertible elements of $A$.
\end{definition}
By the above lemma, there is a ball $B$ around $1$ contained in $\GL(A)$. By continuity of translation, we can carry $B$ to be centered at any point of $\GL(A)$. Therefore $\GL(A)$ is an open set.

\begin{proposition}
    The function $z \mapsto (z - a)^{-1}$ is holomorphic on $\rho(a) \cup \infty$.
\end{proposition}
    In this case, holomorphy is indicated by local existence of a convergent power series.
\begin{proof}
    We have
    $$(a - z)^{-1} = \sum_{n=0}^\infty (a - z_0)^{-n-1}(z-z_0)^n$$
    for each $z_0 \in \rho(a)$ and $z$ close enough to $z_0$ that the power series converges. To see that the function is still holomorphic at $\infty$, notice that
    $$(a - z^{-1})^{-1} = z(1 - az)^{-1}$$
    which vanishes as $z \to 0$. Replacing $z$ by $z^{-1}$, we see that the function is bounded close to infinity, and continuous, so holomorphic there.
\end{proof}

We now observe that the usual proofs of Cauchy's integral formula and its friends such as Cauchy's estimate and Liouville's theorem go through even in case of holomorphic functions $U \to A$, $U \subseteq \CC$ open.

We now come to the famous Gelfand-Mazur theorem, which can be thought of as a ``restatement of the fundamental theorem of algebra" for our purposes. For the notation, recall that the map $z \mapsto z1$ is an embedding of $\CC$ in any Banach algebra.
\begin{theorem}[Gelfand-Mazur]
    \index{Gelfand-Mazur theorem}
    If $A = \GL(A) \cup 0$, then $A = \CC$.
\end{theorem}
\begin{proof}
    Let $a \in \GL(A)$ and assume towards contradiction that $a \notin \CC$. Then the resolvent $z \mapsto (a - z)^{-1}$ is a holomorphic function defined on the Riemann sphere, so constant. Taking $z = \infty$, the resolvent is identically $0$, but also identically $a^{-1}$ (taking $z = 0$). This is a contradiction.
\end{proof}
Notice that this fails over $\RR$, as witnessed by $\CC$ as well as the quaternions $\mathbb H$. This is why we study Banach algebras over $\CC$.

\section{Ideals}
Let $I$ be an ideal of $A$. It is immediate that the norm-closure $\overline I$ is an ideal. Moreover, since $\GL(A) \ni 1$ is open, if $I$ is a proper ideal, then $I$ does not meet $\GL(A)$ and so $\overline I$ does not contain $1$, so $\overline \cdot$ preserves propriety. Therefore maximal ideals are closed. Moreover, for continuous morphisms, kernels are closed, so we might as well only study closed ideals.

If $I$ is a (left, right) ideal then $A/I$ is a (left, right) module over $A$, equipped with the seminorm
$$||a|| = \inf_{d \in I} ||a - d||.$$
In case $I$ is closed, this seminorm is actually a norm, and complete since $A$ is complete. So we end up with a Banach space.

\begin{definition}
    A \dfn{Banach module} over $A$ is an $A$-module $M$ which is a Banach space, such that
    $$||am||_M \leq ||a||_A ||m||_M.$$
\end{definition}
It is not very hard to check that $M = A/I$ is a Banach module. In fact, for $b, c \in I$, we have
$$||am||_M \leq ||(a - c)(m - d)||_A \leq ||a - c||_A ||m - d||_A.$$
Taking the $\inf$ over $c, d$ of both sides, we have
$$||am||_M \leq ||a||_A ||m||_M.$$
In case $I$ is two-sided, $M$ is a Banach $(A, A)$-bimodule, or in other words, a Banach algebra.

In what follows we use $\Hom(A, B)$ to mean the $K$-algebra of morphisms of $K$-algebras $A \to B$ over some field $K$ (which is usually $\CC$).

If $I$ is a maximal ideal, therefore, $A/I$ is a field, and so $A/I = \CC$ by the Gelfand-Mazur theorem. But a maximal ideal gives a epimorphism $A \to \CC$, and conversely, the kernel of a such an epimorphism is a maximal ideal. This gives a bijection between the maximal spectrum of $A$ and $\Hom(A, \CC) \setminus 0$, which we call $\hat A$.

\begin{lemma}
    Let $K$ be a field and $A$ a unital $K$-algebra. Let $\varphi \in \Hom(A, K)$. Then if $a \in A$, $\varphi(a) \in \sigma(a)$.
\end{lemma}
\begin{proof}
    We have $\varphi(a - \varphi(a)) = 0$.
\end{proof}
\begin{lemma}
    If $\varphi: A \to \CC$ is a nonzero morphism, then $||\varphi|| \leq 1$.
\end{lemma}
\begin{proof}
    $\varphi(a) \in \sigma(a)$ so $||\varphi(a)|| \leq ||a||$.
\end{proof}

Therefore $\hat A$ is contained in the unit ball $A'_1$ of the dual $A'$. Since nets in $\hat A$ act continuously on $A$, their pointwise convergence preserves operations of $A$. So $\hat A$ is closed. In particular, the Banach-Alaoglu theorem implies that $\hat A$ is a weakstar compact Hausdorff space.

\begin{definition}
    Let $a \in A$. The \dfn{Gelfand transform} $\hat a$ is the function
    $$\hat a(\varphi) = \varphi(a),$$
    for $\varphi \in \hat A$.
\end{definition}
Notice that $||\hat a||_{L^\infty(\hat A)} \leq ||a||_A$ and $\hat a(\hat A) \subseteq \sigma_A(a)$. Conversely, let $\lambda \in \sigma_A(a)$. Then $a - \lambda$ is not invertible, so there is a maximal ideal $I \supseteq (a - \lambda)$ and an epimorphism $\varphi$ such that $\ker \varphi = I$. Thus $\lambda \in \hat a(\hat A)$. Therefore $\hat A = \sigma_A(a)$, but the proof of this is highly nonconstructive.

\begin{example}
    Recall that $c_0(\NN)$, the set of $x \in \ell^\infty(\NN)$ such that $x_n \to 0$ as $n \to \infty$, is a closed ideal of $A = \ell^\infty(\NN)$. Therefore there is a $\varphi$ such that $\varphi(c_0(\NN)) = 0$. But, in fact, $\hat A = \beta\NN$, where $\beta$ is the Stone-Cech functor. It follows that it is consistent with ZF without the axiom of choice that $\varphi$ does not exist.
\end{example}

\begin{definition}
    Let $a \in A$. The \dfn{spectral radius} of $a$ is
    $$r(a) = \max_{\lambda \in \sigma(a)} |\lambda|.$$
\end{definition}
Equivalently, $r(a) = ||\hat a||_{L^\infty(\hat A)}$. Therefore we have $r(ab) \leq r(a)r(b)$.

\section{The holomorphic functional calculus}
    As usual, let $A$ be a commutative Banach algebra.
\begin{definition}
    Let $a \in A$ and let $f$ be a holomorphic function on $D(0, ||a|| + \varepsilon)$. Put
    $$f(z) = \sum_{n=0}^\infty \alpha_n z^n.$$
    The \dfn{holomorphic functional calculus} is the morphism $f \mapsto f(a)$ defined by
    $$f(a) = \sum_{n=0}^\infty \alpha_n a^n.$$
\end{definition}
    The Taylor series of $f$ converges uniformly absolutely on $D(0, ||a||)$, so the partial sums of $f(a)$ form a Cauchy sequence in $A$. Therefore $f(a)$ is a well-defined element of $A$, and we can think of $f$ as a mapping $U \to A$, where $U$ consists of elements of $A$ that are small enough. If $f$ is entire, then $f$ lifts to a function $A \to A$.
\begin{theorem}[spectral mapping theorem]
    \index{spectral mapping theorem}
    If $\lambda \in \sigma(a)$ then $f(\lambda) \in \sigma(f(a))$.
\end{theorem}
\begin{proof}
    We have
    \begin{align*}
        f(a) - f(\lambda) &= \sum_{n=0}^\infty \alpha_n(a^n - \lambda^n)
            = \sum_{n=0}^\infty \alpha_n(a - \lambda)(a^{n-1} + a^{n-2}\lambda + \dots + \lambda^{n-1})\\
            &= (a - \lambda)b
    \end{align*}
    for some $b$, if we can show that the partial sums are a Cauchy sequence. In fact
    $$||a^{n-1} + \dots + \lambda^{n-1}|| \leq n||a||^{n-1}$$
    which is the right-hand side of $f'(||a||)$ (which clearly converges, so partial sums are Cauchy). Therefore $f(a) - f(\lambda) = (a-\lambda)b$. So if $f(a) - f(\lambda)$ is invertible, then so is $a - \lambda$.
\end{proof}





\chapter{$C^*$-algebras}
\section{Weights}
\begin{definition}
\index{weight}
Let $\mathcal A$ be a sub-$*$-algebra of $\mathcal B(\mathcal H)$. A map $\omega: \mathcal A^+ \to [0, \infty]$ is a \emph{weight} if $\omega$ is additive and if $\omega(ta) = t\omega(a)$ whenever $t \geq 0$.
\end{definition}

Fix a weight $\omega$. By $m_\omega$ we mean the span of the set of positive $a$ such that $\omega(a) < \infty$, and by $m_\omega^{sa}$ we mean the closure of the set of positive $a$ such that $\omega(a) < \infty$ under subtraction. Clearly $\omega$ extends uniquely to $m_\omega^{sa}$ by $\omega(b-c) = \omega(b) - \omega(c)$. So $\omega$ extends to a positive linear functional on $m_\omega$ in the obvious way. On the other hand, if $\varphi$ is any positive linear functional on $\mathcal B(\mathcal H)$, then $\varphi$ is a weight such that $m_\varphi = \mathcal B(\mathcal H)$.

Now we define $n_\omega$ to be the set of $a \in \mathcal A$ such that $\omega(a^*a) < \infty$, which is clearly a subspace of $\mathcal A$.
\begin{lemma}
    $n_\omega$ is a left ideal of $\mathcal A$.
\begin{proof}
    If $T \geq 0$ then
    \begin{align*}
        \langle S^*TS\xi, \xi\rangle
            &= \langle TS\xi, S\xi\rangle
            \leq ||T|| ||S\xi||^2\\
            &= ||T|| \langle S\xi, S\xi\rangle
            = ||T|| \langle S^*S\xi, \xi\rangle.
    \end{align*}
    So if $d \in \mathcal A$ and $a \in n_\omega$ then
    \begin{align*}
        (da)^*da = a^*d^*da \leq ||d^*d|| a^*a = ||d||^2 a^*a
    \end{align*}
    whence
    $$||\omega((da)^*(da)) \leq ||d||^2 \omega(a^*a) < \infty.$$
\end{proof}
\end{lemma}
\begin{definition}
\index{tracial weight}
If $\omega(a^*a) = \omega(aa^*)$ then $\omega$ is \emph{tracial}.
\end{definition}
Clearly if $\omega$ is tracial then $n_\omega$ is a two-sided ideal. For example, if $\omega$ is actually the trace,
$$\omega(x) = \sum_j \langle x^*xe_j, e_j\rangle$$
for $\{e_j\}_j$ an orthonormal basis of the separable Hilbert space $\mathcal H$, then $\omega$ is tracial and $n_\omega$ is just the space of trace-class operators and $\omega$ is tracial.

\index{polarization identity}
Recall the polarization identity:
$$4b^*a = \sum_{k=0}^3 i^k(a+i^kb)^*(a+i^kb).$$
From this we are justified in defining, on $n_\omega$,
$$\langle a, b\rangle_\omega = \omega(b^*a).$$
This would be an inner product if $N_\omega = \{a \in \mathcal A: \omega(a^*a) = 0\}$ were trivial. Clearly $N_\omega$ is a subspace, so we can take the completion of $n_\omega/N_\omega$ and recover a Hilbert space.

\section{The GNS construction}
\begin{definition}
    The completion of $n_\omega/N_\omega$ is denoted $L^2(\mathcal A, \omega)$.
\end{definition}
If $b \in n_\omega$, then
$$\langle ab, ab\rangle_\omega
    = \omega(b^*a^*ab)
    \leq ||a^*a|| \omega(b^*b) = ||a||^2 ||b||_\omega^2.$$
So if $a \in \mathcal A$ then $\xi \mapsto a\xi$ is a well-defined, bounded operator on $n_\omega/N_\omega$ and so extends to $L^2(\omega)$.

\begin{definition}
    \index{$*$-representation}
    A \emph{$*$-representation} is a morphism of $*$-algebras (i.e. a morphism of algebras preserving $*$) into $\BB(\HH)$.
\end{definition}
If $a \in \AAA$, $b,c \in n_\omega$, then
$$\langle ab, c\rangle_\omega = \omega(c^*ab) = \omega((a^*c)^*b) = \langle b, a^*c\rangle_\omega,$$
which descends to $L^2(\omega)$. So we can define a $*$-representation
\begin{align*}
    L: \AAA &\to \BB(L^2(\omega))\\
        a &\mapsto (\xi \mapsto a\xi).
\end{align*}
\begin{definition}
    \index{GNS construction}
    \index{left regular representation}
    The map $L$ is called the \emph{GNS construction} (for Gelfand-Neimark-Segal) of $\AAA$, or the \emph{left regular representation} of $\AAA$.
\end{definition}
The GNS construction allows us to assume that $\AAA$ is actually acting on a Hilbert space, namely $L^2(\omega)$. So a $C^*$-algebra is always an operator algebra.

We can also define a right regular representation\index{right regular representation},
\begin{align*}
    R: \AAA &\to \BB(L^2(\omega))\\
        a &\mapsto (\xi \mapsto \xi a).
\end{align*}
Notice that $R$ is an antihomomorphism.

\begin{lemma}
    Assume that for each positive $a \in \AAA$, $\sqrt a$ exists. Then $m_\omega \subseteq n_\omega$, and in particular $m_\omega$ is a sub-$*$-algebra of $\AAA$.
    \end{lemma}
\begin{proof}
If $a \in m_\omega$ is positive,
    $$\omega(\sqrt a^2) = \omega(\sqrt a^* \sqrt a) = \omega(a)$$
so $\sqrt a \in n_\omega$. Since $n_\omega$ is a left ideal, $a \in n_\omega$.
\end{proof}


\begin{example}
    Let $X$ be a measure space and $K \in L^2(X \times X)$. Then the integral operator $T_K: L^2(X) \to L^2(X)$ has $||T_K||_{\BB^2} = ||K||_{L^2}$. Indeed, if $\{\xi_n\}_n$ is a Hilbert basis for $L^2(X)$ then
\begin{align*}
    \sum_n ||T_k\xi_n||^2
        &= \sum_{m,n} |\langle T_k\xi_m, \xi_n\rangle|^2
        = \sum_{m,n} \left|\iint_{X \times X} K(x, y) \xi_n(y) \xi_m(x) ~dx ~dy\right|^2\\
        &= \sum_{m,n} |\langle K, \xi_m \otimes \xi_n\rangle|^2
        = ||K||_{L^2}
\end{align*}
    since the $\xi_m \otimes \xi_n$ form a Hilbert basis for $L^2(X \times X) = L^2(X) \otimes L^2(X)$.
\end{example}
\begin{example}
    If $\AAA = C([0, 1])$ and
    $$\omega(f) = \int_0^1 f(t) ~dt$$
    then $\omega$ is a tracial weight on $\AAA$ such that $n_\omega = \AAA$. But of course $n_\omega$ is a Banach space when given the $\BB^2 = L^2$ norm. Its completion is $L^2([0, 1])$.
\end{example}
\begin{proposition}
    Let $\omega$ be a tracial weight and $\AAA$ be a sub-$*$-algebra of $\BB_0(\HH)$. If $b \geq 0$ and $b \in m_\omega$, and $a \in \BB(\HH)$ then
    $$|\omega(ab)| \leq ||a||_{op}|\omega(b)|.$$
\end{proposition}
\begin{proof}
    We have
\begin{align*}
    |\omega(ab)|^2
        &= |\omega(a \sqrt b \sqrt b)|^2
        = |\omega(\sqrt b a \sqrt b)|^2
        = |\langle a \sqrt b, \sqrt b\rangle|^2\\
        &\leq \langle a\sqrt b, a\sqrt b\rangle \langle \sqrt b, \sqrt b\rangle
        = \omega(\sqrt b a^* a\sqrt b)\omega(b)
        \leq ||a||^2 \omega(b)^2.
\end{align*}
\end{proof}

We summarize the GNS construction, and the Gelfand transform, in the following theorem. We define $C_\infty(X)$ to be a subspace of the space of continuous functions $C(X)$. If $X$ is compact we define $C_\infty(X) = C(X)$. Otherwise, we let $C_\infty(X)$ be those functions in $C(X)$ which vanish at the point at infinity given by the one-point compactification of $X$. For example, $C_\infty(\RR)$ consists of functions on $\RR$ which go to zero as $|x| \to \infty$.
\begin{theorem}[Gelfand-Naimark]
    \index{Gelfand-Naimark theorem}
    For every $C^*$-algebra $A$, there is a faithful $*$-representation of $A$. Moreover, if $A$ is commutative, then there is a locally compact Hausdorff space $X$ such that $A = C_\infty(X)$ (which gives a representation of $A$ on $L^2(X)$. If $A$ is also unital, then $X$ is compact and is naturally in bijection with the maximal ideal space of $A$. Moreover, the map $A \mapsto X$ is a contravariant equivalence of categories between compact Hausdorff spaces and commutative, unital $C^*$-algebras.
\end{theorem}

\section{The $\BB^p$ spaces}
\begin{definition}
    If $T \in \BB^2(\HH)$ then $T$ is called a \dfn{Hilbert-Schmidt operator}.
\end{definition}
\begin{example}
    Let $\HH$ be the separable Hilbert space. Take $\AAA = \BB_0(\HH)$ and $\omega$ to be the trace. Since $\BB_0(\HH)$ has square roots and $\omega$ is tracial, we can apply the above result to prove that $n_\omega$ and $m_\omega$ are two-sided ideals and hence sub-$*$-algebras.

    If we write $|T| = \sqrt T^2$, and let $\BB^p(\HH)$ be the space of $T \in \BB_0(\HH)$ such that $\omega(|T|^p) < \infty$, then $\BB^1(\HH) = m_\omega(\HH)$ and $\BB^2(\HH) = n_\omega(\HH)$.

    We think of $\BB^p(\HH)$ as the noncommutative analogue of $\ell^p$.
\end{example}
Let's check that that example actually makes sense.
\begin{theorem}
    $\BB^2(\HH)$ is a Banach space.
\end{theorem}
\begin{proof}
    First observe that $||T||_{op} \leq ||T||_2$. To do this, compute the trace of $T$ by using an orthonormal basis containing a $\xi$ such that $||T\xi|| \geq ||T||_{op} - \varepsilon$. As this is possible for any $\varepsilon > 0$ the claim holds.

    Now assume that $\{T_n\}_n$ is $2$-Cauchy, so in particular $op$-Cauchy. So there is a $T \in \BB_0(\HH)$ such that $T_n \to^{op} T$.

    If $P$ is a finite-rank projection then $(T-T_n)P$ is a finite-rank operator, hence $\in \BB^2(\HH)$. So
\begin{align*}
    ||(T-T_n)P||_2^2
        &= tr P(T-T_n)^*(T-T_n)P
        = tr (T-T_n)P(T-T_n)^*
        = \lim_{k \to \infty} tr (T_k-T_n)P(T_k -T_n)^*\\
        &\leq \limsup_{k \to \infty} (T_k-T_n)(T_k-T_n)^*
        = \limsup_{k \to \infty} ||T_k-T_n||_2^2.
\end{align*}
Let $C_n = \limsup_{k \to \infty} ||T_k-T_n||_2^2$. Then $C_n \to 0$ and
$$||(T-T_n)P||_2^2 \leq C_n$$
regardless of the choice of $n$ and $P$. Since $T-T_n$ is a compact operator, we can approximate it arbitrarily well by $(T-T_n)P$ by choosing $P$. So $||T-T_n||_2^2 \to 0$.
\end{proof}
Recall the \dfn{polar decomposition} of $T \in \BB_0(\HH)$ is the factorization
$$T = V|T|$$
where $|T| = \sqrt T^2$ and $V$ is a \dfn{partial isometry}, i.e. an isometry on its cokernel.
\begin{lemma}
    If $T \in \BB^1(\HH)$ and $A \in \BB(\HH)$ then
    $$|tr(AT)| \leq ||A||_{op} tr|T|.$$
\end{lemma}
\begin{proof}
    Write $T = V|T|$. Then
\begin{align*}
    |tr(AT)|
        &= |tr(AV|T|)|
        \leq ||AV||_{op} tr|T|
        \leq ||A||_{op} tr|T|.
\end{align*}
\end{proof}
\begin{lemma}
    $||\cdot||_1 = tr|\cdot|$ is a norm on $\BB^1(\HH)$.
\end{lemma}
\begin{proof}
    Let $S,T \in \BB^1(\HH)$ and $S+T = W|S+T|$. Then
\begin{align*}
    tr|S+T|
        &= tr W^*(S+T)
        = tr(W^*S) + tr(W^*T)
        \leq |trW^*S| + |trW^*T|
        \leq tr|S| + tr|T|.
\end{align*}
\end{proof}
\begin{theorem}
    $\BB^1(\HH)$ is a Banach algebra.
\end{theorem}
\begin{proof}
    Since $||T||_{op} \leq ||T||_1$ the proof is basically the same as for Hilbert-Schmidt operators.
\end{proof}
\begin{theorem}
    $\BB^1(\HH)^* = \BB(H)$.
\end{theorem}
\begin{proof}
    If $A \in \BB(\HH)$, let $\Psi_A(T) = tr(AT)$. Then
$$||\Psi_A(T)|| \leq ||A||||T||_1.$$
    So $A \mapsto \Psi_A$ is an isometry and so $\BB(\HH) \subseteq \BB^1(\HH)^*$.

    Let $\Psi \in \BB^1(\HH)^*$ and $\xi,\eta \in \HH$. Define a bounded operator $\langle \xi, \eta\rangle_O$ by
    $$\langle \xi, \eta\rangle_O\zeta = \xi \langle \eta, \zeta\rangle.$$
    (So $\langle\cdot,\cdot\rangle_O$ is an operator-valued pseudo-inner product (the pseudo- here means that it could be zero).
    Define a semilinear form
    $$B_\Psi(xi, \eta) = \Psi\langle \xi, \eta\rangle_O.$$
    So $|B_\Psi(\xi, \eta)| \leq ||\Psi||||\xi||||\eta||$. Therefore by the Riesz representation theorem, there is an operator $A$ such that $B_\Psi(\xi, \eta) = \langle A\xi, \eta\rangle$. Therefore $||A|| = ||\Psi||$ and $\Psi = \Psi_A$. So $\BB^1(\HH)^* \subseteq \BB(\HH)$.
\end{proof}

\section{Representation theory of groups}
Let $G$ be a group with a good topology (so $G$ admits a Haar measure).

\begin{definition}
A \dfn{unitary representation} of $G$ is a continuous morphism of groups $G \to U(H)$. It is \dfn{irreducible} if the only $G$-invariant subspaces are trivial.
\end{definition}
For $\pi$ a unitary representation, we have $\pi(x)^* = \pi(x)^{-1}$.

\begin{example}
    Let $G = \SL(3, \ZZ)$. Then the ``obvious" map $G \to \SL(3, \CC)$ is not a unitary representation. In fact $G$ has very few finite-dimensional unitary representations, because $G$ is not compact.
\end{example}

\begin{definition}
    The \dfn{left regular representation} of $G$ is the map $G \to U(L^2(G))$ given by
    $$\pi(x)(\xi)(y) = \xi(x^{-1}y).$$
\end{definition}

It is natural to want to study the subalgebra of $B(H)$ generated by $\pi(G)$ for $\pi$ a unitary representation. This will be given by linear combinations of the $\pi(x)$s as $x \in G$, which we identify with the space $C_c(G)$ of compactly supported continuous functions on $G$. Namely, for $f \in G$ we define
$$\pi(f) = \int_G f(x) \pi(x) ~dx.$$
\begin{definition}
    The norm-closure of $\pi(C_c(G))$ is the \dfn{reduced C$^*$-algebra} of $G$.
\end{definition}
Now an easy computation shows
$$\pi(f)\pi(g) = \pi(f*g)$$
and of course $\pi(f)^* = \pi(f^*)$ where we define $f^*(x) = \overline f(x^{-1})$. Finally, we observe that
$$||\pi(f)|| \leq ||f||_{L^1(G)}$$
so $\pi$ is a $*$-Banach algebra morphism which extends to a map
$$\pi: L^1(G) \to B(L^2(G)).$$
This leads to the abstract theory of Fourier transform.

\section{Compact operators}
Let $B_0(H)$ denote the algebra of compact operators in $H$. This is a closed ideal of $H$, hence a $C^*$ algebra (proof: it is the closure of the ideal $B_f(H)$ of finite rank operators in $H$.) It will be one of our main examples of a noncommutative, nonunital $C^*$ algebra.

We now study the representation theory of $B_0(H)$.

Like any $C^*$ algebra, $B_0(H)$ has a normalized approximate identity, sequential if $H$ is separable. Decompose $H$ by transfinite recursion as
$$H = \bigoplus_{\alpha < \kappa} \CC$$
where $\kappa$ is some cardinal ($\kappa = \aleph_0$ if $H$ separable) and the biproduct is in the category of Hilbert spaces. For $\lambda < \kappa$, let $H_\lambda = \bigoplus_{\alpha < \lambda} H_\lambda$ so $H$ is the injective limit of the $H_\lambda$; then let $e_\lambda$ be the natural projection $H \to H_\lambda$. The $e_\lambda$ form a net with respect to the natural ordering on $\kappa$ and are obviously an approximate identity.

Recall that if we fix a representation $\pi: A \to B(H)$, we can view $H$ as a module over $A$ by defining $a\xi = \pi(a)(\xi)$. Recall also that a representation is said to be nondegenerate if $HA$ is dense in $A$.

In fact, any representation of $B_0(H)$ is faithful. Since representations are continuous, and $B_0(H)$ has no closed ideals (since $B_f(H)$ contains all proper ideals of $B_0(H)$, and is dense in $H$), any representation of $B_0(H)$ is faithful.

There is a natural $*$-representation $B_0(H) \to B(H)$ given by the inclusion map. Since $e_\lambda\xi \to \xi$, this representation is nondegenerate. In some sense this is the only such representation.
\begin{lemma}
    A nondegenerate $*$-representation of $B_0(H)$ is isomorphic to a direct sum to copies of the representation $B_0(H) \to B(H)$. In particular, the only irreducible such representation is the representation $B_0(H) \to B(H)$.
\end{lemma}
\begin{proof}
    Let $\langle \xi, \eta\rangle_0$ be the $B_0(H)$-valued inner product
    $$\langle \xi, \eta\rangle_0 \zeta = \xi \langle \eta, \zeta\rangle.$$
    In fact such an inner product has values in rank-$1$ operators since $\xi \langle \eta, \zeta\rangle$ lies in the span of $\xi$.

    For $T \in B(H)$, $T\langle \xi, \eta\rangle_0\zeta = (T\xi)\langle \eta, \zeta\rangle$ so $T\langle \xi, \eta\rangle_0 = \langle T\xi, \eta\rangle_0$, and $\langle \xi, \eta\rangle_0T = \langle \xi, T^*\eta\rangle_0$.

    Let $\pi: B_0(H) \to B(V)$ be a nondegenerate $*$-representation, $\xi \in H$ a unit vector. Then $\langle \xi, \xi\rangle_0$ is a rank-$1$ projection. Since $\pi$ is faithful, $\pi(\langle \xi, \xi\rangle_0)$ is a nonzero projection. Let $v$ be a unit vector of $\langle \xi, \xi\rangle_0(V)$ and define $Q: H \to V$ by $Q\eta = \langle \eta, \xi\rangle_0 v$. Then by a tedious computation, $Q$ is an isometry.

    We now show that $Q$ commutes the representations. Let $T = \langle \omega, \zeta\rangle_0$. Any operator in $B_0(H)$ can be written as an infinite linear combination of rank-$1$ operators so it suffices to show that $QT = TQ$. In fact,
    $$Q(T\eta) = \langle T\eta, \zeta\rangle_0 v = TQ(\eta).$$
    Also, $Q(H)^\perp$ is $\pi$-invariant, so we repeat the argument on $Q(H)^\perp$ to see that we have
    $$V = Q(H) \oplus Q(H)^\perp$$
    as $B_0(H)$-modules. Now run Zorn's lemma to keep decomposing $Q(H)^\perp$ until we hit an irreducible representation.
\end{proof}

This is a very remarkable property of $B_0(H)$. To see why, we need something stronger than ZFC.
\begin{definition}
    A \dfn{$\Diamond$-sequence} is a net of sets $\alpha \mapsto A_\alpha$, for $\alpha < \aleph_1$, such that for any $A \subseteq \aleph_1$,
    $$\hat A = \{\alpha < \aleph_1: A \cap \alpha = A_\alpha\}$$
    is stationary in $\aleph_1$.
\end{definition}
In other words, for every closed and unbounded (``club") set $C \subseteq \aleph_1$, $C \cap \hat A$ is nonempty. The existence of a $\Diamond$-sequence implies that $V = L$, in particular implying GCH.

Naimark conjectured that if $A$ was a $C^*$-algebra with only one irreducible representation, then $A = B_0(H)$. This is true if $A$ is separable.
\begin{theorem}[Ackemann-Weaver]
    If there is a $\Diamond$-sequence, then there is a $C^*$-algebra $A$ which has only one irreducible representation such that $A \neq B_0(H)$.
\end{theorem}

In this lemma, you should read $A = B(H)$ and $I = B_0(H)$.
\begin{lemma}
    Let $A$ be a $*$-normed algebra and $I$ a $*$-ideal of $A$ with normalized approximate unit. Then every nondegenerate $*$-representation of $I$ extends uniquely to $A$.
\end{lemma}
\begin{proof}
    Let $\pi: I \to B(H)$ be such a representation. Define
    $$\tilde \pi(a) \sum_\alpha \pi(d_\alpha) \xi_\alpha = \sum_\alpha \pi(ad_\alpha) \xi_\alpha$$
    where the $\xi_\alpha$ are a Hilbert basis of $H$. Then $\tilde \pi$ is a well-defined function since if $\sum_\alpha \pi(d_\alpha)\xi_\alpha = 0$, then
    $$\sum_\alpha \pi(ad_\alpha)\xi_\alpha = \lim_\lambda \sum_\alpha \pi(ae_\lambda d_\alpha) \xi_\alpha = \lim_\lambda \pi(ae_\lambda) \sum_\alpha \pi(d_\alpha)\xi_\alpha = 0$$
    since the $e_\lambda$ are a normalized approximate unit. This is unique because
    $$\tilde \pi(a)\pi(d)\xi = \pi(ad)\xi$$
    and the $\pi(ad)\xi$ are dense in $H$.
\end{proof}

\begin{lemma}
    Let $A,I$ be as above and let $\pi$ be an irreducible representation of $A$. Then either $I \subseteq \ker\pi$ or $\pi$ is an irreducible representation of $I$.
\end{lemma}
\begin{proof}
    Assume $I$ is not contained in $\ker \pi$. Then $\overline{IH}$ is nonzero and $A$-invariant. Since $\pi$ is irreducible, $\overline{IH} =H$. Therefore $\pi$ is a nondegenerate representation of $I$.

    We have $e_\lambda\xi \to \xi$ for any $\xi$ since $\pi$ is nondegenerate. Let $K \subseteq H$ be nondegenerate and nonzero. Using $e_\lambda$, $\overline{IK} = K$ so $\overline{IK}$ is $A$-invariant. Therefore since $\pi$ is irreducible, $K = H$.
\end{proof}

\begin{lemma}
    Let $A,I$ be as above. Let $\pi: A \to B(H)$ and $\rho: A \to B(K)$ be irreducible representations. If $\pi \cong \rho$ as representations of $I$, then $\pi \cong \rho$ as representations of $A$.
\end{lemma}
\begin{proof}
    Let $U: H \to K$ be an isomorphism of $I$-modules. For $d \in I$,
    $$U(\pi(a)\pi(d)\xi) = U(\pi(ad)\xi) = \rho(ad)U\xi = \rho(a)U(\pi(d)\xi).$$
    So $U\pi = \rho U$.
\end{proof}

\begin{theorem}[Burnside]
    Assume $H \neq \CC$. Let $A \subseteq B_0(H)$ be a $C^*$-algebra. If $A$ acts on $H$ irreducibly, then $A = B_0(H)$.
\end{theorem}
\begin{proof}
    By assumption on $H$, $A \neq 0$. Let $T \neq 0$. Then $T^*T \in A$ is nonzero, so we can assume without loss of generality that $T$ is self-adjoint. Moreover, $C^*(T, 1) = C(\sigma(T))$ acts on $H$ as an abelian monoid.

    Let $\lambda \in C(\sigma(T))$. Taking bump functions centered on $\lambda$ we can find a sequence of $\xi_n \in H$ such that $(T - \lambda)\xi_n \to 0$. Since $T$ is comapct, $T$ sends the unit ball of $H$ to a precompact set. So the $T\xi_n$ have a weak limit $\eta$. Thus $T\eta = \lambda\eta$. Therefore $T$ has an eigenvector for $\lambda$. So the only limit point of $\sigma(T)$ is $0$, because the other eigenvectors are all orthogonal.

    Let $P$ be a projection of minimal rank in $A$. We claim that $P$ is a rank-$1$ projection. In fact, $PTP$ is a self-adjoint operator on the finite-dimensional space $PH$, so has spectral projections in $A$ whose rank is the same as that of $P$ by minimality. Thus there is a unique such spectral projection; i.e. there is an $s \in \RR$ such that $PTP = sP$. Moreover, if $\xi,\eta \in PH$ are orthonormal, $R \in A$, then $\langle R\xi, \eta\rangle = s\langle\xi, \eta\rangle = 0$. Therefore $\langle A\xi, \eta\rangle = 0$, yet $A$ acts irreducibly, which is a contradiction. Therefore $P$ is a rank-$1$ projection.

    Now $R,S \in A$ implies that $RPS$ is a rank-$1$ projection. Thus $\overline{APAH} = H$. So $\overline{APA}$ is the set of rank-$1$ operators. Any compact operator can be written as an infinite linear combination of those.
\end{proof}

\begin{corollary}
    Let $A$ be a $C^*$-algebra and $\pi: A \to B(H)$ an irreducible representation. If $\pi(A)$ contains a nonzero compact operator then $\pi(A)$ contains $B_0(H)$.
\end{corollary}
\begin{proof}
    Let $I = \pi(A) \cap B_0(H)$. Then $I$ acts irreducibly on $H$ so by Burnside's theorem, $\overline{\pi(I)} = B_0(H)$. But $\pi$ takes $I/\ker \pi$ to $B_0(H)$ isometrically (since it is injective), so $\pi(I)$ is closed since $I$ is complete. Thus $\pi(I) = B_0(H)$.
\end{proof}

We write $\hat A$ to denote the set of all isomorphism classes of irreducible representations of $A$.
\begin{definition}
    Let $\pi: A \to B(H)$ range over $\hat A$. We say that $A$ is \dfn{liminal} or \dfn{CCR} if for every $\pi$, $\pi(A) = B_0(H)$. We say $A$ is \dfn{postliminal} or \dfn{GCR} if for every $\pi$, $B_0(H) \cap \pi(A) \neq 0$. We say $A$ is \dfn{antiliminal} or \dfn{NCR} if for every $\pi$, $B_0(H) \cap \pi(A) = 0$.

    If $A$ is a von Neumann algebra, we say that $A$ is \dfn{type-I} if $A$ is postliminal. We say that $A$ is \dfn{non-type-I} if $A$ is not postliminal.
\end{definition}
    We will prove later that if $G$ is a semisimple Lie group or a nilpotent Lie group, then $C^*(G)$ (by which we really mean $C^*(L^1(G))$) is CCR. But if $G$ is solvable we cannot even prove that $C^*(G)$ is GCR.
\begin{example}
    Let $\alpha$ be the Lie action of $\RR$ on $\CC^2$ by $\alpha(t)(z, w) = (e^{2\pi it}z, e^{2\pi i\mu t}w)$ where $\mu$ is irrational. Then we take the outer semidirect product $G = \CC^2 \times_\alpha \RR$. Then $C^*(G)$ is not GCR.
\end{example}
\begin{theorem}
    Let $\pi: A \to B(H)$ be an irreducible representation and let $I = \ker \pi$. If $\pi(A) \cap B_0(H) \neq 0$ then for every irreducible representation $\rho$ of $A$ such that $\ker \rho = I$, $\rho \cong \pi$.
\end{theorem}
\begin{proof}
    Let $J = \pi^{-1}(B_0(H))$, so $I \subseteq J$. Then $\pi$ is an irreducible representation of $J$ with kernel $I$, so drops to an irreducible representation of $J/I$. By Burnside's theorem, $\pi$ is an isomorphism $J/I \to B_0(H)$.

    But by assumption on $\rho$, $\rho$ is an isomorphism $J/I \cong B_0(H)$. Since there can be only one irreducible representation of $B_0(H)$, $\rho \cong \pi$ as representations of $J$. But $\rho,\pi$ extend uniquely to $A$, so $\rho \cong \pi$ as representations of $A$.
\end{proof}
\begin{corollary}
    If $A$ is a GCR algebra, then every representation of $A$ is uniquely determined by its kernel.
\end{corollary}
\begin{definition}
    A \dfn{primitive ideal} is a kernel of some irreducible representation.
\end{definition}
    So if $A$ is GCR, then there is a bijection between $\hat A$ and the set of primitive ideals of $A$. On the other hand, NCR algebras are very bad:
\begin{theorem}
    Let $\pi: A \to B(H)$ be an irreducible representation and assume $\pi(A) \cap B_0(H) = 0$. Then there are uncountably many irreducible representations $\rho$ of $A$ such that $\ker \rho = \ker \pi$.
\end{theorem}
    In fact Mackey showed that in some sense the set of representations sharing a kernel with $\pi$ is ``unclassifiable."
\begin{example}
    Let $A$ be a unital, infinite-dimensional, simple $C^*$-algebra and $\pi: A \to B(H)$ an irreducible representation. Since $A$ is simple, it has no proper ideals; yet $\pi^{-1}(B_0(H))$ is an ideal. If $\pi^{-1}(B_0(H)) = A$ then $A$ is not unital, a contradiction. So $\pi(A) \cap B_0(H) = 0$. Therefore $A$ is an NCR algebra.
\end{example}
\begin{theorem}
    Every primitive ideal is prime.
\end{theorem}
\begin{proof}
    Let $I$ be a primitive ideal of $A$, say $I = \ker \pi$. Let $J_1,J_2$ be ideals of $A$ such that $J_1J_2 \subseteq I$. If $J_1 \subseteq I$ there is nothing to prove. Otherwise, $\pi(J_1) \neq 0$, so $\overline{\pi(J_1)H} \neq 0$ is $\pi$-invariant. By irreducibility, $\overline{\pi(J_1)H} = H$, so $\overline{\pi(J_2)H} = \overline{\pi(J_2)\pi(J_1)H} = \overline{\pi(J_1J_2)H} \subseteq \overline{\pi(I)H} = 0$. So $J_2 \subseteq I$.
\end{proof}
    Recall that the prime spectrum $\Spec R$ of a ring $R$ is defined by the Zariski (or Jacobson, or hull-kernel) topology is given by declaring that $S \subset \Spec R$ is closed if there is an ideal $I$ such that $S = \{J \in \Spec R: J \subseteq I\}$. By the above theorem, if $R$ is a $C^*$-algebra, then the Zariski topology drops to a topology on the set of primitive ideals $\Prim R$. This topology is far from Hausdorff in general, but is at least locally compact. If $R$ is a commutative $C^*$-algebra, then $\Prim R$ is naturally the maximal ideal space of $R$ (since every primitive ideal is maximal in that case) and this just generalizes the Gelfand-Naimark theorem in that case.

    In fact, if $A$ is separable, then every closed prime ideal of $A$ is primitive. This follows from applying the Baire category theorem to $\Prim A$. This does not work in general: Weaver used transfinite induction to find a counterexample if $A$ is not separable. (It may be undecidable whether there is a counterexample in ZF alone.)

\chapter{Generators and relations}
We now study $C^*$-algebras determined by generators and relations.

\section{Construction of maximally free algebras}
Let $\{a_i\}$ be a set of generators, and take the free $*$-algebra $F$ over $\CC$ generated by the $a_i$. This is the set of noncommutative polynomials in the $a_i$ and $a_i^*$ (where $a_i^*$ is just a formal symbol for now).

Given a set $R$ of relations, we can view $R$ as noncommutative polynomial equations. So we take the ideal $(R, R^*)$, which is the $*$-ideal generated by $R$. Then we let $A = F/(R, R^*)$, which is a $*$-algebra still. We consider the set $\Pi$ of all $*$-representations of $A$. Then for $a \in A$ we set
$$||a|| = \sup_{\pi \in \Pi} ||\pi(a)||.$$
A priori we have $||a|| = \infty$. This happens if $F$ is the free $*$-algebra on one generator. So we need $R$ to force $||a_i|| < \infty$ for each generator $a_i$.

Assume that $R$ forces $||a_i|| < \infty$ (for example, if the generators are sent to unitary operators). Since the image of every $\pi \in \Pi$ is a $C^*$-algebra, $||\cdot||$ is a seminorm satisfying the $C^*$-identity $||a^*a|| = ||a||^2$. So we take the completion with respect to $||\cdot||$; i.e. we annihilate the kernel of the seminorm and then complete. What is left over is a $C^*$-algebra.

Just because a set of generators and relations gives a valid $C^*$-algebra of $A$ does not mean that we necessarily can find a natural, faithful representation. Moreover, even if we have a natural, faithful representation of $A$, the norm arising from that representation is not necessarily the norm given by taking the supremum over $\Pi$.

\begin{example}
Let $G$ be a discrete group, which we view as a set of generators. We impose relations corresponding to each true relation in $G$. (This can be viewed as taking the first-order theory of $G$.) We also take the relations $x^* = x^{-1}$ for $x \in G$ so the resulting maximally free operator algebra $A$ naturally represents the elements of $G$ as unitary operators. Therefore for $x \in G$ we have $||x|| = 1$; so $A$ is actually a $C^*$-algebra. In fact, it is easy to see that $A$ is the completion of $C_c(G)$, i.e. finitely supported functions on $G$. In other words, $A = C^*_r(G)$ is the reduced group $C^*$-algebra.

Moreover, $A$ acts on $\ell^2(G)$ by left translation, since $G$ does. This representation is faithful, so $A$ is unusual amongst the free $C^*$-algebras in that it has a natural faithful representation. In general, the norm in $\ell^2(G)$ does not always agree with the norm on $A$; this happens if and only if $G$ is an \dfn{amenable group}. (It turns out that since $G$ is discrete, $G$ is amenable if and only if there is a finitely additive, left-invariant probability measure on $G$. For example, this fails if $G$ is the free group on $n$ letters, $n \geq 2$.)
\end{example}

\section{Tensor products of $C^*$-algebras}
An important application of generators and relations is the ability to define the tensor product of $C^*$-algebras.
\begin{definition}
    Let $A,B$ be unital $C^*$-algebras. Their \dfn{tensor product} $A \otimes B$ is the $C^*$-algebra with generators $A \cup B$ and relations consisting of all true relations in $A$, all true relations in $B$,
    $$\forall a\in A~\forall b\in B~ab = ba,$$
    and $\forall x ~(1_A1_B)x = x$.
\end{definition}
The relation $ab = ba$ is the ``tensor product relation", so we can reasonably think of $ab$ as $a \otimes b$. Similarly, the relation $(1_A1_B)x = x$ requires that $1_{A \otimes B} = 1_A1_B$, so $A \otimes B$ is unital. We have embeddings $A \to A \otimes B$, $B \to A \otimes B$ given by $a \mapsto a \otimes 1_B$ and $b \mapsto 1_A \otimes b$.

To see that this is actually well-defined (i.e. has finite norm), let $\pi$ be a $*$-representation of $A \otimes B$. Then $\pi$ restricts to a $*$-representation of $A$ (and similarly to $B$) along the mapping $A \to A \otimes B$. Therefore for any $a \in A$,
$$||\pi(a \otimes 1_B)|| \leq ||a||.$$
Similarly for $B$; so there is an upper bound on the norm of any generator. Therefore $A \otimes B$ is a $C^*$-algebra.

\begin{example}
    Let $X,Y$ be compact Hausdorff spaces. Then we have $C(X \times Y) = C(X) \otimes C(Y)$.
\end{example}

We now consider the representation theory of tensor products.
\begin{definition}
    Let $\pi: A \to B(H)$ and $\rho: B \to B(K)$ be representations. The \dfn{tensor product of representations} $\pi \otimes \rho: A \otimes B \to B(H \otimes K)$ is defined by
    $$\pi \otimes \rho(a \otimes b)(\xi \otimes \eta) = \pi(a)\xi \otimes \rho(b)\eta.$$
\end{definition}
The norm induced by the tensor product of representations is also a $C^*$-norm, so we have two reasonable $C^*$-norms on $A \otimes B$. If we use the norm obtained by taking the supremum over representations, we emphasize this by writing $A \otimes_{max} B$. If we are using the norm obtained by the tensor product of Gelfand-Naimark-Segal representations, we write $A \otimes_{min} B$.
\begin{example}[Takesaki, 1959]
    Let $G$ be the free group on $2$ generators, $\lambda$ its left regular representation, $\rho$ its right regular representation. Consider the tensor product $C_r^*(G) \otimes C_r^*(G)$, represented by $\lambda \otimes \rho$. Then the two norms given above are the ``minimum" (i.e. tensor product $\lambda \otimes \rho$) and ``maximum" norms respectively, but there are many intermediate norms between the two that have been studied in recent years. In particular, the tensor product does not have a unique norm.
\end{example}
\begin{definition}
    A \dfn{nuclear $C^*$-algebra} $A$ is a $C^*$-algebra such that for every $C^*$-algebra $B$, the norm on $A \otimes B$ is unique.
\end{definition}
\begin{example}
    $B_0(H)$ is nuclear, since it uniquely embeds in $B(H)$. It follows that any GCR algebra is nuclear. But if $G$ is a discrete group, then $C^*(G)$ is nuclear if and only if $G$ is amenable. Thus the free group on $2$ generators is not nuclear.
\end{example}
\begin{definition}
    An \dfn{exact $C^*$-algebra} $A$ is a $C^*$-algebra such that $A \otimes_{min} \cdot$ is an exact functor.
\end{definition}
    Note that $A \otimes_{max} \cdot$ is an exact functor for any $A$.

We now treat the coproduct in the category of $C^*$-algebras.
\begin{definition}
    The \dfn{free product} of $C^*$-algebras $A, B$, $A*B$, is defined to have generators $A \cup B$ and relations induced from $A,B$ as well as $1_A = 1_B$.
\end{definition}
The norm is defined as in the case of tensor products. A representation of $A*B$ consists of a pair of representations of $A$ and $B$ on the same Hilbert space, by the universal property.

\chapter{Representation theory of locally compact groups}
\section{Noncommutative dynamical systems}
\begin{definition}
    A \dfn{noncommutative dynamical system} is an action of a group $G$ on a $C^*$-algebra $A$.
\end{definition}

    Often it is convenient that a noncommutative dynamical system is taking place inside a Hilbert space $H$.
\begin{definition}
    Let $\alpha: G \to \Aut(A)$ be a noncommutative dynamical system. Let $\pi: A \to B(H)$ a $*$-representation, $U: G \to U(H)$ a unitary representation. If the dynamical system satisfies the \dfn{covariance relation}
    $$\alpha(x)a = U_x\pi(a)U_x^*,$$
    then $(\pi, U)$ is called a \dfn{covariant representation} of $\alpha$.
\end{definition}

    we now define $*$-operations on $C_c(G \to A)$. If $f,g \in C_c(G \to A)$ then
\begin{align*}\left(\sum_x f(x)x\right)\left(\sum_y g(y)y\right) &= \sum_{x,y} f(x)xg(y)y \\
    &= \sum_{x,y} f(x)\alpha(x)(g(y))xy\\
    &= \sum_{x,y} f(x)\alpha(x)(g(x^{-1}y))y.
\end{align*}
    This motivates the following definition.
\begin{definition}
    If $\alpha$ is a noncommutative dynamical system, then for $f,g \in C_c(G \to A)$, we define
    $$(f*g)(y) = \sum_x f(x) \alpha(x)(g(x^{-1}y)),$$
    the \dfn{twisted convolution} of $f,g$ by $\alpha$.
\end{definition}
\begin{definition}
    If $\alpha$ is a noncommutative dynamical system, $f \in C_c(G \to A)$, then
    $$f^*(x) = \alpha(x)(f(x^{-1})^*),$$
    the \dfn{twisted involution} of $f$ by $\alpha$.
\end{definition}
    With twisted convolution and involution, $C_c(G \to A)$ is a $*$-algebra. Moreover, we have an injective mapping $A \to C(G \to A)$ give by $a \mapsto a\delta_1$, $1$ the identity of the group. We also have an injective mapping $G \to C(G \to M(A))$ with $x \mapsto 1x$ (here $1$ is the identity of the noncommutative Stone-Cech compactification $M(A)$; if $A$ is unital then $A = M(A)$.

\begin{definition}
    Given any covariant representation $(\pi, U)$ of $\alpha: G \to \Aut(A)$ we call the \dfn{integrated form} $\sigma$, a $*$-representation $\sigma: C_c(G, A) \to B(H)$ by
    $$\sigma(f) = \sum_x \pi(f(x))U_x.$$
\end{definition}
    The set of all possible integrated forms is bounded. Therefore we can make the following definition by generators and relations:
\begin{definition}
    The $C^*$-algebra generated by $C_c(G, A)$ and $A$ is called $C^*(G, A)$ or $A \rtimes_\alpha G$. It is called the \dfn{crossed product $C^*$-algebra} or the \dfn{covariance $C^*$-algebra}.
\end{definition}
    The generators of $A \rtimes_\alpha G$ will consist of all elements of $G$ and of $A$, the relations will be all relations in $G$ and $A$ as well as
    $$xa = \alpha(x)ax$$
    since $G$ acts on $A$ by convolution (since $\alpha$ has a covariant representation).

    To see that that there are, in fact, covariant representations, we give an explicit one.
\begin{definition}
    Let $\rho: A \to B(H_0)$ be a $*$-representation, and define
    $$H = \ell^2(G \to H_0) = \ell^2(G) \otimes H_0.$$
    Now define a unitary representation of $G$ on $H$ by (for $x,y \in G$, $\xi \in H$)
    $$U(x)(\xi)(y) = \xi(x^{-1}y).$$
    We then define $\pi: A \to B(H)$ by
    $$\pi(a)(\xi)(x) = \rho(\alpha(x^{-1})(a))(\xi)(x),$$
    which is a $*$-representation. It is called the \dfn{induced covariant representation} of $\alpha$ by $\rho$.
\end{definition}
\begin{lemma}
    The induced covariant representation is covariant.
\end{lemma}
\begin{proof}
    We have
\begin{align*}U(x)(\pi(a))(\xi)(y) &= (\pi(a)(\xi))(x^{-1}y) = \rho(\alpha((x^{-1}y)^{-1}))(U(x)(\xi))(y)\\
    &= \rho(\alpha(y^{-1}x)(a)) = \rho(\alpha(y^{-1})(\alpha(x))(a))(U(x)(\xi))(y)\\
    &= (\pi(\alpha(x)(a))U(x)(\xi))(y).
\end{align*}
    So
    $$U(x)(\pi(a))(\xi) = \pi(\alpha(x)(a)U(x))(\xi)$$
    which implies
    $$U(x)(\pi(a)) = \pi(\alpha(x)(a))U(x)$$
    so that $(\pi, U)$ is a covariant representation.
\end{proof}
    If $K$ is a subgroup of $G$, $(\rho, V)$ a covariant representation of $\alpha|_K$ on $H_0$, then the induced representation of $(\rho, V)$ is a covariant representation of $\alpha$. Here $H = \ell^2(G/H) \otimes H_0$.
\begin{definition}
    The \dfn{reduced group $C^*$-algebra of the representation} $\alpha$, $C^*_r(A, G, \alpha)$, is the $C^*$-algebra generated by $C_c(G \to A)$ with $||f||$ defined to be the supremum of $||\sigma(f)||$ for $\sigma$ ranging over the integrated forms of induced covariant representations.
\end{definition}
\begin{definition}
    If $C^*(A, G, \alpha) = C^*_r(A, G, \alpha)$, then $\alpha$ is said to be an \dfn{amenable action}.
\end{definition}
    If $\alpha$ is trivial, then
    $$C^*(A, G, \alpha) = A \otimes C^*(G)$$
    so the theory of covariant representations includes the theory of unitary representations.

    Though all the above theory was developed for discrete groups, it works fine for locally compact groups, as we now describe in detail. In fact, if $G$ is a locally compact group, then we consider the left Haar measure $\mu$, which is a Radon measure that is unique up to positive scalars (but need not be two-sided). Moreover, if $y \in G$, if we let
    $$\nu(f) = \int_G f(xy) ~d\mu(x)$$
    then $\nu$ is a left Haar measure, so we can find a $\Delta(y)$ such that $\nu = \Delta(y)\mu$.
\begin{definition}
    The function $\Delta: G \to \RR^+$ is called the \dfn{modular function} of $G$.
\end{definition}
    The modular function is a continuous morphism of groups.
\begin{definition}
    A group is \dfn{unimodular} if $\Delta = 1$.
\end{definition}
    For unimodular groups, the left and right Haar measures coincide. So any abelian group is unimodular. Moreover, $\RR^+$ has no compact subgroups, so any compact group is unimodular. Semisimple Lie groups and nilpotent Lie groups can also be shown to be unimodular. Solvable Lie groups are often not unimodular.
\begin{example}
    Let $(M, \omega)$ be a Poisson manifold (e.g. $\omega$ is a symplectic form on $M$). We let $h$ be a semiclassical parameter for a family of noncommutative ring structures on $C_c^\infty(M \to \CC)$, so we can view $(M, \omega, h)$ as a family of operator algebras. There is a notion of modular function for $M$.
\end{example}
    The trouble with locally compact groups is that the maps $G \to U(H)$ are typically not norm-continuous. In fact, this is already apparent for the left regular representation $\RR \to U(L^2(\RR))$, where we take a function with compact support and translate it by far outside its support.
\begin{definition}
    An action $\alpha$ of $G$ on a Banach space $A$ is \dfn{strongly continuous} if for every $a \in A$, $x \mapsto \alpha(x)(a)$ is continuous.
\end{definition}
    This is the correct definition of the continuity of a representation of a locally compact group.
\begin{definition}
    If $G$ is a locally compact group, a $C^*$-dynamical system for $G$ is a strongly continuous action of $G$ on a $C^*$-algebra. A \dfn{covariant representation} of the $C^*$-dynamical system is one which is also strongly continuous.
\end{definition}
    The integrated form of a covariant representation $(\pi, U)$ of a $C^*$-dynamical system is given by
    $$\sigma_f(\xi) = \int_G \pi(f(x))U(x)(\xi) ~dx.$$
    Using Bochner integration, we can define the integrated form for any $f \in C_c(G \to A)$. Now
    $$||\sigma_f(\xi)|| \leq \int_G ||f(x)|| ~dx ~||\xi||$$
    so it follows that $||\sigma_f|| \leq ||f||_{L^1(G)}$. So $\sigma$ extends to $L^1(G \to A)$. Though the elements of $C_c(G \to A)$ are functions, it makes sense to think of them as Radon-Nikodym derivatives of $A$-valued measures on $G$ with respect to Haar measure. As in the theory of discrete groups, $\sigma_f\sigma_g = \sigma_{f*_\alpha g}$ where the twisted convolution is defined by
    $$f*_\alpha g(x) = \int_G f(y) \alpha(y)(g(y^{-1}x))(y) ~dy.$$
    Therefore $||f*_\alpha g||_{L^1(G)} \leq ||f||_{L^1(G)}||g||_{L^1(G)}$.

    Another complication comes in the form of groups that are not unimodular. This happens because
\begin{align*}
    \sigma_f^* &= \int_G \pi(f(x))U_x ~dx = \int_G U_x^* \pi(f(x))^* ~dx\\
    &= \int_G U_{x^{-1}} \pi(f(x))^* ~dx = \Delta(x) \int_G U_x\pi(f(x^{-1})^*) ~dx.
\end{align*}
    Therefore we must define the twisted involution
    $$f^*(x) = \Delta(x)\alpha(x)(f(x^{-1})^*)$$
    if we want $\sigma_f^* = \sigma_{f^*}$.
\begin{example}
    Let $G$ be the group of affine transformations of $\RR$ (the ``$ax + b$ group"). This group is far from unimodular, and its action on $\RR$ is important in the theory of wavelets. So the modular function is important in signal processing.
\end{example}
    We write $C^*(A, G, \alpha) = A \rtimes_\alpha G$ for the completion of $C_c(G \to A)$ with respect to the norm given by taking the supremum over all integrated forms. The reduced algebra $A \rtimes_\alpha^r G$ is given by taking the supremum over all integrated forms arising from covariant representations.

    Now $L^1(G)$ does not have an identity since it does not have a delta function if $G$ is not discrete. But we could always take an approximate delta function. Specifically, we let $\Lambda$ denote the filter of all open sets containing the identity $1$ of $G$. (A filter-base also suffices.) Then given a neighborhood $\lambda$ of $1$, let $f_\lambda \in C_c(G \to \RR^+)$ be supported in $\lambda$ with $||f_\lambda||_{L^1(G)} = 1$. We view $f_\lambda$ as a probability measure carried by $\lambda$. Obviously the $(f_\lambda)_\lambda$ are an approximate delta function in $L^1(G)$.

    Let $\alpha$ be an action of $G$ on a $C^*$-algebra $A$ with approximate identity $(e_\mu)_\mu$. We then define
    $$h_{\mu,\lambda}(x) = f_\lambda(x) e_\mu$$
    to obtain an approximate identity for $L^1(G \to A)$.

\begin{theorem}
    There is a natural bijection between nondegenerate $*$-representations of $A \rtimes_\alpha G$ and covariant representations of $\alpha$.
\end{theorem}
\begin{proof}[Proof sketch]
    Let $\sigma$ be a nondegenerate $*$-representation of $A \rtimes_\alpha G$ and consider its multiplier algebra $M(A \rtimes_\alpha G)$. We have an injection $G \to M(A \rtimes_\alpha G)$ by $x \mapsto \delta_x$, using the fact that $L^1(G)$ is a $2$-sided ideal in the multiplier algebra $M(G)$, where we think of double centralizers as finite Radon measures (this is true up to natural isomorphism). We also have an injection $A \to M(A \rtimes_\alpha G)$, by $a \mapsto a \delta_{1_G}$. We can therefore obtain a covariant representation $(\pi, U)$ of $\alpha$ obtained by restricting $\sigma$ to $A,G$. It follows that $\sigma$ is the integrated form of $(\pi, U)$.
\end{proof}
    In particular, it makes sense to talk about the hull-kernel topology on the set of covariant representations of $\alpha$.

\begin{definition}
    For $G$ a locally compact group, we let $(A, \alpha)$ and $(B, \beta)$ be $C^*$-dynamical systems. Let $\varphi: A \to B$ be a $*$-morphism. Then $\varphi$ is a \dfn{equivariant morphism} with respect to $\alpha,\beta$ if for every $a \in A$,
    $$\varphi(\alpha(x)(a)) = \beta(x)(\varphi(a)).$$
\end{definition}
    The category of $C^*$-dynamical systems over $G$ has equivariant morphisms as its morphisms by definition. Equivariant morphisms $\varphi:A \to B$ give rise to maps $C_c(G \to A) \to C_c(G \to B)$ given by
    $$\varphi(f)(x) = \varphi(f(x)).$$
    This naturally extends to the group $C^*$-algebras, so gives rise to a morphism $\varphi: A \rtimes_\alpha G \to B \rtimes_\alpha G$. Therefore the following theorem holds.
\begin{theorem}
    The map $\alpha \mapsto A \rtimes_\alpha G$ is a functor from the category of $C^*$-dynamical systems over $G$ to the category of $C^*$-algebras.
\end{theorem}
    Using category theory, we can obtain the following theorem.
\begin{theorem}
    Let $(A, \alpha)$ be a $C^*$-dynamical system. Let $I$ be an $\alpha$-invariant ideal of $A$. Then the natural action of $\alpha$ on $A/I$is a $C^*$-dynamical system, and the natural arrows
    $$0 \to I \rtimes_\alpha G \to A \rtimes_\alpha G \to (A/I) \rtimes_\alpha G \to 0$$
    form a short exact sequence.
\end{theorem}
    In other words, the functor $\cdot \rtimes_\alpha G$ is exact on $\alpha$-invariant ideals. This is not true for the reduced product.
\begin{proof}
    Consider the short exact sequence
$$\begin{tikzcd}0 \arrow[r]& I\arrow[r,"i"] & A\arrow[r,"p"] & A/I\arrow[r] &0\end{tikzcd}.$$
    Straight from the definitions, the induced map $p^*$ is onto and $p^*(i^*(C_c(G, A))) = 0$. So $i^*$ maps into the kernel of $p^*$.

    We now claim that the induced map $i^*$ is injective. Let $\sigma$ be a faithful representation of $I \rtimes_\alpha G$. Let $(\pi, U)$ the covariant representation of $\sigma$. Then if $\sigma$ is nondegenerate, so is $(\pi, U)$. Then $\pi$ extends to a representation $\tilde \pi$ of $A$. Since $\pi$ is nondegenerate, we can restrict to the image of $\pi(I)$ without any loss of generality when proving that $(U, \tilde \pi)$ is covariant. In fact,
    $$U(x)\tilde \pi(a)(\pi(d)\xi) = \pi(\alpha(x)(ad))U(x)\xi$$
    which proves covariance of $(U, \tilde \pi)$. So let $\tilde \sigma$ be a representation of $A \rtimes_\alpha G$ for which $(U, \tilde \pi)$ is a covariant representation. Then
    $$\tilde\sigma|_{i^*(I \rtimes_\alpha G)} = \sigma,$$
    so $\tilde \sigma \circ i^*$ is faithful on $I \rtimes_\alpha G$. Therefore $\ker i^* = 0$. In particular $I \rtimes_\alpha G$ is isomorphic to an ideal of $A \rtimes_\alpha G$, and without loss of generality we assume that they are equal (i.e. $i^*$ is the identity).

    Finally we show exactness at $A \rtimes_\alpha G$; i.e. $\ker p^* \subseteq I \rtimes_\alpha G$. Since $I \rtimes_\alpha G$ is a $C^*$-algebra, $A \rtimes_\alpha G/I \rtimes_\alpha G$ exists, and has a faithful representation $\sigma$. Pulling $\sigma$ back along the quotient map $A \rtimes_\alpha G \to A \rtimes_\alpha/I \rtimes_\alpha G$, we obtain a representation of $A \rtimes_\alpha G$. Let $(\pi, U)$ be the corresponding covariant representation of $A \rtimes_\alpha G$.

    Let $d \in I$, $h \in C_c(G)$. Let $f \in C_c(G \to I) \subseteq I \rtimes_\alpha G$ be defined by
    $$f(x) = h(x)d$$
    so $\sigma(f) = 0$. Thus
    $$0 = \int_G \pi(d)f(x)U(x) ~dx = \pi(d) \int_G f(x)U(x) ~dx.$$
    Since the integral on the right does not have to be zero, $\pi(d) = 0$. So $I \subseteq \ker \pi$. Therefore $\pi$ drops to a representation $\tau$ of $A/I \rtimes_\alpha G$. It is routine to prove that $\sigma = \tau \circ p^*$. So $\ker \sigma = \ker p^* \subseteq I \rtimes_\alpha G$.
\end{proof}
    Note that the representation $\tau$ may not exist on the reduced product, which explains why the theorem fails there.

\section{Group actions on locally compact spaces}
    Let $A$ be a commutative $C^*$-algebra and let $G$ be a locally compact group which acts on $A$ by $\alpha$. Then we can find a locally compact Hausdorff space $M$ such that $A = C_\infty(M)$. We have an action $\alpha$ of $G$ on $M$ by homeomorphisms, and $G \times M$ is locally compact. We will assume that $\alpha$ is \dfn{jointly continuous}, i.e. the map
\begin{align*}
    G \times M &\to M\\
    (x, m) &\mapsto \alpha(x)m
\end{align*}
    is continuous. Thus the action
    $$\alpha(x)(f)(y) = f(\alpha(x)^{-1}y)$$
    of $G$ on $C_c(M) \subseteq A$ is isometric, in particularly, strongly continuous in $L^\infty$-norm. Moreover, $\alpha$ is continuous in the inductive limit topology of $C_c(M)$. Therefore if $\mu$ is an $\alpha$-invariant Radon measure on $M$, $\alpha$ acts strongly continuously on $L^p(\mu)$.

    We now consider $A \rtimes_\alpha G$, which contains $C_c(G \to A)$. Since
    $$(f *_\alpha g)(y) = \int_G f(x)\alpha(x)(g(x^{-1}y)) ~dx,$$
    it follows that
    $$(f *_\alpha g)(y)(m) = \int_G f(x)(m) g(x^{-1}y)(\alpha(x)^{-1}m) ~dx.$$
\begin{theorem}
    Let $M$ be a second-countable locally compact Hausdorff space, $A = C_\infty(M) \rtimes_\alpha G$. Let $\sigma$ be an irreducible nondegenerate representation of $A$ and let $(\pi, U)$ be the covariant representation of $\sigma$, $I = \ker \pi$. Let $Z \subseteq M$ be the hull of $I$. Then there is an $\alpha$-orbit whose closure is $Z$.
\end{theorem}
    Notice that $I$ is $\alpha$-invariant, hence an ideal of $A$. Expanding out the definitions, $I$ is the set of $f$ whose supports are disjoint from $Z$. In particular, $Z$ is closed and $\alpha$-invariant and $A/I = C_\infty(Z)$. The theorem says that there is a $m_0 \in M$ such that
    $$Z = \overline{\{\alpha(x)m_0: x \in G\}}.$$
\begin{example}
    For the irrational rotation, every orbit-closure is the entire circle, so for every ideal, the hull is the entire space. This generalizes to various ergodic actions.
\end{example}
\begin{proof}[Proof of theorem]
    Since $M$ is second-countable, so is $Z$. Let $\{B_n\}_n$ be open subsets of $M$ such that the $B_n \cap Z$ form a countable base for the topology of $Z$, $B_n \cap Z$ nonempty. Let
    $$O_n = \bigcup_{x \in G} \alpha(x)(B_n).$$
    Then the $O_n$ are open, $\alpha$-invariant, and meet $Z$.

    Let $J_n = C_\infty(O_n) \subseteq C_\infty(M)$. Since $O_n \cap Z$ is nonempty, there is a $f \in J_n$ which is not identically zero on $Z$, by Urysohn's lemma. So $\pi(J_n)$ is nonzero. Since $\sigma$ is nondegenerate, so is $\pi$, and $\pi(J_n)H$ generates a nonzero closed $\sigma$-invariant subspace. Since $\sigma$ is irreducible, $\pi(J_n)H$ generates $H$.

    Let $\xi$ be a unit vector of the representation space $H$. Define a Radon probability measure $\mu$ on $M$ by
    $$\int_M f~d\mu = \langle \pi(f)\xi, \xi\rangle.$$
    If $f \in I$, $f = 0$ $\mu$-almost everywhere. Therefore $Z$ contains the support of $\mu$. Since $M$ is second-countable, $J_n$ is a separable $C^*$-algebra and we can find a countable normalized approximate unit $\{e_{n,m}\}_m$ of $J_n$. We can assume that the $e_{n,m}$ are compactly supported in $O_n$. Since $\pi|_{J_n}$ is nondegenerate,
    $$\lim_{m \to \infty} \pi(e_{n,m})\xi = \xi.$$
    The $e_{n,m}$ are supported on $O_n$, so $O_n$ contains the support of $\mu$. Therefore
    $$\supp \mu \subseteq \bigcap_{n=1}^\infty O_n \cap Z.$$
    (Here we are using the cardinality assumption; the complements must be $\mu$-null and there are only countably many of them.)
    Since $\mu$ is a probability measure, $\supp \mu$ is nonempty.

    Let $m_0 \in \supp \mu$. Each of the $O_n$ is $\alpha$-invariant, so $\alpha_G(m_0) \subseteq \supp \mu$. So for each $n$, $\alpha_G(m_0) \subseteq O_n \cap Z$. Since $\alpha_G(m_0)$ is contained in every element of an open base of $Z$, $\alpha_G(m_0)$ is dense in $Z$.
\end{proof}
\begin{example}
    Let $M$ be the two-point compactification of $\RR$. Then the action of $\RR$ on $M$ by translation is jointly continuous, and $\RR$ is a dense orbit, but the boundary points $\pm \infty$ are fixed points. So not every point has a dense orbit. We will study $C(M) \rtimes \RR$ soon.
\end{example}
    Let $\alpha$ be an action of $G$ on $M$. For $m \in M$, let $G_m$ be the stabilizer of $m$. By the orbit-stabilizer theorem, the map
\begin{align*}
    G/G_m &\to \alpha_G(m)\\
    x &\mapsto \alpha_{xG_m}(x)
\end{align*}
    is a bijection (where $xG_m$ is the coset of $G_m$ by $x$). Now $G_m$ is a closed normal subgroup so $G/G_m$ is a locally compact group. In general the orbit-stabilizer map $G/G_m \to \alpha_G(m)$ is not a homeomorphism. It is favorable that the orbit of $x$ is open in its closure, in which case the orbit is locally compact.
\begin{theorem}
    Let $m_0 \in M$. If $G$ is a second countable group which acts on $M$ by $\alpha$, and $\alpha_G(m_0)$ is locally compact, then $G/G_{m_0} \to \alpha_G(m_0)$ is a homeomorphism.
\end{theorem}
\begin{proof}
    Use the Baire category theorem on the locally compact space $\alpha_G(m_0)$.
\end{proof}
    Let $H$ be a closed normal subgroup and let $M = G/H$. Then $G$ acts on $M$ by left translation, and $A = C_\infty(M) \rtimes G$ is a $C^*$-algebra. If $H = G$, then $A = C^*(G)$. If $H = 0$, then $A = C_\infty(G) \rtimes G$.

    In case $H = 0$, we study the covariant representation on $L^2(G)$ given by $U$ the left regular representation, $\pi$ the representation by pointwise representation; i.e.
        $$\pi(f)(\xi)(x) = f(x)\xi(x).$$
    To see covariance, we compute
    $$U(x)\pi(f)(\xi)(y) = \pi(f)(\xi)(x^{-1}y) = f(x^{-1}y) \xi(x^{-1}y) = \pi(\alpha(x)(f))(U(x)(\xi))(y).$$
\begin{definition}
    The \dfn{Schrodinger representation} of a group $G$ is the covariant representation $(\pi, U)$ of $C_\infty(G) \rtimes_G$ on $L^2(G)$ given above.
\end{definition}
    We compute the integrated form $\sigma$ of the Schrodinger representation by realizing that
    $$C_c(G \to A) = C_c(G \to C_\infty(G))$$
    is generated by $C_c(G \times G)$. Given $F \in C_c(G \times G)$ we have
    \begin{align*}\sigma(F)(\xi)(x) &= \left(\int_G \pi(F(y))U(y)(\xi) ~dy \right)(x)
    \\&= \int_G F(y, x) \xi(y^{-1}x) ~dy.
    \end{align*}
    Now $f, g \in C_c(G)$ can be viewed as elements of $L^2(G)$, which has a rank-$1$-operator-valued inner product $\langle\cdot,\rangle\cdot_0$. In fact,
    \begin{align*}\langle f, g\rangle_0(\xi)(x) &= f(x) \langle g, \xi\rangle(x) \\
    &= f(x) \int_G \overline{g(y)}\xi(y) ~dy
    \\&= \int_G f(x) \overline{g(y^{-1})} \xi(y^{-1}) \Delta(y^{-1}) ~dy\\
    &= \int_G f(x) g(y^{-1}x) \xi(y^{-1}x) \Delta(y^{-1}x) ~dy.
    \end{align*}
    We define
    $$\langle f, g\rangle_E (x, y) = f(x) \overline{g(y^{-1}x)} \Delta(y^{-1}x).$$
    Then this is an inner product which has values in $C_c(G \times G)$. Let $E$ be the (algebraic) span of
    $$\{\langle f, g\rangle_E: f,g \in C_c(G)\}.$$
    Thus
    $$\langle f, g\rangle_E * \langle h, k\rangle_E = \langle \langle f, g\rangle_E h, k\rangle_E$$
    and
    \begin{align*}\pi(\langle f, g\rangle_E) \pi(\langle h, k\rangle_E) &= \langle f, g\rangle_0 \langle f, g\rangle_0 = \langle \langle f, g\rangle_0 h, k\rangle_0 \\
    &= \langle g, h\rangle \langle f, k\rangle_0 = \pi(\langle g, h\rangle \langle f, k\rangle_E)
    \end{align*}
    allows us to define
    $$\langle f, g\rangle_E * \langle h, k\rangle_E = \langle g, h\rangle \langle f, k\rangle_E.$$
    This defines a convolution on $E$ which is compatible with the convolution on $C^*(G, C_\infty(G))$. Therefore $E$ is a subalgebra of $C^*(G, C_\infty(G))$. Clearly $E$ is a $*$-algebra since
    $$\langle f, g\rangle_E = \langle g, f\rangle_E.$$
    Moreover, $C_c(G)$ is dense in $L^2(G)$, so $\pi(E)$ is dense in $B_0(H)$.

    We claim that $E$ is also closed under pointwise multiplication. In fact,
\begin{align*}
    \langle f, g\rangle_E\langle h, k\rangle_E(x, y) &= f(x)\overline{(\Delta g)}(y^{-1}x) h(x) \overline{(\Delta k)}(y^{-1}x)\\
        &= f(x)g(x) \overline{\Delta g}k(y^{-1}x) \Delta(y^{-1}x).
\end{align*}
    Clearly $E$ is closed under complex conjugation and separates points of $G \times G$ from zero. Thus we can apply the Stone-Weierstrass theorem, but this is not very interesting because we actually want to prove that $E$ is dense in $C_c(G \times G)$ for the inductive limit topology. In fact, if $O$ is an open, precompact subset of $G \times G$, we can find $V \times W \subseteq G \times G$, where $V,W$ are open, precompact subsets of $G$ and consider the algebraic span of $\langle C_c(V), C_c(W)\rangle_E$. By the Stone-Weierstrass theorem, this is $L^\infty$-dense in $C_c(V \times W)$. One can then check that $E$ is dense in $L^1(G \to C_\infty(G))$ and hence dense in $C^*(G, C_\infty(G))$.

    We claim that $E$ has the same operator norm as $C^*(G, C_\infty(G))$. In fact if $f_1, \dots, f_n$ are an $L^2$-orthonormal set in $C_c(G)$ then the $\langle f_j, f_k\rangle_0$ span the $C^*$-algebra $\CC^{n \times n}$ once we choose a basis. On $C^*$-algebras the operator norm is uniquely determined, so $E$ agrees with $C^*(G, C_\infty(G))$ in operator norm on any finite-dimensional subalgebra. Such matrix algebras can be used to approximate $C^*(G, C_\infty(G))$ so we have proven the claim. We consider that we have proven the following theorem.
\begin{theorem}
    $C_\infty(G) \rtimes G = B_0(L^2(G))$.
\end{theorem}
    Since $B_0(L^2(G))$ has no proper ideals, one also has $C_\infty(G) \rtimes^r G = B_0(L^2(G))$. Therefore the translation action is amenable.
\begin{example}
    If $G$ is not an amenable group, then $G$ still admits an amenable action by translation.
\end{example}
    Now if $G$ acts on $X$ by $\alpha$, and $O$ is an orbit of $\alpha$, then if $O$ is an orbit which is open in its closure, $C_\infty(O) \subseteq C(\overline O)$. Moreover, $C_\infty(O) \rtimes_\alpha G = B_0(L^2(G))$.
\begin{example}
    Let $\RR$ act on its two-point compactification $X$ by translation. Then $C_\infty(X) = C(X)$ contains $C_\infty(\RR)$. So
    $$C(X) \rtimes \RR \supset C_\infty(\RR) \rtimes \RR = B_0(L^2(\RR))$$
    which gives a GCR representation of $C(X)$ on $L^2(\RR)$. It is not CCR because $C(X)$ is unital.

    If we instead look at the orbits of $\pm \infty$, we see that $C^*(\RR) = C(\pm\infty) \rtimes_\alpha \RR$. By the Fourier transform, $\widehat{C^*(\RR)} = \RR$. (More generally, if $G$ is a locally compact abelian group, then $C^*(G)$ is a commutative $C^*$-algebra, $\widehat{C^*(G)}$ consists of one-dimensional representations of $G$, which are exactly the continuous morphisms $G \to S^1$.)
\end{example}
\begin{definition}
    Let $H$ be a closed normal subgroup of $G$. For simplicity we assume that the Haar measure on $G/H$ is $G$-invariant. Let $V: H \to U(K)$ be a unitary representation. We define a Hilbert space by taking all functions $\xi: G \to K$ such that for all $x \in G$, $s \in H$, $\xi(xs) = V(s)^* \xi(x)$ where we define $\langle \xi, \eta\rangle(x) = \langle \xi(x), \eta(x)\rangle$. Since $\langle \xi, \eta\rangle$ is constant on cosets, it drops to a function on $G/H$ such that
    $$\langle \xi, \eta\rangle(x) = \int_{G/H} \langle \xi, \eta\rangle(x) ~dx.$$
    We take the Hilbert space to be all $\xi$ such that $\langle \xi, \xi\rangle < \infty$, which $G$ acts on by left translation. This action of $G$ is called the \dfn{induced representation} of $G$ from $V$, $\Ind V$.
\end{definition}
    If $H,V$ are as above, $\alpha$ the action of $G$ on $C_\infty(G/H)$ by left translation, then we obtain a covariant representation of $\alpha$ on the induced representation space by letting $C_\infty(G/H)$ act by left translation and $\lambda$ be the left action of $G$ on $G/H$. Then $(C_\infty(G/H), \lambda)$ is a covariant representation of $\alpha$.

    Let $G$ be a unimodular group (though the same argument goes through without too much trouble otherwise). Recall that $C_c(G/H \to G) \subseteq C(G/H) \rtimes_\alpha G$. We define for $f, g \in C^*(H)$,
    $$\langle f, g\rangle_{C^*(H)} = f^* * g|_H.$$
    Here we are using continuity; if $H$ is a Haar null set then the restriction map is not defined for measurable functions in general. Let $B = C_c(G/H \to G)$. Then we can define
    $$\langle f,g\rangle_B h = f \langle g, h\rangle_{C^*(H)}.$$
    One can then prove $C(G/H) \rtimes_\alpha G$ is strongly Morita equivalent to $C^*(H)$. This theory generalizes to when $G$ is merely a groupoid rather than a group.
\section{Semidirect products of groups}
\begin{definition}
    If $N$ and $Q$ are locally compact groups, and $\alpha: Q \to \Aut(N)$ a jointly continuous action, then we define $N \rtimes_\alpha Q$ as follows. As a Hausdorff space, $N \rtimes_\alpha Q$ is the product of topological spaces $N \times Q$. The group operation is defined by
    $$(n_1, q_1)(n_2, q_2) = (n_1\alpha(q_1)(n_2), q_1q_2).$$
    Then $N \rtimes_\alpha Q$ is the \dfn{semidirect product} of locally compact groups.
\end{definition}
    We remember the group operation on $N \rtimes_\alpha Q$ by recalling that ``whenever we want to commute an $n$ and a $q$, the $q$ must act on the $n$."

    Let $G = N \rtimes_\alpha Q$. Then $Q$ and $N$ embed in $G$ in the obvious way and we have a split exact sequence
    $$0 \to N \to G \to Q \to 0.$$
    Therefore any representation of $G$ restricts to representations of $N$ and $Q$. Moreover, $Q$ acts on $N$ by inner automorphisms, i.e.
    $$qnq^{-1} = \alpha(q)(n).$$
    The action of $q$ does not preserve Haar measure, but it does send Haar measure to a translation-invariant measure; i.e. it multiplies Haar measure by a scalar, say $\sigma(q)$.

    We now define an action of $Q$ on $C^*(N)$. In fact, for $f \in C_c(N)$, $n \in N$, $q \in Q$,
    $$\alpha(q)(f)(n) = \sigma(q)f(\alpha(q)(n)).$$
    Then $\alpha(q)$ is an isometry in $L^1$-norm, so $Q$ acts on $L^1(N)$ by isometries. This action $\alpha$ immediately extends to $C^*(N)$.

    Let $U$ be a unitary representation of $G$. Restricting, we obtain representations of $C^*(N)$ and $Q$, and $U$ is a covariant representation of $\alpha: Q \to \Aut(C^*(N))$. Conversely, a representation of $C^*(N) \rtimes_\alpha Q$ gives rise to a covariant representation of $\alpha$. Then
    $$C^*(N) \rtimes_\alpha Q = C^*(G).$$
    This can be remembered as $C^*(N) \rtimes_\alpha Q = C^*(N \rtimes_\alpha Q)$.

    Let $N$ be an abelian group. Then $C^*(N) = C_\infty(\hat N)$ (where $\hat N$ denotes the Fourier transform, $\hat N = \Hom(N, S^1)$ in the category of locally compact groups). Since $Q$ acts on $N$ and hence $C^*(N)$, $Q$ also acts on $\hat N$. In fact if $\varphi \in \hat N$, then
    $$\alpha(q)(\varphi)(n) = \varphi(\alpha(q^{-1})(n)).$$
    Then $C^*(N \rtimes_\alpha Q) = C_\infty(\hat N) \rtimes_\alpha Q$. Thus we are back in the original situation of a locally compact group acting on a locally compact space.
\begin{example}[Wigner 1939]
    Let $L$ be the \dfn{Lorentz group}, the automorphism group of Minkowski spacetime (linear automorphisms that preserve the Lorentzian metric $g(x, y) = -x_0y_0 + x_1y_1 + x_2y_2 + x_3y_3$.) Then $L$ acts on $\RR^4$, so we have a semidirect product $\RR^4 \rtimes L$, the \dfn{Poincare group}. The unitary representations of the Poincare group are important in relativistic quantum mechanics. Elementary particles ``should be" completely determined by their symmetries, so correspond to representations of certain stabilizers of $\RR^4 \rtimes L$. This paper led to the discovery that electrons have spin. In principle one could use the representation theory of $\RR^4 \rtimes L$ to rederive the periodic table of elements.
\end{example}

\section{The Heisenberg commutation relations}
We now look at an algebra with ``invalid" generators and relations.

In quantum physics, the position $q$ and momentum $p$ observables act on certain dense subspaces of tensor powers of $L^2(\RR)$ with
    $$[p, q] = i\hbar.$$
They must be unbounded operators, since their commutator is a scalar. So $q, p$ are not elements of a $C^*$-algebra. This relation is called the \dfn{Heisenberg commutation relation}.

We want to be able to form the holomorphic functional calculus for an unbounded operator $T$. In particular, we would like to define a one-parameter unitary group by the group morphism $t \mapsto e^{itT}$.
\begin{example}
    The Schrodinger equation is the PDE that says that if $H$ is the Hamiltonian, the action of its one-parameter unitary group $t \mapsto e^{itH}$ is the time-advance map.
\end{example}
Reasoning just formally about how the holomorphic functional calculus should behave, we let $U(s) = e^{isP}$ and $V(t) = e^{itQ}$. Weyl observed that
$$U(s) V(t) U(s)^* = e^{itU(s)QU(s)^*}$$
so we let
$$\varphi(s) = U(s)QU(-s).$$
Then
$$\varphi'(s) = iU(s)(PQ - QP)U(-s) = -\hbar.$$
So $\varphi(s) = Q - s\hbar$ whence
$$U(s)V(t) = e^{itQ}e^{-i\hbar ts}U(s) = e^{-i\hbar st} V(t) U(s).$$

Recall that $\hat \RR \cong \RR$ (noncanonically). If we let $\langle\cdot,\cdot\rangle$ be the pairing of $\RR$ and $\hat \RR$, then we have just proved
$$U(s)V(t) = \langle s, t\rangle V(t)U(s).$$
Here the choice of isomorphism is induced by some normalization of the Fourier transform.

\begin{definition}
Let $G$ be an locally compact abelian group. By a \dfn{representation for the Heisenberg commutation relations} of $G$ we mean a pair of unitary representations $(U, V)$, $U: G \to \Aut(H)$, $V: \hat G \to \Aut(H)$, such that
$$U(s)V(t) = \langle s, t\rangle V(t)U(s).$$
\end{definition}
Any unitary representation $V$ of $\hat G$ lifts to a representation $\pi$ of the commutative $C^*$-algebra $C^*(\hat G)$, which is $C_\infty$ of the double dual of $G$. By the Pontryagin duality theorem, the double dual of any locally compact abelian group is itself, so $C^*(\hat G) = C_\infty(G)$. Let $f = \hat h \in C_\infty(G)$ for some $h \in L^1(G)$. Then
$$\pi(f) = \int_{\hat G} h(t) V(t) ~dt.$$
Therefore
\begin{align*}
    U(s)\pi(f)U(s)^* &= \int_{\hat G} h(t) U(s)V(t)U(s)^* ~dt = \int_{\hat G} h(t) \langle s, t\rangle V(t) ~dt\\
        &= \pi(\alpha_s(f))
\end{align*}
where $\alpha$ is the action of $G$ on $C_\infty(G)$ by left translation. So $(\pi, U)$ is a covariant representation of $\alpha$ and hence a representation of $C_\infty(G) \rtimes_\alpha G = B_0(L^2(G))$. But $B_0(L^2(G))$ only has one irreducible representation, which turns out to be the Schrodinger representation. This shows that the Heisenberg picture and the Schrodinger picture are equivalent. This is a theorem of von Neumann which was important to the foundations of physics.

\section{Projective representations}
Let $W: G \times \hat G \to U(H)$ be defined by
$$W(s, t) = U(s)V(t),$$
where $(U, V)$ is a representation for the Heisenberg commutation relations. Then
\begin{align*}
W(s, t) W(s', t') &= U(s)V(t)U(s')V(t') = -U(s + s') V(t) \langle s', t\rangle V(t')\\
&= \langle s', t\rangle U(s + s') V(t + t') = \langle s', t\rangle W(s + s', t + t').
\end{align*}

\begin{definition}
    Let $G$ be a group. A \dfn{projective representation} of $G$ is a continuous function $W: G \to U(H)$ defined by
    $$W(x)W(y) = c(x, y)W(xy)$$
    for some $c: G^2 \to S^1$.
\end{definition}
Here we are thinking of $S^1$ as the unit circle subgroup of $\CC$. It is a morphism up to a harmless constant. In fact, if $\PP H$ is the projective space of some Hilbert space, then every projective representation drops to a morphism of groups $G \to \Aut(\PP H)$, since it permutes the one-dimensional subspaces. Wigner proved that every automorphism of $\PP H$ is given by a unitary or antiunitary (i.e. conjugate linear) operator. If $P$ is a rank-$1$ projection, then $P$ is sent to $UPU^{-1}$ by any such automorphism, for $U$ a unitary or antiunitary operator.
\begin{example}
    Charge-conjugation, parity, and time-reversal are examples of antiunitary operators in quantum field theory.
\end{example}
So we have constructed a projective representation of $G \times \hat G$.

Up to a normalization we may assume $U_1 = 1$.

Given $d: G \to S^1$, set $V(x) = d(x)U(x)$. Then
$$V(x)V(y) = d(x)d(y)U(x)U(y) = d(x)d(y)\overline{d(xy)}V(xy).$$
Associativity of $\Aut(\PP H)$ manifests as
$$c(xy, z) c(x, y) = c(x, yz) c(y, z).$$
We say that $c$ is a \dfn{$2$-cocycle} for $G$ valued in $S^1$.

\begin{example}
    Let $C_k$ be the set of all functions $G^k \to S^1$. The \dfn{boundary operator for the homology of groups} with values in $S^1$ is defined by $\partial: C_1 \to C_2$ by
    $$\partial d(x, y) = d(x)d(y)d(xy),$$
    and $\partial: C_2 \to C_3$ by
    $$\partial d(x, y, z) = c(xy, z)c(x, y)\overline{c(x, yz)c(y, z)}.$$
    This extends to a homology theory for all $k$. Here $S^1$ can be replaced by any abelian group. If $c' = (\partial d) c$, then $c'$ and $c$ are homologous.
\end{example}

Assume that for all $\xi \in H$, $x \mapsto U(x)(\xi)$ is measurable. Then for every $f \in L^1(G)$, we define
$$U(f)(\xi) = \int_G f(x)U(x)(\xi) ~dx.$$
Then $||U(f)|| = ||f||_{L^1}$ and $U(f)U(g) = U(f*_cg)$ where $*_c$ is the twisted convolution defined by
$$f *_c g(x) = \int_G f(y) g(y^{-1}x)c(y, y^{-1}x) ~dy.$$
So $L^1(G, c)$ (which is $L^1$ with the twisted convolution $*_c$) is a Banach algebra, which is not commutative even if $G$ is abelian. If $c$ is homologous to $c'$ then we have an isomorphism $L^1(G, c) \to L^1(G, c')$.

Given a $2$-cocycle $c$ we can consider all projective representations of $G$ where the $2$-cocycle is homologous to $c$. In these cases, the isomorphism of Banach algebras above implies that we can assume that the $2$-cocycle is actually $c$. For $f \in L^1(G, c)$, this defines the $C^*$-norm by
$$||f||_{C^*(G, c)} = \sup_U ||U(f)||$$
where $U$ ranges over all projective representations whose cocycles are homologous to $c$. Here we have a twisted adjoint, which for unimodular groups can be explicitly expressed as
$$f^*(x) = \overline{f(x) c(x, x^{-1})}.$$

We have a left regular representation $L: L^1(G, c) \to L^2(G)$ defined by
$$L(f)(\xi)(x) = \int_G f(y) \xi(y^{-1}x) c(y, y^{-1}x) ~dy.$$
This gives rise to the reduced $C^*$-algebra $C^*_r(G, c)$.

\begin{example}
    Let $G = \RR^n \times \widehat{\RR^n}$. We define the cocycle $c((x, s), (y, t)) = \langle (x, s), (y, t)\rangle$. We already saw that $C^*(G, c) \cong B_0(L^2(G))$ in an unnatural way, by uniqueness of the Heisenberg commutation relation. Now $G$ is abelian, but $B_0(L^2(G))$ is far from commutative.
\end{example}

\begin{example}
    Let $G = \ZZ^n$. Then every cocycle is homologous to a \dfn{bicharacter}, a cocycle $c$ of the form
    $$c(m, \ell) = e^{\langle im, \Theta \ell\rangle}$$
    where $\Theta \in \RR^{n \times n}$.

    When we study $G$ we will assume without loss of generality that $c$ is a bicharacter. Henceforth we will mainly be interested in discrete groups, but really we are actually studying $\ZZ^n$.
\end{example}
    If $G$ is a discrete group with a cocycle $c$, we can define a faithful tracial state $\tau$ on $\ell^1(G, c)$ (hence on $C^*(G, c)$) by $\tau(f) = f(e)$. Moreover, $\delta_1$ is the identity of $\ell^1(G, c)$, and $\tau(f *_c f^*) = \sum_{y \in G} |f(y)|^2$. From this it follows that the GNS construction for $\tau$ gives a faithful representation $\ell^1(G, c) \to \ell^2(G)$, so extends to a representation $C^*(G, c) \to \ell^2(G)$.

    If $G$ is discrete and abelian, then $\hat G$ is compact, and we have an action $\hat \alpha: G \to \Aut(C^*(G, c))$,
    $$\hat \alpha(t)(f)(x) = \langle x, t\rangle f(x).$$
    Then
    $$\hat \alpha(t)(f *_c g)(x) = \hat\alpha(t)(f) *_c \hat\alpha(t)(g)(x).$$

    To study the properties of this action $\hat \alpha$, $G$ be a compact group with its Haar probability measure, and $\alpha: G \to \Aut A$ an action. Then we define $P: A \to A$,
    $$P(a) = \int_G \alpha(x)(a) ~dx.$$
    Then $P$ is $\alpha$-invariant, $\alpha(y)(P(a)) = P(a)$. In particular, if we let $A^G$ be the algebra of all fixed points of $\alpha$, $P$ carries $A$ into $A^G$. Conversely, if $a$ is actually a fixed point, then $P(a) = a$. So $P$ is the projection map $A \to A^G$.
\begin{definition}
    Assume $B \subseteq A$, and $P: A \to B$ is a projection. If for every $b \in B$, $a \in A$, $P(ab) = P(a)b$ and $P(ba) = bP(a)$, we say that $P$ is a \dfn{conditional expectation}.
\end{definition}
    It is easy to check that the projection $P: A \to A^G$ is a conditional expectation. Moreover, if $a > 0$ then $P(a) > 0$.

    If $G$ is a compact abelian group (in applications, $G$ is usually a torus), then $\hat G$ is discrete (in applications, $\ZZ^n$). We let $\alpha$ be an action of $G$ on $A$. For each $t \in \hat G$, set
    $$a_t = \int_G \langle x, t\rangle \alpha_x(a) ~dx.$$
    So the $a_t$ are the generalized Fourier coefficients of $a$. We have
    $$\alpha(y)(a_t) = \alpha(y)\left(\int_G \overline{\langle x, t\rangle} \alpha(x)(a) ~dx\right) = \int_G \overline{\langle y^{-1x}, t\rangle} \alpha(x)(a) ~dx = \langle y, t\rangle a_t.$$
    We now set $A_t = \{a \in A: \forall y ~\alpha(y)(a) = \langle y, t\rangle a\}$. Then $a_t \in A_t$ and $A_t$ is a closed subspace, hence a $C^*$-algebra. For $a \in A_t$, $b \in A_s$, we have
    $$\alpha(y)(ab) = \alpha(y)(a)\alpha(y)(b) = \langle y, t\rangle \langle y, s\rangle ab = \langle y, ts\rangle ab$$
    so $ab \in A_{ts}$.

\chapter{Noncommutative geometry}
\section{Quantum tori}
Fix $\Theta \in \RR^{d \times d}$ and let $c_\Theta(m, n) = e^{2\pi im\cdot\Theta n}$, for $(m, n) \in \ZZ^{d + d}$. Then $c_\Theta$ is a cocycle for the duality of $\ZZ^d$ and the torus $T^d$. In fact the pairing is given by
$$c_\Theta = \langle m, \Theta n\rangle.$$
\begin{definition}
    The $C^*$-algebra $A_\Theta = C^*(\ZZ^d, c_\Theta)$ is called the algebra of functions on the \dfn{noncommutative torus} or \dfn{quantum torus} of dimension $d$.
\end{definition}
Now
$$\delta_m * \delta_n * \delta_m^* = \langle n, (\Theta - \Theta^t)m\rangle \delta_n.$$
So we can reasonably define
$$\rho_\Theta(m) = (\Theta - \Theta^t)m \in T^d.$$
We now define
$$H_\Theta = \overline{\{\rho_\Theta(m) \in T^d: m \in \ZZ^d\}}.$$
Then $H_\Theta$ is an subgroup of $T^d$, so $\rho_\Theta: \ZZ \to T^d$ is a morphism of groups. It gives rise to an action $\alpha$ of $H_\Theta$ on $A$ defined by
$$\delta_m a \delta_m^* = \alpha(\rho_\Theta(m))(a).$$

\section{The 2-torus}
\begin{theorem}
    If $H_\Theta = T^d$ then $A_\Theta$ is simple.
\end{theorem}
\begin{proof}
If $I$ is a closed ideal of $A_\Theta$ then $I$ is closed under conjugation, hence under the action of $H_\Theta$. Now $A_\Theta$ is a space of noncommutative functions on $T^d$ so we're done.
\end{proof}

\begin{example}
    If $d = 2$, we take $\Theta = \begin{bmatrix}&\theta\\0&\end{bmatrix}$. Then $\Theta - \Theta^t = \begin{bmatrix}&\theta\\-\theta&\end{bmatrix}$. So
    $$\rho_\Theta(m_1, m_2) = (\theta m_2, -\theta m_1).$$
    If $\theta$ is irrational then $\{\theta m_2: m_2 \in \ZZ\}$ is dense in $S^1$. Therefore $\rho_\Theta$ acts densely on $T^2$. So if $\Theta$ is irrational then $A_\Theta$ is simple.

    Let $M = S^1$, $\theta \in \RR$, $\alpha$ the action of $\ZZ$ on $M$ by rotation by $\theta$. Then $C(M) \rtimes_\alpha \ZZ$ is a rotation algebra, i.e. it is the universal $C^*$-algebra generated by a unitary, namely $U = e^{2\pi i t}$. If $V$ is the unitary acting on $C(M)$ by $Vf = \alpha(1)(f)$ (so translation by $\theta$), then $VU = e^{2\pi i\theta}UV$. One can then show that
    $$C(M) \rtimes_\alpha \ZZ = C^*(\ZZ^2, c_\Theta)$$
    where $\Theta = \begin{bmatrix}&\theta\\-\theta&\end{bmatrix}$. So this is another construction of $A_\Theta$.

    If $\theta$ is irrational, then $\alpha$ is a free action (i.e. all stabilizers are trivial). We now define a morphism $C(M) \to C_b(\ZZ)$ by $\tilde f(n) = f(\alpha(n)(t_0))$ for some fixed $t_0 \in M$. Now $C_b(\ZZ)$ acts on $\ell^2(\ZZ)$ by multiplication and $\ZZ$ acts on $\ell^2(\ZZ)$ by translation. This gives a covariant representation of $\alpha$ on $\ell^2(\ZZ)$. One can then show using certain commutation relations that the covariant representation is irreducible, hence gives an irreducible representation of $A_\Theta$. This depends on the orbit of $t_0$, so we construct uncountably many irreducible representations of $A_\Theta$, all of which have kernel $0$ since $A$ is simple.

    Thus we have constructed a $C^*$-algebra with lots of irreducible representations that have the same primitive ideal but are not unitarily equivalent. There are even more irreducible representations that we have not treated.

    If $\theta$ is rational then every orbit is finite, and $C(M) \rtimes_\alpha \ZZ$ is a continuous field of $d \times d$ matrix algebras, which is not isomorphic to $C(T^2 \to M^d)$.
\end{example}

\begin{example}
    Let $M$ be a compact Hausdorff space, $G$ a finite group, $\alpha$ a free action of $G$ on $M$. Then $M/\alpha$ is a compact Hausdorff space and we have a Morita equivalence $C(M) \rtimes_\alpha G \to C(M/\alpha)$. ``Most" of noncommutative algebraic topology is only defined up to Morita equivalence, so from the point of view of an algebraic topologist, $C(M) \rtimes_\alpha F = C(M/\alpha)$. This is a very unusual situation!

    If $G$ is an infinite group instead, then $M/\alpha$ may not be Hausdorff (for example, if $G$ is a Lie group which foliates $M$ badly). Then $C(M/\alpha)$ may not be a $C^*$-algebra, so we have no way of studying its algebraic topology. We can still find topological invariants of the dynamical system $\alpha$ by instead studying the topology of $C(M) \rtimes_\alpha G$.
\end{example}

\begin{example}
    Let $H = U + U^* + r(V + V^*)$ where $U,V$ are the generating unitaries of the $2$-dimensional quantum torus. In physics, $U + U^*$ is the potential energy, $r(V + V^*)$ is the kinetic energy, $r$ ``electron coupling", and $H$ the Hamiltonian. Hofstadter (of Godel-Escher-Bach fame) showed that if $\theta$ is rational but with large denominators, then the spectrum of $H$ approximates a Cantor set. So he conjectured that if $\theta$ is irrational then the spectrum is Cantor space. Katz offered $10$ martinis for anyone who could prove this, which was known as the \dfn{ten martinis conjecture}. Avila et al. proved the ten martinis conjecture.
\end{example}

If $\delta_m \in Z(A_\Theta)$ then $\alpha(\rho_\Theta(m))$ is the identity. So for all $n$,
$$1 = \langle n (\Theta - \Theta^t)m \rangle = \langle(\Theta^t - \Theta)n, m\rangle$$
so $m$ lies in the dual group $H_\Theta^\perp$ of $H_\Theta$. One can then show that $Z(A_\Theta) = C^*(H_\Theta^\perp) = C(\widehat{H_\Theta^\perp})$. Then we can express $A_\Theta$ as a continuous field over $C(\widehat{H_\Theta^\perp})$.

\section{Noncommutative smooth manifolds}
We now take the theory of Lie groups and smooth manifolds and turn it into a noncommutative theory.

Let $G$ be a Lie group. We can always assume that $G$ is a closed, connected subgroup of $\GL(\RR^n)$. In fact $\GL(\RR^n)$ can be obtained by applying the exponential map to $\RR^{n \times n}$; i.e. the exponential of a matrix is an invertible matrix. We therefore define $\Lie G = \{X \in \RR^{n \times n}: \forall t \in \RR~e^{tX} \in G\}$. Then $\Lie G$ is a Lie algebra and $\Lie$ is the functor that sends a Lie group to its Lie algebra. Besides, $\exp: \Lie G \to G$ is the exponential map (in the sense of Riemannian geometry), so is close to the identity of $G$ a homeomorphism.

Given $X \in \Lie G$, $t \mapsto e^{tX}$ is a morphism of groups $\RR \to G$; i.e. a \dfn{smooth one-parameter subgroup} of $G$. In fact every one-parameter subgroup is of this form, though we note that $t \mapsto e^{tX}$ may not be injective. (For example $S^1$ is a one-parameter subgroup which is periodic.)

\begin{example}
    $\Lie T^d = \RR^d$.
\end{example}
Let $\alpha$ be a strongly continuous action of $\RR$ by isometries on a Banach space $B$. Let $b \in B$. Then we have a one-parameter semigroup $r \mapsto \alpha(r)(b)$.

\begin{definition}
    Let $G$ be a Lie group and $\alpha$ a strongly continuous action of $G$ by isometries on a Banach space $B$. Given $X \in \Lie G$, $b \in B$, $X \mapsto \alpha(X)(b)$, the \dfn{directional derivative} is
    $$D_Xb = \lim_{r \to 0} \frac{\alpha(\exp(rX))(b) - b}{r}.$$
    We let $B^\infty$ be those $b \in B$ such that every higher directional derivative $D_{X_1} \cdots D_{X_n} b$ exists.
\end{definition}
\begin{theorem}[Garding]
    \index{Garding's theorem}
    Let $f \in C^\infty_{comp}(G)$, $f$ supported in a small enough neighborhood of the identity. Given $b \in B$, then the integrated form $\alpha(f)(b) \in B^\infty$.
\end{theorem}
\begin{proof}
    Let $X \in \Lie G$. Then
\begin{align*}D_X(\alpha(f)(b)) &= \lim_{t \to 0} \frac{\alpha(\exp(tX))(\alpha(f)(b)) - \alpha(f)(b)}{t} \\
    &= \lim_{t \to 0} \frac{1}{t}\left(\alpha(\exp(tX))\int_G f(x)\alpha(x)(b) ~dx - \int_G f(x)\alpha(x)(b) ~dx \right)\\
    &= \lim_{t \to 0} \frac{1}{t}\left(\int_G f(\exp(-tX)x) \alpha(x)(b) ~dx - \int_G f(x) \alpha(x)(b) ~dx\right)\\
    &= \lim_{t \to 0} \int_G \frac{f(\exp(-tX)x) - f(x)}{t} - \alpha(x)(b) ~dx\\
    &= \int_G D_{-X}f(x)\alpha(x)(b) ~dx.
    \end{align*}
    So $\alpha(f)(b)$ is once differentiable. Now use the fact that
    $$D_YD_X\alpha(f)(b) = \alpha(D_YD_Xf)(b)$$
    to see that $\alpha(f)(b)$ twice differentiable and induct.
\end{proof}
\begin{corollary}
    $B^\infty$ is dense in $B$.
\end{corollary}
\begin{proof}
    Choose an action $\alpha$ and let $f_n \in C^\infty_{comp}(G)$ be an approximate identity for $L^1(G)$. Then the sequence of $\alpha(f_n)(b)$ approximates $b$ arbitrarily well.
\end{proof}
    Let $A$ be a Banach algebra. If $\alpha$ is a strongly continuous action by algebra homomorphisms of the Lie group $G$, then for $a,b \in A^\infty$, $X \in \Lie G$, the directional derivative $D_X$ is a derivation of $A$. It is reasonable to think of the space of derivations of $A$ as ``vector fields on the noncommutative space $\hat A$," assuming that the space of derivations has the structure of a $A$-module. But in general, it is only a $Z(A)$-module. Therefore, in general, we cannot define the tangent bundle of a noncommutative smooth manifold.
\begin{example}
    Let $G = T^d$ so $\hat G = \ZZ^d$ and $\Lie G = \RR^d$, and let $\alpha$ be an action of $G$ on a Banach space $B$. Let $b \in B^\infty$. Then
\begin{align*}
    \alpha(f)(D_Xb) &= \lim_{t \to 0} \int_G f(x) \alpha(x)\left(\frac{\alpha(\exp(tX))(b) - b}{t}\right) ~dx\\
        &= \lim_{t \to 0} \int_G \frac{(f(x \exp(-tX)) - f(x))\alpha(x)(b)}{t} ~dx = \alpha(D_Xf)(b)
\end{align*}
    where we used the fact that $G$ is abelian, hence unimodular. (This formula is therefore true for any unimodular group.) The Fourier transform of $b$ is given by
    $$(D_Xb)_n = \alpha(e_n)(D_Xb) = -\alpha(D_Xe_n)(b) = 2\pi inXb_n$$
    where
    $$e_n(t) = e^{2\pi int},$$
    and the multiplication of $\ZZ^d$ and $\RR^d$ is given by the dot product.

    Let the Laplacian $\Delta$ act on $B^\infty$ by
    $$(\Delta b)_n = \sum_j (2\pi)^2 (nE_j)^2 b_n$$
    where the $E_j$ form a basis for $\RR^d$. Then for $k \in \NN$,
    $$((1 + \Delta)^k b)_n = \left(1 + (2\pi)^2 \sum_j (nE_j)^2\right)^k b_n$$
    so
    $$b_n = \frac{((1 + \Delta)^kb)_n}{(1 + (2\pi)^2\sum_j(nE_j)^2)^k}$$
    whence
    $$||b_n|| \leq \frac{||(1+ \Delta)^kb||}{(1 + (2\pi)^2\sum_j (nE_j)^2)^k}.$$
    Therefore if $p$ is a polynomial on $\ZZ^d$,
    $$p(n)||b_n|| \leq \frac{|p(n)|\cdot ||(1 + \Delta)^kb||}{(1 + (2\pi)^2 \sum_j (nE_j)^2)^k}$$
    and if $k$ is large enough, it follows that $n \mapsto |p(n)|||b_n||$ is bounded (since $(1 + \Delta)^kb$ is independent of $n$). Since $p$ can grow arbitrarily fast, the function $n \mapsto ||b_n||$ lies in $C_\infty(\ZZ^d)$. In fact it lies in the Schwartz space of $C_\infty(\ZZ^d)$.
\end{example}
\begin{theorem}
    Let $b \in B$. Then $b \in B^\infty$ if and only if $n \mapsto ||b_n||$ is a Schwartz function on $\ZZ^d$.
\end{theorem}
\begin{example}
    Recall that $A_\Theta = C^*(\ZZ^d, c_\Theta)$. Then $a_n \in \CC$ and $c_\Theta$ is an action of $T^d$, so $A_\Theta^\infty$ is isomorphic to the Schwartz space of $\ZZ^d$.
\end{example}
    We now introduce noncommutative differential forms. Given $a \in A^\infty$, let $da: \Lie G \to A$ be given by
    $$da(X) = \alpha(X)(a).$$
    Then $d$ is a derivation. We let $\tilde \Omega$ be the space of linear maps $\Lie G \to A$, viewed as a $(A, A)$-bimodule. Then let $\Omega$ be the submodule generated by $d$; i.e. linear combinations of elements of the form $a~db$, i.e. $1$-forms on $A$.
\begin{definition}
    Let $A$ be a $C^*$-algebra. The $(A,A)$-bimodule $\Omega$ is known as the \dfn{noncommutative cotangent bundle} of $A$.
\end{definition}
    From this it is not difficult to define the higher exterior power $\Omega^k$ and define the boundary map
    $$d: \Omega^k \to \Omega^{k+1}.$$

\section{Noncommutative vector bundles}
    Let $X$ be a compact Hausdorff space, $E$ a vector bundle over $X$, and let $\Gamma(E)$ be the vector space of continuous sections of $E$. Given $\xi \in \Gamma(E)$ and $f \in C(X)$, $(f\xi)(x) = f(x)\xi(x)$ by scalar multiplication so $\Gamma(E)$ is a $C(X)$-module.

    In this section we will assume all $C(X)$-modules are finitely generated.
\begin{definition}
    Let $R$ be a unital ring. A \dfn{projective module} over $R$ is a $R$-module $V$ which is isomorphic to a direct summand of a free $R$-module.
\end{definition}
    In other words, if $V$ is free then there is a $R$-module $W$ and a free $R$-module $F$ such that $V \oplus W \cong F$.
\begin{theorem}[Swan]
    \index{Swan's theorem}
    Let $X$ be a compact Hausdorff space. A $C(X)$-module $V$ is projective if and only if there is a vector bundle $E$ such that $V \cong \Gamma(E)$. Moreover, we have an isomorphism of vector bundles $E \cong F$ if and only if $\Gamma(E) \cong \Gamma(F)$.
\end{theorem}
    So we have an equivalence of categories relating projective $C(X)$-modules and vector bundles over $X$. Though Swan proved this result in 1962, by this time Grothendieck had already started identifying projective modules with vector bundles over algebraic varieties.

    Let $R$ be a ring. If we view $R^n$ as a right $R$-module, then $\End_R(R^n) \cong M_n(R)$, where $M_n(R)$ is viewed as acting on $R^n$ from the left. So we usually will view $R^n$ as a right $R$-module. Henceforth we assume that every ring acts on its modules from the right.

    If $V$ is a projective module which appears as a direct summand in $R^n$, then there is a projection $P \in \End_R(R^n)$ such that $P(R^n) \cong V$. This is not a bijection between projections and projective modules, but it is often useful. Indeed, for any projection $P$, $P(R^n)$ is a projective $R$-module.
\begin{example}
    Let $R = C(X)$, $P \in M_n(R)$ a projection. Then $P$ acts on $R^n = C(X \to \CC^n)$. So $P(R^n)$ is a projective module, and we can find the vector bundle from Swan's theorem by looking at its localizations.
\end{example}
    If $R$ is a unital ring, let $S(R)$ be the space of isomorphism classes of projective $R$-modules. For $V,W$ isomorphism classes, define $V + W = V \oplus W$. Then $S(R)$ is an abelian monoid, and $S$ is a functor from unital rings to abelian monoids. But $S(R)$ is badly behaved because $V \oplus W \cong V' \oplus W$ does not imply $V \cong V'$.
\begin{example}
    Let $T$ be the circle, $A = C(T \to \RR)$. So $A$ consists of periodic functions $\RR \to \RR$ which are continuous of period $1$. Let $\Xi_n^\pm$ be the set of continuous $\xi: \RR \to \RR$ such that $\xi(t+n) = \pm \xi(t)$. Then
    $$S(R) = \{\Xi_n^\pm: n \in \ZZ\}.$$
    From the space $\Xi_1^-$ we can recover the Moebius strip.
\end{example}
\begin{example}
    Let $A = C(T^2)$, viewed as continuous functions $f: \RR^2 \to \CC$ which are periodic of period $1$ in both variables. Now let $\Xi_{m,q}$ be the space of $\xi \in C(\RR^2 \to \CC)$ such that $\xi(s + 1, t) = \xi(s, t)$ and
    $$\xi(s, t+m) = e^{2\pi iqs} \xi(s, t).$$
    Together with the free $A$-modules, we recover all projective modules over $A$, i.e. all $\CC$-vector bundles over $T^2$. So we have classified vector bundles on a torus.
\end{example}
\begin{example}
    Let $S^2$ be the $2$-sphere, viewed as the unit sphere of $\RR^3$, and let $A = C(S^2 \to \RR)$. Let $\Xi$ be the $A$-module of continuous sections of the tangent bundle $TS^2$ of $S^2$. This can be viewed as the set of $\xi \in C(S^2 \to \RR^3)$ such that for all $x \in S^2$, $\langle \xi(x), x\rangle = 0$. By the hairy ball theorem, $TS^2$ is a nontrivial bundle. But the normal bundle $NS^2$, whose continuous sections consist of $\xi: S^2 \to \RR^3$ such that $\xi(x) \in \RR x$, is isomorphic to $S^2 \times \RR$, so is a trivial bundle. Therefore, if $S(A)$ was a cancellative monoid, then
    $$\Xi \oplus A \cong \Xi \oplus NS^2 \cong A^3 \cong A^2 \cong A$$
    so we could conclue that $\Xi \cong A^2$ and hence $TS^2$ is trivial, a contradiction. Therefore $S(A)$ is noncancellative, and constructing its Grothendieck group will be quite difficult.
\end{example}
    Let $C(R)$ be the universal cancellative abelian monoid for $S(R)$. In other words, for every $V,V'$ for which there exists $W$ with $V \oplus W \cong V' \oplus W$, we impose the relation $V = V'$. We then take the universal abelian group containing $C(R)$, say $K_0(R)$, i.e. the Grothendieck group\footnote{According to Rieffel, Grothendieck does not deserve to have such a trivial construction named after him.} of $S(R)$. (Constructively, elements of $K_0(R)$ are pairs $(V, W)$ where $V,W$ are elements of $C(R)$, and we are thinking of $(V, W)$ as meaning $V - W$.) We define the positive elements of $K_0(R)$ to be those in $C(R)$.

    In fact one constructs \dfn{K-theoretic group}s $K_n(R)$ for every $n \in \NN$. The first groups $n \in \{0, 1, 2\}$ were well-known previously, but Quillen introduced K-theoretic groups for any ring and any natural number. But we are not interested in Quillen's K-theoretic groups. We introduce, for $R$ a Banach algebra, the topological K-theoretic group $K_1^{top}(R)$. If one tries to define $K_2^{top}(R)$ over $\CC$ we find $K_2^{top}(R) = K_0(R)$; similarly over $R$ we have $K_8^{top}(R) = K_0(R)$. This is the \dfn{Bott periodicity theorem}.
\begin{example}
    Let $A_\theta = C^*(\ZZ^2, c_\theta)$ be the quantum $2$-torus. Then $A_0 = C(T^2)$. In case of the $2$-torus, $A_\theta = C(T) \rtimes_{\alpha_\theta} \ZZ$ acts on $L^2(T)$. We have projections on $A_\theta$ provided that $\theta \in (0, 1)$, namely
    $$P = U_{-1}M_h + M_f + M_gU_1$$
    for certain multiplication operators $M_f,M_g,M_h$. If $\varepsilon > 0$ and $\theta + \varepsilon < 1$, then the function $f$ is supported on $[0, \theta + \varepsilon]$ and identically $1$ on $[\varepsilon, \theta]$ and the trace $t(P)$ of the projection is given by
    $$t(P) = \int_0^1 f = \theta.$$
    If there is a unitary equivalence $P \cong P'$, then $t(P') = t(P)$. By a theorem on the homework, there are only countably many projections in a separable $C^*$-algebra up to unitary equivalence, so $A_\theta$ contains countably many traces, and its set of traces is determined by $\theta$. Yet there are uncountably many choices of $\theta$. So \dfn{Rieffel's theorem} says that there are uncountably many quantum tori up to $C^*$-isomorphism. In fact $(\ZZ + \ZZ\theta) \cap [0, 1]$ indexes the quantum tori that embed in $A_\theta$. But when Rieffel showed this result to Voicolescu, he proved \dfn{Voicolescu's theorem}, which shows that
    $$K_0(A_\theta) \cong \ZZ^2$$
    where $C(A_\theta) = (Z + Z\theta) \cap [0, \infty)$. So the K-theoretic group is not a complete invariant of the quantum tori, but the positive elements of the K-theoretic group give more information.
\end{example}


\chapter{Holomorphy in several complex variables}
This chapter follows Hormander's SCV book, Chapter II, and Zworski's lectures on SCV.

\section{Cauchy-Riemann equations}
Let us generalize the Cauchy-Riemann equations to higher dimensions.
\begin{definition}
    Let $f: \CC^n \to \CC$ be a function. We write $z = x + iy$ and define the partial derivatives
    $$\frac{\partial f}{\partial z_j} = \frac{1}{2}\left(\frac{\partial f}{\partial x_j} - i\frac{\partial f}{\partial y_j}\right),$$
    and
    $$\frac{\partial f}{\partial \overline z_j} =
    \frac{1}{2}\left(\frac{\partial f}{\partial x_j} + i\frac{\partial f}{\partial y_j}\right).$$
    We define the \dfn{Wirtinger differential} of $f$ by $\partial f = \sum_j \partial_{z_j}f dz_j$ and $\overline \partial f = \sum_j \partial_{\overline z_j} f d\overline z_j$. Finally, we define the \dfn{total differential} $df = \partial f + \overline \partial f$.
\end{definition}
For ease of notation we frequently make the decomposition $dz_j = dx_j + idy_j$ and $d\overline z_j = dx_j - idy_j$. Then $df = \sum_j \partial_{x_j}f ~dx_j + \partial_{y_j}f ~dy_j$, as it should be.

Notice that if $n = 1$ and $f$ is holomorphic, then $\overline \partial f = 0$. Indeed, $f = u + iv$ solves the Cauchy-Riemann equations, so
$$\frac{\partial f}{\overline \partial z} = \frac{1}{2}\left(\frac{\partial u}{\partial x} - \frac{\partial v}{\partial y} + \frac{\partial u}{\partial y} + \frac{\partial v}{\partial x}\right) = 0.$$
This motivates the general definition of holomorphy.
\begin{definition}
    The \dfn{Cauchy-Riemann equation} is the equation
    $$\overline \partial f = 0.$$
    If $f: \CC^n \to \CC$ solves the Cauchy-Riemann equation, then $f$ is said to be a \dfn{holomorphic function} of several complex variables. If $f = (f_1, \dots, f_m)$ is a function $\CC^n \to \CC^m$ such that each $f_j$ is holomorphic, then $f$ itself is said to be holomorphic.
\end{definition}
It is easy to check that the composite of holomorphic functions is holomorphic.

Recall that the implicit function theorem guarantees that a $C^r$ relation between $\RR^n$ and $\RR^m$ that ``passes the vertical hyperplane test" is actually the graph of a $C^r$ function $\RR^n \to \RR^m$. In particular, this holds if $r = \infty$, but demanding holomorphy of the function is actually a much stronger condition, so we must check that it holds.
\begin{theorem}[implicit function theorem]
    \index{implicit function theorem}
    Let $U$ be a neighborhood of $(w_0, z_0) \in \CC^m \times \CC^n$, and let $f: U \to \CC^m$ be holomorphic. Suppose that $f(w_0, z_0) = 0$ and $\det(\partial f_j/\partial w_k)_{j,k=1}^m \neq 0$. Then there is a unique holomorphic function $g: \CC^n \to \CC^m$ such that $f(g(z), z) = 0$ and $g(z_0) = w_0$.
\end{theorem}
\begin{proof}
    By replacing $\CC$ with $\RR^2$, we can apply the classical implicit function theorem. To do this, we write $f = u + iv$ and consider the Jacobian
$$\frac{\partial(u,v)}{\partial(x,y)} = \begin{bmatrix}
    \Re \partial f & -\Im \partial f\\
    \Im \partial f & \Re \partial f.
\end{bmatrix}$$
    One easily checks that the determinant of this matrix is $|\det(\partial f_j/\partial w_k)|_{j,k=1}^m$ which is nonzero. This gives a function $g$ with $f(g(z), z) = 0$.

    To prove holomorphy, we apply $\dbar$. This is
$$\dbar_k f_j(g(z), z) = \sum_\ell \partial_{w_\ell} f_j \dbar g_\ell(z)$$
    by the chain rule, using that $\dbar f = 0$. We have $\dbar_k f_j(g(z), z) = 0$, so by linear algebra, $\dbar g_\ell = 0$. So $g$ is holomorphic.
\end{proof}
\begin{corollary}[inverse function theorem]
    \index{inverse function theorem}
    Let $z_0 \in \CC^m$ and $f: \CC^m \to \CC^m$ be a  holomorphic function whose Jacobian does not vanish at $z_0$. Then there is a neighborhood $U \ni z_0$ such that $f$ is a holomorphic diffeomorphism of $U$ into its image.
\end{corollary}

    Recall that if $\alpha = (\alpha_1, \dots, \alpha_n)$ is a multiindex, then the differential form $dx_\alpha$ is given by
    $$dx_\alpha = \bigwedge_{j=1}^n dx_{\alpha_j}.$$
\begin{definition}
    A differential form $\omega$ is \dfn{type} $(p, q)$ if it can be written
    $$\omega = \sum_{|\alpha| = p} \sum_{|\beta| = q} f_{\alpha,\beta} dz_\alpha d\overline z_\beta.$$
    For each function space $\mathcal F$, we let $\mathcal F_{p,q}$ denote the space of differential forms of type $(p, q)$ over $\mathcal F$.
\end{definition}

\section{Basic properties}
\begin{theorem}[Hartogs]
    \index{Hartogs' theorem}
    Let $u: \Omega \to \CC$ be holomorphic in each variable alone; then $u$ is holomorphic.
\end{theorem}
Allegedly this theorem is useless (though it comes up in dynamical systems). Its proof is very difficult, using the Baire category theorem and the Schwarz lemma. However, if we are allowed to assume that $u$ is continuous, then the proof is almost trivial.
\begin{lemma}
    Assume that we are given a sequence $u_j$ of uniformly bounded subharmonic functions, such that $\limsup_j u_j$ is bounded from above. Then the $u_j$ are locally uniformly bounded from above by $\sup \limsup_j u_j + \varepsilon$.
\end{lemma}
\begin{proof}
    Without loss of generality, we may assume that the $u_j \leq 0$. Let $K$ be a compact set with $d(K, \Omega^c) \geq 3r$. Let $z \in K$, $\varepsilon > 0$. Then we can find a $n_0$ such that for any $n > n_0$,
$$\int_{|z' - z| < r} u_n(z') ~dz' \leq (C + \frac{\varepsilon}{2})\pi r^2$$
    by Fatou's lemma. If $\delta$ is small enough and $|w - z| < \delta$, then
$$\pi(r + \delta)^2 u_n(w) \leq \int_{|z' - z| \leq r + \delta} u_n(z') ~dz' \leq \int_{|z' - z| < \delta} u_n(z') ~dz' \leq (C + \varepsilon/2) \pi r^2$$
for $n$ large enough. Therefore
$$u_n(w) \leq (C + \varepsilon/2)\left(\frac{ r + \delta}{r}\right)^2 \leq C + \varepsilon$$
for $\delta$ small enough. This does not depend on $z, w$.

Cover $K$ by discs $D(z, \delta)$, so reduce to a finite subcover to find a uniform $\delta$.
\end{proof}

\begin{definition}
    A \dfn{polydisk} in $\CC^n$ is a set $D$ of the form
    $$D = \prod_{j=1}^n D_j$$
    where each $D_j \subset \CC$ is an open disk. The \dfn{distinguished boundary} $\partial_0 D$ is the set
    $$\partial_0 D = \prod_{j=1}^n \partial D_j.$$
    If $u: \overline D \to \CC^m$ is a continuous function which is holomorphic on $D$, we simply say $u$ is \dfn{holomorphic} on $\overline D$.
\end{definition}
Notice that $\partial_0 D$ is in general a torus (since it is a product of circles) and a very small subset of $\partial D$.

By induction on $n$, one easily proves the following.
\begin{theorem}[Cauchy's integral formula in a polydisk]
    \index{Cauchy's integral formula}
    Let $D$ be a polydisk in $\CC^n$ and let $u: \overline D \to \CC$ is holomorphic in each variable separately and continuous on $\overline D$. Then one has
    $$u(z) = \frac{1}{(2\pi i)^n} \int_{\partial_0 D} \frac{u(\zeta) ~d\zeta}{(\zeta_1 - z_1)\cdots(\zeta_n - z_n)}.$$
\end{theorem}
\begin{corollary}
    Let $U \subseteq \CC^n$ be an open set and $f: U \to \CC^m$ be a holomorphic function. Then $f \in C^\infty(U)$, and for every compact set $K \subset U$, every multiindex $\alpha$, and every open neighborhood $V$ of $K$, we have
    $$||\partial^\alpha u||_{L^\infty(K)} \preceq_{K,\alpha} ||u||_{L^1(V)}.$$
\end{corollary}
\begin{proof}
    Since $K$ is compact, it can be covered by finitely many sets contained in polydiscs contained in $V$. Now use Cauchy's integral formula (the implicit constant arising from the denominator of the formula, the measures of the distinguished boundary, and the obligatory factors of $2\pi$).
\end{proof}
\begin{corollary}[Montel]
    \index{Montel's theorem}
    Let $(u_k)_k$ be a sequence of holomorphic functions on an open set $U \subseteq \CC^n$. If $u_k \to u$ locally uniformly, then $u$ is holomorphic. On the other hand, if one has
    $$||u_k||_{L^\infty(K)} \preceq_K 1$$
    for $K \subset U$ compact, then $(u_k)_k$ has a locally uniformly convergent subsequence (which, in particular, has a holomorphic limit).
\end{corollary}
The proof is the same as in one variable.

We view $\dbar$ as an exterior derivative. For $(0, 1)$-forms $f$ we have
$$\dbar f = \sum_{j<k} (\dbar_j f_k - \dbar_k f_j) d\overline z_j \wedge d\overline z_k$$
and if $\dbar u = f$ for some $(0, 0)$-form $u$, $\dbar f = 0$.

\begin{theorem}
    Let $\Omega \subset \CC^n$ be bounded, $n > 1$, and assume that $\CC^n \setminus \Omega$ is connected. If there is a $\rho \in C^4(\CC^n \to \RR)$ such that $\partial \Omega$ is the zero set of $\rho$ and $d\rho|_{\partial \Omega} \neq 0$, and there is a $u \in C^4(\overline \Omega)$ such that $\dbar u \wedge \dbar \rho = 0$ on $\partial \Omega$, then there is a $U \in A(\Omega) \cap C^1(\overline \Omega)$ such that $U = u$ on $\partial \Omega$.
\end{theorem}
    Note that if $U$ exists, then $U - u = \rho h$ for some $h \in C^1(\overline \Omega)$, with $-\dbar u = \dbar \rho h$ on $\partial \Omega$. This is what we mean by $\dbar u \wedge \dbar \rho = 0$ on $\partial \Omega$, which is hence a necessary condition as well as sufficient.
\begin{definition}
    For $\rho$ as above, the equation $\dbar u \wedge \dbar \rho = 0$ is called the \dfn{tangential Cauchy-Riemann equation}.
\end{definition}
    Clearly if $u$ is holomorphic then it solves the tangential Cauchy-Riemann equations. Otherwise, what this condition is saying is that $\dbar u$ and $\dbar \rho$ are proportional. The intuition is that if $\sum_j t_j \dbar_j \rho = 0$ then $\sum_j t_j \dbar_j u = 0$, and the hypothesis here is that the vector field $\sum_j t_j \dbar_j$ is tangent (in the sense of the complexified tangent bundle) to $\partial \Omega$. Thus, what the tangential Cauchy-Riemann equations say is that any section $V$ of the tangent bundle of $\partial \Omega$ which only contain antiholomorphic coordinates has $Vu = 0$, so annihilates $u$.
\begin{example}
    Take the unit ball $B$ of $\CC^2$. Here $\rho(z) = |z|^2 - 1$. If $V = t_1 \dbar_1 + t_2 \dbar_2$ then $V\rho(z) = 2t\cdot z$. Thus the antiholomorphic vector fields which are tangent to $\partial B$ are elements of the ideal generated by $z_2 \dbar_1 - z_1 \dbar_2$.
\end{example}
\begin{proof}[Proof of theorem]
    We construct $U_0 \in C^1(\overline \Omega)$ such that $U_0 = u$ on $\partial \Omega$ and $\dbar U_0 = O(\rho^2)$ on $\partial \Omega$. By the tangential Cauchy-Riemann equations, we have a $h_0 \in C^3$ such that $\dbar u = h_0\dbar \rho$ on $\partial \Omega$. Thus $\dbar u = h_0 \dbar \rho + \rho h_1$ for some $h_1 \in C^3_{0,1}$. Then $\dbar(u-h_0\rho) = \dbar u - h_0\dbar \rho - \dbar h_0\rho = \rho h_1 - \dbar h_0 \rho = \rho(h_1 - \dbar h_0) = \dbar \rho h_2$ for $h_2 \in C^2_{0, 1}$, and $0 =\dbar(\rho h_2) = \dbar \rho \wedge h_2 + \rho \dbar h_2$. Now $\dbar h_2 = 0$ on $\partial \Omega$ so $\dbar \rho \wedge h_2 = 0$ on $\partial \Omega$, i.e. $h_2 = h_3 \dbar \rho + \rho h_4$ for some $h_3 \in C^2$ and $h_4 \in C^2_{0, 1}$. Now let $h_5 = -h_3/2$, then $2\dbar \rho h_5 = -h_2 + O(\rho)$, so
    $$\dbar(u-h_0\rho+h_5\rho^2) = \rho(h_3\dbar \rho + \rho h_4) - \rho^2\dbar h_3/2 + 2\rho \dbar \rho(-h_3/2)=\rho^2h_4 - \rho^2 \dbar h_3/2 = O(\rho^2).$$
    Thus let $U_0 = u - h_0\rho + h_5\rho^2$. So there is an $f \in C^1$ such that $\dbar U_0 = \rho^2 f$.

    Now let $F = \rho^2f$ on $\overline \Omega$ and $0$ away from $\Omega$. Then $F \in C^1(\CC)$. So there is a $v \in C^1_{comp}$ such that $\dbar v = F$. By continuity, $v = 0$ on $\partial \Omega$. Let $U = U_0 - v$.
\end{proof}





\section{Plurisubharmonicity and domains of holomorphy}
\begin{definition}
    Let $\Omega \subseteq \CC^n$ be open, and $u: \Omega \to [-\infty, \infty)$ be a upper-semicontinuous function. We say $u$ is a \dfn{plurisubharmonic function} or \dfn{plush function} if for every $a, b \in \CC^n$, the function $z \mapsto u(az + b)$ is subharmonic whenever it is defined.
\end{definition}
\begin{lemma}If $u$ is $C^2$, then $u$ is plush iff the Hessian matrix
$$\lambda_{ij} = \partial_i \dbar_j f$$
is positive-semidefinite everywhere.\end{lemma}
\begin{proof}
    We have $0 \leq \Delta_\tau u(z + \tau w) = 4\partial_\tau \dbar_\tau u(z + \tau w)$. By the chain rule,
    $$0 \leq 4 \partial_\tau \sum_j \dbar_j u (z + \tau w)\overline w_j = 4 \sum_{k,j} \partial_k \dbar_j u(z + \tau w) \overline w_j w_k.$$
    Now divide both sides by $4$.
\end{proof}
We let $P(\Omega)$ denote the set of plush functions on $\Omega$. It is easy to see from this characterization that for any holomorphic function $f$, $\log |f| \in P(\Omega)$. A tensor product of subharmonic functions, $u \otimes v(z, w) = u(z) v(w)$ is also plush. Any convex function is plush.

Recall that the decreasing sequence of a subharmonic functions converges to a subharmonic function. Restricting to a line we see:
\begin{corollary}
    The limit of a decreasing sequence of plush functions is plush.
\end{corollary}
    Thus we could define a plush function to be a decreasing limit of smooth plush functions (where a smooth plush function is one satisfying the Hessian characterization.)

\begin{corollary}
Let $\varphi$ be a standard mollifier and $u$ plush. Let $u_\varepsilon$ denote the $\varepsilon$-mollification of $u$. Then $u_\varepsilon$ decreases to $u$ as $\varepsilon \to 0$ and $u_\varepsilon \in C^\infty(\Omega_\varepsilon) \cap P(\Omega_\varepsilon)$ for $\Omega_\varepsilon = \{z \in \Omega: d(z, \partial \Omega) > \varepsilon\}$.
\end{corollary}
The proof is the same as in one dimension (i.e. for subharmonic functions).

\begin{corollary}
    Let $\Omega \subseteq \CC^n$, $\Omega' \subseteq \CC^m$, $f: \Omega \to \Omega'$ a holomorphic mapping, and $u \in P(\Omega')$. Then the pullback $f^*u \in P(\Omega)$.
\end{corollary}
\begin{proof}
    Without loss of generality we can assume that $u$ is smooth (since the pullback of a decreasing sequence is decreasing). By the Hessian characterization, the pullback is plush.
\end{proof}


\begin{definition}
    An open set $\Omega \subseteq \CC^n$ is a \dfn{domain of holomorphy} if there are no $\Omega_1, \Omega_2 \subset \CC^n$, $\Omega_1$ is nonempty, $\Omega_1 \subset \Omega_2 \cap \Omega$, $\Omega_2$ is not contained in $\Omega_1$, and for every $u \in A(\Omega)$ there is a $\tilde u \in A(\Omega_2)$ such that $u|_{\Omega_1} = u_1|_{\Omega_1}$.
\end{definition}
    For example, a polydisk $D = \prod_j D_j$ is a domain of holomorphy because we could always find $f_j \in A(D_j)$ which cannot extend to any open set beyond $D_j$, and then $f(z) = \sum_j f_j(z_j)$ cannot be extended to any open set. More generally, a product of domains of holomorphy is a domain of holomorphy. If $n = 1$ then every set is a domain of holomorphy, so this is the distinction between one and many variables.

We choose a function $\delta: \CC^n \to [0, \infty)$ such that $\delta(z) > 0$ for $z \neq 0$, $\delta(tz) = |t|\delta(z)$ for $\delta \in \CC$. For example, $\delta(z) = |z|$, or $\delta(z) = \max_j |z_j|r_j^{-1}$ for $r_j > 0$. Put
$$\delta(z, \Omega) = \inf_{w \in \Omega} \delta(z - w).$$

Recall that for $K \subset \Omega$ a compact set, the holomorphic hull $\hat K_\Omega$ is defined by the set of $z \in \Omega$ such that for every $f \in A(\Omega)$, $|f(z)| \leq ||f||_{L^\infty(K)}$.

\begin{lemma}
    Suppose $\Omega$ is a domain of holomorphy, $f \in A(\Omega)$, and $K$ compactly contained in $\Omega$. If for every $z \in K$,
    $$|f(z)| \leq \delta(z, \CC^n \setminus \Omega),$$
    then this estimate is also true for $z \in \hat K_\Omega$, the holomorphic hull of $K$. In particular,
    $$f(z) = \inf_{\substack{w \in \CC \setminus \Omega\\z \in K}} \delta(z - w) = \inf_{\substack{w \in \CC \setminus \Omega\\z \in \hat K_\omega}}
 \delta(z - w).$$\end{lemma}
\begin{proof}
    For $D = \{|z_j| < r_j\}$ a polydisc, let
    $$\Delta_\Omega^D(z) = \sup_{z + rD \subseteq \Omega} r = \delta(z, \CC^n \setminus \Omega)$$
    where $\delta(z) = \max_j |z_j|r_j^{-1}$.

    Suppose $f \in A(\Omega)$ and $|f(z)| \leq \Delta_\Omega^D(z)$ for $z \in K$. We claim that for each $\zeta \in \hat K_\Omega$ and $u \in A(\Omega)$,
    $$u(z) = \sum_\alpha \frac{(z - \zeta)^\alpha}{\alpha!} \partial^\alpha u(\zeta)$$
    on $\zeta + |f(\zeta)|D$. Since $\Omega$ is a domain of holomorphy, this Taylor series cannot converge on a large enough polydisc, so $|f(\zeta)| \leq \Delta_\Omega^D(z)$, which proves the claim in case $\Omega$ itself is a polydisc.

    To prove the claim, let $L_t = \{z \in \Omega: \exists w \in K~|z_j - w_j| \leq tr_j|f(w)|\}$. If $t < 1$, then $L_t$ is compact. In fact, if $\{z^k\}_j$ is a sequence in $L_t$ then there is a sequence of $w^k$ witnessing that $z^k \in L_t$. Since $K$ is compact we can choose a limit $w$ of the $w^k$. Thus for $k$ large,
    $$|z^k_j - w_j| \leq tr_j |f(w)| + \varepsilon.$$
    Taking $\varepsilon$ small, we see that the $z^k$ lie in a compact set in $K$ since $|z_j^k - w_j^k| \leq \Delta_\Omega^D(w)$, so have a limit $z \in K$.

    Let $M_t = ||u||_{L^\infty(L_t)}$. By the Cauchy inequality applied to the polydisc $\zeta + |f(\zeta)|D$,
    $$|\partial^\alpha u(w)| t^{|\alpha|} \frac{r^\alpha}{\alpha!} |f(w)|^{|\alpha|} \leq M_t.$$
    Let $F(\zeta) = \partial^\alpha u(\zeta) t^{|\alpha|} r^\alpha (\alpha!)^{-1} f(\zeta)^{|\alpha|}$. Then $F$ is holomorphic and $|F(\zeta)| \leq M_t$ for $\zeta \in K$. So by the definition of $\hat K_\Omega$, $||F||_{L^\infty(\hat K_\Omega)} \leq M_t$. Therefore
    $$\frac{|\partial^\alpha u(\zeta)|}{\alpha!} \leq M_t t^{|\alpha|} r^\alpha |f(\zeta)|^{|\alpha|}$$
    so if $|z_j - \zeta_j| \leq tr_j |f(\zeta)|$ then the Taylor series converges.

    The above argument proves the lemma when $\delta(z) = \max_j |z_j|r_j^{-1}$. The general $\delta$ satisfies
    $$\delta(z, \CC^n \setminus \Omega) = \sup \{r \in R: \forall w \in \CC ~\delta(w) \leq 1 \implies \forall a \in B(0, r) ~z + aw \in \Omega\}.$$
    Thus $\delta(z, \CC^n \setminus \Omega) = \inf_{\delta(w) \leq 1} \delta_w(z, \CC^n \setminus \Omega)$ where $\delta_w(z, \CC^n \setminus \Omega) = \sup_{\forall a ~z \in aw \in \Omega} r$ where $a$ ranges over $B(0, r)$. Take $w = (1, 0, \dots, 0)$ and $D_k = \{z: |z_1| < 1, j \neq 1 \implies |z_j| < 1/k\}$. Thus $\Delta^{D_k}_\Omega(z) \to \delta_w(z, \CC^n \setminus \Omega)$ as $k \to \infty$. Moreover $\Delta^{D_{k+1}}_\Omega(z) \geq \Delta^{D_k}_\Omega(z)$ so by Dini's theorem the convergence of the $\Delta^{D_k}_\Omega$ is uniform.

    If $|f(z)| \leq \delta_w(z, \CC^n \setminus \Omega)$ then $|f(z)| \leq (1 + \varepsilon) \Delta^{D_k}_\Omega(z)$ for $k$ large enough and $z \in K$. By the lemma, $|f(z)| \leq (1 + \varepsilon)\Delta_\Omega^{D_k}(z)$ and $z \in \hat K_\Omega$. By Dini's theorem again one has
    $$|f(z)| \leq \delta_w(z, \CC^n \setminus \Omega).$$
    Take the infimum over $w$ of both sides. So $|f(z)| \leq \delta(z, \CC^n \setminus \Omega)$ even for ``weird" choices of $\delta$.
\end{proof}


\begin{theorem}
The following are equivalent:
\begin{enumerate}
    \item $\Omega$ is a domain of holomorphy.
    \item For every compact subset $K$ of $\Omega$, the holomorphic hull $\hat K_\Omega$ is compactly contained in $\Omega$, and for every $f \in A(\Omega)$,
    $$\sup_K \frac{|f(z)|}{\delta(z, \CC^n \setminus \Omega)} = \sup_{\hat K_\Omega} \frac{|f(z)|}{\delta(z, \CC^n \setminus \Omega)}.$$
    \item For every compact subset $K$ of $\Omega$, the holomorphic hull $\hat K_\Omega$ is compactly contained in $\Omega$.
    \item There is a $f \in A(\Omega)$ which does not extend to any larger set.
\end{enumerate}
\end{theorem}
\begin{proof}
    Obviously 2 implies 3 and 4 implies 1. By the lemma above, 1 implies 2. So we just need to show 3 implies 4.

    Let $\{K_j\}$ a compact exhaustion of $\Omega$ (so for any compact $L \subset \Omega$ and every $j$ large enough $L \subseteq K_j$). Define $D_\zeta = \{\zeta\} + rD$ for $D = \{|z_j| < r_j\}$ and $r = \sup_{\{\zeta\} + \rho D \subset \Omega} \rho$.

    Let $M$ be the set of rational points of $\Omega$; then we choose a sequence of $\zeta_j \in M$ so that every element of $M$ appears infinitely often. We also choose a sequence of $z_j \in D_{\zeta_j}$ such that $z_j \notin K_j$; this is possible because $D_{\zeta_j}$ touches $\partial \Omega$ but $K_j$ does not because it is compact.

    By (3), there is a $f_j \in A(\Omega)$ such that $f_j(z_j) = 1$ and $||f_j||_{L^\infty(K_j)} < 1$. Replacing $f_j$ by a power of $f_j$, we can assume without loss of generality that $||f_j|_{L^\infty(K_j)} < 2^{-j}$.

    Now let
    $$f(z) = \prod_j (1 - f_j(z))^j.$$
    To see that this product converges we just have to show convergence in any compact set $L$, and we can assume without loss of generality that there is a $J$ such that $L = K_J$. For $j > J$ we have $|f_j(z)| < 2^{-j}$ and since we only care about the tail we can assume $J = 1$. Then
    $$\log f(z) = \sum_j j|\log(1-f_j(z))| \leq \sum_j j |f_j(z)| \leq \sum_j j2^{-j} < \infty.$$
    Therefore convergence is locally uniform so $f \in A(\Omega)$.

    For every $D_{\zeta_j}$ there is a $w_N$ such that for every $|\alpha| \leq N$, $\partial^\alpha f(z_N) = 0$. Therefore $f$ has a zero of order $N$ in $D_{\zeta_j}$. In particular, the zeroes of $f$ have higher and higher order as we approach $\partial \Omega$. Therefore if $f$ is defined at a point $z$ of $\partial \Omega$ then $z$ is an infinite-order zero of $f$. So $f = 0$. But $f$ is nonzero so this is a contradiction.
\end{proof}
\begin{example}
    As a counterexample, notice that if $\Omega = B(0, 3) \setminus \overline{B(0, 1)}$ and $K = \partial B(0, 2)$ then $\hat K_\Omega = \overline{B(0, 2) \setminus B(0, 1)}$ by Hartogs' theorem and the maximum principle for $n \geq 2$. This is not a compact subset of $\Omega$, so $\Omega$ is not a domain of holomorphy.
\end{example}
\begin{corollary}
    If $\Omega$ is convex then $\Omega$ is a domain of holomorphy.
\end{corollary}
\begin{proof}
    Recall that $\hat K_\Omega$ is contained in the convex hull of $K$, which is a compact subset of $\Omega$.
\end{proof}
\begin{corollary}
    If $\Omega_\alpha$ are domains of holomorphy then the interior of $\bigcap_\alpha \Omega_\alpha$ is a domain of holomorphy.
\end{corollary}
\begin{proof}
    Let $K \subset \Omega$ be a compact set. Then $\hat K_\Omega$ is a compact subset of $\hat K_{\Omega_\alpha}$ for every $\alpha$, in particular of the compact set $\bigcap_\alpha \hat K_{\Omega_\alpha}$, which is a compact subset of $\Omega$.
\end{proof}
\begin{corollary}
    Let $\Omega$ be a domain of holomorphy, $f_1, \dots, f_N \in A(\Omega)$. Then
    $$\Omega_f = \{z \in \Omega: |f_j(z)| < 1\}$$
    is a domain of holomorphy.
\end{corollary}
\begin{proof}
    Let $K$ be a compact subset of $\Omega_f$. Then by compactness, there is a $r < 1$ such that $K \subset \{z \in \Omega: |f_j(z)| < r\}$. Thus for any $z \in \hat K_{\Omega_f}$, $|f_j(z)| \leq r$. Moreover
    $$\hat K_{\Omega_f} \subseteq \hat K_\Omega \subset \{|f_j| \leq r\} \subseteq \Omega_f.$$
    Therefore $\hat K_{\Omega_f}$ is a compact subset of $\Omega_f$.
\end{proof}
\begin{corollary}
    Let $u: \Omega \to \CC^m$ be a holomorphic transformation, $\Omega' \subseteq \CC^m$ a domain of holomorphy. Then the pullback $u^{-1}(\Omega')$ is a domain of holomorphy.
\end{corollary}

We now relate domains of holomorphy to plurisubharmonicity.
\begin{theorem}
    If $\Omega$ is a domain of holomorphy and $\delta$ as above, then $z \mapsto -\log \delta(z, \CC^n \setminus \Omega)$ is a continuous plurisubharmonic function.
\end{theorem}
\begin{proof}
    For $z_0 \in \Omega$, $w \in \CC^n$, take $D = \{z_0 + \tau w: \tau \in \CC, ~|\tau|\leq r\}$. If $r$ is small enough then $D \subseteq \Omega$. Let $f$ be a polynomial in $\tau$ and $-\log \delta(z_0 + \tau w, \CC^n \Omega) \leq \Re f(\tau)$ for every $|\tau| = r$. We must show this is true for $|\tau| < r$ as well.

    Let $F$ be a polynomial on $\CC^n$ such that $F(z_0 + \tau w) = f(\tau)$. Then
    $$|e^{-F(z)}| \leq \delta(z, \CC^n \setminus \Omega)$$
    for $z \in \partial D$, hence for $z \in \widehat{\partial D}_\Omega$, in particular for $z\in D$. Thus the claim holds for $|\tau| < r$.
\end{proof}
    In fact the converse of this result holds, using Hormander $L^2$-estimates with plush weights on $\Omega$.
\begin{definition}
    Let $K \subset \Omega$ be a compact set. The \dfn{plurisubharmonic hull}
    $$\hat K^p_\Omega = \{z \in \Omega: \forall u \in P(\Omega) ~u(z) \leq \sup_K u.\}$$
    If $K = \hat K^p_\Omega$, we say that $K$ is \dfn{plurisubharmonically convex}.
\end{definition}
    Now if $f \in A(\Omega)$ we have $\log |f| \in P(\Omega)$, so we are testing by fewer functions that in the case of the analytic hull $\hat K_\Omega$. Thus $\hat K^p_\Omega \subseteq \hat K_\Omega$.
\begin{theorem}
    The following are equivalent:
\begin{enumerate}
    \item $z \mapsto -\log \delta(z, \CC^n \setminus \Omega)$ is plush.
    \item There is a $u \in P(\Omega)$ such that for every $c \in \RR$, $\Omega_c = \{z \in \Omega: u(z) < c\}$ is $\Omega$-precompact.
    \item For every $K \subset \Omega$ compact, $\hat K^p_\Omega$ is compact in $\Omega$.
\end{enumerate}
\end{theorem}
\begin{definition}
    $\Omega$ is \dfn{pseudoconvex} if one (and hence) all of the above conditions hold. The function $u$ appearing in (2) is called a \dfn{plurisubharmonic exhaustion function}.
\end{definition}
    If $\Omega$ is a domain of holomorphy, then $\Omega$ is pseudoconvex.
\begin{proof}[Proof of theorem]
    To see 1 implies 2, let $u(z) = -\log \delta(z, \CC^n \setminus \Omega) + |z|^2$. This is clearly plush and blows up at the boundary. So it is a plush exhaustion function.

    To see 2 implies 3, let $c = \sup_K u$. Then $\{z \in \Omega: u(z) \leq \sup_K u\}$ is clearly compact and contains $\hat K^p_\Omega$.

    So we just need to prove 3 implies 1. Take $z_0 \in \Omega$ and $w \in \CC^n \setminus 0$. We need to show that for every $|\tau| = r$, if $-\log(z_0 + \tau w, \CC^n \setminus \Omega) \leq \Re f(\tau)$ then for every $|\tau| \leq r$ we have $-\log(z_0 + \tau w, \CC^n \setminus \Omega)$.

    If $|\tau| = r$ then $\delta(z_0 + \tau w, \CC^n \setminus \Omega) \geq |e^{-f(\tau)}|$. Let $a \in \CC^n$ be such that $\delta(a) < 1$. Let $F_\lambda(\tau) = z_0 + \tau w + \lambda z e^{-f(t)}$ and let $D_\lambda = F_\lambda(D(0, r))$ and $\Lambda = \{\lambda \in [0, 1]: D_\lambda \subseteq \Omega\}$. We must show $\Lambda = [0, 1]$ by showing that $\Lambda$ is clopen and nonempty.

    If $\lambda \in \Lambda$ and we perturb $\lambda$, then we do not move $D_\lambda$ by much, so it remains in the open set $\Omega$. Therefore $\Lambda$ is open. Moreover, $0 \in \Lambda$ by assumption 3, so $\Lambda$ is nonempty.

    Let $K = \{z_0 + \tau w + \lambda a e^{-f(\tau)}: |\tau| = r, ~\lambda \in [0, 1]\}$. Then $K$ is $\Omega$-compact since by assumption 3, $\delta(z_0 + \tau w, \CC^n \setminus \Omega) \geq |e^{-f(\tau)}|$. Since $|a| < 1$ we have $|a\lambda e^{-f(\tau)}| < 1$ whence $\delta(z_0 + \tau w + \lambda ae^{-f(t)}) > 0$. Thus the function
    $$\tau \mapsto u(z_0 + \tau w + a \lambda e^{-f(\tau)}$$
    is subharmonic near $|\tau| \leq r$. Thus
    $$u(z_0 + \tau w + \lambda ae^{-f(\tau)}) \leq \sup_K u$$
    for $|\tau| \leq r$, by the maximum principle, since $K$ contains the boundary of $\hat K^p_\Omega$. Thus $D_\lambda \subseteq \hat K^p_\Omega$.

    If we have a sequence of $\lambda_j \in \Lambda$, say $\lambda_j \to \lambda_0 \in [0, 1]$, then the $D_{\lambda_j} \subseteq \hat K^p_\Omega$, giving a continuous family of closed sets which converge to a closed set $D_{\lambda_0}$. So $D_{\lambda_0} \subseteq \hat K^p_\Omega$, so $\lambda_0 \in \Lambda$. So $\Lambda$ is closed, which proves the theorem.
\end{proof}
\begin{corollary}
    If $(\Omega_\alpha)_\alpha$ is a family of pseudoconvex domains then the interior $\Omega$ of $\bigcap_\alpha \Omega_\alpha$ is pseudoconvex.
\end{corollary}
\begin{proof}
    One has
    $$\delta(z, \CC^n \setminus \Omega) = \delta\left(z, \CC^n \setminus \bigcap_\alpha \Omega_\alpha\right) \inf_\alpha \delta(z, \CC^n \setminus_\alpha).$$
    Taking $-\log$ of both sides we arrive at
    $$-\log \delta(z, \CC^n \setminus \Omega) = \sup_\alpha -\log \delta(z, \CC^n \setminus \Omega_\alpha)$$
    and the right-hand side is plush since the supremum of plush functions is plush.
\end{proof}
\begin{corollary}
    $\Omega$ is pseudoconvex if and only if for every $z \in \hat \Omega$ there is a neighborhood $\omega \ni z$ such that $\Omega \cap \omega$ is pseudoconvex.
\end{corollary}
\begin{proof}
    If $\Omega$ is pseudoconvex, let $\omega$ be a convex neighborhood of $z$. So $\omega$ is pseudoconvex; use the previous corollary.

    For the converse, notice that this is trivial if $z \in \Omega$ by (3) of the above theorem. If $z_0 \in \partial \Omega$ and $\omega \ni z_0$ we notice that $\delta(z, \CC^n \setminus \Omega) = \delta(z, \CC^n \setminus (\Omega \cap \omega))$ if $|z - z_0|$ is small. Thus the function $z \mapsto -\log \delta(z, \CC^n \setminus \Omega)$ is plush near $z_0$. But plurisubharmonicity is a local property so the function is plush on a neighborhood of $\partial \Omega$, i.e. there is a closed set $F \subset \Omega$ such that $z \mapsto -\log \delta(z, \CC^n \setminus \Omega)$ is plush on $\Omega \setminus C$.

    Let
    $$\Phi(r) = \max_{\substack{|\zeta| \leq r\\\zeta \in F}} -\log \delta(\zeta, \CC^n \setminus \Omega)$$
    so $\Phi$ is increasing. Now let $\Phi_1$ be a convex increasing function such that $\Phi_1 \geq \Phi$. So we define $\varphi(z) = \Phi_1(|z|)$. So $\varphi$ is a plush function and we can put
    $$u(z) = \max(\varphi(z), -\log\delta(z, \CC^n\setminus\Omega))$$
    which is a supremum of plush functions, hence plush. Clearly $u$ satisfies (2).
\end{proof}
    So pseudoconvexity is a local property.
\begin{theorem}
    \label{Levi condition}
    Let $\rho \in C^2(\CC^n)$ with $d\rho|_{\rho = 0} = 0$ and let $\Omega = \{z \in \CC: \rho(z) < 0\}$. Then $\Omega$ is pseudoconvex if and only if for every $z \in \partial \Omega$, $w \in \CC^n$ such that $\sum_j \partial_j \rho(z) w_j = 0$
    we have
    $$\sum_{j,k=1}^n \partial_j \dbar_k \rho(z) w_j \overline w_k \geq 0.$$
\end{theorem}
\begin{definition}
    If $\Omega$ satisfies the hypotheses of Theorem \ref{Levi condition} and is pseudoconvex then $\Omega$ satisfies the \dfn{Levi condition}.
\end{definition}
\begin{example}
    Let $\Omega = \{z \in \CC^2: |z_1|^2 + 2 \Im z_2 < 0\}$. Then
    $$\rho(z) = |z_1|^2 + 2 \Im z_1.$$
    Calculating, we see that $$\partial\dbar \rho = \begin{bmatrix}1&0\\0&0\end{bmatrix}$$
    so $\Omega$ satisfies the \emph{strict} Levi condition (where $\geq$ is replaced with $>$). In this case, for any $z_0 \in \partial \Omega$ there is a $U \in A(\Omega)$ which has a singularity at $z_0$, so $U$ does not extend beyond $\Omega$. In fact, we put
    $$U(z) = (z_1\overline a) - iz_2 - |a|^2/2 + ib)^{-1}$$
    where $a \in \CC$, $b \in \RR$. We put $z_0 = (a, b -i|a|^2/2)$, then $U$ has a singularity at $z_0$. However, as we will prove in a later theorem, any function on $\CC^2 \setminus \Omega$ admits an analytic continuation to $\Omega$.

    For this example, the tangential Cauchy-Riemann equation is $(\dbar_1 + iz_1\dbar_2)u(z) = 0$. That tangential Cauchy-Riemann operator, viewed as an operator on a $3$-dimensional manifold (since $\partial \Omega$ is a $3$-dimensional manifold) was used by Levi to disprove the version of the Cauchy-Kovaleskai theorem for smooth (rather than analytic) functions because for the generic $f \in C^\infty(\partial \Omega)$ we do not have $Pu = f$.
\end{example}
\begin{proof}[Proof of Theorem \ref{Levi condition}]
    First, a one-line lemma: Let $\rho_1 = h\rho$, $\rho > 0$, $h \in C^2$ with $h > 0$. Then if $\rho$ satisfies the Levi condition then $\rho_1$ does as well. So we can replace the $\rho$ in the hypotheses in theorem with any $\rho$ satisfying the same conditions, and we willl.

    To prove that if $\Omega$ is pseudoconvex with a $C^2$ boundary then the Levi condition holds, we let $\rho(z) = -\inf_{w \notin \Omega} |z - w|$ for $z \in \Omega$ and $\rho(z) = \inf_{w \in \Omega} |z - w|$ for $z \notin \Omega$. Thus $\rho(z)$ is the ``signed distance" from $z$ to $\partial \Omega$. If $z \notin \partial \Omega$, we have
    $$z = w + \rho(z)n(w)$$
    for a minimizer $w \in \partial \Omega$ (which exists since $\partial \Omega$ is closed) and $n$ the unit normal. Since $\Omega$ has a $C^2$-boundary, $n$ is a $C^1$ function of the element $w'$ of $\RR^k$ where $k$ is the dimension of $\partial \Omega$ as a real manifold. Let $f(w') = w$. Then
    $$F(z, w', \rho(z)) = z - (f(w'), w') - \rho(z)\frac{(\nabla f(w'), 1)}{\sqrt{|\nabla f(w')|^2 + 1}}$$
    and $F$ is $C^1$. By rotating the manifold $\partial \Omega$ so that the tangent plane near $w$ is horizontal and translate so $w = 0$, i.e. $w' = 0$, $f(0) = 0$, $\nabla f(0) = 0$. So
    $$\frac{\partial F}{\partial(w', \rho(z))}(z_0, 0, \rho(z_0)) = \begin{bmatrix}\frac{\partial F}{\partial w'} & \frac{\partial F}{\partial \rho}\end{bmatrix} = \begin{bmatrix}I + O(\rho(z_0)) & 0 \\ O(\rho(z_0)) & -1\end{bmatrix}$$
    which is invertible if $\rho(z_0)$ is small. Therefore we can use the implicit function theorem to see that $\rho$ is well-defined and $C^1$. We now implicitly differentiate $x = y + \rho n$ to see that
    $$e_j = \partial_j(y'(x) + \rho(x)\left(y'(x) + \frac{-\nabla f(y'(x))}{\sqrt{1 + |\nabla f(y'(x))|}}, f(y') + \rho(1 + O(y')^2)\right)$$ which evaluates at $x = x_0$, $y' = 0$ to show that
    $$\delta_{Nj} = \partial_j \rho(x_0).$$
    Rotating back to the original coordinate frame,
    $$\nabla \rho(x) = n(y(x))$$
    whenever $x$ is close to $\partial \Omega$. Therefore $\nabla \rho \in C^1$ so $\rho \in C^2$.

    Now for $z \in \Omega$ and the standard $\delta$ (namely $\delta(z, w) = |z - w|$) we have $\rho(z) = -\delta(z, \CC^n \setminus \Omega)$. Then $-\delta = \rho$ so $\delta \in C^2$ whence
    $$-\partial_j \dbar_k \log \delta = \delta^{-2} \partial_j \delta \dbar_k \delta - \delta^{-1} \partial_j \dbar_k \delta$$
    so it follows that
    $$\sum_{j,k} \delta^{-1} \partial_j \delta w_j \dbar_k \delta \overline w_k - \partial_j \dbar_k \delta w_j \overline w_k \geq 0.$$
    We know that $\partial_j \rho(z_0) w_j = 0$ for $z_0 \in \partial \Omega$ (hypothesis of the Levi condition) so
    $$\sum_j \partial_j \rho(z) w_j = O(|z - z_0|).$$
    Taking $z \to z_0$,
    $$\sum_{j,k} \partial_j \dbar_k \rho w_j \overline w_k \geq 0$$
    (conclusion of the Levi condition).

    We prove the converse by contradiction. Assume that $-\log \delta(z, \CC^n \setminus \Omega)$ is not plush in $z$, in any neighborhood of $\partial \Omega$. Then $\partial \dbar \log(z + \tau w, \CC^n \setminus \Omega) > 0$ for some $z$ close to $\partial \Omega$. We will expand this function in a Taylor series in $\tau$. In fact,
    $$\log \delta(z + \tau w, \CC^n \setminus \Omega) = \log \delta(z, \CC^n \setminus \Omega) + \Re (A \tau) + \Re (B \tau^2) + C|\tau|^2 + o(|\tau|^2)$$
    for some $A, B \in \CC$ and $C > 0$.

    Now let $z(\tau) = z + \tau w + ae^{A\tau + B\tau^2}$ for some $a \in \CC$. Then
\begin{align*}
    \delta(z(\tau), \CC^n \setminus \Omega) &\geq \delta(z + \tau w, \CC^n \setminus \Omega) - \delta(a)|e^{A\tau + B\tau^2}|\\
    &\geq \delta(a) (e^{C|\tau|^2/2} - 1)|e^{A\tau + B\tau^2}| \sim |\tau|^2
\end{align*}
    for $|\tau|$ small enough. Choose $a$ so that $\delta(a) = d(z, \CC^n \setminus \Omega)$. Then $z(0) = z + a \in \partial \Omega$. Since the function looks like $|\tau|^2$ we have
\begin{align*}\partial_\tau \delta(z(\tau), \CC^n \setminus \Omega)|_{\tau = 0} &= 0\\
\partial_\tau^2 \delta(z(\tau), \CC^n \setminus \Omega)|_{\tau = 0} &> 0.
\end{align*}
    By the chain rule,
\begin{align*}
    0 &= \partial_\tau \delta(z(\tau), \CC^n \setminus \Omega)|_{\tau = 0} = -\sum_j \partial_j \rho z(0) z_j'(0)\\
    0 &< \partial_\tau^2 \delta(z(\tau), \CC^n \setminus \Omega)|_{\tau = 0} = -\sum_{j,k} \partial_j \dbar_k \rho z_j'(0) \overline z_k'(0).
\end{align*}
    Since $\rho < 0$, this contradicts the Levi condition.
\end{proof}
    We will not bother to prove the following theorem, but it is true. The proof uses the theory of Hormander $L^2$ estimates on complex manifolds with boundary, which is technically complicated but not very interesting.
\begin{theorem}[Levi problem]
    \index{Levi problem}
    If $\Omega$ is pseudoconvex then $\Omega$ is a domain of holomorphy.
\end{theorem}

\begin{theorem}
    Assume that $\omega$ is a neighborhood of $z_0$ and $\rho \in C^4(\omega)$, $\rho(z_0) = 0$, $d\rho(z_0) = 0$. Suppose that there is a $w \in \CC^n$ such that
    $$\sum_{j,k} \partial_j \dbar_k \rho(z_0) w_j \overline w_k < 0$$
    and
    $$\sum_j \partial_j \rho(z_0) w_j = 0.$$
    Then there is an $\omega' \subseteq \omega$, $z_0 \in \omega'$, such that for every $u \in C^4(\omega')$ such that for every $z \in \omega'$, $\rho(z) = 0$ implies
    $$\dbar u \wedge \dbar \rho(z) = 0.$$
    Let $\omega_+' = \{z \in \omega': \rho(z) > 0\}$. Then there is a function $U$ defined on $\omega'$ such that if $\rho(z) = 0$ then $U(z) = u(z)$, and such that $U \in A(\omega_+')$.
\end{theorem}
\begin{proof}
    Write $z = (z_1, z', z_n)$. There is an affine change of coordinates such that
    $$\rho(z) = \Im z_n + A_{11}|z_1|^2 + O(|z_1|^3) + O(|z'|^2)$$
    for some $A_{11} < 0$. In particular, we can find $\delta, \varepsilon > 0$ such that
    $$\omega' = \{z \in \CC^n: |z_1| < \delta, ~|z'| + |z_n| < \varepsilon\} \subseteq \omega$$
    satisfies $\rho(z) < 0$ if $|z_1| = \delta$, $z \in \omega'$, and also such that
    $$\partial_1 \dbar_1 \rho < 0$$
    on $\omega'$. Now the set of $z_1 \in \CC$ such that $|z'| + |z_n| < \varepsilon$ implies $\rho(z) < 0$ is connected: otherwise, $\rho$ would have a local minimum in the second connected component, yet $\Delta \rho < 0$ so this is impossible.
\begin{lemma}
    For every $f \in C^k_{(0, 1)}(\omega')$ such that $f|_{\omega' \setminus \omega_+'} = 0$, $k \geq 1$, $\dbar f = 0$, there is a $v \in C^k(\omega')$ such that $\dbar v = f$ and $v|_{\omega' \setminus \omega_+'} = 0$.
\end{lemma}
\begin{proof}
    We define
    $$v(z) = \frac{1}{2\pi i} \int_{|z_1| < \delta} \frac{f_1(\tau, z')}{\tau - z_1} ~d\tau \wedge d\overline \tau$$
    so $\dbar v = f$. In particular $v$ is analytic whenever $\rho \leq 0$. We claim $v = 0$ on $\omega' \setminus \omega_+'$. Near the boundary (except for the top) of $\omega'$, $\dbar v = 0$, and $v = 0$ at the bottom, so $v = 0$ near the bottom. The set of points where $\rho < 0$ is connected, so $v = 0$ there.
\end{proof}
    Let $v$ be as in the lemma. Let $U_0 \in C^2(\omega')$ be such that $\dbar U_0 = O(\rho^2)$ and $U_0|_{\rho=0} = u|_{\rho = 0}$. Then let $U = U_0 + v$. So $\dbar U = \dbar U_0 + \dbar v$, $\dbar v = - \dbar U_0$, and $v|_{\omega' \setminus \omega_+'} = 0$ so we're done.
\end{proof}

\section{Hormander $L^2$ estimates}
We want to use the method of a priori estimates to show that $\dbar u = f$ has a solution, but the Hilbert space $L^2(\CC^n)$ contains no holomorphic functions except $0$. Therefore we must weight the inner product to apply Hilbert space theory. Fix $\Omega \subseteq \CC^n$ open. We let $\lambda$ denote Lebesgue measure.

\begin{definition}
    Let $\varphi \in C^2(\Omega)$. We say that $\varphi$ is a \dfn{strictly plurisubharmonic function} or simply that $\varphi$ is \dfn{strictly plush} if
    $$\inf_{t \in \CC^n \setminus 0} \frac{\sum_j\sum_k \partial_j \dbar_k \varphi(z) t_j \overline{t_k}}{\sum_j |t_j|^2} > 0.$$
\end{definition}
\begin{definition}
Let $\varphi$ be a strictly plush function on $\Omega$. We define the \dfn{weighted inner product} with strictly plush weight $\varphi$ to be the inner product $(\cdot,\cdot)_\varphi$ of the Hilbert space $L^2(\varphi)$ corresponding to the Borel measure $e^{-\varphi} ~d\lambda$ on $\Omega$.
\end{definition}
In other words,
$$(f, g)_\varphi = \int_\Omega f(z) g(z) e^{-\varphi(z)} ~dz.$$
To motivate this definition, notice that
$$\int |u|^2 e^{-\varphi} = \int e^{2\log |u| - \varphi}$$
so we must have $\varphi > 2 \log |u|$ at infinity. But $\log |u|$ is subharmonic, so the point is that $\varphi$ must have an especially strong form of subharmonicity for this to work on as many holomorphic functions as possible. This leads us to consider $\varphi$ as a plush function.

Suppose that we have solved the equation $\dbar u = f$, for $f: \Omega \to \CC^n$ a good function. Then for $f$ to be the ``gradient" of $u$ with respect to $\dbar$, it must be the case that
$$\dbar_j f_k = \dbar_k f_j$$
since $u$ is smooth and so has equality of mixed partials. We call this condition the \dfn{Cauchy-Riemann constraint equation}. One could view it as the statement that the $1$-form $\overline du = \sum_k f_j ~dx_j$ is closed. Of course, $\overline du$ is an exact form, hence closed.

\begin{theorem}[Hormander's estimates]
    \index{Hormander's $L^2$-estimates}
    Let $\varphi$ be strictly plush and let
    $$\kappa(z) = \inf_{t \in \CC^n \setminus 0} \frac{\sum_j\sum_k \partial_j \dbar_k \varphi(z) t_j \overline{t_k}}{\sum_j |t_j|^2}$$
    witness that $\varphi$ is strictly plush. Assume that $f \in L^2(\varphi + \log \kappa, \Omega \to \CC^n)$ satisfies the Cauchy-Riemann constraints $\dbar_j f_k = \dbar_k f_j$. Then there is a $u \in L^2_\varphi$ such that $\dbar_j u = f_j$ and
    $$||u||_{L^2(\varphi, \Omega \to \CC)} \leq ||f||_{L^2(\varphi + \log \kappa, \Omega \to \CC^n)}.$$
\end{theorem}
Before proving the theorem, we need the notion of weak solution for the operator $\dbar$. We introduce the differential operators
$$\delta_j = \partial_j - (\partial_j \varphi).$$
If $u$ is a smooth solution to the equation $\dbar u = f$ and $g$ is smooth, then
$$(\delta_j g, u)_\varphi
    = \int_\Omega \partial g \overline u e^{-\varphi} - \int_\Omega \partial_j \varphi g \overline u e^{-\varphi}
    = (g, f_j)_\varphi.$$
In other words, $\dbar_j^* = \delta_j$.


For functions $\Omega \to \CC^n$ we define
$$(g, h)_\varphi = \sum_j (g_j, h_j)_\varphi.$$
\begin{proof}[Proof of the Hormander estimates]
    Fix a convex function $\phi \geq 0$ such that
    $$\phi(z) \geq |z| \log \kappa(z),$$
    which is possible because $\kappa > 0$. Then for any $\varepsilon > 0$, we have $\log \kappa \leq \varepsilon \phi$ on $|z| > 1/\varepsilon$. Thus
    $$\int_{|z| > 1/\varepsilon} |f|^2 e^{-\varphi - \varepsilon\phi} \leq \int_{|z| > 1/\varepsilon} |f|^2 e^{-\varphi-\log\kappa} < \infty,$$
    and the set $\{|z| \leq 1/\varepsilon\}$ is no problem for integrability, so $f \in L^2_{\varphi + \varepsilon\phi}$. Since $\phi$ is convex, it is plush, so $\varphi_\varepsilon = \varphi + \varepsilon \phi$ is strictly plush. Taking weakstar limits as $\varepsilon \to 0$, we can assume that $f \in L^2(\varphi)$.

    Fix $g \in C^\infty_c(\Omega \to \CC^n)$. Then
    \begin{align*}
(\delta_j g_j, \delta_k g_k)_\varphi &= -(\dbar_j \delta_k g_j, g_k)_\varphi \\
&= -(\delta_k \partial_{\overline{z}_j} g_j,g_k)_\varphi - ([\dbar_j, \delta_k]g_j, g_k)_\varphi \\
&= (\dbar_j g_j, \dbar_k g_k)_\varphi + (\partial_j \dbar_k \varphi g_j, g_k)_\varphi
\end{align*}
    so, summing both sides over $j, k$,
    $$(\delta g, \delta g)_\varphi + \frac{1}{2} \sum_{j \neq k} ||\dbar_j g_k - \dbar_k g_j||_\varphi^2 = \sum_j ||\dbar_j g||_\varphi^2 + \sum_{jk} (g_j \partial_j \dbar_k \varphi, g_k).$$
    By the Cauchy-Schwartz inequality,
    $$(g, f)_\varphi^2 = (g\kappa, f)_{\varphi + \log \kappa}^2 \leq ||g\kappa||_{\varphi + \log \kappa} ||f||_{\varphi + \log \kappa}$$
    and
\begin{align*}
    ||g\kappa||_{\varphi + \log \kappa}
        &= \inf_{t \neq 0} ||t||^{-2} \sum_{jk} \int_\Omega g\overline g e^{-\varphi} \partial_j \dbar_k \varphi t_j \overline{t_k}
        \leq \sum_{jk} \int_\Omega g\overline g e^{-\varphi} \partial_j \dbar_k \varphi\\
        &\leq \sum_j ||\dbar_j g||_\varphi^2 + \sum_{jk} (g_j\partial_j \dbar_k \varphi, g_k)
\end{align*}
    In conclusion,
$$(g, f)_\varphi^2 \leq ||\delta g||^2_\varphi ||f||_{\varphi + \log \kappa}^2 + \frac{1}{2} \sum_{j \neq k} ||\dbar_j g_k - \dbar_k g_j||_\varphi^2 ||f||_{\varphi + \log \kappa}^2.$$

    Let $N$ be the subspace of $L^2(\varphi, \Omega \to \CC^n)$ consisting of $g$ which satisfy the Cauchy-Riemann constraints. If $h \in N^\perp$ (with respect to $(\cdot, \cdot)_\varphi$) and $\psi$ is a test function, then $\dbar \psi \in N$, so $0 = (h, \dbar \psi)_\varphi = (\delta h, \psi)_\varphi$. But $\psi$ was arbitrary, so $\delta h = 0$.

    Let $P: L^2(\varphi, \Omega \to \CC^n) \to N$ be the canonical projection. Then
    $$|(g, f)_\varphi| = |(Pg, f)_\varphi| \leq ||f||_{\varphi + \log \kappa}^2 ||\delta g||_\varphi.$$
    Let us define the space $D$ of elements of $L^2(\varphi, \Omega \to \CC^n)$ of the form $\delta g$ for some $g \in L^2(\varphi)$, and define $T \in D^*$ by $T(\delta g) = (g, f)_\varphi$. Then $||T|| \leq ||f||_{\varphi + \log \kappa} < \infty$. So by the Hanh-Banach theorem, $T$ admits a linear extension $\tilde T$ to $L^2(\varphi, \Omega \to \CC^n)$. Then $\tilde T$ has a Riesz representation, say $u$, which completes the proof.
\end{proof}



\chapter{Multivariable holomorphic functional calculus}
This chapter follows Hormander's SCV book, Chapter III.

\section{The Gelfand transform}
Let $B$ be an abelian, unital Banach algebra. One of the fundamental problems of the theory of such algebras is to consider to what extent that $B$ can be approximated by algebras of the form $C(K)$, for $K$ a compact Hausdorff space.
\begin{definition}
    A \dfn{Banach algebra representation} of $B$ on $K$ is a continuous morphism of algebras $B \to C(K)$.
\end{definition}

To classify representations, we consider the space of characters on $B$.
\begin{definition}
    A \dfn{character} or \dfn{multiplicative functional} on $B$ is a continuous morphism of algebras $m: B \to \CC$ which is not identically $0$. The space of all characters on $B$ is denoted $M_B$.

    For each $f \in B$, the function $\hat f$ defined on characters by $\hat f(m) = m(f)$ is the \dfn{Gelfand transform} of $f$. We give the space $M_B$ the weakstar topology, namely the weakest topology such that for each $f \in B$, the Gelfand transform $\hat f$ is continuous. The resulting map $B \to C(M_B)$ is called the \dfn{Gelfand representation} of $f$.
\end{definition}
    By the Banach-Alaoglu theorem, $M_B$ is a compact Hausdorff space.

    The Gelfand representation is universal among representations of $B$.
\begin{lemma}
    If $T: B \to C(K)$ is a representation of $B$, then there is a map $\varphi$ so that for every $f \in B$,
    $$Tf = \hat f \circ \varphi.$$
\end{lemma}
\begin{proof}
    Recall that $Te$ is idempotent, so its image $(Te)(K)$ consists only of $0$ or $1$. Let $K_0$ be the kernel of $Te$ and $K_1$ be its complement in $K$. Then $\{K_0, K_1\}$ is a partition of $K$ into compact, open sets. But then for any $f \in B$, $Tf = 0$ on $K_0$, so we might as well assume $K = K_1$. Under this assumption, for each $x \in K$, the map $f \mapsto Tf(x)$ is a character, which we denote $\varphi(k)$.
\end{proof}

\begin{definition}
    Let $f \in B$. The \dfn{resolvent function} of $f$ is the meromorphic function $R_f: U \to B$ given by
    $$R_f(\lambda) = \frac{1}{f - \lambda e}.$$
    The domain $U$ of the resolvent function of $f$ is called the \dfn{resolvent set} of $f$. The complement in $B$ of $U$ of $f$ is called the \dfn{spectrum} of $f$, denoted $\sigma(f)$.
\end{definition}
    In case $B$ is a space of matrices, then the spectrum of $f$ is exactly the set of eigenvalues of $f$ viewed as a linear operator, since then $f$ solves the eigenvalue equation
    $$f(x) = \lambda x.$$
\begin{theorem}[spectral radius theorem]
    \index{spectral radius theorem}
    For each $f \in B$, $\sigma(f) = \{\hat f(m): m \in M_B\}$. Moreover,
    $$\sup_{m \in M_B} |\hat f(m)| = \lim_{n \to \infty} ||f^n||^{1/n}.$$
\end{theorem}
    The proof of this theorem uses some complex analysis, which we now consider.
\begin{lemma}
    Let $g \in B$. If $g$ is invertible, then the mapping
    $$\lambda \mapsto (g - \lambda h)^{-1}$$
    is continuous on the disk $D$ of all $\lambda$ such that
    $$|\lambda| < \frac{1}{||g^{-1}h||}.$$
    Assume $\omega \subseteq D$ is bounded by a finite number of $C^1$ arcs. If $\varphi$ is holomorphic on $\omega$ and $C^1$ on $\overline \omega$, then
    $$\int_{\partial \omega} (g - \lambda h)^{-1} \varphi(\lambda) d\lambda = 0.$$
\end{lemma}
\begin{proof}
    Let $H = g^{-1}h$ and
    $$I(\lambda) = g^{-1}\sum_{n=0}^\infty \lambda^n H^n.$$
    This series converges locally uniformly on $D$, in fact by definition of $D$. Also,
    $$I(\lambda)(g - \lambda h) = I(\lambda)g(e - \lambda H) = e.$$
    Therefore we can integrate term by term after multiplying by $\varphi(\lambda)$, and each term is holomorphic.
\end{proof}
\begin{lemma}
    If $(e - \lambda f)^{-1}$ exists for every $|\lambda| \leq R$, then for each $n \geq 0$,
    $$R^n ||f^n|| \leq \sup_{|\lambda| = R} ||(e-\lambda f)^{-1}||.$$
\end{lemma}
\begin{proof}
    By homotopy invariance and the above lemma, the integral
    $$\frac{1}{2\pi i} \int_{|\lambda| = r} (e-\lambda f)^{-1}\lambda^{-n-1} ~d\lambda$$
    is independent of $r$ if $r \leq R$, and if $r||f|| < 1$, then the integral is equal to $f^n$.
\end{proof}
\begin{lemma}
    For each $f \in B$, $\sigma(f)$ is nonempty.
\end{lemma}
\begin{proof}
    Assume $\sigma(f)$ is empty. Then $(e - \lambda f)^{-1}$ exists for every $\lambda \in \CC$, and the holomorphic function
    $$||(e - \lambda f)^{-1}|| \leq \frac{||f^{-1}||}{|\lambda|}||(e - \lambda^{-1}f^{-1})^{-1}||$$
    is bounded as $\lambda\to \infty$, contradicting Liouville's theorem.
\end{proof}
\begin{corollary}
    If $B$ is a field, then $B = \CC$.
\end{corollary}
\begin{proof}
    By the lemma, for every $f \in B$ we can find $\lambda \in \CC$ such that $f - \lambda e$ is not invertible, but since $B$ is a field, it follows that $f = \lambda e$.
\end{proof}
\begin{lemma}
    If $I$ is a proper ideal of $B$ then there is a $m \in M_B$ such that $m(f) = 0$ for every $f \in I$.
\end{lemma}
\begin{proof}
    By Zorn's lemma, we can find a maximal ideal $\mathfrak m \supseteq I$. The natural map $B \to B/\mathfrak m = \CC$ is a character, call it $m$.
\end{proof}
    Finally we are ready to prove the spectral radius theorem.
\begin{proof}[Proof of spectral radius theorem]
    We first claim that $\{\hat f(m): m \in M_B\} \subseteq \sigma(f)$: if $\lambda \notin \sigma(f)$, then there is a $g \in B$ so that $g(f - \lambda e) = e$, so $\hat g(\hat f - \lambda) = 1$. Therefore $\hat f(m) \neq \lambda$ for any $m$.

    Now let
    $$1/R \leq \sup_{z \in \sigma(f)} |z|.$$
    So if $|\lambda| \leq R$, $(e-\lambda f)^{-1}$ exists by the above lemmata. So
    $$\left(\limsup_{n \to \infty} R^n||f^n||\right)^{1/n} =  R\limsup_{n \to \infty}||f^n||^{1/n} \leq 1$$
    and it follows that
    $$\limsup_{n \to \infty} ||f^n||^{1/n} \leq \frac{1}{R} \leq \sup_{z \in \sigma(f)} |z|.$$

    Third, if $\lambda \in \sigma(f)$, then $f - \lambda e$ is not a unit, so the ideal $I = (f - \lambda e)$ is proper. Therefore we can find a $m \in M_B$ which is annihilated by $I$. Then $\lambda = m(f)$, so $\sigma(f) \subseteq \{\hat f(m): m \in M_B\}$. This proves the first assertion of the spectral radius theorem.

    Since the Gelfand representation is continuous, there is a $C \geq 1$ such that
    $$\sup_{m \in M_B} |\hat f(m)| \leq C||f||.$$
    So
    $$\sup_{m \in M_B} |\hat f(m)| \leq C^{1/n} ||f^n||^{1/n} \leq C^{1/n}||f||$$
    implying that
    $$\sup_{m \in M_B} |\hat f(m)| \leq \liminf_{n \to \infty} ||f^n||^{1/n}.$$
    Therefore by the lemmata
    $$\sup_{m \in M_B} |\hat f(m)| \leq \liminf_{n \to \infty} ||f^n||^{1/n} \leq \limsup_{n \to \infty} ||f^n||^{1/n} \leq \sup_{z \in \sigma(f)} |z| = \sup_{m \in M_B} |\hat f(m)|.$$
    This proves the second assertion.
\end{proof}
    Now we generalize the notion of a spectrum to several complex variables. A version of the spectral radius theorem holds still.
\begin{definition}
    The \dfn{joint spectrum} $\sigma(f_1, \dots, f_n)$ is the set of all $\lambda \in \CC^n$ such that the ideal
    $$(f_1 - \lambda_1 e, \dots, f_n -\lambda_n e)$$
    is proper.
\end{definition}
\begin{corollary}
    For $f_1, \dots, f_n \in B$,
    $$\sigma(f_1, \dots, f_n) = \{(\hat f_1(m), \dots, \hat f_n(m): m \in M_B\}.$$
\end{corollary}
    The proof is essentially the same. This generalization of the spectral radius theorem will allow us to classify the Gelfand representations of finitely generated Banach algebras.
\begin{theorem}
    Let $B$ be the Banach algebra generated by $f_1, \dots, f_n$. Then the mapping
\begin{align*}
    \varphi: M_B &\to \sigma(f_1, \dots, f_n)\\
    m &\mapsto (\hat f_1(m), \dots, \hat f_n(m))
\end{align*}
    is a homeomorphism. Moreover, $\sigma(f_1, \dots, f_n)$ is polynomially convex, and for each $f \in B$, one can approximate $\hat f \circ \varphi^{-1}$ uniformly by polynomials on $\sigma(f_1, \dots, f_n)$.
\end{theorem}
\begin{proof}
    Since $M_B$ carries the weakstar topology, $\varphi$ is continuous, and injective since if $p$ is a polynomial,
    $$m(p(f_1, \dots, f_n)) = p(\hat f_1(m), \dots, \hat f_n(m))$$
    (and polynomials in the $f_j$ are dense in $B$, by definition of $B$). By the corollary of the spectral radius theorem, $\varphi$ is surjective.

    To prove the second statement, let $K = \sigma(f_1, \dots, f_n)$, $z \in \hat K$, and define a map on the generators by $f_j \mapsto z_j$. This will extend to a character on all of $B$ if it is continuous; and indeed
    $$|p(z)| \leq \sup_{w \in K} |p(w)| \leq \sup_{m \in M_B} |m(p(f))| \leq |p(f)|.$$
    Therefore $z \in K$, proving the second claim. The final claim follows because polynomials are dense in $B$.
\end{proof}

\section{The holomorphic functional calculus}
We now show that holomorphic functions can be extended to an (abelian, unital) Banach algebra $B$, even in several complex variables.
\begin{theorem}[holomorphic functional calculus in several complex variables]
    \index{holomorphic functional calculus in several complex variables}
    Let $f_1, \dots, f_n \in B$ and let $\varphi$ be a holomorphic function on a neighborhood of the joint spectrum $\sigma(f_1, \dots, f_n)$. Then there is a $g \in B$ such that
    $$\hat g = \varphi(\hat f_1, \dots, \hat f_n).$$
\end{theorem}
One generally writes $g = \varphi(f_1, \dots, f_n)$, so we think of $\varphi$ as a function $B^n \to B$.

\begin{lemma}
    Let $\Omega \subseteq \CC^n$ be open and contain $\sigma(f_1, \dots, f_n)$. Then there is a finitely generated Banach subalgebra $B'$ of $B$ such that $f_1, \dots, f_n \in B'$ and $\sigma_{B'}(f_1, \dots, f_n) \subseteq \Omega$.
\end{lemma}
\begin{proof}
    As $B'$ increases, $\sigma_{B'}(f_1, \dots, f_n)$ decreases. So we show that for every $z \notin \sigma(f_1, \dots, f_n)$, we can find $B'$ so that $z \notin \sigma_{B'}(f_1, \dots, f_n)$. Indeed, we can find $f_{j+n} \in B$ so that
    $$e = \sum_{j=1}^n f_{j+n}(f_j - z_je).$$
    Now let $B'$ be the Banach algebra generated by the $f_j$.
\end{proof}
\begin{lemma}
    Let $B'$ be as above. Let $f_1, \dots, f_v$ be the generators of $B'$, and let $\pi: \CC^v \to \CC^n$ be the projection which annihilates $(0, \dots, 0, z_{n+1}, \dots, z_v)$. Then there are polynomials $p_k$ such that for each $z \in \CC^v$ such that for each $j$, $|z_j| \leq ||f_j||$, if
    $$|p_k(z)| \leq ||p_k(f_1, \dots, f_v)||,$$
    then $\pi(z) \in \Omega$.
\end{lemma}
\begin{proof}
    Assume $\pi(z) \notin \sigma_{B'}(f_1, \dots, f_n)$. Then the map $f_j \mapsto z_j$ cannot extend to a character on $B'$, so is discontinuous if we were to try to extend it; i.e. there is a $p$ so that
    $$|p(z_1, \dots, z_v)| \geq ||p(f_1, \dots, f_v)||.$$
    This is still true close to $z$, so use compactness of the closed polydisk
    $$\{z \in \CC^v: |z_j| \leq ||f_j||\}.$$
\end{proof}
    Now we come to the theorem that we will use to prove the holomorphic functional calculus.
\begin{theorem}
    Let $f_1, \dots, f_n \in B$ and let $\varphi$ be holomorphic in a neighborhood of $\sigma(f_1, \dots, f_n)$. Then there are $f_{n+1}, \dots, f_N$ and a holomorphic function $\Phi$ on a neighborhood of the polydisk
    $$\{z \in \CC^N: |z_j| \leq ||f_j||\}$$
    such that $\varphi(f_1, \dots, f_n) = \Phi(f_1, \dots, f_N)$.
\end{theorem}
\begin{proof}
    Let $\Omega$ be a neighborhood of $\sigma(f_1, \dots, f_n)$. By the lemma, we can find $f_{n+1}, \dots, f_v$ and $p_1, \dots, p_\mu$ satisfying certain conditions. Let $N = v + \mu$ and $f_{v+k} = p_k(f_1, \dots, f_v)$. The function $\varphi \circ \pi$ is holomorphic in a neighborhood of the compact set of all $z \in \CC^v$ such that $|z_j| \leq ||f_j||$ and $|p_k(z)| \leq ||f_{k+v}||$. Therefore by results in several complex variables, we can find the desired $\Phi$.
\end{proof}
\begin{proof}[Proof of holomorphic functional calculus]
    Let $\Phi$ and $f_{n+1}, \dots, f_N$ be as above. By holomorphy, we can write
    $$\Phi(z) = \sum_\alpha a_\alpha z^\alpha$$
    such that
    $$\sum_\alpha |a_\alpha| R^\alpha < \infty$$
    where $R = (||f_1||, \dots, ||f_N||)$ and $z = (z_1, \dots, z_N)$. Therefore the series
    $$g = \sum_\alpha a_\alpha f^\alpha$$
    norm-converges. Moreover,
    $$\hat g = \sum_\alpha a_\alpha \hat f^\alpha = \Phi(\hat f_1, \dots, \hat f_N) = \varphi(\hat f_1, \dots , \hat f_n).$$
\end{proof}



\chapter{Algebraic geometry}
Throughout this chapter, we assume that all rings are commutative and unital.

\section{Schemes and varieties}
\begin{definition}
    A \dfn{ringed space} $X = (X, \mathcal F)$ consists of a sheaf of rings $\mathcal F$ on $X$. If the stalks of $\mathcal F$ are all local rings, then we say that $X$ is a \dfn{locally ringed space}.

    A \dfn{morphism of ringed spaces} $\psi: (X, \mathcal F) \to (Y, \mathcal G)$ consists of a continuous map $\psi: X \to Y$ and for each $U \in \Open(Y)$, a morphism of rings $\psi_U: \mathcal G(U) \to \mathcal F(\psi^{-1}(U))$ such that for every open set $V \subseteq U$, the diagram
$$\begin{tikzcd}
\mathcal G(V) \arrow[r,"\psi_V"] \arrow[d] &\mathcal F(\psi^{-1}(V)) \arrow[d]\\
\mathcal G(U) \arrow[r,"\psi_U"] &\mathcal F(\psi^{-1}(U))
\end{tikzcd}$$
    commutes.

    Let $(X, \mathcal F)$ and $(Y, \mathcal G)$ be locally ringed spaces. A \dfn{morphism of locally ringed spaces} $\psi: (X, \mathcal F) \to (Y, \mathcal G)$ is a morphism of ringed spaces such that for every $x \in X$, the maximal ideal $\mathfrak m$ of the stalk $\mathcal F_x$ is given by $\mathfrak m = \psi_x(\mathfrak n)$ where $\mathfrak n$ is the maximal ideal of the stalk $\mathcal G_{\psi(x)}$ and $\psi_x$ is the colimit of morphisms $\psi_U$ as $U$ ranges over the directed set $\mathcal D_{\psi(x)}$ of all open sets $U \ni \psi(x)$.
\end{definition}
So a morphism of locally ringed spaces is a morphism of ringed spaces, whose domain and codomain are locally ringed, which preserves the maximal ideals at each stalk.

\begin{definition}
    Let $R$ be a ring. Let $X$ be the spectrum of $R$, equipped with the Zariski topology. If $U = D(f)$ is a distinguished open set, let $\mathcal F(U)$ be the localization of $R$ at $R \setminus D(f)$. Let $\mathcal F$ be the induced sheaf. Then $\Spec R = (X, \mathcal F)$ is called the \dfn{affine scheme} associated to $R$.
\end{definition}
\begin{proposition}
    Let $R$ be a ring. The affine scheme $\Spec R$ is a locally ringed space, and $R$ is the ring of global sections of $\Spec R$.
\end{proposition}
\begin{definition}
    A \dfn{scheme} is a locally ringed space which is locally an affine scheme. A \dfn{morphism of schemes} is a morphism of locally ringed spaces.
\end{definition}
    If $X$ is a scheme, then there is a unique morphism $X \to \Spec \ZZ$, which is the categorical dual of the morphisms $\ZZ \to R$ for each ring $R$ appearing in the definition of $X$.
\begin{definition}
    Let $S$ be a scheme. A \dfn{scheme over} $S$, $X$, is a scheme $X$ such that there is a morphism of schemes $\pi: X \to S$.

    Let $X$ and $Y$ be schemes over $S$, witnessed by morphisms $\pi: X \to S$ and $\varphi: Y \to S$. A \dfn{morphism of schemes over} $S$, $\psi: X \to Y$, is a morphism of schemes $\psi: X \to Y$ such that $\pi = \varphi \circ \psi$.

    In case $S = \Spec \CC$, we say that $X$ is a \dfn{complex scheme}.
\end{definition}
    Notice that if $X = (X, \mathcal F)$ is a complex scheme and $U \subseteq X$ is an affine subscheme, then $\mathcal F(U)$ admits a morphism of rings $\CC \to \mathcal F(U)$. This gives rise to a complex algebra structure on $\mathcal F(U)$.
\begin{definition}
    If $X$ is a complex scheme and every algebra $\mathcal F(U)$ is finitely generated over $\CC$, we say that $X$ is a \dfn{complex scheme of finite type}.
\end{definition}
\begin{definition}
    A \dfn{reduced scheme} $X = (X, \mathcal F)$ is a scheme such that for every open set $U \subseteq X$, the ring $\mathcal F(U)$ has no nilpotents.
\end{definition}
\begin{definition}
    Let $\pi: X \to Y$ be a morphism of schemes. The \dfn{diagonal morphism} $\delta_\pi: X \to X \times_Y X$ is the fiber product of the identity $X \to X$ with itself induced by $\pi$. If $\delta_\pi(X)$ is closed in $X \times_Y X$, we say that $\pi$ is a \dfn{separated morphism}.

    If the unique morphism $X \to \Spec \ZZ$ is separated, we say that $X$ is a \dfn{separated scheme}.
\end{definition}
\begin{definition}
    A \dfn{variety} is a reduced, separated complex scheme of finite type.
\end{definition}

\section{Formal power series}
Let $\sum_\alpha a_\alpha z^\alpha$ be a formal power series on $\CC^n$ with domain of (absolute) convergence $D$. Let $B$ be the set of $z$ such that $|a_\alpha z^\alpha|$ is uniformly bounded in $\alpha$. Clearly $D \subseteq B$.
\begin{lemma}
    Assume $w \in B$ and $U = \{z \in \CC^n: |z_j| < |w_j|\}$. Then $U \subseteq D$.
\end{lemma}
\begin{theorem}
    $D^* = \{\xi \in \RR^n: (e^{\xi_1}, \dots, e^{\xi_n}) \in D\}$ is an open, convex set. If $\xi \in D^*$ and $|\eta_j| \leq |\xi_j|$ then $\eta \in D^*$. Moreover, $z \in D$ if and only if $|z_j| \leq e^{\xi_j}$ for some $\xi \in D^*$.
\end{theorem}
\begin{proof}
    $D^*$ is the interior of $B^*$. We will show $B^*$ is convex. There is an $M$ such that $\xi, \eta \in B^*$ if and only if
    $$|a_\alpha e^{\alpha\xi}| \leq M$$
    and similarly for $e^{\alpha\eta}$. This remains true when we raise both sides to the $t$ or $1-t$ power. Then
    $$|a_\alpha|e^{\alpha(t\xi + (1-t)\eta)} \leq M.$$
    Thus $t\xi + (1-t)\eta \in B^*$.

    The other claims follow from the definition or are obvious from the lemma.
\end{proof}
\begin{definition}
    A \dfn{Reinhardt domain} is a set $\Omega \subset \CC^n$ such that for every $z \in \Omega$ and every $\theta \in \RR^n$, $(e^{i\theta_1} z_1, \dots, e^{i\theta_n} z_n) \in \Omega$.
\end{definition}


\section{Bergman kernels}
Suppose $\Phi$ is a strictly plush quadratic form on $\CC^n$. That is,
$$\Phi(z) = \Re \langle Az, z\rangle + \langle Cz, \overline z \rangle$$
where $A \in \CC^{n \times n}$ and $C$ is a positive matrix. (The inner product is antilinear in $\overline z$, hence why we needed complex conjugation). Then we define
$$L^2_\Phi(\CC^n) = \left\{u: \int_{\CC^n} |u(z)|^2 e^{-2\Phi(z)} ~dz < \infty\right\}.$$
Thus we can define $H_\Phi(\CC^n) = L^2_\Phi(\CC^n) \cap A(\CC^n)$. Taking the holomorphic part of a compactly supported function (which is always possible for smooth functions by the Hormander $L^2$ estimates), we see that $H_\Phi(\CC^n)$ is nonempty. $H_\Phi(\CC^n)$ is closed, hence a Hilbert space.

It is often useful to have a semiclassical parameter, so we put
$$||u||_\Phi^2 = \int_{\CC^n} |u|^2 e^{-2\Phi/h} ~dm.$$

Putting
$$u_\alpha(z) = e^{-\langle Az, z\rangle/h} z^\alpha$$
we recover an orthonormal basis
$$f_\alpha = C (h^{n + |\alpha|} \alpha!)^{-1/2}u_\alpha$$
of $H_\Phi$.

Let
$$\Pi_\Phi: L^2_\Phi \to H_\Phi$$
be the orthogonal projection. Then one can check that
$$\Pi_\Phi u(z) = C \int_{\CC^n} e^{2\Psi(z, \overline w)/h - 2\Phi(w)/h} u(w) ~dm(w),$$
where $\Psi: \CC^{2n} \to \CC$ is the unique analytic quadratic function such that
$$\Psi(z, \overline z) = \Phi(z).$$
\begin{example}
    If $\Phi(z) = |z|^2/2$ then $\Psi(z, w) = \langle z, w\rangle/2$ and
$$\Pi u(z) = \frac{1}{(\pi h)^n} \int_{\CC^n} e^{-\langle z, \overline w\rangle/h - |w|^2/h}u(w) ~dm(w).$$
\end{example}
    In quantum mechanics, $H$ is the space of wavefunctions. If $g \in L^\infty$ is a classical observable, we quantize $g$ by
    $$T_g = \Pi g \Pi.$$
    In fact, $g$ itself cannot be holomorphic by Liouville's theorem, but $T_g$ preserves $A$.
\begin{definition}
    The projection $T_g$ is called the \dfn{Toeplitz operator} of $g$.
\end{definition}
\begin{example}
    Let $g(z) = \overline z_j$, $u \in H_{\Phi + \varepsilon|z|^2}$, where $\Phi(z) = |z|^2/2$. Then $T_g$ carries $H$ to itself, and
    $$T_gu(z) = \frac{1}{(h\pi)^n} \int_{\CC^n} e^{-\langle z, \overline w \rangle/h - |w|^2/h} \overline w_j u(w) ~dm(w).$$
    We first see that
    $$\overline w_j e^{-\langle z, \overline w\rangle - |w|^2/h} = -h\partial_{w_j} e^{-\langle z, \overline w\rangle - |w|^2/h}$$
    and since the integrals converge, we integrate by parts to see that
    $$T_gu(z) = \frac{1}{(h\pi)^n} e^{-\langle z, \overline w\rangle/h - |w|^2/h} h\partial_{w_j}u(w) ~dm(w) = \Pi(h\partial_ju(z)) = h\partial_j u(z).$$
    That is, multiplication by $\overline z_j$ is the same as differentiating in $z_j$. So this is, in fact, a quantization.
\end{example}




\section{Quillen's theorem}
\begin{theorem}[Catlin-d'Angelo-Quillen]
    \index{Catlin-d'Angelo-Quillen theorem}
    Let
    $$f(z, \overline z) = \sum_{|\alpha| = |\beta| = m} c_{\alpha\beta} z^\alpha \overline z^\beta$$
    be a bihomogeneous quadratic form on $\CC^n$ such that $f(z, \overline z) > 0$ for every $z \neq 0$. Then there is a $N \in \NN$ and a polynomial
    $$P_j(z) = \sum_{|\alpha| = m} p_\alpha^j z^\alpha$$
    such that
    $$|z|^{2N} f(z, \overline z) = \sum_{j=1}^J |P_j(z)|^2.$$
\end{theorem}
    Quillen developed this theorem to prove the complex Nullstellensatz. It was also used by Polya for the following theorem.
\begin{theorem}[Polya]
    \index{Polya's theorem}
    Suppose $p$ is a real homogeneous polynomial on $\RR^n$ and $p(x) > 0$ if $\forall j ~x_j \geq 0$ and $\sum_j x_j = 1$. Then there is a $N \in \NN$ such that $(x_1 + \dots + x_n)^Np(x)$ has positive coefficients.
\end{theorem}
\begin{proof}
    Write $x_j = z_j \overline{z_j}$ and put $c_{\alpha\beta} = 0$ for $\alpha \neq \beta$, then use Quillen's theorem.
\end{proof}
    To prove Quillen's theorem, we develop the theory of real quadratic forms.

    Let $q: \CC^n \to \RR$ be a quadratic form.
\begin{lemma}
    There is a decomposition $q = h + \ell$ such that $h(iz) = -ih(z)$ and $\ell(z) = \ell(iz)$.
\end{lemma}
\begin{definition}
    $\ell$ is the \dfn{Levi form} of $q$.
\end{definition}
\begin{proof}
    Define $Jq(z) = q(iz)$. Then $J^2 = 1$. Put $h = (q - Jq)/2$, $\ell = (q + Jq)/2$. Then $h$ is \dfn{pluriharmonic} in the sense that for any $j$ in any coordinate system, $\partial_j \dbar_j h(iz)$ is constant because $h$ is a quadratic form. Moreover,
    $$\partial_j \dbar_j h(iz) = -\partial_j(\dbar_jh(iz)) = \partial_j \dbar_j h(iz)$$
    which proves the claim for $h$.
\end{proof}
    To prove Quillen's theorem, let
    $$P_f(z) = \sum_{|\alpha| = |\beta| = m} c_{\alpha\beta} z^\alpha (h\partial)^\beta$$
    be a quantization of $f$. Then if $h = 1/N$, $P_f$ carries the space $\mathcal P_{m + N}$ of homogeneous polynomials of degree $m + N$ to itself, as we will prove, and this will prove Quillen's theorem.
\begin{lemma}
    There is a polynomial $P$ such that
    $$f(z, \overline z) = \sum_{j=1}^J |P_j(z)|^2$$
    if and only if $A = (c_{\alpha\beta})_{\alpha\beta}$ is a positive-definite matrix.
\end{lemma}
    Note that this is not trivial because $A$ acts on the vector space $\CC^K$, where $K$ is the set of partitions of $m$, which by some combinatorics can be very large!
\begin{proof}
    For any symmetric matrix $A \in \CC^{K \times K}$ we can find $w_k \in \CC^K$, $\lambda_j$, $j,k \in K$, such that
    $$A = \sum_{j,k} \lambda_j w_k w_k^*$$
    by the spectral theorem. Here $f$ is real-valued, so $c_{\alpha\beta} = \overline{c_{\beta\alpha}}$ whence $A$ is symmetric. Thus we can use that decomposition.

    Let $Z = (z^\alpha)_{\alpha \in K}$. Then
    $$f(z, \overline z) = Z^*AZ = \sum_{\alpha \in K} \lambda_\alpha Z^*w_\alpha w_\alpha^* Z$$
    so $f(z, \overline z)$ is the sum over $\alpha$ of the sign of $\lambda_\alpha$ times $|P_\alpha(z)|^2$, where
    $$P_\alpha(z) = |\lambda_\alpha|^{1/2} w_\alpha^* Z = |\lambda_\alpha|^{1/2} \sum_{\beta \in K} \overline w_\alpha^\beta z^\beta.$$
    So if $A$ is positive then the $\lambda_\alpha \geq 0$.

    For the converse, just run the same argument in reverse.
\end{proof}
\begin{example}
    Let $z \in \CC^2$,
    $$f(z, \overline z) = |z_1|^4 + |z_2|^4 + c|z_1|^2 |z_2|^2$$
    for some $|c| \in (0, 2)$. Then $f$ is a positive quadratic form. There are only $3$ partitions of $2$, so the matrix $A$ acts on $\CC^3$ by
    $$A = \begin{bmatrix}
        1 &&\\&c&\\&&1
    \end{bmatrix}.$$
    Then $A$ is positive-definite if and only if $c > 0$, so that is exactly when $f$ is a positive quadratic form.
\end{example}
    Let $\mathcal P_{n + M}$ be the space of homogeneous polynomials of degree $n + M$. Let $\Phi(z) = |z|^2/2$ as in the theory of Bergman kernels and $T_\cdot$ be the quantization operator.
\begin{lemma}
    Let
    $$|z|^{2N} f(z, \overline z) = \sum_{|\alpha| = |\beta| = N + m} c_{\alpha\beta}^N z^\alpha \overline z^\beta$$
    be a bihomogeneous form. Let
    $$P_f(z) = \sum_{|\alpha| = |\beta| = m} c_{\alpha\beta}z^\alpha T_{\overline z^\beta}.$$
    Then $A = (c_{\alpha\beta})$ is positive if and only if there is a $c > 0$ such that for every $u \in \mathcal P_{n + M}$ we have
    $$\langle P_f u, u\rangle_\Phi \geq c||u||_\Phi$$
\end{lemma}
\begin{proof}
    The orthonormal projection $\Pi: L^2 \to H$ is self-adjoint so
    $$T_g^* = T_{\overline g}.$$
    Moreover, it is easy to check that $P_f$ is self-adjoint, and $P_f = \sum_{\alpha,\beta} c_{\alpha\beta} z^\alpha (h\partial)^\beta$. Then
\begin{align*}P_fg(z) &= \sum_{\alpha,\beta} c_{\alpha\beta} z^\alpha \Pi(\overline z^\beta g(z)) \\
&= \sum_{L=0}^\infty \frac{1}{(\pi h)^n} \int_{\CC^n} \frac{\langle z, \overline w\rangle^L}{h^LL!} f(z, \overline w) g(w) e^{-|w|^2/h} ~dm(w)\\
&= \sum_L (h\pi)^{-n} \int_{\CC^n} \sum_{|\mu| = L} \frac{z^\mu \overline w^\mu}{\mu! h^{|\mu|}} f(z, \overline w) g(w) e^{-|w|^2/2h} ~dm(w).
\end{align*}
Assume $g \in \mathcal P_{m + N}$. By homogeneity one can only get a nonzero contribution to the integral when $L = N$. So
$$P_fg(z) = (\pi h)^{-n} \int_{\CC^n} \sum_{|\mu| = N} \frac{z^\mu \overline w^\mu}{\mu! h^{|\mu|}} f(z, \overline w) g(w) e^{-|w|^2/h} ~dm(w).$$
Let $u \in \mathcal P_{m + N}$ expand as
$$u(z) = \sum_{|\gamma| = m + N} u_\gamma(z) z^\gamma.$$
Then
\begin{align*}\langle P_fu, u\rangle &=
h^{N + 2m}\sum_{\substack{|\alpha| = |\beta| = m\\|\mu| = N}} c_{\alpha\beta} \frac{(\alpha + \mu)!(\beta + \mu)!}{\mu!}u_{\beta + \alpha} \overline u_{\alpha + \mu}\\
&= (\pi h)^{-n} h^{N+2m} \sum_{|\gamma| = |\rho| = N + m} c^N_{\rho\gamma} \rho! \gamma! u_\rho \overline u_\gamma
\end{align*}
and $(c^N_{\rho\gamma}\rho!\gamma!)$ is positive iff $A$ is positive. Since
$$\frac{\langle z, \overline z\rangle^N}{N!} \sum_{|\alpha|=|\beta|=m} c_{\alpha\beta}z^\alpha z^\beta = \sum_{\substack{|\alpha| = |\beta| =m\\|\mu| = N}} \frac{c_{\alpha\beta}}{\mu!} z^{\alpha + \mu} \overline z^{\beta + \mu}$$
we plug this back into $\langle Pf u, u\rangle$.
\end{proof}
\begin{lemma}
    For every $\delta > 0$ and $m > 1$ there is a $\varepsilon > 0$ such that if $|Mh - 1| < \varepsilon$, then for every $u \in \mathcal P_M$,
    $$||~|z|^mu||_{L^2_\Phi(|z| \leq 1 - \delta)} \leq O(h^\infty)||u||_\Phi.$$
\end{lemma}
\begin{proof}
    Let
    $$\Pi_M: L^2_\Phi \to \mathcal P_M$$
    be the orthogonal projection. Now
    $$||\Pi_M |z|^m 1_{|z| \leq 1/2} \Pi_M||_{L^2_\Phi \to L^2_\Phi} = O(h^\infty)$$
    if $hM$ is close to $1$. In fact, it is no trouble to reduce to when $m = 0$ since $|z|^m < 2^{-m}$ and then
    $$||u||_{L^2_\Phi(|z| < 1/2)}^2 = \langle 1_{|z| < 1/2}u, u\rangle_\Phi = \langle \Pi_M 1_{|z| < 1/2}\Pi_Mu, u\rangle_\Phi.$$
    We now use Schur's criterion with $p(w) = |w|^M$. In fact
    $$\Pi_Mu(z) = (\pi h)^n \int_{\CC^n} \langle z, \overline w\rangle^M\frac{u(w)}{M!h^M} e^{-|w|^2/h} ~dw$$
    so if $K$ is the integral kernel defined by
    $$\Pi_M1_{|z|\leq 1/2}\Pi_M = \int_M K(z, w) u(w) e^{-|w|^2/h} ~dw$$
    we have
    $$K(z, w) = (h\pi)^{-2n} \int_{\CC^n} \frac{\langle z, \overline \zeta\rangle^M}{M!h^M} 1_{|\zeta|<1/2} \frac{\langle \zeta, \overline w\rangle^M}{M!h^M} e^{-|\zeta|^2/h} ~d\zeta$$
    since by Fubini's theorem the integral kernel of a product of integral operators is the product of the integral kernels. Thus
\begin{align*}|K(z, w)| &\leq (\pi h)^{-2n} \int_{|\zeta|<1/2} \frac{|z|^M|\zeta|^{2M}|w|^M}{M!h^M M!h^M} e^{-|\zeta|^2/h} ~d\zeta
\end{align*}
    implies
\begin{align*}
    \int_{\CC^n} |K(z, w)| |w|^M e^{-|w|^2/h} ~dw &\leq \frac{|z|^M}{(\pi h)^{2M}} \int_{\CC^n} \int_{|\zeta| < 1/2} \frac{|\zeta|^{2M}}{M!h^M} \frac{|w|^{2M}}{M!h^M} e^{-|\zeta|^2/h} e^{-|w|^2/h} ~d\zeta ~dw\\
    &=: I_1I_2
\end{align*}
    where we are thinking of the $e^{-|\cdot|^2/h}$ as a Radon-Nikodym derivative of a certain measure (so we can use Schur's criterion on that measure). Using polar coordinates,
\begin{align*}I_1 &\leq \frac{C}{M!h^{M+n}} \int_0^\infty r^{2M+2n-1} e^{-r^2/h} ~dr\\
&= O\left(\frac{1}{M!}\right) \int_0^{\infty} t^{M+n-1} e^{-t} ~dt\\
&\leq O\left(\frac{1}{M!}\right) (M + n - 1)! \leq O((M + n)^n).
\end{align*}
    Since $Mh$ is close to $1$, $e^{(M+m+n-1)t}$ looks like $e^{t/h}$, and if $1 - \rho = Mh$ then $\rho$ is small. We recall that $\log 1/4 - 1/4 \approx -1.6 < 1$. So if we take $r^2 = t$,
\begin{align*}
    I_2 &\leq C\int_0^{1/2} \frac{r^{2M + 2n - 1}}{M!h^{M+n}} e^{-r^2/h} ~dr \\
    &= C\int_0^{1/4} \frac{t^{M+m+n-1}}{M!h^{M+n}} e^{-t/h} ~dt
    = \frac{Ce^{\delta/4h}}{M!h^{M+n}} \int_0^{1/4} (te^{-t})^{1/h} ~dt\\
    &= \frac{Ce^{\delta/4h}}{M!h^{M+n}} \int_0^{1/4} e^{(\log t - t)/h} ~dt
    \leq C\frac{e^{\delta/4h}-1.5/h}{M!h^{M+n}}\\
    &\leq C\frac{e^{-1.2/h}}{M!(1-\rho)^{-(n+M)}M^{-(n+M)}}
    \leq C\frac{(1+\rho)^{n+M}e^{-1.2/h}}{M^{M+1/2}M^{-M}M^{-n}}\\
    &\leq O(M^n (1 + \rho)^{n+M}) e^{-(1-\rho)M}
    \leq O(e^{-CM}) = O(e^{-C/h})
\end{align*}
    by Stirling's formula.
\end{proof}
\begin{lemma}
    Let $|\alpha| = |\beta| = m$. There are $p_j \in \mathcal P_{m - j}$ such that if
    $$q_{\alpha\beta}(z, \overline z) = z^\alpha \overline z^\beta + \sum_{j=1}^n h^j p_j^{\alpha\beta}(z, \overline z)$$
    then $z^\alpha T_{\overline z^\beta} = T_{q_{\alpha\beta}}$.
\end{lemma}
\begin{proof}
    This is obvious if $m = 0$. Otherwise, assume that it is true for $m$, and use the fact that
    $$T_{z^\alpha \overline z^\beta} = (h \partial_z)^\beta z^\alpha = z^\alpha(h\partial_z)^\beta \sum_{j=1}^m h^j \sum_{|\mu|=m-j} p_j^{\alpha\beta} z^\mu (h\partial_z)^\mu$$
    to prove the lemma by induction.
\end{proof}
\begin{proof}[Proof of Quillen's theorem]
    By a previous lemma, we have $f = T_q$ where
    $$q(z, \overline z) = f(z, \overline z) + \sum_{j=1}^m h^j p_j(z, \overline z)$$
    for some $p_j \in \mathcal P_{m - j}$ which depend on our choice of $h$. We claim that for some $N > 1$, $h < 1$ and every $u \in \mathcal P_{m + N}$, there is a $c > 0$ such that
    $$\langle T_qu, u\rangle_\Phi \geq c||u||^2_\Phi.$$
    In fact we take $c$ so that $f(z, \overline z) > 2c$ for $2|z| > 1$, which is possible by homogeneity of $f$. So if $h > 0$ is small enough, then
    $$q(z, \overline z) > c$$
    for every $z$ with $2|z| > 1$. In particular we have the luxury to choose $h$ so that we can find $N$ such that $N = 1/h$. Now $\Pi$ is self-adjoint and $u \in H$, so
\begin{align*}
    \langle T_qu, u\rangle_\Phi &= \langle \Pi q\Pi u, u\rangle_\Phi = \langle q\Pi u, \Pi u\rangle_\Phi \\
    &= \langle qu, u\rangle_\Phi \geq c\langle 1_{|z| \geq 1/2}u, u\rangle_\Phi - c||u||_{L^2_\Phi(|z| < 1/2)}\\
    &= c||1_{|z| \geq 1/2} u||_{L^2_\Phi(|z| \geq 1/2)}^2 - O(h^\infty)||u||_\Phi\\
    &\geq c\frac{||u||_\Phi^2}{2}.
\end{align*}
\end{proof}

\chapter{Line bundles over complex varieties}
\begin{definition}
    An \dfn{immersion} is an injective continuous function.
\end{definition}

\begin{definition}
    An \dfn{holomorphic atlas} on a topological space $M$ is an open cover $(U_\alpha)_\alpha$ equipped with open immersions $\tau_\alpha: U_\alpha \to \CC^n$ such that whenever $U_\alpha \cap U_\beta$ is nonempty, the mapping
    $$\tau_\alpha \circ \tau_\beta^{-1}: \tau_\beta(U_\alpha \cap U_\beta) \to \tau_\alpha(U_\alpha \cap U_\beta)$$
    is holomorphic.
\end{definition}

\begin{definition}
    A \dfn{complex manifold} $M$ is a Hausdorff space equipped with a holomorphic atlas.
\end{definition}

\begin{definition}
    A function $f: M \to \CC^m$ is a \dfn{holomorphic function} if for every $\alpha$, $f \circ \tau^{-1}: \tau_\alpha(U_\alpha) \to \CC^,$ is holomorphic.
\end{definition}

\begin{example}
    Let
    $$M = \{z \in \CC^{n+1}: z_0^2 + \dots + z_n^2 = 1\}.$$
    Then there is a $j$ such that $z_j \neq 0$. By the implicit function theorem, we can show that $M$ is a complex manifold. But it is not compact, and in fact has complex codimension $1$.
\end{example}

\begin{definition}
    Let $V$ be a complex vector space, which the multiplicative group $\CC^*$ acts on by scalar multiplication. Let $\PP(V) = (V\setminus 0)/\CC^*$, the \dfn{complex projective space}. Define for $x \in \PP(V)$ the equivalence class $[x] \in \PP(V)$. Define $\Omega_j = \{[x] \in \PP(V): x_j \neq 0\}$. Then define
    $$\tau_j^{-1}([x]) = (x_0x_j^{-1}, \dots, x_{j-1}x_j^{-1}, x_{j+1}x_j^{-1}, \dots, x_nx_j^{-1}).$$
    Then for $z \in \tau_j(\Omega_i \cap \Omega_j)$,
    $$\tau_i \circ \tau_j^{-1}(z) = (z_0z_i^{-1}, \dots, z_{j-1}z_i^{-1}, z_i^{-1}, z_{j+1}z_i^{-1}, \dots, z_{i-1}z_i^{-1}, z_{i+1}z_i^{-1}, \dots, z_nz_i^{-1}).$$
\end{definition}
    Then if $V = \CC^{n+1}$, $\PP(V) = S^{2n-1}/S^1$ where $S^1$ acts on $S^{2n-1}$ by scalars. So $\PP(V)$ is a compact complex manifold.
\begin{example}
    Let $M = \{x \in \CC^{n+1}: x_0^2 + \dots + x_{n+1}^2 = 0\}/\CC^*$, where $\CC^*$ acts on $\CC^{n+1}$ by scalars. By a similar argument as with projective space, $M$ is a compact complex manifold.
\end{example}

\begin{theorem}
    The only holomorphic functions on a compact connected complex manifold are constant.
\end{theorem}
\begin{proof}
    A function would have to attain a maximum since the manifold is compact, and by the maximum principle if a function attains its maximum on the interior of a connected set then it is constant there.
\end{proof}

\section{Holomorphic line bundles}
\begin{definition}
    A \dfn{holomorphic line bundle} is a holomorphic projection of complex manifolds $\pi: E \to M$, such that for every $x \in M$, $E_x = \{\pi^{-1}(x)\}$ is isomorphic to $\CC$, and that there is an open set $U \ni x$ and a holomorphic function $\Theta: \pi^{-1}(U) \to U \times \CC$ such that $\Theta|_{E_x}: E_x \to \{x\} \times \CC$ is an isomorphism of vector spaces.
\end{definition}
    We define $\Theta_{j\ell} = \Theta_j \cap \Theta_\ell^{-1}$ whenever $U_j \cap U_\ell$ is nonempty. Then $\Theta_{j\ell}(x, \cdot)$, $x \in M$, is a linear map $\CC \to \CC$, so there is a scalar $g_{j\ell}(x)$ such that $\Theta_{j\ell}(x, t) = g_{j\ell}(x)t$.
\begin{definition}
    The functions $g_{j\ell}: \Theta_j \cap \Theta_\ell^{-1} \to \CC$ are called \dfn{transition functions} for the holomorphic line bundle.
\end{definition}
    We have $g_{j\ell}g_{\ell j} = 1$. In fact, on $U_j \cap U_k \cap U_\ell$, then $g_{j\ell}g_{\ell k}g_{kj} = 1$. On the other hand, we have a holomorphic line bundle for any family of functions with these compatibility conditions. Let $J$ be the index set, $(j, x, t) \sim (j', x, t')$ whenever $t' = g_{j'j}t$. Then $J \times M \times \CC$ projects by $\sim$ to a holomorphic line bundle $E$.
\begin{example}
    The \dfn{trivial line bundle} is $E = M \times \CC$.
\end{example}
\begin{example}
    The \dfn{tautological line bundle} $O(-1)$ over $\PP^n$, the moduli space of all lines through $0 \in \CC^{n+1}$, sends the line corresponding to each point of $\PP^n$ to itself. It is defined by
    $$O(-1) = \{([x], \xi) \in \PP^n \times \CC^{n+1}: \xi \in [x]\}.$$
    (Here we allow $0 \in [x]$.) We define $\Theta_j([x], \xi) = \xi_j$. Then $\Theta_j \circ \Theta_\ell^{-1}([x], \xi) = x_j\xi/x_\ell$. That is,
    $$g_{j\ell}(x) = \frac{x_j}{x_\ell}.$$
\end{example}

\begin{definition}
    Let $\pi: E \to M$ be a holomorphic line bundle. A \dfn{section} is a right inverse of $\pi$.
\end{definition}
    So a section carries $x \in E$ to the complex line $\CC \times \{x\}$. We denote by $C^\infty(M, E)$ the space of smooth sections of $\pi$. The space of holomorphic sections is denoted by $H^0(M, E)$, for sheaf-theoretic reasons.
\begin{example}
    For the tautological line bundle, let $x \in U_j$. We define a vector space isomorphism $e_j: E_x \to \CC$ that sends a point in $[x]$ to its $j$th coordinate. This allows us to express as section $s$ as $s(x) = s_j(x) e_j(x)$ whenever $x \in U_j$. If $x \in U_j \cap U_\ell$, then $e_\ell(x) = g_{j \ell}(x)e_j(x)$ and $s_j(x) e_j(x) = s_\ell(x) e_\ell(x)$. So $s_j(x) e_j(x) = s_j(x) g_{\ell j}(x) e_\ell(x) = s_\ell(x) e_\ell(x)$ whence $s_j(x)/x_j = s_\ell(x)/x_\ell$. (Clearly this does not depend on the choice of $x \in [x]$.)

    A section $s$ of $O(-1)$ defines a function on $\CC^{n+1} \setminus 0$, $\tilde s$, by, for $[x] \in U_j$,
    $$\tilde s(x) = \frac{s([x])}{x_j}.$$
    Then $\tilde s$ is well-defined, does not depend on the choice of $j$, and is homogeneous of degree $-1$. So $C^\infty(\PP^n, O(-1))$ is infinite-dimensional while $H^0(\PP^n, O(-1)) = 0$.
\end{example}
    We now define operations on the category of line bundles. The goal is to construct line bundles that have holomorphic sections.
\begin{definition}
    Let $M$ be a complex manifold. If $E$ is a holomorphic line bundle, we define the \dfn{dual line bundle} $E^*$ by requiring that the fibers $E_x^*$ are dual vector spaces to the fibers $E_x$. Given $g_{ij}$ the transition maps for $E$, we define the dual transition maps $g_{ij}^* = g_{ij}^{-1}$.
\end{definition}
    Then $E^*$ is a holomorphic line bundle, and given $x \in U_j \cap U_k \subset E$, $\xi \in E_x$, $f \in E_x^*$, $f(\xi)$ does not depend on whether we compute $f(\xi)$ in $U_k$ or in $U_j$.
\begin{example}
    The dual of the tautological line bundle is by definition $O(1) = O(-1)^*$. It has the transition maps
    $$g_{ij}^*([x]) = \frac{x_j}{x_i}.$$
    As above, the sections are functions on $\CC^{n+1} \setminus 0$ which are homogeneous of degree $1$. So $H^0(\PP^n, O(1))$ consists of linear forms on $\CC^{n+1}$ (since every holomorphic function extends over $0$ on $\CC^{n+1}$.)
\end{example}

\begin{definition}
    Let $F,E$ be holomorphic line bundles over $M$. We define the \dfn{tensor product of line bundles} by $(F \otimes E)_x = F_x \otimes E_x$ on fibers and
    $$g_{ij}^{F \otimes E} = g_{ij}^F \otimes g_{ij}^E.$$
\end{definition}
    Here the tensor product of holomorphic functions is defined by pointwise multiplication.
\begin{example}
    Let $O(k) = O(1)^{\otimes k}$ (for $k \geq 0$; for $k < 0$ we have $O(k) = O(-1)^{\otimes -k}$). Then we have transition maps $g_{ij}^k([x]) = x_j^k x_i^{-k}$ so $H^0(\PP^n, O(k))$ consists of $k$th degree forms on $\CC^{n+1}$. The dimension of $H^0(\PP^n, O(k))$ is $(n+k)!/n!$.
\end{example}
\begin{definition}
    The \dfn{Picard group} is the group $\{O(k): k \in \ZZ\}$ of line bundles on $\PP^n$. with tensor product as multiplication and duality as inversion.
\end{definition}
    So the Picard group is isomorphic to $\ZZ$. The identity of the Picard group $O(0)$ is the trivial bundle. The sections on $O(0)$ are exactly those functions $\PP^n \to \CC$.


\begin{definition}
    A \dfn{Hermitian metric} on a holomorphic line bundle $h = (h_j)_j$ is a vector of smooth $h_j: U_j \to [0, \infty)$ satisfying the compatibility condition $h_j(x) = |g_{ij}(x)|^2 h_i(x)$, such that with $e_j(x) = \Theta^{-1}_j(x, 1)$ defined for $x \in U_j$, we have norms $||\cdot||$ on each fiber defined by
    $$||e_j(x)||^2_h = h_j(x).$$
\end{definition}
    The norm is well-defined because any element of a fiber $E_x$ is a scalar multiple of $e_j(x)$. For a section $s$ we have
    $$||s(x)||_h^2 = h_j(x) |s_j(x)|^2$$
    in $U_j$.
\begin{definition}
    The \dfn{Fubini-Study metric} is the Hermitian metric on $O(1)$ given by
    $$h_j([x]) = \frac{|x_j|^2}{|x|^2}$$
    for $[x] \in \PP^n$.
\end{definition}
\begin{example}
    Let
    $$f(x, \overline x) = \sum_{|\alpha| = |\beta| = k} c_{\alpha\beta} x^\alpha \overline x^\beta$$
    be such that $f(x, \overline x) > 0$. Then we can define
    $$h_j(x) = \frac{|x_j|^{2k}}{f(x, \overline x)}.$$
    By Quillen's theorem, if $k$ is large enough then there is a polynomial such that the metric we have just defined is the pullback of the Fubini-Study metric.
\end{example}
\begin{definition}
    A \dfn{Hermitian line bundle} is a holomorphic line bundle equipped with a Hermitian metric.
\end{definition}
\begin{definition}
    A \dfn{positive line bundle} is a Hermitian line bundle $(E, h)$ such that $h_j = e^{-\varphi_j}$ for some strictly plush functions $\varphi_j$.
\end{definition}
    On a positive line bundle, there will be lots of holomorphic sections.
\begin{example}
    Take $U_0 \subset \PP^n$ to be the set of $[x]$ such that $x_0 \neq 0$. Then $h_0([x]) = |x_0|^2|x|^{-2}$ for the Fubini-Study metric. The Fubini-Study metric makes $O(1)$ into a positive line bundle by
    $$\varphi_0(z) = - \log(|x_0|^2 |x|^{-2}) = \log(1+|z|^2).$$
    In fact,
    $$\partial_j \dbar_k \varphi_0(z) = \frac{\delta_{jk}}{1 + |z|^2} - \frac{z_k \overline z_j}{(1 + |z|^2)^2}.$$
    It is easy to check that this matrix is positive-semidefinite, so $(O(1), h)$ is a positive line bundle.
\end{example}

\section{Integration on Hermitian line bundles}
Let $h$ be the Hermitian line bundle defined by $h_\alpha = e^{-\varphi_\alpha}$ for some strictly plurisubharmonic function $\varphi_\alpha$.
\begin{definition}
The \dfn{Levi form} of $h$ is given by
$$\omega_\alpha = \frac{1}{2i}\partial\dbar \partial_\alpha.$$
\end{definition}
The Levi form is a $(1, 1)$-form, given by
$$\omega_\alpha = \frac{1}{2i} \sum_{j,k} \partial_j\dbar_k \varphi_\alpha dx_j \wedge d\overline x_k.$$
Then
$\varphi_\alpha = 2\log|g_{\alpha\beta}| + \varphi_\beta$ for transition functions $\varphi$. It follows that $\omega_\alpha = \omega_\beta$ on overlapping patches so $\omega$ is a real $2$-form.

\begin{example}
    Let $h$ be the Fubini-Study metric. Then the Levi form of $h$ is
\begin{align*}\omega_\ell &= \frac{i}{2} \sum_j \frac{dx_j \wedge d\overline x_j}{1 + |x|^2} - \sum_{\substack{j \neq \ell\\k\neq \ell}} \frac{\overline x_j x_k ~dx_j \wedge d\overline x_k}{(1+|x|^2)^2}\\
    &= \frac{i}{2}\left(\frac{d(x~d\overline x)}{(1 + |x|^2)} - \frac{(\overline x ~dx)\wedge(x ~d\overline x)}{(1 + |x|^2)^2}\right).
\end{align*}
\end{example}

Given a Levi form $\omega$ on an $n$-dimensional manifold, we obtain a volume form $dV = \omega^n/n!$ (where $\omega^n$ is the exterior power).
\begin{example}
    The Fubini-Study Levi form is invariant under the unitary group. We can use this diagonalize the Fubini-Study metric so that the double sum is only a single sum, and prove that
    $$dV = \frac{i^n}{2^n} \frac{dx_1 \wedge d\overline x_1 \wedge \dots \wedge dx_n \wedge d\overline x_n}{(1+|x|^2)^{n+1}}.$$
\end{example}
Because we have a volume form, integration makes sense.
\begin{definition}
    The space $L^2(M, E, h)$ is the completion of the space of $s \in C^\infty(M, E)$ for which the norm defined by the inner product
    $$\langle s, s'\rangle_h = \int_M \langle s(x), s'(x)\rangle_h ~dV.$$
\end{definition}
Here the inner product in the integral is the inner product induced on each line appearing in the line bundle induced by the Hermitian metric $h$.

We thus introduce the Bergman projection $L^2(M, E) \to H^0(M, E)$, which is orthogonal. To construct it, we compute an orthonormal basis of $H^0(M, E)$.
\begin{example}
    For the Fubini-Study metric on $O(k)$, $O$ the Picard group, consider the basis $s_\alpha^k$ such that on the patch $\Omega_0$, $s_\alpha^k(x) = x_1^{\alpha_1}\dots x_n^{\alpha_n}$, $|\alpha| = k$. Then
\begin{align*}
    \langle s_\alpha, s_\beta\rangle_{FS} &= \int_{\PP^n} x^\alpha \overline x^\beta h_{FS,0}(x) ~dV_{FS} \\&= \frac{i}{2} \int_{\CC^n} \frac{x_1^{\alpha_1} \cdots x_n^{\alpha_n} \overline x_1^{\beta_1} \cdots \overline x_n^{\beta_n}}{(1+|x|^2)^{k+n+1}} ~dx_1\wedge d\overline x_1\wedge \cdots \wedge dx_n \wedge d\overline x_n\\
    &= \delta_{\alpha\beta} \pi^n \frac{\alpha!}{(n+k)!}
\end{align*}
    where we used polar coordinates. Therefore we normalize by dividing $s^k_{\alpha,0}$ by $\sqrt{(k+n)!/(\alpha!\pi^n)}$ to get an orthonormal basis. The Bergman projection is given by
    $$\Pi_ku(x) = \sum_{|\alpha| = k} s_\alpha^k(x) \langle u, s_\alpha^k\rangle_k.$$
    The \dfn{Bergman kernel} $B_k$ is defined by
    $$\Pi_ku(x) = \int_{\PP^n} \langle u(y), B_k(y, x)\rangle_h ~dV.$$
    So we have
    $$B_k(y, x) = \sum_\alpha s_\alpha^k(y) \overline{s_\alpha^k(x)}.$$
    Therefore $B_k$ is holomorphic in the first variable and antiholomorphic in the second variable. We conclude that $B_k$ can be given by the explicit formula
    $$B_k(x, y) = \sum_{|\alpha| = k } \frac{(k+n)!}{\alpha!\pi^n} z^\alpha \overline w^\alpha = \frac{(k+n)!}{k!\pi^n} \langle z, w\rangle^k$$
    where $x = [z]$, $y = [w]$. In particular, the Bergman kernel is given by the Hermitian metric:
    $$B_k(x, x) = e^{-2k\varphi_0(x)} = \frac{(k+n)!}{k!\pi^n}.$$
\end{example}

\section{Asymptotics for the Bergman kernel}
    Here we let, for $L$ a line bundle, $L^k$ denote the $k$th tensor power of $L$ with itself.
\begin{theorem}[Fefferman, Boutet, de Monvel, Sjostrand, Zelditch, Catlin, Berman, Bernsom]
    Let $L \to M$ be a positive Hermitian line bundle and let $s_j^k$ be an orthonormal basis of $H^0(M, L^k)$. Then there is an asymptotic expansion
    $$\sum_j |s_j^k(x)|^2_h = \frac{k^n}{\pi^n} + \sum_{\ell=1}^\infty a_\ell(x)k^{n-\ell}.$$
\end{theorem}
    Towards this theorem, we work locally. We identify a coordinate patch with $B_{\CC^n}(0, 1)$ and fix a strictly plurisubharmonic weight $\varphi$ on $B_{\CC^n}(0, 1)$ and a semiclassical parameter $h \in (0, 1)$. This gives a $(1, 1)$-form
    $$\omega = -i\partial\dbar\varphi = \sum_{jk} \partial_j \dbar_k \varphi ~dx_j \wedge d\overline x_k.$$
    We then have a volume form defined by $n! ~dV = \omega^k$, the right hand side being an exterior power. Then
    $$||u||_\varphi^2 = \int_{B_{\CC^n}(0, 1)} |u(x)|^2 e^{-2\varphi(x)/h} ~dV(x)$$
    is a $L^2$-norm and we denote its inner product by $(\cdot,\cdot)_\varphi$. We will obtain a local reproducing kernel $K$ given some $\varepsilon > 0$ and $\chi$ a $B_{\CC^n}(0, \varepsilon)$-cutoff function, i.e.
    $$u(x) = \int_{B_{\CC^n}(0, 1)} \chi(y)u(y) \overline{K(y, x)} e^{-2\varphi(y)/h} ~dV(y).$$
\begin{definition}
    The reproducing kernel $K$ is called a \dfn{Bergman kernel} if $K(x, \cdot)$ is holomorphic.
\end{definition}
\begin{example}
    Reproducing kernels also have applications in machine learning.
\end{example}
    To construct the Bergman kernel, we obtain an approximate reproducing kernel, which is correct up to an error of order $h^\infty$.

    Recall that for $A: L^2_\varphi \to L^2_\varphi$ an integral operator, $e^-\varphi A e^\varphi$ sends $L^2$ to itself an its integral kernel is given by $e^{-\varphi(x)/h} K_A(x, y) e^{-\varphi(y)/h}$.

\begin{definition}
    Let $\Lambda \subseteq B_{\CC^n}(0, 1) \times \CC^n$, $\Lambda = \{(y, \xi)\}$ where $\xi$ is a function of $(x, y)$. Then $\Lambda$ is a \dfn{good contour} if
    $$\Im \langle \xi, x - y\rangle \geq \delta|x - y|^2 + \varphi(y) - \varphi(x).$$
\end{definition}
\begin{example}
    If $\varphi(x) = |x|^2/2$ then
    $$\overline{K(y, x)} = (\pi h)^{-n} e^{\langle x, \overline y\rangle/h}$$
    and this Bergman kernel is valid on $\CC^n$. Here $\omega = i/2\sum_j dz_j \wedge d\overline z_j$ and $dV$ is the Euclidean volume form. We obtain a good contour by $\xi(x, y) = i\overline y$. Then
    $$2\Im \langle i\overline y, x - y\rangle = |y|^2 + |x|^2 + |x - y|^2.$$
\end{example}

    Taylor expanding,
    $$\varphi(x) = \varphi(y) - \Im Q(x, y)(x - y) + (\partial\dbar\varphi(x)(x-y))\overline{x - y} + O(|x - y|^3)$$
    where $Q$ is some holomorphic function and
    $$2\delta |x - y|^2 \leq (\partial\dbar\varphi(x)(x-y))\overline{x-y}.$$
    We now define a good contour by $\xi = Q$.
\begin{lemma}
    Suppose $\Lambda$ is a good contour for $\varphi$. Then there is a $\varepsilon > 0$ and a $B(0, \varepsilon)$-cutoff $\chi$ such that for every $u \in A(B(0, 1))$,
    $$u(x) = i^{-n^2}(2\pi h)^{-n} \int_\Lambda e^{i\langle \xi, x - y\rangle/h} u(y) \chi(y) ~d\xi \wedge dy + O(e^{(\varphi(x) - \delta)/h})||u||_\varphi.$$
\end{lemma}
    This lemma can be intuited by deforming $\Lambda$ into a flat space and then applying the Fourier inversion formula. That is not a proof, however. We put $c_n = i^{-n^2}$ for simplicity.
\begin{proof}
    Define the form
    $$\eta = c_n(2\pi h)^{-n} e^{i\langle \xi, x - y\rangle/h} u(y)\chi(y) ~d\xi \wedge dy.$$
    Define $$I_s = \int_{\Lambda_s} \eta$$
    where $\Lambda_s$ is the form defined by $\xi_s(x, y) = \xi(x, y) + i\overline{s(x - y)}$.
\begin{lemma}
    As $s \to \infty$, $I_s \to u(x)$ for $x \in B(0, 1/2)$.
\end{lemma}
\begin{proof}
    We have the expression
    $$e^{i\langle\xi, x - y\rangle/h} e^{-s|x - y|^2/h} u(y) \chi(y)(c_n ~d_{\overline y} \xi(x, y) \wedge dy + s^n~dm(y)).$$
    By the dominated convergence theorem $e^{-s|x-y|^2/h}c_n~d_{\overline y}\xi(x, y) \wedge dy \to 0$ as $s \to \infty$. As a distribution,
    $$s^n(2\pi h)^{-n} e^{-s|x - y|^2/h} \to \delta_0(x - y).$$
\end{proof}
Now $\Omega = \CC^n \times [0, s]$ and $\partial \Omega = \CC^n \times s - \CC^n \times 0$ (as a chain). We introduce the homotopy $h(y, t) = \langle y, \xi\rangle + it\overline{x - y}$, so $h: \CC^n \to [0, \infty) \to \CC^{2n}$ and put $\omega = h^*\eta$. Since $\omega$ is compactly supported (since $\eta$ has a factor of $\chi$) we can apply Stokes' theorem to see that
    $$\int_\Omega d\omega = \int_{\partial \Omega} \omega.$$
    Since $\eta$ has a factor of $d\xi \wedge dy$, $\partial \eta = 0$. So
    $$d\eta = \dbar \eta = c_n(2\pi h)^{-n} e^{-\langle \xi, x-y\rangle/h} u(y) \dbar \chi \wedge dy \wedge d\xi.$$
    Then
$$
    I_s - I_0 = \int_{\CC^n \times s} h^*\eta = \int_{\CC^n \times 0} h^*\eta = \int_{\CC^n \times [0, s]} h^*~d\eta
$$
    Since $I_s \to u(x)$ we can estimate $d\eta$. If $|y| < 1/2$ then $d\chi = 0$, and
    $$\Re (i\langle x - y, \xi\rangle/h) \leq -\delta|x-y|^2 + \varphi(x) - \varphi(y)$$
    since $\Lambda$ is good. Since $|x| < \varepsilon$ and we can take $\varepsilon < 1/4$ we only need to consider $|y| > 1/2$, in which case
    $$\Re(i\langle x - y, \xi\rangle)/h \leq -\delta/C - \varphi(y) + \varphi(x).$$
    Now
\begin{align*}
    \left|\int_{\CC^n \times [0, s]} h^*(d\eta)\right| &\leq Ch^{-n} e^{-\delta/16h + \varphi(x)/h} \int_{|y| \geq 1/2} |\chi'(y)| |u(y)|e^{-\varphi(y)/h} ~dm(y) \int_0^s (1 + t^n)e^{-t/16h} ~dt\\
    &\leq Ce^{-\delta/32h + \varphi(x)/h}||u||_\varphi.
\end{align*}
    We conclude that
    $$I_s - I_0 = O(e^{-\delta/h}e^{\varphi(x)/h})||u||_\varphi.$$
    The right hand side does not depend on $s$ so we take $s \to \infty$ to see that we have constructed an approximate kernel.
\end{proof}
    Unfortunately the approximate kernel we constructed here may not be holomorphic. So we consider for which $a$ do we have
    $$u(x) = c_n(2\pi h)^{-n} \int_\Lambda e^{i\langle \xi, x-y\rangle} u(y)(1+a)\chi(y) ~d\xi \wedge dy + O(\varphi(x)/h - \delta/h)||u||_\varphi.$$
    If
    $$ae^{i\langle \xi, x - y\rangle/h} ~d\xi = d\xi(e^{i\langle \xi ,x -y\rangle}A)$$
    where $A = \sum_j A_j(x, y, \xi, h) ~d\hat \xi_j$ and $d\hat \xi_j$ is the unique $(n-1,0)$-form such that $\xi_j \wedge d\hat \xi_j = d\xi$ then
\begin{align*}
    \int_\Lambda e^{i\langle \xi, x-y\rangle/h} u(y) \chi(y) a(x, y, \xi, h) ~d\xi\wedge dy &= \int_\Lambda u(y)\chi(y) d(e^{i\langle \xi, x-y\rangle/h}A) \wedge dy\\
    &= \int_\Lambda \chi(y)d(u(y)e^{i\langle \xi, x-y\rangle/h}A) \wedge dy\\
    &= -\int_\Lambda d\chi \wedge u(y)e^{i\langle \xi,x-y\rangle/h}A \wedge dy\\
    &= O(e^{-\delta/h}e^{\varphi(x)/h})||u||_\varphi
\end{align*}
    using that $\Lambda$ is a good contour. Intuitively we are replacing a pseudodifferential symbol that depends on $(x, y, \xi)$ with a pseudodifferential operator that depends on $(x, \xi)$.
\begin{definition}
    A symbol $a(x, y, \xi, h)$ is a \dfn{negligible symbol} if $a = \sum_j \D_{\xi_j}A_j + (x-y)A/h$ for some $A$.
\end{definition}
    We introduce the differential operator $\tilde \nabla = \partial_\xi + i(x - y)/h$. Then $ia = \tilde \nabla A$ and we absorb the $i$ into the $A$.

    In this case the error term introduced by adding $a$ is $O(e^{-\delta/h}e^{\varphi(x)/h})||u||_\varphi$. We are specifically interested in when $A$ has an asymptotic sum $A = \sum_m A_m h^m$. What this means is that for every $N$,
    $$A - \sum_{m\leq N} A_m h^m = O(h^{N+1})$$
    or in other words the sum ``converges" modulo $O(h^\infty)$. We write $A^{(N)}$ for the $N$th partial sum. Then $a^{(N)} = \tilde \nabla A^{(N)} + O(h^{N+1})$.

    We now introduce the pseudodifferential operator
    $$S = e^{ih\langle D_y, D_\xi\rangle} = \sum_{m=0}^\infty \frac{(ih)^m}{m!} D_y \cdot D_\xi.$$
    A priori this makes no sense, but it does make sense modulo $O(h^\infty)$. In fact if $a = \sum_m a_m h^m$, $b = \sum_m b_m h^m$ we say $Sa = b$ if and only if
    $$b_j = \sum_{m+p = j} \frac{i^m}{m!}(D_\xi \cdot D_y)^m a_p$$
    where the dot product of differential operators is defined by $D_\xi \cdot D_y = \sum_j D_{\xi_j}D_{y_j}$. Moreover $S^{-1} = e^{-ihD_y\cdot D_\xi}$. Then $S^{-1}Sa = a$ up to $O(h^\infty)$.
\begin{lemma}
    $a$ is a negligible symbol if and only if $Sa|_{x = y} = 0$.
\end{lemma}
\begin{proof}
    $S\partial_\xi = \partial_\xi S$ and, thinking of $y_j$ as the multiplication operator, $Sy_j = y_jS + hD_{\xi_j}S$. Therefore
    $$S((x-y)A) = (x-y)SA - hD_{\xi_j}SA.$$
    If $a$ is negligible, say $a = \tilde \nabla A$, then
    $$Sa = S\tilde \nabla A = \partial_{\xi_j}SA + i(x-y)SA/h - \partial_{\xi_0}SA = i(x-y)SA/h$$
    which restricts to $0$ on $x=y$. Conversely, if $Sa|_{x=y} = 0$ we can write $Sa = (x-y)B$. Now $S\tilde \nabla = i(x-y)S/h$ so
    $$\tilde \nabla S^{-1} = iS^{-1}(x-y)/h.$$
    Therefore
    $$\tilde \nabla S^{-1}B = iS^{-1}(x-y)B/h.$$
    So $ia/h = \tilde \nabla(S^{-1}B)$.
\end{proof}
\begin{definition}
    Let $Y \subset \CC^N$ be a $\RR$-linear subspace. Then $Y$ is a \dfn{totally real space} if $Y \cap iY = 0$.
\end{definition}
    Assume $\varphi$ is analytic as a function on $\RR^n \times \RR^n$. We extend $\varphi$ to a function on the maximal totally real subspace $\{(x, \overline x) \in \CC^n \times \CC^n\}$. To do this, let $\iota(x, y) = (x + y, -i(x - y))$. Then let $\iota^*\tilde \varphi = \varphi$. We then can find a $\psi$ which is holomorphic near $(0, 0)$ such that $\psi(z, \overline z) = \varphi(z)$. Then $\psi(x, \overline y) = \psi(y, \overline x)$ with
    $$2 \Re \psi(x, \overline y) - \varphi(x) - \varphi(y) \leq -\delta|x - y|^2.$$
    We solve for $\xi$ in the equation
    $$i\langle \xi, x - y\rangle = 2(\psi(x, z) - \psi(y, z)).$$
    This is possible by the lemma on negligible symbols and $\xi$ is a function of $(x, y, z)$. By Taylor's theorem,
    $$\xi(x, y, z) = -2i\int_0^1 \partial_x \psi(tx + (1-t)y, z) ~dt$$
    so $\xi(x, x, z) = -2i\partial_x \psi(x, z)$. In fact $(x, y, z) \mapsto (x, y, \xi)$ is a biholomorphic function close to $0$ on $\CC^{3n}$ by the analytic implicit function theorem. To see this we must show that the quadratic form given by the matrix $\partial_z \xi(0)$ is nondegenerate. Now
    $$\partial_z \xi(0) = -2i \partial_x\partial_z \psi(0) = -2i \partial_x\dbar_x \varphi(0)$$
    which is nondegenerate since $\varphi$ is strictly plush. By our previous estimate on $\psi$, the contour
    $$\Lambda = \{(y, \xi): \xi = \xi(x, y, \overline y)\}$$
    is good.
\begin{lemma}
    We have
    $$u(x) = (\pi h)^{-n} \int_{B(0, 1)} \chi(y) e^{i(2\psi(x, \overline y) - 2\psi(y, \overline y))/h} \frac{i}{2} \det \dbar_y \xi (1 + a^{(N)})u(y) ~d\overline y \wedge dy + O(e^{\varphi(x)/h}h^{N+1})||u||_\varphi.$$
\end{lemma}

\begin{definition}
    The \dfn{Bergman function} is defined on a compact complex manifold equipped with a positive line bundle by
    $$B(x) = ||K(x, x)|| = K(x, x)e^{-2\varphi(x)}.$$
\end{definition}
\begin{lemma}
    We have
    $$B(x) = \sup_{\substack{s \in H^0(M \to L)\\||s||_{L^2} \leq 1}} ||s(x)||^2.$$
\end{lemma}
\begin{proof}
    We have
    $$K(x, y) = \sum_\alpha u_\alpha(x) \overline{u_\alpha(y)},$$
    the sum ranging over an orthonormal basis $u_\alpha$ of $H^0(M \to L)$. It follows that
    $$B(x) = \sum_\alpha ||u_\alpha(x)||^2$$
    and now we use the Cauchy-Schwarz inequality.
\end{proof}
    From this we can easily compute
    $$\int_M B ~dV = \dim H^0(M \to L).$$

    Given $L \to M$ fixed, we let $B_k$ denote the Bergman function determined by $L^k$.
\begin{lemma}
    The Bergman function satisfies
    $$|B_k(x)| \leq Ck^n.$$
\end{lemma}
    Therefore $\dim H^0(M \to L^k)$ is finite and grows like $k^n$. Morally this lemma is a restatement of the uncertainty principle because it describes how many independent states can be on a compact set, with $h = 1/k$. The uncertainty principle says we cannot localize past $\sqrt{1/h} = \sqrt k$. Here we cannot localize too much because we only have so many holomorphic sections.
\begin{proof}
    Let $s \in H^0(M \to L^k)$. We want to show
    $$||s(x)||^2 \leq Ck^n ||s||_{L^2_k}.$$
    Then the claim follows from the previous lemma. Since $||s||_{L^2_k} \leq ||s||_{L^2_k(\Omega_j)}$ we might as well prove the claim locally.

    Choose an atlas so $\theta_j(x_0) = 0$ and write $s(x) = (x, u(x))$ where $u$ is holomorphic near $x_0 = 0$. Now $\varphi_j$ is strictly plurisubharmonic so up to a linear change of variables
    $$\varphi_j(x) = \sum_{j=1}^n \lambda_j|x_j|^2 + \Re Q(x) + \varphi(0) + O(|x|^3)$$
    where $Q(x) = \langle Ax, x\rangle + \langle a, x\rangle$ is holomorphic. Now $\varphi(0)$ does not affect our computation so we might as well take $\varphi(0) = 0$. By absorbing the holomorphic term $Q$ into $u$ we may assume $Q = 0$. Close to $0$ we may assume $O(|x|^3) = 0$. Now
    $$\frac{||u(0)||^2}{||u|_{L^2_k(\Omega_j)}} \leq \frac{||u(0)||^2}{||u||_{L^2_k(B(0, R_k))}}$$
    and we now make the change of variables
    $$f(w) = u(w/\sqrt k)$$
    and
    $$k\varphi(w) = \varphi_0(w) = \frac{1}{2}\sum_j \lambda_j |w_j|^2 + O(|w|^3k^{-1/2}).$$
    By our assumptions on $\varphi$, $u(0) = f(0)$. Here $dm$ and $dV$ are absolutely continuous to each other so we can replace $dV$ with $dm$ up to a constant error. Hence
\begin{align*}\frac{||u(0)||^2}{||u||_{L^2_k(\Omega_j)}} &\leq \frac{Ck^n||f(0)||^2}{\int_{|w| \leq B_k(\sqrt k)} |f(w)|^2 e^{-2\varphi_0(w)}  ~dm(w)} \\
    &\leq \frac{Ck^n||f(0)||^2}{\int_{|w| \leq B_k(\sqrt k)} |f(w)|^2 e^{-\sum_j \lambda_j|w_j|^2} ~dm(w)}.
\end{align*}
    Now $|f|^2$ is strictly plush so by the mean value theorem for measures,
$$\int_{|w| \leq B_k(\sqrt k)} e^{-\sum_j \lambda_j |w_j|^2} |f(0)|^2 ~dm(w) \leq \int_{|w| \leq B_k(\sqrt k)} |f(w)|^2 e^{-\sum_j \lambda_j |w_j|^2} ~dm(w).$$
    Pulling out $|f(0)|^2$ and taking $k$ big enough that the integral converges we see our lemma.
\end{proof}
    Actually Mike Christ showed that
    $$|K(x, y)| \leq Ce^{-\sqrt{k \log k}}d(x, y).$$
    But we have only proved $|K(x, y)| = O(k^{-\infty})$.
\begin{theorem}
    Assume $\varphi$ is real analytic. If $d(x, y) < \varepsilon$ then in a suitable trivialization,
    $$\partial^\alpha_{x,y}(K(x, y) - K^{(N)}(x, y)) = O(k^{-N-1+n+|\alpha|}e^{k(\varphi(x)+k\varphi(y)})||u||_k.$$
\end{theorem}
    To prove the theorem we rephrase the Hormander $L^2$-estimates to work for manifolds.
\begin{lemma}
    For every $(0, 1)$-form $f \in C^\infty_{0,1}(M \to L^k)$ such that $\dbar f = 0$ there is a $u \in C^\infty(M \to L^k)$ such that $\dbar u = f$ and $||u||_{L^2_k} \leq C||f||_{L^2_k}$, and $C$ can be chosen independently of $k$.
\end{lemma}
\begin{lemma}
    One has
    $$K(y, x) = (xK_x, K_y^{(N)})_{L^2_k} + O(k^{-N-1+n})e^{k\varphi(x) + k\varphi(y)}||K_x||_{L^2_k}.$$
\end{lemma}
\begin{proof}
    We use the fact that
    $$u(y) = (\chi_yu, K^N(\cdot, y))_{L^2_k} + O(k^{-N-1+n}e^{k\varphi(y)})||u||_\varphi$$
    on the function $u(y) = K_x(y)$. Then
    $$K(y, x) = (\chi K_x, K_y^{(N)})_{L^2_k} + O(k^{N-1})e^{k\varphi(y)}||u||_{L^2_k}.$$
    Here
    $$||K_x||_{L^2_k}^2 = B_k(x)e^{2k\varphi(x)}$$
    so
    $$||K_x||_{L^2_k} \leq Ck^{n/2}e^{k\varphi(x)}.$$
    Now use
    $$e^{k\varphi(y)}||u||_{L^2_k}^2 \leq e^{k\varphi(x) + k\varphi(y)} k^{n/2}$$
    to prove the lemma.
\end{proof}
    Recall also that if $P: H \to \ker A$ is an orthogonal projection and $Pu = v$, $w = u - Pu$, then $A(u - Pu) = Au$ and
    $$||w|| = \min_{A\tilde w = Au} ||\tilde w||.$$
\begin{proof}[Proof of theorem]
    We may assume that $\chi$ is real-valued. Then
    $$(\chi K_x, K_y^{(N)})_{L^2_k} = (K_x, \chi K_y^{(N)})_{L^2_k} = \overline{P_k(\chi K_x^{(N)})(y)}.$$
    Then we define
    $$u_y(x) = K_y^{(N)}(x) - (\chi K_y^{(N)}, K_x)_{L^2_k} = K_y^{(N)}(x) - P_k(\chi K_y^{(N)})(x).$$
    By the linear algebra above, $u_y$ is the $L^2_k$-minimal solution to the problem
    $$\dbar u_y = \dbar \chi K_y^{(N)} + \chi \dbar K_y^{(N)} = \dbar \chi K_y^{(N)}$$
    because $\varphi$ was assumed analytic. Moreover, $\dbar \chi = 0$ away from the diagonal. So
    $$\dbar u_y(x) = O(e^{-\delta k}e^{k\varphi(x) + k\varphi(y)}).$$
    Since $u_y$ is the minimal solution, $||u_y||_{L^2_k} \leq ||\tilde u||_{L^2_k}$ where $\tilde u$ is the solution to the $\dbar$-problem yielded by the Hormander estimate. Therefore the Hormander estimate gives
    $$||u_y||_{L^2_k} \leq Ce^{-\delta k}e^{k\varphi(y)}.$$
    By the Cauchy-Green formula we have, for any compactly supported $\psi$ which is identically $1$ near $0$,
    $$|u(0)| \leq \sup_D e^{k\varphi(y)} O(e^{-\delta/k} e^{k\varphi(y)})$$
    for any open neighborhood $D$ of $0$. If we take $D \subset B(0, k^{-1})$ and translate appropriately we have
    $$|u_y(x)| \leq e^{-\delta k}e^{k\varphi(y)} e^{k\varphi(x)}.$$
\end{proof}
    Therefore we can actually approximate the Bergman kernel by its asymptotic expansion. We use this theorem to find an asymptotic expansion for the Bergman function.
\begin{corollary}
    One has
    $$B_k(x) = \left(\frac{k}{\pi}\right)^n(1 + k^{-1}b_1(x, \overline x) + k^{-2}b_2(x, \overline x) + \dots)$$
    where the sum is meant in the asymptotic sense.
\end{corollary}
\begin{corollary}
    One has
    $$\dim H^0(M \to L^k) = \frac{k^n}{\pi^n}(1 + O(k^{-1})) \int_M ~dV.$$
\end{corollary}
    All that remains is to extend from the real-analytic case to the $C^\infty$ case. To do this we use a technique introduced by Hormander and Nirenberg.
\begin{lemma}
    Suppose $f \in C^\infty_{comp}(\RR^m)$. Then there is a $\tilde f \in C^\infty(\CC^m)$ such that $\tilde f|_{\RR^m} =f$ and $\dbar \tilde f(z) = O(|\Im z|^\infty)$.
\end{lemma}
\begin{proof}
    By a Paley-Weiner type theorem, we can define
    $$\tilde f(x + iy) = \frac{1}{(2\pi)^m} \int_{\RR^m} e^{ix\xi} \chi(|\xi||y|)e^{-y\xi} \hat f(\xi) ~d\xi$$
    for some compactly supported function $\chi$ which is identically $1$ at $0$. Obviously $\tilde f$ is smooth, and $\dbar$ only falls on the $\chi$ term. Then we end up with $\chi'$, which is not supported at $0$, and a factor of $|\xi|y_j|y|^{-1}$. We can then add factors of $|\xi||y|^{-1}$ with impunity since $\chi^{(N)}$ is $0$ close to $0$. Iterating we get the decay we need.
\end{proof}
\begin{definition}
    The function $\tilde f$ is called the \dfn{almost analytic extension} of $f$.
\end{definition}
    Using the theory of almost analytic extensions, one can prove the following lemma.
\begin{lemma}
    Assume $\varphi$ is smooth. There is a function $\psi$ which is $C^\infty$ in a neighborhood of $0 \in \CC^{n + n}$ such that $\psi(x, \overline x) = \varphi(x)$ and $\dbar_x\psi(x, \overline y)$ and $\partial_y\psi(x, \overline y) = O(|x - y|^\infty)$.
\end{lemma}
    Now notice that we can replace any assumption that $\dbar \varphi = 0$ with $\dbar \varphi$ with the property in the above lemma anywhere in the above construction of the approximate Bergman kernel, since the function was only an approximation away from the diagonal anyways. This completes the proof of the theorem of Fefferman, Boutet, de Monvel, etc.

\section{Morphisms into projective space}
    Throughout this section, let $M$ be a compact complex manifold, $L \to M$ a line bundle. Let $\omega_{FS}$ be the Fubini-Study metric on $O(1)$, the dual of the tautological line bundle, of $\PP^n$ for some $n$. We let $(s_j^k)_j$ be an orthonormal basis of $H^0(M \to L^k)$. Define the morphism
\begin{align*}
    \varphi_k: M &\to \PP^{d_k - 1}\\
    x &\mapsto [s_0^k(x), \dots, s_{d_k}^k(x)].
\end{align*}

\begin{theorem}[Catlin-Tian-Yau-Zelditch asymptotics]
    \index{Catlin-Tian-Yau Zelditch asymptotics}
    Let $\omega$ be a positive Hermitian metric on $L$. Then there is an asymptotic expansion
    $$\omega = \frac{1}{k}\varphi^*_k\omega_{FS} + \frac{\omega_2}{k^2} + \frac{\omega_3}{k^3} + \dots$$
    for some $\omega_j$ which are $(1, 1)$-forms.
\end{theorem}
\begin{proof}
    Write $s_j^k(z) = f_j^k(z)e_k$ for some holomorphic functions $f_j^k$. Then
    $$\varphi_k^*\omega_{FS} = i\partial \dbar\left(\log \sum_{j=1}^{d_k} |f_j^k(z)|^2 \right).$$
    On the other hand, the Bergman projector is given by
    $$B_k(z) = e^{-2k\varphi(z)} \sum_{j=1}^{d_k} |f_j^k(z)|^2.$$
    Therefore
    $$i\partial\dbar \log B_k(z) = -k \omega + \varphi_k^* \omega_{FS}.$$
    So
    \begin{align*}
        \omega = \frac{1}{k} \varphi_k^*\omega_{FS} + \frac{1}{ik} \partial \dbar \log B_k(z)\\
            &= \frac{1}{k} \varphi_k^*\omega_{FS} + \frac{1}{ik} \partial \dbar \frac{a_1(x)}{k} + \dots
    \end{align*}
    as $k \to \infty$.
\end{proof}

\begin{definition}
    A \dfn{Hodge metric} is a positive $(1, 1)$-form $\omega$ such that $[\omega] \in H^2(M, \QQ)$.
\end{definition}
    In other words, $\omega$ is in the second rational-valued cohomology class of $M$.

\begin{theorem}[Kodiara embedding theorem]
    \index{Kodiara embedding theorem}
    The following are equivalent:
\begin{enumerate}
    \item If $n$ is large enough, then $M$ can be embedded in $\PP^n$.
    \item $M$ admits a positive Hermitian line bundle.
    \item $M$ admits a Hodge metric.
\end{enumerate}
\end{theorem}
    In the below proof we use $\varphi$ to mean the weight and $\varphi_k$ to mean the embedding, oops.
\begin{proof}
    Obviously if we have an embedding in projective space we can just pull back $(O(1), \omega_{FS})$ to $M$. The proof that positive Hermitian line bundles are equivalent to Hodge metrics uses Chern classes.

    If $M$ has a positive Hermitian line bundle, then $\varphi_k$ is an immersion $M \to \PP^{d_k}$. Suppose that $\varphi_k$ is never injective. Then there are sequences $x_k,y_k$ such that $x_k \neq y_k$ and $\varphi_k(x_k) = \varphi_k(y_k)$.

    First suppose $d(x_k, y_k)\sqrt k \to \infty$. Since $\varphi_k(x_k) = \varphi_k(y_k)$, for all $x$, the Bergman kernel $K$ has $K(x, x_k) = K(x, y_k)$. Then
$$\int_{B(x_k, r_{k/2})} |K_k(x, x_k)|^2 e^{-2k\varphi(x)} ~dV(x) \sim e^{2k\varphi(x_k)} \frac{k^n}{\pi^n}$$
yet
$$\int_{B(x_k, r_{k/2})} |K_k(x, x_k)|^2 e^{-2k\varphi(x)} ~dV(x) = \int_{B(y_k, r_{k/2})} |K_k(x, y_k)|^2 e^{-2k\varphi(x)} ~dV(x).$$
    Now $d(x_k, y_k) \gg k^{-1/2}$ so the balls can be taken to be disjoint. Then
    $$K_k(x_k, x_k) = \int_M |K_k(x, x_k)|^2 e^{-2k\varphi(x)} ~dV(x) \geq \frac{k^n}{\pi^n}(e^{2k\varphi(x_k)} + e^{2k\varphi(y_k)}).$$
    But
    $$K_k(x_k, y_k) \sim \frac{k^n}{\pi^n} e^{2k\varphi(x_k)}.$$
    Without loss of generality we can assume $\varphi(y_k) \geq \varphi(x_k)$. But
    $$e^{2k\varphi(x_k)} \gtrsim e^{2k\varphi(x_k)} + e^{2k\varphi(y_k)}$$
    which is a contradiction.

    Otherwise we can assume $d(x_k, y_k) \leq Ck^{-1/2}$. By changing coordinates we may assume that $x_k = 0$, $y_k = w_kk^{-1/2}$, $|w_k| \leq C$, and $x_k \neq y_k$. Now put
    $$f_k(t) = \frac{|K_k(0, tw_kk^{-1/2})|^2}{K_k(0, 0)K_k(tw_kk^{-1/2}, tw_kk^{-1/2})}.$$
    Then $f_k$ is smooth on $[0, 1]$ and by the Cauchy-Schwarz inequality $0 \leq f_k(t) \leq 1$. In fact $f_k(0) = 1$ and $f_k(1) = 1$. So there is a $t_k \in [0, 1]$ such that $f_k''(t_k) \geq 0$. But
\begin{align*}
    f_k(t) &= \exp(-2k\psi(0, tw_kk^{-1/2} - \varphi(0) - \varphi(tw_kk^{-1/2})))(1 + r(tw_k)k^{-1} + \dots)\\
        &= \exp(-t^2\langle Aw_k, \overline w_k\rangle + O(t^3|w_k|^3))(1 + r_k(t_wk)k^{-1} + \dots)
\end{align*}
    which implies $|w_k| = O(k^{-1/2})$.
    So
    $$f_k''(t) = -2\langle Aw_k, \overline w_k\rangle O(|w_k|^3k^{-1/}2 + |w_k|^2k^{-1}) \leq -C|w_k|^2 < 0$$
    which is a contradiction.
\end{proof}
\begin{theorem}[Chow]
    \index{Chow's theorem}
    If $M$ admits a positive Hermitian line bundle, then $M$ is a projective variety.
\end{theorem}
\begin{proof}
    A compact submanifold of $\PP^n$ is a projective variety.
\end{proof}


\chapter{Lorentzian geometry}
Lorentzian geometry is a generalization of Riemannian geometry that was motivated by physical axioms.

\section{Axioms of special relativity}
\begin{definition}
    A \dfn{reference frame} is a coordinate system for $\RR^4 = \RR \times \RR^3$. A reference frame is said to be \dfn{inertial} if the motion of a body without external influence forms a straight line in $\RR^4$. Otherwise, the reference frame is said to be \dfn{accelerated}.
\end{definition}
\begin{axiom}
    All laws of physics are invariant under change of inertial reference frame.
\end{axiom}
\begin{axiom}
    The speed of light in a vacuum is invariant under change of inertial reference frame.
\end{axiom}
    We denote the speed of light in a vacuum by $c$.

\section{Lorentz transformations}
\begin{definition}
    A \dfn{Lorentz transformation} is a smooth transformation which fixes the origin and is homotopic to the identity, which carries an inertial reference frame to an inertial reference frame.
\end{definition}
\begin{definition}
    Let $p = (t, x), q = (t', x') \in \RR^4$. The \dfn{spacetime interval} $\Delta s = [p, q]$ between $p, q$ is the distance
    $$\Delta s^2 = \Delta x^2 - c^2\Delta t^2$$
    where $\Delta t = t' - t$ and $\Delta x = x' - x$.
\end{definition}
It is not hard to check that Lorentz transformations are linear (since they preserve straight-line trajectories). Moreover, spacetime intervals are also preserved by Lorentz transformations. By the second axiom of relativity, the quadratic polynomials associated to $\Delta s$ and its Lorentz transform, say $\Delta s'$, have the same roots. So there is an $\alpha \neq 0$ such that
$$(\Delta s)^2 = \alpha (\Delta s')^2.$$
Moreover, this constant appears for any choice of $s, s'$, so by ``reciprocity", $\alpha^2 = 1$. Since Lorentz transformations are homotopic to the identity, which clearly has $\alpha > 0$, we have $\alpha = 1$. Therefore the claim holds.

\begin{example}[twin paradox]
\index{twin paradox}
Let $A, B$ be two twins born in space. They are separated at birth (spacetime $P$), and $B$ moves away from $A$ but then suddenly turns around (at spacetime $Q$) and meets $A$ again (at spacetime $R$). Then it appears that both $A$ is older than $B$ (from the point of view of $A$) and $B$ is older than $A$ (from the point of view of $B$). However, one can check that in fact $A$ is older than $B$, since $B$ had an accelerated reference frame (when he turned around at $Q$) and so has an incorrect perception of the universe. This can be checked using the spacetime interval invariance.
\end{example}

Let us consider the Lorentzian metric
$$ds^2 = dx^2 - c^2 dt^2,$$
where as usual we write $s = (t, x) \in \RR \times \RR^3 = \RR^4$ for a point in spacetime. This is a linear combination of the Riemannian metrics $dx^2$ and $dt^2$. We will also write $m$ for the indefinite quadratic form induced by $ds^2$ on the tangent bundle. On the other hand we will write $\delta$ for the positive-definite quadratic form induced by the Riemannian metric $dx^2$. Therefore $(\RR^3, \delta)$ is just Euclidean space.
\begin{definition}
    A \dfn{Lorentzian manifold} is a smooth manifold equipped with a smoothly varying quadratic form on each tangent space. The Lorentzian manifold $(\RR^4, m)$ is known as \dfn{Minkowski spacetime}.
\end{definition}

Now let $\gamma$ be a curve in $\RR^4$, which we think of as parametrized by $[0, 1]$. We denote the tangent vector by $\dot \gamma$.
\begin{definition}
The \dfn{proper time} of the curve $\gamma$ is
$$\int_0^1 \frac{\sqrt{-m(\dot \gamma(\sigma), \dot \gamma(\sigma))}}{c} ~d\sigma.$$
\end{definition}
\begin{definition}
    Let $v$ be a tangent vector over $\RR^4$. If $m(v, v) < 0$, we say that $v$ is \dfn{timelike}. If $m(v, v) = 0$, then $v$ is \dfn{lightlike} or \dfn{null}. Otherwise, $v$ is \dfn{spacelike}. If $v$ is not spacelike, we say that $v$ is \dfn{causal}. If every tangent vector to a curve is timelike (lightlike, etc.), we say that the curve itself is timelike (lightlike, etc.)
\end{definition}
Notice that a vector $v$ has speed $\leq c$ iff $v$ is causal. So causal curves are those trajectories of objects which are allowed by the laws of physics.

\section{Riemannian geometry}
We dive deeper into Lorentzian geometry. Throughout this chapter we fix a Lorentzian spacetime $(M, g)$. In other words, $(M, g)$ is locally isomorphic as a Lorentzian manifold to the Minkowski spacetime $(\RR^4, m)$. We always assume that $M$ is orientable.

Throughout these notes we use Einstein's conventions that a repeated index is summed over:
$$\omega_\mu v^\mu = \langle \omega, v\rangle.$$
We let $\{\partial_0, \dots, \partial_3\}$ be the standard basis of the tangent bundle $TM$, and $\{dx_0, \dots, dx_3\}$ be the standard basis of the cotangent bundle $T^*M$. Thus we have a pairing
$$dx^\mu \partial_\nu = \delta_\nu^\mu.$$

\begin{definition}
A $(p, q)$-\dfn{tensor} at $x \in M$ is an element of $(T_x^*M)^{\otimes p} \otimes (T_xM)^{\otimes q}$. A $(p, q)$-\dfn{tensor field} is a section of the tensor bundle $$(T^*M)^{\otimes p} \otimes (TM)^{\otimes q} \to M.$$
\end{definition}
In local coordinates, we write
$$T_{\alpha_1, \dots, \alpha_p}^{\beta_1, \dots, \beta_q}(x)$$
for the $(\alpha_1, \dots, \alpha_p; \beta_1, \dots, \beta_q)$th coordinate of a $(p, q)$-tensor field evaluated at $x$.

We now need the notion of a linear connection. A linear connection, morally, is a ``way to differentiate a vector field against another vector field." Let $\mathcal T(M)$ denote the space of vector fields $M \to TM$.
\begin{definition}
    A \dfn{linear connection} is a $(C^\infty(M), \RR)$-bilinear map $\nabla: \mathcal T(M)^2 \to \mathcal T(M)$, written $(X, Y) \mapsto \nabla_XY$ (though we write $\nabla_\alpha = \nabla_{\partial_\alpha}$) satisfying the Leibniz rule
    $$\nabla_X (f Y) = (\nabla_X f) Y + f\nabla_XY = df(X)Y + f\nabla_XY.$$
\end{definition}
Let's consider the easy example of a Euclidean connection.
\begin{definition}
    Assume $(M, g)$ is Euclidean space. The \dfn{Euclidean connection} on $M$ is the linear connection
    $$\nabla_X Y^j \partial_j = XY^j \partial_j.$$
\end{definition}
Since $X$ is a first-order derivation at each point, $XY^j$ is the partial derivative of $Y$ in the direction of $X$. Thus Euclidean connections are a very natural thing to study, and in case $X = \partial_\alpha$, $\nabla_X$ is just the map that sends $Y$ to its derivative in some direction. The Euclidean connection has the useful property that $\nabla g = 0$.
\begin{definition}
    The \dfn{Levi-Civita connection} $\nabla$ is the unique connection on $M$ such that $\nabla g = 0$ and which satisfies $\nabla_XY - \nabla_YX = [X, Y]$.
\end{definition}
\begin{theorem}
    The Levi-Civita connection is well-defined.
\end{theorem}
\begin{definition}
    The \dfn{Riemann curvature tensor} is the $(3, 1)$-tensor field
    $$R_{\alpha\beta\gamma}^\delta \partial_\delta = \nabla_\alpha \nabla_\beta \partial_\gamma - \nabla_\beta \nabla_\alpha \partial_\gamma.$$
\end{definition}
It is pretty clear that
$$R_{\alpha\beta\gamma}^\delta = -R_{\beta\alpha\gamma}^\delta,$$
and
$$R_{\alpha\beta\gamma\delta} = R_{\alpha\beta\delta\gamma}.$$
\begin{theorem}[Bianchi]
    \index{Bianchi's identities}
    One has
    $$R_{\alpha\beta\gamma\delta} + R_{\beta\gamma\alpha\delta} + R_{\gamma\beta\alpha\delta} = 0$$
    and
    $$\nabla_\alpha R_{\beta\gamma\delta\epsilon} + \nabla_\beta R_{\gamma\alpha\delta\epsilon} + \nabla_\gamma R_{\alpha\beta\delta\epsilon} = 0.$$
\end{theorem}
\begin{definition}
    The \dfn{Christoffel symbol} $\Gamma$ is defined by
    $$\nabla_\alpha \partial_\beta = \Gamma_{\alpha\beta}^\gamma \partial_\gamma.$$
\end{definition}
We have the \dfn{Kozsul formula}
$$\Gamma_{\alpha\beta}^\gamma = \frac{g^{\gamma\delta}}{2}(\partial_\alpha g_{\beta\delta} + \partial_\beta g_{\delta\alpha} - \partial_\delta g_{\alpha\beta}).$$

\section{Causality}
Let $(M, g)$ be a Lorentzian spacetime as above. Recall that we have a causal structure on the tangent bundle of $M$, which gives rise to a pair of light cones in each tangent space. Taking the exponential map, we get a causal structure on curves in $M$.

\begin{definition}
    The spacetime $(M, g)$ is \dfn{time-orientable} if there is a continuous choice of light cone for each tangent space.
\end{definition}
    Let us fix a time-orientation. The vectors in the chosen lightcone point to the ``future."
\begin{definition}
    Let $S \subseteq M$. The \dfn{chronological future} $I^+(S)$ is the set of points in $M$ that can be reached by a curve through the exponential map of the open future light cones of $S$. The chronological past is defined similarly, but with the open past light cone. The \dfn{causal future} and causal past are defined similarly, but for the closed light cones.
\end{definition}

\chapter{The Cauchy problem in general relativity}
We pose the Einstein equation as an initial-value problem on the Lorenztian manifold $(M, g)$.

\section{The Einstein equation}
If $\mathcal L$ is a Lagrangian density, then $\mathcal L$ does not have to be integrable, so long as we only take only compactly supported perturbations when we carry out the calculus of variations. That is why we emphasize that $\mathcal L$ is only a ``density" rather than a summable quantity.

Throughout, we let $\delta g_{\alpha\beta}$ be a compcatly supported perturbation of the metric tensor $g_{\alpha\beta}$.
\begin{definition}
    Let $\mathcal L$ be a Lagrangian density. The \dfn{energy-momentum tensor} associated to $\mathcal L$ is the tensor $T_{\alpha\beta}$ given by
    $$\int \frac{d\mathcal L}{ds}(\cdot, g + s\delta g) ~dV(g + s \delta g) + \int T^{\alpha\beta} \delta g_{\alpha\beta} dV = 0.$$
\end{definition}
\begin{theorem}[Noether]
    If $T_{\alpha\beta}$ is an energy-momentum tensor, then the divergence
    $$\nabla^\alpha T_{\alpha\beta} = 0.$$
\end{theorem}
Noether's theorem can be interpreted as a generalization of the conservation laws of energy, mass (which is just a form of energy by Einstein's special theory of relativity), and momentum. It is a special case of Noether's theorem that for any Lie action of $\RR$ on a Lagrangian density, there is an associated conserved quantity in the respective Euler-Lagrange equations.

A key point in general relativity is that ``curvature is energy-momentum", yet the energy-momentum tensor $T_{\alpha\beta}$ is divergence-free. So the curvature tensor appearing in general relativity should be a divergence-free covariant $2$-tensor. Thus we must define a $2$-tensor which measures curvature.
\begin{definition}
    Let $R^\alpha_{\beta\gamma\delta}$ be the Riemann curvature tensor of $(M, g)$. The \dfn{Ricci curvature tensor} of $(M, g)$ is given by
    $$\Ric_{\alpha\beta} = R^\mu_{\alpha\mu\beta}.$$
    The \dfn{scalar curvature} is given by $R = \Ric_\alpha^\alpha$.
\end{definition}
\begin{definition}
    The \dfn{Einstein tensor} of $(M, g)$ is
    $$G_{\alpha\beta} = \Ric_{\alpha\beta} - \frac{1}{2}g_{\alpha\beta} R.$$
    The \dfn{Einstein equation} is the equation
    $$G_{\alpha\beta} = \frac{8\pi G}{c^4} T_{\alpha\beta}.$$
\end{definition}
We will normalize $c = 1$, and then take $G = 1/(4\pi)$, so the Einstein equation will read as $G_{\alpha\beta} = 2T_{\alpha\beta}$. Notice the similarity to the Gauss-Poisson equation for gravity
$$4\pi G \nabla^\alpha g_\alpha = \rho$$
where $\rho$ is the mass density of the universe and $g_\alpha$ is the gravitational field. We have
$$\nabla^\alpha G_{\alpha\beta} = 0$$
by the second Bianchi identity.

We interpret $T_{\alpha\beta} = 0$ as meaning that the universe is a vacuum. In this case we have $\Ric_{\alpha\beta} = 0$. Now $\Ric_{\alpha\beta} = 0$ is a geometric PDE, but we want to frame it as an initial-value problem where the initial data consists of a $3$-manifold, and the resulting $4$-manifold comes from gluing together the $3$-manifolds together in time.

This interpretation gives another derivation of the Einstein equation. Assume $T_{\alpha\beta} = 0$; then the universe should have no curvature.
\begin{definition}
    The \dfn{Einstein-Hilbert Lagrangian density} is $\mathcal L_{EH} = R ~dV$.
\end{definition}
The Einstein equation should be the Euler-Lagrange equation minimizing the Einstein-Hilbert action.
\begin{theorem}
    The Euler-Lagrange equation corresponding to the Einstein-Hilbert Lagrangian density is the Einstein equation.
\end{theorem}
\begin{proof}
    Let $\delta g$ be a compactly supported perturbation of hte metric tensor as above. Then
    $$\frac{\delta}{\delta s} (g + s \delta g)^{\mu\nu} = -\delta g^{\mu\nu}.$$
    Similarly
    $$\frac{\delta}{\delta g} dV(g) = \delta g^{\alpha\beta} ~dV(g).$$
    In coordinates, we have
    $$\Ric_{\beta\nu} = \partial_\alpha \Gamma^\alpha_{\beta\nu} - \partial_\beta \Gamma^\alpha_{\alpha\nu} + \Gamma^\mu_{\alpha\gamma} \Gamma^\gamma_{\mu\nu} + \Gamma_{\beta\gamma}^\alpha\Gamma_{\alpha\nu}^\gamma.$$
    After a tedious computation in normal coordinates (where $g = m$ and $\Gamma = 0$ at the origin) we have
    $$\frac{\delta}{\delta s} \Ric_{\alpha\beta} (g + s\delta g)| = \frac{1}{2}g^{\mu\nu} (\partial_\alpha\delta g_{\beta\nu} + \partial_\beta \delta g_{\nu\alpha} - \partial_\nu\delta g_{\alpha\beta}).$$
    Applying the Lebiniz rule we have
    $$\int_M \frac{\delta}{\delta s} \mathcal L_{EH}(g + s\delta g) = \frac{1}{2} \int_M g^{\alpha\beta} R \delta g_{\alpha\beta} - \Ric^{\alpha\beta} \delta g_{\alpha\beta} ~dV$$
    so the claim follows by lowering indices.
\end{proof}

\section{Initial-data sets}
Henceforth we fix a time-orientation of $(M, g)$.
\begin{definition}
A \dfn{time function} is a smooth function $t$ on $M$ such that for every future-pointing timelike vector field $X^\alpha$, $g_{\alpha\beta} \nabla^\alpha t X^\beta > 0$.
\end{definition}
We fix a time function as well.

\begin{definition}
The \dfn{initial-time slice} $\Sigma_0$ is the level set of the equation $t = 0$.
\end{definition}
Because the metric signature is $(-, +, +, +)$, the initial-time slice will be a $3$-manifold. We denote its induced Riemannian metric by $\overline g$ and induced Levi-Civita connection by $\overline \nabla$.

\begin{definition}
    Let $\Sigma$ be a Riemannian submanifold of codimension $1$. Let $n$ denote the future-pointing unit normal vector to $\Sigma$. Then the \dfn{second fundamental form} $\overline k_{\alpha\beta}$ is given by
    $$\overline k_{\alpha\beta} u^\alpha v^\beta = -g_{\alpha\beta} u^\alpha \cdot \nabla_v n^\beta.$$
\end{definition}
The second fundamental form measures how fast the unit normal vectors change as we move along unit tangent vectors; it is a measure of the extrinsic curvature of $\Sigma$ in $M$.

\begin{theorem}[Gauss-Codazzi]
    \index{Gauss-Codazzi equations}
    One has $R(\overline g)_{ijk\ell} + \overline k_{ij} \overline k_{j\ell} - \overline k_{i\ell} \overline k_{jk} = R(g)_{ijk\ell}$ and $\overline \nabla_i \overline k_{j\ell} - \overline \nabla_j \overline k_{i\ell} = C R(g)_{ik\ell0}$.
\end{theorem}
    Thus, if $(\Sigma, \overline g, \overline k)$ is to be an initial-data set, it had better satisfy the Gauss-Codazzi equations for the $4$-manifold we want to embed it into.
\begin{definition}
    Let $T_{\alpha\beta}$ be a symmetric, divergence-free $2$-tensor. An \dfn{initial-data set} $(\Sigma, \overline g, \overline k)$ corresponding to the energy-momentum tensor $T_{\alpha\beta}$ is the data of:
\begin{enumerate}
    \item A $3$-manifold $\Sigma$,
    \item a Riemannian metric $\overline g$ on $\Sigma$ with Riemann curvature tensor $\overline R$,
    \item and a symmetric $2$-tensor $\overline k$ on $\Sigma$,
\end{enumerate}
    satisfying the Gauss-Codazzi constraints
\begin{align*}
    \overline R + (\operatorname{tr} \overline k)^2 + \overline k^{ij} \overline k_{ij} &= 4 \varphi^2 T_{tt}\\
    \nabla^i \overline k_{ij} - \nabla_j \operatorname{tr} \overline k &= 2\varphi T_{jt}
\end{align*}
    where
    $$\varphi = \frac{\overline k_{ij}}{2\partial_t \overline g_{ij}}.$$
\end{definition}
\begin{definition}
    Let $(\Sigma, \overline g, \overline k)$ be an initial-data set corresponding to the energy-momentum tensor $T_{\alpha\beta}$. A \dfn{development} of $(\Sigma, \overline g, \overline k)$ is an isometric embedding $\iota: \Sigma \to M$, where $M = (M, g)$ is a Lorentzian $(1+3)$-manifold solving the Einstein equation
    $$\Ric_{\alpha\beta} - \frac{1}{2} g_{\alpha\beta} R = T_{\alpha\beta}$$
    and $\overline k$ is the second fundamental form of $\iota$.
\end{definition}
    We think of the initial-data set as being the initial conditions of the Einstein equation and $(M, g)$ as being the solution.
\begin{definition}
    Let $S \subseteq M$ be a spacelike hypersurface. Then $S$ is a {Cauchy hypersurface} if every maximal causal curve in $M$ intersects $S$ at exactly one point. Moreover, the \dfn{domain of dependence} is the maximal submanifold $D \subseteq M$ such that $S$ is a Cauchy hypersurface of $D$.
\end{definition}
    For example, the initial-time slice $\Sigma_0$ of Minkowski spacetime is a Cauchy hypersurface. In fact, if $B$ is a ball in $\Sigma_0$, then the future-pointing causal cone based at $B$ is the domain of dependence of $B$.
\begin{definition}
    Let $(\Sigma, \overline g, \overline k)$ be a respective initial-data set. Let $(M, g)$ be a Lorentzian $(1+3)$-manifold. A \dfn{globally hyperbolic development} $\iota: \Sigma \to M$ is a development of $(\Sigma, \overline g, \overline k)$ such that $\iota(\Sigma)$ is a Cauchy hypersurface of $(M, g)$.
\end{definition}

\section{Well-posedness for the vacuum equation}
Throughout this section we work with the Einstein vacuum equation $\Ric_{\alpha\beta} = 0$.

First, we recall the theorem that quasilinear wave equations are well-posed.
\begin{theorem}
    Fix a Lorentzian metric $g$ and consider the PDE
\begin{align*}
    g^{\mu\nu}(x, \varphi(x)) \partial_\mu \partial_\nu \varphi(x) &= N(x, \varphi(x), \partial \varphi(x))\\
    (\varphi, \partial_t\varphi)(x) &= (\varphi_0, \varphi_1)(x)
\end{align*}
    where $\varphi$ is an unknown. Let $s > d/2 + 1$. If $(\varphi_0, \varphi_1) \in H^s \times H^{s-1}(\Sigma_0)$, then there is an maximal eclipse time $T > 0$ and a unique solution $\varphi \in H^s([0, T] \times \Sigma_0)$.
\end{theorem}

For $T_{\alpha\beta}$ a tensor, we write
$$\hat T_{\alpha\beta} = T_{\alpha\beta} - \frac{1}{2} g_{\alpha\beta} g^{\alpha\beta} T_{\alpha\beta}.$$

\begin{theorem}[Choquet-Bruhat-Geroch]
    \index{Choquet-Bruhat-Geroch theorem}
    Let $(\Sigma, \overline g, \overline k)$ be a smooth initial-data set with $\Ric_{\alpha\beta} = 0$. Then there is a unique \dfn{maximal globally hyperbolic development} $(M, g, \iota)$ of $(\Sigma, \overline g, \overline k)$; i.e. a globally hyperbolic development such that for any globally hyperbolic development $(\hat M, \hat g, \hat \iota)$, there is an isometric embedding $\Phi: \hat M \to M$ such that the diagram
$$\begin{tikzcd}
    \hat M \arrow[rr, "\Phi"] && M\\
    &\Sigma \arrow[lu, "\hat \iota"] \arrow[ru, "\iota"]
    \end{tikzcd}$$
    commutes.
\end{theorem}
    By lower-order terms we mean those of first or zeroth order (those which may serve as quasilinear perturbations of the d'Alembertian, which is a second-order linear operator). The idea of the proof is to write $\Ric_{\alpha\beta}$ as a quasilinear wave equation and use local well-posedness to construct local solutions, then glue all the local solutions together using Zorn's lemma.
\begin{proof}
    We have
\begin{align*}
    \Ric_{\alpha\beta} &= \partial_\mu \Gamma^\mu_{\alpha\beta} - \partial_\alpha \Gamma^\mu_{\mu\beta}\\
    &= \frac{1}{2} g^{\mu\nu} \partial_\mu \partial_\nu g_{\alpha\beta} - \frac{1}{2} g^{\mu\nu} \partial_\alpha\partial_\beta g_{\mu\nu} + \frac{1}{2} g^{\mu\nu}\partial_\alpha\partial_\nu g_{\beta\nu} + \frac{1}{2}g^{\mu\nu} \partial_\beta \partial_\mu g_{\alpha\nu}\\
    &= \frac{1}{2} \partial^\nu \partial_\nu g_{\alpha\beta} + \partial_\alpha \Gamma_\beta + \partial_\beta \Gamma_\alpha
\end{align*}
where
$$\Gamma_\beta = \frac{1}{2} g^{\mu\nu} \partial_\mu g_{\beta\nu} - \frac{1}{2} \partial_\beta g_{\mu\nu} = \frac{1}{2} g^{\mu\nu} \Gamma_{\mu\nu}^\alpha g_{\alpha\beta}.$$
Let
$$S_{\alpha\beta} = \Ric_{\alpha\beta} - \partial_\alpha \Gamma_\beta - \partial_\beta \Gamma_\alpha.$$
Then the equation $S_{\alpha\beta} = 0$ is a quasilinear wave equation, so is locally well-posed, and has a solution on a submanifold $M$ of $\RR \times \Sigma$.

Recall that $\widehat \Ric_{\alpha\beta}$ is the Einstein tensor and hence
$$\nabla^\alpha \widehat \Ric_{\alpha\beta} = 0.$$
Taking the hat and divergence of both sides of the definition of $S_{\alpha\beta}$, we have
$$\nabla^\alpha (\widehat{\nabla_\alpha \Gamma_\beta + \nabla_\beta \Gamma_\alpha}) = 0.$$
But
$$\widehat{\nabla_\alpha \Gamma_\beta + \nabla_\beta \Gamma_\alpha} = \nabla_\alpha \Gamma_\beta + \nabla_\beta \Gamma_\alpha - g_{\alpha\beta} g^{\mu\nu} \nabla_\mu \Gamma_\nu$$
so
$$0 = \nabla^\alpha \nabla_\alpha - \Gamma_\beta + \nabla^\alpha \nabla_\beta \Gamma_\alpha - g^{\mu\nu} \nabla_\beta \nabla_\mu \Gamma_\nu = \nabla^\alpha \nabla_\alpha \Gamma_\beta.$$
Therefore $\Gamma$ solves the wave equation. Since the wave equation is well-posed, it suffices to show therefore that $\Gamma|_\Sigma = 0$ and $\partial_t \Gamma|_\Sigma = 0$. For $i,j$ spatial coordinates, we set $g_{ij}|_\Sigma = 0$ and $g_{tt}|_\Sigma = -1$, $g_{ti}|_\Sigma = 0$, $\partial_t g_{ij}|_\Sigma|_\Sigma = 2\overline k_{ij}$, and $\partial_t g_{t\alpha}|_\Sigma = 0$. Then $\Gamma|_\Sigma = \partial_t\Gamma|_\Sigma = 0$ by the Gauss-Codazzi equations. Thus with these initial conditions, $S_{\alpha\beta} = \Ric_{\alpha\beta}$ so the Einstein equation reduces to the quasilinear wave equation $S_{\alpha\beta} = 0$, and the solution manifold $M$ solves the vacuum Einstein equation, which is therefore locally well-posed.

Now let $\mathcal M$ be the class of globally hyperbolic developments of $(\Sigma, \overline g, \overline k)$, ordered by isometric embeddings which commute with the inclusions $\iota$. This class is proper, but taking a quotient by isometry, we arrive at a poset. Taking injective limits, we show that every chain has an upper bound, so $\mathcal M$ has a maximal element $\iota: \Sigma \to M$, the \dfn{set-theoretic maximal globally hyperbolic development}. It remains to show that $\iota$ is maximal in the sense of the definition of maximal globally hyperbolic development (and hence unique).

Let $\hat \iota: \Sigma \to \hat M$ be a set-theoretically maximal globally hyperbolic development. We must construct a isometric embedding $\Phi: \hat M \to M$ making the diagram commute. By a \dfn{partial isometric embedding} of $\hat M$ into $M$ we mean a isometric embedding $\hat U \to M$ for some open set $\hat U \subseteq \hat M$. By local well-posedness, every point is contained in a neighborhood which admits a partial isometric embedding that makes the diagram commute, and by local uniqueness, they satisfy the cohomological conditions in the definition of a sheaf. Therefore there is a global partial isometric embedding, which is of course $\Phi$.

We define the \dfn{development-theoretic union} $M \cup \hat M = M \coprod \hat M/\Phi$, where the coproduct $\coprod$ is the sense of disjoint union. All conditions in the definition of a globally hyperbolic development are easily checked for $M \cup \hat M$ except that $M \cup \hat M$ is Hausdorff.

Assume that $M \cup \hat M$ is not Hausdorff at a point $x \in M \cup \hat M$. Then $x \in \partial (M \cup \hat M)$, and by a difficult computation in Lorentzian geometry, there is a spacelike hypersurface $S$ which touches $\partial M$ exactly at $x$. Away from $x$, $S$ and $\Psi(S)$ determine the same initial-data set. But by continuity, $S$ and $\Psi(S)$ determine the same initial-data set at $x$ as well.

But $x$ lifts to a regular point in $M \coprod \hat M$ (and let us assume without loss of generality that $x$ then lifts to a regular point in $M$), so there is a globally hyperbolic development extending from a $S$-neighborhood of $x$ by local well-posedness. Since $M$ is set-theoretically maximal, $x$ does not lift to a point of $\partial M$. Therefore $x \notin \partial (M \cup \hat M)$, a contradiction.

It follows that $M \cup \hat M$ is Hausdorff, and hence a globally hyperbolic development which contains $M$. So $M \cup \hat M = M$, and it follows that $\hat M = M$. So $M$ is a globally hyperbolic development.
\end{proof}

\chapter{General relativity in spherical symmetry}
We now make a simplifying assumption to get rid of annoying obstructions in Lorentzian geometry: that of spherical symmetry.

\begin{definition}
    A spacetime $(M, g)$ is \dfn{spherically symmetric} if there is a $SO(3)$-action on $M$ by $g$-isometries such that every $SO(3)$-orbit is a manifold of dimension at most $2$.
\end{definition}
Then the only possible orbits are fixed points and spheres of positive radius. For $p \in M$ we let $S_p$ denote the orbit of $p$, and let $r(p)$ denote the radius of $S_p$, which can be intrinsically defined by
$$r(p) = \sqrt{\frac{\mu(S_p)}{4\pi}},$$
$\mu$ denoting area. Then zeroes of $r$ are fixed points of $SO(3)$.

When $r(p) > 0$, the induced metric on $S_p$ is given by
$$g|_{S_p} = r^2 \underline g$$
where $\underline g$ is the Riemannian metric of the unit $2$-sphere $S^2$.

We let $\Q = M/SO(3)$, so $\Q$ is a Lorentzian $(1+1)$-manifold with boundary $\Gamma = \partial \Q$. Then
$$g = g_\Q + r^2 \underline g.$$

\section{Double-null pairs}
\begin{definition}
    A \dfn{double-null pair} on $M$ is a pair of $SO(3)$-invariant smooth functions $u, v: M \to \RR$, increasing in time, such that
    $$g^{\alpha\beta} du_\alpha du_\beta = g^{\alpha\beta} dv_\alpha dv_\beta = 0$$
    and such that $du, dv$ are linearly independent on every cotangent space. If we view $u,v$ as coordinates on $M$ and let $\theta, \varphi$ be the usual polar coordinates on $S^2$, the tuple $(u, v, \theta, \varphi)$ is known as a system of \dfn{double-null pair coordinates}.
\end{definition}
    Assuming that $M$ has double-null pair coordinates,
    $$\underline g = d\theta^2 + \sin^2 \theta ~d\varphi^2$$
and
    $$g_\Q = - \Omega^2 du \cdot dv$$
    for some function $\Omega$.
\begin{definition}
    The function $\Omega$ determined by double-null pair coordinates is called the \dfn{null lapse} of the double-pull pair.
\end{definition}
    Let us assume that every spacetime admits a double-null pair coordinate system.

    We have $g_{uu} = g_{vv} = 0$ and
    $$g_{uv} = - \frac{1}{2} \Omega^2.$$
In particular we have
$$g = \begin{bmatrix}
&2^{-1}\Omega^2\\
2^{-1}\Omega^2\\
&&r^2\\
&&&r^2\end{bmatrix}$$
so
$2\sqrt{-\det g} = $
    We will always write the angular coordinates with capital letters.

    By reparametrizing $u,v$ to have bounded range, we can embed $\Q$ into a compact subset of the Minkowski spacetime $\RR^{1+1}$.
\begin{theorem}
    Let $(u, v, \theta, \varphi)$ be a double-null coordinate system with null lapse $\Omega$ and let $\phi$ be a spherically symmetric scalar field. Then
    $$\Box_g \phi = -4 \Omega^{-2}(\partial_u \partial_v \phi + r^{-1}\partial_u r \partial_v \phi + r^{-1} \partial_v r \partial_u \phi).$$
    Besides this, we can write the Einstein tensor $G_{\alpha\beta}$ as
\begin{align*}
    G_{uu} &= \Ric_{uu} &= -2r^{-1} \Omega^2 \partial_u (\Omega^{-2} \partial_u r)\\
    G_{vv} &= \Ric_{vv} &= -2r^{-1} \Omega^2 \partial_v (\Omega^{-2} \partial_v r)\\
    G_{uv} &= \Ric_{uv} - \frac{1}{2} g_{uv}R &= 2r^{-1} \partial_u \partial_v r + 2r^{-2} \partial_u r \partial_v r + \Omega^2r^{-2}\\
    G_{AB} &= \Ric_{AB} - \frac{1}{2}g_{AB} &= -4r^2 Omega^2\underline g_{AB}(\Omega^{-1} \partial_u \partial_v \Omega - \Omega^{-2} \partial_u \Omega \partial_v \Omega + r^{-1} \partial_u \partial_v r)
\end{align*}
    and all other entries determined by symmetry or vanishing.
\end{theorem}
From this, it is easy to see that the Einstein vacuum equation in spherical symmetry can be expressed as
\begin{align*}
    \partial_u(\Omega^{-2}\partial_u r) &= 0\\
    \partial_v(\Omega^{-2}\partial_v r) &= 0\\
    \partial_u \partial_v r + 2r^{-1} \partial_u r \partial_v r + (2r)^{1} \Omega &= 0\\
    \partial_u \partial_v \Omega - \Omega \partial_u \Omega \partial_v \Omega - \Omega r^{-2}(2\partial_u r \partial_b r + 2^{-1}\Omega^2) &= 0.
\end{align*}
The wave operator in spherical symmetry has principal part $\partial_u \partial_v$. So we view the first two Einstein equations as constraint equations (called \dfn{Raychaudhuri equations}) and the last two Einstein equations as quasilinear wave equations.

\section{Local rigidity}
We show that the Raychaudhuri equations form a strong constraint on the sort of solutions we are allowed to study.
\begin{lemma}
    Consider the quasilinear wave equation
    $$\partial_u\partial_v \Phi = N(\phi, \partial \Phi)$$
    where $N$ is $C^1$. If $\Phi$ is a $C^1$ solution to the equation with $\Phi$ prescribed on the future-pointing lightcone centered at a point $p \in M$, then $\Phi$ is unique.
\end{lemma}
Since this is a wave equation, we expect to need $\partial \Phi$ as initial data as well in order for $\Phi$ to be unique. But the lightcone consists exactly of characteristic curves of $\Phi$, one of which determines $\partial_u \Phi$ and the other determines $\partial_v \Phi$ -- and they must be compatible since $p$ touches both curves.
\begin{proof}
    Notice that
    $$\partial_u \partial_v \Psi (\partial_u + \partial_v) \Psi = .5 \partial_v(\partial_u \Psi)^2 + .5 \partial_u(\partial_v \Psi)^2.$$
    Assume that $\Phi, \Phi'$ are solutions and let $\Psi = \Phi - \Phi'$. Then
    $$\partial_u \partial_v \Psi = \Psi\partial_\Phi N(\Phi, \partial \Phi) +  O(\Psi, \partial \Psi).$$
    Integrate along a future-pointing ``diamond" whose first vertex is $p = (0, 0)$ and whose sides are given by (or are perpendicular to -- we call these $C_u$ and $C_v$) the characteristic curves $C_0, \underline C_0$). This gives
$$\frac{1}{2}\left(\int_{C_u} (\partial_v \Psi)^2 + \int_{C_v} (\partial_u \Psi)^2  - \int_{C_0} (\partial_v \Psi)^2 - \int_{\underline C_0} (\partial_u \Psi)^2 \right) \leq C \int_D |\Psi| |\partial \Psi| + C\int_D |\partial \Psi|^2.$$
    The integrals along $C_0$ and $\overline C_0$ are $0$ because $\Phi = \Phi'$ there by assumption. So we have
    $$\int_{C_u} (\partial_v \Psi)^2 + \int_{C_v} (\partial_u \Psi)^2 \leq C \int_D |\Psi| |\partial \Psi| + \int_D |\partial \Psi|^2.$$
    We use Gronwall's inequality to control the right-hand side.
    Now
    $$\Psi(u, v) = \int_0^v \partial_v \Psi(u, \cdot)$$
    which then vanishes. So this energy estimate gives $\Psi = 0$.
\end{proof}
\begin{theorem}[Birkhoff]
    \index{Birkhoff's theorem on local rigidity}
    Up to gauge symmetry, the solution to the Einstein vacuum equation in spherical symmetry near a point $p \in M$ is determined by $r(p)$, the signs $\sigma_\nu, \sigma_\lambda$ of $\partial_u r(p)$ and $\partial_v r(p)$, and $g^{\mu\nu} \partial_\mu \partial_\nu r(p)$.
\end{theorem}
\begin{proof}
    If $u,v$ are a future-pointing double-null pair, and we transform them to $(\tilde u, \tilde v)$ where $\tilde u > 0$ only depends on $u$ and similarly for $\tilde v$, then $(\tilde u, \tilde v)$ are a future-pointing double-null pair. This transformation results in the transformation of $\Omega$ by
$$-\Omega^2 ~du~dv = -\tilde \Omega^2 ~d\tilde u ~d\tilde v = -\tilde \Omega^2 \tilde u' \tilde v' ~du ~dv$$
so $\tilde \Omega^2 \tilde u' \tilde v' = \Omega^2$.
    Let $\underline c_p$, $c_p$ be the curves of the future-pointing lightcone along $u, v$ from $p$. Then we can choose $\tilde u, \tilde v$ so that $\tilde \Omega = 1$ on $\underline c_p, c_p$. Let us henceforth work in the coordinates $\tilde u, \tilde v$ (so $\Omega = \tilde \Omega$).

    By the lemma, we only need to determine $\Omega$ and $r$ along $\underline c_p, c_p$, and this will uniquely determine the solution inside any diamond with two sides that lie along $\underline c_p, c_p$; then make the diamond as big as we need.

    Assume $\sigma_\nu = \sigma_\lambda = 0$. Applying the Raychaudhuri equations along $\underline c_p, c_p$, we have $\partial_u\partial_u r = 0$ on $\underline c_p$, so by the initial conditions we have $\partial_u r = 0$ on $\underline c_p$. Similarly $\partial_v r = 0$ on $c_p$. Thus $r = r(p)$.

    If $\sigma_\nu \neq 0$, $\sigma_\lambda \neq 0$, then we again have $\partial_u \partial_u r = 0$ along $\underline c_p$. The only remaining degree of freedom is our freedom to choose $g^{\mu\nu} \partial_\mu \partial_\nu r(p)$, which ends up determining $\partial_u r$. Therefore we know the value of $r$ along $\underline c_p, c_p$.

    Finally assume $\sigma_\nu \neq 0$ but $\sigma_\lambda = 0$. The proof is similar to the previous cases.

    We have proven uniqueness in the future-pointing lightcone of $p$. By time-reversal symmetry we obtain uniqueness in the past-pointing lightcone. By spherical symmetry, we can switch $u$ and $v$ with $-u$ and $-v$ and run the same argument for the ``left-pointing lightcone" and the ``right-pointing lightcone" which is all four cones that are around $p$.
\end{proof}
    The point is that $r,\sigma_\nu,\sigma_\lambda$, and $\mu_0 = g^{\mu\nu}\partial_\mu r \partial_\nu r(p)$ are geometric data, and don't depend on the choice of coordinates; but everything that isn't determined by these terms is determined by our choice of coordinates. Actually, $\mu_0$ is only needed when $\sigma_\nu$ and $\sigma_\lambda$ are nonzero.
\begin{definition}
    The \dfn{Hawking mass} is a function $m$ on spherically symmetric spacetime defined by
    $$g^{\mu\nu} \partial_\mu r \partial_\nu r = 1 - 2mr^{-1}.$$
\end{definition}
    Note that $m$ does not depend on a choice of coordinates since neither does $r$.
\begin{lemma}
    The Hawking mass is constant on connected components.
\end{lemma}
\begin{proof}
    We have
    $$-4\partial_ur\partial_vr\Omega^{-2} = g^{\mu\nu} \partial_\mu r \partial_\nu r$$
    so by implicit differentation we have
    $$-2r^{-1}\partial_u m + 2 \partial_u r r^{-2} m = -4 \partial_ur \Omega^{-2} \partial_u\partial_v r.$$
    Doing a bunch of algebra we see $\partial_u m = \partial_v m = 0$.
\end{proof}
When $m = 0$ we will end up with Minkowski spacetime. If $m > 0$ one can show that the null lapse is given by
$$\Omega^2 = -\sigma_\nu\sigma_\lambda(1 - 2mr^{-1}).$$
This gives a certain metric that we call the Schwarzschild metric.
\begin{definition}
The \dfn{Schwarzschild metric} is the metric
$$g = -\sigma_\nu\sigma_\lambda (1 - 2mr^{-1}) ~dudv + r^2 \underline g.$$
\end{definition}
One can construct a maximal Schwarzschild spacetime. In fact if we define $\mu$ by $1 - \mu = g^{\alpha\beta}\partial_\alpha\partial_\beta r$, then $r\mu$ is constant, and in fact we take $r\mu = 2m$. Doing some algebra and using the Raychaudhuri equations, we have $\Omega^2 = |1-2mr^{-1}|$. If $r \to r_0$ as $v \to \infty$, then $\partial_v r \to 0$, i.e. $r_0 = 2m$.

We now rephrase Birkhoff's theorem.
\begin{corollary}
    If $(M, g)$ is a spherically symmetric solution to the Einstein vacuum equation then $(M, g)$ is locally isometric to an open subset of a Schwarzschild spacetime.
\end{corollary}

We think of Schwarzschild spacetime as an easy example of a black hole spacetime, for $m > 0$. Choosing our signs correctly, we have
$$g = -(1-2mr^{-1})~dudv + r^2 \underline g = -dudv + r^2 \underline g + o(1)$$
so that $g$ approximates Minkowski spacetime for $r$ large enough (or $m$ small enough; if $m = 0$ it is Minkowski spacetime, with the singularity $r = 0$ artifically added by the choice of coordinates). That is, $g$ is asymptotically flat, so models a gravitational system (where the mass is concentrated in a compact set -- say, all the mass is inside some star.) In particular, if the observer is not massless, then the observer is at $r = \infty$. Drawing the Penrose diagram, our causal past, looking in from $r = \infty$, is $r > 2m$. Thus no matter how far in the future we are, we lie in the causal complement of the region $r < 2m$.
\begin{definition}
    The boundary $r = 2m$ of a black hole is called the \dfn{event horizon}. The region $r < 2m$ is called a \dfn{black hole}.
\end{definition}
In fact we have $R_{\mu\nu\alpha\beta}R^{\mu\nu\alpha\beta} \geq Cm^2r^{-6}$, so the spacetime has infinite curvature at $0$.

But assume $m < 0$. Then $r = 0$ is a ``naked singularity", which lies in our causal past. A major conjecture, the \dfn{weak cosmic censorship conjecture}, is that for any physically meaningful spacetime, naked singularities do not exist. Note that the Schwarzschild spacetime with $m < 0$ is not a counterexample, because such a spacetime somehow has negative mass, which is absurd.

\section{Einstein-Maxwell equations}
Let $F_{\mu\nu}$ be a real-valued $2$-form on $M$, the \dfn{electromagnetic field}. If $(M, g)$ is Minkowski spacetime, we can take $E_i = F_{0i}$ and $B_i = \epsilon_{ijk}F^{jk}/2$, the Hodge dual of $E$, to recover the electric and magnetic fields.
\begin{definition}
    The \dfn{Maxwell equations} are the system $\nabla^\mu F_{\nu\mu} = 0$, $dF = 0$.
\end{definition}
Let us assume that $F$ is spherically symmetric; i.e. if $R \in SO(3)$ then $R^*F_{\mu\mu} = F_{\mu\nu}$, where we think of $SO(3)$ as the symmetry group of $(M, g)$. We will write
$$F = F_{uv} ~du\wedge dv + F_{\theta\varphi} ~d\theta \wedge d\varphi.$$
One can use algebraic topology to prove that that $F_{uv}$ is completely determined by $u,v$ and $F_{\theta\varphi}$ is completely determined by a function of $u,v$ as well as $\sin \theta$. Since $dF = 0$, $\partial_u F_{\theta\varphi} = 0$, and $\partial_v F_{\theta\varphi} = 0$. So actually $F_{\theta\varphi} = m \sin\theta ~d\theta \wedge d\varphi$ for some constant $m$. Also,
$$0 = \nabla^\mu F_{u\mu} = -2\Omega^{-2}\partial_u(r^2\Omega^{-2} F_{uv})$$
and similarly for $v$. Thus $\partial_u(r^2\Omega^{-2}F_{uv}) = 0$ and similarly for $v$. Thus $F_{uv} = e\Omega^2r^{-2}$ for some constant $e$.

\begin{theorem}[Weyl?]
    Every spherically symmetric solution $F$ of the Maxwell equation is given by
    $$F = e\Omega^2r^{-2}~du\wedge dv + b\sin \theta ~d\theta \wedge d\varphi.$$
\end{theorem}
Thus an electromagnetic field is completely determined by the pair $(e, m)$. In Minkowski spacetime, $e\Omega^2r^{-2} ~du\wedge dv = er^{-2} ~dt \wedge dr$ which is the electric field given by a point charge at the origin, while $b\sin \theta ~d\theta \wedge d\varphi$ is the magnetic flux through a sphere.

We now derive the Einstein-Maxwell system from the principle of least action. The Lagrangian density of the Einstein vacuum equation was $R~dV$ while the Lagrangian density of the Maxwell equation is $-F_{\alpha\beta}F^{\alpha\beta}/2$. Thus we have
\begin{align*}
    \Ric_{\alpha\beta} - \frac{1}{2}g_{\alpha\beta}R &= 2T_{\alpha\beta}\\
    T_{\alpha\beta} &= F_{\alpha\mu}F^\mu_\beta - \frac{1}{4}g_{\alpha\beta} F_{\mu\nu} F^{\mu\nu}\\
    \nabla^\alpha F_{\alpha\beta} = dF &= 0.
\end{align*}
As usual, $T_{\alpha\beta}$ is the energy-momentum tensor of the Maxwell equation. We now compute $T_{\alpha\beta}$ by
\begin{align*}
    T_{uu} = T_{vv} &= 0\\
    T_{uv} &= 4^{-1} \Omega^2 r^{-4}(b^2 - e^2)
\end{align*}
since
$$F_{\mu\nu} F^{\mu\nu} = 2(g^{uv})^2 (F_{uv})^2 + g^{AA'}g^{BB'} F_{AB} F_{A'B'} = 2r^{-4}(e^2 + b^2).$$
Moreover, $T_{AB}$ is proportional to $T_{uv}$. Thus, the Einstein-Maxwell equations in spherical symmetry, obtained by plugging into the Raychaudhuri and Einstein spherically symmetric equations, is
\begin{align*}
    0 &= -2r^{-1}\Omega^2 \partial_u (\Omega^{-2} \partial_ur)\\
    0 &= -2r^{-1}\Omega^2 \partial_v (\Omega^{-2} \partial_vr)\\
    0 &= 2r^{-1}\partial_u\partial_v r + 2r^{-2}\partial_ur\partial_vr + \Omega^22^{-1}r^{-2} - 2^{-1}\Omega^2r^{-4}(e^2 + b^2).
\end{align*}
\begin{theorem}
    Let $(M, g, F)$ be a spherically symmetric solution to the Einstein-Maxwell system. Then for any $p \in M$, the solution is determined in an open neighborhood $O$ of $p$ by $r$, $\mu$, $\sigma_\nu$, $\sigma_\lambda$, $e = 2r^2\Omega^{-2}F_{uv}$, and $b = \csc \theta F_{\theta \varphi}$.
\end{theorem}
\begin{proof}
    The electromagnetic field is rigid since we are in spherical symmetry. Now run the proof of Einstein vacuum equation rigidity (using the Raychaudhuri equations, which are the same as before) but with $T_{\alpha\beta}$ given by $(e, b)$.
\end{proof}
\begin{lemma}
    $d(1 - \mu) \wedge dr = 0$.
\end{lemma}
\begin{proof}
    Same as in the vacuum case, because $\Ric_{uu} = \Ric_{vv} = 0$.
\end{proof}
As a result, there are constants $C$ such that $1 - \mu = 1 - 2Cr^{-1} + 2C(e^2 + b^2)r^{-2}$. In fact, $1 - \mu = g^{\alpha\beta} \partial_u r \partial_v r = -4\partial_u r \partial_vr \Omega^{-2}$ so we have
$$d(1 - \mu) = -4d(\partial_u r \partial_v r\Omega^{-2}) = -4 \partial_u\partial_vr \Omega^{-2} (\partial_ur ~du + \partial_vr ~dv).$$
Therefore
$$d(1 - \mu) = -4\Omega^{-2}\partial_u\partial_vr ~dr.$$
Using the Einstein-Maxwell equations we see that if $f(r) = -r^{-1}(1 - \mu) + r^{-1} - (e^2 + b^2)r^{-3}$ then $d(1 - \mu) = f(r) ~dr$. In addition, if $h = 1 - \mu$ then $h' = f$ so $d(rh)/dr = 1 - (e^2 + b^2)r^{-2}$ whence $rh = C + r + (e^2 + b^2)r^{-1}$. This proves the above claim.

We now search for a global solution to the Einstein-Maxwell equation. We do this in Eddington-Finkelstein coordinates, which just means that $\partial_ur \Omega^{-2}$ is constant in one dimension and $\partial_vr \Omega^{-2}$ is constant in the other dimension. This is possible because of the Raychaudhuri equations. In these coordinates, $\Omega^2 = |1 - 2Cr^{-1} + (e^2 + b^2)r^{-2}|$ and $g = -\Omega^2 du~dv + r^2 \underline g$.

We look at the sign of the discriminant $C^2 - Q^2$ where $Q^2 = e^2 + b^2$. If $0 < |Q| < C$ then there are two solutions to the equation $1 - 2CR^{-1} + Q^2r^{-2} = 0$, namely $r_\pm = C \pm \sqrt{C^2 - Q^2}$. This is called the \dfn{subextremal case}.

By the Raychaudhuri equations the signs of $\partial_ur$ and $\partial_vr$ cannot change along the $u$ and $v$ directions respectively. So we can fix a sign for each and see what happens.

First take the case $\partial_ur < 0$, $\partial_vr > 0$. Assume that $r \to r_+$ as $u \to \infty$, $v \to -\infty$; then $r \to \infty$ as $u \to -\infty$, $v \to \infty$. Thus along every null curve, $r$ tends to $r_+$ in one direction and $\infty$ in the other direction. On the other hand, if we take ``initial data" $r = r_-$, then we hit $r = 0$ for some finite $u$, which is a singularity.

If $\partial_ur < 0$, $\partial_vr < 0$, then as $u \to -\infty$, $v \to -\infty$, $r \to r_+$. Similarly as $u \to \infty$, $v \to \infty$, $r \to r_-$.

Gluing together the above Penrose diagrams we construct all possible solutions to the Einstein-Maxwell equations in spherical symmetry. We have to make sure that $\Omega$ is continuous along the gluings, which can be guaranteed by a clever change of coordinates. The maximal such simply connected solution is called the \dfn{maximal Reissner-Nordstrom spacetime}. It is not compact.

In the \dfn{superextremal case} $Q^2 > C^2$ we recover the negative-mass Schwarzschild solution.

Finally we consider the \dfn{extremal case} $Q^2 = C^2$. The resulting maximal solution is the \dfn{Bertotti-Robinson spacetime}. One can show that
$$\partial_u\partial_v \log \Omega = 2r^{-2} \partial_u \partial_v r + ((2r^2)^{-1} - e^2r^{-4})\Omega^2$$
using the equation for the angular Einstein tensor $\Ric_{AB} - g_{AB}R/2$ and the angular energy-momentum $T_{AB} = Q^2$. One then shows that $\partial_u\partial_v \log \Omega = K\Omega^2$ for some $K = (2r_0^2)^{-1} - e^2 r_0^{-4}$. This is a constant-curvature spacetime.

\begin{theorem}[Birkhoff for Einstein-Maxwell]
    \index{Birkhoff's theorem}
    If $(M, g, F)$ is a spherically symmetric solution to the Einstein-Maxwell equation, then each point of $M$ is contained in an open set which is isometric to an open set of either the maximal Reissner-Nordstrom spacetime, the Bertotti-Robinson spacetime, a Schwarzschild spacetime, or the Minkowski spacetime.
\end{theorem}

\chapter{Cosmic censorship}
In GR, we are interested in two regimes: isolated gravitational systems (asymptotically flat spacetimes; there is a singularity at one point and everything else is a vacuum, so we are studying the dynamical structure) and cosmological systems (where we are modeling the entire universe, and we want to study the topological structure). For now, we will study the isolated case, and view it as a Cauchy problem.

Notice that the Cauchy problem behaves quite strange in the negative Schwarzschild spacetime $(M, g)$. Suppose we have an initial-data set $\Sigma$ for $M$; then, a geodesic in $\Sigma$ along which $r \to 0$ cannot be extended to the future. Drawing the Penrose diagram we see that the negative Schwarzschild spacetime is ``not deterministic," i.e. $\Sigma$ does not uniquely determine the future because we cannot extend it into the future-pointing lightcone of the black hole.

At least in a positive Schwarzschild spacetime, these ``incomplete geodesics" are inside the black hole region. Therefore the observer at infinity cannot see the singularity, where we cannot extend an initial data set to the future. But in the negative Schwarzschild spacetime, the observer sees the singularity. But negative Schwarzschild spacetimes have negative mass by definition, which makes no sense physically.

We thus state the weak cosmic censorship conjecture: an observer at infinity cannot see a singularity in a ``typical" physically meaningful spacetime.

We call the future boundary of a Penrose diagram (limiting points of radial null geodesics along which $r \to \infty$) the \dfn{null infinity} of the spacetime. A spacetime has \dfn{complete null infinity} if the lengths of geodesics parallel to null infinity tend to $\infty$ as $r \to \infty$. In the negative Schwarzschild spacetime, the null infinity was incomplete because the null infinity was the limit of the causal future of the initial-data set, which was compact.

We will be deliberately vague about what we mean by a \dfn{reasonable Einstein-matter system}, but it will be the Einstein equation coupled to physically meaningful Lagrangian densities (i.e. the Maxwell density, the vacuum density, etc.) Similarly for \dfn{physically-meaningful initial-data set} but in particular the initial-data set should be a \dfn{geodesically complete manifold}. (This means that you can ``follow a geodesic forever"; or in other words the domain of the exponential map $T\Sigma \to \Sigma$ is defined on all of $T\Sigma$.) This rules out the punctured line and manifolds with boundary, because those have singularities we can run into in finite distance, which does not seem physically reasonable.

By generic we mean in the sense of the Baire category theorem. In fact, Christodoulou has proven that a naked singularity is unstable, and under a slight perturbation of $g$ necessarily collapses into a black hole, and so is hidden from the observer at infinity, as in the positive Schwarzschild spacetime.
\begin{conjecture}[weak cosmic censorship]
    Given a generic physically-meaningful initial-data set to a reasonable Einstein-matter system in an asympotically flat universe, the future maximal globally hyperbolic development has complete null infinity.
\end{conjecture}

Recall that by definition, the maximal globally hyperbolic development ends at the spacelike hypersurface wherein the development fails to be unique. This region is called a \dfn{Cauchy horizon}. In a Reissner-Nordstrom black hole, there is a Cauchy horizon, so that a test particle falling into a black hole is NOT unique.
\begin{conjecture}[strong cosmic censorship]
    Given a generic physically-meaningful initial-data set to a reasonable Einstein-matter system in an asymptotically flat universe, the future maximal globally hyperbolic development is inextendible as a smooth Lorentzian manifold.
\end{conjecture}

\section{Einstein-Maxwell-charged scalar field equations}
The most complicated model of the cosmic censorship conjectures is the \dfn{Einstein-Maxwell-charged scalar field} equation. A scalar field $\phi$ is a section of a complex line bundle $E$ whose structure group is the orthogonal group $O(1)$. This gives rise to a connection $D$ on $E$ and
$$F_{\alpha\beta} = [D_\alpha, D_\beta].$$
Locally, we have $D_\alpha = \partial_\alpha + iA_\alpha$. The action is given by
$$\rho(\phi, D, g) = \int R ~dV(g) - \int F^{\alpha\beta}F_{\alpha\beta} ~dV_g - 2 \int \langle D^\alpha\phi, D_\alpha\phi\rangle ~dV_g$$
where $\langle \phi, \psi \rangle = \Re(\phi\overline\psi)$ is the natural real-valued inner product on a complex line bundle.

In case $\phi = 0$, the Einstein-Maxwell-charged scalar field reduces to the Einstein-Maxwell system, but it is dynamical because it solves the wave equation with connection $D$, namely
$$D_\alpha D^\alpha \phi = 0.$$
However, the Einstein-Maxwell-charged scalar field is too hard to study directly, so we restrict to subsystems thereof.
\begin{example}
    The \dfn{Einstein-scalar field} equation or \dfn{Christodoulou model} is the Einstein-Maxwell-charged scalar field with trivial Maxwell tensor, $F_{\alpha\beta} = 0$. Then $D = \partial$, so we do not need to worry about the curvature of the line bundle. That is, we can think of $\phi$ as a mapping $\phi: M \to \RR$. It is the model that we will study when we consider the weak cosmic censorship conjecture.
\end{example}
\begin{example}
    The \dfn{Einstein-Maxwell-uncharged scalar field} equation or \dfn{Daferemos model} is the system obtained by decoupling $\phi$ from $F$. In other words, $\phi: M \to \RR$ (so the curvature of the line bundle is trivial). It is the model where Reichner-Nordstrom spacetimes make sense, so we study the strong cosmic censorship here.
\end{example}

We now study the (relativistic) kinetic theory of the Einstein equation. Let $M$ be a spacetime, so $T^*M$, the cotangent bundle, has a natural symplectic form, namely
$$\omega = dx^\alpha \wedge dp_\alpha.$$
Here $x^\alpha$ is a coordinate system on $U \subseteq M$ and we view a covector as $p_\alpha ~dx^\alpha$. Given $H \in C^\infty(T^*M)$ we define the \dfn{Hamiltonian vector field} by
$$(X^H)^\alpha = \omega^{\alpha\beta}~dH_\beta.$$
In case $H = 2^{-1}p^\alpha p_\alpha$ then $X^H$ is the vector field on the cotangent bundle whose flow restricts to the Hamiltonian flow on $M$. We apply the Legrende transform $(x^\alpha, p_\alpha) \mapsto (x^\alpha, p^\alpha)$ we get a flow on $TM$ for which $\dot x^\alpha = p^\alpha$, $\dot p^\alpha = -\Gamma^\alpha_{\beta\gamma} p^\beta p^\gamma$. We let $\Phi^H$ denote the induced flow of $X_H$.

\begin{definition}
    A \dfn{Vlasov field} is a positive measure $\mu$ on $T^*M$ which is invariant under the pullback by $\Phi^H_t$ for every $t$; that is,
    $$\mu = (\Phi_t^H)^* \mu.$$
\end{definition}
    In Newtonian mechanics, one assumes that the Vlasov field is absolutely continuous with respect to the natural volume form $\epsilon$ induced by the symplectic form $\mu$. Using the Radon-Nikodym theorem, we find an $f$ so that $\mu = f \epsilon$. Since $\Phi^H$ preserves $\epsilon$ we just need to check that $X^Hf = 0$, the \dfn{Vlasov equation}.

    Now $T^*M$ is foliated by level hypersurfaces of $H$, and $X^HH = 0$, so $\Phi^H$ preserves the foliation of $T^*M$. Now a null geodesic is one arising from the flow restricted to $H = 0$, and timelike geodesics are those for which $H = -1$. To restrict to future-pointing geodesics we assume $p^0 < 0$. Thus we define $P_0^+$ to be the $(x, p)$ with $H(x, p) = 0$ and $p^0 < 0$. Similarly for $P_1^+$ where we have $H(x, p) = -1$. These level hypersurfaces are $7$-manifolds and we search for a top form on them. Now
    $$\epsilon_{P^+_\sigma} = c~dH\wedge\omega\wedge\omega\wedge\omega$$
    for some function $c$ allowed to depend on $\sigma \{0, -1\}$. For $\mu = f \epsilon_{P^+_\sigma}$, $X^Hf = 0$ iff
    $$p^\alpha \partial_\alpha f = \partial_\alpha g^{\beta\gamma} p_\beta p_\gamma \partial_\alpha f = 0.$$
    Of course if $\sigma = 0$ then we are thinking of our particle as a photo (no mass) so we say that this is the ``massless" case and $\sigma = -1$ is the ``massive" case.
\begin{definition}
    Let $\mu$ be a Vlasov field which is absolutely continuous with respect to $\epsilon_{P^+_\sigma}$. The \dfn{associated energy-momentum tensor} $T_{\alpha\beta}$ of $\mu$ is given weakly by (with $\varphi$ a test function)
    $$\int_M T_{\alpha\beta})x \varphi(x) ~dV(g) = \int_{T^*M} p_\alpha p_\beta \varphi(x) ~d\mu.$$
    The \dfn{number density} is
    $$\int_M N_\alpha(x) \varphi(x) ~dV(g) = \int_{T^*M} p_\alpha \varphi(x) ~d\mu.$$
\end{definition}
    Since $\mu$ is absolutely continuous, it is supported on the $7$-manifold which is a level hypersurface of $H$. If $f$ is the Radon-Nikodym derivative, then
    $$T_{\alpha\beta}(x) = \int_{T^*M} p_\alpha p_\beta f(-\det g)^{-1/2}\epsilon_{P^+_\sigma}|_{T^*_xM}$$
    and
    $$N_\alpha(x) = \int_{T^*M} p_\alpha f(-\det g)^{-1/2}\epsilon_{P^+_\sigma}|_{T^*_xM}.$$

    To couple the Vlasov field to the Maxwell equation we take $N_\beta = \nabla^\alpha F_{\alpha\beta}$ and
    $$2H = g^{\alpha\beta}(p_\alpha + A_\alpha)(p_\beta + A_\beta).$$

\begin{example}
    A subsystem of the Einstein-Vlasov system is the \dfn{Einstein-null dust system}. It is too simple to be realistic but is useful to demonstrate computations. The interpretation is that everything travels along radial null geodesics. So there is no mass, and the physical system consists solely of radiation moving radially.

    An \dfn{null dust field} which is outgoing is characterized by having energy-momentum tensor $T_{\alpha\beta}$ such that
    $$T^{out}_{uu} \geq 0$$
    with other components zero. Similarly $T^{in}_{vv} \leq 0$ for incoming null dust fields. Now $\nabla^\alpha T_{\alpha\beta} = 0$ so $\partial_v T_{uu}^{out} = 0$. (Similarly $\partial_u T_{vv}^{in} = 0$.)

    We will assume that there are two noninteracting null dusts, one incoming and one outgoing. That is, the Einstein-null dust equation is given by
    $$\Ric_{\alpha\beta} -\frac{1}{2}g_{\alpha\beta}R = 2(T^{in}_{\alpha\beta} + T^{out}_{\alpha\beta}).$$
\end{example}

\section{The structure of toy models}
    Two basic papers about the a priori characterizations of solutions to spherically symmetric toy models are Daferemos ``Spherically symmmetric spacetimes with a trapped surface" and Komnemi ``The global structure of a spherically symmetric charged scalar field spacetime". Let us give a shallow introduction to this theory.

    We will let $(M, g)$ be the $1+3$-dimensional maximal globally hyperbolic development with a spherically symmetric initial data set $\Sigma_0$. Let $(Q, g_Q)$ be the quotient of $(M, g)$ by $SO(3)$. We will assume that $\Sigma_0$ is diffeomorphic to $\RR^3$ or $\RR \times S^2$. (The latter is the initial-data set of the spacetimes for which we will study the strong cosmic censorship conjecture.)

    If $\Sigma_0 = \RR^3$, then by algebraic topology, there is a fixed point of $SO(3)$. In other words, the set $\Gamma = \{r = 0\}$ has
    $$\Gamma \cap \Sigma_0 = \{p\}.$$
    On the other hand, if $\Sigma_0 = \RR \times S^2$, then $SO(3)$ cannot have any fixed points.

    By global hyperbolicity, there is a future-pointing double null pair $(u, v)$ on $Q$. The existence of a double null pair implies that there is an embedding $Q \to \RR^2$, i.e. a Penrose diagram. We will write $\overline Q$ for the closure of $Q$ inside $\RR^2$; i.e. if $Q$ is not a closed manifold then we will take it to be a manifold with boundary.

\begin{definition}
    $T_{\alpha\beta}$ obeys the \dfn{dominant energy condition} if for every causal, future-pointing vectors $x, y$,
    $$T_{\alpha\beta}x^\alpha y^\beta \geq 0.$$
\end{definition}
    In fact, $T_{\alpha\beta} \dot \gamma^\alpha = J_\beta$ should be interpreted as the ``energy-momentum" along $\gamma$. In fact, the coordinate of $J_\beta$ along $\gamma$ is the energy along $\gamma$. So the dominant energy condition says that there is positive energy.

    In spherical symmetry, the dominant energy condition is equivalent to $T_{uu} \geq 0$, $T_{vv} \geq 0$, $T_{uv} \geq 0$. It follows that $\Ric_{uu} \geq 0$, $\Ric_{vv} \geq 0$. Since
    $$\Ric_{uu} = -2r^{-1}\Omega^2 \partial_u (\Omega^{-2}\partial_ur)$$
    it must be that the sign of $\partial_ur$ is preserved, and similarly for $\partial_vr$. Thus $r$ is monotone in $u$ and $v$ separately.

    Henceforth we assume the dominant energy condition. It therefore makes sense to also assume the antitrapping condition:
\begin{definition}
    $\Sigma_0$ obeys the \dfn{antitrapping condition} if: if $\Sigma_0 = \RR^3$ then $\partial_ur < 0$ on $\Sigma_0$; if $\Sigma_0 = \RR \times S^2$ then $\partial_ur < 0$ on some $\Sigma_0'$ a connected subset of $\Sigma$ which meets the ideal endpoint of $\Sigma_0$ on the right.
\end{definition}
    Let
    $$Q' = \{(u, v) \in Q: \exists u_0~(u_0(v), v) \in \Sigma_0\}.$$
    Then the antitrapping condition implies that $\partial_ur < 0$.
\begin{definition}
    Assume the antitrapping condition. $(u, v) \in Q'$ is \dfn{trapped} if $\partial_vr < 0$. $(u, v) \in Q'$ is \dfn{regular} if $\partial_ur > 0$. $(u, v)$ is \dfn{marginally trapped} if $\partial_vr = 0$.

    Because of these conventions, we say that $u$ is incoming and $v$ is outgoing.
\end{definition}
    In a black hole, every point is trapped. The event horizon is marginally trapped. Formally, if $T$ is the set of trapped points, and $(u, v')$ lies in the future of $(u, v)$, then $(u, v) \in T$ implies $(u, v') \in T$.
\begin{theorem}[Penrose singularity theorem]
    \index{Penrose singularity theorem}Suppose that $T$ is nonempty. Then there is an incomplete outgoing null geodesic.
\end{theorem}
    Recall that a geodesic $\gamma$ is complete if for every $t$ such that $\gamma_{\dot \gamma}\dot \gamma(t) = 0$ ($t$ is an \dfn{affine paramter}), $\gamma(t)$ exists. This is not the case if $\gamma$ runs into a boundary. That is, there is an incomplete geodesic the exponential map $TM \to M$ fails to be defined far away from the origin of each tangent space. Incomplete null geodesics can be interpreted physically as meaning that a light wave fails to exist after traveling a finite distance.
\begin{proof}
    Let $(u_0, v_0) \in T$ be trapped, and let $(u_1, v_1)$ be the endpoint of the outgoing null geodesic from $(u_0, v_0)$. This is finite because we embedded $Q$ in $\RR^2$. Now we compute
    $$\int_{v_0}^{v_1} \Omega^2(u, v) ~dv$$
    and use the Raychaudhuri equations and the trapping conditions to conclude that the integral is the integral of a bounded function over a compact set. So it's finite, hence an affine parameter.
\end{proof}
    Actually the Penrose singularity theorem holds in much greater generality. The existence of trapped surfaces is an open condition on the moduli space of all initial-data sets, which implies that there is a \emph{stable} singularity, which necessarily follows from the existence of black holes.

\section{Penrose inequalities}
We generalize the result that says that the mass of the universe is positive if there are no black holes, to bound the mass of the universe in terms of the radius of the black hole. We follow Daferemos's paper ``Spherically symmetric spacetimes with a trapped surface".

Let $(M, g)$ be a spherically symmetric solution to the Einstein equation, which is the future maximally globally hyperbolic development of a spherically symmetric initial-data set $\Sigma_0$, where $\Sigma_0$ is either homeomorphic to $\RR^3$ to $\RR \times S^2$. Let $Q = M/SO(3)$ be the Penrose diagram of $(M, g)$. We will assume the dominant energy condition on the energy-momentum tensor $T$ (i.e. $T_{uu} \geq 0$, $T_{vv} \geq 0$, $T_{uv} \geq 0$). We also assume that there are no antitrapped spheres (which for $\Sigma_0 = \RR^3$ means that $\partial_ur < 0$ on $\Sigma_0$.)

Let $Q'$ be the set of points in the Penrose diagram which are in the image of an incoming null curve from $\Sigma_0$. By the Raychaudhuri equations and the assumption on antitrapped spheres, $\partial_ur < 0$ on $Q'$.

Let $A$ be the apparent horizon, i.e. those $(u, v)$ for which $\partial_vr(u, v) = 0$. Every event horizon is contained in the apparent horizon.

Let $U$ be the set of $u$ such that $\sup_v r(u, v) = \infty$. Thinking of $Q$ as a bounded subset of $\RR^2$ we let $\zeta^+$ be the set of $(u,v) \in \partial U$ such that $u \in U$.
\begin{definition}
    $\zeta^+$ is the \dfn{future null infinity} of $Q$.
\end{definition}
\begin{lemma}
    If $\zeta^+$ is nonempty, then it is a connected incoming null curve emanating from the interior.
\end{lemma}
\begin{proof}
    If $(u, v) \in \zeta^+$ then we can find a point on $\Sigma_0$ whose lightcone includes $(u, v)$.
\end{proof}
\begin{lemma}
    $J^-(\zeta^+) \subseteq R$, the set of regular points.
\end{lemma}
    So a particle cannot end up in the future null infinity if it is trapped or lies on an event horizon.

Recall that the Hawking mass $m$ at $(u, v)$ is defined by
$$1 - \frac{2m}{r} = -4 \frac{\partial_ur\partial_vr}{\Omega^2}.$$
\begin{lemma}
    One has $\partial_um = 2r^2\Omega^{-2} (T_{uv}\partial_ur - T_{uu}\partial_vr)$ and similarly for $v$.
\end{lemma}
\begin{proof}
    Use the Einstein equations in spherical symmetry and the dominant energy and no-antitrapping conditions.
\end{proof}
\begin{corollary}
    Inside $R \cup A$, $\partial_um \leq 0$ and $\partial_vm \geq 0$.
\end{corollary}
\begin{lemma}
    Inside $Q'$, the sign of $\partial_vr$ is the sign of $1 - 2mr^{-1}$.
\end{lemma}
    So in particular, a point is trapped provided that $1 - 2mr^{-1}$.
\begin{definition}
    Fix $(u, v) \in J^{-1}(\zeta^+)$. Define the \dfn{Bondi mass}
    $$M(u) = \lim_{v \to v_{\zeta^+}} m(u, v).$$
    The \dfn{ADM mass} is
    $$M_{ADM} = \lim_{u \to \Sigma_0} M(u).$$
\end{definition}
    So the Bondi mass is the mass observed by someone standing at the future end of a curve for which $u$ is constant. The ADM mass is the mass observed by an observer at $\partial \Sigma_0$. Here ``feeling mass" means experiencing a gravitational field.
\begin{theorem}[positive mass theorem]
    \index{positive mass theorem}
    If $\Sigma_0 = \RR^3$ then $M_{ADM} \geq 0$.
\end{theorem}
\begin{proof}
    Either $\Sigma_0 \subseteq R$ or not. If not, then there is a point on $\Sigma_0$ which does not end up at $\zeta^+$, and in particular there is a point $(u_0, v_0)$ on $\partial R \cap \Sigma_0$ which lies in the apparent horizon. So at that point, $1 = 2mr^{-1}$. Since $r > 0$, $m > 0$, and the monotonicity properties above guarantee that the regular points also have positive mass. The observer at infinity can only feel things in his causal past, in particular $\Sigma_0 \cap R$, so we're done.

    If $\Sigma_0 \subseteq R$, note that since $g$ is smooth, $r$ is Lipschitz. So
    $$1 - 2mr^{-1} = g(\partial r, \partial r)$$
    is bounded, whence $m \to 0^+$ as $r \to 0$ on $\Sigma_0$. By monotonicity, $m \geq 0$ on $\Sigma_0$, and so $M_{ADM} \geq 0$.
\end{proof}
    Note that in the case $\Sigma_0 \subseteq R$, we used the fact that $\Sigma_0 = \RR^3$ so show that $g$ is smooth and that $\Sigma_0$ is connected. We used the dominant energy condition and the antitrapping to guarantee monotonicity.
\begin{corollary}[Riemannian Penrose inequality]
    \index{Riemannian Penrose inequality}
    Let $S_R$ be a minimal sphere in $\Sigma_0$ of radius $R > 0$, and assume that the second fundamental form is $0$. Then
    $$M_{ADM} \geq \frac{R}{2}.$$
\end{corollary}
    In $\RR^3$ there are no minimal spheres so we take $R \to 0$. The positive mass theorem is sharp, because the ADM mass of Minkowski spacetime is $0$.
\begin{definition}
    The \dfn{generalized extension principle} is the assumption that for every $p \in \overline Q$, $q \in I^-(p)$, $q \neq p$, if
    $$D = J^+(q) \cap J^-(p) \setminus p,$$
    then $D$ has finite volume and $$0 < \inf_D r < \sup_D r < \infty.$$
\end{definition}
    The generalized extension principle holds for any reasonable spacetime.
\begin{example}
    The generalized extension principle is not true for the Einstein null dust spacetime.
\end{example}
\begin{definition}
    The \dfn{event horizon} $H^+$ is the future boundary of $J^-(\zeta^+)$.
\end{definition}
    Then one has
    $$\lim_{v \to \zeta^+} r = \sup_{H^+} r.$$
\begin{definition}
    The \dfn{final Bondi mass} is
    $$M_f = \lim_{u \to u_\Box} M(u) = \inf_u M(u),$$
    the limit taken as $u$ goes to the future.
\end{definition}
    The fact that this is an infimum follows from the monoticity assumptions.
\begin{theorem}[Penrose event horizon inequality]
    \index{Penrose event horizon inequality}
    Assume the generalized extension principle. Then
    $$\sup_{H^+} r \leq 2 M_f.$$
\end{theorem}
    We think of the sup as the radius of the black hole. Unravelling the definitions, we obtain a lower bound on all Bondi masses that follows from the size of the black hole.

    The idea of the proof is that if we have control of $1 - 2mr^{-1}$, then we use the Raychaudhuri equation
    $$-4 \partial_ur\Omega^{-2} = \frac{1 - 2mr^{-1}}{\partial_vr}$$
    to control $\partial_vr$ in terms of $\partial_ur$. We then use the definition of the Hawking mass to control the integral of the energy-momentum tensor. So if the conclusion of the Penrose event horizon inequality fails, we can find a $(u, v)$ on the event horizon such that $r > 2M_f$. Since $M$ and $r$ obey similar monoticity conditions, we obtain an absurd bound on the mass.

\section{Recent progress on strong censorship}
\begin{conjecture}
    For a generic asymptotically flat initial data set for a ``reasonable" Einstein-matter Lagrangian, the future maximal globally hyperbolic development is inextendible as a ``suitably regular" Lorentzian manifold.
\end{conjecture}
\begin{example}
    The Reichner-Nordstrom spacetime is a (highly nongeneric) counterexample.
\end{example}
    To prove the strong cosmic censorship conjecture one would need to show that the Cauchy horizon of any counterexample must be unstable in the sense that a small perturbation of the initial-data set would necessarily destroy the Cauchy horizon. To do this, we first characterize its stability properties.
\begin{example}
    Let us study the stability properties of Cauchy horizons in the Einstein-Maxwell null dust Lagrangian. Recall that null dust is defined by its energy momentum tensor. That is, $T_{uu}^{out}$ is nonzero and can depend on $u$ and $v$ but all other $T_{\alpha\beta}^{out} = 0$. This is the outgoing null dust. By Noether's theorem, $\nabla^\mu T_{\mu\nu}^{out} = 0$ which implies that $T_{uu}$ does not depend on $v$. We also have a term $T_{vv}^{out}$ which can be nonzero and only depends on $v$.

    The Einstein-Maxwell-null dust equation is $G_{\alpha\beta} = 2T_{\alpha\beta}$ like usual, where
    where
    $$T_{\alpha\beta} = T^{out}_{\alpha\beta} + T^{in}_{\alpha\beta} + T_{\alpha\beta}^{Max}$$
    where $T^{Max}$ is the Einstein-Maxwell energy-momentum tensor. But $T_{uu}^{Max} = T_{vv}^{Max} = 0$, so the null dust coordinates do not interact with the Einstein-Maxwell coordinates.

    First we treat the case $T^{out}|_{\Sigma_0} = 0$. Assume that $T^{in}|_{\Sigma_0}$ is small and rapidly decaying. We claim that this will deform into a large perturbation of the Cauchy horizon. In these assumptions, the metric tensor is given by the \dfn{Vaidya metric}. In Eddington-Finkelstein coordinates for $v$ (i.e. $v$ is normalized so $\Omega^2(\partial_ur)^{-1} = -2$, which is possible by the Raychaudhuri equations),
\begin{align*}
    g &= -\Omega^2 ~dudv + r^2\underline g = -\Omega^2((\partial_ur)^{-1} \partial_ur + (\partial_vr)^{-1} \partial_vr) ~dudv + r^2\underline g\\
        &= 2~drdv - 2\partial_vr dv^2 + r^2\underline g.
\end{align*}
    We introduce the \dfn{modified Hawking mass} $\varpi$ defined by
    $$g^{\alpha\beta} \partial_\alpha r \partial_\beta r = 1 -2\varpi r^{-1} + \varpi^2 r^{-2}.$$
    Then $\partial_u\varpi = -2r^2 \partial_vr\Omega^{-2} T_{uu}$ and similarly for $v$. In Eddington-Finkelstein coordinates,
    $$\partial_v \varpi = r^2 T_{vv}.$$
    Thus we arrive at the Vaidya metric
\begin{align*}
    g &= 2~drdv - (1 - 2\varpi r^{-1} + Q^2r^{-2} ~dv^2 + r^2\underline g)\\
    \partial_v\varpi &= r^2T_{vv}.
\end{align*}
    We now normalize so $T_{vv}|_{\Sigma_0}(v) = \varepsilon v^{-p}$ for $p,\varepsilon$ parameters. Then $\partial_v \varpi = \varepsilon v^{-p}$. We assume $p > 1$; then $\partial_v \varpi$ is $L^1$ and $\varpi$ stays finite up to the Cauchy horizon. Now we do not have $r \to 0$at the Cauchy horizon, so the Vaidya metric cannot blow up. But $v \to \infty$ at the Cauchy horizon, and $\Ric$ blows up at the Cauchy horizon. In double-null coordinates,
    $$\Omega^2 = O(e^{-2Kv})$$
    for some $K$, when $u$ is held fixed. So if $L = \Omega^2 ~\partial_v$, $L$ ``should be" a well-behaved vector field, yet $L$ experiences exponential growth as we approach the Cauchy horizon. Now $\Ric(L, L) = 2r^{-2}v^{-p} O(e^{4Kv})$ which blows up at the Cauchy horizon.

    So $\Ric$ is highly unstable near the Cauchy horizon, even though the Vaidya metric itself is stable. But a theorem of Poisson and Israel in the early 90's shows that the blowup of $\varpi$ near the Cauchy horizon is generic, so the Einstein-Maxwell null dust system with no outgoing radiation is highly unrealistic. In fact if $T^{out}$ is nonzero on $\Sigma_0$ then $\varpi = \infty$ on the Cauchy horizon. This phenomenon is known as the \dfn{mass inflation scenario}.
\end{example}

\begin{example}
    We now treat the more realistic Einstein-Maxwell uncharged scalar field system. Here we have introduced a scalar field $\phi$ on $M$ which is governed by the wave equation $\Box_g\phi = 0$. When we say it is uncharged we mean that we do not use the Maxwell tensor to introduce curvature on the line bundle that $\phi$ maps $M$ into (so we can take that line bundle to just be $\RR$).
\begin{theorem}[Kommemi]
    Assume that $\Sigma_0$ is homeomorphic to $\RR \times S^2$, and is asymptotically flat on two ends. Then there are at most two Cauchy horizons, and there must be a complete null infinity. If the spacetime is $C^2$-extendible, then it must be extendible through the Cauchy horizon in the sense that there must be a timeline geodesic $\gamma$ in the extension which meets the Cauchy horizon on the interior of $\gamma$.
\end{theorem}
    Let us consider the initial-value problem inside the black hole region. We normalize $v$ so that $-2\partial_vr = \Omega^{-2}$ on the event horizon, and we assume that there is a $p > 1/2$ such that $\phi = O(v^{-p})$ on the event horizon. If $\phi$ is smooth, then Daferemos proved that there is a Cauchy horizon, and that $g$ is continuous up to the Cauchy horizon. Moreover, if $\phi$ is compactly supported on $\Sigma_0$ then $\partial_v\phi = O(v^{-4})$. This estimate on $\phi$ is known as the \dfn{Price law rate}.
\end{example}






\newpage
\printindex

\end{document}
