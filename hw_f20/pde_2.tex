
% --------------------------------------------------------------
% This is all preamble stuff that you don't have to worry about.
% Head down to where it says "Start here"
% --------------------------------------------------------------

\documentclass[10pt]{article}

\usepackage[margin=.7in]{geometry}
\usepackage{amsmath,amsthm,amssymb}
\usepackage{enumitem}
\usepackage{tikz-cd}
\usepackage{mathtools}
\usepackage{amsfonts}
\usepackage{listings}
\usepackage{algorithm2e}
\usepackage{verse,stmaryrd}
\usepackage{fancyvrb}

% Number systems
\newcommand{\NN}{\mathbb{N}}
\newcommand{\ZZ}{\mathbb{Z}}
\newcommand{\QQ}{\mathbb{Q}}
\newcommand{\RR}{\mathbb{R}}
\newcommand{\CC}{\mathbb{C}}
\newcommand{\PP}{\mathbb P}
\newcommand{\FF}{\mathbb F}
\newcommand{\DD}{\mathbb D}
\renewcommand{\epsilon}{\varepsilon}

\newcommand{\Aut}{\operatorname{Aut}}
\newcommand{\cl}{\operatorname{cl}}
\newcommand{\ch}{\operatorname{ch}}
\newcommand{\Con}{\operatorname{Con}}
\newcommand{\coker}{\operatorname{coker}}
\newcommand{\CVect}{\CC\operatorname{-Vect}}
\newcommand{\Cantor}{\mathcal{C}}
\newcommand{\D}{\mathcal{D}}
\newcommand{\card}{\operatorname{card}}
\newcommand{\dbar}{\overline \partial}
\newcommand{\diam}{\operatorname{diam}}
\newcommand{\dom}{\operatorname{dom}}
\newcommand{\End}{\operatorname{End}}
\DeclareMathOperator*{\esssup}{ess\,sup}
\newcommand{\GL}{\operatorname{GL}}
\newcommand{\Hom}{\operatorname{Hom}}
\newcommand{\id}{\operatorname{id}}
\newcommand{\Ind}{\operatorname{Ind}}
\newcommand{\Inn}{\operatorname{Inn}}
\newcommand{\interior}{\operatorname{int}}
\newcommand{\lcm}{\operatorname{lcm}}
\newcommand{\mesh}{\operatorname{mesh}}
\newcommand{\LL}{\mathcal L_0}
\newcommand{\Leb}{\mathcal{L}_{\text{loc}}^2}
\newcommand{\Lip}{\operatorname{Lip}}
\newcommand{\ppGL}{\operatorname{PGL}}
\newcommand{\ppic}{\vspace{35mm}}
\newcommand{\ppset}{\mathcal{P}}
\DeclareMathOperator{\proj}{proj}
\DeclareMathOperator*{\Res}{Res}
\newcommand{\Riem}{\mathcal{R}}
\newcommand{\RVect}{\RR\operatorname{-Vect}}
\newcommand{\Sch}{\mathcal{S}}
\newcommand{\SL}{\operatorname{SL}}
\newcommand{\sgn}{\operatorname{sgn}}
\newcommand{\spn}{\operatorname{span}}
\newcommand{\Spec}{\operatorname{Spec}}
\newcommand{\supp}{\operatorname{supp}}
\newcommand{\TT}{\mathcal T}
\DeclareMathOperator{\tr}{tr}

% Calculus of variations
\DeclareMathOperator{\pp}{\mathbf p}
\DeclareMathOperator{\zz}{\mathbf z}
\DeclareMathOperator{\uu}{\mathbf u}
\DeclareMathOperator{\vv}{\mathbf v}
\DeclareMathOperator{\ww}{\mathbf w}

% Categories
\newcommand{\Ab}{\mathbf{Ab}}
\newcommand{\Cat}{\mathbf{Cat}}
\newcommand{\Group}{\mathbf{Group}}
\newcommand{\Module}{\mathbf{Module}}
\newcommand{\Set}{\mathbf{Set}}
\DeclareMathOperator{\Fun}{Fun}
\DeclareMathOperator{\Iso}{Iso}

% Complex analysis
\renewcommand{\Re}{\operatorname{Re}}
\renewcommand{\Im}{\operatorname{Im}}

% Logic
\renewcommand{\iff}{\leftrightarrow}
\newcommand{\Henkin}{\operatorname{Henk}}
\newcommand{\PA}{\mathbf{PA}}
\DeclareMathOperator{\proves}{\vdash}

% Group
\DeclareMathOperator{\Gal}{Gal}
\DeclareMathOperator{\Fix}{Fix}
\DeclareMathOperator{\Out}{Out}

% Other symbols
\newcommand{\heart}{\ensuremath\heartsuit}
\newcommand{\club}{\ensuremath\clubsuit}

\DeclareMathOperator{\atanh}{atanh}

% Theorems
\theoremstyle{definition}
\newtheorem*{corollary}{Corollary}
\newtheorem*{falselemma}{Grader's ``Lemma"}
\newtheorem{exer}{Exercise}
\newtheorem{lemma}{Lemma}[exer]
\newtheorem{theorem}[lemma]{Theorem}

\def\Xint#1{\mathchoice
{\XXint\displaystyle\textstyle{#1}}%
{\XXint\textstyle\scriptstyle{#1}}%
{\XXint\scriptstyle\scriptscriptstyle{#1}}%
{\XXint\scriptscriptstyle\scriptscriptstyle{#1}}%
\!\int}
\def\XXint#1#2#3{{\setbox0=\hbox{$#1{#2#3}{\int}$ }
\vcenter{\hbox{$#2#3$ }}\kern-.6\wd0}}
\def\ddashint{\Xint=}
\def\dashint{\Xint-}

\usepackage[backend=bibtex,style=alphabetic,maxcitenames=50,maxnames=50]{biblatex}
\renewbibmacro{in:}{}
\DeclareFieldFormat{pages}{#1}

\begin{document}
\noindent
\large\textbf{PDE, HW 2} \hfill \textbf{Aidan Backus} \\

% --------------------------------------------------------------
%                         Start here
% --------------------------------------------------------------\

\begin{exer}[2.12a]
Show that the heat equation is invariant under the scaling $u(x, t) \mapsto u(\lambda x, \lambda^2 t)$.
\end{exer}

Let $v(x, t) = u(\lambda x, \lambda^2 t)$. Then
$$v_t(x, t) = \partial_t u(\lambda x, \lambda^2 t) = \lambda^2 u_t(\lambda x, \lambda^2 t) = \lambda^2 (\Delta u)(\lambda x, \lambda^2 t) = \Delta v(x, t).$$

\begin{exer}[2.12b]
Show that if $u$ solves the heat equation then $v(x, t) = x \cdot \nabla u(x, t) + 2t u_t(x, t)$ solves the heat equation.
\end{exer}

Let $u^\lambda(x, t) = u(\lambda x, \lambda^2 t)$. Then
$$\partial_\lambda u^\lambda (x, t) = x \cdot \nabla u(\lambda x, \lambda^2 t) + 2t \lambda u_t(\lambda x, \lambda^2 t).$$
Then by Exercise 2.12a, $u^\lambda$ and hence $\partial_\lambda u^\lambda$ solve the heat equation.
Setting $\lambda = 1$ we see that
$$\partial_\lambda u^\lambda |_{\lambda = 1} = v$$
also solves the heat equation.

\begin{exer}[2.18]
Suppose that $u$ solves the wave equation with initial data $u = 0$, $u_t = h$.
Show that $v = u_t$ solves the wave equation with initial data $v = h$, $v_t = 0$.
\end{exer}

Commuting mixed partials we immediately see that $v$ solves the wave equation with initial data $v = h$.
Moreover, we have initial data
$$v_t = u_{tt} = \Delta u = 0$$
since $u = 0$.

\begin{exer}[2.19a]
Show that the general solution of $u_{xy} = 0$ is
$$u(x, y) = F(x) + G(y).$$
\end{exer}

We are given
$$\frac{\partial^2 u}{\partial x\partial y} (x, y) = 0.$$
Taking the antiderivative in $y$ alone we see that
$$\frac{\partial u}{\partial x} (x, y) = F(x)$$
for some function $F$. Taking the antiderivative in $x$ alone now,
$$u(x, y) = F(x) + G(y)$$
for some function $G$.

\begin{exer}[2.19b]
Suppose that $\xi = x + t$, $\eta = x - t$. Show that $u_{tt} = u_{xx}$ iff $u_{\xi\eta} = 0$.
\end{exer}

One has
$$u_\xi = \xi \cdot \nabla u = u_x + u_t,$$
so
$$u_{\xi\eta} = \nabla u_x \cdot \eta + \nabla u_t \cdot \eta = u_{xx} - u_{xt} + u_{tx} - u_{tt} = u_{xx} - u_{tt}.$$

\begin{exer}[2.19c]
Rederive d'Alembert's formula.
\end{exer}

We must show that if $u$ solves the wave equation with initial data $u = g$, $u_t = h$, then
$$2u(x, t) = g(x + t) + g(x - t) + \int_{x - t}^{x + t} h(y) ~dy.$$
We are given
$$u(x, t) = F(x + t) + G(x - t),$$
where $F(x) + G(x) = g(x)$ and $F'(x) - G'(x) = h(x)$. Therefore
$$F(x) - G(x) = \int_0^x h(y)~dy.$$
So
$$2F(x) = g(x) + \int_0^x h(y)~dy$$
and
$$2G(x) = g(x) + \int_x^0 h(y)~dy.$$
Plugging into $2u$,
$$2u(x, t) = g(x + t) + \int_0^{x+t}h(y)~dy + g(x - t) + \int_{x - t}^0 h(y)~dy$$
which was desired.

\begin{exer}[2.19d]
When are solutions to the wave equation right-moving or left-moving?
\end{exer}

The wave equation is right-moving provided that $F = 0$, i.e. for every $x$,
$$g(x) + \int_0^x h(y)~dy= 0.$$
Similarly the equation is left-moving provided that $G = 0$, thus
$$g(x) + \int_x^0 h(y)~dy = 0.$$

\begin{exer}[2.21a]
Assume that $\mathbf E, \mathbf B$ solve Maxwell's equations. Show that $\mathbf E,\mathbf B$ solve the wave equation.
\end{exer}

Recall that if $\mathbf A$ is a vector field on a Riemannian $3$-manifold then
$$\nabla \times (\nabla \times \mathbf A) = \nabla(\nabla \cdot \mathbf A) - \Delta \mathbf A$$
as can be proven by manipulating differential forms.
Therefore
$$\mathbf E_{tt} = \partial_t(\nabla \times \mathbf B) = \nabla \times \mathbf B_t = -\nabla \times(\nabla \times \mathbf E) = -\nabla(\nabla \cdot \mathbf E) + \Delta \mathbf E = \Delta \mathbf E.$$
Replacing $\mathbf B$ with $-\mathbf E$ and vice versa, we see the same result for $\mathbf B$.

\begin{exer}[2.21b]
Assume that
$$\mathbf u_{tt} = \mu\Delta \mathbf u + (\lambda + \mu)\nabla(\nabla \cdot \mathbf u).$$
Show that $w = \nabla \cdot \mathbf u$ and $\mathbf w = \nabla \times \mathbf u$ solve wave equations of different speeds of propagation.
\end{exer}

One has
\begin{align*}
w_{tt} &= \nabla \cdot \mathbf u_{tt} = \nabla \cdot(\mu \Delta \mathbf u + (\lambda + \mu)\nabla(\nabla \cdot \mathbf u) )\\
&= \mu\Delta w + (\lambda + \mu)\Delta w = (\lambda + 2\mu)\Delta w.
\end{align*}
Recall that $f$ is a scalar field then
$$\nabla \times \nabla f = 0$$
so
\begin{align*}
\mathbf w_{tt} &= \nabla \times \mathbf u_{tt} = \nabla \times (\mu\Delta \mathbf u + (\lambda + \mu)\nabla(\nabla \cdot \mathbf u))\\
&= \mu \Delta \mathbf w + (\nabla \times \nabla)(\nabla \cdot \mathbf u) = \mu \Delta \mathbf w.
\end{align*}

\begin{exer}[Extra 1a]
Let $\Omega$ be the closed lower half-plane. Given $\xi \in \CC^2$, find a solution of the heat equation of the form
$$u(x, y, t) = e^{i\xi_1x}e^{i\xi_2y}e^{i\lambda t}.$$
For which $\xi$ is $u$ bounded in $\Omega$? For which does the solution decay as $y \to -\infty$, or as $t \to +\infty$?
Find all harmonic solutions and observe that they only depend on $z = x + iy$ if $\Re \xi_1 < 0$ and $\overline z $ if $\Re \xi_1 > 0$.
\end{exer}

Suppose that $\Delta u = \partial_t u$. Now
$$\frac{\partial u}{\partial t} (x, y, t) = i\lambda u(x, y, t)$$
while
$$\frac{\partial^2 u}{\partial x^2} (x, y, t) = -\xi_1^2 u(x, y, t)$$
so that
$$\Delta u(x, y, t) = -(\xi_1^2 + \xi_2^2) u(x, y, t).$$
That means that
$$\xi_1^2 + \xi_2^2 + i\lambda = 0$$
determines $\lambda$, thus
$$\lambda = i(\xi_1^2 + \xi_2^2).$$

Now $u$ is bounded in $\Omega$ iff $\Re i\xi_2 \leq 0$ and $\Re i\xi_1 = 0$, thus $\Im \xi_1 \geq 0$ and $\Im \xi_2 = 0$.
Moreover $u$ decays at $y = -\infty$ iff $\Re i\xi_2 < 0$, thus $\Im \xi_2 > 0$; $u$ decays at $t = +\infty$ iff $\Re i\lambda < 0$, thus $\Re(\xi_1^2 + \xi_2^2) > 0$.

Suppose that $u$ is harmonic. Then $\partial_t u = 0$, since $u$ is a steady state of the heat equation.
So $i\lambda = 0$, which implies that
$$\xi_1^2 + \xi_2^2 = 0.$$
Conversely, any such $\xi$ on that variety gives a harmonic $u$.

Suppose that $u$ is harmonic. Then $\lambda = 0$, so the Cauchy-Riemann derivative
$$\frac{\partial u}{\partial \overline z}(x, y, t) = \left(\frac{\partial}{\partial x} + i\frac{\partial}{\partial y}\right)u(x, y, t) = \frac{1}{2}(i\xi_1 - \xi_2)e^{i(\xi_1x + \xi_2y)}$$
vanishes exactly if $i\xi_1 = \xi_2$.
If $\Re \xi_1 < 0$, then we can choose a branch of the square root such that $\xi_1^2 + \xi_2^2 = 0$ implies $i\xi_1 = \xi_2$.
Similarly, if $\Re \xi_2 > 0$, then we can choose a branch of the square root $i\xi_1 = -\xi_2$, and the conjugate of the Cauchy-Riemann derivative
$$\frac{\partial u}{\partial z}(x, y, t) = \left(\frac{\partial}{\partial x} - i\frac{\partial}{\partial y}\right)u(x, y, t) = \frac{1}{2}(i\xi_1 + \xi_2)e^{i(\xi_1x + \xi_2y)}$$
vanishes.
So if $u$ is harmonic and $\xi_1$ is not imaginary, then $u$ is either holomorphic or antiholomorphic, depending on the sign of $u$.

\begin{exer}[Extra 1b]
Solve the Dirichlet problem for the heat equation, $u(x, 0, t) = g(x, t)$, $u(x, -\infty, 0) = 0$, on $\Omega$, provided $g$ is smooth enough and periodic in time of period $T$.
If $g$ is real, give an explicitly real formula for $u$.
\end{exer}

Using the Fourier transform in $\RR \times (\RR/T\ZZ)$, we obtain Fourier coefficients $\hat g_k$ such that
\begin{equation}
\label{fourier inversion formula}
g(x, t) = \frac{1}{\sqrt{2\pi}} \sum_{k=-\infty}^\infty \int_{-\infty}^\infty \hat g_k(\xi) e^{i\xi x} e^{2\pi i kt/T}~d\xi.
\end{equation}
We first treat the case that the Fourier transform of $g$ is a Dirac mass (i.e. $g$ is a plane wave).

\begin{lemma}
Suppose that $\hat g_\ell(\xi) = \sqrt{2\pi}\delta(\xi - \eta)$, where $\delta$ is the Dirac mass, and $\hat g_k = 0$ if $\ell \neq k$.
Then
$$u(x, y, t) = e^{i\eta x} e^{-y\sqrt{\eta^2 + 2\pi i\ell/T}} e^{2\pi i\ell t/T}.$$
\end{lemma}
\begin{proof}
One has
$$g(x, t) = e^{i\eta x}e^{2\pi i\ell t/T}.$$
Then
$$e^{i\xi_1x}e^{i\lambda t} = e^{i\eta x} e^{2\pi i\ell t/T}$$
so that $\eta = \xi_1$ and $\lambda = 2\pi \ell/T$.
In particular,
$$\xi_2^2 = -\eta^2 - 2\pi i\frac{\ell}{T}.$$
In order to get good decay in $y$, we require that $\Im \xi_2 > 0$; there is a unique branch of the square root $\sqrt\cdot$ which satisfies this property as long as $-\eta^2 - 2\pi i\ell/T$ is positive and real.
But that would imply that $\eta$ is imaginary, which is nonsense.
\end{proof}

Now we treat the general case. Suppose that $g$ satisfies (\ref{fourier inversion formula}).
Then
$$u(x, y, t) = \frac{1}{\sqrt{2\pi}} \sum_{k=-\infty}^\infty \int_{-\infty}^\infty \hat g_k(\xi) e^{i\xi x} e^{-y\sqrt{\xi^2 + 2\pi ik/T}} e^{2\pi ik t/T}~d\xi.$$
Here $\sqrt\cdot$ is the unique branch of the square root which maps $\CC \setminus (-\infty, 0]$ into the right half-plane.
Provided that $g$ has enough regularity, this follows from the lemma.

Now if $g$, in addition to having good regularity properties, is real, then $u$ must be real as well.
Let $g = g^0 + g^1$ be the decomposition of $g$ into its even and odd parts.
Then $\hat g_k = \hat g_k^0 + \hat g_k^1$, and $\hat g_k^0$ is even and imaginary, $\hat g_k^1$ is odd and real.
Let $\alpha + i\beta = \sqrt{\xi^2 + 2\pi ik/T}$.
Then
$$u(x, y, t) = \frac{1}{\sqrt{2\pi}} \sum_{k=-\infty}^\infty \int_{-\infty}^\infty e^{-\alpha y} (\hat g_k^0(\xi) + \hat g_k^1(\xi))
 \left(\cos\left(\xi x + \beta y + 2\pi\frac{kt}{T}\right) +i \sin\left(\xi x + \beta y + 2\pi\frac{kt}{T}\right)\right) ~d\xi.$$
Since the decomposition $L^2 = L^2_0 \oplus L^2_1$ into even and odd parts is an orthogonal decomposition ($L^2_0 \perp L^2_1$), lots of terms cancel out, and
$$u(x, y, t) = \frac{1}{\sqrt{2\pi}} \sum_{k=-\infty}^\infty \int_{-\infty}^\infty
e^{-\alpha y} \hat g_k^0(\xi) \cos\left(\xi x + \beta y + 2\pi\frac{kt}{T}\right) - ie^{-\alpha y} \hat g_k^1(\xi) \sin\left(\xi x + \beta y + 2\pi\frac{kt}{T}\right) ~d\xi.$$
Since $\hat g_k^1$ is imaginary and $\alpha,\beta$ are real functions of $k,\xi$, this decomposition implies that $u$ is real.

\begin{exer}[Extra 2a]
Find the Green function of the first octant using Couloumb's law.
\end{exer}

Let $U$ be the first octant. We must find a harmonic error term $\varphi_x(y)$ such that
$$\varphi_x(y) = \frac{1}{4\pi|x - y|}$$
on $\partial U$. To this end we let $X_i$ denote the $i$th axis, and $R_i$ reflection across $X_i$.
Then we let $R_\alpha = \prod_k R_{\alpha_k}$ for a multiindex $\alpha$.

Consider the charge density
$$\rho_x(y) = \sum_{i=1}^3 \delta(R_ix - y) - \sum_{1 \leq i < j \leq 3} \delta(R_{ij}x - y) + \delta(R_{123}x - y).$$
Here $\delta$ is a Dirac mass, so $\rho_x$ consists of point charges at $R_ix$, $R_{ij}x$, and $R_{123}x$, where $i < j$.
Let $\varphi_x$ denote the electric potential given $\rho_x$, thus by Couloumb's law,
$$\varphi_x(y) = \sum_{i=1}^3 \Phi(R_ix - y) - \sum_{1 \leq i < j \leq 3} \Phi(R_{ij}x - y) + \Phi(R_{123}x - y).$$
Clearly $\varphi_x$ is harmonic on $U$.

We must now compute $\varphi_x|\partial U$. Since $R_\alpha$ is a product of reflections, $R_\alpha(R_\alpha x - y) = x - R_\alpha y$.
Moreover, $R_\alpha$ preserves norms. Therefore $\Phi(R_\alpha x - y) = \Phi(x - R_\alpha y)$, so $\varphi_x(y) = \varphi_y(x)$.
Therefore we must show that if $x \in \partial U$ then for every $y \in \RR^3$, $\varphi_y(x) = \Phi(x - y)$.

Since $x \in \partial U$, there is an $i$ such that $x \in X_i$.
Then $R_ix = x$, $R_{ij}x = R_jx$, and $R_{ijk}x = R_{jk}x$.
So, if $\ell,m$ are the indices $\in \{1, 2, 3\}$ not equal to $i$, then
\begin{align*}
\Phi(x - y) - \varphi_x(y) &= \Phi(x - y) - \varphi_y(x)\\
&= \Phi(x - y) - \Phi(R_ix - y) + \sum_{\substack{1\leq j \leq 3\\ i \neq j}} \Phi(R_jx - y) - \\
&\qquad\sum_{1 \leq j < k \leq 3} \Phi(R_{jk}x - y) + \Phi(R_{i\ell m}x - y)\\
&= \Phi(R_\ell x - y) + \Phi(R_m x - y) - \Phi(R_\ell x - y) - \Phi(R_m x - y) -
\\&\qquad \Phi(R_{\ell m} x - y) + \Phi(R_{\ell m}x - y)\\
&= 0
\end{align*}
which was desired.

\begin{exer}[Extra 2b]
Let $g \in C^2_c(\RR_+^2)$, $\RR^2_+$ the first quadrant. Extend $g$ to a doubly odd function $G$ on $\RR^2$. Solve $\Delta U = G$, with $U \to 0$ at $\infty$.
\end{exer}

Set
$$U = \Phi * G.$$
Then clearly $\Delta U = G$.
Moreover, since $G$ is doubly odd with compact support, $\int_{\RR^2} G = 0$, so
$$U(x) = \frac{1}{2\pi} \log |x| \int_{\RR^2} G(y)~dy - \frac{1}{2\pi} \int_{\RR^2} \log|x - y| G(y)~dy.$$
Since $g$ has compact support, so does $G$, say $\supp G \subset B(0, R)$.
Now
$$\lim_{x \to \infty} \log\left|\frac{|x|}{|x-y|}\right| = 0$$
and since $|y| \leq R$, this estimate is uniform, thus for every $\varepsilon > 0$ there is a $M> 0$ such that if $|x| > M$ then for every $y \in \supp G$, $\left|\log\left||x|/|x-y|\right|\right| < \varepsilon$. Therefore if $|x| > M$,
\begin{align*}
|U(x)| &\leq \frac{1}{2\pi} \int_{\RR^2} \left|\log|x| - \log|x-y|\right| |G(y)| ~dy\\
&\leq \frac{1}{2\pi} \int_{\RR^2} \log\left|\frac{|x|}{|x-y|}\right| |G(y)| ~dy\\
&< \frac{\varepsilon}{2\pi} ||G||_{L^1(\RR^2)}.
\end{align*}

\begin{exer}[Extra 2c]
Compute $U(0, x_2)$ and $U(x_1, 0)$.
\end{exer}

We claim $U(0, x_2) = U(x_1, 0) = 0$, and by a symmetry argument it suffices to show that $U(0, x_2) = 0$.
Let $R$ be the reflection across the vertical axis; then $G(x) + G(Rx) = 0$ since $G$ is doubly odd.
Suppose that $x_1 = 0$. Then
$$\log |x - y| = \frac{1}{2} \log\left(|y_1|^2 + |x_2 - y_2|^2\right).$$
So
\begin{align*}
U(0, x_2) &= -\frac{1}{2\pi} \int_{\RR^2} \log|x-y| G(y)~dy
\\&= -\frac{1}{4\pi} \int_{\RR^2} \log\left(|y_1|^2 + |x_2 - y_2|^2\right) G(y)~dy
\\&= -\frac{1}{4\pi} \int_{x_1 > 0} \log\left(|y_1|^2 + |x_2 - y_2|^2\right) (G(y) + G(Ry)) ~dy = 0.
\end{align*}

\begin{exer}[Extra 2d]
Solve $\Delta u = g$ on $\RR^2_+$, $u = 0$ on $\partial \RR^2_+$, $u \to 0$ at $\infty$.
\end{exer}

By the above, if $U = \Phi * G$ then $U = \Phi * G$ restricts to a function $u$ with the desired properties.
Here $G$ is the doubly odd extension of $g$, as before.

\begin{exer}[Extra 2e]
Compute the Green function of $\RR^2_+$. What is its behavior near $0$?
\end{exer}

We use Couloumb's law again.
Let $R_1,R_2,R_{12}$ be reflections as in Extra 2a.
Consider the charge density
$$\sigma_x(y) = \delta(R_1 x - y) + \delta(R_2 x - y) - \delta(R_{12} x - y).$$
Then by Couloumb's law in two dimensions,
$$\varphi_x(y) = \Phi(R_1 x - y) + \Phi(R_2 x - y) - \Phi(R_{12} x - y).$$
By a similar argument to Extra 2a, we see that $\varphi_x(y) = \Phi(x - y)$ if $x \in \partial \RR^2_+$ and $y \in \RR^2$, and also that $\varphi_x$ is harmonic.
So $G(x, y) = \Phi(x - y) - \varphi_x(y)$ is the Green function.
As $(x, y) \to 0$ in $\RR^4$, $G(x, y) \to 0$, so $G(x, y)$ extends by continuity to $G(0, 0) = 0$.

\begin{exer}[Extra 2f]
Solve the Neumann problem $\Delta u = g$, $\partial_\nu u = 0$ on $\partial \RR^2_+$, and $|u(x)| \lesssim \log|x|$.
\end{exer}

Following Evans' derivation of the definition of the Green function, we suppose that $x \in \RR_+^2$ and choose $\varepsilon > 0$ sufficiently small, and set $U_\varepsilon = \RR^2_+ \setminus B(x, \varepsilon)$. Then
$$\int_{U_\varepsilon} u(y)\Delta \Phi(x - y) - \Phi(y - x)\Delta u(y)~dy = \int_{\partial U_\varepsilon} u(y) \frac{\partial \Phi}{\partial y}(y - x) - \Phi(y - x) \frac{\partial u}{\partial \nu}(y)~dS(y).$$
But $\Phi$ is harmonic on $U_\varepsilon$, so the $\Delta \Phi$ term drops out. Moreover,
$$\left|\int_{\partial B(x, \varepsilon)} \Phi(y - x) \frac{\partial u}{\partial \nu}(y)~dS(y)\right| \lesssim \varepsilon$$
and
$$\lim_{\varepsilon \to 0}\int_{\partial B(x, \varepsilon)} u(y) \frac{\partial \Phi}{\partial \nu}(y - x)~dS(y) = \lim_{\varepsilon \to 0}\dashint_{\partial B(x, \varepsilon)} u~dS = u(x).$$
Summing up,
$$u(x) = \int_{\partial \RR_+^2} \Phi(y - x) \frac{\partial u}{\partial \nu}(y) - u(y) \frac{\partial \Phi}{\partial \nu}(y - x)~dS(y) - \int_{\RR_+^2} \Phi(y - x)\Delta u(y)~dy.$$
Since we are given $\Delta u = g$ and $\partial_\nu u = 0$,
$$u(x) = -\int_{\partial \RR_+^2} u(y) \frac{\partial \Phi}{\partial \nu}(y - x)~dS(y) - \int_{\RR_+^2} \Phi(y-x) g(y)~dy.$$
Therefore if we can find a harmonic error term $\varphi_x(y)$ and consider the integral kernel $K(x, y) = \Phi(x, y) - \varphi_x(y)$, with
$$\frac{\partial \Phi}{\partial \nu}(y-x) = \frac{\partial \varphi_x}{\partial \nu}(y),$$
then
\begin{align*}\int_{\RR_+^2} \varphi^x(y) \Delta u(y)~dy &= -\int_{\partial \RR_+^2} u(y) \frac{\partial \varphi^x}{\partial \nu}(y) - \varphi^x(y) \frac{\partial u}{\partial \nu}(y) ~dS(y)\\
&= -\int_{\partial \RR^2_+} u(y) \frac{\partial \Phi}{\partial \nu}(y-x)~dS(y).
\end{align*}
Adding up the above equations,
$$u(x) = \int_{\RR_+^2} \varphi^x(y) g(y) - \Phi(y - x) g(y)~dy = -\int_{\RR_+^2} K(x, y)g(y)~dy.$$
Now we claim that, in fact, $K = -G$. We already showed that
$$G(x, y) = \Phi(x - y) - \Phi(R_1x - y) - \Phi(R_2x - y) + \Phi(R_{12}x - y).$$
If $x = (x_1, 0)$ then
$$\frac{\partial G}{\partial \nu}(x, y) = \frac{\partial \Phi}{\partial \nu}(x - y) - \frac{\partial \Phi}{\partial \nu}(-x-y) - \frac{\partial \Phi}{\partial \nu}(x - y) + \frac{\partial \Phi}{\partial \nu}(-x-y) = 0$$
which was desired.
So one has
$$u(x) = \int_{\RR^2_+} G(x, y) g(y)~dy.$$

\begin{exer}[Extra 3]
Suppose that $u_j$ are smooth solutions of the $1$-dimensional heat equation with initial data $f_j$. Show that
$$v(x_1, \dots, x_d, t) = \prod_{j=1}^d u_j(x_j, t)$$
is a solution of the $d$-dimensional heat equation with initial data $x \mapsto \prod_j f_j(x_j)$.
What happens to the fundamental solution?
\end{exer}

This follows by induction on $d$. If $d = 1$, this is clear.
Suppose that $v$ is as written, $v$ is harmonic with the desired initial data, and
$$w(x_1, \dots, x_d, x_{d+1}, t) = \prod_{j=1}^{d+1} u_j(x_j, t).$$
Then
$$\frac{\partial w}{\partial t} = \frac{\partial v}{\partial t}u_{d+1} + v\frac{\partial u_{d+1}}{\partial t} = u_{d+1}\Delta v + v\Delta u_{d+1}$$
by our induction hypothesis.
But
$$\Delta w = u_{d+1} \Delta v + v \Delta u_{d+1} + \langle \nabla v, \nabla u_{d+1}\rangle = \frac{\partial w}{\partial t}$$
since $v$ does not depend on $x_{d+1}$ while $u_{d+1}$ only depends on $x_{d+1}$, so that $\nabla v \perp \nabla u_{d+1}$.
Plugging in $t = 0$,
$$w(x, 0) = v(x, 0) u_{d+1}(x_{d+1}, 0) = f_{d+1}(x_{d+1})\prod_{j=1}^d f_j(x_j) = \prod_{j=1}^{d+1} f_j(x_j).$$


\end{document}
