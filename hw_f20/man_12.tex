
% --------------------------------------------------------------
% This is all preamble stuff that you don't have to worry about.
% Head down to where it says "Start here"
% --------------------------------------------------------------

\documentclass[10pt]{article}

\usepackage[margin=.7in]{geometry}
\usepackage{amsmath,amsthm,amssymb}
\usepackage{enumitem}
\usepackage{tikz-cd}
\usepackage{mathtools}
\usepackage{amsfonts}
\usepackage{listings}
\usepackage{algorithm2e}
\usepackage{verse,stmaryrd}
\usepackage{fancyvrb}

% Number systems
\newcommand{\NN}{\mathbb{N}}
\newcommand{\ZZ}{\mathbb{Z}}
\newcommand{\QQ}{\mathbb{Q}}
\newcommand{\RR}{\mathbb{R}}
\newcommand{\CC}{\mathbb{C}}
\newcommand{\PP}{\mathbb P}
\newcommand{\FF}{\mathbb F}
\newcommand{\DD}{\mathbb D}
\renewcommand{\epsilon}{\varepsilon}

\newcommand{\Aut}{\operatorname{Aut}}
\newcommand{\cl}{\operatorname{cl}}
\newcommand{\ch}{\operatorname{ch}}
\newcommand{\Con}{\operatorname{Con}}
\newcommand{\coker}{\operatorname{coker}}
\newcommand{\CVect}{\CC\operatorname{-Vect}}
\newcommand{\Cantor}{\mathcal{C}}
\newcommand{\D}{\mathcal{D}}
\newcommand{\card}{\operatorname{card}}
\newcommand{\dbar}{\overline \partial}
\newcommand{\diam}{\operatorname{diam}}
\newcommand{\dom}{\operatorname{dom}}
\newcommand{\End}{\operatorname{End}}
\DeclareMathOperator*{\esssup}{ess\,sup}
\newcommand{\Hess}{\operatorname{Hess}}
\newcommand{\Hom}{\operatorname{Hom}}
\newcommand{\id}{\operatorname{id}}
\newcommand{\Ind}{\operatorname{Ind}}
\newcommand{\Inn}{\operatorname{Inn}}
\newcommand{\interior}{\operatorname{int}}
\newcommand{\lcm}{\operatorname{lcm}}
\newcommand{\mesh}{\operatorname{mesh}}
\newcommand{\LL}{\mathcal L_0}
\newcommand{\Leb}{\mathcal{L}_{\text{loc}}^2}
\newcommand{\Lip}{\operatorname{Lip}}
\newcommand{\ppic}{\vspace{35mm}}
\newcommand{\ppset}{\mathcal{P}}
\DeclareMathOperator{\proj}{proj}
\DeclareMathOperator*{\Res}{Res}
\newcommand{\Riem}{\mathcal{R}}
\newcommand{\RVect}{\RR\operatorname{-Vect}}
\newcommand{\Sch}{\mathcal{S}}
\newcommand{\sgn}{\operatorname{sgn}}
\newcommand{\spn}{\operatorname{span}}
\newcommand{\Spec}{\operatorname{Spec}}
\newcommand{\supp}{\operatorname{supp}}
\newcommand{\TT}{\mathcal T}
\DeclareMathOperator{\tr}{tr}

% Calculus of variations
\DeclareMathOperator{\pp}{\mathbf p}
\DeclareMathOperator{\zz}{\mathbf z}
\DeclareMathOperator{\uu}{\mathbf u}
\DeclareMathOperator{\vv}{\mathbf v}
\DeclareMathOperator{\ww}{\mathbf w}

% Categories
\newcommand{\Ab}{\mathbf{Ab}}
\newcommand{\Cat}{\mathbf{Cat}}
\newcommand{\Group}{\mathbf{Group}}
\newcommand{\Module}{\mathbf{Module}}
\newcommand{\Set}{\mathbf{Set}}
\DeclareMathOperator{\Fun}{Fun}
\DeclareMathOperator{\Lie}{Lie}
\DeclareMathOperator{\Iso}{Iso}

% Complex analysis
\renewcommand{\Re}{\operatorname{Re}}
\renewcommand{\Im}{\operatorname{Im}}

% Logic
\renewcommand{\iff}{\leftrightarrow}
\newcommand{\Henkin}{\operatorname{Henk}}
\newcommand{\PA}{\mathbf{PA}}
\DeclareMathOperator{\proves}{\vdash}

% Group
\DeclareMathOperator{\Gal}{Gal}
\DeclareMathOperator{\Fix}{Fix}
\DeclareMathOperator{\Out}{Out}

\DeclareMathOperator{\Diffeo}{Diffeo}

\DeclareMathOperator{\ad}{ad}
\DeclareMathOperator{\Ad}{Ad}
\newcommand{\GL}{\operatorname{GL}}
\newcommand{\ppGL}{\operatorname{PGL}}
\newcommand{\SL}{\operatorname{SL}}
\newcommand{\SO}{\operatorname{SO}}
\newcommand{\iprod}{\mathbin{\lrcorner}}


% Other symbols
\newcommand{\heart}{\ensuremath\heartsuit}
\newcommand{\club}{\ensuremath\clubsuit}

\DeclareMathOperator{\atanh}{atanh}
\DeclareMathOperator{\codim}{codim}

% Theorems
\theoremstyle{definition}
\newtheorem*{corollary}{Corollary}
\newtheorem*{falselemma}{Grader's ``Lemma"}
\newtheorem{exer}{Exercise}
\newtheorem{lemma}{Lemma}[exer]
\newtheorem{theorem}[lemma]{Theorem}

\def\Xint#1{\mathchoice
{\XXint\displaystyle\textstyle{#1}}%
{\XXint\textstyle\scriptstyle{#1}}%
{\XXint\scriptstyle\scriptscriptstyle{#1}}%
{\XXint\scriptscriptstyle\scriptscriptstyle{#1}}%
\!\int}
\def\XXint#1#2#3{{\setbox0=\hbox{$#1{#2#3}{\int}$ }
\vcenter{\hbox{$#2#3$ }}\kern-.6\wd0}}
\def\ddashint{\Xint=}
\def\dashint{\Xint-}

\usepackage[backend=bibtex,style=alphabetic,maxcitenames=50,maxnames=50]{biblatex}
\renewbibmacro{in:}{}
\DeclareFieldFormat{pages}{#1}

\begin{document}
\noindent
\large\textbf{Manifolds, HW 12} \hfill \textbf{Aidan Backus} \\

% --------------------------------------------------------------
%                         Start here
% --------------------------------------------------------------\

\begin{exer}[20.9]
Show that
$$\exp(tX)\exp(tY) = \exp\left(t(X + Y)) + \frac{t^2}{2} [X, Y] + O(t^3)\right).$$
\end{exer}

If $G$ is a linear group then
\begin{align*}
\exp(tX)\exp(tY) &= \left(1 + tX + \frac{t^2}{2}X + O(t^3)\right)\left(1 + tY + \frac{t^2}{2}Y + O(t^3)\right)\\
&= 1 + t(X + Y) + \frac{t^2}{2}(X^2 + Y^2 + 2XY) + O(t^3).
\end{align*}
On the other hand,
$$\exp\left(t(X + Y) + \frac{t^2}{2} f(X, Y)\right) = 1 + t(X + Y) + \frac{t^2}{2} f(X, Y) + \frac{t^2}{2} (X^2 + XY + YX + Y^2) + O(t^3)$$
whenever $f$ is a smooth function.
In order that those be equal one must have $f(X, Y) = XY - YX$.

If $G$ is any Lie group, then there is a linear group $H$ such that $\Lie G = \Lie H$, by Ado's theorem.
In particular, if $|t|$ is small enough, then $\exp(tX)\exp(tY)$ is in a neighborhood of $1_G$ that can be identified with a neighborhood of $1_H$, and the above proof remains valid for $G$.

\begin{exer}[20.11]
Let $G, H$ be Lie groups. Show that every continuous morphism $\gamma: \RR \to H$ is smooth.
Show that every continuous morphism $F: G \to H$ is smooth.
Show that a Lie group's smooth structure is determined by its topological group structure.
\end{exer}

First, when $G = \RR$, let $V \subseteq \Lie H$ be a neighborhood of $0$ such that $\exp|2V$ is an embedding.
Let $t_0 \in \RR$ be such that if $|t| \leq t_0$ then $\gamma(t) \in \exp V$.
Then let $X_0 = \exp^{-1}(\gamma(t_0)) \in \Lie H$.

Let $q = m/2^n$ be a dyadic rational $\leq 1$. We claim that $\gamma(qt_0) = \exp(qX_0)$.
This follows by induction on $n$, and is a tautology when $n = 0$.
Assume that it holds for $n$ but not $n + 1$, so there is an $m$ such that $\gamma(mt_0/2^n) = \exp(mX_0/2^n)$ but $\gamma(mt_0/2^{n+1}) \neq \exp(mX_0/2^{n+1})$.
After rescaling we can assume $m = 1$, $n = 0$, thus $\gamma(t_0) = \exp X_0$ but $\gamma(t_0/2) \neq \exp(X_0/2)$.
Then $\gamma(t_0) \neq \gamma(t_0/2)^2 = \exp(X_0)$, a contradiction.

Now by continuity it follows that for any $|t| \leq t_0$, $\gamma(t) = \exp(tX_0)$, so $\gamma$ is the one-parameter subgroup generated by $X_0$, and hence $\gamma$ is smooth.

Now we claim that there is a linear $\varphi: \Lie G \to \Lie H$ which is conjugated by $\exp$ to $F$.
Indeed, if $\gamma$ is a one-parameter subgroup of $\Lie G$, $F \circ \gamma$ is a one-parameter subgroup of $H$, by the previous part.
Suppose that $X \in \Lie G$ generates $\gamma$; then let $Y$ generate $F \circ \gamma$, and set $\varphi(X) = Y$.
Then $\exp(\gamma(\varphi(X))) = F(\gamma(1)) = F(\exp X)$, so indeed $\varphi$ exists.
One-parameter subgroups are continuous in their generators, and $F$ is continuous, so $\varphi$ is continuous.
Suppose that $X_i$ generates $\gamma_i$; then $\varphi(X_1 + X_2)$ generates $F(\gamma_1 \gamma_2) = F(\gamma_1) F(\gamma_2)$, which is generated by $\varphi(X_1) + \varphi(X_2)$, so $\varphi$ is additive.
A continuous additive function is linear, by a modification of the above dyadic rationals argument, so $\varphi$ is linear.
So $\varphi = \Lie F$, so $F$ is smooth.

Now suppose that $F: G \to H$ is an isomorphism of topological groups.
We just showed that $F$ is a diffeomorphism, so $F$ is an isomorphism of Lie groups.

\begin{exer}[20.20]
Let $G$ be a connected Lie group.
Show that the center of $G$ is the kernel of $\Ad$.
\end{exer}

Let $\Psi: G \to \Inn(G)$ be the conjugation action.
Then $\Ad(g) = \Lie \Psi(g)$ where $\Lie$ is the Lie functor on connected Lie groups.
If $g \in Z(G)$ then $\Psi(g) = 1$ so $\Lie \Psi(g) = 0$.
Conversely if $\Lie \Psi(g) = 0$ then $\Psi(g) = 1$ so $g \in Z(G)$, since $\Lie$ is faithful.

\begin{exer}[20.22]
Let $G$ be a connected Lie group.
Show that $Z(\Lie G) = \Lie Z(G)$.
\end{exer}

We showed $Z(G) = \ker \Ad$. Since $\Lie$ is a faithful functor, it commutes with $\ker$, thus $\Lie Z(G) = \ker \Lie \Ad = \ker \ad = Z(\Lie G)$.





\end{document}
