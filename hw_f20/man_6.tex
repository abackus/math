
% --------------------------------------------------------------
% This is all preamble stuff that you don't have to worry about.
% Head down to where it says "Start here"
% --------------------------------------------------------------

\documentclass[10pt]{article}

\usepackage[margin=.7in]{geometry}
\usepackage{amsmath,amsthm,amssymb}
\usepackage{enumitem}
\usepackage{tikz-cd}
\usepackage{mathtools}
\usepackage{amsfonts}
\usepackage{listings}
\usepackage{algorithm2e}
\usepackage{verse,stmaryrd}
\usepackage{fancyvrb}

% Number systems
\newcommand{\NN}{\mathbb{N}}
\newcommand{\ZZ}{\mathbb{Z}}
\newcommand{\QQ}{\mathbb{Q}}
\newcommand{\RR}{\mathbb{R}}
\newcommand{\CC}{\mathbb{C}}
\newcommand{\PP}{\mathbb P}
\newcommand{\FF}{\mathbb F}
\newcommand{\DD}{\mathbb D}
\renewcommand{\epsilon}{\varepsilon}

\newcommand{\Aut}{\operatorname{Aut}}
\newcommand{\cl}{\operatorname{cl}}
\newcommand{\ch}{\operatorname{ch}}
\newcommand{\Con}{\operatorname{Con}}
\newcommand{\coker}{\operatorname{coker}}
\newcommand{\CVect}{\CC\operatorname{-Vect}}
\newcommand{\Cantor}{\mathcal{C}}
\newcommand{\D}{\mathcal{D}}
\newcommand{\card}{\operatorname{card}}
\newcommand{\dbar}{\overline \partial}
\newcommand{\diam}{\operatorname{diam}}
\newcommand{\dom}{\operatorname{dom}}
\newcommand{\End}{\operatorname{End}}
\DeclareMathOperator*{\esssup}{ess\,sup}
\newcommand{\Hess}{\operatorname{Hess}}
\newcommand{\Hom}{\operatorname{Hom}}
\newcommand{\id}{\operatorname{id}}
\newcommand{\Ind}{\operatorname{Ind}}
\newcommand{\Inn}{\operatorname{Inn}}
\newcommand{\interior}{\operatorname{int}}
\newcommand{\lcm}{\operatorname{lcm}}
\newcommand{\mesh}{\operatorname{mesh}}
\newcommand{\LL}{\mathcal L_0}
\newcommand{\Leb}{\mathcal{L}_{\text{loc}}^2}
\newcommand{\Lip}{\operatorname{Lip}}
\newcommand{\ppic}{\vspace{35mm}}
\newcommand{\ppset}{\mathcal{P}}
\DeclareMathOperator{\proj}{proj}
\DeclareMathOperator*{\Res}{Res}
\newcommand{\Riem}{\mathcal{R}}
\newcommand{\RVect}{\RR\operatorname{-Vect}}
\newcommand{\Sch}{\mathcal{S}}
\newcommand{\sgn}{\operatorname{sgn}}
\newcommand{\spn}{\operatorname{span}}
\newcommand{\Spec}{\operatorname{Spec}}
\newcommand{\supp}{\operatorname{supp}}
\newcommand{\TT}{\mathcal T}
\DeclareMathOperator{\tr}{tr}

% Calculus of variations
\DeclareMathOperator{\pp}{\mathbf p}
\DeclareMathOperator{\zz}{\mathbf z}
\DeclareMathOperator{\uu}{\mathbf u}
\DeclareMathOperator{\vv}{\mathbf v}
\DeclareMathOperator{\ww}{\mathbf w}

% Categories
\newcommand{\Ab}{\mathbf{Ab}}
\newcommand{\Cat}{\mathbf{Cat}}
\newcommand{\Group}{\mathbf{Group}}
\newcommand{\Module}{\mathbf{Module}}
\newcommand{\Set}{\mathbf{Set}}
\DeclareMathOperator{\Fun}{Fun}
\DeclareMathOperator{\Iso}{Iso}

% Complex analysis
\renewcommand{\Re}{\operatorname{Re}}
\renewcommand{\Im}{\operatorname{Im}}

% Logic
\renewcommand{\iff}{\leftrightarrow}
\newcommand{\Henkin}{\operatorname{Henk}}
\newcommand{\PA}{\mathbf{PA}}
\DeclareMathOperator{\proves}{\vdash}

% Group
\DeclareMathOperator{\Gal}{Gal}
\DeclareMathOperator{\Fix}{Fix}
\DeclareMathOperator{\Lie}{Lie}
\DeclareMathOperator{\Out}{Out}

\DeclareMathOperator{\Diffeo}{Diffeo}

\newcommand{\GL}{\operatorname{GL}}
\newcommand{\ppGL}{\operatorname{PGL}}
\newcommand{\SL}{\operatorname{SL}}
\newcommand{\SO}{\operatorname{SO}}

% Other symbols
\newcommand{\heart}{\ensuremath\heartsuit}
\newcommand{\club}{\ensuremath\clubsuit}

\DeclareMathOperator{\atanh}{atanh}

% Theorems
\theoremstyle{definition}
\newtheorem*{corollary}{Corollary}
\newtheorem*{falselemma}{Grader's ``Lemma"}
\newtheorem{exer}{Exercise}
\newtheorem{lemma}{Lemma}[exer]
\newtheorem{theorem}[lemma]{Theorem}

\def\Xint#1{\mathchoice
{\XXint\displaystyle\textstyle{#1}}%
{\XXint\textstyle\scriptstyle{#1}}%
{\XXint\scriptstyle\scriptscriptstyle{#1}}%
{\XXint\scriptscriptstyle\scriptscriptstyle{#1}}%
\!\int}
\def\XXint#1#2#3{{\setbox0=\hbox{$#1{#2#3}{\int}$ }
\vcenter{\hbox{$#2#3$ }}\kern-.6\wd0}}
\def\ddashint{\Xint=}
\def\dashint{\Xint-}

\usepackage[backend=bibtex,style=alphabetic,maxcitenames=50,maxnames=50]{biblatex}
\renewbibmacro{in:}{}
\DeclareFieldFormat{pages}{#1}

\begin{document}
\noindent
\large\textbf{Manifolds, HW 6} \hfill \textbf{Aidan Backus} \\

% --------------------------------------------------------------
%                         Start here
% --------------------------------------------------------------\

\begin{exer}[8.25]
Show that if $G$ is abelian then $\Lie G$ is abelian.
\end{exer}

Since $G$ is abelian, $i(xy) = i(y)i(x) = i(x)i(y)$, thus $i$ is a morphism of Lie groups, and so induces a morphism of Lie algebras $i_*$.
But $i_*(X) = di(X) = -X$, so if $X, Y$ are invariant vector fields, then
$$[X, Y] = [-X, -Y] = [i_*X, i_*Y] = i_*[X, Y] = -[X, Y].$$
Adding $[X, Y]$ to both sides we see that $[X, Y] = 0$.

\begin{exer}[8.27]
Let $F: G \to H$ be a morphism of Lie groups.
Show that if $F$ is a local diffeomorphism then $F_*: \Lie G \to \Lie H$ is an isomorphism of Lie algebras.
\end{exer}

Since $F$ is a local diffeomorphism, there is an open set $e_G \in U \subseteq G$ such that $F|U$ is an embedding, and a submersion.
In particular, $F_* = (F|U)_*$, since $\Lie G$ is identified with the tangent space $T_eG$.
Since $F|U$ is a diffeomorphism and $e_H \in F(U)$, it induces an isomorphism of vector spaces $\Lie G = T_eG \to T_eH = \Lie H$.
Since $F$ is a morphism of Lie groups, $F_*$ is a morphism of Lie algebras, which is a bijection since it is an isomorphism of vector spaces.
Therefore $F_*$ is an isomorphism of Lie algebras.

\begin{exer}[8.29]
Show that:
\begin{enumerate}
\item $\Lie \SL(n, \RR) = \mathfrak{sl}(n, \RR)$.
\item $\Lie \SO(n) = \mathfrak o(n, \RR)$.
\item $\Lie \SL(n, \CC) = \mathfrak{sl}(n, \CC)$.
\item $\Lie U(n) = \mathfrak u(n)$.
\item $\Lie SU(n) = \mathfrak{su}(n)$.
\end{enumerate}
\end{exer}

We recall that if $H$ is a Lie subgroup of $\GL(n)$ defined by $H = \varphi^{-1}(c)$, then
$$\Lie H = \ker d\varphi_1.$$

\begin{enumerate}
\item Consider $\varphi: \GL(n) \to \GL(1)$ with $\varphi(A) = \det A$; then $\SL(n) = \varphi^{-1}(1)$.
Assume that the Jordan canonical form of $A$ has diagonal entries $\lambda_1, \dots, \lambda_n$. Then
\begin{align*}
d\varphi_1(A) &= \lim_{h \to 0} \frac{\det(1 + hA) - \det 1}{h} \\
&= \lim_{h \to 0} \frac{1}{h}\left(\prod_{j=1}^n (1 + h\lambda_j) - 1\right)\\
&= \lim_{h \to 0} \frac{h \tr A + O(h^2)}{h} = \tr A
\end{align*}
since $\tr A = \sum_j \lambda_j$. Therefore $\Lie \SL(n, \RR) = \ker \tr$, as desired.
\item Consider $\varphi: \SL(n, \RR) \to \SL(n, \RR)$ with $\varphi(A) = A^tA$; then $\SO(n) = \varphi^{-1}(1)$. Then
\begin{align*}
d\varphi_1(A) &= \lim_{h \to 0} \frac{(1 + hA)^t(1 + hA) - 1^t1}{h} \\
&= \lim_{h \to 0} \frac{1 + hA^t + hA + O(h^2) - 1}{h}\\
&= A^t + A
\end{align*}
so $\ker \varphi_1 = \mathfrak{o}(n, \RR) \cap \mathfrak{sl}(n, \RR)$.
However, $\mathfrak o(n, \RR) \subseteq \mathfrak{sl}(n, \RR)$ since if $A^t + A = 0$ then $A$ has zero diagonal and hence zero trace.
So this was as desired.
\item The proof that $\Lie \SL(n, K) = \mathfrak{sl}(n, K)$ works regardless of if $K = \RR$ or $K = \CC$.
\item Consider $\varphi: \GL(n, \CC) \to \GL(n, \CC)$ with $\varphi(A) = A^*A$; then $U(n) = \varphi^{-1}(1)$. Then
\begin{align*}
d\varphi_1(A) &= \lim_{h \to 0} \frac{(1 + hA)^*(1 + hA) - 1^t1}{h} \\
&= \lim_{h \to 0} \frac{1 + hA^* + hA + O(h^2) - 1}{h}\\
&= A^* + A
\end{align*}
so $\ker \varphi_1 = \mathfrak u(n)$. Here we used that since $d$ is a Fr\'echet derivative, $h = \overline h$.
\item One has $SU(n) = \SL(n, \CC) \cap U(n)$ so $\Lie SU(n) = \mathfrak{sl}(n, \CC) \cap \mathfrak u(n) = \mathfrak{su}(n)$.
\end{enumerate}

\begin{exer}[9.5]
Let $M$ be a compact manifold with a nowhere vanishing vector field. Show that there is a smooth map $F: M \to M$ such that $F$ has no fixed points and $F$ is homotopic to the identity.
\end{exer}

Let $X$ be a nowhere vanishing vector field on $M$.
Since $M$ is compact, $X$ is complete, so let $\psi$ be the flow induced by $X$.
For every point $x \in M$ there is an open set $U_x \ni x$ and a $\delta_x > 0$ such that for every $y \in U_x$ and $0 < t < \delta_x$, $\psi_t(y) \neq y$.
Since $M$ is compact, we can use the cover by $U_x$s to find a \emph{minimal} $\delta > 0$ such that for every $0 < t < \delta$, $\psi_t$ has no fixed points.
So let $F = \psi_{\delta/2}$.
Then since $\psi_0$ is the identity, $\psi$ induces a homotopy between $F$ and the identity.

\begin{exer}[9.7]
Let $M$ be a connected manifold. Show that $\Diffeo M$ acts transitively on $M$.
\end{exer}

If $X$ is a compactly supported vector field, let $\psi^X$ denote the flow induced by $X$.
We will call a chart \emph{convex} if its image in $\RR^d$ is convex.
Every point is the center of a convex chart, since we can always restrict a chart to a small ball around the origin.

We first show that the theorem is at least true locally:
\begin{lemma}
For every $x \in M$ there is an open set $U \ni x$ such that for every $y \in U$ there is a vector field $X \in \mathcal X_c(U)$ such that $\psi^X_1(x) = y$.
\end{lemma}
\begin{proof}
Let $\Psi: U \to \RR^d$ be a convex chart centered at $x$ with image $\Psi(U) = V$.
Then $\Psi(x) = 0$. Given $y \in U$, let $Y$ be the constant vector field $\Psi(y)$, so $\Psi^{-1}(\psi^Y_1(\Psi(x))) = y$.
Then $Y$ pulls back to a vector field $F^{-1}(Y)$ on $U$ such that $\psi^{F^{-1}(Y)}_1(x) = y$.
By multiplying $F^{-1}(Y)$ by a smooth function with compact support which is $1$ on an open set containing $\{\psi^{F^{-1}(Y)}_t(x): t \in [0, 1]\}$, we can replace $F^{-1}(Y)$ by a vector field with compact support but the desired flow properties.
\end{proof}

Now let $x, y \in M$.
Since $M$ is path-connected, let $\gamma$ be a path from $x$ to $y$.
Then the image of $\gamma$ is compact, and so is covered by finitely many convex charts $U_1, \dots, U_n$, chosen so that $U_1$ is centered at $x$, $U_n$ is centered at $y$, and $U_i \cap U_{i+1}$ is nonempty whenever $1 \leq i < n$.
Let $z_0 = x$, $z_n = y$, and $z_i \in U_i \cap U_{i+1}$.
Then choose vector fields $X_i \in \mathcal X_c(U_i)$ such that $\psi^{X_i}_1(z_i) = z_{i+1}$.
Flowing along each $X_i$ in succession, one obtains a path $\kappa$ from $x$ to $y$ which passes through each $z_i$ and is smooth except possibly at the $z_i$.
One can smooth a path at finitely many points, so it is no loss of generality to suppose that $\kappa$ is smooth.

To complete the proof, it suffices to show that $\kappa$ is an integral curve.
Indeed, if $X$ is a vector field for which $\kappa$ is an integral curve, then there is a $t > 0$ such that $\psi^X_t(x) = y$, and so $\psi^X_t$ witnesses that $\Diffeo M$ acts transitively.
Taking a partition of unity, it suffices to show that $\kappa$ is locally integral.

For each $t \in [0, 1]$, let $\delta_t > 0$ be sufficiently small based on criteria to be chosen later.
Let $\eta_t: B_{d-1}(0, \delta_t) \to M$ be an embedding of the $d-1$-dimensional ball such that $\eta_t(0) = \kappa(t)$ and the image of $\eta_t$ does not intersect the image of $\eta_s$ if $s \neq t$.
This can be done by working in flowbox coordinates, since we only need to show that $\kappa$ was locally integral and were allowed to take $\delta_t$ sufficiently small, and taking $\eta_t$ to be perpendicular to the straight line $\kappa$.
Since the image of $\kappa$ is compact, we can even assume that the images of $\eta_t$ and $\kappa$ only intersect at $\kappa(t)$ by taking $\delta_t$ small enough.
Then let $X|\eta_t = \kappa'(t)$.
Since $\kappa$ is smooth, so is $X$, so we're done.


\begin{exer}[9.14]
Prove the Whitney immersion theorem for manifolds with nonempty boundary.
\end{exer}

Let $M$ be a manifold with boundary.
We already proved this when $\partial M$ was empty, so let $M'$ be the double of $M$.
Then $\partial M'$ is empty, so there is an immersion $f: M' \to \RR^d$.
The restriction of an immersion is an immersion, so $f|M$ is the desired immersion.

\end{document}
