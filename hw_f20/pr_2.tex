
% --------------------------------------------------------------
% This is all preamble stuff that you don't have to worry about.
% Head down to where it says "Start here"
% --------------------------------------------------------------

\documentclass[10pt]{article}

\usepackage[margin=.7in]{geometry}
\usepackage{amsmath,amsthm,amssymb}
\usepackage{enumitem}
\usepackage{tikz-cd}
\usepackage{mathtools}
\usepackage{amsfonts}
\usepackage{listings}
\usepackage{algorithm2e}
\usepackage{verse,stmaryrd}
\usepackage{fancyvrb}

% Number systems
\newcommand{\NN}{\mathbb{N}}
\newcommand{\ZZ}{\mathbb{Z}}
\newcommand{\QQ}{\mathbb{Q}}
\newcommand{\RR}{\mathbb{R}}
\newcommand{\CC}{\mathbb{C}}
\newcommand{\PP}{\mathbb P}
\newcommand{\FF}{\mathbb F}
\newcommand{\DD}{\mathbb D}
\renewcommand{\epsilon}{\varepsilon}

\newcommand{\Aut}{\operatorname{Aut}}
\newcommand{\cl}{\operatorname{cl}}
\newcommand{\ch}{\operatorname{ch}}
\newcommand{\Con}{\operatorname{Con}}
\newcommand{\coker}{\operatorname{coker}}
\newcommand{\CVect}{\CC\operatorname{-Vect}}
\newcommand{\Cantor}{\mathcal{C}}
\newcommand{\D}{\mathcal{D}}
\newcommand{\card}{\operatorname{card}}
\newcommand{\dbar}{\overline \partial}
\newcommand{\diam}{\operatorname{diam}}
\newcommand{\End}{\operatorname{End}}
\DeclareMathOperator*{\esssup}{ess\,sup}
\newcommand{\GL}{\operatorname{GL}}
\newcommand{\Hom}{\operatorname{Hom}}
\newcommand{\id}{\operatorname{id}}
\newcommand{\Ind}{\operatorname{Ind}}
\newcommand{\Inn}{\operatorname{Inn}}
\newcommand{\interior}{\operatorname{int}}
\newcommand{\lcm}{\operatorname{lcm}}
\newcommand{\mesh}{\operatorname{mesh}}
\newcommand{\LL}{\mathcal L_0}
\newcommand{\Leb}{\mathcal{L}_{\text{loc}}^2}
\newcommand{\Lip}{\operatorname{Lip}}
\newcommand{\ppGL}{\operatorname{PGL}}
\newcommand{\ppic}{\vspace{35mm}}
\newcommand{\ppset}{\mathcal{P}}
\DeclareMathOperator*{\Res}{Res}
\newcommand{\Riem}{\mathcal{R}}
\newcommand{\RVect}{\RR\operatorname{-Vect}}
\newcommand{\Sch}{\mathcal{S}}
\newcommand{\SL}{\operatorname{SL}}
\newcommand{\sgn}{\operatorname{sgn}}
\newcommand{\Spec}{\operatorname{Spec}}
\newcommand{\supp}{\operatorname{supp}}
\newcommand{\TT}{\mathcal T}
\DeclareMathOperator{\tr}{tr}

% Calculus of variations
\DeclareMathOperator{\pp}{\mathbf p}
\DeclareMathOperator{\zz}{\mathbf z}
\DeclareMathOperator{\uu}{\mathbf u}
\DeclareMathOperator{\vv}{\mathbf v}
\DeclareMathOperator{\ww}{\mathbf w}

% Categories
\newcommand{\Ab}{\mathbf{Ab}}
\newcommand{\Cat}{\mathbf{Cat}}
\newcommand{\Group}{\mathbf{Group}}
\newcommand{\Module}{\mathbf{Module}}
\newcommand{\Set}{\mathbf{Set}}
\DeclareMathOperator{\Fun}{Fun}
\DeclareMathOperator{\Iso}{Iso}

% Complex analysis
\renewcommand{\Re}{\operatorname{Re}}
\renewcommand{\Im}{\operatorname{Im}}

% Logic
\renewcommand{\iff}{\leftrightarrow}
\newcommand{\Henkin}{\operatorname{Henk}}
\newcommand{\PA}{\mathbf{PA}}
\DeclareMathOperator{\proves}{\vdash}

% Group
\DeclareMathOperator{\Gal}{Gal}
\DeclareMathOperator{\Fix}{Fix}
\DeclareMathOperator{\Out}{Out}

% Other symbols
\newcommand{\heart}{\ensuremath\heartsuit}

\DeclareMathOperator{\atanh}{atanh}

% Theorems
\theoremstyle{definition}
\newtheorem*{corollary}{Corollary}
\newtheorem*{falselemma}{Grader's ``Lemma"}
\newtheorem{exer}{Exercise}
\newtheorem{lemma}{Lemma}[exer]
\newtheorem{theorem}[lemma]{Theorem}


\usepackage[backend=bibtex,style=alphabetic,maxcitenames=50,maxnames=50]{biblatex}
\renewbibmacro{in:}{}
\DeclareFieldFormat{pages}{#1}

\begin{document}
\noindent
\large\textbf{Probability, HW 2} \hfill \textbf{Aidan Backus} \\

% --------------------------------------------------------------
%                         Start here
% --------------------------------------------------------------\

\begin{exer}
Let $\mu^*$ be an outer measure on $\Omega$ such that every set has a $\mu^*$-measurable cover. Show that for every increasing sequence $(A_n)_n$, $\mu^*(\bigcup_n A_n) = \lim_n \mu^*(A_n)$. Find a counterexample for decreasing sequences.
\end{exer}

We first treat increasing sequences. One has $A_m \subseteq \bigcup_n A_n$ so
$$\mu^*(A_m) \leq \mu^*\left(\bigcup_n A_n\right)$$
and taking the limit in $m$ on the left we see
$$\lim_n \mu^*(A_n) \leq \mu^*\left(\bigcup_n A_n\right).$$
As for the converse, let $E_n$ be a measurable cover of $A_n$, and let $\mu$ be the measure induced by $\mu^*$. Then $\mu$ commutes with limits of increasing chains and, since $\bigcup_n E_n$ is a measurable set containing $\bigcup_n A_n$,
$$\mu^*\left(\bigcup_n A_n\right) = \mu^*\left(\bigcup_n E_n\right) = \mu\left(\bigcup_n E_n\right) = \lim_n \mu(E_n) = \lim_n \mu^*(E_n) = \lim_n \mu^*(A_n).$$

For the counterexample, we will construct a sequence of modified Vitali sets $A_n$, as follows.
Let $S^1$ be the circle with Lebesgue outer measure $\mu^*$, $\mu$ Lebesgue measure.
We first observe that every subset of $S^1$ has a measurable cover: if $E \subseteq S^1$ is nonmeasurable, let $U_n \supseteq E$ be an open set such that $\mu^*(U_n) \leq \mu^*(E) + 1/n$. Then $\bigcap_n U_n$ is a Borel (therefore measurable) cover of $E$.

Let $B_0$ be a Vitali set in $S^1$ and identify a rational number $q_n \in \QQ$ with the rotation of $S^1$ by $q_n$.
Here $(q_n)_n$ is an enumeration of $\QQ$.
Then $B_0 + q_n$ is a Vitali set of the same outer measure as $A_0$, and if we define $B_n = B_{n-1} \cup (B_{n-1} + q_n)$, $\bigcup_n B_n = S^1$.
Now let $A_n = \bigcap_{m\geq n} B_m$.
Then $\bigcap_n A_n = \bigcap_m B_m = \emptyset$ and $(A_n)_n$ is a decreasing sequence. However,
$$\mu^*(A_n) \geq \mu^*(B_m) = \mu^*(B_0) > 0,$$
since $B_0$ is nonmeasurable and $\mu$ is complete.

\begin{exer}
Let $\Omega$ be a metric space with Borel $\sigma$-algebra $\mathcal B$.
Let $\mathcal F \supseteq \mathcal B$ be a $\sigma$-algebra, and let $\mu$ be a regular measure on $\mathcal F$.
Show that for every $D \in \mathcal F$, there are Borel sets $B_1 \subseteq D \subseteq B_2$ such that $\mu(B_2 \setminus B_1) = 0$, as well as Borel sets $E,B$ and a set $F \subseteq B$ such that $\mu(B) = 0$ and $D = E \cup F$.
\end{exer}

Fix $D$. For every $n$ there is a closed set $B_1^n \subseteq D$ such that $\mu(D \setminus B_1^n) < 1/n$.
Their union $B_1 = \bigcup_n B_1^n$ is a Borel set $\subseteq D$ such that for every $n$,
$$\mu(D) - \frac{1}{n} < \mu(B_1) \leq \mu(D)$$
so $\mu(B_1) = \mu(D)$. A similar argument using open sets $B_2^n \supseteq D$ furnishes the Borel set $B_2$. Then $\mu(B_1) = \mu(B_2)$ so $\mu(B_2 \setminus B_1) = 0$.

Now set $E = B_1$ and $F = D \setminus E$. Then $\mu(F) = 0$ since $\mu(D) = \mu(E)$.
Therefore we can run the construction of $B_2$ given above with $F$ as input instead of $D$ to get a Borel set $B \supseteq F$ of measure $0$.

\begin{exer}
Let $\Omega$ be a metric space and $\mu$ a Borel measure which is finite on bounded Borel sets. Show that $\mu$ is regular, and the completion $\overline \mu$ of $\mu$ is regular.
\end{exer}

We first note that $\mu$ is $\sigma$-finite. In fact, we can choose an outcome $\omega \in \Omega$; then $\Omega$ is the union of the balls $\Omega_n =
\overline{B(\omega, n)}$, which have finite measure by hypothesis and are closed.
This suggests that we can consider the case when $\mu$ is finite.

We will call the double inclusion $K \subseteq E \subseteq U$ an $\varepsilon$-sandwich with bologna $E$ provided that $K$ is closed, $U$ is open, and $\mu(U \setminus K) < \varepsilon$.
\begin{lemma}
If $\mu$ is finite, then $\mu$ is regular.
\end{lemma}
\begin{proof}
Let $\mathcal E$ be the set of all $E \subseteq \Omega$ such that for every $\varepsilon$, $E$ is the bologna of some $\varepsilon$-sandwich.

To see that $\mathcal E$ contains every closed set, let $E$ be closed.
Let $E_\delta = \{\omega \in \Omega: \rho(\omega, E) < \delta\}$; then $E_\delta$ is open, $\overline E_\delta$ is closed, and $\bigcap_n \overline E_{1/n} = E$.
Since $\mu$ is finite, for every $\varepsilon > 0$ we can find a $\delta$ such that
$$\mu(E) \leq \mu(E_\delta) \leq \mu(\overline E_\delta) < \mu(E) + \varepsilon.$$
So $E \subseteq E \subseteq E_\delta$ is an $\varepsilon$-sandwich of bologna $E$, and $E \in \mathcal E$.

Now we show that $\mathcal E$ is closed under complements.
Let $E \in \mathcal E$ and $\varepsilon > 0$.
Then there is a $\varepsilon$-sandwich $K \subseteq E \subseteq U$.
Taking complements, we obtain a double inclusion $U^c \subseteq E^c \subseteq K^c$.
But
$$K^c \setminus U^c = K^c \cap U = U \setminus K$$
so $U^c \subseteq E^c \subseteq K^c$ is a $\varepsilon$-sandwich of bologna $E^c$ since $U^c$ is clearly closed and $K^c$ open.

The fact that $\mathcal E$ follows from Zeno's paradox. Suppose that we are given $\varepsilon/2^n$-sandwiches $K_n \subseteq E_n \subseteq U_n$ with bologna $E_n \in \mathcal E$.
Let $E = \bigcup_n E_n$ and $U = \bigcup_n U_n$. Then $U \supseteq E$ is open.
Since $\mu$ is finite, $\mu(\bigcup_n K_n)$ is finite, so there must be some $N$ such that $\mu(\bigcup_n K_n \setminus \bigcup_{n \leq N} K_n) < \varepsilon$. Then set $K = \bigcup_{n \leq N} K_n$, so $K$ is closed $\subseteq E$.
Summing the geometric series,
$$\mu(U \setminus K) \leq \mu\left(U \setminus \bigcup_n K_n\right) + \mu\left(\bigcup_n K_n \setminus K\right) < 2\varepsilon.$$
So $K \subseteq E \subseteq U$ is a $2\varepsilon$-sandwich of bologna $E$.

It follows that $\mathcal E$ is a $\sigma$-algebra containing the closed sets, therefore the Borel sets.
\end{proof}

We now show that $\mu$ is regular without any finiteness hypothesis.
Let $E$ be a Borel set and $E_n = E \cap \Omega_n$.
Since each $\Omega_n$ is a finite measure space, the lemma gives $\varepsilon/2^n$-sandwiches $K_n \subseteq E_n \subseteq U_n$, where $K_n$ is closed in $\Omega$ and $U_n$ is open in $\Omega$ (so that $E_n$ is bologna in $\Omega$).
Here we use the fact that, since $\Omega_n$ is closed, $K_n$ is closed in $\Omega_n$ iff it is closed in $\Omega$; for the open sets $U_n$, we may need to replace them with $U_n^\delta = \{\omega \in \Omega_{n+2}: \rho(\omega, U_n) < \delta\}$, which is an open set whose measure can be made arbitrarily close to $\mu(U_n)$ using measure continuity.

We let $U = \bigcup_n U_n$ and $K = \bigcup_n K_n$. To finish the proof, we must check that $K$ is closed.
Let $(\omega_m)_m$ be a sequence in $K$ which converges to $\omega \in \Omega$.
Then $(\omega_m)_m$ is Cauchy, so it is bounded; so it remains inside $\bigcup_{n \leq N} \Omega_n$ for some sufficiently large $N$, and so in $\bigcup_{n \leq N} K_n$, a finite union of closed sets.
Therefore $\omega \in K_n$ for some $n \leq N$, and hence $\omega \in K$.

Finally we show that $\overline \mu$ is regular.
If $E$ is a measurable set, then by definition there are disjoint sets $E^\flat$ and $N_E$ such that $E = E^\flat \cup N_E$, $E^\flat$ is Borel, and $N_E$ is contained a Borel set $N_E^\sharp$ such that $\mu(N_E^\sharp) = 0$.
Choose $\varepsilon$-sandwiches $K^\flat \subseteq E^\flat \subseteq U^\flat$ and $K^\sharp \subseteq N_E^\sharp \subseteq U^\sharp$; then let $K = K^\flat \cup K^\sharp$ and $U = U^\flat \cup U^\sharp$.
Then $K \subseteq E \subseteq U$ is a $2\varepsilon$-sandwich.

\begin{exer}
Let $F$ be an increasing, right-continuous function, and let $\mu_F$ be the induced Stieltjes measure. Show that $\mu_F(\{a\}) = F(a) - F(a-)$, $\mu_F([a, b)) = F(b-) - F(a-)$, $\mu_F((a,b)) = F(b-) - F(a)$, $\mu_F([a,b]) = F(b) - F(a-)$.
\end{exer}

One has
$$\mu_F(\{a\}) = \lim_n \mu_F((a-1/n,a]) = \lim_n F(a) - F(a-1/n) = F(a) - F(a-).$$
Similarly
$$\mu_F([a, b)) = \lim_n \mu_F((a-1/n,b-1/n]) = F(b-) - F(a-),$$
$$\mu_F((a,b)) = \lim_n \mu_F((a, b-1/n]) = F(b-) - F(a),$$
and
$$\mu_F([a,b]) = \lim_n \mu_F((a-1/n,b]) = F(b) - F(a-).$$

\begin{exer}
Let $\mu^*$ be an outer measure which is finitely additive. Show that $\mu^*$ is a measure.
\end{exer}

Let $(A_n)_n$ be a disjoint sequence. If $\mu^*(\bigcup_n A_n) = \infty$ then
$$\sum_n \mu^*(A_n) \leq \infty = \mu^*\left(\bigcup_n A_n\right) \leq \sum_n \mu^*(A_n).$$
Otherwise, for every $\varepsilon > 0$ there is an $N$ so large that if $n > N$,
$$ \mu^*\left(\bigcup_{n > N} A_n \right) \leq \sum_{n > N} \mu^*(A_n) < \varepsilon.$$
That implies
\begin{align*}
 \mu^*\left(\bigcup_n A_n\right) &\leq \sum_n \mu^*(A_n) = \sum_{n \leq N} \mu^*(A_n) + \sum_{n > N} \mu^*(A_n)\\
 &< \sum_{n \leq N} \mu^*(A_n) + \varepsilon = \mu^*\left(\bigcup_{n \leq N} A_n\right) + \varepsilon\\
 &\leq \mu^*\left(\bigcup_n A_n\right) + \varepsilon.
\end{align*}
Either way, $\mu^*$ is countably additive.

\begin{exer}
Let $(\mu_n)_n$ be an increasing sequence of measures. Show that $\mu = \lim_n \mu_n$ is a measure.
\end{exer}

Recall that one can commute an infinite sum with an increasing limit. Therefore if $(A_m)_m$ is a disjoint sequence,
\begin{align*}
\mu\left(\bigcup_m A_m\right) &= \lim_n \mu_n\left(\bigcup_m A_m\right) = \lim_n \sum_m \mu_n(A_m)\\
&= \sum_m \lim_n \mu_n(A_m) = \sum_m \mu(A_m).
\end{align*}
Therefore $\mu$ is countably additive.



\end{document}
