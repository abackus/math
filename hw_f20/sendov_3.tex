\documentclass[12pt]{article}
\usepackage[pdftex,pagebackref,letterpaper=true,colorlinks=true,pdfpagemode=none,urlcolor=blue,linkcolor=blue,citecolor=blue,pdfstartview=FitH]{hyperref}

\usepackage{amsmath,amsfonts}
\usepackage{graphicx}
\usepackage{color}

% Number systems
\newcommand{\NN}{\mathbb{N}}
\newcommand{\ZZ}{\mathbb{Z}}
\newcommand{\QQ}{\mathbb{Q}}
\newcommand{\RR}{\mathbb{R}}
\newcommand{\CC}{\mathbb{C}}
\newcommand{\PP}{\mathbb P}
\newcommand{\FF}{\mathbb F}
\newcommand{\DD}{\mathbb D}
\newcommand{\EE}{\mathbb E}
\renewcommand{\epsilon}{\varepsilon}

\newcommand{\Aut}{\operatorname{Aut}}
\newcommand{\cl}{\operatorname{cl}}
\newcommand{\ch}{\operatorname{ch}}
\newcommand{\Con}{\operatorname{Con}}
\newcommand{\coker}{\operatorname{coker}}
\newcommand{\CVect}{\CC\operatorname{-Vect}}
\newcommand{\Cantor}{\mathcal{C}}
\newcommand{\D}{\mathcal{D}}
\newcommand{\card}{\operatorname{card}}
\newcommand{\dbar}{\overline \partial}
\newcommand{\diam}{\operatorname{diam}}
\newcommand{\dom}{\operatorname{dom}}
\newcommand{\End}{\operatorname{End}}
\DeclareMathOperator*{\esssup}{ess\,sup}
\newcommand{\Hess}{\operatorname{Hess}}
\newcommand{\Hom}{\operatorname{Hom}}
\newcommand{\id}{\operatorname{id}}
\newcommand{\Ind}{\operatorname{Ind}}
\newcommand{\Inn}{\operatorname{Inn}}
\newcommand{\interior}{\operatorname{int}}
\newcommand{\lcm}{\operatorname{lcm}}
\newcommand{\mesh}{\operatorname{mesh}}
\newcommand{\LL}{\mathcal L_0}
\newcommand{\Leb}{\mathcal{L}_{\text{loc}}^2}
\newcommand{\Lip}{\operatorname{Lip}}
\newcommand{\ppic}{\vspace{35mm}}
\newcommand{\ppset}{\mathcal{P}}
\DeclareMathOperator{\proj}{proj}
\DeclareMathOperator*{\Res}{Res}
\newcommand{\Riem}{\mathcal{R}}
\newcommand{\RVect}{\RR\operatorname{-Vect}}
\newcommand{\Sch}{\mathcal{S}}
\newcommand{\sgn}{\operatorname{sgn}}
\newcommand{\spn}{\operatorname{span}}
\newcommand{\Spec}{\operatorname{Spec}}
\newcommand{\supp}{\operatorname{supp}}
\newcommand{\TT}{\mathcal T}
\DeclareMathOperator{\tr}{tr}

% Calculus of variations
\DeclareMathOperator{\pp}{\mathbf p}
\DeclareMathOperator{\zz}{\mathbf z}
\DeclareMathOperator{\uu}{\mathbf u}
\DeclareMathOperator{\vv}{\mathbf v}
\DeclareMathOperator{\ww}{\mathbf w}

% Categories
\newcommand{\Ab}{\mathbf{Ab}}
\newcommand{\Cat}{\mathbf{Cat}}
\newcommand{\Group}{\mathbf{Group}}
\newcommand{\Module}{\mathbf{Module}}
\newcommand{\Set}{\mathbf{Set}}
\DeclareMathOperator{\Fun}{Fun}
\DeclareMathOperator{\Iso}{Iso}

% Complex analysis
\renewcommand{\Re}{\operatorname{Re}}
\renewcommand{\Im}{\operatorname{Im}}

% Logic
\renewcommand{\iff}{\leftrightarrow}
\newcommand{\Henkin}{\operatorname{Henk}}
\newcommand{\PA}{\mathbf{PA}}
\DeclareMathOperator{\proves}{\vdash}

% Group
\DeclareMathOperator{\Gal}{Gal}
\DeclareMathOperator{\Fix}{Fix}
\DeclareMathOperator{\Lie}{Lie}
\DeclareMathOperator{\Out}{Out}

\DeclareMathOperator{\Diffeo}{Diffeo}

\newcommand{\GL}{\operatorname{GL}}
\newcommand{\ppGL}{\operatorname{PGL}}
\newcommand{\SL}{\operatorname{SL}}
\newcommand{\SO}{\operatorname{SO}}

% Other symbols
\newcommand{\heart}{\ensuremath\heartsuit}
\newcommand{\club}{\ensuremath\clubsuit}

\DeclareMathOperator{\atanh}{atanh}

\def\Xint#1{\mathchoice
{\XXint\displaystyle\textstyle{#1}}%
{\XXint\textstyle\scriptstyle{#1}}%
{\XXint\scriptstyle\scriptscriptstyle{#1}}%
{\XXint\scriptscriptstyle\scriptscriptstyle{#1}}%
\!\int}
\def\XXint#1#2#3{{\setbox0=\hbox{$#1{#2#3}{\int}$ }
\vcenter{\hbox{$#2#3$ }}\kern-.6\wd0}}
\def\ddashint{\Xint=}
\def\dashint{\Xint-}

\setlength{\oddsidemargin}{0pt}
\setlength{\evensidemargin}{0pt}
\setlength{\textwidth}{6.0in}
\setlength{\topmargin}{0in}
\setlength{\textheight}{8.5in}

\setlength{\parindent}{0in}
\setlength{\parskip}{5px}

%%%%%%%%% For wordpress conversion

\def\more{}

\newif\ifblog
\newif\iftex
\blogfalse
\textrue


\usepackage{ulem}
\def\em{\it}
\def\emph#1{\textit{#1}}

\def\image#1#2#3{\begin{center}\includegraphics[#1pt]{#3}\end{center}}

\let\hrefnosnap=\href

\newenvironment{btabular}[1]{\begin{tabular} {#1}}{\end{tabular}}

\newenvironment{red}{\color{red}}{}
\newenvironment{green}{\color{green}}{}
\newenvironment{blue}{\color{blue}}{}

%%%%%%%%% Typesetting shortcuts

\def\B{\{0,1\}}
\def\xor{\oplus}

\def\P{{\mathbb P}}
\def\E{{\mathbb E}}
\def\var{{\bf Var}}

\def\N{{\mathbb N}}
\def\Z{{\mathbb Z}}
\def\R{{\mathbb R}}
\def\C{{\mathbb C}}
\def\Q{{\mathbb Q}}
\def\eps{{\epsilon}}

\def\bz{{\bf z}}

\def\true{{\tt true}}
\def\false{{\tt false}}

%%%%%%%%% Theorems and proofs

\newtheorem{exercise}{Exercise}
\newtheorem{theorem}{Theorem}
\newtheorem{lemma}[theorem]{Lemma}
\newtheorem{definition}[theorem]{Definition}
\newtheorem{corollary}[theorem]{Corollary}
\newtheorem{proposition}[theorem]{Proposition}
\newtheorem{example}{Example}
\newtheorem{remark}[theorem]{Remark}
\newenvironment{proof}{\noindent {\sc Proof:}}{$\Box$ \medskip} 


\begin{document}

% --------------------------------------------------------------
%                         Start here
% --------------------------------------------------------------\

In this post we're going to complete the proof of case zero and continue the proof of case one. In the last two posts we managed to prove:

\begin{theorem}
Let $n$ be a nonstandard natural, and let $f$ be a monic polynomial of degree $n$ on $\mathbf C$ with all zeroes in $\overline{D(0, 1)}$.
Suppose that $a$ is a zero of $f$ such that:
\begin{enumerate}
\item Either $a$ or $a - 1$ is infinitesimal, and
\item $f$ has no critical points on $\overline{D(a, 1)}$.
\end{enumerate}
Let $\lambda$ be a random zero of $f$ and $\zeta$ a random critical point of $f$.
Let $z$ be a complex number outside some measure zero set and let $\gamma$ be a contour that misses the zeroes of $f$,$f'$. Then:
\begin{enumerate}
\item $\zeta \in \overline{D(0, 1)} \setminus \overline{D(a, 1)}$.
\item If $\mu = \mathbf E \lambda$ then $\mu = \mathbf E \zeta$.
\item One has
$$U_\lambda(z) = -\frac{1}{n} \log |f(z)|$$
and
$$U_\zeta(z) = -\frac{\log n}{n - 1} - \frac{1}{n-1} \log |f'(z)|.$$
\item One has
$$s_\lambda(z) = \frac{1}{n} \frac{f'(z)}{f(z)}$$
and
$$s_\zeta(z) = \frac{1}{n - 1} \frac{f''(z)}{f'(z)}.$$
\item One has
$$U_\lambda(z) - \frac{n - 1}{n} U_\zeta(z) = \frac{1}{n} \log |s_\lambda(z)|$$
and
$$s_\lambda(z) - \frac{n - 1}{n} s_\zeta(z) = -\frac{1}{n} \frac{s_\lambda'(z)}{s_\lambda(z)}.$$
\item One has
$$f(\gamma(1)) = f(\gamma(0)) \exp \left(n \int_\gamma s_\lambda(z) ~dz\right)$$
and
$$f'(\gamma(1)) = f(\gamma(0)) \exp\left((n-1) \int_\gamma s_\zeta(z) ~dz\right).$$
\end{enumerate}
Moreover,
\begin{enumerate}
\item If $a$ is infinitesimal (case zero), then $\lambda^{(\infty)},\zeta^{(\infty)}$ are identically distributed and almost surely lie in
$$C = \{e^{i\theta}: 2\theta \in [\pi, 3\pi]\}.$$
Moreover, if $K$ is any compact set which misses $C$, then
$$\mathbf P(\lambda \in K) = O\left(a + \frac{\log n}{n^{1/3}}\right),$$
so $d(\lambda, C)$ is infinitesimal in probability.
\item If $a - 1$ is infinitesimal (case one), then $\lambda^{(\infty)}$ is uniformly distributed on $\partial D(0, 1)$ and $\zeta^{(\infty)}$ is almost surely zero.
Moreover,
$$\mathbf E \log \frac{1}{|\lambda|}, \mathbf E \log |\zeta - a| = O(n^{-1}).$$
\end{enumerate}
\end{theorem}






\end{document}
