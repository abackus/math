
% --------------------------------------------------------------
% This is all preamble stuff that you don't have to worry about.
% Head down to where it says "Start here"
% --------------------------------------------------------------

\documentclass[10pt]{article}

\usepackage[margin=.7in]{geometry}
\usepackage{amsmath,amsthm,amssymb}
\usepackage{enumitem}
\usepackage{tikz-cd}
\usepackage{mathtools}
\usepackage{amsfonts}
\usepackage{listings}
\usepackage{algorithm2e}
\usepackage{verse,stmaryrd}
\usepackage{fancyvrb}

% Number systems
\newcommand{\NN}{\mathbb{N}}
\newcommand{\ZZ}{\mathbb{Z}}
\newcommand{\QQ}{\mathbb{Q}}
\newcommand{\RR}{\mathbb{R}}
\newcommand{\CC}{\mathbb{C}}
\newcommand{\PP}{\mathbb P}
\newcommand{\FF}{\mathbb F}
\newcommand{\DD}{\mathbb D}
\renewcommand{\epsilon}{\varepsilon}

\newcommand{\Aut}{\operatorname{Aut}}
\newcommand{\cl}{\operatorname{cl}}
\newcommand{\ch}{\operatorname{ch}}
\newcommand{\Con}{\operatorname{Con}}
\newcommand{\coker}{\operatorname{coker}}
\newcommand{\CVect}{\CC\operatorname{-Vect}}
\newcommand{\Cantor}{\mathcal{C}}
\newcommand{\D}{\mathcal{D}}
\newcommand{\card}{\operatorname{card}}
\newcommand{\dbar}{\overline \partial}
\newcommand{\diam}{\operatorname{diam}}
\newcommand{\End}{\operatorname{End}}
\DeclareMathOperator*{\esssup}{ess\,sup}
\newcommand{\GL}{\operatorname{GL}}
\newcommand{\Hom}{\operatorname{Hom}}
\newcommand{\id}{\operatorname{id}}
\newcommand{\Ind}{\operatorname{Ind}}
\newcommand{\Inn}{\operatorname{Inn}}
\newcommand{\interior}{\operatorname{int}}
\newcommand{\lcm}{\operatorname{lcm}}
\newcommand{\mesh}{\operatorname{mesh}}
\newcommand{\LL}{\mathcal L_0}
\newcommand{\Leb}{\mathcal{L}_{\text{loc}}^2}
\newcommand{\Lip}{\operatorname{Lip}}
\newcommand{\ppGL}{\operatorname{PGL}}
\newcommand{\ppic}{\vspace{35mm}}
\newcommand{\ppset}{\mathcal{P}}
\DeclareMathOperator*{\Res}{Res}
\newcommand{\Riem}{\mathcal{R}}
\newcommand{\RVect}{\RR\operatorname{-Vect}}
\newcommand{\Sch}{\mathcal{S}}
\newcommand{\SL}{\operatorname{SL}}
\newcommand{\sgn}{\operatorname{sgn}}
\newcommand{\Spec}{\operatorname{Spec}}
\newcommand{\supp}{\operatorname{supp}}
\newcommand{\TT}{\mathcal T}
\DeclareMathOperator{\tr}{tr}

% Calculus of variations
\DeclareMathOperator{\pp}{\mathbf p}
\DeclareMathOperator{\zz}{\mathbf z}
\DeclareMathOperator{\uu}{\mathbf u}
\DeclareMathOperator{\vv}{\mathbf v}
\DeclareMathOperator{\ww}{\mathbf w}

% Categories
\newcommand{\Ab}{\mathbf{Ab}}
\newcommand{\Cat}{\mathbf{Cat}}
\newcommand{\Group}{\mathbf{Group}}
\newcommand{\Module}{\mathbf{Module}}
\newcommand{\Set}{\mathbf{Set}}
\DeclareMathOperator{\Fun}{Fun}
\DeclareMathOperator{\Iso}{Iso}

% Complex analysis
\renewcommand{\Re}{\operatorname{Re}}
\renewcommand{\Im}{\operatorname{Im}}

% Logic
\renewcommand{\iff}{\leftrightarrow}
\newcommand{\Henkin}{\operatorname{Henk}}
\newcommand{\PA}{\mathbf{PA}}
\DeclareMathOperator{\proves}{\vdash}

% Group
\DeclareMathOperator{\Gal}{Gal}
\DeclareMathOperator{\Fix}{Fix}
\DeclareMathOperator{\Out}{Out}

% Other symbols
\newcommand{\heart}{\ensuremath\heartsuit}

\DeclareMathOperator{\atanh}{atanh}

% Theorems
\theoremstyle{definition}
\newtheorem*{corollary}{Corollary}
\newtheorem*{falselemma}{Grader's ``Lemma"}
\newtheorem{exer}{Exercise}
\newtheorem{lemma}{Lemma}[exer]
\newtheorem{theorem}[lemma]{Theorem}


\usepackage[backend=bibtex,style=alphabetic,maxcitenames=50,maxnames=50]{biblatex}
\renewbibmacro{in:}{}
\DeclareFieldFormat{pages}{#1}

\begin{document}
\noindent
\large\textbf{Smooth manifolds, HW 1} \hfill \textbf{Aidan Backus} \\

% --------------------------------------------------------------
%                         Start here
% --------------------------------------------------------------\

\begin{exer}[1.6]
Let $M$ be a nonempty topological manifold of dimension $\geq 1$.
Show that if $M$ has a smooth structure, then $M$ has uncountably many.
\end{exer}

Following the hint, let
\begin{align*}
F_s: \overline{\DD^d} &\to \overline{\DD^d}\\
x &\mapsto |x|^{s-1}x.
\end{align*}
\begin{lemma}
For every $s \in (0, 1)$, $F_s$ is a homeomorphism of $\DD^d$ with itself, but is not differentiable at $0$.
\end{lemma}
\begin{proof}
Certainly $F_s$ is continuous. If $F_s(x) = F_s(y)$ then
$$\frac{|x|^{s-1}}{|y|^{s-1}}x = y,$$
so $y$ is a scalar multiple of $x$; but $F_s$ is injective on each line through $0$, and hence $y = x$.
We will shortly check that $F_s$ is surjective, and since $\overline{\DD^d}$ is compact, that implies that $F_s$ is a homeomorphism of $\overline{\DD^d}$ to itself.

If $|x| = 1$, then $F_s(x) = x$, so if $F_s$ fails to be surjective, then $F_s$ is a retract of $\overline{\DD^d}$ onto a compact subset $K$ of $\overline{\DD^d}$ which contains $\partial \DD^d$ but not some $x \in \overline{\DD^d}$.
A loop whose image is $\partial \DD^d$ is not homotopic to a point in $K$ since any such homotopy must include $x$ in its image, yet we assumed that $x \notin K$.
So $\pi_1(K)$ is nontrivial, yet there is a retract $\overline{\DD^d} \to K$, a contradiction.

Since $F_s$ preserves $\partial \DD^d$ and is a bijection, in particular it preserves $\DD^d$.
Therefore $F_s$ is a homeomorphism of $\DD^d$ to itself.

But $F_s$ is not a diffeomorphism, or even differentiable.
If it were differentiable, then it would still be differentiable when restricted to the line through $0$ and the first basis vector $e_1$, yet its directional derivative along that line is
$$\nabla F_s \cdot e_1(x) = \sigma \frac{d}{dx} |x|^s = \tau s|x|^{s-1}$$
where $\sigma,\tau$ are appropriately chosen signs; since $0^{s-1}$ makes no sense $F_s$ canot possibly be differentiable at the origin.
\end{proof}

Let $\mathcal A$ be a smooth atlas of $M$, and choose a chart $(W, \varphi) \in \mathcal A$.
Then $\varphi(W)$ is a nonempty open subset of $\RR^d$, so contains a compact ball $\tilde K$.
Let $K = \varphi^{-1}(\tilde K)$ and let $U$ be the interior of $K$; then $U$ is nonempty since $\varphi(U)$ is an open ball.
Then $\mathcal A$ determines an open cover $\{V: (V, \psi) \in \mathcal A\}$ of $K$, so there are finitely many charts $(V_1, \psi_1), \dots, (V_n, \psi_n)$ such that $\{V_1, \dots, V_n\}$ is an open cover of $K$.

For every $x \in K$, we say that $x$ is \emph{efficiently covered} by $\{V_1, \dots, V_n\}$ if there is a unique $i$ such that $x \in V_i$.
Let us prove two facts about efficiently covered points.
\begin{lemma}
The property ``is efficiently covered" is an open property.
Moreover, it is no loss of generality to assume that there exists $x_1 \in V_1$ which is efficiently covered by $\{V_1, \dots, V_n\}$.
\end{lemma}
\begin{proof}
If $x_i \in V_i$ is efficiently covered, then $W_i = V_i \setminus \bigcup_{j\neq i} \overline V_j$ is nonempty and open, and is exactly the set of elements of $V_i$ which are efficiently covered.
The union of the $W_i$ is open and is the set of elements of $K$ which are efficiently covered.

If, for every $x \in K$, $x$ is inefficiently covered, pick an $x_1 \in K$ and let
$$V_{n+1} = \bigcap_{j:x_1 \in V_j} V_j$$
(so that $V_{n+1}$ is nonempty because, in particular, $x_1 \in V_{n+1}$).
Since all sets involved are open, and there are only finitely many (so that they cannot come arbitrarily close to $x_1$ in any metric) we can find a smaller open set, $Y \subseteq V_{n+1}$, such that $x_1 \in Y$ and $\bigcup_k \partial V_k$ does not meet $\overline Y$.
It follows that $\{V_1 \setminus \overline Y, \dots, V_n \setminus Y, V_{n+1}\}$ is an open cover of $K$ which efficiently covers $x_1$.
Taking the obvious restrictions, we can find suitable charts $V_i \setminus \overline Y \to \RR^d$ and $V_{n+1} \to \RR^d$ which are smoothly compatible with $\mathcal A$.
So we can add $V_i \setminus \overline Y$ and $V_{n+1}$ to $\mathcal A$ and replace the cover $\{V_1, \dots, V_n\}$ with $\{V_1 \setminus \overline Y, \dots, V_n \setminus Y, V_{n+1}\}$.
The claim now follows by reordering the $V_i$ so that $x_1 \in V_1$.
\end{proof}

Let $x_1 \in V_1$ be efficiently covered, per the lemma.
Since $\psi_1(V_1)$ is an open subset of $\RR^d$, $\psi_1(x_1)$ is contained in an open ball $\tilde B$; let $B = \psi_1^{-1}(\tilde B)$.
Since $x_1$ is efficiently covered and ``is efficiently covered" is an open property, we can take the radius of $\tilde B$ to be so small that if $B \cap V_i$ is nonempty, then $i = 1$.

After an affine change of coordinates on $\RR^d$, we can assume that $\psi_1$ maps $x_1$ to $0$, $B$ to the unit ball, and $V_1$ to an open set containing the unit ball.
Let $\chi$ be a smooth function $\RR^d \to [0, 1]$ be a smooth function such that $\{\chi = 1\}$ is the ball of radius $1/4$ around $0$, and $\{\chi > 0\}$ is the ball of radius $1/2$ around $0$.
Then let
$$\tilde G_s(x) = \chi F_s(x) + (1 - \chi)x.$$
In polar coordinates,
$$\tilde G_s(r, \Theta) = (\chi r^s + (1-\chi)r, \Theta).$$
It follows that $\tilde G_s$ is a homeomorphism of $X = \psi_1(V_1)$ to itself such that $\tilde G_s|\psi_i(V_i \cap V_j)$ is smooth whenever $j \neq i$.

Let $G_s$ be the identity on $M \setminus \overline B$ and on $V_1$, be defined so that the diagram
$$\begin{tikzcd}
V_1 \arrow[r,"G_s"] \arrow[d,"\psi_1"] & V_1 \arrow[d,"\psi_1"]\\
X \arrow[r,"\tilde G_s"] & X
\end{tikzcd}$$
commutes. Since $\tilde G_s$ is the identity on $\psi_1(V_1 \setminus B)$, $G_s$ is well-defined, and is a homeomorphism from $M$ to itself.

Now we are ready to define uncountably many smooth structures on $M$.
Let
$$\mathcal A_s = \{(G_s(A), \alpha \circ G_s^{-1}): (A, \alpha) \in \mathcal A\}.$$
Since $G_s$ is a homeomorphism, $G_s(A)$ is open and $\alpha \circ G_s^{-1}$ is a homeomorphism, so $\mathcal A_s$ is an atlas.
To see that it is smooth, let $\alpha, \beta$ be (homeomorphisms for) charts in $\mathcal A$; then
$$(\beta \circ G_s^{-1}) \circ (\alpha \circ G_s^{-1})^{-1} = \beta \circ \alpha^{-1}$$
and $\beta \circ \alpha^{-1}$ is a diffeomorphism since $\mathcal A$ is a smooth atlas, so $\alpha \circ G_s^{-1}$ and $\beta \circ G_s^{-1}$ are smoothly compatible.
Therefore $\mathcal A_s$ determines a smooth structure $\mathcal S_s$ on $M$.

It suffices, by Cantor's diagonal argument, to show that $s \mapsto \mathcal S_s$ is injective.
This will follow from the following lemma.

\begin{lemma}
If $G_t^{-1} \circ G_s$ is smooth, then $s = t$.
\end{lemma}
\begin{proof}
By assumption, $G_t^{-1} \circ G_s|B$ is smooth; since the above diagram commutes and $\psi_1$ is a diffeomorphism, the same is true of $\tilde G_t^{-1} \circ \tilde G_s$.

Suppose that $s < t$.
In polar coordinates,
$$\tilde G_t^{-1}(\tilde G_s(r, \Theta)) = (\chi r^{s/t} + (1-\chi) r, \Theta) = \tilde G_{s/t}(r, \Theta).$$
But $\tilde G_{s/t}|\{\chi = 1\} = F_{s/t}|\{\chi = 1\}$ is not differentiable at $0$ by the first lemma; indeed, $\chi(0) = 1$.
That is a contradiction.
\end{proof}

Suppose that $\mathcal S_s = \mathcal S_t$, and let $\alpha$ be a chart in $\mathcal A$.
Since $\mathcal S_s = \mathcal S_t$, $\alpha \circ G_s^{-1}$ and $\alpha \circ G_t^{-1}$ are smoothly compatible, thus
$$(\alpha \circ G_t^{-1}) \circ (\alpha \circ G_s^{-1})^{-1} = \alpha \circ G_t^{-1} \circ G_s \circ \alpha^{-1}$$
is smooth. But $\alpha$ is a diffeomorphism, so $G_t^{-1} \circ G_s$ is smooth. By the lemma, then, $s = t$, which proves the claim.

\begin{exer}[1.9]
Show that $\PP_\CC^n$ can be given the structure of a compact smooth manifold of dimension $2n$.
\end{exer}

Let $\pi: \CC^{n+1} \setminus \{0\} \to \PP_\CC^n$ be the natural projection.
Since $\pi$ is a projection under the action of a group, namely $\CC^*$, $\pi$ is an open map.
Since $\pi$ is continuous, and remains surjective even when restricted to the (compact) closed unit ball of $\CC^{n+1}$, it immediately follows that $\PP_\CC^n$ is compact and second-countable.

\begin{lemma}
$\PP_\CC^n$ is Hausdorff.
\end{lemma}
\begin{proof}
Let $z = \pi(z_1, \dots, z_{n+1})$ and similarly for $w$.
We must show that $z,w$ are separated by open sets.
If there exists a chart $U_j$ such that $z, w \in U_j$, then $z,w$ are separated by open sets.
So suppose otherwise; that is, if $z_j \neq 0$ then $w_j = 0$.
After reordering the basis vectors of $\CC^{n+1}$ we can then assume that there is a $k$ such that $z_{k+1} = z_{k+2} = \cdots = z_{n+1} = 0$ and $w_1 = w_2 = \cdots = w_k = 0$.
If $B$ is a sufficiently small ball in $\CC^{n+1}$ around $(z_1, \dots, z_k, 0, \dots, 0)$, then $\overline B$ does not meet the hyperplane $z_{k+1} = z_{k+2} = \cdots = z_{n+1} = 0$.
Thus no representative of $w$ in $\CC^{n+1}$ is in $\overline B$, so $\pi(\overline B) = \overline{\pi(B)}$ does not contain $w$.
Thus $\pi(B)$ contains $z$ but not $w$, while the complement of $\overline{\pi(B)}$ contains $w$ but not $z$.
Therefore $z, w$ are separated by open sets.
\end{proof}

Now let $U_j = \{\pi(z_1, \dots, z_{n+1}) \in \PP_\CC^n: z_j \neq 0\}$.
The defining property of $U_j$ is invariant under scaling, so $U_j$ is well-defined.
We turn $U_j$ into a chart by defining
\begin{align*}
\varphi_j: U_j &\to \CC^n\\
\pi(z_1, \dots, z_{n+1}) &\mapsto \left(\frac{z_1}{z_j}, \dots, \frac{z_{j-1}}{z_j}, \frac{z_{j+1}}{z_j}, \dots, \frac{z_{n+1}}{z_j}\right).
\end{align*}
Since the value of $\varphi_j \circ \pi$ is invariant under rescaling, $\varphi_j$ is well-defined.

\begin{lemma}
The $\varphi_j$ are compatible smooth charts.
\end{lemma}
\begin{proof}
Stronger, we claim that the $\varphi_j$ are compatible \emph{holomorphic} charts in $n$ complex dimensions, which immediately implies smoothness in $2n$ real dimensions.

We first check that $\varphi_j$ is continuous. By the universal property of $\pi$, if
$$V_j = \{(z_1, \dots, z_{n+1}) \in \CC^{n+1}: z_j \neq 0\},$$
then $\varphi_j$ is continuous iff
$$\varphi \circ \pi: V_j \to \CC^n$$
is. But that is clear, since $\varphi \circ \pi$ is just projection onto the coordinates that are not equal to $j$, followed by division by the nonzero number $z_j$.

We now check that the change-of-coordinates maps are holomorphic. Indeed,
$$\varphi_j(\varphi_k^{-1}(z_1, \dots, z_n)) = \varphi_j(\pi(z_1, \dots, 1, \dots, z_n))$$
where the $1$ is in the $k$th entry of the vector. But that means that
$$\varphi_j(\varphi_k^{-1}(z_1, \dots, z_n)) = \left(\frac{z_1}{z_j}, \dots, \frac{1}{z_j}, \dots, \frac{z_n}{z_j}\right).$$
But $z_j$ is a nonzero coordinate function on the domain $\varphi_k(U_j \cap U_k)$ of $\varphi_j \circ \varphi_k^{-1}$, so division by $z_j$ is holomorphic.
\end{proof}

We conclude that the $\varphi_j$ generate a smooth structure on $\PP_\CC^n$.




\end{document}
