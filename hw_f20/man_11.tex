
% --------------------------------------------------------------
% This is all preamble stuff that you don't have to worry about.
% Head down to where it says "Start here"
% --------------------------------------------------------------

\documentclass[10pt]{article}

\usepackage[margin=.7in]{geometry}
\usepackage{amsmath,amsthm,amssymb}
\usepackage{enumitem}
\usepackage{tikz-cd}
\usepackage{mathtools}
\usepackage{amsfonts}
\usepackage{listings}
\usepackage{algorithm2e}
\usepackage{verse,stmaryrd}
\usepackage{fancyvrb}

% Number systems
\newcommand{\NN}{\mathbb{N}}
\newcommand{\ZZ}{\mathbb{Z}}
\newcommand{\QQ}{\mathbb{Q}}
\newcommand{\RR}{\mathbb{R}}
\newcommand{\CC}{\mathbb{C}}
\newcommand{\PP}{\mathbb P}
\newcommand{\FF}{\mathbb F}
\newcommand{\DD}{\mathbb D}
\renewcommand{\epsilon}{\varepsilon}

\newcommand{\Aut}{\operatorname{Aut}}
\newcommand{\cl}{\operatorname{cl}}
\newcommand{\ch}{\operatorname{ch}}
\newcommand{\Con}{\operatorname{Con}}
\newcommand{\coker}{\operatorname{coker}}
\newcommand{\CVect}{\CC\operatorname{-Vect}}
\newcommand{\Cantor}{\mathcal{C}}
\newcommand{\D}{\mathcal{D}}
\newcommand{\card}{\operatorname{card}}
\newcommand{\dbar}{\overline \partial}
\newcommand{\diam}{\operatorname{diam}}
\newcommand{\dom}{\operatorname{dom}}
\newcommand{\End}{\operatorname{End}}
\DeclareMathOperator*{\esssup}{ess\,sup}
\newcommand{\Hess}{\operatorname{Hess}}
\newcommand{\Hom}{\operatorname{Hom}}
\newcommand{\id}{\operatorname{id}}
\newcommand{\Ind}{\operatorname{Ind}}
\newcommand{\Inn}{\operatorname{Inn}}
\newcommand{\interior}{\operatorname{int}}
\newcommand{\lcm}{\operatorname{lcm}}
\newcommand{\mesh}{\operatorname{mesh}}
\newcommand{\LL}{\mathcal L_0}
\newcommand{\Leb}{\mathcal{L}_{\text{loc}}^2}
\newcommand{\Lip}{\operatorname{Lip}}
\newcommand{\ppic}{\vspace{35mm}}
\newcommand{\ppset}{\mathcal{P}}
\DeclareMathOperator{\proj}{proj}
\DeclareMathOperator*{\Res}{Res}
\newcommand{\Riem}{\mathcal{R}}
\newcommand{\RVect}{\RR\operatorname{-Vect}}
\newcommand{\Sch}{\mathcal{S}}
\newcommand{\sgn}{\operatorname{sgn}}
\newcommand{\spn}{\operatorname{span}}
\newcommand{\Spec}{\operatorname{Spec}}
\newcommand{\supp}{\operatorname{supp}}
\newcommand{\TT}{\mathcal T}
\DeclareMathOperator{\tr}{tr}

% Calculus of variations
\DeclareMathOperator{\pp}{\mathbf p}
\DeclareMathOperator{\zz}{\mathbf z}
\DeclareMathOperator{\uu}{\mathbf u}
\DeclareMathOperator{\vv}{\mathbf v}
\DeclareMathOperator{\ww}{\mathbf w}

% Categories
\newcommand{\Ab}{\mathbf{Ab}}
\newcommand{\Cat}{\mathbf{Cat}}
\newcommand{\Group}{\mathbf{Group}}
\newcommand{\Module}{\mathbf{Module}}
\newcommand{\Set}{\mathbf{Set}}
\DeclareMathOperator{\Fun}{Fun}
\DeclareMathOperator{\Iso}{Iso}

% Complex analysis
\renewcommand{\Re}{\operatorname{Re}}
\renewcommand{\Im}{\operatorname{Im}}

% Logic
\renewcommand{\iff}{\leftrightarrow}
\newcommand{\Henkin}{\operatorname{Henk}}
\newcommand{\PA}{\mathbf{PA}}
\DeclareMathOperator{\proves}{\vdash}

% Group
\DeclareMathOperator{\Gal}{Gal}
\DeclareMathOperator{\Fix}{Fix}
\DeclareMathOperator{\Lie}{Lie}
\DeclareMathOperator{\Out}{Out}

\DeclareMathOperator{\Diffeo}{Diffeo}

\newcommand{\GL}{\operatorname{GL}}
\newcommand{\ppGL}{\operatorname{PGL}}
\newcommand{\SL}{\operatorname{SL}}
\newcommand{\SO}{\operatorname{SO}}
\newcommand{\iprod}{\mathbin{\lrcorner}}


% Other symbols
\newcommand{\heart}{\ensuremath\heartsuit}
\newcommand{\club}{\ensuremath\clubsuit}

\DeclareMathOperator{\atanh}{atanh}
\DeclareMathOperator{\codim}{codim}

% Theorems
\theoremstyle{definition}
\newtheorem*{corollary}{Corollary}
\newtheorem*{falselemma}{Grader's ``Lemma"}
\newtheorem{exer}{Exercise}
\newtheorem{lemma}{Lemma}[exer]
\newtheorem{theorem}[lemma]{Theorem}

\def\Xint#1{\mathchoice
{\XXint\displaystyle\textstyle{#1}}%
{\XXint\textstyle\scriptstyle{#1}}%
{\XXint\scriptstyle\scriptscriptstyle{#1}}%
{\XXint\scriptscriptstyle\scriptscriptstyle{#1}}%
\!\int}
\def\XXint#1#2#3{{\setbox0=\hbox{$#1{#2#3}{\int}$ }
\vcenter{\hbox{$#2#3$ }}\kern-.6\wd0}}
\def\ddashint{\Xint=}
\def\dashint{\Xint-}

\usepackage[backend=bibtex,style=alphabetic,maxcitenames=50,maxnames=50]{biblatex}
\renewbibmacro{in:}{}
\DeclareFieldFormat{pages}{#1}

\begin{document}
\noindent
\large\textbf{Manifolds, HW 11} \hfill \textbf{Aidan Backus} \\

% --------------------------------------------------------------
%                         Start here
% --------------------------------------------------------------\

\begin{exer}[19.2]
Let $D$ be a distribution of rank $k$ which is annihilated by linearly independent $1$-forms $\omega^1, \dots, \omega^{n-k}$.
Show that $D$ is involutive iff for every $i \in \{1, \dots, n - k\}$ one has
\begin{equation}
\label{coframe involutive}
d\omega^i \wedge \bigwedge_{j=1}^{n-k} \omega^j = 0.
\end{equation}
\end{exer}

If $D$ is involutive, then it is completely integrable, so $\omega^i$ is a linear combination of $dx^j$, say $j \in \{k+1, \dots, n\}$.
But then it follows that $d\omega^i = 0$.

Conversely, if (\ref{coframe involutive}) holds then $d\omega^i$ is a linear combination of the $\omega^j$, so $d\omega^i$ annihilates $D$.
Since $i$ was arbitrary this implies that $D$ is involutive.

\begin{exer}[19.3]
Let $\omega$ be a nowhere vanishing $1$-form. Show that $\omega$ locally admits an integrating factor iff $d\omega \wedge \omega = 0$.
Show that on a surface, $\omega$ locally admits an integrating factor.
\end{exer}

Suppose that $d\omega \wedge \omega = 0$. Let $D$ be the distribution annihilated by $\omega$; then by the previous exercise, $D$ is involutive.
So $D$ is completely integrable, and we may choose coordinates in which $D$ is spanned by $\partial_2, \dots, \partial_n$.
So $\omega$ annihilates $\partial_2, \dots, \partial_n$, but $\omega$ vanishes nowhere, so there is a $\mu$ such that $\omega = \mu~dx^1$, which is obviously closed.
But locally, de Rham cohomology is trivial, so that implies that $\mu$ is an integrating factor.

Conversely, suppose that $\omega$ has an integrating factor in a small open set.
Then $d\omega = 0$, so $d\omega \wedge \omega = 0$.

If $\omega$ is a $1$-form on a surface, then $d\omega$ is a top form, so $\wedge d\omega$ kills $1$-forms. Therefore by the previous part, $\omega$ locally admits an integrating factor.

\begin{exer}[19.4]
Let $U$ be the upper octant of $\RR^3$ and let $D$ be the distribution spanned by
\begin{align*}
X &= y\frac{\partial}{\partial z} - z\frac{\partial}{\partial y},\\
Y &= z\frac{\partial}{\partial x} - x\frac{\partial}{\partial z}.
\end{align*}
Find a global flat chart for $D$ on $U$.
\end{exer}

Define $x' = x^2/2$. Then
$$\frac{\partial x'}{\partial x} = x$$
so
$$\frac{1}{x} \frac{\partial}{\partial x} = \frac{\partial}{\partial x'}.$$
Similarly define $y',z'$. Now notice that $D$ is spanned by
\begin{align*}
X' &= \frac{1}{x} \frac{\partial}{\partial x} - \frac{1}{z}\frac{\partial}{\partial z} = \frac{\partial}{\partial x'} - \frac{\partial}{\partial z'},\\
Y' &= \frac{1}{y} \frac{\partial}{\partial y} - \frac{1}{z} \frac{\partial}{\partial z} = \frac{\partial}{\partial y'} - \frac{\partial}{\partial z'}.
\end{align*}
Now setting $x'' = x' - z'$, $y'' = y' - z'$, $z'' = z'$ we see that $X' = x''$ and $Y' = y''$, so that $(x'', y'', z'')$ is a global coordinate frame for $D$.

\begin{exer}
Let $G$ be a connected Lie group with Lie algebra $\mathfrak g$.
Let $X, Y \in \mathfrak g$.
Show that $[X, Y] = 0$ iff for every $s, t \in \RR$, $\exp(tX)\exp(sY) = \exp(sY)\exp(tX)$.
Show that $G$ is abelian iff $\mathfrak g$ is.
Show that this fails if $G$ is not connected.
\end{exer}

For the first claim: by rescaling and renaming we may assume that either $s = t = 1$, $s = 1$ and $t = 0$, or $s = t = 0$.
The latter two cases are clear since $e^0 = 1$.
Now suppose that $[X, Y] = 0$. Then the flow $\varphi_X\varphi_Y = \varphi_{X + Y}$, but $\exp X = \varphi_X(1)$ and $\exp Y = \varphi_Y(1)$, so $\exp X \exp Y = \exp(X + Y) = \exp Y \exp X$.
Conversely if $\exp X \exp Y = \exp Y \exp X$ then $\varphi_X\varphi_Y = \varphi_{X + Y}$, so $[X, Y] = 0$.

Now if $G$ is abelian then it immediately follows that $\mathfrak g$ is abelian.
Conversely if $\mathfrak g$ is abelian, we just showed that $G$ is abelian near $1$.
If $G$ is connected then $G$ is generated by a small open set around $1$.

For the final claim, let $G$ be a nonabelian countable discrete group. Then $G$ is trivially Lie of dimension $0$ (so $\mathfrak g$ is abelian since $\mathfrak g = 0$).
To make this example less stupid one can take the product of $G$ by an abelian Lie group of positive dimension.

\end{document}
