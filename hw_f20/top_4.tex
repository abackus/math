
% --------------------------------------------------------------
% This is all preamble stuff that you don't have to worry about.
% Head down to where it says "Start here"
% --------------------------------------------------------------

\documentclass[10pt]{article}

\usepackage[margin=.7in]{geometry}
\usepackage{amsmath,amsthm,amssymb}
\usepackage{enumitem}
\usepackage{tikz-cd}
\usepackage{mathtools}
\usepackage{amsfonts}
\usepackage{listings}
\usepackage{algorithm2e}
\usepackage{verse,stmaryrd}
\usepackage{fancyvrb}

% Number systems
\newcommand{\NN}{\mathbb{N}}
\newcommand{\ZZ}{\mathbb{Z}}
\newcommand{\QQ}{\mathbb{Q}}
\newcommand{\RR}{\mathbb{R}}
\newcommand{\CC}{\mathbb{C}}
\newcommand{\PP}{\mathbb P}
\newcommand{\FF}{\mathbb F}
\newcommand{\DD}{\mathbb D}
\renewcommand{\epsilon}{\varepsilon}

\newcommand{\Aut}{\operatorname{Aut}}
\newcommand{\cl}{\operatorname{cl}}
\newcommand{\ch}{\operatorname{ch}}
\newcommand{\Con}{\operatorname{Con}}
\newcommand{\coker}{\operatorname{coker}}
\newcommand{\CVect}{\CC\operatorname{-Vect}}
\newcommand{\Cantor}{\mathcal{C}}
\newcommand{\D}{\mathcal{D}}
\newcommand{\card}{\operatorname{card}}
\newcommand{\dbar}{\overline \partial}
\newcommand{\diam}{\operatorname{diam}}
\newcommand{\dom}{\operatorname{dom}}
\newcommand{\End}{\operatorname{End}}
\DeclareMathOperator*{\esssup}{ess\,sup}
\newcommand{\GL}{\operatorname{GL}}
\newcommand{\Hess}{\operatorname{Hess}}
\newcommand{\Hom}{\operatorname{Hom}}
\newcommand{\id}{\operatorname{id}}
\newcommand{\Ind}{\operatorname{Ind}}
\newcommand{\Inn}{\operatorname{Inn}}
\newcommand{\interior}{\operatorname{int}}
\newcommand{\lcm}{\operatorname{lcm}}
\newcommand{\mesh}{\operatorname{mesh}}
\newcommand{\LL}{\mathcal L_0}
\newcommand{\Leb}{\mathcal{L}_{\text{loc}}^2}
\newcommand{\Lip}{\operatorname{Lip}}
\newcommand{\ppGL}{\operatorname{PGL}}
\newcommand{\ppic}{\vspace{35mm}}
\newcommand{\ppset}{\mathcal{P}}
\DeclareMathOperator{\proj}{proj}
\DeclareMathOperator*{\Res}{Res}
\newcommand{\Riem}{\mathcal{R}}
\newcommand{\RVect}{\RR\operatorname{-Vect}}
\newcommand{\Sch}{\mathcal{S}}
\newcommand{\SL}{\operatorname{SL}}
\newcommand{\sgn}{\operatorname{sgn}}
\newcommand{\spn}{\operatorname{span}}
\newcommand{\Spec}{\operatorname{Spec}}
\newcommand{\supp}{\operatorname{supp}}
\newcommand{\TT}{\mathcal T}
\DeclareMathOperator{\tr}{tr}

% Calculus of variations
\DeclareMathOperator{\pp}{\mathbf p}
\DeclareMathOperator{\zz}{\mathbf z}
\DeclareMathOperator{\uu}{\mathbf u}
\DeclareMathOperator{\vv}{\mathbf v}
\DeclareMathOperator{\ww}{\mathbf w}

% Categories
\newcommand{\Ab}{\mathbf{Ab}}
\newcommand{\Cat}{\mathbf{Cat}}
\newcommand{\Group}{\mathbf{Group}}
\newcommand{\Module}{\mathbf{Module}}
\newcommand{\Set}{\mathbf{Set}}
\DeclareMathOperator{\Fun}{Fun}
\DeclareMathOperator{\Iso}{Iso}
\DeclareMathOperator{\Map}{Map}

% Complex analysis
\renewcommand{\Re}{\operatorname{Re}}
\renewcommand{\Im}{\operatorname{Im}}

% Logic
\renewcommand{\iff}{\leftrightarrow}
\newcommand{\Henkin}{\operatorname{Henk}}
\newcommand{\PA}{\mathbf{PA}}
\DeclareMathOperator{\proves}{\vdash}

% Group
\DeclareMathOperator{\Gal}{Gal}
\DeclareMathOperator{\Fix}{Fix}
\DeclareMathOperator{\Out}{Out}

% Other symbols
\newcommand{\heart}{\ensuremath\heartsuit}
\newcommand{\club}{\ensuremath\clubsuit}

\DeclareMathOperator{\atanh}{atanh}

% Theorems
\theoremstyle{definition}
\newtheorem*{falselemma}{Grader's ``Lemma"}
\newtheorem{exer}{Exercise}
\newtheorem{lemma}{Lemma}[exer]
\newtheorem{theorem}[lemma]{Theorem}
\newtheorem{corollary}[lemma]{Corollary}

\def\Xint#1{\mathchoice
{\XXint\displaystyle\textstyle{#1}}%
{\XXint\textstyle\scriptstyle{#1}}%
{\XXint\scriptstyle\scriptscriptstyle{#1}}%
{\XXint\scriptscriptstyle\scriptscriptstyle{#1}}%
\!\int}
\def\XXint#1#2#3{{\setbox0=\hbox{$#1{#2#3}{\int}$ }
\vcenter{\hbox{$#2#3$ }}\kern-.6\wd0}}
\def\ddashint{\Xint=}
\def\dashint{\Xint-}

\usepackage[backend=bibtex,style=alphabetic,maxcitenames=50,maxnames=50]{biblatex}
\renewbibmacro{in:}{}
\DeclareFieldFormat{pages}{#1}

\begin{document}
\noindent
\large\textbf{Algebraic Topology, HW 4} \hfill \textbf{Aidan Backus} \\

% --------------------------------------------------------------
%                         Start here
% --------------------------------------------------------------\

I talked about most of these problems with Megan Chang-Lee, Steven Creech, Matthew Emerson and Nate Gillman.

\begin{exer}
Show that the second barycentric subdivision of a $\Delta$-complex is a simplicial complex.
\end{exer}

Following the hint, we first check the first barycentric subdivision.
Let $X$ be a $\Delta$-complex.

\begin{lemma}
The first barycentric subdivision of a $\Delta$-complex is a $\Delta$-complex such that every simplex has distinct vertices.
\end{lemma}
\begin{proof}
Let $Y$ be a simplex of $SX$ of dimension $n$.
We prove the lemma by induction on $n$. If $n = 1$, then $Y$ was obtained by subdividing a $1$-dimensional simplex $Y'$ of $X$ at its midpoint $y'$; $y'$ is not a vertex of $Y'$, so $y'$ is distinct from the other vertex of $Y$, which must be a vertex of $Y$.

If $n > 1$, suppose that
$$Y = [b, w_0, \dots, w_{n-1}]$$
was obtained from the $n$-dimensional simplex $Y'$. Then $b$ is a barycenter of $Y'$, so is not a vertex of $Y$; thus $b \neq w_i$.
Moreover, $[w_0, \dots, w_{n-1}]$ is the first barycentric subdivision of a $n-1$-dimensional simplex, so the $w_i$ are disinct.
\end{proof}

Since the second barycentric subdivision is a first barycentric subdivision, it follows that every simplex of $SSX$ has distinct vertices.
Thus we must show that any two simplices of $SSX$ have different vertex sets.
Let $Y,Z$ be two simplices of dimension $n$ of $SSX$ with the same vertex set $Y = [b, w_0, \dots, w_{n-1}]$, and let $Y',Z'$ be the simplicies of $SX$ we obtained $Y,Z$ from via barycentric subdivision; we show $Y = Z$ by induction on $n$.

If $n = 1$, so $Y = [b, w_0]$, either $Z = [b, w_0]$ or $Z = [w_0, b]$.
If $Z = [b, w_0]$ then $Y'$ and $Z'$ had the same vertex $w_0$ and the same vertex $w_1$ such that $(w_0 + w_1)/2 = b$; here $w_0 \neq w_1$ since $SX$ is a first barycentric subdivision.
In that case $Y'$ and $Z'$ were glued at $b$, so $Y' = Z'$ and hence $Y = Z$.
Otherwise, $Z'$ was glued to $Y'$ at the barycenter of $Y'$, a contradiction.

If $n > 1$, induction gives that $w_0, \dots, w_{n-1}$ determine the $n-1$-simplex $[w_0, \dots, w_{n-1}]$.
Reasoning as in the $n = 1$ case we see that $Z = [b, w_{\sigma(0)}, \dots, w_{\sigma(n-1)}]$ where $\sigma$ is a permutation of $n$.
By induction, the $n-1$-simplices $\tilde Z = [w_{\sigma(0)}, \dots, w_{\sigma(n-1)}]$ and $\tilde Y = [w_0, \dots, w_{n-1}]$ are equal.
Then $Y'$ and $Z'$ had the same vertex $w_n$ such that $\sum_{i \leq n} w_i/(n+1) = b$, so $Y'$ and $Z'$ were glued at $b$, thus $Y' = Z'$ and hence $Y = Z$.

\begin{exer}
Show that every $n$-simplex in the barycentric subdivision of $\Delta^n$ is defined by inequalities $t_{i_0} \leq \cdots \leq t_{i_n}$ in barycentric coordinates.
\end{exer}

We first need a fact about barycentric subdivision:
\begin{lemma}
If $[b, w_0, \dots, w_{n-1}]$ is a simplex in the barycentric subdivision of $[v_0, \dots, v_n]$ then there is a permutation $\sigma$ of $n$ such that for every $i \in \{0, \dots, n-1\}$,
$$w_{(n-1)-i} = \frac{1}{i+1} \sum_{j=0}^i v_{\sigma(j)},$$
where $b = \sum_i v_i/(n+1)$ is the barycenter of $[v_0, \dots, v_n]$.
\end{lemma}
\begin{proof}
We induct on $n$.
If $n = 0$, then the lemma then merely says that there exists an empty permutation (true by convention) and that the barycenter of $[v_0]$ is $v_0$ (obvious).

Now suppose that $n > 0$, that the result holds for any $n-1$-simplex, and let $[b, w_0, \dots, w_{n-1}]$ be a barycentric subdivision of $[v_0, \dots, v_{n-1}]$.
By definition, $[w_0, \dots, w_{n-1}]$ is the barycentric subdivision of a face $[v_0', \dots, v_{n-2}'] = [v_0, \dots, \widehat v_i, \dots, v_{n-1}]$ of $[v_0, \dots, v_n]$; we thus write $b' = w_0$, $w_i' = w_{i+1}$ for $i \in \{0, \dots, n-2\}$.
Then $b'$ is the barycenter of $[v_0', \dots, v_{n-2}']$, so by induction, there is a permutation $\sigma'$ of $n - 1$ such that for every $i \in \{0, \dots, n - 2\}$,
$$w_{(n-1)-i} = w_{((n-1)-1)-i}' = \frac{1}{i+1} \sum_{j=0}^i v_{\sigma'(j)}'.$$
For every $j$ there is a unique $k$ such that $v_{\sigma'(j)}' = v_k$ for some $k$, so we can set $\sigma(j) = k$.
Then, since $\sigma'$ is a permutation of $n - 1$, $\sigma(n - 1)$ is not defined; we set $\sigma(n) = i$, where $[v_0', \dots, v_{n-2}'] = [v_0, \dots, \widehat v_i, \dots, v_{n-1}]$.
Then
$$w_{(n-1)-i} = \frac{1}{i+1} \sum_{j=0}^i v_{\sigma(j)}$$
whenever $i \in \{0, \dots, n - 2\}$. As for the case $i = n -1$, we obtain
$$w_{(n-1)-i} = w_0 = b' = \frac{1}{n} \sum_{j=0}^{n-2} v_j' = \frac{1}{i+1} \sum_{j=0}^{i-1} v_{\sigma(j)}$$
since $\sigma(n-1)$ was omitted from the $n-1$-simplex in consideration, so that $v_{\sigma(j)}$ ranges over $v_k'$, $k \in \{0, \dots, n - 2\}$.
This implies the lemma.
\end{proof}

We induct on $n$. If $n = 1$, then the left half-segment is defined by $t_{i_1} \leq t_{i_0}$ and the right is defined by $t_{i_0} \leq t_{i_1}$.

If $n > 1$, consider the $n$-simplex $[b, w_0, \dots, w_{n-1}]$ arising from a barycentric subdivision of $[e_0, \dots, e_n]$, and let $\sigma$ be as in the lemma applied to $v_i = e_i$.
Let $x \in [b, w_0, \dots, w_{n-1}]$. In barycentric coordinates with respect to $[b, w_0, \dots, w_{n-1}]$, one has
$$x = \frac{1}{n+1}\left(x_nb_n + \sum_{j=0}^{n-1} x_jw_j \right).$$
Expanding using the lemma,
$$x = \frac{1}{n+1}\left(\frac{x_n}{n+1}\sum_{k=0}^n e_{\sigma(k)} + \sum_{j=0}^{n-1} \frac{x_j}{n - j} \sum_{k=0}^{n-1-j} e_{\sigma(k)} \right).$$
Now we let $t_j$ be the coefficient of $e_j$ in the above expansion.
Then $(t_j)$ gives barycentric coordinates for $x$ in $[e_0, \dots, e_n]$, and since
\begin{align*}
x &= \frac{1}{n+1}\left(\frac{x_n}{n+1}\sum_{k=0}^n e_{\sigma(k)} + \sum_{j=0}^{n-1} \frac{x_j}{n - j} \sum_{k=0}^{n-1-j} e_{\sigma(k)} \right)\\
&= \frac{1}{n+1}\left(\frac{x_n}{n+1}\sum_{k=0}^n e_{\sigma(k)} + \sum_{j=0}^{n-1} \sum_{k=0}^{n-1} \delta_{0 \leq k \leq n-j-1} \frac{x_j}{n-j} e_{\sigma(k)}\right)\\
&= \frac{1}{n+1} \sum_{k=0}^n \left(x_n + \sum_{j=n-k-1}^{n-1} \frac{x_j}{n-j}\right) e_{\sigma(k)},
\end{align*}
one has
$$t_{\sigma(k)} = \frac{1}{n+1} \left(x_n + \sum_{j=n-k-1}^{n-1} \frac{x_j}{n-j}\right).$$
As $k$ increases, there are more terms in the above sum, and since all terms are nonnegative,
$$t_{\sigma(0)} \leq t_{\sigma(1)} \leq \cdots \leq t_{\sigma(n)},$$
which means that we can set $i_j = \sigma(j)$ to complete the proof.

\begin{exer}
Find a noninductive formula for the barycentric subdivision operator $S: C_n(X) \to C_n(X)$.
\end{exer}

Since $C_n(X)$ is a free abelian group, we can compute $S\lambda$ just in case $\lambda: \Delta^n \to X$ is a generator of $C_n(X)$, thus $\lambda$ has the form
$$\lambda = [v_0, \dots, v_n].$$
Let $n!$ be the symmetric group on $n$, viewed as acting on $\{0, \dots, n-1\}$.
We view $\sgn$ as a morphism of groups $\sgn: n! \to \{-1, 1\}$ (thus $\{-1, 1\}$ is multiplicative).
In that case,
$$S[v_0, \dots, v_n] = \sum_{\sigma \in (n+1)!} \sgn \sigma \cdot
\left[\frac{1}{n+1} \sum_{j=0}^n v_{\sigma(j)}, \frac{1}{n} \sum_{j=1}^n v_{\sigma(j)}, \dots, \frac{v_{\sigma(n-1)} + v_{\sigma(n)}}{2}, v_{\sigma(n)} \right].$$
In general, the $i$th vertex in the $\sigma$th summand in the above formula is
$$\frac{1}{n+1-i} \sum_{j=i}^n v_{\sigma(j)}.$$
We prove that this formula is valid by induction on $n$. If $n = 0$, then $S[v_0] = [v_0]$, $1! = \{\id\}$, and
$$\left[\frac{1}{n+1} \sum_{j=0}^n v_{\sigma(j)}, \frac{1}{n} \sum_{j=1}^n v_{\sigma(j)}, \dots, \frac{v_{\sigma(n-1)} + v_{\sigma(n)}}{2}, v_{\sigma(n)} \right] = [v_0],$$
so the claim holds.
If $n > 0$, suppose that for every $n-1$-simplex $[w_0, \dots, w_{n-1}]$,
$$S[w_0, \dots, w_{n-1}] = \sum_{\sigma \in n!} \sgn \sigma \cdot \left[\frac{1}{n} \sum_{j=0}^{n-1} w_{\sigma(j)}, \dots, w_{\sigma(n)}\right].$$
Let $b$ be the barycenter of $\lambda$, which we identify with the map
\begin{align*}
b: C_{n-1}(X) &\to C_n(X)\\
[w_0, \dots, w_{n-1}] &\mapsto [b, w_0, \dots, w_{n-1}]
\end{align*}
as in the proof of the excision theorem. By definition,
$$S\lambda = b(S\partial\lambda)$$
and one has
$$\partial\lambda = \sum_{i=0}^n (-1)^i [v_0, \dots, \widehat v_i, \dots, v_n].$$
If $\tau \in n!$ we define $\tau_i: \{0, \dots, \widehat i, \dots, n\} \to \{0, \dots, \widehat i, \dots, n\}$ by $\tau_i = f_i^{-1}\tau f_i$ where $f_i$ is the unique order-preserving bijection $\{0, \dots, \widehat i, \dots, n\} \to \{0, \dots, n - 1\}$. Then
$$S\partial\lambda = \sum_{i=0}^n (-1)^i \sum_{\tau \in n!} \sgn \tau \cdot
\left[\frac{1}{n} \sum_{\substack{0 \leq j \leq n\\j \neq i}} v_{\tau_i(j)}, \dots, v_{\tau_i(n)}\right].$$
If $\tau \in n!$ then
$$b = \frac{1}{n+1} \left(v_i + \sum_{\substack{0 \leq j \leq n\\j \neq i}} v_{\tau_i(j)}\right)$$
so it follows that
$$S\lambda = \sum_{i=0}^n (-1)^i \sum_{\tau \in n!} \sgn \tau \cdot
\left[\frac{1}{n+1} \left(v_i + \sum_{\substack{0 \leq j \leq n\\j \neq i}} v_{\tau_i(j)}\right) , \frac{1}{n} \sum_{\substack{0 \leq j \leq n\\j \neq i}} v_{\tau_i(j)}, \dots, v_{\tau_i(n)} \right].
$$
Given $i$ and $\tau \in n!$ we define $\sigma \in (n+1)!$ by first extending $\tau_i$ to an element of $(n+1)!$ by setting $\tau_i(i) = i$, setting $\rho_i$ to swap $0, i$, and defining $\sigma = \rho_i\tau_i$. Then
$$\sgn \sigma = (-1)^i \sgn \tau$$
and so the claim follows.

\begin{exer}
Let $f: (X, A) \to (Y, B)$ be a map of pairs such that $f: X \to Y$ and $f|A: A \to B$ are both homotopy equivalences.
Show that $f_*: H_n(X, A) \to H_n(Y, B)$ is an isomorphism.
Show that in case $f$ is the inclusion $(D^n, S^{n-1}) \to (D^n, D^n \setminus 0)$, $f$ is not a homotopy equivalence of pairs.
\end{exer}

Since $f$ is a map of pairs, $f_*$ is a chain map, thus the diagram
$$\begin{tikzcd}
\cdots \arrow[r]& H_n(A) \arrow[d,"f_*"] \arrow[r]& H_n(X) \arrow[d,"f_*"] \arrow[r]& H_n(X, A) \arrow[d,"f_*"] \arrow[r]& H_{n-1}(A) \arrow[d,"f_*"] \arrow[r]& H_{n-1}(X) \arrow[d,"f_*"] \arrow[r]&\cdots \\
\cdots \arrow[r]& H_n(B) \arrow[r]& H_n(Y) \arrow[r]& H_n(Y, B) \arrow[r]& H_{n-1}(B) \arrow[r]& H_{n-1}(Y) \arrow[r]&\cdots
\end{tikzcd}$$
commutes with exact rows.
Since $f$ is a homotopy equivalence on $X$ and on $A$, the maps $f_*: H_k(A) \to H_k(B)$ and $f_*: H_k(X) \to H_k(Y)$ are all isomorphisms, so by the five lemma, $f_*: H_n(X, A) \to H_n(Y, B)$.

Now if the inclusion $f: (D^n, S^{n-1}) \to (D^n, D^n \setminus 0)$ is a homotopy equivalence of pairs, there is then a homotopy equivalence
$$g: (D^n, D^n \setminus 0) \to (D^n, S^{n-1}).$$
Viewed as a map on $D^n$, $f$ is the identity, so since $gf$ is homotopic to the identity of $D^n$ through maps of pairs, so is $g$. Thus $g$ is homotopic to the identity on $D^n$ through maps which map $D^n \setminus 0$ to $S^{n-1}$.
But $0$ is a limit point of $D^n \setminus 0$, so all the maps in the homotopy $g \to \id_{D^n}$ must map $0$ into $S^{n-1}$.
This is a contradiction since $\id_{D^n}$ does not map $0$ to $S^{n-1}$.

\begin{exer}
Let $X$ be the cone on the $1$-skeleton of $\Delta^3$. Compute the local homology of $X$.
Let $\partial X$ be the space of points of $X$ with zero local homology and compute the local homology of $\partial X$.
Determine which subsets $A \subseteq X$ have the property that for every homeomorphism $f: X \to X$, $f(A) \subseteq A$.
\end{exer}

Let $A$ denote the $1$-skeleton of $\Delta^3$.
By definition, $X$ consists of $A$, an additional vertex $x_0$, and a line segment from each point of $A$ to $x_0$.
The line segments from vertices to $x_0$ are edges, and the line segments from points on edges of $A$ to $x_0$ comprise faces which are bounded by the edges from vertces to $x_0$.
There are five types of points in $X$:
\begin{enumerate}
\item $x_0$.
\item A vertex of $A$.
\item A point on the interior of an edge of $A$.
\item A point on the interior of an edge not of $A$.
\item A point on the interior of a face.
\end{enumerate}
When we say ``manifold" we mean ``manifold with boundary"
We will need two preliminary computations.
\begin{lemma}
If $M$ is an manifold then the reduced local homology at $x \in M$ satisfies
$$\widetilde H_k(M, M \setminus x) = \begin{cases}
\ZZ, k = \dim M\text{ and } x \notin \partial M\text{ and } \dim M \geq 2\\
\ZZ^2, k = \dim M = 1\text{ and } x \notin \partial M\\
0, \text{ else}.
\end{cases}$$
\end{lemma}
\begin{proof}
First suppose $x \notin \partial M$. Then we can find a contractible chart $U \ni x$, and, excising $M \setminus U$,
$$\widetilde H_k(M, M \setminus x) = \widetilde H_k(U, U \setminus x) = \widetilde H_k(D^n, D^n \setminus 0).$$
Using the long exact sequence
$$\begin{tikzcd}
\cdots \arrow[r] & H_k(D^n) \arrow[r] & H_k(D^n, D^n \setminus 0) \arrow[r] & H_{k-1}(D^n \setminus 0) \arrow[r] & H_{k-1}(D^n) \arrow[r]& \cdots
\end{tikzcd}$$
which simplifies to $$\begin{tikzcd}
\cdots \arrow[r] & 0 \arrow[r] & H_k(D^n, D^n \setminus 0) \arrow[r] & H_{k-1}(S^{n-1}) \arrow[r] & 0 \arrow[r] &\cdots
\end{tikzcd}$$
since $S^{n-1}$ is a strong deformation retract of $D^n \setminus 0$ and $D^n$ is contractible, we get an isomorphism $H_k(D^n, D^n \setminus 0) \to H_{k-1}(S^{n-1})$, which solves the problem in this case, since the homology of $S^n$ is concentrated in degree $n$ with
$$H_n(S^n) = \begin{cases}
\ZZ^2, n = 1\\
\ZZ, \text{ else}.
\end{cases}$$

If $x \in \partial M$, we repeat the same argument but get a long exact sequence
$$\begin{tikzcd}
\cdots \arrow[r] & H_k(D^n_+) \arrow[r] & H_k(D^n_+, D^n_+ \setminus 0) \arrow[r] & H_{k-1}(D^n_+ \setminus 0) \arrow[r] & H_{k-1}(D^n_+) \arrow[r]& \cdots
\end{tikzcd}$$
where $X_+ = X \cap \overline{\RR^d_+}$ whenever $X \subseteq \RR^d$ and $\RR^d_+$ is a half-space. This long exact sequence simplifies to
$$\begin{tikzcd}
\cdots \arrow[r] & 0 \arrow[r] & H_k(D^n_+, D^n_+ \setminus 0) \arrow[r] &0 \arrow[r] & 0 \arrow[r]& \cdots
\end{tikzcd}$$
since $S^n_+ \cong D^n$ is contractible, which implies that $H_k(D^n_+, D^n_+ \setminus 0) = 0$.
\end{proof}
\begin{lemma}
The homology of $A$ satisfies
$$\widetilde H_k(A) = \begin{cases}
\ZZ^3, k = 1\\
0, \text{ else}.
\end{cases}$$
\end{lemma}
\begin{proof}
Since $\Delta^3$ has $6$ edges and $4$ vertices, $A$ is the complete graph on $4$ vertices $v_0, \dots, v_3$.
The edges $(v_0, v_1), (v_1, v_2), (v_2, v_3)$ form a spanning tree, and retracting the spanning tree leaves $3$ loops based at a single vertex; thus $A$ is homotopy equivalent to a $3$-petaled rose $(S^1)^{\vee 3}$.
Applying the Hurewicz theorem to $\pi_1((S^1)^{\vee 3}) = \ZZ^{*3}$ immediately gives the claim.
\end{proof}

We compute the local homology of each such point $x \in X$:
\begin{enumerate}
\item There is a strong deformation retract $X \setminus x_0 \to A$ obtained by retracting all faces, and $X$ is contractible, so the long exact sequence
$$\begin{tikzcd}
\cdots \arrow[r] & H_n(X) \arrow[r] & H_n(X, X \setminus x_0) \arrow[r] & H_{n-1}(X \setminus x_0) \arrow[r] & H_{n-1}(X) \arrow[r] & \cdots
\end{tikzcd}$$
simplifies to
$$\begin{tikzcd}
\cdots \arrow[r] & 0 \arrow[r] & H_n(X, X \setminus x_0) \arrow[r] & H_{n-1}(A) \arrow[r] & 0 \arrow[r] & \cdots
\end{tikzcd}$$
and so induces an isomorphism $H_n(X, X \setminus x_0) \to H_{n-1}(A)$. Therefore local homology at $x_0$ is concentrated in degree $2$ with $H_2(X, X \setminus x_0) \cong \ZZ^3$.
\item Let $x$ be a vertex of $A$; then $x$ meets three faces. Let $B$ be a small ball around $X$, and excise all of $X$ except $B$. Retracting one of the faces, we see that $(B, B \setminus x)$ is homotopy equivalent to a pointed manifold with boundary, with $x$ on the boundary. Therefore the local homology at $x$ is trivial.
\item Let $x$ be in the interior of an edge of $A$; then $x$ is on the boundary of a single face, and excising all but a small ball $B$ around $x$, we again see that $(B, B \setminus x)$ is a pointed manifold with boundary containing $x$, so the local homology at $x$ is trivial.
\item Let $x$ be in the interior of an edge connecting a vertex $v_0 \in A$ to $x_0$. Then $x$ meets three faces, one for each edge from $v$ to another vertex $v_i$, $i \geq 1$, in $A$.
We first excise the remaining three faces, so we just need to treat the homology of the pair $(Y, Y \setminus x)$ where $Y$ consists of three faces glued along an edge $[v_0, x_0] \ni x$.
As in the first case,
$$H_n(Y, Y \setminus x) = H_{n-1}(Y \setminus x)$$
since $Y$ is contractible. But there is a strong deformation retract $Y \setminus x \to Y \setminus (v, x_0)$.
Now $Y \setminus (v_0, x_0)$ consists of three faces that are glued at the two endpoints $v_0, x_0$.
We can then retract each face $[x_0, v_0, v_i]$ onto the edges $[v_0, v_i] \cup [v_i, x_0]$, and absorb the edge $v_i$ into the edges $[v_0, v_i] \cup [v_i, x_0]$ to replace $Y \setminus (v_0, x_0)$ with a graph with vertices $v_0, x_0$ and three edges from $v_0$ to $x_0$.
Selecting such an edge to be a spanning tree and retracting it, we obtain a graph with one vertex and two loops. So the homology of $Y \setminus x$ is concentrated in degree $1$ with $H_1(Y \setminus x) \cong \ZZ^2$, by the Hurewicz theorem.
So the local homology at $x$ is concentrated in degree $2$ with $H_2(X, X \setminus x) \cong \ZZ^2$.
\item A face is a $2$-manifold, so if $x$ is an interior point of a face then homology is concentrated in degree $2$ with $\widetilde H_2(X, X \setminus x) \cong \ZZ$.
\end{enumerate}

In particular $\partial X$ consists exactly of the edges and vertices of $\Delta^3$, thus $\partial X = A$.
We again split into cases to compute the local homology $H(\partial X, \partial X \setminus x)$:
\begin{enumerate}
\item If $x$ is a vertex of $\Delta^3$, then we can excise all but a small ball $B$ around $x$ in $A$; then $B$ consists of three edges coming out of $x$.
So $B$ is contractible and there is a strong deformation retract $B \setminus x \to Z$ where $Z$ is a three-point space.
The usual long exact sequence argument gives an isomorphism $H_n(B, B \setminus x) \to H_{n-1}(\{0, 1, 2\})$ so the local homology at $x$ is concentrated in degree $1$ with $H_1(\partial X, \partial X \setminus x) \cong \ZZ^3$.
\item If $x$ is a point on an edge of $\Delta^3$, then we can excise all but the interior of the edge to obtain a $1$-manifold. Therefore homology is concentrated in degree $1$ with $H_1(\partial X, \partial X \setminus x) \cong \ZZ^2$.
\end{enumerate}

We therefore conclude:
\begin{theorem}
Let $F: X \to X$ be a homeomorphism. Then:
\begin{enumerate}
\item The barycenter $x_0$ is a fixed point of $F$.
\item If $x \in X$ is in the interior of an edge linking $x_0$ to a vertex of $A$, then so is $F(x)$.
\item If $x \in X$ is on a face linking $x_0$ to an edge of $A$, then so is $F(x)$.
\item If $x \in X$ is a vertex of $A$, then so is $F(x)$.
\item If $x \in X$ is in the interior of an edge of $A$, then so is $F(x)$.
\end{enumerate}
In particular, $F$ restricts to a homeomorphism $A \to A$.
\end{theorem}
\begin{proof}
First, $x_0$ is the unique point with local homology $H_2(X, X \setminus x_0) \cong \ZZ^3$. Similarly, the interior of edges in $X$ that are not in $A$ are the only places with local homology $H_2(X, X \setminus x_0) \cong \ZZ^2$; and the interiors of faces those with $H_2(X, X \setminus x_0) \cong \ZZ$.

In particular, $A = \partial X$ is sent to itself, so we can consider which local homology must also be preserved by $F|\partial X$.
So let $x \in \partial X$.
If $x$ is a vertex then $H_1(\partial X, \partial X \setminus x) \cong \ZZ^2$; otherwise $H_1(\partial X, \partial X \setminus x) \cong \ZZ^3$, so $F|\partial X$ must preserve vertices of $\Delta^3$.
\end{proof}


\begin{exer}
Show that $S^1 \times S^1$ and $S^1 \vee S^1 \vee S^2$ have isomorphic homology but their universal covers do not.
\end{exer}

First,
$$\widetilde H_n(S^1 \times S^1) = \begin{cases}
\ZZ^2, &n = 1\\
\ZZ, &n = 2\\
0, &n \geq 3
\end{cases}$$
since $S^1 \times S^1$ is a torus. Since $S^1 \vee S^1 \vee S^2$ is a wedge sum,
$$\widetilde H_n(S^1 \vee S^1 \vee S^2) = \widetilde H_n(S^1)^{\oplus 2} \oplus \widetilde H_n(S^2) = \begin{cases}
\ZZ^2, &n = 1\\
\ZZ, &n = 2\\
0, &n \geq 3.
\end{cases}$$
Therefore the homologies of the two spaces are isomorphic.

However, the universal cover of $S^1 \times S^1$ is contractible, while the universal cover $X$ of $S^1 \vee S^1 \vee S^2$ has, as a $1$-skeleton, the Cayley graph of $\ZZ * \ZZ$; one then obtains $X$ by adjoining an $S^2$ to every vertex.
Therefore $\widetilde H_2(X)$ is not even finitely generated.



\end{document}
