
% --------------------------------------------------------------
% This is all preamble stuff that you don't have to worry about.
% Head down to where it says "Start here"
% --------------------------------------------------------------

\documentclass[10pt]{article}

\usepackage[margin=.7in]{geometry}
\usepackage{amsmath,amsthm,amssymb}
\usepackage{enumitem}
\usepackage{tikz-cd}
\usepackage{mathtools}
\usepackage{amsfonts}
\usepackage{listings}
\usepackage{algorithm2e}
\usepackage{verse,stmaryrd}
\usepackage{fancyvrb}

% Number systems
\newcommand{\NN}{\mathbb{N}}
\newcommand{\ZZ}{\mathbb{Z}}
\newcommand{\QQ}{\mathbb{Q}}
\newcommand{\RR}{\mathbb{R}}
\newcommand{\CC}{\mathbb{C}}
\newcommand{\PP}{\mathbb P}
\newcommand{\FF}{\mathbb F}
\newcommand{\DD}{\mathbb D}
\renewcommand{\epsilon}{\varepsilon}

\newcommand{\Aut}{\operatorname{Aut}}
\newcommand{\cl}{\operatorname{cl}}
\newcommand{\ch}{\operatorname{ch}}
\newcommand{\Con}{\operatorname{Con}}
\newcommand{\coker}{\operatorname{coker}}
\newcommand{\CVect}{\CC\operatorname{-Vect}}
\newcommand{\Cantor}{\mathcal{C}}
\newcommand{\D}{\mathcal{D}}
\newcommand{\card}{\operatorname{card}}
\newcommand{\dbar}{\overline \partial}
\newcommand{\diam}{\operatorname{diam}}
\newcommand{\dom}{\operatorname{dom}}
\newcommand{\End}{\operatorname{End}}
\DeclareMathOperator*{\esssup}{ess\,sup}
\newcommand{\Hess}{\operatorname{Hess}}
\newcommand{\Hom}{\operatorname{Hom}}
\newcommand{\id}{\operatorname{id}}
\newcommand{\Ind}{\operatorname{Ind}}
\newcommand{\Inn}{\operatorname{Inn}}
\newcommand{\interior}{\operatorname{int}}
\newcommand{\lcm}{\operatorname{lcm}}
\newcommand{\mesh}{\operatorname{mesh}}
\newcommand{\LL}{\mathcal L_0}
\newcommand{\Leb}{\mathcal{L}_{\text{loc}}^2}
\newcommand{\Lip}{\operatorname{Lip}}
\newcommand{\ppic}{\vspace{35mm}}
\newcommand{\ppset}{\mathcal{P}}
\DeclareMathOperator{\proj}{proj}
\DeclareMathOperator*{\Res}{Res}
\newcommand{\Riem}{\mathcal{R}}
\newcommand{\RVect}{\RR\operatorname{-Vect}}
\newcommand{\Sch}{\mathcal{S}}
\newcommand{\sgn}{\operatorname{sgn}}
\newcommand{\spn}{\operatorname{span}}
\newcommand{\Spec}{\operatorname{Spec}}
\newcommand{\supp}{\operatorname{supp}}
\newcommand{\TT}{\mathcal T}
\DeclareMathOperator{\tr}{tr}

% Calculus of variations
\DeclareMathOperator{\pp}{\mathbf p}
\DeclareMathOperator{\zz}{\mathbf z}
\DeclareMathOperator{\uu}{\mathbf u}
\DeclareMathOperator{\vv}{\mathbf v}
\DeclareMathOperator{\ww}{\mathbf w}

% Categories
\newcommand{\Ab}{\mathbf{Ab}}
\newcommand{\Cat}{\mathbf{Cat}}
\newcommand{\Group}{\mathbf{Group}}
\newcommand{\Module}{\mathbf{Module}}
\newcommand{\Set}{\mathbf{Set}}
\DeclareMathOperator{\Fun}{Fun}
\DeclareMathOperator{\Iso}{Iso}

% Complex analysis
\renewcommand{\Re}{\operatorname{Re}}
\renewcommand{\Im}{\operatorname{Im}}

% Logic
\renewcommand{\iff}{\leftrightarrow}
\newcommand{\Henkin}{\operatorname{Henk}}
\newcommand{\PA}{\mathbf{PA}}
\DeclareMathOperator{\proves}{\vdash}

% Group
\DeclareMathOperator{\Gal}{Gal}
\DeclareMathOperator{\Fix}{Fix}
\DeclareMathOperator{\Lie}{Lie}
\DeclareMathOperator{\Out}{Out}

\DeclareMathOperator{\Diffeo}{Diffeo}

\newcommand{\GL}{\operatorname{GL}}
\newcommand{\ppGL}{\operatorname{PGL}}
\newcommand{\SL}{\operatorname{SL}}
\newcommand{\SO}{\operatorname{SO}}
\newcommand{\iprod}{\mathbin{\lrcorner}}


% Other symbols
\newcommand{\heart}{\ensuremath\heartsuit}
\newcommand{\club}{\ensuremath\clubsuit}

\DeclareMathOperator{\atanh}{atanh}
\DeclareMathOperator{\codim}{codim}

% Theorems
\theoremstyle{definition}
\newtheorem*{corollary}{Corollary}
\newtheorem*{falselemma}{Grader's ``Lemma"}
\newtheorem{exer}{Exercise}
\newtheorem{lemma}{Lemma}[exer]
\newtheorem{theorem}[lemma]{Theorem}

\def\Xint#1{\mathchoice
{\XXint\displaystyle\textstyle{#1}}%
{\XXint\textstyle\scriptstyle{#1}}%
{\XXint\scriptstyle\scriptscriptstyle{#1}}%
{\XXint\scriptscriptstyle\scriptscriptstyle{#1}}%
\!\int}
\def\XXint#1#2#3{{\setbox0=\hbox{$#1{#2#3}{\int}$ }
\vcenter{\hbox{$#2#3$ }}\kern-.6\wd0}}
\def\ddashint{\Xint=}
\def\dashint{\Xint-}

\usepackage[backend=bibtex,style=alphabetic,maxcitenames=50,maxnames=50]{biblatex}
\renewbibmacro{in:}{}
\DeclareFieldFormat{pages}{#1}

\begin{document}
\noindent
\large\textbf{Manifolds, HW 10} \hfill \textbf{Aidan Backus} \\

% --------------------------------------------------------------
%                         Start here
% --------------------------------------------------------------\

\begin{exer}[17.2]
Let $M$ be an oriented compact Riemannian manifold. Let $\Delta: \Omega^p(M) \to \Omega^p(M)$ be the Laplace-Beltrami operator
$$\Delta = d^*d + dd^*$$
where $d^*$ is the codifferential. Say that $\omega$ is harmonic iff $\Delta \omega = 0$. Show that the following are equivalent:
\begin{enumerate}
\item $\omega$ is harmonic.
\item $\omega$ is closed and coclosed.
\item $\omega$ is closed and $\omega$ is the unique minimizer in the de Rham cohomology class $[\omega]$ of the $L^2$-norm.
\end{enumerate}
\end{exer}

Clearly $2 \implies 1$. For the converse, note that
$$(\Delta \omega, \omega) = ||d\omega||_{L^2}^2 + ||d^*\omega||_{L^2}^2 \geq 0.$$
So if $\Delta \omega = 0$ it must be the case that $d\omega = d^*\omega = 0$.
Therefore $1 \iff 2$.

\begin{lemma}
If $[\omega]$ is a de Rham cohomology class, there is at most one representative, say $\omega$, of $[\omega]$, such that $\Delta \omega = 0$.
\end{lemma}
\begin{proof}
Let $\omega, \omega'$ be harmonic representatives of $[\omega]$, and assume that $\omega$ is a $p$-form.
If $p = 0$ then $\omega,\omega'$ are functions, which are constant since $1 \implies 2$, but $[\omega]$ contains at most one constant representative, so $\omega = \omega'$.

If $p \neq 0$, then since $\omega - \omega'$ is exact, there is a $p-1$-form $\eta$ such that $d\eta = \omega - \omega'$. Then ]
$$||\omega - \omega'||_{L^2}^2 = (\omega - \omega', d\eta) = (d^*(\omega - \omega'), \eta) = 0$$
since $1 \implies 2$.
\end{proof}

Now we can show $2 \iff 3$. The Euler-Lagrange equation for $||\cdot||_{L^2}^2$ are
$$0 = \frac{d}{dt}(\eta + t~d\beta, \eta + t~d\beta)|_{t=0}.$$
But this simplifies to
$$0 = (\eta, d\beta)$$
thus $d^*\eta = 0$.

\begin{exer}[17.6]
Let $M$ be a connected manifold, $\dim M \geq 3$, $x \in M$, $p \in [0, n - 2]$.
Show that the pullback $H^p(M) \to H^p(M \setminus x)$ is an isomorphism.
Show that the same is true if $M$ is compact and orientable, and $p = n - 1$.
\end{exer}

Let $U = M \setminus x$ and $V$ a small ball around $x$.
Let $n = \dim M$.
Then $V$ is contractible and $U \cap V = V \setminus x$ is homotopic to $S^{n-1}$, so
$$H^p(U \cap V) = \begin{cases}
\RR, p \in \{0, n - 1\}\\
0, \text{ else}
\end{cases}$$
and the de Rham cohomology of $V$ is concentrated in degree $0$ with $H^0(V) = 1$.
Let $i: U \to M$ and $j: V \to M$ be the inclusion maps.

First suppose that $p \in [2, n - 2]$, so that
$$H^{p-1}(U \cap V) = H^p(U \cap V) = H^p(V) = 0.$$
The Mayer-Vietoris sequence of $\{U, V\}$ reads
$$\begin{tikzcd}
\cdots \arrow[r] & H^{p-1}(U \cap V) \arrow[r] & H^p(M) \arrow[r,"i^* \oplus j^*"] & H^p(U) \oplus H^p(V) \arrow[r] & H^p(U \cap V) \arrow[r] & \cdots;
\end{tikzcd}$$
by assumption, $j^* = 0$ and $i^*$ is an isomorphism $H^p(M) \to H^p(U)$, which was desired.

Now we treat the edge cases; it will be helpful to have the following lemma:
\begin{lemma}
$i^*$ is injective.
\end{lemma}
\begin{proof}
Let $\omega$ be a closed $p$-form in $M$ which is exact in $M \setminus x$. Suppose that $d\omega = \eta$ in $M$. Then the support of $\eta$ is concentrated at $\{x\}$, since $\eta|M \setminus x = 0$.
However, in a chart $U$ near $x$, $\eta = f_I ~dy^I$ for some coordinate frame $dy$ and smooth map $f: U \to \RR^p$.
So $f$ is supported at a point, which contradicts that $f$ is continuous.
Therefore $\eta = 0$.
So if $i^*[\omega] = 0$, then $[\omega] = 0$.
\end{proof}

\begin{lemma}
The claim holds for $p = 0$.
\end{lemma}
\begin{proof}
We can show that $\dim H^0(M) = \dim H^0(U) = 1$; since $i^*$ is injective, it follows that $i^*$ is an isomorphism.
Since $M$ is connected, $H^0(M) = \RR$; we must show, then, that $M \setminus x$ is connected.
Let $y_1, y_2 \in M \setminus x$; then, since $M$ is connected, there is a path $\gamma$ from $y_1$ to $y_2$ in $M$.
If $\gamma$ misses $x$, then we're done, so assume that $\gamma$ passes through $x$.
We need to perturb $\gamma$ to avoid $x$; working in local coordinates, this follows from the assumption that $\dim M \geq 3 > 1$.
\end{proof}

\begin{lemma}
The claim holds for $p = 1$.
\end{lemma}
\begin{proof}
We must show that $i^*$ is surjective, so let $\eta$ be a closed $1$-form on $M \setminus x$ which is not exact.
(If no such form exists, then $H^1(M \setminus x) = 0$ and there is nothing to prove.)
We claim that $\eta$ is cohomologous to a $1$-form $\omega$ which extends to all of $M$; thus $i^*([\omega]) = [\eta]$, completing the proof.

Suppose that $\omega$ does not exist.
Let $A$ be a ball around $x$ minus $x$, and select an open set $B \subseteq M$ which misses a smaller ball around $x$.
Then there is a strong deformation retract $A \cap B \to S^{n-1}$, where the $S^{n-1}$ is a sphere around $x$.
Consider the Mayer-Vietoris sequence
$$
\cdots \to H^0(A) \oplus H^0(B) \to H^0(A \cap B) \to H^1(M \setminus x) \to H^1(A) \oplus H^1(B) \to H^1(A \cap B) \to \cdots.
$$
Since $A \cap B$ is homotopic to $S^{n-1}$, $H^0(A \cap B) \cong \RR$.
Therefore every smooth function on $A \cap B$ is cohomologous to a constant function $c$, which can be written as the difference of $c \in C^\infty(A)$ and $0 \in C^\infty(B)$.
Therefore the map $H^0(A) \oplus H^0(B) \to H^0(A \cap B)$ is surjective, so the map $H^0(A \cap B) \to H^1(M \setminus x)$ is zero, so $H^1(M \setminus x) \to H^1(A) \oplus H^1(B)$ is injective.
Since $H^1(S^2) = 0$, so is $H^1(A \cap B)$, so $H^1(M \setminus x) \to H^1(A) \oplus H^1(B)$ is surjective and hence an isomorphism.

Since $A$ is a punctured ball, $A$ is homotopy equivalent to $S^{n-1}$, so $H^1(A) = H^1(S^{n-1}) = 0$ since $n \geq 3$.
Therefore $\eta$ is exact on $A$.
In particular, $\eta$ is cohomologous to a $1$-form $\omega$ which is supported in $B$, so $\omega$ extends to a $1$-form on $M$, so $i^*([\omega]) = [\eta]$.
\end{proof}

\begin{lemma}
The claim holds for $p = n - 1$ provided that $M$ is compact and orientable.
\end{lemma}
\begin{proof}
By assumption and the computation of top cohomology, $H^n(M) \cong \RR$ and $H^n(U) = 0$.
The Mayer-Vietoris sequence reads
$$\cdots \to H^{n-1}(M) \to H^{n-1}(U) \oplus H^{n-1}(V) \to H^{n-1}(U \cap V) \to H^n(M) \to H^n(U) \oplus H^n(V) \to \cdots,$$
which simplifies to
$$\cdots \to H^{n-1}(M) \to H^{n-1}(M \setminus x) \to \RR \to \RR \to 0 \to \cdots.$$
Here the map $H^{n-1}(M) \to H^{n-1}(M \setminus x)$ is $i^*$.
The map $\RR \to \RR$ is surjective, so its kernel is $0$, so the map $H^{n-1}(M \setminus x) \to \RR$ is zero.
That implies that $i^*$ is surjective.
\end{proof}

\begin{exer}[17.11]
Let $M, N$ be connected, oriented $n$-manifolds. Let $F: M \to N$ be a proper smooth map. Show that there is a unique $k \in \ZZ$ such that for every $\omega \in \Omega^n_c(N)$,
$$\int_M F^*\omega = k\int_N \omega,$$
and for every regular value $q$ of $F$,
$$k = \sum_{x \in F^{-1}(q)} \sgn dF_x.$$
Considering $F,G$ self-maps of $\CC$ defined by $F(z) = z$ and $G(z) = z^2$, show that degree is not a homotopy invariant.
\end{exer}

Two forms with compact support are cohomologous if they have the same integral, since the top cohomology with compact support of $M, N$ is $1$-dimensional.
Arguing as in the existence of degree on compact manifolds we see that for any $F$ there is a unique
$$k = \int_M F^*\theta,$$
where $\theta$ is any compactly supported top form with $\int_M \theta = 1$, with the desired property.
We show now that $k$ satisfies the regular value characterization, which implies that $k$ is an integer as well.

Let $q$ be a regular value of $F$. Since $F$ is proper, $F^{-1}(q)$ is compact, and since $F^{-1}(q)$ is a $0$-dimensional manifold, it follows that $F^{-1}(q)$ is finite, say
$$F^{-1}(q) = \{x_1, \dots, x_m\}.$$
Arguing as usual we get disjoint pairwise neighborhoods $U_i \ni x_i$ with $q \in F(U_i)$, and then we can choose a compact set $K \subseteq M$ which meets none of the $U_i$, and then $F(K)$ is compact and misses $q$.
Reason as usual, adding $K^c$ to the $U_i$, to get that $F$ is either an orientation-preserving or orientation-reserving covering map of some neighborhood of $q$.
Now the claim follows by the same argument as in the book.

Since $\CC$ is contractible, $F$ and $G$ are homotopic. On the other hand if $\omega = e^{-x^2-y^2}~dx\wedge dy$ is the Gaussian volume form (not compactly supported, but of course we can cut it off if we really wanted to) $\int_\CC F^*\omega = 2\pi$ while $\int_\CC G^*\omega = 4\pi$.

\begin{exer}[17.12]
Let $M,N$ be compact, connected, oriented $n$-manifolds, and $F: M \to N$ smooth.
Show that if there is an $\eta \in \Omega^n(N)$ such that $\int_M F^*\eta \neq 0$ then $F$ is surjective.
Show that the converse is not true.
\end{exer}

The first part is clear since if $F$ is not surjective then $\deg F = 0$.
Indeed, since $M$ is compact, if $F$ is not surjective then there is an open set $U$ missed by $F$, and if $\eta$ is a top form supported in $U$, then $F^*\eta = 0$.

For the converse, let $F: S^1 \to S^1$ be defined by $\theta \mapsto 2\pi\sin \theta$.
Then $f$ is surjective but nullhomotopic.





\end{document}
