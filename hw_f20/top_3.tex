
% --------------------------------------------------------------
% This is all preamble stuff that you don't have to worry about.
% Head down to where it says "Start here"
% --------------------------------------------------------------

\documentclass[10pt]{article}

\usepackage[margin=.7in]{geometry}
\usepackage{amsmath,amsthm,amssymb}
\usepackage{enumitem}
\usepackage{tikz-cd}
\usepackage{mathtools}
\usepackage{amsfonts}
\usepackage{listings}
\usepackage{algorithm2e}
\usepackage{verse,stmaryrd}
\usepackage{fancyvrb}

% Number systems
\newcommand{\NN}{\mathbb{N}}
\newcommand{\ZZ}{\mathbb{Z}}
\newcommand{\QQ}{\mathbb{Q}}
\newcommand{\RR}{\mathbb{R}}
\newcommand{\CC}{\mathbb{C}}
\newcommand{\PP}{\mathbb P}
\newcommand{\FF}{\mathbb F}
\newcommand{\DD}{\mathbb D}
\renewcommand{\epsilon}{\varepsilon}

\newcommand{\Aut}{\operatorname{Aut}}
\newcommand{\cl}{\operatorname{cl}}
\newcommand{\ch}{\operatorname{ch}}
\newcommand{\Con}{\operatorname{Con}}
\newcommand{\coker}{\operatorname{coker}}
\newcommand{\CVect}{\CC\operatorname{-Vect}}
\newcommand{\Cantor}{\mathcal{C}}
\newcommand{\D}{\mathcal{D}}
\newcommand{\card}{\operatorname{card}}
\newcommand{\dbar}{\overline \partial}
\newcommand{\diam}{\operatorname{diam}}
\newcommand{\dom}{\operatorname{dom}}
\newcommand{\End}{\operatorname{End}}
\DeclareMathOperator*{\esssup}{ess\,sup}
\newcommand{\GL}{\operatorname{GL}}
\newcommand{\Hess}{\operatorname{Hess}}
\newcommand{\Hom}{\operatorname{Hom}}
\newcommand{\id}{\operatorname{id}}
\newcommand{\Ind}{\operatorname{Ind}}
\newcommand{\Inn}{\operatorname{Inn}}
\newcommand{\interior}{\operatorname{int}}
\newcommand{\lcm}{\operatorname{lcm}}
\newcommand{\mesh}{\operatorname{mesh}}
\newcommand{\LL}{\mathcal L_0}
\newcommand{\Leb}{\mathcal{L}_{\text{loc}}^2}
\newcommand{\Lip}{\operatorname{Lip}}
\newcommand{\ppGL}{\operatorname{PGL}}
\newcommand{\ppic}{\vspace{35mm}}
\newcommand{\ppset}{\mathcal{P}}
\DeclareMathOperator{\proj}{proj}
\DeclareMathOperator*{\Res}{Res}
\newcommand{\Riem}{\mathcal{R}}
\newcommand{\RVect}{\RR\operatorname{-Vect}}
\newcommand{\Sch}{\mathcal{S}}
\newcommand{\SL}{\operatorname{SL}}
\newcommand{\sgn}{\operatorname{sgn}}
\newcommand{\spn}{\operatorname{span}}
\newcommand{\Spec}{\operatorname{Spec}}
\newcommand{\supp}{\operatorname{supp}}
\newcommand{\TT}{\mathcal T}
\DeclareMathOperator{\tr}{tr}

% Calculus of variations
\DeclareMathOperator{\pp}{\mathbf p}
\DeclareMathOperator{\zz}{\mathbf z}
\DeclareMathOperator{\uu}{\mathbf u}
\DeclareMathOperator{\vv}{\mathbf v}
\DeclareMathOperator{\ww}{\mathbf w}

% Categories
\newcommand{\Ab}{\mathbf{Ab}}
\newcommand{\Cat}{\mathbf{Cat}}
\newcommand{\Group}{\mathbf{Group}}
\newcommand{\Module}{\mathbf{Module}}
\newcommand{\Set}{\mathbf{Set}}
\DeclareMathOperator{\Fun}{Fun}
\DeclareMathOperator{\Iso}{Iso}
\DeclareMathOperator{\Map}{Map}

% Complex analysis
\renewcommand{\Re}{\operatorname{Re}}
\renewcommand{\Im}{\operatorname{Im}}

% Logic
\renewcommand{\iff}{\leftrightarrow}
\newcommand{\Henkin}{\operatorname{Henk}}
\newcommand{\PA}{\mathbf{PA}}
\DeclareMathOperator{\proves}{\vdash}

% Group
\DeclareMathOperator{\Gal}{Gal}
\DeclareMathOperator{\Fix}{Fix}
\DeclareMathOperator{\Out}{Out}

% Other symbols
\newcommand{\heart}{\ensuremath\heartsuit}
\newcommand{\club}{\ensuremath\clubsuit}

\DeclareMathOperator{\atanh}{atanh}

% Theorems
\theoremstyle{definition}
\newtheorem*{falselemma}{Grader's ``Lemma"}
\newtheorem{exer}{Exercise}
\newtheorem{lemma}{Lemma}[exer]
\newtheorem{theorem}[lemma]{Theorem}
\newtheorem{corollary}[lemma]{Corollary}

\def\Xint#1{\mathchoice
{\XXint\displaystyle\textstyle{#1}}%
{\XXint\textstyle\scriptstyle{#1}}%
{\XXint\scriptstyle\scriptscriptstyle{#1}}%
{\XXint\scriptscriptstyle\scriptscriptstyle{#1}}%
\!\int}
\def\XXint#1#2#3{{\setbox0=\hbox{$#1{#2#3}{\int}$ }
\vcenter{\hbox{$#2#3$ }}\kern-.6\wd0}}
\def\ddashint{\Xint=}
\def\dashint{\Xint-}

\usepackage[backend=bibtex,style=alphabetic,maxcitenames=50,maxnames=50]{biblatex}
\renewbibmacro{in:}{}
\DeclareFieldFormat{pages}{#1}

\begin{document}
\noindent
\large\textbf{Algebraic Topology, HW 3} \hfill \textbf{Aidan Backus} \\

% --------------------------------------------------------------
%                         Start here
% --------------------------------------------------------------\

I talked about most of these problems with Megan Chang-Lee, Steven Creech, Matthew Emerson and Nate Gillman.
It goes without saying that I consulted Hatcher, as well as Jill Clayburn, to review some of the later proofs.

\begin{exer}
Show that a finitely generated group has only a finite number of subgroups of a given finite index.
\end{exer}

Let $G$ be a finitely generated group, and choose a presentation
$$1 \to A \to \ZZ^{*n} \to G \to 1$$
for some (possibly not finitely generated) group $A$, and let $m \in \NN$.
If $H$ is a subgroup of $G$ of index $m$, then $H$ lifts to a subgroup of $\ZZ^{*n}$, of index $m$.
So we may assume that $G \cong \ZZ^{*n}$ is free, and in fact that $G = \pi_1((S^1)^{\vee n})$, the fundamental group of the $n$-petaled rose.

Under that assumption, the subgroups of $G$ of index $m$ are in bijection with the $m$-sheeted covering spaces of $(S^1)^{\vee n}$.
Since $(S^1)^{\vee n}$ is a graph, so are its covering spaces $G$.
In fact $G$ is a graph with $m$ vertices, each of degree $2n$; thus by the handshaking lemma $G$ has $n$ edges.
But there are only finitely many such graphs.
One can introduce a covering space structure on such a graph by choosing a vertex and then choosing an orientation for each edge; there are only finitely many choices here, so there are only finitely many $m$-sheeted covering spaces of $(S^1)^{\vee n}$.

\begin{exer}
Find all abelian groups $A$ making the sequence
$$0 \to \ZZ/p \to A \to \ZZ/q \to 0$$
exact.
\end{exer}

In order that the above sequence be exact, we must have
$$\frac{\ZZ}{q} \cong \frac{A}{\ZZ/p}.$$
Thus $|A| = pq$.
If $p \neq q$, then by a Sylow-type theorem, there is an embedding $\ZZ/q \to A$, and so the short exact sequence splits.
That implies that $A \cong \ZZ/p \oplus \ZZ/q$.
Otherwise, $p = q$. If $A \cong \ZZ/(p^2)$ then we're done.

Otherwise, $|A| = p^2$ and $A$ is not isomorphic to $\ZZ/p^2$.
So all elements have order at most $p$, and since $p$ is prime that means that all elements have order $p$ except the identity.
So $A$ is a $\ZZ/p$-vector space of dimension $2$, thus $A \cong (\ZZ/p)^2$.

\begin{exer}
Let $V$ be a graded vector space whose homogeneous parts are finite-dimensional and almost always zero. Define the Euler characteristic
$$\chi(V) = \sum_{n=-\infty}^\infty (-1)^n \dim V_n.$$
Suppose that $V', V''$ are two more such graded vector spaces with a long exact sequence
\begin{equation}
\label{LES}
\cdots \to V''_{n+1} \to V_n' \to V_n \to V_n'' \to V'_{n-1} \to \cdots.
\end{equation}
Show that $\chi(V) = \chi(V') + \chi(V'')$.

Show that if $V$ is a chain complex then $\chi(V) = \chi(H(V))$.
\end{exer}

The first claim follows from a stronger statement that we will now prove.
We will say that a graded sequence has compact support if all but finitely many entries are $0$.
\begin{lemma}
Let $(V, d)$ be a long exact sequence with compact support. Then $\chi(V) = 0$.
\end{lemma}
\begin{proof}
Let the support of $V$ be indexed
$$V_n \to V_{n-1} \to \cdots \to V_2 \to V_1.$$
One has
$$V_i \cong \ker d_i \oplus d_i(V_i)$$
by the rank-nullity theorem, and by exactness, $d_i(V_i) = \ker d_{i+1}$. Therefore
$$\chi(V) = \sum_{i=1}^n (-1)^n\dim \ker d_i + (-1)^n\dim \ker d_{i+1}$$
which telescopes since $n$ is finite.
\end{proof}
Now we prove the first claim. By assumption, the long exact sequence $(W, d)$ defined by (\ref{LES}) has compact support.
Therefore $\chi(W) = 0$.
Moreover,
$$\chi(V) - \chi(V') - \chi(V'') = \sum_{n=-\infty}^\infty (-1)^n(\dim V_n - \dim V'_n - \dim V''_n)$$
and setting $W_{3n} = V'_n$, $W_{3n+1} = V_n$, $W_{3n+2} = V''_n$, we see that
$$0 = \chi(W) = \sum_{m=-\infty}^\infty (-1)^m \dim W_m = \sum_{n=-\infty}^\infty (-1)^{3n}(\dim V'_n + \dim V''_n) + (-1)^{3n+1} \dim V_n $$
which is what we wanted, since $3$ is odd and so $3n$ has the same parity as $n$.

For the second claim, we have
\begin{align*}
\chi(H(V)) &= \sum_n (-1)^n \dim H_n = \sum_n (-1)^n \dim \frac{\ker d_n}{d_{n+1}(V_{n+1})}\\
&= \sum_n (-1)^n (\dim \ker d_n - \dim d_{n+1}(V_{n+1}) = \sum_n (-1)^n (\dim \ker d_n + \dim d_n(V_n))
\end{align*}
after reiindexing using compact support.
One has
$$\chi(V) = \sum_n (-1)^n (\ker d_n + d_n(V_n))$$
by the rank-nullity theorem.

\begin{exer}
Consider a diagram of abelian groups
$$\begin{tikzcd}
G \arrow[r,"a"] \arrow[d,"g"] & H \arrow[r,"b"] \arrow[d,"h"] & I \arrow[r,"c"] \arrow[d,"i"] & J \arrow[r,"d"] \arrow[d,"j"] & K \arrow[d,"k"]\\
G' \arrow[r,"a'"] & H' \arrow[r,"b'"] & I' \arrow[r,"c'"] & J' \arrow[r,"d'"] & K'
\end{tikzcd}$$
in which the rows are exact. Show that if $h,j$ are surjective and $k$ is injective then $i$ is surjective.
Show that if $h,j$ are injective and $g$ is surjective, then $i$ is injective.
\end{exer}

We prove the first claim first.
Let $y' \in I'$. Then $c'(y') \in J'$ and since $j$ is surjective there is a $y \in J$ such that $j(y) = c'(y')$.
Since
$$0 = d'c'(y') = d'j(y) kd(y)$$
and $k$ is injective, $d(y) = 0$.
Since the top row is exact this implies that there is a $x \in I$ with $y = c(x)$.
Now
$$c'(i(x) - y') = c'i(x) - c'(y') = jc(x) - c'(y) = j(y) - c'(y) = 0.$$
Therefore since the bottom row is exact, there is a $x' \in H'$ with $b'(x') = i(x) - y'$.
Since $h$ is surjective, there is a $z$ in $H$ with $h(z) = x'$, thus $b'h(z) = i(x) - y'$.
But then
$$i(b(z) + x) = b'h(z) + i(x) = y'$$
so $i$ is surjective.

For the second claim, note that what we really proven in the first claim is that in an abelian category, if $h,j$ are epic and $k$ is monic, then $i$ is epic.
Flipping all the arrows and noting that we have done nothing but dualize all the commutative diagrams involved, one sees that if $h,j$ are monic and $g$ is epic, then $i$ is monic.
This is what we wanted to show.

Considering how many classes have had me prove these homological facts I should probably just save this document somewhere.
But every time I take a class that requires me to prove these facts, I think it will be the last such class and then delete the document.
I will not make the same mistake this time.

\begin{exer}
Consider a diagram of abelian groups
$$\begin{tikzcd}
0 \arrow[r] & A \arrow[r] \arrow[d,"\alpha"] & B \arrow[r] \arrow[d,"\beta"] & C \arrow[r] \arrow[d,"\gamma"] & 0\\
0 \arrow[r] & A' \arrow[r] & B' \arrow[r] & C' \arrow[r] & 0
\end{tikzcd}$$
in which the rows are exact. Show that there is an exact sequence
$$0 \to \ker \alpha \to \ker \beta \to \ker \gamma \to \coker \alpha \to \coker \beta \to \coker \gamma \to 0.$$

Let
$$0 \to C' \to C \to C'' \to 0$$
be a short exact sequence of chain complexes. Show that there is a long exact sequence
$$\cdots \to H_{n+1}(C'') \to H_n(C') \to H_n(C) \to H_n(C'') \to H_{n-1}(C') \to \cdots$$
in homology.
\end{exer}

\begin{theorem}
The snake lemma is true.
\end{theorem}
\begin{proof}
We extend our diagram to
$$\begin{tikzcd}
& 0 \arrow[d] & 0 \arrow[d] & 0 \arrow[d] \\
& \ker \alpha \arrow[d] & \ker \beta \arrow[d] & \ker \gamma \arrow[d] \\
0 \arrow[r] & A \arrow[r,"f"] \arrow[d,"\alpha"] & B \arrow[r,"g"] \arrow[d,"\beta"] & C \arrow[r] \arrow[d,"\gamma"] & 0\\
0 \arrow[r] & A' \arrow[r,"f'"] \arrow[d] & B' \arrow[r, "g'"] \arrow[d] & C' \arrow[r] \arrow[d] & 0\\
 & \coker \alpha \arrow[d] & \coker \beta \arrow[d] & \coker \gamma \arrow[d]\\
 & 0 & 0 & 0
\end{tikzcd}$$
in which the columns are exact and rows three and four are exact.
If $a \in \ker \alpha$ then $\beta f(a) = f' \alpha(a) = 0$ so $f$ restricts to a map $\ker \alpha \to \ker \beta$. Similarly we can restrict $g$; and then by a duality argument, $f'$ and $g'$ drop to maps in row five, thus we add arrows
$$\begin{tikzcd}
& 0 \arrow[d] & 0 \arrow[d] & 0 \arrow[d] \\
0 \arrow[color=red,r] & \ker \alpha \arrow[r,"f",color=red]\arrow[d] & \ker \beta \arrow[r,"g",color=red]\arrow[d] & \ker \gamma \arrow[d] \\
0 \arrow[r] & A \arrow[r,"f"] \arrow[d,"\alpha"] & B \arrow[r,"g"] \arrow[d,"\beta"] & C \arrow[r] \arrow[d,"\gamma"] & 0\\
0 \arrow[r] & A' \arrow[r,"f'"] \arrow[d] & B' \arrow[r, "g'"] \arrow[d] & C' \arrow[r] \arrow[d] & 0\\
 & \coker \alpha \arrow[d] \arrow[r,"f'", color=red] & \coker \beta \arrow[r,"f'", color=red]\arrow[d] & \coker \gamma \arrow[d] \arrow[r,color=red] & 0\\
 & 0 & 0 & 0
\end{tikzcd}$$
to the diagram.

We now introduce a snake arrow
$$d: \ker \gamma \to \coker \alpha.$$
Let $c \in \ker \gamma$. By exactness of row three, $g$ is surjective, so there is a $b \in B$ such that $c = g(b)$.
Then
$$g'\beta(b) = \gamma(c) = 0$$
so $\beta(b) \in \ker g' = f'(A')$ and thus there is a unique $a' \in A'$ with $f'(a') = \beta(b)$, where we have repeatedly used exactness of row four.

We claim that $a'$ is uniquely defined modulo $\alpha(A)$.
The only ambiguity was in our choice of $b$; so let $b_1, b_2 \in B$ satisfy $g(b_1) = g(b_2) = c$.
Then $b_1 - b_2 \in \ker g = f(A)$.
Let $a'_1, a_2'$ be the corresponding elements of $A'$ to the choices of $b_1, b_2$.
Then, by exactness of row four, $a'_1 - a'_2 = \alpha(f^{-1}(b_1 - b_2)) \in \alpha(A)$, where $f^{-1}(b_1 - b_2)$ exists (and may not be unique) by exactness of row three.
Therefore we are entitled to let $d(c)$ be the unique element of $\coker \alpha$ which lifts to an $a' \in A'$ with $f'(a') = \beta(g^{-1}(c))$.
This guarantees that all the maps in the long exact sequence exist.

Now we check that the long exact sequence is in fact exact.
Since $f$ restricts to an injective arrow, one has exactness at $\ker \alpha$.
Similarly $g'$ drops to a surjective arrow, so one has exactness at $\coker \gamma$.

To see exactness at $\ker \beta$, note that $gf = 0$.
If $b \in \ker \beta \cap \ker g$, then since row three is exact, there is an $a \in A$ with $f(a) = b$.
So it suffices to show that $a \in \ker \alpha$.
In fact
$$f'(\alpha(a)) = \beta(f(m)) = \beta(b) = 0$$
so by exactness of row four, $\alpha(a) = 0$.

To see exactness at $\ker \gamma$, first let $c \in \ker \beta$.
Then $\beta(c) = 0 = f'(0)$ so $dg(c) = 0$. Therefore $dg = 0$.
Conversely suppose that $c \in \ker \gamma \cap \ker d$, thus there is a $b \in \ker \beta$ with $c = g(b)$.
Then there is an $a \in A$ with $f'\alpha(a) = \beta(b)$. So $g(b-f(a)) = c$.

To see exactness at $\coker \alpha$, let $c \in \ker \gamma$, say $c = g(b)$ and $\beta(b) = f'(a)$, thus $d(c) = a$.
But $f'(a) = \beta(b)$ so $f'(a) \in \beta(B)$, so $f' d(c) = 0$.
Conversely suppose that $a' \in \ker f'$, thus there is a $b \in B$ such that $f'(a') = \beta(b)$.
Let $c = g(b)$; then $\gamma g(b) = 0$ implies $c \in \ker \gamma$, so by definition of $d$, $d(c) = a$.

To see exactness at $\coker \beta$, we are given $g'f' = 0$.
Let $b' \in \ker g'$, thus there is a $c \in C$ with $g'(b) = \gamma(c)$, and an $b \in B$ with $g(b) = c$.
Also $b' - \beta(b) = b'$ in $\coker \beta$.
But $g'(b' - \beta(b)) = 0$, so $g'(b') = 0$ in $\coker \beta$.
Then $b' = f'(a')$ for some $a' \in A$, which was desired.
\end{proof}

\begin{theorem}
A short exact sequence of chain complexes induces a long exact sequence in homology.
\end{theorem}
\begin{proof}
We are given a diagram
$$\begin{tikzcd}
& 0 \arrow[d] & 0 \arrow[d] & 0 \arrow[d] \\
\cdots \arrow[r] & A_{n+1} \arrow[d,"f"] \arrow[r] & A_n \arrow[d,"f"] \arrow[r] & A_{n-1} \arrow[d,"f"] \arrow[r]&\cdots \\
\cdots \arrow[r] & B_{n+1} \arrow[d,"g"] \arrow[r] & B_n \arrow[d,"g"] \arrow[r] & B_{n-1} \arrow[d,"g"] \arrow[r]&\cdots \\
\cdots \arrow[r] & C_{n+1} \arrow[d] \arrow[r] & C_n \arrow[d] \arrow[r] & C_{n-1} \arrow[r] \arrow[d]&\cdots \\
& 0 & 0 & 0
\end{tikzcd}$$
with exact columns. In particular, $fd = df$ and $dg = gd$, so $f,g$ drop to maps $f_*,g_*$ in homology.
We now define
$$\partial: H_n(C) \to H_{n-1}(A)$$
by setting $\partial = f^{-1}dg^{-1}$.
We must show that there is a unique $a$ with $a = \partial(c)$.
Let $c \in C_n$.
Since columns are exact, there is a $b \in B_n$ with $g(b) = c$; then $b$ drops to an element $db$ of $B_{n-1}$ which then lifts to an element $a$ of $A_{n-1}$.
So $a$ exists. The only ambiguities in the construction of $a$ are the choices of $b,c$.
If we instead chose, say, $b'$, then $b - b' \in f(A_n)$, which lifts to $a' \in A_n$; this replaces $a$ with $a - da'$, which has no effect on homology.
If instead of $c$ we chose $c + dc'$ then there is a $b'$ with $g(b') = c'$. Thus $c + dc'$ lifts to $b + db'$, which then drops to $db$ in $B_{n-1}$ and has no effect on our choice of $a$.
Thus we obtain a chain complex
$$\begin{tikzcd}
\cdots \arrow[r,"\partial"] & H_n(A) \arrow[r,"f_*"] & H_n(B) \arrow[r,"g_*"] & H_n(C) \arrow[r,"\partial"] & H_{n-1}(A) \arrow[r,"f_*"] & \cdots.
\end{tikzcd}$$
We claim that it is a long exact sequence.

First we check exactness at $H_n(A)$. Clearly $g_*f_* = 0$.
If $[b] \in \ker g_*$ then $g(b) = dc$ for some $c \in C_{n+1}$.
Thus there is a $b' \in B_{n+1}$ with $dg(b') = g(b)$.
So $g(db' - b) = 0$ since $g$ is a chain map, thus there is an $a \in A_n$ with $f(a) = b - db'$.
But then $fd(a) = db = 0$, but $f$ is injective so $da = 0$, thus $f_*[a] = [b]$.

Now we check exactness at $H_n(B)$. By construction $\partial$ kills $B$, so $\partial g_* = 0$.
If $[c] \in \ker \partial$ then the $a \in A$ obtained in the construction of $\partial$ satisfies $da' = a$ for some $a' \in A_n$.
Thus $d(b - f(a')) = db - fd(a') = 0$ and $g(b - f(a')) = g(b) = c$ so $g_*[b - f(a')] = [c]$.

Finally we check exactness at $H_n(C)$. One has $f_*\partial ([c]) = [\partial b] = 0$.
Now if $a \in A_{n-1}$ and there is a $b \in B_n$ satisfying $f(a) = db$, then $dg(b) = g(db) = 0$, and $\partial[g(b)] = a$.
\end{proof}



\end{document}
