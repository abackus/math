
% --------------------------------------------------------------
% This is all preamble stuff that you don't have to worry about.
% Head down to where it says "Start here"
% --------------------------------------------------------------

\documentclass[10pt]{article}

\usepackage[margin=.7in]{geometry}
\usepackage{amsmath,amsthm,amssymb}
\usepackage{enumitem}
\usepackage{tikz-cd}
\usepackage{mathtools}
\usepackage{amsfonts}
\usepackage{listings}
\usepackage{algorithm2e}
\usepackage{verse,stmaryrd}
\usepackage{fancyvrb}

% Number systems
\newcommand{\NN}{\mathbb{N}}
\newcommand{\ZZ}{\mathbb{Z}}
\newcommand{\QQ}{\mathbb{Q}}
\newcommand{\RR}{\mathbb{R}}
\newcommand{\CC}{\mathbb{C}}
\newcommand{\PP}{\mathbb P}
\newcommand{\FF}{\mathbb F}
\newcommand{\DD}{\mathbb D}
\renewcommand{\epsilon}{\varepsilon}

\newcommand{\Aut}{\operatorname{Aut}}
\newcommand{\cl}{\operatorname{cl}}
\newcommand{\ch}{\operatorname{ch}}
\newcommand{\Con}{\operatorname{Con}}
\newcommand{\coker}{\operatorname{coker}}
\newcommand{\CVect}{\CC\operatorname{-Vect}}
\newcommand{\Cantor}{\mathcal{C}}
\newcommand{\D}{\mathcal{D}}
\newcommand{\card}{\operatorname{card}}
\newcommand{\dbar}{\overline \partial}
\newcommand{\diam}{\operatorname{diam}}
\newcommand{\dom}{\operatorname{dom}}
\newcommand{\End}{\operatorname{End}}
\DeclareMathOperator*{\esssup}{ess\,sup}
\newcommand{\Hess}{\operatorname{Hess}}
\newcommand{\Hom}{\operatorname{Hom}}
\newcommand{\id}{\operatorname{id}}
\newcommand{\Ind}{\operatorname{Ind}}
\newcommand{\Inn}{\operatorname{Inn}}
\newcommand{\interior}{\operatorname{int}}
\newcommand{\lcm}{\operatorname{lcm}}
\newcommand{\mesh}{\operatorname{mesh}}
\newcommand{\LL}{\mathcal L_0}
\newcommand{\Leb}{\mathcal{L}_{\text{loc}}^2}
\newcommand{\Lip}{\operatorname{Lip}}
\newcommand{\ppic}{\vspace{35mm}}
\newcommand{\ppset}{\mathcal{P}}
\DeclareMathOperator{\proj}{proj}
\DeclareMathOperator*{\Res}{Res}
\newcommand{\Riem}{\mathcal{R}}
\newcommand{\RVect}{\RR\operatorname{-Vect}}
\newcommand{\Sch}{\mathcal{S}}
\newcommand{\sgn}{\operatorname{sgn}}
\newcommand{\spn}{\operatorname{span}}
\newcommand{\Spec}{\operatorname{Spec}}
\newcommand{\supp}{\operatorname{supp}}
\newcommand{\TT}{\mathcal T}
\DeclareMathOperator{\tr}{tr}

% Calculus of variations
\DeclareMathOperator{\pp}{\mathbf p}
\DeclareMathOperator{\zz}{\mathbf z}
\DeclareMathOperator{\uu}{\mathbf u}
\DeclareMathOperator{\vv}{\mathbf v}
\DeclareMathOperator{\ww}{\mathbf w}

% Categories
\newcommand{\Ab}{\mathbf{Ab}}
\newcommand{\Cat}{\mathbf{Cat}}
\newcommand{\Group}{\mathbf{Group}}
\newcommand{\Module}{\mathbf{Module}}
\newcommand{\Set}{\mathbf{Set}}
\DeclareMathOperator{\Fun}{Fun}
\DeclareMathOperator{\Iso}{Iso}

% Complex analysis
\renewcommand{\Re}{\operatorname{Re}}
\renewcommand{\Im}{\operatorname{Im}}

% Logic
\renewcommand{\iff}{\leftrightarrow}
\newcommand{\Henkin}{\operatorname{Henk}}
\newcommand{\PA}{\mathbf{PA}}
\DeclareMathOperator{\proves}{\vdash}

% Group
\DeclareMathOperator{\Gal}{Gal}
\DeclareMathOperator{\Fix}{Fix}
\DeclareMathOperator{\Lie}{Lie}
\DeclareMathOperator{\Out}{Out}

\DeclareMathOperator{\Diffeo}{Diffeo}

\newcommand{\GL}{\operatorname{GL}}
\newcommand{\ppGL}{\operatorname{PGL}}
\newcommand{\SL}{\operatorname{SL}}
\newcommand{\SO}{\operatorname{SO}}

% Other symbols
\newcommand{\heart}{\ensuremath\heartsuit}
\newcommand{\club}{\ensuremath\clubsuit}

\DeclareMathOperator{\atanh}{atanh}

% Theorems
\theoremstyle{definition}
\newtheorem*{corollary}{Corollary}
\newtheorem*{falselemma}{Grader's ``Lemma"}
\newtheorem{exer}{Exercise}
\newtheorem{lemma}{Lemma}[exer]
\newtheorem{theorem}[lemma]{Theorem}

\def\Xint#1{\mathchoice
{\XXint\displaystyle\textstyle{#1}}%
{\XXint\textstyle\scriptstyle{#1}}%
{\XXint\scriptstyle\scriptscriptstyle{#1}}%
{\XXint\scriptscriptstyle\scriptscriptstyle{#1}}%
\!\int}
\def\XXint#1#2#3{{\setbox0=\hbox{$#1{#2#3}{\int}$ }
\vcenter{\hbox{$#2#3$ }}\kern-.6\wd0}}
\def\ddashint{\Xint=}
\def\dashint{\Xint-}

\usepackage[backend=bibtex,style=alphabetic,maxcitenames=50,maxnames=50]{biblatex}
\renewbibmacro{in:}{}
\DeclareFieldFormat{pages}{#1}

\begin{document}
\noindent
\large\textbf{Manifolds, HW 7} \hfill \textbf{Aidan Backus} \\

% --------------------------------------------------------------
%                         Start here
% --------------------------------------------------------------\

\begin{exer}[10.6]
Let $(U_\alpha)_\alpha$ be an open cover of $M$.
Suppose that for every $\alpha, \beta$ there is a smooth map $\tau_{\alpha\beta}: U_\alpha \cap U_\beta \to \GL(k)$ such that
$$\tau_{\alpha\beta}\tau_{\beta\gamma} = \tau_{\alpha\gamma}$$
on $U_\alpha \cap U_\beta \cap U_\gamma$.
Show that there is a rank-$k$ vector bundle $E \to M$ whose transition functions are the given maps.
\end{exer}

Let $E'$ be the disjoint union of all spaces of the form $U_\alpha \cap \RR^k$.
Let $(p, v, \alpha)$ denote $(p, v) \in U_\alpha$.
One has a binary relation $R$ on $E'$, defined by setting $(p, v, \alpha)R(p, \tau_{\alpha\beta}v, \beta)$.
\begin{lemma}
$R$ is an equivalence relation.
\end{lemma}
\begin{proof}
Using the triple intersection property of $\tau$:
\begin{enumerate}
\item If $\alpha = \beta$ then $\tau_{\alpha\alpha}(p)^2 = \tau_{\alpha\alpha}(p)$ which implies that $\tau_{\alpha\alpha}(p) = 1$ so $(p, v, \alpha)R(p, v, \alpha)$.
\item Since $\tau_{\alpha\beta}(p) \tau_{\beta\alpha}(p) = \tau_{\alpha\alpha}(p) = 1$, $(p, v, \alpha)R(p, \tau_{\alpha\beta}v, \beta)$ implies $(p, \tau_{\alpha\beta}v, \beta)R(p, v, \alpha)R$.
\item It follows immediately from the relation $\tau_{\alpha\beta}(p)\tau_{\beta\gamma}(p) = \tau_{\alpha\gamma}(p)$ that
if $(p, v, \alpha)R(p, \tau_{\alpha\beta}v, \beta)$ and $(p, \tau_{\alpha\beta}v, \beta)R(p, \tau_{\beta\gamma}\tau_{\alpha\beta}v, \gamma)$ then $(p, v, \alpha)R(p, \tau_{\beta\gamma}\tau_{\alpha\beta}v, \gamma)$.
\end{enumerate}
This was desired.
\end{proof}

Let $[\cdot]$ be the map that sends a point of $E'$ to its equivalence class.
Since $R$ does not identify basepoints, for $[(p, v, \alpha)]$ we may instead write $(p, [v, \alpha])$.
Thus $E'/R$ is a disjoint union of vector spaces
$$\frac{E'}{R} = \coprod_{p \in M} \RR^k.$$
Let $\pi: E'/R \to M$ be the obvious projection.
Define $\Phi_\alpha: \pi^{-1}(U_\alpha) \to U_\alpha \times \RR^k$ to be the map which maps $(p, [v, \alpha])$ to $(p, v)$.
Then $\Phi_\alpha|\pi^{-1}(\{p\})$ is certainly a linear isomorphism and
$$\Phi_\alpha(\Phi_\beta^{-1}(p, v)) = \Phi_\alpha(p, [v, \beta]) = \Phi_\alpha(p, [\tau_{\beta\alpha}v, \alpha]) = (p, \tau_{\beta\alpha}v)$$
so the $\Phi_\alpha$ are local trivializations of a vector bundle $E \to M$ with underlying space $E'/R$.

\begin{exer}[11.5]
$T^*M$ is trivial iff $TM$ is trivial.
\end{exer}

Both directions follow immediately from the below lemma.
\begin{lemma}
Let $E$ be a vector bundle over $M$. If $E$ is trivial then the dual bundle $E^*$ is also trivial.
\end{lemma}
\begin{proof}
Let $k$ be the rank of $E$.
Since $E$ is trivial, there is a bundle isomorphism $E \to (M \times \RR^k)$.
In particular, the only transition function is the identity, whose inverted transpose is also the identity.
So $(M \times \RR^k)^*$ is bundle isomorphic to $M \times \RR^k$.
Therefore $E^*$ is bundle isomorphic to $M \times \RR^k$ and so is trivial.
\end{proof}

\begin{exer}[11.16]
Show that if $M$ is compact of positive dimension, then every exact covector field on $M$ vanishes on at least two points of every component of $M$.
\end{exer}

Let $\omega = df$ be an exact $1$-form.
By passing to a component, assume that $M$ is connected; since $M$ has positive dimension it has at least two points.
If $f$ is constant then $\omega = 0$ and we're done, so assume that $f$ is not; then $f$ attains a maximum, say at $p$, and a minimum, say at $q$, and $p \neq q$.
So $p,q$ are distinct critical points of $f$, and thus $\omega = 0$ at $p,q$.

\begin{exer}[12.10]
Show that if $k, \ell \geq 0$ and $F: M \to N$ is a diffeomorphism, one has isomorphisms $F_*: \Gamma(T^{k,\ell}TM) \to \Gamma(T^{k,\ell}TN)$ and $F^* \Gamma(T^{k,\ell}TN) \to \Gamma(T^{k,\ell}TM)$ such that $F_*$ and $F^*$ extend the usual pushforward and pullback morphisms,
$F_* = (F^*)^{-1}$, $F^*(A \otimes B) = F^*A \otimes F^*B$, $(F \circ G)_* = F_* \circ G_*$, $(F \circ G)^* = G^* \circ F^*$, $1_M^* = (1_M)_*$ is the identity, and
$$F^*(A(X_1, \dots, X_k)) = F^*A(F^{-1}_*(X_1), \dots, F^{-1}_*(X_k)).$$
\end{exer}

The criterion $F^*(A \otimes B) = F^*A \otimes F^*B$ defines $F^*$ on covariant tensor fields, hence also $F_*$ on contravariant tensor fields using the criterion $F_* = (F^*)^{-1}$.
Obviously this definition of $F^*$ agrees with the definition of $F^*$ on differential $1$-forms.
We have a definition of $F_*$ for vector fields, thus also $F^*$. The criterion $F^*(A \otimes B) = F^*A \otimes F^*B$ then extends $F^*$ to contravariant tensor fields; so $F_*$ also extends to contravariant tensor fields, in a way which agrees with the definition on vector fields.
Appealing to the first two rules again, we define $F^*$ on tensor fields, and then $F_*$ on tensor fields.

Since $F_*$ and $F^*$ were isomorphisms on their original domains of definition, and what we have obtained is merely a tensor product of that, $F_*$ and $F^*$ remain isomorphisms after extension.
Since the identity induced identity pushforwards and pullbacks, the same remains true after extension.

Since $(F \circ G)^* = G^* \circ F^*$ on differential $1$-forms, this remains true for covariant tensor fields.
Inserting inverses as appropriate, we also see $(F \circ G)_* = F_* \circ G_*$ on covariant tensor fields.
A similar argument works for contravariant tensor fields, thus for tensor fields.

Now let $A$ be a covariant $k$-tensor field on $N$ and $X_1, \dots, X_k$ be vector fields on $N$.
By linearity, we may assume that $A = \bigotimes_{i=1}^k \omega_i$, where $\omega_i$ are differential $1$-forms.
Then $F^*A = \bigotimes_{i=1}^k F^*\omega_i$, and $F^*\omega_i(Y) = \omega_i(F_*Y)$.
In particular,
$$F^*\omega_i(F_*^{-1} X_i) = F^*(\omega_i(X_i)),$$
so
\begin{align*}F^*A(F_*^{-1}X_1, \dots, F_*^{-1}X_k) &= \bigotimes_{i=1}^k F^*\omega_i(F_*^{-1}X_1, \dots, F_*^{-1}X_k) \\
&= \prod_{i=1}^k F^*\omega_i(F_*^{-1}X_i) \\
&= F^*\prod_{i=1}^k \omega_i(X_i)\\
&= F^*\bigotimes_{i=1}^k \omega_i(X_1, \dots, X_k)\\
&= F^*(A(X_1, \dots, X_k))\end{align*}
as desired.


\begin{exer}[12.11]
Show that $[\mathcal L_V, \mathcal L_W] A = \mathcal L_{[V, W]} A$.
\end{exer}

Let $A$ be a covariant $k$-tensor field.
If $k = 1$ then
\begin{align*}\mathcal L_V \mathcal L_W AX &= V \mathcal L_W AX - \mathcal L_W A[V, X]\\
&= VWAX - VA[W, X] - WA[V, X] + A[W, [V, X]]\end{align*}
and so standard properties of Lie algebras give
$$[\mathcal L_V, \mathcal L_W] A = [V, W] A - A[[V, W], \cdot] = \mathcal L_{[V, W]} A.$$
Otherwise, by linearity, we can assume that $A$ is a pure tensor, $A = A_1 \otimes A_2$, where $A_1,A_2$ are covariant $k_1, k_2$-tensor fields respectively, $k_i < k$.
So by induction and the Leibniz formula, the claim holds.







\end{document}
