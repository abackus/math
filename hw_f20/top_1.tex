
% --------------------------------------------------------------
% This is all preamble stuff that you don't have to worry about.
% Head down to where it says "Start here"
% --------------------------------------------------------------

\documentclass[10pt]{article}

\usepackage[margin=.7in]{geometry}
\usepackage{amsmath,amsthm,amssymb}
\usepackage{enumitem}
\usepackage{tikz-cd}
\usepackage{mathtools}
\usepackage{amsfonts}
\usepackage{listings}
\usepackage{algorithm2e}
\usepackage{verse,stmaryrd}
\usepackage{fancyvrb}

% Number systems
\newcommand{\NN}{\mathbb{N}}
\newcommand{\ZZ}{\mathbb{Z}}
\newcommand{\QQ}{\mathbb{Q}}
\newcommand{\RR}{\mathbb{R}}
\newcommand{\CC}{\mathbb{C}}
\newcommand{\PP}{\mathbb P}
\newcommand{\FF}{\mathbb F}
\newcommand{\DD}{\mathbb D}
\renewcommand{\epsilon}{\varepsilon}

\newcommand{\Aut}{\operatorname{Aut}}
\newcommand{\cl}{\operatorname{cl}}
\newcommand{\ch}{\operatorname{ch}}
\newcommand{\Con}{\operatorname{Con}}
\newcommand{\coker}{\operatorname{coker}}
\newcommand{\CVect}{\CC\operatorname{-Vect}}
\newcommand{\Cantor}{\mathcal{C}}
\newcommand{\D}{\mathcal{D}}
\newcommand{\card}{\operatorname{card}}
\newcommand{\dbar}{\overline \partial}
\newcommand{\diam}{\operatorname{diam}}
\newcommand{\dom}{\operatorname{dom}}
\newcommand{\End}{\operatorname{End}}
\DeclareMathOperator*{\esssup}{ess\,sup}
\newcommand{\GL}{\operatorname{GL}}
\newcommand{\Hess}{\operatorname{Hess}}
\newcommand{\Hom}{\operatorname{Hom}}
\newcommand{\id}{\operatorname{id}}
\newcommand{\Ind}{\operatorname{Ind}}
\newcommand{\Inn}{\operatorname{Inn}}
\newcommand{\interior}{\operatorname{int}}
\newcommand{\lcm}{\operatorname{lcm}}
\newcommand{\mesh}{\operatorname{mesh}}
\newcommand{\LL}{\mathcal L_0}
\newcommand{\Leb}{\mathcal{L}_{\text{loc}}^2}
\newcommand{\Lip}{\operatorname{Lip}}
\newcommand{\ppGL}{\operatorname{PGL}}
\newcommand{\ppic}{\vspace{35mm}}
\newcommand{\ppset}{\mathcal{P}}
\DeclareMathOperator{\proj}{proj}
\DeclareMathOperator*{\Res}{Res}
\newcommand{\Riem}{\mathcal{R}}
\newcommand{\RVect}{\RR\operatorname{-Vect}}
\newcommand{\Sch}{\mathcal{S}}
\newcommand{\SL}{\operatorname{SL}}
\newcommand{\sgn}{\operatorname{sgn}}
\newcommand{\spn}{\operatorname{span}}
\newcommand{\Spec}{\operatorname{Spec}}
\newcommand{\supp}{\operatorname{supp}}
\newcommand{\TT}{\mathcal T}
\DeclareMathOperator{\tr}{tr}

% Calculus of variations
\DeclareMathOperator{\pp}{\mathbf p}
\DeclareMathOperator{\zz}{\mathbf z}
\DeclareMathOperator{\uu}{\mathbf u}
\DeclareMathOperator{\vv}{\mathbf v}
\DeclareMathOperator{\ww}{\mathbf w}

% Categories
\newcommand{\Ab}{\mathbf{Ab}}
\newcommand{\Cat}{\mathbf{Cat}}
\newcommand{\Group}{\mathbf{Group}}
\newcommand{\Module}{\mathbf{Module}}
\newcommand{\Set}{\mathbf{Set}}
\DeclareMathOperator{\Fun}{Fun}
\DeclareMathOperator{\Iso}{Iso}

% Complex analysis
\renewcommand{\Re}{\operatorname{Re}}
\renewcommand{\Im}{\operatorname{Im}}

% Logic
\renewcommand{\iff}{\leftrightarrow}
\newcommand{\Henkin}{\operatorname{Henk}}
\newcommand{\PA}{\mathbf{PA}}
\DeclareMathOperator{\proves}{\vdash}

% Group
\DeclareMathOperator{\Gal}{Gal}
\DeclareMathOperator{\Fix}{Fix}
\DeclareMathOperator{\Out}{Out}

% Other symbols
\newcommand{\heart}{\ensuremath\heartsuit}
\newcommand{\club}{\ensuremath\clubsuit}

\DeclareMathOperator{\atanh}{atanh}

% Theorems
\theoremstyle{definition}
\newtheorem*{corollary}{Corollary}
\newtheorem*{falselemma}{Grader's ``Lemma"}
\newtheorem{exer}{Exercise}
\newtheorem{lemma}{Lemma}[exer]
\newtheorem{theorem}[lemma]{Theorem}

\def\Xint#1{\mathchoice
{\XXint\displaystyle\textstyle{#1}}%
{\XXint\textstyle\scriptstyle{#1}}%
{\XXint\scriptstyle\scriptscriptstyle{#1}}%
{\XXint\scriptscriptstyle\scriptscriptstyle{#1}}%
\!\int}
\def\XXint#1#2#3{{\setbox0=\hbox{$#1{#2#3}{\int}$ }
\vcenter{\hbox{$#2#3$ }}\kern-.6\wd0}}
\def\ddashint{\Xint=}
\def\dashint{\Xint-}

\usepackage[backend=bibtex,style=alphabetic,maxcitenames=50,maxnames=50]{biblatex}
\renewbibmacro{in:}{}
\DeclareFieldFormat{pages}{#1}

\begin{document}
\noindent
\large\textbf{Algebraic Topology, HW 1} \hfill \textbf{Aidan Backus} \\

% --------------------------------------------------------------
%                         Start here
% --------------------------------------------------------------\

I talked about most of these problems with Megan Chang-Lee, and the last few with Steven Creech and Nate Gillman as well.

\begin{exer}
Show that TFAE:
\begin{enumerate}
\item Every map $S^1 \to X$ is nulhomotopic.
\item Every map $S^1 \to X$ extends to a map $D^2 \to X$.
\item For every $x \in X$, $\pi_1(X, x) = 0$.
\end{enumerate}
Conclude that $X$ is simply connected iff every map $S^1 \to X$ is nulhomotopic.
\end{exer}

Clearly 3 implies 1.
We first show 1 implies 2.
Let us write elements of $D^2$ in polar coordinates, thus $(r, \theta) \in (0, 1] \times [0, 2\pi)$ denotes the point usually written $(r \cos \theta, r \sin \theta) \in D^2 \setminus 0$.

\begin{lemma}
If $f: S^1 \to X$ is nulhomotopic, then $f$ extends to a map $D^2 \to X$.
\end{lemma}
\begin{proof}
Let $H: I \times S^1 \to X$ be a homotopy $x \to f$, where $x \in X$ is a constant map.
Let
$$f^\sharp(r, \theta) = H(r, e^{i\theta})$$
and $f^\sharp(0) = x$. Then $f^\sharp$ is clearly continuous, since $H$ is a homotopy from $x$, and
$$f^\sharp(1, \theta) = H(1, e^{i\theta}) = f(e^{i\theta})$$
so $f^\sharp|S^1 = f$.
\end{proof}

This lemma obviously implies that 1 implies 2.
If 2 holds, then let $f$ be a loop in $X$ based at $x$, and $f^\sharp$ its extension to $D^2$.
Writing
$$H(r, e^{i\theta}) = f^\sharp(r, \theta)$$
we obtain a homotopy $f \to y$ where $y = H(0)$. But $y, x$ are both in the continuous image of $D^2$, thus in the same path-component; so there is a homotopy $y \to x$, so there is a homotopy $f \to x$. This implies 1 and 3.

Now if $X$ is simply connected, then for every $x \in X$, $\pi_1(X, x) = 0$, so every map $f: S^1 \to X$ is nulhomotopic, say to a point $x_f \in X$. But also $\pi_0(X)$ is trivial so $x_f$ is homotopic to $x_g$ for any $g: S^1 \to X$, thus $f,g$ are homotopic.
Conversely, if any two maps $f,g$ are homotopic, then in particular every map is nulhomotopic, so $\pi_1(X, x) = 0$, and any two points are homotopic, thus in the same path-component, so $\pi_0(X)$ is trivial.

\begin{exer}
Let $[S^1, X]$ denote the space of homotopy classes of maps $S^1 \to X$, and let
$$\Phi: \pi_1(X, x_0) \to [S^1, X]$$
be the map that forgets basepoints. Show that if $X$ is path connected then $\Phi$ is surjective, and $\Phi([f]) = \Phi([g])$ iff $[f]$ and $[g]$ are conjugate in $\pi_1(X, x_0)$.
\end{exer}

Suppose that $X$ is path connected.
Let $f: S^1 \to X$, $x = f(1)$, and let $p$ be a path $x \to x_0$. Then $p^{-1}fp$ is a loop based at $x_0$, so its homotopy class is in the image of $\Phi$. Moreover $f$ and $p^{-1}fp$ are homotopic obtained by moving the basepoint along $p$.
Therefore $\Phi$ is surjective.

If $\Phi([f]) = \Phi([g])$, then there is a homotopy $H: f \to g$; say that $f$ is the top of $I^2$ and $g$ is the bottom. Since $f,g$ have the same basepoint $x_0$, the left and right sides of $H$ are a path $x_0 \to x_0$. Therefore $[p][f] = [g][p]$, so $[f],[g]$ are conjugate.

Conversely if $[p][f] = [g][p]$, then we can freely homotopy $[p][f][p^{-1}]$ to $[p^{-1}][p][f]$ and thus to $[f]$ by rotating the domain $S^1$ by $2\pi/3$ radians.

\begin{exer}
Show that $S^1 \wedge S^n \cong S^{n+1}$.

Let $(M, \mu, e)$ be an $A_2$-space. Show that for any two loops $\alpha, \beta \in \pi_1(M, e)$, $\alpha*\beta$ is homotopic to $\mu(\alpha, \beta)$.

Show that if $(M, \mu, e)$ is an $A_2$-space, then $\pi_1(M, e)$ is abelian.

Let $X$ be based with homotopy group $\pi_n(X)$. Show that if $n \geq 2$, then $\pi_n(X)$ is abelian.
\end{exer}

I could not figure out for the life of me what the smash product $S^1 \wedge I^n$ is supposed to be, so I attacked this problem a somewhat different way.
We will need the following result from functional analysis.

\begin{lemma}
Let $X$ be a compact Hausdorff space.
Up to homeomorphism, $X$ is determined by the abelian $C^*$-algebra $C(X)$ of continuous functions $X \to \CC$ under complex conjugation and the $L^\infty$-norm.
\end{lemma}
Here a morphism of abelian $C^*$-algebras is a morphism of $\CC$-algebras which also respects the involution (which here is complex conjugation) and is continuous with respect to the norm (here the $L^\infty$-norm).
This can be proven by showing that the space of maximal ideals of $C(X)$ with the Zariski topology (equivalently, the weakstar topology) is $X$ under the identification $x \mapsto \{f \in C(X): f(x) = 0\}$, a fact that, once all the definitions have been put in place, is basically trivial.
(So this result is really just a version of the Nullstellensatz, equivalently the Yoneda lemma, for compact Hausdorff spaces.)
However, this is not a course in functional analysis, so we omit the details.
I learned it from my functional analysis course two years ago, and then again the following semester when I took a course on $C^*$-algebras.

\begin{theorem}
One has $S^1 \wedge S^n = S^{n+1}$.
\end{theorem}
\begin{proof}
Following the problem statement, we note that a based function $f: S^1 \wedge S^n \to \CC$ (where the basepoint of $\CC$ is $0$) is the same as a function $Af: S^1 \times S^n \to \CC$ such that for every $(x, y) \in S^1 \times S^n$, $Af(1, y) = Af(x, 1) = 0$.
Then $A$ is an isomorphism of $C^*$-algebras, which follows because $A$ clearly respects addition, scalar multiplication, function multiplication, complex conjugation, and the $L^\infty$-norm.

But $S^k$ is by definition the one-point compactification of $\RR^k$, so $S^1 \times S^n$ is a compactification of $\RR^{1+n}$.
That means that a function $g: S^1 \times S^n \to \CC$ satisfying $g(1, y) = g(x, 1) = 0$ is the same as a function $Bg: \RR^{1 + n} \to \CC$ such that for every $(x, y) \in \RR^{1+n}$,
\begin{equation}
\label{Bg}
\lim_{|y'| \to \infty} Bg(x, y') = \lim_{|x'| \to \infty} Bg(x', y) = 0.
\end{equation}
As with $B$, $A$ is an isomorphism of $C^*$-algebras.
(Here we identify bounded functions on $\RR^{1+n}$ with their extension to a sufficiently large compactification, which has a $C^*$-algebra.)

Therefore there is a canonical, $C^*$-structure-preserving, identification between based functions on $S^1 \wedge S^n$ and functions on $\RR^{1+n}$ satisfying (\ref{Bg}).
But if $h = Bg$ satisfies (\ref{Bg}), then in fact
$$\lim_{|(x', y')| \to \infty} h(x', y') = 0.$$
So $h$ extends to a based function $Ch$ on the one-point compactification $S^{1+n}$, in a way that again preserves the $C^*$-structure.
That means that we have an isomorphism $CBA$ relating based functions $S^1 \wedge S^n$ to based functions on $S^{n+1}$.
So $S^1 \wedge S^n = S^{n+1}$ up to homeomorphism, by the lemma.
\end{proof}

For the second statement, we recall the Eckmann-Hilton lemma.

\begin{lemma}[Eckmann-Hilton]
Let $(M, *, 1)$ and $(M, *', 1')$ be unital magmas such that for every $(a, b, c, d) \in M^4$,
\begin{equation}
\label{EH}
(a * b) *' (c * d) = (a *' c) * (b *' d).
\end{equation}
Then $(M, *, 1)$ and $(M, *', 1')$ are both the same unital magma $M$, and $M$ is an abelian monoid.
\end{lemma}
\begin{proof}
One has $1 = 1 * 1 = (1' *' 1) * (1 *' 1) = (1' * 1) *' (1 * 1') = 1' *' 1' = 1'$.
Similarly one has
$$a * b = (1 *' a) * (b *' 1) = b * a.$$
So the units coincide and $*$ is commutative. Also
$$b * a = (b * 1) *' (1 * a) = (b *' 1) * (1 *' a) = b * a$$
so the operations coincide. Plugging in $c = 1$ in (\ref{EH}) immediately gives associativity.
\end{proof}

\begin{theorem}
Let $(M, \mu, e)$ be an $A_2$-space. For any $\alpha,\beta \in \pi_1(M, e)$, $\alpha*\beta \sim \mu(\alpha, \beta)$, and $\pi_1(M, e)$ is abelian.
\end{theorem}
\begin{proof}
Obviously $\pi_1(M, e)$ is a unital magma under its group operation.
The definition of an $A_2$-space implies that $\pi_1(M, e)$ is also a unital magma under $\mu$. Now if $t < 1/2$,
$$\mu(a*c, b*d)(t) = \mu(a, b)(t).$$
Similarly one has $\mu(a*c,b*d)(t) = \mu(c, d)(t)$ if $t > 1/2$. Putting it together we see that
$$\mu(a*c, b*d) = \mu(a, b) * \mu(c, d)$$
which is just a restatement of (\ref{EH}).
By the Eckmann-Hilton lemma, $\mu$ is the group operation on $\pi_1(M, e)$ and that $\pi_1(M, e)$ is abelian.
\end{proof}

\begin{theorem}
Let $X$ be a based space. If $n \geq 2$ then $\pi_n(X)$ is an abelian group.
\end{theorem}
\begin{proof}
Following the hint, we note that $\pi_n(X) = \pi_0(F(S^n, X))$ and $F(S^n, X)$ is an $A_2$-space.
We first show that the $A_2$-structure on $F(S^n, X)$ drops to an $A_2$-structure on $\pi_0(F(S^n, X))$.
Let $[\cdot]$ be the natural map $F(S^n, X) \to \pi_0(F(S^n, X))$.

If $f,g$ are based maps, and $[f] = [f']$ then let $H: f \to f'$ be a homotopy.
Then $H$ can be viewed as a map $H: I \times S^n \to X$, which does not affect the other copy of $S^n$ in $S^n \vee S^n$ since they only interact at the basepoint $\heart \in S^n$, and $f(\heart) = f'(\heart) = *$ since $f'$ is based.
That is, $H(\cdot, \heart) = *$.
So $H$ extends by the trivial homotopy $g \to g$ to a map $H: I \times (S^n \vee S^n) \to X$ which is a homotopy $f \vee g \to f' \vee g$.
Similarly if $[g] = [g']$, one can show that $f \vee g$ is homotopic to $f' \vee g'$.
It follows (from the definition of the $A_2$-structure $+$ in $F(S^n, X)$) that $f + g$ is homotopic to $f' + g'$.
So $+$ drops to an $A_2$-structure on $\pi_n(X) = \pi_0(F(S^n, X))$.

Now we show that $+$ is abelian. By construction, $f + g$ is the map which, for every $x \in$ the equator $E$, satisfies $(f+g)(x) = *$; in the north hemisphere is equal to a copy of $f|\RR^n$ (where we think of $S^n = \RR^n \cup \{\heart\}$ as a one-point compactification of $\RR^n$, and the north hemisphere $\overline{D^n}$ as a compactification), and in the south hemisphere is a copy of $g|\RR^n$.
By shrinking the northern hemisphere $\overline{D^n}$ to a small neighborhood of the north pole and mapping the rest of the northern hemisphere to $\heart$, then similarly shrinking the southern hemisphere to a small neighborhood of the south pole, one homotopies $f + g$ to a map which is constant except in a neighborhood of the poles.
One can then rotate the sphere to swap the poles, and reverse the previous homotopy, to obtain a homotopy from $f + g$ to $g + f$.
\end{proof}

\begin{exer}
Let $p$ be a loop in $\CC$, $a \in \CC$ not in the image of $p$. The winding number of $p$ is the degree of
$$\hat p(z) = \frac{p(z) - a}{|p(z) - a|}$$
as a map $\hat p: S^1 \to S^1$. Let $f$ be a rational function with no poles or zeroes on the circle $\partial B(0, r)$.
Show that the total number of zeroes minus poles of $f$ in $B(0, r)$ is the winding number around $0$ of $t \mapsto f(re^{2\pi it})$.
\end{exer}

It will be convenient to prove this in the higher generality that $f$ is meromorphic on a neighborhood of $\partial B(0, r)$, simply because rationality is a distracting hypothesis.

Define the \emph{analytic winding number} $A_\gamma$ of a loop $\gamma$ in $\CC$ by
$$A_\gamma = \frac{1}{2\pi i} \int_0^1 \frac{\gamma'(t)}{\gamma(t)} ~dt.$$
By the argument principle, $A_{f \circ p}$ is the number of zeroes (minus poles, counted with multiplicity) of $f$ inside the loop $p$.
For this problem we take $p(t) = re^{2\pi it}$, and $\gamma(t) = f(p(t)) = f(re^{2\pi it})$.
Since $f$ has no zeroes on $\partial B(0, r)$, $\gamma$ has no zeroes, so $\gamma$ is a loop in the punctured plane $\CC^*$.
We must show that $A_\gamma$ is equal to the topological winding number $T_\gamma$ of $\gamma$.

But $S^1$ is a strong deformation retract of $\CC^*$.
In particular, $\gamma$ is homotopic to the loop $\hat \gamma$ that is purely in $S^1$, where the homotopy is the retraction map.
Complex integration is homotopy invariant, so $A_\gamma = A_{\hat \gamma}$, and we can replace $\hat \gamma$ with any element of its homotopy class in $\pi_1(S^1)$; thus we might as well assume that $\hat \gamma(t) = e^{2\pi int}$ where $n \in \ZZ$ is the degree of $\hat \gamma$.
So $A_\gamma = n$, but by definition, $T_\gamma = n$.

\begin{exer}
Let $f: X \to Y$ be a map between path-connected spaces. Let $M_f$ the mapping cylinder of $f$; that is, the pushout of the diagram
$$\begin{tikzcd}X \times \{0\} \arrow[r] \arrow[d,"f"] & X \times I\\ Y\end{tikzcd}.$$
Show that $Y$ is a strong deformation retract of $M_f$.

Let $CX$ be the cone on $X$, namely
$$CX = \frac{X \times I}{X \times \{1\}}.$$
Show that $CX$ is contractible.
\end{exer}

We note that $M_f$ consists of a disjoint union of $X \times I$ and $Y$, glued together along the bottom of the cylinder using the identification $f$.
In fact, this is already true on the level of sets, and all the maps involved are continuous so this is also true in the category of topological spaces.

We introduce a homotopy $H: M_f \times I \to M_f$ which fixes $Y$ and acts on $X \times I$ by
\begin{align*}
H: X \times I \times I &\to X \times I\\
(x, t, s) &\mapsto (x, \kappa(t, s))
\end{align*}
where $\kappa$ is any homotopy from the identity on $I$ to the constant map $1$.
Since $X \times \{1\}$ is identified with $f(X) \subseteq Y$, $H(X \times I \times \{1\}) \subseteq Y$, so $H$ is a deformation retract of $M_f$ into $Y$.
Since $H$ fixes $Y$, $H$ is strong.

To see that $CX$ is contractible, we again use the homotopy $H$ defined above. This squishes $X \times I$ into $X \times \{1\}$, which was crushed to a point. Therefore $H$ is a homotopy from the identity to a constant map.

\begin{exer}
Let $f: X \to Y$ be a map between path-connected spaces. Compute $\pi_1(M_f)$.

Let $C_f$ be the mapping cone of $f$; that is, the pushout of the diagram
$$\begin{tikzcd}X \arrow[r] \arrow[d,"f"] & CX\\ Y\end{tikzcd}.$$
Compute $\pi_1(C_f)$.

Let $g: X \to Y$ be another map. Let $T_{f,g}$ be the mapping torus of $(f,g)$; that is, the pushout of the diagram
$$\begin{tikzcd}
X \times \partial I \arrow[r,"f;g"] \arrow[d] & Y\\
X \times I\end{tikzcd}.$$
Compute $\pi_1(T_{f,g})$.

Now suppose that $f: A \to X$, $g: A \to Y$ are two maps and $A$ is path-connected. Let $M_{f,g}$ be the double mapping cylinder of $(f,g)$; that is, the pushout of the diagram
$$\begin{tikzcd}
A \times \partial I \arrow[r,"f;g"] \arrow[d] & X \coprod Y \\
A \times I\end{tikzcd}.$$
Compute $\pi_1(M_{f,g})$.
\end{exer}

Since $Y$ is a strong deformation retract of $M_f$, $\pi_1(Y) = \pi_1(M_f)$, so $\pi_1(M_f) = \pi_1(Y)$.

\begin{theorem}
$\pi_1(C_f) = \pi_1(Y/f(X))$.
\end{theorem}
\begin{proof}
The mapping cone $C_f$ is defined by gluing $CX$ to $Y$ by identifying the base $X \times \{0\}$ of the cone with $f(X) \subseteq Y$.
Let $U$ be the image of $X \times [0, 3/4)$ in $C_f$, which is open, and $V$ be $Y$ unioned with the image of $X \times (1/2, 1]$.
Then $U \cap V = X \times (1/2, 3/4)$, which is clearly path-connected. Fix a basepoint in $U \cap V$.
By the Seifert-van Kampen theorem,
$$\pi_1(C_f) = \pi_1(U) *_{\pi_1(U \cap V)} \pi_1(V).$$
Since $CX$ is contractible, so is $U$, so $\pi_1(U) = 0$.
On the other hand, it easily follows from the fact that $Y$ is a strong deformation retract of $M_f$ that $Y$ is a strong deformation retract of $V$, so $\pi_1(Y) = \pi_1(V)$.
In particular $\pi_1(C_f)$ is a quotient of $\pi_1(Y)$.
The resulting map $\pi_1(Y) \to \pi_1(C_f)$ is therefore surjective and does nothing but kill loops in $f(X)$, since they are identified with elements of $\pi_1(X) = 0$. Therefore the claim follows.
\end{proof}

\begin{theorem}
$\pi_1(T_{f,g}) = \pi_1(X) * \pi_1(Y/(X \vee X))$.
\end{theorem}
\begin{proof}
The double mapping torus $T_{f,g}$ is the pushout of the diagram
$$\begin{tikzcd}
X \vee X \arrow[r, "f \vee g"] \arrow[d] & Y\\
X \wedge I_+.
\end{tikzcd} $$
We first compute $X \wedge I_+$. Let $x_0$ be the basepoint of $X$. Then $X \wedge I_+ = X \times I/(\{x_0\} \times I)$.
So $T_{f,g}$ consists of $Y$ glued to $(X \times I/\{x\} \times I)$, where $f(X)$ is identified with $X \times \{0\}$ and $X \times \{1\}$ is identified with $g(X)$.
Indeed, $X \vee X = (X \times \{0, 1\})/(\{x_0\} \times \{0, 1\})$, so $(f \vee g)(X \vee X)$ is mapped to the correct subsets of $X \wedge I_+$ and $Y$.

Let
$$U = Y \cup \{(x, i) \in X \wedge I_+: d(i, \partial I) < 1/4\}.$$
Here we take the convention that $x_0$ is only paired with $0$.
Then $U$ is path-connected; indeed, if $(x, i) \in X \wedge I_+$ and $y \in Y$ then one can take a path from $(x, i)$ to $f(X)$ or $g(X)$, depending on if $i < 1/4$ or $i > 3/4$.
Then one can take a path from $f(X)$ or $g(X)$ to $y$, since $Y$ is path-connected.
Moreover, one can homotope $\{(x, i) \in X \wedge I_+: d(i, \partial I) < 1/4\}$ to $X \vee X$, which is identified with $f(X) \cup g(X) \subseteq Y$.
Therefore $Y$ is a strong deformation retract of $U$, so $\pi_1(U) = \pi_1(Y)$.

Let
$$V = \{(x, i) \in X \wedge I_+: d(i, \partial I) > 1/8\}$$
(so in particular that $x_0 \notin V$). Since $X$ is path-connected, so is $V$.
Moreover, $V \cap Y$ is empty, and $X \times \{1/2\} \cong X$ is a strong deformation retract of $V$, so $\pi_1(V) = \pi_1(X)$.

As for $U \cap V = \{(x, i) \in X \wedge I_+: d(i, \partial I) \in (1/4, 1/8)$, $X \times \{3/8\} \cong X$ is a strong deformation retract of $U \cap V$, so $U \cap V$ is path-connected and $\pi_1(U \cap V) = \pi_1(X)$.
Thus the Seifert-van Kampen theorem gives
$$\pi_1(T_{f,g}) = \pi_1(X) *_{\pi_1(X)} \pi_1(Y)$$
where the amalgamation refers to the two maps
$$f_*, g_* \in \Hom(\pi_1(X), \pi_1(Y)).$$
Therefore the claim holds.
\end{proof}

\begin{theorem}
$\pi_1(M_{f,g}) = \pi_1(X) *_{\pi_1(A)} \pi_1(Y)$, where we have maps $f_*: \pi_1(A) \to \pi_1(X)$ and $g_*: \pi_1(A) \to \pi_1(Y)$.
\end{theorem}
\begin{proof}
The double mapping cylinder of $(f, g)$ is defined by gluing $f(A)$ to $A \times \{0\}$ and $g(A)$ to $A \times \{1\}$.
Let $U = X \cup (A \times [0, 3/4))$ and $V = Y \cup (A \times (1/4, 1])$ and fix a basepoint in $A \times \{1/2\}$.
Then $U \cap V = A \times (1/4, 3/4)$ is path-connected, and the Seifert-van Kampen theorem provides the claim.
\end{proof}

\begin{exer}
Compute the fundamental group of the space obtained from two tori $S^1 \times S^1$ by identifying $S^1 \times \{x_0\}$ in one torus with the corresponding circle in the other torus.
\end{exer}

We recall that the fundamental group of $S^1 \times S^1$ is $\ZZ^2$, with one generator generated by each circle.
Let $a,b$ be the generators of the first torus and $c,d$ be the generators of the second.
After choosing representatives correctly and reordering the generators, we can assume that the image of $b$ was glued to the image of $d$. Therefore $b$ is identified with $d$ in the glued space $A$.

Let $U$ be an open neighborhood of the first torus in $A$ and $V$ of the second. Then $U \cap V$ can be chosen to be the gluing of two cylinders along the image of $b$, so $S^1$ is a strong deformation retract of $U \cap V$ and thus $\pi_1(U \cap V) = \pi_1(S^1)$ is the free group generated by $b$.
By the Seifert-van Kampen theorem,
$$\pi_1(A) = \pi_1(S^1 \times S^1) *_{\pi_1(S^1)} \pi_1(S^1 \times S^1)$$
where $\pi_1(S^1)$ maps into the subgroup generated by $b$ in the left factor and $d$ in the right factor.
Therefore $\pi_1(A)$ admits a presentation
$$(a, b, c:~ ab = ba,~ bc=cb).$$

\end{document}
