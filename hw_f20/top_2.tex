
% --------------------------------------------------------------
% This is all preamble stuff that you don't have to worry about.
% Head down to where it says "Start here"
% --------------------------------------------------------------

\documentclass[10pt]{article}

\usepackage[margin=.7in]{geometry}
\usepackage{amsmath,amsthm,amssymb}
\usepackage{enumitem}
\usepackage{tikz-cd}
\usepackage{mathtools}
\usepackage{amsfonts}
\usepackage{listings}
\usepackage{algorithm2e}
\usepackage{verse,stmaryrd}
\usepackage{fancyvrb}

% Number systems
\newcommand{\NN}{\mathbb{N}}
\newcommand{\ZZ}{\mathbb{Z}}
\newcommand{\QQ}{\mathbb{Q}}
\newcommand{\RR}{\mathbb{R}}
\newcommand{\CC}{\mathbb{C}}
\newcommand{\PP}{\mathbb P}
\newcommand{\FF}{\mathbb F}
\newcommand{\DD}{\mathbb D}
\renewcommand{\epsilon}{\varepsilon}

\newcommand{\Aut}{\operatorname{Aut}}
\newcommand{\cl}{\operatorname{cl}}
\newcommand{\ch}{\operatorname{ch}}
\newcommand{\Con}{\operatorname{Con}}
\newcommand{\coker}{\operatorname{coker}}
\newcommand{\CVect}{\CC\operatorname{-Vect}}
\newcommand{\Cantor}{\mathcal{C}}
\newcommand{\D}{\mathcal{D}}
\newcommand{\card}{\operatorname{card}}
\newcommand{\dbar}{\overline \partial}
\newcommand{\diam}{\operatorname{diam}}
\newcommand{\dom}{\operatorname{dom}}
\newcommand{\End}{\operatorname{End}}
\DeclareMathOperator*{\esssup}{ess\,sup}
\newcommand{\GL}{\operatorname{GL}}
\newcommand{\Hess}{\operatorname{Hess}}
\newcommand{\Hom}{\operatorname{Hom}}
\newcommand{\id}{\operatorname{id}}
\newcommand{\Ind}{\operatorname{Ind}}
\newcommand{\Inn}{\operatorname{Inn}}
\newcommand{\interior}{\operatorname{int}}
\newcommand{\lcm}{\operatorname{lcm}}
\newcommand{\mesh}{\operatorname{mesh}}
\newcommand{\LL}{\mathcal L_0}
\newcommand{\Leb}{\mathcal{L}_{\text{loc}}^2}
\newcommand{\Lip}{\operatorname{Lip}}
\newcommand{\ppGL}{\operatorname{PGL}}
\newcommand{\ppic}{\vspace{35mm}}
\newcommand{\ppset}{\mathcal{P}}
\DeclareMathOperator{\proj}{proj}
\DeclareMathOperator*{\Res}{Res}
\newcommand{\Riem}{\mathcal{R}}
\newcommand{\RVect}{\RR\operatorname{-Vect}}
\newcommand{\Sch}{\mathcal{S}}
\newcommand{\SL}{\operatorname{SL}}
\newcommand{\sgn}{\operatorname{sgn}}
\newcommand{\spn}{\operatorname{span}}
\newcommand{\Spec}{\operatorname{Spec}}
\newcommand{\supp}{\operatorname{supp}}
\newcommand{\TT}{\mathcal T}
\DeclareMathOperator{\tr}{tr}

% Calculus of variations
\DeclareMathOperator{\pp}{\mathbf p}
\DeclareMathOperator{\zz}{\mathbf z}
\DeclareMathOperator{\uu}{\mathbf u}
\DeclareMathOperator{\vv}{\mathbf v}
\DeclareMathOperator{\ww}{\mathbf w}

% Categories
\newcommand{\Ab}{\mathbf{Ab}}
\newcommand{\Cat}{\mathbf{Cat}}
\newcommand{\Group}{\mathbf{Group}}
\newcommand{\Module}{\mathbf{Module}}
\newcommand{\Set}{\mathbf{Set}}
\DeclareMathOperator{\Fun}{Fun}
\DeclareMathOperator{\Iso}{Iso}
\DeclareMathOperator{\Map}{Map}

% Complex analysis
\renewcommand{\Re}{\operatorname{Re}}
\renewcommand{\Im}{\operatorname{Im}}

% Logic
\renewcommand{\iff}{\leftrightarrow}
\newcommand{\Henkin}{\operatorname{Henk}}
\newcommand{\PA}{\mathbf{PA}}
\DeclareMathOperator{\proves}{\vdash}

% Group
\DeclareMathOperator{\Gal}{Gal}
\DeclareMathOperator{\Fix}{Fix}
\DeclareMathOperator{\Out}{Out}

% Other symbols
\newcommand{\heart}{\ensuremath\heartsuit}
\newcommand{\club}{\ensuremath\clubsuit}

\DeclareMathOperator{\atanh}{atanh}

% Theorems
\theoremstyle{definition}
\newtheorem*{falselemma}{Grader's ``Lemma"}
\newtheorem{exer}{Exercise}
\newtheorem{lemma}{Lemma}[exer]
\newtheorem{theorem}[lemma]{Theorem}
\newtheorem{corollary}[lemma]{Corollary}

\def\Xint#1{\mathchoice
{\XXint\displaystyle\textstyle{#1}}%
{\XXint\textstyle\scriptstyle{#1}}%
{\XXint\scriptstyle\scriptscriptstyle{#1}}%
{\XXint\scriptscriptstyle\scriptscriptstyle{#1}}%
\!\int}
\def\XXint#1#2#3{{\setbox0=\hbox{$#1{#2#3}{\int}$ }
\vcenter{\hbox{$#2#3$ }}\kern-.6\wd0}}
\def\ddashint{\Xint=}
\def\dashint{\Xint-}

\usepackage[backend=bibtex,style=alphabetic,maxcitenames=50,maxnames=50]{biblatex}
\renewbibmacro{in:}{}
\DeclareFieldFormat{pages}{#1}

\begin{document}
\noindent
\large\textbf{Algebraic Topology, HW 2} \hfill \textbf{Aidan Backus} \\

% --------------------------------------------------------------
%                         Start here
% --------------------------------------------------------------\
I talked about most of these problems with Megan Chang-Lee, Steven Creech, Matthew Emerson and Nate Gillman.
In addition, I consulted the books of Hatcher, Bredon, and Dugundji, as well as Terry Tao's notes on Galois theory.

\begin{exer}
Show that if $v - e + f = 2$, then there is a CW-structure on $S^2$ with $v$ vertices, $e$ edges, and $f$ faces.
\end{exer}

We prove this by induction on $e = v + f - 2$. When $e = 0$ then, because any CW-structure on a surface must have $f \geq 1$, and if there are two faces then their common boundary is an edge (so $f \leq 1$ if $e = 0$), we see that $v = f = 1$, which is only possible if $S^2$ is realized as the one-point compactification of a face; that is, $S^2$ consists of a face with its boundary glued together into a vertex.

Now suppose that any solution of $v - e + f = 2$ can be realized when $e = e_0$.
Then one can realize $(v + 1) - (e_0 + 1) + f = 2$ by splitting an edge into two edges and putting a vertex in their intersection.
One can realize $v - (e_0 + 1) + (f + 1) = 2$ by splitting a face into two faces and putting an edge along their intersection.

\begin{exer}
Describe all connected covering spaces of $\PP^n$.
\end{exer}

We first consider when $n = 1$. Then $\PP^1 = S^1$ has fundamental group $\ZZ$, which is abelian and hence has all subgroups normal.
Let $n\ZZ$ be a normal subgroup of $\ZZ$; then $n\ZZ$ determines a covering space of $S^1$, given by the self-map $z \mapsto z^n$ of $S^1$.

Now suppose that $n \geq 2$. Then $\PP^n = S^n/(\ZZ/2)$, since $\pi_1(\PP^n) = \ZZ/2$, and $S^n$ is simply connected.
So the only covering space is the double cover $S^n \to \PP^n$, except for the trivial cover $\PP^n \to \PP^n$.

\begin{exer}
Describe all connected covering spaces of the Klein bottle.
\end{exer}

Consider the flat Klein bottle $[0, 1]^2/R$ where $R$ is an equivalence relation that glues the vertical edges in an orientation-preserving way and the horizontal edges in an orientation-reversing way.
Then gluing two flat bottles along two horizontal edges of opposite orientation gives a torus.
This map gives a double cover $S^1 \times S^1 \to K$, where $K$ is the Klein bottle.
Gluing an infinite strip of Klein bottles along vertical edges we get a cover by the M\"obius band with deck transformation group $\ZZ$.

Covering spaces of the torus are easy to classify: each of the factors $S^1$ can be covered by $S^1$ winding around itself some number of times, or by $\RR$. So all covering spaces are given by $S^1 \times S^1$, $S^1 \times \RR$, $\RR \times S^1$, or $\RR^2$, where the maps down from $S^1$ are of the form $z \mapsto z^n$ and the maps down from $\RR$ are the universal covering maps for $S^1$.
Since the torus is a double cover of $K$, each of these covering maps induces two covering maps of $K$.

Covering spaces of the M\"obius band $M$ induce one covering space of $K$ for each element of $\ZZ$.
Since $M$ is ``$S^1 \times \RR$ with $\ZZ/2$-torsion", we get a trivial cover $\RR \to \RR$ for the second factor, and covers $S^1 \to S^1$, $z \mapsto z^n$ for the second factor.

It seems like there could be other covering spaces obtained by some trickery with fundamental polygons but I sure don't see how.
The trouble is that even conjugacy classes of $\pi_1(K)$ seem totally intractable to compute.
Clearly $\pi_1(K) = \langle a, b: aba^{-1} = b^{-1}\rangle$, which is a semidirect product of $\ZZ$ with itself according to the $\ZZ/2$-action $1 \mapsto -1$.
However, we then run into the following technical issue:
\begin{lemma}
I don't have the faintest idea how to recover conjugacy classes of a (nonabelian) semidirect product.
\end{lemma}
\begin{proof}
I'm an analyst.
\end{proof}

\begin{exer}
Find all connected, $3$-sheeted covering spaces of the figure eight.
Consider the consequences for its fundamental group.
\end{exer}

A connected, $3$-sheeted covering space of the figure eight $X$ is given by a graph which is homotopy equivalent to one with $3$ vertices, that all have degree $4$.
Indeed, the figure eight is a graph with $1$ vertex of degree $4$.
So now we classify such graphs.

\begin{lemma}
If $G \to X$ is a $3$-sheeted connected covering space of $X$, then $G$ is a graph, and up to graph isomorphism, $G$ is one of:
\begin{enumerate}
\item The graph $H_{58,3}$ depicted in diagram (3) on page 58 of Hatcher.
\item A triangle equipped with one loop at each vertex.
\item A triangle equipped with a loop at one vertex, and two additional edges between its other two vertices.
\item The graph $H_{58,5}$ depicted in diagram (5) on page 58 of Hatcher.
\end{enumerate}
\end{lemma}
\begin{proof}
Let the vertices of $G$ be denoted $a,b,c$.
Since $G$ is connected, after reordering we can find edges $ a \to b$ and $ b \to c$.
Let $\alpha,\beta$ freely generate $\pi_1(X)$.

We first claim that either there is an edge $a \to c$, or $G$ is the graph with two edges $a \to b$ and $b \to c$, and loops at $a, c$; thus $G$ is isomorphic to $H_{58,3}$.
Suppose that there are no edges $a \to c$.
For parity reasons, there must then be a second edge $a \to b$.
If there is a third edge $a \to b$, then because there was already an edge $b \to c$, there cannot be a fourth edge $a \to b$, but nor can there be a loop based at $a$, so one has a contradiction.
Therefore the last edge out of $b$ must be spent on an edge $b \to c$.
There are two more edges out of $a,c$ to spend, so by assumption there must be loops at $a,c$.
Therefore $G$ is isomorphic to $H_{58,3}$.
As discussed in Hatcher, $H_{58,3}$ is a cover of $X$.

So if $G$ is not isomorphic to $H_{58,3}$, then $G$ contains a copy of the triangle $K_3$.
Let $\alpha$ lift to the loop that runs around the triangle.
All vertices in $K_3$ have degree $4$, so there are still $2$ degrees left to ``spend".
It is possible that every vertex has a loop on it, in which $G$ covers $X$: let $\beta$ lift to the loops based at each vertex.

Suppose that $G$ also has a vertex which does not have a loop on it, say $a$.
Suppose also that there is another vertex, say $b$, such that there are three edges $a \to b$.
In that case the degrees of $a,b$ are all spent, and $c$ must have a loop on it.
Then $G$ covers $X$: let $\beta$ lift to the cycle $a \to b \to a$ which does not include any edge of  $K_3$.

Otherwise, there are two edges $a \to b$ and two edges $a \to c$.
In that case the degree of $a$ is all spent, and only one of the degree of $b,c$ is left.
So there must be an edge $b \to c$. Clearly this graph is isomorphic to $H_{58,5}$.
\end{proof}

\begin{theorem}
Let $G \to X$ be a covering space of $X$. Then $G$ is, up to isomorphism of covering spaces, one of the following seven covering spaces:
\begin{enumerate}
\item The graph $H_{58,3}$, with basepoint taken to be a vertex with a loop.
\item The graph $H_{58,3}$, with basepoint taken to be the vertex without a loop.
\item A triangle equipped with one loop at each vertex, one of which we take as a basepoint.
\item A triangle equipped with a loop at one vertex, which we take as a basepoint, and two additional edges between its other two vertices.
\item A triangle equipped with a loop at one vertex, and two additional edges between its other two vertices, one of which we take as a basepoint.
\item The graph $H_{58,5}$, with basepoint taken to be a vertex, and any two edges between two vertices oriented in the same direction.
\item The graph $H_{58,5}$, with basepoint taken to be a vertex, and any two edges between two vertices oriented in opposite directions.
\end{enumerate}
\end{theorem}
\begin{proof}
We have already shown that up to graph isomorphism, these are the only covering spaces, and each of the listed covering spaces is indeed a valid covering space.
So we just need to show that on those graphs, these are the only possible covering spaces, and each description determines exactly one covering space up to isomorphism.
Let $a,b$ be the generators of $\pi_1(X)$.

First consider $H_{58,3}$. Either the basepoint $x$ has a loop or not.
If so, then up to a relabeling we can take that loop to be $a$; then the other edges from $x$ must be $b$, which passes through the loopless vertex $y$. Then the other two edges passing through $y$ must be $a$, and the other loop must be $b$.
Otherwise, the basepoint does not have a loop, so two of the edges out of $x$, corresponding to $a$, must pass through a looped vertex $y$; the other two edges must correspond to $b$. So the loop at $y$ must correspond to $b$ and the other loop must correspond to $a$.

Now if $G$ is a triangle with all vertices looped, $G$ is sufficiently symmetric that it does not matter which vertex we take as $x$.
If the loop at $x$ is $a$, then the $K_3$ must be $b$, and the other loops must also be $a$.

If $G$ is a triangle which only has one vertex looped, suppose that $x$ is looped, with loop $a$.
Then the $K_3$ must be $b$, and the two leftover edges must be $a$.
If $x$ is not looped, then let the $K_3$ be $b$; then the loop and the two leftover edges must be $a$.
The orientation of the leftover edges does not matter; one is always oriented opposite to $K_3$ and the other is oriented identically to $K_3$.

Finally consider $H_{58,5}$. It is sufficiently symmetric that the choice $x$ does not matter.
Choose a $K_3$ in $H_{58,5}$, and let $a$ correspond to that $K_3$.
Then $b$ corresponds to the other $K_3$.
Either $a,b$ have the same orientation, or reversed orientations; either uniquely determines a covering space.
\end{proof}

\begin{lemma}
If $G \to X$ is a covering space, then $G$ is homotopy equivalent to the $4$-petaled rose $(S^1)^{\vee 4}$.
\end{lemma}
\begin{proof}
Let $G$ be a graph which admits a covering space $G \to X$.
Any such graph $G$ is homotopy equivalent to a $4$-petaled rose.
Indeed, if $T$ is a spanning tree of $G$ and the vertices of $G$ are $a,b,c$, then after reordering we can assume that $T$ contains edges $e: a \to b$ and $f: b \to c$.
We claim that $T$ is in fact a maximal spanning tree.
Indeed, if $d$ is an edge which does not appear in $T$ and $d$ is not a loop, then either $d$ is an edge $a \to b$, $b \to c$, or $a \to c$.
The former two are unacceptable because $T$ already contains edges $a \to b$ and $b \to c$. The latter is unacceptable because then $T$ contains a copy of the triangle $K_3$.

After retracting $e,f$ to points, we obtain a graph $G'$ with one vertex; thus $G'$ is a rose.
Each of the edges except $e,f$ becomes a loop in $G'$.
By the handshaking lemma, $G$ has $6$ edges; we retracted two to points, so $G'$ is a $4$-petaled rose which is a retract of $G$.
Moreover, $G$ and $G'$ are homotopy equivalent.
\end{proof}

\begin{theorem}
Every index-$3$ subgroup of $\ZZ^{*2}$ is free on four generators.
\end{theorem}
\begin{proof}
Let $G \to X$ be a covering space.
By the previous lemma and the Seifert-van Kampen theorem, $\pi_1(G) \cong \ZZ^{*4}$.
On the other hand, $\pi_1(X) \cong \ZZ^{*2}$.
If $H$ is a subgroup of $\ZZ^{*2}$ of index $3$, then $H$ is the fundamental group of a $3$-sheeted cover, hence $H \cong \pi_1(G) \cong \ZZ^{*4}$.
\end{proof}

\begin{exer}
Suppose that $G$ is a topological group, and let $p: \widetilde G \to G$ be its universal cover.
Show that there is a unique group structure on $\widetilde G$ such that $p$ is a morphism of groups.
\end{exer}

Let $m,i$ be the group operations of $G$. Let $\tilde e$ be an element of $p^{-1}(e) \subseteq \widetilde G$.
Then the map $m \circ (p \times p)$ has a unique lift $\tilde m: \widetilde G \times \widetilde G \to \widetilde G$ making the diagram
$$\begin{tikzcd}
\widetilde G \times \widetilde G \arrow[r, "\tilde m"] \arrow[d,"p \times p"] & \widetilde G \arrow[d,"p"]\\
G \times G \arrow[r,"m"] & G
\end{tikzcd}$$
commute such that $\tilde m(\tilde e, \tilde e) = \tilde e$.
Similarly, one can find a lift $\tilde i: \widetilde G \to \widetilde G$ of $i \circ p$ such that $\tilde i(\tilde e) = \tilde e$.
Then clearly $p$ is a morphism of groups, if $\tilde G$ is a group.
The map $f(x) = \tilde m(\tilde e, x)$ is a lift of $p$, but the identity map of $\tilde G$ is a lift of $p$, so $f$ is the identity.
Similarly $x \mapsto \tilde m(x, \tilde e)$ is the identity.
A similar argument, lifting different versions of the map $(x, y, z) \mapsto p(x)p(y)p(z)$, shows that $\tilde m$ is associative, which was to be shown.
Since the universal covering group obviously is initial in the category of covering groups, it is unique by general abstract nonsense.

\begin{exer}
Say that a covering space is abelian if it is a normal cover whose Galois group is abelian.

Let $X$ be a path-connected, locally path-connected, semilocally simply connected space.
Show that $X$ has a unique universal abelian covering space.

Compute the universal abelian covering space of $S^1 \vee S^1$ and $S^1 \vee S^1 \vee S^1$.
\end{exer}

We recall the following form of the Galois correspondence. By $\Gal(A/B)$ we mean the group of deck transformations of a covering space $B \to A$.

\begin{lemma}[fundamental theorem of Galois theory]
Suppose that $M \to N$ and $N \to L$ are covering spaces, and $\Gal(M/N)$ is a normal subgroup of $\Gal(M/L)$.
Then $\Gal(N/L) = \Gal(M/L)/\Gal(M/N)$.
\end{lemma}

Here we take $L = X$, $M \to L$ the universal cover of $X$, and $N \to L$ the covering space corresponding to the commutator of $\pi_1(X) = \Gal(M/L)$, which is always normal.
Then $\Gal(N/L)$ is the abelianization of $\pi_1(X)$.
By the universal property of the abelianization (it is initial among abelian quotients of $\pi_1(X)$) and the universal property of the universal cover (it is initial among connected covering spaces over $X$), it follows that $N \to L$ is initial among abelian connected covering spaces over $X$, so satisfies the universal property of the universal abelian covering space.
Moreover, since its definition is by a universal property, this implies that $N \to L$ is unique.

It will be convenient to dispense of both universal abelian covers at the same time:
\begin{theorem}
Let $(S^1)^{\vee n}$ be the $n$-petaled rose, and let $A_n \to (S^1)^{\vee n}$ be its universal abelian cover.
Then $A_n$ is the Cayley graph of $\ZZ^n$.
\end{theorem}
\begin{proof}
By the Seifert-van Kampen theorem, $F_n = \pi((S^1)^{\vee n})$ is free on $a_1, \dots, a_n$, where $a_i$ is a loop around the $i$th petal.
In particular, the universal cover of $(S^1)^{\vee n}$ is the Cayley graph $C_n$ of $F_n$.
We have already shown that we can obtain $A_n$ by taking a quotient of $C_n$ by the deck action of the commutator of $F_n$.
The vertices of $C_n$ are exactly the reduced words in $a_1, \dots, a_n$, and the vertices of $A_n$ will therefore consist of reduced words of the form $a_1^{k_1} \cdots a_n^{k_n}$.
The map
$$a_1^{k_1} \cdots a_n^{k_n} \mapsto (k_1, \dots, k_n)$$
is a bijection between the vertices of $A_n$ and $\ZZ^n$.

Now let $(k_1, \dots, k_n)$ be a vertex in $A_n$.
If $j \in \{1, \dots, n\}$, then there are edges
$$(k_1, \dots, k_n) \to (k_1, \dots, k_{j-1}, k_j \pm 1, k_{j+1}, \dots, k_n)$$
obtained by taking the projection of the edge $a_1^{k_1} \cdots a_n^{k_n} \to a_1^{k_1} \cdots a_n^{k_n} a_j$ in $C_n$.
In fact there are $2n$ such edges.
On the other hand, $A_n$ is a covering graph of $(S^1)^{\vee n}$, so every vertex of $A_n$ has degree $2n$, and thus there can be no other edges.
\end{proof}



\begin{exer}
Say that $i: A \to X$ has the homotopy extension property with respect to $Y$ if for every map $f: X \to Y$ and homotopy $H: A \times I \to Y$ with $H(\cdot, 0) = f \circ i$, there is an extended homotopy $\tilde H: X \times I \to Y$ with $\tilde H(\cdot, 0) = f$ and $\tilde H(i(a), t) = H(a, t)$.
Say that $i$ is a cofibration if for every $Y$, $i$ has the homotopy extension property with respect to $Y$.

Suppose that
$$\begin{tikzcd}
A \arrow[r,"f"] \arrow[d,"i"] & B \arrow[d,"j"]\\
X \arrow[r,"g"] & Y
\end{tikzcd}$$
is a pushout diagram. Show that if $i$ is a cofibration, then so is $j$.

Suppose that $i: A \to X$ is a closed inclusion. Show that $i$ is a cofibration iff the mapping cylinder $M_i$ is a strong deformation retract of $X \times I$.

Let $i: A \to X$ be a cofibration. Show that $i$ is injective, and if $X$ is Hausdorff, then $i(A)$ is closed.

Suppose that
$$\begin{tikzcd}
A \arrow[r,"f"] \arrow[d,"i"] & B \arrow[d,"j"]\\
X \arrow[r,"g"] & Y
\end{tikzcd}$$
is a pushout diagram. Show that if $i$ is a cofibration, then $Y$ is homotopy equivalent to the homotopy pushout $M_{i,f}$, the double mapping cylinder of $(i, f)$.
\end{exer}

\begin{theorem}
If $f: A \to B$, $i: A \to X$, $g: X \to Y$, and $j: B \to Y$ are given, and $Y$ is the pushout of the above maps, and $i$ is a cofibration, then $j$ is a cofibration.
\end{theorem}
\begin{proof}
Let $Z$ be a topological space. We must show that $j$ has the $Z$-homotopy extension property.
Let $\Map(I, Z) \to Z$ be the evaluation map $\alpha \mapsto \alpha(0)$, let $h: Y \to Z$ be a map, and suppose we are given a homotopy $H: B \to \Map(I, Z)$ making the diagram
$$\begin{tikzcd}
B \arrow[r,"H"] \arrow[d,"j"] & \Map(I, Z) \arrow[d] \\
Y \arrow[r,"h"] & Z
\end{tikzcd}$$
commute. We must find a homotopy $\tilde H: Y \to \Map(I, Z)$ which commutes with all the maps in the above homotopy diagram.
Combining the homotopy diagram and the pushout diagram we obtain a diagram
$$\begin{tikzcd}
& A \arrow[d, "i"] \arrow[rr,"f"] && B \arrow[d,"j"] \arrow[dr,"H"] \\
\Map(I, Z) \arrow[drr] & X \arrow[rr,"g"] && Y \arrow[dl,"h"] & \Map(I, Z) \arrow[dll]\\
&&Z
\end{tikzcd}$$
where the two copies of $\Map(I, Z)$ have an identity arrow connecting them (not shown because it would be too hard to draw in tikz).
We fill in the above diagram to get a diagram
$$\begin{tikzcd}
& A \arrow[d, "i"] \arrow[rr,"f"] \arrow[dl,color=orange] && B \arrow[d,"j"] \arrow[dr,"H"] \\
\Map(I, Z) \arrow[drr] & X \arrow[rr,"g"] \arrow[dr,color=orange] \arrow[l,color=blue] && Y \arrow[dl,"h"] \arrow[r,color=red] & \Map(I, Z) \arrow[dll]\\
&&Z
\end{tikzcd}$$
where first the orange arrows were added by composition, then the fact that $i$ is a cofibration gave us the blue arrow, and finally the red arrow was added using the universal property of the pushout.
The red arrow is a witness to the fact that $j$ has the $Z$-HEP.
\end{proof}

We now prove a long sequence of implications, which together will imply the following theorem, which comprised two parts of the exercise.
The reason for doing it in this awkward way is that we needed some of the facts from part ``c" to prove part ``b", and we needed to be careful that our argument not be circular.

\begin{theorem}
Let $i: A \to X$ be a map. If $i$ is a cofibration, then $i$ is an embedding. In fact, if $X$ is Hausdorff, then $i$ is a closed embedding.

Moreover, if $i$ is a closed embedding, then $i$ is a cofibration if and only if the mapping cylinder $M_i$ is a strong deformation retract of $X \times I$.
\end{theorem}

\begin{lemma}
If $i$ is a cofibration, then there is a surjective map $r: X \times I \to M_i$ which makes the diagram
$$\begin{tikzcd}
A \arrow[rr] \arrow[dd,"i"] && A \times I \arrow[dl] \arrow[dd]\\
& X \times I \arrow[dr,"r"]\\
X \arrow[ur] \arrow[rr] && M_i
\end{tikzcd}$$
commute.
\end{lemma}
\begin{proof}
By assumption, $i$ has the $M_i$-HEP. In particular, we can fill in the diagram
$$\begin{tikzcd}
A \arrow[rr] \arrow[dd,"i"] && A \times I \arrow[dl] \arrow[dd]\\
& X \times I \\
X \arrow[ur] \arrow[rr] && M_i
\end{tikzcd}$$
with an arrow
$$\begin{tikzcd}
A \arrow[rr] \arrow[dd,"i"] && A \times I \arrow[dl] \arrow[dd]\\
& X \times I \arrow[dr,"r",color=red]\\
X \arrow[ur] \arrow[rr] && M_i
\end{tikzcd}.$$
Let $(x, t) \in M_i$. If $t = 0$ then, since the bottom triangle commutes, $r(x, 0) = (x, 0)$.
Otherwise, if $t > 0$, then, since the right triangle commutes and $(x, t)$ lifts to an element of $A \times I$, $r(x, t) = (x, t)$.
\end{proof}

\begin{lemma}
If $i$ is a cofibration, then $i$ is an embedding.
\end{lemma}
\begin{proof}
We now show that $i$ is an embedding. Let $r: X \times I \to M_i$ be the surjective map from the previous result.
Then $r \circ (i, 1)$ maps $A$ onto the top of $M_i$.
But the top of $M_i$ is homeomorphic to $A$ in a natural way, by the universal property of $M_i$, and can be identified with $A$.
Therefore $r \circ (i, 1)$ is the identity on $A$, so, identifying $i$ with $(i, 1)$, $r$ is a left inverse of $i$.
So $i$ must be an embedding.
\end{proof}

\begin{lemma}
If $i$ is a cofibration, then $r$ is a retraction.
\end{lemma}
\begin{proof}
We showed that $r$ is surjective and $i$ is an embedding. Therefore $r$ is a surjective map from $X \times I$ to a subspace of $X \times I$.
Since $r$ was defined to make the diagram
$$\begin{tikzcd}
A \arrow[rr] \arrow[dd,"i"] && A \times I \arrow[dl] \arrow[dd]\\
& X \times I \arrow[dr,"r"]\\
X \arrow[ur] \arrow[rr] && M_i
\end{tikzcd}$$
commute, and the arrows $X \to M_i$ and $A \times I \to M_i$ are inclusions, commutativity of the bottom and right triangles imply that $r$ fixes $M_i$.
So $r$ is a retraction.
\end{proof}

\begin{lemma}
If $i$ is a cofibration and $X$ is Hausdorff, then $i$ is a closed embedding.
\end{lemma}
\begin{proof}
Since $M_i$ is a retract of the Hausdorff space $X \times I$, $M_i$ is closed in $X \times I$.
In particular,
$$U = \{(x, t) \in X \times (0, 1]: x \notin A\}$$
is open, so $V = (X \setminus A) \times \{1\}$ is open.
Therefore $V$ is open when viewed as a subset of $X$. So $A$ is closed.
\end{proof}

\begin{lemma}
If $i$ a cofibration and a closed embedding, then $M_i$ is a strong deformation retract of $X \times I$.
\end{lemma}
\begin{proof}
Consider the homotopy
$$H(x, s, t) = (p_X(r(x, st)), (1 - t)s + tp_I(r(x, s))).$$
Then $H(x, s, 0) = (p_X(r(x, 0)), s) = (x, s)$ since $p_X \circ r$ is the identity (since the diagram that defined $r$ commuted).
Also $H(x, s, 1) = (p_X(r, s), p_I(r(x, s))) = r(x, s)$.
So $H$ is a homotopy from the identity to $r$, and hence a deformation retraction.
It is strong since if $x \in A$ then $p_X(r(x, st)) = x$ and $(1 - t)s + tp_I(r(x, s)) = (1 - t + t)s = s$.
\end{proof}

\begin{lemma}
If $M_i$ is a strong deformation retract of $X \times I$, then $i$ is a cofibration.
\end{lemma}
\begin{proof}
Let $r: X \times I \to M_i$ be the induced retraction.
Let $Y$ be a topological space; we must show that $i$ has the $Y$-HEP.
Consider the homotopy extension diagram
$$\begin{tikzcd}
A \arrow[rr] \arrow[dd,"i"] && A \times I \arrow[dl] \arrow[dd,"H"]\\
& X \times I \\
X \arrow[ur] \arrow[rr,"f"] && Y
\end{tikzcd}.$$
Deleting the middle we obtain a diagram
$$\begin{tikzcd}
A \arrow[r] \arrow[d,"i"] & A \times I \arrow[d,"H"]\\
X \arrow[r,"f"] & Y
\end{tikzcd}$$
and by the universal property of the pushout $M_i$, we obtain a map $\alpha: M_i \to Y$.
Then the diagram
$$\begin{tikzcd}
A \arrow[rr] \arrow[dd,"i"] && A \times I \arrow[dl] \arrow[dd] \arrow[dddr,"H"]\\
& X \times I \arrow[dr,"r"] \\
X \arrow[ur] \arrow[rr] \arrow[drrr,"f"] && M_i \arrow[dr,"\alpha"]\\
&&& Y
\end{tikzcd}$$
commutes. Therefore $\tilde H = \alpha \circ r$ gives an arrow that we can insert into our original diagram
$$\begin{tikzcd}
A \arrow[rr] \arrow[dd,"i"] && A \times I \arrow[dl] \arrow[dd,"H"]\\
& X \times I \arrow[dr,"\tilde H",color=red] \\
X \arrow[ur] \arrow[rr,"f"] && Y
\end{tikzcd}$$
to complete the proof.
\end{proof}

\begin{corollary}
If $i$ is a cofibration and
$$\begin{tikzcd}
A \arrow[r,"f"] \arrow[d,"i"] & B \arrow[d,"j"]\\
X \arrow[r,"g"] & Y
\end{tikzcd}$$
is a pushout diagram, then $Y$ is homotopy equivalent to the homotopy pushout $M_{i,f}$.
\end{corollary}
\begin{proof}
First, $M_{i,f}$ is given by the pushout
$$\begin{tikzcd}
A \arrow[r,"i \times \{1\}"] \arrow[d,"f"] & M_i \arrow[d] \\
B \arrow[r] & M_{i,f}
\end{tikzcd}$$
which attaches $B$ to the top of the mapping cylinder $M_i$.

Let $Z$ be defined by the pushout diagram
$$\begin{tikzcd}
A \arrow[r,"f \times \{1\}"] \arrow[d,"i"] & B \arrow[d,"j"]\\
X \times I \arrow[r,"g"] & Z
\end{tikzcd}.$$
Then $Z$ is clearly homotopy equivalent to $Y$.
Since $i$ is a cofibration, $M_i$ is a strong deformation retract of $X \times I$; say by $H: 1 \sim r$.
Let $X \times I/f$ denote $X \times I$ modulo the equivalence relation $(a, 1) \sim (a', 1)$ whenever $f(a) = f(a')$ and $a \in A$.
Then $H(t)$ drops to a suitable map $H/f: X \times I/f \to M_i/f$, and $H/f: 1 \sim r/f$ is a strong deformation retraction.
Gluing $B$ to the top of $M_i/f$ to obtain $M_{i,f}$ using the first pushout diagram in this proof, we can extend $H/f$ to a strong deformation retraction of $Z$ onto $M_{i,f}$.

Therefore $Z$ and $M_{i,f}$ are homotopy equivalent, and $Z$ is clearly homotopy equivalent to $Y$, so we're done.
\end{proof}



\end{document}
