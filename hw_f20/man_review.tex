
% --------------------------------------------------------------
% This is all preamble stuff that you don't have to worry about.
% Head down to where it says "Start here"
% --------------------------------------------------------------

\documentclass[10pt]{article}

\usepackage[margin=.7in]{geometry}
\usepackage{amsmath,amsthm,amssymb}
\usepackage{enumitem}
\usepackage{tikz-cd}
\usepackage{mathtools}
\usepackage{amsfonts}
\usepackage{listings}
\usepackage{algorithm2e}
\usepackage{verse,stmaryrd}
\usepackage{fancyvrb}

% Number systems
\newcommand{\NN}{\mathbb{N}}
\newcommand{\ZZ}{\mathbb{Z}}
\newcommand{\QQ}{\mathbb{Q}}
\newcommand{\RR}{\mathbb{R}}
\newcommand{\CC}{\mathbb{C}}
\newcommand{\PP}{\mathbb P}
\newcommand{\FF}{\mathbb F}
\newcommand{\DD}{\mathbb D}
\renewcommand{\epsilon}{\varepsilon}

\newcommand{\Aut}{\operatorname{Aut}}
\newcommand{\cl}{\operatorname{cl}}
\newcommand{\ch}{\operatorname{ch}}
\newcommand{\Con}{\operatorname{Con}}
\newcommand{\coker}{\operatorname{coker}}
\newcommand{\CVect}{\CC\operatorname{-Vect}}
\newcommand{\Cantor}{\mathcal{C}}
\newcommand{\D}{\mathcal{D}}
\newcommand{\card}{\operatorname{card}}
\newcommand{\dbar}{\overline \partial}
\newcommand{\diam}{\operatorname{diam}}
\newcommand{\dom}{\operatorname{dom}}
\newcommand{\End}{\operatorname{End}}
\DeclareMathOperator*{\esssup}{ess\,sup}
\newcommand{\Hess}{\operatorname{Hess}}
\newcommand{\Hom}{\operatorname{Hom}}
\newcommand{\id}{\operatorname{id}}
\newcommand{\Ind}{\operatorname{Ind}}
\newcommand{\Inn}{\operatorname{Inn}}
\newcommand{\interior}{\operatorname{int}}
\newcommand{\lcm}{\operatorname{lcm}}
\newcommand{\mesh}{\operatorname{mesh}}
\newcommand{\LL}{\mathcal L_0}
\newcommand{\Leb}{\mathcal{L}_{\text{loc}}^2}
\newcommand{\Lip}{\operatorname{Lip}}
\newcommand{\ppic}{\vspace{35mm}}
\newcommand{\ppset}{\mathcal{P}}
\DeclareMathOperator{\proj}{proj}
\DeclareMathOperator*{\Res}{Res}
\newcommand{\Riem}{\mathcal{R}}
\newcommand{\RVect}{\RR\operatorname{-Vect}}
\newcommand{\Sch}{\mathcal{S}}
\newcommand{\sgn}{\operatorname{sgn}}
\newcommand{\spn}{\operatorname{span}}
\newcommand{\Spec}{\operatorname{Spec}}
\newcommand{\supp}{\operatorname{supp}}
\newcommand{\TT}{\mathcal T}
\DeclareMathOperator{\tr}{tr}

% Calculus of variations
\DeclareMathOperator{\pp}{\mathbf p}
\DeclareMathOperator{\zz}{\mathbf z}
\DeclareMathOperator{\uu}{\mathbf u}
\DeclareMathOperator{\vv}{\mathbf v}
\DeclareMathOperator{\ww}{\mathbf w}

% Categories
\newcommand{\Ab}{\mathbf{Ab}}
\newcommand{\Cat}{\mathbf{Cat}}
\newcommand{\Group}{\mathbf{Group}}
\newcommand{\Module}{\mathbf{Module}}
\newcommand{\Set}{\mathbf{Set}}
\DeclareMathOperator{\Fun}{Fun}
\DeclareMathOperator{\Lie}{Lie}
\DeclareMathOperator{\Iso}{Iso}

% Complex analysis
\renewcommand{\Re}{\operatorname{Re}}
\renewcommand{\Im}{\operatorname{Im}}

% Logic
\renewcommand{\iff}{\leftrightarrow}
\newcommand{\Henkin}{\operatorname{Henk}}
\newcommand{\PA}{\mathbf{PA}}
\DeclareMathOperator{\proves}{\vdash}

% Group
\DeclareMathOperator{\Gal}{Gal}
\DeclareMathOperator{\Fix}{Fix}
\DeclareMathOperator{\Out}{Out}

\DeclareMathOperator{\Diffeo}{Diffeo}

\DeclareMathOperator{\ad}{ad}
\DeclareMathOperator{\Ad}{Ad}
\newcommand{\GL}{\operatorname{GL}}
\newcommand{\ppGL}{\operatorname{PGL}}
\newcommand{\SL}{\operatorname{SL}}
\newcommand{\SO}{\operatorname{SO}}
\newcommand{\iprod}{\mathbin{\lrcorner}}


% Other symbols
\newcommand{\heart}{\ensuremath\heartsuit}
\newcommand{\club}{\ensuremath\clubsuit}

\DeclareMathOperator{\atanh}{atanh}
\DeclareMathOperator{\codim}{codim}

% Theorems
\theoremstyle{definition}
\newtheorem*{corollary}{Corollary}
\newtheorem*{falselemma}{Grader's ``Lemma"}
\newtheorem{exer}{Exercise}
\newtheorem{lemma}{Lemma}[exer]
\newtheorem{theorem}[lemma]{Theorem}

\def\Xint#1{\mathchoice
{\XXint\displaystyle\textstyle{#1}}%
{\XXint\textstyle\scriptstyle{#1}}%
{\XXint\scriptstyle\scriptscriptstyle{#1}}%
{\XXint\scriptscriptstyle\scriptscriptstyle{#1}}%
\!\int}
\def\XXint#1#2#3{{\setbox0=\hbox{$#1{#2#3}{\int}$ }
\vcenter{\hbox{$#2#3$ }}\kern-.6\wd0}}
\def\ddashint{\Xint=}
\def\dashint{\Xint-}

\usepackage[backend=bibtex,style=alphabetic,maxcitenames=50,maxnames=50]{biblatex}
\renewbibmacro{in:}{}
\DeclareFieldFormat{pages}{#1}

\begin{document}
\noindent
\large\textbf{Manifolds, Practice Exam} \hfill \textbf{Aidan Backus} \\

% --------------------------------------------------------------
%                         Start here
% --------------------------------------------------------------\

\begin{exer}
State the Whitney embedding theorem, the tubular neighborhood theorem, Sard's theorem, and the Frobenius theorem.
\end{exer}

\begin{theorem}[Whitney embedding]
Every manifold can embedded in a sufficiently high-dimensional euclidean space.
\end{theorem}

\begin{theorem}[tubular neighborhood]
Let $M \subseteq N$ be an embedded submanifold. There is an open $M \subseteq U \subseteq N$ equipped with a projection $\pi: U \to M$ such that locally, $U$ is a neighborhood of $M$ in the normal bundle of $M$ with respect to $N$.
\end{theorem}

\begin{theorem}[Sard]
Let $M$ be a manifold, $\mu$ a Borel measure on $M$, and $f$ a smooth function on $M$.
If $f$ is nowhere constant and $\mu$ is absolutely continuous with respect to the Lebesgue measure of every chart of $M$, then the set of critical points of $f$ is $\mu$-null.
\end{theorem}

\begin{theorem}[Frobenius]
A distribution is involutive iff it can be written in coordinates as the span of $\partial_{x_1}, \dots, \partial_{x_n}$.
\end{theorem}

\begin{exer}
Let $E$ be a closed subset of $M$, $\delta: M \to (0, \infty)$ continuous, $h \in C(M)$ with $h|E$ smooth.
Show that there is a smooth $u$ on $M$ with $h|E = u|E$ and $|h - u| < \delta$ on $M$.
\end{exer}

\begin{exer}
Let $G$ be a positive-dimensional Lie group. Show that $G$ has an orientation with respect to which left translation is orientation preserving.
Show that if $G$ is connected then $G$ has an orientation with respect to with translation is orientation preserving.
\end{exer}

\begin{exer}
Let $M$ be a positive-dimensional manifold. Define the tangent bundle $TM$ of $M$ and show that $TM$ is orientable.
\end{exer}

Given $p \in M$ we define $m_p$ to be the space of smooth functions which vanish to first order at $p$, and $m_p^2$ to be the space of functions that vanish to second order.
Then we define the cotangent space $T^*_pM = m_p/m_p^2$.
The cotangent bundle $T^*M$ is the disjoint union of all cotangent spaces, glued together by identifying $0 \in T^*_pM$ with $p$ in $M$; then $T^*M$ is a vector bundle of rank $\dim M$ over $M$.
One can show that a basis for $T^*M$ in coordinates $(x_i)_i$ is $(dx_i)_i$.
The dual bundle $TM$ is known as the tangent bundle.

It suffices to show that the cotangent bundle $T^*M$ is orientable, since there is a bundle isomorphism $TM \to T^*M$ induced by a choice of Riemannian metric, which in particular is a diffeomorphism.
To show that $T^*M$ is orientable, it suffices to show that it has a top form; we in fact claim that $T^*M$ naturally carries the structure of a symplectic manifold, which implies this, since if $(N, \omega)$ is a symplectic manifold, then since $\omega$ is locally of the form $\sum_i dx_i \wedge d\xi^i$, $\omega^{\wedge n} = \bigwedge_i dx_i \wedge d\xi^i$ is a top form.

We let $\pi: T^*M \to M$ be the projection map, $v \in T^*M$. Then $v$ is a linear map $TM \to \RR$ and $d\pi: TT^*M \to TM$ induces a linear map $v \circ d\pi: TT^*M \to \RR$. Thus $v \circ d\pi \in T^*(T^*M)$.
Since $\theta(v) = v \circ d\pi$ varies smoothly in $v$, $\theta$ is a $1$-form on $T^*M$.
In local coordinates, $\theta(v) = v \circ d\pi$ is the $1$-form $\theta = \sum_i x_i ~d\xi^i$, where $(x_i)_i$ are coordinates on $M$ and $\xi^i = dx_i$ are the induced coordinates on each cotangent space of $M$.
Indeed, it suffices to check this when $M = \RR^n$, $T^*M = \RR^n \times \RR^n$, and $\pi = d\pi$ is projection onto the first coordinate, in which case $(x_i)_i \oplus (\xi_i)_i$ is a basis for $T^*M$, and the claim follows by definition.
So $d\theta = \sum_i dx_i \wedge d\xi^i$ is a symplectic form on $T^*M$.

\begin{exer}
Classify connnected abelian Lie groups up to isomorphism.
\end{exer}

Let $G$ be a connected abelian Lie group of dimension $n$ and $\mathfrak g$ its Lie algebra.
Then $\mathfrak g$ is the unique abelian Lie algebra of dimension $n$, so the Lie correspondence implies that the universal covering group of $G$ is $\RR^n$.
Let $p: \RR^n \to G$ be the universal covering morphism and $\RR^n = \RR_1 \oplus \cdots \oplus \RR_n$.
Then $G = p(\RR_1) \oplus p(\RR_n)$, so the problem reduces to the case when $n = 1$.
In that case, $\ker p$ is a Lie subgroup of $\RR$ of dimension $\leq 1$.
If $\dim \ker p = 1$ then $\ker p$ contains a neighborhood of $0 \in \RR$, so $\ker p = \RR$ and so $n = 0$, a contradiction.
Therefore $\ker p$ is a discrete subgroup of $\RR$.
But the only discrete subgroups of $\RR$ are isomorphic to $\ZZ$, in which case $G$ is the circle group $S$, or the trivial group, in which case $G = \RR$.
Thus in general $G = \RR^k \oplus S^m$ where $S$ is the circle group.

\begin{exer}
Let $A$ be a subset of $\RR^n$ of cardinality $m$ and $W = \RR^n \setminus A$. Compute the cohomology and cohomology with compact support of $W$.
\end{exer}

In both cases $W$ is connected, so $H^0(W) = H^0_c(W) = \RR$.
For cohomology without compact support, since $W$ is homotopy equivalent to a wedge of $m$ spheres of dimension $n - 1$, each centered on a point in $A$, $H^{n-1}(W) \cong \RR^m$, and cohomology vanishes in all other degrees.
For cohomology with compact support, the homotopy equivalence in question is not a proper map, so the argument does not generalize.
TODO

\begin{exer}
Let $M$ be a Riemannian manifold with boundary.
Show that there is a smooth function $f$ on $M$ with Dirichlet boundary condition and $f < 0$ on the interior such that $\nabla f$ is the outer unit normal on the boundary.
\end{exer}

It suffices to check that this can be done locally. Indeed, if $\mathcal U$ is a locally finite open cover of $M$ by charts $U$ on which a suitable function $f_U$ has been defined, we can apply a partition of unity subordinate to $\mathcal U$ and glue together all the scraps $f_U$ to obtain a suitable function $f$.
Moreover, if $U$ is an open set which does not meet the boundary $\partial M$ we may set $f_U = -1$, so we just need to check when $U \cap \partial M$ is nonempty. In that case, we may assume that $U$ is diffeomorphic to $V = \{x \in \RR^d: |x| < 1,~x_d \geq 0\}$, so we can push forward the Riemannian metric of $M$ to $V$, and then it suffices to check when $M = V$.

Let $g$ be the Riemannian metric on $V$. Then for any smooth function $h$,
$$(\nabla h)^j = g^{ij} \frac{\partial h}{\partial x^i}.$$
We want $(\nabla f)^j = 0$ for $j < d$ and $(\nabla f)^d = -1$, so that
$$\frac{\partial f}{\partial x^i} = -g_{ij} \delta^j_d = -g_{id}.$$
Now let $v = (g_{1d}, \dots, g_{dd})$ and set $f(y + tv) = -t$ whenever $y_d = 0$ and $t \geq 0$.
To check that this definition makes sense, we recall that $g_{dd} \neq 0$ whenever the boundary $\partial V = \RR^d$, so $y,v$ are linearly independent.
Then $f(y) = 0$ and $f < 0$ on the interior.

\begin{exer}
Let $M$ be a positive-dimensional oriented Riemannian manifold.
Define the Laplace-Beltrami operator $\Delta$ acting on functions in terms of the Hodge star on $M$.
Show that there is a sign $\epsilon$ such that for every smooth function $f$ on $M$,
$$\epsilon \int_M f \Delta f ~dV \leq 0.$$
Show that if $M$ is compact and connected and $\Delta f \geq 0$ then $f$ is constant.
\end{exer}

We define $\Delta = *d*d$, at least on functions.
In particular, one has an integration by parts formula
$$\int_M f \Delta g ~dV = \int_M \langle df, dg\rangle ~dV$$
where the inner product is the dual of $g$; that is, the inner product induced by $g$ on cotangent spaces.
To see this, we note that
$$\int_M \langle df, dg\rangle ~dV = \int_M df \wedge *dg$$
and
$$\int_M f \Delta g~dV = \int_M f(*d*dg) ~dV = \int_M f \wedge d*dg.$$
But
$$d(f \wedge *dg) = df \wedge *dg - f \wedge d*dg$$
is a closed form, so by Stokes' theorem one has
$$\int_M f \wedge d*dg = \int_M df \wedge *dg$$
proving the claim.
So
$$\int_M f \Delta f ~dV = \int_M ||df||^2 ~dV \geq 0$$
where the norm $||\cdot||$ is the norm on the cotangent space induced by $g$.
Thus the claim holds with $\epsilon = -1$.

Now assume $\Delta f \geq 0$. Then for every Borel set $E \subseteq M$
$$\int_E d*df = \int_E *\Delta f = \int_E \Delta f~dV \geq 0.$$
But no matter one partitions a closed curve $\gamma$ into connected sets $E_1, \dots, E_n$, $\int_{E_i} d*df \geq 0$, so if $\int_E d*df > 0$ for some $E \subset \gamma$ then $\int_\gamma d*df > 0$, even though $d*df$ is exact.
This implies that $*\Delta f = d*df = 0$, so $\Delta f = 0$.
Therefore $f$ is harmonic.
If $[\alpha]$ is the de Rham cohomology class of $\alpha$, then there is a unique harmonic form $\beta \in [\alpha]$.
In this case, since $M$ is connected, $H^0(M)$ is one-dimensional.
Since constant functions are harmonic, it follows that $f$ is cohomologous to a constant function, and therefore is itself constant.





\end{document}
