
% --------------------------------------------------------------
% This is all preamble stuff that you don't have to worry about.
% Head down to where it says "Start here"
% --------------------------------------------------------------

\documentclass[10pt]{article}

\usepackage[margin=.7in]{geometry}
\usepackage{amsmath,amsthm,amssymb}
\usepackage{enumitem}
\usepackage{tikz-cd}
\usepackage{mathtools}
\usepackage{amsfonts}
\usepackage{listings}
\usepackage{algorithm2e}
\usepackage{verse,stmaryrd}
\usepackage{fancyvrb}

% Number systems
\newcommand{\NN}{\mathbb{N}}
\newcommand{\ZZ}{\mathbb{Z}}
\newcommand{\QQ}{\mathbb{Q}}
\newcommand{\RR}{\mathbb{R}}
\newcommand{\CC}{\mathbb{C}}
\newcommand{\PP}{\mathbb P}
\newcommand{\FF}{\mathbb F}
\newcommand{\DD}{\mathbb D}
\renewcommand{\epsilon}{\varepsilon}

\newcommand{\Aut}{\operatorname{Aut}}
\newcommand{\cl}{\operatorname{cl}}
\newcommand{\ch}{\operatorname{ch}}
\newcommand{\Con}{\operatorname{Con}}
\newcommand{\coker}{\operatorname{coker}}
\newcommand{\CVect}{\CC\operatorname{-Vect}}
\newcommand{\Cantor}{\mathcal{C}}
\newcommand{\D}{\mathcal{D}}
\newcommand{\card}{\operatorname{card}}
\newcommand{\dbar}{\overline \partial}
\newcommand{\diam}{\operatorname{diam}}
\newcommand{\End}{\operatorname{End}}
\DeclareMathOperator*{\esssup}{ess\,sup}
\newcommand{\GL}{\operatorname{GL}}
\newcommand{\Hom}{\operatorname{Hom}}
\newcommand{\id}{\operatorname{id}}
\newcommand{\Ind}{\operatorname{Ind}}
\newcommand{\Inn}{\operatorname{Inn}}
\newcommand{\interior}{\operatorname{int}}
\newcommand{\lcm}{\operatorname{lcm}}
\newcommand{\mesh}{\operatorname{mesh}}
\newcommand{\LL}{\mathcal L_0}
\newcommand{\Leb}{\mathcal{L}_{\text{loc}}^2}
\newcommand{\Lip}{\operatorname{Lip}}
\newcommand{\ppGL}{\operatorname{PGL}}
\newcommand{\ppic}{\vspace{35mm}}
\newcommand{\ppset}{\mathcal{P}}
\DeclareMathOperator*{\Res}{Res}
\newcommand{\Riem}{\mathcal{R}}
\newcommand{\RVect}{\RR\operatorname{-Vect}}
\newcommand{\Sch}{\mathcal{S}}
\newcommand{\SL}{\operatorname{SL}}
\newcommand{\sgn}{\operatorname{sgn}}
\newcommand{\Spec}{\operatorname{Spec}}
\newcommand{\supp}{\operatorname{supp}}
\newcommand{\TT}{\mathcal T}
\DeclareMathOperator{\tr}{tr}

% Calculus of variations
\DeclareMathOperator{\pp}{\mathbf p}
\DeclareMathOperator{\zz}{\mathbf z}
\DeclareMathOperator{\uu}{\mathbf u}
\DeclareMathOperator{\vv}{\mathbf v}
\DeclareMathOperator{\ww}{\mathbf w}

% Categories
\newcommand{\Ab}{\mathbf{Ab}}
\newcommand{\Cat}{\mathbf{Cat}}
\newcommand{\Group}{\mathbf{Group}}
\newcommand{\Module}{\mathbf{Module}}
\newcommand{\Set}{\mathbf{Set}}
\DeclareMathOperator{\Fun}{Fun}
\DeclareMathOperator{\Iso}{Iso}

% Complex analysis
\renewcommand{\Re}{\operatorname{Re}}
\renewcommand{\Im}{\operatorname{Im}}

% Logic
\renewcommand{\iff}{\leftrightarrow}
\newcommand{\Henkin}{\operatorname{Henk}}
\newcommand{\PA}{\mathbf{PA}}
\DeclareMathOperator{\proves}{\vdash}

% Group
\DeclareMathOperator{\Gal}{Gal}
\DeclareMathOperator{\Fix}{Fix}
\DeclareMathOperator{\Out}{Out}

% Other symbols
\newcommand{\heart}{\ensuremath\heartsuit}

\DeclareMathOperator{\atanh}{atanh}

% Theorems
\theoremstyle{definition}
\newtheorem*{corollary}{Corollary}
\newtheorem*{falselemma}{Grader's ``Lemma"}
\newtheorem{exer}{Exercise}
\newtheorem{lemma}{Lemma}[exer]
\newtheorem{theorem}[lemma]{Theorem}


\usepackage[backend=bibtex,style=alphabetic,maxcitenames=50,maxnames=50]{biblatex}
\renewbibmacro{in:}{}
\DeclareFieldFormat{pages}{#1}

\begin{document}
\noindent
\large\textbf{Probability, HW 3} \hfill \textbf{Aidan Backus} \\

% --------------------------------------------------------------
%                         Start here
% --------------------------------------------------------------\

\begin{exer}
Let $f: X \to Y$ be a measurable function and $\mathcal G$ the $\sigma$-algebra on $Y$. Let $\sigma(f) = \{f^{-1}(E): E \in \mathcal G\}$.
Show that $\sigma(f)$ is the smallest $\sigma$-algebra on $X$ making $f$ measurable, and if $\mathcal G = \sigma(\mathcal E)$ then
$$\sigma(f) = \sigma(\{f^{-1}(E): E \in \mathcal E\}).$$
\end{exer}

We first check that $\sigma(f)$ is a $\sigma$-algebra.
Indeed, if $E_n \in \mathcal G$, then $f^{-1}(\bigcup_n E_n) = \bigcup_n f^{-1}(E_n)$ so $\sigma(f)$ is closed under countable union. Similarly $\sigma(f)$ is closed under complement.
By definition, $f$ is measurable under $\sigma(f)$.
If $\mathcal K$ is a $\sigma$-algebra under which $f$ is measurable, then for every $E \in \mathcal G$, $f^{-1}(E) \in \mathcal K$, so $\mathcal K$ is contained in $\sigma(f)$.
So $\sigma(f)$ is the smallest $\sigma$-algebra making $f$ measurable.

Now if $\mathcal G = \sigma(\mathcal E)$ then $\mathcal E \subseteq \mathcal G$, so $f$ pulls back $\mathcal E$ to a subset of $\sigma(f)$.
Moreover, if $\mathcal K$ is a $\sigma$-algebra containing the pullback of $\mathcal E$, then $f$ pushes $\mathcal K$ forward to a $\sigma$-algebra that contains $\mathcal G$. So $\sigma(f)$ is the smallest $\sigma$-algebra containing the pullback of $\mathcal E$, as desired.

\begin{exer}
Let $f, g$ be measurable functions on $\Omega$. Show that $\sigma(g) \subseteq \sigma(f)$ iff there is a Borel function $h$ on $\RR$ such that $g = h \circ f$.
\end{exer}

Suppose that $\sigma(g) \subseteq \sigma(f)$. Then for every Borel set $G$ there is a Borel set $F$ such that, $g^{-1}(G) = f^{-1}(F)$.=

If $g$ is a simple function with image $\{x_1, \dots, x_n\}$, and $g^{-1}(\{x_i\}) = E_i$, then let $F_i$ be Borel sets such that $f^{-1}(F_i) = E_i$, and let $h|F_i = x_i$.
If $y$ is not in any $F_i$, let $h(y) = 0$.
Then $h(f|E_i) = h|F_i = x_i$. Therefore $g = h \circ f$.

If $g$ is an arbitrary measurable function, let $g_n \to g$ be a sequence of simple functions which converge pointwise.
Let $h_n$ be the corresponding functions such that $g_n = h_n \circ f$. Then if there is an $h$ such that $h_n \to h$ pointwise, then $g = h \circ f$.
So let $y \in \RR$; we will show that the sequence $n \mapsto h_n(y)$ is Cauchy, so it converges to an $h(y)$.
Let $x_i^n$ be in the image of $g_n$, and let $E_i^n = g_n^{-1}(x_i^n)$.
Then let $F_i^n$ be the corresponding Borel set to $E_i^n$.
If $y \in F_i^n$, then $h_n(y) = x_i^n$, so $h_n(y)$ converges to a point in the image of $g$, and hence is Cauchy.
Otherwise, $y$ is not in the image of $f$ and so one has $h(y) = 0$.

Conversely, suppose that $h$ exists and let $G$ be Borel. Then $g^{-1}(G)$ is Borel, and $g^{-1}(G) = f^{-1}(h^{-1}(G))$, so $G \in \sigma(f)$.

\begin{exer}
Let $\mu, \nu$ be two measures on $\Omega$. Let $f$ be a nonnegative function on $\Omega$. Show that if $\mu \leq \nu$ then $\int_\Omega f ~d\mu \leq \int_\Omega f~d\nu$ and $\lambda(E) = \int_E f~d\mu$ defines a measure on $\Omega$ such that
$$\int_\Omega g~d\lambda = \int_\Omega fg~d\mu$$
whenever $g$ is nonnegative and measurable.
\end{exer}

For the first claim, since $f$ is nonnegative there is an increasing sequence of simple functions that approximate $f$, and by monotone converge we can replace $f$ with a simple function.
Then by linearity we can replace $f$ with an indicator function $1_E$. But then the claim is
$$\mu(E) \leq \nu(E)$$
which is true by definition.

Now we check that $\lambda$ is a measure. Clearly $\lambda$ is nonnegative and not identically $\infty$; now, if $E_n$ are disjoint sets then let $F_n = \bigcup_{m \leq n} E_m$, so $F_n$ is an increasing sequence, and by monotone convergence
\begin{align*}
\lambda\left(\bigcup_n E_n\right) &= \int_\Omega f1_{\bigcup_n E_n}~d\mu = \int_\Omega f1_{\bigcup_n F_n} ~d\mu \\
&= \int_\Omega f\lim_n 1_{F_n}~d\mu = \lim_n \int_\Omega f1_{F_n}~d\mu = \lim_n \sum_{m \leq n} \lambda(E_m) \\
&= \sum_n \lambda(E_n).
\end{align*}

Now to compute $\int_\Omega g~d\lambda$, we note that by monotone convergence and linearity it suffices to check when $g = 1_E$. But then
$$\int_\Omega g~d\lambda = \lambda(E) = \int_E f~d\mu = \int_\Omega fg~d\mu.$$

\begin{exer}
Let $f: [a, b] \times \Omega \to \RR$ be measurable such that $f(t) \in L^1(\Omega, \mu)$. Let
$$F(t) = \int_\Omega f(t, \omega)~d\mu(\omega).$$
Show that if there is a $g \in L^1(\Omega)$ such that for every $(t, \omega) \in [a, b] \times \Omega$ one has $|f(t, \omega)| \leq g(\omega)$, and $f(\omega)$ is continuous for every $\omega$, then $F$ is continuous.

Suppose that $|\partial_t f(t, \omega)| \leq h(\omega)$ for some $h \in L^1(\Omega)$. Show that $F$ is differentiable and
$$F'(t) = \int_\Omega \frac{\partial}{\partial t} f(t, \omega)~d\mu(\omega).$$
\end{exer}

First, we must show that $F$ commutes with limits. The hypothesis on $g$ allows us to use dominated convergence to show that
$$\lim_{s \to t} F(s) = \lim_{s \to t} \int_\Omega f(s, \omega)~d\mu(\omega) = \int_\Omega \lim_{s \to t} f(s, \omega) ~d\mu(\omega) = \int_\Omega f(t, \omega) ~d\mu(\omega) = F(t).$$

For the second claim, we use the hypothesis on $h$ to apply dominated convergence and show that
\begin{align*}
F'(t) &= \lim_{\varepsilon \to 0} \frac{1}{\varepsilon} \int_\Omega f(t + \varepsilon, \omega) - f(t, \omega)~d\mu(\omega) = \int_\Omega \lim_{\varepsilon \to 0} \frac{f(t + \varepsilon, \omega) - f(t, \omega)}{\varepsilon}~d\mu(\omega)\\
&= \int_\Omega \frac{\partial}{\partial t} f(t, \omega) ~d\mu(\omega)
\end{align*}
which was desired.

\begin{exer}
Suppose that $f \in L^1(\mu)$. Show that for every $\varepsilon > 0$ there is a $\delta > 0$ such that if $\mu(E) < \delta$ then $\int_E |f|~d\mu < \varepsilon$.
\end{exer}

Since $f \in L^1(\mu)$ and $L^\infty(\mu)$ is dense in $L^1(\mu)$ (since it contains the simple functions), there is a $g \in L^\infty(\mu)$ such that $||f - g||_1 < \varepsilon$. Then
$$\int_E |f|~d\mu \leq ||f - g||_1 + \int_E |g|~d\mu \leq ||f - g||_1 + \mu(E) ||g||_\infty < \varepsilon + O(\mu(E)).$$
This implies the claim.

\begin{exer}
Suppose that $f \in L^1(\RR)$. Show that
$$\lim_{\varepsilon \to 0} \int_{-\infty}^\infty |f(x + \varepsilon) - f(x)|~dx = 0.$$
\end{exer}

We first check when $f = 1_U$ is the indicator function of an open interval.
Then for every $x \in U$ there is a $\varepsilon$ so small that $x + \varepsilon \in U$ and so $f(x + \varepsilon) = f(x)$.
On the other hand, if $x \notin U$, then either $x \in \partial U$ -- a finite set, which we can ignore -- or $x$ is in the interior of $\RR \setminus U$, which is open, so there is a $\varepsilon$ so small that $x + \varepsilon$ is also in the interior of $\RR \setminus U$.
So $f(x + \varepsilon) = f(x)$.
In any case,
$$\lim_{\varepsilon \to 0} f(x + \varepsilon) = f(x)$$
pointwise, and $f \in L^1(\RR)$, and $2|f|$ dominates $x \mapsto |f(x + \varepsilon) - f(x)|$, so by dominated convergence, the claim holds.

Now if $U$ is an open set, we can write $U$ as the countable union of open intervals, and so $1_U$ as the increasing limit of indicator functions of finite unions of open intervals. Then monotone convergence implies the claim for $f = 1_U$.

Now let $E$ be any measurable set, and $U_n$ open sets such that $\mu(U_n \setminus E) < 1/n$.
Then $1_{U_n} \to 1_E$ pointwise, and we can choose $U_1$ so that $1_{U_n} \in L^1(\RR)$, and for every $n$, $U_n \subseteq U_1$.
Then dominated convergence with domitor $4\cdot 1_{U_1}$ implies that
$$\lim_{\varepsilon \to 0} \int_{-\infty}^\infty |1_E(x + \varepsilon) - 1_E(x)|~dx = \lim_{n \to \infty} \lim_{\varepsilon \to 0} \int_{-\infty}^\infty |1_{U_n}(x + \varepsilon) - 1_{U_n}(x)|~dx = \lim_{n \to \infty} 0 = 0.$$

Now if $f = c1_E + d1_F$, and $c,d > 0$, $E,F$ measurable sets, then
$$|f(x + \varepsilon) - f(x)| \leq c |1_E(x + \varepsilon) - 1_E(x)| + d|1_F(x + \varepsilon) - 1_F(x)|.$$
It follows that $f$ satisfies the claim.
By induction, then, any simple nonnegative function satisfies the claim.
So by monotone convergence, any measurable nonnegative function satisfies the claim.
If $f$ is any integrable function, let $f_\pm$ be its unsigned parts; then
$$|f(x + \varepsilon) - f(x)| \leq |f_+(x + \varepsilon) - f_+(x)| + |f_-(x + \varepsilon) - f_-(x)|.$$
Since the $f_\pm$ are nonnegative measurable functions, they satisfy the claim; so $f$ does as well.

\end{document}
