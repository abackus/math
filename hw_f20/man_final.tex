
% --------------------------------------------------------------
% This is all preamble stuff that you don't have to worry about.
% Head down to where it says "Start here"
% --------------------------------------------------------------

\documentclass[10pt]{article}

\usepackage[margin=.7in]{geometry}
\usepackage{amsmath,amsthm,amssymb}
\usepackage{enumitem}
\usepackage{tikz-cd}
\usepackage{mathtools}
\usepackage{amsfonts}
\usepackage{listings}
\usepackage{algorithm2e}
\usepackage{verse,stmaryrd}
\usepackage{fancyvrb}

% Number systems
\newcommand{\NN}{\mathbb{N}}
\newcommand{\ZZ}{\mathbb{Z}}
\newcommand{\QQ}{\mathbb{Q}}
\newcommand{\RR}{\mathbb{R}}
\newcommand{\CC}{\mathbb{C}}
\newcommand{\PP}{\mathbb P}
\newcommand{\FF}{\mathbb F}
\newcommand{\DD}{\mathbb D}
\renewcommand{\epsilon}{\varepsilon}

\newcommand{\Aut}{\operatorname{Aut}}
\newcommand{\cl}{\operatorname{cl}}
\newcommand{\ch}{\operatorname{ch}}
\newcommand{\Con}{\operatorname{Con}}
\newcommand{\coker}{\operatorname{coker}}
\newcommand{\CVect}{\CC\operatorname{-Vect}}
\newcommand{\Cantor}{\mathcal{C}}
\newcommand{\D}{\mathcal{D}}
\newcommand{\card}{\operatorname{card}}
\newcommand{\dbar}{\overline \partial}
\newcommand{\diam}{\operatorname{diam}}
\newcommand{\dom}{\operatorname{dom}}
\newcommand{\End}{\operatorname{End}}
\DeclareMathOperator*{\esssup}{ess\,sup}
\newcommand{\Hess}{\operatorname{Hess}}
\newcommand{\Hom}{\operatorname{Hom}}
\newcommand{\id}{\operatorname{id}}
\newcommand{\Ind}{\operatorname{Ind}}
\newcommand{\Inn}{\operatorname{Inn}}
\newcommand{\interior}{\operatorname{int}}
\newcommand{\lcm}{\operatorname{lcm}}
\newcommand{\mesh}{\operatorname{mesh}}
\newcommand{\LL}{\mathcal L_0}
\newcommand{\Leb}{\mathcal{L}_{\text{loc}}^2}
\newcommand{\Lip}{\operatorname{Lip}}
\newcommand{\ppic}{\vspace{35mm}}
\newcommand{\ppset}{\mathcal{P}}
\DeclareMathOperator{\proj}{proj}
\DeclareMathOperator*{\Res}{Res}
\newcommand{\Riem}{\mathcal{R}}
\newcommand{\RVect}{\RR\operatorname{-Vect}}
\newcommand{\Sch}{\mathcal{S}}
\newcommand{\sgn}{\operatorname{sgn}}
\newcommand{\spn}{\operatorname{span}}
\newcommand{\Spec}{\operatorname{Spec}}
\newcommand{\supp}{\operatorname{supp}}
\newcommand{\TT}{\mathcal T}
\DeclareMathOperator{\tr}{tr}

% Calculus of variations
\DeclareMathOperator{\pp}{\mathbf p}
\DeclareMathOperator{\zz}{\mathbf z}
\DeclareMathOperator{\uu}{\mathbf u}
\DeclareMathOperator{\vv}{\mathbf v}
\DeclareMathOperator{\ww}{\mathbf w}

% Categories
\newcommand{\Ab}{\mathbf{Ab}}
\newcommand{\Cat}{\mathbf{Cat}}
\newcommand{\Group}{\mathbf{Group}}
\newcommand{\Module}{\mathbf{Module}}
\newcommand{\Set}{\mathbf{Set}}
\DeclareMathOperator{\Fun}{Fun}
\DeclareMathOperator{\Lie}{Lie}
\DeclareMathOperator{\Iso}{Iso}

% Complex analysis
\renewcommand{\Re}{\operatorname{Re}}
\renewcommand{\Im}{\operatorname{Im}}

% Logic
\renewcommand{\iff}{\leftrightarrow}
\newcommand{\Henkin}{\operatorname{Henk}}
\newcommand{\PA}{\mathbf{PA}}
\DeclareMathOperator{\proves}{\vdash}

% Group
\DeclareMathOperator{\Gal}{Gal}
\DeclareMathOperator{\Fix}{Fix}
\DeclareMathOperator{\Out}{Out}

\DeclareMathOperator{\Diffeo}{Diffeo}

\DeclareMathOperator{\ad}{ad}
\DeclareMathOperator{\Ad}{Ad}
\newcommand{\GL}{\operatorname{GL}}
\newcommand{\ppGL}{\operatorname{PGL}}
\newcommand{\SL}{\operatorname{SL}}
\newcommand{\SO}{\operatorname{SO}}
\newcommand{\iprod}{\mathbin{\lrcorner}}


% Other symbols
\newcommand{\heart}{\ensuremath\heartsuit}
\newcommand{\club}{\ensuremath\clubsuit}

\DeclareMathOperator{\atanh}{atanh}
\DeclareMathOperator{\codim}{codim}
\DeclareMathOperator{\PD}{PD}

% Theorems
\theoremstyle{definition}
\newtheorem*{corollary}{Corollary}
\newtheorem*{falselemma}{Grader's ``Lemma"}
\newtheorem{exer}{Exercise}
\newtheorem{lemma}{Lemma}[exer]
\newtheorem{theorem}[lemma]{Theorem}

\def\Xint#1{\mathchoice
{\XXint\displaystyle\textstyle{#1}}%
{\XXint\textstyle\scriptstyle{#1}}%
{\XXint\scriptstyle\scriptscriptstyle{#1}}%
{\XXint\scriptscriptstyle\scriptscriptstyle{#1}}%
\!\int}
\def\XXint#1#2#3{{\setbox0=\hbox{$#1{#2#3}{\int}$ }
\vcenter{\hbox{$#2#3$ }}\kern-.6\wd0}}
\def\ddashint{\Xint=}
\def\dashint{\Xint-}

\usepackage[backend=bibtex,style=alphabetic,maxcitenames=50,maxnames=50]{biblatex}
\renewbibmacro{in:}{}
\DeclareFieldFormat{pages}{#1}

\begin{document}
\noindent
\large\textbf{Manifolds, Final Exam} \hfill \textbf{Aidan Backus} \\

% --------------------------------------------------------------
%                         Start here
% --------------------------------------------------------------\

\begin{exer}[1; Lee 18.7]
Let $M$ be an oriented $n$-manifold. Define
$$\PD(\omega)(\eta) = \int_M \omega \wedge \eta.$$
Show that $\PD$ descends to a linear map $H^p(M) \to H^{n-p}_c(M)^*$.
Show that $\PD$ is an isomorphism on cohomology for all $p$.
\end{exer}

For the first claim, we recall that since $M$ has no boundary,
$$0 = \int_M d(\alpha \wedge \beta) = \int_M d\alpha \wedge \beta \pm \int_M \alpha \wedge d\beta$$
where the sign depends on the degrees of the forms, so one has an integration by parts formula
$$\int_M d\alpha \wedge \beta = \pm \int_M \alpha \wedge d\beta.$$
In particular, if $\omega$ is exact, say $\omega = d\kappa$, then
$$\PD(\omega)(\eta) = \int_M \omega \wedge \eta = \pm \int_M \kappa \wedge d\eta = \pm \PD(\kappa)(d\eta),$$
so $\PD(\omega)$ annihilates closed forms.
Thus if $C^{n-p}_c(M)$ denotes the space of closed forms with compact support, then the corestriction of $\PD$ to $C^{n-p}_c(M)$ contains all exact $p$-forms in its kernel, so $\PD$ drops to a linear map $H^p(M) \to C^{n-p}_c(M)^*$.
However, if $\eta = d\rho$ is exact and $\omega$ is closed, then by the same argument, $\PD(\omega)(\eta) = \PD(d\omega)(\rho) = 0$, so $\PD$ drops to a linear map $H^p(M) \to H^{n-p}_c(M)^*$.

Now we show that $\PD$ is an isomorphism.
Let us say that a smooth manifold $N$ is Poincar\'e if $\PD: H^p(N) \to H^{n-p}_c(N)^*$ is an isomorphism.
Since pullbacks by diffeomorphisms induce isomorphisms on $H^p(N)$ and $H^{n-p}_c(N)$ which commute with the wedge product, the space of Poincar\'e manifolds is closed under diffeomorphism.
As in the proof of de Rham's theorem, we will say that an open cover $\mathcal U$ is Poincar\'e if any finite intersection of scraps in $\mathcal U$ is a Poincar\'e manifold, and say that a basis is Poincar\'e if it is a Poincar\'e open cover.

Throughout we will need to use the fact that if $F: N \to M$ is a smooth proper map then the pullback $F^*: H^p_c(M) \to H^p_c(N)$ dualizes to a map
$$F_*: H^{n-p}_c(N)^* \to H^{n-p}_c(M)^*$$
which we view as a pushforward map.

\begin{lemma}
\label{disjoint poincare}
The disjoint union of countably many Poincar\'e manifolds is Poincar\'e.
\end{lemma}
\begin{proof}
Let $(N_k)$ be a sequence of Poincar\'e manifolds with disjoint union $N$, and inclusion maps $i_k: N_k \to N$.
Then $i_k^*$ induces an isomorphism $i_k^*: H^p(N) \to \prod_k H^p(N_k)$, and since $i_k$ is proper (since it is a diffeomorphism onto its image),
$i_k^*$ is also an isomorphism on compact support, so the pushforward $i_*$ is an isomorphism on the dual spaces.
Since the $N_k$ are Poincar\'e manifolds, there are inverses $\varphi_k: H^{n-p}_c(N_k)^* \to H^p(N_k)$ to $\PD$ on each $k$, so there is an inverse $\varphi: \prod_k H^{n-p}_c(N_k)^* \to \prod_k H^p(N_k)$.
The isomorphisms $i_k^*$ and $(i_k)_*$ are natural and so commute with wedge product; therefore $\varphi$ pulls back along $i_k^*$ and $(i_k)_U$ to an inverse $\psi$ of $\PD$.
\end{proof}

\begin{lemma}
\label{convex poincare}
A convex open subset of $\RR^n$ is a Poincar\'e manifold.
\end{lemma}
\begin{proof}
Let $U$ be such a set.
By the Poincar\'e lemma, if $p > 0$ then $H^p(U) = 0$ and $H^{n-p}_c(U) = 0$ (so $H^{n-p}_c(U)^* = 0$).
So we just need to check when $p = 0$. In fact, $H^0(U)$ is generated by the constant function $1$, while $H^n_c(U)$ is generated by a compactly supported top form, say $\omega$, that does not integrate to $0$. Then $1 \wedge \omega = \omega$, so $\PD(1)$ is nonzero.
Since $\PD$ is a map between one-dimensional spaces and is nonzero, $\PD$ is an isomorphism.
\end{proof}

\begin{lemma}
\label{poincare covers}
A manifold with a finite Poincar\'e cover is Poincar\'e.
\end{lemma}
\begin{proof}
By induction on the cardinality of the Poincar\'e cover, we may reduce to the case $N = U \cup V$ where $U, V$ are open Poincar\'e submanifolds of $N$ and $U \cap V$ is a Poincar\'e manifold.
The Mayer-Vietoris diagram is
\begin{equation}
\label{MV}
\begin{tikzcd}
H^{p-1}_c(U)^* \oplus H^{p-1}_c(V)^* \arrow[r] \arrow[d] & H^{p-1}_c(U \cap V)^* \arrow[r] \arrow[d] & H^p_c(U \cup V)^* \arrow[r] \arrow[d] & H^p_c(U)^* \oplus H^p_c(V)^* \arrow[r] \arrow[d] & H^p_c(U \cap V)^* \arrow[d] \\
H^{p-1}(U) \oplus H^{p-1}(V) \arrow[r] & H^{p-1}(U \cap V) \arrow[r] & H^p(U \cup V) \arrow[r] & H^p(U) \oplus H^p(V) \arrow[r] & H^p(U \cap V)
\end{tikzcd}
\end{equation}
where the rows are exact and the columns are inverses of $\PD$.

To see why (\ref{MV}) commutes, we must show that each square commutes, or equivalently, that the diagram
$$\begin{tikzcd}
H^p(Q) \arrow[r,"F^*"] \arrow[d,"\PD"] & H^p(R) \arrow[d,"\PD"]\\
H^{n-p}_c(Q)^* \arrow[r,"F_*^{-1}"] & H^{n-p}_c(R)^*
\end{tikzcd}$$
commutes whenever $F: R \to Q$ is a smooth proper map between equidimensional manifolds, (in particular, $F$ an inclusion map), and that if $\delta, \delta_c$ are the connecting morphisms in the Mayer-Vietoris sequences without and with compact support respectively, then the diagram
$$\begin{tikzcd}
H^{p-1}(U \cap V) \arrow[r,"\delta"] \arrow[d,"\PD"] & H^p(N) \arrow[d,"\PD"]\\
H^{n+1-p}_c(U \cap V)^* \arrow[r,"\delta_c^*"] & H^{n-p}_c(N)^*
\end{tikzcd}$$
commutes.

For the former diagram, the claim is
\begin{equation}
\label{diagram 1}
F_*(\PD(F^*(\omega)))(\eta) = \PD(\omega)(\eta)
\end{equation}
whenever $\omega,\eta$ are forms of suitable degree on $Q$ and $\eta$ has compact support. In fact,
\begin{align*}
F_*(\PD(F^*(\omega)))(\eta) &= \PD(F^*\omega)(F^*\eta) = \int_R F^* \omega \wedge F^*\eta \\
&= \int_R F^*(\omega \wedge \eta) = \int_Q \omega \wedge \eta \\
&= \PD(\omega)(\eta)
\end{align*}
which proves (\ref{diagram 1}).

For the latter diagram, the claim is
\begin{equation}
\label{diagram 2}
\PD(\delta(\omega))(\eta) = (\delta_c^*(\PD(\omega)))(\eta).
\end{equation}
TODO: Do a diagram chase with the Mayer-Vietoris to make this hold.

By assumption each of the columns of (\ref{MV}) are isomorphisms except $H^p_c(N)^* \to H^p(N)$.
The claim then follows by the five lemma and the commutativity of (\ref{MV}).
\end{proof}

We are now in a position to finish the proof.
Since $M$ is a manifold, $M$ has a basis of charts, which in turn each have a basis of open sets which, in coordinates, are open convex subsets of $\RR^n$.
So, by Lemma \ref{convex poincare}, it suffices to show that every manifold with a Poincar\'e basis is itself Poincar\'e.
But this holds by Lemma \ref{poincare covers} for the exact same reason that de Rham's theorem is true.

To be more precise, if $M$ has a Poincar\'e basis $\mathcal U$, and $f: M \to \RR$ is an exhaustion, then we may define $A_m = \{m \leq f \leq m + 1\}$ and $B_m = \{m - 1/2 < f < m + 3/2\}$.
Then for any $q \in A_m$ we can find a basic open set $U \in \mathcal U$ such that $q \in U$ and $U \subseteq B_m$.
In particular, the basic open sets contained in $B_m$ form an open cover $\mathcal U_m$ of $A_m$.
Since $f$ is an exhaustion, $A_m$ is compact, so we can find finitely many basic open sets $U_{1,m}, \dots, U_{\ell_m,m}$ which cover $A_m$.
Then the $U_{i,m}$ are Poincar\'e manifolds, so by Lemma \ref{poincare covers}, $A_m$ is contained in a Poincar\'e manifold $C_m = \bigcup_i U_{i,m}$.
Thus one has $A_m \subseteq C_m \subseteq B_m$, so $B_m$ only meets $B_j$ when $|j - m| \leq 1$.
Thus if one defines $U = \bigcup_s B_{2s}$, $V = \bigcup_s B_{2s + 1}$, $U \cap V$ is empty, and $U,V$ are disjoint unions of Poincar\'e manifolds, so $U,V$ are themselves Poincar\'e manifolds. Clearly $U \cup V = M$ so $M$ is a Poincar\'e manifold.


\begin{exer}[6]
Let $M$ be a Riemannian manifold with boundary.
Show that there is a smooth function $f$ on $M$ with Dirichlet boundary condition and $f < 0$ on the interior such that $\nabla f$ is the outer unit normal on the boundary.
\end{exer}

It suffices to check that this can be done locally. Indeed, if $\mathcal U$ is a locally finite open cover of $M$ by charts $U$ on which a suitable function $f_U$ has been defined, we can apply a partition of unity subordinate to $\mathcal U$ and glue together all the scraps $f_U$ to obtain a suitable function $f$.
Moreover, if $U$ is an open set which does not meet the boundary $\partial M$ we may set $f_U = -1$, so we just need to check when $U \cap \partial M$ is nonempty. In that case, we may assume that $U$ is diffeomorphic to $V = \{x \in \RR^d: |x| < 1,~x_d \geq 0\}$, so we can push forward the Riemannian metric of $M$ to $V$, and then it suffices to check when $M = V$.

Let $g$ be the Riemannian metric on $V$. Then for any smooth function $h$,
$$(\nabla h)^j = g^{ij} \frac{\partial h}{\partial x^i}.$$
We want $(\nabla f)^j = 0$ for $j < d$ and $(\nabla f)^d = -1$, so that
$$\frac{\partial f}{\partial x^i} = -g_{ij} \delta^j_d = -g_{id}.$$
Now let $v = (g_{1d}, \dots, g_{dd})$ and set $f(y + tv) = -t$ whenever $y_d = 0$ and $t \geq 0$.
To check that this definition makes sense, we recall that $g_{dd} \neq 0$ whenever the boundary $\partial V = \RR^d$, so $y,v$ are linearly independent.
Then $f(y) = 0$ and $f < 0$ on the interior.

TODO: Fix up this answer





\end{document}
