
% --------------------------------------------------------------
% This is all preamble stuff that you don't have to worry about.
% Head down to where it says "Start here"
% --------------------------------------------------------------

\documentclass[10pt]{article}

\usepackage[margin=.7in]{geometry}
\usepackage{amsmath,amsthm,amssymb}
\usepackage{enumitem}
\usepackage{tikz-cd}
\usepackage{mathtools}
\usepackage{amsfonts}
\usepackage{listings}
\usepackage{algorithm2e}
\usepackage{verse,stmaryrd}
\usepackage{fancyvrb}

% Number systems
\newcommand{\NN}{\mathbb{N}}
\newcommand{\ZZ}{\mathbb{Z}}
\newcommand{\QQ}{\mathbb{Q}}
\newcommand{\RR}{\mathbb{R}}
\newcommand{\CC}{\mathbb{C}}
\newcommand{\PP}{\mathbb P}
\newcommand{\FF}{\mathbb F}
\newcommand{\DD}{\mathbb D}
\renewcommand{\epsilon}{\varepsilon}

\newcommand{\Aut}{\operatorname{Aut}}
\newcommand{\cl}{\operatorname{cl}}
\newcommand{\ch}{\operatorname{ch}}
\newcommand{\Con}{\operatorname{Con}}
\newcommand{\coker}{\operatorname{coker}}
\newcommand{\CVect}{\CC\operatorname{-Vect}}
\newcommand{\Cantor}{\mathcal{C}}
\newcommand{\D}{\mathcal{D}}
\newcommand{\card}{\operatorname{card}}
\newcommand{\dbar}{\overline \partial}
\newcommand{\diam}{\operatorname{diam}}
\newcommand{\dom}{\operatorname{dom}}
\newcommand{\End}{\operatorname{End}}
\DeclareMathOperator*{\esssup}{ess\,sup}
\newcommand{\GL}{\operatorname{GL}}
\newcommand{\Hess}{\operatorname{Hess}}
\newcommand{\Hom}{\operatorname{Hom}}
\newcommand{\id}{\operatorname{id}}
\newcommand{\Ind}{\operatorname{Ind}}
\newcommand{\Inn}{\operatorname{Inn}}
\newcommand{\interior}{\operatorname{int}}
\newcommand{\lcm}{\operatorname{lcm}}
\newcommand{\mesh}{\operatorname{mesh}}
\newcommand{\LL}{\mathcal L_0}
\newcommand{\Leb}{\mathcal{L}_{\text{loc}}^2}
\newcommand{\Lip}{\operatorname{Lip}}
\newcommand{\ppGL}{\operatorname{PGL}}
\newcommand{\ppic}{\vspace{35mm}}
\newcommand{\ppset}{\mathcal{P}}
\DeclareMathOperator{\proj}{proj}
\DeclareMathOperator*{\Res}{Res}
\newcommand{\Riem}{\mathcal{R}}
\newcommand{\RVect}{\RR\operatorname{-Vect}}
\newcommand{\Sch}{\mathcal{S}}
\newcommand{\SL}{\operatorname{SL}}
\newcommand{\sgn}{\operatorname{sgn}}
\newcommand{\spn}{\operatorname{span}}
\newcommand{\Spec}{\operatorname{Spec}}
\newcommand{\supp}{\operatorname{supp}}
\newcommand{\TT}{\mathbb T}
\DeclareMathOperator{\tr}{tr}

% Calculus of variations
\DeclareMathOperator{\pp}{\mathbf p}
\DeclareMathOperator{\zz}{\mathbf z}
\DeclareMathOperator{\uu}{\mathbf u}
\DeclareMathOperator{\vv}{\mathbf v}
\DeclareMathOperator{\ww}{\mathbf w}

% Categories
\newcommand{\Ab}{\mathbf{Ab}}
\newcommand{\Cat}{\mathbf{Cat}}
\newcommand{\Group}{\mathbf{Group}}
\newcommand{\Module}{\mathbf{Module}}
\newcommand{\Set}{\mathbf{Set}}
\DeclareMathOperator{\Fun}{Fun}
\DeclareMathOperator{\Iso}{Iso}
\DeclareMathOperator{\Map}{Map}

% Complex analysis
\renewcommand{\Re}{\operatorname{Re}}
\renewcommand{\Im}{\operatorname{Im}}

% Logic
\renewcommand{\iff}{\leftrightarrow}
\newcommand{\Henkin}{\operatorname{Henk}}
\newcommand{\PA}{\mathbf{PA}}
\DeclareMathOperator{\proves}{\vdash}

% Group
\DeclareMathOperator{\Gal}{Gal}
\DeclareMathOperator{\Fix}{Fix}
\DeclareMathOperator{\Out}{Out}

% Other symbols
\newcommand{\heart}{\ensuremath\heartsuit}
\newcommand{\club}{\ensuremath\clubsuit}

\DeclareMathOperator{\atanh}{atanh}

% Theorems
\theoremstyle{definition}
\newtheorem*{falselemma}{Grader's ``Lemma"}
\newtheorem{exer}{Exercise}
\newtheorem{lemma}{Lemma}[exer]
\newtheorem{theorem}[lemma]{Theorem}
\newtheorem{corollary}[lemma]{Corollary}

\def\Xint#1{\mathchoice
{\XXint\displaystyle\textstyle{#1}}%
{\XXint\textstyle\scriptstyle{#1}}%
{\XXint\scriptstyle\scriptscriptstyle{#1}}%
{\XXint\scriptscriptstyle\scriptscriptstyle{#1}}%
\!\int}
\def\XXint#1#2#3{{\setbox0=\hbox{$#1{#2#3}{\int}$ }
\vcenter{\hbox{$#2#3$ }}\kern-.6\wd0}}
\def\ddashint{\Xint=}
\def\dashint{\Xint-}

\usepackage[backend=bibtex,style=alphabetic,maxcitenames=50,maxnames=50]{biblatex}
\renewbibmacro{in:}{}
\DeclareFieldFormat{pages}{#1}

\begin{document}
\noindent
\large\textbf{Algebraic Topology, HW 5} \hfill \textbf{Aidan Backus} \\

% --------------------------------------------------------------
%                         Start here
% --------------------------------------------------------------\

I talked about most of these problems with Megan Chang-Lee, Steven Creech, Matthew Emerson and Nate Gillman.

\begin{exer}
Let $T$ denote the torsion of $H$ and $MT$ denote $H/T$.
Show that $T$ and $MT$ do not define a homology theory.
\end{exer}

Suppose that $T$ defines a homology theory, and consider the good pair $(\PP^2, \PP^1)$.
One has $\PP^2/\PP^1 = S^2$ and $\PP^1 = S^1$.
By the axiom on long exact sequences, there is a long exact sequence
$$\cdots \to T_2(S^2) \to T_1(S^1) \to T_1(\PP^2) \to T_1(S^2) \to \cdots.$$
But spheres have no torsion, so one obtains an isomorphism $0 \cong T_1(\PP^2)$.
Since $H_1(\PP^2) = \ZZ/2$, this is a contradiction.

As for $MT$, we again use $(\PP^2, \PP^1)$.
Then we get a long exact sequence
$$\cdots \to MT_2(S^1) \to MT_2(\PP^2) \to MT_2(S^2) \to MT_1(S^1) \to MT_1(\PP^2) \to \cdots$$
which simplifies to a natural isomorphism $\varphi: H_2(S^2) = H_1(S^1)$.
However, both of those are isomporhic to $\ZZ$, and the $\varphi$ is modulo torsion of the map $MT_1(S^1) \to MT_1(\PP^2)$, so $\varphi = 2$, but $2$ is not an isomorphism.

\begin{exer}
Let $0 \to \pi \to \rho \to \sigma \to 0$ be a short exact sequence of abelian groups.
Let $C$ be a chain complex of torsion-free abelian groups. Construct a long exact sequence
$$\cdots \to H_q(C, \pi) \to H_q(C, \rho) \to H_q(C, \sigma) \to H_{q-1}(C, \pi) \to \cdots.$$
\end{exer}

By the zigzag lemma, it suffices to show that there is a short exact sequence of chain complexes
\begin{equation}
\label{sescc}
0 \to H(C, \pi) \to H(C, \rho) \to H(C, \sigma) \to 0.
\end{equation}
Since $C_n$ is torsion-free, it is a flat $\ZZ$-module (this follows from Stacks Lemma 15.22.11, since $\ZZ$ clearly is Dedekind), so the functor $C_n \otimes \cdot$ is exact.
That implies that, since
$$0 \to \pi \to \rho \to \sigma \to 0$$
is exact, so is
$$0 \to H_n(C, \pi) \to H_n(C, \rho) \to H_n(C, \sigma) \to 0.$$
Since changing coefficients commutes with the arrows $H_n(C) \to H_{n-1}(C)$, the above short exact sequence induces (\ref{sescc}).

\begin{exer}
Compute $H(S^n \setminus X)$ if $X = S^k \vee S^\ell$ or $X = S^k \amalg S^\ell$.
\end{exer}

We first note that
$$\tilde H_i(S^n \setminus S^k) \oplus \tilde H_i(S^n \setminus S^\ell) \cong \begin{cases}
\ZZ^2, &i = n - k - 1 = n - \ell - 1\\
\ZZ, &i = n - k - 1 \neq n - \ell - 1\\
\ZZ, &i = n - \ell - 1 \neq n - k - 1\\
0, &\text{else}.
\end{cases}$$
We denote that group by $G_{i,n,k,\ell}$.
This computation will be used throughout.

Let $A = S^n \setminus S^k$, $B = S^n \setminus S^\ell$. Then $A \cap B = X$; if $X = S^k \vee S^\ell$, $A \cup B = \RR^n$, and otherwise $A \cup B = S^n$.
If $i \neq 0$ then one has a Mayer-Vietoris sequence
\begin{equation}
\label{MV}
\cdots \to H_{i+1}(\RR^n) \to H_i(S^n \setminus (S^k \vee S^\ell)) \to G_{i,n,k,\ell} \to H_i(\RR^n) \to \cdots
\end{equation}
and since $\RR^n$ is contractible we obtain an isomorphism
$$H_i(S^n \setminus (S^k \vee S^\ell)) \cong G_{i,n,k,\ell}.$$
If in addition $i \leq n - 2$ then we obtain an isomorphism
$$H_i(S^n \setminus (S^k \amalg S^\ell)) \cong G_{i,n,k,\ell}$$
since then $H_i(S^n) = H_{i+1}(S^n) = 0$, and so the $\RR^n$'s in the Mayer-Vietoris sequence (\ref{MV}) can be replaced with $S^n$ without affecting the resulting computation.

Now we treat the case $i = 0$, $X = S^k \amalg S^\ell$.
If $n = 1$ then $k = \ell = 0$, and
$$H_0(S^1 \setminus (S^0 \amalg S^0)) = H_0(\RR \setminus \{0, 1, 2\}) \cong \ZZ^4.$$
Otherwise, assume $n \geq 2$.
Since $H_0(Y, R)$ is the free module over $R$ generated by $\pi_0(Y)$, if $H_0(Y, \QQ)$ has dimension $r$, then $H_0(Y) \cong \ZZ^r$.
Therefore we compute with coefficients in $\QQ$.
The Mayer-Vietoris sequence over $\QQ$ terminates as
$$\cdots \to H_1(S^n, \QQ) \to H_0(S^n \setminus (S^k \amalg S^\ell), \QQ) \to (G_{0,n,k,\ell} \oplus \ZZ^2) \otimes \QQ \to H_0(S^n, \QQ) \to 0.$$
Here we had to replace $G_{0, n, k, \ell}$ with $G_{0,n,k,\ell} \oplus \ZZ^2$ since $G_{0, n, k, \ell}$ was a reduced homology group.
Since $n \geq 2$, $\tilde H_1(S^n, \QQ) = \tilde H_0(S^n, \QQ) = 0$, so we get a short exact sequence of vector spaces over $\QQ$
\begin{equation}
\label{ses1}
0 \to H_0(S^n \setminus (S^k \amalg S^\ell), \QQ) \to (G_{0,n,k,\ell} \otimes \QQ) \oplus \QQ^2 \to \QQ \to 0.
\end{equation}
By a dimension-counting argument, (\ref{ses1}) induces an isomorphism $H_0(S^n \setminus (S^k \amalg S^\ell), \QQ) \cong (G_{0,n,k,\ell} \otimes \QQ) \oplus \QQ$ and hence one has $H_0(S^n \setminus (S^k \amalg S^\ell)) \cong G_{0,n,k,\ell} \oplus \ZZ$.

Now we treat the case $i = 0$, $X = S^k \vee S^\ell$.
The Mayer-Vietoris sequence over $\QQ$ terminates as
$$\cdots \to H_1(\RR^n, \QQ) \to H_0(S^n \setminus (S^k \vee S^\ell), \QQ) \to (G_{0,n,k,\ell} \oplus \ZZ) \otimes \QQ \to H_0(\RR^n, \QQ) \to 0.$$
This simplifies to the exact same short exact sequence (\ref{ses1}) of vector spaces as previously, so one again obtains $H_0(S^n \setminus (S^k \vee S^\ell)) \cong G_{0,n,k,\ell} \oplus \ZZ$.

Now we treat the case $i = n$, $X = S^k \amalg S^\ell$. The Mayer-Vietoris sequence begins
$$0 \to H_n(S^n \setminus (S^k \amalg S^\ell)) \to G_{n,n,k,\ell} \to \cdots$$
but $G_{n,n,k,\ell} = 0$ so $H_n(S^n \setminus (S^k \amalg S^\ell)) = 0$.

Finally we treat the case $i = n - 1$, $X = S^k \amalg S^\ell$. As in the previous case, $G_{n,n,k,\ell} = 0$; since in addition $H_{n-1}(S^n) = 0$ the Mayer-Vietoris sequence simplifies to a short exact sequence
\begin{equation}
\label{ses2}
0 \to H_n(S^n) \to H_{n-1}(S^n \setminus (S^k \amalg S^\ell)) \to G_{n-1,n,k,\ell} \to 0.
\end{equation}
Thus the problem now breaks up into cases based on $k,\ell$. If $k = \ell = 0$ then $G_{n-1,n,k,\ell} \cong \ZZ^2$, so (\ref{ses2}) fails to be helpful.
In this case, $S^n \setminus (S^0 \amalg S^0) \cong \RR^n \setminus (\RR^0)^{\amalg 3} \cong (S^{n-1})^{\vee 3}$.
Therefore $H_{n-1}(S^n \setminus (S^0 \amalg S^0)) \cong \ZZ^3$.
If $k = 0$ and $\ell \neq 0$, then, for similar reasons as above, we ignore (\ref{ses2}).
Then $S^n \setminus (S^0 \amalg S^\ell) \cong \RR^n \setminus (\RR^0 \amalg S^\ell)$.
We set up a new Mayer-Vietoris sequence with $A = \RR^n \setminus \RR^0$, $B = \RR^n \setminus S^\ell$, which reads
$$\cdots \to H_n(\RR^n) \to H_{n-1}(\RR \setminus (\RR^0 \amalg S^\ell)) \to H_{n-1}(\RR^n \setminus \RR^0) \oplus H_{n-1}(S^\ell) \to H_{n-1}(\RR^n) \to \cdots$$
and induces an isomorphism $H_{n-1}(\RR \setminus (\RR^0 \amalg S^\ell)) \cong H_{n-1}(S^{n-1}) \oplus H_{n-1}(S^\ell)$,
thus $H_{n-1}(\RR \setminus (\RR^0 \amalg S^{n-1})) \cong \ZZ^2$ and, if $\ell \neq n - 1$, $H_{n-1}(\RR \setminus (\RR^0 \amalg S^{n-1})) \cong \ZZ$.
Simplifying,
$$H_{n-1}(S^n \setminus (S^0 \amalg S^\ell)) \cong \begin{cases}
\ZZ^2, &\ell = n - 1\\
\ZZ, \text{ else}.
\end{cases}$$
Finally, if $k \neq 0$ and $\ell \neq 0$, then we can apply (\ref{ses2}) with $G_{n-1,n,k,\ell} = 0$ to conclude $H_{n-1}(S^n \setminus (S^k \amalg S^\ell)) \cong H_n(S^n) \cong \ZZ$.

Summarizing, $H_i(S^n \setminus (S^k \vee S^\ell))$ is a free abelian group of rank $r$, where
$$r = \begin{cases}
\begin{cases}
3, &n - k = n - \ell = 1\\
2, &n - k = 1 \text{ and } k \neq \ell\\
2, &n - \ell = 1 \text{ and } k \neq \ell\\
1, &\text{else}
\end{cases}, &i = 0\\
\begin{cases}
2, &i = n - k - 1 = n - \ell - 1\\
1, &i = n - k - 1 \text { and } k \neq \ell\\
1, &i = n - \ell - 1 \text { and } k \neq \ell\\
0, &\text{else}
\end{cases}, &\text{else}.
\end{cases}$$
Similarly, $H_i(S^n \setminus (S^k \amalg S^\ell))$ is a free abelian group of rank $s$, where
$$s = \begin{cases}
4, &i = 0 \text{ and } n = 1\\
\begin{cases}
3, &n - k = n - \ell = 1\\
2, &n - k = 1 \text{ and } k \neq \ell\\
2, &n - \ell = 1 \text{ and } k \neq \ell\\
1, &\text{else}
\end{cases}, &i = 0 \text{ and } n \geq 2\\
\begin{cases}
3, &k = \ell = 0\\
2, &k = 0 \text{ and } \ell = n - 1\\
2, &\ell = 0 \text{ and } k = n - 1\\
1, &\text{else}
\end{cases}, &i = n - 1\\
0, &i = n\\
\begin{cases}
2, &i = n - k - 1 = n - \ell - 1\\
1, &i = n - k - 1 \text { and } k \neq \ell\\
1, &i = n - \ell - 1 \text { and } k \neq \ell\\
0, &\text{else}
\end{cases}, &\text{else}.
\end{cases}$$

\begin{exer}
Show that $\RR^{2n+1}$ is not a division algebra over $\RR$ whenever $n > 0$.
\end{exer}

Let $m = 2n + 1$, so $m \geq 3$ is odd.
Assume that $\RR^{2n + 1}$ is an unital algebra, so it has $\pm 1$.
We will show that $\RR^{2n + 1}$ has a zero divisor.
Let $\gamma$ be a path from $1$ to $-1$ in $\RR^{2n + 1} \setminus 0$, and let $f(a)$ denote the determinant over $\RR$ of the action of $a \in \RR^{2n + 1}$ on $\RR^{2n + 1}$ by multiplication.
Then $g = f \circ \gamma: [0, 1] \to \RR$ has $g(0) = 1$ and $g(1) = -1$ so there is a $t \in [0, 1]$ with $g(t) = 0$.
Then $\gamma(t)$ has a kernel, so $\gamma(t)$ is a zero divisor.

\begin{exer}
Use the transfer sequence for the double cover $S^\infty \to \PP^\infty$ to compute $H(\PP^\infty, \FF_2)$.
Do the same to show
$$H_n(X \times \PP^\infty, \FF_2) \cong \bigoplus_{i \leq n} H_i(X, \FF_2).$$
\end{exer}

We need that $S^\infty$ is contractible.
This follows because $S^\infty$ is the direct limit of $S^n$ as $n \to \infty$ with respect to the equatorial maps $S^n \to S^{n+1}$.
One may then contract $S^n$ in $S^{n+1}$ in the usual way, and so reason inductively to conclude that $S^\infty$ is contractible.

Since $S^\infty$ is contractible, all transfer morphisms are $0$, and the transfer sequence reads
$$\cdots \to 0 \to H_n(\PP^\infty, \FF_2) \to H_{n-1}(\PP^\infty, \FF_2) \to 0 \to \cdots.$$
Thus by induction,
$$H_n(\PP^\infty, \FF_2) \cong H_0(\PP^\infty, \FF_2)$$
and since $S^\infty$ is connected, so is $\PP^\infty$, so
$$H_n(\PP^\infty, \FF_2) \cong \FF_2.$$

For $X \times \PP^\infty$, we again show that all transfer morphisms are $0$.
Let $\sigma$ be a simplex in $X \times \PP^\infty$ and let $p: X \times S^\infty \to X \times \PP^\infty$ be the covering space induced by the covering space $S^\infty \to \PP^\infty$.
Then $p^{-1}(\sigma)$ is the disjoint union of two simplices $\sigma_0,\sigma_1$ since $p$ is a double cover.
Since $S^\infty$ is contractible, one has a natural isomorphism $\psi: H(X \times S^\infty) \to H(X)$. Then
$$\psi(\tau([\sigma])) = \psi([\sigma_0] + [\sigma_1]) = 2\psi([\sigma_0]) = 0$$
so $\tau([\sigma]) = 0$. Therefore $\tau = 0$.

Since transfer morphisms are $0$ and $S^\infty$ is contractible, the transfer sequence
$$\cdots \to H_i(X \times \PP^\infty) \to H_i(X \times S^\infty) \to H_i(X \times \PP^\infty) \to H_{i-1}(X \times \PP^\infty) \to H_{i-1}(X \times S^\infty) \to \cdots$$
simplifies to a short exact sequence
$$0 \to H_i(X) \to H_i(X \times \PP^\infty) \to H_{i-1}(X \times \PP^\infty) \to 0.$$
Since the coefficient ring is a field, this short exact sequence splits unnaturally, inducing an isomorphism
$$H_i(X \times \PP^\infty) \cong H_i(X) \oplus H_{i-1}(X \times \PP^\infty).$$
The claim now follows by induction after one carries out the base case
$$H_0(X \times \PP^\infty) \cong H_0(X) \oplus H_{-1}(X \times \PP^\infty) = H_0(X) \oplus 0 = H_0(X).$$



\begin{exer}
Let $X$ be homotopy equivalent to a finite simplicial complex, $Y$ a countable simplicial complex.
Show that there are countably many homotopy classes of maps $X \to Y$.

Show that there are countably many homotopy types of finite CW complexes.
\end{exer}

For the first part: we can choose a simplicial representative from each homotopy class; so it suffices to show that there are countably many simplicial maps $X \to Y$.
Any simplicial map restricts to a map on vertices $X^0 \to Y^0$. Now $Y^0$ is countable, so simplicial maps can be identified with vectors in $(Y^0)^{X_0}$, which has cardinality
$$\card (Y^0)^{X_0} = (\card Y_0)^{\card X_0} \leq \aleph_0^{\card X_0} = \aleph_0$$
since $X_0$ is finite.

For the second part, we choose a simplicial representative from each homotopy type; this simplicial representative can always be chosen to be finite.
So it suffices to show that the set of simplicial complexes (up to ``simplicial isomorphism", that is, homeomorphism by simplicial map with simplicial inverse) is finite. Let $C_n$ denote the set of all finite $n$-dimensional simplicial complexes.
Since every finite simplicial complex is finite-dimensional, the set $C$ of finite simplicial complexes is $C = \bigcup_n C_n$.
Let us prove by induction that $C_n$ is countable; then $C$ will also be countable.

A $0$-dimensional simplicial complex is just a finite set of points; thus $C_0$ is in bijection with $\NN$, with a complex being mapped to its cardinality.

Suppose that $C_k$ is countable, and let $X \in C_k$. To obtain a simplicial complex $Y \in C_{k+1}$ with $Y^k = X$, one must, for each $k$-simplex $\sigma$ in $X$, specify which $k$-simplices are also faces of the same $k+1$-simplex as $\sigma$.
Let $\sigma_1, \dots, \sigma_n$ be the $k$-simplices of $X$, and let $\tau_1, \dots, \tau_m$ be the $k+1$-simplices of $Y$.
Then $\tau_i$ is characterized by a vector $(\tau_i^1, \dots, \tau_i^{k+2}) \in \NN^{k+2}$, since there are $k+2$ many $k+1$-faces of $\tau_i$, where $\sigma_{\tau_i^j}$ is the $j$th $k+1$-face of $\tau_i$.
So a $k+1$-dimensional simplicial complex $Y$ with $Y^k = X$ and $m$ many $k+1$-simplices can be described by an element of $\NN^{m(k+2)}$.
So a $k+1$-dimensional simplicial complex $Y$ with $Y^k = X$ can be described by an element of $\bigcup_m \NN^{m(k+2)}$.
So there is a surjection
$$C_{k+1} \to C_k \times \bigcup_m \NN^{m(k+2)},$$
so $C_{k+1}$ is countable.

\begin{exer}
Let $f \in \SL(2, \ZZ)$ satisfy $f(x, y) = (-x-y, x)$, so $f$ drops to a toral automorphism.
Find $f_\sharp: \pi_1(\TT^2) \to \pi_1(\TT^2)$ and $f_*: H_1(\TT^2) \to H_1(\TT^2)$.
Find $f_*: H_2(\TT^2) \to H_2(\TT^2)$.
Show that every self-map of $\TT^2$ homotopic to $f$ has a fixed point.
\end{exer}

Let $e,g$ be basis vectors for $\RR^2$.
Let $\alpha, \beta$ be generators for $\pi_1(\TT^2)$ where $\alpha$ is a loop in the $S^1$ in the direction of $e$, and $\beta$ in the direction of $g$.
Then $f_\sharp(\alpha) = \alpha^{-1}\beta^{-1}$ and $f_\sharp(\beta) = \alpha$..
Since $\pi_1(\TT^2)$ is already abelian, $f_*(\alpha) = -\alpha - \beta$ and $f_*(\beta) = \alpha$, viewing $\alpha,\beta$ now as generators of $H_1(\TT^2)$ by taking their images under the Hurcewiz morphism.

Since $f$ is a homeomorphism, $f_*$ is an automorphism in each degree.
Since $H_j(\TT^2) \cong \ZZ$, it follows that either $f_* = 1$ or $f_* = -1$ in degree $j$, whenever $j \in \{0, 2\}$.
I think there's a purely cellular way to prove that in fact $f_* = 1$ in degree $2$ but I'm lazy so I'll use results from the first-year manifolds course.
Since $\TT^2$ is a closed orientable manifold, fix an orientation form $\omega$; then its cohomology class $[\omega]$ generates the top de Rham cohomology $H^2(\TT^2, \RR)$.
Moreover, $f$ is orientation-preserving, so the pullback $f^*$ satisfies, for any top cell $\sigma$,
$$\int_\sigma \omega = \int_\sigma f^*\omega,$$
and $f^*[\omega] = [\omega]$.
That implies that $f^* = 1$ on $H^2(\TT^2, \RR)$. By de Rham's theorem, one obtains a natural isomorphism
\begin{align*}
H^2(\TT^2, \RR) &\to \Hom(H_2(\TT^2, \RR), \RR)\\
[\alpha] &\mapsto \left([\sigma] \mapsto \int_\sigma \alpha \right)
\end{align*}
where $\sigma$ is a cell such that $[\sigma]$ generates $H_2(\TT^2, \RR)$.
In particular, we obtain an isomorphism $\psi: H_2(\TT^2, \RR) \to \RR$ by
$$\psi([\sigma]) = \int_\sigma \omega.$$
Therefore
$$\psi([f_*\sigma]) = \int_{f_*\sigma} \omega = \int_\sigma f^* \omega = \int_\sigma \omega = \psi([\sigma])$$
but $\psi$ is an isomorphism, so $f_*[\sigma] = [f_*\sigma] = [\sigma]$.
Therefore $f_* = 1$ on $H_2(\TT^2, \RR) = H_2(\TT^2, \ZZ) \otimes \RR$, which is impossible if $f_* = -1$; so $f_* = 1$.

Now it's easy to compute the trace of $f_*|H_1$, which is $-1$. We showed that $f_*|H_2 = 1$ and $f_*|H_0 \in \{-1, 1\}$ so it follows that the Lefschetz number of $f$ is $\geq 1$.
That implies that elements of the homotopy class $[f]$ all have fixed points.




\end{document}
