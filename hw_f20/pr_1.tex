
% --------------------------------------------------------------
% This is all preamble stuff that you don't have to worry about.
% Head down to where it says "Start here"
% --------------------------------------------------------------

\documentclass[10pt]{article}

\usepackage[margin=.7in]{geometry}
\usepackage{amsmath,amsthm,amssymb}
\usepackage{enumitem}
\usepackage{tikz-cd}
\usepackage{mathtools}
\usepackage{amsfonts}
\usepackage{listings}
\usepackage{algorithm2e}
\usepackage{verse,stmaryrd}
\usepackage{fancyvrb}

% Number systems
\newcommand{\NN}{\mathbb{N}}
\newcommand{\ZZ}{\mathbb{Z}}
\newcommand{\QQ}{\mathbb{Q}}
\newcommand{\RR}{\mathbb{R}}
\newcommand{\CC}{\mathbb{C}}
\newcommand{\PP}{\mathbb P}
\newcommand{\FF}{\mathbb F}
\newcommand{\DD}{\mathbb D}
\renewcommand{\epsilon}{\varepsilon}

\newcommand{\Aut}{\operatorname{Aut}}
\newcommand{\cl}{\operatorname{cl}}
\newcommand{\ch}{\operatorname{ch}}
\newcommand{\Con}{\operatorname{Con}}
\newcommand{\coker}{\operatorname{coker}}
\newcommand{\CVect}{\CC\operatorname{-Vect}}
\newcommand{\Cantor}{\mathcal{C}}
\newcommand{\D}{\mathcal{D}}
\newcommand{\card}{\operatorname{card}}
\newcommand{\dbar}{\overline \partial}
\newcommand{\diam}{\operatorname{diam}}
\newcommand{\End}{\operatorname{End}}
\DeclareMathOperator*{\esssup}{ess\,sup}
\newcommand{\GL}{\operatorname{GL}}
\newcommand{\Hom}{\operatorname{Hom}}
\newcommand{\id}{\operatorname{id}}
\newcommand{\Ind}{\operatorname{Ind}}
\newcommand{\Inn}{\operatorname{Inn}}
\newcommand{\interior}{\operatorname{int}}
\newcommand{\lcm}{\operatorname{lcm}}
\newcommand{\mesh}{\operatorname{mesh}}
\newcommand{\LL}{\mathcal L_0}
\newcommand{\Leb}{\mathcal{L}_{\text{loc}}^2}
\newcommand{\Lip}{\operatorname{Lip}}
\newcommand{\ppGL}{\operatorname{PGL}}
\newcommand{\ppic}{\vspace{35mm}}
\newcommand{\ppset}{\mathcal{P}}
\DeclareMathOperator*{\Res}{Res}
\newcommand{\Riem}{\mathcal{R}}
\newcommand{\RVect}{\RR\operatorname{-Vect}}
\newcommand{\Sch}{\mathcal{S}}
\newcommand{\SL}{\operatorname{SL}}
\newcommand{\sgn}{\operatorname{sgn}}
\newcommand{\Spec}{\operatorname{Spec}}
\newcommand{\supp}{\operatorname{supp}}
\newcommand{\TT}{\mathcal T}
\DeclareMathOperator{\tr}{tr}

% Calculus of variations
\DeclareMathOperator{\pp}{\mathbf p}
\DeclareMathOperator{\zz}{\mathbf z}
\DeclareMathOperator{\uu}{\mathbf u}
\DeclareMathOperator{\vv}{\mathbf v}
\DeclareMathOperator{\ww}{\mathbf w}

% Categories
\newcommand{\Ab}{\mathbf{Ab}}
\newcommand{\Cat}{\mathbf{Cat}}
\newcommand{\Group}{\mathbf{Group}}
\newcommand{\Module}{\mathbf{Module}}
\newcommand{\Set}{\mathbf{Set}}
\DeclareMathOperator{\Fun}{Fun}
\DeclareMathOperator{\Iso}{Iso}

% Complex analysis
\renewcommand{\Re}{\operatorname{Re}}
\renewcommand{\Im}{\operatorname{Im}}

% Logic
\renewcommand{\iff}{\leftrightarrow}
\newcommand{\Henkin}{\operatorname{Henk}}
\newcommand{\PA}{\mathbf{PA}}
\DeclareMathOperator{\proves}{\vdash}

% Group
\DeclareMathOperator{\Gal}{Gal}
\DeclareMathOperator{\Fix}{Fix}
\DeclareMathOperator{\Out}{Out}

% Other symbols
\newcommand{\heart}{\ensuremath\heartsuit}

\DeclareMathOperator{\atanh}{atanh}

% Theorems
\theoremstyle{definition}
\newtheorem*{corollary}{Corollary}
\newtheorem*{falselemma}{Grader's ``Lemma"}
\newtheorem{exer}{Exercise}
\newtheorem{lemma}{Lemma}[exer]
\newtheorem{theorem}[lemma]{Theorem}


\usepackage[backend=bibtex,style=alphabetic,maxcitenames=50,maxnames=50]{biblatex}
\renewbibmacro{in:}{}
\DeclareFieldFormat{pages}{#1}

\begin{document}
\noindent
\large\textbf{Probability, HW 1} \hfill \textbf{Aidan Backus} \\

% --------------------------------------------------------------
%                         Start here
% --------------------------------------------------------------\

\begin{exer}[1a]
$\mu(E) + \mu(F) = \mu(E \cup F) + \mu(E \cap F)$.
\end{exer}
First note that $E = (E \cap F) \cup (E \setminus F)$, a disjoint union. Similarly for $F$. So
$$\mu(E) + \mu(F) = \mu(E \setminus F) + \mu(F \setminus E) + 2 \mu(E \cap F).$$
But $(E \setminus F) \cup (F \setminus E) \cup (E \cap F) = E \cup F$, the left-hand side a disjoint union. Thus
$$\mu(E \cup F) = \mu(E \setminus F) + \mu(F \setminus E) + \mu(E \cap F).$$
This implies
$$\mu(E) + \mu(F) = \mu(E \cup F) + \mu(E \cap F)$$
which was desired.

\begin{exer}[1b]
If $\mu_A(E) = \mu(A \cap E)$ then $\mu_A$ is a measure.
\end{exer}

First, $\mu_A(\emptyset) = \mu(\emptyset) = 0$. Second, if $(E_n)_n$ is a disjoint sequence, then
$$\mu_A\left(\bigcup_n E_n\right) = \mu\left(A \cap \bigcup_n E_n\right) = \mu\left(\bigcup_n A \cap E_n\right) = \sum_n \mu(A \cap E_n) = \sum_n \mu_A(E_n)$$
since the $(A \cap E_n)_n$ form a disjoint sequence.

\begin{exer}[1ci]
$\mu(\liminf_n E_n) \leq \liminf_n \mu(E_n)$.
\end{exer}

If $\liminf_n \mu(E_n)$ is infinite then there is nothing to prove, so suppose that $\liminf_n \mu(E_n) < \infty$; then there is a subsequence $E_{k_n}$ such that $\lim_n \mu(E_{k_n})$ exists and is finite.

Now let
$$F_n = \bigcap_{k \geq n} E_k.$$
Then $\liminf_n E_n = \bigcup_n F_n$ and $F_n \subseteq F_{n+1}$. By measure continuity,
$$\mu\left(\liminf_n E_n\right) = \mu\left(\bigcup_n F_n\right) = \lim_n \mu(F_n).$$
In particular, for any subsequence $F_{j_n}$ such that $\lim_n \mu(E_{j_n})$ exists,
$$\mu\left(\liminf_n E_n\right) = \lim_n \mu(F_{j_n}) \leq \lim_n \mu(E_{j_n}).$$
Minimizing over all such subsequences, we see
$$\mu\left(\liminf_n E_n\right) \leq \liminf_n \mu(E_n).$$

\begin{exer}[1cii]
If there is an $n$ such that $\mu(\bigcup_{k \geq n} E_n)$ is finite, then
$\mu(\limsup_n E_n) \geq \limsup_n \mu(E_n)$.
\end{exer}

Let
$$F_n = \bigcup_{k \geq n} E_k.$$
Then $\limsup_n E_n = \bigcap_n F_n$ and $F_{n+1} \subseteq F_n$. By hypothesis, measure continuity is applicable, thus
$$\mu\left(\limsup_n E_n\right) = \mu\left(\bigcap_n F_n\right) = \lim_n \mu(F_n).$$
The same argument as 1ci (this time \emph{maximizing} over subsequences) now applies.

\begin{exer}[1ciii - Borel-Cantelli]
If $\sum_n \mu(E_n)$ is finite then $\mu(\limsup_n E_n) = 0$.
\end{exer}

For any $N$,
$$\mu\left(\limsup_n E_n\right) \leq \mu\left(\bigcup_{n \geq N} E_n\right) \leq \sum_{n \geq N} \mu(E_n).$$
If $\sum_n \mu(E_n)$ is finite then the tails $\sum_{n \geq N} \mu(E_n)$ vanish as $N \to \infty$. So $\mu(\limsup_n E_n) = 0$.

\begin{exer}[2a]
If $\mu(E \Delta F) = 0$ then $\mu(E) = \mu(F)$.
\end{exer}

One has
$$\mu(E) = \mu(E \cap F) + \mu(E \setminus F) \leq \mu(F) + \mu(E \Delta F) = \mu(F).$$
By a symmetric argument one has $\mu(F) \leq \mu(E)$.

\begin{exer}[2b]
Let $E \sim F$ mean that $\mu(E \Delta F) = 0$. Then $\sim$ is an equivalence relation.
\end{exer}

$E \Delta E$ is empty, proving symmetry. Moreover, $\Delta$ is commutative, proving reflection. Finally, since $\Delta$ is associative, transitivity follows.

\begin{exer}[2c]
Let $\rho(E, F) = \mu(E \Delta F)$. Show that $\rho$ is a metric on the space of $\sim$-equivalence classes.
\end{exer}

We first check that $\rho$ is a semimetric on the $\sigma$-algebra; then it clearly drops to a metric once we identify equivalent sets.
Since $\Delta$ is commutative, $\rho$ is symmetric, and since $\mu$ is nonnegative, so is $\rho$.
So to see that $\rho$ is a semimetric it suffices to check the triangle inequality, and that follows from the set-theoretic formula
$$E \Delta G \subseteq (E \Delta F) \cup (F \Delta G).$$
This can be seen by letting $x \in E \Delta G$. If $x \in E \setminus G$ then either $x \in F$, so $x \in F \setminus G$, or $x \notin F$, so $x \in E \setminus F$. Therefore $x \in (E \Delta F) \cup (F \Delta G)$. A symmetric argument shows $G \setminus E \subseteq (E \Delta F) \cup (F \Delta G)$.

\begin{exer}[3]
If $\mu_1, \dots, \mu_n$ are measures and $\alpha_1, \dots, \alpha_n \geq 0$, then $\sum_i \alpha_i \mu_i$ is a measure.
\end{exer}

Let $\mu = \sum_i \alpha_i \mu_i$. By linearity $\mu(\emptyset) = 0$. Let $(E_k)_k$ be a disjoint sequence of sets; then
$$\mu\left(\bigcup_k E_k\right) = \sum_i \alpha_i \mu_i\left(\bigcup_k E_k\right) = \sum_i \alpha_i \sum_k \mu_i(E_k) = \sum_k \sum_i \alpha_i \mu_i(E_k) = \sum_k \mu(E_k).$$
Here we commuted the sums using the fact that the sum over $i$ only had finitely many terms.

\begin{exer}[4]
Let $f: \Omega \to [0, \infty)$, and define
$$\mu(E) = \sum_{\omega \in E} f(\omega).$$
Let $A$ be the support of $f$. Then if $A$ is uncountable, then $\mu(A) =\mu(\Omega) = \infty$. Moreover, $\mu$ is a measure, and $\mu$ is $\sigma$-finite iff $A$ is countable.
\end{exer}

We first suppose that $A$ is uncountable. By the infinite pigeonhole principle, there is an $n$ such that $S = \{f \geq 1/n\}$ is infinite (since there are only countably many $n$), so
$$\mu(\Omega) \geq \mu(A) = \sum_{\omega \in A} f(\omega) \geq \sum_{\omega \in S} f(\omega) \geq \frac{\card S}{n} = \infty.$$

Now we check that $\mu$ is a measure. In fact, if $(E_n)_n$ is a disjoint sequence whose union is $E$, then
$$\mu(E) = \sum_{\omega \in E} f(\omega) = \sum_n \sum_{\omega \in E_n} f(\omega) = \sum_n \mu(E_n).$$
We are entitled to split up the sum since all summands are nonnegative (and we may assume that the set of postive summands is countable, by another appeal to the infinite pigeonhole principle), and since for each $\omega \in E$ there is a unique $n$ such that $\omega \in E_n$. Besides, $\mu(\emptyset)$ is the empty sum, which is $0$.

Now suppose that $\mu$ is $\sigma$-finite, so there are $E_n \subseteq E_{n+1}$ with $\mu(E_n) < \infty$ and $\bigcup_n E_n = \Omega$.
Then
$$\mu(E_n \cap A) = \sum_{\omega \in E_n \cap A} f(\omega) = \sum_{\omega \in E_n \cap A} f(\omega) + \sum_{\omega \in E_n \setminus A} f(\omega) = \sum_{\omega \in E_n} f(\omega) = \mu(E_n) < \infty$$
since the sum over $E_n \setminus A$ is a sum over zeroes.
By an argument similar to the appeal to the infinite pigeonhole principle above, we see that $E_n \cap A$ is countable, and since $\bigcup_n E_n = \Omega$, $\bigcup_n E_n \cap A = A$.
Therefore $A$ is countable.

Conversely, if $A$ is countable, then choose an enumeration $\{\omega_n\}_n$ of $A$ and let
$$E_n = (\Omega \setminus A) \cup \{\omega_1, \dots, \omega_n\}.$$
Then
$$\mu(E_n) = \mu((\Omega \setminus A)) + \mu(\{\omega_1, \dots, \omega_n\}) = \sum_{i=1}^n f(\omega_i) < \infty.$$
Meanwhile $\bigcup_n E_n = \Omega$, so $\Omega$ is $\sigma$-finite.

\begin{exer}[5]
Let $\mu$ be a finitely additive function. Then TFAE: $\mu$ is a measure, $\mu$ is continuous from below, and $\mu$ is continuous from above at $\emptyset$ (provided that $\mu$ is finite).
\end{exer}

Suppose that $\mu$ is a measure. We first check that $\mu$ is continuous from below. Indeed, if $(E_n)_n$ is an increasing sequence, let $F_n = E_n \setminus E_{n-1}$, $E_0 = \emptyset$. Then the $F_n$ are a disjoint sequence with $\bigcup_n F_n = \bigcup_n E_n$. So
$$\mu\left(\bigcup_n E_n\right) = \mu\left(\bigcup_n F_n\right) = \mu\left(\bigcup_n E_n \setminus E_{n-1}\right) = \sum_n \mu(E_n) - \mu(E_{n-1}).$$
Telescoping the series we see that $\mu(\bigcup_n E_n) = \lim_n \mu(E_n)$. Applying de Morgan's law and using the finiteness hypothesis to prevent an appearance of $\infty - \infty$ we see that $\mu$ is continuous from above at $\emptyset$.

Now suppose that $\mu$ is continuous from below; we must show that $\mu$ is countably additive.
Let $(E_n)_n$ be a disjoint sequence and let $F_n = \bigcup_{j \leq n} E_j$. The union defining $F_n$ is finite so
$$\mu(F_n) = \sum_{j \leq n} \mu(E_j).$$
Moreover $\bigcup_n F_n = \bigcup_n E_n$ and by measure continuity,
$$\mu\left(\bigcup_n E_n\right) = \lim_n \mu(F_n) = \lim_n \sum_{j \leq n} \mu(E_j) = \sum_n \mu(E_n).$$
Therefore $\mu$ is a measure.

Finally suppose that $\mu$ is continuous from above at $0$. We first show that $\mu$ is continuous from above; then that $\mu$ is continuous from below; then we may conclude from the above paragraph that $\mu$ is a measure.

Since $\mu$ is continuous from above at $0$, suppose that $E_{n+1} \subseteq E_n$, let $E = \bigcup_n E_n$, and let $F_n = E_n \setminus E$. Then $\bigcap_n F_n = \emptyset$ and $E_n = E \cup F_n$, so by continuity from above at $0$,
$$\lim_n \mu(E_n) = \lim_n \mu(E) + \mu(F_n) = \mu(E) + \lim_n \mu(F_n) = \mu(E).$$
Therefore $\mu$ is continuous from above.

Now to see that $\mu$ is continuous from below and complete the proof, suppose that $F_n \subseteq F_{n+1}$; we claim that $\lim_n \mu(F_n) = \mu(\bigcup_n F_n)$.
Let $E_n = \Omega \setminus F_n$, and recall that $\mu(\Omega) < \infty$. Thus
$$\mu\left(\bigcup_n F_n \right) = \mu\left(\bigcup_n \Omega \setminus E_n \right) = \mu(\Omega) - \mu\left(\bigcap_n E_n\right) = \mu(\Omega) - \lim_n \mu(E_n).$$
Therefore
$$\mu\left(\bigcup_n F_n \right) = \lim_n \mu(\Omega) - \mu(E_n) = \lim_n \mu(\Omega \setminus E_n) = \lim_n \mu(F_n).$$

\begin{exer}[6]
Let $E \subseteq \NN$, and let $\gamma_n(E)$ be the cardinality of $E \cap \{1, \dots, n\}$. Let $\mathcal D$ be the set of $E$ such that
$$\nu(E) = \lim_n \frac{\gamma_n(E)}{n}$$
exists. Show that $\nu$ is finitely additive on $\mathcal D$ but not countably so.
\end{exer}

To see finite additivity, let $E_1, \dots, E_k$ be disjoint with union $E$. Then
$$\gamma_n(E) = \card(E \cap \{1, \dots, n\}) = \card\left(\bigcup_j E_j \cap \{1, \dots, n\}\right) = \sum_j \card(E_j \cap \{1, \dots, n\}) = \sum_j \gamma_n(E_j).$$
So
$$\nu(E) = \lim_n \frac{\gamma_n(E)}{n} = \lim_n \frac{1}{n} \sum_j \gamma_n(E_j) = \sum_j \lim_n \frac{\gamma_n(E_j)}{n} = \sum_j \nu(E_j).$$

Now to see that $\nu$ is not countably additive, let $E_n = \{n\}$; then $\bigcup_n E_n = \NN$ and $\nu(E_n) = 0$. However, $\nu(\NN) = 1$.

\end{document}
