
% --------------------------------------------------------------
% This is all preamble stuff that you don't have to worry about.
% Head down to where it says "Start here"
% --------------------------------------------------------------

\documentclass[10pt]{article}

\usepackage[margin=.7in]{geometry}
\usepackage{amsmath,amsthm,amssymb}
\usepackage{enumitem}
\usepackage{tikz-cd}
\usepackage{mathtools}
\usepackage{amsfonts}
\usepackage{listings}
\usepackage{algorithm2e}
\usepackage{verse,stmaryrd}
\usepackage{fancyvrb}

% Number systems
\newcommand{\NN}{\mathbb{N}}
\newcommand{\ZZ}{\mathbb{Z}}
\newcommand{\QQ}{\mathbb{Q}}
\newcommand{\RR}{\mathbb{R}}
\newcommand{\CC}{\mathbb{C}}
\newcommand{\PP}{\mathbb P}
\newcommand{\FF}{\mathbb F}
\newcommand{\DD}{\mathbb D}
\renewcommand{\epsilon}{\varepsilon}

\newcommand{\Aut}{\operatorname{Aut}}
\newcommand{\cl}{\operatorname{cl}}
\newcommand{\ch}{\operatorname{ch}}
\newcommand{\Con}{\operatorname{Con}}
\newcommand{\coker}{\operatorname{coker}}
\newcommand{\CVect}{\CC\operatorname{-Vect}}
\newcommand{\Cantor}{\mathcal{C}}
\newcommand{\D}{\mathcal{D}}
\newcommand{\card}{\operatorname{card}}
\newcommand{\dbar}{\overline \partial}
\newcommand{\diam}{\operatorname{diam}}
\newcommand{\dom}{\operatorname{dom}}
\newcommand{\End}{\operatorname{End}}
\DeclareMathOperator*{\esssup}{ess\,sup}
\newcommand{\GL}{\operatorname{GL}}
\newcommand{\Hom}{\operatorname{Hom}}
\newcommand{\id}{\operatorname{id}}
\newcommand{\Ind}{\operatorname{Ind}}
\newcommand{\Inn}{\operatorname{Inn}}
\newcommand{\interior}{\operatorname{int}}
\newcommand{\lcm}{\operatorname{lcm}}
\newcommand{\mesh}{\operatorname{mesh}}
\newcommand{\LL}{\mathcal L_0}
\newcommand{\Leb}{\mathcal{L}_{\text{loc}}^2}
\newcommand{\Lip}{\operatorname{Lip}}
\newcommand{\ppGL}{\operatorname{PGL}}
\newcommand{\ppic}{\vspace{35mm}}
\newcommand{\ppset}{\mathcal{P}}
\DeclareMathOperator{\proj}{proj}
\DeclareMathOperator*{\Res}{Res}
\newcommand{\Riem}{\mathcal{R}}
\newcommand{\RVect}{\RR\operatorname{-Vect}}
\newcommand{\Sch}{\mathcal{S}}
\newcommand{\SL}{\operatorname{SL}}
\newcommand{\sgn}{\operatorname{sgn}}
\newcommand{\spn}{\operatorname{span}}
\newcommand{\Spec}{\operatorname{Spec}}
\newcommand{\supp}{\operatorname{supp}}
\newcommand{\TT}{\mathcal T}
\DeclareMathOperator{\tr}{tr}

% Calculus of variations
\DeclareMathOperator{\pp}{\mathbf p}
\DeclareMathOperator{\zz}{\mathbf z}
\DeclareMathOperator{\uu}{\mathbf u}
\DeclareMathOperator{\vv}{\mathbf v}
\DeclareMathOperator{\ww}{\mathbf w}

% Categories
\newcommand{\Ab}{\mathbf{Ab}}
\newcommand{\Cat}{\mathbf{Cat}}
\newcommand{\Group}{\mathbf{Group}}
\newcommand{\Module}{\mathbf{Module}}
\newcommand{\Set}{\mathbf{Set}}
\DeclareMathOperator{\Fun}{Fun}
\DeclareMathOperator{\Iso}{Iso}

% Complex analysis
\renewcommand{\Re}{\operatorname{Re}}
\renewcommand{\Im}{\operatorname{Im}}

% Logic
\renewcommand{\iff}{\leftrightarrow}
\newcommand{\Henkin}{\operatorname{Henk}}
\newcommand{\PA}{\mathbf{PA}}
\DeclareMathOperator{\proves}{\vdash}

% Group
\DeclareMathOperator{\Gal}{Gal}
\DeclareMathOperator{\Fix}{Fix}
\DeclareMathOperator{\Out}{Out}

% Other symbols
\newcommand{\heart}{\ensuremath\heartsuit}
\newcommand{\club}{\ensuremath\clubsuit}

\DeclareMathOperator{\atanh}{atanh}

% Theorems
\theoremstyle{definition}
\newtheorem*{corollary}{Corollary}
\newtheorem*{falselemma}{Grader's ``Lemma"}
\newtheorem{exer}{Exercise}
\newtheorem{lemma}{Lemma}[exer]
\newtheorem{theorem}[lemma]{Theorem}


\usepackage[backend=bibtex,style=alphabetic,maxcitenames=50,maxnames=50]{biblatex}
\renewbibmacro{in:}{}
\DeclareFieldFormat{pages}{#1}

\begin{document}
\noindent
\large\textbf{Smooth manifolds, HW 3} \hfill \textbf{Aidan Backus} \\

% --------------------------------------------------------------
%                         Start here
% --------------------------------------------------------------\

\begin{exer}[Lee 3.3]
Show that $T(M \times N) = TM \times TN$.
\end{exer}

Let $d_1 = \dim M$ and $d_2 = \dim N$.
Choose atlases $\mathcal U$ and $\mathcal V$ of $M$ and $N$; then $\mathcal U \otimes \mathcal V = \{U \times V: (U, V) \in \mathcal U \times \mathcal V\}$ is an atlas of $M \times N$.
Therefore $\mathcal U \otimes \mathcal V$ extends to an atlas of $T(M \times N)$, namely
$$\{T(U \times V): U \times V \in \mathcal U \otimes \mathcal V\}.$$
Thus $T(U \times V)$ is diffeomorphic to $U \times V \times \RR^{d_1} \times \RR^{d_2}$ and hence to $(U \times \RR^{d_1}) \times (V \times \RR^{d_2})$, which is diffeomorphic to $TU \times TV$.
Let
$$(\varphi, \psi): T(U \times V) \to TU \times TV$$
be the relevant diffeomorphism, thus $\varphi$ is the projection of $T(U \times V)$ onto $TU$ and similarly for $\psi$.

If $(\varphi_i, \psi_i): T(U_i \times V_i) \to TU_i \to TV_i$ are such diffeomorphisms, and $A = T(U_1 \times V_1) \cap T(U_2 \times V_2)$ is nonempty, we must show that $\varphi_1|A = \varphi_2|A$ and similarly for $\psi_i$, so that $(\varphi, \psi)$ can be extended to a diffeomorphism
$$(\varphi, \psi): T(M \times N) \to TM \times TN.$$
We just check this for $\varphi$. Let $(x, y, v) \in T(U \times V)$, where $x \in M$, $y \in N$, and $v$ is a tangent vector.
Then $\varphi_i(x, y, v)$ is independent of $y$ and is a tangent vector $w$ to $x$.
The fact that $w$ does not depend on $i$ follows from the fact that the $\varphi_i$ come from the same atlas and so are smoothly compatible, so they make the same identification of $T_xU$ with $\RR^{d_1}$.

\begin{exer}[Lee 3.8]
Let $M$ be a manifold with boundary.
Let $V_pM$ denote the space of equivalence classes of smooth curves starting at $p$, where $\gamma_1 \sim \gamma_2$ means that for every smooth germ $f$ at $p$, $(f \circ \gamma_1)'(0) = (f \circ \gamma_2)'(0)$.
Show that $\Psi: V_pM \to T_pM$, $\Psi([\gamma]) = \gamma'(0)$ is a well-defined bijection.
\end{exer}

Let $U$ be a chart near $p$. If $\gamma$ is a curve from $p$, we can replace $\gamma$ with a shorter curve contained in $U$ by truncating it at a point $q \neq p$ such that $q \in \gamma([0, 1]) \cap U$ without affecting, for any smooth germ $f$, $(f \circ \gamma)'(0)$, so it suffices to check curves contained in $U$.
Moreover, $(f \circ \gamma)'(0)$ is a diffeomorphism invariant, since $f$ would be pulled back by a diffeomorphism $\psi$ to $f \circ \psi^{-1}$ while $\gamma$ would push forward to $\psi \circ \gamma$, and
$$f \circ \psi^{-1} \circ \psi \circ \gamma = f \circ \gamma.$$
So we might as well assume that $p$ is contained in an open submanifold of $\RR^d$.

We now show that $\Psi([\gamma])$ is well-defined.
Assume that $p$ is contained in an open submanifold of $\RR^d$.
The projection $\proj_i$ onto the $i$th factor $\RR^d$ is a smooth germ, so $(\proj_i \circ \gamma)'(0)$ is completely determined by $[\gamma]$ and not its representative $\gamma$, so that
$$\Psi([\gamma]) = \gamma'(0) = ((\proj_1 \circ \gamma)'(0), \dots, (\proj_d \circ \gamma)'(0))$$
is well-defined.

Now we show that $\Psi$ is injective. Assume that $\gamma'(0) = \kappa'(0)$. Then for any smooth germ $f$,
$$(f \circ \gamma)'(0) = df_p(\gamma'(0)) = df_p(\kappa'(0)) = (f \circ \kappa'(0)).$$
Therefore $\gamma \sim \kappa$.

Finally we show that $\Psi$ is surjective. Let $v$ be a tangent vector at $p$, and assume that $p$ is contained in an open submanifold $U$ of $\RR^d$.
Then we can view $v$ as a point in $\RR^d$, and $p + tv$ is well-defined and an element of $U$ for every $t \in [0, \delta]$ if $\delta > 0$ is small enough.
Let $\gamma(t) = p + tv$. Then $\gamma'(0) = v$.

\begin{exer}[Lee 4.5]
Show that the quotient map $\pi: \CC^{n+1}\setminus 0 \to \PP^n_\CC$ is a surjective submersion and $\PP_\CC^1$ is diffeomorphic to $S^2$.
\end{exer}

Since $\pi$ is a quotient map, $\pi$ is surjective.
To see that $\pi$ is smooth, we recall the charts
$$U_k = \{\pi(z_1, \dots, z_{n+1}) \in \PP_\CC^n: z_k \neq 0\}$$
that we already defined for $\PP^n_\CC$, where
$$\varphi_k \circ \pi(z_1, \dots, z_{n+1}) = \left(\frac{z_1}{z_k}, \dots, \frac{z_{k-1}}{z_k}, \frac{z_{k+1}}{z_k}, \dots, \frac{z_{n+1}}{z_k}\right).$$
We let $V_k = \pi^{-1}(U_k)$; then $V_k$ is a chart for $\CC^{n+1}$ via the identity map, and $\pi|V_k \to U_k$ is smooth since
$$\pi \circ \varphi_k|V_k \to \CC^n$$
is rational with no poles and hence smooth.

Now to see that $\pi$ is a submersion, we compute $d\pi$ in coordinates. Let $\ell' = \ell$ if $\ell < k$ and $\ell' = \ell + 1$ otherwise (thus $j \neq k$); then, as a complex linear map,
$$(d(\varphi_k \circ \pi)_p)_{j\ell} = \frac{\partial}{\partial z_j} (\varphi_k)_\ell(\pi(z_1, \dots, z_{n+1})) = \frac{\partial}{\partial z_j} \frac{z_{\ell'}}{z_k} = \begin{cases}
1/z_k, & j = \ell'\\
-z_{\ell'}/z_k^2, & j = k\\
0, & \text{else}
\end{cases}.$$
The rows of $d(\varphi_k \circ \pi)_p$ range over $\ell$; we claim that they are linearly independent over $\CC$, which implies that $d(\varphi_k \circ \pi)_p$ and hence $d\pi_p$ is surjective as a map of \emph{complex} vector spaces (but that implies that it is surjective on the level of sets, thus also the realization is surjective).
The $\ell$th row vector has two nonzero entries, namely $1/z_k$ in the $\ell'$th entry and $-z_\ell/z_k^2$ in the $k$th entry.
Therefore the $\ell$th row vector with its $k$th entry removed $a_\ell$ satisfies $a_{\ell,\ell} = 1/z_k$ and $a_{\ell,m}$ if $m \neq \ell$.
Then $A = (a_{\ell,m})_{\ell,m}$ is a multiple of the identity and hence is surjective. But $A$ is just $d(\varphi_k \circ \pi)$ with one column removed, so $d(\varphi_k \circ \pi)$ is also surjective.


\begin{exer}[Lee 4.6]
Let $M$ be a compact manifold. Show that there is no submersion $M \to \RR^k$, $k \geq 1$.
\end{exer}

It will be convenient to show the following stronger statement.

\begin{theorem}
Let $M$ be a compact manifold and $N$ a connected manifold.
If there is a submersion $\pi: M \to N$, then $N$ is compact.
\end{theorem}

In particular, this shows that if there is a submersion $M \to \RR^k$, $\RR^k$ is compact, so $k = 0$.
We will need the following lemma:

\begin{lemma}
Let $M$, $N$ be manifolds and $\pi: M \to N$ be a submersion.
Then $\pi$ is open.
\end{lemma}
\begin{proof}
We first show that the situation is local.
Assume that $\pi$ is locally open in the sense that for every $y \in N$ there is an open set $V_y$ such that $\pi|\pi^{-1}(V_y)$ is open.
Let $U \subseteq M$ be open. Then
$$U = \bigcup_{y \in \pi(U)} U \cap \pi^{-1}(V_y)$$
and, since $\pi$ is continuous, $U \cap \pi^{-1}(V_y)$ is open, so $\pi(U \cap \pi^{-1}(V_y))$ is open.
But then
$$\pi(U) = \bigcup_{y \in \pi(U)} \pi(U \cap \pi^{-1}(V_y))$$
is a union of open sets, so open. Therefore $\pi$ is open.

So it suffices to show that $\pi$ is locally open; thus we can restrict to a chart $V \subseteq N$ such that $\pi^{-1}(V)$ is contained in a chart $U$ of $M$.
Then we may replace $U, V$ with open submanifolds of $\RR^{d_i}$ for appropriate $d_i$.

According to the rank theorem, possibly after replacing $U, V$ with open subsets thereof, there are diffeomorphisms $\alpha: U \to \tilde U$ and $\beta: V \to \tilde V$ such that the diagram
$$\begin{tikzcd}
U \arrow[r, "\pi|U"] \arrow[d,"\alpha"]& V \arrow[d,"\beta"]\\
\tilde U \arrow[r,"\proj_{d_2}"] &\tilde V
\end{tikzcd}
$$
commutes, where $\proj_{d_2}$ is projection onto the first $d_2$ coordinates, $\tilde U \subseteq \RR^{d_1}$, and $\tilde V \subseteq \RR^{d_2}$.
But $\proj_{d_2}$ is clearly open, so $\pi|U$ is as well.
Since $U$ was an arbitrary chart, it follows that $\pi$ is locally open, which was to be shown.
\end{proof}

Since $M$ is compact and $N$ is Hausdorff, $\pi$ is closed.
So $\pi$ is clopen, and since $M$ is clopen, $\pi(M)$ is clopen.
Since $N$ is connected and $M$ is nonempty, $\pi(M) = N$.
That implies that $N$ is compact.




\end{document}
