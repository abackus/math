
% --------------------------------------------------------------
% This is all preamble stuff that you don't have to worry about.
% Head down to where it says "Start here"
% --------------------------------------------------------------

\documentclass[10pt]{article}

\usepackage[margin=.7in]{geometry}
\usepackage{amsmath,amsthm,amssymb}
\usepackage{enumitem}
\usepackage{tikz-cd}
\usepackage{mathtools}
\usepackage{amsfonts}
\usepackage{listings}
\usepackage{algorithm2e}
\usepackage{verse,stmaryrd}
\usepackage{fancyvrb}

% Number systems
\newcommand{\NN}{\mathbb{N}}
\newcommand{\ZZ}{\mathbb{Z}}
\newcommand{\QQ}{\mathbb{Q}}
\newcommand{\RR}{\mathbb{R}}
\newcommand{\CC}{\mathbb{C}}
\newcommand{\PP}{\mathbb P}
\newcommand{\FF}{\mathbb F}
\newcommand{\DD}{\mathbb D}
\renewcommand{\epsilon}{\varepsilon}

\newcommand{\Aut}{\operatorname{Aut}}
\newcommand{\cl}{\operatorname{cl}}
\newcommand{\ch}{\operatorname{ch}}
\newcommand{\Con}{\operatorname{Con}}
\newcommand{\coker}{\operatorname{coker}}
\newcommand{\CVect}{\CC\operatorname{-Vect}}
\newcommand{\Cantor}{\mathcal{C}}
\newcommand{\D}{\mathcal{D}}
\newcommand{\card}{\operatorname{card}}
\newcommand{\dbar}{\overline \partial}
\newcommand{\diam}{\operatorname{diam}}
\newcommand{\dom}{\operatorname{dom}}
\newcommand{\End}{\operatorname{End}}
\DeclareMathOperator*{\esssup}{ess\,sup}
\newcommand{\GL}{\operatorname{GL}}
\newcommand{\Hom}{\operatorname{Hom}}
\newcommand{\id}{\operatorname{id}}
\newcommand{\Ind}{\operatorname{Ind}}
\newcommand{\Inn}{\operatorname{Inn}}
\newcommand{\interior}{\operatorname{int}}
\newcommand{\lcm}{\operatorname{lcm}}
\newcommand{\mesh}{\operatorname{mesh}}
\newcommand{\LL}{\mathcal L_0}
\newcommand{\Leb}{\mathcal{L}_{\text{loc}}^2}
\newcommand{\Lip}{\operatorname{Lip}}
\newcommand{\ppGL}{\operatorname{PGL}}
\newcommand{\ppic}{\vspace{35mm}}
\newcommand{\ppset}{\mathcal{P}}
\DeclareMathOperator{\proj}{proj}
\DeclareMathOperator*{\Res}{Res}
\newcommand{\Riem}{\mathcal{R}}
\newcommand{\RVect}{\RR\operatorname{-Vect}}
\newcommand{\Sch}{\mathcal{S}}
\newcommand{\SL}{\operatorname{SL}}
\newcommand{\sgn}{\operatorname{sgn}}
\newcommand{\spn}{\operatorname{span}}
\newcommand{\Spec}{\operatorname{Spec}}
\newcommand{\supp}{\operatorname{supp}}
\newcommand{\TT}{\mathcal T}
\DeclareMathOperator{\tr}{tr}

% Calculus of variations
\DeclareMathOperator{\pp}{\mathbf p}
\DeclareMathOperator{\zz}{\mathbf z}
\DeclareMathOperator{\uu}{\mathbf u}
\DeclareMathOperator{\vv}{\mathbf v}
\DeclareMathOperator{\ww}{\mathbf w}

% Categories
\newcommand{\Ab}{\mathbf{Ab}}
\newcommand{\Cat}{\mathbf{Cat}}
\newcommand{\Group}{\mathbf{Group}}
\newcommand{\Module}{\mathbf{Module}}
\newcommand{\Set}{\mathbf{Set}}
\DeclareMathOperator{\Fun}{Fun}
\DeclareMathOperator{\Iso}{Iso}

% Complex analysis
\renewcommand{\Re}{\operatorname{Re}}
\renewcommand{\Im}{\operatorname{Im}}

% Logic
\renewcommand{\iff}{\leftrightarrow}
\newcommand{\Henkin}{\operatorname{Henk}}
\newcommand{\PA}{\mathbf{PA}}
\newcommand{\Var}{\operatorname{Var}}
\DeclareMathOperator{\proves}{\vdash}

% Group
\DeclareMathOperator{\Gal}{Gal}
\DeclareMathOperator{\Fix}{Fix}
\DeclareMathOperator{\Out}{Out}

% Other symbols
\newcommand{\heart}{\ensuremath\heartsuit}
\newcommand{\club}{\ensuremath\clubsuit}

\DeclareMathOperator{\atanh}{atanh}

% Theorems
\theoremstyle{definition}
\newtheorem*{corollary}{Corollary}
\newtheorem*{falselemma}{Grader's ``Lemma"}
\newtheorem{exer}{Exercise}
\newtheorem{lemma}{Lemma}[exer]
\newtheorem{theorem}[lemma]{Theorem}

\def\Xint#1{\mathchoice
{\XXint\displaystyle\textstyle{#1}}%
{\XXint\textstyle\scriptstyle{#1}}%
{\XXint\scriptstyle\scriptscriptstyle{#1}}%
{\XXint\scriptscriptstyle\scriptscriptstyle{#1}}%
\!\int}
\def\XXint#1#2#3{{\setbox0=\hbox{$#1{#2#3}{\int}$ }
\vcenter{\hbox{$#2#3$ }}\kern-.6\wd0}}
\def\ddashint{\Xint=}
\def\dashint{\Xint-}

\usepackage[backend=bibtex,style=alphabetic,maxcitenames=50,maxnames=50]{biblatex}
\renewbibmacro{in:}{}
\DeclareFieldFormat{pages}{#1}

\begin{document}
\noindent
\large\textbf{Probability, HW 8} \hfill \textbf{Aidan Backus} \\

% --------------------------------------------------------------
%                         Start here
% --------------------------------------------------------------\


\begin{exer}
Let $S$ be Polish. Let $(X_n)$ be a sequence of random variables in $S$, and $\phi$ a tightness function.
Show that if
$$\sup_n E(\phi(X_n)) < \varepsilon$$
then $(X_n)$ is tight.
\end{exer}

We reason by contrapositive. Suppose that $(X_n)$ is not tight, so there is a $\varepsilon > 0$ such that for every compact set $K$, there is a $n$ such that $\PP(X_n \in K) \leq 1 - \varepsilon$.
Since $\phi$ is a tightness function, for every $c > 0$ there is an $n$ with
$$\PP(\phi(X_n) \leq c) \leq 1 - \varepsilon,$$
so
$$E(\phi(X_n)) > (1 - \varepsilon)c.$$
Taking $c \to \infty$ we see the claim.

\begin{exer}
Let $\mathcal A$ be the set of finite-dimensional cylinder sets
$$\{f \in \mathcal C: (f(t_1), \dots, f(t_n)) \in B\}$$
where $B \subseteq \RR^n$ is Borel, $t_i \in [0, T]$, and $n \geq 1$. Let $\mathcal E$ be the set of all sets of the form $\{f \in \mathcal C: f(t) \in G\}$ where $G \subseteq \RR$ is open and $t \in [0, T]$.
Show that $\mathcal B(\mathcal C) = \sigma(\mathcal A) = \sigma(\mathcal E)$.
\end{exer}

It is immediate that
$$\sigma(\mathcal E) \subseteq \sigma(\mathcal A) \subseteq \mathcal B(\mathcal C),$$
since every element of $\mathcal E$ is a finite-dimensional cylinder set, and every finite-dimensional cylinder set is Borel in $\mathcal C$.
So we just need to show that $\mathcal B(\mathcal C) \subseteq \sigma(\mathcal E)$.
By definition,
$$\mathcal B(\mathcal C) = \sigma(\{\{g \in \mathcal C: \sup |f - g| < \varepsilon\}: f \in \mathcal C,~\varepsilon > 0\})$$
So we must show that the $L^\infty$-balls $\{g \in \mathcal C: \sup |f - g| < \varepsilon\}$ can be expressed in terms of sets of the form $\{g \in \mathcal C: g(t) \in G\}$, which we call basic Borel sets, using only $\sigma$-algebra operations.
So fix $f$ and $\varepsilon$, and an enumeration $(q_n)$ of $\QQ$. Let
$$G_n = \{g \in \mathcal C: |g(q_n) - f(q_n)| < \varepsilon\};$$
here $|g(q_n) - f(q_n)| < \varepsilon$ is an open condition, so $G_n$ is a basic Borel set. Thus $\bigcap_n G_n$ consists of $g$ such that
$$\sup_{q \in \QQ} |g - f| < \varepsilon;$$
since $\QQ$ is dense in $[0, T]$, this implies that $\sup |g - f| < \varepsilon$, so $\bigcap G_n = \{g \in \mathcal C: \sup |f - g| < \varepsilon\}$, which was desired.

\begin{exer}
Let $\mu$ be a Borel probability measure on $\mathcal C$.
We say that $\mu_{t_1, \dots, t_k}$, a Borel probability measure on $\RR^k$, is a finite-dimensional distribution of $\mu$ if it is a pushforward of $\mu$ by the evaluation map $f \mapsto (f(t_1), \dots, f(t_k))$ and $0 \leq t_1 < t_2 < \cdots < t_k \leq T$. Show that $\mu$ is determined by its finite-dimensional distributions.
\end{exer}

Since $\mathcal C$ is Polish, every open subset of $\mathcal C$ can be written as a countable union of $L^\infty$-balls $B(f, \varepsilon)$.
The topology of $\mathcal C$ is a $\pi$-system which generates $\mathcal B(\mathcal C)$, so it follows that it suffices to show that the value of $\mu(B(f, \varepsilon))$ is determined by the finite-dimensional distributions of $\mu$ whenever $f \in \mathcal C$ and $\varepsilon > 0$.
Moreover,
$$\mu(B(f, \varepsilon)) = \mu\left(\bigcap_{t \in \QQ} \{g \in \mathcal C: |g(t) - f(t)| < \varepsilon\}\right)$$
since $\QQ$ is dense in $[0, T]$.

If $S = \{t_1, \dots, t_k\}$ is a finite set of rationals, and $t_i < t_{i+1}$, we let $\mu^S = \mu_{t_1, \dots, t_k}$ and define the evaluation map
$$\delta_S(f) = (f(t_1), \dots, f(t_k)).$$
We claim that
\begin{equation}
\label{measure formula}
\mu(B(f, \varepsilon)) = \inf_S \mu^S\left(\prod_{t \in S} (f(t) - \varepsilon, f(t) + \varepsilon)\right)
\end{equation}
where the infimum ranges over all finite sets $S$ and the product is indexed in increasing order.
Indeed,
$$\delta_S\left(\bigcap_{t \in S} \{g \in \mathcal C: |g(t) - f(t)| < \varepsilon\}\right) = \prod_{t \in S} (f(t) - \varepsilon, f(t) + \varepsilon),$$
so by definition of pushforward,
$$\mu\left(\bigcap_{t \in S} \{g \in \mathcal C: |g(t) - f(t)| < \varepsilon\}\right) = \mu^S\left(\prod_{t \in S} (f(t) - \varepsilon, f(t) + \varepsilon)\right).$$
If $(S_n)$ is a sequence of finite sets that increases to $\QQ$, one has
$$\lim_{n \to \infty} \bigcap_{t \in S_n} \{g \in \mathcal C: |g(t) - f(t)| < \varepsilon\} = \bigcap_{t \in \QQ} \{g \in \mathcal C: |g(t) - f(t)| < \varepsilon\}$$
where the limit is in the sense of a decreasing chain of sets.
Therefore measure continuity implies that
$$\lim_{n \to \infty} \mu\left(\bigcap_{t \in S_n} \{g \in \mathcal C: |g(t) - f(t)| < \varepsilon\}\right) = \mu\left(\bigcap_{t \in \QQ} \{g \in \mathcal C: |g(t) - f(t)| < \varepsilon\}\right)$$
where the limit is again decreasing. Therefore
$$\mu\left(\bigcap_{t \in \QQ} \{g \in \mathcal C: |g(t) - f(t)| < \varepsilon\}\right) = \lim_{n \to \infty} \mu^{S_n}\left(\prod_{t \in S_n} (f(t) - \varepsilon, f(t) + \varepsilon)\right);$$
the limit is decreasing and the sequence $(S_n)$ was arbitrary, so this implies (\ref{measure formula}).
Thus $\mu$ is determined by $(\mu^S)_S$.

\begin{exer}
Let $(\mu_n)$ and $\mu$ be Borel probability measures on $\mathcal C$. Show that if $\mu_n \implies \mu$ then every finite-dimensional distribution of $\mu_n$ converges to that of $\mu$.
\end{exer}

By the portmanteau theorem, it suffices to show that for every open set $U \subseteq \mathcal C$ and every finite set $S$, in the notation of the previous solution,
\begin{equation}
\label{WTS 4}
\mu^S(U) \leq \liminf_{n \to \infty} \mu_n^S(U).
\end{equation}
Since the $\ell^\infty$ norm on $\RR^{|S|}$ is comparable to the $\ell^2$ norm, it suffices to check (\ref{WTS 4}) when $U$ is an $\ell^\infty$-ball,
$$U = \prod_{j=1}^{\card S} (y_j - \varepsilon, y_j + \varepsilon),$$
for some $y \in \RR^{\card S}$. Writing $S = \{t_1, \dots, t_{\card S}\}$, there is a polynomial $f: [0, T] \to \RR^{\card S}$ such that for every $j$, $f(t_j) = y_j$. Then
$$U = \delta_S\left(\bigcap_{t \in S} \{g \in \mathcal C: |g(t) - f(t)| < \varepsilon\}\right)$$
so by definition of a pushforward measure and the portmanteau theorem applied to $(\mu_n)$,
\begin{align*}\mu^S(U) &= \mu\left(\bigcap_{t \in S} \{g \in \mathcal C: |g(t) - f(t)| < \varepsilon\}\right)\\
&\leq \liminf_{n \to \infty} \mu_n\left(\bigcap_{t \in S} \{g \in \mathcal C: |g(t) - f(t)| < \varepsilon\}\right)\\
&= \liminf_{n \to \infty} \mu^S_n(U),
\end{align*}
thus (\ref{WTS 4}) holds.

\begin{exer}
Let $(\mu_n)$ and $\mu$ be Borel probability measures on $\mathcal C$. Show that if every finite-dimensional distribution of $(\mu_n)$ converges to that of $\mu$, and $(\mu_n)$ is tight, then $\mu_n \implies \mu$.
\end{exer}

By the portmanteau theorem, it suffices to show that for every open set $U \subseteq \mathcal C$,
$$\mu(U) \leq \liminf_{n \to \infty} \mu_n(U).$$
As usual, we can replace the arbitrary open set $U$ with the $L^\infty$-ball $B(f, \varepsilon)$, and we we work within the framework we set up in Exercise 3.

Let $(S_m)$ be a sequence of finite sets that increase to $\QQ$; then, as in the solution of Exercise 3,
\begin{align*}
\mu(B(f, \varepsilon)) &= \lim_{m \to \infty} \mu^{S_m} \left(\prod_{t \in S_m} (f(t) - \varepsilon, f(t) + \varepsilon)\right)\\
&= \lim_{m \to \infty} \left(\lim_{n \to \infty} \mu^{S_m}_n \right) \left(\prod_{t \in S_m} (f(t) - \varepsilon, f(t) + \varepsilon)\right)
\end{align*}
where the limit in $n$ is weak.

Since $(\mu_n)$ is tight, Prokohov's theorem gives a subsequence $(\mu_{n_k})_k$ which converges weakly to a measure $\nu$, thus for every open set $U \subseteq \mathcal C$,
$$\nu(U) \leq \liminf_{k \to \infty} \mu_{n_k}(U).$$
As in Exercise 3, one then has
$$\nu(B(f,\varepsilon)) \leq \liminf_{k \to \infty} \lim_{m \to \infty} \mu_{n_k}^{S_m} \left(\prod_{t \in S_m} (f(t) - \varepsilon, f(t) + \varepsilon)\right).$$
The sequence on the right-hand side is decreasing in $m$, and after passing to a further subsequence $\mu_{n_{k_j}}^{S_m}$ can be assumed monotone (and hence convergent!) in $j$. Therefore we may commute the limits,
$$\nu(B(f,\varepsilon)) \leq \lim_{m \to \infty} \lim_{j \to \infty} \mu_{n_{k_j}}^{S_m}
\left(\prod_{t \in S_m} (f(t) - \varepsilon, f(t) + \varepsilon)\right).$$
Then $\lim_j \mu_{n_{k_j}}^{S_m} \implies \nu$, and so since the weak topology of measures is Hausdorff, $\nu = \mu$.
Thus every monotone subsequence of the convergent subsequence $(\mu_{n_k})$ converges to $\mu$, which implies $\mu_{n_k} \implies \mu$.

Thus if $(\mu_{n_k})$ is an arbitrary subsequence of $(\mu_n)$ there is a subsequence $(\mu_{n_{k_j}})$ such that $\mu_{n_{k_j}} \implies \mu$; that implies $\mu_n \implies \mu$.

\end{document}
