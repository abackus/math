
% --------------------------------------------------------------
% This is all preamble stuff that you don't have to worry about.
% Head down to where it says "Start here"
% --------------------------------------------------------------

\documentclass[10pt]{article}

\usepackage[margin=.7in]{geometry}
\usepackage{amsmath,amsthm,amssymb}
\usepackage{enumitem}
\usepackage{tikz-cd}
\usepackage{mathtools}
\usepackage{amsfonts}
\usepackage{listings}
\usepackage{algorithm2e}
\usepackage{verse,stmaryrd}
\usepackage{fancyvrb}

% Number systems
\newcommand{\NN}{\mathbb{N}}
\newcommand{\ZZ}{\mathbb{Z}}
\newcommand{\QQ}{\mathbb{Q}}
\newcommand{\RR}{\mathbb{R}}
\newcommand{\CC}{\mathbb{C}}
\newcommand{\PP}{\mathbb P}
\newcommand{\FF}{\mathbb F}
\newcommand{\DD}{\mathbb D}
\renewcommand{\epsilon}{\varepsilon}

\newcommand{\Aut}{\operatorname{Aut}}
\newcommand{\cl}{\operatorname{cl}}
\newcommand{\ch}{\operatorname{ch}}
\newcommand{\Con}{\operatorname{Con}}
\newcommand{\coker}{\operatorname{coker}}
\newcommand{\CVect}{\CC\operatorname{-Vect}}
\newcommand{\Cantor}{\mathcal{C}}
\newcommand{\D}{\mathcal{D}}
\newcommand{\card}{\operatorname{card}}
\newcommand{\dbar}{\overline \partial}
\newcommand{\diam}{\operatorname{diam}}
\newcommand{\dom}{\operatorname{dom}}
\newcommand{\End}{\operatorname{End}}
\DeclareMathOperator*{\esssup}{ess\,sup}
\newcommand{\GL}{\operatorname{GL}}
\newcommand{\Hom}{\operatorname{Hom}}
\newcommand{\id}{\operatorname{id}}
\newcommand{\Ind}{\operatorname{Ind}}
\newcommand{\Inn}{\operatorname{Inn}}
\newcommand{\interior}{\operatorname{int}}
\newcommand{\lcm}{\operatorname{lcm}}
\newcommand{\mesh}{\operatorname{mesh}}
\newcommand{\LL}{\mathcal L_0}
\newcommand{\Leb}{\mathcal{L}_{\text{loc}}^2}
\newcommand{\Lip}{\operatorname{Lip}}
\newcommand{\ppGL}{\operatorname{PGL}}
\newcommand{\ppic}{\vspace{35mm}}
\newcommand{\ppset}{\mathcal{P}}
\DeclareMathOperator{\proj}{proj}
\DeclareMathOperator*{\Res}{Res}
\newcommand{\Riem}{\mathcal{R}}
\newcommand{\RVect}{\RR\operatorname{-Vect}}
\newcommand{\Sch}{\mathcal{S}}
\newcommand{\SL}{\operatorname{SL}}
\newcommand{\sgn}{\operatorname{sgn}}
\newcommand{\spn}{\operatorname{span}}
\newcommand{\Spec}{\operatorname{Spec}}
\newcommand{\supp}{\operatorname{supp}}
\newcommand{\TT}{\mathcal T}
\DeclareMathOperator{\tr}{tr}

% Calculus of variations
\DeclareMathOperator{\pp}{\mathbf p}
\DeclareMathOperator{\zz}{\mathbf z}
\DeclareMathOperator{\uu}{\mathbf u}
\DeclareMathOperator{\vv}{\mathbf v}
\DeclareMathOperator{\ww}{\mathbf w}

% Categories
\newcommand{\Ab}{\mathbf{Ab}}
\newcommand{\Cat}{\mathbf{Cat}}
\newcommand{\Group}{\mathbf{Group}}
\newcommand{\Module}{\mathbf{Module}}
\newcommand{\Set}{\mathbf{Set}}
\DeclareMathOperator{\Fun}{Fun}
\DeclareMathOperator{\Iso}{Iso}

% Complex analysis
\renewcommand{\Re}{\operatorname{Re}}
\renewcommand{\Im}{\operatorname{Im}}

% Logic
\renewcommand{\iff}{\leftrightarrow}
\newcommand{\Henkin}{\operatorname{Henk}}
\newcommand{\PA}{\mathbf{PA}}
\newcommand{\Var}{\operatorname{Var}}
\DeclareMathOperator{\proves}{\vdash}

% Group
\DeclareMathOperator{\Gal}{Gal}
\DeclareMathOperator{\Fix}{Fix}
\DeclareMathOperator{\Out}{Out}

% Other symbols
\newcommand{\heart}{\ensuremath\heartsuit}
\newcommand{\club}{\ensuremath\clubsuit}

\DeclareMathOperator{\atanh}{atanh}

% Theorems
\theoremstyle{definition}
\newtheorem*{corollary}{Corollary}
\newtheorem*{falselemma}{Grader's ``Lemma"}
\newtheorem{exer}{Exercise}
\newtheorem{lemma}{Lemma}[exer]
\newtheorem{theorem}[lemma]{Theorem}

\def\Xint#1{\mathchoice
{\XXint\displaystyle\textstyle{#1}}%
{\XXint\textstyle\scriptstyle{#1}}%
{\XXint\scriptstyle\scriptscriptstyle{#1}}%
{\XXint\scriptscriptstyle\scriptscriptstyle{#1}}%
\!\int}
\def\XXint#1#2#3{{\setbox0=\hbox{$#1{#2#3}{\int}$ }
\vcenter{\hbox{$#2#3$ }}\kern-.6\wd0}}
\def\ddashint{\Xint=}
\def\dashint{\Xint-}

\usepackage[backend=bibtex,style=alphabetic,maxcitenames=50,maxnames=50]{biblatex}
\renewbibmacro{in:}{}
\DeclareFieldFormat{pages}{#1}

\begin{document}
\noindent
\large\textbf{Probability, HW 6} \hfill \textbf{Aidan Backus} \\

% --------------------------------------------------------------
%                         Start here
% --------------------------------------------------------------\


\begin{exer}
Suppose $X_n \to X$ in distribution. Show that if $X = a$ almost surely then $X_n \to X$ in probability.
\end{exer}

By assumption, the cdf $F$ of $X$ is identically $0$ on $(-\infty, a)$ and $1$ on $(a, \infty)$.
Therefore the cdfs $F_n$ of $X_n$ converge to $F$ pointwise on $\RR \setminus \{a\}$.
So for every $\varepsilon > 0$, and every $n$ large enough, $F_n < \varepsilon$ on $(-\infty, a - \varepsilon)$ and $F_n > \varepsilon$ on $(a + \varepsilon, \infty)$.
Therefore $\PP(X_n < a - \varepsilon) < \varepsilon$ and $\PP(X_n > a + \varepsilon) < \varepsilon$.
This was desired.

\begin{exer}
Show that a weak limit of Borel probability measures is unique.
\end{exer}

Suppose that $\mu_n \to \mu$ and $\mu_n \to \mu'$ weakly.
After subtracting $\mu'$ from both sides we may assume that $\mu_n \to 0$ weakly.
Let $E$ be a Borel set; we must show that $\mu(E) = 0$.
For every bounded continuous function $f$,
$$\lim_{n \to \infty} \int_{-\infty}^\infty f~d\mu_n = 0.$$
In particular this is true if $f$ majorizes $1_E$. So $\mu(E) = 0$.

\begin{exer}
Suppose $X_n \to X$ in distribution. Show that $E|X| \leq \liminf_n E|X_n|$.
\end{exer}

Let $Y_n,Y$ be the Skohorod representations of $X_n,X$, thus $Y_n \to Y$ almost surely and $Y_n,Y$ are identically distributed to $X_n,X$.
Then by Fatou's lemma,
$$E|Y| \leq \liminf_{n \to \infty} E|Y_n|.$$
Passing back to $X_n,X$ (since $Y_n,Y$ are identically distributed to $X_n,X$) the claim follows.

\begin{exer}
Suppose $X_n \to X$ in distribution and $B$ is a Borel set. Show that
$$\PP(X \in B^o) \leq \liminf_{n \to \infty} \PP(X_n \in B) \leq \limsup_{n \to \infty} \PP(X_n \in B) \leq \PP(X \in \overline B).$$
\end{exer}

Let $Y_n,Y$ be the Skohorod representations of $X_n,X$.
Then $\PP(Y \in B^o) = \PP(X \in B^o)$, $\PP(Y \in \overline B) = \PP(X \in \overline B)$, and $\PP(Y_n \in B) = \PP(X_n \in B)$.
So we can assume that $X_n \to X$ almost surely.
The inequality
$$\liminf_{n \to \infty} \PP(X_n \in B) \leq \limsup_{n \to \infty} \PP(X_n \in B)$$
is clear.

Let $X_{n_k}$ be a subsequence such that $\liminf_n \PP(X_n \in B) = \lim_n \PP(X_{n_k} \in B)$, and $X_{n_k'}$ similarly for $\limsup_n \PP(X_n \in B)$.
But then $X_{n_k} \to X$ almost surely still, and $\PP(X_{n_k} \in B^o) \leq \PP(X_{n_k} \in B)$, so this remains true when passing to the limit, thus
$$\PP(X \in B^o) \leq \liminf_{n \to \infty} \PP(X_n \in B).$$
For analogous reasons,
$$\limsup_{n \to \infty} \PP(X_n \in B) \leq \PP(X \in \overline B).$$


\begin{exer}
Suppose that $\mu^n$ is a sequence of Borel probability measure on $\RR^2$.
Show that $\mu^n$ is tight iff both its marginals $\mu^n_1,\mu^n_2$ are.
\end{exer}

Suppose that $\mu^n$ is tight, let $\varepsilon > 0$, and select $K$ compact such that for every $n$, $\mu^n(K) > 1 - \varepsilon$.
By passing to the smallest rectangle containing $K$, we may assume that $K$ itself is a rectangle.
Let $\pi_1,\pi_2$ be projections. Then, given $j \in \{1, 2\}$,
$$\mu^n_j(\RR \setminus \pi_j(K)) \leq \mu^n(\RR \setminus K) \varepsilon$$
since $x_j \notin \pi_j(K)$ implies $x \notin K$.
So $\mu^n_j(\pi_j(K)) > 1 - \varepsilon$, so $\mu^n$ has tight marginals.

Conversely, suppose that $\mu^n$ has tight marginals, let $\varepsilon > 0$ and select $K_j$ compact such that for every $n$, $\mu^n_j(K) > 1 - \varepsilon$.
Let $K = K_1 \times K_2$; then
$$\mu^n(\RR^2 \setminus K) \leq \mu^n_1(\RR \setminus K_1) + \mu^n_2(\RR \setminus K_2) < 2\varepsilon$$
so $\mu^n$ is tight.














\end{document}
