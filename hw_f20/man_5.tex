
% --------------------------------------------------------------
% This is all preamble stuff that you don't have to worry about.
% Head down to where it says "Start here"
% --------------------------------------------------------------

\documentclass[10pt]{article}

\usepackage[margin=.7in]{geometry}
\usepackage{amsmath,amsthm,amssymb}
\usepackage{enumitem}
\usepackage{tikz-cd}
\usepackage{mathtools}
\usepackage{amsfonts}
\usepackage{listings}
\usepackage{algorithm2e}
\usepackage{verse,stmaryrd}
\usepackage{fancyvrb}

% Number systems
\newcommand{\NN}{\mathbb{N}}
\newcommand{\ZZ}{\mathbb{Z}}
\newcommand{\QQ}{\mathbb{Q}}
\newcommand{\RR}{\mathbb{R}}
\newcommand{\CC}{\mathbb{C}}
\newcommand{\PP}{\mathbb P}
\newcommand{\FF}{\mathbb F}
\newcommand{\DD}{\mathbb D}
\renewcommand{\epsilon}{\varepsilon}

\newcommand{\Aut}{\operatorname{Aut}}
\newcommand{\cl}{\operatorname{cl}}
\newcommand{\ch}{\operatorname{ch}}
\newcommand{\Con}{\operatorname{Con}}
\newcommand{\coker}{\operatorname{coker}}
\newcommand{\CVect}{\CC\operatorname{-Vect}}
\newcommand{\Cantor}{\mathcal{C}}
\newcommand{\D}{\mathcal{D}}
\newcommand{\card}{\operatorname{card}}
\newcommand{\dbar}{\overline \partial}
\newcommand{\diam}{\operatorname{diam}}
\newcommand{\dom}{\operatorname{dom}}
\newcommand{\End}{\operatorname{End}}
\DeclareMathOperator*{\esssup}{ess\,sup}
\newcommand{\GL}{\operatorname{GL}}
\newcommand{\Hess}{\operatorname{Hess}}
\newcommand{\Hom}{\operatorname{Hom}}
\newcommand{\id}{\operatorname{id}}
\newcommand{\Ind}{\operatorname{Ind}}
\newcommand{\Inn}{\operatorname{Inn}}
\newcommand{\interior}{\operatorname{int}}
\newcommand{\lcm}{\operatorname{lcm}}
\newcommand{\mesh}{\operatorname{mesh}}
\newcommand{\LL}{\mathcal L_0}
\newcommand{\Leb}{\mathcal{L}_{\text{loc}}^2}
\newcommand{\Lip}{\operatorname{Lip}}
\newcommand{\ppGL}{\operatorname{PGL}}
\newcommand{\ppic}{\vspace{35mm}}
\newcommand{\ppset}{\mathcal{P}}
\DeclareMathOperator{\proj}{proj}
\DeclareMathOperator*{\Res}{Res}
\newcommand{\Riem}{\mathcal{R}}
\newcommand{\RVect}{\RR\operatorname{-Vect}}
\newcommand{\Sch}{\mathcal{S}}
\newcommand{\SL}{\operatorname{SL}}
\newcommand{\sgn}{\operatorname{sgn}}
\newcommand{\spn}{\operatorname{span}}
\newcommand{\Spec}{\operatorname{Spec}}
\newcommand{\supp}{\operatorname{supp}}
\newcommand{\TT}{\mathcal T}
\DeclareMathOperator{\tr}{tr}

% Calculus of variations
\DeclareMathOperator{\pp}{\mathbf p}
\DeclareMathOperator{\zz}{\mathbf z}
\DeclareMathOperator{\uu}{\mathbf u}
\DeclareMathOperator{\vv}{\mathbf v}
\DeclareMathOperator{\ww}{\mathbf w}

% Categories
\newcommand{\Ab}{\mathbf{Ab}}
\newcommand{\Cat}{\mathbf{Cat}}
\newcommand{\Group}{\mathbf{Group}}
\newcommand{\Module}{\mathbf{Module}}
\newcommand{\Set}{\mathbf{Set}}
\DeclareMathOperator{\Fun}{Fun}
\DeclareMathOperator{\Iso}{Iso}

% Complex analysis
\renewcommand{\Re}{\operatorname{Re}}
\renewcommand{\Im}{\operatorname{Im}}

% Logic
\renewcommand{\iff}{\leftrightarrow}
\newcommand{\Henkin}{\operatorname{Henk}}
\newcommand{\PA}{\mathbf{PA}}
\DeclareMathOperator{\proves}{\vdash}

% Group
\DeclareMathOperator{\Gal}{Gal}
\DeclareMathOperator{\Fix}{Fix}
\DeclareMathOperator{\Out}{Out}

% Other symbols
\newcommand{\heart}{\ensuremath\heartsuit}
\newcommand{\club}{\ensuremath\clubsuit}

\DeclareMathOperator{\atanh}{atanh}

% Theorems
\theoremstyle{definition}
\newtheorem*{corollary}{Corollary}
\newtheorem*{falselemma}{Grader's ``Lemma"}
\newtheorem{exer}{Exercise}
\newtheorem{lemma}{Lemma}[exer]
\newtheorem{theorem}[lemma]{Theorem}

\def\Xint#1{\mathchoice
{\XXint\displaystyle\textstyle{#1}}%
{\XXint\textstyle\scriptstyle{#1}}%
{\XXint\scriptstyle\scriptscriptstyle{#1}}%
{\XXint\scriptscriptstyle\scriptscriptstyle{#1}}%
\!\int}
\def\XXint#1#2#3{{\setbox0=\hbox{$#1{#2#3}{\int}$ }
\vcenter{\hbox{$#2#3$ }}\kern-.6\wd0}}
\def\ddashint{\Xint=}
\def\dashint{\Xint-}

\usepackage[backend=bibtex,style=alphabetic,maxcitenames=50,maxnames=50]{biblatex}
\renewbibmacro{in:}{}
\DeclareFieldFormat{pages}{#1}

\begin{document}
\noindent
\large\textbf{Manifolds, HW 5} \hfill \textbf{Aidan Backus} \\

% --------------------------------------------------------------
%                         Start here
% --------------------------------------------------------------\

\begin{exer}[7.5]
Let $G$ be a Lie group. Show that the universal covering group of $G$ is unique.
\end{exer}

Let $\pi: \widetilde G \to G$ and $\pi': \widetilde G' \to G$ be universal covering groups.
Then $\widetilde G$ and $\widetilde G'$ are universal covering spaces.
Since $G$ is a connected manifold, $G$ is semilocally simply connected, path connected, and locally path-connected.
Therefore there is a map of covering spaces $\Psi: \widetilde G \to \widetilde G'$.
In particular, if $\tilde e'$ denotes the identity of $\widetilde G'$ then $\Psi^{-1}(\tilde e')$ lies in the fiber $\pi^{-1}(e)$ of the identity $e$ of $G$.
The action of $\pi_1(G)$ on fibers in $\widetilde G$ is transitive, so there is a deck transformation $\kappa: \widetilde G \to \widetilde G$ such that $\kappa(\Psi^{-1}(\tilde e')) = \tilde e$ is the identity of $\widetilde G$. Therefore $\Psi(\kappa^{-1}(\tilde e)) = \tilde e'$.

So let $\Phi = \Psi \circ \kappa^{-1}$. Since $\Psi$ is a morphism of covering spaces, and $\kappa$ is a deck transformation, $\Phi$ is a morphism of covering spaces, and besides, $\Phi(\tilde e) = \tilde e'$.
By abstract nonsense in the category of covering spaces over $G$, $\Phi$ is the unique morphism of covering spaces with this property.

We now must show that $\Phi$ is a morphism of covering groups; that is, if $m$ is the multiplication of $G$, $\tilde m$ the multiplication of $\widetilde G$, and $\tilde m'$ the multiplication in $\widetilde G'$, then the diagram of smooth manifolds
\begin{equation}
\label{Phi is covering morphism}
\begin{tikzcd}
\widetilde G \times \widetilde G \arrow[r,"\Phi \times \Phi"] \arrow[d,"\tilde m"] & \widetilde G' \times \widetilde G' \arrow[d,"\tilde m'"]\\
\widetilde G \arrow[r,"\Phi"] & \widetilde G'
\end{tikzcd}
\end{equation}
commutes. However, we also have a commutative diagram of manifolds
$$\begin{tikzcd}
\widetilde G \times \widetilde G \arrow[r,"\pi \times \pi"] \arrow[d,"\tilde m"] & G \times G \arrow[d,"m"] & \widetilde G' \times \widetilde G' \arrow[l,"\pi' \times \pi'"] \arrow[d,"\tilde m'"]\\
\widetilde G \arrow[r,"\pi"] & G & \widetilde G' \arrow[l,"\pi'"]
\end{tikzcd}$$
which shows that $m$ lifts to $\pi \circ \tilde m$ and also to $\pi' \circ \tilde m'$.
So by more abstract nonsense, (\ref{Phi is covering morphism}) commutes up to a deck transformation $\eta$.
Moreover, since $\Phi(\tilde e) = \tilde e'$, $\eta(\tilde e) = \tilde e$.
But $\eta$ is a deck transformation, so this implies that $\eta$ is the identity, and (\ref{Phi is covering morphism}) actually commutes.

\begin{exer}[7.13]
Show that $U(n)$ is a properly embedded $n^2$-dimensional Lie subgroup of $\GL(n, \CC)$.
\end{exer}

Let $\iota: U(n) \to \GL(n, \CC)$ be the inclusion map.
We must check:
\begin{itemize}
\item $U(n)$ is a subgroup of $\GL(n, \CC)$.
\item $U(n)$ is a manifold of dimension $n^2$.
\item $U(n)$ is a Lie group.
\item $\iota$ is a smooth embedding.
\item $\iota$ is proper.
\end{itemize}
The first of these claims is clear.

For the next few, consider the smooth map $f(A) = A^*A$. Then $f(A) = 1$ iff $A \in U(n)$, and $f(A)$ is always self-adjoint.
The space $\GL^*(n, \CC)$ of self-adjoint matrices has dimension $n^2$.
Indeed, the diagonal entries are real, the entries above the diagonal are free, and the entries below the diagonal are given by relations, and a simple counting argument then shows $\dim \GL^*(n, \CC) = n^2$.

We claim that if $A \in U(n)$ then $f$ is a submersion near $A$.
In fact,
$$df_A(X) = \frac{d}{dt} f(A + tX)\bigg|_{t=0} = \frac{d}{dt} (A + tX)^*(A + tX)\bigg|_{t=0} = X^*A + A^*X.$$
We must show that $df_A$ is surjective.
Let $B \in \GL^*(n, \CC)$. Then
$$df_A(AB) = (AB)^*A + A^*AB = BA^*A + A^*AB = 2B$$
so $df_A(AB/2) = B$. This implies that near $A$, $f$ is a submersion, and so $1$ is a regular value of $f$.
Therefore $U(n) = f^{-1}(1)$ is a manifold, and
$$\dim U(n) = \dim \GL(n, \CC) - \dim \GL^*(n, \CC) = 2n^2 - n^2 = n^2.$$
In particular, $\iota$ is a smooth embedding.
Moreover, the proof that $U(n)$ is Lie -- thus, multiplication and inversion are smooth -- is the same as the proof for $\GL(n, \CC)$.

Finally, to show that $\iota$ is proper, it suffices to show that $U(n)$ is compact.
Since $U(n)$ is the preimage of $1$, it is closed.
Since unitary matrices have operator norm $1$, $U(n)$ is bounded.
So the Heine-Borel theorem completes the proof.


\begin{exer}[7.16]
Show that $SU(2)$ is diffeomorphic to $S^3$.
\end{exer}

Let $A \in SU(2)$. Then $\det A = 1$ and there are $w, z \in \CC$, $\theta \in S^1$, such that
$$A = \begin{bmatrix}
w & z \\
-e^{i\theta}\overline z & e^{i\theta}\overline w
\end{bmatrix}.$$
Solving the determinant equation
$$1 = |w|^2 e^{i\theta} + |z|^2 e^{i\theta}$$
for $e^{i\theta}$ we see that $e^{-i\theta} \geq 0$, so $\theta = 0$ and one has
$$|w|^2 + |z|^2 = 1.$$
Thus $(w, z) \in S^3$, thinking of $S^3$ as the unit sphere of $\CC^2$.
The map $A \mapsto (w, z)$ is obtained by projecting $A$ onto its first row and is clearly smooth; similarly the map $(w, z) \mapsto A$ is clearly smooth.
We just need to show that $A \mapsto (w, z)$ is surjective; that is, given $|w|^2 + |z|^2 = 1$, the matrix $A$ is unitary. To check this, we compute
$$A^*A = \begin{bmatrix}
\overline w & -e^{-i\theta}z\\
\overline z & e^{-i\theta} w
\end{bmatrix}\begin{bmatrix}
w & z \\
-e^{i\theta}\overline z & e^{i\theta}\overline w
\end{bmatrix}
= \begin{bmatrix}|w|^2 + |z|^2 \\ & |w|^2 + |z|^2\end{bmatrix}
= 1.
$$
Thus $A$ is unitary.


\end{document}
