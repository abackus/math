
% --------------------------------------------------------------
% This is all preamble stuff that you don't have to worry about.
% Head down to where it says "Start here"
% --------------------------------------------------------------

\documentclass[10pt]{article}

\usepackage[margin=.7in]{geometry}
\usepackage{amsmath,amsthm,amssymb}
\usepackage{enumitem}
\usepackage{tikz-cd}
\usepackage{mathtools}
\usepackage{amsfonts}
\usepackage{listings}
\usepackage{algorithm2e}
\usepackage{verse,stmaryrd}
\usepackage{fancyvrb}

% Number systems
\newcommand{\NN}{\mathbb{N}}
\newcommand{\ZZ}{\mathbb{Z}}
\newcommand{\QQ}{\mathbb{Q}}
\newcommand{\RR}{\mathbb{R}}
\newcommand{\CC}{\mathbb{C}}
\newcommand{\PP}{\mathbb P}
\newcommand{\FF}{\mathbb F}
\newcommand{\DD}{\mathbb D}
\renewcommand{\epsilon}{\varepsilon}

\newcommand{\Aut}{\operatorname{Aut}}
\newcommand{\cl}{\operatorname{cl}}
\newcommand{\ch}{\operatorname{ch}}
\newcommand{\Con}{\operatorname{Con}}
\newcommand{\coker}{\operatorname{coker}}
\newcommand{\CVect}{\CC\operatorname{-Vect}}
\newcommand{\Cantor}{\mathcal{C}}
\newcommand{\D}{\mathcal{D}}
\newcommand{\card}{\operatorname{card}}
\newcommand{\dbar}{\overline \partial}
\newcommand{\diam}{\operatorname{diam}}
\newcommand{\dom}{\operatorname{dom}}
\newcommand{\End}{\operatorname{End}}
\DeclareMathOperator*{\esssup}{ess\,sup}
\newcommand{\Hess}{\operatorname{Hess}}
\newcommand{\Hom}{\operatorname{Hom}}
\newcommand{\id}{\operatorname{id}}
\newcommand{\Ind}{\operatorname{Ind}}
\newcommand{\Inn}{\operatorname{Inn}}
\newcommand{\interior}{\operatorname{int}}
\newcommand{\lcm}{\operatorname{lcm}}
\newcommand{\mesh}{\operatorname{mesh}}
\newcommand{\LL}{\mathcal L_0}
\newcommand{\Leb}{\mathcal{L}_{\text{loc}}^2}
\newcommand{\Lip}{\operatorname{Lip}}
\newcommand{\ppic}{\vspace{35mm}}
\newcommand{\ppset}{\mathcal{P}}
\DeclareMathOperator{\proj}{proj}
\DeclareMathOperator*{\Res}{Res}
\newcommand{\Riem}{\mathcal{R}}
\newcommand{\RVect}{\RR\operatorname{-Vect}}
\newcommand{\Sch}{\mathcal{S}}
\newcommand{\sgn}{\operatorname{sgn}}
\newcommand{\spn}{\operatorname{span}}
\newcommand{\Spec}{\operatorname{Spec}}
\newcommand{\supp}{\operatorname{supp}}
\newcommand{\TT}{\mathcal T}
\DeclareMathOperator{\tr}{tr}

% Calculus of variations
\DeclareMathOperator{\pp}{\mathbf p}
\DeclareMathOperator{\zz}{\mathbf z}
\DeclareMathOperator{\uu}{\mathbf u}
\DeclareMathOperator{\vv}{\mathbf v}
\DeclareMathOperator{\ww}{\mathbf w}

% Categories
\newcommand{\Ab}{\mathbf{Ab}}
\newcommand{\Cat}{\mathbf{Cat}}
\newcommand{\Group}{\mathbf{Group}}
\newcommand{\Module}{\mathbf{Module}}
\newcommand{\Set}{\mathbf{Set}}
\DeclareMathOperator{\Fun}{Fun}
\DeclareMathOperator{\Iso}{Iso}

% Complex analysis
\renewcommand{\Re}{\operatorname{Re}}
\renewcommand{\Im}{\operatorname{Im}}

% Logic
\renewcommand{\iff}{\leftrightarrow}
\newcommand{\Henkin}{\operatorname{Henk}}
\newcommand{\PA}{\mathbf{PA}}
\DeclareMathOperator{\proves}{\vdash}

% Group
\DeclareMathOperator{\Gal}{Gal}
\DeclareMathOperator{\Fix}{Fix}
\DeclareMathOperator{\Lie}{Lie}
\DeclareMathOperator{\Out}{Out}

\DeclareMathOperator{\Diffeo}{Diffeo}

\newcommand{\GL}{\operatorname{GL}}
\newcommand{\ppGL}{\operatorname{PGL}}
\newcommand{\SL}{\operatorname{SL}}
\newcommand{\SO}{\operatorname{SO}}
\newcommand{\iprod}{\mathbin{\lrcorner}}


% Other symbols
\newcommand{\heart}{\ensuremath\heartsuit}
\newcommand{\club}{\ensuremath\clubsuit}

\DeclareMathOperator{\atanh}{atanh}
\DeclareMathOperator{\codim}{codim}

% Theorems
\theoremstyle{definition}
\newtheorem*{corollary}{Corollary}
\newtheorem*{falselemma}{Grader's ``Lemma"}
\newtheorem{exer}{Exercise}
\newtheorem{lemma}{Lemma}[exer]
\newtheorem{theorem}[lemma]{Theorem}

\def\Xint#1{\mathchoice
{\XXint\displaystyle\textstyle{#1}}%
{\XXint\textstyle\scriptstyle{#1}}%
{\XXint\scriptstyle\scriptscriptstyle{#1}}%
{\XXint\scriptscriptstyle\scriptscriptstyle{#1}}%
\!\int}
\def\XXint#1#2#3{{\setbox0=\hbox{$#1{#2#3}{\int}$ }
\vcenter{\hbox{$#2#3$ }}\kern-.6\wd0}}
\def\ddashint{\Xint=}
\def\dashint{\Xint-}

\usepackage[backend=bibtex,style=alphabetic,maxcitenames=50,maxnames=50]{biblatex}
\renewbibmacro{in:}{}
\DeclareFieldFormat{pages}{#1}

\begin{document}
\noindent
\large\textbf{Manifolds, HW 9} \hfill \textbf{Aidan Backus} \\

% --------------------------------------------------------------
%                         Start here
% --------------------------------------------------------------\

\begin{exer}[15.5]
Let $M$ be a manifold with boundary. Show that $TM$ and $T^*M$ are orientable.
\end{exer}

Since $M$ admits a Riemannian metric, we may choose a bundle isomorphism $TM \to T^*M$; it thus suffices to show that $T^*M$ is orientable.
Let $\pi: T^*M \to M$ be the projection map.
We consider the tautological $1$-form
$$\eta_{x,\xi} = \pi_x^*\xi.$$
In coordinates $(x, \xi)$, $\eta = \xi_i ~dx^i$, so if $\omega = d\eta$, $\omega = d\xi_i \wedge dx^i$.
Thus $\omega^{\wedge n} = \bigwedge_i d\xi_i \wedge dx^i$.
This gives $\omega^{\wedge n} \neq 0$ locally and hence globally, so $\omega^{\wedge n}$ is a top form on $T^*M$.

\begin{exer}[16.6]
Prove the hairy ball theorem by showing that the following are equivalent on $n$, where $\alpha$ is the antipode of $S^n$:
\begin{enumerate}
\item There is a nowhere vanishing vector field on $S^n$.
\item There is a continuous map $V: S^n \to S^n$ such that for every $x \in S^n$, $V(x) \perp x$.
\item $\alpha$ is homotopic to the identity.
\item $\alpha$ is orientation-preserving.
\item $n$ is odd.
\end{enumerate}
\end{exer}

We prove in part $n$ that $n \implies n + 1$ (here $n - 1\in \ZZ/5$):
\begin{enumerate}
\item Let $X$ be nonvanishing and let $V(x) = X_x/||X_x||$. Then $V(x) \perp x$ since $x$ is a normal vector and $V(x)$ is a tangent vector.
\item Let $H(x, t) = \cos(2\pi t) x + \sin(2\pi t)V(x)$.
\item Homotopies preserve orientation.
\item $\alpha$ is a composite of $n+1$ reflections.
\item Let $X_x = (-x_1, x_0, \dots, -x_{2k-1}, x_{2k})$ if $n = 2k - 1$.
\end{enumerate}

\begin{exer}[16.9]
Let $\omega$ be the $n-1$-form on $(\RR^n)^*$ defined by
$$\omega = |x|^{-n} \sum_{i=1}^n (-1)^{i-1} \bigwedge_{j \neq i} dx^j.$$
Let $\iota: S^{n-1} \to (\RR^n)^*$.
Show that $\iota^*\omega$ is the Riemannian top form of $S^{n-1}$.
Show that $\omega$ is closed but not exact.
\end{exer}

Let $\nu$ be the unit normal field of $S^{n-1}$, which exists since $\codim S^{n-1} = 1$.
Let $dA$ be the Riemannian top form of $S^{n-1}$ and $dV$ the top form of $\RR^n$. Then
$$dA = \iota^*(\nu \iprod dV)$$
but $\nu(x) = x$, $dV = \bigwedge_j dx^j$, and we need to commute $\partial_{x_i}$ with $i-1$ exterior factors to cancel it, so
$$dA = \sum_{i=1}^n x_i \partial_{x_i} \iprod dV = \sum_{i=1}^n (-1)^{i-1} \bigwedge_{j \neq i} dx^j = \iota^*\omega.$$
Since $\iota^*\omega$ is a top form it is closed, thus $\omega$ is closed as $[d, \iota^*] = 0$.
But
$$0 \neq \int_{S^{n-1}} \iota^*\omega$$
since $\iota^*\omega$ is a Riemannian top form and $S^{n-1}$ is a compact manifold without boundary, so $\iota^*\omega$ is not exact; thus neither is $\omega$.

\begin{exer}[16.17]
Let $M$ be a compact connected Riemannian manifold with nonempty boundary.
Show that $u \in C^\infty(M)$ is harmonic iff $u$ minimizes $||\nabla u||_{L^2(M)}$ among all smooth functions with the same boundary data.
\end{exer}

Assume that $\Delta u = 0$ and $w$ has the same boundary data as $u$. Then
$$0 = \int_M \Delta uw.$$
Integrating by parts,
$$0 = \int_M \langle \nabla u, \nabla w\rangle,$$
where there is no boundary term since $w$ has the same boundary data as $u$.
Then
$$\int_M |\nabla u|^2 = \int_M \langle \nabla u, \nabla w\rangle \leq \frac{1}{2}(||\nabla u||_{L^2(M)}^2 + ||\nabla w||_{L^2(M)}^2).$$
This implies that $u$ is a minimizer.

If $u$ is a minimizer, let $f(t) = ||\nabla(u + tv)||_{L^2(M)}$ where $v$ is any test function. Then $f'(0) = 0$, but
$$f(t) = \int_M |\nabla u|^2 + t\langle \nabla u, \nabla v\rangle + \frac{t^2}{2}|\nabla v|^2.$$
So
$$0 = f'(t) = \int_M \Delta uv$$
but $v$ was arbitrary so this implies $\Delta u = 0$.

\begin{exer}[16.18]
Let $M$ be an oriented Riemannian manifold of dimension $n$.
\begin{enumerate}
\item Show that there is a unique inner product on $k$-covectors satisfying
$$\langle \omega^1 \wedge \cdots \wedge \omega^k, \eta^1 \wedge \cdots \wedge \eta^k\rangle = \det((\langle (\omega^i)^\sharp, (\eta^j)^\sharp)_{ij}).$$
\item Show that the Riemannian volume form $dV$ is the unique unit-length positively oriented top form.
\item Show that there is a unique smooth morphism of bundles $*$, the Hodge star, from $k$-forms to $n-k$-forms satisfying
$$\omega \wedge *\eta = \langle \omega, \eta\rangle ~dV.$$
\item Show that $*$ acts on $0$-forms by $*f = f~dV$.
\item Show that $**$ acts on $k$-forms by $**\omega = (-1)^{k(n-k)}\omega$.
\end{enumerate}
\end{exer}

We induct on $k$. When $k = 1$ this is trivial. Assume we have an inner product on $k-1$-forms; then consider the cofactor expansion
$$\det((\langle (\omega^i)^\sharp, (\eta^j)^\sharp)_{ij}) = \sum_{i^*=1}^k (-1)^{i^* - 1} \omega^{i^*}\det((\langle (\omega^i)^\sharp, (\eta^j)^\sharp)_{\substack{i\neq i^*\\j \neq 1}}).$$
Each of the summands above is an inner product by assumption, and a sum of inner products is an inner product.
This extends by linearity to all $k$-covectors, giving uniqueness.

We now compute $||dV||^2$ with respect to this inner product. In coordinates, $dV = \sqrt{|\det g|} \bigwedge_i dx^i$, so
$$||dV||^2 = \langle dV, dV\rangle = |\det g| \det((\langle (dx^i)^\sharp, (dx^j)^\sharp)_{ij}) = \frac{|\det g|}{\det g} \det((\langle \partial_i, \partial_j)_{ij}) = \det 1 = 1.$$
Conversely if $\omega$ is a unit-length positively oriented top form, then $\omega$ is $f~\bigwedge_i dx^i$ in coordinates, where $|f| = \sqrt{|\det g|}$ and $f$ and $\det g$ have the same sign. So $\omega = dV$.

We first define the Hodge star on the basic $k$-forms $dx^I$ where $dx$ is an orthonormal coframe, where $I$ ascends with length $k$, by letting $*I$ ascend with length $n-k$ and entries exactly those not in $I$, and $*(dx^I) = (-1)^{\sgn(I*I)} dx^{*I}$ where $I*I$ is the permutation $i_1i_2 \cdots i_k (*i)_1 (*i)_2 \cdots (*i)_{n-k}$ where $I = (i_1, \dots, i_k)$ and $*I = ((*i)_1, \dots, (*i)_{n-k})$.
Then we define $*(f~dx^I) = f(*dx^I)$, and glue everything together with a partition of unity to get a morphism of bundles.
Locally,
$$\omega \wedge *(f~dx^I) = f\omega \wedge dx^{*I}.$$
So it suffices to check the axiomatic definition of $*$ when $\eta = dx^I$.
Similarly we may assume $\eta = dx^J$. Here $I,J$ ascend and $dx$ is an orthonormal coframe.
$$dx^J \wedge dx^{*I} = \sigma ~\bigwedge_i dx^i$$
where $\sigma$ is a sign, equal to $(-1)^m$ where $m$ is the number of swaps needed to reorder the concatenation $J*I$ to ascend.
Then $\sigma$ is also the sign that appears in the definition of the cofactor expansion. Also $\langle f, g\rangle = \sqrt{|\det g|}fg$. So this was as desired.

For uniqueness of the Hodge star, we compute $*(dx^I)$ from the axiomatic definition, assuming that $I$ ascends and $dx$ is an orthonormal coframe.
If $J$ also ascends then
$$dx^J \wedge *(dx^I) = \langle dx^J, dx^I\rangle ~dV = \det((\langle \partial_{j_\ell}, \partial_{i_m}\rangle)_{\ell m}) ~dV = \sigma \bigwedge_i dx^i$$
where $\sigma$ is $(-1)^s$ where $s$ the number of swaps needed to reorder $J*I$ to ascend. This was as desired.

Now we compute $*f$. We can check when $f = 1$, and indeed $*1 = dx^{*\Box} = dV$ where $\Box$ is the empty multiindex.

Now we compute $**$. Indeed,
$$\langle \omega, \eta\rangle~dV = \langle *\omega, *\eta\rangle~dV = *\omega \wedge **\eta = (-1)^{k(n-k)} **\eta \wedge *\omega$$
since we constructed $*$ to preserve orthonormal coframes (hence $*$ is unitary), $**\eta$ is a $k$-form and $*\omega$ is an $n-k$-form.
This implies
$$\langle \omega, \eta\rangle~dV = (-1)^{k(n-k)}\langle \omega, **\eta\rangle ~dV$$
so $**\eta = (-1)^{k(n-k)}\eta$. Therefore $** = (-1)^{k(n-k)}$.

\begin{exer}[16.22]
Introduce the codifferential $d^*$ which maps $k$-forms to $k-1$-forms by $d^* = (-1)^{n(k+1)+1}*d*$.
\begin{enumerate}
\item Show that $(d^*)^2 = 0$.
\item Show that
$$(\omega, \eta) = \int_M \langle \omega, \eta\rangle ~dV$$
defines an inner product on $k$-forms.
\item Show that $(d^*\omega, \eta) = (\omega, d\eta)$ whenever $\omega$ is a $k$-form and $\eta$ is a $k-1$-form.
\end{enumerate}
\end{exer}

First, one has
$$(d^*)^2 = *d**d* = (-1)^{k(n-k)} *dd* = 0.$$
That $(\cdot, \cdot)$ is an inner product follows from the definition of $L^2$ for Hilbert space-valued functions.
Also
$$(d^*\omega, \eta) = \int_M \langle d^*\omega, \eta\rangle ~dV = \int_M d^*\omega \wedge *\eta = \int_M *d^*\omega \wedge \eta.$$
Now
$$*d^* = (-1)^{n(k+1)+1}**d* = (-1)^{(n(k+1)+1)k(n+k)} d*.$$
The expansion of $(n(k+1)+1)k(n+k)$ has six terms, so is even, so $*d^* = d*$. But
$$(\omega, d\eta) = \int_M \langle \omega, d\eta\rangle~dV = \int_M \omega \wedge d\eta = \int_M d*\omega \wedge \eta = (d^*\omega, \eta).$$


\end{document}
