\documentclass[reqno,12pt,letterpaper]{amsart}
\RequirePackage{amsmath,amssymb,amsthm,graphicx,mathrsfs,url}
\RequirePackage[usenames,dvipsnames]{color}
\RequirePackage[colorlinks=true,linkcolor=Red,citecolor=Green]{hyperref}
\RequirePackage{amsxtra}
\usepackage{cancel}
\usepackage{tikz-cd}

\setlength{\textheight}{9.00in} \setlength{\oddsidemargin}{-0.25in}
\setlength{\evensidemargin}{-0.25in} \setlength{\textwidth}{6.58in}
\setlength{\topmargin}{-0.25in} \setlength{\headheight}{0.00in}
\setlength{\marginparwidth}{0.5in}
\setlength{\abovedisplayskip}{0.2in}
\setlength{\belowdisplayskip}{0.2in}
\setlength{\parskip}{0.05in}
\renewcommand{\baselinestretch}{1.10}

\title{Midterm 1 practice problems}
\author{Aidan Backus}
\date{October 2021}

\newcommand{\NN}{\mathbf{N}}
\newcommand{\ZZ}{\mathbf{Z}}
\newcommand{\QQ}{\mathbf{Q}}
\newcommand{\RR}{\mathbf{R}}
\newcommand{\CC}{\mathbf{C}}
\newcommand{\DD}{\mathbf{D}}
\newcommand{\PP}{\mathbf P}
\newcommand{\MM}{\mathbf M}
\newcommand{\II}{\mathbf I}
\newcommand{\Hyp}{\mathbf H}

\DeclareMathOperator{\card}{card}
\DeclareMathOperator{\cent}{center}
\DeclareMathOperator{\ch}{ch}
\DeclareMathOperator{\codim}{codim}
\DeclareMathOperator{\diag}{diag}
\DeclareMathOperator{\diam}{diam}
\DeclareMathOperator{\dom}{dom}
\DeclareMathOperator{\Gal}{Gal}
\DeclareMathOperator{\Hom}{Hom}
\DeclareMathOperator{\Jac}{Jac}
\DeclareMathOperator{\Lip}{Lip}
\DeclareMathOperator{\Met}{Met}
\DeclareMathOperator{\id}{id}
\DeclareMathOperator{\rad}{rad}
\DeclareMathOperator{\rank}{rank}
\DeclareMathOperator{\Radon}{Radon}
\DeclareMathOperator*{\Res}{Res}
\DeclareMathOperator{\sgn}{sgn}
\DeclareMathOperator{\singsupp}{sing~supp}
\DeclareMathOperator{\Spec}{Spec}
\DeclareMathOperator{\supp}{supp}
\DeclareMathOperator{\Tan}{Tan}
\newcommand{\tr}{\operatorname{tr}}

\newcommand{\Ric}{\mathrm{Ric}}
\newcommand{\Riem}{\mathrm{Riem}}

\newcommand{\dbar}{\overline \partial}

\DeclareMathOperator{\arcosh}{arcosh}
\DeclareMathOperator{\atanh}{atanh}
\DeclareMathOperator{\csch}{csch}
\DeclareMathOperator{\sech}{sech}

\DeclareMathOperator{\Div}{div}
\DeclareMathOperator{\grad}{grad}
\DeclareMathOperator{\Ell}{Ell}
\DeclareMathOperator{\WF}{WF}

\newcommand{\Hilb}{\mathcal H}
\newcommand{\normal}{\mathbf n}
\newcommand{\vol}{\mathrm{vol}}

\newcommand{\pic}{\vspace{30mm}}
\newcommand{\dfn}[1]{\emph{#1}\index{#1}}

\renewcommand{\Re}{\operatorname{Re}}
\renewcommand{\Im}{\operatorname{Im}}


\newtheorem{theorem}{Theorem}[section]
\newtheorem{badtheorem}[theorem]{``Theorem"}
\newtheorem{prop}[theorem]{Proposition}
\newtheorem{lemma}[theorem]{Lemma}
\newtheorem{claim}[theorem]{Claim}
\newtheorem{proposition}[theorem]{Proposition}
\newtheorem{corollary}[theorem]{Corollary}
\newtheorem{conjecture}[theorem]{Conjecture}
\newtheorem{axiom}[theorem]{Axiom}

\theoremstyle{definition}
\newtheorem{definition}[theorem]{Definition}
\newtheorem{remark}[theorem]{Remark}
\newtheorem{example}[theorem]{Example}
\newtheorem{notation}[theorem]{Notation}
\newtheorem{assumption}[theorem]{Assumption}

\newtheorem{exercise}[theorem]{Discussion topic}
\newtheorem{homework}[theorem]{Homework}
\newtheorem{problem}[theorem]{Problem}

\newtheorem{ack}{Acknowledgements}

\numberwithin{equation}{section}


% Mean
\def\Xint#1{\mathchoice
{\XXint\displaystyle\textstyle{#1}}%
{\XXint\textstyle\scriptstyle{#1}}%
{\XXint\scriptstyle\scriptscriptstyle{#1}}%
{\XXint\scriptscriptstyle\scriptscriptstyle{#1}}%
\!\int}
\def\XXint#1#2#3{{\setbox0=\hbox{$#1{#2#3}{\int}$ }
\vcenter{\hbox{$#2#3$ }}\kern-.6\wd0}}
\def\ddashint{\Xint=}
\def\dashint{\Xint-}

%\usepackage{color}
%\hypersetup{%
%    colorlinks=true, % make the links colored%
%    linkcolor=blue, % color TOC links in blue
%    urlcolor=red, % color URLs in red
%    linktoc=all % 'all' will create links for everything in the TOC
%Ning added hyperlinks to the table of contents 6/17/19
%}

% style=alphabetic
\usepackage[backend=bibtex,maxcitenames=50,maxnames=50]{biblatex}
%\addbibresource{topics.bib}
\renewbibmacro{in:}{}
\DeclareFieldFormat{pages}{#1}

\begin{document}
%\begin{abstract}
%Practice for Midterm 1.
%x\end{abstract}

%\maketitle

%%%%%%%%%%%%%%%%%%%%%%%%%%%%%%%%%%%%%%%%%%%%%%%%%%%%%%%

% \tableofcontents

A lot of these problems were adapted from Paul's Online Math Notes.

\section{Antiderivatives}
\begin{enumerate}
\item Compute $\int 4x^6 - 2x^3 + 7x - 4 ~dx$.
\item Compute $\int 4x^3 + 10x^{-3} + 12x^{-9} ~dx$.
\item Compute $\int \sqrt{x^7} - 7\sqrt[6]{x^5} + 17\sqrt[3]{x^{10}} ~dx$.
\item Compute $\int \frac{x^4 - \sqrt[3]{x}}{6 \sqrt x} ~dx$.
\item Compute $\int 2\cos x - \sec x \tan x ~dx$.
\item Compute $\int \frac{7 - 6\sin^2 x}{\sin^2 x} ~dx$.
\item Suppose that $f'(x) = 12 + \csc x(\sin x + \csc x) ~dx$ and $f(0) = 0$. What is $f$?
\item Suppose that $f''(x) = 24x^2 - 48x + 2$, $f(1) = -9$, and $f(-2) = -4$. What is $f$?
\item Let $y(t)$ denote the altitude of a freefalling object at time $t$, and let $g \approx 9.8 m/s^2$ denote the gravitational field at sea level. According to \dfn{Galileo's law of falling bodies},\footnote{Newton introduced calculus in 1666 in order to, among other things, generalize Galileo's law of falling bodies to show that the orbit of a planet around the Sun is a conic section.}
$$y''(t) = -g.$$
Show that if I drop an object from height $y = y_0$, it will take time $\sqrt{2y_0/g}$ for the object to hit the ground $y = 0$.
\item Suppose that I have a fresh Petri dish, and I introduce $1000$ bacteria at time $0$ to the Petri dish. Let $f(t)$ denote the number of bacteria growing in the Petri dish at time $t$. One can show that if $t > 0$ is small then $f'(t) = kf(t)$ where $k > 0$ is a constant that depends on the species' biology. What is $f(t)$ if $t$ is small?
\end{enumerate}

\section{Integrals}
Do the following problems without using the fundamental theorem.

\begin{enumerate}
\item Suppose that we have a piecewise function $f$ where $f = g$ on $[0, 1]$, $f = h$ on $(1, 2]$, $\int_0^1 g(x) ~dx = 6$, and $\int_1^2 h(x) ~dx = 12$. Compute $\int_0^2 f(x) ~dx$.
\item Compute $\int_0^{2\pi} \sin x ~dx$.\footnote{Hint: Please don't actually write out a Riemann sum here. I respect you too much to ask you to do that. Think about what happens to $\sin x$ when $x = \pi$.}
\item The \dfn{mean}, or \dfn{average}, $\dashint_a^b f(x) ~dx$ of a function $f$ on $[a, b]$ is
$$\dashint_a^b f(x) ~dx = \frac{1}{b - a} \int_a^b f(x) ~dx.$$
Compute the mean of $f(x) = x$ on $[0, 1]$.\footnote{Hint: What does the graph of $f$ look like?}
\item Compute $\sum_{j=1}^5 j^2$.
\item Compute $\int_0^2 x^2 + 1 ~dx$ using the following method. Let $\Delta x = 1/n$ and write down a Riemann sum $R(n)$ with mesh size $\Delta x$. Then take the limit of $R(n)$ as $n \to \infty$.
\end{enumerate}

\section{The fundamental theorem}
\begin{enumerate}
\item Compute $\int_1^6 12x^3 - 9x^2 + 2 ~dx$.
\item Compute $\int_{-2}^5 5x^2 - 7x + 3 ~dx$.
\item Compute $\int_0^3 15x^4 - 13x^2 + w ~dx$.
\item Compute $\int_{\pi/6}^{\pi/3} 2 \sec^2 x - 8 \csc x \cot x ~dx$.
\item Compute
$$\frac{d}{dx} \int_{\tan x}^0 \frac{dt}{1 + t^2}.$$
\item Compute
$$\lim_{x \to \infty} \int_1^x \frac{dt}{\sqrt x}.$$
\item Compute $\int_0^1 dx/x$, if possible.
\item Compute $\int_{0.5}^1 dx/x$, if possible.
\item In statistics, the \dfn{standard normal distribution} $\mathscr N(x)$ is the derivative of $e^{-x^2/2}$. Compute its \dfn{expected value}, which by definition is the limits of the means
$$\lim_{N \to +\infty} \int_{-N}^N \mathscr N(x) ~dx.$$
\end{enumerate}

\section{Substitution}
\begin{enumerate}
\item Compute $\int \sec(1 - x) \tan(1 - x) ~dx$.
\item Compute $\int e^{1 + \tan x} \sec^2 x ~dx$.
\item Compute
$$\int \frac{4x + 3}{4x^2 + 6x - 1} ~dx.$$
\item Compute $\int_{e^2}^{e^6} (\ln x)^4 x^{-1} ~dx$, if possible.
\item Compute $\int_{-5}^5 4x/(2 - x^2) ~dt$, if possible.
\item Compute $\int_{-\pi}^{\pi/2} \cos x \cos(\sin x)$, if possible.
\item In economics, the \dfn{Lorenz function} of society is a function $L$ on $[0, 1]$ defined by letting $x$ denote a proportion of the population and $L(x)$ the proportion of the wealth they have. For example, if the bottom $99$ percent only control $50$ percent of the wealth, then $L(.99) = .5$. The \dfn{Gini coefficient} is defined by
$$G = 2 \int_0^1 x - L(x) ~dx$$
and measures how much inequality society has: if $G = 0$ then everyone has an equal amount of wealth and if $G = 1$ then a single individual holds all the wealth.
Suppose that
$$L(x) = 1 - \sqrt{1 - x^2}$$
and compute the Gini coefficient.
\end{enumerate}


\end{document}
