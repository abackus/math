\documentclass[reqno,12pt,letterpaper]{amsart}
\RequirePackage{amsmath,amssymb,amsthm,graphicx,mathrsfs,url}
\RequirePackage[usenames,dvipsnames]{color}
\RequirePackage[colorlinks=true,linkcolor=Red,citecolor=Green]{hyperref}
\RequirePackage{amsxtra}
\usepackage{cancel}
\usepackage{tikz-cd}

\setlength{\textheight}{8.50in} \setlength{\oddsidemargin}{0.00in}
\setlength{\evensidemargin}{0.00in} \setlength{\textwidth}{6.08in}
\setlength{\topmargin}{0.00in} \setlength{\headheight}{0.18in}
\setlength{\marginparwidth}{1.0in}
\setlength{\abovedisplayskip}{0.2in}
\setlength{\belowdisplayskip}{0.2in}
\setlength{\parskip}{0.05in}
\renewcommand{\baselinestretch}{1.10}

\title[Word problems]{Application and culture-oriented word problems for undergraduate mathematics}
\author{Aidan Backus}
\date{October 2021}

\newcommand{\NN}{\mathbf{N}}
\newcommand{\ZZ}{\mathbf{Z}}
\newcommand{\QQ}{\mathbf{Q}}
\newcommand{\RR}{\mathbf{R}}
\newcommand{\CC}{\mathbf{C}}
\newcommand{\DD}{\mathbf{D}}
\newcommand{\PP}{\mathbf P}
\newcommand{\MM}{\mathbf M}
\newcommand{\II}{\mathbf I}
\newcommand{\Hyp}{\mathbf H}

\DeclareMathOperator{\card}{card}
\DeclareMathOperator{\cent}{center}
\DeclareMathOperator{\ch}{ch}
\DeclareMathOperator{\codim}{codim}
\DeclareMathOperator{\diag}{diag}
\DeclareMathOperator{\diam}{diam}
\DeclareMathOperator{\dom}{dom}
\DeclareMathOperator{\Gal}{Gal}
\DeclareMathOperator{\Hom}{Hom}
\DeclareMathOperator{\Jac}{Jac}
\DeclareMathOperator{\Lip}{Lip}
\DeclareMathOperator{\Met}{Met}
\DeclareMathOperator{\id}{id}
\DeclareMathOperator{\rad}{rad}
\DeclareMathOperator{\rank}{rank}
\DeclareMathOperator{\Radon}{Radon}
\DeclareMathOperator*{\Res}{Res}
\DeclareMathOperator{\sgn}{sgn}
\DeclareMathOperator{\singsupp}{sing~supp}
\DeclareMathOperator{\Spec}{Spec}
\DeclareMathOperator{\supp}{supp}
\DeclareMathOperator{\Tan}{Tan}
\newcommand{\tr}{\operatorname{tr}}

\newcommand{\Ric}{\mathrm{Ric}}
\newcommand{\Riem}{\mathrm{Riem}}

\newcommand{\dbar}{\overline \partial}

\DeclareMathOperator{\atanh}{atanh}
\DeclareMathOperator{\csch}{csch}
\DeclareMathOperator{\sech}{sech}

\DeclareMathOperator{\Div}{div}
\DeclareMathOperator{\grad}{grad}
\DeclareMathOperator{\Ell}{Ell}
\DeclareMathOperator{\WF}{WF}

\newcommand{\Hilb}{\mathcal H}
\newcommand{\normal}{\mathbf n}
\newcommand{\vol}{\mathrm{vol}}

\newcommand{\pic}{\vspace{30mm}}
\newcommand{\dfn}[1]{\emph{#1}\index{#1}}

\renewcommand{\Re}{\operatorname{Re}}
\renewcommand{\Im}{\operatorname{Im}}


\newtheorem{theorem}{Theorem}[section]
\newtheorem{badtheorem}[theorem]{``Theorem"}
\newtheorem{prop}[theorem]{Proposition}
\newtheorem{lemma}[theorem]{Lemma}
\newtheorem{claim}[theorem]{Claim}
\newtheorem{proposition}[theorem]{Proposition}
\newtheorem{corollary}[theorem]{Corollary}
\newtheorem{conjecture}[theorem]{Conjecture}
\newtheorem{axiom}[theorem]{Axiom}

\theoremstyle{definition}
\newtheorem{definition}[theorem]{Definition}
\newtheorem{remark}[theorem]{Remark}
\newtheorem{example}[theorem]{Example}
\newtheorem{notation}[theorem]{Notation}
\newtheorem{assumption}[theorem]{Assumption}

\newtheorem{exercise}[theorem]{Discussion topic}
\newtheorem{homework}[theorem]{Homework}
\newtheorem{problem}[theorem]{Problem}

\newtheorem{ack}{Acknowledgements}

\numberwithin{equation}{section}


% Mean
\def\Xint#1{\mathchoice
{\XXint\displaystyle\textstyle{#1}}%
{\XXint\textstyle\scriptstyle{#1}}%
{\XXint\scriptstyle\scriptscriptstyle{#1}}%
{\XXint\scriptscriptstyle\scriptscriptstyle{#1}}%
\!\int}
\def\XXint#1#2#3{{\setbox0=\hbox{$#1{#2#3}{\int}$ }
\vcenter{\hbox{$#2#3$ }}\kern-.6\wd0}}
\def\ddashint{\Xint=}
\def\dashint{\Xint-}

%\usepackage{color}
%\hypersetup{%
%    colorlinks=true, % make the links colored%
%    linkcolor=blue, % color TOC links in blue
%    urlcolor=red, % color URLs in red
%    linktoc=all % 'all' will create links for everything in the TOC
%Ning added hyperlinks to the table of contents 6/17/19
%}

% style=alphabetic
\usepackage[backend=bibtex,maxcitenames=50,maxnames=50]{biblatex}
%\addbibresource{topics.bib}
\renewbibmacro{in:}{}
\DeclareFieldFormat{pages}{#1}

\begin{document}
\begin{abstract}
In this document I want to give some examples of applied problems in undergraduate mathematics.
I tend to get annoyed with problems involving a ladder sliding down a building, or more generally problems that are not tailored to the interests of the students taking the class in question.
This includes applications, but also issues in popular culture.
\end{abstract}

\maketitle

%%%%%%%%%%%%%%%%%%%%%%%%%%%%%%%%%%%%%%%%%%%%%%%%%%%%%%%

\tableofcontents

\section{Limits}
In general, limits, especially limits at infinity, arise naturally when we are interested in some quantity $x$ that is very large, for example:
\begin{enumerate}
\item $x$ is the number of cells in a macroorganism.
\item $x$ is the number of particles in an ideal gas.
\item $x$ is the amount of data fed into an algorithm.
\item $x$ is the number of agents in society.
\end{enumerate}

\subsection{Runtime analysis of sorting problems}
Suppose we are software engineers and are given a list of numbers, and want to sort them in increasing order.
Thus, for example, if we are given $1,3,2,9,5,4$ we want to output $1,2,3,4,5,9$.
We will assume that the length $x$ of the list is very large.
There are two (among many other) popular \emph{sorting algorithms} in the literature:
\begin{enumerate}
\item \emph{Insertion sort}, which can be assumed to take $x^2$ steps.
\item \emph{Merge sort}, which can be assumed to take $15x \log_2 x$ steps\footnote{The constant $15$ here is a little BS, but the point is that it is much larger than the constant on insertion sort, which we just normalized to $1$.}.
\end{enumerate}
Which should we use?

Well, we're interested in the ratio of the numbers of steps for insertion sort to the number of steps for merge sort, which is $x^2/(15x \log_2 x)$.
And since $x$ is large, we're actually interested in the limit of this quantity as $x \to \infty$, that is, we're interested in
$$L = \lim_{x \to \infty} \frac{x^2}{15x \log_2 x}.$$
If $L > 1$ then for $x$ very large, insertion sort takes more steps, and we should use merge sort; if $L < 1$, then for $x$ very large, merge sort takes more steps and we should use insertion sort.

In fact,
$$L = \frac{1}{15} \lim_{x \to \infty} \frac{x}{\log_2 x}$$
which can be seen to be $+\infty$, owing to the fact that when we double $x$, $\log_2 x$ only increases by $1$\footnote{You can prove this more rigorously using l'H\^opital's rule, a tool that probably has not been covered at this point in a calculus course.}.

\subsection{Extinction of epidemic viruses}
Recall that the \emph{basic reproduction number} $\mathcal R_0$ of an epidemic is the number of persons infected by a newly infected person.
Suppose that $\mathcal R_0 < 1$ (thus the virus doesn't spread very well) and the first generation of the virus infected $10^6$ people. Will the epidemic eventually die out, even if we don't come up with any medicine for it?

Yes! The second generation of the virus will infect $\mathcal R_0 10^6$ people, the third generation will infect ${\mathcal R_0}^2 10^6$ people, and in general the $x$th generation will infect ${\mathcal R_0}^{x - 1} \cdot 10^6$ people.
Therefore as the time $x$ goes to infinity, the number of people that will be infected at time $x$ is
$$\lim_{x \to \infty} {\mathcal R_0}^{x - 1} \cdot 10^6 = \frac{10^6}{\mathcal R_0} \lim_{x \to \infty} {\mathcal R_0}^x = 0.$$

I worked in a lab in San Diego for a summer studying a model of the spread of HIV, and one thing I studied was under what conditions could we get $\mathcal R_0 < 1$ and conclude that the epidemic would wipe itself out.
The model we presented here is really oversimplified because it assumes that you can break up time into discrete generations, and it assumes that $\mathcal R_0$ is unchanging.
In practice a virus that spreads really fast runs out of people to infect, tanking $\mathcal R_0$, for example\footnote{The converse to this was an early theory (at the time of writing I do not know if it was validated for evidence) why COVID-19 is so dangerous: it spread \emph{asymptomatically}, so it didn't kill off the people it needed to be active in public in order to spread.}.

\section{Related rates}
Whenever we have a interesting formula $\mathscr F$, it's reasonable to take the derivative with respect to time of both sides of $\mathscr F$.
This is the essence of \emph{any} related rates problem: $\mathscr F$ could be the ideal gas law, the formula for the luminosity of the Milky Way, or yes, the angle that a ladder makes as it slides down a wall.

\subsection{The ideal gas law}
In high school chemistry we learn that $n$ mol of an ideal gas in a container of volume $V$ under pressure $P$ and at temperature $T$ satisfies
$$PV = nRT$$
where $R \approx 8.3 J/(\text{mol}\cdot K)$.

Suppose that we are designing a pressure cooker. We want to know how fast it is heating up.
Assume that the air in the cooker is an ideal gas, there are $50$ mol of air in the cooker, and the cooker is not leaky.
Suppose that right now, the pressure is $10^6 J/L$ and is increasing at a rate of $10^5 J/(L \cdot \text{min})$.
Similarly the volume is $1 L$ and is decreasing at a rate of $.07 L/\text{min}$, as the cooker squishes the gas inside it.

Differentiating the ideal gas law in time we get
$$(PV)' = (nRT)'.$$
According to Lebiniz' product rule this is
$$P'V + PV' = n'RT + nRT'$$
since $R' = 0$ (the laws of the universe are not changing).
If our pressure cooker does not have any leaks, then the amount of gas going in or out is given by $n' = 0$.
Thus we have
$$P'V + PV' = nRT'$$
or in other words
$$T' = \frac{1}{nR}(P'V + PV').$$
We could also obtain this by solving the ideal gas law for $T$ and then differentiating but this way turns out to generalize more nicely.

Plugging everything in, we get
$$T' = \frac{1 K}{50 \cdot 8.3 J}(10^5 J/\text{min} -7 \cdot 10^4 J/\text{min}) = \frac{1 K}{415 J}(3 \cdot 10^5 J/\text{min}) \approx 723 \frac{K}{\text{min}},$$
so this pressure cooker is heating up really fast.

\section{l'H\^opital's rule}
\subsection{Mean Girls}
Before coming to class, watch the climactic scene of the famous Lindsay Lohan movie \emph{Mean Girls} at \url{https://www.youtube.com/watch?v=oDAKKQuBtDo}.
Then see if you could find a way to prove that the limit in that scene does not exist without using l'H\^opital's rule\footnote{I couldn't do it without using the small-angle formula, in any case.}.
I'm not actually sure how Lohan's character solved this problem so quickly, given that she apparently didn't even remember what a limit was going into the scene; maybe having an epiphany about how you've been horribly mistreating your friends powers up your math ability, Shounen Battle Anime-style, and I just have never experienced the Power of Friendship while working on research.

Anyways, the limit in question is
$$L = \lim_{x \to 0} \frac{\log(1 - x) - \sin x}{1 - \cos^2 x}.$$
By the Pythagorean theorem, we can simplify this
$$L = \lim_{x \to 0} \frac{\log(1 - x) - \sin x}{\sin^2 x} = \lim_{x \to 0} \frac{\log(1 - x)}{\sin^2 x} - \lim_{x \to 0} \csc x.$$
Now $\lim_{x \to 0\pm} \csc x = \pm\infty$.
Here we worry that the logarithm term will simplify to $\pm \infty$ as well, in which case $L$ would have indeterminate form $\infty - \infty$ on each side.
But $\lim_{x \to 0} \log(1-x)/\sin^2 x$ has indeterminate form $0/0$, so by l'H\^opital's rule,
$$\lim_{x \to 0\pm} \frac{\log(1 - x)}{\sin^2 x} = \lim_{x \to 0\pm} \frac{-(1 - x)^{-1}}{2 \sin x \cos x} = -0.5 \frac{\lim_{x \to 0\pm} 1}{\lim_{x \to 0\pm} \sin x \cos x} = -(\pm \infty).$$
Thus
$$\lim_{x \to 0\pm} \frac{\log(1 - x) - \sin x}{1 - \cos^2 x} = - (\pm \infty)$$
and so $L$ does not exist.


\printbibliography


\end{document}
