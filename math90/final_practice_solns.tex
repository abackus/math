\documentclass[reqno,12pt,letterpaper]{amsart}
\RequirePackage{amsmath,amssymb,amsthm,graphicx,mathrsfs,url}
\RequirePackage[usenames,dvipsnames]{color}
\RequirePackage[colorlinks=true,linkcolor=Red,citecolor=Green]{hyperref}
\RequirePackage{amsxtra}
\usepackage{cancel}
\usepackage{tikz-cd}

\setlength{\textheight}{9.00in} \setlength{\oddsidemargin}{-0.25in}
\setlength{\evensidemargin}{-0.25in} \setlength{\textwidth}{6.58in}
\setlength{\topmargin}{-0.25in} \setlength{\headheight}{0.00in}
\setlength{\marginparwidth}{0.5in}
\setlength{\abovedisplayskip}{0.2in}
\setlength{\belowdisplayskip}{0.2in}
\setlength{\parskip}{0.05in}
\renewcommand{\baselinestretch}{1.10}

\title{Midterm 1 practice problems}
\author{Aidan Backus}
\date{October 2021}

\newcommand{\NN}{\mathbf{N}}
\newcommand{\ZZ}{\mathbf{Z}}
\newcommand{\QQ}{\mathbf{Q}}
\newcommand{\RR}{\mathbf{R}}
\newcommand{\CC}{\mathbf{C}}
\newcommand{\DD}{\mathbf{D}}
\newcommand{\PP}{\mathbf P}
\newcommand{\MM}{\mathbf M}
\newcommand{\II}{\mathbf I}
\newcommand{\Hyp}{\mathbf H}

\DeclareMathOperator{\card}{card}
\DeclareMathOperator{\cent}{center}
\DeclareMathOperator{\ch}{ch}
\DeclareMathOperator{\codim}{codim}
\DeclareMathOperator{\diag}{diag}
\DeclareMathOperator{\diam}{diam}
\DeclareMathOperator{\dom}{dom}
\DeclareMathOperator{\Gal}{Gal}
\DeclareMathOperator{\Hom}{Hom}
\DeclareMathOperator{\Jac}{Jac}
\DeclareMathOperator{\Lip}{Lip}
\DeclareMathOperator{\Met}{Met}
\DeclareMathOperator{\id}{id}
\DeclareMathOperator{\rad}{rad}
\DeclareMathOperator{\rank}{rank}
\DeclareMathOperator{\Radon}{Radon}
\DeclareMathOperator*{\Res}{Res}
\DeclareMathOperator{\sgn}{sgn}
\DeclareMathOperator{\singsupp}{sing~supp}
\DeclareMathOperator{\Spec}{Spec}
\DeclareMathOperator{\supp}{supp}
\DeclareMathOperator{\Tan}{Tan}
\newcommand{\tr}{\operatorname{tr}}

\newcommand{\Ric}{\mathrm{Ric}}
\newcommand{\Riem}{\mathrm{Riem}}

\newcommand{\dbar}{\overline \partial}

\DeclareMathOperator{\arcosh}{arcosh}
\DeclareMathOperator{\atanh}{atanh}
\DeclareMathOperator{\arcsec}{arcsec}
\DeclareMathOperator{\csch}{csch}
\DeclareMathOperator{\sech}{sech}

\DeclareMathOperator{\Div}{div}
\DeclareMathOperator{\grad}{grad}
\DeclareMathOperator{\Ell}{Ell}
\DeclareMathOperator{\WF}{WF}

\newcommand{\Hilb}{\mathcal H}
\newcommand{\normal}{\mathbf n}
\newcommand{\vol}{\mathrm{vol}}

\newcommand{\pic}{\vspace{30mm}}
\newcommand{\dfn}[1]{\emph{#1}\index{#1}}

\renewcommand{\Re}{\operatorname{Re}}
\renewcommand{\Im}{\operatorname{Im}}


\newtheorem{theorem}{Theorem}[section]
\newtheorem{badtheorem}[theorem]{``Theorem"}
\newtheorem{prop}[theorem]{Proposition}
\newtheorem{lemma}[theorem]{Lemma}
\newtheorem{claim}[theorem]{Claim}
\newtheorem{proposition}[theorem]{Proposition}
\newtheorem{corollary}[theorem]{Corollary}
\newtheorem{conjecture}[theorem]{Conjecture}
\newtheorem{axiom}[theorem]{Axiom}

\theoremstyle{definition}
\newtheorem{definition}[theorem]{Definition}
\newtheorem{remark}[theorem]{Remark}
\newtheorem{example}[theorem]{Example}
\newtheorem{notation}[theorem]{Notation}
\newtheorem{assumption}[theorem]{Assumption}

\newtheorem{exercise}[theorem]{Discussion topic}
\newtheorem{homework}[theorem]{Homework}
\newtheorem{problem}[theorem]{Problem}

\newtheorem{ack}{Acknowledgements}

\numberwithin{equation}{section}


% Mean
\def\Xint#1{\mathchoice
{\XXint\displaystyle\textstyle{#1}}%
{\XXint\textstyle\scriptstyle{#1}}%
{\XXint\scriptstyle\scriptscriptstyle{#1}}%
{\XXint\scriptscriptstyle\scriptscriptstyle{#1}}%
\!\int}
\def\XXint#1#2#3{{\setbox0=\hbox{$#1{#2#3}{\int}$ }
\vcenter{\hbox{$#2#3$ }}\kern-.6\wd0}}
\def\ddashint{\Xint=}
\def\dashint{\Xint-}

%\usepackage{color}
%\hypersetup{%
%    colorlinks=true, % make the links colored%
%    linkcolor=blue, % color TOC links in blue
%    urlcolor=red, % color URLs in red
%    linktoc=all % 'all' will create links for everything in the TOC
%Ning added hyperlinks to the table of contents 6/17/19
%}

% style=alphabetic
\usepackage[backend=bibtex,maxcitenames=50,maxnames=50]{biblatex}
%\addbibresource{topics.bib}
\renewbibmacro{in:}{}
\DeclareFieldFormat{pages}{#1}

\begin{document}
%\begin{abstract}
%Practice for Midterm 1.
%x\end{abstract}

%\maketitle

%%%%%%%%%%%%%%%%%%%%%%%%%%%%%%%%%%%%%%%%%%%%%%%%%%%%%%%

% \tableofcontents

\begin{enumerate}
\item Recall that $f_b$ is continuous iff for every $x$,
$$f_b(x) = \lim_{y \to x} f_b(y).$$
This is clearly true if $x \neq 0$ (since the pieces that define $f$ are continuous there).
Now, if $x = 0$, then we must find $b$ so that
$$\lim_{y \to 0} f_b(y) = 3b + 5.$$
If $y > 0$ is going to $0$, then $3b + e^y + 4$ clearly is going to $3b + 5$. What about from the left? Well,
$$\lim_{y \to 0-} f_b(y) = \lim_{y \to 0-} \frac{\sin((b^2 + 1)y)}{y}$$
and recall that $\sin z/z \to 1$ as $z \to 0$. (You could also use l'H\^opital's rule here.) Setting $z = (b^2 + 1)y$ (so $y = z/(b^2 + 1)$) we get
$$\lim_{y \to 0-} f_b(y) = \lim_{z \to 0-} \frac{\sin z}{z/(b^2 + 1)} = (b^2 + 1) \lim_{z \to 0} \frac{\sin z}{z} = b^2 + 1.$$
Therefore
$$b^2 + 1 = 3b + 5$$
which we can solve to get $b = 4$ or $b = -1$.
\item There are two ways to skin this cat. For one, we could set
$$y = \arcsec(e^x)$$
and notice that this is equivalent to
$$\sec y = e^x.$$
Taking the derivative of both sides in $x$ (and making sure to use the chain rule! I want to get a smock that says ``Never forget the chain rule" and wear it every time I teach), we get
$$\tan y \sec y \frac{dy}{dx} = e^x.$$
To get this back in terms of just $x$ we recall that $\sec y = e^x$. Moreover,
$$\tan y = \sec y \sin y = e^x \sqrt{1 - \cos^2 y} = e^x \sqrt{1 - \frac{1}{\sec^2 y}} = e^x \sqrt{1 - e^{-2x}}$$
and hence
$$\frac{dy}{dx} = \frac{e^{-x}}{\sqrt{1 - e^{-2x}}}.$$
This approach might leave you wondering where the absolute value signs were supposed to come from.
To do that, we'll use the other approach. Recall that you can look at your notes on the exam, so in particular, you should be able to look up that
$$\frac{d}{du} \arcsec u = \frac{1}{|u|\sqrt{u^2 - 1}}.$$
Now we set $u = e^x$ and use the chain rule, to get
$$\frac{d}{dx} \arcsec(e^x) = \frac{1}{|e^x| \sqrt{e^{2x} - 1}} \frac{d}{dx} e^x.$$
The point is that $|e^x| = e^x$ because $e^x$ is alwyas positive.
Also the derivative of $e^x$ is itself. Therefore
$$\frac{d}{dx} \arcsec(e^x) = \frac{1}{e^x \sqrt{e^{2x} - 1}} e^x = \frac{1}{\sqrt{e^{2x} - 1}}.$$
You might worry that this is not equal to the other solution, but it is. The reason is that you can pull out an $e^x$ from the square root and get
$$\sqrt{e^{2x} - 1} = e^x\sqrt{1 - e^{-2x}}$$
whence
$$\frac{1}{\sqrt{e^{2x} - 1}} = \frac{e^{-x}}{\sqrt{1 - e^{-2x}}}.$$
\item You might be tempted to try to integrate $\frac{\sqrt{x^2 - 1}}{x} ~dx$.
Don't do it. You know what you're expected to get, so you can work backwards from there by differentiating.
This illustrates a few important principles.
First, in Calc I, if $F'(x) = f(x)$, then differentiating $F$ to get $f$ is almost always much easier than integrating $f$ to get $F$.
Second, if you integrate and you're not sure if your answer is correct, just differentiate your answer and see if you got the original function.

OK now let's do the problem. You have your notes in front of you, so you know that
$$\frac{d}{du} \arctan u = \frac{1}{1 + u^2}.$$
Setting $u = \sqrt{x^2 - 1}$ and using the chain rule (and the fact that $d/dx$ of $100$ is $0$) we get
\begin{align*}
F'(x) &= \frac{d}{dx}\sqrt{x^2 - 1} - \frac{1}{1 + (x^2 - 1)} \frac{d}{dx} \sqrt{x^2 - 1} \\
&= \left(1 - \frac{1}{x^2}\right) \frac{d}{dx} \sqrt{x^2 - 1}\\
&= \left(1 - \frac{1}{x^2}\right) \frac{x}{\sqrt{x^2 - 1}}.
\end{align*}
That doesn't look like $\sqrt{x^2 - 1}/x$, but luckily we can pull an $x$ out of the square root to get
$$\frac{x}{\sqrt{x^2 - 1}} = \frac{1}{\sqrt{1 - \frac{1}{x^2}}} = \left(1 - \frac{1}{x^2}\right)^{-1/2}$$
and plugging that back into our formula for $F'(x)$ we get
$$F'(x) = \left(1 - \frac{1}{x^2}\right)  \left(1 - \frac{1}{x^2}\right)^{-1/2} = \sqrt{1 - \frac{1}{x^2}}.$$
Pulling an $1/x$ out of the square root we get
$$F'(x) = \frac{\sqrt{x^2 - 1}}{x} = f(x)$$
which in particular implies that
$$\int f(x) ~dx = F(x).$$
\item OK, so this problem says to ``prove" something, so we should use words and not just write symbol soup.
You should have the same reaction to seeing such command verbs as ``show" and ``explain", among others.
(If you became a STEM major explicitly to avoid the use of the English language, this is the end of the line for you.)
One thing we could try is to integrate $\arctan(e^x) ~dx$, but if we wanted you to do that we would have just said to compute $\int \arctan(e^x) ~dx$.
Actually, that integral is just way too hard to compute; it involves a function called the ``second polylogarithm" which is way outside the scope of Calc I.

Recall that if $f$ is a continuous function on $[a, b]$, $m$ is the global minimum, and $M$ is the global maximum of $f$ on $[a, b]$, then
$$m(b - a) \leq \int_a^b f(x) ~dx \leq M(b - a).$$
Here $a = 0$, $b = \ln(\sqrt 3)$, so it suffices to show that the global minimum $m$ of $\arctan(e^x)$ on $[0, \ln(\sqrt 3)]$ satisfies
$$m \leq \frac{\pi}{4},$$
and similarly that the global maximum $M$ on $[0, \ln(\sqrt 3)]$ satisfies
$$M \geq \frac{\pi}{3}.$$
To see this, we recall that the global extremizers must either be at $0$ or $\ln(\sqrt 3)$, or must be critical points of $\arctan(e^x)$.
But $\arctan x$ and $e^x$ are both increasing functions, that is, their derivatives are positive.
Therefore by the chain rule
$$\frac{d}{dx} \arctan(e^x) = \arctan'(e^x) \frac{d}{dx} e^x$$
is the product of two positive functions. So it must be positive and $\arctan(e^x)$ has no critical points.
But the global extrema must exist by the extreme value theorem, ergo $0$ and $\ln(\sqrt 3)$ must be global extremizers.
Since the derivative is positive, $0$ is the global minimizer and $\ln(\sqrt 3)$ is the global maximizer.
Therefore
$$m = \arctan(e^0) = \arctan 1 = \frac{\pi}{4}$$
and
$$M = \arctan(e^{\ln(\sqrt 3)}) = \arctan(\sqrt 3) = \frac{\pi}{3}.$$
(The latter equality follows from properties of 30-60-90 triangles.)
This is exactly what we wanted to prove, so we're done.

By the way, this problem illustrates an important issue.
The ``global maximum" and ``global minimum" of a function is only defined once you choose a domain for it.
Sometimes there's a natural domain to work with, like $(-\infty, \infty)$, but here the domain was the domain of integration $[0, \ln(\sqrt 3)]$, since outside that domain how the function behaves doesn't affect the integral we were trying to estimate.
\item Blindly going in and plugging in $x = \infty$ gives us $\infty - \infty$ which is nonsense.
There's a standard trick to fix limits of the form $\infty - \infty$ however, and it is to rationalize.
More precisely,
$$\lim_{x \to \infty} \sqrt{x^2 + \sin x} - x = \lim_{x \to \infty} \frac{(\sqrt{x^2 + \sin x} - x)(\sqrt{x^2 + \sin x} + x)}{x + \sqrt{x^2 + \sin x}} = \lim_{x \to \infty} \frac{\sin x}{x + \sqrt{x^2 + \sin x}}.$$
It's tempting to use l'H\^opital, but the numerator isn't going to $\infty$ even though the denominator is, so we can't.

(Incidentally, always make sure that you explicitly indicate why you are allowed to use l'H\^opital's rule, which will ALWAYS either be that the limit takes the form $0/0$ or takes the form $\infty/\infty$.
l'H\^opital's rule doesn't work on limits of other forms, and if you just apply l'H\^opital's rule willy-nilly without indicating why you're doing it, either your answer will be wrong, or it will be right and we'll assume that you didn't actually understand why l'H\^opital's rule worked, and probably take off a point or two.)

Anyways, since the numerator stays bounded while the denominator goes to $\infty$, the limit is $0$.
\item The area of $\Delta$, which I will denote $|\Delta|$, satisfies
$$|\Delta| = (1 + x)y.$$
We also have the constraint equations $0 < x < 1$, $0 < y < 1$, $x^2 + y^2 = 1$. Setting $y = \sqrt{1 - x^2}$ we get
$$|\Delta| = (1 + x)\sqrt{1 - x^2}.$$
The derivative is
$$\frac{d}{dx} |\Delta| = \frac{1 - x - 2x^2}{\sqrt{1 - x^2}}$$
and since $0 < x < 1$, the denominator is never $0$. So the critical points are zeroes of $1 - x - 2x^2$.
That means the critical points are $x = -1, x = 1/2$, but as $x > 0$ this means that the only critical point is $x = 1/2$.

Now we show that the critical point $x = 1/2$ is a maximizer. (A priori it could be a minimizer or a saddle. Don't you dare let me catch you skipping this step when I grade your final exam.)
The second derivative is
$$\frac{d^2}{dx^2} |\Delta| = \frac{1 + 2x - 2x^2}{(x - 1)\sqrt{1 - x^2}}$$
and if we plug in $x = 1/2$ this is
$$\frac{d^2}{dx^2} |\Delta| = \frac{1 + 1 - 1/2}{-1/2 \cdot \sqrt{1 - 1/4}} < 0.$$
Therefore by the first and second derivative tests, $|\Delta|$ is maximimized when $x = 1/2$.

In that case, $y = \sqrt 3/2$ and $|\Delta| = (1 + 1/2)\sqrt 3/2$.
Don't bust out a calculuator and try to round that off to however many decimal digits.
This is a perfectly good answer as it is (in fact, it is better than anything a calculator could give you, since it has infinite precision, while any decimal answer comes with an error term.)
\item Let $y$ be the altitude of the balloon and $x$ the distance the bicycle has traveled since it passed under the balloon.
Thus $y = 9 + t$ and $x = 3t$. The angle of elevation satisfies
$$\tan \theta = \frac{y}{x} = \frac{9 + t}{3t}$$
so at time $t = 3$ we have
$$\theta = \arctan\left(\frac{12}{9}\right) = \arctan\left(\frac{4}{3}\right).$$
To find the rate of change of $\theta$ we implicitly differentiate:
$$\sec^2 \theta \frac{d\theta}{dt} = \frac{d}{dt} \frac{9 + t}{3t} = -3t^{-2}$$
so
$$\frac{d\theta}{dt} = -3\frac{\cos^2 \theta}{t^2} = -3t^{-2} \cos^2\left(\arctan\frac{9 + t}{3t}\right).$$
How to simplify $\cos^2 \arctan z$? We use the Pythagorean theorem.
It says that if $w = \arctan z$, then $z = \tan w$ so $z^2 = \sin^2 w/\cos^2 w$ and hence
$$z^2 + 1 = \frac{\sin^2 w + \cos^2 w}{\cos^2 w}$$
or in other words
$$\cos^2 w = \frac{1}{z^2 + 1}.$$
Therefore
$$\frac{d\theta}{dt} = -3\frac{t^{-2}}{z^2 + 1}$$
where $z = (9 + t)/3t$. Plugging in $t = 3$ we get $z = 12/9 = 4/3$ and $t^{-2} = 1/9$, so
$$\frac{d\theta}{dt} = -\frac{1}{3} \cdot \frac{1}{16/9 + 1}.$$
\item A few na\"ive substitutions are worth trying but none of them seem to work.
Luckily we have our old friend the Pythagorean theorem, which is always helpful for integrals of the form $\sin^k x ~dx$ or $\cos^k x ~dx$ where $k \geq 3$ is an odd number, thus
$$\sin^5 x = (\sin^2 x)^2 \sin x = (1 - \cos^2 x)^2 \sin x.$$
Now we substitute $u = -\cos x$, so $du = \sin x ~dx$, and hence
\begin{align*}\int \sin^5 x ~dx &= \int (1 - u^2)^2 ~du \\
&= \int 1 - 2u^2 + u^4 ~du \\
&= u - \frac{2}{3}u^3 + \frac{u^5}{5} \\
&= -\cos x + \frac{2}{3} \cos^3 x - \frac{\cos^5 x}{5}.
\end{align*}
Evaluating we get
$$\int_0^{\pi/4} \sin^5 x ~dx = \frac{64}{120} - \frac{43\sqrt 2}{120}$$
which is a good final answer.
\item Let
$$f(x) = 20x - e^{-4x}.$$
We want to show that there is one, and only one, zero of $f$.
To show that a function has a zero, we use the intermediate value theorem.
In fact
$$\lim_{x \to -\infty} f(x) = -20\infty - e^{4\infty} = -\infty - \infty = -\infty$$
and
$$\lim_{x \to \infty} f(x) = 20\infty - e^{-4\infty} = \infty + 0 = \infty$$
so by the intermediate value theorem, $f$ has at least one zero.
Let $Z$ be the number of zeroes that $f$ has; then $Z \geq 1$.

We need to show that $f$ has at most one zero, that is, $Z \leq 1$.
Let's see what happens if that's not true. Then $Z \geq 2$, so $f$ has at least two zeroes, say $x_1 < x_2$.
By Rolle's mean value theorem, $f$ has a critical point $y$ between $x_1$ and $x_2$, say
$$f'(y) = 0.$$
But also
$$f'(y) = 20 + 4e^{-4y} > 20 > 0$$
since $e^{-4y}$ is positive. Therefore $y$ is not a critical point... but $y$ is a critical point.
This cannot be! So it could not possibly be true that $Z \geq 2$, and hence $Z \leq 1$.
\item Don't try to integrate $e^{-t^2} ~dt$. This is called the \emph{Gaussian integral}, and will probably be treated in Calc III or a statistics class (since it's closely related to the central limit theorem). Anyways it's too hard for Calc I.

By the fundamental theorem of calculus, we have
$$F'(x) = e^{-x^2} > 0.$$
The extremizers must either be critical points or on the boundary, and we just saw that $F$ has no critical points.
So they must be on the boundary, namely $x = 1$ must define the global minimum and $x = 10$ must define the global maximum, since the derivative is always positive.

\end{enumerate}


\end{document}
