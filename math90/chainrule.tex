\documentclass[reqno,12pt,letterpaper]{amsart}
\RequirePackage{amsmath,amssymb,amsthm,graphicx,mathrsfs,url}
\RequirePackage[usenames,dvipsnames]{color}
\RequirePackage[colorlinks=true,linkcolor=Red,citecolor=Green]{hyperref}
\RequirePackage{amsxtra}
\usepackage{cancel}
\usepackage{tikz-cd}

\setlength{\textheight}{8.50in} \setlength{\oddsidemargin}{0.00in}
\setlength{\evensidemargin}{0.00in} \setlength{\textwidth}{6.08in}
\setlength{\topmargin}{0.00in} \setlength{\headheight}{0.00in}
\setlength{\marginparwidth}{1.0in}
\setlength{\abovedisplayskip}{0.2in}
\setlength{\belowdisplayskip}{0.2in}
\setlength{\parskip}{0.05in}
\renewcommand{\baselinestretch}{1.10}

\title{Midterm 1 practice problems}
\author{Aidan Backus}
\date{October 2021}

\newcommand{\NN}{\mathbf{N}}
\newcommand{\ZZ}{\mathbf{Z}}
\newcommand{\QQ}{\mathbf{Q}}
\newcommand{\RR}{\mathbf{R}}
\newcommand{\CC}{\mathbf{C}}
\newcommand{\DD}{\mathbf{D}}
\newcommand{\PP}{\mathbf P}
\newcommand{\MM}{\mathbf M}
\newcommand{\II}{\mathbf I}
\newcommand{\Hyp}{\mathbf H}

\DeclareMathOperator{\card}{card}
\DeclareMathOperator{\cent}{center}
\DeclareMathOperator{\ch}{ch}
\DeclareMathOperator{\codim}{codim}
\DeclareMathOperator{\diag}{diag}
\DeclareMathOperator{\diam}{diam}
\DeclareMathOperator{\dom}{dom}
\DeclareMathOperator{\Gal}{Gal}
\DeclareMathOperator{\Hom}{Hom}
\DeclareMathOperator{\Jac}{Jac}
\DeclareMathOperator{\Lip}{Lip}
\DeclareMathOperator{\Met}{Met}
\DeclareMathOperator{\id}{id}
\DeclareMathOperator{\rad}{rad}
\DeclareMathOperator{\rank}{rank}
\DeclareMathOperator{\Radon}{Radon}
\DeclareMathOperator*{\Res}{Res}
\DeclareMathOperator{\sgn}{sgn}
\DeclareMathOperator{\singsupp}{sing~supp}
\DeclareMathOperator{\Spec}{Spec}
\DeclareMathOperator{\supp}{supp}
\DeclareMathOperator{\Tan}{Tan}
\newcommand{\tr}{\operatorname{tr}}

\newcommand{\Ric}{\mathrm{Ric}}
\newcommand{\Riem}{\mathrm{Riem}}

\newcommand{\dbar}{\overline \partial}

\DeclareMathOperator{\arcosh}{arcosh}
\DeclareMathOperator{\atanh}{atanh}
\DeclareMathOperator{\csch}{csch}
\DeclareMathOperator{\sech}{sech}

\DeclareMathOperator{\Div}{div}
\DeclareMathOperator{\grad}{grad}
\DeclareMathOperator{\Ell}{Ell}
\DeclareMathOperator{\WF}{WF}

\newcommand{\Hilb}{\mathcal H}
\newcommand{\normal}{\mathbf n}
\newcommand{\vol}{\mathrm{vol}}

\newcommand{\pic}{\vspace{30mm}}
\newcommand{\dfn}[1]{\emph{#1}\index{#1}}

\renewcommand{\Re}{\operatorname{Re}}
\renewcommand{\Im}{\operatorname{Im}}


\newtheorem{theorem}{Theorem}[section]
\newtheorem{badtheorem}[theorem]{``Theorem"}
\newtheorem{prop}[theorem]{Proposition}
\newtheorem{lemma}[theorem]{Lemma}
\newtheorem{claim}[theorem]{Claim}
\newtheorem{proposition}[theorem]{Proposition}
\newtheorem{corollary}[theorem]{Corollary}
\newtheorem{conjecture}[theorem]{Conjecture}
\newtheorem{axiom}[theorem]{Axiom}

\theoremstyle{definition}
\newtheorem{definition}[theorem]{Definition}
\newtheorem{remark}[theorem]{Remark}
\newtheorem{example}[theorem]{Example}
\newtheorem{notation}[theorem]{Notation}
\newtheorem{assumption}[theorem]{Assumption}

\newtheorem{exercise}[theorem]{Discussion topic}
\newtheorem{homework}[theorem]{Homework}
\newtheorem{problem}[theorem]{Problem}

\newtheorem{ack}{Acknowledgements}

\numberwithin{equation}{section}


% Mean
\def\Xint#1{\mathchoice
{\XXint\displaystyle\textstyle{#1}}%
{\XXint\textstyle\scriptstyle{#1}}%
{\XXint\scriptstyle\scriptscriptstyle{#1}}%
{\XXint\scriptscriptstyle\scriptscriptstyle{#1}}%
\!\int}
\def\XXint#1#2#3{{\setbox0=\hbox{$#1{#2#3}{\int}$ }
\vcenter{\hbox{$#2#3$ }}\kern-.6\wd0}}
\def\ddashint{\Xint=}
\def\dashint{\Xint-}

%\usepackage{color}
%\hypersetup{%
%    colorlinks=true, % make the links colored%
%    linkcolor=blue, % color TOC links in blue
%    urlcolor=red, % color URLs in red
%    linktoc=all % 'all' will create links for everything in the TOC
%Ning added hyperlinks to the table of contents 6/17/19
%}

% style=alphabetic
\usepackage[backend=bibtex,maxcitenames=50,maxnames=50]{biblatex}
%\addbibresource{topics.bib}
\renewbibmacro{in:}{}
\DeclareFieldFormat{pages}{#1}

\begin{document}
%\begin{abstract}
%Practice for Midterm 1.
%x\end{abstract}

%\maketitle

%%%%%%%%%%%%%%%%%%%%%%%%%%%%%%%%%%%%%%%%%%%%%%%%%%%%%%%

% \tableofcontents

A lot of these problems were adapted from Paul's Online Math Notes, or from Jim Belk's posts on MathOverflow.

\section{Chain rule}
Compute the derivatives of the following functions.

\begin{enumerate}
\item $2\sin(3x+ \tan x)$.
\item $\tan(4+10x)$.
\item $e^{1- \cos x}$.
\item $x^2 \log(x^5)$.
\item $\tan^4(x^2 + 1)$.
\item $\cos(x^2e^x)$.
\end{enumerate}


\section{Implicit differentiation}
\begin{enumerate}
\item Compute $dy/dx$ if
$$\cos(x^2 + 2y) + xe^{y^2} = 1.$$
\item Compute $dy/dx$ if
$$\tan(x^2 y^4) = 3x + y^2.$$
\item Compute $dy/dx$ if
$$e^x = x + \sin y.$$
\item Find the equation of the tangent line to the plane curve
$$y^2 e^{2x} = 3y + x^2$$
at $(0, 3)$.
\item Compute the derivative of $\arccos x$.
\item Two functions which are frequently useful in applications are ``hyperbolic sine", or $\sinh x$, and ``hyperbolic cosine", or $\cosh x$. The derivative of $\sinh x$ is $\cosh x$ and the derivative of $\cosh x$ is $\sinh x$.
Moreover, the ``hyperbolic arcosine" function, $\arcosh x$, is the inverse of $\cosh x$, that is, $\arcosh \cosh x = x = \cosh \arcosh x$.
Prove that
$$\frac{d}{dx} \arcosh x = \frac{1}{\sqrt{x^2 - 1}}.$$
\end{enumerate}

\section{Related rates}
You can use a calculator for these problems.
A correct solution should include the appropriate units.

\begin{enumerate}
\item Let $M_0 = 4.75$ be the magnitude of the Sun and let $L_0 = 3.3839 \cdot 10^{26}$ watts be the luminosity of the Sun.
Then the magnitude $M$ of a star with luminosity $L$ is given by
$$M = M_0 - 2.5 \log_{10} \frac{L}{L_0}.$$
(Thus a brighter star has \emph{lower} magnitude).
Suppose that astronomers have discovered that a variable star is losing magnitude at a rate of $0.09$ per week, and currently has magnitude $3.8$.
Compute the rate of change of the luminosity of the star.
\item Let $R = 8.314 J/(K\cdot \text{mol})$ be the ideal gas constant, so that (assuming that kinetic molecular theory is valid) if we store $n$ mol of a gas in a container with volume $V$, the pressure $P$ and temperature $T$ of the gas are related by
$$PV = nRT.$$
Suppose that a container full of a high-pressure gas is slowly expanding, but no gas is entering or exiting the tank.
Currently, the pressure inside the container is $10^6 J/m^3$ and the temperature of the gas is $300 K$.
The volume of the tank is increasing at a rate of $0.1 m^3/s$ and the temperature of the gas is increasing by $3 K/s$.
Is the pressure of the gas increasing or decreasing? By how much?\footnote{Hint: Try taking the logarithm of both sides of the ideal gas law, so that you can avoid using the product rule.}
\item Suppose that James Bond is chasing an agent of the notorious supervillain organization SPECTRE.
Bond is driving his west towards an intersection at a rate of $90$ mph.
At the intersection, the agent suddenly turns north.
According to a radar gadget that MI-6 Q attached to Bond's car, the distance between Bond's car and the agent is now increasing at a rate of $15$ mph.
How fast is the agent driving?
\end{enumerate}

\printbibliography


\end{document}
