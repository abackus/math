\documentclass[reqno,12pt,letterpaper]{amsart}
\RequirePackage{amsmath,amssymb,amsthm,graphicx,mathrsfs,url}
\RequirePackage[usenames,dvipsnames]{color}
\RequirePackage[colorlinks=true,linkcolor=Red,citecolor=Green]{hyperref}
\RequirePackage{amsxtra}
\usepackage{cancel}
\usepackage{tikz-cd}

\setlength{\textheight}{8.50in} \setlength{\oddsidemargin}{0.00in}
\setlength{\evensidemargin}{0.00in} \setlength{\textwidth}{6.08in}
\setlength{\topmargin}{0.00in} \setlength{\headheight}{0.18in}
\setlength{\marginparwidth}{1.0in}
\setlength{\abovedisplayskip}{0.2in}
\setlength{\belowdisplayskip}{0.2in}
\setlength{\parskip}{0.05in}
\renewcommand{\baselinestretch}{1.10}

\title{Midterm 1 practice problems}
\author{Aidan Backus}
\date{October 2021}

\newcommand{\NN}{\mathbf{N}}
\newcommand{\ZZ}{\mathbf{Z}}
\newcommand{\QQ}{\mathbf{Q}}
\newcommand{\RR}{\mathbf{R}}
\newcommand{\CC}{\mathbf{C}}
\newcommand{\DD}{\mathbf{D}}
\newcommand{\PP}{\mathbf P}
\newcommand{\MM}{\mathbf M}
\newcommand{\II}{\mathbf I}
\newcommand{\Hyp}{\mathbf H}

\DeclareMathOperator{\card}{card}
\DeclareMathOperator{\cent}{center}
\DeclareMathOperator{\ch}{ch}
\DeclareMathOperator{\codim}{codim}
\DeclareMathOperator{\diag}{diag}
\DeclareMathOperator{\diam}{diam}
\DeclareMathOperator{\dom}{dom}
\DeclareMathOperator{\Gal}{Gal}
\DeclareMathOperator{\Hom}{Hom}
\DeclareMathOperator{\Jac}{Jac}
\DeclareMathOperator{\Lip}{Lip}
\DeclareMathOperator{\Met}{Met}
\DeclareMathOperator{\id}{id}
\DeclareMathOperator{\rad}{rad}
\DeclareMathOperator{\rank}{rank}
\DeclareMathOperator{\Radon}{Radon}
\DeclareMathOperator*{\Res}{Res}
\DeclareMathOperator{\sgn}{sgn}
\DeclareMathOperator{\singsupp}{sing~supp}
\DeclareMathOperator{\Spec}{Spec}
\DeclareMathOperator{\supp}{supp}
\DeclareMathOperator{\Tan}{Tan}
\newcommand{\tr}{\operatorname{tr}}

\newcommand{\Ric}{\mathrm{Ric}}
\newcommand{\Riem}{\mathrm{Riem}}

\newcommand{\dbar}{\overline \partial}

\DeclareMathOperator{\atanh}{atanh}
\DeclareMathOperator{\csch}{csch}
\DeclareMathOperator{\sech}{sech}

\DeclareMathOperator{\Div}{div}
\DeclareMathOperator{\grad}{grad}
\DeclareMathOperator{\Ell}{Ell}
\DeclareMathOperator{\WF}{WF}

\newcommand{\Hilb}{\mathcal H}
\newcommand{\normal}{\mathbf n}
\newcommand{\vol}{\mathrm{vol}}

\newcommand{\pic}{\vspace{30mm}}
\newcommand{\dfn}[1]{\emph{#1}\index{#1}}

\renewcommand{\Re}{\operatorname{Re}}
\renewcommand{\Im}{\operatorname{Im}}


\newtheorem{theorem}{Theorem}[section]
\newtheorem{badtheorem}[theorem]{``Theorem"}
\newtheorem{prop}[theorem]{Proposition}
\newtheorem{lemma}[theorem]{Lemma}
\newtheorem{claim}[theorem]{Claim}
\newtheorem{proposition}[theorem]{Proposition}
\newtheorem{corollary}[theorem]{Corollary}
\newtheorem{conjecture}[theorem]{Conjecture}
\newtheorem{axiom}[theorem]{Axiom}

\theoremstyle{definition}
\newtheorem{definition}[theorem]{Definition}
\newtheorem{remark}[theorem]{Remark}
\newtheorem{example}[theorem]{Example}
\newtheorem{notation}[theorem]{Notation}
\newtheorem{assumption}[theorem]{Assumption}

\newtheorem{exercise}[theorem]{Discussion topic}
\newtheorem{homework}[theorem]{Homework}
\newtheorem{problem}[theorem]{Problem}

\newtheorem{ack}{Acknowledgements}

\numberwithin{equation}{section}


% Mean
\def\Xint#1{\mathchoice
{\XXint\displaystyle\textstyle{#1}}%
{\XXint\textstyle\scriptstyle{#1}}%
{\XXint\scriptstyle\scriptscriptstyle{#1}}%
{\XXint\scriptscriptstyle\scriptscriptstyle{#1}}%
\!\int}
\def\XXint#1#2#3{{\setbox0=\hbox{$#1{#2#3}{\int}$ }
\vcenter{\hbox{$#2#3$ }}\kern-.6\wd0}}
\def\ddashint{\Xint=}
\def\dashint{\Xint-}

%\usepackage{color}
%\hypersetup{%
%    colorlinks=true, % make the links colored%
%    linkcolor=blue, % color TOC links in blue
%    urlcolor=red, % color URLs in red
%    linktoc=all % 'all' will create links for everything in the TOC
%Ning added hyperlinks to the table of contents 6/17/19
%}

% style=alphabetic
\usepackage[backend=bibtex,maxcitenames=50,maxnames=50]{biblatex}
%\addbibresource{topics.bib}
\renewbibmacro{in:}{}
\DeclareFieldFormat{pages}{#1}

\begin{document}
%\begin{abstract}
%Practice for Midterm 1.
%x\end{abstract}

%\maketitle

%%%%%%%%%%%%%%%%%%%%%%%%%%%%%%%%%%%%%%%%%%%%%%%%%%%%%%%

% \tableofcontents

A lot of these problems were adapted from Paul's Online Math Notes.

\section{Algebra}
\begin{enumerate}
\item Find the domain and range of
$$f(x) = -3\sqrt{14+3x}.$$
\item Solve the equation
$$8 \tan(2x) - 5 = 3$$
where $x$ is in $[-\pi/2, 3\pi/2]$.
\item Find all solutions $x$ to the equation
$$\log x + \log(x - 21) = 2$$
where the logarithm is base $e$.
\item Sketch a graph of the piecewise function
$$f(x) = \begin{cases} x - 2, &x < 0 \\
x^2 + 6, x \geq 0.
\end{cases}$$
\end{enumerate}

\section{Limits}
In the below problems you cannot use l'H\^opital's rule.
\begin{enumerate}
\item Compute
$$\lim_{w \to 3} \frac{w^2 - 8w}{4 - 7w}.$$
\item Compute
$$\lim_{x \to 0} \frac{\sin(6x)}{x + x^2}.$$
\item Find all one-sided limits at $\theta = 3\pi/2$ of $\tan \theta$.
\item Compute
$$\lim_{x \to -\infty} \frac{e^{-3x} - 2e^{8x}}{9e^{8x} - 7e^{-3x}}.$$
\item Prove that there is a solution $t$ to the equation
$$4t + 10e^t = e^{2t}$$
with $t$ in $[1, 3]$.
\end{enumerate}

\section{Derivatives}
In the below problem if it does not say ``Do not use the derivative rules," you can use the derivative rules (but not the chain rule).
\begin{enumerate}
\item Find the slope of the tangent line to $y = x^3$ at $x = 1$. Do not use the derivative rules.
\item Compute $f'(x)$ where
$$f(x) = \sin x.$$
Do not use the derivative rules.
\item Compute $f'(x)$ where
$$f(x) = x^2 + 2x^3 + 3x^4.$$
\item Compute $f'(x)$ where
$$f(x) = \frac{6}{\sqrt x} + \frac{1}{8x^4} + 26.$$
\item Compute $f'(x)$ where
$$f(x) x^3 \sin x.$$
\item Compute the equation of the tangent line to
$$y = \frac{1 + 2x + 3x^2}{16x^3}$$
at $x = 5$.
\item If a particle is at position $f(t) = t^2$ at time $t$, when is the particle moving to the right?
\end{enumerate}


\printbibliography


\end{document}
