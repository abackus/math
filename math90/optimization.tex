\documentclass[reqno,12pt,letterpaper]{amsart}
\RequirePackage{amsmath,amssymb,amsthm,graphicx,mathrsfs,url}
\RequirePackage[usenames,dvipsnames]{color}
\RequirePackage[colorlinks=true,linkcolor=Red,citecolor=Green]{hyperref}
\RequirePackage{amsxtra}
\usepackage{cancel}
\usepackage{tikz-cd}

\setlength{\textheight}{9.00in} \setlength{\oddsidemargin}{-0.25in}
\setlength{\evensidemargin}{-0.25in} \setlength{\textwidth}{6.58in}
\setlength{\topmargin}{-0.25in} \setlength{\headheight}{0.00in}
\setlength{\marginparwidth}{1.0in}
\setlength{\abovedisplayskip}{0.2in}
\setlength{\belowdisplayskip}{0.2in}
\setlength{\parskip}{0.05in}
\renewcommand{\baselinestretch}{1.10}

\title{Midterm 1 practice problems}
\author{Aidan Backus}
\date{October 2021}

\newcommand{\NN}{\mathbf{N}}
\newcommand{\ZZ}{\mathbf{Z}}
\newcommand{\QQ}{\mathbf{Q}}
\newcommand{\RR}{\mathbf{R}}
\newcommand{\CC}{\mathbf{C}}
\newcommand{\DD}{\mathbf{D}}
\newcommand{\PP}{\mathbf P}
\newcommand{\MM}{\mathbf M}
\newcommand{\II}{\mathbf I}
\newcommand{\Hyp}{\mathbf H}

\DeclareMathOperator{\card}{card}
\DeclareMathOperator{\cent}{center}
\DeclareMathOperator{\ch}{ch}
\DeclareMathOperator{\codim}{codim}
\DeclareMathOperator{\diag}{diag}
\DeclareMathOperator{\diam}{diam}
\DeclareMathOperator{\dom}{dom}
\DeclareMathOperator{\Gal}{Gal}
\DeclareMathOperator{\Hom}{Hom}
\DeclareMathOperator{\Jac}{Jac}
\DeclareMathOperator{\Lip}{Lip}
\DeclareMathOperator{\Met}{Met}
\DeclareMathOperator{\id}{id}
\DeclareMathOperator{\rad}{rad}
\DeclareMathOperator{\rank}{rank}
\DeclareMathOperator{\Radon}{Radon}
\DeclareMathOperator*{\Res}{Res}
\DeclareMathOperator{\sgn}{sgn}
\DeclareMathOperator{\singsupp}{sing~supp}
\DeclareMathOperator{\Spec}{Spec}
\DeclareMathOperator{\supp}{supp}
\DeclareMathOperator{\Tan}{Tan}
\newcommand{\tr}{\operatorname{tr}}

\newcommand{\Ric}{\mathrm{Ric}}
\newcommand{\Riem}{\mathrm{Riem}}

\newcommand{\dbar}{\overline \partial}

\DeclareMathOperator{\arcosh}{arcosh}
\DeclareMathOperator{\atanh}{atanh}
\DeclareMathOperator{\csch}{csch}
\DeclareMathOperator{\sech}{sech}

\DeclareMathOperator{\Div}{div}
\DeclareMathOperator{\grad}{grad}
\DeclareMathOperator{\Ell}{Ell}
\DeclareMathOperator{\WF}{WF}

\newcommand{\Hilb}{\mathcal H}
\newcommand{\normal}{\mathbf n}
\newcommand{\vol}{\mathrm{vol}}

\newcommand{\pic}{\vspace{30mm}}
\newcommand{\dfn}[1]{\emph{#1}\index{#1}}

\renewcommand{\Re}{\operatorname{Re}}
\renewcommand{\Im}{\operatorname{Im}}


\newtheorem{theorem}{Theorem}[section]
\newtheorem{badtheorem}[theorem]{``Theorem"}
\newtheorem{prop}[theorem]{Proposition}
\newtheorem{lemma}[theorem]{Lemma}
\newtheorem{claim}[theorem]{Claim}
\newtheorem{proposition}[theorem]{Proposition}
\newtheorem{corollary}[theorem]{Corollary}
\newtheorem{conjecture}[theorem]{Conjecture}
\newtheorem{axiom}[theorem]{Axiom}

\theoremstyle{definition}
\newtheorem{definition}[theorem]{Definition}
\newtheorem{remark}[theorem]{Remark}
\newtheorem{example}[theorem]{Example}
\newtheorem{notation}[theorem]{Notation}
\newtheorem{assumption}[theorem]{Assumption}

\newtheorem{exercise}[theorem]{Discussion topic}
\newtheorem{homework}[theorem]{Homework}
\newtheorem{problem}[theorem]{Problem}

\newtheorem{ack}{Acknowledgements}

\numberwithin{equation}{section}


% Mean
\def\Xint#1{\mathchoice
{\XXint\displaystyle\textstyle{#1}}%
{\XXint\textstyle\scriptstyle{#1}}%
{\XXint\scriptstyle\scriptscriptstyle{#1}}%
{\XXint\scriptscriptstyle\scriptscriptstyle{#1}}%
\!\int}
\def\XXint#1#2#3{{\setbox0=\hbox{$#1{#2#3}{\int}$ }
\vcenter{\hbox{$#2#3$ }}\kern-.6\wd0}}
\def\ddashint{\Xint=}
\def\dashint{\Xint-}

%\usepackage{color}
%\hypersetup{%
%    colorlinks=true, % make the links colored%
%    linkcolor=blue, % color TOC links in blue
%    urlcolor=red, % color URLs in red
%    linktoc=all % 'all' will create links for everything in the TOC
%Ning added hyperlinks to the table of contents 6/17/19
%}

% style=alphabetic
\usepackage[backend=bibtex,maxcitenames=50,maxnames=50]{biblatex}
%\addbibresource{topics.bib}
\renewbibmacro{in:}{}
\DeclareFieldFormat{pages}{#1}

\begin{document}
%\begin{abstract}
%Practice for Midterm 1.
%x\end{abstract}

%\maketitle

%%%%%%%%%%%%%%%%%%%%%%%%%%%%%%%%%%%%%%%%%%%%%%%%%%%%%%%

% \tableofcontents

A lot of these problems were adapted from Paul's Online Math Notes, or Math24's ``Optimization in Economics".

\section{First and second derivative tests}
\begin{enumerate}
\item Find all critical points of $1 + 80x^3 + 5x^4 - 2x^5$.
\item Find all critical points of $4 \cos x - x$.
\item Find all critical points of the function which is $\sin x$ for $x \leq 0$ and $x$ for $x \geq 0$.
\item Find the intervals on which $e^{-x^2/2}$ is increasing, decreasing, convex, and concave.
\item Find the local extrema of $2x^3 (x + 2)^5$ on $[-5/2, 1/2]$. Are they global extrema?
\item Find the local extrema of $e^{x^3 - 2x^2 - 7x}$ on $[-1/2, 5/2]$. Are they global extrema?
\item Sketch the graph of $x - 2 \ln(1 + x^2)$.
\item Sketch the graph of $3x - 5 \sin(2x)$.
\item Sketch the graph of $\ln(x^2 + x + 1)$.
\end{enumerate}

\section{Mean value theorem}
\begin{enumerate}
\item Prove that there is exactly one real number $x$ such that
$$x^3 - 7x^2 + 25x + 8 = 0.$$
\item Suppose that $f$ is differentiable on $[-7, 0]$, $f(-7) = -3$, and $f'(x) \leq 2$. What can you say about $f(0)$?
\end{enumerate}

\section{l'H\^opital's rule}
\begin{enumerate}
\item Watch the climactic scene of the famous Lindsay Lohan movie \emph{Mean Girls} at \url{https://www.youtube.com/watch?v=oDAKKQuBtDo}.
Then see if you could find a way to prove that the limit in that scene does not exist without using l'H\^opital's rule\footnote{I'm not actually sure how Lohan's character solved this problem so quickly, given that she apparently didn't even remember what a limit was going into the scene; maybe having an epiphany about how you've been horribly mistreating your friends powers up your math ability, Sh\=onen Battle Anime-style, and I just have never experienced the Power of Friendship while working on research.}.
Then, notice that l'H\^opital's rule makes the limit much, much easier to find.
\item In class much earlier this semester, we were talking about the \emph{sorting problem} from computer science, and we needed the fact that
$$\lim_{x \to \infty} \frac{x}{\log_2 x} = + \infty.$$
At the time I said some vague waffle about the doubling rate of these two functions to explain why this is true.
Now that you actually have the tools to compute this limit, do it.
\item Compute the limit
$$\lim_{x \to 2} \frac{x^3 - 7x + 10x}{x^2 + x - 6}.$$
\item Compute the limit
$$\lim_{x \to -\infty} \frac{x^2}{e^{1 - x}}.$$
\item Compute the limit
$$\lim_{x \to 6} \frac{x - 6}{x}.$$
\item Compute the limit
$$\lim_{x \to +\infty} (e^x + x)^{1/x}.$$
\end{enumerate}

\section{Optimization}
\begin{enumerate}
\item Suppose that we want to design an aluminum cylindrical can with a bottom but no top, which needs to hold $3$ liters ($3000$ cubic centimeters) of condensed milk. What should the radius and the height of the can be, so that we use the least amount of aluminum?
\item Two $10$-meter tall poles are $30$ meters apart. A length of wire is attached to the top of each pole and is staked to the ground somewhere between the two poles. Where should the wire touch the ground, so that the least amount of wire is needed?
\item In economics, the \emph{demand function} $P$ is defined by declaring that if the price of a commodity is $P(x)$, then consumers will buy $x$ units of that commodity.
Suppose that you are a luthier. It cost you $1,000,000$ dollars to set up your violin shop, and the demand function for your violins is
$$P(x) = 5000 - x \frac{\text{dollars}}{\text{violin}}.$$
On the other hand, it costs you $1000$ dollars to make a violin.
How much should you charge per violin to maximize your profit?\footnote{Hint: Does the ``sunk cost" of one million dollars have anything to do with this calculation?}
\item In computer science, the \emph{method of steepest descent} is a popular algorithm for finding the minimizers of a function $f$, or for solving the equation $f'(x) = 0$.
The idea is that if you're lost in the mountains and want to get down, you should choose the path which goes down more steeply.
More precisely, we some number $x_0$ and some very small number $\varepsilon > 0$, and let $x_1 = x_0 - \varepsilon f'(x_0)$, $x_2 = x_1 - \varepsilon f'(x_1)$, et cetra.
Then, if $n$ is large, $x_n$, should be approximately a minimizer.

I'm going to graph three functions on the board: a ``twisty" function, a convex function, and a rapidly oscillating function.
For which do you think the method of steepest descent will work? Why?\footnote{Hint: The ``lost in the mountains" metaphor is helpful here. Once I was lost in the mountains and wandered for about six hours, because taking the paths of descent didn't seem to help.}
\end{enumerate}

\end{document}
