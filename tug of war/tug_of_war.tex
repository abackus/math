\documentclass[reqno,11pt]{amsart}
\usepackage[letterpaper, margin=1in]{geometry}
\RequirePackage{amsmath,amssymb,amsthm,graphicx,mathrsfs,url,slashed,subcaption}
\RequirePackage[usenames,dvipsnames]{xcolor}
\RequirePackage[colorlinks=true,linkcolor=Red,citecolor=Green]{hyperref}
\RequirePackage{amsxtra}
\usepackage{cancel}
\usepackage{tikz-cd}
%\usepackage[T1]{fontenc}

% \setlength{\textheight}{9.3in} \setlength{\oddsidemargin}{-0.25in}
% \setlength{\evensidemargin}{-0.25in} \setlength{\textwidth}{7in}
% \setlength{\topmargin}{-0.25in} \setlength{\headheight}{0.18in}
% \setlength{\marginparwidth}{1.0in}
% \setlength{\abovedisplayskip}{0.2in}
% \setlength{\belowdisplayskip}{0.2in}
% \setlength{\parskip}{0.05in}
%\renewcommand{\baselinestretch}{1.05}

\title{Tug-of-war solutions to $\infty$-elliptic systems}
\author{Aidan Backus}
\address{Department of Mathematics, Brown University}
\email{aidan\_backus@brown.edu}
\date{\today}

\newcommand{\NN}{\mathbf{N}}
\newcommand{\ZZ}{\mathbf{Z}}
\newcommand{\QQ}{\mathbf{Q}}
\newcommand{\RR}{\mathbf{R}}
\newcommand{\CC}{\mathbf{C}}
\newcommand{\DD}{\mathbf{D}}
\newcommand{\PP}{\mathbf P}
\newcommand{\MM}{\mathbf M}
\newcommand{\II}{\mathbf I}
\newcommand{\Hyp}{\mathbf H}
\newcommand{\Sph}{\mathbf S}
\newcommand{\Group}{\mathbf G}
\newcommand{\GL}{\mathbf{GL}}
\newcommand{\Orth}{\mathbf{O}}
\newcommand{\SpOrth}{\mathbf{SO}}
\newcommand{\Ball}{\mathbf{B}}

\newcommand*\dif{\mathop{}\!\mathrm{d}}

\DeclareMathOperator{\card}{card}
\DeclareMathOperator{\dist}{dist}
\DeclareMathOperator{\id}{id}
\DeclareMathOperator{\supp}{supp}
\DeclareMathOperator{\Teich}{Teich}
\DeclareMathOperator{\tr}{tr}

\newcommand{\Leaves}{\mathscr L}
\newcommand{\Lagrange}{\mathcal L}
\newcommand{\Hypspace}{\mathscr H}

\newcommand{\Chain}{\underline C}

\newcommand{\Two}{\mathrm{I\!I}}

\newcommand{\normal}{\mathbf n}
\newcommand{\radial}{\mathbf r}
\newcommand{\evect}{\mathbf e}
\newcommand{\vol}{\mathrm{vol}}

\newcommand{\diam}{\mathrm{diam}}
\newcommand{\Ell}{\mathrm{Ell}}
\newcommand{\inj}{\mathrm{inj}}
\newcommand{\Lip}{\mathrm{Lip}}
\newcommand{\Riem}{\mathrm{Riem}}
\newcommand{\Ric}{\mathrm{Ric}}

\newcommand{\Min}{\mathrm{Min}}
\newcommand{\Max}{\mathrm{Max}}

\newcommand{\dfn}[1]{\emph{#1}\index{#1}}

\renewcommand{\Re}{\operatorname{Re}}
\renewcommand{\Im}{\operatorname{Im}}

\newcommand{\loc}{\mathrm{loc}}
\newcommand{\cpt}{\mathrm{cpt}}
\newcommand{\euc}{\mathrm{euc}}

\def\Japan#1{\left \langle #1 \right \rangle}

\newtheorem{theorem}{Theorem}[section]
\newtheorem{badtheorem}[theorem]{``Theorem"}
\newtheorem{prop}[theorem]{Proposition}
\newtheorem{lemma}[theorem]{Lemma}
\newtheorem{sublemma}[theorem]{Sublemma}
\newtheorem{proposition}[theorem]{Proposition}
\newtheorem{corollary}[theorem]{Corollary}
\newtheorem{conjecture}[theorem]{Conjecture}
\newtheorem{axiom}[theorem]{Axiom}
\newtheorem{assumption}[theorem]{Assumption}

\newtheorem{mainthm}{Theorem}
\renewcommand{\themainthm}{\Alph{mainthm}}

\newtheorem{claim}{Claim}[theorem]
\renewcommand{\theclaim}{\thetheorem\Alph{claim}}
% \newtheorem*{claim}{Claim}

\theoremstyle{definition}
\newtheorem{definition}[theorem]{Definition}
\newtheorem{remark}[theorem]{Remark}
\newtheorem{example}[theorem]{Example}
\newtheorem{notation}[theorem]{Notation}

\newtheorem{exercise}[theorem]{Discussion topic}
\newtheorem{homework}[theorem]{Homework}
\newtheorem{problem}[theorem]{Problem}

\makeatletter
\newcommand{\proofpart}[2]{%
  \par
  \addvspace{\medskipamount}%
  \noindent\emph{Part #1: #2.}
}
\makeatother



\numberwithin{equation}{section}


% Mean
\def\Xint#1{\mathchoice
{\XXint\displaystyle\textstyle{#1}}%
{\XXint\textstyle\scriptstyle{#1}}%
{\XXint\scriptstyle\scriptscriptstyle{#1}}%
{\XXint\scriptscriptstyle\scriptscriptstyle{#1}}%
\!\int}
\def\XXint#1#2#3{{\setbox0=\hbox{$#1{#2#3}{\int}$ }
\vcenter{\hbox{$#2#3$ }}\kern-.6\wd0}}
\def\ddashint{\Xint=}
\def\dashint{\Xint-}

\usepackage[backend=bibtex,style=alphabetic,giveninits=true]{biblatex}
\renewcommand*{\bibfont}{\normalfont\footnotesize}
\addbibresource{tug_of_war.bib}
\renewbibmacro{in:}{}
\DeclareFieldFormat{pages}{#1}

\newcommand\todo[1]{\textcolor{red}{TODO: #1}}


\begin{document}
\begin{abstract}
Viscosity solutions don't work
\end{abstract}

\maketitle

%%%%%%%%%%%%%%%%%%%%%%%%%%%%%%%%%%%%%%%%%%%%%%%%%%%%%%%
\section{Introduction}
Let's do $\dim M = 3$ for simplicity.

First recall that $\infty$-harmonic maps $v: M \to \Sph^1$ solve the $\infty$-Laplace equation 
\begin{equation}\label{infinity laplacian}
\nabla_{(\dif v)^\sharp} |\dif v|^2 = 0.
\end{equation}
However, $\infty$-harmonic maps need not be $C^2$, and (\ref{infinity laplacian}) cannot be written in divergence form, so $\infty$-harmonic maps need to be understood as viscosity solutions of (\ref{infinity laplacian}) \cite{Crandall2008}.

If $F$ is a tight $2$-form (say), it solves 
$$\nabla_{(\iota_X F)^\sharp} |F|^2 = 0$$
for every vector field $X$ (in particular, the coordinate vector fields).
Locally we have $F = \dif A$ and so this equation is of the same form as (\ref{infinity laplacian}).
However, viscosity solutions are defined in terms of the maximum principle and so are only suitable for systems in diagonal form, as in \cite{Ishii92}.

%%%%%%%%%%%%%%%%%%%%%%%
\section{pMaxwell}
We introduce the \dfn{$p$-Maxwell equation}
$$\dif^*(|\dif A|^{p - 2} \dif A) = 0$$
which is elliptic \emph{up to a gauge transformation}. 
In particular for a domain $U$ and a continuous $1$-form $B$ on $\partial U$ there exists a continuous $1$-form $A$ on $U$ with $\iota_{\partial U}^* A = B$ which solves the $p$-Maxwell equation.
If we in addition impose that $A$ is in Coulomb gauge 
$$\dif^* A = 0$$
then $A \in W^{1, p}$ is H\"older continuous ($p > 3$) and is uniquely determined, since in particular $A$ is $p$-harmonic.

\subsection{Continuity of the data-to-solution map}
Let's assume $\Ric_M \leq 0.$

\begin{proposition}
Let $D$ be a small open set in $M$.
Let $u, v$ be continuous solutions of the $p$-Maxwell equation in Coulomb gauge on a neighborhood of $\overline D$.
Then
$$\|u - v\|_{C^0(\overline D)} \leq \|u - v\|_{C^0(\partial D)}.$$
\end{proposition}
\begin{proof}
Suppose towards contradiction that
$$\|u - v\|_{C^0(\overline D)} > \|u - v\|_{C^0(\partial D)} =: M,$$
and consider the open set
$$V := \{|u - v| > M\}$$
which is then nonempty and satisfies $V \Subset D$ and $|u - v||_{\partial V} = M$.

We now let $\varepsilon > 0$ and choose a smooth $1$-form $\varphi$ on $D$ of unit length such that on $\overline V$,
$$\left|\varphi - \frac{u - v}{|u - v|}\right| < \varepsilon.$$
This is possible because $u - v$ has no zeroes near $\overline V$.
We can then introduce 
$$w := u - v - M\varphi$$
so that on $\partial V$, we can estimate by the Cauchy-Schwarz and Young inequalities
\begin{align*}
|w|^2 &= |u - v|^2 + M^2 - 2\langle u - v, M\varphi\rangle  \\
&= 2M^2 - 2\langle u - v, u - v\rangle + 2M \langle u - v, \frac{u - v}{|u - v|} + \varphi\rangle \\
&\leq 0 + 2M^2 \varepsilon = 2M^2 \varepsilon.
\end{align*}

Let
$$\psi := |\dif u|^{p - 2} \dif u - |\dif v|^{p - 2} \dif v.$$
Integrating the $p$-Maxwell equation by parts,
$$0 = \int_V \langle \dif^* \psi, w\rangle \dif V = \int_{\partial V} \langle \psi, w\rangle \dif S - \int_V \langle \psi, \dif u - \dif v\rangle \dif V.$$
Then Cauchy-Schwarz implies that 
$$\left|\int_V \langle \psi, \dif u - \dif v\rangle \dif V\right| = \left|\int_{\partial V} \langle \psi, w\rangle \dif S\right| \leq 2M^2 \varepsilon \|\psi\|_{L^1(\partial V)}.$$
But the left-hand side of this inequality is independent of $\varepsilon$, so we actually have 
$$\int_V \langle |\dif u|^{p - 2} \dif u - |\dif v|^{p - 2} \dif v, \dif u - \dif v\rangle \dif V = 0.$$
By the Cauchy-Schwarz and Young inequalities,
$$\langle |\dif u|^{p - 2} \dif u - |\dif v|^{p - 2} \dif v, \dif u - \dif v\rangle \geq |\dif u|^p + |\dif v|^p - |\dif u|^p - |\dif v|^p = 0.$$
So we must have $\dif u = \dif v$ almost everywhere in $V$.
Since we are given $\dif^* u = \dif^* v$, $\gamma := u - v$ is harmonic.
Moreover, $|\gamma| = M$ on $\partial V$.

To obtain a contradiction, we show that $|\gamma|^2$ is subharmonic. In fact,
$$\Delta |\gamma|^2 = \nabla^\alpha(\partial_\alpha |\gamma|^2) = \nabla^\alpha (\gamma^\beta \nabla_\alpha \gamma_\beta) = |\nabla \gamma|^2 + \langle \gamma, \nabla^* \nabla \gamma\rangle \geq \langle \gamma, \nabla^* \nabla \gamma\rangle$$
where $\nabla^* \nabla$ is Bochner's Laplacian.
By the Weitzenb\"ock formula,
$$\langle \gamma, \nabla^* \nabla \gamma\rangle = \Delta \gamma - \Ric(\gamma^\sharp, \gamma^\sharp) \geq \Delta \gamma$$
where $\Delta$ is Hodge's Laplacian and we used the fact that $\Ric \leq 0$.
In conclusion, we have 
$$\Delta |\gamma|^2 \geq 0$$
so by the maximum principle, $|\gamma| \leq M$ on $V$.
In particular, $|u - v| \leq V$. This is a contradiction.
\end{proof}

%%%%%%%%%%%%%%%%%%%%%%%%
\section{Variational solutions}

Let $B \in C^\infty(\partial U, \Omega^1)$ be given and $A_p$ be the solution of the $p$-Maxwell equation with $\iota_{\partial U}^* A_p = B$, which we assume to be in Coulomb gauge.
Then 
$$\|\nabla A_p\|_{L^p} \lesssim \|(\dif + \dif^*) A_p\|_{L^p} \lesssim \|B\|_{C^0}.$$
Then we obtain a limiting form $A$, with the limit taken weakly in $W^{1, q}$, $d < q < \infty$, and strongly in $C^\alpha$, $0 \leq \alpha < 1$.

\begin{definition}
We call $A$ a \dfn{variational solution of the $\infty$-Maxwell equation}
$$\langle\langle \nabla \dif A, \dif A\rangle, \dif A\rangle = 0$$
where the inner contraction is between the two $\dif A$s and the outer contraction is between $\nabla$ and an index of $\dif A$, or in other words:
$${(\dif A)_i}^j (\dif A)^{k\ell} \nabla_j (\dif A)_{k \ell} = 0.$$
\end{definition}

To derive this equation we write the $p$-Maxwell equation in nondivergence form
$$0 = \dif(|\dif A|^{p - 2} \star \dif A) = \dif(|\dif A|^{p - 2}) \wedge \star \dif A - |\dif A|^{p - 2} \dif(\star \dif A).$$
Since $\nabla$ is a metric connection,
$$\partial_j (|\dif A|^{p - 2}) = (p - 2) |\dif A|^{p - 4} (\dif A)^{k \ell} \nabla_j (\dif A)_{k \ell}$$
and from the definition of $\star$ it follows that
$$(\star \dif(|\dif A|^{p - 2}) \wedge \star \dif A)_i = \pm (p - 2) |\dif A|^{p - 4} (\dif A)^{k \ell} \nabla_j (\dif A)_{k \ell} {(\dif A)_i}^j.$$
Dividing through we get the renormalized nondivergence form $p$-Maxwell equation 
$${(\dif A)_i}^j (\dif A)^{k\ell} \nabla_j (\dif A)_{k \ell} = \pm \frac{(\dif^* \dif A)_i}{(p - 2) |\dif A|^2}.$$
If $\dif^* \dif A/|\dif A|^2$ grows only sublinearly as $p \to \infty$, then it follows that the limiting equation is the $\infty$-Maxwell equation.
But this only makes sense variationally.

%%%%%%%%%%%%%%%%%%%%%%
\subsection{Absolutely best comass}
\begin{definition}
$\dif A$ has \dfn{absolutely best comass} if for every open ball $D \subseteq U$ and every $B \in C^\infty(D, \Omega^1)$ with $A|_{\partial D} = B|_{\partial D}$,
one has
$$L(\dif A) \leq L(\dif B).$$
\end{definition}

This condition is gauge-invariant.
Indeed, if $\dif \chi$ is a gauge transformation, then it does not affect the comass and we can add it to $B$.
Note that we take the full trace on $\partial D$, rather than the pullback, since this is still gauge invariant.

\begin{proposition}
Suppose that $A \in C^2(U, \Omega^1)$ and $\dif A$ has absolutely best comass.
Then $A$ is a classical solution of the $\infty$-Maxwell equation.
\end{proposition}
\begin{proof}
For a covariant $2$-tensor $T$, let $T^{\rm as}$ be its antisymmetrization, and let
$$f(x, T) := |T^{\rm as}|_{g(x)}^2.$$
Since $A$ has absolutely best comass and $(\nabla A)^{\rm as} = \dif A$, $A$ is an absolute minimizer (see \cite[Definition 5.1]{Barron2001}) of the essential supremum of $f(\cdot, \nabla A)$.
By \cite[Theorem 5.2]{Barron2001},
\begin{equation}\label{ELA}
	\frac{\partial f}{\partial T_{ij}}(x, \nabla A(x)) \partial_j (f(x, \nabla A(x))) = 0.
\end{equation}
We then compute
\begin{align*}
\partial_j (f(x, \nabla A(x))) &= \nabla_j |\dif A|^2 = 2 \langle \dif A, \nabla_j \dif A\rangle = (\dif A)^{k\ell} \nabla_j (\dif A)_{k\ell} \\
\frac{\partial f}{\partial T_{ij}}(x, \nabla A(x)) &= (\nabla A^{\rm as})^{ij} = (\dif A)^{ij}. \qedhere
\end{align*}
\end{proof}

\begin{conjecture}
If $A$ is Lipschitz and $\dif A$ has absolutely best comass, then $A$ is a variational solution of the $\infty$-Maxwell equation.
\end{conjecture}

Let's now prove the converse.
Let $\cdot_\euc$ denote the euclidean version of something in some coordinates.

\begin{definition}
A form $A$ is in \dfn{flat Coulomb gauge} with respect to some coordinates if $\dif^*_\euc A = 0$.
\end{definition}

\begin{lemma}[comparison principle]
Suppose that $A, B$ are in flat Coulomb gauge and solve the $p$-Maxwell equation, $3 < p < \infty$, on a neighborhood of $\overline D$, and on $\partial D$, $A_i \leq B_i$.
Then on $\overline D$, $A_i \leq B_i$.
\end{lemma}
\begin{proof}
We follow \cite[Theorem 2.15]{lindqvist2019notes} which is analogous for the $p$-Laplacian.
Since $p > 3$ and
$$\|A\|_{C^0} \lesssim \|\nabla A\|_{L^p} \lesssim \|(\dif + \dif_\euc^*) A\|_{L^p} = \|\dif A\|_{L^p},$$
$A$ is continuous.
Let $\varepsilon > 0$, so either
$$V_i(\varepsilon) := \{A_i > B_i + \varepsilon\}$$
is empty, or it is nonempty and satisfies $V_i(\varepsilon) \Subset D$.
If the claim fails, then there exist $j, \varepsilon$ such that $V_i(\varepsilon)$ is nonempty; we let 
$$V := \bigcup_i V_i(\varepsilon)$$
which is then nonempty and satisfies $V \Subset D$. 

Let
$$w_i := \max(A_i - B_i + \varepsilon, 0)$$
which is a $1$-form with support $\overline V$, and if we test it against the $p$-Maxwell equation, we get
$$\int_V \star \langle |\dif A|^{p - 2} \dif A - |\dif B|^{p - 2} \dif B, \dif A - \dif B\rangle = 0.$$
By the Cauchy-Schwarz and Young inequalities, the integrand is nonnegative, and positive iff $\dif A \neq \dif B$.
Therefore $\dif A = \dif B$ on almost all of $V$.
Since $A, B$ are in flat Coulomb gauge and have the same full trace on $\partial D$, the flat Hodge decomposition implies that $v := A - B$ is flatly harmonic on $D$.
On $D \setminus \overline V$, $v_i \leq \varepsilon$, and $v$ is continuous, so $v_i \leq \varepsilon$ on $\partial V$.
By the (trivial, by flatness) Weitzenb\"ock formula, $v_i$ is flatly harmonic, so by the maximum principle, $v_i \leq \varepsilon$ on $\overline V$.
But then $A_i \leq B_i + \varepsilon$ on $\overline V$, so $V$ is empty, a contradiction.
\end{proof}

\begin{proposition}
Let $A$ be a variational solution of the $\infty$-Maxwell equation.
Then $\dif A$ has absolutely best comass.
\end{proposition}
\begin{proof}
We follow \cite[Theorem 5]{lindqvist2015notes}.
Let $A_p$ solve the $p$-Maxwell equation in $U$ with $\iota_{\partial U}^* A_p = \iota^*_{\partial U} A$, let $D \subseteq U$ be a small ball, and let $B_p$ solve the $p$-Maxwell equation in $D$ with $B_p|_{\partial D} = A|_{\partial D}$.

We claim that $B_p \to A$ in $C^0(\overline D)$.
Let $\varepsilon > 0$; then for $p$ large enough, $\|A_p - A\|_{C^0} \leq \varepsilon$, hence
$$\|B_p - A\|_{C^0} \leq \|B_p - A_p\|_{C^0} + \|A_p - A\|_{C^0} \leq \|B_p - A_p\|_{C^0} + \varepsilon.$$
On $\partial D$, $(B_p)_i \leq (A_p)_i + \varepsilon$, so by the comparison principle, the same holds on $\overline D$.
In particular, 
$$\|B_p - A_p\|_{C^0} \leq \sqrt 3 \varepsilon.$$
So $\|B_p - A\|_{C^0} \leq (1 + \sqrt 3) \varepsilon$ as desired.

Now let $B \in C^\infty(D, \Omega^1)$ be arbitrary with $B|_{\partial D} = A|_{\partial D}$.
Then for any $\varepsilon > 0$, any $q$ large enough depending on $\varepsilon, U$, and any $p \geq q$,
$$\|\dif B_p\|_{L^q} \leq (1 + \varepsilon) \|\dif B_p\|_{L^p} \leq (1 + \varepsilon) \|\dif B\|_{L^p} \leq (1 + 2\varepsilon) \|\dif B\|_{C^0}$$
by H\"older's inequality and the fact that $B_p$ is a variational solution of the $p$-Maxwell equation.
In particular, $\dif B_p \to \dif A$ weakly in $W^{1, q}$, hence by Fatou's lemma
$$\|\dif A\|_{L^q(D)} \leq \|\dif B_p\|_{L^q} \leq (1 + 2\varepsilon) \|\dif B\|_{C^0} = (1 + 2\varepsilon) L(\dif B).$$
Taking $\varepsilon \to 0$ (hence $q \to \infty$) we conclude that $L(\dif A) \leq L(\dif B)$.
\end{proof}

\printbibliography
\end{document}