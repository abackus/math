\documentclass[reqno,11pt]{amsart}
\usepackage[letterpaper, margin=1in]{geometry}
\RequirePackage{amsmath,amssymb,amsthm,graphicx,mathrsfs,url,slashed,subcaption}
\RequirePackage[usenames,dvipsnames]{xcolor}
\RequirePackage[colorlinks=true,linkcolor=Red,citecolor=Green]{hyperref}
\RequirePackage{amsxtra}
\usepackage{cancel}
\usepackage{tikz-cd}
%\usepackage[T1]{fontenc}

% \setlength{\textheight}{9.3in} \setlength{\oddsidemargin}{-0.25in}
% \setlength{\evensidemargin}{-0.25in} \setlength{\textwidth}{7in}
% \setlength{\topmargin}{-0.25in} \setlength{\headheight}{0.18in}
% \setlength{\marginparwidth}{1.0in}
% \setlength{\abovedisplayskip}{0.2in}
% \setlength{\belowdisplayskip}{0.2in}
% \setlength{\parskip}{0.05in}
%\renewcommand{\baselinestretch}{1.05}

\title{Tug-of-war solutions to $\infty$-elliptic systems}
\author{Aidan Backus}
\address{Department of Mathematics, Brown University}
\email{aidan\_backus@brown.edu}
\date{\today}

\newcommand{\NN}{\mathbf{N}}
\newcommand{\ZZ}{\mathbf{Z}}
\newcommand{\QQ}{\mathbf{Q}}
\newcommand{\RR}{\mathbf{R}}
\newcommand{\CC}{\mathbf{C}}
\newcommand{\DD}{\mathbf{D}}
\newcommand{\PP}{\mathbf P}
\newcommand{\MM}{\mathbf M}
\newcommand{\II}{\mathbf I}
\newcommand{\Hyp}{\mathbf H}
\newcommand{\Sph}{\mathbf S}
\newcommand{\Group}{\mathbf G}
\newcommand{\GL}{\mathbf{GL}}
\newcommand{\Orth}{\mathbf{O}}
\newcommand{\SpOrth}{\mathbf{SO}}
\newcommand{\Ball}{\mathbf{B}}

\newcommand*\dif{\mathop{}\!\mathrm{d}}

\DeclareMathOperator{\card}{card}
\DeclareMathOperator{\dist}{dist}
\DeclareMathOperator{\id}{id}
\DeclareMathOperator{\supp}{supp}
\DeclareMathOperator{\Teich}{Teich}
\DeclareMathOperator{\tr}{tr}

\newcommand{\Leaves}{\mathscr L}
\newcommand{\Lagrange}{\mathcal L}
\newcommand{\Hypspace}{\mathscr H}

\newcommand{\Chain}{\underline C}

\newcommand{\Two}{\mathrm{I\!I}}

\newcommand{\normal}{\mathbf n}
\newcommand{\radial}{\mathbf r}
\newcommand{\evect}{\mathbf e}
\newcommand{\vol}{\mathrm{vol}}

\newcommand{\diam}{\mathrm{diam}}
\newcommand{\Ell}{\mathrm{Ell}}
\newcommand{\inj}{\mathrm{inj}}
\newcommand{\Lip}{\mathrm{Lip}}
\newcommand{\Riem}{\mathrm{Riem}}

\newcommand{\Min}{\mathrm{Min}}
\newcommand{\Max}{\mathrm{Max}}

\newcommand{\dfn}[1]{\emph{#1}\index{#1}}

\renewcommand{\Re}{\operatorname{Re}}
\renewcommand{\Im}{\operatorname{Im}}

\newcommand{\loc}{\mathrm{loc}}
\newcommand{\cpt}{\mathrm{cpt}}

\def\Japan#1{\left \langle #1 \right \rangle}

\newtheorem{theorem}{Theorem}[section]
\newtheorem{badtheorem}[theorem]{``Theorem"}
\newtheorem{prop}[theorem]{Proposition}
\newtheorem{lemma}[theorem]{Lemma}
\newtheorem{sublemma}[theorem]{Sublemma}
\newtheorem{proposition}[theorem]{Proposition}
\newtheorem{corollary}[theorem]{Corollary}
\newtheorem{conjecture}[theorem]{Conjecture}
\newtheorem{axiom}[theorem]{Axiom}
\newtheorem{assumption}[theorem]{Assumption}

\newtheorem{mainthm}{Theorem}
\renewcommand{\themainthm}{\Alph{mainthm}}

\newtheorem{claim}{Claim}[theorem]
\renewcommand{\theclaim}{\thetheorem\Alph{claim}}
% \newtheorem*{claim}{Claim}

\theoremstyle{definition}
\newtheorem{definition}[theorem]{Definition}
\newtheorem{remark}[theorem]{Remark}
\newtheorem{example}[theorem]{Example}
\newtheorem{notation}[theorem]{Notation}

\newtheorem{exercise}[theorem]{Discussion topic}
\newtheorem{homework}[theorem]{Homework}
\newtheorem{problem}[theorem]{Problem}

\makeatletter
\newcommand{\proofpart}[2]{%
  \par
  \addvspace{\medskipamount}%
  \noindent\emph{Part #1: #2.}
}
\makeatother



\numberwithin{equation}{section}


% Mean
\def\Xint#1{\mathchoice
{\XXint\displaystyle\textstyle{#1}}%
{\XXint\textstyle\scriptstyle{#1}}%
{\XXint\scriptstyle\scriptscriptstyle{#1}}%
{\XXint\scriptscriptstyle\scriptscriptstyle{#1}}%
\!\int}
\def\XXint#1#2#3{{\setbox0=\hbox{$#1{#2#3}{\int}$ }
\vcenter{\hbox{$#2#3$ }}\kern-.6\wd0}}
\def\ddashint{\Xint=}
\def\dashint{\Xint-}

\usepackage[backend=bibtex,style=alphabetic,giveninits=true]{biblatex}
\renewcommand*{\bibfont}{\normalfont\footnotesize}
\addbibresource{tug_of_war.bib}
\renewbibmacro{in:}{}
\DeclareFieldFormat{pages}{#1}

\newcommand\todo[1]{\textcolor{red}{TODO: #1}}


\begin{document}
\begin{abstract}
Viscosity solutions don't work
\end{abstract}

\maketitle

%%%%%%%%%%%%%%%%%%%%%%%%%%%%%%%%%%%%%%%%%%%%%%%%%%%%%%%
\section{Introduction}
We have motivated the definition of tight forms by geometric topology.
However, we expect them to be analytically interesting as well.
Here we propose a program to study tight forms analytically, which we hope to carry out in a later work.

First recall that $\infty$-harmonic maps $v: M \to \Sph^1$ solve the $\infty$-Laplace equation 
\begin{equation}\label{infinity laplacian}
\nabla_{(\dif v)^\sharp} |\dif v|^2 = 0.
\end{equation}
However, $\infty$-harmonic maps need not be $C^2$, and (\ref{infinity laplacian}) cannot be written in divergence form, so $\infty$-harmonic maps need to be understood as viscosity solutions of (\ref{infinity laplacian}) \cite{Crandall2008}.

If $F$ is a tight $2$-form (say), it solves 
$$\nabla_{(\iota_X F)^\sharp} |F|^2 = 0$$
for every vector field $X$ (in particular, the coordinate vector fields).
Locally we have $F = \dif A$ and so this equation is of the same form as (\ref{infinity laplacian}).
However, viscosity solutions are defined in terms of the maximum principle and so are only suitable for systems in diagonal form, as in \cite{Ishii92}.

There is an equivalent notion of solution to viscosity solution for the $\infty$-Laplace equation, that of a \dfn{tug-of-war solution}.
There exists a continuous-time stochastic game called tug-of-war such that a function is a viscosity solution of (\ref{infinity laplacian}) iff it is a value function of tug-of-war \cite{Peres_2008}.
While the notion of ``value differential form" (as opposed to value function) is somewhat conceptually awkward, it is not unworkable, and since it is exactly the trilinear structure of (\ref{infinity laplacian}) which is used in the proof of equivalence of viscosity and tug-of-war solutions, a notion of tug-of-war solution of (\ref{infinity Maxwell}) should be in reach.

\section{Derivation of the \texorpdfstring{$\infty$-Maxwell}{infinity-Maxwell} equation}

\subsection{The \texorpdfstring{$\infty$-Maxwell equation}{infinity-Maxwell equation}}
We now consider the Euler-Lagrange equation for $\infty$-tight forms.
We shall not need this material in the sequel, but it could make for a very interesting future research project.

\begin{proposition}
Suppose that $F$ has absolutely best comass, regularity $C^1$, and no points with $F = 0$. Then
\begin{equation}\label{infinity Maxwell}
	F^{ij} F^{k\ell} \nabla_j F_{k \ell} = 0.
\end{equation}
\end{proposition}
\begin{proof}
For a covariant $2$-tensor $T$, let $T^{\rm as}$ be its antisymmetrization, and let
$$f(x, T) := |T^{\rm as}|_{g(x)}^2.$$
Working locally, we may write $F = \dif A$ for some $A$, which we may assume to be in Coulomb gauge and therefore $C^2$ by \todo{elliptic regularity}.
Since $A$ has absolutely best curl and $(\nabla A)^{\rm as} = \dif A$, $A$ is an absolute minimizer (see \cite[Definition 5.1]{Barron2001}) of the essential supremum of $f(\cdot, \nabla A)$.
By \cite[Theorem 5.2]{Barron2001},
\begin{equation}\label{ELA}
	\frac{\partial f}{\partial T_{ij}}(x, \nabla A(x)) \partial_j (f(x, \nabla A(x))) = 0.
\end{equation}
We then compute
\begin{align*}
\partial_j (f(x, \nabla A(x))) &= \nabla_j |\dif A|^2 = 2 \langle \dif A, \nabla_j \dif A\rangle = F^{k\ell} \nabla_j F_{k\ell} \\
\frac{\partial f}{\partial T_{ij}}(x, \nabla A(x)) &= (\nabla A^{\rm as})^{ij} = F^{ij}. \qedhere
\end{align*}
\end{proof}

For the sake of discussion, let $d = 3$.
Then (\ref{infinity Maxwell}) is the $L^\infty$ analogue of the Maxwell equation $\dif^* F = 0$ (where $F$ is a closed $2$-form), so we call it the \dfn{$\infty$-Maxwell equation}.
It is somewhat easier to visualize if $d = 3$ and we view $F = \dif X^\flat$ for a vector field $X$.
Then the $\infty$-Maxwell equation is equivalent to
$$(\nabla \times X) \times \nabla_{\nabla \times X} (\nabla \times X) = 0.$$

\printbibliography
\end{document}