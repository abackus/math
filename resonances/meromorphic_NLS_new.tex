\documentclass[reqno,12pt,letterpaper]{amsart}
\RequirePackage{amsmath,amssymb,amsthm,graphicx,mathrsfs,url}
\RequirePackage[usenames,dvipsnames]{color}
\RequirePackage[colorlinks=true,linkcolor=Red,citecolor=Green]{hyperref}
\RequirePackage{amsxtra}
\usepackage{tikz-cd}

\setlength{\textheight}{8.50in} \setlength{\oddsidemargin}{0.00in}
\setlength{\evensidemargin}{0.00in} \setlength{\textwidth}{6.08in}
\setlength{\topmargin}{0.00in} \setlength{\headheight}{0.18in}
\setlength{\marginparwidth}{1.0in}
\setlength{\abovedisplayskip}{0.2in}
\setlength{\belowdisplayskip}{0.2in}
\setlength{\parskip}{0.05in}
\renewcommand{\baselinestretch}{1.10}

\title[Meromorphic continuation of the NLS resolvent]{Meromorphic continuation of the NLS resolvent}
\author{Aidan Backus}
\date{May 2021}

\newcommand{\NN}{\mathbf{N}}
\newcommand{\ZZ}{\mathbf{Z}}
\newcommand{\QQ}{\mathbf{Q}}
\newcommand{\RR}{\mathbf{R}}
\newcommand{\CC}{\mathbf{C}}
\newcommand{\DD}{\mathbf{D}}
\newcommand{\PP}{\mathbf P}

\DeclareMathOperator{\card}{card}
\DeclareMathOperator{\ch}{ch}
\DeclareMathOperator{\diag}{diag}
\DeclareMathOperator{\dom}{dom}
\DeclareMathOperator{\Gal}{Gal}
\DeclareMathOperator{\id}{id}
\DeclareMathOperator{\rank}{rank}
\DeclareMathOperator*{\Res}{Res}
\DeclareMathOperator{\sgn}{sgn}
\DeclareMathOperator{\singsupp}{sing~supp}
\DeclareMathOperator{\Spec}{Spec}
\DeclareMathOperator{\supp}{supp}
\newcommand{\tr}{\operatorname{tr}}

\newcommand{\dbar}{\overline \partial}

\DeclareMathOperator{\atanh}{atanh}
\DeclareMathOperator{\csch}{csch}
\DeclareMathOperator{\sech}{sech}

\DeclareMathOperator{\Ell}{Ell}
\DeclareMathOperator{\WF}{WF}

\newcommand{\pic}{\vspace{30mm}}
\newcommand{\dfn}[1]{\emph{#1}\index{#1}}

\renewcommand{\Re}{\operatorname{Re}}
\renewcommand{\Im}{\operatorname{Im}}


\newtheorem{theorem}{Theorem}[section]
\newtheorem{badtheorem}[theorem]{``Theorem"}
\newtheorem{prop}[theorem]{Proposition}
\newtheorem{lemma}[theorem]{Lemma}
\newtheorem{proposition}[theorem]{Proposition}
\newtheorem{corollary}[theorem]{Corollary}
\newtheorem{conjecture}[theorem]{Conjecture}
\newtheorem{axiom}[theorem]{Axiom}

\theoremstyle{definition}
\newtheorem{definition}[theorem]{Definition}
\newtheorem{remark}[theorem]{Remark}
\newtheorem{example}[theorem]{Example}

\newtheorem{exercise}[theorem]{Discussion topic}
\newtheorem{homework}[theorem]{Homework}
\newtheorem{problem}[theorem]{Problem}

\newtheorem*{ack}{Acknowledgements}
\newtheorem*{notate}{Notation}

%\usepackage{color}
%\hypersetup{%
%    colorlinks=true, % make the links colored%
%    linkcolor=blue, % color TOC links in blue
%    urlcolor=red, % color URLs in red
%    linktoc=all % 'all' will create links for everything in the TOC
%Ning added hyperlinks to the table of contents 6/17/19
%}

\usepackage[backend=bibtex,style=alphabetic,maxcitenames=50,maxnames=50]{biblatex}
\addbibresource{resonances.bib}
\renewbibmacro{in:}{}
\DeclareFieldFormat{pages}{#1}

\begin{document}
\begin{abstract}
\end{abstract}

\maketitle

\tableofcontents

\section{Preliminaries}
\subsection{Algebraic geometry}
A regular map $F$ is a \dfn{submersion} if $\nabla F$ is always surjective on tangent spaces, and if $F$ is a submersion from a variety of dimension $n$ to a variety of dimension $k$, the fibers of $F$ are isomorphic varieties of dimension $n - k$.

If $f: U \to \CC^m$ is a holomorphic map and $F(z, f(z)) = 0$ for all $z$, we can define the \emph{Riemann surface} of $f$ to be the smooth manifold $X = F^{-1}(0)$. Then the map $p: X \to \CC^m$, $p(z, w) = w$, defines an analytic continuation $(X, p, \iota)$ of $f$ after one chooses an embedding $\iota: U \to X$, at least as long as $F$ is a submersion.

If $f$ is a holomorphic function, we define its maximal analytic continuation in the obvious way.
We will be interested in the case that the maximal analytic continuation $f: X \to \CC$ satisfies that $X$ is the Riemann surface of $f$, and $X$ is a variety.

We need the following lemma.
\begin{lemma}
% https://math.stackexchange.com/questions/2554046/extending-isomorphism-of-punctured-riemann-surfaces
\label{compact Riemann surfaces}
Let $X, Y$ be algebraic curves over $\CC$ and $S \subset X$ a finite set. Then if $\iota: X \setminus S \to Y$ is an embedding, $\iota$ extends to an isomorphism $\overline \iota: X \to Y$.
\end{lemma}

\subsection{Infinite-dimensional K\"ahler geometry}
The definition of $\dbar$ makes sense even in infinite dimensions.
Indeed if $B$ is a Banach space over $\CC$, then for any Banach basis $(z_\alpha = x_\alpha + iy_\alpha)_\alpha$, we have
$$\frac{\partial f}{\dbar z} = \frac{1}{2} \sum_\alpha \frac{\partial f}{\partial x_\alpha} + i\frac{\partial f}{\partial y_\alpha}$$
whenever the infinite series converges.
Thus the notion of holomorphy makes sense in infinite dimensions.
As a sanity check, any bounded linear functional
$$z \mapsto \sum_\alpha c_\alpha z_\alpha$$
is holomorphic.
So the notion of complex manifold still makes sense in infinite dimensions.

We recall some K\"ahler geometry, that still makes sense in this setting.
\begin{definition}
Let $X$ be a complex manifold. A \dfn{hermitian form} on $X$ is a smoothly varying inner product on each tangent space of $X$.
If $h$ is a hermitian form on $X$ and $\Im h$ is closed, we say that $(X, h)$ is a \dfn{K\"ahler manifold}, with symplectic form $\Im h$ and Riemannian metric $\Re h$.
\end{definition}
Here these definitions make sense since $h(u, u)$ is always real for any tangent vector $u$, by definition of inner product.
Note that in coordinates $z = x + iy$ we can write
$$h(u, v) = \sum_\alpha x_\alpha(u) x_\alpha(v) + y_\alpha(u) y_\alpha(v) + i\sum_\alpha x_\alpha(v) y_\alpha(u) - x_\alpha(u) y_\alpha(v)$$
whence the symplectic form
$$\omega(u, v) = \sum_\alpha x_\alpha(v) y_\alpha(u) - x_\alpha(u) y_\alpha(v).$$
So it really is locally a symplectic form!
In particular it is exact -- the condition that it is closed ensures that $\omega$ is globally a symplectic form.
If $X$ is a Hilbert space over $\CC$ for example, so $X$ is canonically identified with its tangent spaces, $X$ is K\"ahler, since $X$ has a global chart.

So let $(\mathcal H, h)$ be a Hilbert space over $\CC$, viewed as a K\"ahler manifold.
Then we get a Riemannian metric $g$ and symplectic form $\omega$, which we can view as inner and symplectic products on $\mathcal H$ since we identify $\mathcal H$ with its tangent spaces. In particular $\omega$ induces an isomorphism $\mathcal H^* \to \mathcal H$.
So as to not mix it up with the isomorphism induced by $h$ we write $\eta \mapsto \eta^\sharp$ for the isomorphism induced by $\omega$.

Thus if $H: \mathcal H \to \RR$ is any smooth function, its differential $dH: \mathcal H \to \mathcal H^*$ induces a vector field
$$dH^\sharp: \mathcal H \to \mathcal H.$$
In particular, an integral curve $u$ of $dH^\sharp$ is tangent to $dH^\sharp$ and hence is $\omega$-orthogonal to $\ker dH$; since $\omega$ is a symplectic form that means that $\dot u \in \ker dH$; that is, $H$ is constant along integral curves of $dH^\sharp$.

\subsection{Linearizing the cubic NLS}
In this section we linearize the cubic NLS with small Dirac impurity
$$i \frac{\partial u}{\partial t} + \frac{1}{2} \frac{\partial^2 u}{\partial x^2} + q\delta_0(x)u + u|u|^2 = 0$$
where $0 < |q| \ll 1$. This was adapted from Holmer-Zworski.

Let $\mathcal H_\CC = H^1(\RR \to \CC)$. By Morrey's inequality we have
$$H^1(\RR) \subseteq C^{0+1/2}(\RR);$$
thus we may introduce the $q$-Hamiltonian
$$H_q(u) = \frac{1}{4} \int_{-\infty}^\infty \left|\frac{\partial u}{\partial x}\right|^2 + |u(x)|^4 ~dx - q|u(0)|^2.$$
Then one has
$$\cdot u(x) = i\left(\frac{1}{2} \frac{\partial^2 u}{\partial x^2} + u(x)|u(x)|^2 + q\delta_0(x)u(x)\right)$$
This reflects that solutions of the cubic NLS obey conservation of energy.
One has also conservation of mass
$$\frac{\partial}{\partial t} ||u||_{L^2}^2 = 2 \Re \int_{-\infty}^\infty iu(x) \left(\frac{\partial^2}{\partial x^2} + q\delta_0(x) + |u(x)|^2\right)\overline u(x) ~dx$$
which follows because the operator $\frac{\partial^2}{\partial x^2} + q\delta_0(x) + |u(x)|^2$ is real, so that if $u$ is totally real or totally imaginary then the integrand is totally imaginary; this then extends to arbitrary $u$ by linearity.

To see well-posedness we recall the bound
$$||u(t)||_{H^1}^2 \lesssim H_q(u) + ||u(t)||_{L^2}^6 + ||u(t)||_{L^2}^2.$$
This follows because the cubic NLS is subcritical (the quintic is critical).
Since this is a uniform bound, we can show (see Holmer-Zworski \S2.2) that there is a $T > 0$ which only depends on the \emph{initial} data $u(0)$ such that if $u(t)$ exists then so does $u(s)$ for $t < s < T$.
Indeed, if $\Phi$ is the nonlinear transformation $H^1 \to H^1$ whose fixed points are exactly solutions to the cubic NLS, then
\begin{align*}
||\Phi(u) - \Phi(v)||_{L^\infty([0, T] \to H^1)} &\lesssim T(||u||_{L^\infty([0, T] \to H^1)} - ||v||_{L^\infty([0, T] \to H^1)})^2 ||u - v||_{L^\infty([0, T] \to H^1)} \\
&\lesssim T(H_q(u) + ||u(0)||_{L^2} + ||u(0)||_{L^2}^3 + H_q(v) + ||v(0)||_{L^2} + ||v(0)||_{L^2}^3),
\end{align*}
so we can apply Banach's fixed point theorem provided that
$$T \lesssim (H_q(u) + ||u(0)||_{L^2} + ||u(0)||_{L^2}^3 + H_q(v) + ||v(0)||_{L^2} + ||v(0)||_{L^2}^3)^{-1}.$$
Therefore we have global well-posedness in $H^1$.

To define the linearization, we consider the modified Hamiltonian
$$H(u, \lambda) = H_q(u) + \frac{\lambda^2}{4} ||u||_{L^2}^2$$
with $\lambda^2/4$ a Lagrange multiplier. Let $F(\cdot, \lambda)$ be the linearization of the Hamiltonian flow of $H(\cdot, \lambda)$. Then the following are equivalent:
\begin{enumerate}
\item $u$ is a ground state of the cubic NLS with Dirac impurity.
\item $H_q(u)$ is minimized subject to $||u||_{L^2}^2$ given.
\item There is a unique $\lambda$ for which $u$ is a stationary point of $H(\cdot, \lambda)$ and $F(u, \lambda)$ is positive-definite.
\end{enumerate}
By a rescaling argument and the formula
$$u_\lambda(x) = \lambda \sech(\lambda|x| + \tanh^{-1}(q/\lambda))$$
for the ground state, we see that as long as we choose $\lambda > |q|$ (since $\atanh$ has a pole at $1$!) we can find a ground state with Lagrange multiplier $\lambda^2/4$.
In this case Holmer-Zworski show that
$$F(\lambda) = -i\begin{bmatrix}L_+ \\ & L_-\end{bmatrix}.$$
where
\begin{align*}
2L_+ &= \lambda^2 - \partial^2 - 6v^2 - 2q\delta_0\\
2L_- &= \lambda^2 - \partial^2 - 2v^2 - 2q\delta_0.
\end{align*}
Here this matrix is written in the basis $u = (\Re u, \Im u)$. Conjugating to the basis $u = (u, \overline u)$ we obtain the linearized matrix in the next section.
The point is to consider the ground state as a steady state in the Hamiltonian flow of the modified Hamiltonian.

\subsection{Notation}
Let $j = \diag(1, -1)$. If $A$ is defined on $\RR$ and can be pushed forward or pulled back along $\tanh: \RR \to (-1, 1)$, we define $A^\flat$ to be the pushforward or pullback of $A$.
We let $\sharp$ denote the inverse of $\flat$.

If $V$ is a Radon measure which is absolutely continuous with respect to Lebesgue measure near $x$, we write $V(x)$ for the value of the Radon-Nikod\'ym derivative of $V$ at $x$.
In particular, by a smooth Radon measure we mean a Radon measure whose Radon-Nikody\'m derivative is smooth.

\section{Analytic continuation of the free resolvent}
Let $D = -i\partial$ be the momentum observable.
Then
$$H_0 - z = \begin{bmatrix}
D^2 + 1 - z\\
&-D^2 - 1 -z
\end{bmatrix}.$$
The equation $H_0u = f$ is the linearized free NLS.
We are interested in the resolvent $R_0(z) = (H_0 - z)^{-1}$.

Let
$$S = (-\infty, -1] \cup [1, \infty)$$
be the spectrum of $H_0$. Thus $R_0(z)$ has poles at $\pm 1$ and branch cuts along $S$.

Let $\Sigma \subset \CC^2$ be the affine variety defined by
\begin{equation}
\label{definition of affine variety}
w_1^2 + w_2^2 + 2 = 0.
\end{equation}
On $\Sigma$ we define
\begin{equation}
\label{free analytic continuation}
\overline R_0(x, y, w)_{jj} = (-1)^{j-1} \frac{i}{2} \frac{\exp(iw_j|x-y|)}{w_j},\quad j \in \{1, 2\}
\end{equation}
and $\overline R_0(x, y, w)_{jk} = 0$ if $j \neq k$.

\begin{lemma}
The resolvent $R_0(x, y)$ is holomorphic on $\CC \setminus S$.
Furthermore, the regular map
\begin{equation}
\label{recovering the standard branch}
\psi(w) = \frac{w_1^2 - w_2^2}{2}
\end{equation}
satisfies $\overline R_0(x, y) = R_0(x, y) \circ \psi$.
\end{lemma}
\begin{proof}
Let $R$ be the resolvent of the classical Schrodinger operator,
$$R(\lambda)(D - \lambda^2) = 1.$$
Then
$$R(x, y, \lambda) = \frac{i}{2\lambda} e^{i\lambda|x-y|}.$$
Here we take $\lambda = \sqrt z$ where we are taking the branch of $\sqrt\cdot$ which forces $\Im \lambda \geq 0$; i.e.
$$\sqrt{re^{i\theta}} = e^{i\theta/2}\sqrt r,\quad \theta \in (0, 2\pi)$$
(so that $\theta/2 \in [0, \pi]$). This branch is discontinuous on $\RR_+$.

We need to treat the two resolvents $D^2 + 1 - z$ and $-D^2 - 1 - z$ separately.
For the first resolvent, $D^2 + 1 - z$, we are plugging in $\lambda^2 = z - 1$, and we need $\Im \lambda \geq 0$.
Taking $\widetilde z = z - 1$, we see that we need $\widetilde z \notin \RR_+$, i.e. $z \notin [1, \infty)$.
Similarly, for $-D^2 - 1 - z = -(D^2 + 1 + z)$, this is defined when $z \notin (-\infty, -1]$. Thus
\begin{equation}
\label{uncontinued resolvent}
R_0(x, y, z)_{jj} = (-1)^{1-j}\frac{i}{2} \exp(i\sqrt{(-1)^{1-j}z -1}|x-y|)((-1)^{1-j}z -1)^{-1/2}
\end{equation}
and $R_0(x,y,z)_{jk} = 0$ if $j \neq k$.
This is clearly holomorphic in $z$ for $z \notin S$.

Now fix $(x, y)$; then we can view $R_0$ as a map $\CC \setminus S \to \CC^2$ by
$$R_0(z)_j = (-1)^{1-j} \frac{i}{2} \exp(i\sqrt{(-1)^{1-j}z -1}|x-y|)((-1)^{1-j}z -1)^{-1/2}).$$
The map $w \mapsto \exp(iw|x-y|)/w$ is clearly holomorphic on $\CC \setminus 0$ and has a simple pole at $0$.
In particular no choice of branch is made for it; so to understand the Riemann surface of $R_0$ it suffices to consider the Riemann surface of the function $z \mapsto w$,
$$w_j = \sqrt{(-1)^{1-j}z-1}.$$
That is, $w_j^2 = (-1)^{1-j}z - 1$, and thus can be inverted as $\psi(w) = z$.
Provided that $w$ satisfies that condition for $z$ given, one has (\ref{definition of affine variety}).
Plugging $w$ into (\ref{uncontinued resolvent}) we get (\ref{free analytic continuation}).
\end{proof}

Let $\overline \Sigma$ be the projective completion of $\Sigma$, defined as the copy of $\PP^1$ in $\PP^2$ cut out by
$$2w_0^2 + w_1^2 + w_2^2 = 0.$$

\begin{lemma}
One has a rational parametrization $\zeta$ of $\overline \Sigma$ with
$$w(\zeta) = [2\zeta:-i(\zeta^2 + 2):2 - \zeta^2)].$$
\end{lemma}
\begin{proof}
Set $W_0 = w_0/w_1$, $W_2 = w_2/w_1$, and consider the line
$$W_2 = i - \zeta W_0$$
through $(W_0, W_2) = (0, i)$.
Plugging $w_2$ into the affine curve $2W_0^2 + W^2 + 1 = 0$ and solving for $W_0$, we get
$$W_0 = 2i \frac{\zeta}{\zeta^2 + 2}.$$
Setting $w_1 = -i(\zeta^2 + 2)$ the claim now follows.
\end{proof}

In this parametrization we have
\begin{align}\label{analytic continuation in zeta 1}
\overline R_0(x, y, \zeta)_1 &= \frac{\zeta}{\zeta^2 + 2} \exp\left(|x - y|\frac{\zeta^2 + 2}{2\zeta}\right)\\
\label{analytic continuation in zeta 2}
\overline R_0(x, y, \zeta)_2 &= i \frac{\zeta}{\zeta^2 - 2} \exp\left(-i|x - y|\frac{\zeta^2 - 2}{2\zeta}\right).
\end{align}

Now it is easy to see that there are four poles of $\overline R_0(x, y)$, namely $\zeta = \sigma \sqrt{2\tau}$ where $\sigma,\tau \in S^0 = \{\pm 1\}$; and two essential singularities, namely $\zeta = 0, \infty$.
The large amount of symmetry in the poles, and the location of the essential singularities, are why this parametrization is especially convenient.

\subsection{Branches of the analytic continuation}
We now construct embeddings of $\CC \setminus S$ into $\Sigma$.

Let
$$\sqrt{re^{i\theta}}_\pm = e^{i\theta/2}\sqrt r,\quad\theta \in (0, \pm 2\pi),$$
so $\sqrt{re^{i\theta}}_\pm \in \CC_\pm$, where $\CC_\pm = \{x + iy: \pm y > 0\}$.

\begin{definition}
A \dfn{branch pair} is a pair $(\sigma, \tau) \in (S^0)^2$.
Given a branch pair $(\sigma, \tau)$, define the \dfn{induced embedding}
$$\varphi_{(\sigma, \tau)}(z) = (\sqrt{z-1}_\sigma, \sqrt{-z-1}_\tau)$$
and let $\Sigma_{(\sigma, \lambda)} = (\CC_\sigma \times \CC_\lambda) \cap \Sigma$ be the \dfn{induced sheet}.
The \dfn{physical sheet} of $\Sigma$ is $\Sigma_{(+,+)}$.
\end{definition}

\begin{lemma}
For every branch pair $\kappa$, the induced embedding $\varphi_\kappa$ is an isomorphism
$$\varphi_\kappa: \CC \setminus S \to \Sigma_\kappa$$
which is a section of $\psi$.
\end{lemma}
\begin{proof}
Let $\kappa = (\sigma, \tau)$.
By definition of $S$, $\varphi_{(\sigma, \tau)}$ is holomorphic on $\CC \setminus S$ and is a section of $\psi$.
Moreover, as $\sqrt \cdot_\sigma$ and $\sqrt\cdot_\tau$ are injective, $\varphi_{(\sigma, \tau)}$ is injective on $\CC \setminus S$ and thus an embedding by the holomorphic open mapping theorem. Let $U_\kappa$ be its image.
Then $U_\kappa = \Sigma_\kappa$; to see this, we first note that the inclusion $U_\kappa \subseteq \Sigma_\kappa$ follows from the definition of $\varphi_\kappa$.
Conversely, if $w \in \Sigma_\kappa$, then we get $\varphi_\kappa(\psi(w)) = w$, so $w \in U_\kappa$.
\end{proof}

Let $\mathring \Sigma = \bigcup_{\kappa \in (S^0)^2} \Sigma_\kappa$; then $\Sigma \setminus \mathring \Sigma$ consists of $w \in \Sigma$ such that $\Im w_1 = 0$ or $\Im w_2 = 0$, so $\mathring \Sigma$ is an open dense subset of $\Sigma$.

All four sheets $\Sigma_{(\sigma,\tau)}$ are isomorphic to $\CC \setminus S$, so they each have two cuts, a \emph{left cut} defined by $\Re z = 0, \Im z \leq -1$, and a \emph{right cut} defined by $\Re z = 0, \Im z \geq 1$.
If one crosses the left cut of $\Sigma_{(\sigma, \tau)}$, they move to $\Sigma_{(\sigma, -\tau)}$ and if one crosses the right cut they move to $\Sigma_{(-\sigma, \tau)}$.
Thus we have the following diagram, where a line between two branch pairs $\kappa,\delta$ means that one can cross a cut in $\Sigma_\kappa$ to get to $\Sigma_\delta$ (a symmetric relation).
$$\begin{tikzcd}
&(+,+) \arrow[dash,dl] \arrow[dash,dr]\\
(+, -) \arrow[dash,dr] && (-,+) \arrow[dash,dl]\\
&(-,-)
\end{tikzcd}$$

\begin{proposition}
Let $\kappa$ be a branch pair. Then $\kappa$ defines a maximal meromorphic continuation of $R_0(x, y)$ by
\begin{equation}\begin{tikzcd}\label{continuationDiagram}
\CC \setminus S \arrow["\varphi_\kappa", r] \arrow["R_0{(x, y)}", dr, swap] & \Sigma \arrow["\overline R_0{(x, y)}", d] \\
& \PP^1 \times \PP^1.
\end{tikzcd}\end{equation}
\end{proposition}
\begin{proof}
First, (\ref{continuationDiagram}) actually defines a meromorphic continuation, since $\varphi_\kappa$ is an embedding which is a section of $\psi$, and $\overline R_0(x, y) = R_0(x, y) \circ \psi$.
Furthermore, it is clear that (\ref{analytic continuation in zeta 1}, \ref{analytic continuation in zeta 2}) define meromorphic maps on $\Sigma \cong \PP^1 \setminus \{0, \infty\}$ which do not extend further.
\end{proof}


\subsection{Bounds on the free resolvent}
Let $\mathcal H^k = H^k \oplus H^k$ where $H^k$ is the Sobolev space $H^k = W^{2,k}$.
We are interested in the $\mathcal H^0 \to \mathcal H^2$ norm of $R_0(z)$.

\begin{lemma}
\label{bounds on partial resolvent}
Let $L > 0$ and let $\rho$ be a cutoff such that $\supp \rho \subseteq (-L, L)$.
Let $\chi$ be a cutoff such that $\chi = 1$ on $\supp \rho$.
Fix a sign $\sigma$.
Let
$$G_\pm(z) = \chi(x) \frac{\exp(i\sqrt{\pm z - 1}_\sigma|x-y|)}{\sqrt{\pm z - 1}_\sigma} \rho(y).$$
Then for every $z$ such that $\pm z - 1$ is in the domain of $\sqrt\cdot_\sigma$, one has
$$||G_\pm(z)||_{L^2 \to L^2} \lesssim L\frac{\exp(4|\Im \sqrt{\pm z - 1}_\sigma)}{|\sqrt{\pm z - 1}_\sigma}.$$
\end{lemma}
\begin{proof}
By Schur's test,
\begin{equation}
\label{schur bound}
||G_\pm(z)||_{L^2 \to L^2}^2 \leq \sup_{(x, y) \in \RR^2} ||G_\pm (\cdot, y, z)||_{L^1}\cdot||G_\pm (x, \cdot, z)||_{L^1}.
\end{equation}
Since
\begin{align*}
||G_\pm (\cdot, y, z)||_{L^1} &= \int_{-\infty}^\infty \chi(x) \frac{|\exp(i\sqrt{\pm z-1}|x-y|)|}{|\sqrt{\pm z-1}|}\rho(y)~dx\\
&\leq \frac{\rho(y)}{|\sqrt{\pm z-1}|} \int_{-2L}^{2L} |\exp(i\sqrt{\pm z-1}|x-y|)~dx\\
&\leq \frac{1}{|\sqrt{\pm z-1}|} \int_{-2L}^{2L} |\exp(4Li\sqrt{\pm z-1})\\
&\lesssim \frac{L}{|\sqrt{\pm z-1}|} \exp(4 |\Im\sqrt{\pm z-1}|).
\end{align*}
The same bound is clearly valid on $||G_\pm(x, \cdot, z)||_{L^1}$ by symmetry. Taking the square root of (\ref{schur bound}),
$$||G(z)||_{L^2 \to L^2} \lesssim \frac{L}{|\sqrt{z-1}|} \exp(4 |\Im\sqrt{z-1}|)$$
follows and the proof is complete.
\end{proof}

\begin{lemma}
\label{exponential bound on free resolvent}
Let $L > 0$ and let $\rho$ be a cutoff such that $\supp \rho \subseteq (-L, L)$.
Let $\chi$ be a cutoff such that $\chi = 1$ on $\supp \rho$.
Then for every $w \in \Sigma$,
$$||\chi R_0(x, y, w) \rho||_{\mathcal H^0(\RR) \to \mathcal H^2((-L, L))} \lesssim 1 + L\langle w_1^2 - w_2^2\rangle\max\left(\frac{\exp(4 |\Im w_1|)}{|w_1|}, \frac{\exp(4 |\Im w_2|)}{|w_2|} \right).$$
In particular, $\rho R_0(x, y) \rho$ is a meromorphic family of compact operators on $\Sigma$ acting on $\mathcal H^0$, with poles exactly at $p_\pm^j$.
\end{lemma}
\begin{proof}
If $\chi$ is a cutoff, $\chi = 1$ on $\supp \rho$, $u \in \mathcal H^2$, then by elliptic regularity,
$$||\rho u||_{\mathcal H^2} \lesssim_\chi ||\chi u||_{\mathcal H^0} + ||\chi D^2 u||_{\mathcal H^0}.$$
Taking $u = R_0(w)\rho f$ we see that
$$||\rho R_0(w)\rho f||_{\mathcal H^2} \lesssim ||\chi R_0(w) \rho f||_{\mathcal H^0} + ||\chi D^2 R_0(w) \rho f||_{\mathcal H^0}.$$
First assume $w \in \mathring \Sigma$, so that if we define $z$ by (\ref{recovering the standard branch}), one has
$$R_0(w) = \diag(G_+(z), G_-(z)).$$
By Lemma \ref{bounds on partial resolvent}, one has
$$||\chi R_0(w) \rho f||_{\mathcal H^0} \lesssim L \max\left(\frac{\exp(4 |\Im w_1|)}{|w_1|}, \frac{\exp(4 |\Im w_2|)}{|w_2|} \right) ||f||_{\mathcal H^0}.$$
Treating the second term,
$$D^2R_0(w) = \diag(1, - 1) - \diag(1-z,1+z)R_0(w)$$
so
$$||\chi D^2 R_0(w) \rho f||_{\mathcal H^0} \leq ||\rho f||_{\mathcal H^0} + ||\chi R_0(w) \rho f||_{\mathcal H^0} + ||\chi z R_0(w) \rho f||_{\mathcal H^0}.$$
The second term in this sum was already bounded and the first is trivially bounded by $||f||_{\mathcal H^0}$.
The final term is
$$\lesssim |z|L \max\left(\frac{\exp(4 |\Im w_1|)}{|w_1|}, \frac{\exp(4 |\Im w_2|)}{|w_2|} \right) ||f||_{\mathcal H^0}$$
so we can bound
$$||\chi R_0(w) \rho||_{\mathcal H^0 \to \mathcal H^2} \lesssim 1 + L\langle z\rangle\max\left(\frac{\exp(4 |\Im w_1|)}{|w_1|}, \frac{\exp(4 |\Im w_2|)}{|w_2|} \right).$$
Plugging in (\ref{recovering the standard branch}) proves the first claim.
The second claim follows by the Rellich-Kondrachov theorem.
\end{proof}

Now we treat the special case of the standard branch of $R_0$; here we identify $\CC_+$ with the upper-half plane of $\CC \setminus S$.
TODO: Shift this to $\Sigma_{(+, +)}$.

\begin{proposition}
\label{sharp bound on upper half free resolvent}
For every $z \in \CC_+$ and $s \in [0, 2]$, let $2s' = s - 1$; then
$$||R_0(z)||_{\mathcal H^0 \to \mathcal H^s} \lesssim \max_{0 \leq t_1,t_2 \leq s} \frac{|1-z|^{t_1'}}{\Im \sqrt{1 - z}_+} + \frac{|1+z|^{t_2'}}{\Im \sqrt{1+ z}_+}.$$
In particular, $R_0$ is a holomorphic family of compact operators acting on $\mathcal H_0$ on $\CC_+$.
\end{proposition}
\begin{proof}
We must bound $||R_0(z)_j||_{L^2 \to H^s}$, and by interpolation it suffices to check when $s \in \NN$. In that case,
$$||R_0(z)_j||_{L^2 \to H^s} = \sum_{0 \leq t \leq s} ||D^tR_0(z)_j||_{L^2 \to L^2}.$$
Letting $K(t, z)$ denote the integral kernel of $D^tR_0(z)_j$ and $\mathcal F$ the Fourier transform,
$$K(t, x, y, z) = \int_{-\infty}^\infty \frac{\xi^t}{\pm(\xi^2 + 1)-z}e^{i\xi(x-y)}~d\xi = \mathcal F^{-1}\left(\xi \mapsto \frac{\xi^t}{\pm(\xi^2 +1)-z}\right)(x-y).$$
Here we take $+$ in the $\pm$ if $j = 1$ and $-$ otherwise. We now consider the case $j = 1$; the other case is similar. Letting $w = 1 - z$, we have
$$K(t, x, y, z) = \mathcal F^{-1}\left(\xi \mapsto \frac{\xi^t}{\xi^2 - w}\right)(x-y).$$

When $t = 0$ we just recover the usual integral kernel of the resolvent,
$$|K(0, x, y, z)| = \frac{e^{i|x-y|\sqrt w}}{\sqrt w}.$$
Moreover,
\begin{align*}
  |K(t, x, y, z)| &= |\mathcal F^{-1}(\xi \mapsto \xi^t)(x-y) * K(0, x, y, w)|.\\
  &= \left|\delta^{(t)}_0(x-y) * \frac{e^{i|x-y|\sqrt w}}{\sqrt w}\right|\\
  &= \left|\partial_{x-y}^{(t)} \frac{e^{i|x-y|\sqrt w}}{\sqrt w}\right|\\
  &= \left|\sqrt{w^{t-1}} \exp(i|x-y|\sqrt w)\right|
\end{align*}
where the square roots above satisfy $\Im \sqrt w > 0$ for any $w \notin [0, \infty)$, and in particular when $\Im z > 0$. Changing variables back,
$$|K(t, x, y, z)| = \left|\sqrt{(1-z)^{t-1}} \exp(i|x-y|\sqrt{1-z})\right|.$$

We now apply Schur's test. In fact,
$$||K(t, z)||_{L^2 \to L^2}^2 \leq \sup_{(x, y) \in \RR^2} ||K(t, x, \cdot, z)||_{L^1} \cdot ||K(t, \cdot, y, z)||_{L^1} = ||K(t, 0, \cdot, z)||_{L^1}^2$$
by symmetry. Provided that $t \leq 2$,
\begin{align*}
  ||K(t, 0, \cdot, z)||_{L^1} &\leq \sqrt{|1-z|^{t-1}} \left|\int_{-\infty}^\infty \exp(i|y|\sqrt{1-z}) ~dy\right|\\
  &\leq 2\sqrt{|1-z|^{t-1}} \int_0^\infty \exp(-y\Im\sqrt{1-z})~dy\\
  &= \frac{2\sqrt{|1-z|^{t-1}}}{\Im \sqrt{1-z}}.
\end{align*}
Letting $K'(t, z)$ denote the integral kernel of $D_jR_0(z)_2$ we find a similar estimate, and unifying them proves the first claim.
Again, the second claim follows by the Rellich-Kondrachov theorem.
\end{proof}

\section{Analytic continuation for rapidly decaying potentials}
\begin{definition}
By a \dfn{potential} we mean a matrix-valued distribution on $\RR$.
A potential $V$ is \dfn{rapidly decaying} if $\singsupp V$ is compact and there is a constant $\gamma > 0$ such that for every $x \notin \singsupp V$,
$$|V(x)| \lesssim e^{-\gamma |x|}$$
in the sense of operator norm.
\end{definition}

\begin{example}
In the application to Holmer-Zworski we have
$$V(x) = v(x)^2\begin{bmatrix}-4 & -2\\2 & 4\end{bmatrix} - 2q\delta_0 j$$
where
$$v(x) = \sech(|x| + \tanh^{-1}(q))$$
and $0 \leq |q| \ll 1$. Here $\singsupp V = \{0\}$ and $\gamma = 2$.
\end{example}

Let $R_V$ be the resolvent of $P_V = j(D^2 + 1) + V$, which by definition is
$$R_V(z) = (P_V - z)^{-1}.$$
Then one has
$$(P_V - z)R_0(z) = (P_0 - z + V)R_0(z) = 1 + VR_0(z).$$
TODO: What if $V$ has singular support?
If $z \in \Sigma_{(+,+)}$ then by Proposition \ref{sharp bound on upper half free resolvent}, (TODO: Check)
$$||VR_0(z)||_{\mathcal H^0 \to \mathcal H^0} \lesssim ||V||_{L^\infty} \left(\frac{1}{\sqrt{|1 + z|} \Im \sqrt{1 + z}_+ + \sqrt{|1 - z|} \Im \sqrt{1 - z}_+}\right) \to 0$$
as $\Im z \to \infty$.
Thus $1 + VR_0(z)$ is invertible, and
$$R_V(z) = (P_V - z)^{-1} = R_0(z)(1 + VR_0(z))^{-1}.$$
In particular (TODO: Check)
$$R_V(z) = R_0(z) - R_0(z)VR_V(z)$$
so as in Froese if $S_V(z) = \sqrt V R_0(z) \sqrt{|V|}$ (TODO: How to choose a branch of the square root on matrices??) then
$$S_V(z) = (1 + S_V(z)) \sqrt V R_V(z) \sqrt{|V|}$$
or in other words
$$R_V(z) = (\sqrt V)^{-1} (1 + S_V(z))^{-1} S_V(z) \sqrt{|V|}$$
where this equation needs to be understand as an equation of meromorphic families of Fredholm operators, with poles when $1 + S_V(z)$ is not invertible.

However, $S_V(z) =$








\printbibliography


\end{document}
