
\documentclass[12pt]{book}
\usepackage[utf8]{inputenc}
\usepackage[margin=1in]{geometry}
\usepackage{amsmath,amsthm,amssymb}
\usepackage{mathrsfs}

\usepackage{enumitem}
%\usepackage[shortlabels]{enumerate}
\usepackage{tikz-cd}
\usepackage{mathtools}
\usepackage{amsfonts}
\usepackage{amscd}
\usepackage{makeidx}
\usepackage{enumitem}
\title{PDE notes}
\author{Aidan Backus}
\date{2021}


\newcommand{\NN}{\mathbb{N}}
\newcommand{\ZZ}{\mathbb{Z}}
\newcommand{\QQ}{\mathbb{Q}}
\newcommand{\RR}{\mathbb{R}}
\newcommand{\CC}{\mathbb{C}}
\newcommand{\PP}{\mathbb{P}}
\newcommand{\DD}{\mathbb{D}}

\newcommand{\Torus}{\mathbb{T}}

\newcommand{\AAA}{\mathcal A}
\newcommand{\BB}{\mathcal B}
\newcommand{\HH}{\mathcal H}

\newcommand{\Grp}{\mathbf{Grp}}
\newcommand{\Open}{\mathbf{Open}}
\newcommand{\Vect}{\mathbf{Vect}}
\newcommand{\Set}{\mathbf{Set}}

\newcommand{\Cau}{\mathbf{Cau}}
\newcommand{\ISF}{\mathbf{ISF}}
\newcommand{\Simp}{\mathbf{Simp}}
\newcommand{\Sch}{\mathscr S}

\DeclareMathOperator{\atanh}{atanh}
\DeclareMathOperator{\sech}{sech}
\DeclareMathOperator{\sinc}{sinc}
\DeclareMathOperator{\dom}{dom}

\DeclareMathOperator{\curl}{curl}
\DeclareMathOperator{\osc}{osc}

\DeclareMathOperator{\card}{card}
\DeclareMathOperator{\coker}{coker}
\DeclareMathOperator{\diag}{diag}
\DeclareMathOperator{\diam}{diam}
\DeclareMathOperator{\Gal}{Gal}
\DeclareMathOperator{\Hom}{Hom}
\DeclareMathOperator{\id}{id}
\DeclareMathOperator{\tr}{tr}
\DeclareMathOperator{\sgn}{sgn}
\DeclareMathOperator{\supp}{supp}
\DeclareMathOperator{\rank}{rank}

\newcommand{\dbar}{\overline\partial}

\def\Xint#1{\mathchoice
{\XXint\displaystyle\textstyle{#1}}%
{\XXint\textstyle\scriptstyle{#1}}%
{\XXint\scriptstyle\scriptscriptstyle{#1}}%
{\XXint\scriptscriptstyle\scriptscriptstyle{#1}}%
\!\int}
\def\XXint#1#2#3{{\setbox0=\hbox{$#1{#2#3}{\int}$ }
\vcenter{\hbox{$#2#3$ }}\kern-.6\wd0}}
\def\ddashint{\Xint=}
\def\dashint{\Xint-}

\renewcommand{\Re}{\operatorname{Re}}
\renewcommand{\Im}{\operatorname{Im}}
\newcommand{\dfn}[1]{\emph{#1}\index{#1}}

\usepackage[backend=bibtex,style=alphabetic,maxcitenames=50,maxnames=50]{biblatex}
\renewbibmacro{in:}{}
\DeclareFieldFormat{pages}{#1}

\usepackage{color}
\usepackage{hyperref}
\hypersetup{
    colorlinks=true, % make the links colored
    linkcolor=blue, % color TOC links in blue
    urlcolor=red, % color URLs in red
    linktoc=all % 'all' will create links for everything in the TOC
    %Ning added hyperlinks to the table of contents 6/17/19
}

\theoremstyle{definition}
\newtheorem{theorem}{Theorem}[chapter]
\newtheorem{lemma}[theorem]{Lemma}
\newtheorem{sublemma}[theorem]{Sublemma}
\newtheorem{proposition}[theorem]{Proposition}
\newtheorem{corollary}[theorem]{Corollary}
\newtheorem{axiomx}[theorem]{Axiom}
\newtheorem{theoremxx}[theorem]{Theorem}
\newtheorem{conjecture}[theorem]{Conjecture}
\newtheorem{definitionx}[theorem]{Definition}
\newtheorem{remark}[theorem]{Remark}
\newtheorem{examplex}[theorem]{Example}
\newtheorem{exercisex}{Exercise}[chapter]
\newtheorem{problem}[theorem]{Problem}

\newenvironment{axiom}
  {\pushQED{\qed}\renewcommand{\qedsymbol}{$\diamondsuit$}\axiomx}
  {\popQED\endexamplex}

\newenvironment{definition}
  {\pushQED{\qed}\renewcommand{\qedsymbol}{$\diamondsuit$}\definitionx}
  {\popQED\endexamplex}

\newenvironment{example}
  {\pushQED{\qed}\renewcommand{\qedsymbol}{$\diamondsuit$}\examplex}
  {\popQED\endexamplex}

  \newenvironment{exercise}
    {\pushQED{\qed}\renewcommand{\qedsymbol}{$\diamondsuit$}\exercisex}
    {\popQED\endexamplex}

  \newenvironment{theoremx}
        {\pushQED{\qed}\renewcommand{\qedsymbol}{$\diamondsuit$}\theoremxx}
        {\popQED\endexamplex}

\makeindex

\begin{document}

\maketitle

\tableofcontents

\chapter{The Navier-Stokes equations}
This chapter is based on lectures of Huy Quang Nguyen.

\section{Derivation of the Navier-Stokes equations}
Let us study the motion of inviscid fluids at a macroscopic level. Thus we are only interested in averaged behavior of the fluid molecules, regarding fluid as a continuum, and discarding the behavior of individual molecules.
The continuum assumption implies that an infinitesimal volume element $dV$ is large enough to contain a large number of fluid molecules, and in particular is much larger than the distance between any two adjacent molecules.
On the other hand, $dV$ must be much smaller than the volume of the domain $\Omega$.
Here we assume $\Omega \subseteq \RR^d$, where $d \in \{2, 3\}$ is the spatial dimension.

The \dfn{Eulerian quantities} of the fluid are the velocity field $u: \RR^{1 + d} \to T\RR^d = \RR^d$, the pressure field $p: \RR^{1 + d} \to \RR$, and the mass density field $\rho: \RR^{1 + d} \to \RR$. Thus the Eulerian quantities are macroscopic in nature.

One also defines the microscopic \dfn{Lagrangian quantities}.
To do this, one fixes an \dfn{initial configuration} of a particle $a$, and introduce the \dfn{particle trajectory} $X(t, a)$, which is the position of $a$ at time $t$.
We typically identify $a$ with its initial position $X(0, a)$, so $X$ is a function $\RR^{1 + d} \to \RR^d$.
The \dfn{path} of $X$ is the function $X(a): \RR \to \RR^d$. The \dfn{flow map} at time $t$ is the vector field $X(t): \RR^d \to \RR^d$.
Thus under suitable assumptions, the flow maps $X$ induce a smooth action of $\RR$ on $\RR^d$ by time-translation.
The \dfn{back-to-labels map} $A$ is the inverse of the flow map, thus
$$A(t, X(t, a)) = a.$$
By construction one has
\begin{equation}
\label{Lagrangian ODE}
\frac{\partial X}{\partial t}(t, a) = u(t, X(t, a))
\end{equation}
with $X(0, a) = a$, so the Eulerian velocity field is tangent to the Lagrangian flow map.
If $u$ is continuous in time and Lipschitz in space, then applying the Picard-Linde\"of theorem, we may find a unique solution $X$ to (\ref{Lagrangian ODE}).
This is used to show that the Eulerian and Lagrangian formulations of fluid mechanics are equivalent (assuming enough regularity).

In order to study fluid mechanics, we need a notion of derivative which makes sense when taken by an observer moving according to the flow map $X$.
If $f: \RR^{1 + d} \to \RR^d$, the chain rule reads
$$\frac{\partial}{\partial t} f(X(t, a), t) = \frac{\partial f}{\partial t}(t, X(t, a)) + \sum_{j=1}^d \partial_jf(t, X(t, a)) \frac{\partial X_j}{\partial t}(t, a).$$
Plugging into (\ref{Lagrangian ODE}), we have
\begin{align*}
\frac{\partial}{\partial t} f(X(t, a), t) &= \frac{\partial f}{\partial t}(t, X(t, a)) + \sum_{j=1}^d u_j(t, X(t, a)) \partial_j f(t, X(t, a))\\
& = \frac{\partial f}{\partial t}(t, X(t, a)) + (u \cdot \nabla)f(t, X(t, a))\\
&=\left(\frac{\partial}{\partial t} + u\cdot\nabla\right)f(t, X(t, a)).
\end{align*}
This motivates the definition of the material derivative.
\begin{definition}
The \dfn{material derivative} or \dfn{convective derivative} of $f: \RR^{1 + d} \to \RR^d$ with respect to a velocity field $u: \RR^{1 + d} \to \RR^d$ is
$$D_tf(t, x) = (\partial_t + u\cdot \nabla)f(t, x).$$
The \dfn{convection term} of $D_tf$ is $(u \cdot \nabla)f$.
\end{definition}
We have proven that the derivative of $f$ along the flow map is $D_tf$.

We now impose a new assumption on our fluid, namely incompressibility.
Let $V$ be a small open set and let $V(t) = \{X(t, a): a \in V\}$ be its image under the flow map.
If our fluid is not dilating then the Lebesgue measure $|\cdot|$ satisfies $|V(t)| = |V(0)|$.

\begin{definition}
An \dfn{incompressible fluid} is a fluid whose flow map $X$ satisfies $|V(t)| = |V(0)|$ for all small open sets $V$.
\end{definition}

In what follows we will need that if $f: \RR^d \to \RR$ is measurable then the change-of-variables formula for $x = X(a, t)$ gives
$$\int_{V(t)} f(x) ~dx = \int_V f(X(t, a)) \det(\nabla X(t, a)) ~da$$
where by definition
$$(\nabla X(t, a))_{ij} = \frac{\partial X_i}{\partial a_j}(t, a).$$
Let $J = \det(\nabla X)$.
If $u \in C^1$ then
\begin{equation}
\label{Jacobian ODE}
\frac{\partial J}{\partial t}(t, a) = J(t, a) (\nabla \cdot u)(t, X(t, a)).
\end{equation}

\begin{proposition}
A fluid with velocity field $u$ is incompressible iff $\nabla \cdot u = 0$.
\end{proposition}
\begin{proof}
Plugging in $f = 1$ we get
$$|V(t)| = \int_V J(t, a) ~da$$
so $u$ is incompressible iff $J = 1$. Thus it suffices to show that $J = 1$ iff $\nabla \cdot u = 0$.
Solving (\ref{Jacobian ODE}) with the initial data $J(0, a) = 1$ we get
$$J(t, a) = \exp((\nabla \cdot u)(t, X(t, a)))$$
so $J(t, a) = 1$ iff $(\nabla \cdot u)(t, X(t, a)) = 0$.
Since $X(t)$ is a bijection, this happens iff $\nabla \cdot u = 0$.
\end{proof}

\begin{proposition}[transport formula]
Let $X$ be the flow map of a velocity field $u$.
For every function $f \in C^1(\RR^{1+d} \to \RR)$ and open set $V$,
$$\frac{\partial}{\partial t} \int_{V(t)} f(t, x) ~dx = \int_{V(t)} \frac{\partial f}{\partial t}(t, x) + \nabla \cdot(uf)(t, x) ~dx.$$
\end{proposition}
\begin{proof}
Using (\ref{Jacobian ODE}),
\begin{align*}
\frac{\partial}{\partial t} \int_{V(t)} f(t, x) ~dx &= \frac{\partial}{\partial t} \int_V f(t, X(t, a)) J(t, a) ~da\\
&= \int_V J(t, a) \frac{\partial}{\partial t}(f(t, X(t, a))) + f(t, X(t, a)) \frac{\partial J}{\partial t}(t, a) ~da\\
&= \int_V J(t, a) \left(\frac{\partial f}{\partial t} + u \cdot \nabla f + f \nabla \cdot u)\right)(t, X(t, a)) ~da\\
&= \int_V J(t, a) \left(\frac{\partial f}{\partial t} + \nabla \cdot(uf)\right)(t, X(t, a)) ~da\\
&= \int_{V(t)} \frac{\partial f}{\partial t}(t, x) + \nabla \cdot(uf)(t, x) ~dx
\end{align*}
which was desired.
\end{proof}

\begin{corollary}
If $u$ is an incompressible velocity field then
$$\frac{\partial}{\partial t} \int_{V(t)} f(x) ~dx = \int_{V(t)} D_tf(t, x) ~dx.$$
\end{corollary}
\begin{proof}
Plug $\nabla \cdot u = 0$ into the transport formula.
\end{proof}

We now impose conservation of mass.
The mass of $V$ is
$$m(t, V) = \int_V \rho(t, x) ~dx.$$
Conservation of mass says that mass is neither created nor destroyed.
Since $V(t)$ moves with the flow map, conservation of mass can be expressed as
$$\frac{\partial}{\partial t} m(t, V(t)) = 0.$$
In other words,
$$\frac{\partial}{\partial t} \int_{V(t)} \rho(t, x) ~dx = 0.$$
By the transport formula, if $u$ is incompressible, then
$$0 = \frac{\partial}{\partial t} \int_{V(t)} \rho(t, x) ~dx = \int_{V(t)} \left(\frac{\partial \rho}{\partial t} + \nabla \cdot (u\rho)\right)(t, x) ~dx.$$
Since $V$ is an arbitrarily small open set, we can take $V \to \{x\}$ and conclude
$$\frac{\partial \rho}{\partial t} + \nabla \cdot (u\rho) = 0.$$
By incompressibility of $u$, this equation to simplifies to
\begin{equation}
\label{conservation of mass}
\frac{\partial \rho}{\partial t} + u \cdot \nabla \rho = 0.
\end{equation}
Therefore if the initial data of $\rho$ is constant in space, then $\rho$ is constant in spacetime.

Finally we impose conservation of momentum.
Let $S$ be a closed hypersurface in $\Omega$ and let $\vec n$ be the outer unit normal field to $S$.
The force of stress exerted on $S$ per unit area at the point $x \in S$ is given by the pressure density $p(t, x) \vec n(x)$.
So if $S = \partial V$, then the force of stress acting on $V$ is
$$-\int_{\partial V} p(t, x) \vec n(x) ~dx = -\int_V \nabla p(t, x) ~dx$$
by the divergence theorem. By Newton's second law, the integral of acceleration is given by
$$\frac{\partial}{\partial t} \int_{V(t)} \rho(t, x) u(t, x) ~dx = -\int_{V(t)} \nabla p(t, x) ~dx.$$
Let $f$ be a component of $u$. By the transport formula,
\begin{align*}\frac{\partial}{\partial t} \int_{V(t)} \rho(t, x) f(t, x) ~dx &= \int_{V(t)} \partial_t(\rho f)(t, x) + \nabla \cdot(u\rho f)(t, x) ~dx\\
&= \int_{V(t)} f\frac{\partial \rho}{\partial t} + f \nabla \cdot(u\rho) + \rho \frac{\partial f}{\partial t} + u\rho \cdot \nabla f ~dx\\
&= \int_{V(t)} f\left(\frac{\partial \rho}{\partial t} + \nabla \cdot(u\rho)\right) + \rho D_t f ~dx\\
&= \int_{V(t)} \rho(t, x) D_tf(t, x)~dx
\end{align*}
where we used mass conservation (\ref{conservation of mass}). Plugging into Newton's second law,
$$\int_{V(t)} \rho(t, x) D_tu(t, x) ~dx = -\int_{V(t)} \nabla p(t, x) ~dx.$$
Shrinking $V$ to a point, we get
\begin{equation}
\label{conservation of momentum}
\rho D_tu = -\nabla p.
\end{equation}
To encorporate a force field $F$ (say, $F$ is the gravitational field), we instead have
$$\frac{\partial}{\partial t} \int_{V(t)} \rho(t, x) u(t, x) ~dx = \int_{V(t)} \rho(t, x)F(t, x) - \nabla p(t, x) ~dx$$
and so (\ref{conservation of momentum}) becomes instead
\begin{equation}
\label{conservation of momentum with forcing}
\rho D_tu = \rho F - \nabla p.
\end{equation}

Summarizing what we have derived so far: if $u$ is the velocity field of an ideal incompressible fluid not in an external force field, $p$ denotes the pressure, and $\rho$ denotes the mass density, then we have
\begin{align*}
\rho D_t u &= -\nabla p\\
\nabla \cdot u &= 0\\
\partial_t \rho + \nabla \cdot (u\rho) &= 0.
\end{align*}
This is a system of $2 + d$ equations for $2 + d$ unknowns (since $u$ is a vector field).
If $\rho$ is constant, say $\rho = 1$, then this system simplifies to
\begin{align*}
D_t u &= -\nabla p\\
\nabla \cdot u &= 0.
\end{align*}
This system is known as the \dfn{Euler equations}.

Now we modify the Euler equations to account for friction, which occurs when two fluid layers slide over each other.
This phenomenon is known as \dfn{viscosity}.
In this situation, we have the momentum balance
$$\frac{\partial}{\partial t} \int_{V(t)} \rho(t, x) u(t, x) ~dx = -\int_{\partial V(t)} \sigma(t, x) n(x) ~dS(x)$$
where $n$ is the outward unit normal field of $\partial V(t)$ and $\sigma$ is a tensor field known as the \dfn{Cauchy stress tensor}, given by
$$\sigma(t, x) = -p(t, x) + \sigma_{visc}(t, x)$$
where $\sigma_{visc}$ is a quantity known as the \dfn{viscous stress tensor}.
In a Newtonian fluid, we assume that $\sigma_{visc}$ is linear in $\nabla u$.
If the fluid is also isotropic, thus the fluid properties are invariant under rotation, then there are constants $\lambda, \nu > 0$ such that
$$\sigma_{visc} = \lambda \nabla \cdot u + \nu(\nabla u + \nabla u^t)$$
but we know that $\nabla \cdot u = 0$ so $\lambda$ is irrelevant. Here $\nabla u^t$ makes sense because $\nabla u$ is a matrix field (since $u$ is a vector field).
It follows that
$$\rho D_t u = -\nabla p + \nu \nabla \cdot(\nabla u + \nabla u^t) = -\nabla p + \nu\Delta u$$
since $\nabla \cdot u = 0$. Here $\nu$ is known as the \dfn{dynamic viscosity}.
Summarizing, we arrive at the system
\begin{align*}
\rho D_t u &= -\nabla p + \nu \Delta u\\
\nabla \cdot u &= 0\\
\partial_t \rho + \nabla \cdot (u\rho) &= 0.
\end{align*}
Since we are mainly interested in the case $\rho = \rho_0$, where $\rho_0 > 0$ is a constant, we arrive at the system
\begin{align*}
D_t u &= -\frac{1}{\rho_0}\nabla p + \frac{\nu}{\rho_0}\Delta u\\
\nabla \cdot u &= 0.
\end{align*}
This system is known as the \dfn{Navier-Stokes equations}, and $\nu/\rho_0$ is known as the \dfn{kinematic viscosity}.
We will mainly normalize $\rho_0 = 1$, thus
\begin{align*}
D_t u &= -\nabla p + \nu\Delta u\\
\nabla \cdot u &= 0.
\end{align*}

\section{Vorticity}
If $u$ is a vector field on $\RR^2$, we will regard $u$ as a map $\RR^2 \to \RR^3$ by $u(x_1, x_2) = (u_1(x_1, x_2), u_2(x_1, x_2), 0)$.
Therefore the curl of a vector field on $\RR^2$ makes sense, and is a scalar field.
In this case, if
$$D_tu + \nabla p = \nu \Delta u,$$
then $\partial_3p = 0$, and $p$ can be regarded as a map $\RR^2 \to \RR^3$ or $\RR^2 \to \RR^2$.

\begin{definition}
Let $u$ be a velocity field. The \dfn{vorticity} of $u$ is
$$\omega = \nabla \times u = \curl u$$
where $\times$ denotes cross product.
\end{definition}

In two dimensions, we regard $\omega$ as a scalar field and have
$$\omega = \partial_1 u_2 - \partial_2 u_1.$$
Thus we write
$$\omega = \nabla^\perp \cdot u$$
where, since $\nabla = (\partial_1, \partial_2)$, $\nabla^\perp = (-\partial_2, \partial_1)$.
The fact that the vorticity is a scalar field on $\RR^2$ justifies why there are so many differences between the behavior of the Navier-Stokes equations in two and three dimensions.

In full generality, let $(t_0, x_0) \in \RR^{1 + 3}$, and suppose that $u \in C^1$. Then
$$u(t_0, x) = u(t_0, x_0) + \nabla u(t_0, x_0) (x - x_0) + o(|x - x_0|).$$
Let us write
$$\nabla u(t_0, x_0) = \mathscr D_u + \Omega_u$$
where $2\mathscr D_u = \nabla u + \nabla u^t$ is the symmetric part of the matrix $\nabla u$ and $\Omega_u$ is the antisymmetric part, thus
$$u(t_0, x) = u(t_0, x_0) + \Omega_u(x - x_0) + \mathscr D_u(x - x_0) + o(|x - x_0|).$$
In the degenerate case $u(t_0, x) = u(t_0, x_0)$ the Lagrangian equations
\begin{align*}
\partial_t X(t, a) &= u(t, X(t, a))\\
X(0, a) &= a
\end{align*}
become
\begin{align*}
\partial_t X(t, a) &= u(t_0, x_0)\\
X(0, a) &= a
\end{align*}
thus
$$X(t, a) = a + u(x_0, t_0)(t - t_0).$$
Thus $u(t_0, x_0)$ induces translation in $X$.

Meanwhile, the antisymmetric part satisfies
$$2\Omega_u h = \omega \times h$$
for any vector field $h$. In particular in the degenerate case $u = \Omega_u$,
$$2\partial_t X(t, a) = \omega(t_0, x_0) \times X(t, a).$$
After a change of coordinates we may assume $\omega(t_0, x_0) = (0, 0, \overline \omega)$ for some $\overline \omega \in \RR$.
Then
$$\frac{1}{2}\omega \times X = (-\overline \omega X_2, \overline \omega X_1, 0)$$
and thus
\begin{align*}
2\partial_t X_1(t, a) &= -\overline \omega X_2(t, a)\\
2\partial_t X_2(t, a) &= \overline \omega X_1(t, a)\\
\partial_t X_3(t, a) &= 0.
\end{align*}
Solving for $X$, we get $X_3(t, a) = a_3$, and with $X' = (X_1, X_2)$,
$$2X'(t) = Q(\overline \omega t)a'$$
where $a' = (a_1, a_2)$ and $Q(\theta)$ is the rotation operator by $\theta$.
Thus $\Omega_u$ induces rotation in some plane, here the plane $x_3 = 0$ since we did a change of coordinates.

As for the symmetric part, we can find an orthogonal matrix $Q$ with
$$Q^t \mathscr D_u Q = \diag(\gamma)$$
where $\gamma \in \RR^3$. Moreover the trace of $\mathscr D_u$ is $\nabla \cdot u = 0$, so $\gamma_1 + \gamma_2 + \gamma_3 = 0$.
After a change of coordinates, then, we may assume
$$\mathscr D_u = \diag(\gamma_1, \gamma_2, -\gamma_1 - \gamma_2).$$
Therefore if $u = \mathscr D_u$,
\begin{align*}
X_1(t, a) &= ae^{\gamma_1t}\\
X_2(t, a) &= ae^{\gamma_2t}\\
X_3(t, a) &= ae^{-(\gamma_1 + \gamma_2)t}
\end{align*}
(assuming $x_0 = t_0 = 0$).
After a change of coordinates again, we can assume $\gamma_1, \gamma_2 < 0$, then the fluid converges onto the axis $x_3 > 0$, $x_1 = x_2 = 0$.

The key term is the antisymmetric part $\Omega_u$, since it induces rotation.
Taking the curl of the Navier-Stokes equation
$$\partial_t u + (u \cdot \nabla) u + \nabla p = \nu \Delta u$$
and using $\nabla \times \nabla = 0$ and
\begin{align*}
(u \cdot \nabla) u &= \omega \times u + \frac{1}{2} \nabla(|u|^2)\\
\curl(\omega \times u) &= \omega(\nabla \cdot u) - u(\nabla \cdot \omega) + (u \cdot \nabla)\omega - (\omega \cdot \nabla)u\\
\nabla \cdot u &= 0\\
\nabla \cdot \omega &= 0
\end{align*}
we get
\begin{equation}
\label{vorticity equation}
\partial_t \omega + (u \cdot \nabla)\omega = (\omega \cdot \nabla)u + \nu \Delta \omega,
\end{equation}
the \dfn{vorticity equation}. In two dimensions, $(\omega \cdot \nabla)u = 0$ since $\omega = \nabla^\perp u$. Thus we get the simplified equation
\begin{equation}
\label{2D vorticity equation}
\partial_t \omega + (u \cdot \nabla)\omega = \nu\Delta \omega.
\end{equation}
Here (\ref{vorticity equation}) is a system of $3$ equations and $6$ unknowns.
We need to be able to recover $u$ from $\omega$ to fix this problem, which depends on the domain $\Omega$.

So let us solve the \dfn{div-curl system}
\begin{align*}
\nabla \cdot u &= 0\\
\curl u &= \omega.
\end{align*}
Taking the curl of both sides,
$$\curl \omega = \curl \curl u = -\Delta u + \nabla(\nabla \cdot u) = -\Delta u.$$
Thus $u$ solves the Laplace equation with forcing term $\curl \omega$.
If $\Omega = \RR^3$ then $\Delta^{-1}$ exists and commutes with $\curl$, thus
$$u = -\curl \Delta^{-1}\omega.$$
\begin{definition}
The \dfn{stream function} $\psi$ of a vorticity $\omega$ is $\psi = \Delta^{-1}\omega$.
\end{definition}
In that case, $u = -\curl \psi$.
If $N_d$ is the Newtonian kernel of $\RR^d$,
$$N_d(x) = \begin{cases}
\frac{\log|x|}{2\pi}, &d=2,\\
\frac{1}{4\pi|x|}, &d=3,
\end{cases}$$
then
$$\psi(x) = \int_{\RR^3} N_3(x - y) \omega(y) ~dy$$
since $\psi = \Delta^{-1}\omega$, thus
$$u(x) = -\frac{1}{4\pi} \int_{\RR^3} \frac{1}{|x - y|} \curl \omega(y) ~dy.$$
Integrating by parts and using the Levi-Civita symbol (thus $\epsilon_{ijk} = 1$ if $(i, j, k)$ is an even permutation of $3$, $-1$ if $(i, j, k)$ is an odd permutation of $3$, and $0$ if an index is repeated),
\begin{align*}
u(x) &= -\frac{1}{4\pi} \int_{\RR^3} \frac{1}{|x - y|} \epsilon_{ijk} \partial_i \omega_j(y) e_k ~dy\\
&= \frac{1}{4\pi} \int_{\RR^3} \partial_{y_i}\left(\frac{1}{|x - y|}\right) \varepsilon_{ijk} \epsilon_{ijk} \omega_j(y) e_k ~dy\\
&= \int_{\RR^3} K_3(x - y) \omega(y) ~dy
\end{align*}
where
$$K_3(x)h = \frac{1}{4\pi} \frac{x \times h}{|x|^3}.$$
This formula is known as the \dfn{Biot-Savart law}.

Now consider the case $\Omega = \RR^2$. In this case, $\omega$ is a scalar field and the div-curl system reads
\begin{align*}
\nabla \cdot u &= 0,\\
\nabla^\perp u &= \omega.
\end{align*}
Again we introduce the stream function $\psi$, which is again a scalar field, and satisfies $u = \nabla^\perp \psi$.
We can then again derive a Biot-Savart law
$$u(x) = \int_{\RR^2} K_2(x - y) \omega(y) ~dy$$
where
$$K_2(x) = \frac{1}{2\pi} \frac{x^\perp}{|x|^2}.$$

In the above derivations we have assumed that $u$ decays at infinity.
This was necessary to impose that $\Delta^{-1}$ exists, and since we will eventually want to require that $u$ lies in a suitable Sobolev space, this is not a big loss.

Now let us derive a Biot-Savart law with $\Omega = \Torus^d$.
Let us take the fundamental domain $\Torus^d = [-\pi, \pi]^d$ and impose
$$\int_{\Torus^d} u(t, x)~dx = \int_{\Torus^d} p(t, x) ~dx = 0.$$
If we integrate
$$\partial_t u + u \cdot \nabla u + \nabla p = \nu \Delta u$$
we get
$$\partial_t \int_{\Torus^d} u(t, x) ~dx + \int_{\Torus^d} u \cdot \nabla u + \nabla p = \nu \int_{\Torus^d} \Delta u$$
and it follows that
$$\int_{\Torus^d} u(t, x) ~dx = \int_{\Torus^d} u(0, x) ~dx$$
so once we assume that $u(0, x)$ has mean zero this remains true over all time.

Finally let us consider the case when $\Omega$ is an open, simply connected subset of $\RR^d$ and $\partial \Omega$ is a smooth closed manifold.
For ideal flows we will assume that no fluid leaves $\Omega$, the \dfn{nonpenetrating boundary condition}
$$u(t, x) \cdot n(x) = 0$$
where $n$ is the unit normal field of $\partial \Omega$, for the Euler equations.
For the Navier-Stokes equations we instead assume the \dfn{no-slip boundary condition}, thus $u = 0$ on $\partial \Omega$.
To derive a Biot-Savart law, let us solve the div-curl problem
\begin{align*}
\nabla \cdot u &= 0 \\
\curl u &= \omega.
\end{align*}
Suppose $d = 2$. We define the stream function to solve the Laplace equation
\begin{align*}
\Delta \psi &= \omega\\
\psi|\partial \Omega &= 0.
\end{align*}
We then set $u = \nabla^\perp \psi$. We claim $u$ is nonpenetrating.
Since $\psi|\partial \Omega = 0$, $\partial \Omega$ is a level curve of $\psi$.
Therefore $\nabla^\perp \psi \perp n$, as desired.

Now suppose $d = 3$. Again we define the stream function to solve the Laplace equation, and let $u = -\curl \psi$.
It is easy to check that $u$ is no-slip.

We note that the pressure $p$ does not appear in the vorticity equation.
However we are given
$$\partial_t u + u\cdot u + \nabla p = \nu \Delta u;$$
taking the divergence and using $\nabla \cdot u = 0$, $\nabla \cdot \nabla = \Delta$, we get
$$\Delta p = -\nabla \cdot(u \cdot \nabla u).$$
After choosing a gauge for $p$ (which is quite a tricky issue if $\partial \Omega$ is a closed manifold) this uniquely defines $p$ on $\partial \Omega$.

\section{Vorticity in ideal fluids}
Assume that $\nu = 0$, thus $u$ defines an ideal fluid. Then the vorticity equation says
$$\partial_t \omega + u \cdot \nabla \omega = (\omega \cdot \nabla)u,$$
but
$$((\omega \cdot \nabla)u)_i = \omega_j \partial_j u_i = \partial_j u_i \omega_j.$$
Thus we will write
$$(\omega \cdot \nabla)u = \nabla u \omega$$
where we view $\nabla u$ as a matrix acting on the vector $\omega$. In $\RR^2$, of course, we instead have
$$\partial_t \omega + u \cdot \nabla \omega = 0.$$

In the below formula, we think of $\partial_a X$ as a matrix field (since $a$ is a vector and $X$ is a vector field).
\begin{proposition}[vorticity transport formula]
Let $u$ be an ideal fluid. Then
$$\omega(t, X(t, a)) = \partial_a X(t, a) \omega(0, a).$$
\end{proposition}
\begin{proof}
We claim that both sides solve the same differential equation. Indeed,
$$\omega(0, X(0, a)) = \omega(0, a) = \partial_a X(0, a) \omega(0, a)$$
since $\partial_a X(0, a) = \partial_a a = 1$.
Also
$$\frac{\partial}{\partial t} \omega(t, X(t, a)) = \nabla u(t, X(t, a))\omega(t, X(t, a)) \omega(t, X(t, a))$$
while
$$\frac{\partial}{\partial t}(\partial_a X(t, a) \omega(0, a)) = \frac{\partial}{\partial t} \partial_a X(t, a) \omega(0, a)$$
but the derivatives commute so
$$\frac{\partial}{\partial t}(\partial_a X(t, a) \omega(0, a)) = \partial_au(t, X(t, a)) \omega(0, a) = \nabla u(t, X(t, a)) \omega(t, X(t, a)).$$
So by uniqueness of solutions for ODE the claim holds.
\end{proof}

In two dimensions, then, vorticity is constant along stream lines.

\begin{corollary}
Let $u$ be a two-dimensional ideal fluid. Then
$$\omega(t, X(t, a)) = \omega(0, a).$$
\end{corollary}
\begin{proof}
From the vorticity transport formula,
$$\omega(t, X(t, a)) = \partial_a X(t, a) \omega(0, a)$$
but $\omega = (0, 0, \overline \omega)$ for the scalar field $\overline \omega = \nabla^\perp u$.
Similarly
$$\partial_a X = \begin{bmatrix}\partial_1 X_1 & \partial_2 X_1 & 0\\\partial_1 X_2 & \partial_2 X_2 & 0 \\ 0 & 0 & 1\end{bmatrix}$$
so $\partial_a X(t, a) \omega(0, a) = \overline \omega(0, a)$.
\end{proof}

\begin{definition}
Let $C$ be a closed curve in $\RR^d$, parametrized by a function $y$.
We call $C$ a \dfn{vortex line} at time $t$ if $C$ is tangent to $\omega(t)$, thus
$$\dot y(s) = \lambda(s) \omega(t, y(s))$$
for all $y \in [0, 1]$ and some $\lambda: [0, 1] \to \RR$.
\end{definition}

Vortex lines move with ideal fluids.

\begin{proposition}
Let $C$ be a vortex line at time $0$. Let
$$C(t) = X(t, C).$$
Then $C(t)$ is a vortex line at time $C$.
\end{proposition}
\begin{proof}
One has
$$\frac{\partial}{\partial s} X(t, y(s)) = \partial_a X(t, y(s)) \cdot y(s) = \partial_a X(t, y(s)) \lambda(s) \omega(0, y(s))$$
but by the vorticity transport formula,
$$\partial_a X(t, y(s)) \omega(0, y(s)) = \omega(t, X(t, y(s)))$$
so
$$\frac{\partial}{\partial s} X(t, y(s)) =  \lambda(s)\omega(t, X(t, y(s)))$$
gives a vortex line at time $t$.
\end{proof}

\begin{definition}
The \dfn{velocity circulation} along a vortex line $C$ is the line integral
$$\Gamma_C = \int_C u(t, x) ~dx.$$
\end{definition}

\begin{proposition}
Let $u$ be a smooth velocity field (possibly not incompressible),
$$\frac{\partial}{\partial t} \Gamma_{C(t)} = \int_{C(t)} D_tu(t, x) ~dx.$$
\end{proposition}
\begin{proof}
Since $C$ is a closed curve, we can write $C = a(\Torus)$ where $a: \Torus \to \Omega$.
Then $C(t) = X(t, a(\Torus))$ and
$$\Gamma_{C(t)} = \int_{C(t)} u(t, x) ~dx = \int_\Torus u(t, X(t, a(s))) \partial_s X(t, a(s)) ~ds.$$
Differentiating,
$$\frac{\partial}{\partial t} \Gamma_{C(t)} = \int_\Torus D_t u(t, X(t, a(s))) \partial_s X(t, a(s)) + u(t, X(t, a(s)))\partial_s u(t, X(t, a(s)))~ds$$
but
$$2u(t, X(t, a(s)))\partial_s u(t, X(t, a(s))) = \partial_s |u(t, X(t, a(s)))|^2$$
so we are integrating a derivative over the torus, so by the fundamental theorem it has average zero.
Thus
$$\frac{\partial}{\partial t} \Gamma_{C(t)} = \int_\Torus D_t u(t, X(t, a(s))) \partial_s X(t, a(s))~ds = \int_{C(t)} D_tu(t, x)~dx .$$
This was desired.
\end{proof}

\begin{corollary}[Kelvin's theorem: conservation of circulation]
Let $u$ be a smooth solution of the Euler equations and let $C$ be a vortex line. Then
$$\Gamma_{C(t)} = \Gamma_C.$$
\end{corollary}
\begin{proof}
One has
$$\Gamma_{C(t)} = \frac{\partial}{\partial t}\Gamma_{C(t)} = \int_{C(t)} D_tu(t, x) ~dx = -\int_{C(t)} \nabla p = 0$$
since we are integrating a gradient along the closed curve $C(t)$.
\end{proof}

Let $S$ be a compact surface and let $\partial S = C$, where $C$ is a vortex line.
Let $S(t) = X(t, S)$.
By the Stokes curl formula, the vorticity flux along $S$ is
$$\int_S \omega \cdot n ~dS = \int_C u~ds$$
so by conservation of circulation, we also have conservation of vorticity flux:
\begin{corollary}[Helmholtz' theorem: conservation of vorticity flux]
One has
$$\int_{S(t)} \omega(t) \cdot n(t) ~dS = \int_S \omega(0) \cdot n ~dS.$$
\end{corollary}

\section{Conservation laws}
Hereinafter ideal fluids are those that obey the Euler equations; otherwise we call them viscous.
\begin{proposition}[ideal conservation laws]
Let $u$ be a Schwartz solution of the Euler equations, $d = 3$.
Then the following quantities are constant in time:
\begin{enumerate}
\item The \dfn{kinetic energy}
$$\frac{1}{2} ||u(t)||_{L^2}^2.$$
\item The \dfn{total velocity flux}
$$\int_{\RR^3} u(t,x )~dx.$$
\item The \dfn{total vorticity flux}
$$\int_{\RR^3} \omega(t, x) ~dx.$$
\item The \dfn{helicity}
$$\int_{\RR^3} u(t, x) \cdot \omega(t, x) ~dx.$$
\item The \dfn{fluid impulse}
$$\int_{\RR^3} x \times u(t, x) ~dx.$$
\item The \dfn{moment of fluid impulse}
$$\int_{\RR^3} x \times (x \times u(t, x)) ~dx.$$
\end{enumerate}
\end{proposition}
\begin{proof}
Let us prove this for kinetic energy; the other proofs are similar. One has
$$\frac{\partial}{\partial t} \frac{1}{2} ||u(t)||_{L^2}^2 = \int_{\RR^3} \partial_t u \cdot u.$$
However, by the Euler equation, this is
$$\frac{\partial}{\partial t} \frac{1}{2} ||u(t)||_{L^2}^2 = -\int_{\RR^3} ((u \cdot \nabla)u) \cdot u - \int_{\RR^3} \nabla p \cdot u.$$
By Green's theorem and incompressibility,
$$\int_{\RR^3} \nabla p \cdot u = \langle p, \nabla \cdot u\rangle = 0.$$
Thus
$$\frac{\partial}{\partial t} \frac{1}{2} ||u(t)||_{L^2}^2 = -\int_{\RR^3} ((u \cdot \nabla)u) \cdot u = \int_{\RR^3} \partial_j u_j |u_i|^2 = \langle \nabla \cdot u , |u_i|^2\rangle = 0.$$
\end{proof}

The \dfn{Onsager conjecture} says that we don't need $u$ to be Schwartz, but only that $u$ has the H\"older regularity $C^{1/3}$, at least to show that the kinetic energy is conserved.

\begin{proposition}
Let $u$ be a Schwartz solution of the Euler equations and $d = 2$. Then the following quantities are conserved:
\begin{enumerate}
\item The \dfn{total velocity flux} $\int_{\RR^2} u$.
\item The \dfn{total vorticity flux} $\int_{\RR^2} \omega$
\item The \dfn{kinetic energy} $||u||_{L^2}^2/2$.
\item $||\omega(t)||_{L^p}$.
\end{enumerate}
\end{proposition}
\begin{proof}
Let us prove that the $L^p$ norm of $\omega(t)$ is conserved.
We may assume $p \in (1, \infty)$ and otherwise use a limiting argument.

One has
$$\frac{\partial}{\partial t} ||\omega(t)||_{L^p}^p = p \int_{\RR^2} |\omega(t, x)|^{p-2} \omega(t,x ) \partial_t \omega(t, x)~dx.$$
Using $d = 2$ to compute $\partial_t \omega$ we get
$$\frac{\partial}{\partial t} ||\omega(t)||_{L^p}^p = -p \int_{\RR^2} |\omega(t, x)|^{p-2} \omega(t,x) u \cdot \nabla \omega(t,x )~dx = -\int_{\RR^2} u \cdot \nabla|\omega(t,x)|^p ~dx$$
but integrating by parts again we get
$$\frac{\partial}{\partial t} ||\omega(t)||_{L^p}^p = \int_{\RR^2} \nabla \cdot u |\omega|^p = 0$$
which is what we wanted.
\end{proof}

\section{Explicit solutions in two dimensions}
Let us first consider the case $d = 2$, and look for explicit steady solutions to the Euler equation
$$\partial_t \omega + u \cdot \nabla \omega = 0$$
written in vorticity form.
If $(u, \omega)$ is a time-independent solution then
$$u \cdot \omega = 0.$$
Let us restrict to a region $\Omega$ and consider the stream function
$$\begin{cases}
\Delta \psi &= \omega\\
\psi|\partial \Omega &= 0.
\end{cases}$$
Then $u = (-\partial_2 \psi, \partial_1\psi)$, so
$$u \cdot \nabla \omega = \begin{pmatrix}\partial_1 \psi & \partial_2 \psi \\
\partial_1 \Delta \psi & \partial_2 \Delta \psi
\end{pmatrix}$$
so we write $u \cdot \nabla \omega = J(\psi, \Delta \psi)$.
Since $u \cdot \nabla \omega = 0$, the gradients $\nabla \psi$ and $\nabla(\Delta \psi)$ are linearly dependent.
In particular, $\nabla^\perp \psi$ and $\nabla^\perp(\Delta \psi)$ are linearly dependent, so $\psi$ and $\Delta \psi$ have the same level sets.
Therefore there is some diffeomorphism $F: \RR \to \RR$ such that
\begin{equation}
\label{functional equation}
\Delta \psi = F \circ \psi.
\end{equation}
Conversely, if (\ref{functional equation}) holds, even if $F$ is not necessarily a diffeomorphism, then $J(\psi, \Delta \psi) = 0$, inducing a steady solution of the Euler equations.

Thus we have reduced the problem of finding steady solutions to the Euler equations to:
\begin{proposition}
Let $d = 2$.
A stream function $\psi: \Omega \to \RR$ defines a steady solution of the Euler equations if and only if there is a smooth function $F: \RR \to \RR$ such that the functional equation (\ref{functional equation}) holds.
\end{proposition}

For steady flows, the particle trajectory $X$ satisfies
$$\partial_t X = \nabla^\perp \psi \circ X$$
which is a Hamiltonian system with $\psi$ as the Hamiltonian; in particular $\psi$ is constant along the trajectories of $X$.

\begin{definition}
A \dfn{stagnation point} in a steady solution $u$ of the Euler equations in velocity form is a point $a$ at which $u(a) = 0$.
\end{definition}

Equivalently, $a$ is stagnant iff $\nabla \psi(a) = 0$, and that happens iff the trajectory of $X$ at $a$ is $X(t, a) = a$.0
If $a$ is a local extremum of $\psi$ then close to $a$, the trajectories of $X$ form ellipses around $a$.
If instead $a$ is a saddle point of $\psi$ then $a$ is a hyperbolic fixed point of $X$; that is, there are coordinates in which $a$ is a sink of one axis, a source of another axis, and elsewhere the trajectories form hyperbolas.

\begin{example}
Let us consider the \dfn{radial eddy solution}.
Suppose that $\omega_0$ is a smooth radial vorticity and $d = 2$.
Since $\Delta$ is rotationally symmetric, we will look for radial stream functions $\psi_0$.
In polar coordinates, we get the ODE
$$\psi_0''(r) + \frac{\psi_0'(r)}{r} = \omega_0(r)$$
which can be rewritten as
$$r \psi_0'(r) = \int_0^r s\omega_0(s) ~ds + C$$
or equivalently,
\begin{equation}
\label{radial eddy stream ODE}
\psi_0'(r) = \frac{1}{r} \int_0^r s\omega_0(s) ~ds + \frac{C}{r}.
\end{equation}
Since $\psi_0, \Delta \psi_0 = \omega_0$ are both radial, they have parallel gradients, so $J(\psi_0, \Delta \psi_0) = 0$.
In particular, any radial vorticity defines a steady solution to the Euler equations with stream function (\ref{radial eddy stream ODE}), $C = 0$.
(Here we take $C = 0$ so avoid a singularity at $r = 0$.)
In that case,
$$u(x) = \frac{x^\perp}{|x|^2} \int_0^{|x|} s\omega_0(s) ~ds$$
which defines a rotation.
If $\omega_0 \geq 0$ then $u$ rotates counterclockwise.
\end{example}

\begin{example}
Now let us look for radial eddy solutions of the Navier-Stokes equations in vorticity form, thus
$$\partial_t \omega^\nu + u \cdot \nabla \omega^\nu = \nu \Delta \omega^\nu$$
with $\nu > 0$ and $d = 2$.
We cannot get a steady solution due to the viscosity, but we will let $\omega_0$ be a radial vorticity and assume the initial condition $\omega^\nu(0) = \omega_0$.
Since $\Delta$ is rotationally invariant, we look for radially symmetric solutions $\omega^\nu$.
Then
$$u^\nu \cdot \nabla \omega^\nu = J(\psi^\nu, \Delta \psi^\nu) = 0.$$
Therefore the Navier-Stokes equations reduce to the heat equation
$$\partial_t \omega^\nu = \nu \partial_r^2 \omega^\nu$$
with $d = 1$, so that the heat equation has the explicit solution
$$\omega^\nu(t, r) = \int_{-\infty}^\infty H(\nu t, r - s) \omega_0(s) ~ds$$
where the heat kernel satisfies the normalization
$$H(t, x) = \frac{1}{\sqrt{4\pi t}} \exp\left(-\frac{|x|^2}{4t}\right).$$
Then $\omega^\nu$ is a solution.
\end{example}

Let us now consider the \dfn{inviscid limit problem}: if $\omega^\nu$ is a solution to the Navier-Stokes equations and $\omega$ is the solution to the Euler equations then does $\omega^\nu \to \omega$ as $\nu \to 0$?
In general this is quite hard.

Let us solve the inviscid limit problem for radial eddies. The question is, does $\omega^\nu$ converge to its initial data $\omega_0$ as $\nu \to 0$?
This is easy to solve since this is the special case of the heat equation.

\begin{lemma}
Let $\omega_0$ be a radial eddy solution to the Euler equations and $\omega^\nu$ the induced solution to the Navier-Stokes equations.
Then
$$||\omega^\nu(t) - \omega_0||_{L^\infty} \lesssim ||\omega_0||_{W^{1,\infty}} \sqrt{\nu t}$$
and
$$||\omega^\nu(t) - \omega_0||_{L^2} \lesssim (||\omega_0||_{L^2} + ||\omega_0''||_{L^2}) \nu t$$
(so in particular $||\omega^\nu(t) - \omega||_{L^2} \lesssim ||\omega_0||_{W^{2,\infty}} \nu t$).
\end{lemma}
\begin{proof}
The analogous result is true for solutions of the heat equation.
\end{proof}

Let us now consider steady solutions on the torus $\Torus^2$. We normalize the Fourier transform of $f$ as
$$\hat f(k) = \int_{\Torus^2} f(x) e^{-ik\cdot x} ~dx.$$
We look for stream functions $\psi: \Torus^2 \to \RR$ with $\Delta \psi = F \circ \psi$.
In particular this holds for $F(y) = \lambda y$ if $\psi$ is a $\lambda$-eigenfunction of $\Delta$.
So eigenfunctions of $\Delta$ give rise to steady solutions of the Euler equations.
The real eigenfunctions of $\Delta$ have basis.
$$\psi(x) = a_k \cos kx + b_k \sin kx$$
with $a_k, b_k \in \RR$, and $k$ ranges over $\ZZ^2$.
Here $\lambda = -|k|^2$.

That is, for every $\ell \in \ZZ^+$, we can define
\begin{equation}
\label{torus steady stream}
\psi(x) = \sum_{|k|^2 = \ell} a_k \cos kx + b_k \sin kx
\end{equation}
and $u = \nabla^\perp \psi$ is a steady solution to the Euler equations on $\Torus^2$.

\begin{example}
Let $\ell = 2$. The solution
$$\psi(x) = \sin x_1 \cos x_2$$
has $6$ local extrema and $6$ saddle points.
\end{example}

\begin{example}
Now we solve the Navier-Stokes equation
$$\partial_t \omega^\nu + u^\nu \cdot \omega^\nu + \nu \Delta \omega^\nu$$
on $\Torus^2$, with streamline $\psi_0$ defined by (\ref{torus steady stream}).
Let us use the ansatz
$$\omega^\nu(t, x) = a(t) \psi_0(x)$$
where $a(0) = 1$.
First
$$u^\nu \cdot \nabla \omega^\nu = a(t)^2 u_0 \cdot \nabla \omega_0(x)$$
where $u_0 = \nabla^\perp \psi_0$.
But $u_0 \perp \nabla \omega_0$ so the Navier-Stokes equation reduces to the heat equation with
$$\Delta \omega^\nu = a(t) \Delta \omega_0 = a(t) \Delta^2 \psi_0.$$
But with $\psi_0$ a steady stream given by $\ell$,
$$\Delta \omega^\nu = -\ell a(t) \Delta \psi_0 = \ell^2 a(t) \psi_0.$$
Meanwhile
$$\partial_t \omega^\nu = -\ell a'(t) \psi_0.$$
Therefore the heat equation reduces to
$$a'(t) = -\nu \ell a(t).$$
With our initial data $a(0) = 1$, it follows that $a(t) = e^{-\nu \ell t}$ so
$$\omega^\nu(t, x) = e^{-\nu \ell t} \omega_0(x).$$
We get a bound
$$|\omega^\nu(t, x) - \omega_0(x)| = |e^{-\nu \ell t} - 1|\cdot|\omega_0(x)| \leq \nu \ell t ||\omega_0||_{L^\infty}$$
by the mean value theorem.
Thus these solutions satisfy the inviscid limit condition
$$||\omega^\nu(t) - \omega_0||_{L^\infty} \leq \nu \ell t ||\omega_0||_{L^\infty}$$
in $L^\infty$, which in particular implies the same for $L^2$.
\end{example}

\section{Explicit solutions in three dimensions}
Let us first treat the case ``$d = 2.5$".
This means that $u(x)$ only depends on $\tilde x = (x_1, x_2)$ even though $x \in \RR^3$.

\begin{theoremx}
Let $\tilde \psi$ be a stream function of a steady solution to the Euler equations with $d = 2$.
Then for any smooth function $W: \RR \to \RR$,
$$u(x) = (-\partial_2 \tilde \psi(\tilde x), \partial_1 \psi(\tilde x), W(\tilde \psi(\tilde x)))$$
is a steady solution of the Euler equations with $d = 3$.
\end{theoremx}

This theorem is an easy exercise once we recall that a steady solution to the Euler equations with $d = 3$ satisfies
$$u \cdot \nabla \omega = \omega \cdot \nabla u.$$

\begin{definition}
A \dfn{Beltrami field} is a vector field $u$ such that there is a scalar field $\lambda$ such that $\curl u = \lambda u$.
\end{definition}

If $u$ is an incompressible velocity field such that
$$\nabla \lambda \cdot u = 0$$
then $u$ is Beltrami, and in fact solves the steady Euler equations with $d = 3$ and $\omega = \lambda u$.
Indeed we must show
$$u \cdot \nabla \omega = \omega \cdot \nabla u.$$
But
$$u \cdot \nabla \omega = u \cdot \nabla(\lambda u) = (u \cdot \nabla \lambda)u + \lambda u \cdot \nabla u = \omega \cdot \nabla u.$$
The converse also holds.

Assume $\psi: \RR^2 \to \RR$, $\Delta \psi = F(\psi)$, $W,F$ smooth functions. Then
$$u(x) = (-\partial_2 \tilde \psi(\tilde x), \partial_1 \psi(\tilde x), W(\tilde \psi(\tilde x)))$$
is a solution of the Euler equations on $\RR^3$ with vorticity
$$\omega = (W'(\psi)\partial_2\psi, -W'(\psi) \partial_1\psi, W(\psi)).$$
If $u$ is Beltrami, then $u \times \omega = 0$, and hence
$$W'(\psi)W(\psi) = -F(\psi).$$

\begin{example}
Let
$$\psi(\tilde x) = \sum_{|k|^2 = \ell} a_k \cos kx + b_k \sin kx$$
be an eigenfunction of $\Delta$ on $\Torus^2$.
Then $\Delta \psi = \ell\psi$. We then need to solve the equation
$$W'(\psi)W(\psi) = \ell\psi$$
to get a Beltrami flow $u$. We get
$$(W(\psi)^2)' = 2\ell \psi$$
and hence
$$W(\psi)^2 = \ell \psi^2 + C.$$
If $C = 0$ we get $W(\psi) = \psi \sqrt \ell$.
We get a Beltrami flow with $d = 2.5$,
$$u = (-\partial_2\psi, \partial_1\psi, \psi \sqrt \ell),$$
on $\Torus^3$.
\end{example}

\chapter{Local well-posedness for Navier-Stokes}
Given an initial vector field $u(0)$, we would like to find a solution $u: [0, T) \to C^\infty$ to the Euler or Navier-Stokes equations, where $T > 0$ is maximal possible, and ideally $T = \infty$.
We would like to show that the solution is also unique.
In case the maximal time, say $T_*$, is finite, we would like to quantify the singularity at $T = T_*$.

Let $\nu \geq 0$, $d \geq 2$, and consider the initial-value problem for
$$\partial_t u + u \cdot \nabla u + \nabla p = \nu \Delta u$$
on $\RR^d$. Assume $u \in C^\infty([0, T) \to \Sch(\RR^d))$ where $\Sch(\RR^d)$ is the Schwartz space.
We showed that the kinetic energy
$$E(t) = \frac{1}{2} ||u(t)||_{L^2}^2$$
satisfies
$$E'(t) = -\nu ||\nabla u(t)||_{L^2}^2 \leq 0.$$
Therefore the $L^2$ norm of $u$ is decreasing.

The proof of local well-posedness will work as follows.
We first show a closed a priori estimate in some function space $X$, and then show the existence of an approximate equation to the Euler equations which satisfies the same a priori estimates in $X$.
Then if $u_\varepsilon$ is a solution to the $\varepsilon$-approximate equation, we can use compactness or similar to show that $u_\varepsilon \to u$ where $u$ is a solution to the Euler equations.
Note however the convergence may be weaker than convergence in $X$, but for example, if $X = L^2$ and convergence is weak convergence in $L^2$, then $u \in L^2$.

\section{Harmonic analysis review}
In our case we will need the function space $X$ to be a Sobolev space.
Let
$$||u||_{k,\Sch} = \sup_{|\alpha| \leq k} (1 + |x|^k) |\partial^\alpha u(x)|_\infty$$
be the $k$th Schwartz seminorm.
Let $\Sch'$ denote the dual of Schwartz space, thus the space of tempered distributions on $\Sch$.
We take the normalization
$$\hat u(\xi) = \int_{\RR^d} u(x) e^{-ix\xi} ~dx$$
of the Fourier transform, $u \in \Sch$.
Thus
$$u(x) = \frac{1}{2\pi} \int_{\RR^d} \hat u(\xi) e^{ix\xi} ~d\xi.$$
We define the Sobolev space to be the space of tempered distributions $u$ such that $\hat u \in L^2_{loc}$ with
$$||u||_{H^s}^2 = \int_{\RR^d} |\hat u(\xi)|^2 (1 + |\xi|^2)^s ~d\xi$$
finite; thus for example the Dirac mass $\delta$ satisfies $\delta \in H^{-d/2-\varepsilon}$ and this is best possible, since $\hat \delta = 1$.
Let
$$(\xi)^s = (1 + |\xi|^2)^{s/2}$$
denote the Japanese angle bracket.

\begin{proposition}
If $s > d/2$ then
$$||u||_{L^\infty} \lesssim ||u||_{H^s},$$
thus $H^s \subseteq L^\infty$.
\end{proposition}
\begin{proof}
One has
\begin{align*}|u(x)| &\lesssim \int_{\RR^d} |\hat u(\xi)| ~d\xi\\
&= \int_{\RR^d} (1 + |\xi|^2){-s/2} (1 + |\xi|^2)^{s/2} |\hat u(\xi)| ~d\xi\\
&\leq ||1||_{H^{-s}} ||\hat u||_{H^s}
\end{align*}
by H\"older's inequality.
\end{proof}

\begin{theorem}
One has
$$||fg||_{H^m} \lesssim ||f||_{L^\infty} ||g||_{H^m} + ||f||_{H^m} ||g||_{L^\infty}$$
and hence if $m > d/2$ then
$$||fg||_{H^m} \lesssim ||f||_{H^m} ||g||_{H^m}$$
and so $H^m$ is an algebra.

Moreover, if $|\alpha| = m$ then
$$||\partial^\alpha(fg) - f\partial^\alpha g||_{L^2} \lesssim ||\nabla f||_{L^\infty} ||g||_{H^{m-1}} + ||\nabla f||_{H^{m-1}} ||g||_{L^\infty}.$$
\end{theorem}
\begin{proof}
By H\"older's inequality,
$$||\partial^\alpha(fg)||_{L^2} \lesssim_\alpha \sum_{0 \leq \beta \leq \alpha} ||\partial^\beta f \partial^{\alpha - \beta} g|||_{L^2}
\leq \sum_{0 \leq \beta \leq \alpha} ||\partial^\beta f||_{L^{2|\alpha|/|\beta|}} ||\partial^{\alpha - \beta} g||_{L^{2|\alpha|/(|\alpha| - |\beta|)}}.$$
The Gagliardo-Nirenberg inequality
$$||\nabla^i u||_{L^{2r/i}} \lesssim ||u||_{L^\infty}^{1-i/r} ||\nabla^r u||_{L^2}^{i/r},$$
valid when $0 < i < r$, implies with $|\alpha| = r$, $|\beta| = i$,
\begin{align*}
||\partial^\alpha(fg)||_{L^2}
&\lesssim_\alpha ||f||_{L^\infty} ||\partial^\alpha g||_{L^2} + ||\partial^\alpha f||_{L^2} ||g||_{L^\infty} \\
&\qquad+ \sum_{0 < \beta < \alpha} ||\partial^\beta f||_{L^{2|\alpha|/|\beta|}} ||\partial^{\alpha - \beta} g||_{L^{2|\alpha|/(|\alpha| - |\beta|)}}\\
&=  ||f||_{L^\infty} ||\partial^\alpha g||_{L^2} + ||\partial^\alpha f||_{L^2} ||g||_{L^\infty} \\
&\qquad+
\sum_{0 < \beta < \alpha} ||f||_{L^\infty}^{1-1/|\beta||\alpha|} ||\partial^\beta f||_{L^2}^{|\beta|/|\alpha|} ||g||_{L^\infty}^{1-(|\alpha| - |\beta|)/|\alpha|} ||\partial^\alpha g||_{L^2}^{(|\alpha| - |\beta|)/|\alpha|}.
\end{align*}
Since $1 - 1/|\beta||\alpha| = (|\alpha - |\beta|)/|\alpha|$ and similarly for $|\beta|/|\alpha|$, we simplify to
\begin{align*}
||\partial^\alpha(fg)||_{L^2}
&\lesssim_\alpha ||f||_{L^\infty} ||\partial^\alpha g||_{L^2} + ||\partial^\alpha f||_{L^2} ||g||_{L^\infty}\\
&\qquad+ \sum_{0 < \beta < \alpha} (||f||_{L^\infty} ||\partial^\alpha g||_{L^2})^{1-|\beta|/|\alpha|} (||g||_{L^\infty} ||\partial^\alpha f||_{L^2})^{|\beta|/|\alpha|}\\
&\leq ||f||_{L^\infty} ||\partial^\alpha g||_{L^2} + ||\partial^\alpha f||_{L^2} ||g||_{L^\infty}\\
&\qquad+\sum_{0 < \beta < \alpha} ||f||_{L^\infty} ||\partial^\alpha g||_{L^2} + ||\partial^\alpha f||_{L^2} ||g||_{L^\infty}\\
&\leq ||f||_{L^\infty} ||\partial^\alpha g||_{L^2} + ||\partial^\alpha f||_{L^2} ||g||_{L^\infty}.
\end{align*}
This completes the proof of the product estimate.

For the commutator estimate,
$$||\partial^\alpha(fg) - f\partial^\alpha g||_{L^2} \lesssim_\alpha \sum_{0 < \beta \leq \alpha} ||\partial^\beta f \partial^{\alpha - \beta}g||_{L^2}$$
by the product rule. Without loss of generality let us assume $\beta_1 \geq 1$ and then set $\beta' = \beta - (1, 0, \dots, 0)$, $0 < \beta \leq |\alpha|$.
Thus $\alpha_1 \geq 1$, so set $\alpha' = \alpha - (1, 0, \dots, 0)$.
Then
$$\partial^\beta f \partial^{\alpha - \beta}g = \partial^{\beta'}\partial_1f \partial^{\alpha' - \beta'}g.$$
Then by the proof of the product estimate,
\begin{align*}
||\partial^{\beta'}\partial_1f \partial^{\alpha' - \beta'}g||_{L^2} &\lesssim ||\partial_1f||_{L^\infty} ||g||_{H^{m-1}} + ||\partial_1f||_{H^{m-1}} ||g||_{L^\infty}\\
&\lesssim ||\nabla f||_{L^\infty} ||g||_{H^{m-1}} + ||f||_{H^m} ||g||_{L^\infty}.
\end{align*}
This gives the proof of the commutator estimate.
\end{proof}

If $s \in \RR$, let $\Lambda^s$ be the Fourier multiplier
$$\widehat{\Lambda^sf}(\xi) = (1 + |\xi|^2)^{s/2} \hat f(\xi).$$
Then $\Lambda^s$ is an isometry $H^s \to L^2$.

\begin{definition}
The \dfn{Leray projector} is the Fourier multiplier $\PP$ defined by
$$\widehat{(\PP u)^j}(\xi) = \widehat{u^j}(\xi) - \frac{1}{|\xi|^2} \sum_{k=1}^d \xi_j\xi_k \widehat{u^k}(\xi)$$
whenever $u$ is a tempered vector field on $\RR^d$.
\end{definition}

\begin{theorem}[Kato-Ponce estimates]
Let $s > 0$, $1 < p < \infty$. Then
$$||fg||_{H^s} \lesssim_s ||f||_{L^\infty} ||g||_{H^s} + ||f||_{H^s} ||g||_{L^\infty}.$$
Moreover, if $[\Lambda^s, f]g = \Lambda^s(fg) - f\Lambda^sg$, then
$$||[\Lambda^s, f]g||_{L^p} \lesssim ||\nabla f||_{L^\infty} ||\Lambda^{s-1}g||_{L^p} + ||\Lambda^sf||_{L^p} ||g||_{L^\infty}.$$
\end{theorem}
\begin{proof}
Hard.
\end{proof}

\begin{theorem}[Leray estimates]
If $u$ is a vector field, $\nabla \cdot(\PP u) = 0$.
Moreover if $s \geq 0$ and $u \in H^s$ then there is a tempered scalar field $\varphi$ such that
$$u = \PP u + \nabla \varphi,$$
and $\PP u \perp \nabla \varphi$ in $H^r$ whenever $0 \leq r \leq s$.
In particular, $\PP$ is the orthogonal projection from $H^s$ onto the subspace of divergence-free vector fields in $H^s$.
\end{theorem}
\begin{proof}
One can directly compute $\widehat{\nabla \cdot(\PP u)} = 0$.
We then set
$$\hat \varphi(\xi) = \frac{1}{i|\xi|^2} \sum_k \xi_k \widehat{u^k}(\xi)$$
It's easy to check that $\PP$ is bounded on $H^s$; one can then use integration by parts on
$$(\Lambda^s \PP u, \Lambda^s \nabla \varphi)_{L^2} = (\nabla \cdot(\PP\Lambda^su), \Lambda^s\varphi)_{L^2} = 0.$$
This easily gives the claim.
\end{proof}

\section{Local well-posedness for Euler in Sobolev spaces}
The modern notion of well-posedness for Cauchy problems was introduced by Hadamard in 1902.
A Cauchy problem is well-posed if there is a unique solution, and the solution must depend continuously on the initial data.

We will prove continuous dependence on initial data later on.
Recall that if $s > 1 + d/2$ then $H^s(\RR^d) \subset W^{1, \infty}(\RR^d)$, and in that case the flow map is uniquely determined since every element of $W^{1,\infty}$ is Lipschitz.

\begin{theorem}
Let $d \geq 2$, $s > 1+d/2$. Consider the Euler system
\begin{equation}
\label{Euler well posed}
\begin{cases}
(\partial_t + u \cdot \nabla)u + \nabla p = 0\\
\nabla \cdot u = 0 \\
u(0) = u_0.
\end{cases}
\end{equation}
For every $u_0 \in H^s(\RR^d)$ with $\nabla \cdot u_0 = 0$, there is a
$$T \gtrsim_s \frac{1}{||u_0||_{H^s(\RR^d)}}$$
and unique $u \in C([0, T] \to H^s(\RR^d))$.
\end{theorem}

We start by proving a priori estimates in $H^s$.
The Leray projection
\begin{equation}
\label{Leray Euler well posed}
0 = \PP((\partial_t + u \cdot \nabla)u + \nabla p) = \partial_t u + \PP(u \cdot \nabla u)
\end{equation}
eliminates the pressure from the Euler system.

\begin{lemma}
Let $u$ be a Schwartz solution to (\ref{Euler well posed}) with
$$u \in C([0, T] \to H^{s+1+d/2})$$
and $s > 1 + d/2$.
Then
$$\partial_t ||u(t)||_{H^s}^2 \lesssim_s ||\nabla u(t)||_{L^\infty} ||u(t)||_{H^s}.$$
\end{lemma}
\begin{proof}
By (\ref{Leray Euler well posed}) because $\PP$ is self-adjoint (since $\PP$ has real symbol),
$$\frac{1}{2}\partial_t ||u(t)||_{L^2}^2 = (\partial_t u, u)_{L^2} = -(\PP(u \cdot \nabla u), u)_{L^2} = -(u \cdot \nabla u, u)_{L^2}$$
which gives
$$\partial_t ||u(t)||_{L^2}^2 = -\int_{\RR^d} u \cdot \nabla |u|^2 = \int_{\RR^d}|u|^2 \nabla \cdot u = 0.$$
This estimate is good in $L^2$ and can easily be adapted to $H^s$ by replacing $u$ with $\Lambda^su$.

Apply $\Lambda^s$ to (\ref{Leray Euler well posed}) to get
$$0 = \partial_t \Lambda^s u + \Lambda^s \PP(u \cdot \nabla u) = \partial_t \Lambda^s u + u \cdot \nabla \Lambda^s u + \Lambda^s \PP(u \cdot \nabla u) - u \cdot \nabla \Lambda^s u.$$
Then
\begin{align*}
\frac{1}{2} \partial_t ||\Lambda^s u||_{L^2}^2 &= (\partial_t \Lambda^s u, \Lambda^s u)_{L^2}\\
&= -(u\cdot \nabla \Lambda^s u, \Lambda^s u)_{L^2} - (\Lambda^s \PP(u \cdot \nabla u), \Lambda^s u)_{L^2} + (u \cdot \nabla \Lambda^s u, \Lambda^s u)_{L^2}.
\end{align*}
Then
$$(u\cdot \nabla \Lambda^s u, \Lambda^s u)_{L^2} = \frac{1}{2}\int_{\RR^d} u \cdot \nabla |\Lambda^s u|^2 = -\frac{1}{2}\int_{\RR^d} (\nabla \cdot u) |\Lambda^s u|^2 = 0.$$
This was justified because $u \in H^{s+1+d/2}$ implies $\nabla \Lambda^s u \in H^{d/2}$.
So by the Sobolev embedding theorem says $\Lambda^s u \in L^\infty$ so $\Lambda^s u \nabla \Lambda^s u \in L^2$.
Therefore
\begin{align*}\frac{1}{2} \partial_t ||\Lambda^s u||_{L^2}^2 &= - (\Lambda^s \PP(u \cdot \nabla u), \Lambda^s u)_{L^2} + (u \cdot \nabla \Lambda^s u, \Lambda^s u)_{L^2}\\
&= -(\Lambda^s(u \cdot \nabla u), \PP \Lambda^s u) + (u \cdot \nabla \Lambda^s u, \Lambda^s u)\\
&= -(\Lambda^s(u \cdot \nabla u), \Lambda^s u) + (u \cdot \nabla \Lambda^s u, \Lambda^s u)\\
&= -(\Lambda^s(u \cdot \nabla u) - u \cdot \nabla \Lambda^s u, \Lambda^s u)\\
&= -([\Lambda^s, u] \cdot \nabla u, \Lambda^s u).
\end{align*}
By the Kato-Ponce commutator estimate,
$$||[\Lambda^s, u] \cdot \nabla u||_{L^2} \lesssim ||\nabla u||_{L^\infty} ||\Lambda^{s-1} \nabla u||_{L^2} + ||\Lambda^s u||_{L^\infty} ||\nabla u||_{L^\infty} \lesssim ||\nabla u||_{L^\infty} ||u||_{H^s}.$$
We deduce
$$\frac{1}{2} \partial_t ||\Lambda^s u||_{L^2}^2 \lesssim ||\nabla u||_{L^\infty} ||u||_{H^s}^2$$
which was desired.
\end{proof}

The significance of this estimate is that since $s > 1 + d/2$,
$$||\nabla u|_{L^\infty} \lesssim ||u||_{H^s}$$
and thus
$$\partial_t ||u(t)||_{H^s}^2 \lesssim ||u(t)||_{H^s}^3.$$
It's tempting to rewrite this as
$$\partial_t ||u(t)||_{H^s} \lesssim ||u(t)||_{H^s}^2$$
but this argument is invalid because a priori $u(t)$ could be zero.
Instead, we set $f(t) = ||u(t)||_{H^s}^2$ so
$$f' \lesssim f^{3/2}.$$
Solving the differential inequality we get
$$f^{-1/2} \geq f^{-1/2}(0) - Ct$$
for some $C$ which only depends on $s,d$.
If $f^{-1/2}(0) \geq Ct$
(that is,
$$t \lesssim \frac{1}{||u_0||_{H^s}})$$
then
$$\sqrt{f(t)} \leq \frac{1}{||u_0||_{H^s} - Ct}.$$
This gives us control on the norm of the solution based on the initial data.

Let us now approximate the Euler equation
$$\partial_t u + \PP(u \cdot \nabla u) = 0$$
by mollifying the frequency space.
Let $\chi$ be a smooth function of compact support on $\RR^d$ such that $\chi(\xi) = 1$ if $|\xi| < 1/2$.
Introduce the mollifier
$$J_\varepsilon = \chi(\varepsilon D)$$
where $D$ is meant in the sense of pseudodifferential operators, so that if $f$ is tempered,
$$\widehat{J_\varepsilon f}(\xi) = \chi(\varepsilon \xi) \hat f(\xi).$$

Recall that $T_n \to T$ in the strong operator topology provided that for every $f$, $T_nf \to Tf$.

\begin{lemma}
Let $\varepsilon > 0$ be small enough, $r \geq 0$, $1 \leq p \leq \infty$.
Then
$$||J_\varepsilon||_{L^2 \to H^r} \lesssim_r \frac{1}{\varepsilon^r}.$$
Moreover, $J_\varepsilon$ converges in the strong $H^r$ topology to the identity.
Similarly, $||J_\varepsilon||_{L^p \to L^p} \lesssim_p 1$ and $J_\varepsilon$ converges in the strong $L^p$ topology to the identity.
\end{lemma}
\begin{proof}
An easy exercise with pseudodifferential operators.
\end{proof}

Consider the approximate Euler system
$$\begin{cases}
\partial_t u^\varepsilon + \PP J_\varepsilon(J_\varepsilon u^\varepsilon \cdot \nabla(J_\varepsilon u^\varepsilon)) = 0\\
\nabla \cdot u^\varepsilon = 0\\
u^\varepsilon(0) = J_\varepsilon u_0
\end{cases}$$
with the compatibility condition $\nabla \cdot u_0 = 0$.
For any $s \in \RR$ we let $H^s_\sigma$ be the space of $u \in H^s$ with $\cdot \nabla u = 0$.
Then $H^s_\sigma$ is a closed subspace of $H^s$ and $\PP$ is the orthogonal projection $H^s \to H^s_\sigma$.
We let $L^2_\sigma = H^0_\sigma$.

Recall the Picard-Lindel\"of theorem on Banach spaces.
\begin{theorem}[Picard-Lindel\"of]
Let $X$ be a Banach space, $O \subseteq X$ open.
Suppose that $F: O \to X$ is a locally Lipschitz map in the sense that for every $a \in O$ there is a $L_a > 0$ and $U_a \ni a$ such that
$$||F(b_1) - F(b_2)|| \leq L_a ||b_1 - b_2||$$
whenever $b_1, b_2 \in U_a$. Then for every $f_0 \in O$, there is a $T > 0$ such that the ODE
$$f'(t) = F(f(t))$$
with initial data $f(0) = f_0$ has a unique solution $f \in C^1((-T, T) \to O)$.
Let $T^*$ be the maximal possible $T$. Then either $T^* = \infty$, or $f(t) \to \partial O$ as $t \to T^*$.
\end{theorem}
\begin{proof}
Use Picard iteration, which works since Cauchy sequences in $X$ converge.
\end{proof}

\begin{lemma}
Let $u_0 \in L^2_\sigma(\RR^d)$.
Then for every $\varepsilon > 0$ small enough, $r \geq 0$, and $T > 0$, the approximate Euler system has a unique global solution
$$u^\varepsilon \in C([0, T] \to H^r_\sigma(\RR^d)).$$
\end{lemma}
\begin{proof}
Let use the Picard-Lindel\"of theorem with $O = X = H^r_\sigma(\RR^d)$ and
$$F(v) = -\PP J_\varepsilon(J_\varepsilon v \cdot \nabla J_\varepsilon v).$$

We need to show that $F$ sends $X$ to itself. If $v \in H^r_\sigma$, then
\begin{align*}
||F(v)||_{H^r} &\lesssim ||J_\varepsilon(J_\varepsilon v \cdot \nabla J_\varepsilon v)||_{H^r}\\
&\lesssim ||J_\varepsilon v \cdot \nabla J_\varepsilon v||_{H^r}\\
&\lesssim ||J_\varepsilon v||_{L^\infty} ||\nabla J_\varepsilon v||_{H^r} + ||\nabla J_\varepsilon v||_{L^\infty} ||J_\varepsilon v||_{H^r}.
\end{align*}
By Sobolev embedding, then,
\begin{align*}
||F(v)||_{H^r} &\lesssim ||J_\varepsilon v||_{H^{d/2 + 2}} + ||J_\varepsilon v||_{H^{r+1}} + ||J_\varepsilon v||_{H^r} ||J_\varepsilon v||_{H^{d/2 + 2}}\\
&\lesssim \frac{||v||_{L^2} ||v||_{H^r}}{\varepsilon^{d/2+2}}  + \frac{||v||_{H^r} ||v||_{L^r}}{\varepsilon^{d/2+2}}\\
&\lesssim \frac{||v||_{H^r}^2}{\varepsilon^{d/2+2}} < \infty.
\end{align*}
In particular, $F$ is locally Lipschitz, so we may use the Picard-Lindel\"of theorem.
Let $T^*$ be the maximal time of existence.
If $T^* < \infty$ then, since $\partial O = \{\infty\}$,
$$\limsup_{t \to T^*} ||u^\varepsilon(t)||_{H^r} = \infty.$$
But we showed that $\partial_t u^\varepsilon \in H^\infty(\RR^d)$ where $H^\infty = \bigcap_\mu H^\mu$.
Thus
\begin{align*}
\frac{1}{2} \frac{d}{dt} ||u^\varepsilon(t)||_{L^2}^2 &= (\partial_t u^\varepsilon, u^\varepsilon) = -(\PP J_\varepsilon(J_\varepsilon u^\varepsilon \cdot \nabla J_\varepsilon u^\varepsilon), u^\varepsilon)\\
&= -(J_\varepsilon(J_\varepsilon u^\varepsilon \cdot \nabla J_\varepsilon u^\varepsilon), u^\varepsilon)\\
&= -(J_\varepsilon u^\varepsilon \cdot \nabla J_\varepsilon u^\varepsilon, J_\varepsilon u^\varepsilon)
\end{align*}
since $J_\varepsilon$ is self-adjoint (since it is a pseudodifferential operator of real symbol). But
$$(u \cdot \nabla v) \cdot v = u \cdot \nabla |v|^2/2$$
so
\begin{align*}
\frac{1}{2} \frac{d}{dt} ||u^\varepsilon(t)||_{L^2}^2 = - \frac{1}{2}\int_{\RR^d} J_\varepsilon u^\varepsilon \cdot \nabla |J_\varepsilon u^\varepsilon|^2 = 0
\end{align*}
where we integrated by parts to get a term with $\nabla \cdot (J_\varepsilon u^\varepsilon) = 0$, using $[J_\varepsilon, \nabla \cdot] = 0$.
Thus
$$||u^\varepsilon(t)||_{L^2} \lesssim ||u_0||_{L^2}$$
whenever $t < T^*$.
Therefore the solution cannot escape $L^2$.
But
\begin{align*}
\frac{1}{2} \frac{d}{dt} ||u^\varepsilon(t)||_{H^r}^2 &= (\partial_t u^\varepsilon, u^\varepsilon)_r \\
&= (F(u^\varepsilon), u^\varepsilon)_r \leq ||F(u^\varepsilon)||_{H^r} ||u^\varepsilon||_{H^r}\\
&\lesssim \frac{||u^\varepsilon||_{L^2} ||u^\varepsilon||_{H^r}^2}{\varepsilon^{d/2+2}}\\
&\lesssim_\varepsilon ||u^\varepsilon(t)||_{L^2} ||u^\varepsilon(t)||_{H^r}^2\\
&\lesssim ||u_0||_{L^2} ||u^\varepsilon(t)||_{H^r}^2
\end{align*}
which implies
\begin{align*}
||u^\varepsilon(t)||_{H^r}^2 &\leq ||u^\varepsilon(0)||_{H^r}^2 e^{tC_\varepsilon ||u_0||_{L^2}}\\
&\lesssim \frac{||u_0||_{L^2}^2}{\varepsilon} e^{tC_\varepsilon ||u_0||_{L^2}}
\end{align*}
provided $t < T^*$.
If
$$||u^\varepsilon(t)||_{H^r}^2 \to \infty$$
as $t \to T^*$ then so does $e^{Ct}$, so $T^* = \infty$.
\end{proof}

\begin{lemma}
Let $u_0 \in H^s(\RR^d)$, $s > d/2 + 1$. Then the unique global solution $u^\varepsilon$ of the approximate Euler system with initial data $u_0$ satisfies
$$\partial_t ||u^\varepsilon(t)||_{H^r} \lesssim ||\nabla u^\varepsilon||_{L^\infty} ||u^\varepsilon(t)||_{H^r}^2.$$
\end{lemma}
\begin{proof}
The proof is the same as the proof that the Euler system satisfies this bound.
\end{proof}

\begin{corollary}
There is a $C_0$ depending on $s$ (but not on $\varepsilon$) such that if
$$||u_0||_{H^s} < \frac{1}{C_0T}$$
then for every $t \in [0, T]$,
$$||u^\varepsilon(t)||_{H^s} \leq \frac{||u_0||_{H^s}}{1 - C_0T ||u_0||_{H^s}}.$$
\end{corollary}
\begin{proof}
Same as that for the Euler system.
\end{proof}

Let
$$T_0 = \frac{1}{2C_0||u_0||_{H^s}}.$$
Then for every $\varepsilon > 0$ and $t \leq T_0$,
$$||u^\varepsilon(t)||_{H^s} \leq 2||u_0||_{H^s}.$$
We are going to show that $u^\varepsilon$ converges to a solution to the Euler system in $L^2$, and then use interpolation to show that $u \in H^s$.

To do this, we show that $u^\varepsilon$ is a Cauchy sequence in $C([0, T_0], L^2_\sigma(\RR^d))$.
\begin{lemma}
For every $\varepsilon, \varepsilon' > 0$,
$$||u^\varepsilon - u^{\varepsilon'}||_{L^\infty([0, T_0] \to L^2_\sigma)} \lesssim_{||u_0||_{H^s},s} \max(\varepsilon, \varepsilon').$$
\end{lemma}
\begin{proof}
One has
\begin{align*}
\frac{1}{2} \partial_t ||u^\varepsilon(t) - u^{\varepsilon'}(t)||_{L^2}^2 &= -(\PP J_\varepsilon(J_\varepsilon u^\varepsilon \cdot \nabla J_\varepsilon u^\varepsilon) - \PP J_{\varepsilon'} u^{\varepsilon'} \cdot \nabla J_{\varepsilon'} u^{\varepsilon'}, u_\varepsilon - u_{\varepsilon'})\\
& \qquad -(J_\varepsilon(J_\varepsilon u^\varepsilon \cdot \nabla J_\varepsilon u^\varepsilon) - J_{\varepsilon'} u^{\varepsilon'} \cdot \nabla J_{\varepsilon'} u^{\varepsilon'}, \PP(u^\varepsilon - u^{\varepsilon'})).
\end{align*}
OK I can't write out this mess but the point is that you can break this up into five terms and the last one is
\begin{align*}R_5 &= (J_{\varepsilon'}(J_{\varepsilon'}u^{\varepsilon'} \cdot \nabla J_{\varepsilon'}(u^\varepsilon - u^{\varepsilon'})), u^\varepsilon - u^{\varepsilon'})\\
&= (J_{\varepsilon'}u^{\varepsilon'} \cdot \nabla J_{\varepsilon'}(u^\varepsilon - u^{\varepsilon'}), J_{\varepsilon'}(u^\varepsilon - u^{\varepsilon'}))\\
&= \frac{1}{2}\int_{\RR^d} J_{\varepsilon'} u^{\varepsilon'} \cdot \nabla |J_{\varepsilon'}(u^\varepsilon - u^{\varepsilon'})^2 \\
&= 0
\end{align*}
by integration by parts.
Let us now estimate
\begin{align*}
|R_1| &= |((J_\varepsilon - J_{\varepsilon'})(J_\varepsilon u^\varepsilon \cdot \nabla J_\varepsilon u^{\varepsilon}), u^\varepsilon - u^{\varepsilon'})|\\
&\leq ||(J_\varepsilon - J_{\varepsilon'})(J_\varepsilon u^\varepsilon \cdot \nabla J_\varepsilon u^{\varepsilon})||_{L^2} ||u^\varepsilon - u^{\varepsilon'}||_{L^2}.
\end{align*}
But
$$||J_\varepsilon - J_{\varepsilon'}||_{H^1 \to L^2} \lesssim \max(\varepsilon, \varepsilon').$$
Therefore
\begin{align*}
||(J_\varepsilon - J_{\varepsilon'})(J_\varepsilon u^\varepsilon \cdot \nabla J_\varepsilon u^{\varepsilon})||_{L^2} &\lesssim \max(\varepsilon, \varepsilon') ||J_\varepsilon u^\varepsilon \cdot \nabla u^\varepsilon||_{H^1}\\
&\lesssim \max(\varepsilon, \varepsilon') (||J_\varepsilon u^\varepsilon||_{L^\infty} ||\nabla u^\varepsilon||_{H^1} + ||J_\varepsilon u^\varepsilon||_{H^1} ||\nabla u^\varepsilon||_{L^\infty})\\
&\lesssim \max(\varepsilon, \varepsilon') ||u^\varepsilon||_{H^{s-1}} ||u^\varepsilon||_{H^2} + ||u^\varepsilon||_{H^1} ||u^\varepsilon||_{H^s}\\
&\lesssim \max(\varepsilon, \varepsilon') ||u^\varepsilon||_{H^s}^2.
\end{align*}
Here we used $s > 1 + d/2 \geq 2$.
We have similar bounds on $R_2, R_4$ by a similar argument.
Finally we bound
\begin{align*}
|R_3| &= (J_{\varepsilon'}(u^\varepsilon - u^{\varepsilon'})\cdot \nabla u^\varepsilon,u^\varepsilon - u^{\varepsilon'}) \\
&\leq ||J_{\varepsilon'}(u^\varepsilon - u^{\varepsilon'})\cdot \nabla u^\varepsilon||_{L^2} ||u^\varepsilon - u^{\varepsilon'}||_{L^2}\\
&\lesssim ||J_\varepsilon(u^\varepsilon - u^{\varepsilon'})||_{L^2} ||\nabla u^\varepsilon||_{L^\infty} ||u^\varepsilon - u^{\varepsilon'}||_{L^2}\\
&\lesssim ||u^\varepsilon - u^{\varepsilon'}||_{L^2}^2 ||u^\varepsilon||_{H^s}.
\end{align*}
On $[0, T_0]$,
$$||u^\varepsilon||_{L^\infty_tH^s_x} \leq 2||u_0||_{H^s}.$$
Therefore
\begin{align*}
\partial_t ||u^\varepsilon - u^{\varepsilon'}||_{L^2}^2 &\lesssim \max(\varepsilon, \varepsilon') ||u^\varepsilon||_{H^s}^2 ||u^\varepsilon - u^{\varepsilon'}||_{L^2} + ||u^\varepsilon||_{H^s} ||u^\varepsilon - u^{\varepsilon'}||_{L^2}^2\\
&\lesssim \max(\varepsilon, \varepsilon')||u^\varepsilon - u^{\varepsilon'}||_{L^2} + ||u^\varepsilon - u^{\varepsilon'}||_{L^2}^2
\end{align*}
where the implied constant depends on $||u_0||_{H^s}$.
By Gr\"onwall's inequality,
$$f(t) \leq (f(0) + Ct\max(\varepsilon, \varepsilon')^2)e^{Ct}$$
where $f(t) = ||u^\varepsilon - u^{\varepsilon'}||_{L^\infty([0, T_0] \to L^2_\sigma)}^2$.
But
$$f(0) = ||u^\varepsilon(0) - u^{\varepsilon'}(0)||_{L^2}^2 = ||J_\varepsilon u_0 - J_{\varepsilon'} u_0||_{L^2}^2 \lesssim (\varepsilon, \varepsilon') ||u_0||_{H^1}^2.$$
Now take square roots.
\end{proof}

Now set $u_n = u^{\varepsilon_n}$ where $\varepsilon_n \to 0$. Then
$$||(u_m)_m||_{\ell^\infty_n L^\infty_t H^s_x} \leq 2 ||u_0||_{H^s} \lesssim 1.$$
Since the dual of $L^1$ is $L^\infty$, the Banach-Alaoglu theorem implies that there is $u \in L^\infty_t H^s_x$ such that $u_n \to u$ in weakstar $L^\infty_t H^s_x$, possibly after passing to a subsequence.
By the $L^2$ estimates that we just proved, it follows that $u_n \to u$ in (strong) $C_t L^2_x$.
We want to interpolate to get strong convergence in $C_t H^{s - \varepsilon}_x$.

\begin{lemma}
Let $s_1 < s_0 < s_2$.
Then for every $f \in H^{s_2}$,
$$||f||_{H^{s_0}} \leq ||f||_{H^{s_1}}^{\frac{s_2 - s_0}{s_2 - s_1}} \cdot ||f||_{H^{s_2}}^{\frac{s_0 - s_1}{s_2 - s_1}}.$$
\end{lemma}
\begin{proof}
Exercise.
\end{proof}

Applying the lemma with $f = u_k(t) - u_\ell(t)$ we obtain for every $0 < r < s$,
\begin{align*}
||u_k(t) - u_\ell(t)||_{H^r} &\leq ||u_k(t) - u_\ell(t)||_{L^2}^{1-r/s} ||u_k(t) - u_\ell(t)||_{H^s}^{r/s}\\
&\leq ||u_k(t) - u_\ell(t)||_{L^s}^{1-r/s} (4||u_0||_{H^s})^{r/s}.
\end{align*}
This implies, using the strong $C_t L^2_x$ convergence, shows that $(u_n)$ is Cauchy in $C_t H^r_x$.
Hence $u_n \to u$ strongly in $C_t H^r_x$.

\begin{lemma}
If $u$ is the limit just obtained then $u$ solves the Euler system, $u(0) = u_0$, $\nabla \cdot u = 0$, and
$$\partial_t u + \PP(u \cdot \nabla u) = 0.$$
\end{lemma}
\begin{proof}
Easy exercise, use the fact that $u^\varepsilon$ solves the approximate Euler system in a weak norm and converges strongly to $u$.
\end{proof}

The solution that we have constructed is
$$u \in L^\infty([0, T_0] \to H^s) \cap C([0, T_0] \to H^r).$$
We need to replace that $r$ with an $s$.

Let $X$ be a Hilbert space and $I$ a time interval. Then $C_w(I \to X)$ denotes the space of continuous functions $I \to X$, where $X$ is given the weak topology.
That is, for every $\varphi \in X^*$ and $u \in C_w(I \to X)$, the map
$$t \mapsto (\varphi, u(t))$$
is continuous $I \to \CC$.

\begin{lemma}
One has $u \in C_w([0, T_0] \to H^s)$.
\end{lemma}
\begin{proof}
For every $r < s$, $u_n \to u$ in $C([0, T_0] \to H^r)$.
Therefore for every $\varphi \in H^{-r} = (H^r)^*$,
$$(\varphi, u_n(t))_{L^2} \to (\varphi, u(t))_{L^2}$$
uniformly in $t$.
Now suppose $\varphi \in H^{-s}$.
Since $H^{-r}$ is dense in $H^{-s}$, for every $\eta > 0$ there is $\tilde \varphi \in H^{-r}$ such that
$$||\varphi - \tilde \varphi||_{H^{-s}} < \eta.$$
So
$$(\varphi, u_n(t)) - (\varphi, u(t)) = (\tilde \varphi, u_n(t)) - (\tilde \varphi, u(t)) + (\varphi - \tilde \varphi, u_n(t)) - (\varphi - \tilde \varphi, u(t)).$$
Also
$$||(\varphi - \tilde \varphi, u_n(t)) - (\varphi - \tilde \varphi, u(t))|| \leq 4\eta ||u_0||_{H^s}.$$
Since $-r > -s$, the duality pairing $(\tilde \varphi, u_n(t)) - (\tilde \varphi, u(t))$, ostensibly a pairing between $H^s$ and $H^{-s}$, is also a pairing between $H^r$ and $H^{-r}$.
Therefore the convergence in $H^r$ shows that it converges to $0$ uniformly in $t$.
Therefore
$$t \mapsto (\varphi, u(t))$$
is continuous.
\end{proof}

Now we turn the weak convergence into strong convergence:

\begin{lemma}
One has $u \in C([0, T_0], H^s)$.
\end{lemma}
\begin{proof}
Suppose $t_n \to t_0 \in [0, T_0]$. It suffices to show that
$$||u(t_n)||_{H^s} \to ||u(t_0)||_{H^s}$$
since we already know $u(t_n) \to u(t_0)$ in weak $H^s$, by the previous lemma.
Indeed,
$$||u(t_n) - u(t_0)||_{H^s}^2 = ||u(t_n)||_{H^s}^2 - 2(u(t_n), u(t_0))_{H^s} + ||u(t_0)||_{H^s}^2.$$
If we have
$$||u(t_n)||_{H^s} \to ||u(t_0)||_{H^s}$$
then
$$||u(t_n) - u(t_0)||_{H^s}^2 \to 2||u(t_0)||_{H^s}^2 - 2||u(t_0)||_{H^s}^2 = 0.$$

We now show
$$||u(t_n)||_{H^s} \to ||u(0)||_{H^s}$$
assuming $t_n \to 0$.
Then since $u \in C_w([0, T_0] \to H^s)$, $u(t_n) \to u(0)$ in weak $H^s$ and hence
$$||u(0)||_{H^s} \leq \liminf_{n \to \infty} ||u(t_n)||_{H^s}.$$
Recall that if $T \leq T_0$ then
$$\sup_{t \in [0, T]} ||u_{n_k}(t)||_{H^s} \leq \frac{||u_0||_{H^s}}{1 - CT||u_0||_{H^s}}$$
where $u_{n_k}$ is a subsequence of the mollified solution $u^\varepsilon$, and hence converges in weakstar $H^s$ to $u$.
Therefore
$$\sup_{t \in [0, T]} ||u(t)||_{H^s} \leq \frac{||u_0||_{H^s}}{1 - CT||u_0||_{H^s}}$$
Taking $T = t_n$ and $n \to \infty$ we conclude
$$\limsup_{n \to \infty} ||u(t_n)||_{H^s} \leq ||u_0||_{H^s}$$
and hence
$$\lim_{n \to \infty} ||u(t_n)||_{H^s} = ||u_0||_{H^s}$$
(since for example $u(0) = u_0$).
Therefore $u$ is right-continuous at $0$.

Now take $t_0 > 0$.
View $u(t_0)$ to be the initial data of the Euler system with initial time $t_0$, and let $\overline u$ be the induced solution on $[t_0, t_0 + T']$ where $T'$ is less than the blowup time of the initial data $u(t_0)$.
The above argument shows that $\overline u$ is right-continuous at $t_0$.
By uniqueness for the Euler system, $u = \overline u$ on $[t_0, t_0 + T']$ if $t_0 + T' \leq T_0$.
Therefore $u$ is right-continuous at $t_0$.

Finally we prove left-continuity at $t_0$.
In fact $\tilde u(x, t) = u(x, 2t_0 - t)$ solves
$$\partial_t \tilde u - \PP(\tilde u \cdot \nabla \tilde u) = 0.$$
The energy estimates of the Euler system do not care whether we have $\PP$ or $-\PP$ (since we have conservation of energy), so we can use right-continuity of $\tilde u$ to get left-continuity of $u$.
\end{proof}

This completes the proof of the local well-posedness of the Euler system.

\begin{theorem}
Let $u_0 \in H^s_\sigma(\RR^d)$ satisfy $s > 1 + d/2$ and $d \geq 2$.
Then there is a unique $u \in C([0, T] \to H^s(\RR^d)) \cap L^2([0, T] \to H^{s+1}(\RR^d))$
to the Navier-Stokes system
$$\partial_t u + \PP(u \cdot \Delta u) = \nu \Delta u$$
assuming that the viscosity $\nu > 0$.
Moreover,
$$T \gtrsim_s ||u_0||_{H^s}^{-1}$$
uniformly in $\nu$.
\end{theorem}

We omit the details of the proof since they are very similar to the Euler local well-posedness.
In fact,
$$\frac{1}{2} \partial_t ||u(t)||_{H^s}^2 = -(\PP(u \cdot \nabla u), u)_{H^s} + \nu (\Delta u, u)_{H^s}.$$
The Euler estimates already take care of the term $-(\PP(u \cdot \nabla u), u)_{H^s}$.
The viscosity term integrates by parts to
$$\nu (\Delta u, u)_{H^s} = -\nu ||\nabla u||_{H^s}^2 \leq 0.$$
Therefore the Euler energy estimates still apply to the Navier-Stokes system.
This gives the uniform bound
$$T \gtrsim_s ||u_0||_{H^s}^{-1}$$
on the blowup time $T$, which in turn gives
$$||u||_{L^\infty([0, T] \to H^s)} \leq \frac{||u_0||_{H^s}}{1 - CT||u_0||_{H^s}}$$
where $C$ depends on $s$ but not $\nu$.
Integrating the Navier-Stokes energy estimates in time, we get
$$\frac{1}{2} ||u(t)||_{H^s} + \nu \int_0^t ||\nabla u(s)||_{H^s}^2 ~ds \leq \frac{1}{2} ||u_0||_{H^s} + C \int_0^t ||\nabla u(s)||_{L^\infty} ||u(s)||_{H^s}^2 ~ds.$$
Since $||u(t)||_{H^s} \geq 0$ and $s > 1 + d/2$ we can use Sobolev embedding theory to get
$$\nu \int_0^t ||\nabla u(s)||_{H^s}^2 ~ds \leq \frac{1}{2} ||u_0||_{H^s} + Ct||u||_{L^\infty([0, t] \to H^s)}^3.$$
By energy estimates we get a bound on $||\nabla u||_{L^2_tH^s_x}$ which is allowed to depend on $s$ and the viscosity $\nu$, but at least is finite.

\section{Global well-posedness}
Of course we can't prove global well-posedness in general because this is a Millennium Problem.
But we can give criteria under which the solutions cannot blow up in the short term.

\begin{proposition}
Let $T > 0$, possibly larger than the blowup time given by the local well-posedness construction above.
Let $u \in C([0, T] \to H^s)$ be a solution of the Euler system with $s > 1 + d/2$ and $d \geq 2$.
Then
$$||u||_{L^\infty([0, T] \to H^s(\RR^d))} \leq ||u(0)||_{H^s} \exp\left(C \int_0^T ||\nabla u(s)||_{L^\infty} ~ds\right)$$
where $C$ only depends on $s$.
\end{proposition}
\begin{proof}
Use Gr\"onwall's inequality.
\end{proof}

\begin{corollary}
Let $u \in C([0, T) \to H^s)$ be a solution of the Euler system with $s > 1 + d/2$, $d \geq 2$, and $T < \infty$.
Then there is a $\delta > 0$ such that $u$ can be extended to a solution in $C([0, T + \delta] \to H^s)$ iff
$$\int_0^T ||\nabla u(t)||_{L^\infty(\RR^d)} ~dt < \infty.$$
\end{corollary}
\begin{proof}
If $u$ can be extended, then
$$\int_0^T ||\nabla u(t)||_{L^\infty} ~dt \lesssim T ||u||_{C([0, T] \to H^s)} < \infty$$
by Sobolev embedding.
Conversely, if
$$\int_0^T ||\nabla u(t)||_{L^\infty(\RR^d)} ~dt < \infty$$
and $T' < T$ then we have
$$||u||_{L^\infty([0, T'] \to H^s)} \lesssim_T ||u(0)||_{H^s} < \infty.$$
Therefore $M = ||u||_{L^\infty([0, T) \to H^s} < \infty$ as well since $T'$ is arbitrary.
Then for any data we can find a unique solution for time $\gtrsim 1/M$.
Taking $u(T - 1/2CM)$ as an initial datum of size $\leq M$, we get a solution up to time $T + 1/2CM > T$.
\end{proof}

Recall that if $d = 3$ then
$$u = -\nabla \times \Delta^{-1} \omega$$
where $\omega$ is the vorticity; then
$$\nabla u = - \nabla\nabla \times \Delta^{-1}\omega.$$
The operator $-\nabla\nabla \times \Delta^{-1}$ is a pseudodifferential operator of order $0$ so it is bounded on $L^p$ if $1 < p < \infty$, by Calder\'on-Zygmund theory.
Therefore
$$||\nabla u||_{L^p} \lesssim_p ||\omega||_{L^p}.$$
Unfortunately we cannot take $p \to \infty$ uniformly, and instead we get
$$||\nabla u||_{L^\infty} \lesssim ||\omega||_{BMO}.$$
Therefore it is not obvious that $||\omega||_{L^\infty}$ controls the blowup time of $u$.
Miraculously, however, it is.

\begin{theorem}[Beale-Kato-Majda criterion]
Let $u_0 \in H^s_\sigma(\RR^d)$ where $2 \leq d \leq 3$ and $s > 1 + d/2$.
Assume that $u \in C([0, T) \to H^s)$ is a solution of the Euler system with initial datum $u_0$.
Then $u$ can be continued to time $T + \delta$, for some $\delta > 0$, iff the vorticity $\omega = \nabla \times u$ satisfies $\omega \in L^1([0, T] \to L^\infty)$.
\end{theorem}

We will prove the Beale-Kato-Majda criterion shortly.

Thus the vorticity is the quantity we have to control if we want global well-posedness.
This is exactly what we need in $L^2$:

\begin{corollary}
If $d = 2$ then the Euler system is well-posed for all time.
\end{corollary}
\begin{proof}
If $d = 2$ then for every $t \geq 0$,
$$||\omega(t)||_{L^\infty} = ||\omega(0)||_{L^\infty}$$
as long as $u$ exists to time $t$ and $\omega$ is Schwartz.
Approximation arguments allow us to remove the assumption on $\omega$.
Assume that $u$ blows up at time $T^*$.
If $T^* < \infty$ then
$$T^* ||\omega(0)||_{L^\infty} = \int_0^{T^*} ||\omega(t)||_{L^\infty} ~dt = \infty$$
by the Beale-Kato-Majda criterion, and solving for $T^*$ we get $T^* = \infty$.
\end{proof}

Coming up with something similar in $d = 3$ would not quite solve the Millennium Problem since it would just work for the Euler system, not Navier-Stokes, but it would come very close.

Let us begin the proof of the Beale-Kato-Majda criterion.

Let
$$J = \begin{bmatrix}0 & -1\\1 & 0\end{bmatrix}, \qquad \sigma(z) = \begin{bmatrix}2z_1z_2 & z_2^2 - z_1^2 \\ z_1^2 - z_2^2 & -2z_1z_2\end{bmatrix}.$$
We need to differentiate the Biot-Savart law
$$u(x) = \int_{\RR^2} K_2(x - y) \omega(y) ~dy$$
where $K_2(z) = z^\perp/2\pi|z|^2$. But we cannot deduce
$$\nabla u(x) = \int_{\RR^2} \partial_x K_2(x - y) \omega(y) ~dy$$
at least in the sense of the Lebesgue integral. In fact $\partial_x K_2(x - y) \sim |x - y|^{-2}$, so we cannot commute the derivative with the integral.

\begin{lemma}
Let $K \in L^1_{loc}(\RR^d)$ and $d \geq 2$ satisfy $K(\lambda x) = \lambda^{1-d} K(x)$.
Then the derivative of $K$ satisfies for every test function $\varphi$,
$$(\partial_j K, \varphi) = \lim_{\varepsilon \to 0} \int_{|x| > \varepsilon} \partial_j K(x) \varphi(x) ~dx - c_j\varphi(0)$$
where
$$c_j = \int_{|x| = 1} K(x) x_j ~dS(x).$$
Moreover, all limits involved exist, so $\partial_j K$ is a distribution.
\end{lemma}
\begin{proof}
By definition
$$(\partial_j K, \varphi) = -(K, \partial_j\varphi).$$
Since $K \in L^1_{loc}$,
$$-(K, \partial_j\varphi) = -\int_{\RR^d} K(x) \partial_j \varphi(x) ~dx$$
so by dominated convergence, this is
$$-(K, \partial_j\varphi) = \lim_{\varepsilon \to 0} \int_{|x| > \varepsilon} K(x) \partial_j \varphi(x) ~dx.$$
Integrating by parts,
$$ \int_{|x| > \varepsilon} K(x) \partial_j \varphi(x) ~dx = -\int_{|x| > \varepsilon} \partial_j K(x) \varphi(x) ~dx + \int_{|x| = \varepsilon} K(x) \varphi(x) \frac{x_j}{|x|} ~dx.$$
But, if $x = \varepsilon z$,
$$\lim_{\varepsilon \to 0} \int_{|x| = \varepsilon} K(x) \varphi(x) \frac{x_j}{|x|} ~dx = \lim_{\varepsilon \to 0} \int_{|z| = 1} \varepsilon^{1 - d} K(z) \varphi(\varepsilon z) z_j \varepsilon^{d - 1} ~dz$$
using the homogeneity.
This is
\begin{align*}
\lim_{\varepsilon \to 0} \int_{|x| = \varepsilon} K(x) \varphi(x) \frac{x_j}{|x|} ~dx &= \lim_{\varepsilon \to 0} \int_{|z| = 1} K(z) \varphi(\varepsilon z) z_j ~dz\\
&= \int_{|z| = 1}  K(z) \varphi(0) z_j ~dz = \varphi(0) c_j
\end{align*}
by dominated convergence.
\end{proof}

\begin{lemma}
If $d = 2$ and $\omega$ is Schwartz then
$$\nabla u(x) = \frac{\omega(x)J}{2} + \frac{1}{2\pi} \int_{\RR^2} \frac{1}{|x - y|^2}\sigma\left(\frac{x - y}{|x -y|}\right) \omega(y) ~dy$$
where the integral is in the sense of a Cauchy principal value.
\end{lemma}
\begin{proof}
Taking the distributional derivative of the Biot-Savart law
$$u(x) = \int_{\RR^2} K_2(x - y) \omega(y) ~dy$$
with respect to a test function $\varphi$,
$$(\partial_1 u, \varphi) = -(u, \partial_1 \varphi).$$
By assumption on $\omega$, $u \in L^1_{loc}$ so this is
\begin{align*}
-(u, \partial_1 \varphi) &= -\int_{\RR^d} u(x) \partial_1 \varphi(x) ~dx\\
&= -\iint_{\RR^{d + d}} K_2(x - y) \omega(y) ~dy \partial_1 \varphi(x) ~dx \\
&= - \iint_{\RR^{d + d}} K_2(x - y) \omega(y) \partial_1 \varphi(x) ~dx ~dy\\
&= -\iint_{\RR^{d + d}} K_2(x) \partial_1 \varphi(x + y) ~dx \omega(y) ~dy\\
&= -\int_{\RR^d} K_2(x) \int_{\RR^d} \partial_1 \varphi(x + y) \omega(y) ~dy ~dx.
\end{align*}
We used Fubini's theorem.
Integrating by parts,
$$\int_{\RR^d} \partial_1 \varphi(x + y) \omega(y) ~dy = -\int_{\RR^d} \varphi(x + y) \partial_1 \omega(y) ~dy.$$
Therefore
\begin{align*}
-(u, \partial_1 \varphi) &= \iint_{\RR^{d + d}} K_2(x) \varphi(x + y) ~dx \partial_1 \omega(y) ~dy\\
&= \iint_{\RR^{d + d}} K_2(x - y) \varphi(x) ~dx \partial_1 \omega(y) ~dy\\
&= (\int_{\RR^d} K_2(\cdot - y) \partial_1 \omega(y) ~dy, \varphi)
\end{align*}
which makes sense since $K_2 \in L^1_{loc}$ and $\omega$ is Schwartz.
Thus we can drop $\varphi$ and deduce
$$\partial_1 u(x) = \int_{\RR^d} K_2(x - y) \partial_1 \omega(y) ~dy.$$
Since $K_2$ is odd,
\begin{align*}
\partial_1 u(x) &= -\int_{\RR^2} K_2(y - x) \partial_1 \omega(y) ~dy\\
&= -\int_{\RR^2} K_2(y) \partial_1 \omega(y + x) ~dy\\
&= -(K_2, \partial_1 \omega(x + \cdot)).
\end{align*}
By the previous lemma,
$$\partial_1 u(x) = \int_{\RR^2} \partial_1 K(y) \omega(y + x) ~dy - c_1\omega(x)$$
where
$$c_1 = \int_{|x| = 1} K_2(z) z_1 ~dz.$$
Now the proof is just a computation.
\end{proof}

\begin{lemma}
For every $h \in \RR^3$,
$$\nabla u(x)h = \frac{1}{3} \omega(x) \times h - \frac{1}{4\pi} \int_{\RR^3} \frac{\omega(y) \times h + 3(\frac{x - y}{|x -y|} \times \omega(y))\frac{x - y}{|x - y|}h}{|x - y|^3} ~dy.$$
\end{lemma}
\begin{proof}
Similar to the above.
\end{proof}

Now we show that singular integral operators are bounded into $L^\infty$ modulo logarithmic terms.
\begin{definition}
A \dfn{convolution-type Calder\'on-Zygmund operator} is an operator $T$ defined by
$$Tf(x) = \int_{\RR^d} \frac{f(x - y)}{|y|^d} \Omega(y/|y|) ~dy$$
where the integral is a principal value and $\Omega$ is a smooth function on $S^{d-1}$ such that $\int_{S^{d-1}} \Omega = 0$.
\end{definition}

\begin{theorem}
Let $f \in L^2 \cap L^\infty \cap C^\alpha$ where $0 < \alpha < 1$ is a H\"older exponent.
Let $T$ be the convolution-type Calder\'on-Zygmund operator induced by some $\Omega$. Then
$$||Tf||_{L^\infty} \lesssim_{\alpha, d, \Omega} ||f||_{L^2} + ||f||_{L^\infty}(1 + \log^+ \frac{[f]_\alpha}{||f||_{L^\infty}})$$
where $[f]_\alpha$ is the H\"older seminorm of $f$.
\end{theorem}
\begin{proof}
Let $\phi$ be a cutoff to $\{x: |x| \leq 1\}$.
Let $r \in (0, 1]$.
Write
\begin{align*}Tf(x) &= \int_{\RR^d} \phi(|y|/r) \frac{f(x - y)}{|y|^d} \Omega(y/|y|) ~dy \\
&\qquad+ \int_{\RR^d} (1 - \phi(|y|/r))\frac{f(x - y)}{|y|^d} \Omega(y/|y|) \phi(|y|) ~dy\\
&\qquad + \int_{\RR^d}  (1 - \phi(|y|/r))\frac{f(x - y)}{|y|^d} \Omega(y/|y|) (1 - \phi(|y|)) ~dy
\\
&= (Tf)_{in}(x) + (Tf)_{med}(x) + (Tf)_{out}(x).
\end{align*}
For $(Tf)_{in}$ (thus $|y| \lesssim 1$), we let $\varepsilon > 0$ and note that
\begin{align*}\int_{|y| \geq \varepsilon} \phi(|y|/r) \frac{1}{|y|^d} \Omega(y/|y|) ~dy
&= \int_\varepsilon^r \rho^{-d} \int_{S^{d-1}} \Omega(z) \phi(\rho/r) \rho^{d - 1} ~dz ~d\rho\\
&= \int_\varepsilon^r \rho(\rho/r) \rho^{-1} \int_{S^{d-1}} \Omega(z) ~dz ~d\rho \\
&= 0
\end{align*}
so
$$(Tf)_{in} = \int_{\RR^d} |y|^{-d} \Omega(y/|y|) \phi(|y|/r)(f(x - y) - f(x)) ~dy$$
implies
\begin{align*}
||(Tf)_{in}||_{L^\infty} &\leq \int_{|y| < r} y^{-d} ||\Omega||_{L^\infty} [f]_\alpha |y|^\alpha ~dy\\
&\lesssim_d ||\Omega||_{L^\infty} [f]_\alpha r^\alpha.
\end{align*}

For $(Tf)_{med}$, we bound
\begin{align*}
||Tf_{med}||_{L^\infty} &\leq \int_{r < |y| < 2} |y|^{-d} ||\Omega||_{L^\infty} |f(\cdot - y)| ~dy \\
&\leq ||\Omega||_{L^\infty} ||f||_{L^\infty} \int_{r < |y| < 2} \frac{dy}{|y|^d} \\
&\lesssim_d ||\Omega||_{L^\infty} ||f||_{L^\infty} \log(2/r).
\end{align*}

For $(Tf)_{out}$ we use the $L^2$ assumption on $f$ to get
\begin{align*}
||(Tf)_{out}||_{L^\infty} &\leq \int_{|y| > 1} |y|^{-d} ||\Omega||_{L^\infty} |f(x - y)| ~dy\\
&\leq ||\Omega||_{L^\infty} ||f||_{L^2} \left(\int_{|y| > 1} \frac{dy}{|y|^{2d}}\right)^{1/2}
\end{align*}
by the Cauchy-Schwarz inequality. Thus
$$||(Tf)_{out}||_{L^\infty} \lesssim_d ||\Omega||_{L^\infty} ||f||_{L^2}.$$
Summing up,
$$||Tf||_{L^\infty} \lesssim_d ||\Omega||_{L^\infty} ([f]_\alpha r^\alpha + ||f||_{L^\infty} \log(2/r) + ||f||_{L^2}).$$
Let
$$r^\alpha = \min(1, \frac{||f||_{L^\infty}}{[f]_\alpha}).$$
This gives the bound we want.
\end{proof}

\begin{corollary}
If $\alpha \in (0, 1)$ and $d \in \{2, 3\}$,
$$||\nabla u||_{L^\infty} \lesssim_d ||\omega||_{L^2} + ||\omega||_{L^\infty}(1 + \log^+ \frac{[\omega]_\alpha}{||\omega||_{L^\infty}}).$$
\end{corollary}
\begin{proof}
The differentiated Biot-Savart operator $\omega \mapsto \nabla u$ is a convolution-type Calder\'on-Zygmund operator.
\end{proof}

\begin{proof}[Proof of Beale-Kato-Majda criterion]
If $u$ can be continued past $T$ to a solution in $C([0, T + \delta) \to H^s)$ then
\begin{align*}
\int_0^T ||\omega(t)||_{L^\infty} ~dt &\lesssim \int_0^T ||\omega(t)||_{H^{s-1}} ~dt\\
& = \int_0^T ||u(t)||_{H^s} ~dt \leq ||u||_{C([0, T] \to H^s)} < \infty.
\end{align*}
Conversely, suppose $C_T = ||\omega||_{L^1([0, T) \to L^\infty)} < \infty$.
By local well-posedness, $u$ can be continued past $T$ provided that
$$u \in L^\infty([0, T) \to H^s).$$

If $t < T$ then
$$||u(t)||_{H^s} \leq ||u_0||_{H^s} \exp \int_0^t ||\nabla u(r)||_{L^\infty} ~dr.$$
Since $s > 1 + d/2$, if $\alpha < s - 1 - d/2$ then $\omega \in C^\alpha$ (since $\omega \in H^{s-1}$).
Therefore the Calder\'on-Zygmund estimates give
$$||\nabla u(t)||_{L^\infty} \lesssim ||\omega(t)||_{L^2} + ||\omega(t)||_{L^\infty}(1 + \log^+ \frac{||u||_{H^s}}{\omega(t)||_{L^\infty}}).$$
To bound $||\omega(t)||_{L^2}$ we use the vorticity equation
$$\partial_t \omega + u \cdot \nabla \omega = \omega \cdot \nabla u$$
to prove
$$.5 \partial_t ||\omega(t)||_{L^2}^2 = -(u \cdot \nabla \omega, \omega)_{L^2} + (\omega \cdot \nabla u, \omega)_{L^2}.$$
Using the divergence-free condition, $(u \cdot \nabla \omega, \omega)_{L^2} = 0$.
Also by Gr\"onwall's lemma,
$$||\omega(t)||_{L^2}^2 \leq ||\omega(0)||_{L^2}^2 \exp \int_0^t ||\omega(r)||_{L^\infty} ~dr = M^2$$
is finite, so
\begin{align*}
||\nabla u(t)||_{L^\infty} &\lesssim M + ||\omega(t)||_{L^\infty}(1 + \log^+ \frac{||u(t)||_{H^s}}{||\omega||_{L^\infty}})\\
&\lesssim M + (1 + ||\omega(t)||_{L^\infty})(1 + \log^+ ||u(t)||_{H^s})\\
&\lesssim M + (1 + ||\omega(t)||_{L^\infty}) \log(||u||_{H^s} + e).
\end{align*}
We conclude
$$||u(t)||_{H^s} \leq ||u_0||_{H^s} \exp(CM + C(1 + ||\omega||_{L^\infty})\log (||u(t)||_{H^s} + e))$$
so that
$$||u(t)||_{H^s} + e \leq (||u_0||_{H^s} + e)\exp(CM + C(1 + ||\omega||_{L^\infty})\log(||u(t)||_{H^s} + e)).$$
Taking the logarithm of both sides,
$$Z(t) = \log(||u(t)||_{H^s} + e)$$
satisfies
$$Z(t) \leq Z(0) + CM + C(1 + ||\omega(t)||_{L^\infty})Z(t).$$
By Gr\"onwall's lemma,
\begin{align*}Z(t) &\leq (Z(0) + CMt)\exp \int_0^t C(1 + ||\omega(r)||_{L^\infty}) ~dr\\
&\leq (Z(0) + CMt) \exp(Ct + ||\omega||_{L^1([0, T] \to L^\infty)}).
\end{align*}
Therefore
$$||u||_{L^\infty([0, T] \to H^s)} < \exp( (\log(||u_0||_{H^s} + e) + CMt) \exp(Ct + ||\omega||_{L^1([0, T] \to L^\infty)})) < \infty$$
which was desired.
\end{proof}

If $d = 2$, then if $Z$ is as in the preceding proof,
$$Z(t) \lesssim e^{C_0t}$$
where $C_0$ only depends on $||u_0||_{H^s}$.
Taking the exponential of both sides we conclude that $||u(t)||_{H^s}$ is doubly exponential in $t$:
$$||u(t)||_{H^s} \lesssim \exp \exp(C_0t)$$
where again $C_0$ and the implied constants only depend on $||u_0||_{H^s}$.
It is an open problem whether this bound is sharp in the sense that there is initial data $u_0$ satisfying $d = 2$ and
$$||u(t)||_{H^s} \sim \exp \exp(C_0 t).$$
The same bound holds for bounded open submanifolds of $\RR^2$.
In fact Kiselev and Sverak showed that the doubly-exponential growth is sharp on a disk, but the doubly-exponential growth happens at the boundary, so this argument does not transfer to $\RR^2$.

\begin{theorem}
The Beale-Kato-Majda criterion also holds for the Navier-Stokes equation.
In particular, if $d = 2$ then the Navier-Stokes equation is globally well-posed.
\end{theorem}
\begin{proof}
The proof of the Beale-Kato-Majda criterion only used energy estimates on the Euler equation, which are also valid for the Navier-Stokes equations.
\end{proof}

Later we shall show that we can get much sharper bounds on the Navier-Stokes equation, using the viscosity term to get decay in energy.

\chapter{Steady Navier-Stokes equations}
This chapter is based on lectures of Huy Quang Nguyen.
We consider steady states of the Navier-Stokes system, thus
$$-\nu \Delta u = u \cdot \Delta u + \nabla p = f$$
where $u$ is a divergence-free vector field on $\Omega \subseteq \RR^d$ such that
$$u|\partial \Omega = u_*$$
where $u_*$ is given, and $\partial \Omega$ is a closed Lipschitz manifold.

\section{Notation}
Let $1 < p < \infty$. We let $W^{1,p}_\sigma(\Omega)$ be the completion of the space $C_{c,\sigma}^\infty(\overline \Omega)$ of compactly supported, divergence-free vector fields on the compact manifold with boundary $\overline \Omega$ with respect to the Sobolev norm
$$||u||_{W^{1,p}(\Omega)} = ||u||_{L^p(\Omega)} + ||\nabla u||_{L^p(\Omega)}.$$
We let $W^{1,p}_{0,\sigma}(\Omega)$ be the completion of $C_{c,\sigma}^\infty(\Omega)$, the space of vector fields which are compactly supported in $\Omega$ (rather than $\overline \Omega$).
We let
$$V = W^{1,2}_{0,\sigma}(\Omega)$$
and let $V'$ be the dual space to $V$.
Then there is a Riesz representation isomorphism $V \to V'$.
Let $\langle \cdot, \cdot \rangle$ denote the pairing of $V'$ and $V$.

We fix $u_* \in L^p(\partial \Omega)$.

\begin{theorem}
There exists a unique bounded linear trace
$$\tr: W^{1,p}(\Omega) \to L^p(\partial \Omega)$$
which extends the classical trace; thus if $u$ is smooth then $\tr u = u|\partial \Omega$.
\end{theorem}
\begin{proof}
The proof is the same as in Evans' book; he incorrectly assumes that $\partial \Omega$ is $C^1$ but only uses Lipschitz.
\end{proof}
Henceforth we abuse notation and write $u|\partial \Omega$ to mean $\tr u$.

The Navier-Stokes operator is a second-order nonlinear operator due to the viscosity term.
Therefore the Navier-Stokes system only makes pointwise for functions in $W^{2,p}$.
Let us now define weak solutions of the Navier-Stokes system.

\begin{lemma}
Let $u$ be a smooth solution of the Navier-Stokes system. Then for every $\varphi \in C^\infty_{c,\sigma}(\Omega)$,
$$\int_\Omega -\nu\nabla u \nabla \varphi + (u \cdot \nabla u)\cdot \varphi = \int_\Omega f \cdot \varphi.$$
\end{lemma}
\begin{proof}
By assumption,
$$\int_\Omega -\nu\varphi\cdot\Delta u + (u\cdot\nabla u)\cdot\varphi + \nabla p \cdot \varphi = \int_\Omega f \cdot \varphi.$$
Integrating by parts,
$$\int_\Omega - \nu\varphi \cdot \Delta u = \int_\Omega \nu \nabla u \nabla \varphi$$
and
$$\int_\Omega f \cdot \varphi = -\int_\Omega p \nabla \cdot \varphi = 0$$
since $\varphi$ is divergence-free.
Thus
$$\int_\Omega -\nu\nabla u \nabla \varphi + (u \cdot \nabla u)\cdot \varphi = \int_\Omega f \cdot \varphi.$$
\end{proof}

\begin{definition}
Let $f \in V'$ and $u_* \in L^2(\partial \Omega)$.
A \dfn{weak solution to the steady-state Navier-Stokes system} is a vector field $u \in W^{1,2}_\sigma(\Omega)$ such that for every $\varphi \in C^\infty_{c,\sigma}(\Omega)$,
$$\int_\Omega -\nu\nabla u \nabla \varphi + (u \cdot \nabla u)\cdot \varphi = \langle f, \varphi\rangle$$
and $u|\partial \Omega = u_*$.
\end{definition}

\section{Small-large uniqueness}
\begin{lemma}
Let $u,w \in W^{1,2}_0(\Omega)$ and $v \in W^{1,2}(\Omega)$.
Then
$$\left|\int_\Omega (u \cdot \nabla v) \cdot w\right| \leq ||u||_{L^4} ||\nabla v||_{L^2} ||w||_{L^4} \lesssim_\Omega ||\nabla u||_{L^2} ||\nabla v||_{L^2} ||\nabla w||_{L^2}.$$
Moreover if $u$ is divergence-free then
$$\int_\Omega v \cdot \nabla u = 0.$$
\end{lemma}
\begin{proof}
We prove the first claim by H\"older duality and the Sobolev estimate
$$||u||_{L^p} \lesssim_\Omega ||u||_{W^{1,p}} \lesssim ||\nabla u||_{L^p}$$
where the estimate $ ||u||_{W^{1,p}} \lesssim ||\nabla u||_{L^p}$ follows from the Poincar\'e inequality since $u$ is trace-free.
This is valid whenever
$$1 \leq p \leq \frac{2d}{d - 2}$$
and either $d \neq 2$ or $p \neq \infty$.

The second claim holds if $u \in C^\infty_c(\Omega)$ since then
$$\int_\Omega (v \cdot \nabla u)u = \frac{1}{2} \int_\Omega v \cdot \nabla |u|^2 = -\frac{1}{2} \int_\Omega (\nabla \cdot v) |u|^2 = 0.$$
If $u \in W^{1,2}_0(\Omega)$ then let $(u_n) \in C^\infty_c(\Omega)$ satisfy $u_n \to u$. By the first part,
\begin{align*}
\left|\int_\Omega (v \cdot \nabla u_n) \cdot u_n - \int_\Omega (v \cdot \nabla u) \cdot u\right|
&\leq \left|\int_\Omega (v \cdot \nabla(u_n - u)) \cdot u_n\right| + \left|\int_\Omega (v \cdot \nabla u) \cdot (u_n - u)\right|\\
&\lesssim ||\nabla v||_{L^2} ||\nabla(u_n - u)||_{L^2} ||\nabla u_n||_{L^2} + ||\nabla v||_{L^2} ||\nabla u||_{L^2} ||\nabla(u_n - u)||_{L^2}
\end{align*}
which goes to $0$.
\end{proof}

\begin{theorem}[small-large uniqueness]
Let $u,v$ be two weak solutions of the steady Navier-Stokes system.
Let $C_1$ be the constant in the previous lemma. If
$$C_1 ||\nabla v||_{L^2} < 1$$
then $u = v$.
\end{theorem}
\begin{proof}
Let $w = u - v$; then $w$ is trace-free. From the definition of weak solution, for every $\varphi \in C^\infty_{c,\sigma}(\Omega)$,
\begin{align*}
0 &= \nu(\nabla w, \nabla \varphi) + (u \cdot \nabla u, \varphi) - (v \cdot \nabla v, \varphi)\\
&= \nu(\nabla w, \nabla \varphi) + (w \cdot \nabla w, \varphi) + (w \cdot \nabla v, \varphi) + (v \cdot \nabla w, \varphi).
\end{align*}
By the lemma, if $\varphi_n \to w$ in $W^{1,2}_0(\Omega)$, and $n \to \infty$, we conclude
$$0 = \nu(\nabla w, \nabla w) + (w \cdot \nabla w, w) + (w \cdot \nabla v, w) + (v \cdot \nabla w, w)$$
with $(w \cdot \nabla w, w) = (v \cdot \nabla w, w) = 0$. That is,
$$0 = \nu||\nabla w||_{L^2}^2 + (w \cdot \nabla v, w).$$
Using the lemma again,
$$0 \leq \nu ||\nabla w||_{L^2}^2 \leq C_1 ||\nabla w||_{L^2}^2 ||\nabla v||_{L^2}.$$
Thus
$$||\nabla w||_{L^2}^2(C_1||\nabla v||_{L^2} - C_1) \geq 0$$
but $C_1||\nabla v||_{L^2} - C_1 < 0$ by assumption. That means $\nabla w = 0$.
Since $w$ is gradient-free and trace-free, $w = 0$.
\end{proof}

In general solutions may not be unique however.

\section{Existence of solutions}
Let us first treat the case of zero-boundary solutions using the Galerkin method.

\begin{lemma}
Let $B_R = B(0, R)$ be a ball in $\RR^m$, $m \geq 1$, and $P$ a continuous vector field on $\overline B_R$ such that for every $x \in \partial B_R$,
$$P(x) \cdot x > 0.$$
Then there is $x_0 \in B_R$ with $P(x_0) = 0$.
\end{lemma}
\begin{proof}
This is a consequence of Brouwer's fixed point theorem.
If this fails to be true, then by the boundary hypothesis, for every $x \in \overline B_R$, $P(x) \neq 0$
Let
$$\Pi(x) = -R\frac{P(x)}{|P(x)|}.$$
Then $\Pi$ is continuous since $P \neq 0$ and $\Pi$ sends $\overline B_R$ to itself.
So $\Pi$ has a fixed point. But
$$P(x_0) \cdot x_0 = P(x_0) \Pi(x_0) = -R|P(x_0)| < 0$$
contradicting the boundary hypothesis.
\end{proof}

Let us recall the Galerkin method.
We search for approximate solutions in high-dimensional spaces, and pass to the limit in the infinite-dimensional Hilbert space where the exact solution lives.

\begin{theorem}
If $u_* = 0$ and $f \in V'$ then there is a weak solution $u \in V$ of the steady-state Navier-Stokes system such that
$$||\nabla u||_{L^2(\partial \Omega)} \leq \frac{||f||_{V'}}{\nu}.$$
\end{theorem}
\begin{proof}
Since $V$ is a separable Hilbert space, there is a countable orthonormal basis of $V$ consisting of $\psi_k \in C^\infty_{c,\sigma}(\Omega)$.
Let us find an approximate solution $u_m$ in the span of $\psi_1, \dots, \psi_m$.
We solve
\begin{equation}
\label{Galerkin NS}
\nu(\nabla u_m, \nabla \psi_k) + (u_m \cdot \nabla u_m, \psi_k) = \langle f, \psi_k\rangle.
\end{equation}
If $u_m$ is a solution, there is $\alpha \in \RR^m$ such that
$$u_m = \sum_{k=1}^m \alpha_k \psi_k.$$
Let $P: \RR^m \to \RR^m$ satisfy
$$(P\alpha)_k = \nu(\nabla v, \nabla \psi_k) + (v \cdot \nabla v, \psi_k) - \langle f,\psi_k\rangle$$
where $v = \sum_k \alpha_k \psi_k$.

Clearly $P$ is a continuous vector field, and
\begin{align*}
P(\alpha) \cdot \alpha &= \sum_{k=1}^m (P(\alpha))_k \alpha_k\\
&= \sum_{k=1}^\infty \nu(\nabla v, \nabla \alpha_k \psi_k) + (v \cdot \nabla v, \alpha_k \psi_k) - \langle f, \alpha_k \psi_k\rangle\\
&= \nu(\nabla v, \nabla v) + (v \cdot \nabla v, v) - \langle f,v\rangle\\
&\geq \nu||\nabla v||_{L^2}^2 + 0 - ||f||_{V'} ||\nabla v||_{L^2}\\
&= ||\nabla v||_{L^2}(\nu ||\nabla v||_{L^2} - ||f||_{V'})\\
&\gtrsim ||\nabla v||_{L^2}(\nu ||v||_{L^2} - ||f||_{V'})
\end{align*}
by Poincar\'e's inequality. Since $||v||_{L^2} = ||\alpha||_{\ell^2}$, if $||\alpha||_{\ell^2} = M \gg 1$ then $P(\alpha) \cdot \alpha > 0$.
So by the lemma there is $\alpha^*$ such that $||\alpha^*||_{\ell^2} \leq M$ and $P\alpha^* = 0$.
Therefore $\alpha^*$ defines $u_m = \sum_{k \leq m} \alpha_k^*\psi_k$ which is a solution of the approximate equation (\ref{Galerkin NS}).

Since $P(\alpha^*) = 0$, $P(\alpha^*) \cdot \alpha^* = 0$.
Therefore
$$\nu ||\nabla u_m||_{L^2}^2 = \langle f, u_m\rangle \leq ||f||_{V'} ||u_m||_V$$
but $||u_m||_V = ||\nabla u_m||_{L^2}$ so
$$\nu ||\nabla u_m||_{L^2} \leq ||f||_{V'}.$$
Thus there is a subsequence of $u_m$ which converges weakly in $V = W^{1,2}_{0,\sigma}(\Omega)$ to $u \in V$.
But since $\Omega$ is bounded, the embedding $W^{1,2} \to L^2$ is compact, so passing to a subsequence again we can get strong convergence.
Then
$$\nu(\nabla u, \nabla \psi_k) + (u \cdot \nabla u, \psi_k) = \langle f, \psi_k\rangle$$
but the $\psi_k$ form a basis so for every $\psi$,
$$\nu(\nabla u, \nabla \psi), (u \dot \nabla u, \psi) = \langle f, \psi\rangle.$$
Therefore $u$ is a weak solution.

By the weak convergence, we can pass to a subsequence of $u_m$ which attains the limit inferior and get
$$||\nabla u||_{L^2(\Omega)} \leq \liminf_{m \to \infty} ||\nabla u_m||_{L^2} \leq \nu^{-1} ||f||_{V'}$$
as desired.
\end{proof}

One could also prove the above result using Leray-Schauder fixed point theory.

\chapter{Weak solutions of the Navier-Stokes equations}
We now repeat the above theory but allow for time-dependence.

\section{Notation}
Let $\Omega$ be $\RR^d$ or a bounded open set with Lipschitz boundary and $f: \Omega \times [0, T]$ a force.
If $\Omega \neq \RR^d$ we add the no-slip boundary data $u|\partial \Omega = 0$.
We fix an initial data $u(0) = u_0$.
Let us show that there is a weak solution $u$ to the Navier-Stokes system
$$\partial_t u + u\cdot \nabla u + \nabla p = f$$
with the given data.

Recall that if $u$ is smooth, we have a kinetic energy balance
$$\frac{1}{2} \int_\Omega |u(x, t)|^2 ~dx + \nu \int_0^t \int_\Omega |\nabla u(x, s)|^2 ~dx ~ds = \frac{1}{2} \int_\Omega |u_0(x)|^2 ~dx + \int_0^t \int_\Omega f(x, s) \cdot u(x, s) ~dx ~ds.$$

Let $V$ be the closure of $C^\infty_{c,\sigma}(\RR^d)$ in the seminormed space with seminorm $||u||_V = ||\nabla u||_{L^2}$.
This is the seminorm of the homogeneous Sobolev space $\dot H^1$.
Thus we sometimes write $V = \dot H^1_{c,\sigma}$.

\begin{definition}
Assume $f \in L^2([0, T] \to V')$.
We say
$$u \in L^\infty([0, T] \to L^2(\Omega)) \cap L^2([0, T] \cap V) \cap C_w([0, T] \to L^2(\Omega))$$
is a \dfn{weak solution to the Navier-Stokes system} if $u(t) \to u_0$ in the weak topology of $L^2(\Omega)$ as $t \to 0$ and for every $\varphi \in C^\infty_{c,\sigma}(\Omega)$
$$\int_0^T \int_\Omega (u \cdot \nabla u) \cdot \varphi - u \cdot \partial_t \varphi + \nu \nabla u \nabla \varphi = \int_0^T \langle f(t), \varphi(t)\rangle ~dt$$
and for every $\varphi \in C^\infty_c(\Omega)$,
$$\int_\Omega u(x, t) \cdot \nabla\varphi(x) ~dx = 0.$$
If $u$ is a weak solution to the Navier-Stokes system and for every $t \in [0, T]$,
$$\frac{1}{2} \int_\Omega |u(x, t)|^2 ~dx + 2\nu \int_0^t \int_\Omega |\nabla u(x, s)|^2 ~dx ~ds \leq \frac{1}{2} \int_\Omega |u_0(x)|^2 ~dx + \int_0^t \langle f(t), u(t)\rangle ~dt,$$
we say that $u$ is a \dfn{Leray-Hopf solution to the Navier-Stokes system}.
\end{definition}

\section{Existence of Leray-Hopf solutions}
We now show that the Navier-Stokes system has a solution provided that the initial data has finite kinetic energy.
However, if $d = 3$ we will not be able to show that it is unique, as this is a major open problem.

This was proven by Leray when $\Omega = \RR^d$, and then Hopf proved it for bounded $\Omega$ using the Galerkin method.
Leray did not use the Galerkin method, but rather used asymptotics for the heat equation.
We will give Hopf's proof.

\begin{lemma}[Aubin-Lions-Simon]
Suppose $X_0 \to X_1 \to X_2$ are embeddings of Banach spaces.
Assume $X_1 \to X_2$ is bounded and $X_0 \to X_1$ is compact.
If $p, r \in [1, \infty]$, let $E_{p,r}$ be the space of $f \in L^p([0, T] \to X_0)$ such that $\partial_t f \in L^r([0, T] \to X_2)$.

If $p < \infty$ then the embedding $E_{p,r} \to L^p([0, T] \to X_1)$ is compact.

If $r > 1$ then the embedding $E_{\infty, r} \to C([0, T] \to X_1)$ is compact.
\end{lemma}

Aubin and Lions proved this lemma under the assumption that the $X_j$ were reflexive.
Simon showed this hypothesis was unnecessary.
We omit the proof.

\begin{lemma}
For every $u \in \dot H^1_{0,\sigma}(\Omega)$ and $v, w \in H^1(\Omega)$,
$$\int_\Omega (u \cdot \nabla v) \cdot w = -\int_\Omega (u \cdot \nabla w) \cdot \nabla v$$
and
$$\left|\int_\Omega (u \cdot \nabla w) \cdot v\right| \lesssim_d ||\nabla w||_{L^2} ||u||_{L^2}^{1-d/4} ||u||_{H^1}^{d/4} ||v||_{L^2}^{1 - d/4} ||v||^{d/4}_{H^1}.$$
\end{lemma}
\begin{proof}
First integrate by parts.
Now to get the estimate, we first bound
$$\left|\int_\Omega (u \cdot \nabla w) \cdot v\right| \leq ||\nabla w||_{L^2} ||u||_{L^4} ||v||_{L^4}.$$
Now use Gagliardo-Nirenberg interpolation
$$||u||_{L^4} \lesssim_d ||u||_{L^2}^{1-d/4}||u||_{H^1}^{d/4}$$
to get the claim.
\end{proof}

For the ease of presentation we will assume that $\Omega$ is bounded.
The argument can be extended to the case $\Omega = \RR^d$ without too much work but in that case, $\dot H^1$ is not a subspace of $L^2$, and one needs to do some not very interesting work to overcome this issue, involving interpreting $\dot H^1$ as a quotient modulo constant functions so it becomes a Hausdorff space.

\begin{lemma}
Let $X$ be a dense subspace of $L^2(\Omega)$.
If $v \in L^\infty([0, T] \to L^2)$ and $\partial_t v \in L^1([0, T] \to X')$ then there is
$$\tilde v \in C_w([0, T] \to L^2) \cap C([0, T] \to X')$$
such that $v = \tilde v$ for almost all time.
\end{lemma}
We omit the proof, as this result is standard.
Henceforth we identify $v$ with $\tilde v$ since they are almost surely equal.

\begin{theorem}[Leray 1934, Hopf 1951]
For every $u_0 \in L^2_\sigma(\Omega)$ and $f \in L^2([0, T] \to V')$, and $0 < T < \infty$, there is a Leray-Hopf solution $u$ to the Navier-Stokes system with $u(0) = u_0$ and forcing $f$.
Moreover, $\partial_t u \in L^{4/d}([0, T] \to V')$.
\end{theorem}
\begin{proof}
Let $\psi_k \in C^\infty_{c,\sigma}(\Omega)$ be a basis of $V$ which is orthonormal in $L^2$.
We will find approximate solutions $u_m(t) = \sum_{j=1}^m g_m^j(t) \psi_j$ which satisfy
$$u_m(0) = \sum_{j=1}^m (u_0, \psi_j)\psi_j$$
and $u_m$ solves the approximate Navier-Stokes system
\begin{equation}
\label{Galerkin NS ODE}
\int_\Omega \partial_t u_m \cdot \psi_k + \nu \nabla u_m \nabla \psi_k + (u_m \cdot \nabla u_m) \cdot \psi_k = \langle f(t), \psi_k\rangle
\end{equation}
whenever $t \in [0, T]$ and $1 \leq k \leq m$.

The system (\ref{Galerkin NS ODE}) is an ordinary differential system for $g_m$.
Indeed, it can be expressed as
$$(g^k_m)' + \nu \sum_{j=1}^m g_m^j \int_\Omega \nabla \psi_m \nabla \psi_k(x) ~dx + \sum_{j,\ell=1}^m g_m^j g_m^\ell \int_\Omega (\psi_j \cdot \nabla \psi_\ell) \cdot \psi_k(x) ~dx = \langle f, \psi_k\rangle$$
which is of the form
$$g_m' + N(g) = F$$
where $N$ is a quadratic nonlinearity and $F \in L^2$ (since $f \in L^2$).
The initial data is
$$g_m(0) = (u_0, \psi).$$
Since $N$ is locally Lipschitz, by t  he Cauchy-Lipschitz theorem, there is a unique maximal solution $g_m \in C([0, T_*))$ where $T_* \leq T$.
We now control the blowup time $T_*$.

If $T_* < T$, then
$$\sup_{t < T_*} ||g_m(t)||_{\ell^2} = \infty$$
since otherwise we could continue $g_m$ over $T_*$ by the Cauchy-Lipschitz theorem.
But if we multiply (\ref{Galerkin NS ODE}) by $g_m^k$ and sum over $k$,
we get
$$\int_\Omega \partial_t u_m \cdot u_m + (u_m \cdot \nabla u_m) \cdot u_m + \nu \nabla u_m \nabla u_m = \langle f, u_m\rangle.$$
Therefore
$$\frac{\partial_t}{2} \int_\Omega |u_m(x, t)|^2 ~dx + \nu \int_\Omega |\nabla u_m(x, t)|^2 ~dx = \langle f, u_m\rangle.$$
So
\begin{align*}
\frac{1}{2} ||u_m(t)||_{L^2}^2 + \nu\int_0^t ||\nabla u_m||_{L^2}^2 &\leq \frac{1}{2} ||u_m(0)||_{L^2}^2 = \int_0^t ||f||_{V'} ||u_m||_V\\
&\leq \frac{1}{2} ||u_m(0)||_{L^2}^2 + \frac{\nu}{2} \int_0^t ||u_m||_V^2 + \frac{1}{2\nu} \int_0^t ||f||_{V'}^2
\end{align*}
by Young's inequality.
Consequently,
$$||u_m(t)||_{L^2}^2 + \nu\int_0^t ||\nabla u_m||_{L^2}^2 \leq ||u_0||_{L^2}^2 + \nu^{-1} \int_0^t ||f||_{V'}^2$$
so it follows that $||g_m||_{\ell^2} \lesssim 1$.
This is a contradiction.

So $T_* = T$ and
$$||u_m||_{L^\infty([0, T] \to L^2)} + ||\nabla u_m||_{L^2([0, T] \to L^2)} \lesssim 1$$
uniformly in $m$.
Therefore there is a $u \in L^2([0, T] \to L^2_\sigma) \cap L^2([0, T] \to V)$ such that $u_m \to u$ in the weakstar topology of $L^\infty([0, T] \to L^2)$ and $u_m \to L^2([0, T] \to \dot H^1)$.
(Note that weak convergence with codomain $V$ is weaker than weak convergence with codomain $\dot H^1$).

Now we show that $u$ is a weak solution. Let $\theta \in C^\infty_c((0, T))$ and let $\varphi_k(x, t) = \psi_k(x) \theta(t)$.
Then $\varphi_k$ is a spacetime test function.
Integrating the approximate equation against $\varphi_k$ we get
$$\int_0^T \int_\Omega \partial_t u_m \cdot \varphi_k + (u_m \cdot \nabla u_m) \cdot \varphi_k + \nu \nabla u_m \nabla \varphi_k = \int_0^T \langle f(t), \varphi_k(t) ~dt.$$
The right-hand side does not depend on $m$.
Integrating by parts in $t$ with $k$ fixed and taking the limit as $m \to \infty$,
$$\int_0^T \int_\Omega \partial_t u_m \cdot \varphi_k = -\int_0^T \int_\Omega u_m \cdot \partial_t \varphi_k \to -\int_0^T \int_\Omega u \cdot \partial_t \varphi_k$$
owing to the weak convergence of $u_m$ in $L^2([0, T] \to L^2)$.
The same argument works for $\nabla u_m$.

Finally we show
$$\lim_{m \to \infty} \int_0^T \int_\Omega (u_m \cdot \nabla u_m) \cdot \varphi_k = \int_0^T \int_\Omega (u \cdot \nabla u) \cdot \varphi_k.$$
We know $\nabla u_m \to \nabla u$ weakly, but in order to conclude $u_m \cdot \nabla u_m \to u \cdot \nabla u$ we need $u_m \to u$ strongly.
This is a problem because we only have compactness of $u_m$ via the Rellich-Kondrachov theorem in space, not time.
Instead we show $u_m \to u$ strongly in $L^2([0, T] \to L^2(\Omega_k))$ where $\Omega_k$ is the support of $\psi_k$.
This follows from the Aubin-Lions-Simon lemma with $X_0 = \dot H^1_0(\Omega_k)$ and $X_1 = L^2(\Omega_k)$, as we now show.

The embedding $X_0 \to X_1$ is compact by the Rellich-Kondrachov theorem.
To compute $X_2$ we bound $\partial_t u_m$.
Since $u_m$ is an approximate solution,
$$\int_0^T \langle \partial_t u_m + u_m \cdot \nabla u_m, \nu \Delta u_m - f, \varphi_k\rangle = 0.$$
Since the span of the $\varphi_k$ is all of $L^2([0, T] \to V)$,
$$\partial_t u_m + u_m \cdot \nabla u_m = \nu \Delta u_m + f$$
in the space $L^2([0, T] \to V')$.
Since $u_m \in L^2([0, T] \to V)$, $\Delta u_m \in L^2([0, T] \to V')$.
If $\varphi \in L^{4/(4-d)}([0, T] \to V)$ then
$$\left|\int_0^T \int_\Omega (u_m \cdot \nabla u_m) \cdot \varphi\right| \lesssim \int_0^T ||\nabla u||_{L^2}^{2-d/2} ||u_m||_{H^1}^{d/2}.$$
Now $(4-d)/4+1/\infty+d/4=1$ so we get a bound
$$ \int_0^T ||\nabla u||_{L^2}^{2-d/2} ||u_m||_{H^1}^{d/2} \lesssim ||\nabla \varphi||_{L^{4/4-d}([0, T] \to L^2)} ||u_m||_{L^\infty([0, T] \to L^2)^{2-d/2}} ||u_m||_{L^2([0, T] \to H^1)} \lesssim ||\varphi||_{L^{4/(4-d)([0, T] \to V)}}.$$
Since $4/(4-d) < 2$, we get
$$||u_m \cdot \nabla u_m||_{L^{4/d}([0, T] \to V')} \lesssim 1$$
so
$$||\partial_t u_m||_{L^{4/d}([0, T] \to V')} \lesssim 1$$
uniformly in $m$.
Thus we take $X_2 = V'$ with $r = 4/d$ and $p = 2$.

By the Aubin-Lions-Simon lemma, after passing to a subsequence $u_m \to u$ in the strong topology of $L^2([0, T] \to L^2(\Omega_k))$.
Summarizing, for every $k$,
$$\int_0^T \int_\Omega -u \cdot \partial_t \varphi_k + (u \cdot \nabla u) \cdot \varphi_k + \nu \nabla u \nabla \varphi_k = \int_0^T \langle f, \varphi_k\rangle.$$
Since the $\varphi_k$ are dense by the Stone-Weierstrass theorem, we conclude that $u$ is a weak solution of the Navier-Stokes equation.

To check the initial conditions we will assume that $\Omega$ is bounded.
Then $\partial \Omega$ is a Lipschitz manifold and $V = H^1_{0,\sigma}(\Omega)$.

We know that
$$u \in L^\infty([0, T] \to L^2) \cap L^2([0, T] \cap V) \cap \dot W^{1,4/d}([0, T] \to V').$$
Now we prove $u \in C_w([0, T] \to L^2_\sigma)$.
We apply the third lemma with $X = V$ to get
$$u \in C_w([0, T] \to L^2) \subseteq C_w([0, T] \to L^2_\sigma).$$

Now we need to show that $u(t)$ converges weakly to $u_0$ as $t \to 0$.
First since $||u_m||_{L^\infty([0, T] \to L^2)} \lesssim 1$ and $||\partial_t u_m||_{L^{4/d}([0, T] \to L^2)} \lesssim 1$, the Aubin-Lions-Simon lemma with $X_0 = L^2$, $X_1 = X_2 = V'$ applies.
Indeed, $V \to L^2$ is a compact embedding, so its adjoint $L^2 \to V'$ is also a compact embedding.
Therefore $u_m \to u$ strongly in $C([0, T] \to V')$.
Therefore $u_m(0) \to u(0)$ in $V'$.
But
$$u_m(0) = \sum_{j=1}^m (u_0, \psi_j)\psi_j \to u_0$$
in $L^2$ so $u(0) = u_0$ since $L^2$ is Hausdorff.

Next we check that $u$ is a Leray-Hopf solution, thus $u$ satisfies the energy estimate
$$\frac{1}{2} ||u(t)||_{L^2}^2 + \nu\int_0^t ||\nabla u(s)||_{L^2}^2 ~ds \leq \frac{1}{2} ||u_0||_{L^2}^2 + \int_0^t \langle f(s), u(s)\rangle ~ds.$$
Recall that
$$\frac{1}{2} ||u_m(t)||_{L^2}^2 + \nu\int_0^t ||\nabla u_m(s)||_{L^2}^2 ~ds = \frac{1}{2} ||u_m(0)||_{L^2}^2 + \int_0^t \langle f(s), u_m(s)\rangle ~ds.$$
Taking the limit as $m \to \infty$ we conclude
$$\lim_{m \to \infty} ||u_m(0)||_{L^2}^2 = ||u_0||_{L^2}^2.$$
Using the weak convergence of $u_m$ in $L^2([0, T] \to L^2)$ and $V$ respectively, we compute the dissipation
$$\lim_{m \to \infty} \int_0^t \langle f(s), u_m(s)\rangle ~ds = \int_0^t \langle f(s), u(s)\rangle ~ds$$
and forcing
$$\liminf_{m \to \infty} \int_0^t \int_\Omega |\nabla u_m(x, s)|^2 ~dx ~ds \geq \int_0^t \int_\Omega |\nabla u(x, s)|^2 ~dx ~ds$$
where we used Fatou's lemma.
So it remains to show
\begin{equation}
\label{energy liminf}
||u(t)||_{L^2}^2 \leq \liminf_{m \to \infty} ||u_m(t)||_{L^2}^2.
\end{equation}
Unfortunately this does NOT follow from the weakstar convergence in $L^\infty([0, T] \to L^2)$, since we would need to take a limit of a Dirac sequence in $L^1$ to get this time-pointwise bound, but there is no reason that limit would converge.

To prove (\ref{energy liminf}), let $\alpha \in C^\infty_c([0, T])$. Multiplying the approximate energy estimtate by $\alpha(t)^2$ and integrating in $t$ we get
$$\int_0^T \alpha(t)^2 ||u_m(t)||_{L^2}^2 ~dt + \nu \int_0^T \int_0^t ||\nabla u_m(s)||_{L^2}^2 ~ds ~dt = ||u_m(0)||_{L^2}^2 \int_0^T \alpha(t)^2 ~dt + \int_0^T \alpha(t)^2 \int_0^t \langle f(s), u_m(s)\rangle ~ds~dt.$$
By Fatou's lemma and the weak convergence,
\begin{align*}
\liminf_{m \to \infty} \int_0^T \alpha(t)^2 \int_0^t \int_\Omega |\nabla u_m(x, s)|^2 ~dx ~ds~dt &\geq \int_0^T \alpha(t)^2 \liminf_{m \to \infty}\int_0^t \int_\Omega |\nabla u_m(x, s)|^2 ~dx ~ds~dt\\
&\geq \int_0^T \alpha(t)^2 \int_0^t \int_\Omega |\nabla(x, s)|^2 ~dx ~ds ~dt.
\end{align*}
Since $u_m \to u$ in $L^2$,
$$\lim_{m \to \infty} ||u_m(0)||_{L^2}^2 \int_0^T \alpha(t)^2 ~dt = ||u_0||_{L^2}^2 \int_0^T \alpha(t)^2 ~dt$$
By dominated convergence,
$$\lim_{m \to \infty} \int_0^T \alpha(t)^2 \int_0^t \langle f(s), u_m(s)\rangle ~ds ~dt = \int_0^T \alpha(t)^2 \int_0^t \langle f(s), u(s)\rangle ~ds ~dt.$$
Moreover,
$$\int_0^T \alpha(t)^2 ||u(t)||_{L^2}^2 = ||\alpha u||_{L^2([0, T] \to L^2)} \leq \liminf_{m \to \infty} \int_0^T \alpha(t)^2 ||u_m(t)||_{L^2}^2 ~dt.$$
Therefore
$$\int_0^T \alpha(t)^2 ||u(t)||_{L^2}^2 ~dt + \nu\int_0^T \int_0^t ||\nabla u(s)||_{L^2} ~ds ~dt \leq ||u_0||_{L^2}^2 \int_0^T \alpha(t)^2 ~dt + \int_0^T \int_0^t \langle f(s), u(s)\rangle ~ds ~dt.$$

Now we choose a suitable $\alpha$. Let $t_0 \in (0, t)$. Let $\theta \in C^\infty_c((0, 1))$ satisfy
$$\int_0^\infty \theta(t)^2 ~dt = 1.$$
Let
$$\alpha_\delta(t) = \delta^{-1/2} \theta\left(\frac{t - t_0}{\delta}\right).$$
If $\delta$ is small enough depending on $t_0$ and $T$, then $\alpha_\delta \in C^\infty_c((0, T))$ and
$$\int_0^T \alpha_\delta(t)^2 ~dt = 1.$$
So $\alpha_\delta^2$ is an approximation to the Dirac measure at $t_0$ in the sense that if $g \in L^1_{loc}(\RR)$ and $t_0$ is a Lebesgue point of $g$, then
$$\lim_{\delta \to 0} \int_0^T \alpha_\delta(t)^2 g(t) ~dt = g(t_0).$$
Applying to
$$g(t) = \int_0^t \langle f(s), u(s)\rangle ~ds$$
and
$$g(t) = \int_0^t \int_\Omega |\nabla u(x, s)|^2 ~dx ~ds$$
we get
$$\lim_{\delta \to 0} \int_0^T \int_0^t \langle f(s), u(s)\rangle ~ds ~dt = \int_0^{t_0} \langle f(s), u(s)\rangle ~ds$$
and
$$\lim_{\delta \to 0} \int_0^T \int_0^t ||\nabla u(s)||_{L^2}^2 ~ds ~dt = \int_0^{t_0} \langle ||\nabla u(s)||_{L^2}^2 ~ds.$$
We get (\ref{energy liminf}) whenever $t_0$ is a Lebesgue point of the remaining term, which happens almost surely if a point of $[0, T]$ is drawn at random.
Now suppose that $t_0$ is not a Lebesgue point and let $t_n \to t_0$.
Then since $u \in C_w([0, T] \to L^2_\sigma)$ we get
$$||u(t_0)||_{L^2} \leq \liminf_{n \to \infty} ||u(t_n)||_{L^2}$$
which proves (\ref{energy liminf}) for general $t_0 \in (0, T)$.
Now taking the limit as $t_n \to 0$ and $t_n \to T$ we eliminate the last two cases.
\end{proof}

\begin{proposition}
If $f \in L^2([0, T] \to V)$ and $u$ is a weak solution on $[0, T]$ then $\partial_t u \in L^{4/d}([0, T] \to V')$.
\end{proposition}
\begin{proof}
One can check
$$\left|\int_0^T \int_\Omega -u\partial_t \varphi\right| = \left|\int_0^T \int_\Omega -(u \cdot \nabla u) \cdot \varphi + \nu \nabla u \nabla \varphi - \langle f, \varphi\rangle\right| \lesssim ||\varphi||_{L^{4/(4-d);V}}.$$
Therefore for every test function $\varphi$,
$$\langle \partial_t u, \varphi\rangle \lesssim ||\varphi||_{L^{4/(4-d);V}}$$
and since the space of test functions is dense in $L^{4/(4-d)};V$ we can extend $\partial_t u \in L^{4/d}V'$.
\end{proof}

\section{Properties of Leray-Hopf solutions, $d = 2$}
Assume $d = 2$ and let $u$ be a Leray-Hopf solution.
Notice $4/d = 2$.

\begin{lemma}
Let $X \subseteq L^2$ be a dense subspace.
If $v \in L^2([0, T] \to X)$ and $\partial_t v \in L^2([0, T] \to X')$ then there is $\tilde v \in C([0, T] \to L^2)$ such that $v = \tilde v$ almost everywhere.
\end{lemma}
We again omit the proof.

\begin{theorem}
One has $u \in C([0, T] \to L^2_\sigma)$.
\end{theorem}
\begin{proof}
Since $4/d = 2$ then $u \in L^2([0, T] \to V)$ and $\partial_t u \in L^2([0, T] \to V')$, we can apply the previous lemma.
\end{proof}

\begin{lemma}[Lions-Magenes]
Let $X \subseteq L^2$ be a dense subspace.
Let $p, q \in [1, \infty]$, $p',q'$ the H\"older duals to $p,q$ respectively, and
$$E_{p,q'} = \langle f \in L^p([0, T] \to X): \partial_t f \in L^{q'}([0, T] \to X')\}.$$
Assume $v \in E_{p,q'}$ and $w \in E_{p',q}$.
Then $t \mapsto (v(t), w(t))_{L^2}$ is a continuous map $[0, T] \to \CC$ and satisfies the fundamental theorem of calculus in the sense that
$$(v(t_2), w(t_2))_{L^2} - (v(t_1), w(t_1))_{L^2} = \int_{t_1}^t \langle \partial_t v(t), w(t)\rangle + \langle v(t), \partial_t w(t)\rangle ~dt.$$
\end{lemma}
We again omit the proof.

\begin{theorem}
We have a strong energy EQUALITY: For every $t_0, t_1 \in [0, T]$,
$$\frac{1}{2} ||u(t_1)||_{L^2}^2 + \nu\int_{t_0}^{t_1} ||\nabla u(s)||_{L^2}^2 ~ds = \frac{1}{2} ||u(t_0)||_{L^2}^2 + \int_{t_0}^{t_1} \langle f(s), u(s)\rangle ~ds.$$
\end{theorem}
\begin{proof}
Since $\partial_t u \in L^2([0, T] \to V')$ (which happens since $d = 2$) the weak equation is
$$-\int_0^T \langle \partial_t u, \varphi\rangle + \nu(\nabla u, \nabla \varphi) + \langle u \cdot \nabla u, \varphi\rangle = \int_0^T \langle f(t), \varphi(t)\rangle ~dt.$$
This holds if $\varphi \in C^\infty_{c,\sigma}(\Omega \times (0, T))$, but makes sense even if $\varphi \in L^2(\Omega \times (0, T))$ and by density extends to that case.
Thus we may take $\varphi = u1_{[t_0, t_1]}$.
Therefore
$$\int_{t_0}^{t_1} -\langle \partial_t u, u\rangle + \nu ||\nabla u||_{L^2}^2 + \langle u \cdot \nabla u, u\rangle = \int_{t_0}^{t_1} \langle f, u\rangle.$$
But
$$\langle u \cdot \nabla u, u\rangle = \int_\Omega (u \cdot \nabla u) \cdot u = 0$$
since we have enough regularity to integrate by parts.
The theorem now follows from the fact that
$$\int_{t_0}^{t_1} \langle \partial_t u, u\rangle = \frac{(u(t_1), u(t_1)) - (u(t_0), u(t_0))}{2}$$
which is a consequence of the Lions-Magenes lemma with $v = w = u$.
\end{proof}

Assume $u \in V$ and $v, w \in H^1_0(\Omega)$.
Then we can define a cubic form
$$b(u, v, w) = \int_\Omega (u \dot \nabla v) \cdot w.$$
We already showed that $b(u, v, w) = -b(u, w, v)$ and
$$|b(u, v, w)| = |b(u, w, v)| \lesssim ||\nabla w||_{L^2} ||u||_{L^2}^{d/4} ||\nabla u||_{L^2}^{1-d/4} ||v||_{L^2} ||\nabla v||_{L^2}.$$
Therefore $u \cdot \nabla v \in H^{-1}$, since that space is the dual of $H^1_0$, and in fact $H^{-1} \subseteq V'$.
The duality pairing here is
$$\langle u \cdot \nabla v, w\rangle = b(u, v, w).$$
Thus the preceding estimates are given by
$$||u \cdot \nabla v||_{V'} \leq ||u \cdot \nabla v||_{H^{-1}} \leq ||u||_{L^2}^{d/4} ||\nabla u||_{L^2}^{1-d/4} ||v||_{L^2}^{d/4} ||\nabla v||_{L^2}^{1-d/4}.$$

\begin{theorem}
Let $d = 2$.
Then the Leray-Hopf weak solution $u \in \dot H^1([0, T] \to V')$ is unique.
\end{theorem}
\begin{proof}
The proof of the Leray-Hopf solution says that
$$\int_0^T \langle \partial_t u, \varphi\rangle + \langle u \dot \nabla u, \varphi\rangle + \nu(\nabla u, \nabla \varphi) = \int_0^T \langle f, \varphi\rangle$$
whenever $f \in L^2([0, T] \to V)$.
This makes sense because $\partial_t \in L^2([0, T] \to V')$ by hypothesis.

Let $u_1, u_2$ be Leray-Hopf solutions with the same initial data. Then $w = u_1 - u_2$ satisfies $w(0) = 0$ and
$$\int_0^T \langle \partial_t w, \varphi\rangle + \langle w \cdot \nabla u, \varphi\rangle + \langle v \cdot \nabla w, \varphi\rangle + \nu(\nabla w, \nabla \varphi) = 0$$
so if $t \in (0, T]$ and $\varphi = w1_{(0, t)}$,
$$\int_0^T \langle \partial_t w, w\rangle + b(w, u, w) + b(v, w, w) + \nu||\nabla w||_{L^2}^2 = 0$$
but $b$ is antisymmetric in the final variables, so $b(v, w, w) = 0$.
Thus
\begin{align*}
\frac{1}{2}||w(t)||_{L^2}^2 - \frac{1}{2} ||w(0)||_{L^2}^2 + \nu\int_0^t ||\nabla w||_{L^2}^2 &= -\int_0^t b(w, u, w)\\
&\lesssim \int_0^t ||\nabla u||_{L^2} ||w||_{L^2} ||\nabla w||_{L^2}.
\end{align*}
Taking Cauchy's inequality,
$$\int_0^t ||\nabla u||_{L^2} ||w||_{L^2} ||\nabla w||_{L^2} \leq \frac{\nu}{2} \int_0^t ||\nabla w||_{L^2}^2 + \frac{1}{\nu} \int_0^t ||w||_{L^2}^2 ||\nabla u||_{L^2}^2.$$
By the Cauchy-Schwarz inequality,
$$\frac{1}{2}||w(t)||_{L^2}^2 + \frac{\nu}{2} \int_0^t ||\nabla w||_{L^2}^2 \leq \frac{1}{2} ||w(0)||_{L^2}^2 + \frac{C}{\nu} \int_0^t ||\nabla u||_{L^2}^2 ||w||_{L^2}^2.$$
We can discard the second left-hand term and get
$$\frac{1}{2}||w(t)||_{L^2}^2 \leq \frac{1}{2} ||w(0)||_{L^2}^2 + \frac{C}{\nu} \int_0^t ||\nabla u||_{L^2}^2 ||w||_{L^2}^2.$$
So by Gr\"onwall's inequality,
$$||w(t)||_{L^2}^2 \leq ||w(0)||_{L^2}^2 \exp\left(\frac{C}{\nu} \int_0^t ||\nabla u(s)||_{L^2}^2 ~ds\right) = 0$$
since $w(0) = 0$.
\end{proof}

\section{Weak solutions with $d = 3$}
The trouble with the case $d = 3$ is that we get $\varphi \in L^4([0, T] \to V)$ which is too weak.

Here is a heuristic argument for why the Navier-Stokes equation
$$\partial_t u + u \cdot \nabla u + \nabla p = \nu \Delta u$$
is badly behaved with $\Omega = \RR^3$.
If $u$ is a solution with pressure $p$, then for every $\lambda > 0$,
$$u_\lambda(x, t) = \lambda u(\lambda x, \lambda t)$$
is also a solution with pressure
$$p_\lambda(x, t) = \lambda^2 p(\lambda x, \lambda^2 t).$$
For that scaling, $L^\infty([0, T] \to L^d)$, $L^\infty([0, T] \to \dot H^{d/2 - 1})$, $L^2([0, T] \to \dot H^{d/2})$, and $L^4([0, T] \to \dot H^{(d-1)/2})$ are scaling-invariant seminorms.

In particular, if $d = 2$ then $L^\infty([0, T] \to L^2)$ is scaling-invariant.
The uniform $L^2$ bound from the energy inequality gives control of the $L^2$ norm for all time, so in $d = 2$ the Navier-Stokes equations are critical for $L^\infty([0, T] \to L^2)$.
But this is not true for any of the scaling-invariant seminorms if $d = 3$, so the Navier-Stokes are supercritical in all such seminorms.
This is why Leray-Hopf uniqueness is open if $d = 3$.

\begin{theorem}[Buckmaster-Vicol 2018]
There is small $\beta > 0$ such that for any smooth $\rho: [0, T] \to [0, \infty)$ there is a weak solution $u \in C([0, T] \to H^\beta(\Torus^3))$ such that for every $t \in [0, T]$,
$$\frac{1}{2} ||u(t)||_{L^2} = \rho(t).$$
\end{theorem}

This gives nonuniqueness on $\Torus^3$, but it is not Leray-Hopf uniqueness.
Indeed, let $\rho = 0$ on $[0, T/3]$ and $\rho \gtrsim 1$ on $[T/2, T]$.
Then there is a weak solution $u$ with $u(0) = 0$ and
$$\frac{1}{2} ||u(t)||_{L^2} = \rho(t)$$
so having zero initial data does not imply that energy is eventually conserved.
However such a solution exhibits ``spontaneous creation of energy" and so is nonphysical.

Let $\Omega$ be a bounded Lipschitz connected open set in $\RR^d$, $2 \leq d \leq 3$.
(The important hypothesis is ``connected!")
Let $\psi \in V$ and $\theta \in C^\infty_c((0, T))$ be test functions and $f \in L^2([0, T] \to V')$ be a forcing term.
Let $\varphi(x, t) = \psi(x) \theta(t)$, so
$$\int_0^T \theta(t) \left(\langle \partial_t u, \psi\rangle + b(u, u, \psi) + \nu(\nabla u, \nabla \psi) - \langle f,\psi\rangle\right) = 0.$$
Since $\theta$ is arbitrary,
$$\langle \partial_t u + u \cdot \nabla u - \nu \Delta u - f, \psi\rangle = 0$$
Integrating in time,
$$(u(t), \psi) - (u(0), \psi) + \left\langle \int_0^t u \cdot \nabla u - \nu \Delta u - f, \psi\right\rangle = 0$$
for almost every $t \in [0, T]$.
Thus if
$$G(t) = u(t) - u(0) + \int_0^t u \cdot \nabla u - \nu \Delta u - f$$
then $\langle G(t), \psi\rangle = 0$ for almost every $t$.

Assume $f \in L^2([0, T] \to H^{-1})$. Then we can replace the pairing $\langle G(t), \psi\rangle = 0$ in $(V', V)$ with the same pairing in $(H^{-1}, H^1_0)$.
Using weak continuity of $u$ we can then show $G(t) \in H^{-1}$ for every $t$.
Using de Rham cohomology it follows that there is a unique $\Pi(t) \in L^2_0(\Omega)$ with zero mean such that $G(t) = -\nabla \Pi(t)$.
We claim that $\Pi \in C_w([0, T] \to L^2)$.
Indeed, if $g \in L^2(\Omega)$, the below lemma implies there is $h \in H^1_0(\Omega)^d$ such that
$$\nabla \cdot h = g - \int_\Omega g.$$
Since $\Pi$ has zero mean,
$$(\Pi(t), g) = (\Pi(t), \nabla \cdot h) = -\langle \nabla \Pi(t), h\rangle_{H^{-1},H^1_0}.$$
Since $(\Pi(t), g)$ is continuous in $t$, we conclude $\Pi \in C_w([0, T] \to L^2)$ and hence $\Pi \in L^\infty([0, T] \to L^2)$.
Now let
$$p = \partial_t \Pi \in W^{-1,\infty}([0, T] \to L^2_0)$$
where $W^{-1, \infty}$ is the dual of $W^{1,1}_0$. Thus $G = -\nabla \Pi$ and hence
$$u(t) - u(0) = \int_0^t (u \cdot \nabla u - \nu \Delta u - f) + \nabla \Pi = 0$$
in $H^{-1}$.
Testin that equation against $\partial_t\varphi$ where $\varphi \in C^\infty_c((0, T) \times \Omega)$, we can integrate by parts, to get
$$\langle \partial_t u + u \cdot \nabla u - \nu \Delta u - f + \nabla \partial_t \Pi, \varphi\rangle = 0$$
where the dual pairing is between the test space and distributions on spacetime.
But $\partial_t \Pi = p$, so in the distribution space,
$$\partial_t u + u \cdot \nabla u + \nabla p = \nu \Delta u + f.$$
Summarizing:

\begin{proposition}
Let $f \in L^2([0, T] \to H^{-1})$ and $\Omega$ bounded, Lipschitz, connected open set in $\RR^d$.
Then the Leray-Hopf weak solution with forcing $f$ satisfies the Navier-Stokes equation in the sense of distributions, with the pressure $p \in W^{-1,\infty}([0, T] \to L^2_0)$.
\end{proposition}

We used the following lemma:

\begin{lemma}
There is a bounded linear operator $\Pi: L^2_0(\Omega) \to H^1_0(\Omega)^d$ such that for every $q \in L^2_0(\Omega)$, $v = \Pi q$ satisfies $\nabla \cdot v = q$.
\end{lemma}

We omit the proof, which is in the book of Boyer and Fabrie.

\chapter{Fixed-point iteration and Navier-Stokes}
When $d = 3$, $L^\infty_t,L^3_x,L^\infty_t\dot H^{1/2}$ are scaling-invariant for Navier-Stokes.
So we will use fixed-point iteration to construct Navier-Stokes solutions in these spaces.
Still following lectures of Huy Quang Nguyen, who is in turn following the book on ``Fourier analysis and nonlinear PDE" by Bahouri, Chemin, and Danchia.

\section{Notation}
As usual we let $\dot H^s$ be the space of tempered distributions $f$ such that $\hat f \in L^1_{loc}$ and
$$||f||_{\dot H^s}^2 = \int_{\RR^d} |\hat f(\xi)|^2 |\xi|^{2s} ~d\xi$$
is finite.

\begin{theorem}
If $s \in (0, d/2)$, then the space $\Sch_0(\RR^d)$ of Schwartz functions whose Fourier transform vanishes near $0$ is dense in $\dot H^s(\RR^d)$.
Moreover, $\dot H^s(\RR^d)$ is a Hilbert space iff $s < d/2$, and if $|s| < d/2$ then $(\dot H^s)' = \dot H^{-s}$.
Finally if $s \in (0, d/2)$ then
$$\dot H^s(\RR^d) \subseteq L^{\frac{2d}{d - 2s}}(\RR^d).$$
\end{theorem}

If $f_0$ is a tempered distribution, let $e^{\nu t\Delta} f_0$ be the solution of the heat equation with initial data $f_0$ at time $t$ with diffusivity $\nu > 0$.
Thus $e^{\nu t \Delta}$ is the heat propagator.

Write $(v \otimes w)_{ij} = v_iw_j$ for vector fields $v,w$.
In particular
$$(\nabla \cdot (v \otimes w))_i = \sum_{j=1}^d \partial_j (v_i w_j).$$

Let us write $L^p_TX$ to mean $L^p([0, T] \to X)$.

\section{Fixed-point iteration methods}
Consider the unforced Navier-Stokes equation
$$\partial_t u + \PP(u \cdot \nabla u) = \nu \Delta u$$
where $u$ is divergence-free. Our goal is to treat this equation as a nonlinear perturbation of the heat equation.

If $v, w$ are vector fields on $\RR^d$, define the symmetric bilinear map
$$Q(v, w) = -\frac{1}{2}\PP(\nabla \cdot (v \otimes w) + \nabla \cdot(w \otimes v)).$$
If $v,w$ are divergence-free then in particular
$$Q(v, w) = -\frac{1}{2} \PP(w \cdot \nabla v + v \cdot \nabla w).$$
So if $\nabla \cdot u = 0$ then $Q(u, u) = -\PP(u \cdot \nabla u)$.
The point of defining $Q$ more general was so that we could see that it is symmetric.

Let $B(u, u) = g$ define the unique solution of the inhomogeneous heat equation
$$\partial_t g = \nu \Delta g + Q(u, u)$$
with $g(0) = 0$.
Then if $u$ is a solution of the unforced Navier-Stokes equation,
$$u(t) = e^{\nu t \Delta} u_0 + B(u, u).$$
Also if $u$ is a fixed point of $e^{\nu t \Delta} u + B(u ,u)$ and $\nabla \cdot u_0 = 0$, then $\nabla \cdot u = 0$.
Indeed, $\nabla \cdot B(u, u) = 0$.

Using Duhamel's formula for the heat propagator,
$$B(u(t), u(t)) = \int_0^t e^{\nu(t - t')\Delta}Q(u(t'), u(t')) ~dt'.$$
That is, the fixed-point equation is
$$u(t) = e^{\nu t\Delta} u_0 + \int_0^t e^{\nu(t - t')\Delta}Q(u(t'), u(t')) ~dt'.$$
To find fixed points we appeal to the following Picard-type theorem:

\begin{proposition}
Let $E$ be a Banach space, $\mathscr B: E \otimes E \to E$ a bounded linear map, and
$$0 < \alpha < \frac{1}{4||\mathscr B||}.$$
Then for every $x_0 \in B_E(0, \alpha)$, there is a unique fixed point
$$x_* = x_0 + \mathscr B(x_*, x_*)$$
such that $x_* \in B_E(0, 2\alpha)$.
\end{proposition}
\begin{proof}
Consider the Picard iteration
$$x_{n+1} = x_0 + \mathscr B(x_n, x_n).$$
Then $x$ is a Cauchy sequence.
\end{proof}

Thus we must find a Banach space $E$ which satisfies the above hypothesis with $\mathscr B = B$.

\section{$\dot H^{1/2}$}
We first apply this to $E = \dot H^{1/2}(\RR^3)$.

We can prove local well-posedness and stability for any data, and global well-posedness for data which is small depending on the viscosity.

\begin{lemma}
There is $C > 0$ such that
$$||Q(a, b)||_{\dot H^{-1/2}} \lesssim ||a||_{\dot H^1} ||b||_{\dot H^1}.$$
\end{lemma}
\begin{proof}
We get
\begin{align*}
||Q(a, b)||_{\dot H^{-1/2}} &\lesssim ||\nabla \cdot (a \otimes b)||_{H^{-1/2}} + ||\nabla \cdot(b \otimes a)||_{H^{-1/2}}\\
&\lesssim \sup_{i,j} ||a_i \nabla b_j||_{\dot H^{-1/2}} + ||b_j \nabla a_i||_{\dot H^{-1/2}}.
\end{align*}
Furthermore we have an embedding $\dot H^{1/2} \to L^3$ which dualizes to an embedding $L^{3/2} \to \dot H^{-1/2}$.
So
\begin{align*}
||Q(a, b)||_{\dot H^{-1/2}} &\lesssim \sup_{i,j} ||a_i \nabla b_j||_{L^{3/2}} + ||b_i \nabla a_j||_{L^{3/2}}\\
&\lesssim ||a||_{L^6} ||\nabla u||_{L^2} + ||b||_{L^6} ||\nabla a||_{L^2}\\
&\lesssim ||a||_{\dot H^1} ||b||_{\dot H^1}
\end{align*}
since there is an embedding $\dot H^1 \to L^6$.
\end{proof}

\begin{lemma}
Let $v_0 \in \dot H^s$ and $h \in L^2_T \dot H^{s-1}$. Let $v$ satisfy
$$\partial_t v = \nu \Delta v + h$$
with $v(0) = v_0$. Then
$$v \in C([0, T] \to \dot H^s) \cap L^p_T \dot H^{s + 2/p}$$
if $p \in [2, \infty]$. In fact,
$$||v(t)||_{\dot H^s}^2 + 2\nu \int_0^t ||\nabla v(t')||_{\dot H^s}^2 ~dt' = ||v_0||_{\dot H^s}^2 + \int_0^t \langle h(t'), v(t')\rangle_s ~dt'$$
where
$$\langle a, b\rangle_s = \int_{\RR^d} |\xi|^{2s} \hat a(\xi) \overline{\hat b(\xi)} ~d\xi.$$
Moreover,
$$||v||_{L^p_T \dot H^{s+2/p}} \leq \nu^{-1/p} \left(||v_0||_{\dot H^s} + \nu^{-1/2} ||h||_{L^2_T \dot H^{s-1}}\right)$$
and
$$\int_{\RR^d} |\xi|^{2s} \sup_{0 \leq t' \leq t} |\hat v(\xi, t')|^2 ~d\xi \leq \left(||v_0||_{\dot H^s} + (2\nu)^{-1/2} ||h||_{L^2_T \dot H^{s-1}} \right).$$
\end{lemma}
\begin{proof}
We sketch the argument.
Here $v$ is given by the Duhamel formula for the heat propagator, thus
$$v(t) = e^{\nu t\Delta} v_0 + \int_0^t e^{\nu(t - t')\Delta} h(t') ~dt'$$
so that
$$\hat v(\xi, t) = e^{-\nu t|\xi|^2} + \int_0^t e^{\nu(t - t')|\xi|} * \hat h(\xi, t') ~dt'$$
which implies the estimates that appear in the lemma.
\end{proof}

\begin{theorem}
Let $u_0 \in \dot H^{1/2}_\sigma(\RR^3)$. Then there is $T > 0$ such that the Navier-Stokes equation has a unique solution $u \in L^4([0, T] \to \dot H^1)$. Furthermore,
$$u \in C([0, T] \to \dot H^{1/2}) \cap L^2([0, T] \to \dot H^{3/2}).$$
If $T_{u_0}$ is the blowup time of $u_0$, then:
\begin{enumerate}
\item There is a constant $c > 0$ which does not depend on $u_0$ such that if $||u_0||_{\dot H^{1/2}} \leq c\nu$, then $T_{u_0} = \infty$.
\item If $T_{u_0} < \infty$ then
$$\int_0^{T_{u_0}} ||u(t)||_{\dot H^1}^4 ~dt = \infty.$$
\end{enumerate}
Furthermore there is $C > 0$ such that if $u,v$ are solutions with initial data $u_0, v_0$, then
$$||u(t) - v(t)||_{\dot H^{1/2}}^2 + \nu \int_0^t ||u(s) - v(s)||_{\dot H^{3/2}}^2 ~ds \leq ||u_0 - v_0||_{\dot H^{1/2}}^2 \exp \left(\frac{C}{\nu^3} \int_0^t ||u(s)||_{\dot H^1}^4 + ||v(s)||_{\dot H^1}^4 ~ds\right).$$
\end{theorem}
\begin{proof}
Let us show that $E = L^4_T\dot H^1$ satisfies the constraints of the Picard-type theorem.
The second lemma above implies that for every $T > 0$,
$$||B(u, v)||_{L^4_T \dot H^1} \lesssim ||u||_{L^4_T \dot H^1} ||v||_{L^4_T \dot H^1}.$$
Indeed,
$$||v||_{L^p_T \dot H^{s+2/p}} \leq \nu^{-1/p} \left(||v_0||_{\dot H^s} + \nu^{-1/2} ||h||_{L^2_T \dot H^{s-1}}\right)$$
so we plug in $v_0 = 0$, $h = Q(u, v)$, to get
$$||B(u, v)||_{L^4_T \dot H^1} \lesssim ||Q(u, v)||_{L^2 \dot H^{-1/2}} \lesssim ||u||_{L^4 \dot H^1} ||v||_{L^4 \dot H^1}.$$
Let $C_0 > 0$ be the implied constant here.
This satisfies the constraints if
$$||e^{\nu t \Delta} u_0||_{L^4_T \dot H^1} \leq \frac{1}{4C_0}.$$
I missed the rest of this proof.
\end{proof}

I missed a day of class due to illness (covid?) so some lemmata are missing here.

\section{$L^3(\RR^3)$}
\begin{theorem}
Let $u_0 \in L^3_\sigma(\RR^3)$. Then there is $T > 0$ such that the Navier-Stokes equations have a unique solution $u([0, T] \to L^3)$.
Moreover, there is $c > 0$ which does not depend on $u_0$ such that if $||u_0||_{L^3} \leq c\nu$ then $T = +\infty$.
\end{theorem}
This proof cannot be obtained by a fixed-point argument in $L^\infty;L^3$.
Indeed, the quadratic form $B$ does not send $L^\infty;L^3$ to itself.
Instead, we will use ``Kato's space", which is modified to account for the heat flow.

\begin{definition}
Let $p \in [3, \infty]$ and $T \in (0, \infty)$. Then \dfn{Kato's space} $K_p(T)$ has the norm
$$||u||_{K_p(T)} = \sup_{t \in (0, T]} (\nu t)^{(1 - 3/p)/2}||u(t)||_{L^p}$$
and is a subspace of $C((0, T] \to L^p)$.

Let $p \in [1, 3)$ and $T \in (0, \infty)$. Then $K_p(T)$ has the norm
$$||u||_{K_p(T)} = \sup_{t \in [0, T]} (\nu t)^{(1 - 3/p)/2}||u(t)||_{L^p}$$
and is a subspace of $C([0, T] \to L^p)$.

We define the same space for $T = \infty$ by taking the supremum over $(0, \infty)$ or $[0, \infty)$.
\end{definition}

Notice that in the case $p < 3$ we measure time all the way up to $0$ but in the case $p \geq 3$ we discard time zero.

Let $u_0 \in L^3(\RR^3)$ and $p \geq 3$.
As
$$e^{\nu t \Delta} u_0(x) = (3\pi \nu t)^{-3/2} e^{-|x|^2/4\nu t} * u_0(x)$$
Young's inequality gives
$$||e^{\nu t \Delta} u_0||_{L^p} \leq (4 \pi \nu t)^{-3/2} ||e^{-|\cdot|^2/4\nu t}||_{L^r} ||u_0||_{L^3}$$
where $1 + 1/p = 1/r + 1/3$. Then
$$||e^{\nu t \Delta} u_0||_{L^p} \lesssim (\nu t)^{(1- 3/p)/2}||u_0||_{L^3}.$$
That is,
$$||e^{\nu t \Delta} u_0||_{L^p} \lesssim ||u_0||_{L^3}.$$
Then for every $\varepsilon > 0$, there is $\phi \in L^p$ such that $||u_0 - \phi||_{L^3} < \varepsilon$; then if $T \in (0, \infty)$,
\begin{align*}
||e^{\nu t \Delta} u_0||_{K_p(T)} &\leq ||e^{\nu t \Delta}(u_0 - \phi)||_{K_p(T)} + ||e^{\nu t \Delta} \phi||_{K_p(T)}\\
&\lesssim ||u_0 - \phi||_{L^3} + (\nu T)^{(1 - 3/p)/2} ||\phi||_{L^p}\\
&\lesssim \varepsilon + (\nu T)^{(1-3/p)/2} ||\phi||_{L^p}
\end{align*}
which implies that
$$\lim_{T \to 0} ||e^{\nu t \Delta} u_0||_{K_p(T)} = 0$$
whenever $u_0 \in L^3$ and $p > 3$.

To prove the main theorem of this section we need a new theorem, which is a subcritical well-posedness theorem (while $L^3$ is critical):

\begin{theorem}
For every $p > 3$ and every tempered $u_0$, $\nabla \cdot u_0 = 0$, and $T > 0$ such that
$$||e^{\nu t \Delta} u_0||_{K_p(T)} \lesssim_p \nu,$$
there is a unique solution of the Navier-Stokes system in $B(0, 2||e^{\nu t\Delta}u_0||_{K_p(T)}) \subseteq K_p(T)$.
\end{theorem}

To prove this theorem we also need a lemma.

\begin{lemma}
Suppose
$$\frac{1}{r} \leq \frac{1}{p} + \frac{1}{q} \leq 1$$
and
$$0 < \frac{1}{p} + \frac{1}{q} < \frac{1}{3} + \frac{1}{r}.$$
Then for every $T > 0$,
$$||B(u, v)||_{K_r(T)} \lesssim \nu^{-1} ||u||_{K_p(T)} ||v||_{K_q(T)}.$$
\end{lemma}

The subcritical well-posedness theorem follows by applying this lemma with
$$3 < p = q = r < \infty.$$
That will allow us to carry out a Picard iteration.
To prove this bound on the bilinear form we will prove a bound on solutions of the heat equation:

\begin{lemma}
Let $\ell \in \{1, 2, 3\}$ and $g: \RR^3 \times [0, \infty) \to \RR^3$ satisfy
$$\partial_t g - \nu \Delta g = \PP \partial_\ell h$$
where $h: \RR^3 \times [0, \infty) \to \RR^3$ and $g(0) = 0$. Then there is an integral kernel $\Gamma_{j,k} \in C((0, \infty) \to \bigcap_{1 < s < \infty} L^s)$ such that
$$g^j(x, t) = \sum_{k=1}^3 \int_0^t \int_{\RR^3} \Gamma_{j,k}(t - t', y) h^k(x - y, t') ~dy ~dt'$$
and
$$|\Gamma_{j,k}(x, t)| \lesssim (\sqrt{\nu t} + |x|)^{-4}.$$
\end{lemma}

This heat equation bound is Lemma 5.30 in Bahouri's book and is based on an explicit computation.

\begin{proof}[Proof of bilinear form bounds]
It follows from the hypotheses that
$$\frac{1}{r} \leq \frac{1}{p} + \frac{1}{q} \leq 1$$
so there is $s \in [1, \infty]$ such that
$$1 + \frac{1}{r} = \frac{1}{s} + \frac{1}{p} + \frac{1}{q}.$$
Let
$$\frac{1}{m} = \frac{1}{p} + \frac{1}{q},$$
so $m \in [1, \infty)$.
By Young's inequality and H\"older's inequality applied to the heat equation bounds,
\begin{align*}||B(u, v)(t)||_{L^r} &\lesssim \int_0^t ||(\sqrt{\nu (t - t')} + |\cdot|)^{-4}||_{L^s} ||u \otimes v||_{L^m}(t') ~dt'\\
&\leq \int_0^t \frac{||u(t')||_{L^p} ||v(t')||_{L^q}}{(\nu(t - t'))^{2 - 3/2s}}~dt'\\
&\leq \int_0^t \frac{1}{\sqrt{\nu(t - t')}^{4-3/(1 + 1/r - 1/p - 1/q)}} \frac{1}{\sqrt{\nu t}^{2-3(1/p+1/q)}} ||u||_{K_p(t)}(t') ||v||_{K_q(t)}(t') ~dt'\\
&\lesssim \frac{||u||_{K_p(T)} ||v||_{K_p(T)}}{\nu\sqrt{\nu t}^{1-1/r}}.
\end{align*}
\end{proof}

\begin{proof}[Proof of well-posedness in $L^3$]
Applying the well-posedness in Kato space, we get a solution in $K_6(T)$, assuming
$$||e^{\nu t \Delta} u_0||_{K_6(T)} \lesssim \nu.$$
We need to find criteria for that to hold. In fact
$$||e^{\nu t \Delta} u_0||_{K_6(T)} \lesssim ||u_0||_{L^3}$$
so what we need is $||u_0||_{L^3} \lesssim \nu$, in which case we get a global solution in $K_6(T_0)$ for any $T_0 > 0$.
On the other hand, if $||u_0||_{L^3}$ is large then we can take $T$ small enough that $||e^{\nu t \Delta} u_0||_{K_6(T)} \ll 1$, in which case we get a local solution in $K_6(T_0)$ for every $T_0 > 0$ small enough.

So now it suffices to show that the solution in $K_6(T_0)$ is actually in $C([0, T] \to L^3)$, and that this solution is unique.
Recall that $u(t) = e^{\nu t \Delta} u_0 + B(u, u)$ and $e^{\nu t \Delta} u_0 \in C([0, T] \to L^3)$. Set $w = B(u, u)$.
Applying the lemma with $p = q = 6$ and $r = 3$ we get $w \in K_3(T)$ since $u \in K_6(T)$.
To get the continuity of $w$ at $T = 0$, we use
$$w(0) = u(0) - u_0 = 0$$
but since $w \in K_3(T)$,
$$||w||_{L^\infty([0, t] \to L^3)} \lesssim \frac{||u||_{K_6(t)}^2}{\nu}.$$
The solution that we obtained belongs to the ball $B(0, 2||e^{\nu t \Delta} u_0||_{K_6(t)})$ whence
$$||w||_{L^\infty([0, t] \to L^3)} \lesssim \frac{||e^{\nu (t - t') \Delta} u_0||_{K_6(t)}}{\nu}.$$
Taking the limit $t \to 0^+$ we get convergence to $0$.

For the uniqueness, suppose $u_1, u_2$ have initial data $u_0$ and are in $C([0, T] \to L^3)$.
Let $u_{12} = u_1 - u_2$ and $w_j = B(u_j, u_j)$. Then $u_{12} = w_2 - w_1 = w_{12}$ by uniqueness for the heat equation.
Also
$$\partial_t u_{12} = \nu \Delta u_{12} + Q(u_2, u_2) - Q(u_1, u_1)$$
thus
$$\partial_t u_{12} = \nu \Delta u_{12} + f$$
where
$$f = Q(e^{\nu t \Delta} u_0, u_{12}) + Q(u_{12}, e^{\nu t \Delta} u_0) + Q(w_2, w_{12}) + Q(u_{12}, w_1).$$
We have an inhomogeneous heat equation for $u_{12}$ with zero initial data and forcing $f$.
Thus $w_j \in K_2(T)$ where we used the lemma with $r = 2$, $p = q = 3$.
Also $K_2(T) \subseteq C([0, T] \to L^2)$.
Also
$$||Q(a, b)||_{\dot H^{-3/2}} \lesssim \sup_{1 \leq k, j \leq 3} ||a_k b_j||_{\dot H^{-1/2}} \lesssim \sup_{1 \leq k, j \leq 3} ||a_k b_j||_{L^{3/2}} \lesssim ||a||_{L^3} ||b||_{L^3}$$
using the adjoint of the Sobolev embedding $\dot H^{1/2} \to L^3$, hence $f \in L^\infty([0, T] \to H^{-3/2})$.
Thus heat equation estimates and interpolation gives
$$u_{12} \in C([0, T] \to \dot H^{-1/2}) \cap L^2([0, T] \to \dot H^{1/2})$$
but we also have the estimate
$$M(t) \leq 2 \int_0^t \langle f(t'), u_{12}(t')\rangle_{\dot H^{-1/2}} ~dt'.$$
Since $C_c(\RR^3)$ is dense in $L^3(\RR^3)$, we can decompose the initial data into a part $u_0^a$ which is small in $L^3$ and a part $u_0^b$ which is in $L^6$, thus
$$||u_0^a||_{L^3} \leq c\nu$$
where $c > 0$ is a small constant.
Set
$$g(t) = f(t) - Q(e^{\nu t\Delta} u_0^b, u_{12}) - Q(u_{12}, e^{\nu t\Delta} u_0^b).$$
Using the bilinear estimate for $Q$,
\begin{align*}
||g(t)||_{\dot H^{-3/2}} &\lesssim ||e^{\nu \Delta t} u_0^a||_{L^3} + ||w_1||_{L^3} + ||w_2||_{L^3}\\
&\lesssim (||u_0^a||_{L^3} + ||w_1||_{L^3} + ||w_2||_{L^3}) ||u_{12}||_{L^3}.
\end{align*}
Now choosing $c$ small, $t \leq t_0$ where $t_0$ is small and using the embedding $\dot H^{-1/2} \to L^3$, we get
$$||g(t)||_{\dot H^{-3/2}} \leq \frac{\nu}{4}||u_{12}(t)||_{\dot H^{-1/2}}.$$
On the other hand,
$$||Q(a, b)||_{\dot H^{-3/2}} \lesssim \sup_{1 \leq k,\ell \leq 3} ||a_k b_\ell||_{L^{3/2}} \leq ||a||_{L^6} ||b||_{L^2}$$
since $1/2 + 1/6 = 2/3$.
That is,
$$||Q(e^{\nu t \Delta} u_0^b, u_{12})||_{\dot H^{-3/2}} \lesssim ||e^{\nu t\Delta}u_0^b||_{L^6} ||u_{12}||_{L^2} \leq ||u_0^b||_{L^6} ||u_{12}||_{L^2}.$$
Combining this with the estimate on $g$ we get an estimate on $f$, namely
\begin{align*}
||f||_{\dot H^{-3/2}} &\leq \frac{\nu}{4}||u_{12}||_{\dot H^{1/2}} + C||u_0^b||_{L^6} ||u_{12}||_{L^2}\\
&\leq \frac{\nu}{4} ||u_{12}(t)||_{\dot H^{1/2}} + C||u_0^b||_{L^6} ||u_{12}||_{\dot H^{-1/2}}^{1/2} ||u_{12}||_{\dot H^{1/2}}^{1/2}
\end{align*}
assuming that $t \leq t_0 \ll 1$. Therefore
$$||u_{12}(t)||_{\dot H^{-1/2}}^2 + 2\nu \int_0^t ||u_{12}(t')||_{\dot H^{1/2}}^2 ~dt' \leq 2\int_0^t \frac{\nu}{4} ||u_{12}(t')||_{\dot H^{1/2}}^2 ~dt + C||u_0^b||_{L^6}||u_{12}(t')||_{\dot H^{-1/2}}^{1/2} ||u_{12}(t')||_{\dot H^{1/2}}^{3/2} ~dt.$$
Gr\"onwall's inequality now gives $u_{12} = 0$ on $[0, t_0]$.
Let $t_1 > 0$ be the maximal time such that $u_{12} = 0$ on $[0, t_1)$.
By continuity we get $u_{12}(t_1) = 0$, and then by repeating the argument we see that the set on which $u_{12} = 0$ must be open, but we just showed that it is closed.
\end{proof}

Now suppose that $u_0 \in L^p$ with $p > 3$.
In the homework we show that if the blowup time $T_{u_0} > 0$ for initial data $u_0$ in $K_6(T)$, then
$$||u(t)||_{L^p(\RR^3)} \gtrsim \frac{1}{(T_{u_0} - t)^{(1 - 3/p)/2}}$$
provided $t < T_{u_0}$.
The trouble is that $1-3/p \to 0$ as $p \to 3$ so this doesn't tell us anything about blowup time in $L^3$. In fact, a recent theorem of Escauraiaza, Seregin, and Sverak, implies
$$\limsup_{t \to T_{u_0}} ||u(t)||_{L^3} = \infty.$$
But this is difficult to prove.

How far can the fixed-point method take us?
The maximal such space is called $BMO^{-1}$, as proved by Koch and Tataru.
Here $BMO^{-1}$ is the space of functions
$$f = \sum_{j=1}^N \partial_j f^j$$
where $f^j \in BMO$.
If $||u_0||_{BMO^{-1}} \ll 1$ then we get a global solution.
But a local solution is not known to exist for large data in $BMO^{-1}$.

\chapter{Weak Euler solutions with $L^\infty$ vorticity and $d = 2$}
Let $d = 2$.
We have proven that for smooth initial data for the Euler equations we get global smooth solutions.
However, in reality we may experience vortex patches, where the vorticity is cut off discontinuously to an open set $\Omega$, or vortex sheets, where the vorticity is concentrated in a lower-dimensional curve.
These were studied by Yudovich in the 1960s.
Still following Huy Nguyen's lectures.

The simplest case is the vortex patch
$$\omega = \alpha 1_\Omega$$
where $\alpha$ is a constant. In this case, Yudovich showed that vortex patches are initial data for weak solutions in $L^p(\RR^2)$ for every $p$.

Recall the vorticity formulation
$$\partial_t \omega + u \cdot \nabla \omega = 0$$
where
$$u = \nabla^\perp \Delta^{-1} \omega$$
where the convolution kernel $K$ of $\nabla^\perp \Delta^{-1}$ is
$$K(x) = \frac{1}{2\pi} \frac{x^\perp}{|x|^2}.$$
Suppose $\omega_0 \in L^1 \cap L^\infty$.

\section{Potential-theoretic estimates}
For $\omega \in L^1 \cap L^\infty$, we write
$$||\omega||_{L^1 \cap L^\infty} = ||\omega||_{L^1} + ||\omega||_{L^\infty}.$$
Set $u = K * \omega$.

\begin{lemma}
For every $\omega \in L^1 \cap L^\infty$,
$$||u||_{L^\infty} \lesssim ||\omega||_{L^1 \cap L^\infty}.$$
\end{lemma}
\begin{proof}
One has $|K(x)| \leq 1/|x|$. Split up
$$u(x) = \int_{|x - y| \leq 1} K(x - y)\omega(y) ~dy + \int_{|x - y| \geq 1} K(x - y) \omega(y) ~dy = I_1 + I_2.$$
By Young's inequality,
$$||I_1||_{L^\infty} \leq ||K||_{L^1(B(0, 1))} ||\omega||_{L^\infty} \lesssim ||\omega||_{L^\infty}.$$
Also
$$||I_2||_{L^\infty} \leq ||K||_{L^\infty(B(0, 1))} ||\omega||_{L^1} \lesssim ||\omega||_{L^1}$$
as desired.
\end{proof}
In fact if
$$1 < p < 2 < q < \infty,$$
then
$$||u||_{L^\infty} \lesssim ||\omega||_{L^p \cap L^q}$$
using the same proof.
Note that
$$\nabla u = \nabla \nabla^\perp \Delta^{-1} \omega$$
and $\nabla \nabla^\perp \Delta^{-1}$ is a singular integral and hence unbounded on $L^\infty$.
Therefore it does not follow from $\omega \in L^\infty$ that $u$ is Lipschitz.
But we get the following estimate, which is Lipschitz modulo logarithmic singularities:

\begin{lemma}
For every $\omega \in L^1 \cap L^\infty$ and $x, x' \in \RR^2$,
$$|u(x) - u(x')| \lesssim ||\omega||_{L^1 \cap L^\infty} \varphi(|x - x'|)$$
where
$\varphi(r) = r(1 - \log r)$ if $r < 1$ and otherwise $\varphi(r) = 1$.
\end{lemma}
\begin{proof}
If $r = |x - x'| \geq 1$ then by the previous lemma,
$$|u(x) - u(x')| \leq 2||u||_{L^\infty} \lesssim ||\omega||_{L^1 \cap L^\infty} = ||\omega||_{L^1 \cap L^\infty} \varphi(r)$$
as desired. Otherwise
$$u(x) - u(x') = \left[\int_{B(x, 2r)} + \int_{B(x, 2r)^c}\right] (K(x - y) - K(x' - y)) \omega(y) ~dy = I_1 + I_2.$$
For $I_1$, use the fact that $K$ is homogeneous of degree $-1$ so
\begin{align*}
|I_1| &\leq \int_{|x - y| \leq 2r} \left(\frac{1}{|x - y|} + \frac{1}{|x' - y|}\right) |\omega(y)| ~dy\\
&\leq \int_{|y - x| \leq 2r} \frac{|\omega(y)|}{|x - y|} ~dy + \int_{|x' - y| \leq 3r} \frac{|\omega(y)|}{|x' - y|} ~dy\\
&\lesssim ||\omega||_{L^\infty} \left(\int_0^{2r} dr + \int_0^{3r} ~dr\right)\\
&\lesssim r||\omega||_{L^\infty}.
\end{align*}
For $I_2$, we get a logarithmic singularity; namely, by the mean-value theorem, there is $x'' \in [x, x']$ such that
\begin{align*}
|K(x - y) - K(x' - y)| &\leq |x - x'| |\nabla K(x'' - y)| \lesssim \frac{r}{|x'' - y|^2}.
\end{align*}
If $y \notin B(x, 2r)$ then
$$|x - y| \leq |x - x''| + |x'' - y| \lesssim |x'' - y|$$
which implies
$$|K(x - y) - K(x' - y)| \lesssim \frac{r}{|x - y|^2}$$
and so
\begin{align*}
|I_2| &\lesssim \int_{B(x, 2r)^c} r\frac{|\omega(y)|}{|x - y|^2} ~dy\\
&\lesssim \int_{B(x, 2) \setminus B(x, 2r)} r\frac{|\omega(y)|}{|x - y|^2} ~dy + \int_{B(x, 2)^c} r|\omega(y)| ~dy\\
&= r||\omega||_{L^\infty} \int_{2r}^2 \frac{dr}{r} + r||\omega||_{L^1}\\
&= r||\omega||_{L^\infty}(\log 2 - \log(2r)) + r||\omega||_{L^1}.
\end{align*}
Combining this with the bound on $I_1$ we get
$$|u(x) - u(x')| \lesssim r||\omega||_{L^1 \cap L^\infty}(2 + \log 2 - \log(2r))$$
which gives the bound.
\end{proof}

\begin{definition}
A function $u$ is \dfn{log-Lipschitz} if
$$|u(x) - u(x')| \lesssim \varphi(|x - x'|)$$
where $\varphi(r) = r(1 - \log r)$ if $r < 1$ and otherwise $\varphi(r) = 1$.
We write
$$||u||_{LL} = \sup_{x \neq x'} \frac{|u(x) - u(x')|}{\varphi(x - x')}$$
for the \dfn{log-Lipschitz seminorm}.
\end{definition}

We have just shown that if the vorticity is in every $L^p$ space then the velocity is log-Lipschitz.
We will now show that every log-Lipschitz velocity generates a unique trajectory, generalizing the Picard-Linde\"of theorem.

\begin{lemma}[Osgood's lemma]
Let $\mu: [0, a] \to [0, \infty)$ be an \dfn{Osgood continuity modulus} in the sense that $\mu$ is nondecreasing, $\mu(0) = 0$, $\mu(r) > 0$, and for every $r \in (0, a]$ such that
$$\int_0^a \frac{dr}{\mu(r)} = \infty.$$
Let $c \geq 0$, $f: [t_0, T] \to [0, a]$ continuous, $\gamma \in L^1([t_0, T])$, and
$$f(t) \leq c + \int_{t_0}^t \gamma(t')\mu(f(t')) ~dt'.$$
Let $t \in [t_0, T]$.
Then if $c > 0$ and
$$M(x) = \int_x^a \frac{dr}{\mu(r)},$$
then
$$-M(f(t)) + M(c) \leq \int_{t_0}^t \gamma(t') ~dt'$$
and if $c = 0$, then $f(t) = 0$.
\end{lemma}
We omit the proof.

\begin{theorem}[log-Picard iteration]
Let $b \in L^\infty([0, T] \to LL \cap L^\infty)$ and $T > 0$.
Then for every $x_0 \in \RR^d$,
$$\dot x(t) = b(x(t), t)$$
has a solution with initial data $x(0) = x_0$ which is unique in $W^{1, \infty}([0, T]) \subseteq C([0, T])$.
\end{theorem}
\begin{proof}
Define $x_0(t) = x_0$ and
$$x_{n+1}(t) = x_0 + \int_0^t b(x_n(s), s) ~ds.$$
Then $(x_n)$ is bounded in $C([0, T])$.
Indeed, this follows because $b \in L^\infty([0, T] \times \RR^d)$.
Now set
$$\rho_{k,n}(t) = \sup_{t' \in [0, t]} |x_{k+n}(t') - x_k(t')|.$$
Then
$$|\rho_{k+1,n}(t)| \leq \int_0^t \varphi(\rho_{k, n}(t')) ~dt'$$
so $\rho_k = \sup_n \rho_{k,n}$ satisfies
$$\rho_{k+1}(t) \leq \int_0^t \varphi(\rho_k(t')) ~dt'$$
by monotone convergence (since $\varphi$ is nondecreasing).
Then
$$\tilde \rho(t) = \limsup_{k \to \infty} \rho_k(t) = \int_0^t \varphi(\tilde \rho(t')) ~dt'.$$
By Osgood's lemma (case $c = 0$), $\tilde \rho = 0$, so $(x_n)$ is a Cauchy sequence, say $x = \lim_n x_n$.
Then $x$ is a solution, and by Osgood's lemma again is unique.
\end{proof}

Now we show a form of H\"older continuity, which deterioriates as $T \to \infty$.

\begin{proposition}
Let $b \in L^\infty(\RR^d \times [0, T]) \cap L^\infty([0, T] \to LL)$ and denote by $X_t(x_0)$ the flow map for
$$\dot x(t) = b(x(t), t)$$
with $x(0) = x_0$.
Let
$$L = \sup_{t \in [0, T]} ||b(t)||_{LL}.$$
If $|x - y| < e^{1 - \exp(LT)}$, then for every $t \in [0, T]$,
$$|X_t(x) - X_t(y)| \leq |x - y|^{\exp(-LT)} e^{1 - \exp(-LT)}$$
and
$$|X_t(x) - X_t(y)| \geq |x - y|^{\exp(LT)} e^{1 - \exp(LT)}.$$
\end{proposition}
\begin{proof}
Let
$$0 < |x - y| < e^{1 - \exp(LT)}.$$
Because $X_t^{-1} = X_t$, $X_t$ is injective so $X_t(x) \neq X_t(y)$.
Thus $F(t) = |X_t(x) - X_t(y)| > 0$.
Then
\begin{align*}
|(F^2)'(t)| &= 2F(t) |X_t'(x) - X_t'(y)|
&\leq 2F(t) \int_0^t \varphi(F(t')) ||b(t')||_{LL} ~dt'\\
&\leq 2L F(t)\int_0^t \varphi(F(t')) ~dt'.
\end{align*}
But $|F(0)| = |x - y| < 1$.
Define $T_*$ so $F < 1$ on $[0, T_*)$ and $F(T_*) = 1$.
On $[0, T_*)$,
$$|F'(t)| \leq LF(t)(1 - \log F(t))$$
and hence
$$|\partial \log F(t)| \leq L(1 - \log F(t)).$$
That is,
$$F(0)^{\exp(LT)}e^{1 - \exp(LT)} \leq F(t) \leq F(0)^{\exp(-LT)} e^{1 - \exp(-LT)}.$$
Thus $F(T_*) \leq F(0) e^{\exp(-LT_*)} e^{1 - \exp(-LT_*)}$.
So $T_* = T$ provided that $F(0) e^{\exp(-LT_*)} e^{1 - \exp(-LT_*)} < 1$, or equivalently,
$$F(0) < \exp(-1 \exp(-LT_*))$$
which is given.
\end{proof}

\begin{proposition}
Let $b$ be as above with $\nabla \cdot b = 0$. Then $X$ is a measure-preserving action.
\end{proposition}
\begin{proof}
Pushforward by $X$ preserves the integral of any function in $C_c^\infty(\RR^d)$.
\end{proof}

\section{Existence and uniqueness}

\begin{theorem}[Yudovich]
Let $\omega$ be a smooth Euler vorticity in $\RR^2$ with initial data $\omega_0$.
Let $X$ be the flow map associated to $u$. If $u \in L^\infty([0, T] \to W^{1, \infty}(\RR^2))$ then $X$ is Lipschitz and invertible.
\end{theorem}
\begin{proof}
One has
$$\partial_t \omega(X_t(x), t) = 0$$
so $\omega$ is invariant along trajectories of $X$.
Thus $\omega(x, t) = \omega_0(X_t^{-1}(x))$.
This makes $X_t^{-1}$ well-defined as long as $\omega \in L^\infty([0, T] \to L^1 \cap L^\infty)$.
\end{proof}

\begin{theorem}[Yudovich 1963]
Let $\omega_0 \in L^1 \cap L^\infty$. Then there are unique $\omega,u,X$ such that $\omega \in L^\infty([0, T] \to L^1 \cap L^\infty)$, for every $T > 0$, $X$ is invertible and measure-preserving, $X$ is the flow map for the velocity field $u$, $u$ satisfies the Biot-Savart law for $\omega$, $\omega$ solves the Euler equation with inital data $\omega_0$, and
$$\omega(x, t) = \omega_0(X_t^{-1}(x)).$$
\end{theorem}
\begin{proof}
We first construct approximate solutions. Set $\omega^0(x, t) = \omega_0(x)$ and given $(\omega^n, u^n, X^n)$, set $u^n(t) = K * \omega^{n-1}(t)$, and
$$\dot X_t^n(x) = u^n(X_t^n(x), t)$$
with initial data $\dot X_0^n(t) = x$, and $\omega^n(x, t) = \omega_0((X_t^n)^{-1}(x))$.
Thus we are solving the system of infinitely many linear pseudodifferential equations
\begin{align*}
\partial_t \omega^n + u^n \cdot \nabla \omega^n &= 0\\
u^n &= K * \omega^{n - 1}.
\end{align*}
If $X^n$ is volume-preserving, then $||\omega^n(t)||_{L^p} = ||\omega_0||_{L^p}$.
Thus the previous lemmata yield
$$||u^n(t)||_{L^\infty} \lesssim ||\omega^n(t)||_{L^1 \cap L^\infty} \lesssim ||\omega||_{L^1 \cap L^\infty}$$
uniformly in $n,t$, and also
$$||u^n(t)||_{LL} \lesssim ||\omega_0||_{L^1 \cap L^\infty}$$
uniformly for similar reasons.
Let $R, T > 0$. The H\"older continuity proposition shows that there is some $\alpha$ which only depends on $T,R$, and $||\omega_0||_{L^1 \cap L^\infty}$ such that
$$||X^n_t||_{C^\alpha(\overline B_R \times [0, T])} \lesssim_{R, T} \lesssim 1.$$
According to Ascoli's compactness theorem, after passing to a subsequence, there is $X \in C(\overline B_R \times [0, T])$ with $X_n \to X$.
From $X$ we can reconstruct $\omega$ according to
$$\omega(x, t) = \omega_0(X_{-t}(x))$$
and thus $u$ by $u = K * \omega$.
We must check that $(\omega, u, X)$ is an Euler solution; in fact it suffices to check
$$\dot X_t(x) = u(X_t(x), t).$$
That is, we must show
$$X_t(x) = x + \int_0^t u(X_s(x), s) ~ds.$$
By construction,
$$X_t^n(x) = x + \int_0^t u^n(X_s^n(x), s) ~ds$$
and $X^n \to X$ locally uniformly, so we just need to get a suitably strong velocity convergence.
In fact this follows from dominated convergence, as the convergence of $X^n$ implies
$$\lim_{n \to \infty} ||u^n(t) - u(t)||_{L^\infty} = 0$$
as necessary.
To see this, by the proposition, $u^n(t) \in LL$. Then
$$|u^n(x, t) - u(x, t)| \leq \int_{|y - x| \leq \delta} |K(x - y)||\omega^n(y, t) - \omega(y, t)| ~dy + \int_{|y - x| > \delta} \bullet = I_1 + I_2.$$
Clearly
$$|I_1| \leq 2||\omega_0||_{L^\infty} \int_{|z| \leq \delta} \frac{dz}{|z|} \lesssim \delta ||\omega_0||_{L^\infty}.$$
Also
$$|I_2| \leq \int_{|x - y| \geq \delta} (|\omega^n(y, t)| + |\omega(y, t)|)\frac{dy}{|x - y|} \lesssim \frac{||\omega^n(t)||_{L^1} + ||\omega(t)||_{L^1})}{\delta}.$$
That is,
$$|u^n(x, t) - u(x, t)| \lesssim \delta ||\omega||_{L^\infty} + \frac{||\omega^n(t) - \omega(t)||_{L^1}}{\delta}.$$
As $\omega^n \to \omega$ in $L^1$ (since $\omega^n$ is defined in terms of $X^n$, and passing to suitable subsequences we can upgrade the convergence of $X^n$ from $L^\infty_{loc}$ to $L^\infty$).
Thus we get along a subsequence,
$$\limsup_{n \to \infty} |u^n(x, t) - u(x, t)| \lesssim \delta ||\omega_0||_{L^\infty}.$$
As the left-hand side does not depend on $\delta$, we conclude $u^n \to u$ in $L^\infty$ which gives the velocity convergence.

Now we prove uniqueness.
For simplicity we assume $\omega_0 \in L^\infty$ has compact support.
Let $(\omega^j, u^j, X^j)$ be two solutions with initial vorticity $\omega_0$.
Assume that $X^1 = X^2$, then because vorticity and velocity can be recovered from $X$, we have uniqueness.
So, it suffices to show $X^1 = X^2$.
Let $F = X^1 - X^2$.
Then
\begin{align*}
F(x, t) &\leq \int_0^t |u^1(X^1_s(x), s) - u^2(X^s_(x), s)| ~ds\\
&\leq \int_0^t |u^1(X^1_s(x), s) - u^1(X^s_(x), s)| ~ds + \int_0^t |u^1(X^2_s(x), s) - u^2(X^2_s(x), s)| ~ds\\
&= I_1 + I_2.
\end{align*}
Since
$$||u_1||_{LL} \lesssim ||\omega_0||_{L^1 \cap L^\infty}$$
we get
$$|I_1| \lesssim ||\omega_0||_{L^1 \cap L^\infty} \int_0^t \varphi(F(x, s)) ~ds.$$
Since $u^j = K * \omega^j$ we get
\begin{align*}
|I_2| &\leq \int_0^t \left|\int_{\RR^2} K(X^1_s(x) - y) \omega^1(y, s) - K(X^2_s(x) - y) \omega^2(y, s) ~dy\right| ~ds\\
&= \int_0^t \left|\int_{\RR^2} K(X^2_s(x) - y) \omega_0(X^1_{-s})(y)) - K(X^2_s(x) - y) \omega_0(X^2_{-s}(y)) ~dy\right| ~ds.
\end{align*}
Suppose that $\omega_0$ is zero outside $B_{R_0}(0)$. Then there is $R_T$ such that $\supp \omega(t) \subseteq B_{R_T}(0)$ if $t \leq T$.
In particular $|X^j_t(x)| \leq M(T)$ for some $M(T) > 0$ and every $x \in \supp \omega_0$.
Also the $X^j$ are measure-preserving, so
\begin{align*}
|I_2| &\leq \int_0^t \left|\int_{\RR^2} K(X^2_s(x) - X^1_s(y))\omega_0(y) - K(X^2_s(x) - K^2_s(y))\omega_0(y) ~dy\right| ~ds\\
&\leq \int_0^t \int_{\RR^2} |K(X^2_s(x) - K^1_s(y)) - K(X^2_s(x) - X^2_s(y))| |\omega_0(y)| ~dy ~ds.
\end{align*}
Then, if $R = R(T)$,
\begin{align*}
\int_{B_R(0)} |I_2(x, t)| ~dx &\leq \int_0^t \iint_{B_{R_0}(0)^2} |K(X^2_s(x) - X^1_s(y)) - K(X^2_s(x) - X^2_s(y))| ~dx ~dy ~ds\\
&\leq \int_0^t \int_{B_R(0)} |\omega_0(y)| \int_{X_s^2(B_R(0))} |K(x - X^1_s(y)) - K(x - X^2_s(y))| ~dx ~dy ~ds.
\end{align*}
Since $X_s^2(B_R(0))$ is bounded,
$$\int_{X_s^2(B_R(0))} |K(x - a) - K(x - b)| ~dx \lesssim \varphi(|a - b|)$$
whenever $a,b \in \RR^2$.
Plugging this in we get
$$\int_{B_R(0)} |I_2(x, t)| ~dx \lesssim \int_0^t \int_{B_{R_0}(0)} |\omega_0(y)| \varphi(F(y, s)) ~dy ~ds.$$
Combining these we get
$$\int_{B_R(0)} |F(x, t)| ~dx \lesssim ||\omega_0||_{L^1 \cap L^\infty} \int_0^t \int_{B_R(0)} \varphi(F(y, s)) ~dy ~ds.$$
Dividing both sides by the volume of $B_R(0)$ and using Jensen's inequality for the concave function $\varphi$,
$$\int_{B_R(0)} \varphi(f(x)) ~\frac{dx}{|B_R(0)|} \leq \varphi\left(\int_{B_R(0)} f(x) ~\frac{dx}{|B_R(0)|}\right).$$
Now let
$$\eta(t) = \int_{B_R(0)} |F(x, t)| ~\frac{dx}{|B_R(0)|}$$
so that
$$\eta(t) \lesssim ||\omega_0||_{L^1 \cap L^\infty} \int_0^t \varphi(\eta(s)) ~ds.$$
So by Osgood's lemma, $\eta = 0$.
Therefore $F = 0$.
\end{proof}

\section{Vortex patches with constant vorticity}

Consider the initial vorticity distribution $\omega_0(x) = a1_{\Omega_0}(x)$ where $\Omega_0$ is bounded, simply connected, open set and $\partial \Omega_0$ is mooth.
By Yudovich theory, there is a unique global solution $(\omega, u, X)$.
Furthermore,
$$\omega(x, t) = \omega_0(X_{-t}(x)).$$
Let $\Omega(t) = X_t(\Omega_0)$; then
$$\omega_0(x) = a1_{\Omega(t)}(x).$$
Thus the vorticity distribution is completely determined by $\partial \Omega(t)$.

Let us parametrize $\partial \Omega_0 = \{z_0(\alpha) \in \RR^2: \alpha \in [0, 2\pi)\}$ and $\partial \Omega(t) = \{z(\alpha, t): \alpha \in [0, 2\pi)\}$ where
$$z(\alpha, t) = X_t(z_0(\alpha)).$$
Since $d = 2$ the stream function $\psi$ satisfies
$$\Delta \psi(x, t) = a1_{\Omega(t)}(x)$$
which implies
$$\psi(x, t) = \frac{a}{2\pi} \int_{\Omega(t)} \log|x-y| ~dy.$$
Furthermore, $u = \nabla^\perp \psi$, so by the Green formula
$$\int_\Omega f\partial_ig = \int_{\partial \Omega} fg n_i \cdot dS - \int_\Omega \partial_ifg,$$
we get
$$u(x, t) = -\frac{a}{2\pi} \int_0^{2\pi} \log|x - z(\beta, t)| \partial_\beta z(\beta, t) ~d\beta.$$
Since $\dot z(\alpha, t) = u(z(\alpha, t), t)$, it follows that
$$\dot z(\alpha, t) = -\frac{a}{2\pi} \int_0^{2\pi} \log|z(\alpha, t) - z(\beta, t)| \partial_\beta z(\beta, t) ~d\beta.$$
This is known as the \dfn{contour dynamics equation}.

\begin{theorem}
Let $z_0 \in C^{1 + \gamma}(S^1)$, $\gamma \in (0, 1)$, satisfy the \dfn{arc-chord condition}
$$|z_0|_* = \inf_{\alpha_1 \neq \alpha_2} \frac{|z_0(\alpha_1) - z_0(\alpha_2)|}{|\alpha_1 - \alpha_2|} > 0.$$
Then there is $T > 0$ depending only on $||z_0||_{C^{1 + \gamma}}$ and $|z_0|_*$ such that the contour dynamics equation has a unique solution
$$z \in C([0, T] \to C^{1 + \gamma}(S^1))$$
which satisfies the arc-chord condition $|z(t)|_* > 0$.

Furthermore, let $z \in C([0, T] \to C^{1 + \gamma}(S^1))$ be a solution to the contour dynamics equation.
If $z_0 \in C^\eta(S^1)$, $\eta > 2$, then $z \in C([0, T] \to C^\eta(S^1))$.
\end{theorem}
\begin{proof}
Omitted due to lack of time; see Majda and Bertozzi.
\end{proof}

The concern is that $\partial \Omega$ becomes singular in finite time, either in the sense that $\partial \Omega$ self-intersects or loses its H\"older regularity.
However, as Chemin proved in 1993 using paradifferential calculus, this is impossible.
In fact, there is a proof which only uses harmonic analysis, by Constantin and Bertozzi in 1993.

Assume that the initial patch $\Omega_0 = \{x \in \RR^2: \varphi_0(x) > 0\}$ with $\partial \Omega_0 = \{x \in \RR^2: \varphi_0(x) = 0\}$.
Let $\varphi(x, t) > 0$ on $\Omega(t)$ with $\varphi(x, t) = 0$ on $\partial \Omega(t)$.

\begin{theorem}
Suppose that $\varphi_0 \in C^{1 + \gamma}(\RR^2)$ and
$$\inf_{\varphi_0(x) = 0} |\nabla \varphi_0(x)| \geq m > 0.$$
Then there is $C > 0$ which only depends on $||\varphi_0||_{C^{1 + \gamma}},m$ such that
$$||\nabla u(t)||_{L^\infty} \leq ||\nabla u(0)||_{L^\infty} e^{Ct},$$
also
$$||\nabla \varphi(t)||_{C^\gamma} \leq ||\nabla \varphi(0)||_{C^\gamma} \exp((C_0 + \gamma) e^{\gamma t}),$$
and
$$||\nabla \varphi(t)||_{L^\infty} \leq ||\nabla \varphi(0)||_{L^\infty} e^{Ct}.$$
Finally,
$$\inf_{\varphi(x, t) = 0} |\nabla \varphi(x, t)| \geq \inf_{\varphi_0(x) = 0} |\nabla \varphi_0(x)| e^{-Ct}$$
where $C_0$ is a \emph{universal} constant.
If in addition $\varphi_0 \in C^\eta(\RR^2)$ where $\eta > 2$, then $\varphi(t) \in C^\eta$ for every $t \geq 0$.
\end{theorem}

\chapter{Navier-Stokes stability}
Consider the dynamical system
$$\dot x(t) = F(x(t))$$
with initial data $x(0) = x_0$.
A point $x_* \in \RR^d$ is stationary if $F(x_*) = 0$.

\begin{definition}
A \dfn{stable point} $x_*$ is a stationary point if for every $\varepsilon > 0$ there is $\delta > 0$ such that if $|x_0 - x_*| < \delta$, then
$$\sup_{t \geq 0} |x(t) - x_*| < \varepsilon.$$
If $x_*$ is a stable point such that
$$\lim_{t \to \infty} x(t) = x_*$$
then $x_*$ is an \dfn{asymptotically stable point}.
\end{definition}

We can extend these notions to PDE, but will need to make a choice of norm.

\section{Euler equations with $d = 2$}
Suppose that $(u, \omega)$ is a stationary Euler solution with $d = 2$. Then
$$u \cdot \nabla \omega = 0.$$
In other words, if $\psi$ is the stream function, there is $F$ such that
$$F \circ \psi = \Delta \psi.$$

\begin{example}
$\omega_* = 0$ is stationary and stable in $L^\infty(\Omega)$ if $\Omega$ is bounded.
Indeed, by Yudovich theory, if $\omega_0 \in L^\infty(\Omega)$, there is $\omega \in L^\infty([0, \infty) \to L^\infty)$ such that $||\omega(t)||_{L^\infty} = ||\omega_0||$.
However, $\omega_*$ is not stable in $L^p$, since the solution may fail to exist with initial data in $L^p$.]
This is in spite of the fact that $L^p$ norm is conserved (since $L^p$ is not locally compact).
\end{example}

\begin{theorem}[Arnold stability condition]
Let $\Omega$ be a bounded connected set in $\RR^2$ such that $\partial \Omega$ has smooth components $\Gamma_0, \dots, \Gamma_n$.
Let $(\omega^*, u^*)$ be a stationary Euler solution in $\Omega$.
Suppose that there are $0 < c_1 < c_2$ such that
$$c_1 < -\frac{u^*}{\nabla^\perp \omega^*} < c_2.$$
Then $u^*$ is stable in $H^1(\Omega)$.
\end{theorem}

The condition makes sense here since $u^* \cdot \nabla^\perp \omega^*$, so that the vectors are collinear and we can divide them.
Also $u \in H^1(\Omega)$ iff $\omega \in L^2(\Omega)$.
One direction of this is obvious; conversely if $\omega \in L^2$ then $\psi \in H^2$ by elliptic regularity, so $u \in H^1$.

\begin{example}
Consider a shear flow in a periodic channel $\Omega = [0, L] \times [-A, A]$, with periodic boundary condition in $[0, L]$ and no-penetration boundary condition in $[-A, A]$.
Here the shear flow condition means
$$u^*(x) = (u_1^*(x_2), 0).$$
Then $u^*$ is stationary. The no-penetration boundary condition means $u_2(x_1, \pm A) = 0$.
Since $u^*$ is shear, $\omega^*(x) = -\partial_2 u_1^*(x_2)$, thus $\nabla^\perp \omega^*(x) = (\partial_2^2 u_1^*(x_2), 0)$, thus the Arnold stability condition is
$$c_1 < -\frac{u_1^*(x_2)}{\partial_2^2 u_1^*(x_2)} < c_2.$$
By adding a constant we can make $u_1^*$ strictly negative, since $\Omega$ is compact.
Thus the Arnold stability condition simply requires that $\partial_2^2 u_1^*(x_2)$ be strictly positive.
Thus if the shear flow has no inflection points -- that is, the \dfn{Rayleigh condition} -- then the shear flow is stable in $H^1(\Omega)$.
\end{example}



\chapter{Variational calculus and gradient flows}
Following lectures of Francesco Maggi.

Let $\Omega$ be an open set in $\RR^n$ and $u \in C^1(\overline \Omega)$. We want to infimize $\mathcal F(u)$ where $u$ is subject to a Dirichlet constraint and
$$\mathcal F(u) = \int_\Omega f(\nabla u(x)) ~dx$$
is some action. We will refer to $f$ as the \dfn{integrand} and $\mathcal F$ as the \dfn{energy}.

\begin{definition}
If $f(z) = |z|^2/2$ we call $\mathcal F$ the \dfn{Dirichlet energy}, and if $f(z) = \sqrt{1 + |z|^2}$ then $\mathcal F$ is the \dfn{area functional}.
\end{definition}

If $\mathcal F$ is the area functional then $\mathcal F$ is the measure of the graph of $u$ over $\Omega$.

Other than Dirichlet constraints we can apply an \dfn{integral constraint}, namely
$$\int_\Omega g(u(x)) ~dx = M$$
where $g,M$ are given.

\begin{definition}
A \dfn{gradient flow} is the flow of an ODE
$$X'(t) = -\nabla F(X(t))$$
with Cauchy data, $X: [0, \infty) \to \RR^n$.
\end{definition}

The point is that the flowout is given by a gradient.
Since $-\nabla F$ points in the direction of steepest descent of $F$, the stationary points of $X$ are the critical points of $X$.
A local maximum is unstable while a local minimum is stable (this is why we use the negative sign condition).
Thus local maxima are unphysical.

Let $\mathcal F$ be some action, and consider the gradient flow
$$\frac{\partial u}{\partial t} = -\nabla F(u(x, t)).$$
Here $-\nabla \mathcal F$ must be defined on some kind of infinite-dimensional manifold, which needs to be made precise.
Since we are mainly interested in critical points of $-\nabla \mathcal F$ this is a sign that we need to use variational calculus to study it!
It turns out that the heat equation
$$\partial_t u = \Delta u$$
is a gradient flow.
The function $u$ flows to a steady state $\Delta u = 0$ which minimizes the Dirichlet energy.
This suggests that we can find minimizers of $\mathcal F$ by flowing along the gradient flow $-\nabla F$.

\section{Necessary conditions of minimality}
Let $P$ be a problem in the variational calculus with energy $\mathcal F$, domain $\Omega$, and Dirichlet data $u_0$.

\begin{definition}
The set
$$\mathcal A = \{u \in C^1(\overline \Omega): u|\partial \Omega = u_0\}$$
is called the \dfn{competition class} of $P$.
\end{definition}

Let us study what properties minimizers have.

The idea here is to consider $\varphi \in C^\infty_c(\Omega)$.
(If we were using an integral rather than Dirichlet competition class we would need a different $\varphi$.)
Then $u + t\varphi \in \mathcal A$ whenever $t \in \RR$ and we can take the derivative
$$\delta^k \mathcal F_u(\varphi) = \frac{d^k}{dt^k}\bigg|_{t = 0}\mathcal F(u + t\varphi).$$

\begin{definition}
The $k$th \dfn{variation} is $\delta^k \mathcal F_u$.
\end{definition}

Then $\delta \mathcal F_u(\varphi) = 0$ since $u$ is a critical point, and $\delta^2 \mathcal F_u(\varphi) \geq 0$ since $u$ is a minimizer.

Let us get a Taylor formula for the variations.
Suppose
$$\mathcal F(u + t\varphi) = \int_\Omega f(\nabla u(x) + t\nabla \varphi(x)) ~dx,$$
then
$$f(z + tw) = f(z) + t\langle \nabla f(z), w\rangle + \frac{t^2}{2} \langle w, \nabla^2 f(z)w\rangle + O(t^3)$$
so plugging in $z = \nabla u(x)$, $w = \nabla \varphi(x)$,
$$\mathcal F(u + t\varphi) \approx \mathcal F(u) + t\int_\Omega \langle \nabla f(\nabla u(x)), \nabla \varphi(x)\rangle ~dx + \frac{t^2}{2} \int_\Omega \langle \nabla \varphi(x), \nabla^2 f(\nabla u(x))\nabla \varphi(x)\rangle ~dx.$$
Thus, using dominated convergence, if everything is nice we get
$$\delta \mathcal F_u(\varphi) = \int_\Omega \nabla f(\nabla u(x)) \cdot \nabla \varphi(x) ~dx$$
and
$$\delta^2 \mathcal F_u(\varphi) = \int_\Omega \nabla \varphi(x) \cdot \nabla^2 f(\nabla u(x))\nabla \varphi(x) ~dx.$$
Notice the first variation is linear in $\varphi$ and the second is quadratic in $\varphi$.

If $f$ is convex (e.g. $\mathcal F$ is the Dirichlet energy or the area functional) then $\nabla^2 f(z) \geq 0$.
Thus the condition $\delta^2 \mathcal F_u \geq 0$ is trivial if $f$ is convex.
So, for the Dirichlet problem with a convex integrand, the condition on the second variation is trivially satisfied.

Now we need a fact which was shocking to mathematicians two centuries ago, but now is essentially a triviality.

\begin{theorem}[fundamental lemma of the calculus of variations]
For every $u \in L^1_{loc}(\Omega)$ such that for every $\varphi \in C^\infty_c(\Omega)$, $\int_\Omega u\varphi = 0$, $u = 0$ almost everywhere.
\end{theorem}

We will also use the divergence theorem
$$\int_\Omega \nabla \cdot X = \int_{\partial \Omega} X \cdot \nu_\Omega$$
a lot. I just put the fundamental lemma in a separate display so that I could say ``(fundamental lemma of the calculus of variations)".

To apply these two theorems, say
$$0 = \int_\Omega \nabla f(\nabla u(x)) \cdot \nabla \varphi(x) = \int_\Omega X \cdot \nabla \varphi$$
where $X = \nabla f \circ \nabla u$.
Since
$$\nabla \cdot (\varphi X) = X \cdot \nabla \varphi + \varphi \nabla \cdot X$$
we get
$$0 = \int_\Omega \nabla \cdot (\varphi X) - \int_\Omega \varphi \nabla \cdot X = \int_{\partial \Omega} \varphi X \cdot \nu_\Omega - \int_\Omega \varphi \nabla \cdot X.$$
But $\varphi = 0$ near $\partial \Omega$ so we get
$$\int_\Omega \varphi \nabla \cdot X = 0.$$
Since $\varphi$ was arbitrary, the fundamental lemma gives $\nabla \cdot X = 0$ almost everywhere, but $X \in C^1$ so this just means that $X$ is divergence-free.
We conclude:

\begin{theorem}[Euler-Lagrange]
If $u$ is a minimizer of an energy with integrand $f$ and everything is smooth then
$$\nabla \cdot (\nabla f(\nabla u(x))) = 0.$$
\end{theorem}

\begin{example}
Consider the problem of minimizing the area functional of $u$ subject to $u = 0$ on $\partial \Omega$ and $\int_\Omega |u(x)| ~dx = M$, thus the volume under the graph of $|u|$ is equal to $M$.
This is a problem of isoperimetric type, which may not have a solution.
Indeed if $M \geq |B_R|/2$ and $\Omega = B_R$ then there is no solution, since the fact that $\Omega$ is an $n$-ball implies that the graph of $u$ is an $n$-sphere by the isoperimetric inequality, and we have an upper bound on $M$, which must be the volume of an $n+1$-ball.
\end{example}

\begin{example}
Consider the problem of minimizing the Dirichlet energy of $u$ subject to $|u|^2/2 = M$ and $u|\partial \Omega = 0$.
Suppose $\varphi$ satisfies for every $t$
$$\int_\Omega g(u + t\varphi) = M.$$
Then a Taylor series computation suggests that we want $\int_\Omega g'(u(x))\varphi(x) ~dx = 0$.

Let $\varphi \in C^\infty(\overline \Omega)$ satisfy $\int_\Omega g'(u(x)) \varphi(x) ~dx = 0$.
Let $\zeta \in C^\infty(\Omega)$ satisfy $g'(u(x)) \zeta(x) ~dx = 1$.
Then
$$u + t\varphi + s\zeta \in C^\infty(\Omega)$$
is a ``double-parametrized variation".
According to the implicit function theorem, there is $\varepsilon > 0$ and $s(t)$ defined if $|t| < \varepsilon$ such that if we set
$$u_t = u + t\varphi + s(t)\zeta$$
then
$\int_\Omega g(u_t) = M$ and $s(0) = 0$.
We get
\begin{align*}
\frac{d}{dt} g(u_t) &= g'(u_t)(\varphi + s'(t)\zeta)\\
\frac{d^2}{dt^2} g(u_t) &= g''(u_t)(\varphi + s'(t)\zeta)^2 + g'(u_t)(s''(t)\zeta).
\end{align*}
Plugging in $t = 0$ we get $s'(0) = 0$ and $s''(0) = -\int_\Omega g''(u(x))\varphi(x)^2 ~dx$.

Now let $\psi \in C^\infty(\overline \Omega)$ and
$$\varphi = \psi - \frac{\int_\Omega g'(u)\psi}{\int_\Omega g'(u)^2}g'(u).$$
Then $\varphi$ is a projection to $g'(u)$ so $\int_\Omega g'(u)\varphi = 0$.
Let $\zeta = g'(u)/\int_\Omega g'(u)^2$, then $\int_\Omega g'(u)\zeta = 1$.

If $u$ is a local minimum then $\mathcal F(u_t) = 0$. Since $s$ has a double zero at $0$ we can discard it when we take the integral form of the Euler-Lagrange equations, namely we get
$$\int_\Omega \nabla f(\nabla u(x)) \cdot \nabla \varphi(x) ~dx = 0.$$
However, at the second variation we get an additional term, namely
$$\int_\Omega \nabla \varphi(x) \cdot (\nabla^2 f(\nabla u(x))\nabla\varphi(x)) + s''(0)\nabla f(\nabla u(x)) \cdot \nabla(\zeta(x)) ~dx \geq 0.$$
We get a Lagrange multiplier
$$\lambda(u) = \frac{\int_\Omega g''(u)(\nabla u \cdot \nabla f(\nabla u))}{\int_\Omega g'(u)^2}$$
which satisfies
$$0 = \int_\Omega \nabla f(\nabla u) \nabla \psi - \lambda(u)\int_Omega g'(u)\psi.$$
Calling $X = \nabla f(\nabla u)$ and using Stokes' theorem,
$$\int_\Omega X \cdot \nabla \psi = \int_{\partial \Omega} \psi(X \cdot \nu_\Omega) - \int_\Omega \psi \nabla \cdot X.$$
Then
$$0 = \int_{\partial \Omega} \psi \nu_\Omega \cdot \nabla f(\nabla u) + \int_\Omega \psi \cdot 0$$
using the Lagrange multiplier.
Since $\psi$ was arbitrary we get the Neumann problem
\begin{align*}
\lambda g'(u) + \nabla \cdot(\nabla f(\nabla u)) &= 0\\
\nu_\Omega \cdot \nabla f(\nabla u) &= 0.
\end{align*}
Simplifying we get
\begin{align*}
-\Delta u &= \lambda u\\
\nabla u \cdot \nu_\Omega &= 0.
\end{align*}
Thus the minimizers are the Neumann eigenvectors of the Laplacian!
\end{example}

\begin{example}
By a similar argument, if $f$ is the area functional, then we get a Neumann condition
$$\frac{\nabla u \cdot \nu_\Omega}{\sqrt{1 + |\nabla u|^2}} = 0.$$
Here the denominator is always positive so we can replace it with the classical Neumann condition $\nabla u \cdot \nu_\Omega = 0$, but this is not typically the case.
\end{example}

\section{Existence of minimizers}
At this point we have developed lots of useful properties of minimizers, but do not yet know that minimizers exist.

Consider the problem of minimizing $\mathcal F(u)$ subject to $u \in C^1(\overline \Omega)$, $u|\partial \Omega = u_0$, and $\int_\Omega g(u) = M$. The competition class is denoted $\mathcal A$.
This problem can be formulated classically, without the Lebesgue integral or the Sobolev trace theorem, so we avoid a lot of stupid technicalities.
The \dfn{direct method} is a technique to find a minimizer, without referring to the Euler-Lagrange equations.

The direct method has the following steps:
\begin{enumerate}
\item Show $\mathcal A$ is nonempty and $\inf_{\mathcal A} \mathcal F > -\infty$. Therefore there exists a minimizing sequence $(u_j)$.
\item Use a compactness argument to pass to a subsequence of $(u_j)$ which converges to some $u$ in a suitably weak function space $X$.
\item Show that $\mathcal F$ is lower semicontinuous in $X$, so that
$$\mathcal F\left(\lim_{j \to \infty} v_j\right) \leq \liminf_{j \to \infty} \mathcal F(v_j).$$
\item Show that $u \in \mathcal A$.
\end{enumerate}
If these steps hold then we conclude
$$\inf_{\mathcal A} \mathcal F \leq \mathcal F(u) \leq \liminf_{j \to \infty} \mathcal F(u_j) = \inf_{\mathcal A} \mathcal F.$$
Therefore $u$ is a minimizer of $\mathcal F$ subject to $\mathcal A$.

\begin{example}
Let
$$\mathcal F(u) = \int_0^1 u^2 + ((u')^2 - 1)^2.$$
We minimize $\mathcal F(u)$ subject to $u(0) = u(1) = 0$.
Notice that $\mathcal F(0) = 1$ but if $\mathcal F(u) = 0$ then $u = 0$, contradiction.
So $\inf \mathcal F \geq 0$. In fact $\inf \mathcal F = 0$, using functions $u_j$ with $u_j' = 1$ on $[0, 1/j]$, $u_j' = -1$ on $[1/j, 2/j]$, $u_j' = 1$ on $[2/j, 3/j]$, etc.
These functions give a minimizing sequence which does not converge.
\end{example}

\begin{example}
Let us minimize the Dirichlet energy of $u \in C^1(\RR^n)$ subject to $||u||_{L^q(\RR^n)} = 1$, where $1 \leq q < \infty$.
If $\mathcal F(u) = 0$ then $u$ is a constant so $||u||_{L^q} \neq 1$.
Assume $n \leq 2$ or $q \neq 2n/(n-2)$.
The rescaling $u_\lambda(x) = \lambda^{n/q} u(\lambda x)$ preserves $||u||_{L^q}$.
But
$$\int_{\RR^n} |\nabla u_\lambda|^2 = \lambda^{2n/q + 2} \int_{\RR^n} |\nabla u(\lambda x)|^2 ~dx = \lambda^{2(n/q + 1) - n} \int_{\RR^n} |\nabla u|^2.$$
By taking $\lambda$ to a boundary point of $[0, \infty]$ we can send the energy to $0$.
But at $\lambda = 0$ or $\lambda = \infty$ we destroy $||u||_{L^q}$ so we get a minimizing sequence that does not converge.
\end{example}

Now for the rest of the section, suppose we want to minimize the Dirichlet energy subject to $u|\partial \Omega = u_0|\partial \Omega$ and $u_0 \in C^1(\overline \Omega)$.
Let $(u_j)$ be a minimizing sequence, which necessarily satisfies
$$\sup_j \int_\Omega |\nabla u_j|^2 < \infty.$$
By the Banach-Alaoglu theorem, $(\nabla u_j)$ converges weakly.
This does NOT imply that $(u_j)$ converges weakly a priori, but we have fixed the boundary data $u_0$.
Since we have a bound on the Dirichlet energy (which bounds the oscillation of $u$) we expect a bound on $||u_j||_{L^2}$.
Indeed, we can use the Sobolev trace theorem to fix this.

\begin{theorem}[Sobolev trace theorem]
If $\Omega$ is bounded and $\partial \Omega$ is a smooth manifold then
$$\int_\Omega |u|^2 \lesssim_{\diam \Omega, n} \int_{\partial \Omega} |\partial u|^2 + \int_\Omega |\nabla u|^2.$$
\end{theorem}
\begin{proof}
Let $x_0 \in \RR^n$, then
\begin{align*}
n\int_\Omega u^2 &= \int_\Omega u(x)^2 \nabla \cdot(x - x_0) ~dx \\
&= \int_\Omega \nabla \cdot((x - x_0)u^2(x)) ~dx - 2\int_\Omega u(x) \nabla u(x) \cdot (x - x_0) ~dx\\
&\leq \int_{\partial \Omega} u(x)^2 (x - x_0) \cdot \nu_\Omega + 2\int_\Omega |u(x)||\nabla u(x)||x - x_0| ~dx\\
&\leq 2 \diam \Omega \left(\int_{\partial \Omega} u^2 + 2\int_\Omega |u||\nabla u|\right)\\
&\leq 2 \diam \Omega \int_{\partial \Omega} u^2 + 2\diam \Omega \left(\varepsilon \int_\Omega u^2 + \frac{1}{\varepsilon} \int_\Omega |\nabla u|^2\right).
\end{align*}
Therefore
$$(n - 2\varepsilon\diam \Omega) \int_\Omega u^2 \leq 2 \diam \Omega \left(\int_{\partial \Omega} u^2 + \frac{1}{\varepsilon} \int_\Omega |\nabla u|^2\right)$$
so if we take $\varepsilon = n/(4 \diam \Omega)$ we win.
\end{proof}

The above proof is motivated by finding the constant which makes the units all match up.
Any good proof in analysis has this property.

By the trace inequality, $\{u_j\}$ is bounded in $L^2$, so by the Banach-Alaoglu theorem, $u_j \to u$ in the weak topology of $H^1$.
It remains to show
$$||\nabla u||_{L^2}^2 \leq \liminf_{j \to \infty} ||\nabla u_j||_{L^2}^2.$$
Let $T \in L^2(\Omega \to \RR^n)$; then
$$\int_\Omega (\nabla u) \cdot T = \lim_{j \to \infty} \int_\Omega (\nabla u_j) \cdot T \leq \lim_{j \to \infty} ||\nabla u_j||_{L^2} ||T||_{L^2}.$$
Now pick $T = \nabla u/||\nabla u||_{L^2}$. Then we get
$$||\nabla u||_{L^2} \leq \lim_{j \to \infty} ||\nabla u_j||_{L^2}.$$
Summing up, the direct method gives us a minimizer but no control on its regularity except $H^1$.

\begin{theorem}[Serrin 1961]
If $(u_j)$ is a sequence in $L^1_{loc}$ which converges to $u \in L^1_{loc}$, and $\nabla u_j,\nabla u \in L^1$, then for every convex function $f$,
$$\int_\Omega f(\nabla u(x)) ~dx \leq \liminf_{j \to \infty} \int_\Omega f(\nabla u_j(x)) ~dx.$$
\end{theorem}



\section{Hilbert's 19th problem}
Hilbert's 19th problem asks us to show that solutions of ``well-behaved" variational PDE always have analytic solutions.
This is a generalization of the fact that solutions of the Laplace equation are always analytic.

\begin{definition}
$C^\omega$ is the space of functions that are locally representatable by power series.
\end{definition}

It's not just the Laplace equation that has solutions in $C^\omega$. The steady states of the mean-curvature flow are also $C^\omega$, for example; these are minimizers of the area functional.
So, in particular, soap films are $C^\omega$.

Hilbert observed that equations with $C^\omega$ solutions tended to be Euler-Lagrange equations for the energy
$$E = \int_\Omega F(\nabla u(x)) ~dx$$
where $F$ has extremely high regularity:

\begin{definition}
We say that $F$ has \dfn{extremely high regularity} if $F \in C^\omega$ is convex and $\det(\nabla^2F) > 0$ .
\end{definition}

\begin{example}
If $F(x) = |x|^2$ then we get the Laplace equation, if $F(x) = \sqrt{1 + |x|^2}$ we get the minimal-surface equation.
Both of these examples have extremely high regularity.
The Euler-Lagrange equations can be written as
$$\nabla \cdot (\nabla F(\nabla u(x))) = \sum_{i,j} \partial_i \partial_j F(\nabla u(x)) \cdot \partial_i \partial_j u(x) = 0.$$
\end{example}

\begin{conjecture}[Hilbert's 19th problem]
If $F$ has extremely high regularity and $u$ solves the Euler-Lagrange equations induced by $F$ with Dirichlet data then $u \in C^\omega$.
\end{conjecture}

Here we do NOT assume that the Dirichlet data is $C^\omega$!
Hilbert did not define what he meant by a solution of the Euler-Lagrange equations, so we need to define that.

Hilbert didn't know what a weak solution is, so Bernstein made some assumptions to guarantee that the solution had enough regularity to be a strong solution.

\begin{theorem}[Bernstein 1904]
If $n = 2$ and we have the a priori regularity $u \in C^3$, then $u \in C^\omega$.
\end{theorem}

\begin{definition}
A \dfn{weak solution of the Euler-Lagrange equations} is a Lipschitz function $u$ such that
$$\int_\Omega \nabla F(\nabla u(x)) \cdot \nabla \psi(x) ~dx = 0$$
whenever $\psi \in C^\infty_c$.
\end{definition}

Here $\nabla F(\nabla u(x))$ is well-defined by Rademacher's theorem.
If $u$ is Lipschitz, then $u$ is a weak solution iff $u$ is a minimizer of the energy
$$J = \int_\Omega F(\nabla u(x)) ~dx$$
in the class of Lipschitz functions subject to some Dirichlet constraint.

\begin{theorem}[Leray-Hopf, 1930s]
If we have the a priori regularity $u \in C^{1 + \alpha}$, $\alpha > 0$, then $u \in C^\omega$.
\end{theorem}
\begin{proof}[Proof that $u \in C^\infty$]
Let us write the Euler-Lagrange equations in nondivergence form
$$F_{ij}(\nabla u(x)) u_{ij}(x) = 0$$
where $g_{ij} = \partial_i\partial_jg$ and we use Einstein notation.
Suppose that $u \in C^{1+\alpha}$, so $F_{ij} \circ \nabla u \in C^\alpha$.
Thus $\nabla^2 u$ solves an elliptic equation with coefficients in $C^\alpha$.
By the interior Schauder estimates, $\nabla^2 u \in C^\alpha$.
That implies that $F_{ij} \circ \nabla u \in C^{1 + \alpha}$, so $u \in C^{3 + \alpha}$.
Iterating, we get $u \in C^\infty$.
\end{proof}

To extend $u \in C^\omega$, we can either make careful inductive estimate on the higher derivatives of $u$ (the approach of Bernstein) or show that smooth solutions of the Euler-Lagrange equations extend to \emph{holomorphic} solutions of a complex-analytic analogue of the variational calculus (the approach of Hopf and Leray).
We omit the details.

The trouble is that, a priori, we may not have solutions in $C^{1 + \alpha}$, and Leray and Hopf did not know how to prove that.
We need to solve the Dirichlet problem for the equations
$$\nabla \cdot (\nabla F(\nabla u(x))) = 0$$
subject to the constraint $u \in C^{1 + \alpha}$, but by the 1930s one could only show that if the Dirichlet data was $C^2$ then $u$ was Lipschitz, i.e. $u \in C^L$.
Thus we just need an estimate on $||\nabla u||_{C^\alpha}$.

Our goal in this chapter is the following:
\begin{theorem}
Suppose that $u \in C^\infty(B_1)$ is a solution to the Euler-Lagrange equations.
Then there exists $\alpha$ such that
$$||\nabla u||_{C^\alpha(B_{1/2})} \lesssim 1$$
where $\alpha$ and the implied constant are allowed to depend on $n$, $||\nabla u||_{L^\infty}$, and $F$.
\end{theorem}
One can then replace $1/2$ with $1-\varepsilon$ and deduce $u \in C^{1 + \alpha}_{loc}$.

More precisely, if $a_{ij}(x)$ is a matrix with eigenvalues in the interval $[\lambda^{-1}, \lambda]$,
$$\partial_i (a_{ij}(x) \partial_j v(x)) = 0$$
then we have
$$||v||_{C^\alpha(B_{1/2})} \lesssim_{n, \lambda} ||v||_{L^\infty(B_1)}.$$
Then we set $v = \partial_k u$.
This estimate was obtained by Morrey if $n = 2$ using complex analysis in the 1930s.
Twenty years de Giorgi and Nash proved this independently.
This is considered the solution of the 19th problem.
We will discuss Morrey's proof.

\section{Morrey's interior estimate}
Let $u \in C^\infty(B_1)$, $B_1 \subset \RR^2$.
Suppose that $u$ solves the Euler-Lagrange equations in divergence form,
$$\nabla \cdot \nabla F(\nabla u(x)) = 0.$$
We will see that $u$ is analytic provided that the following theorem holds.

\begin{theorem}[Morrey]
Suppose that $v \in W^{1,p}(B_1)$, $1 < p < \infty$, and
\begin{equation}
\label{Linear Morrey}
\partial_i(a_{ij}\partial_jv) = 0
\end{equation}
in $B_1 \subset \RR^2$ where $a$ is a symmetric matrix-valued function with eigenvalues in $[\lambda, \lambda^{-1}]$ where $\lambda > 0$.
Then
$$||v||_{C^\alpha(B_{1/2})} \lesssim_\lambda ||v||_{L^\infty(B_1)}.$$
\end{theorem}

Since (\ref{Linear Morrey}) is an elliptic equation we start by mimicking the theory of the Laplace equation.

\begin{lemma}[variational characterization]
Solutions $v$ of (\ref{Linear Morrey}) minimize the quadratic form
$$E(v) = \int_{B_1} a_{ij}(x)v_i(x)v_j(x) ~dx.$$
\end{lemma}
\begin{proof}
Let $\psi$ be a test function. Then
$$E(v + \psi) = E(v) + E(\psi) + 2\int_{B_1} a_{ij}(x)v_j(x)\psi_i(x) ~dx.$$
We have $E(\psi) \geq 0$. Meanwhile,
$$2\int_{B_1} a_{ij}(x)v_j(x)\psi_i(x) ~dx = -\int_{B_1} \partial_i(a_{ij}v_j)\psi = 0$$
by (\ref{Linear Morrey}).
Thus we conclude $E(v + \psi) \geq E(v)$.
\end{proof}

\begin{lemma}[maximum principle]
Solutions $v$ of (\ref{Linear Morrey}) satisfy
$$\sup_\Omega v = \sup_{\partial \Omega} v$$
whenever $\Omega \subseteq B_1$.
\end{lemma}
\begin{proof}
If not, $v$ attains a maximum in $\Omega$, say at $x$.
Then we can find $\varepsilon$ so that
$$\sup_\Omega v - \varepsilon > \sup_{\partial \Omega} v.$$
Replacing $v$ by $\min(v, \sup_\Omega v - \varepsilon)$, on $v > \sup_\Omega v - \varepsilon$ there is no energy.
Thus this competitor to $v$ has smaller energy.
\end{proof}

Suppose that $v$ is a subsolution in the sense that
$$\partial_i(a_{ij}v_j) \geq 0.$$
Then $E(v + \psi) \geq E(v)$ provided that $\psi \leq 0$.
The proof is the same.
In particular, one has a one-sided maximum principle (one can't replace $v$ with $-v$ to get a minimum principle).

\begin{theorem}[Courant-Lebesgue lemma]
If $w: B_1 \to \RR$, $n = 2$, and $w$ satisfies the oscillating maximum principle in the sense that
$$\osc_{\partial B_s} w = \sup_{\partial B_s} w - \inf_{\partial B_s} w$$
is nondecreasing in $s$, then for every $0 < r < 1/2$,
$$\left(\osc_{\partial B_r} w\right)^2 \leq \frac{\pi}{\log(2/r)} \int_{B_{1/2}} |\nabla w|^2.$$
\end{theorem}
\begin{proof}
We have
$$\osc_{\partial B_s} w \leq \int_\gamma |\nabla w|$$
where $\gamma$ is some half-circle of radius $s$. By Cauchy-Schwarz, we get
$$\int_\gamma |\nabla w| \leq (\pi s)^{1/2} \sqrt{\int_{\partial B_s} |\nabla w|^2}.$$
Squaring both sides and dividing by $s$, we get
$$\frac{1}{s} \left(\osc_{\partial B_s} w \right)^2 \leq \pi \int_{\partial B_s} |\nabla w|^2.$$
Integrating in $s \in [r, 1/2]$, we get
$$\left(\osc_{\partial B_r} w\right)^2 \log(2/r) \leq \pi \int_{B_{1/2}} |\nabla w|^2.$$
\end{proof}

The Courant-Lebesgue lemma is false if $n \geq 3$.
This is because we have an ``almost Sobolev embedding" $W^{1,2} \to C^0$ if $n = 2$, but this is actually false.
However, the Courant-Lebesgue lemma says that for functions in $W^{1,2}$ which satisfy the maximum principle then we actually get a map to $C^0$.

\begin{theorem}[Caccoppoli inequality]
If $v$ solves (\ref{Linear Morrey}) then for every $\psi \in C^1_0(B_1)$,
$$\int_{B_1} |\nabla v|^2 \psi^2 \leq \frac{4}{\lambda^4} \int_{B_1} v^2 |\nabla \psi|^2.$$
\end{theorem}
\begin{proof}
Since $v$ is a critical point of $E$,
$$\lim_{\varepsilon \to 0} \frac{E(v) - E(v - \varepsilon v\psi^2)}{\varepsilon} = 0.$$
Also
$$\lim_{\varepsilon \to 0}  \frac{E(v) - E(v - \varepsilon v\psi^2)}{\varepsilon} = \int_{B_1} a_{ij} v_i\partial_j(v\psi^2),$$
so that is equal to $0$.
Thus
$$\lambda \int_{B_1} |\nabla v|^2 \psi^2 \leq \int_{B_1} a_{ij} v_i v_j \psi^2 = -2\int_{B_1} a_{ij}v_i\psi_j v\psi|.$$
By Cauchy-Schwarz,
$$-2\int_{B_1} a_{ij}v_i\psi_j v\psi| \leq \frac{2}{\lambda} \int_{B_1} |\nabla v||\psi||v\nabla \psi \leq \frac{2}{\lambda} \sqrt{\int_{B_1} |\nabla v|^2 \psi^2} \sqrt{\int_{B_1} v^2 |\nabla \psi|^2}.$$
So
$$\sqrt{\int_{B_1} |\nabla v|^2 \psi^2} \leq \frac{2}{\lambda^2} \sqrt{\int_{B_1} v^2 |\nabla \psi|^2}.$$
\end{proof}
This inequality is sort of a backwards Poincar\'e inequality.
It can be modified to hold for subsolutions.

\begin{proof}[Proof of Morrey's interior estimate]
If $v$ solves (\ref{Linear Morrey}) then there is $\delta > 0$ such that
\begin{equation}
\label{iterable morrey}
\osc_{B_{\delta r}} v \leq \frac{1}{2} \osc_{B_r} v
\end{equation}
provided that $r \leq 1$.
Iterating with $r = 1$ as initial datum,
$$\osc_{B_{\delta^k}} v \leq 2^{-k} \osc_{B_1} v$$
so we set $\delta^\alpha = 1/2$ so that we get
$$\osc_{B_{\delta^k}} v \leq \delta^{k\alpha} \osc_{B_1} v.$$

To prove (\ref{iterable morrey}), we rescale so $r = 1$ and $\osc_{B_1} v = 1$.
This can be accomplished by replacing $v$ with $x \mapsto v(rx)/\osc_{B_r} v$, following (\ref{Linear Morrey}).
Then we rescale by constants so $||v||_{L^\infty} \leq \osc_{B_1} \leq 1$.
Then
$$\left(\osc_{B_\delta} v\right)^2 = \left(\osc_{\partial B_\delta} v\right)^2$$
by the maximum principle, and by the Courant-Lebesgue lemma,
$$\left(\osc_{\partial B_\delta} v\right)^2 \leq \frac{\pi}{\log(1/2\delta)} \int_{B_{1/2}} |\nabla v|^2$$
and by the Caccoppoli inequality with $\psi$ a cutoff to $B_{1/2}$,
$$\frac{\pi}{\log(1/2\delta)} \int_{B_{1/2}} |\nabla v|^2 \lesssim_\lambda \frac{1}{\log(1/2\delta)} \int_{B_1} v^2$$
using our estimate on $||v||_{L^\infty}$. If $\delta$ is small enough this is $<1/2$.
\end{proof}

The idea is that if $n = 2$, a function that satisfies the maximum principle cannot oscillate a lot on small scales without having a lot of energy, so we get a H\"older estimate.





\newpage
\printindex
\printbibliography

\end{document}
