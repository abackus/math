
\documentclass[12pt]{book}
\usepackage[utf8]{inputenc}
\usepackage[margin=1in]{geometry}
\usepackage{amsmath,amsthm,amssymb}
\usepackage{mathrsfs}

\usepackage{enumitem}
%\usepackage[shortlabels]{enumerate}
\usepackage{tikz-cd}
\usepackage{mathtools}
\usepackage{amsfonts}
\usepackage{amscd}
\usepackage{makeidx}
\usepackage{enumitem}
\title{Harmonic analysis notes}
\author{Aidan Backus}
\date{2021}


\newcommand{\NN}{\mathbb{N}}
\newcommand{\ZZ}{\mathbb{Z}}
\newcommand{\QQ}{\mathbb{Q}}
\newcommand{\RR}{\mathbb{R}}
\newcommand{\CC}{\mathbb{C}}
\newcommand{\PP}{\mathbb{P}}
\newcommand{\DD}{\mathbb{D}}

\newcommand{\Torus}{\mathbb{T}}

\newcommand{\AAA}{\mathcal A}
\newcommand{\BB}{\mathcal B}
\newcommand{\HH}{\mathcal H}

\newcommand{\Grp}{\mathbf{Grp}}
\newcommand{\Open}{\mathbf{Open}}
\newcommand{\Vect}{\mathbf{Vect}}
\newcommand{\Set}{\mathbf{Set}}

\newcommand{\Cau}{\mathbf{Cau}}
\newcommand{\ISF}{\mathbf{ISF}}
\newcommand{\Simp}{\mathbf{Simp}}
\newcommand{\Sch}{\mathscr S}

\DeclareMathOperator{\atanh}{atanh}
\DeclareMathOperator{\sech}{sech}
\DeclareMathOperator{\sinc}{sinc}
\DeclareMathOperator{\dom}{dom}

\DeclareMathOperator{\card}{card}
\DeclareMathOperator{\coker}{coker}
\DeclareMathOperator{\diag}{diag}
\DeclareMathOperator{\Gal}{Gal}
\DeclareMathOperator{\Hom}{Hom}
\DeclareMathOperator{\id}{id}
\DeclareMathOperator{\sgn}{sgn}
\DeclareMathOperator{\supp}{supp}
\DeclareMathOperator{\rank}{rank}
\DeclareMathOperator{\rad}{rad}

\newcommand{\dbar}{\overline\partial}

\def\Xint#1{\mathchoice
{\XXint\displaystyle\textstyle{#1}}%
{\XXint\textstyle\scriptstyle{#1}}%
{\XXint\scriptstyle\scriptscriptstyle{#1}}%
{\XXint\scriptscriptstyle\scriptscriptstyle{#1}}%
\!\int}
\def\XXint#1#2#3{{\setbox0=\hbox{$#1{#2#3}{\int}$ }
\vcenter{\hbox{$#2#3$ }}\kern-.6\wd0}}
\def\ddashint{\Xint=}
\def\dashint{\Xint-}

\renewcommand{\Re}{\operatorname{Re}}
\renewcommand{\Im}{\operatorname{Im}}
\newcommand{\dfn}[1]{\emph{#1}\index{#1}}

\usepackage[backend=bibtex,style=alphabetic,maxcitenames=50,maxnames=50]{biblatex}
\renewbibmacro{in:}{}
\DeclareFieldFormat{pages}{#1}

\usepackage{color}
\usepackage{hyperref}
\hypersetup{
    colorlinks=true, % make the links colored
    linkcolor=blue, % color TOC links in blue
    urlcolor=red, % color URLs in red
    linktoc=all % 'all' will create links for everything in the TOC
    %Ning added hyperlinks to the table of contents 6/17/19
}

\theoremstyle{definition}
\newtheorem{theorem}{Theorem}[chapter]
\newtheorem{lemma}[theorem]{Lemma}
\newtheorem{sublemma}[theorem]{Sublemma}
\newtheorem{proposition}[theorem]{Proposition}
\newtheorem{corollary}[theorem]{Corollary}
\newtheorem{axiomx}[theorem]{Axiom}
\newtheorem{theoremxx}[theorem]{Theorem}
\newtheorem{conjecture}[theorem]{Conjecture}
\newtheorem{definitionx}[theorem]{Definition}
\newtheorem{remark}[theorem]{Remark}
\newtheorem{examplex}[theorem]{Example}
\newtheorem{exercisex}{Exercise}[chapter]
\newtheorem{problem}[theorem]{Problem}

\newenvironment{axiom}
  {\pushQED{\qed}\renewcommand{\qedsymbol}{$\diamondsuit$}\axiomx}
  {\popQED\endexamplex}

\newenvironment{definition}
  {\pushQED{\qed}\renewcommand{\qedsymbol}{$\diamondsuit$}\definitionx}
  {\popQED\endexamplex}

\newenvironment{example}
  {\pushQED{\qed}\renewcommand{\qedsymbol}{$\diamondsuit$}\examplex}
  {\popQED\endexamplex}

  \newenvironment{exercise}
    {\pushQED{\qed}\renewcommand{\qedsymbol}{$\diamondsuit$}\exercisex}
    {\popQED\endexamplex}

  \newenvironment{theoremx}
        {\pushQED{\qed}\renewcommand{\qedsymbol}{$\diamondsuit$}\theoremxx}
        {\popQED\endexamplex}

\makeindex

\begin{document}

\maketitle

\tableofcontents

\chapter{Muckenhoupt weights}
This chapter is based on the Informal Analysis seminar at Brown as well as Chapter 5 of Stein's book on harmonic analysis.

Throughout we work in $\RR^d$ and let $|\cdot|$ denote Lebesgue measure on $\RR^d$.

Recall the definition of the maximal operator:
$$Mf(x) = \sup_{r > 0} \dashint_{B(x, r)} f(y) ~dy$$
whenever $f \geq 0$ is locally integrable and the dashed integral denotes a mean. Thus $Mf(x)$ dominates the means of $f$ taken on balls centered at $x$.
Then one has:
\begin{theorem}[Hardy-Littlewood maximal inequality]
Let $1 < p \leq \infty$. Then $M$ satisfies the strongtype $(p, p)$ inequality; that is,
$$||Mf||_{L^p} \lesssim_{p, d} ||f||_{L^p}.$$
\end{theorem}
\begin{proof}
The strongtype $(\infty, \infty)$ inequality is trivial, so we may assume $p < \infty$.
We first prove the weaktype $(1, 1)$ inequality
\begin{equation}
\label{weaktype 11 HL maximal}
\lambda |\{Mf > \lambda\}| \lesssim_d ||f||_{L^1}
\end{equation}
uniformly in $\lambda > 0$. If $Mf(x) > \lambda$ then by definition there is a ball $B_x$ centered at $x$ such that
$$\dashint_{B_x} |f(y)| ~dy > \lambda.$$
After running the Vitali covering algorithm, one can find a countable set $\mathcal B$ of balls such that
$$\{Mf > \lambda\} \subseteq \bigcup_{B \in \mathcal B} 5B.$$
Therefore
$$|\{Mf > \lambda\}| \leq 5^d \sum_{B \in \mathcal B} |B| \leq \frac{5^d}{\lambda} ||f||_{L^1}.$$
This proves (\ref{weaktype 11 HL maximal}). The general claim follows from Marcinkiewicz interpolation when we note that $M$ is obviously bounded on $L^\infty$.
\end{proof}

This motivates the following definition.
\begin{definition}
Fix $1 < p < \infty$.
A \dfn{Muckenhoupt measure} $\mu$ is a positive Borel measure such that
$$||Mf||_{L^p(\mu)} \lesssim_p ||f||_{L^p(\mu)}.$$
If $d\mu(x) = \omega(x) ~dx$, we call $\omega$ a \dfn{Muckenhoupt weight}.
\end{definition}
In fact, it will turn out that every Muckenhoupt measure is absolutely continuous, so by the Radon-Nikodym theorem, there is a canonical isomorphism between the spaces of Muckenhoupt weights and Muckenhoupt measures.
It is immediately true that $1$ is a Muckenhoupt weight.

\section{The space of weights}
We now consider the space of $A_p$ weights. This section follows Chapter V of Stein.

Throughout, fix $1 < p < \infty$ and let $p'$ denote the H\"older dual of $p$.

\begin{definition}
Let $\omega \in L^1_{loc}$ be nonnegative and nonzero. We define
$$A_p(\omega) = \sup_B \dashint_B \omega(x) ~dx \left(\dashint_B \omega(x)^{-p'/p} ~dx\right)^{p/p'}$$
where the supremum is taken over all balls $B$.
We let $A_p$ be the space of $\omega$ for which $A_p(\omega) < \infty$.
\end{definition}

Notice that if $A_p(\omega) < \infty$ then $\omega > 0$ almost everywhere. Indeed, in that case
$$\dashint_B \omega(x)^{-p'/p} ~dx < \infty$$
which forces $\{\omega = 0\}$ to have measure zero. It also follows straight from the definitions that
$$A_{p'}(\omega^{-p'/p})^{1/p'} = A_p(\omega)^{1/p}.$$

\begin{lemma}
If $p_1 \leq p_2$ then $A_{p_1} \subseteq A_{p_2}$.
\end{lemma}
\begin{proof}
Let $q_i = p_i'/p_i$; then $q_2 \leq q_1$. So
$$A_{p_2}(\omega) = \sup_B \dashint_B \omega \left(\dashint_B \omega^{-q_2}\right)^{q_2} \leq \sup_B \dashint_B \omega \left(\dashint_B \omega^{-q_1}\right)^{q_1} = A_{p_1}(\omega)$$
which was desired.
\end{proof}

For this reason we define $A_\infty = \bigcup_p A_p$, setting
\begin{equation}
\label{Ainfty definition}
A_\infty(\omega) = \inf_p A_p(\omega).
\end{equation}

\section{The reverse H\"older inequality}
This section follows Chapter V of Stein.

To prove the maximal inequality weighted by $A_p$ we first need:
\begin{theoremx}[reverse H\"older inequality]
For every $\omega \in A^\infty$ there is a $r > 1$ such that for all balls $B$,
$$\left(\dashint_B \omega(x)^r ~dx\right)^{1/r} \lesssim \dashint_B \omega(x) ~dx$$
where the constant depends on $\omega$ but not $B$.
\end{theoremx}
Before we prove the reverse H\"older inequality, we need a characterization of $A_\infty$.

We view $\omega$ as a measure whenever $\omega$ is a weight, thus
$$\omega(E) = \int_E \omega(x) ~dx$$
whenever $\omega \in L^1_{loc}$ is nonnegative and $E$ is a Borel set.

\begin{lemma}
\label{Ap is averaging}
$A_p(\omega)$ is the infimum of all constants $C > 0$ such that for every nonnegative $f \in L^1_{loc}$ and every ball $B$,
$$\left(\dashint_B f(x) ~dx\right)^p \leq \frac{C}{\omega(B)} \int_B f^p ~d\omega.$$
\end{lemma}
\begin{proof}
Suppose that $\omega \in A_p$ and $f, B$ are given. Then
$$\dashint_B f(x) ~dx = \dashint_B f(x) \omega(x)^{1/p} \omega(x)^{-1/p} ~dx,$$
so by H\"older's inequality,
$$\left(\dashint_B f(x) ~dx\right)^p \leq |B|^{-p} \int_B f^p ~d\omega \left(\omega(x)^{-p'/p}~dx\right)^{p/p'}.$$
On the other hand,
$$\left(\int_B \omega(x)^{-p'/p}~dx\right)^{p/p'} \leq \frac{A_p(\omega)}{\omega(B)},$$
so the claim follows with $C = A_p(\omega)$.

Conversely, supppose that there is a finite choice of $C$, let $\varepsilon > 0$, and let
$$f = (\omega + \varepsilon)^{-p'/p}.$$
Now $p' - 1 = p'/p$ by definition of $p'$, so
$$f \leq (\omega + \varepsilon)^{-p'}\omega.$$
Since we can take $C$ finite, we then have
$$\int_B f(x) ~dx \leq \int_B (\omega + \varepsilon)^{-p'} \omega \lesssim_\varepsilon \int_B \omega(x) ~dx < \infty$$
since $\omega \in L^1_{loc}$. Now
$$f^p\omega = (\omega + \varepsilon)^{-p'} \omega$$
which gives the bound
$$\dashint_B \omega(x) ~dx \left(\dashint_B (\omega + \varepsilon)^{-p'}\omega\right)^{p/p'} \leq C.$$
This bound is uniform in $\varepsilon$, so taking $\varepsilon \to 0$ we get $A_p(\omega) \leq C$.
\end{proof}

Write $kB$ for the dilation of a ball $B$ by a factor of $k > 0$.

\begin{definition}
A \dfn{doubling measure} is a Radon measure $\mu$ such that
$$\mu(2B) \lesssim \mu(B)$$
uniformly in balls $B$.
\end{definition}

\begin{lemma}
If $\omega \in A_p$, then $\omega$ is a doubling measure with doubling constant $A_p(\omega)2^{dp}$.
\end{lemma}
\begin{proof}
Apply Lemma \ref{Ap is averaging} on the ball $2B$ with $f = 1_B$, thus
$$\left(\dashint_{2B} 1_B(x) ~dx\right)^p \leq \frac{A_p(\omega)}{\omega(2B)} \int_{2B} 1_B ~d\omega.$$
This simplifies to
$$2^{dp} = \left(\frac{|B|}{|2B|}\right)^p \leq \frac{A_p(\omega)\omega(B)}{\omega(2B)}.$$
Solving for $\omega(B)/\omega(2B)$ we get the claim.
\end{proof}

\begin{definition}
Let $\omega$ be a weight such that for every $\alpha \in (0, 1)$ there is a $\beta \in (0, 1)$ such that for every ball $B$ and Borel set $F \subseteq B$ such that $|F| \geq \alpha|B|$,
$$\omega(F) \geq \beta \omega(B).$$
Then we say that $\omega$ is \dfn{fair}.
\end{definition}

The intuition is that $\omega$ does not favor any one region of a ball more than another.

\begin{lemma}
Suppose that $\omega \in A_\infty$. Then $\omega$ is fair.
\end{lemma}
\begin{proof}
Suppose that $\omega \in A_p$. Then by Lemma \ref{Ap is averaging} with $f = 1_F$ we have
$$\frac{|F|^p}{|B|^p} \leq A_p(\omega) \frac{\omega(F)}{\omega(B)}$$
which gives the claim with $\beta = \alpha^p/A_p(\omega)$.
\end{proof}

\begin{lemma}
If $\omega$ is a fair weight, then for every $1 < r \ll 2$,
\begin{equation}
\label{weakly fair weight}
\left(\dashint_Q \omega(x)^r ~dx\right)^{1/r} \lesssim \dashint_Q \omega(x) ~dx
\end{equation}
uniformly in cubes $Q$.
\end{lemma}
\begin{proof}
Let $Q_0 = Q$. After rescaling everything we may assume that $Q_0$ is a dyadic cube and
$$\omega(Q_0) = |Q_0| = 1.$$
In that case, we must obtain a uniform bound
\begin{equation}
\label{normalized weakly fair weight}
\int_{Q_0} \omega(x)^r ~dx \lesssim 1
\end{equation}
to deduce (\ref{weakly fair weight}).
Let $f = \omega 1_{Q_0}$.

We need a dyadic version of the Hardy-Littlewood maximal operator. Namely, we set
$$M^\Delta g(x) = \sup_{x \in Q} \dashint_Q g(y) ~dy$$
whenever $g \in L^1_{loc}$ is nonnegative, and $Q$ ranges over all dyadic cubes.
Note that by convention we take dyadic cubes as open, so the dyadic cubes of a given volume only tile almost all of $\RR^d$.

Let
$$E_k(N) = \{x \in Q_0: M^\Delta f(x) > 2^{Nk}\}.$$
Then, if $Q$ is a dyadic cube in $E_{k-1}(N)$, then
\begin{equation}
\label{bounds on Ek}
|E_k(N) \cap Q| \leq 2^{d - N} |Q|.
\end{equation}
To see this, let $R \subseteq Q$ be a maximal dyadic cube in $E_k(N)$. By the Calder\'on-Zygmund decomposition of $f$ (with $2^{Nk}$ viewed as the cutoff for ``$f$ large"), we have
$$|R| \leq 2^{-Nk} \int_R f(x) ~dx$$
so, summing over all such $R$,
$$|E_k(N) \cap Q| = \sum_R |R| \leq 2^{-Nk} \int_Q f(x) ~dx.$$
On the other hand, the Calder\'on-Zygmund decomposition of $f$ gives
$$\int_Q f(x) ~dx \leq 2^{d + N(k - 1)}$$
so (\ref{bounds on Ek}) holds.

Since $\omega$ is a fair weight, there is a $\beta \in (0, 1)$ such that if $|F| \leq |Q|/2$, then $\omega(F) \leq \beta \omega(Q)$.
By (\ref{bounds on Ek}), if $N$ is so large that $2^{d - N} \leq 1/2$, then
$$\omega(E_k(N) \cap Q) \leq \beta \omega(Q).$$
Taking the union over all dyadic cubes $Q$ in $E_{k-1}(N)$, we get
$$\omega(E_k(N)) \leq \beta \omega(E_{k-1}(N)).$$
But $\omega(E_1(N)) \leq \omega(Q_0) = 1$, so by induction,
$$\omega(E_k(N)) \leq \beta^k.$$
Since $\omega = f$ on $Q_0$ and $f \leq M^\Delta f$ we have $\omega^r \leq (M^\Delta f)^{r - 1}\omega$, which implies
$$\int_{Q_0} \omega(x)^r ~dx \leq \int_{Q_0 \cap \{M^\Delta f \leq 1\}} (M^\Delta f)^{r - 1} ~d\omega + \sum_{k=0}^\infty \int_{E_k(N) \setminus E_{k-1}(N)} (M^\Delta f)^{r - 1} ~d\omega.$$
Trivially,
$$\int_{Q_0 \cap \{M^\Delta f \leq 1\}} (M^\Delta f)^{r - 1} ~d\omega \leq 1.$$
By definition of $E_k(N)$,
$$\int_{E_k(N) \setminus E_{k-1}(N)} (M^\Delta f)^{r - 1} ~d\omega \leq 2^{N(k+1)(r - 1)} \omega(E_k(N)) \leq 2^{N(k+1)(r - 1)} \beta^k.$$
Thus we get a bound
$$\int_{Q_0} \omega(x)^r ~dx \leq 1 + \sum_{k=0}^\infty 2^{N(k+1)(r-1)} \beta^k$$
which does not depend on $Q_0$, at least as long as $1 < r \ll 2$.
\end{proof}

Now we are ready to prove the reverse H\"older inequality.
Since $\omega$ is a doubling measure and for any ball $B$ we can find a cube $Q$ with $B \subseteq Q \subseteq 2B$, we may replace balls with cubes in the statement of the reverse H\"older inequality, thus for every cube $Q$, we must bound
$$\left(\int_Q \omega(x)^r ~dx\right)^{1/r} \lesssim \dashint_Q \omega(x) ~dx$$
uniformly in $Q$.
This is just the content of (\ref{weakly fair weight}), which holds whenever $\omega$ is fair and $1 < r \ll 2$.

\section{Characterization of Muckenhoupt weights}
In this section we prove a characterization of Muckenhoupt weights.
This follows Chapter V of Stein still.

\begin{definition}
A \dfn{radially decreasing convolution kernel} $\Phi$ is a nonnegative radial function which is radially decreasing and satisfies $\int_{\RR^d} \Phi = 1$.
\end{definition}

Now if $\Phi$ is a radially decreasing kernel we write $\Phi_\varepsilon(x) = \varepsilon^{-d} \Phi(x/\varepsilon)$.
Thus $\Phi_\varepsilon$ is also a radially decreasing kernel, and as $\varepsilon \to 0$, $\Phi_\varepsilon$ approximates the convolution identity.

\begin{theoremx}
Let $\mu$ be a Radon measure and $1 < p < \infty$. Then the following are equivalent:
\begin{enumerate}
\item $\mu$ is a Muckenhoupt measure.
\item The inequality
\begin{equation}
\label{Teps is unif bounded}
||\Phi_\varepsilon * f||_{L^p(\mu)} \lesssim ||f||_{L^p(\mu)}
\end{equation}
is uniform in $\varepsilon > 0$, $\Phi$ a radially decreasing kernel, and $f \in L^1_{loc}$ nonnegative.
\item $\mu$ is absolutely continuous, and its Radon-Nikodym derivative $\omega$ satisfies $A_p(\omega) < \infty$.
\item $\mu$ is absolutely continuous and the inequality
\begin{equation}
\label{mean value Ap}
\left(\dashint_B f(x)~dx\right)^p \lesssim \frac{1}{\mu(B)} \int_B f^p ~d\mu
\end{equation}
is uniform in $f \in L^1_{loc}$ nonnegative and $B$ a ball.
\end{enumerate}
\end{theoremx}

We recall that we already showed that (3) and (4) are equivalent with implied constant $A_p(\omega)$.
The equivalence of (1) and (3) is what we really want, since it shows that $A_p$ is canonically isomorphic to the space of Muckenhoupt measures.

We first show that $\{p: A_p(\omega) < \infty\}$ is open.

\begin{lemma}
Suppose $A_p(\omega) < \infty$. Then there is a $q < p$ with $A_q(\omega) < \infty$.
\end{lemma}
\begin{proof}
Let $\sigma = \omega^{-p'/p}$. Then $\sigma \in A_{p'}$, so $\sigma$ satisfies the reverse H\"older inequality
$$\left(\dashint_B \sigma(x)^r ~dx\right)^{1/r} \lesssim \dashint_B \sigma(x) ~dx$$
for some $1 < r \ll 2$. Then there is a $1 < q < p$ such that $rp'/p = q'/q$, and by definition of $\sigma$ it follows that $A_q(\omega) < \infty$.
\end{proof}

\begin{lemma}
\label{Mf is rad dec}
One has
$$Mf(x) = \sup_\Phi |f| * \Phi(x)$$
where the supremum ranges over all radially decreasing kernels.
\end{lemma}
\begin{proof}
If $\Phi$ is a radially decreasing kernel, it is easy to check that $\Phi$ is a limit in $L^1$ of weighted averages of functions of the form $|B_r|^{-1}1_{B_r}$ where $B_r$ is the ball of radius $r > 0$ at $0$.
Thus we may restrict the supremum to be over $\Phi = |B_r|^{-1}1_{B_r}$, thus the claim is
$$Mf(x) = \sup_{r > 0} \frac{1}{|B_r|} |f| * 1_{B_r} = \sup_{r > 0} \frac{1}{|B_r|} \int_{\RR^d} f(y) 1_{B_r}(x - y) ~dy$$
which is obvious.
\end{proof}

\begin{lemma}
\label{muckenhoupt is abs cts}
If (\ref{Teps is unif bounded}) holds uniformly in $\varepsilon$ for $\Phi = |B_1|^{-1} 1_{B_1}$, then $\mu$ is absolutely continuous.
\end{lemma}
\begin{proof}
Let $K$ be a null compact set. We must show that $\mu(K) = 0$, so that $\mu = 0$.

Let $U_n = \{x: d(x, K) < 1/n\}$ and $f_n = 1_{U_n \setminus K}$. Then $(f_n)$ is a decreasing sequence and $f_n \to 0$ everywhere.
Since $\mu$ is a Radon measure, it is finite on compact sets, thus $f_n \in L^p(\mu)$. By dominated convergence, $||f_n||_{L^p(\mu)} \to 0$.
So by (\ref{Teps is unif bounded}),
$$\lim_{n \to \infty}||\Phi_\varepsilon * f_n||_{L^p(\mu)} = 0$$ uniformly in $\varepsilon$, and in particular
\begin{equation}
\label{fn is zero}
\lim_{n \to \infty} ||\Phi_{1/n} * f_n||_{L^p(\mu)} = 0.
\end{equation}

On the other hand, if $x \in K$, then
\begin{equation}
\label{fn is nonzero}
\Phi_{1/n} * f_n(x) = 1.
\end{equation}
Indeed,
$$\Phi_{1/n} * f_n(x) = \frac{1}{|B_{1/n}|} \int_{\RR^d} f_n(y) g_n(x - y) ~dy$$
where $g_n = 1_{B_{1/n}}$. Then
$$\int_{\RR^d} f_n(y) g_n(x - y) ~dy = \int_{U_n \setminus K} g_n(x - y) ~dy = \int_{\RR^d \setminus K} g_n(x - y) ~dy$$
by definition of $U_n$. Since $K$ is a null compact set, it follows that
$$\int_{\RR^d} f_n(y) g_n(x - y) ~dy = |B_{1/n}|.$$
Plugging this back in we get (\ref{fn is nonzero}).
The only way to avoid a contradiction between (\ref{fn is zero}) and (\ref{fn is nonzero}) is if $\mu(K) = 0$.
\end{proof}

Let us prove that (1) implies (3). Suppose that $\mu$ is a Muckenhoupt measure. Applying Lemmata \ref{Mf is rad dec} and \ref{muckenhoupt is abs cts}, and the Radon-Nikodym theorem, we see that $\mu$ has a Radon-Nikodym derivative $\omega$.
Let $f \in L^p(\mu)$ be nonnegative, thus
$$||Mf||_{L^p(\mu)} \lesssim ||f||_{L^p(\mu)}.$$
But
$$\dashint_B f(y) ~dy \lesssim Mf(x)$$
if $x \in B$, so
$$\left(\dashint_B f(y) ~dy\right)^p \lesssim Mf(x)^p,$$
we can integrate both sides in $\omega$ over $B$ to get
$$\left(\dashint_B f(y) ~dy\right)^p \lesssim \frac{1}{\omega(B)} \int_B Mf^p ~d\omega.$$
Since all integrals are taken over $B$ we may assume that $f$ is supported in $B$, whence
$$\int_B Mf^p ~d\omega \lesssim \int_B f^p ~d\omega$$
since $\omega$ is a Muckenhoupt weight. Therefore $A_p(\omega) < \infty$.

Let us now prove that (2) implies (4). By Lemma \ref{muckenhoupt is abs cts}, $\mu$ is absolutely continuous, so by the Radon-Nikodym theorem it is weighted by some $\omega$.
Let $f \geq 0$ and $B$ be given, and let $\delta = \rad B$. If $\Phi = 1_{B_1}$, $B_1$ the ball centered at $0$ of radius $1$, then
$$2^{-d} \dashint_B f \leq \dashint_{B(x, 2\delta)} f = T_\varepsilon f(x)$$
whenever $x \in B$. Taking $L^p$ norms of both sides and using (\ref{Teps is unif bounded}), we deduce (4).
We can drop the assumption that $\Phi = 1_{B_1}$ by first rescaling $\Phi$ so $\Phi = 1_{B_r}$ for some $r > 0$ and then replacing a general $\Phi$ with $\alpha 1_{B_r} \leq \Phi$ for some $\alpha > 0$ and $r > 0$, which exists since $\Phi$ is radially decreasing.
That proves (4) in the general case.

Now we show that (4) implies (2). Again we first consider $\Phi = 1_{B_1}$. In this case it suffices to show that
\begin{equation}
\label{4 implies 2 goal}
||\Phi * f||_{L^p(\omega)} \lesssim ||f||_{L^p(\omega)}
\end{equation}
where the implied constant is only allowed to depend on $A_p(\omega)$. Indeed, by dilation invariance, this will then prove the result for $\Phi_\varepsilon$ as well.
As usual we may assume that $f$ is nonnegative. If $x \in B_1$, then $B(x, 1) \subset B_2$, so
$$\Phi * f(x) \leq 2^d \dashint_{B_2} f.$$
Therefore
$$\int_{B_1} |\Phi * f|^p ~d\mu \leq 2^{dp} \left(\dashint_{B_2} f\right)^p \omega(B_1) \lesssim \int_{B_2} f^p ~d\mu$$
by (4). In other words,
$$\int_{\RR^d} |\Phi * f|^p 1_{B_1} ~d\mu \lesssim \int_{\RR^d} f^p 1_{B_2} ~d\mu.$$
By translation invariance of convolution with $\Phi$ and $A_p(\omega)$, we get
$$\int_{\RR^d} |\Phi * f(x)|^p 1_{B_1}(x - y) ~d\mu(x) \lesssim \int_{\RR^d} f(x)^p 1_{B_2}(x - y) ~d\mu(x)$$
uniformly in $y \in \RR^d$. Integrating in $y$ and using Fubini's theorem, we get (\ref{4 implies 2 goal}).
Using a limiting argument, we can drop the assumption that $\Phi = 1_{B_1}$ to get (2).

\begin{lemma}
\label{HL is weaktype in Ap}
The Hardy-Littlewood maximal operator $M$ is weaktype $(L^p(\mu), L^p(\mu))$ iff $d\mu(x) = \omega(x) ~dx$ for some $\omega$ with $A_p(\omega) < \infty$.
\end{lemma}
\begin{proof}
Suppose that $M$ is weaktype $(L^p(\mu), L^p(\mu))$, thus
\begin{equation}
\label{weaktype maximal operator}
\mu\{Mf > \alpha\} \lesssim \frac{1}{\alpha^p} \int_{\RR^d} |f|^p ~d\mu
\end{equation}
uniformly in $\alpha > 0$. Since $M$ is bounded on $L^\infty(\mu)$, by Marcinkiewicz interpolation, it follows that if $q > p$ then $M$ is bounded on $L^q(\mu)$.
In other words, (\ref{Teps is unif bounded}) holds with $p$ replaced by $q$, so by Lemma \ref{muckenhoupt is abs cts} and the Radon-Nikodym theorem we may write $d\mu(x) = \omega(x)~dx$.
If $f$ is nonnegative and supported in a ball $B$, $x \in B$, then as usual we have
$$\dashint_B f(y)~dy \leq 2^d Mf(x).$$
Plugging $\alpha = 2^{-d-1}\dashint_B f$ into (\ref{weaktype maximal operator}), we get
$$\left(\dashint_B f(y) ~dy\right)^p \omega(B) \lesssim 2^{(d+1)p} \int_B |f|^p ~d\mu$$
which implies that $\omega$ satisfies (4). Since (4) implies (3), it follows that $A_p(\omega) < \infty$.

Conversely, if $A_p(\omega) < \infty$, set
$$M_\omega f(x) = \sup_{\delta > 0} \dashint_{B(x, \delta)} |f| ~d\omega,$$
thus $M_\omega$ is the Hardy-Littlewood maximal operator induced by $\omega$.
Since $\omega$ is a doubling measure, the proof that the Hardy-Littlewood maximal operator is weaktype $(L^1(\omega), L^1(\omega))$ goes through, but with possibly a worse constant than $5$ in the Vitali covering algorithm.
That is,
$$\omega\{M_\omega g > \alpha\} \lesssim \frac{1}{\alpha} \int_{\RR^d} g~d\omega$$
uniformly in $\alpha > 0$. On the other hand, since (3) holds, so does (4), and thus
$$(Mf)^p \lesssim M_\omega|f|^p.$$
Plugging in $g = |f|^p$ and replacing $\alpha$ with $\alpha^p$ we deduce (\ref{weaktype maximal operator}).
\end{proof}

Let us finally show that (3) implies (1).
Suppose that $A_p(\omega) < \infty$, and let $1 < q < p$ be such that $A_q(\omega) < \infty$.
By Lemma \ref{HL is weaktype in Ap}, $M$ is weaktype $(L^q(\omega), L^q(\omega))$, and $M$ is clearly bounded on $L^\infty(\omega)$.
Therefore $M$ is bounded on $L^p(\omega)$, which implies (1).

\section{Sharp weighted bounds for sparse operators}
This section is based on a talk by Tai Borges, itself based on a paper of Kabe Moen.
The idea is to prove bounds for sparse operators in the context of $A_2$, and then conclude the same for Calder\'on-Zygmund operators.

Let $T$ be a Calder\'on-Zygmund operator. We are interested in when
$$||Tf||_{L^p(\omega)} \lesssim ||f||_{L^p(\omega)}$$
where $\omega \in L^1_{loc}$ is a weight and $1 < p < \infty$. In fact we will show
$$||T||_{L^p(\omega)} \lesssim_{p, T} [\omega]_p^{\max(1, p'/p)}.$$

\begin{definition}
A \dfn{sparse family}, for $\eta \in (0, 1)$, is a family $S$ of cubes such that for every $Q \in S$ there is $E_Q \subseteq Q$ such that $|E_Q| \geq \eta |Q|$ and $\{E_Q\}_Q$ is pairwise disjoint.
The \dfn{associated sparse operator} is
$$T^Sf(x) = \sum_{Q \in S} f_Q(x) 1_Q(x)$$
where $f_Q$ is the mean of $f$ over $Q$.
\end{definition}

\begin{theoremx}[Lacey-Hyt\"onen-Rocal-Tapiola]
Let $T$ be a Calder\'on-Zygmund operator. If $f \in L^1$ has compact support then there is a $(2\cdot 3^d)^{-1}$-sparse family $S$ such that
$$|Tf| \lesssim_d (||T||_{L^2} + C_K + ||\gamma||_{Dini}) T^S|f|.$$
\end{theoremx}

Now if $\omega \in A_p$ then
$$||Tf||_{L^p(\omega)} \lesssim_{T, d} ||T^S||_{L^p(\omega)} ||f||_{L^p(\omega)}$$
by the theorem of Lacey et al. So it suffices to bound
$$||T^S||_{L^p(\omega)} \lesssim_{d, \eta, p} [\omega]_p^{\max(1, p'/p)}$$
where $S$ is $\eta$-sparse.

\begin{definition}
A \dfn{dyadic grid} in $\RR^d$ is a collection $D$ of cubes such that if $Q \in D$ then $|Q| = 2^{nk}$ for some $k \in \ZZ$, the cubes are disjoint or dyadically contained in each other, and $\{Q \in D: |Q| = 2^{nk}\}$ partitions $\RR^d$.
\end{definition}

The sparse family from Lacey et al.'s theorem is not necessarily dyadic, but the three lattice theorem shows that there are $3^d$ dyadic lattices $D_j$ and sparse families $S_j \subseteq D_j$ with $||T^S|| \lesssim \sum_j ||T^{S_j}||$.

\begin{lemma}
Let
$$M_\omega^D f(x) = \sup_{x\in Q \in D} \dashint_Q |f|\omega$$
be the weighted dyadic Hardy-Littlewood maximal operator. Then
$$||M_\omega^Df||_{L^p(\omega)} \leq \frac{p}{p - 1} ||f||_{L^p(\omega)}.$$
\end{lemma}
\begin{proof}
Let $F \subset D$ be a finite set. Consider the restricted operator $M_\omega^F$ where $M_\omega^Ff = 0$ away from $\bigcup F$.
Let us show
$$||M^F_\omega||_{L^p(\omega)} \leq \frac{p}{p - 1};$$
by monotone convergence this suffices.
Let us show that
$$\omega\{M_\omega^Ff > \alpha\} \leq \frac{1}{\alpha} ||f||_{L^1(\omega)}.$$
Let $\Omega_\alpha = \{M^F_\omega f > \alpha\}$; then $\Omega_\alpha$ is the union of maximal cubes $Q_j \in F$ with $\dashint_{Q_j} |f|\omega > \alpha$.
Now
$$\omega(\Omega_\alpha) \leq \sum_j \frac{1}{\alpha} \int_{Q_j} |f|\omega = \frac{1}{\alpha} \int_{\Omega_\alpha} |f|\omega$$
so we can use the layer-cake decomposition (really, the Fubini-Tonelli theorem) to deduce the weaktype $(1, 1)$-bound.
\end{proof}

Let us prove the theorem in case $p = 2$ for simplicity though the argument transfers over in general.
Thus we need to show
$$||T^S||_{L^2(\omega)} \lesssim [\omega]_2.$$
We have
$$|T^Sf| = T^S|f|$$
so by H\"older duality
$$||T^Sf||_{L^2(\omega)} \leq \sup_{||g||_{L^2(\omega) = 1}} \int_{\RR^d} T^S|f||g|\omega.$$
If $f,g \geq 0$ then
$$\int_{\RR^d} T^Sfg\omega = \sum_{Q \in S} \int_Q f \dashint_Q g\omega \frac{\omega(Q)}{|Q|} = \sum_{Q \in S} \frac{1}{\omega^{-1}(Q)} \int_Q f\omega \omega^{-1} \dashint_Q g\omega
 \frac{\omega(Q)}{Q} \frac{\omega^{-1}(Q)}{|Q|} |Q|.$$
But $\omega(Q)/Q \leq [\omega]_2$ and $|Q| \leq |E_Q|/\eta$, so
$$\int_{\RR^d} T^Sfg\omega \leq \frac{[\omega]_2}{\eta} \sum_{Q \in S} \int_{E_Q} M_{\omega^{-1}}^D f\omega M^D_\omega g.$$
The genius idea is using $\omega^{-1}$ as a measure as well as $\omega$. This works because $[\omega]_2 = [\omega^{-1}]_2$.
By H\"older's inequality,
$$\int_{\RR^d} T^Sfg\omega \leq \frac{[\omega]_2}{\eta} ||M_{\omega^{-1}}^D(f\omega)||_{L^2(\omega^{-1})} ||M^D_\omega g||_{L^2(\omega)}.$$
Thus we have
$$\int_{\RR^d} T^Sfg\omega \leq 4\frac{[\omega]_2}{\eta} [\omega]_2 ||f||_{L^2(\omega)}.$$

Now we prove the case of general $p$. We reduce to the case $p \geq 2$.
If we know that it holds for $p \geq 2$ and we are given $p < 2$ then let $\sigma = \omega^{1-p'}$. Then $[\sigma]_{p'} = [\omega]_p$, and
$$||T^S||_{L^p(\omega)} = ||T^S||_{L^{p'}(\sigma)} \leq \eta^{-p'/p} pp' [\sigma]_{p'} = pp'\eta^{-p'/p}[\omega]_p^{1/(p-1)}.$$
Therefore
$$||T^S||_{L^p(\omega)} \leq pp' \eta^{-\max(p/p',p'/p)} [\omega]_p^{\max(1, p'/p)}.$$
So it suffices to check $p \geq 2$.

Again we use H\"older duality:
$$||T^S||_{L^p(\omega)} = ||T^S(\cdot ~\sigma)||_{L^p(\sigma) \to L^p(\omega)} =  \sup_{||f||_p = ||g||_{p'} = 1} \int_{\RR^d} T^S(f \sigma) g\omega.$$
Bounding that integral using the definition of a sparse operator,
\begin{align*}\int_{\RR^d} T^S(f \sigma) g\omega &= \sum_{Q \in S} (f\sigma)_Q \int_Q g\omega \\
&\leq [\omega]_p \sum_{Q \in S} \frac{|Q|^p}{\omega(Q)\sigma(Q)^{p-1}} \int_Q g\omega \dashint_Q f\sigma\\
&= [\omega]_p \sum_{Q \in S} A_\omega(g, Q) A_\sigma(f, Q) \frac{|Q|^{p-1}}{\sigma(Q)^{p-1}} \sigma(Q)
\end{align*}
where
$$A_\kappa(h, Q) = \dashint_Q g~d\kappa$$
is the mean of $g$ over $Q$ with respect to the measure $\kappa$.
Thus
$$\int_{\RR^d} T^S(f \sigma) g\omega \leq \frac{[\omega]_p}{\eta^{p-1}} \sum_{Q \in S} A_\omega(g, Q) A_\sigma(f, Q) |E_Q|^{p-1} \sigma(E_Q)^{2-p}.$$
Since $p \geq 2$ we don't get any blowup here.
We need to get rid of the $|E_Q|$, using H\"older's inequlity,
$$|E_Q| = \int_{E_Q} \omega^{1/p} \omega^{-1/p} \leq \omega(E_Q)^{1p} \sigma(E_Q)^{1/p'},$$
thus
$$|E_Q|^{p-1} \sigma(E_Q)^{2-p} \leq \omega(E_Q)^{1/p'} \sigma(E_Q)^{1/p}.$$
By H\"older's inequality again,
$$\int_{\RR^d} T^S(f\sigma) g\omega \leq \frac{[\omega]_p}{\eta^{p-1}} (\sum_Q A_\omega(g, Q)^{p'} \omega(E_Q))^{1/p'} (\sum_Q A_\sigma(f, Q)^p \sigma(E_Q))^{1/p}$$
and since the $E_Q$ are disjoint we can use the Hardy-Littlewood operator
$$\int_{\RR^d} T^S(f\sigma) g\omega \leq \frac{[\omega]_p}{\eta^{p-1}} ||M^D_\omega g||_{L^{p'}(\omega)} ||M^D_\omega f||_{L^p(\sigma)} \leq \frac{[\omega]_p}{\eta^{p-1}} ||g||_{L^{p'}(\omega)} ||f||_{L^p(\sigma)}.$$
This proves the theorem.

\chapter{Bellman functions}
\section{Reverse H\"older inequalities}
In this section we use a Bellman function to prove a reverse H\"older inequality.
This lecture follows a talk by Benoit Pausauder which in turn follows a paper by Nazarov, Treil, and Volberg.
Bellman was a famous control theorist.

We use $(f)_I$ to mean the average of $f$ over $I$.

\begin{definition}
A \dfn{dyadic $A_\infty$ weight} is a positive $\omega \in L^1_{loc}(I_0)$ such that for every dyadic $I \subset I_0$,
$$(\omega)_I \lesssim \exp((\log \omega)_I).$$
The optimal constant is denoted $[\omega]_\infty$.
\end{definition}

Using Jensen's inequality on $\mu = |I|^{-1} 1_I$, we get
$$\exp((\log \omega)_I) \leq \frac{(\omega)_I}{|I|}$$
which implies that $[\omega]_\infty > 1$.

\begin{proposition}
\label{dyadic holder}
Let $\omega$ be a dyadic $A_\infty$ weight. Then there is $p > 1$ such that for every dyadic $I \subset I_0$,
$$(\omega^p)_I \lesssim (\omega)_I^p.$$
\end{proposition}
Note that if this claim holds for some fixed $p$ then it holds for all $1 < q < p$ as well, by H\"older's inequality.

We first show that if there is a function $B$ with certain properties, then Proposition \ref{dyadic holder} holds.

\begin{lemma}
For every $c > 1$, $K \geq 0$, there is a function $B$ defined on
$$\Omega = \{(x, y) \in \RR^2: 1 \leq xe^{-y} \leq c\}$$
satisfying $x^p \leq B(x, y) \leq Kx^p$ and for every $z_1,z_2 \in \Omega$ with $(z_1 + z_2)/2 \in \Omega$ then
$$2B\left(\frac{z_1 + z_2}{2}\right) \geq B(z_1) + B(z_2).$$
\end{lemma}
In particular, if $B$ is smooth, then the Hessian of $B$ is negative-semidefinite; that is, $B$ is concave.
We call $B$ the \dfn{Bellman function} associated to $\omega$.

\begin{definition}
A function is \dfn{dyadically measurable} if it is measurable with respect to the $\sigma$-algebra generated by dyadic intervals.
\end{definition}
For example, every function which is constant on sufficiently small dyadic intervals is dyadically measurable.
The dyadically measurable functions are dense in $L^1_{loc}$, and by a density argument, in what follows we may assume that $\omega$ is dyadically measurable.

Assume that $B$ as in the lemma exists. Fix $J \subset I$ and consider $z_J = x_J + iy_J$ with $x_J = (\omega)_J$ and $y_J \in (\log \omega)_J$.
Then the dyadic $A_\infty$ condition implies $z_J \in \Omega$, where $c,K$ arise from the constants in the definition of $\omega$.
Letting $J^\pm$ be the children of $J$,
$$|J| (\omega)_J = \frac{|J|}{2}((\omega)_{J^-} + (\omega)_{J^+})$$
and
$$|J| (\log \omega)_J = \frac{|J|}{2}((\log \omega)_{J^-} + (\log \omega)^{J^+}).$$
In particular, $2z_J = z_{J^-} + z_{J^+}$ and $z_J, z_{J^+}, z_{J^-} \in \Omega$.

Then
$$B(z_{J^-}) + B(z^{J^+}) \leq 2B(z_J) \lesssim x_J^p = (\omega)_J^p.$$
Iterating this relation $n$ times we get
$$\sum_{\substack{j \subset J\\2^n|j| = |J|}} (\omega_j)^p \lesssim 2^n (\omega)_J^p.$$
Since $\omega$ is dyadically measurable, at sufficiently fine scales (thus $n \gg 1$), $\omega$ becomes constant, thus
$$(\omega^p)_J \lesssim (\omega)_J^p.$$
That completes the proof, assuming the lemma.

It remains to prove the lemma. Let
$$a = xe^{-y}.$$
We want a function $B$ such that $\log (B(x+iy)/x^p)$ is bounded and nonnegative.
We consider the ansatz
$$B(x + iy) = x^p\varphi(xe^{-y}).$$
Given $c$ we will choose $\varphi$ so that $\varphi$ is continuous on $[1, c]$ and the Hessian of $B$ is negative-semidefinite.
Then we can take
$$K = \sup_{a \in [1, c]} \varphi(a)$$
which is finite by continuity.
First we compute the Hessian, which is a mess but implies the conditions:
\begin{enumerate}
\item $a\varphi'(a) + a^2 \varphi''(a) \geq 0$
\item
$$(p(p-1)\varphi(a) a\varphi'(a) - (p^2 + 1)(a \varphi'(a))^2 + p(p-1)\varphi(a)a^2 \varphi''(a) - a^3 \varphi'(a) \varphi''(a))x^{2p-2} \geq 0.$$
\end{enumerate}
The second is the determinant of the Hessian. The first equation simplifies to
$$\frac{d}{da}(a\varphi'(a)) \geq 0.$$
For the second equation, we factor out the $x^p$ and notice that since the condition we're trying to prove gets worse as $p \to 1$, we can take $p$ to be a perturbation of $1$, thus we want
$$0 \leq -2(a\varphi'(a))^2 - (a\varphi'(a))(a^2 \varphi''(a)) + O(p-1)$$
which implied by $a\varphi'(a) \leq 0$, $(a\varphi')' \gg 1$.

Once these conditions are met we observe that if $z_1, z_2 \in \Omega_c$ and $(z_1+z_2)/2 \in \Omega_c$, then $[z_1, z_2] \in \Omega_{c^2}$.
Thus we can do the above construction for $c^2$ instead of $c$ to get a function which is locally concave in $\Omega_{c^2}$.
Now we use the fact that if
$$1 \leq (x-h)e^{k-y}, xe^{-y}, (x+h)e^{-yk} \leq c$$
then
$$1 \leq (x + \theta h)e^{-y} e^{\pm \theta k} \leq c$$
when $\theta \in [0, 1]$.
That implies that if $B$ is locally concave in $\Omega_{c^2}$ then $B$ is concave in $\Omega_c$ which was desired.

\section{The Carleson embedding theorem}
\begin{definition}
A family of numbers $(\mu_I)$, $I$ ranging over dyadic intervals, has the \dfn{Carleson measure condition} if for every $I$,
$$c_\mu(I) = |I|^{-1} \sum_{J \subset I} \mu_J \leq 1$$
where $J$ ranges over dyadic subintervals of $I$.
\end{definition}

For example, if
$$\mu_J = \langle h_J, a\rangle^2$$
where $a$ is BMO (say, $a \in L^\infty$) and $\langle h_J, a\rangle$ is the projection of $a$ onto the Haar basis element corresponding to $J$, then $\mu$ has the Carleson measure condition.

\begin{theoremx}
If $\mu$ satisfies the Carleson measure condition then for every $f \in L^2$,
$$\sum_I \mu_I |f_I|^2 \leq 4||f||_{L^2}^2$$
where $I$ ranges over dyadic intervals and $f_I$ is the mean of $f$ over $I$.
\end{theoremx}

This theorem is surprising because we double-count intervals a lot.
We will prove it by showing the existence of a suitable Bellman function.

Fix $f \in L^2$.
Let $\vec f = (f_I)$ and $\vec F = (f^2_I)$.
We write $I = I^+ \cup I^-$ to decompose an interval into children.
We have
$$2f_I = f_{I_-} + f_{I_+}$$
as usual, similarly for $F$.
Moreover if $c$ is as in the Carleson measure condition, then
$$c(I) = |I|^{-1}\mu_I + 2^{-1}(c(I^-) + c(I^+)).$$
These are the recurrence relations we will need.

We now consider some constraint inequalities. First $f^2_I \leq F_I$ by H\"older's inequality, and also $c(I) \leq 1$ by the Carleson measure condition.

Define
$$B(F_0, f_0, M) = \sup_{F,f,\mu} |I|^{-1} \sum_{J \subset I} \mu_J (f_J)^2$$
be the Bellman function describing the worst case, where the supremum ranges over $F = f^2$, $F_I = F_0$, $f_I = f_0$, $c_\mu(I) = M \leq 1$, $\mu$ with the Carleson measure condition.
We will also write $a = F_0$, $b = f_0$, $c = M$, since otherwise this notation is atrocious.

Let
$$\Omega = \{(a, b, c) \in \RR^3_+: b^2 \leq a, ~0 \leq c \leq 1\}.$$
Suppose there is a function $B: \Omega \to \RR$ such that $0 \leq B(a, b, c) \lesssim a$, and for all $0 \leq \delta \leq c$,
$$B(a, b, c) \geq \delta b^2 + 2^{-1}(B(a_+, b_+, c_+) + B(a_-, b_-, c_-))$$
where $2a = a_+ + a_-$, $2b = b_+ + b_-$, and $2(c - \delta) = c_+ + c_-$.
These conditions are given by the recurrence relations and the constraint inequalities.

If such a function $B$ exists, then the theorem is true. In fact
\begin{align*}
cF_I &= (f^2)_I \geq B(F_I, f_I, M_I) \\
&\geq |I|^{-1}\mu_I (f_I)^2 + 2^{-1}(B(F_{I_-}, f_{I_-}, M_{I_-}) + B(F_{I_+}, f_{I_+}, M_{I_+}))\\
&\geq \sum_{|J| \geq 2^{-n}|I|} |I|^{-1} \mu_J(f_J)^2 + \sum_{|J| = 2^{-n}|I|} B(F_J, f_J, M_J)\frac{|J|}{|I|}\\
&\geq |I|^{-1} \sum_J \mu_J (f_J)^2
\end{align*}
by an induction on scales.
The point is that the constants at each stage are all the same, they're always $c|I|$.

Now to show that such a $B$ exists.
It is easy to show (just draw a picture) that $\Omega$ is convex.
We want $B$ to have negative Hessian and $\partial_c B \geq b^2$.
Given a function $b: \RR^3 \to \RR$ we define
$$u(y, z) = \sup_x b(x, y, z) - cx$$
and call $u$ the partial Legendre transform of $b$.
The point is that we got rid of one variable but we have just as much information as before in the following sense.
The intuition is that the supremum of concave functions may not be concave even though the infimum of concave functions is.
However if we take the supremum in just one of many variables, we still get that a supremum of concaves is concave.

\begin{lemma}
$b$ is concave with $0 \leq b \leq cx$ and $\partial_z b \geq y^2$ iff $u$ is concave and $-cy^2 \leq u \leq 0$ and $\partial_z u \geq y^2$.
\end{lemma}
\begin{proof}
First
$$-cx \leq b - cx \leq u(y, z) \leq 0$$
and taking the supremum in $x \leq y^2$,
$$-cy^2 \leq u(y, z) \leq 0.$$
For the concavity we use a lemma which says that if $\varphi$ is concave and $Z(x) = \sup_y \varphi(x, y)$ then $Z$ is concave.
We then bound $\partial_z b$ similarly.
\end{proof}

So we just need to find a suitable $u$.
If $u$ is concave and $-cy^2 \leq u \leq 0$, $\partial_z u \geq y^2$ then so does the rescaling $a^2 u_a(y, z) = u(ay, z)$ so we mod out by the homogeneity to get a one-variable problem, say finding a $\varphi$ with $u(y, z) = y^2 \varphi(z)$.
We want $\varphi \leq 0$, $\varphi \in L^\infty$, $\varphi' \geq 1$, and $\varphi\varphi'' - 4(\varphi')^2 \geq 0$.
In fact we can satisfy this with $\varphi(z) = -4/(1 + z)$.
Then set
$$b(x, y, z) = -4\frac{y^2}{1 + z^2} + 4x.$$
This was desired.
Also, since we need this by Bellman functions, we can use this proof to show that $4$ is the optimal constant.




\newpage
\printindex
\printbibliography

\end{document}
