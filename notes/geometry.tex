
\documentclass[12pt]{book}
\usepackage[utf8]{inputenc}
\usepackage[margin=1in]{geometry}
\usepackage{amsmath,amsthm,amssymb}
\usepackage{mathrsfs}

\usepackage{enumitem}
%\usepackage[shortlabels]{enumerate}
\usepackage{tikz-cd}
\usepackage{mathtools}
\usepackage{amsfonts}
\usepackage{amscd}
\usepackage{makeidx}
\usepackage{enumitem}
\title{Geometry notes}
\author{Aidan Backus}
\date{2021}


\newcommand{\NN}{\mathbb{N}}
\newcommand{\ZZ}{\mathbb{Z}}
\newcommand{\QQ}{\mathbb{Q}}
\newcommand{\RR}{\mathbb{R}}
\newcommand{\CC}{\mathbb{C}}
\newcommand{\PP}{\mathbb{P}}
\newcommand{\DD}{\mathbb{D}}

\newcommand{\Torus}{\mathbb{T}}

\newcommand{\Mero}{\mathscr M}
\newcommand{\Olo}{\mathscr O}

\newcommand{\AAA}{\mathcal A}
\newcommand{\BB}{\mathcal B}
\newcommand{\HH}{\mathcal H}

\newcommand{\Grp}{\mathbf{Grp}}
\newcommand{\Open}{\mathbf{Open}}
\newcommand{\Vect}{\mathbf{Vect}}
\newcommand{\Set}{\mathbf{Set}}

\newcommand{\Cau}{\mathbf{Cau}}
\newcommand{\ISF}{\mathbf{ISF}}
\newcommand{\Simp}{\mathbf{Simp}}
\newcommand{\Sch}{\mathscr S}

\DeclareMathOperator{\GL}{GL}

\DeclareMathOperator{\atanh}{atanh}
\DeclareMathOperator{\sech}{sech}
\DeclareMathOperator{\sinc}{sinc}
\DeclareMathOperator{\dom}{dom}

\DeclareMathOperator{\Ann}{Ann}
\DeclareMathOperator{\Ass}{Ass}
\DeclareMathOperator{\card}{card}
\DeclareMathOperator{\coker}{coker}
\DeclareMathOperator{\diag}{diag}
\DeclareMathOperator{\Div}{Div}
\DeclareMathOperator{\Gal}{Gal}
\DeclareMathOperator{\Jac}{Jac}
\DeclareMathOperator{\Harm}{Harm}
\DeclareMathOperator{\Hom}{Hom}
\DeclareMathOperator{\id}{id}
\DeclareMathOperator{\Pic}{Pic}
\DeclareMathOperator{\sgn}{sgn}
\DeclareMathOperator{\Spec}{Spec}
\DeclareMathOperator{\supp}{supp}
\DeclareMathOperator{\rank}{rank}
\DeclareMathOperator{\Res}{Res}

\newcommand{\dbar}{\overline\partial}

\def\Xint#1{\mathchoice
{\XXint\displaystyle\textstyle{#1}}%
{\XXint\textstyle\scriptstyle{#1}}%
{\XXint\scriptstyle\scriptscriptstyle{#1}}%
{\XXint\scriptscriptstyle\scriptscriptstyle{#1}}%
\!\int}
\def\XXint#1#2#3{{\setbox0=\hbox{$#1{#2#3}{\int}$ }
\vcenter{\hbox{$#2#3$ }}\kern-.6\wd0}}
\def\ddashint{\Xint=}
\def\dashint{\Xint-}

\renewcommand{\Re}{\operatorname{Re}}
\renewcommand{\Im}{\operatorname{Im}}
\newcommand{\dfn}[1]{\emph{#1}\index{#1}}

\usepackage[backend=bibtex,style=alphabetic,maxcitenames=50,maxnames=50]{biblatex}
\renewbibmacro{in:}{}
\DeclareFieldFormat{pages}{#1}

\usepackage{color}
\usepackage{hyperref}
\hypersetup{
    colorlinks=true, % make the links colored
    linkcolor=blue, % color TOC links in blue
    urlcolor=red, % color URLs in red
    linktoc=all % 'all' will create links for everything in the TOC
    %Ning added hyperlinks to the table of contents 6/17/19
}

\theoremstyle{definition}
\newtheorem{theorem}{Theorem}[chapter]
\newtheorem{lemma}[theorem]{Lemma}
\newtheorem{sublemma}[theorem]{Sublemma}
\newtheorem{proposition}[theorem]{Proposition}
\newtheorem{corollary}[theorem]{Corollary}
\newtheorem{axiomx}[theorem]{Axiom}
\newtheorem{theoremxx}[theorem]{Theorem}
\newtheorem{conjecture}[theorem]{Conjecture}
\newtheorem{definitionx}[theorem]{Definition}
\newtheorem{remark}[theorem]{Remark}
\newtheorem{examplex}[theorem]{Example}
\newtheorem{exercisex}{Exercise}[chapter]
\newtheorem{problem}[theorem]{Problem}

\newenvironment{axiom}
  {\pushQED{\qed}\renewcommand{\qedsymbol}{$\diamondsuit$}\axiomx}
  {\popQED\endexamplex}

\newenvironment{definition}
  {\pushQED{\qed}\renewcommand{\qedsymbol}{$\diamondsuit$}\definitionx}
  {\popQED\endexamplex}

\newenvironment{example}
  {\pushQED{\qed}\renewcommand{\qedsymbol}{$\diamondsuit$}\examplex}
  {\popQED\endexamplex}

  \newenvironment{exercise}
    {\pushQED{\qed}\renewcommand{\qedsymbol}{$\diamondsuit$}\exercisex}
    {\popQED\endexamplex}

  \newenvironment{theoremx}
        {\pushQED{\qed}\renewcommand{\qedsymbol}{$\diamondsuit$}\theoremxx}
        {\popQED\endexamplex}

\makeindex

\begin{document}

\maketitle

\tableofcontents

\chapter{Riemann surfaces}
In this chapter we begin discussing algebraic geometry.
Algebraic geometry is the study of projective varieties over $\CC$; compact Riemann surfaces give the dimension $1$ case.

\begin{definition}
Let $X$ be a surface. If $\varphi_1,\varphi_2$ are charts on $X$ valued in $\CC$, we say that $\varphi_1,\varphi_2$ are \dfn{compatible charts} if the transition functions $\varphi_2 \circ \varphi_1^{-1}$ is holomorphic.
\end{definition}

\begin{definition}
Let $X$ a surface. A \dfn{complex atlas} on $X$ is an open cover of $X$ by compatible charts.
Two complex atlases $A,B$ are \dfn{compatible atlases} if for every $\varphi \in A, \psi \in B$, $\varphi,\psi$ are compatible charts.
\end{definition}

\begin{definition}
A \dfn{complex structure} is an equivalence class of complex atlases with respect to compatibility.
A \dfn{Riemann surface} is a connected surface with a choice of complex structure.
\end{definition}

Let us give examples of Riemann surfaces.
We first note that open subsets of Riemann surfaces are Riemann surfaces, simply by taking restrictions of charts. Trivially $\CC$ is a Riemann surface, where the identity is a global chart.
So every open set in $\CC$ is a Riemann surface.

\begin{definition}
The \dfn{Riemann sphere}, or \dfn{projective line}, $\PP^1$, is the Riemann surface defined by taking the topological one-point compactification $\CC \to S^2$, and taking the charts $\varphi: \CC \to \CC$ to be the identity and $\psi: \PP^1 \setminus \{0\}$ to be $\psi(z) = 1/z$.
\end{definition}

To see that $\PP^1$ is in fact a Riemann surface, we note that $\psi \circ \varphi^{-1}(z) = 1/z$, and similarly for $\varphi^{-1} \circ \psi$.

\begin{definition}
Let $\omega_1,\omega_2 \in \CC$ be linearly independent over $\RR$. Let $\Gamma$ be the free abelian discrete group generated by $\omega_1,\omega_2$, equipped with its action on $\CC$ by translation.
Let $T_\Gamma = \CC/\Gamma$ be the space of orbits of $\Gamma$ and $\pi: \CC \to T_\Gamma$ be the projective map.
We put a complex structure on $T_\Gamma$ as follows.
Let $U$ be a subset of a fundamental domain of the action of the $\Gamma$, so $\pi|U$ is a bijection. Then declare that the charts are $(\pi|U)^{-1}$.
We say that $T_\Gamma$ is a \dfn{complex torus} or \dfn{elliptic curve}.
\end{definition}

\section{Holomorphic maps}
We introduce the morphisms of the category of Riemann surfaces.
\begin{definition}
A map $f: X \to Y$ between Riemann surfaces is a \dfn{holomorphic map} if for every pair of charts $\varphi: U \to X$, $\psi: V \to Y$ there is a commutative diagram
$$\begin{tikzcd}
X \arrow[r,"f"] & Y \\
U \arrow[u,"\varphi"] \arrow[r] & V \arrow[u,"\psi"]
\end{tikzcd}$$
in which the bottom arrow commutes.
\end{definition}

\begin{definition}
A holomorphic bijection $f: X \to Y$ is a \dfn{biholomorphic map} if $f^{-1}$ exists and is holomorphic. If a biholomorphic map $X \to Y$ exists, we say that $X,Y$ are \dfn{biholomorphic surfaces}.
If $f$ is biholomorphic onto its image, we say that $f$ is an \dfn{embedding}.
\end{definition}

The fundamental problem for Riemann surfaces, introduced by Riemann himself, is: if $S$ is a topological surface, how many distinct (i.e. up to biholomorphic isomorphism) complex structures exist on $S$?
Solving this problem in general requires one to use Teichm\"uller theory and moduli stacks of curves, and is still an active area of research.
This is different than in the smooth category, where there is exactly one smooth structure one can put on a surface.

\begin{example}
Liouville's theorem shows that $\CC$ has at least two Riemann surface structures on it: the disc and the plane.
\end{example}

\begin{definition}
A \dfn{holomorphic function} on $X$ is a holomorphic map $X \to \CC$.
The space of holomorphic functions on $X$ is denoted $\Olo(X)$.
\end{definition}

\begin{theoremx}[removal of singularities]
If $f \in \Olo(X \setminus A)$, $A$ is finite, and $f$ is bounded near $A$, then $f \in \Olo(X)$.
\end{theoremx}

\begin{theoremx}[analytic continuation]
If $f, g: X \to Y$ are holomorphic, $A \subseteq X$ is not discrete, and $f|A = g|A$, then $f = g$.
\end{theoremx}

These theorems are both local, so by taking charts we can pass to the case that $X$ is an open subset of $\CC$, in which case they are trivial.
They are especially powerful when $X$ is compact, so that $A$ is either discrete or infinite.

\begin{definition}
We say that $f \in \Olo(U)$ is a \dfn{meromorphic function} on $X$ if $U \subseteq X$ is open, $X \setminus U$ is discrete, and
$$\lim_{z \to z_0} |f|(z_0) = \infty$$
whenever $z_0 \notin U$.
Points in $X \setminus U$ are called \dfn{poles} of $f$.
The space of meromorphic functions on $X$ is denoted $\Mero(X)$.
\end{definition}

We note that $\Olo$ and $\Mero$ both have the structure of a sheaf of rings.
Since, if $(U, z)$ is a chart, $\Mero(U)$ is exactly the space of meromorphic functions in the usual sense on some open subsets of $\CC$, we may expand $f \in \Mero(X)$ as a Laurent series
$$f(z) = \sum_{k=-\infty}^\infty a_k z_k$$
at least in the coordinate chart $(U, z)$.

\begin{theorem}
Let $f \in \Mero(X)$ and define $\hat f \in \Olo(X \to \PP^1)$ by $\hat f = f$ away from the poles of $f$, and $\hat f(p) = \infty$ for every pole $p$ of $f$.
Conversely, $\hat f \in \Olo(X \to \PP^1)$ is either constant or restricts to a meromorphic function on $X$.
\end{theorem}
\begin{proof}
By the definition of a one-point compactification, $\hat f$ is continuous. In particular, $\hat f$ is locally bounded in charts, so by removal of continuities $f$ extends to the holomorphic function $\hat f$.

Conversely, let $A = \hat f^{-1}(\infty)$. If $A = X$ then $\hat f$ is constant. Otherwise, $A$ is discrete, and $X \setminus A$ is open, so $f = \hat f|(X \setminus A)$ is a meromorphic function on $X$.
\end{proof}

We now check that locally, holomorphic maps are monomials.

\begin{theorem}
Let $f: X \to Y$ be a nonconstant holomorphic map, $a \in X$, $b = f(a)$.
Then there are coordinates near $a,b$ in which $a = b = 0$ and
$$f(z) = z^k$$
where $k \in \NN$, $k \geq 1$, and $k$ does not depend on the choice of coordinates.
\end{theorem}
\begin{proof}
Let $\varphi: U \to \CC$ be a chart with $\varphi(a) = 0$ and $\psi: V \to \CC$ a chart with $\psi(b) = 0$.
Let $F: \varphi(U) \to \psi(V)$ be the local representation of $f$.
Then $F$ has a zero at $0$, say of order $k$, thus
$$F(z) = z^k G(z)$$
where $G(0) \neq 0$. Taking a biholomorphic isomorphism does not change the order of a zero, so $k$ does not depend on the choice of coordinates.
Shrinking $U$ if necessary, we can assume that $G$ is nonzero on all of $\varphi(U)$.
We can then ramify the logarithm to find a $H$ with $G = H^k$.
In these coordinates we have
$$F(z) = (zH(z))^k.$$
Let $\alpha(z) = zH(z)$. Then, after shrinking $U$ as necessary again, $\alpha$ is a holomorphic embedding $\varphi(U) \to \CC$.
Indeed, $\alpha'(0) = H(0)$ is nonzero, so the claim follows by the inverse function theorem.
We can then replace $\varphi$ with $\alpha \circ \varphi$, and in those coordinates we have $F(z) = z^k$, as desired.
\end{proof}

\begin{corollary}[open mapping theorem]
Every holomorphic map is open.
\end{corollary}
\begin{proof}
This can be checked locally, and locally the map is $z^k$, which is open.
\end{proof}

\begin{corollary}
Every injective holomorphic map is an embedding.
\end{corollary}
\begin{proof}
By the open mapping theorem, we just need to check that $f^{-1}$ is holomorphic.
But $f(z) = z^k$ and $f$ is injective, so $k = 1$, so $f^{-1}(w) = w$ is holomorphic.
\end{proof}

\begin{corollary}[maximum principle]
Let $f: X \to \CC$ be a holomorphic function. Then $f$ is constant or $|f|$ attains a maximum.
\end{corollary}
\begin{proof}
If $r = \max |f|$ is attained, say $r = |f(a)|$, then $f(X) \subseteq \overline{B(0, r)}$.
By the open mapping theorem, $f(X)$ is open if $f$ is not constant, even though $f(X)$ meets $\partial B(0, r)$.
\end{proof}

\begin{corollary}[Liouville's theorem]
If $X$ is compact and $f: X \to \CC$ is holomorphic then $f$ is constant.
\end{corollary}
\begin{proof}
The maximum principle and extreme value theorem imply this.
\end{proof}

\begin{proposition}[GAGA for $\PP^1$]
If $f \in \Mero(\PP^1)$ then $f$ is a rational map.
\end{proposition}
\begin{proof}
This holds if $f$ is constant; otherwise, the set $A$ of poles of $f$ is discrete and compact, so $A$ is finite, say $A = \{a_1, \dots, a_n\}$.
After replacing $f$ with $1/f$ if necessary (which does not affect whether $f$ is a rational map) we may assume that $A \subset \CC$.
Therefore we may let
$$h_\ell(z) = \sum_{j = -k_\ell}^{-1} c_k (z - a_\ell)^j$$
be the principal part of $f$ at $a_\ell$.
Then $f - (h_1 + \dots + h_n) \in \Olo(\PP^1)$ is constant; since the $h_\ell$ are rational maps the claim then holds.
\end{proof}

This is the essence of the GAGA principle of Serre, which says that a projective variety over $\CC$ is the same thing as a complex compact manifold (that has enough meromorphic functions).
Thus much of complex analysis is equivalent to algebraic geometry.

\section{Covering spaces of Riemann surfaces}
Since we assume that Riemann surfaces are connected, and they are locally euclidean, they are always path-connected, and thus we may define their fundamental group, and consider the covering space of a Riemann surface.

\begin{example}
The map $p_k: \CC^* \to \CC^*$, $p_k(z) = z^k$ is a covering space.
Indeed, $p_k'$ is never zero, so by the inverse function theorem $p_k$ is a local homeomorphism.
If $p_k(b) = a$, then there are $U \ni a$, $V \ni b$ such that $p_k$ is a homeomorphism $V \to U$.
In particular,
$$p_k^{-1}(U) = \bigcup_{j=0}^{k-1} \omega^j V$$
where $\omega$ generates the group of $k$th roots of unity.
If $V$ is taken small enough, then the $\omega^j V$ are disjoint, hence the claim.
\end{example}

\begin{example}
For similar reasons, $\exp: \CC \to \CC^*$ is a covering space.
Indeed, $\exp'$ is never zero, and so for every $a \in \CC^*$ we can find $U \ni a$ such that
$$\log U = \bigcup_{n \in \ZZ} V + 2\pi ni$$
with $V$ sufficiently small.
\end{example}

\begin{example}
The projection $\CC \to T_\Gamma$, $T_\Gamma$ an elliptic curve, is a covering space.
\end{example}

The above examples realize $\CC$ as the universal cover of $\CC^*$ and $T_\Gamma$.
On the other hand, $\PP^1$ is simply connected and so cannot be covered by $\CC$.

The homotopy lifting property tells us when we can take a logarithm of a function. Consider the universal cover
$$\exp: \CC \to \CC^*.$$
If $f: X \to \CC^*$ is holomorphic then we want to lift $f$ to a map $\hat f: X \to \CC$ which commutes with the universal cover.
This is exactly the question of finding a logarithm of $f$.

\begin{lemma}
Let $X,Y,Z$ be connected manifolds.
Let $p: Y \to X$ be a local homeomorphism, $f: Z \to X$ continuous, and $g_1, g_2: Z \to Y$ lifts of $f$ along $p$.
If there is $z_0$ with $g_1(z_0) = g_2(z_0)$ then $g_1 = g_2$.
\end{lemma}
\begin{proof}
Let $T \subseteq Z$ be the equalizer of $g_1,g_2$. Then $T \ni z_0$ is nonempty and closed.
Let $w \in T$. Then there is a neighborhood $U$ of $g_1(w) = g_2(w)$ which is homeomorphic along $p$ to a neighborhood of $f(w)$.
Then $g_i = p^{-1}\circ f$ near $w$ so $T$ is open.
Therefore $T = Z$.
\end{proof}

\begin{lemma}
If $p: Y \to X$ is holomorphic and a local homeomorphism and $g: Z \to Y$ is a continuous lift of a holomorphic map $Z \to X$ along $p$, then $g$ is holomorphic.
\end{lemma}
\begin{proof}
Locally, $g$ is $p^{-1} \circ f$, which is holomorphic by the inverse function theorem.
\end{proof}

In particular, if the logarithm of a function exists, then it is automatically holomorphic.

In general lifts may not exist for local homeomorphisms, but they will for covering spaces.
This motivates why covering spaces are useful for complex analysis.
We recall the path-lifting property: paths always lift along covering spaces, and the lifts are in bijection with the possible starting points of the path.
Similarly, homotopies of paths always lift along covering spaces, determined by the starting points of each of the paths (as long as the starting points of the paths together form a path).
If we impose that the homotopy lift has a single starting point, then it also has a single endpoint.

We recall that if $p: Y \to X$ is a covering space then the pushforward $p_*: \pi_1(Y) \to \pi_1(X)$ is injective.
Indeed, if $\gamma$ is a loop in $Y$ annihilated by $p_*$ then there is a homotopy $H$ in $X$ which witnesses this, and $H$ lifts to a homotopy in $Y$ which witnesses that $\gamma$ is nullhomotopic.

\begin{theorem}
Let $p: Y \to X$ be a covering space, $f: Z \to X$ continuous. Then $f$ lifts to a map $Z \to Y$ iff $f_* \pi_1(Z) \subseteq p_*(\pi_1(Y))$.
\end{theorem}
\begin{proof}
If a lift exists then the claim follows by functoriality.
Conversely, let $z_0, w \in Z$, and let $\alpha: z_0 \to w$ be a path in $Z$.
Then there is a lift $\tilde \alpha: y_0 \to y_1$ of $f_*\alpha$ in $Y$.
Let $\tilde f(w) = y_1$.
If we have another path $\alpha': z_0 \to w$ then $\alpha' \circ \alpha^{-1}$ is a loop at $z_0$, so $f_*(\alpha' \circ \alpha^{-1}) \in p_*(\pi_1(Y))$ by assumption.
Thus there is a $\gamma \in \pi_1(Y)$ with $p_*\gamma = f_*(\alpha' \circ \alpha^{-1})$.
On the other hand, $p_*(\tilde \alpha' \circ \tilde \alpha^{-1}) = p_*(\gamma)$.
Since $p_*$ is injective this implies $\tilde \alpha \circ \tilde \alpha^{-1} = \gamma$.
So $\tilde f(w)$ does not depend on the choice of $\alpha$.
\end{proof}

\begin{corollary}
Suppose that $Z$ is simply connected, $p: Y \to X$ a covering space, $f: Z \to X$ continuous. Then $f$ lifts to a map $Z \to Y$.
\end{corollary}
\begin{proof}
In the previous theorem, $f_* \pi_1(Z) = 0$.
\end{proof}

\begin{corollary}
If $U$ is a simply connected Riemann surface, $f: U \to \CC^*$ holomorphic, then $f$ has a holomorphic logarithm.
\end{corollary}
\begin{proof}
A logarithm of $f$ is a lift of $f$ along the covering space $\exp: \CC \to \CC^*$.
\end{proof}

We recall that every connected manifold has a universal cover; that it is unique by abstract nonsense; and that it is the unique simply connected covering space.

\begin{definition}
A \dfn{Galois covering space} is a covering space $p: Y \to X$ such that the deck group of $p$ acts transitively on each fiber of $p$.
\end{definition}

\begin{example}
The covering space $p_k: \CC^* \to \CC^*$ is Galois. If $p_k(z_1) = p_k(z_2)$ then there is a root of unity $\omega$ such that $z_2 = \omega z_1$.
But $\omega$ is a deck transformation since $(\omega z)^k = z^k$.
\end{example}

\begin{theoremx}[Galois theory for covering spaces]
Let $X$ be a connected manifold and $p: \tilde X \to X$ its universal cover. Then $p$ is Galois and its deck group is $\pi_1(X)$.
Moreover, if $q: Y \to X$ is any covering space, then there is a unique covering space $f: \tilde X \to Y$ with $f: \tilde X \to Y$, and $f$ is the universal cover of $Y$.
In addition, there is a subgroup $G$ of $\pi_1(X)$ such that $Y = \tilde X/G$ with quotient map $f$, and $G$ is a normal subgroup of $\pi_1(X)$ iff $q$ is Galois.
If $q$ is Galois, then $\pi_1(Y) = \pi(X)/G$.
\end{theoremx}

We omit the proof since this is not a topology class.

\begin{example}
The universal cover of $\CC^*$ is $\exp: \CC \to \CC^*$. Its deck group is isomorphic to $\ZZ$ since $\varphi$ commutes with $\exp$ exactly if $\varphi(z) = 2\pi ik$ for some $k \in \ZZ$.
This gives a proof that $\pi_1(\CC^*) = \ZZ$.
\end{example}

A motivation for covering spaces is \dfn{uniformization} -- that is, the classification of Riemann surfaces up to homeomorphism.
Now $\PP^1$ is simply connected, so it is its own universal cover.
If $T_\Gamma$ is an elliptic curve, then $\Gamma$ acts on $\CC$ by translation and $T_\Gamma = \CC/\Gamma$, so the universal cover of $T_\Gamma$ is $\CC$.
If $\HH$ denotes the hyperbolic plane and $\DD$ the disc, then $\exp: \HH \to \DD^*$ is the universal cover.
But $\PP^1$, $\HH$, and $\CC$ are the only possible universal covers of Riemann surfaces, as we will see.

\section{Branched coverings}
In complex analysis it is convenient to generalize the notion of covering space somewhat.

\begin{definition}
A \dfn{discrete map} is a holomorphic map with discrete fibers.
\end{definition}

\begin{definition}
Let $p: Y \to X$ be an open discrete map. A \dfn{branch point} $y \in Y$ is a point such that for every $V \ni y$ open, $p|V$ is not a homeomorphism.
If $p$ has no branch points, we say that $p$ is an \dfn{unbranched map}.
\end{definition}

\begin{lemma}
An open discrete map is unbranched iff it is a local homeomorphism.
\end{lemma}
\begin{proof}
Just expand out the definitions.
\end{proof}

\begin{lemma}
The set of branch points of a nonconstant holomorphic map is discrete.
\end{lemma}
\begin{proof}
One has $f'|B = 0$, so if $B$ clusters then $f' = 0$ identically.
\end{proof}

\begin{lemma}
If $X$ is a Riemann surface, $Y$ a connected surface, $p: Y \to X$ a local homeomorphism, then there is a unique complex structure on $Y$ for which $p$ is holomorphic.
\end{lemma}
\begin{proof}
Let $(U, \varphi)$ be a chart on $X$.
Define charts on $Y$ by $\{(V, \varphi \circ p)$ where $p|V$ is a homeomorphism $V \to U$.
This clearly defines a complex structure on $Y$.
Uniqueness follows because we can invert $p$ locally to get a biholomorphic isomorphism.
\end{proof}

\begin{definition}
A \dfn{proper map} is a holomorphic map $f$ such that the preimage of every compact set under $f$ is compact.
\end{definition}

Between compact Riemann surfaces, every holomorphic map is proper.

\begin{lemma}
Let $f: Y \setminus f^{-1}(A) \to X \setminus A$ be a proper map, $B$ the branch points of $f$, $A = f(B)$.
Then $f$ is a covering space.
\end{lemma}
\begin{proof}
Let $x \in X \setminus A$.
By properness, the preimage of $x$ is finite, say $f^{-1}(x) = \{y_1, \dots, y_n\}$.
Let $V_i \ni x_i$ be such that $f|V_i$ is an embedding and the $V_i$ are disjoint, then $W = \bigcap_i f(V_i)$ is nonempty.
Then $f^{-1}(W)$ is evenly covered.
\end{proof}

\begin{definition}
Let $f: Y \to X$ be a holomorphic map.
If $f(z) = z^k$ near $y$, then we say that the \dfn{order} of $y$, $v(f, y) = k$.
\end{definition}

It follows that
$$\sum_{y \in f^{-1}(x)} v(f, y)$$
does not depend on $x$ provided that $f$ is proper.

\begin{lemma}
Let $f$ be a proper map.
Let $x$ be a branch point of $f$, $f^{-1}(x) = \{y_1, \dots, y_m\}$, with $k_i$ the order of $y_i$.
Then $\sum_i k_i$ is the number of sheets of $f$ as a covering space (away from its branch points).
\end{lemma}
\begin{proof}
The degree of a covering space is constant, and in a punctured neighborhood of $x$, $f$ has degree $\sum_i k_i$.
\end{proof}

\begin{corollary}
Let $X$ be a compact Riemann surface.
Let $f \in \Mero(X)$; then the number of zeroes of $f$ equals the number of poles of $f$, counted with multiplicity.
\end{corollary}
\begin{proof}
Expanding out the definitions, $f$ defines a proper map $X \to \PP^1$.
$$\sum_{x \in f^{-1}(0)} v(f, x) = \sum_{x \in f^{-1}(\infty) v(f, x)}$$
as desired.
\end{proof}

Let us compute the moduli space of covering spaces of $\DD^*$.
Recall that $p_k$ denotes the map $p_k(z) = z^k$.
In what follows we use $\HH$ to mean the left half-plane, which isomorphic to the upper half-plane but makes notation easier.

\begin{theorem}
Let $X$ be a Riemann surface, $f: X \to \DD^*$ a holomorphic covering space.
If $f$ has infinitely many sheets then $f$ is isomorphic to the covering space $\exp: \HH \to \DD^*$.
Otherwise, $f$ is isomorphic to the covering space $p_k: \DD^* \to \DD^*$.
\end{theorem}
\begin{proof}
Since $\exp$ is the universal cover, there is a covering space $\psi: \HH \to X$ which commutes with $\exp, f$.
By Galois theory, the deck group of $\psi$ is a subgroup $G$ of $\pi_1(\DD^*) = \ZZ$.
So either $G$ is the trivial group or $G = k\ZZ$ acts on $\HH$ by translation by $2\pi ink$.
If $G$ is trivial then $\psi$ is an isomorphism.
Otherwise $X$ is a quotient of $\HH$ by the action of $G$, so $X$ is a cylinder.
Every cylinder is isomorphic to $\DD^*$, identifying one end of the tube with $0$ and the other end with $\partial \DD$.
\end{proof}

\begin{theorem}
Let $X$ be a Riemann surface, $f: X \to \DD$ a proper nonconstant map such that $f$ is unbranched over $\DD^*$.
Then there is a $k \geq 1$ and an isomorphism $\varphi: X \to \DD$ such that $f = p_k \circ \varphi$.
\end{theorem}
\begin{proof}
Let $X^* = f^{-1}(\DD^*)$.
Then $f|X^*$ is a covering space of $\DD^*$, with only finitely many sheets since $f$ is proper.
Thus there is an isomorphism $\varphi^*: X^* \to \DD^*$ of covering spaces between $p_k$ and $f$.
We claim that $f^{-1}(0)$ is a point, so that $\varphi^*$ extends to an isomorphism of Riemann surfaces $\varphi: X \to D$.

So suppose that $f^{-1}(0) = \{b_1, \dots, b_n\}$, $B$ a disk around $0$, and $f^{-1}(B) = \bigcup_i V_i$ where $b_i \in V_i$ and the $V_i$ are disjoint.
But $f^{-1}(B)$ is isomorphic to $p_k^{-1}(B)$ which is $D(0, r^{1/k})^*$, which is connected.
Therefore $n = 1$.
\end{proof}

\section{Sheaves on Riemann surfaces}
\begin{definition}
Let $X$ be a topological space and $C$ a category.
A \dfn{presheaf} $\mathscr F$ on $X$ valued in $C$ is a map which assigns each open set $U$ of $X$ to some $\mathscr F(U) \in C$ equipped with \dfn{restrictions}
$$\rho_V^U: \mathscr F(U) \to \mathscr F(V)$$
whenever $V \subseteq U$ which compose, thus $\rho_W^V \circ \rho_V^U = \rho_W^U$ and $\rho_U^U = 1$.
We write $f|V = \rho_V^U(f)$ whenever $f \in \mathscr F(U)$.
\end{definition}

\begin{example}
The continuous functions $C$, holomorphic functions $\Olo$, and meromorphic functions $\Mero$ form presheaves on any Riemann surface valued in algebras over $\CC$.
\end{example}

\begin{definition}
Let $X$ be a topological space and $C$ a category.
A \dfn{sheaf} $\mathscr F$ on $X$ valued in $C$ is a presheaf such that for every $U \subseteq X$ open and open covers $\mathcal U$ of $U$:
\begin{enumerate}
\item If $f,g \in \mathscr F(U)$ such that for every $V \in \mathcal U$, $f|V = g|V$, then $f = g$.
\item If for every $V \in \mathcal U$ there is $f_V \in \mathscr F(V)$ such that $f_V|V \cap W = f_W|V \cap W$ for all $V, W \in \mathcal U$, then there is $f \in \mathscr F(U)$ such that for every $V$, $f|V = f_V$.
\end{enumerate}
\end{definition}

\begin{definition}
Let $X$ be a topological space.
Let $x \in X$ and $\mathscr F$ a presheaf on $X$ valued in a category with colimits.
The \dfn{stalk} $\mathscr F_x$ is the colimit of $\mathscr F(U)$ as $U$ shrinks down to $x$.
A \dfn{germ} at $x$ is an element of $\mathscr F_x$.
\end{definition}

Let $\mathscr F$ be a presheaf, valued in a concrete category.
We will construct a topological space $|\mathscr F|$ which as an underlying set is the disjoint union of the stalks of $\mathscr F$, so we get a fiber bundle structure
$$p: |\mathscr F| \to X$$
which sends all the germs at $x$ to $x$.

\begin{theorem}
There is a topology on the set $|\mathscr F|$ such that $p$ is continuous and a local homeomorphism.
\end{theorem}
\begin{proof}
We define a base of sets
$$[U, f] = \{[z, f]: z \in U\}$$
whenever $U \subseteq X$ is open and $f \in \mathscr F(U)$.
Here $[z, f]$ denotes the image of $f$ in $\mathscr F_x$ when we take the colimit over the diagram of open neighborhoods of $z$, so $[U, f]$ is the set of all germs obtained by restricting $f$.
Every germ is in the image of some $f$ so the $[U, f]$ form a cover of $|\mathscr F|$.

Now if $\varphi \in [U, f] \cap [V, g]$ and $\varphi$ is a germ at $x$ then $\varphi = f_x = g_x$.
Therefore there is a $W \subseteq U \cap V$ such that $f|W = g|W$, by definition of colimit.
Let $h = f|W$; then $[W, h] \subseteq [U, f] \cap [V, g]$.
Therefore the $[U, f]$ form a topological base.

If $W \subseteq X$ is open then
$$p^{-1}(W) = \bigcup_{f \in \mathscr F(W)} [W, f]$$
is open and hence $p$ is continuous. Moreover, if $[U, f]$ is given then
$$p|[U, f]: [U, f] \to U$$
is a homeomorphism.
So $p$ is locally a homeomorphism.
\end{proof}

\begin{definition}
The \dfn{\'etal\'e space} of $\mathscr F$ is $|\mathscr F|$.
\end{definition}

Sheaves are not useful in the smooth category, except in the study of de Rham cohomology.
This is because if there are too many partitions of unity then the \'etal\'e space will fail to be Hausdorff.
(I think this has to do with the fact that sheaf cohomology becomes trivial in that case.)

If $f$ is a section of a sheaf and $x$ is a point, the germ of $f$ at $x$ is denoted $f_x$.

\begin{definition}
A sheaf $\mathscr F$ has the \dfn{identity property} if for every open connected set $Y$ and $f,g \in \mathscr F(Y)$, if there is a $a \in Y$ such that $f_a = g_a$ then $f = g$.
\end{definition}

\begin{example}
The sheaf $\Olo$ has the identity property. Indeed if $f_x = g_x$ then there is an open neighborhood $W$ of $x$ such that $f|W = g|W$, so $f = g$.
\end{example}

Thus sheaves are, in fact, useful in the study of complex analysis, even though they're not in real analysis.

\begin{proposition}
If $\mathscr F$ is a sheaf with the identity property on a manifold, then $|\mathscr F|$ is Hausdorff.
\end{proposition}
\begin{proof}
Suppose that $f_x \neq g_y$.
If $x \neq y$ then there are open sets $U \ni x, V \ni y$ which are disjoint, so $[U, f]$ and $[V, g]$ separate $f_x,g_y$.
Otherwise, there are $U, V \ni x$ such that $f \in \mathscr F(U)$, $g \in \mathscr F(V)$.
So there is a $W \subseteq U \cap V$ which is connected and open, since the underlying space is a manifold.
We claim that $[W,f],[W,g]$ separate $f,g$.
If not, then there is a $z \in W$ such that $f_z = g_z$, so by the identity theorem $f = g$ and hence $f_x = g_y$, a contradiction.
\end{proof}

\begin{corollary}
Let $X$ be a Riemann surface. Then $|\Olo_X|$ has the structure of a disjoint union of Riemann surfaces such that $p: |\Olo_X| \to X$ is holomorphic.
\end{corollary}
\begin{proof}
Since $|\Olo_X|$ is a Hausdorff space and $p$ is a local homeomorphism, $|\Olo_X|$ is a surface and $p$ defines a complex structure on $|\Olo_X|$.
\end{proof}

Technically we should show that each component of $|\Olo_X|$ is second countable.
However this is not actually very important, because this follows from the fact that $|\Olo_X|$ has a complex structure, according to a theorem of Rado.
We omit the proof.

Let us use sheaves to study analytic continuation.
Here we use $I$ to mean the unit interval.
\begin{definition}
Let $X$ be a Riemann surface and $u$ a path in $X$ starting at $a \in X$.
An \dfn{analytic continuation} of a germ $f_a$ on $X$ along $u$ is a lift $\hat u: I \to |\Olo_X|$ with $\hat u(0) = f_a$.
\end{definition}
Note that $p: |\Olo_X| \to X$ is not a covering space.
Indeed, $p: |\Olo_\CC| \to \CC$ is not a covering space because a germ of $1/z$ cannot be analytically continued along any path which ends at $0$.
This is the canonical example of a local homeomorphism which is not a covering space.

What does the definition really mean?
Let $u: I \to X$ be a path and $f_a$ a germ at $a$.
Then if we can lift $u$ to $\hat u$ which starts at $\hat u(0) = f_a$, then for each point $t \in [0, 1]$, there is some basic open set $[V_t, f_t]$ in $|\Olo_X|$ around $\hat u(t)$.
By compactness of $I$ there are finitely many such basic open sets $[V_i, f_i]$.
These drop to open sets $V_i \subseteq X$ and finitely many holomorphic functions $f_i: V_i \to \CC$.
Then they glue together to a function $f: W \to \CC$ where $W = \bigcup_i V_i$ is an open set that contains $u(I)$.
So from a germ and a path which lifts along $p$ we can find an honest-to-god holomorphic function defined closed to that path.

\begin{lemma}
Let $d: |\Olo| \to |\Olo|$ be the differentiation map, thus
$$df_z = (f')_z.$$
Then $d$ is a covering space.
\end{lemma}
\begin{proof}
The map $d$ is well-defined because every germ extends to a holomorphic function on an honest-to-god open set, and as long as that open set is small enough the choice of open set does not matter.

Now let $f_a \in [U, f]$ be a germ.
If $U$ is small enough then $U$ is simply connected so there is a $F \in \Olo(U)$ with $F' = f$.
So
$$d^{-1}([U, f]) = \bigcup_C [U, F + c]$$
and since each of the $[U, F + c]$ locally look like $\Omega$, if $U$ is small enough then $d$ is a covering space.
Here $F$ is far from $F+c$, $c \neq 0$, because they never define the same germ.
\end{proof}

\begin{definition}
Let $\Omega$ be an open subset of $\CC$, $f \in \Olo(\Omega)$, $\gamma$ a path in $\Omega$.
A \dfn{primitive} of $f$ along $\gamma$ is a path of germs $F_t$, $t \in I$, at $\gamma(t)$, such that $F_t' = f_{\gamma(t)}$.
\end{definition}

We abused notation to write $F_t$ for the germ at $\gamma(t)$.
The point is that a primitive along $\gamma$ is only defined in a very small neighborhood of $\gamma$, rather than all of $\Omega$.

\begin{theorem}[fundamental theorem]
Let $X$ be a Riemann surface, $\gamma$ a path in $X$, and $f \in \Olo(X)$. Then
\begin{equation}
\label{primitive Cauchy}
\int_\gamma f(z) ~dz = F_1(\gamma(1)) - F_0(\gamma(0)).
\end{equation}
\end{theorem}
\begin{proof}
If $X$ is convex then this is obvious (in that it can be proven using the Cauchy-Goursat triangle trick or something idk).
But, as in the construction of an analytic continuation, we can cover $\gamma(I)$ by finitely many convex sets and apply (\ref{primitive Cauchy}) in each convex set.
Since any two primitives are equal up to a constant what we get is independent of the choice of primitive.
Thus we get (\ref{primitive Cauchy}) in general.
\end{proof}

\begin{theorem}[Cauchy]
Let $X$ be a Riemann surface, $\Gamma: \gamma_0 \to \gamma_1$ a based homotopy of paths in $X$, and $f \in \Olo(X)$. Then
$$\int_{\gamma_0} f = \int_{\gamma_1} f.$$
\end{theorem}
\begin{proof}
Consider the based homotopy of germs $G(s, t) = f_{\Gamma(s, t)}$ where $\Gamma(s, t) = \gamma_s(t)$.
Lift $G$ to a based homotopy $\tilde G: I^2 \to |\Olo|$ along $d$; thus for every $s \in I$, $\tilde G(s, 0) = \tilde G(0, 0)$.
Then by definition of $d$, $\tilde G$ is a homotopy of primitives of $f$, so
$$\int_{\gamma_0} f(z) ~dz = \tilde G(0, 1)(\Gamma(0, 1)) - \tilde G(0, 0)(\Gamma(0, 0))$$
by (\ref{primitive Cauchy}).
Moreover if $t \{0, 1\}$ then
$$\tilde G(1, t) = \tilde G(0, t)$$
since $\tilde G$ is a based homotopy, so
\begin{align*}\int_{\gamma_1} f(z) ~dz &= \tilde G(1, 1)(\Gamma(1, 1)) - \tilde G(1, 0)(\Gamma(1, 0))\\
&= \tilde G(0, 1)(\Gamma(0, 1)) - \tilde G(0, 0)(\Gamma(0, 0))\\
&= \int_{\gamma_0} f(z) ~dz
\end{align*}
which was desired.
\end{proof}

\begin{theorem}[monodromy theorem]
Let $u: u_0 \to u_1$ is a based homotopy of paths, $u_0(0) = a$, $u_0(1) = b$.
If $\varphi \in \Olo_a$ has an analytic continuation along every path $u_s$, then the analytic continuation along $u_0$ is equal to the analytic continuation along $u_1$.
\end{theorem}
\begin{proof}
Let $\hat u$ be the lift of $u$ to $|\Olo|$.
Then $\hat u$ is unique since homotopy lifts are unique if they exist.
\end{proof}

\begin{corollary}
Let $X$ be a simply connected Riemann surface, $a \in X$, and $\varphi \in \Olo_a$.
If $\varphi$ has an analytic continuation along every path starting at $a$, then there is a global section $f \in \Olo(X)$ such that $f_a = \varphi$.
\end{corollary}
\begin{proof}
Given any $x$, choose a path $\gamma: a \to x$, and let $\varphi_x$ be the germ at $x$ obtained by the analytic continuation of $\varphi$ along $\gamma$.
By monodromy, since $\pi_1(X) = 0$, $\varphi_x$ does not depend on $\gamma$.
So we can define $f(x) = \varphi_x(x)$.
Then $f$ is holomorphic in a neighborhood of $x$, so $f$ is globally holomorphic.
\end{proof}

\section{Analytic continuation to global sections}
Let us now consider how to globally continue a germ, even in the case that $\pi_1(X)$ is nonzero.
There are two approaches to this: multivalued functions, or to define the Riemann surface of the germ.
The latter was the historical motivation for defining Riemann surfaces in the first place.

\begin{definition}
Let $p: Y \to X$ be an unbranched map, $p(y) = x$.
The \dfn{pullback} by $p$
$$p^*_x: \Olo_x \to \Olo_y$$
on the stalks of $\Olo$ defined by $p^*f_x = (f \circ p)_y$.
Since $p$ is unbranched, we can also define the \dfn{pushforward} $p_* = (p^*)^{-1}$.
\end{definition}

\begin{definition}
Let $a \in X$, $\varphi \in \Olo_a$.
An \dfn{analytic continuation} of $\varphi$ is an unbranched covering space
$$p: Y \to X$$
equipped with a global section $f \in \Olo(Y)$ such that if $p(b) = a$ then $p_*(f_b) = \varphi$.
\end{definition}

\begin{example}
Let $a \in X^*$, $g(z) = z^{1/n}$ defined in some open set $U \ni a$. Consider
$$Y^* = \{(z, w) \in (\CC^*)^2: w = z^n\}.$$
Define $p: Y^* \to \CC^*$, $p(z, w) = w$. Then $g(p(z, w)) = g(z^n) = z$.
So we set $f$ to be the projection onto the first coordinate of $Y^*$ onto $\CC$.
Therefore $(p, f)$ is an analytic continuation of $g$.

Notice that $Y^*$ is isomorphic to $\CC^*$. We can extend $Y^*$ to
$$Y = \{(z, w) \in \CC^2: w = z^n\}$$
which is homeomorphic but not isomorphic to $\CC$. We can see this by noting that $p$ is not conformal to $0$.
\end{example}

\begin{example}
Let $Y^* = \{(z, w) \in (\CC^*)^2: w = e^z\}$.
Then $Y^*$ defines the analytic continuation of $\log$, and defines an infinite-sheeted covering space of $\CC^*$.
\end{example}

\begin{definition}
Let $a \in X$.
Let $p: Y \to X$, $f: Y \to \CC$ define an analytic continuation of $\varphi \in \Olo_a$.
We say that $(p, f)$ is the \dfn{maximal analytic continuation} of $\varphi$ if for every analytic continuation $q: Z \to X$, $g: Z \to \CC$, of $\varphi$, there is an $F: Z \to Y$ such that $g = f \circ F$ and $q = p \circ F$.
\end{definition}

By general abstract nonsense, maximal analytic continuations are unique up to unique isomorphism.

\begin{theorem}
For every $\varphi \in \Olo_a$, $a \in X$, there is a unique maximal analytic continuation of $\varphi$.
\end{theorem}
\begin{proof}
Let $Y$ be the connected component of $\varphi$ in the \'etal\'e space $|\Olo|$.
Let $p: Y \to X$ be the map that sends a germ to its basepoint.
Let $f(\eta) = \eta_{p(\eta)}$ whenever $\eta \in Y$.
Then $f(\varphi) = \varphi(a)$.
Maximality follows from the definition of analytic continuation along paths; any larger analytic continuation would extend off of $Y$ into another connected component of $|\Olo|$.
\end{proof}

\chapter{Line bundles on Riemann surfaces}
This chapter follows lectures of Georgios Daskalopolus.

\section{Vector bundles}
Let $K$ be a topological field. Recall that a vector bundle of rank $n$ over $K$ on a topological space $X$ consists of a continuous map
$$p: E \to X$$
such that $p^{-1}(x) = E_x$ is equipped with a homeomorphism $h_x: E_x \to K^n$ such that for every sufficiently small open $U \subseteq X$, $p^{-1}(U)$ is identified with $U \times K^n$ by a homeomorphism
$$h: p^{-1}(U) \to U \times K^n$$
such that $h|E_x = h_x$.
A morphism of vector bundles $E \to F$ will be a continuous map which commutes with the projections $p_E,p_F$.

\begin{definition}
A \dfn{trivial vector bundle} on $X$ is a vector bundle which is isomorphic to $X \times K^n$.
\end{definition}

\begin{example}
The tangent bundle $TS^3$ is trivial since $S^3$ is a Lie group, so we can parallel transport tangent vectors all around $S^3$ using the multiplication action of $S^3$ on $S^3$.
The hairy ball theorem says that $TS^2$ is nontrivial.
\end{example}

Suppose $X$ is a manifold and $K \in \{\RR, \CC\}$.
By a linear chart on $X$ we mean a chart $U$ such that $p^{-1}(U)$ is identified with $U \times K^n$.
If $U,V$ are linear charts we can form the transition map $h_{UV}: U \cap V \to \GL(n, K)$ which identifies the $K^n$ in $U \times K^n$ with the $K^n$ in $V \times K^n$.

Let $X$ be a manifold.
If $(U_i)_i$ is an open cover of $X$ by charts and we are given maps $h_{ij}: U_i \cap U_j \to \GL(n, \CC)$ which satisfy the cocycle relation, then there exists a vector bundle $E$ with transition maps $h_{ij}$.
If $F$ is also a vector bundle with transition maps $g_{ij}$ and $f: E \to F$ is a morphism of vector bundles, then in local coordinates we get maps
$$f_i: U_i \times K^n \to U_i \times K^n$$
which preserves $U_i$, i.e. for every $v \in K^n$ there is a $w \in K^n$ with $f_i(x, v) = (x, w)$.
Conversely, if we are given local morphisms $f_i$ which satisfy the transition relation
$$f_i(h_{ij}(v_j)) = f_i(v_i) = g_{ij}(f_j(v_j))$$
on the overlaps $(U_i \times U_j) \times K^n$; that is, $f_i \circ h_{ij} = g_{ij} \circ f_j$, then we can find a global morphism $f: E \to F$.

\begin{definition}
A \dfn{section} of a vector bundle $p: E \to X$ is a map $s: X \to S$ such that $p(s(x)) = x$ for every $x \in X$.
The sheaf of sections will be denoted $\Gamma(E)$ or $\Omega^0(E)$.
\end{definition}

Every vector bundle has local sections.
Indeed, if $U$ is a linear chart of $E$, we can set
$$e_i(x) = (x, e_i)$$
where the $e_i$ are a coordinate basis of $K^n$.

Suppose that $h_i: E_i \to U_i \times K^n$ is a local trivialization and $s \in \Gamma(E, X)$.
Then we get local sections $s_i \in \Gamma(E, U_i)$ which satisfy
$$s_i = h_{ij} \circ s_j$$
where $h_{ij} = h_i \circ h_j^{-1}$ are the transition maps.
Conversely, if we are given local sections which commute with the transition maps, then we may define a global section.

An important philosophy in the theory of vector bundles is that every operation in the category of vector spaces also holds in the category of vector bundles.

\begin{definition}
Let $E,F$ be vector bundles over $X$ with transition maps $h,g$ respectively.
Define the \dfn{direct sum vector bundle} $E \oplus F$ to be the vector bundle with transition maps $h_{ij} \oplus g_{ij}$.
Define the \dfn{tensor product vector bundle} $E \otimes F$ to be the vector bundle with transition maps $h_{ij} \otimes g_{ij}$.
Define the \dfn{dual vector bundle} $E^*$ to be the vector bundle with transition maps $h_{ji}^*$.
\end{definition}

One can also define the $\Hom$-bundle $\Hom(E, F)$, and presumably any other bundle associated to a functor from the category of vector spaces to itself.

\section{Holomorphic line bundles}
We restrict to the setting of $K = \CC$, $X$ a Riemann surface, and requiring that all maps involved are holomorphic.

\begin{definition}
A \dfn{line bundle} is a vector bundle of rank $1$.
\end{definition}

In the setting of holomorphic line bundles we may not expect global sections to exist.

\begin{lemma}
Let $L \to X$ be a line bundle.
Let $s$ be a holomorphic global section which is never zero.
Then $L$ is trivial, as we can define an isomorphism $h: L \to X \times \CC$, by
$$h(x, \alpha s(x)) = (x, \alpha s(x))$$
whenever $\alpha \in \CC$.
\end{lemma}
\begin{proof}
Indeed, $s(x)$ spans $E_x$, so the definition of $h$ makes sense.
\end{proof}

\begin{proposition}
Let $X$ be a Riemann surface.
The tangent bundle $TX$ and cotangent bundle $T^*X$ are holomorphic line bundles over $X$.
\end{proposition}
\begin{proof}
Suppose we have a cover by open sets $U_i$ and coordinate maps $z_i: U_i \to \CC$.
The transition maps for $T^*X$ are given by $g_{ij} = dz_j/dz_i$.
The chain rule says that the sections of $T^*X$ are $1$-forms, as they should be.
The tangent bundle is as desired, by duality.
\end{proof}

Fix a linear chart $U$.
On $U$ we define the vector fields over $\RR$ $\partial_x,\partial_y \in \Gamma(TX, U)_\RR$ by setting $\partial_x(z) = (z, e_1)$, and $\partial_y = (z, e_2)$.
We also define the vector field over $\CC$, $\partial_z \in \Gamma(TX, U)_\CC$ by setting $\partial_z(z) = (z, 1)$.
This gives a isomorphism over $\RR$ of the real tangent bundle and the holomorphic tangent bundle, say by $\partial_x \mapsto \partial_z$, $\partial_y \mapsto i\partial_z$.
We also define the $1$-form $dz$ to be the dual to $\partial_z$, thus $dz(\partial_z) = 1$.

We will write $\Omega_h^1$ for the sheaf of holomorphic $1$-forms.

Let $X$ be a Riemann surface.
Let $TX$ denote, for now, the real tangent bundle, and let $TX^\CC = TX \otimes \CC$, the tensor product taken over $\RR$.
Then $TX^\CC$ is a rank-$4$ vector bundle over $\RR$, so it can be given the structure of a rank-$2$ vector bundle over $\CC$.
More generally, if $E$ is a real bundle, then we can define a complex bundle $E^\CC$ which has the same transition functions, since $\GL(2, \RR) \subset \GL(2, \CC)$.
If we identify $2\partial_z = \partial_x - i\partial_y$ then $\partial_z,\overline{\partial_z}$ span $TX^\CC$.
We then write
$$TX^\CC = T^{1,0}X \oplus T^{0,1}X$$
where $T^{1,0X}$ is spanned by $\partial_z$, $T^{0,1}X$ is spanned by $\overline{\partial_z}$.
Then $T^{1,0}X,T^{0,1}X$ are line bundles over $\CC$, and we have an $\RR$-isomorphism $T^{1,0}X \to T^{0,1}X$.

We identify the holomorphic tangent space with $T^{1,0}X$.
To see why we are allowed to do this, let $z$ and $w$ be two different local coordinates on $X$, so the vector fields $\partial_z$ and $\partial_w$ both span the complex line $T^{1,0}X$.
Then
$$\frac{\partial}{\partial w} = \frac{\partial z}{\partial w} \frac{\partial}{\partial z}$$
which is exactly how the holomorphic tangent space transforms.

Meanwhile $T^{0,1}X$ is an antiholomorphic line bundle, in the sense that we have an $\RR$-linear isomorphism of bundles $T^{1,0}X \to T^{0,1}X$ given by $z \mapsto \overline z$.
This is clearly not holomorphic but it is the complex conjugate of a holomorphic map.

We similarly decompose the complexified cotangent bundle
$$(T^*X)^\CC = (T^{1,0})^*X \oplus (T^{0,1})^*X$$
where $(T^{1,0})^*X$ is the holomorphic cotangent bundle.
We then write the exterior power
$$\Lambda^k (T^*X)^\CC = \bigoplus_{p + q = k} \Lambda^p (T^{1,0})^*X \wedge \Lambda^q (T^{0,1})^*X.$$
We are only interested in surfaces, so we simply have $\Lambda^0 (T^*X)^\CC$ is the trivial line bundle and
$$\Lambda^2 (T^*X)^\CC = (T^{1,0})^*X \wedge (T^{0,1})^*X.$$
This space is spanned by the $2$-form $dz \wedge d\overline z$.
The above decomposition of $2$-forms is the \dfn{Hodge decomposition} for Riemann surfaces.
Let
$$\Omega^k X = \Lambda^k (T^*X)^\CC.$$

We then introduce the exterior derivative $d: \Omega^k \to \Omega^{k+1}$ which satisfies
$$d(f_I ~dx_I) = df_I \wedge dx_I.$$
In particular, the exterior derivative maps
$$d: \Omega^{p,q} \to \Omega^{p+1,q} \oplus \Omega^{p,q+1}.$$
In particular, if $f$ is a smooth function we can write
$$df = d'f + d''f$$
where $d'f = \partial_z f ~dz$ and $d''f = \overline \partial_z f ~d\overline z$.
More generally, we decompose the map $d$ into
$$d': \Omega^{p,q} \to \Omega^{p+1,q}$$
and
$$d'': \Omega^{p,q} \to \Omega^{p,q+1}$$
which satisfies $(d')^2 = 0$, $(d'')^2 = 0$, and $d'd'' + d''d' = 0$.

A deep theorem of Nirenberg and Newlander shows that every manifold which admits a decomposition of the exterior derivative $d = d' + d''$ which satisfies the above criteria admits a complex structure.
Thus the decomposition $d = d' + d''$ is the same data as a complex structure.

Now we recall that $\Omega^k$ is functorial: if $f: X \to Y$ is a smooth map then the pullback $f^*: \Omega^k(Y) \to \Omega^k(X)$ is defined by
$$(f^*\omega)_x(X_1, \dots, X_k) = \omega_{f(x)}((df)_xX_1, \dots, (df)_xX_k).$$
Then $df^* = f^*d$ and $f^*(\omega \wedge \eta) = f^*\omega \wedge f^*\eta$, so the differential forms define a cohomology theory.
We also have $f^*(dg) = d(g \circ f)$ and more generally $f^*(g\omega) = (g \circ f) \cdot f^*\omega$.

\section{Residues of differential forms}
The invariant definition of a residue requires us to work with $1$-forms rather than functions.

\begin{definition}
A $1$-form is said to be \dfn{holomorphic} if it can be written locally in the form $f~dz$.
\end{definition}

\begin{definition}
Let $a \in X$. Let $\omega$ be a holomorphic $1$-form on $X \setminus a$. The \dfn{residue} of $\omega$ at $a$ is
$$\Res_a \omega = c_{-1}$$
where
$$\omega = \sum_{n=-\infty}^\infty c_n z^n ~dz.$$
\end{definition}

Remarkably, the other coefficients, say $c_{10}$, depend on a choice of coordinates, but $c_{-1}$ does not.

\begin{theorem}
The residue is well-defined.
\end{theorem}
\begin{proof}
We may assume that $X$ is an open submanifold $V$ of $\CC$ and $a = 0$.

Suppose that $\omega = dg$ is exact. Then $g$ is meromorphic with a pole at $0$, and
$$\Res_0 \omega = 0.$$
Indeed, in this case, we can write
$$g(z) = \sum_{n=-\infty}^\infty c_n z^n$$
and thus
$$\omega = \sum_{n-\infty}^\infty nc_n z^{n-1} ~dz.$$
Then $\Res_0 \omega = 0 \cdot c_0 = 0$.

Now if $\varphi \in \Olo(V)$ and $\varphi$ has a single zero at $0$, then
$$\Res_0 \frac{d\varphi}{\varphi} = 1.$$
Indeed, we can write $\varphi = zh$ where $h(0) \neq 0$, so $d\varphi = h~dz + z ~dh$.
Then $d\varphi/\varphi = dz/z + dh/h$. Now we apply the Cauchy integral formula at $0$ and use the fact that $h \neq 0$.

Now suppose $\omega = f~dz$ and
$$f(z) = \sum_{n=-\infty}^\infty c_n z^n.$$
Then
$$\omega = dg + c_{-1} \frac{dz}{z}$$
where
$$g(z) = \sum_{n \neq -1} \frac{c_n}{n+1} z^{n+1}.$$
Now let $h$ be a change of coordinates. Then
$$h^*\omega = d(h \circ g)+ c_{-1} \frac{dh}{h}.$$
Since $d(h \circ g)$ is exact we might as well assume $d(h \circ g) = 0$. Thus
$$\Res_0 h^*\omega = c_{-1} \Res_0 \frac{dh}{h} = c_{-1} = \Res_0 \omega$$
as desired.
\end{proof}

Recall that $H^p_{dR}(X)$ is the space of closed $p$-forms modulo exact $p$-forms on $X$, i.e. the $p$th de Rham cohomology of $X$.
In particular the Hurcewiz theorem implies that $H^1_{dR}(X)$ is the abelianization of $\pi_1(X)$, so that $H^1_{dR}(X) = 0$ if $X$ is simply connected.

\begin{theorem}
Let $\omega$ be a closed $1$-form and $u,v$ two based-homotopic paths in $X$. Then
$$\int_u \omega = \int_v \omega.$$
\end{theorem}
\begin{proof}
Let $\pi: \tilde X \to X$ be the universal cover of $X$.
Since $\tilde X$ is simply connected and $\omega$ is closed, $p^*\omega$ is exact, say $p^*\omega = dF$.
Then
\begin{align*}\int_u \omega &= \int_{p^*u} p^*\omega = \int_{p^*u} dF\\
&= F(p^*u(1)) - F(p^*u(0)) = F(p^*v(1)) - F(p^*v(0))\\
&= \int_{p^*v} dF = \int_v \omega
\end{align*}
as desired.
\end{proof}

This can even be extended to the case that $u,v$ are just homotopic rather than based-homotopic, assuming they are loops.

\begin{corollary}
Let $\omega$ be a closed $1$-form and $u,v$ are homotopic, then
$$\int_u \omega = \int_v \omega.$$
\end{corollary}
\begin{proof}
Let $w$ be a path from the basepoint of $u$ to the basepoint of $v$.
Then $[u] = [w][v][w]^{-1}$ in $\pi_1(X)$. Then
$$\int_u \omega = \int_{wvw^{-1}} \omega = \int_w \omega + \int_v \omega - \int_w \omega$$
as desired.
\end{proof}

That corollary shows that the following definition makes sense.

\begin{definition}
Let $\omega$ be a closed $1$-form. Define a map
$$a_\omega: \pi_1(X) \to \CC$$
by
$$a_\omega(\sigma) = \int_\sigma \omega,$$
the \dfn{period} of $\omega$.
\end{definition}

By additivity, the period is a morphism of groups.
For example it factors through the abelianization $H_1(X)$ of $\pi_1(X)$.

\begin{theorem}
Let $\omega$ be a closed $1$-form. Then the period $a$ of $\omega$ is zero iff $\omega$ is exact.
\end{theorem}
\begin{proof}
If $\omega$ is exact then this is obvious.
Conversely, let $p: \tilde X \to X$ be the universal cover of $X$.
Then $p^*\omega$ is exact since $\pi_1(\tilde X) = 0$, and $\sigma \in \pi_1(X)$ induces a deck map of $\tilde X$, so
$$dF = (p \circ \sigma)^* \omega = \sigma^* p^* \omega = d(F \circ \sigma)$$
provided $dF = p^*\omega$.
If $a_\omega = 0$, write $a_\sigma = F \circ \sigma - F$.
We claim $a_\sigma = a_\omega(\sigma)$. In fact
$$a_\omega(\sigma) = \int_{p^*\sigma} dF = F(p^*\sigma(1)) - F(p^*\sigma(0)) = F(\sigma(x)) - F(x) = a_\sigma(x)$$
where $x = p^*\sigma(0)$ and $\sigma$ is viewed as a deck transformation.
But we are assuming that $a_\omega = 0$ so $a_\sigma = 0$.
Since $\sigma$ was arbitrary, $F$ is invariant under the action of $\pi_1(X)$, thus $F$ is constant on every fiber of $p$.
So $F$ drops to a map $f: X \to \CC$ such that $F = f \circ p$, and hence $df = \omega$.
\end{proof}

Since $X$ is orientable, every $2$-form can be written $f~dV$ where $dV = dz \wedge d\overline z$ is the volume form.
Therefore the zeroth and second de Rham cohomologies of $X$ are
$$H^0(X) = H^2(X) = \CC.$$
We now show that de Rham's theorem holds for $H^1(X)$.
By the above theorem, the map $\Phi(\omega) = a_\omega$ drops from closed forms to $H^1_{dR}(X)$ where we use $dR$ to denote the de Rham cohomology, and so $\Phi$ defines a map
$$\Phi: H^1_{dR}(X) \to H^1_{sing}(X)$$
where $sing$ denotes singular cohomology, thus
$$H^1_{sing}(X) = \Hom(\pi_1(X)^{ab}, \CC)$$
is the Pontraygin dual of the abelianization of $\pi_1(X)$.

We will later show that $\Phi$ is an isomorphism.
Since we know $\Phi$ is injective, if $X$ has genus $g$ then we just need to show
$$\dim H^1_{dR}(X) = 2g.$$
This will be done with sheaf cohomology.

\section{Dolbeault and sheaf cohomology}
Now we treat the case when we break up the de Rham complex into holomorphic and antiholomorphic parts.
A $1$-form is holomorphic iff it is of the form $f~dz$ where $f$ is holomorphic.

One has a commutative square
$$\begin{tikzcd}\Omega^{0,0}(X) \arrow[d,"\partial"] \arrow[r,"\dbar"] & \Omega^{0,1}(X) \arrow[d,"\partial"]\\
\Omega^{1,0}(X) \arrow[r,"\dbar"] & \Omega^{1,1}(X)\end{tikzcd}.$$
Note however that the two paths in this square are not exact.

\begin{lemma}
If $X$ is compact and $\omega_1,\omega_2$ are holomorphic $1$-forms with $a_{\omega_1} = a_{\omega_2}$ then $\omega_1 = \omega_2$.
\end{lemma}
\begin{proof}
Let $\omega = \omega_1 - \omega_2$; then $a_\omega = 0$.
So $\omega$ is exact, say
$$\omega = \partial f + \dbar f.$$
Since $\omega$ is holomorphic, $\dbar f = 0$ and $f$ is holomorphic.
Since $X$ is compact, $f$ is constant, and hence $\omega = 0$.
\end{proof}

The chain complex induced by $\dbar$
$$0 \to \Omega^{1,0}(X) \to \Omega^{1,1}(X) \to 0$$
induces a cohomology theory
$$0 \to H^{1,0}(X) \to H^{1,1}(X) \to 0$$
where $H^{1,0}(X)$ is the kernel of $\dbar$ and $H^{1,1}(X)$ is the cokernel of $\dbar$.

\begin{definition}
The chain complex $\dbar$ is called the \dfn{Dolbeault complex}.
The cohomology theory associated to $\dbar$ is called \dfn{Dolbeault cohomology}.
\end{definition}

\begin{definition}
The \dfn{Hodge number} $h^{p,q}(X)$ is the dimension of $H^{p,q}(X)$.
\end{definition}

We have just showed that $h^{1,0} - h^{1,1}$ is the Fredholm index of $\dbar$.
I think this is a trivial case of the Atiyah-Singer index theorem.

To proceed any further we need to talk about sheaf cohomology.
\begin{definition}
Assume $\mathscr F$ is a sheaf valued in an abelian category on a manifold $X$ and $q \in \NN$.
If $\mathcal U$ is an open cover of $X$, let
$$C^q(\mathcal U, \mathscr F) = \prod_{U_0, \dots, U_q \in \mathcal U} \mathscr F\left(\bigcap_{j=0}^q U_j\right)$$
be the $q$-\dfn{sheaf cochain group} associated to $\mathscr F$.
Define
$$\delta: C^q(\mathcal U, \mathscr F) \to C^{q+1}(\mathcal U, \mathscr F)$$
as follows.
Let
$$(\delta f)_{U_0, \dots, U_q} = \sum_{\ell=0}^{q+1} (-1)^\ell f_{U_0,\dots,\widehat{U_\ell},\dots,U_{q+1}}\bigg|\bigcap_{\ell=0}^{q+1} U_\ell.$$
In particular, $\delta: C^0 \to C^1$ has
$$(\delta f)_{U,V} = (f_V - f_U)|U \cap V$$
and $\delta: C^1 \to C^2$ has
$$(\delta f)_{U,V,W} = (f_{V,W} - f_{U,W} + f_{U,V})|U \cap V \cap W.$$
Since this is an alternating sum $\delta^2 = 0$.
Define the $q$-\dfn{sheaf cocycle group} $Z^q(\mathcal U, \mathscr F)$ to be the kernel of $\delta: C^q \to C^{q+1}$ and the $q$-\dfn{sheaf boundary group} $B^q(\mathcal U, \mathscr F)$ to be the image of $\delta: C^{q-1} \to C^q$.
Then we obtain the $q$-\dfn{sheaf cohomology group}
$$H^q(\mathcal U, \mathscr F) = \frac{Z^q(\mathcal U, \mathscr F)}{B^q(\mathcal U, \mathscr F)}$$
with respect to $\mathcal U$.
\end{definition}
This definition is problematic because it depends on the choice of open cover.
The fix will be to take a limit over finer and finer covers.

Consider $\delta: C^0 \to C^1$.
If $(\delta f)_{U,V} = 0$ then $f_V - f_U = 0$ on $U \cap V$.
Thus $\ker \delta$ is the space of sections that agree on all overlaps, in other words
$$H^0(X, \mathscr F) = \mathscr F(X)$$
is the space of global sections, independent of the choice of cover.

\begin{example}
A holomorphic line bundle consists of the data of an open cover $\mathcal U$ and transition maps $g_{UV}: U \cap V \to \CC^*$, $U,V \in \mathcal U$, where
$$g_{UV}g_{VW} = g_{UW}.$$
Let $\mathscr O^*$ be the sheaf of nonzero holomorphic functions under multiplications.
Then $g_{UV} \in \ker \delta$ where $\delta$ is the boundary map of the sheaf cohomology of $\mathscr O^*$.
So $Z^1(\mathcal U, \mathscr O^*)$ is the space of holomorphic line bundles of $X$ which trivialize under elements of $\mathcal U$.
Similarly $B^1(\mathcal U, \mathscr O^*)$ will be the space of trivial line bundles.
So $H^1(X, \mathscr F)$, once it has been shown to not depend on $\mathcal U$, will be the moduli space of holomorphic line bundles on $X$.
In particular that moduli space has the structure of an abelian group.
\end{example}

Let $\tau: \mathcal U \to \mathcal V$ be a refinement.
Now define $\tau: Z^1(\mathcal U) \to Z^1(\mathcal V)$ which maps a cocycle $(f_{U_1\cap U_2})$ to $(f_{\tau(U_1) \cap \tau(U_2)})$.
This easily can be generalized to $Z^p$.
Once this is done, we have $[\delta, \tau] = 0$, so $\tau$ drops to a map on cohomology, $\tau: H^p(\mathcal U) \to H^p(\mathcal V)$.

\begin{lemma}
Let $\mathcal V$ be a refinement of $\mathcal U$.
The map
$$\tau: H^1(\mathcal U, \mathscr F) \to H^1(\mathcal V, \mathscr F)$$
is independent of the refinement map chosen, and is injective.

Moreover, $\tau$ is functorial in the sense that $\mathcal U \to \mathcal V \to \mathcal W$ is a sequence of refinements, then $\tau_{\mathcal W}^{\mathcal V} \tau_{\mathcal V}^{\mathcal U} = \tau_{\mathcal W}^{\mathcal U}$.
\end{lemma}
\begin{proof}
This is a boring exercise.
As an example we show that $\tau$ is injective on $H^1$.
The other $H^p$ are similar but the notation is worse.

Let $\mathcal U = \{U_\alpha\}_{\alpha \in A}$, $\mathcal V = \{V_i\}_{i \in I}$.
By the first part we may choose a refinement, which induces a map $\tau: I \to A$.
Suppose $\tau[f] = 0$, then there is a cocycle $g$ such that $(\tau f)_{k \ell} = g_k - g_\ell$ on $V_k \cap V_\ell$.
On $U_\alpha \cap V_k \cap V_\ell$ we get
$$g_k - g_\ell = f_{\tau(k) \tau(\ell)} - f_{\tau(k)\alpha} + f_{\alpha \tau(\ell)} = f_{\alpha(\tau) \ell} - f_{\alpha(\tau) k}.$$
That is,
$$g_k + f_{\alpha(\tau) k} = g_\ell + f_{\alpha \tau(\ell)}.$$
Since $\mathscr F$ is a sheaf, there is $h_\alpha \in \mathscr F(U_\alpha)$ such that $h_\alpha|U_\alpha \cap V_k = f_{\alpha \tau(k)} + g_k$.
On $U_\alpha \cap U_\beta \cap V_k$ we have $f_{\alpha \beta} = f_{\alpha(\tau) k} + f_{\tau(k) \beta} = h_\alpha - h_\beta$.
Therefore $f$ is a coboundary and dies in $H^1$.
\end{proof}

Therefore we can consider the poset of all covers with respect to refinement, and it is meaningful to take the colimit over that poset of the cohomology.

\begin{definition}
Let $\mathscr F$ be a sheaf on $X$ as above.
The \dfn{sheaf cohomology} $H^p(X, \mathscr F)$ is defined to be the colimit of $H^p(\mathcal U, \mathscr F)$ taken over all open covers $\mathcal U$.
\end{definition}

Now let's show that sheaf cohomology is totally useless as a tool for studying real analysis.

\begin{definition}
Let $\mathscr F$ be a sheaf. We say that an open set $U$ is \dfn{acyclic} with respect to $\mathscr F$ if $H^p(\mathscr F|U) = 0$.
\end{definition}

\begin{proposition}
Every sheaf of smooth differential forms is acyclic on every open set.
\end{proposition}
\begin{proof}
Let $\mathcal U$ be a locally finite open cover of the open set and $\rho$ a partition of unity subordinate to $\mathcal U$.
Let $f$ be a cocycle and let $g_i = \sum_j \rho_j f_{ij}$.
Then
$$(\delta g)_{ij} = \sum_k \rho_k(f_{ik} - f_{jk}) = \sum_k \rho_k f_{ij} = f_{ij}.$$
Therefore $f$ is a coboundary.
\end{proof}

\begin{theorem}[Leray]
If $\mathscr A$ is an acyclic cover for $\mathscr F$, then
$$H^p(\mathscr A, \mathscr F) = H^p(X, \mathscr F).$$
\end{theorem}
\begin{proof}
Let $\mathscr B$ be a refinement of $\mathscr A$.
It suffices to show that the map $\tau: H^p(\mathscr A, \mathscr F) \to H^p(\mathscr B, \mathscr F)$ is an isomorphism, i.e. that any cocycle $f$ for $\mathscr B$ is a coboundary, or equivalently is a cocycle for $\mathscr A$.

Let $U \in \mathscr A$, then $\mathscr B$ intersects with $U$ to a cover $\mathscr B_U$ of $U$.
Then if $V_1, V_2 \in \mathscr B$, since $\mathscr A$ is cyclic we can find a cocycle $g$ with
$$f_{V_1V_2} = g_{UV_1} - g_{UV_2}.$$
If $U'$ is also an open set then
$$g_{U' V_1} - g_{U V_1} = g_{U' V_2} - g_{U V_2}$$
on the intersection of all these open sets.

Thus we can glue the $g$'s together to get $F_{UU'}$ which restricts to $g_{U' V} - g_{U V}$ for any $V \in \mathscr B$.
Then $F$ is a cocycle for $\mathscr A$.
Let $h_\alpha = g_{\tau(\alpha),\alpha}|V_\alpha$.
We conclude
$$F_{\tau(\alpha) \tau(\beta)} - f_{\alpha\beta} = h_\beta - h_\alpha$$
so $g$ is a cocycle for $\mathscr A$.
\end{proof}

\section{The category of sheaves}
\begin{definition}
Let $\mathscr F$ and $\mathscr G$ be sheaves. A \dfn{morphism of sheaves} $\alpha: \mathscr F \to \mathscr G$ consists of morphisms
$$\alpha(U): \mathscr F(U) \to \mathscr G(U)$$
which commute with the restriction maps $\mathscr F(U) \to \mathscr F(V)$ and $\mathscr G(U) \to \mathscr G(V)$ whenever $V \subseteq U$.
\end{definition}

\begin{definition}
The \dfn{kernel} $\ker \alpha$ of a morphism of sheaves $\alpha$ consists of the sheaf $U \mapsto \ker \alpha(U)$.
If the kernel is trivial, we say that $\alpha$ is \dfn{injective}.
\end{definition}

\begin{definition}
Let $\Omega^p$ be the sheaf of $p$-forms, and $\Omega^{p,q}$ the sheaf of $(p,q)$-forms.
\end{definition}

\begin{example}
The map $\dbar: \Omega^0 \to \Omega^{0,1}$ is a morphism of sheaves, and $\ker \dbar = \Olo$.
The map $\dbar: \Omega^{1,0} \to \Omega^2$ is also a morphism of sheaves with $\ker \dbar$ the sheaf of holomorphic $1$-forms.
\end{example}

\begin{example}
Let $\Olo^*$ be the sheaf of nonvanishing holomorphic functions under multiplication.
Then $\exp: \Olo \to \Olo^*$ is a morphism of sheaves and $\ker \exp$ is $2\pi \ZZ$.
\end{example}

A morphism of sheaves $\alpha: \mathscr F \to \mathscr G$ induces a morphism on stalks $\alpha_x: \mathscr F_x \to \mathscr G_x$ defined by
$$\alpha_x([f]) = [\alpha(f)]$$
whenever $f$ is a representative of the germ $[f]$ at $x$.

\begin{definition}
Let $\alpha$ be a morphism of sheaves. We call $\alpha$ \dfn{surjective} if the maps on stalks are all surjective.
\end{definition}

Thus we can talk about exact sequences of sheaves.

\begin{lemma}
$\alpha$ is injective iff $\alpha$ is injective on stalks.
\end{lemma}
\begin{proof}
If $\alpha$ is injective then obviously $\alpha$ is injective on stalks.
So assume $\alpha: \mathscr F \to \mathscr G$ is injective on stalks and fix $U$ open.
Then if $f \in \mathscr F(U)$, $\alpha(U)f = 0$, then for every $x$, $\alpha_x(f_x) = 0$.
So since $\alpha$ is injective on stalks, $f_x = 0$ and by the sheaf property $f = 0$.
\end{proof}

The same is not true for surjectivity, which is why the definitions of injectivity and surjectivity are not analogous.
These are the definitions of injective and surjective which agree with their categorical definitions.

As a consequence of the above lemma, a sequence of sheaves is exact iff it is exact on stalks.

\begin{example}
Consider the exact sequence of sheaves
$$0 \to \ZZ \to \Olo \to \Olo^* \to 0$$
where as usual $\Olo \to \Olo^*$ is given by the exponential map, and $\ZZ \to \Olo$ is given by $n \mapsto (z \mapsto 2\pi in)$.
To see that it really is exact, we can check on stalks, using the fact that around every point we can locally find a logarithm for any function in $\Olo^*$.
This is not exact on the level of open sets, since the map $\Olo(\CC^*) \to \Olo^*(\CC^*)$ is not surjective (i.e. not every nonvanishing function on $\CC^*$ has a logarithm).
This example motivates the localization functor in commutative algebra, as if we pass to local rings things become much nicer (since every function has a logarithm).
\end{example}

Algebraic topology is the study of getting long exact sequences from short exact sequences.
We expect the same thing to work here.

\begin{theorem}[zigzag lemma in sheaf cohomology]
Let
$$0 \to \mathscr F \to \mathscr G \to \mathscr H \to 0$$
be a short exact sequence of sheaves. Then there is a long exact sequence in cohomology
$$0 \to H^0(\mathscr F) \to H^0(\mathscr G) \to H^0(\mathscr H) \to H^1(\mathscr F) \to \cdots.$$
Here the maps $H^p \to H^p$ are quotients of alternating sums of the maps on local sections.
\end{theorem}
\begin{proof}
The sequence is evidently exact away from the connecting morphisms
$$\delta_p: H^{p-1}(\mathscr H) \to H^p(\mathscr F).$$
We will construct $\delta_1$ on the level of an open cover $\mathcal U$ and then take a colimit over all such covers; the other $\delta_p$ are similar, as the only difference is that the notation is cleaner.
Let $\beta_0: H^0(\mathscr G) \to H^0(\mathscr H)$ and let $h_i \in U_i$, $U_i \in \mathscr U$.
Then $\beta_0$ is surjective so there is $g_i \in H^0(\mathscr G)$ with $\beta_0(g_i) = h_i$.
Then $g$ splits as $g_i - g_j = \alpha(f_{ij})$ where $\alpha: \mathscr F \to \mathscr G$ and $f_{ij} \in \mathscr F(U_i \cap U_j)$.
All these maps are well-defined so we can set $\delta_1(h)_{ij} = f_{ij}$.
Now a boring diagram chase shows that $\delta$ is exact.
\end{proof}

We can use the zigzag lemma after solving some PDE, for example:

\begin{lemma}[$\dbar$-Poincar\'e lemma with compact support]
Let $X$ be a disc of radius $R$. For every $g \in C^\infty_c(X)$ there is $f \in C^\infty(X)$ such that $\dbar f = g$.
\end{lemma}
\begin{proof}
Let
$$f(\zeta) = \frac{1}{2\pi i} \iint_\CC \frac{g(z)}{z - \zeta} ~dz \wedge d\overline z.$$
Then $f \in C^\infty(X)$. To see this, let $z = \zeta + re^{i\theta}$, and
$$dz \wedge d\overline z = -2ri ~dr \wedge d\theta.$$
Thus
\begin{align*}
f(\zeta) &= \frac{-2i}{2\pi i} \iint_\CC \frac{g(\zeta + re^{i\theta})}{re^{i\theta}}r ~dr \wedge d\theta \\
&= -\frac{1}{\pi} \iint_\CC g(\zeta + re^{i\theta}) e^{-i\theta} ~dr \wedge d\theta
\end{align*}
which is clearly smooth in $\zeta$.

Differentiating, we get
\begin{align*}
\dbar f(\zeta) &= -\frac{1}{\pi} \iint_\CC \dbar_\zeta g(\zeta + re^{i\theta}) e^{-i\theta} ~dr \wedge d\theta\\
&= -\frac{1}{\pi} \lim_{\varepsilon \to 0} \int_0^{2\pi} \int_\varepsilon^R \dbar_\zeta g(\zeta + re^{i\theta}) e^{-i\theta} ~dr \wedge d\theta\\
&= \frac{1}{2\pi i} \lim_{\varepsilon \to 0} \iint_{X \setminus B_\varepsilon} \dbar_z(g(\zeta + z)) ~\frac{dz \wedge d\overline z}{z} \\
&= \frac{1}{2\pi i} \lim_{\varepsilon \to 0} \iint_{X \setminus B_\varepsilon} \dbar_z\frac{g(\zeta + z)}{z} ~dz \wedge d\overline z.
\end{align*}
Let
$$\omega = \frac{1}{2\pi i} \frac{g(\zeta + z)}{z} ~dz$$
which is a $1$-form on $X \setminus B_\varepsilon$.
Then
$$d\omega = -\frac{1}{2\pi i} \dbar_z\frac{g(\zeta + z)}{z} ~dz \wedge d\overline z.$$
Thus
\begin{align*}
\dbar f(\zeta) &= -\lim_{\varepsilon \to 0} \iint_{X \setminus B_\varepsilon} d\omega = -\int_{\partial(X \setminus B_\varepsilon)} \omega\\
&= \int_{\partial B_\varepsilon} \omega
\end{align*}
since the orientation of $\partial B_\varepsilon$ is opposite of its orientation as viewed as a subset of $\partial(X \setminus B_\varepsilon)$, and because $g$ is not supported near $\partial X$.
But then
\begin{align*}
\dbar f(\zeta) &= \lim_{\varepsilon \to 0} \frac{1}{2\pi i} \int_0^{2\pi} \frac{g(\zeta + \varepsilon e^{i\theta})}{\varepsilon e^{i\theta}} i\varepsilon e^{i\theta} ~d\theta \\
&= \lim_{\varepsilon \to 0} \frac{1}{2\pi} \int_0^{2\pi} g(\zeta + \varepsilon e^{i\theta}) ~d\theta,
\end{align*}
the mean of $g$ on the circle $\zeta + \partial B_\varepsilon$. Since $g$ is smooth this converges to $g(\zeta)$.
\end{proof}

\begin{theorem}[$\dbar$-Poincar\'e lemma]
Let $X$ be a disc of radius $R$. For every $g \in C^\infty(X)$ there is $f \in C^\infty(X)$ such that $\dbar f = g$.
\end{theorem}
\begin{proof}
Consider a monotone sequence $R_n \to R$ and let $X_n$ be the disc of radius $R_n$.
Let $\psi_n \in C^\infty_c(X_{n+1})$ be a cutoff to $X_n$.
By the $\dbar$-Poincar\'e lemma with compact support, there is $f_n \in C^\infty(X)$ such that $\dbar f_n = g\psi_n$.

Now let $\tilde f_n \in C^\infty(X)$ satisfy $\dbar \tilde f_n = g$ on $X_n$ and
$$||\tilde f_{n+1} - \tilde f_n||_{L^\infty(X_{n-1})} < 2^{-n}.$$
To see that $\tilde f_n$ exists, let $\tilde f_1 = f_1$.
Given $\tilde f_1, \dots, \tilde f_n$,
$$\dbar(f_{n+1} - \tilde f_n) = g - g = 0$$
on $X_n$. Therefore $f_{n+1} - \tilde f_n$ is holomorphic.
If $p$ is a Taylor polynomial for $f_{n+1} - \tilde f_n$ of high degree, then
$$||f_{n+1} - \tilde f_n - p||_{L^\infty(X_{n-1})} < 2^{-n}.$$
So set $\tilde f_{n+1} = f_{n_1} - p$.

Now let
$$F_n = \sum_{k=n}^\infty \tilde f_{k+1} - f_k.$$
Then $F_n$ is holomorphic on $X_n$.
Set
$$f = \tilde f_n + F_n.$$
This does not depend on the choice of $n$.
To see this, let us show
$$\tilde f_n + F_n = \tilde f_{n+1} + F_{n+1}.$$
In other words we must show
$$F_{n+1} - F_n = \tilde f_{n+1} - \tilde f_n $$
which is clear by a telescoping argument.
Also
$$\dbar f = \dbar \tilde f_n = g$$
on $X_n$, but $n$ was arbitrary, so $\dbar f = g$.
\end{proof}

As an amusing consequence we show that we can invert the Laplace operator on any disc.

\begin{corollary}
Let $X$ be a disc of radius $R$. For every $g \in C^\infty(X)$ there is $f \in C^\infty(X)$ such that $\Delta f = g$.
\end{corollary}
\begin{proof}
Recall
$$\Delta = 4\partial \dbar.$$
By the $\delta$-Poincar\'e lemma, there is $f_1,f_2$ with $\dbar f_1 = g$ and $\dbar f_2 = \overline f_1$.
Let $f = \overline f_2/4$. Then $\Delta f = g$.
\end{proof}

I think we can use this to give another proof of the Riemann mapping theorem which avoids the normal families.

\begin{corollary}
We have a short exact sequence of sheaves
$$0 \to \Olo \to \Omega^0 \to \Omega^{0,1} \to 0$$
where $\Omega^0 \to \Omega^{0,1}$ is given by $\dbar$ and $\Olo \to \Omega^0$ is the inclusion map.
\end{corollary}
\begin{proof}
The Cauchy-Riemann equations show exactness at $\Olo$ and $\Omega^0$.
Now if $g ~d\overline z \in \Omega^{0,1}$ then by the $\dbar$-Poincar\'e lemma there is $f \in \Olo$ such that $\dbar f = g$ which shows exactness at $\Omega^{1,0}$.
This works because locally every Riemann surface looks like a disc, so wecan apply the $\dbar$-Poincar\'e lemma on stalks.
\end{proof}

By similar reasoning we get the de Rham exact sequence
$$0 \to \CC \to \Omega^0 \to \Omega^1 \to 0$$
where $\Omega^0 \to \Omega^1$ is given by $d$, as well as
$$0 \to \CC \to \Olo \to \Omega_{hol}^1 \to 0$$
where $\Olo \to \Omega_{hol}^1$ is $d$ and $\Omega_{hol}^1$ is the sheaf of holomorphic $1$-forms.
We also get a de Rham short exact sequence
$$0 \to \Omega_{hol}^1 \to \Omega^{1,0} \to \Omega^2 \to 0$$
where $\Omega_{hol}^1 \to \Omega^{1,0}$ is the inclusion map and $d$ sends $\Omega^{1,0}$ to $\Omega^2$.

\begin{example}
Let us take the long exact sequence in cohomology of the short exact sequence
$$0 \to \Olo \to \Omega^0 \to \Omega^{0,1} \to 0.$$
We get
$$0 \to H^0(\Olo) \to H^0(\Omega^0) \to H^0(\Omega^{0,1}) \to H^1(\Olo) \to H^1(\Omega^0).$$
But $\Omega^0$ is a fine sheaf so $H^1(\Omega^0) = 0$. Therefore
$$H^1(\Olo) = \frac{H^0(\Omega^{0,1})}{H^0(\Omega^0)} = \frac{\Omega^{0,1}(X)}{\Omega^0(X)}$$
by exactness.
The map $H^0(\Omega^0) \to H^0(\Omega^{0,1})$ is $d''$, so $H^1(\Olo)$ is the space of global $(0,1)$-forms modulo the space of global exterior Cauchy-Riemann derivatives.
Since the Dolbeault cohomology is induced by the diagram
$$0 \to \Omega^0(X) \to \Omega^{0,1}(X) \to 0$$
where the middle arrow is $d''$, $H^1(\Olo)$ is the first Dolbeault cohomology of $X$.
\end{example}

\begin{example}
Consider the exact sequence
$$0 \to \CC \to \Omega^0 \to \Omega^1 \to 0$$
which gives a long exact sequence
$$0 \to H^0(\CC) \to H^0(\Omega^0) \to H^0(\Omega^1) \to H^1(\CC) \to H^1(\Omega^0).$$
Again $H^1(\Omega^0) = 0$ so
$$H^1(\CC) = \frac{H^0(\Omega^1)}{H^0(\Omega^0)}$$
is the first de Rham cohomology of $X$.
Using the canonicity of the Eilenberg-Steenrod axioms, this is also the singular cohomology of $X$ with values in $\CC$, as one would expect.
\end{example}

\section{Dolbeault cohomology}
Let us compute the Dolbeault cohomology of a Riemann surface.
We first show that it is finite-dimensional.

Recall Montel's theorem: if $\Omega$ is an open set in $\CC$ and $\mathcal F \subseteq \Olo(\Omega)$ is locally uniformly $L^\infty$, then $\mathcal F$ is precompact with respect to locally $L^\infty$ convergence.
This can also be phrased as the follows:
\begin{theorem}[Montel]
Let $\Omega_1$ be precompact in $\Omega_2 \subseteq X$, and let $L: \Olo(\overline \Omega_2) \to \Olo(\overline \Omega_1)$ be the restriction map.
Then $L$ is a compact operator with respect to the $L^\infty$ norm.
\end{theorem}
Intuitively this is saying that extensible holomorphic functions are a tiny subspace of holomorphic functions.

\begin{theorem}[Schwartz]
If $E, F$ are Banach spaces and $u, v: E \to F$ are bounded linear operators, $u$ surjective, $v$ compact, then the cokernel $F/F'$ of $u + v$ is finite-dimensional.
\end{theorem}
\begin{proof}
Just show that $F/F'$ has a compact unit ball.
\end{proof}

We will write $h^p(X, \mathscr F) = \dim H^p(X, \mathscr F)$.

\begin{theorem}
If $X$ is compact then $h^1(X, \Olo)$ is finite.
\end{theorem}
\begin{proof}
It suffices to show that if $\mathscr A = (U_i)$ is a (necessarily finite, since $X$ is compact) open cover of $X$ equipped with isomorphisms $f_i: U_i \to \DD$, then $h^1(\mathscr A, \Olo) < \infty$.
Indeed, $\mathscr A$ is an acyclic cover by the $\dbar$-Poincar\'e lemma so we may apply Leray's theorem.
Identifying $U_i$ with the unit disc we let $U_i(r)$ be the disc of radius $r$.
Now fix $r_0 < 1$ so large that $\mathscr A(r) = \{U_i(r)\}$ is still a cover and assume that $r_0 < r < 1$.
Let $Z_b^1(r)$ be the space of $1$-cocycles in $Z^1(\mathscr A(r), \Olo)$ which are in $L^\infty$.
Let $B^1_b(r) = B^1(\mathscr A(r), \Olo) \cap Z^1_b(r)$.
Similarly define $C_b^0(r)$ to $Z_b^1(r)$.

\begin{lemma}
The connecting homomorphism $\delta: C^0(\mathscr A(r), \Olo) \to B^1(\mathscr A(r), \Olo)$, $(\delta f)_{ij} = f_j - f_i$, has
$$\delta^{-1}(B_b^1(r)) = C^0_b(r).$$
\end{lemma}
\begin{proof}
We first show $C^0_b(r) \subseteq \delta^{-1}(B_b^1(r))$.
If $f \in C^0_b(r)$ then $(\delta f)_{ij} - f_j - f_i \in B^1_b(r)$.
So $f \in \delta^{-1}(B_b^1(r))$.

Conversely, if $f \in \delta^{-1}(B_b^1(r))$, then $f_j - f_i \in L^\infty(U_i(r) \cap U_j(r))$.
Set
$$M_{ij} = ||f_j - f_i||_{L^\infty} < \infty.$$
We claim that $M_{ij} < \infty$ in fact propagates to $U_i(r) \cup U_j(r)$ to show $f_i, f_j \in L^\infty$!

By the maximum principle, the maximum of $f_i$ is on $\partial U_i(r)$.
By compactness of $\partial U_i(r)$ it suffices to show that for every $a \in \partial U_i(r)$ there is $V \ni a$ on which $f_i$ is bounded to show that $f_i$ extends to $L^\infty(U_i(r))$.
(Note that $f_i$ extends over $U_i(r)$ since $f_i$ is defined on $U_i$!)
In fact
$$|f_i| \leq M_{ij} + |f_j|$$
if $\partial U_i(r) \cap U_j(r)$.
However, there is some $j$ with this property by how we constructed the cover.
This guarantees the finiteness of the $L^\infty$ norm.
\end{proof}

\begin{lemma}
The $L^\infty$ Dolbeault cohomology $H^1_b(r) = Z^1_b(r)/B_b^1(r)$ is finite-dimensional.
\end{lemma}
\begin{proof}
The map
\begin{align*}
Z^1(\mathscr A) \oplus C^0(\mathscr A(r)) &\to Z^1(\mathscr A(r))\\
(f, g) &\mapsto (f_{ij} + (\delta g)_{ij})_{ij}
\end{align*}
is surjective. Since $\mathscr A(r)$ is a refinement of the acyclic cover $\mathscr A$, we have an isomorphism $H^1(\mathscr A, \Olo) \to H^1(\mathscr A(r), \Olo)$.
But
$$H^1(\mathscr A(r), \Olo) = \frac{Z^1(\mathscr A(r))}{\delta C^0(\mathscr A(r))}.$$
Therefore the map
$$Z^1(\mathscr A) \to  \frac{Z^1(\mathscr A(r))}{\delta C^0(\mathscr A(r))}$$
is surjective.
Taking the direct sum of both sides by $\delta C^0(\mathscr A(r))$ we see the claim (since everything is a vector space and so we can invert a quotient by taking a direct sum).

Choose $r < \rho < 1$.
Then the above map factors through the restriction map
$$Z^1(\mathscr A) \oplus C^0(\mathscr A(r)) \to Z^1(\mathscr A(\rho)) \oplus C^0(\mathscr A(r)) \to Z^1(\mathscr A(r)).$$
So the map
$$Z^1(\mathscr A(\rho)) \oplus C^0(\mathscr A(r)) \to Z^1(\mathscr A(r))$$
is surjective.
We now claim that
$$u: Z^1_b(\rho) \oplus C^0_b(r) \to Z^1_b(r)$$
is also surjective.
If $f \in Z^1_b(r)$ then $f_{ij} = h_{ij} + (\delta g)_{ij}$ where $f_{ij} \in L^\infty$ and $h_{ij}$ is extensible off of $U_i(r) \cap U_j(r)$ and therefore $h_{ij} \in L^\infty(U_i(r) \cap U_j(r))$.
So $\delta g_{ij} \in L^\infty$ and therefore $g \in C^0_b(r)$ by the first lemma.
Therefore $u$ is surjective, and has the form
$$u(f, g) = f + \delta g.$$

Meanwhile,
\begin{align*}
v: Z^1_b(\rho) \oplus C^0_b(r) &\to Z^1_b(r)\\
(f, g) \mapsto -f
\end{align*}
is compact by Montel's theorem.
Also $(u + v)(f, g) = \delta g$.
The image of $\delta$ is closed and its cokernel is $H^1_b(r)$, which is finite-dimensional by Schwartz' theorem.
\end{proof}

With these lemmata proven, choose $r < \rho < 1$.
Since $\mathscr A(r)$ is a refinement of $\mathscr A$ and $\mathcal A$ is acyclic, we have a refining isomorphism
$$H^1(\mathscr A, \Olo) \to H^1(\mathscr A(r), \Olo).$$
Therefore we may show $h^1(\mathscr A(r), \Olo) < \infty$.
But the map $H^1(\mathscr A, \Olo) \to H^1(\mathscr A(r), \Olo)$ factors through the restriction map $H^1(\mathscr A, \Olo) \to H^1_b(\mathscr A(\rho), \Olo)$, which is well-defined on $L^\infty$ cohomology by the first lemma.
But by the second lemma, $h^1_b(\mathscr A(\rho), \Olo) < \infty$, so $h^1(\mathscr A(r), \Olo) < \infty$.
\end{proof}

\begin{definition}
The \dfn{geometric genus} of $X$ is
$$g(X) = \dim H^1(X, \Olo)$$
assuming that $X$ is compact.
\end{definition}

Recall that if $X$ is compact then $H^1(X, \CC)$ is finite-dimensional.

\begin{definition}
The \dfn{Betti number} of $X$ is
$$b_1(X) = \dim H^1(X, \CC)$$
assuming that $X$ is compact. The \dfn{topological genus} of $X$ is $g = b_1/2$.
\end{definition}

A priori it is not even obvious that the topological genus is an integer, but in fact we will show that the geometric and topological genera coincide.
The point is that the topological genus could be defined purely entirely in terms of Eilenberg-Steenrod axioms, even though we defined it using de Rham cohomology.
However the geometric genus can only be defined using the sheaf of holomorphic functions.

Before we check this fact let us compute $H^1(\PP, \Olo)$.

\begin{lemma}
One has $H^1(\PP, \Olo) = 0$.
\end{lemma}
\begin{proof}
If $U_1 = \PP^1 \setminus \{\infty\}$ and $U_2 = \PP^1 \setminus \{0\}$, then by the $\dbar$-Poincar\'e lemma, $U_1,U_2$ form an acyclic cover.
Every section $f$ of $\Olo$ determines a cocycle $f_{12} \in Z^1(\{U_1, U_2\}, \Olo)$, but $f_{12} \in \Olo(U_1 \cap U_2) = \Olo(\CC \setminus \{0\})$. Thus $f_{12}$ has a Laurent series
$$f_{12}(z) = \sum_{n=-\infty}^\infty c_n z^n = f_1(z) - f_2(z)$$
where $f_1$ is the sum over positive terms and so is holomorphic on $U_1$, and $f_2$ is the sum over negative terms and is holomorphic on $U_2$.
Therefore $f_{12}$ is a coboundary, so $H^1(\PP, \Olo) = 0$.
\end{proof}

\section{The first Chern class}
Consider the exact sequence
$$0 \to \ZZ \to \Olo \to \Olo^* \to 0$$
given by the exponential map. The long exact sequence includes
$$\cdots \to H^1(X, \ZZ) \to H^1(X, \Olo) \to H^1(X, \Olo^*) \to H^2(X, \ZZ) \to \cdots.$$
Recall that $H^1(X, \Olo^*)$ is the moduli space of line bundles on $L$, and $H^2(X, \ZZ) = \ZZ$.

\begin{example}
If $X$ is a disc, then $X$ is contractible and by the $\dbar$-Poincar\'e lemma and a diagram chase we get
$$H^1(X, \Olo) = H^1(X, \Olo^*) = 0$$
so all line bundles on $X$ are trivial.
This would be rather hard to show otherwise.
\end{example}

\begin{definition}
Let
$$\delta: H^1(X, \Olo^*) \to H^2(X, \ZZ)$$
be the connecting morphism, and $L$ a line bundle.
The first \dfn{Chern class} $c_1(L)$ is $c_1(L) = \delta(L)$.
\end{definition}

The first Chern class is an integer, whose sign is determined by a choice of orientation of $X$.

\begin{example}
Since $H^1(\PP^1, \Olo) = 0$, the long exact sequence for $\PP^1$ is
$$0 \to H^1(\PP^1, \Olo^*) \to H^2(\PP^1, \ZZ) \to 0.$$
Therefore $c_1$ is an isomorphism on $\PP^1$. So the space of line bundles of $\PP^1$ is isomorphic to $\ZZ$, and the Chern class is a complete invariant.
\end{example}

The above example worked because $\PP^1$ has zero genus.
More generally, the kernel of $c_1$ is the image of $H^1(X, \Olo)$ in $H^1(X, \Olo^*)$, thus
$$\ker c_1 = \frac{H^1(X, \Olo)}{H^1(X, \ZZ)} = \frac{\CC^g}{\ZZ^{2g}}.$$
But this is nothing more than a torus.

\begin{definition}
The \dfn{Jacobian variety} of $X$ is $\ker c_1$.
\end{definition}

It follows that the Jacobian variety of any algebraic curve is an elliptic curve.
It is the connected component of the identity in the topological group $H^1(X, \Olo^*)$, and thus the space of holomorphic structures on the trivial line bundle on $X$.

\section{The Riemann-Roch theorem}
\begin{definition}
A \dfn{divisor} on $X$ is a map $X \to \ZZ$ whose support is discrete.
The abelian ordered group of divisors is denoted $\Div X$.
\end{definition}
Therefore a divisor is a choice of integers on finitely many points in each compact subset of $X$.

\begin{definition}
If $f \in \Mero^*(X)$, the space of not identically $0$ meromorphic functions, the \dfn{divisor} of $f$, $(f)$, is defined by setting $(f)(x) = 0$ if $x$ is neither a zero nor pole, $(f)(x) = k$ if $f$ has a zero of order $k$ at $x$, and $(f)(x) = -k$ if $f$ has a pole of order $k$ at $x$.
\end{definition}

Note that $(f) \geq 0$ iff $f$ is holomorphic. Also $(fg) = (f) + (g)$, so $(\cdot)$ is a group homomorphism $\Mero^*(X) \to \Div X$.

\begin{definition}
If $\omega \in \Mero^1(X)^*$, we define the \dfn{divisor} of $\omega$, $(\omega)$, by writing $\omega = f~dz$ and setting $(\omega) = (f)$.
\end{definition}

The divisor of a meromorphic $1$-form is well-defined because multiplying by a Jacobian neither creates nor destroys poles or zeroes.

\begin{definition}
Let $D, D'$ be divisors. We say that $D \sim D'$, or $D,D'$ are \dfn{linearly equivalent divisors}, if there is a meromorphic function $f$ with $D - D' = (f)$.
\end{definition}

We will mainly be interested in $\Div X$ modulo linear equivalence, since this will turn out to be $H^1(X, \Olo^*)$, the moduli space of line bundles on $X$. In that case we can recover the first Chern class by summing up the divisor:

\begin{definition}
If $D$ is a divisor and $X$ is compact we define $\deg D = \sum_x D(x)$.
\end{definition}

\begin{definition}
Let $D$ be a divisor, and $\Olo_D$ be the \dfn{sheaf associated to the divisor} $D$,
$$\Olo_D(U) = \{f \in \Mero(U): (f) \geq -D\}$$
where $(0) \geq -D$ for all $D$ by convention.
\end{definition}

For example $\Olo_0 = \Olo$.

\begin{example}
If $P(p) = 1$ and $P(x) = 0$ if $x \neq p$, then $\Olo_P$ is the sheaf of meromorphic functions with at least a simple pole at $p$ and holomorphic everywhere else.
\end{example}

\begin{proposition}
If $D, D'$ are divisors and $D \sim D'$ then $\Olo_D \cong \Olo_{D'}$.
\end{proposition}
\begin{proof}
Suppose that $D - D' = (\psi)$.
Define a morphism of sheaves by, for every $f \in \Olo_D(U)$, $f \mapsto \psi f$.
Then $\psi f \in \Olo_{D'}(U)$ and this map is an isomorphism since $g \mapsto \psi^{-1}g$ is its inverse.
\end{proof}

Now we give a special case of the Kodaira vanishing theorem from algebraic geometry, which says that divisors of negative degree cannot have nonvanishing sheaf cohomology.

\begin{proposition}
Let $X$ be compact and $\deg D < 0$. Then $H^0(X, \Olo_D) = 0$.
\end{proposition}
\begin{proof}
Assume $f \in H^0(X, \Olo_D)$. If $f$ is nonzero then $(f) \geq -D$. Then $\deg (f) \geq -\deg D > 0$.
So $f$ has strictly more zeroes than poles, violating the residue theorem.
\end{proof}

\begin{definition}
Let $p \in X$. The \dfn{skyscraper} at $p$, denoted $\CC_p$, is $\CC_p(U) = 0$ if $p \notin U$ and $\CC_p(U) = \CC$ if $p \in U$.
\end{definition}

The skyscraper is the sheaf-theoretic form of a Dirac measure.

\begin{proposition}
One has $H^0(X, \CC_p) = \CC$ and $H^1(X, \CC_p) = 0$.
\end{proposition}
\begin{proof}
The value of the zeroth cohomology is by definition.
Now let $\mathscr A$ be an acyclic cover such that $p$ lies in exactly one set $U_0 \in \mathscr A$.
Let $f \in Z^1(\mathscr A, \CC_p)$, thus $f_{ij} + f_{jk} = f_{ik}$.
If $p \notin U_i \cap U_j$ then $f_{ij} = 0$.
Otherwise the relation $f_{00} + f_{00} = f_{00}$, since $p$ is only in $U_0$, gives $f_{00} = 0$.
Therefore $f$ is a coboundary.
\end{proof}

Let $D$ be a divisor, $p \in X$. We want an exact sequence
$$0 \to \Olo_D \to \Olo_{D + p} \to \CC_p \to 0$$
which will give us lots of data about the cohomology of line bundles.
If $f \in \Olo_D(U)$ then $(f) \geq -D \geq -D - p$, so we get an inclusion map $\alpha: \Olo_D \to \Olo_{D + p}$.
To define a map $\beta: \Olo_{D + p} \to \CC_p$, fix $U$.
If $p \notin U$, we set $\beta_U = 0$.
Otherwise, $p \in U$, and we choose coordinates in which $p = 0$. Then if
$$f(z) = \sum_{n=-k-1}^\infty c_n z^n,$$
then $k = D(p)$ and we can let $\beta_U(f) = c_{-k-1}$.
Thus $\ker \beta$ is the sheaf of functions of whose poles are at worst of order $k$ at $p$, which is exactly $\Olo_D$.
Conversely we can always construct $f \in \Olo_{D + p}(U)$ with order at most a pole of order $k + 1$ at $p$,
so we really did construct a short exact sequence.

\begin{theorem}[Riemann-Roch]
Let $g$ be the geometric genus of a compact Riemann surface $X$. Then
$$h^0(X, \Olo_D) - h^1(X, \Olo_D) = 1 - g + \deg D.$$
\end{theorem}
\begin{proof}
We prove this by tree induction: first we prove it when $D = 0$, then for every $p \in X$, we show that if it is true for $D$, then it is true for $D \pm p$.
If $D = 0$ then the claim is that
$$1 - g = 1 - g + 0$$
which is obvious.

Now assume that it is true for $D$ and consider the long exact sequence induced by the skyscraper sequence
$$0 \to H^0(\Olo_D) \to H^0(\Olo_{D+p}) \to \CC \to H^1(\Olo_p) \to H^1(\Olo_{D+p}) \to 0.$$
Let $V$ be the image of $H^0(\Olo_{D+p})$ in $\CC$ and $W = \CC/V$.
Then
$$\dim V + \dim W = 1 = \deg(D+p) - \deg D.$$
Moreover, from the long exact sequence,
$$h^0(\Olo_{D+p}) = h^0(\Olo_D) + \dim V.$$
We also get
$$h^1(\Olo_D) = h^1(\Olo_{D+p}) + \dim W.$$
Summing up we get
$$h^0(\Olo_{D + p}) - h^1(\Olo_{D + p}) = \deg(D+p) + h^0(\Olo_D) - h^1(\Olo_D) - \deg D.$$
By our inductive hypothesis,
$$h^0(\Olo_D) - h^1(\Olo_D) - \deg D = 1 - g$$
so this gives the claim.
Reversing the above argument we get the claim for $D - p$.
\end{proof}

The Riemann-Roch theorem is a simple case of the Atiyah-Singer index theorem, which says that the Fredholm index of an elliptic pseudodifferential operator on a vector bundle $V$ can be computed purely from the cohomology of $V$.
Once we show that $g$ is the topological genus and $\deg D$ is the first Chern class of a line bundle, $1 - g + \deg D$ is a purely topological number.
Meanwhile $h^0(X, \Olo_D) - h^1(X, \Olo_D)$ is the Fredholm index of $\dbar$ acting on smooth sections of the line bundle whose holomorphic sections are $\Olo_D$.
Another famous special case is the Gauss-Bonnet theorem.
To see this, consider the de Rham complex. Taking adjoints with respect to the Riemannian metric, we get an elliptic operator
$$d + d^*: \Omega^0 \oplus \Omega^2 \to \Omega^1.$$
Then the Gauss-Bonnet theorem says that the Fredholm index of $d + d^*$ is the Euler characteristic.
The relationship between $d^*$ and the Gaussian curvature is given by Hodge theory.

The classification of Riemann surfaces of genus zero follows from the Riemann-Roch theorem.

\begin{corollary}
Let $X$ be a compact Riemann surface of geometric genus $g$, and suppose that $a \in X$.
Then there is a nonconstant meromorphic function $f$ with a pole of order $\leq g + 1$ at $a$, such that $f$ is holomorphic on $X \setminus \{a\}$.
\end{corollary}
\begin{proof}
Let $D = (g+1)a$.
By the Riemann-Roch theorem,
$$h^0(X, \Olo_D) - h^1(X, \Olo_D) = (1 - g) + g + 1 = 2.$$
Therefore $h^0(X, \Olo_D) \geq 2$.
The space of constant functions in $H^0(X, \Olo_D)$ has dimension $1$.
\end{proof}

\begin{corollary}
Let $X$ be a compact Riemann surface of geometric genus $g$.
Then $X$ is a branched covering space of $\PP^1$ with at most $g + 1$ sheets.
\end{corollary}
\begin{proof}
Let $f: X \to \PP^1$ be the nonconstant map obtained from the Riemann-Roch theorem.
Since the pole has order $\leq g + 1$, the covering has at most $g + 1$ sheets.
\end{proof}

\begin{corollary}
The only compact Riemann surface of geometric genus $0$ is $\PP^1$.
\end{corollary}
\begin{proof}
If $f$ is a nonconstant branched covering space with at most $1$ sheet, then $f$ is an isomorphism.
\end{proof}

One consequence of the Riemann-Roch theorem is that an elliptic curve can be written as
$$y^2 = x^3 + ax + b.$$
This classifies compact Riemann surfaces of geometric genus $1$.

Let us show that the Riemann-Roch theorem is really a statement about line bundles.
Let $E$ be a vector bundle and $\mathscr A$ a linear atlas.
Then we obtain transition functions $g_{ij}$ which are cocycles of sheaf cohomology with coefficients in $\GL(n, \CC)$.
More precisely, $g_{ij} = h_i h_j^{-1}$ where $h_i$ is the $i$th chart of $\mathscr A$.

If $s: X \to E$ is a section of a vector bundle $p: E \to X$, we obtain local sections $s_i: V \to V \times \CC^n$ whenever $V$ is the image of a linear chart.
The transition relation is that $s_i = g_{ij}s_j$.
Let $\Olo_E$ be the sheaf of holomorphic sections of $E$.
Then $\Olo_E$ is locally free over $\Olo$, in the sense that $\Olo_E(U) = \Olo(U)^{\oplus n}$ whenever $U$ is a linear chart.
Similarly let $\Mero_E$ be the sheaf of meromorphic sections.

\begin{theorem}
Let $X$ be a compact Riemann surface, $E \to X$ a holomorphic vector bundle.
Then $h^1(X, \Olo_E)$ is finite.
\end{theorem}
\begin{proof}
The proof is the same as for when $E$ is trivial, as the point is that $\dbar$ is Fredholm.
\end{proof}

\begin{theorem}
Let $X$ be a compact Riemann surface, $E \to X$ a vector bundle, $a \in X$.
Then there is $f \in \Olo_E(X \setminus a)$ which is meromorphic on $X$.
\end{theorem}
\begin{proof}
Let $U$ be a linear chart at $a$.
Consider the section $s_j(z) = (z^{-j}, 0, \dots, 0)$ on $U$.
Now let $\mathscr A = \{U, X \setminus a\}$.
Then $s_j$ defines a cocycle for $\mathscr A$.
Let $k = h^1(X, \Olo_E)$, so $s_0, \dots, s_k$ are linearly dependent, thus there is a holomorphic chain $\eta$ and $c \in \CC^{k+1}$ such that
$$\sum_{j=0}^k c_j s_j = \eta_2 - \eta_1$$
on $U \setminus a$.
Therefore $\eta_1$ defines a meromorphic global section.
\end{proof}

\begin{definition}
Let $D$ be a divisor. Let $E_D$ be the line bundle such that $\Olo_{E_D} = \Olo_D$.
\end{definition}

To see why $E_D$ exists, let $(U_i)$ be a cover where $D|U_i$ consists of at most one point and $\psi_i$ is a meromorphic function on $U_i$ which is a multiple of $D|U_i$ and has no other zeroes or poles.
Now set $g_{ij} = \psi_i/\psi_j$. Since $\psi_i$ has no zeroes or poles on $U_j$, $g_{ij}$ is a transition map.
Let $E_D$ be the line bundle given by the transition maps $g_{ij}$.

\begin{proposition}
With the above construction and $U$ an open set, $\Olo_{E_D}(U) = \Olo_D(U)$.
\end{proposition}
\begin{proof}
Let $f \in \Olo_D(U)$. Since $(f)_x \geq -D(x) = -(\psi_i)_x$, $(f\psi_i) \geq 0$, so there is $f_i \in \Olo(U_i)$ with $f_i = f\psi_i$.
Then $f_i = g_{ij}f_j$ so $f \in \Olo_{E_D}(U)$.
This process can be reversed to prove the other inclusion.
\end{proof}

Recall that $D,D'$ are linearly equivalent if $D - D'$ is the divisor of a global meromorphic section.
The linearly trivial divisors form a subgroup of $\Div X$, normal since $\Div X$ is abelian.

\begin{definition}
Let $\Pic X$, the \dfn{Picard group} of $X$, to be the abelian group of divisors modulo linear equivalence.
\end{definition}

\begin{theorem}
There is a natural isomorphism $\alpha: H^1(X, \Olo^*) \to \Pic X$.
\end{theorem}
\begin{proof}
Let $\xi \in H^1(X, \Olo^*)$ and fix an open cover of $U_i$s.
Let $E$ be the associated line bundle of $\xi$ and $g_{ij}$ its transition relation.
Let $f$ be a nonzero meromorphic section of $E$, so locally it is $f_i = g_{ij}f_j$.
Let $D$ be the divisor which on $U_i$ is the divisor of $(f_i)$, and set $\alpha(\xi) = [D]$ where $[\cdot]$ is the map that annihilates linearly trivial divisors.

Let $\tilde f$ be another meromorphic section and $\tilde D$ be the associated divisor.
Then if $\varphi = \tilde f/f$, $\tilde D - D = (\varphi)$.
Therefore $[\tilde D - D] = 0$.
So $\alpha$ is well-defined.
Since every divisor has an associated line bundle, $\alpha$ is surjective

Finally suppose $\alpha(\xi) = 0$, thus there is a meromorphic section $f$ such that $\alpha(\xi) = [(f)]$.
Suppose $\xi$ induces a cocycle $g$, then $g_{ij} = f_i/f_j$ so $g$ is a coboundary.
\end{proof}

\section{Serre duality and Hodge theory}
Let $X$ be a compact Riemann surface.
Consider the short exact sequence
$$0 \to \Omega \to \tilde \Omega^{1,0} \to \tilde \Omega^2_{cl} \to 0$$
where $\Omega$ is the sheaf of holomorphic $1$-forms, $\tilde \Omega^{1,0}$ is the sheaf of smooth $1,0$-forms, and $\tilde \Omega^2_{cl}$ is the sheaf of closed $2$-forms.
Then we get a right exact sequence in cohomology
$$H^0(X, \tilde \Omega^{1,0}) \to H^0(X, \tilde \Omega^2_{cl}) \to H^1(X, \Omega) \to 0.$$
Thus $\omega \in H^1(X, \Omega)$ can be represented by an element of $H^0(X, \tilde \Omega^2_{cl})$.

\begin{definition}
Let $\omega \in H^1(X, \Omega)$. The \dfn{residue} of $\omega$ is
$$\Res \omega = \frac{1}{2\pi i} \iint_X \omega$$
where $\omega$ is represented as a closed $2$-form.
\end{definition}

Let $D$ be a divisor. Then multiplication defines a map $\Omega_{-D} \times \Omega_D \to \Omega$ where $\Omega_D(U)$ consists of meromorphic $1$-forms whose poles are no worse than $-D$.

\begin{definition}
The \dfn{Serre pairing} is the map
\begin{align*}
H^0(X, \Omega_{-D}) \times H^1(X, \Olo_D) &\to H^1(X, \Omega)\\
(\omega, (f_{ij})_{ij}) &\mapsto (\omega f_{ij})_{ij}.
\end{align*}
\end{definition}

The Serre pairing induces a linear map
$$i_D: H^0(X, \Omega_{-D}) \to H^1(X, \Olo_D)^*.$$
In the language of line bundles, there is a \dfn{dualizing sheaf} $\Omega$ -- the sheaf of sections of the cotangent bundle of $X$ -- a line bundle such that for every line bundle $L$ we get an isomorphism
$$i_L: H^0(X, \Omega \otimes L^*) \to H^1(X, L)^*.$$

\begin{theorem}[Serre]
The map $i_D$ is a natural isomorphism.
\end{theorem}
\begin{proof}
TODO
\end{proof}

Applying Serre duality to $D = 0$ we get an isomorphism
$$H^1(X, \Olo)^* \to H^0(X, \Omega) = \Omega(X).$$
Since $g = h^1(X, \Olo)$, the space of holomorphic $1$-forms $\Omega(X)$ has dimension $g$.

Taking the dual of Serre duality, we get an isomorphism $H^1(X, \Olo_{-D}) \to H^1(X, \Omega_D)^*$.
Indeed, $\Omega$ is the sheaf of sections of $(T^{1,0}X)^*$ -- that is, global meromorphic $1$-forms.
Let $K$ be a \dfn{canonical divisor} -- that is, the divisor of a global meromorphic $1$-form.
Then $K$ is uniquely defined up to linear equivalence.
So $\Omega = \Olo_K$, so we get an isomorphism $H^0(X, \Omega_{-D-K}) \to H^1(X, \Olo_{D + K})^*$.
But $\Omega_{-D-K} \cong \Olo_{-D}$ and $H^1(X, \Olo_{D + K}) \cong H^1(X, \Omega_D)$ where all the isomorphism are natural.

Now let $K$ be a canonical divisor. Then $\deg K = 2g - 2$.
Indeed, by the Riemann-Roch theorem
$$h^0(X, \Olo_K) - h^1(X, \Olo_K) = 1 - g + \deg K$$
but by Serre duality, $h^0(X, \Olo_K) = h^0(X, \Omega) = h^1(X, \Olo) = g$ while $h^1(X, \Olo_K) = 1$.

We conclude that every elliptic curve has genus $1$, since $dz$ drops to a generator of $H^0(X, (T^{1,0}X)^*)$.
Therefore $\deg K = 0$, so $2g = 2$.

\begin{theorem}[Riemann-Hurwitz]
Let $Y$ be a compact Riemann surface.
Let $f: X \to Y$ be an $n$-sheeted branched covering space.
Let $b(x) = v(x) - 1$ where $f(z) = z^{v(x)}$ for some chart at $x$.
Let $b = \sum_x b(x)$ be the \dfn{branching number} of $x$.
Then
$$g(X) = \frac{b}{2} + n(g(Y) - 1) + 1.$$
\end{theorem}
\begin{proof}
Let $\omega$ be a meromorphic $1$-form on $Y$.
Then we can write $\omega = \psi(w) ~dw$ and $f^*\omega = k\psi(z^k) z^{k-1}~dz$ in coordinates at $x \in X$ where $f(z) = z^k$.
Then
$$(f^*\omega)(x) = k (\omega)(w) + (k-1) = v(x) (\omega)(f(x)) + b(x).$$
Therefore $(f^*\omega)(x) = v(x)(\omega)(f(x)) + b(x)$.
Summing up,
$$\sum_y \sum_{x \in f^{-1}(y)} (f^*\omega)(x) = \sum_y \sum_{x \in f^{-1}(y)} b(x)$$
so
$$\sum_x (f^*\omega)(x) = n \sum_y (\omega)(y) + b$$
so $2g(X) - 2 = n(2g(Y) - 2) + b$.
\end{proof}

We now introduce a Hodge star operator for Riemann surfaces.
Suppose $\omega = f ~dz + g~d\overline z$, then $\overline \omega = \overline g ~dz + \overline f ~d\overline z$.
\begin{definition}
The \dfn{Hodge star operator} is the operator
$$*: \tilde \Omega^1 \to \tilde \Omega^1(X)$$
defined by $*(\omega_1 + \omega_2) = i(\overline \omega_1 - \overline \omega_2)$, where $\omega_1 \in \Omega^{1,0}$ and $\omega_2 \in \Omega^{0,1}$.
\end{definition}
Then we can easily compute $*^2 = -1$.
We also check
$$d*(\partial f + \dbar f) = i\partial \dbar \overline f - i\dbar \partial \overline f = 2i \partial \dbar f.$$

\begin{definition}
A $1$-form $\omega$ is \dfn{harmonic} if $d\omega = d*\omega = 0$.
Let $\Harm X$ be the space of harmonic $1$-forms.
\end{definition}

\begin{proposition}
The following are equivalent:
\begin{enumerate}
\item $\omega$ is harmonic.
\item $\omega$ is the sum of a holomorphic and an antiholomorphic form.
\item For every $a$ there is a $U$ open and a harmonic function $f$ with $\omega = df$.
\end{enumerate}
\end{proposition}
\begin{proof}
The equivalence of the first three is obvious because $\partial,\dbar$ anticommute.
Now suppose $\omega$ is harmonic.
By the Poincar\'e lemma we can locally write $\omega = df$ and $d*\omega = 0$. Therefore $d*df = 0$ but $d*d= -2i\Delta$.
The converse is similar.
\end{proof}

In particular $\Harm X = \Omega(X) \oplus \overline{\Omega(X)}$.
So $\dim \Harm X = 2g$.

\begin{theorem}[Hodge]
One has a natural isomorphism $H^1(X, \CC) \cong \Harm X$.
In particular, the topological and geometric genera are equal.
\end{theorem}

This follows from Hodge theory in general.

\section{Abel's theorem and principal divisors}
Let $X$ be a compact Riemann surface, and $\Div_P(X)$ be the set of principal divisors on $X.$
By a solution to a divisor $D$ we mean $f$ such that $(f) = D$.
Thus a divisor is soluble iff it is principal.
Furthermore, we say that $f$ is a weak solution of $D$ iff $f$ is smooth on $X_D = \{x \in X: D(x) \geq 0\}$ and for every $a \in X$ there are coordinates $z$ near $a$ such that $z(a) = 0$ and $f(z) = \psi(z) z^k$ where $k = D(a)$ and $\psi$ is smooth without a zero.

Let $f$ be a weak solution of $D$. Then $f$ is a solution of $D$ iff $f$ is holomorphic on $X_D$.

\begin{lemma}
Let $a_1, \dots, a_n \in X$, $k_1, \dots, k_n \in \ZZ$, and let $D = \sum_i k_i a_i$.
Suppose that $f$ is a weak solution of $D$. Then there is $g \in C^\infty(X)$ such that
$$\frac{1}{2\pi i} \iint_X \frac{df}{f} \wedge dg = \sum_{i=1}^n k_i g(a_i).$$
\end{lemma}
\begin{proof}
Choose coordinates $z_j$ centered on $a_j$ that witness that $f$ is a weak solution. If $g$ is supported away from $D$,
$$\iint_X \frac{df}{f} \wedge dg = - \iint_X dg \wedge \frac{df}{f} = - \iint_X d(g ~df) = 0$$
since $X$ is a closed surface.
So we can cut off $g$ to $g_j$ close to $a_j$, then
$$\iint_X \frac{df}{f} \wedge dg = \sum_j \iint_{U_j} \frac{df}{f} \wedge dg_j = \sum_j \iint_{U_j} k_j \frac{dz_j}{z_j} \wedge dg_j + \frac{d\psi_j}{\psi_j} \wedge dg_j$$
where $f = z_j^{k_j} \psi_j$ and $U_j$ is an open set near $g_j$.
The second term is exact since $\psi_j$ is nonzero (and so we can rewrite it as $d(\log \psi_j ~dg_j)$), thus
$$\iint_X \frac{df}{f} \wedge dg = \sum_j k_j \iint_{U_j} \frac{dz_j}{z_j} \wedge dg_j.$$
Then
$$\iint_{U_j} \frac{dz_j}{z_j} \wedge dg_j = -\lim_{r \to 0} \iint_{r < |z_j| \leq r_j} d(g_j ~dz_j/z_j) = \lim_{r \to 0} \int_{|z_j| = r} g_j ~\frac{dz_j}{z_j} = 2\pi i g_j(a_j)$$
as desired.
\end{proof}

\begin{lemma}
Let $c$ be a path in $X$. Then there is a weak solution $f$ to $\partial c$ such that for every holomorphic $1$-form $\omega$ such that $d\omega - 0$,
$$\int_c \omega = \frac{1}{2\pi i} \iint_X \frac{df}{f} \wedge \omega.$$
\end{lemma}
\begin{proof}
First assume $c$ is in a chart $z$, and assume then that $c$ is in the unit disc, $c(0) = a$, $c(1) = b$.
Choose $r < 1$ and a $\log$-branch so that $\log((z-b)/(z-a))$ is well-defined on $r < |z| < 1$.
Then set
$$f(z) = \begin{cases}\exp(\psi(z) \log\frac{z-b}{z-a}), &r < |z|\\
\frac{z-b}{z-a}, &|z| \leq r
\end{cases}$$
where $\psi = 1$ on $D_r$ and there is $r' \in (r, 1)$ such that $\psi = 0$ away from $D_{r'}$.
Then by the previous lemma,
$$\frac{1}{2\pi i} \iint_X \frac{df}{f} \wedge \omega = \frac{1}{2\pi i} \iint_X \frac{df}{f} \wedge dg = g(b) - g(a) = \int_c dg$$
where we set $dg = \omega$.
If $c$ is not a chart, use a partition of unity.
\end{proof}

\begin{theorem}[Abel]
A divisor $D$ is principal iff there is a chain $c$ such that $\partial c = D$ and for every holomorphic $1$-form $\omega$,
$$\int_c \omega = 0.$$
\end{theorem}
\begin{proof}
Suppose that there is $c$ with $\partial c = D$ and $\int_c \omega = 0$ for every $\omega$.
We want to show $D = (f)$ for some $f$.
By the previous lemma, there is a weak solution $f$ such that for every $\omega$,
$$\int_c \omega = \frac{1}{2\pi i} \iint_X \frac{df}{f} \wedge \omega = \frac{1}{2\pi i} \iint_X \frac{d''f}{f} \wedge \omega,$$
where the last equality is because $\omega$ is holomorphic. Then there is $\psi$ such that $f(z) = z^k\psi(z)$, thus
$$d''f(z) = d''(z^k\psi(z)) = z^k d''\psi(z).$$
Also
$$\iint_X \frac{d'' \psi}{\psi} \wedge \omega = 0$$
since $(0,1)$-forms are orthogonal to holomorphic $1$-forms.
Now we claim that $F = e^{-g}f$ is holomorphic, hence is the solution to $D$.
In fact
$$\dbar F = \dbar(e^{-g}f) = -e^{-g} \dbar g f + e^{-g} \dbar f$$
but $f\dbar g = \dbar f$ so $\dbar F = 0$.

Conversely, let $D$ be a nonzero divisor, $D = (f)$, and $\omega$ is given.
Let $Y = \PP^1 \setminus \{f(a_1), \dots, f(a_r)\}$, then $f: f^{-1}(Y) \to Y$ is a covering space, say with $n$ sheets.
If $V \subseteq Y$ is a local trivialization of this covering map, $\varphi_i: V \to U_i$ one of the isomorphisms, define the trace of $\omega$ on $V$ to be $\sum_i \varphi_i^* \omega$.
This is a holomorphic $1$-form on $Y$, and extends to $\PP^1$ since it extends as $z^{1/n - 1}$ over each singularity, which is a ``pole" of order $1/n - 1 > -1$ and hence is removable.
Now let $\gamma$ be a path $0 \to \infty$ in $Y$.
Then $f^{-1}(\gamma)$ is a chain $c = c_1 + \dots + c_n$ which connects zeroes to poles and
$$\int_c \omega = \int_\gamma \text{tr } \omega = 0.$$
Here the last inequality follows because $\PP^1$ has genus $0$ and we can use de Rham cohomology.
\end{proof}

Now consider the \dfn{Abel-Jacobi map}
$$\Phi: \Div_0(X) \to \Jac X.$$
Here the Jacobian variety $\Jac X$ is defined as follows: let $\omega_1, \dots, \omega_g$ be a basis of holomorphic $1$-forms on $X$, and consider the \dfn{period lattice} $P(\omega_1, \dots, \omega_g)$, which is generated by
$$\left(\int_a \omega_1, \dots, \int_a \omega_g\right) \in \CC^g$$
where $a \in H_1(X, \ZZ)$. Once we have shown that $P(\omega_1, \dots, \omega_g)$ is a lattice we can set
$$\Jac X = \frac{\CC^g}{P(\omega_1, \dots, \omega_g)}.$$
To construct $\Phi$, let $D \in \Div_0(X)$.
Let $c$ be a $1$-chain with $\partial c = D$.
Set
$$\Phi(D) = \left(\int_c \omega_1, \dots, \int_c \omega_g\right)$$
modulo $P(\omega_1, \dots, \omega_g)$.
Here we need the quotient to make sure that the choice of $c$ doesn't matter.
Then the kernel of $\Phi$ consists of \emph{principal} divisors.

\begin{lemma}
If $g \geq 1$, then there are $a_1, \dots, a_g \in X$ such that for every holomorphic $1$-form $\omega$, if $\omega(a_1) = \dots = \omega(a_g) = 0$, then $\omega = 0$.
\end{lemma}
\begin{proof}
Fix $a \in X$ and let $L_a$ be the evaluation map at $a$.
The kernel of $L_a$, $H_a$, has codimension $\leq 1$. Thus
$$\bigcap_{a \in X} H_a = 0.$$
Thus a counting argument gives the claim.
\end{proof}

\begin{lemma}
$P(\omega_1, \dots, \omega_g)$ is a lattice.
\end{lemma}
\begin{proof}
Let $a_1, \dots, a_g$ be as in the previous lemma, and $z_j$ a coordinate at $a_j$ with $\omega_i = \varphi_{ij} ~dz_j$.
The $\varphi_{ij}(a_j)$ form a matrix $A$. We claim that $A$ has rank $g$.
If not then there are $\lambda_j \in \RR$ such that
$$\sum_j \lambda_j \omega_i(a_j) = 0$$
so by the lemma $\lambda = 0$.
Let $U_i$ be the (necessarily simply connected) domain of $z_i$ and define a map
\begin{align*}
F: U_1 \times \cdots \times U_g &\to \CC^g\\
(x_1, \dots, x_g) &\mapsto \left(\sum_{j=1}^g \int_{a_j}^{x_j} \omega_1, \dots, \sum_{j=1}^g \int_{a_j}x^{x_j} \omega_g\right).
\end{align*}
Let $J$ be the Jacobian matrix of $F$ at $a = (a_1, \dots, a_g)$; then
$$J_{i\ell} = \varphi_{i\ell}(a_\ell).$$
Therefore $J$ has rank $g$, so by the inverse function theorem, $F$ is locally invertible at $a$.
But $F(a) = 0$, so there is open $W \ni 0$ such that $F$ maps $U_1 \times \cdots \times U_g$ to $W$.

If $\Gamma$ is our supposed lattice, we need to show that near $0$, $\Gamma$ is discrete.
That is, we need to show that if the $U_i$ are small enough then $\Gamma \cap W$ is a singleton.
If not, there is $t \in \Gamma \cap W$ which is nonzero.
By Abel's theorem, there is a meromorphic function $f$ such that $(f) = z_j - a_j$.
Let $c_j/z_j$ be the principal part of $f$ at $a_j$; then by the residue theorem,
$$0 = \Res f\omega_i = \sum_{j=1}^k c_i \varphi_{ij}(a_j)$$
so $c_i = 0$ and hence $f = 0$, a contradiction.

Finally we show that $\Gamma$ does not span a proper subspace of $\CC^g$.
If it did, then there would be a covector $\ell \in (\RR^{2g})^*$ which vanishes on $\Gamma$ but $\ell \neq 0$.
Let $\hat \ell$ be the corresponding covector over $\CC$ with $\Re \hat \ell = \ell$.
Then if $c_j$ are the coefficients of $\hat \ell$,
$$\ell\left(\int_a \omega_1, \dots, \int_a \omega_g\right) = \int_a \Re \sum_{j=1}^g c_j \omega_j = 0$$
so that
$$\sum_{j=1}^g c_j \omega = 0$$
which implies $c = 0$, a contradiction.
\end{proof}

\begin{theorem}[Jacobi]
The Abel-Jacobi map $\Phi$ is surjective.
\end{theorem}
\begin{proof}
Let $p \in \Jac X$, say $p = [\xi]$ where $\xi \in \CC^g$.
Recall that if $U_1, \dots, U_g$ are charts, then there is open $W \ni 0$ such that
$$F: U_1 \times \cdots \times U_g \to W$$
is a diffeomorphism.
Let $N$ be a large integer, so that $\xi/N \in W$.
Let $c = \sum_i [a_i, z_i]$ be a $1$-chain such that
$$\frac{\xi}{N} = \left(\int_c \omega_1, \dots, \int_c \omega_g\right).$$
Then $\xi/N = \Phi(c)$.
\end{proof}

\section{Proof of Serre duality}
The proof of Serre duality is long and boring. However, we need to give all the details because otherwise no student will bother to learn them themselves. Or something.

We define a right resolution of the sheaf $\Omega$ of holomorphic $1$-forms by
$$0 \to \Omega \to \tilde \Omega^{1,0} \to \Omega^2_{cl} \to 0$$
where $\Omega^2_{cl}$ is the sheaf of closed $2$-forms, which gives a left resolution of $H^1(X, \Omega)$ as
$$\cdots \to \tilde \Omega^{1, 0}(X) \to \Omega^2_{cl}(X) \to H^1(X, \Omega) \to 0.$$

\begin{definition}
If $\xi \in H^1(X, \Omega)$ can be represented by $\omega \in \Omega^2_{cl}(X)$ we set
$$\Res \xi = \frac{1}{2\pi i} \iint_X \omega,$$
the \dfn{residue} of $\xi$.
\end{definition}

The residue of a first cohomology class is clearly well-defined since $d\tilde \Omega^{1, 0}$ is annihilated by integration.

Now let $D$ be a divisor. We get a map
\begin{align*}
\Omega_{-D} \times \Olo_D &\to \Omega\\
(\omega, f) &\mapsto f\omega
\end{align*}
since the zeroes in $\Omega_{-D}$ and poles in $\Olo_D$ necessarily cancel each other out to give a holomorphic $1$-form.
Now if $\xi \in H^1(X, \Olo_D)$ is a cocycle, say $\xi = (f_{ij})$, we can then define $\omega \xi = (\omega f_{ij})$, whenever $\omega \in \Omega_{-D}(X)$, and we get a sequence of maps
$$\Omega_{-D}(X) \times H^1(X, \Olo_D) \to H^1(X, \Omega) \to \CC$$
given by $\Res: H^1(X, \Omega) \to \CC$.
This sequence of maps induces a bilinear map $\langle \cdot, \cdot\rangle: \Omega_{-D}(X) \times H^1(X, \Olo_D) \to \CC$.

\begin{definition}
We define a map
$$i_D: H^0(X, \Omega_{-D}) \to H^1(X, \Olo_D)^*$$
to be induced by the bilinear map $\langle \cdot, \cdot\rangle: \Omega_{-D}(X) \times H^1(X, \Olo_D) \to \CC$, called the \dfn{Serre isomorphism}.
\end{definition}

We must show that the Serre isomorphism is an isomorphism, so that
$$h^0(X, \Omega_{-D}) = h^1(X, \Olo_D).$$

\begin{definition}
Let $\mathscr A$ be an open cover of $X$.
Let
$$\delta: C^0(\mathscr A, \Mero^1) \to C^1(\mathscr A, \Mero^1)$$
be the coboundary map.
A \dfn{Mittag-Leffler distribution} is a cocycle $\mu \in C^0(\mathscr A, \Mero^1)$ such that $\delta \mu$ is a holomorphic cocycle.
\end{definition}

\begin{definition}
Let $\mu = (\omega_i)$ be a Mittag-Leffler distribution. We define the \dfn{residue} of $\mu$ to be
$$\Res \mu = \sum_{x \in X} \Res_x \omega_i.$$
\end{definition}

The residue is well-defined since $\omega_i - \omega_j$ is holomorphic on $U_i \cap U_j$, so that we can choose a cover where each of the poles is contained in exactly one open set and everywhere else it's holomorphic and has no residue.

Let $\mu$ be a Mittag-Leffler distribution.
Then $\delta \mu \in Z^1(\mathscr A, \Omega)$, so $[\delta \mu] \in H^1(X, \Omega)$, which has a residue in the sense of left resolutions.

\begin{lemma}
\label{Serre theorem 1}
Let $\mu$ be a Mittag-Leffler distribution.
One has $\Res \mu = \Res [\delta \mu]$.
\end{lemma}
\begin{proof}
Fix $\mathscr A = (U_i)$ fine enough, then $\delta \mu \in Z^1(\mathscr A, \tilde \Omega^{1, 0})$.
Since $\omega_j - \omega_i$ is meromorphic but $\tilde \Omega^{1, 0}$ is an acyclic sheaf, we can find $(\sigma_i) \in C^0(\mathscr A, \tilde \Omega^{1, 0})$ such that $\omega_j - \omega_i = \sigma_j - \sigma_i$.
Moreover, $\omega_j - \omega_i$ is holomorphic, thus
$$d(\sigma_j - \sigma_i) = d(\omega_j - \omega_i) = 0$$
which implies that $d\sigma_j = d\sigma_i$ on $U_i \cap U_j$.
That is, there is $\tau \in \tilde \Omega_{cl}^2(X)$ such that $\tau|U_i = d\sigma_i$.

Using the right resolution
$$\cdots \to \tilde \Omega^2_{cl}(X) \to H^1(X, \Omega) \to 0$$
we claim that $\delta \mu$ is represented by $\tau$ in $\tilde \Omega^2_{cl}(X)$.
To do this, recall the proof of the snake lemma.
We know that $\tau = d\sigma_i$, and $\sigma_j - \sigma_i = \omega_j - \omega_i$, but this was how the proof of the snake lemma constructed the right resolution of $H^1(X, \Omega)$.
Thus
$$\Res [\delta \mu] = \frac{1}{2\pi i} \iint_X \tau.$$
Let $a_1, \dots, a_n$ be the poles of $\mu$ and let $X' = X \setminus \{a_1, \dots, a_n\}$.
That is, $\sigma_i - \omega_i = \sigma_j - \omega_j$ on $U_i \cap U_j \cap X'$.
Thus $d\sigma_i = d\sigma_j = \tau|U_i \cap U_j$.
So there is $\sigma \in \tilde \Omega^{1, 0}(X')$ such that $\sigma|U_i = \sigma_i - \omega_i$ which implies $\tau = d\sigma$.
Define $(V_k, z_k)$ to be coordinates around $z_k$ contained in some element $U_{i_k}$ of $\mathscr A$, and let $V_k'$ be a small ball in $V_k$ around $a_k$.
Let $f_k \in C^\infty_c(V_k)$ be a smooth cutoff to $V_k'$.
Let $g = 1 - \sum_k f_k$, thus the germ of $g$ at any pole is $0$.
A priori $\sigma$ is not smooth at $a_k$ since $\sigma \in \tilde \Omega^{1, 0}(X')$.
But then $g\sigma \in \tilde \Omega^{1, 0}(X)$.
Thus
$$\iint_X d(g\sigma) = 0$$
by Stokes' theorem.
Also on $V_k' \setminus X'$,
$$d(f_k\sigma) = d\sigma_{i_k} - d\omega_{i_k} = d\sigma_{i_k}.$$
Hence
$$\tau = d(g\sigma) + \sum_k d(f_k\sigma)$$
and hence
\begin{align*}
\iint_X \tau &= \sum_k \iint_X d(f_k\sigma) = \sum_k \iint_{V_k} d(\sigma_{i_k} - \omega_{i_k})\\
&= -\sum_k \int_{\partial V_k} d\sigma_{i_k} - d\omega_{i_k} = -\sum_k \int_{V_k} d\omega_{i_k} \\
&= \sum_k \iint_{V_k} \omega_{i_k} = \sum_k \Res_{a_{i_k}} \omega_{i_k}\\
&= \Res \mu
\end{align*}
as desired.
\end{proof}

\begin{lemma}
\label{Serre theorem 2}
For every divisor $D$, the Serre isomorphism
$$i_D: H^0(X, \Omega_{-D}) \to H^1(X, \Olo_D)^*$$
is injective.
\end{lemma}
\begin{proof}
We must show that for every $\omega \in H^0(X, \Omega_{-D})$ there is $\xi \in H^1(X, \Olo_D)$ such that $\langle \omega, \xi\rangle$ is nonzero.
Pick $a \in X$ such that $D(a) = 0$, so we can find coordinates $(U_0, z)$ at $a$ such that $D|U_0 = 0$.
Then we can write $\omega = f~dz$ where $f$ is holomorphic on $U_0$.
Let $\mathscr A$ be the open cover $\{U_0, U_1\}$ where $U_1 = X \setminus a$.
Let $\eta = (f_0, f_1)$ where $f_0 = 1/zf$ and $f_1 = 0$ be a cocycle for $\mathscr A$.
Then $\omega\eta = (dz/z, 0)$ is a Mittag-Leffler distribution since on $U_0 \cap U_1$, $dz/z$ is holomorphic.
Then if we set $\xi = [\delta \eta]$, then $\xi \in H^1(X, \Olo_D)$.
Then
$$\omega\xi = \omega[\delta \eta] = [\delta(\omega \eta)]$$
so by Lemma \ref{Serre theorem 1},
$$\langle \omega, \xi\rangle = \Res [\delta(\omega \eta)] = \Res \omega\eta = 1$$
since $dz/z$ clearly has residue $1$ at $a$ and $0$ is holomorphic.
\end{proof}

It remains to show that $i_D$ is surjective. This turns out to be a huge pain in the ass.

\begin{lemma}
\label{Serre theorem 3}
There is $k_0 \in \ZZ$ such that for every divisor $D$,
$$h^0(X, \Omega_D) \geq \deg D + k_0.$$
\end{lemma}
\begin{proof}
Let $K$ be a canonical divisor. Then
$$h^0(X, \Omega_D) = h^0(X, \Olo_{D + K}).$$
Indeed, $\Omega = \Olo_K$ since $K$ is the divisor corresponding in $H^1(X, \Olo^*)$ to the cotangent bundle.
By the Riemann-Roch theorem,
\begin{align*}
h^0(X, \Olo_{D + K}) &= h^1(X, \Olo_{D+K}) + 1 - g + \deg(D + K)\\
&= 1 - g + \deg D + \deg K.
\end{align*}
So let $k_0 = 1 - g + \deg K$.
\end{proof}

The trouble here is that we need Serre duality to show $\deg K = 2g - 2$, but of course when the proof is done we get $k_0 = g - 1$.

If $D' \to D$ then the inclusion map $\Olo_{D'} \to \Olo_D$ gives a short exact sequence
$$0 \to \Olo_{D'} \to \Olo_D \to \mathscr F \to 0$$
where $\mathscr F$ is a skyscraper sheaf, and hence acyclic. So we get a right resolution
$$\cdots \to H^1(X, \Olo_{D'}) \to H^1(X, \Olo_D) \to 0.$$
Dualizing, we get a left resolution
$$0 \to H^1(X, \Olo_D)^* \to H^1(X, \Olo_{D'})^* \to \cdots.$$
In particular, we get a commutative diagram
$$\begin{tikzcd}
0 \arrow[r] & H^1(X, \Olo_D)^* \arrow[r,"i_{D'}^D"] & H^1(X, \Olo_{D'})^*\\
0 \arrow[r] & H^1(X, \Omega_{-D}) \arrow[u,"i_D"] \arrow[r] & H^1(X, \Omega_{D'})^* \arrow[u,"i_{D'}"]
\end{tikzcd}
$$
of injective maps.

\begin{lemma}
\label{Serre theorem 4}
Suppose that $i_{D'}^D(\lambda) = i_{D'}(\omega)$. Then $\omega \in H^0(X, \Omega_{-D})$ and $i_D(\omega) = \lambda$.
\end{lemma}
\begin{proof}
A straightforward diagram chase.
\end{proof}

Let $B$ be a divisor and $\psi \in H^0(X, \Olo_B)$.
Then $\psi$ defines a map $\Olo_{D - B} \to \Olo_B$ given by multiplication against $\psi$.
The adjoint of this map defines a commutative diagram
$$\begin{tikzcd}
H^1(X, \Olo_D)^* \arrow[r] & H^1(X, \Olo_{D - B})^*\\
H^0(X, \Omega_{-D}) \arrow[u,"i_D"] \arrow[r] & H^0(X, \Omega_{B - D}) \arrow[u,"i_{D - B}"]
\end{tikzcd}$$
which commutes by naturality.

\begin{lemma}
\label{Serre theorem 5}
The map $H^1(X, \Olo_D)^* \to H^1(X, \Olo_{D - B})^*$ is injective.
\end{lemma}
\begin{proof}
Let $A = (\psi)$.
Then $A \geq -B$ and the maps induced by $\psi$ and $1/\psi$ give an isomorphism $\Olo_{D+A} \to \Olo_D$.
We also have an inclusion $\Olo_{D - B} \to \Olo_{D + A}$ and hence a commutative diagram
$$\begin{tikzcd}
\Olo_{D - B} \arrow[rr] \arrow[dr] && \Olo_D\\
& \Olo_{D + A} \arrow[ur]
\end{tikzcd}$$
of sheaves, which gives a diagram of the same form in cohomology.
Since the $H^1$ of a skyscraper sheaf vanishes, the map $H^1(X, \Olo_{D - B}) \to H^1(X, \Olo_{D + A})$ is an isomorphism. We already know that $H^1(X, \Olo_{D + A}) \to H^1(X, \Olo_D)$ is an isomorphism, so $H^1(X, \Olo_{D - B}) \to H^1(X, \Olo_D)$ is also an isomorphism.
\end{proof}

To complete the proof of Serre duality, let $p \in X$ and $D_n = D - np$.
Now define, given $\lambda \in H^1(X, \Olo_D)^*$,
$$\Lambda = \{\psi\lambda: \psi \in H^0(X, \Olo_{np})\}.$$
By Lemma \ref{Serre theorem 5}, the natural map $H^0(X, \Olo_{np}) \to \Lambda$ is injective=.
But then
$$\dim \Lambda \geq h^0(X, \Olo_{np}) \geq 1 - g + n$$
by the Riemann-Roch theorem.
Also
$$\dim i_{D_n}(H^0(X, \Omega_{-D_n})) = h^0(X, \Omega_{-D_n}) \geq n + k_0 - \deg D$$
by Lemma \ref{Serre theorem 3}.
By the Riemann-Roch theorem,
$$h^1(X, \Olo_{D_n}) - h^0(X, \Olo_{D_n}) = g - 1 - \deg D + n$$
and if $n$ is large enough then it follows that
$$h^1(X, \Olo_{D_n}) = g - 1 + n - \deg D$$
so
$$\dim \Lambda + \dim i_{D_n}(H^0(X, \Omega_{-D_n})) = 1 - g + 2n + k_0 - \deg D$$
and so the pigeonhole principle implies that
$$\dim \Lambda \cap i_{D_n}(H^0(X, \Omega_{-D_n})) \geq 1.$$
Thus there is $\psi \in H^0(X, \Olo_{np})$ which is nonzero, and $\omega \in H^0(X, \Omega_{-D_n})$, such that $\psi\lambda = i_{D_n}(\omega)$.
If $A = (\psi)$ then $D' = D_n - A$ satisfies $D' \leq D$, thus we have a commutative diagram
$$\begin{tikzcd}
H^1(X, \Olo_D)^* \arrow[r, "i_{D'}^D"] & H^1(X, \Olo_{D'})^*\\
H^0(X, \Omega_{-D}) \arrow[u,"i_D"] \arrow[r] & H^0(X, \Olo_{-D}) \arrow[u,"i_{D'}"]
\end{tikzcd}.$$
Then
$$i_{D'}^D\lambda = \frac{\psi}{\psi}\lambda = \frac{i_{D_n}(\omega)}{\psi} = i_{D_n}(\omega/\psi)$$
which gives $\lambda = i_D(\omega/\psi)$.
This implies that $i_D$ is surjective and so completes the proof of Serre duality.

\chapter{Commutative algebra basics}
Let us begin discussing commutative algebra following Eisenbud. This will probably skip a lot of details as I'm mainly trying to get the intuitions.

All rings are commutative.

\section{Local rings}
We interpret a ring $R$ as follows.
The primes of $R$ are points of the \dfn{affine scheme} $\Spec R$.
For $p \in \Spec R$ and $f \in R$, we write $f(p)$ for the image of $f$ in $R/p$.
The ``geometrically meaningful" points are the maximal ideals.
Thus we have a ``bundle of fields" over $\Spec R$ consisting of \dfn{residue class fields} $R/p$, and an element of $R$ is a section of this bundle of fields.

Affine schemes are equipped with the \dfn{Zariski topology}, i.e. the weakest topology which makes every element of $R$ continuous.
So the closed sets of $\Spec R$ correspond to ideals, namely the closed set corresponding to the ideal $I$ is $\{p \in \Spec R: I \subseteq p\}$.
Bigger ideals correspond to smaller closed sets, since if $I = 0$ then $\{p \in \Spec R: 0 \subseteq p\} = \Spec R$.

A $R$-module $M$ consists of ``sections of a vector bundle".
Indeed, given the projection map $R \to R/p$ we get a map of $R$-modules $M \to M/p$ where $M/p$ carries the structure of a vector space over the residue class field $R/p$.
We write $m(p)$ for the image of $m$ in $M/p$.

A subset of a ring is \dfn{multiplicative} if it is closed under multiplication.
In particular, if $p$ is a prime ideal, then its complement $p^c$ is multiplicative.

\begin{definition}
Let $R$ be a ring, $M$ a $R$-module, and $U$ a multiplicative subset of $R$.
Then the \dfn{localization} of $M$ away from $U$, $M[U^{-1}]$, is the $R$-module obtained from $M$ by first modding out by $\{m \in M: \exists u \in U(mu = 0)\}$ and then adjoining formal inverses $m/u$ for every $m \in M$ and $u \in U$ as freely as possible.
If $p$ is a prime ideal, we write $M_p = M[(p^c)^{-1}]$.
\end{definition}

Let $\varphi: R \to R[U^{-1}]$ be the localization map.
The correspondence $p \mapsto \varphi^{-1}(p)$ is a bijection between the primes of $R[U^{-1}]$ and the primes of $R$ not meeting $U$.
So, the idea of localization away from $U$ is to replace $\Spec R$ with its open subscheme corresponding to the complement of the ideal generated by $U^c$.
In particular $\Spec R_p = \{0, p\}$.
This motivates the following definition.

\begin{definition}
A \dfn{local ring} is a ring with just one maximal ideal.
\end{definition}

In $\Spec R_p$ we have forgotten about all the data of the functions in $R$ except their behavior near $p$.
The fact that we modded out by zero-divisors means that we even forgot about their derivatives.
All we care about is their behavior at $p$.

We have a natural isomorphism $R[U^{-1}] \otimes M \to M[U^{-1}]$, namely $r/u \otimes m \mapsto rm/u$.

\begin{definition}
The \dfn{support} $\supp M$ of a module $M$ is the set of primes $p$ such that $M_p \neq 0$.
The \dfn{annihilator} $\Ann M$ is the largest ideal $I$ such that $IM = 0$.
\end{definition}

Now if $p$ is not in $\supp M$, then $M_p = 0$, so all the elements of $M$ must vanish near $p$.
That is certainly what it means for a section of a vector bundle to not be in the support.

\begin{lemma}
Let $M$ be a finitely generated module.
The support $\supp M$ is the set of all $p \in \Spec R$ such that $\Ann M \subseteq p$.
\end{lemma}
\begin{proof}
This follows from the fact that $M_p = 0$ iff $M$ is annihilated by an element of $p^c$.
But in fact, if $M$ is generated by $m_1, \dots, m_n$, and $m_i$ is annihilated by $u_i \in p^c$, then $\prod_i u_i$ annihilates $M$.
\end{proof}

\begin{definition}
An element $m \in M$ \dfn{stalkwise locally} satisfies a property $P$ if for every maximal ideal $p$, the image of $m$ in $M_p$ satisfies $P$.
\end{definition}

\begin{lemma}
For every $m \in M$, $m = 0$ iff $m$ is stalkwise locally $0$.
\end{lemma}
\begin{proof}
Let $I$ annihilate $m$, then $m_p = 0$ iff $I$ is not contained in $p$.
But $m = 0$ iff $p = R$, which happens iff $I$ is not contained in any maximal ideal of $R$ (since Zorn's lemma says that every prime ideal is contained in a maximal ideal if it's not $R$).
\end{proof}

\section{Localization and flat modules}
It is clear that localization $M \mapsto M[U^{-1}]$ is a functor.
In fact it is a very well-behaved functor, as we now discuss.

If $x,y$ are objects in a category, their biproduct $x \oplus y$ is defined to be both their product and coproduct, if both exist and are equal.

\begin{definition}
A \dfn{preadditive category} is a category $C$ such that every $\Hom$-set is equipped with the structure of an abelian group, such that composition distributes over addition.
An \dfn{abelian category} is a preadditive category which contains a zero object and is closed under finite biproducts, such that every morphism has a kernel and a cokernel, every monomorphism is a kernel, and every epimorphism is a cokernel.
\end{definition}

Clearly the category of $R$-modules is abelian, since every $\Hom$-set has the structure of a $R$-module (and thus an abelian group).
The converse is also true: every abelian category embeds in a category of modules.
We will not prove this fact but we will always have an embedding of an abelian category in a category of modules in mind when we talk about abelian categories.
In any case we will never need an abelian category that is not a full subcategory of some category of modules.

There are two important facts about abelian categories.
The first is that currying (``the fundamental theorem of introductory computer science") makes sense.
Namely, $\Hom$ and $\otimes$ are bifunctors in an abelian category, and we have
$$\Hom(a, \Hom(b, c)) = \Hom(a \otimes b, c).$$
More concretely, a bilinear map $A \times B \to C$ is the same data as a linear map $A \otimes B \to C$.

The second important fact about abelian categories is that, because they have kernels and cokernels, exact sequences make sense.
In particular, the notion of a short exact sequence does:

\begin{definition}
An \dfn{exact functor} is a functor $F: A \to B$, where $A,B$ are abelian categories, such that for every short exact sequence
$$0 \to a \to b \to c \to 0$$
in $A$, the sequence
$$0 \to F(a) \to F(b) \to F(c) \to 0$$
is short exact in $B$.
\end{definition}

\begin{lemma}
Tensoring with a module preserves cokernels.
\end{lemma}
\begin{proof}
By currying, tensoring with a module is left-adjoint to Homming.
Therefore it is cocontinuous, and cokernels are colimits.
\end{proof}

Dually, Homming with a module preserves kernels. The proof is the same.

\begin{example}
Tensoring with $\ZZ/2$ is not exact. We have an exact sequence
$$0 \to 3\ZZ \to \ZZ \to \ZZ/3 \to 0$$
which tensors to
$$0 \to 0 \to \ZZ/2 \to 0 \to 0$$
which is clearly not exact.
\end{example}

\begin{definition}
A \dfn{flat module} is a module $M$ such that tensoring with $M$ is an exact functor.
\end{definition}

A module $M$ is flat iff tensoring with $M$ preserves monomorphisms.
Indeed, a functor is exact iff it preserves monomorphisms and epimorphisms.

Localization preserves flatness:

\begin{theorem}
For every flat module $F$, $F[U^{-1}]$ is flat.
\end{theorem}
\begin{proof}
Let $M' \to M$ be a monomorphism. We must show that the induced map
$$F[U^{-1}] \otimes M' \to F[U^{-1}] \otimes M$$
is a monomorphism as well.
But this induced map is naturally isomorphic to the map
$$F \otimes M'[U^{-1}] \to F \otimes M[U^{-1}]$$
obtained by extending the map $F \otimes M' \to F \otimes M[U^{-1}]$ induced by $F \otimes M' \to F \otimes M$, which is a monomorphism since $F$ is flat.
Furthermore, if $m \in M'$ and $m/u = 0$ in $M[U^{-1}]$ then there is $v \in U$ such that $vm = 0$ in $M$ and hence in $M'$ since everything is monic.
Thus $m/u = 0$ in $M'[U^{-1}]$.
\end{proof}

\section{Associated primes}
We will abuse terminology by writing $\Ann m$ for the annihilator of the \emph{span} $(m)$ of $m$,
$$\Ann m = \Ann(mR).$$

\begin{definition}
A prime $p$ is \dfn{associated} to $M$ if there is $m \in M$ such that $p = \Ann m$.
The set of associated primes to $M$ is called the \dfn{assassin} $\Ass M$ of $M$.
\end{definition}

We will abuse notation and write $\Ass I$ for $\Ass R/I$.

\begin{lemma}
Let $U$ be a multiplicative set.
If $I$ is an ideal which is maximal among those not meeting $U$ then $I$ is prime.
\end{lemma}
\begin{proof}
Suppose $f,g \notin I$. We must show $fg \notin I$.
By maximality of $I$, $I + (f)$ and $I + (g)$ meet $U$, say $i + af, j + bg \in U$ where $i,j \in \Ann m$.
If $fg \in I$ then $(i + af)(bg + j) \in I$, so $I$ meets $U$.
\end{proof}

This is basically just a souped up version of the proof that every maximal ideal is prime.
We record this proof here because we will need the idea again in the following proof:

\begin{lemma}
Let $I$ be an ideal which is maximal in the set $\{\Ann m: m \in M\}$.
Then $I \in \Ass M$.
\end{lemma}
\begin{proof}
We must show that $I$ is prime.
Actually, if $rs \in I$, $s \notin I$, and $I = \Ann m$, then $rsm = 0$ but $sm \neq 0$, so $(r) + I \subseteq \Ann sm$.
Thus $(r) + I = I$ by maximality, so $r \in I$.
\end{proof}

\begin{lemma}
If $R$ is a Noetherian ring and $M$ is a nonzero $R$-module then $\Ass M$ is nonempty.
\end{lemma}
\begin{proof}
Every chain $C$ in $\{\Ann m: m \in M\}$ has only finite length, so $C$ has a maximal element.
So by Zorn's lemma, $\{\Ann m: m \in M\}$ has a maximal element.
\end{proof}

\begin{lemma}
If $0 \to A \to B \to C$ is a short exact sequence of modules then
$$\Ass A \subseteq \Ass B \subseteq \Ass A \cup \Ass C.$$
\end{lemma}
\begin{proof}
That $\Ass A \subseteq \Ass B$ is clear.
Now let $p \in \Ass B \setminus \Ass A$.
If $\Ann m = p$, thus $(m) \cong R/p$, then every nonzero submodule $N$ of $(m)$ also has $\Ann N = p$.
So $(m) \cap A = 0$, so $(m)$ is isomorphic to its image in $C$.
So $p \in \Ass C$.
\end{proof}

\begin{lemma}
If $R$ is a Noetherian ring and $M$ is a finitely generated $R$-module, then there is a filtration
$$0 = M_0 \subset M_1 \subset M_2 \subset \cdots \subset M_n = M$$
with $M_{i+1}/M_i$ isomorphic to $R/p_i$ where $p_i$ are primes.
\end{lemma}
\begin{proof}
If $M \neq 0$ then let $p_1 \in \Ass M$.
So let $M_1$ be isomorphic to $R/p_1$.
Now repeat inductively with $M$ replaced by $M/M_1$.
Since $M$ is a Noetherian module, this process must stop after a finite amount of time $n$.
\end{proof}

\begin{theorem}
If $R$ is a Noetherian ring and $M$ is a finitely generated $R$-module, then $\Ass M$ is a nonempty finite set, and its elements are minimal among those containing $\Ann M$.
\end{theorem}
\begin{proof}
We already proved that this is nonempty, and its relation to $\Ann M$ is just a definition chase.
To see that it is finite, consider a filtration
$$0 = M_0 \subset M_1 \subset M_2 \subset \cdots \subset M_n = M$$
equipped with prime ideals $p_i$ such that $M_{i+1}/M_i \cong R/p_i$.
We induct on $n$; namely we have a short exact sequence
$$0 \to M_{n - 1} \to M_n \to \frac{R}{p_{n - 1}} \to 0$$
so $\card \Ass M_n \leq \card \Ass M_{n - 1} + \card \Ass R/p_{n - 1}$.
If $q$ annihilates $R/p_{n - 1}$ then $q$ contains $p_{n - 1}$, so $\card \Ass R/p_{n - 1} = 1$.
\end{proof}

\begin{theorem}[prime avoidance]
Let $I_1, \dots, I_n, J$ be ideals in $R$ such that $J \subseteq \bigcup_i I_i$ and either:
\begin{enumerate}
\item if $i \geq 3$ then $I_i$ is prime, or
\item $R$ contains an infinite field.
\end{enumerate}
Then there is $i$ such that $J \subseteq I_i$.
\end{theorem}
\begin{proof}
If $R$ contains an infinite field $k$ then the $I_i$ and $J$ are vector spaces over $k$, so the claim is obvious.
Also if $n = 1$ the proof is trivial.
So we might as well assume (by induction and after reindexing) that $n \geq 2$, all but $2$ of the $I_i$ are prime, and $J$ is not a subset of $\bigcup_{i \neq i_0} I_i$ for any $i_0$.
In that case, there are $x_i \in J$ such that $x_i \in I_i \setminus \bigcup_{j \neq i} I_j$.

If $n = 2$ then $x_1 + x_2 \notin I_1$ and $\notin I_2$, so $x_1 + x_2 \notin J$, contradiction.

If $n \geq 3$, then we may assume that $I_1$ is prime after reiindexing, so $x_1 + \prod_{j > 1} x_j \notin I_k$ for any $k$, hence $\notin J$, contradiction.
\end{proof}





\newpage
\printindex
\printbibliography

\end{document}
