\documentclass[reqno,12pt]{amsart}
\usepackage[letterpaper, margin=1in]{geometry}
\RequirePackage{amsmath,amssymb,amsthm,graphicx,mathrsfs,url,slashed}
\RequirePackage[usenames,dvipsnames]{xcolor}
\RequirePackage[colorlinks=true,linkcolor=Red,citecolor=Green]{hyperref}
\RequirePackage{amsxtra}
\usepackage{cancel}
\usepackage{bbm}
\usepackage{tikz-cd}



% \setlength{\textheight}{9.3in} \setlength{\oddsidemargin}{-0.25in}
% \setlength{\evensidemargin}{-0.25in} \setlength{\textwidth}{7in}
% \setlength{\topmargin}{-0.25in} \setlength{\headheight}{0.18in}
% \setlength{\marginparwidth}{1.0in}
% \setlength{\abovedisplayskip}{0.2in}
% \setlength{\belowdisplayskip}{0.2in}
% \setlength{\parskip}{0.05in}
%\renewcommand{\baselinestretch}{1.05}

\title{Notes on the fractal uncertainty principle}
\author{Aidan Backus}
\date{\today}

\newcommand{\NN}{\mathbf{N}}
\newcommand{\ZZ}{\mathbf{Z}}
\newcommand{\QQ}{\mathbf{Q}}
\newcommand{\RR}{\mathbf{R}}
\newcommand{\CC}{\mathbf{C}}
\newcommand{\DD}{\mathbf{D}}
\newcommand{\PP}{\mathbf P}
\newcommand{\MM}{\mathbf M}
\newcommand{\II}{\mathbf I}
\newcommand{\Hyp}{\mathbf H}
\newcommand{\Sph}{\mathbf S}
\newcommand{\Group}{\mathbf G}
\newcommand{\GL}{\mathbf{GL}}
\newcommand{\Orth}{\mathbf{O}}
\newcommand{\SpOrth}{\mathbf{SO}}
\newcommand{\Ball}{\mathbf{B}}

\DeclareMathOperator*{\Expect}{\mathbf E}
\DeclareMathOperator{\Var}{\mathrm{Var}}
\DeclareMathOperator{\Acal}{\mathcal A}
\DeclareMathOperator{\Bcal}{\mathcal B}
\newcommand*\dif{\mathop{}\!\mathrm{d}}

\DeclareMathOperator{\card}{card}
\DeclareMathOperator{\cent}{center}
\DeclareMathOperator{\ch}{ch}
\DeclareMathOperator{\codim}{codim}

\DeclareMathOperator{\diag}{diag}
\DeclareMathOperator{\diam}{diam}
\DeclareMathOperator{\dom}{dom}
\DeclareMathOperator{\Exc}{Exc}
\newcommand{\ext}{\mathrm{ext}}
\DeclareMathOperator{\Gal}{Gal}
\DeclareMathOperator{\Hom}{Hom}
\DeclareMathOperator{\Iso}{Iso}
\DeclareMathOperator{\argmax}{arg\,max}
\DeclareMathOperator{\Lip}{Lip}
\DeclareMathOperator{\Met}{Met}
\DeclareMathOperator{\id}{id}
\DeclareMathOperator{\rad}{rad}
\DeclareMathOperator{\rank}{rank}
\DeclareMathOperator{\Rm}{Rm}
\DeclareMathOperator{\Hess}{Hess}
\DeclareMathOperator{\Hol}{Hol}
\DeclareMathOperator{\Radon}{Radon}
\DeclareMathOperator*{\Res}{Res}
\DeclareMathOperator{\sgn}{sgn}
\DeclareMathOperator{\singsupp}{sing~supp}
\DeclareMathOperator{\Spec}{Spec}
\DeclareMathOperator{\supp}{supp}
\DeclareMathOperator{\Tan}{Tan}
\newcommand{\tr}{\operatorname{tr}}

\newcommand{\dbar}{\overline \partial}

\DeclareMathOperator{\hull}{hull}

\DeclareMathOperator{\Div}{div}
\DeclareMathOperator{\Gram}{Gram}
\DeclareMathOperator{\grad}{grad}
\DeclareMathOperator{\dist}{dist}
\DeclareMathOperator{\spn}{span}
\DeclareMathOperator{\Ell}{Ell}
\DeclareMathOperator{\WF}{WF}

\newcommand{\Lagrange}{\mathscr L}

\newcommand{\Hilb}{\mathcal H}
\newcommand{\Homology}{\mathrm H}
\newcommand{\normal}{\mathbf n}
\newcommand{\radial}{\mathbf r}
\newcommand{\evect}{\mathbf e}
\newcommand{\vol}{\mathrm{vol}}

\newcommand{\pic}{\vspace{30mm}}
\newcommand{\dfn}[1]{\emph{#1}\index{#1}}

\renewcommand{\Re}{\operatorname{Re}}
\renewcommand{\Im}{\operatorname{Im}}

\newcommand{\loc}{\mathrm{loc}}
\newcommand{\cpt}{\mathrm{cpt}}

\def\Japan#1{\left \langle #1 \right \rangle}

\newtheorem{theorem}{Theorem}[section]
\newtheorem{badtheorem}[theorem]{``Theorem"}
\newtheorem{prop}[theorem]{Proposition}
\newtheorem{lemma}[theorem]{Lemma}
\newtheorem{sublemma}[theorem]{Sublemma}
\newtheorem{invariant}[theorem]{Invariant}
\newtheorem{claim}[theorem]{Claim}
\newtheorem{proposition}[theorem]{Proposition}
\newtheorem{corollary}[theorem]{Corollary}
\newtheorem{conjecture}[theorem]{Conjecture}
\newtheorem{axiom}[theorem]{Axiom}
\newtheorem{assumption}[theorem]{Assumption}

\theoremstyle{definition}
\newtheorem{definition}[theorem]{Definition}
\newtheorem{defi}[theorem]{Definition}
\newtheorem{remark}[theorem]{Remark}
\newtheorem{example}[theorem]{Example}
\newtheorem{exa}[theorem]{Example}
\newtheorem{notation}[theorem]{Notation}

\newtheorem{exercise}[theorem]{Discussion topic}
\newtheorem{homework}[theorem]{Homework}
\newtheorem{problem}[theorem]{Problem}

\makeatletter
\newcommand{\proofpart}[2]{%
  \par
  \addvspace{\medskipamount}%
  \noindent\emph{Part #1: #2.}
}
\makeatother

\newtheorem{ack}{Acknowledgements}

\numberwithin{equation}{section}


% Mean
\def\Xint#1{\mathchoice
{\XXint\displaystyle\textstyle{#1}}%
{\XXint\textstyle\scriptstyle{#1}}%
{\XXint\scriptstyle\scriptscriptstyle{#1}}%
{\XXint\scriptscriptstyle\scriptscriptstyle{#1}}%
\!\int}
\def\XXint#1#2#3{{\setbox0=\hbox{$#1{#2#3}{\int}$ }
\vcenter{\hbox{$#2#3$ }}\kern-.6\wd0}}
\def\ddashint{\Xint=}
\def\dashint{\Xint-}

%\usepackage[backend=bibtex,style=numeric]{biblatex}
%\renewcommand*{\bibfont}{\normalfont\footnotesize}
%\addbibresource{fup.bib}
%\renewbibmacro{in:}{}
%\DeclareFieldFormat{pages}{#1}
\usepackage[
    backend=biber,
    style=alphabetic,
    giveninits=true
]{biblatex}
\addbibresource{fup.bib}
\renewcommand*{\bibfont}{\footnotesize}
\renewcommand\UrlFont{\color{black}\rmfamily\itshape}
\DeclareFieldFormat{pages}{#1}
\renewbibmacro{in:}{%
  \ifentrytype{article}
    {}
    {\bibstring{in}%
     \printunit{\intitlepunct}}}
\DeclareFieldFormat
  [article,inbook,incollection,inproceedings,patent,thesis,unpublished]
  {title}{\mkbibemph{#1}}
\DeclareFieldFormat{journaltitle}{#1\isdot}
\DeclareFieldFormat[article]{volume}{\mkbibbold{#1}}
\DeclareFieldFormat[article]{number}{\bibstring{number}\addnbspace #1}
\renewcommand*{\newunitpunct}{\addcomma\space}
\renewbibmacro*{journal+issuetitle}{%
  \usebibmacro{journal}%
  \setunit*{\addspace}%
  \iffieldundef{series}
    {}
    {\newunit
     \printfield{series}%
     \setunit{\addspace}}%
  \printfield{volume}%
  \setunit{\addspace}%
  \usebibmacro{issue+date}%
  \setunit{\addcomma\space}%
  \printfield{number}%
  \setunit{\addcolon\space}%
  \usebibmacro{issue}%
  \setunit{\addcomma\space}%
  \printfield{eid}
  \newunit}
  
  \newcommand\todo[1]{\textcolor{red}{TODO: #1}}

\begin{document}

\maketitle

\tableofcontents 

\section{Motivation from spectral theory}
\subsection{Gamow's resonant states}
Consider an $\alpha$-particle -- that is, a He-4 nucleus that was just ejected from the nucleus of a heavy atom (eg, of U-238).
The physicist Gamow thought of the $\alpha$-particle as a wavefunction $u$, it solves a Schr\"odinger equation 
\begin{equation}\label{Schrodinger}
    (i\partial_t + P)u = 0
\end{equation}
on all of $\RR^d$, where $P$ is uniformly elliptic on $\{|x| \gtrsim 1\}$ (and in fact converges to a rescaled Laplacian near infinity) but may degenerate near the origin due to effects of the strong and electromagnetic forces near the U-238 nucleus.

According to quantum mechanics, steady states of (\ref{Schrodinger}) are eigenfunctions of $P$ with $L^2$ norm one.
But $P$ is basically just the Laplacian, so it has essential spectrum, and so (\ref{Schrodinger}) has no steady states.
Gamow got around this conundrum by considering functions which are formally eigenfunctions of $P$, as detected by the resolvent of $P$.
Recall that the resolvent of an operator $P$ is given by 
$$R_P(z) = (P - z^2)^{-1}$$
and, using the uniform ellipticity of $P$, it admits an analytic continuation to $\CC$ minus a locally finite set of poles \cite{dyatlov2019mathematical}.

\begin{definition}
    A \dfn{resonance} of $P$ is a pole of $R_P$ (counted with multiplicity).
\end{definition}

Resonances $z$ corresponds to particles, called resonant states, which are sort of like formal eigenfunctions of $P$ with eigenvalue $z^2$.
Such a particle has energy $\Re z$ and half-life $-h/\Im z$. (Here $h := \hbar/\sqrt{2m}$ where $m$ denotes mass.)

In general $P$ may admit lots of resonant states, but if $-\Im z \gg h$, they are not very useful, as they are particles of very short half-life, and there are infinitely many distinct such particles.
We want a condition which guarantees that there's only finitely many resonant states worth caring about.

\begin{definition}
    $P$ has an \dfn{essential spectral gap} of size $\beta \geq 0$ if there are only finitely many resonances with $\Im z \geq -\beta$.
\end{definition}

\subsection{The Dyatlov--Zahl theorem}
Taking $h \to 0$ is called the \dfn{semiclassical limit} and corresponds to approximating the quantum world by its classical counterparts.
In that case particles are expected to move along geodesics, if $P$ is the Laplace-Beltrami operator of some Riemannian metric. 
So the study of resonant states would be especially interesting if we assumed that $P$ was the Laplace-Beltrami operator of a negatively curved manifold, since then the geodesic flow would be chaotic: two particles could start with very similar positions and momenta, and end up in very different places in an exponentially small timeframe.

Let $\Gamma$ be a ``nice'' discrete subgroup of the properly orthochronous Lorentz group $SO^+(1, d + 1)$.
This is the group of all spacetime symmetries which preserve orientation and the arrow of time.
In particular it acts on the future unit hyperboloid $\{t = \sqrt{|x|^2 + 1}\}$ by isometries.
But that hyperboloid is hyperbolic space $\mathbf H^{d + 1}$.
So we can take the manifold 
$$M := \mathbf H^{d + 1}/\Gamma$$
whose fundamental group is then $\Gamma$.

Let's assume that $M$ has infinite ends. 
It's negatively curved, so its geodesic flow is chaotic, but also some geodesics are trapped, while some escape to infinity.
Thus we have a highly unpredictable geodesic flow, and the geodesics which do escape correspond to trajectories of resonant states.
So, if we understand the group $\Gamma$, we should be able to say some interesting things about the resonant states.

Dyatlov--Zahl made the above heuristics rigorous using H\"ormander's theory of Fourier integral operators and singularity propagation \cite{Dyatlov_2016}.

\begin{theorem}[Dyatlov--Zahl '16]
    Let $\Gamma$ be a convex cocompact discrete subgroup of $SO^+(1, d + 1)$.
    Let $\Lambda_\Gamma$ be the boundary of an orbit of $\Gamma$ in the sphere at infinity $\mathbf S^d := \partial \mathbf H^{d + 1}$.
    Let 
    $$B_{\chi, h} f(y) := h^{d/2} \int_{\mathbf S^d} |y - x|^{2i/h} \chi(x, y) f(x) \dif x$$
    where $\chi$ has compact support in the off-diagonal of $(\mathbf S^d)^2$.
    If 
    $$\|1_{\Lambda_\Gamma + B_h} B_{\chi, h} 1_{\Lambda_\Gamma + B_h}\|_{L^2(\mathbf S^d) \to L^2(\mathbf S^d)} \lesssim_\chi h^\beta,$$
    then the Laplacian on $\mathbf H^{d + 1}/\Gamma$ has an essential spectral gap $\geq \beta$.
\end{theorem}

Notice that $B_{\chi, h}$ formally resembles the Fourier transform, since 
$$|y - x|^{2i/h} = e^{i\Phi(x, y)/h}$$
where $\Phi(x, y) = \log|x - y|$ is a sort of ``generalized inner product'' which degenerates away from the support of $\chi$.
We will focus on this model case in the sequel.
Also notice that $\Lambda_\Gamma$ is a fractal (topologists love these sets).

\section{The fractal uncertainty principle}
\subsection{Statement of the theorem}
What does it mean for a set to be fractalline? What do we mean when we say that $\Lambda_\Gamma$ is a fractal?
Well, we want some sort of self-similarity, and self-similarity at all scales is what the Ahlfors-David condition says:

\begin{definition}
A compact set $X \subset \RR^d$ of Hausdorff dimension $\delta$ is \dfn{Ahlfors-David regular} if its Hausdorff measure $\mu$ satisfies 
$$\mu(B(x, r)) \sim r^\delta.$$
\end{definition}

If $\Gamma$ is as in the Dyatlov--Zahl theorem, then $\Lambda_\Gamma$ is $\delta$-regular for some $\delta \in (0, \delta)$.

\begin{theorem}[fractal uncertainty principle]
    Let $X$ be $\delta$-regular, $X_h := X + B_h$, and
    $$\mathscr F_h f(y) := h^{d/2} \int_{\RR^d} e^{-ix \cdot y/h} f(x) \dif x.$$
    Assuming ???, there exists $\beta \geq \beta_0$ such that 
    $$\|1_{X_h} \mathscr F_h 1_{X_h}\|_{L^2(\RR^d) \to L^2(\RR^d)} \lesssim h^\beta.$$
\end{theorem}

Here's how to interpret this theorem.
Suppose that $f$ rapidly decays away from a small neighborhood $X_h$ of $X$.
Then $1_{X_h}$ is basically just $f$, and after we take the Fourier transform $\mathscr F_h f \approx \mathscr F_h 1_{X_h} f$, the definition of the fractal uncertainty principle indicates that what's left must avoid $X_h$, in the sense that its contraction with $1_{X_h}$ is tiny in $L^2$.

OK so we need to say what ??? and $\beta_0$ are. First observe that 
$$\|1_{X_h} \mathscr F_h 1_{X_h}\| \leq \|1_{X_h}\|_{L^\infty}^2 \|\mathscr F_h\|_{L^2 \to L^2} \leq 1 = h^0$$
by Plancherel's theorem and H\"older's inequality.
So if FUP is going to be an interesting theorem, we must have $\beta_0 \geq 0$.

Assuming for now that 
$$\|1_{X_h}\|_{L^2} \lesssim h^{\frac{d - \delta}{2}},$$
we get 
\begin{align*}
    \|1_{X_h} \mathscr F_h 1_{X_h}\|
    &\leq \|1_{X_h}\|_{L^\infty \to L^2} \|\mathscr F_h\|_{L^1 \to L^\infty} \|1_{X_h}\|_{L^2 \to L^1}, \\
    &\leq \|1_{X_h}\|_{L^2}^2 \|\mathscr F_h\|_{L^1 \to L^\infty} \\
    &\lesssim h^{d - \delta} h^{-\frac{d}{2}} = h^{\frac{d}{2} - \delta}.
\end{align*}

Thus we have 
$$\beta_0 := \max\left(0, \frac{d}{2} - \delta\right),$$
the so-called \dfn{Patterson-Sullivan spectral gap}.

Now, what is ???:
\begin{itemize}
    \item Dyatlov--Zahl '16 \cite{Dyatlov_2016}: $d = 1$, $\delta = \frac{1}{2}$.
    \item Bourgain--Dyatlov '18 \cite{Bourgain_2018}: $d = 1$, $\frac{1}{2} \leq \delta < 1$.
    \item Dyatlov--Jin '18 \cite{Dyatlov_2018}: $d = 1$, $0 < \delta \leq \frac{1}{2}$.
    \item Han--Schlag '20 \cite{Han_2020}: $\frac{d}{2} \leq \delta < d$, $X = \prod_n X_n$ where $X_n \subset \RR$ is $\delta_n$-regular, $\delta_n \in (0, 1)$.
    \item Cladek--T. Tao '21 \cite{Cladek_2021}: $d$ odd, $\delta = \frac{d}{2}$.
    \item Cohen '22 \cite{cohen2022fractal}: $d = 2$, $1 \leq \delta < 2$, $X$ is an arithmetic Cantor set which does not contain a line.
    \item Backus--Leng--Z. Tao '23: $0 < \delta \leq \frac{d}{2}$ and $X$ is not orthogonal to itself (to be defined later).
\end{itemize}

Roughly speaking there are three cases:
\begin{itemize}
    \item The case $0 < \delta \leq \frac{d}{2}$ is proven using the self-similarity to get lots of cancellations in the Fourier phase; this is called \dfn{Dolgopyat's method}, and this case is called the \dfn{Dyatlov--Jin uncertainty principle}. We'll prove it in detail later.
    \item The case $\delta = \frac{d}{2}$ is proven using additive combinatorics. This case is called the \dfn{Dyatlov--Zahl uncertainty principle}. We'll sketch some of the ideas here.
    \item The case $\frac{d}{2} \leq \delta < d$ is proven using complex analytic or algebro-geometric methods. This case is hardest IMO and is called the \dfn{Bourgain--Dyatlov uncertainty principle}.
\end{itemize}

\subsection{Some examples}
The above theorem is not quite sharp but there are some improvements that we cannot make to it.

\begin{example}
    Let $X = [-3, 3]$, $d = 1$. Then $e^{-x^2/2}$ is microlocalized to $X$ at scale $h$ (if $h$ is small enough), so we cannot take $\delta = d$.
\end{example}

\begin{example}
    $X = \{0\}$. Then $e^{-|x|^2/(2h)}$ is microlocalized to $X$ at scale $h$, so we cannot take $\delta = 0$.
\end{example}

\begin{example}
    Let $X$ be the union of two orthogonal line segments, $d = 2$.
    Then $e^{-x^2/2-y^2/(2h)}$ is microlocalized to the horizontal line segment and its Fourier transform is microlocalized to the vertical part.
    So $X$ cannot have too much ``linearly independent additive structure'', whatever on god's green earth that means.
\end{example}

\section{Sketch of the Dyatlov--Zahl uncertainty principle}
To illustrate some of the ideas used in the proofs of all three cases, we'll sketch Cladek--Tao's argument, especially in the $d = 1$ case where it is particularly simple.

\subsection{Rescaling to a favorable norm}
First observe that $X_h \approx \bigcup_{n = 1}^N B(x_n, h)$ where $\{x_n\}$ is a maximal $O(h)$-separated subset of $X$.
So $|X_h| \sim Nh^d$, but $\mu(B(x_n, h)) \sim h^\delta$ and $\mu(X) \sim 1$, hence $N \sim h^{-\delta}$ and so 
$$|X_h| \sim h^{d - \delta}.$$
In particular we have 
$$\|1_{X_h}\|_{L^p} \sim h^{\frac{d - \delta}{p}}$$
like we promised earlier.

By H\"older's inequality we have 
$$\|1_{X_h} \mathscr F_h 1_{X_h}\|_{L^2 \to L^2} \leq \|1_{X_h}\|_{L^{\frac{8}{3}}} \|\mathscr F_h 1_{X_h}\|_{L^2 \to L^8} \lesssim h^{\frac{3}{8}(d - \delta)} \|\mathscr F_h 1_{X_h}\|_{L^2 \to L^8}.$$
By the Riesz-Thorin inequality,
$$\|\mathscr F_h 1_{X_h}\|_{L^2 \to L^8}^2 \leq \|\mathscr F_h 1_{X_h}\|_{L^1 \to L^\infty} \|\mathscr F_h 1_{X_h}\|_{L^\infty \to L^4}.$$
The first term is 
$$\|\mathscr F_h 1_{X_h}\|_{L^1 \to L^\infty} \leq \|\mathscr F_h\|_{L^1 \to L^\infty} \|1_{X_h}\|_{L^\infty} \lesssim h^{-d/2} \cdot 1 = h^{-d/2}.$$
We bound the second term in terms of the Gowers norm of $1_{X_h}$:

\begin{definition}
    The \dfn{Gowers uniformity space} $U^2$ is defined by 
    $$\|f\|_{U^2}^4 := \iiint_{\RR^{3d}} f(x) \overline{f(x + y) f(x + z)} f(x + y + z) \dif x \dif y \dif z.$$
\end{definition}

Let's recall some properties of this space. We have the \dfn{Gowers inverse theorem}
$$\|f\|_{U^2}^2 = \|f * f\|_{L^2} = \|\hat f\|_{L^4}^2 \leq \|\hat f\|_{L^2} \|\hat f\|_{L^\infty} = \|f\|_{L^2} \|\hat f\|_{L^\infty}.$$
So the $U^2$ norm of $f$ is capturing how much $f$ correlates with a single wave $e^{ix\xi}$: the bigger the norm, the more $f$ is correlated with a sine wave.
There are also $U^s$ norms for $s \geq 3$ that measure how much $f$ correlates with $e^{ip(x, \xi)}$ where $p$ is a polynomial of degree $s - 1$.

Anyways, rescaling the Gowers inverse theorem gives us
$$\|\mathscr F_h (1_{X_h} f)\|_{L^\infty \to L^4} \sim h^{d/4} \|\widehat{1_{X_h} f}\|_{L^4} = h^{d/4} \|1_{X_h} f\|_{U^2}.$$
We observe that
$$\mu * 1_{B_h}(x) = \mu(B(x, h)) \sim h^\delta$$
for $x \in X$, which gives
$$1_{X_h} f \lesssim h^{-\delta} (\mu * 1_{B_h}) \|f\|_{L^\infty}$$
and hence
$$\|\mathscr F_h 1_{X_h}\|_{L^\infty \to L^4} \lesssim h^{\frac{d}{4} - \delta} \|\mu * 1_{B_h}\|_{U^2}.$$
Summing up, we have 
$$\|1_{X_h} \mathscr F_h 1_{X_h}\|_{L^2 \to L^2} \lesssim h^{\frac{3}{8}(d - \delta)} h^{-\frac{d}{4}} h^{\frac{d}{8} - \frac{\delta}{2}} \|\mu * 1_{B_h}\|_{U^2}.$$

\printbibliography
\end{document}
