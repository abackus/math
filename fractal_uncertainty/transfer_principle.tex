\documentclass[reqno,12pt,letterpaper]{amsart}
\RequirePackage{amsmath,amssymb,amsthm,graphicx,mathrsfs,url,slashed}
\RequirePackage[usenames,dvipsnames]{xcolor}
\RequirePackage[colorlinks=true,linkcolor=Red,citecolor=Green]{hyperref}
\RequirePackage{amsxtra}
\usepackage{cancel}
\usepackage{tikz-cd}

\setlength{\textheight}{9.3in} \setlength{\oddsidemargin}{-0.25in}
\setlength{\evensidemargin}{-0.25in} \setlength{\textwidth}{7in}
\setlength{\topmargin}{-0.25in} \setlength{\headheight}{0.18in}
\setlength{\marginparwidth}{1.0in}
\setlength{\abovedisplayskip}{0.2in}
\setlength{\belowdisplayskip}{0.2in}
\setlength{\parskip}{0.05in}
\renewcommand{\baselinestretch}{1.05}

\title{Fractal uncertainty principle via finite abelian groups}
\author{Aidan Backus, James Leng, Zhongkai Tao}
\date{December 2022}

\newcommand{\NN}{\mathbf{N}}
\newcommand{\ZZ}{\mathbf{Z}}
\newcommand{\QQ}{\mathbf{Q}}
\newcommand{\RR}{\mathbf{R}}
\newcommand{\CC}{\mathbf{C}}
\newcommand{\DD}{\mathbf{D}}
\newcommand{\PP}{\mathbf P}
\newcommand{\MM}{\mathbf M}
\newcommand{\II}{\mathbf I}
\newcommand{\Hyp}{\mathbf H}
\newcommand{\Sph}{\mathbf S}
\newcommand{\GL}{\mathbf{GL}}
\newcommand{\Orth}{\mathbf{O}}
\DeclareMathOperator*{\Expect}{\mathbf E}

\newcommand{\Four}{\mathcal F}
\newcommand{\FourD}{\mathcal F^{\mathrm{disc}}}
\newcommand{\disc}{\mathrm{disc}}
\DeclareMathOperator{\hull}{hull}
\newcommand{\dif}{\mathrm d}
\newcommand{\tr}{\operatorname{tr}}
\newcommand{\supp}{\operatorname{supp}}
\newcommand{\dbar}{\overline \partial}
\newcommand{\Tree}{\mathcal T}
\newcommand{\iso}{\xrightarrow{\sim}}

\newcommand{\dfn}[1]{\emph{#1}\index{#1}}

\renewcommand{\Re}{\operatorname{Re}}
\renewcommand{\Im}{\operatorname{Im}}

\newcommand{\loc}{\mathrm{loc}}
\newcommand{\cpt}{\mathrm{cpt}}

\def\Japan#1{\left \langle #1 \right \rangle}

\newtheorem{theorem}{Theorem}[section]
\newtheorem{badtheorem}[theorem]{``Theorem"}
\newtheorem{prop}[theorem]{Proposition}
\newtheorem{lemma}[theorem]{Lemma}
\newtheorem{sublemma}[theorem]{Sublemma}
\newtheorem{claim}[theorem]{Claim}
\newtheorem{proposition}[theorem]{Proposition}
\newtheorem{corollary}[theorem]{Corollary}
\newtheorem{conjecture}[theorem]{Conjecture}
\newtheorem{axiom}[theorem]{Axiom}
\newtheorem{assumption}[theorem]{Assumption}

\theoremstyle{definition}
\newtheorem{definition}[theorem]{Definition}
\newtheorem{remark}[theorem]{Remark}
\newtheorem{example}[theorem]{Example}
\newtheorem{notation}[theorem]{Notation}

\newtheorem{exercise}[theorem]{Discussion topic}
\newtheorem{homework}[theorem]{Homework}
\newtheorem{problem}[theorem]{Problem}

\newtheorem{ack}{Acknowledgements}

\numberwithin{equation}{section}


% Mean
\def\Xint#1{\mathchoice
{\XXint\displaystyle\textstyle{#1}}%
{\XXint\textstyle\scriptstyle{#1}}%
{\XXint\scriptstyle\scriptscriptstyle{#1}}%
{\XXint\scriptscriptstyle\scriptscriptstyle{#1}}%
\!\int}
\def\XXint#1#2#3{{\setbox0=\hbox{$#1{#2#3}{\int}$ }
\vcenter{\hbox{$#2#3$ }}\kern-.6\wd0}}
\def\ddashint{\Xint=}
\def\dashint{\Xint-}

\usepackage[backend=bibtex,style=alphabetic]{biblatex}
\addbibresource{fup.bib}
\renewbibmacro{in:}{}
\DeclareFieldFormat{pages}{#1}


\begin{document}
% \begin{abstract}
% The least-gradient maximum principle, essentially due to Miranda and de Giorgi in the 1960s, shows that least-gradient functions define a minimal lamination of the support of their derivative.
% We extend this result to least-gradient functions on hyperbolic manifolds and propose means for a proof on general Riemannian manifold.
% \end{abstract}

\maketitle

%%%%%%%%%%%%%%%%%%%%%%%%%%%%%%%%%%%%%%%%%%%%%%%%%%%%%%%

Let's try to come up with a transfer principle relating the Fourier transform on $\RR$ with its discretization on $\ZZ/p^n$ in the semiclassical limit $h \to 0$.
Here $p$ is a large but fixed prime, $n$ is a scale parameter, and $h = p^{-n}$ is the semiclassical wavelength.
Roughly speaking we're going to ``throw away frequencies above $1/h$''. In the context of the fractal uncertainty principle this shouldn't be a problem because we're only interested in functions which are bandlimited to a compact fractal $Y$.

We do \emph{not} identify the groups $\ZZ/p^n$ and $\widehat{\ZZ/p^n}$ even though they are isomorphic with the same underlying set $\{0, \dots, p^n - 1\}$.
Rather, we think of $j \in \ZZ/p^n$ as times $x \in [0, 1] = \RR/\ZZ$ by $x = jh$ and think of $k \in \widehat{\ZZ/p^n}$ as frequencies $\xi \in \ZZ$ by $k = \xi$.

\section{Discretization}
We write 
$$e(\theta) = \exp(2\pi i\theta).$$
For $f: \ZZ/p^n \to \CC$ we have the Fourier inversion formula 
$$f(j) = h \sum_{k=0}^{p^n - 1} e(jhk) \sum_{m=0}^{p^n - 1} f(m) e(-mhk).$$
Here $j, m$ are times and $k$ is a frequency. Moving the $h$ inside and using $h^{-1} = p^n$ we can rewrite this sum as 
$$f(j) = \sum_{k=0}^{p^n - 1} e(jhk) \Expect_{m \in \ZZ/p^n} f(m) e(-mhk).$$
This motivates us to define:

\begin{definition}
For $f \in L^2$, we define the \dfn{discretization} $f_h: \ZZ/p^n \to \CC$ by
$$f_h(j) = \sum_{k=0}^{p^n - 1} e(jhk) \int_0^1 f(x) e(-xk) \dif x.$$
\end{definition}

Note that this normalization agrees with Theorem 4 in the original Bourgain--Dyatlov paper, but \emph{not} the semiclassical formalism that seems to have become the industry standard since then.
In accordance with this normalization, we define for $f \in \ZZ/p^n$
$$||f||_{\ell^2_h}^2 = \Expect_{m \in \ZZ/p^n} |f(m)|^2$$
and for $g \in \widehat{\ZZ/p^n}$
$$||g||_{\widehat{\ell^2_h}}^2 = \sum_{k = 0}^{p^n - 1} |g(k)|^2.$$
Then the discrete Fourier transform
$$\Four_hf(k) = \Expect_{m \in \ZZ/p^n} f(m) e(-mhk)$$
is a unitary morphism 
$$\Four_h: \ell^2_h \iso \widehat{\ell^2_h}$$
which respects discretization in the sense that 
$$\Four_h f_h = 1_{\ZZ/p^n} \Four f$$
where we normalize 
$$\Four f(k) = \int_0^1 f(x) e(-xk) \dif x.$$ 

\begin{lemma}
For every $f \in L^2(\RR)$ we have 
$$||f_h||_{\ell^2_h} \leq ||f||_{L^2([0, 1])}.$$
If $f$ is bandlimited to $\{0, \dots, p^n - 1\}$ then this inequality is an equality.
\end{lemma}
\begin{proof}
It follows from bounding
\begin{align*}
||f_h||_{\ell^2_h} &= ||\Four_h f_h||_{\widehat{\ell^2_h}} = ||1_{\ZZ/p^n} \Four f||_{\ell^2} \leq ||\Four f||_{\ell^2} = ||f||_{L^2([0, 1])}. \qedhere
\end{align*}
\end{proof}

We define the discretization of sets of times $X \subseteq [0, 1]$ which support a Radon measure $\mu_X$ by thinking of $[0, 1]$ as a $p$-adic tree.
More precisely, we define the tree of intervals $\Tree$ by setting the root to be $[0, 1]$ and setting the children of an interval $I \in \Tree$ to be the intervals obtained by splitting $I$ into $p$ subintervals of length $|I|/p$.
We number the intervals at level $n$ of the tree to correspond left-to-right to the elements of $\ZZ/p^n$.
Then we set $\Tree(X)$ to be the subtree of active intervals, thus 
$$\Tree(X) = \{I \in \Tree: \mu_X(I) > 0\},$$
and set $X_h$ to be the subset of $\ZZ/p^n$ corresponding to level $n$ of $\Tree(X)$.

We define the discretization of sets of frequencies $Y \subseteq \widehat \RR$ which support a Radon measure $\mu_Y$ more simply: we set 
$$Y_h = \{k \in \ZZ/p^n: \mu_Y([k, k + 1)) > 0\}.$$
Note that unlike a discrete sets of times $X_h$, the discrete set of frequencies $Y_h$ does not depend on $h$ once $h$ is small enough.

\section{Fractal uncertainty principle}
We now show that the fractal uncertainty principle reduces to a statement about abelian groups.

\begin{definition}
We say that $\beta > 0$ is a \dfn{discrete uncertainty exponent} for $(X, Y)$ if:
\begin{enumerate}
\item $X \subseteq [0, 1]$ and $Y \subseteq \widehat \RR$ are compact sets.
\item For every $f: \ZZ/p^n \to \CC$ which is bandlimited to $Y_h$ and every $0 < h \ll 1$,
$$||1_{X_h} f||_{\ell^2_h} \lesssim h^\beta ||f||_{\ell^2_h}.$$
\end{enumerate}
\end{definition}

\begin{definition}
Let $0 < h_1 < h_2 < 1$.
The set of times $X$ is \dfn{elliptic for convolution} on scales $h \in [h_1, h_2]$ if
$$||1_X f||_{L^2} \lesssim ||1_{X_h} f_h||_{\ell^2_h}.$$
\end{definition}

We write $\hull Y$ for the convex hull of a set $Y$. Note that for any $X$,
$$\Four_h(1_{X_h} f_h) = (\Four_h 1_{X_h}) * (\Four_h f_h) = (\Four_h 1_{X_h}) * 1_{\ZZ/p^n} \Four f.$$
Now if $1/h > |\hull Y|$, after modulating $f$ we may assume that $\supp \Four f \subseteq 1_{\ZZ/p^n}$.
We are then interested in when the convolution operator
\begin{align*}
\ell^2_h &\to \ell^2_h \\
g &\mapsto (\Four_h 1_{X_h}) * g
\end{align*}
is elliptic.
There's a chapter in H\"ormander II about convolution operators which might have something to say about this.
My intuition is that $h_1 = N^{-1}$ where $N$ is as in Bourgain--Dyatlov's Theorem 4: this operator should be elliptic if $X$ is well-approximated by its discretization, which should happen on any scale where $X$ looks $\delta$-regular if its discretization is to be a $\delta$-regular set.

\begin{proposition}[transfer principle]
Suppose that $\beta$ is a discrete uncertainty exponent for $(X, Y)$ and $X$ is elliptic for convolution on scales $[h_1, h_2]$.
If $h_2$ is small enough depending on $|\hull Y|$, then for every $f: \RR \to \CC$ which is bandlimited to $Y$ and every $h \in [h_1, h_2]$,
$$||1_X f||_{L^2} \lesssim h^\beta ||f||_{L^2}.$$
\end{proposition}
\begin{proof}
By modulating $f$ we may assume that $Y$ only consists of positive frequencies, and then if we select $h_*$ small enough,
\begin{align*}
||1_X f||_{L^2} &\lesssim ||1_{X_h} f_h||_{\ell^2_h} \lesssim h^\beta ||f_h||_{\ell^2_h} \leq h^\beta ||f||_{L^2}. \qedhere 
\end{align*}
\end{proof}

\printbibliography

\end{document}
