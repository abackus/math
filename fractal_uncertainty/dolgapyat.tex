\documentclass[reqno,10pt]{amsart}
\usepackage[letterpaper, margin=1in]{geometry}
\RequirePackage{amsmath,amssymb,amsthm,graphicx,mathrsfs,url,slashed}
\RequirePackage[usenames,dvipsnames]{xcolor}
\RequirePackage[colorlinks=true,linkcolor=Red,citecolor=Green]{hyperref}
\RequirePackage{amsxtra}
\usepackage{cancel}
\usepackage{tikz-cd}

% \setlength{\textheight}{9.3in} \setlength{\oddsidemargin}{-0.25in}
% \setlength{\evensidemargin}{-0.25in} \setlength{\textwidth}{7in}
% \setlength{\topmargin}{-0.25in} \setlength{\headheight}{0.18in}
% \setlength{\marginparwidth}{1.0in}
% \setlength{\abovedisplayskip}{0.2in}
% \setlength{\belowdisplayskip}{0.2in}
% \setlength{\parskip}{0.05in}
%\renewcommand{\baselinestretch}{1.05}

\title{Dolgopyat}
\author{los tres amigos}
\date{July 2022}

\newcommand{\NN}{\mathbf{N}}
\newcommand{\ZZ}{\mathbf{Z}}
\newcommand{\QQ}{\mathbf{Q}}
\newcommand{\RR}{\mathbf{R}}
\newcommand{\CC}{\mathbf{C}}
\newcommand{\DD}{\mathbf{D}}
\newcommand{\PP}{\mathbf P}
\newcommand{\MM}{\mathbf M}
\newcommand{\II}{\mathbf I}
\newcommand{\Hyp}{\mathbf H}
\newcommand{\Sph}{\mathbf S}
\newcommand{\Group}{\mathbf G}
\newcommand{\GL}{\mathbf{GL}}
\newcommand{\Orth}{\mathbf{O}}
\newcommand{\SpOrth}{\mathbf{SO}}
\newcommand{\Ball}{\mathbf{B}}

\DeclareMathOperator*{\Expect}{\mathbf E}

\DeclareMathOperator{\avg}{avg}
\DeclareMathOperator{\card}{card}
\DeclareMathOperator{\cent}{center}
\DeclareMathOperator{\ch}{ch}
\DeclareMathOperator{\codim}{codim}
\DeclareMathOperator{\Cyl}{Cyl}
\DeclareMathOperator{\diag}{diag}
\DeclareMathOperator{\diam}{diam}
\DeclareMathOperator{\dom}{dom}
\DeclareMathOperator{\Exc}{Exc}
\newcommand{\ext}{\mathrm{ext}}
\DeclareMathOperator{\Gal}{Gal}
\DeclareMathOperator{\Hom}{Hom}
\DeclareMathOperator{\Iso}{Iso}
\DeclareMathOperator{\Jac}{Jac}
\DeclareMathOperator{\Lip}{Lip}
\DeclareMathOperator{\Met}{Met}
\DeclareMathOperator{\id}{id}
\DeclareMathOperator{\rad}{rad}
\DeclareMathOperator{\rank}{rank}
\DeclareMathOperator{\Rm}{Rm}
\DeclareMathOperator{\Hess}{Hess}
\DeclareMathOperator{\Hol}{Hol}
\DeclareMathOperator{\Radon}{Radon}
\DeclareMathOperator*{\Res}{Res}
\DeclareMathOperator{\sgn}{sgn}
\DeclareMathOperator{\singsupp}{sing~supp}
\DeclareMathOperator{\Spec}{Spec}
\DeclareMathOperator{\supp}{supp}
\DeclareMathOperator{\Tan}{Tan}
\newcommand{\tr}{\operatorname{tr}}

\newcommand{\Mink}{\mathbf m}
\newcommand{\Ric}{\mathrm{Ric}}
\newcommand{\Riem}{\mathrm{Riem}}
\newcommand*\dif{\mathop{}\!\mathrm{d}}
\newcommand*\Dif{\mathop{}\!\mathrm{D}}
\newcommand{\LapQL}{\Delta^{\mathrm{ql}}}

\newcommand{\dbar}{\overline \partial}

\DeclareMathOperator{\atanh}{atanh}
\DeclareMathOperator{\csch}{csch}
\DeclareMathOperator{\sech}{sech}

\DeclareMathOperator{\Div}{div}
\DeclareMathOperator{\Gram}{Gram}
\DeclareMathOperator{\grad}{grad}
\DeclareMathOperator{\dist}{dist}
\DeclareMathOperator{\spn}{span}
\DeclareMathOperator{\Ell}{Ell}
\DeclareMathOperator{\WF}{WF}

\newcommand{\Lagrange}{\mathscr L}
\newcommand{\DirQL}{\mathscr D^{\mathrm{ql}}}
\newcommand{\DirL}{\mathscr D}

\newcommand{\Hilb}{\mathcal H}
\newcommand{\Homology}{\mathrm H}
\newcommand{\normal}{\mathbf n}
\newcommand{\radial}{\mathbf r}
\newcommand{\evect}{\mathbf e}
\newcommand{\vol}{\mathrm{vol}}

\newcommand{\pic}{\vspace{30mm}}
\newcommand{\dfn}[1]{\emph{#1}\index{#1}}

\renewcommand{\Re}{\operatorname{Re}}
\renewcommand{\Im}{\operatorname{Im}}

\newcommand{\loc}{\mathrm{loc}}
\newcommand{\cpt}{\mathrm{cpt}}

\def\Japan#1{\left \langle #1 \right \rangle}

\newtheorem{theorem}{Theorem}[section]
\newtheorem{badtheorem}[theorem]{``Theorem"}
\newtheorem{prop}[theorem]{Proposition}
\newtheorem{lemma}[theorem]{Lemma}
\newtheorem{sublemma}[theorem]{Sublemma}
\newtheorem{claim}[theorem]{Claim}
\newtheorem{proposition}[theorem]{Proposition}
\newtheorem{corollary}[theorem]{Corollary}
\newtheorem{conjecture}[theorem]{Conjecture}
\newtheorem{axiom}[theorem]{Axiom}
\newtheorem{assumption}[theorem]{Assumption}

\theoremstyle{definition}
\newtheorem{definition}[theorem]{Definition}
\newtheorem{remark}[theorem]{Remark}
\newtheorem{example}[theorem]{Example}
\newtheorem{notation}[theorem]{Notation}

\newtheorem{exercise}[theorem]{Discussion topic}
\newtheorem{homework}[theorem]{Homework}
\newtheorem{problem}[theorem]{Problem}

\makeatletter
\newcommand{\proofpart}[2]{%
  \par
  \addvspace{\medskipamount}%
  \noindent\emph{Part #1: #2.}
}
\makeatother

\newtheorem{ack}{Acknowledgements}

\numberwithin{equation}{section}


% Mean
\def\Xint#1{\mathchoice
{\XXint\displaystyle\textstyle{#1}}%
{\XXint\textstyle\scriptstyle{#1}}%
{\XXint\scriptstyle\scriptscriptstyle{#1}}%
{\XXint\scriptscriptstyle\scriptscriptstyle{#1}}%
\!\int}
\def\XXint#1#2#3{{\setbox0=\hbox{$#1{#2#3}{\int}$ }
\vcenter{\hbox{$#2#3$ }}\kern-.6\wd0}}
\def\ddashint{\Xint=}
\def\dashint{\Xint-}

\usepackage[backend=bibtex,style=numeric]{biblatex}
\renewcommand*{\bibfont}{\normalfont\footnotesize}
\addbibresource{topics.bib}
\renewbibmacro{in:}{}
\DeclareFieldFormat{pages}{#1}


\begin{document}
\begin{abstract}
    Dolgopyat's method...
\end{abstract}

\maketitle

%%%%%%%%%%%%%%%%%%%%%%%%%%%%%%%%%%%%%%%%%%%%%%%%%%%%%%%

% \tableofcontents

\section{The trees that discretize}
\begin{definition}
A compactly supported Borel measure $\mu$ on $\RR^d$ is \dfn{$\delta$-regular} on scales $[\alpha, \beta]$ if for every square box with side length $\in [\alpha, \beta]$,
$$\mu(I) \leq C_R |I|^{\delta/d},$$
and if in addition $I$ is centered on $x \in \supp \mu$,
$$C_R^{-1} |I|^{\delta/d} \leq \mu(I).$$
\end{definition}

For $q \in L^{-k} \ZZ$ we set $I_k(q)$ to be the half-open box
$$I_k(q) := [q_1 + L^{-k} + q_1) \times [q_2 + L^{-k} + q_2) \times \cdots \times [q_d, L^{-k} + q_d).$$

\begin{definition}
Let $\mu$ be a compactly supported Borel measure on $\RR^d$.
The \dfn{tree which discretizes} $\mu$ is the tree $V$ whose level $V_k$ are nodes which are given by subsets of $\RR^d$ obtained from the following procedure: start with the set of all boxes $I_k(q)$ such that $\mu(I_k(q)) > 0$, then merge any two boxes which are adjacent.
We denote its height function on nodes by $H$.
\end{definition}

So the nodes of $V$ are not necessarily boxes. We don't want to fill them in either.
If we did, and we were studying a fractal which was L-shaped, then we'd end up being really wasteful, at least at scales that aren't too fine: the smallest box containing an L-shape is much larger than the L, and doesn't scale correctly.

Clearly any two distinct nodes in $V_k$ are $L^{-k}$-separated, and $\mu$ is supported on the union of the nodes in $V_k$.
Also it really is a tree. Finally,
$$0 < \mu(I) = \sum_n \mu(I_n)$$
whenever $(I_n)$ are the children of $I$.

Since we're merging adjacent boxes, we could have two boxes in the same node which only touch on a corner, or more generally an edge of codimension $\geq 2$.

\begin{lemma}
Let $\mu$ be $\delta$-regular to scale $L^{-K}$ and $0 < \delta < 2$. Then for every node $I \in V$,
$$L^{-dH(I)} \leq |I| \lesssim_d (C_R)^{\frac{2d}{d - \delta}} L^{-dH(I)}$$
and 
$$C_R^{-1} L^{-\delta H(I)} \leq \mu(I) \lesssim_d C_R^{\frac{2\delta}{d - \delta} + 1} L^{-\delta H(I)}.$$
\end{lemma}
\begin{proof}
The lower bound on $|I|$ is obvious. For the upper bound, let $I$ be the union of $M$ boxes $I_k(q_n)$.
Then the $I_k(q_n)$ are almost disjoint and $\mu(I_k(q_n)) > 0$.
Let $x_n \in I_k(q_n) \cap \supp \mu$ and let $J_n$ be the box of side length $L^{-k}$ centered on $x_n$.
Let $J = \bigcup_n J_n$, then $J \setminus I$ is $\mu$-null since distinct nodes are $L^{-k}$-separated. In particular,
$$\mu(J) \leq \mu(I) \leq C_R (ML^{-kd})^{\delta/d} = C_R M^{\delta/d} L^{-k\delta},$$
and $(J_n)$ covers $J$ efficiently in the sense that for every $x \in J$, $x$ is contained in at most $O_d(1)$ many $J_n$'s.
Here $O_d(1)$ is the combinatorial constant defined to be the number of boxes in the standard grid of side length $L^{-k}$ that are within $2$ boxes of the origin.

So
$$\mu(J) \gtrsim_d M\mu(J_n) \geq MC_R^{-1}L^{-k\delta},$$
and solving for $M$ gives 
$$M \lesssim_d C_R^{\frac{2d}{d - \delta}}.$$
Now we use $|I| = \sum_n |I_n|$, which also gives the upper bound on $\mu(I)$.

For the lower bound on $\mu(I)$, let $J$ be a box of side length $L^{-k}$ centered on $I \cap \supp \mu$.
Then $J \setminus I$ is $\mu$-null so 
\begin{align*}
\mu(I) &\geq \mu(J) \geq C_R^{-1} |J|^{\delta/d} = C_R^{-1} (L^{-kd})^{\delta/d} = C_R^{-1} L^{-k\delta}. \qedhere 
\end{align*}
\end{proof}

\begin{lemma}
Let $J$ be a child of $I$, then 
$$\mu(J) \gtrsim_d C_R^{-1 - \frac{2\delta}{d - \delta}} L^{-\delta} \mu(I).$$
\end{lemma}
\begin{proof}
By the previous lemma, 
$$\mu(J) \geq C_R^{-1} L^{-\delta} L^{-\delta H(I)}$$
and 
$$L^{-\delta H(I)} C_R^{\frac{2\delta}{d - \delta}} \gtrsim_d \mu(I).$$
In particular, 
\begin{align*}
\mu(J) &\gtrsim_d L^{-\delta} \frac{\mu(I)}{C_R^{\frac{2\delta}{d - \delta} + 1}}. \qedhere 
\end{align*}
\end{proof}

Separation to scale $L^{k-2/3}$ needs to be formulated more carefully in this formalism.
The reason is that unlike in one dimension, the nodes do not need to be convex sets. 
Actually, that makes me think that this formalism is fundamentally incorrect: we *don't* want to merge adjacent boxes!

That doesn't mean the above lemmata are worthless though.
In the formalism where we don't merge boxes, Lemma 1.3 is telling you that at most $O_d(C_R^{-\frac{2d}{d - \delta}})$ boxes can touch at each scale.
You could probably get more precise estimates by refining what we mean by touching: in some sense, most of the intersections should have high codimension, so you can get a stronger estimate for codimension-$1$ intersections.

\section{Induction on scale}
A model problem for Prop 3.1 in Dyatlov--Jin is:

\begin{proposition}
Let $0 < \delta < D$ (where $D = D(d)$, maybe $D = 1$), $C_R > 1$.
Let $I_0, J_0 \subset \RR^d$ be boxes containing $X, Y$ repsectively, where $X, Y$ are $\delta$-regular fractals with regularity $C_R$.
Then there exist $L, \varepsilon_1 > 0$ with $L \lesssim C_R^{O_\delta(1)}$ and $\varepsilon_1^{-1} \lesssim C_R^{O_\delta(1)}$, such that if we set $K \in \NN$, $h = L^{-K}$, and
$$\varepsilon_0 := -\frac{\log(1 - \varepsilon)}{\log L}$$
then 
$$||\mathcal F_h f||_{L^2(\mu_X)} \lesssim_{\mu_X(X), \mu_Y(Y)} h^{\varepsilon_0} ||f||_{L^2(\mu_Y)}.$$
\end{proposition}

\printbibliography

\end{document}
