
% --------------------------------------------------------------
% This is all preamble stuff that you don't have to worry about.
% Head down to where it says "Start here"
% --------------------------------------------------------------

\documentclass[10pt]{article}

\usepackage[margin=.7in]{geometry}
\usepackage{amsmath,amsthm,amssymb,mathrsfs}
\usepackage{enumitem}
\usepackage{tikz-cd}
\usepackage{mathtools}
\usepackage{amsfonts}
\usepackage{listings}
\usepackage{algorithm2e}
\usepackage{verse,stmaryrd}
\usepackage{fancyvrb}

% Number systems
\newcommand{\NN}{\mathbb{N}}
\newcommand{\ZZ}{\mathbb{Z}}
\newcommand{\QQ}{\mathbb{Q}}
\newcommand{\RR}{\mathbb{R}}
\newcommand{\CC}{\mathbb{C}}
\newcommand{\PP}{\mathbb P}
\newcommand{\FF}{\mathbb F}
\newcommand{\DD}{\mathbb D}
\renewcommand{\epsilon}{\varepsilon}

\newcommand{\Aut}{\operatorname{Aut}}
\newcommand{\coker}{\operatorname{coker}}
\newcommand{\CVect}{\CC\operatorname{-Vect}}
\newcommand{\Cantor}{\mathcal{C}}
\newcommand{\D}{\mathcal{D}}
\newcommand{\card}{\operatorname{card}}
\newcommand{\dbar}{\overline \partial}
\DeclareMathOperator*{\esssup}{ess\,sup}
\newcommand{\GL}{\operatorname{GL}}
\newcommand{\Hom}{\operatorname{Hom}}
\newcommand{\id}{\operatorname{id}}
\newcommand{\Ind}{\operatorname{Ind}}
\newcommand{\Inn}{\operatorname{Inn}}
\newcommand{\interior}{\operatorname{int}}
\newcommand{\lcm}{\operatorname{lcm}}
\newcommand{\mesh}{\operatorname{mesh}}
\newcommand{\LL}{\mathcal L_0}
\newcommand{\Leb}{\mathcal{L}_{\text{loc}}^2}
\newcommand{\Lip}{\operatorname{Lip}}
\newcommand{\ppGL}{\operatorname{PGL}}
\newcommand{\ppic}{\vspace{35mm}}
\newcommand{\ppset}{\mathcal{P}}
\DeclareMathOperator{\proj}{proj}
\DeclareMathOperator*{\Res}{Res}
\newcommand{\Riem}{\mathcal{R}}
\newcommand{\RVect}{\RR\operatorname{-Vect}}
\newcommand{\Sch}{\mathcal{S}}
\newcommand{\SL}{\operatorname{SL}}
\newcommand{\sgn}{\operatorname{sgn}}
\newcommand{\spn}{\operatorname{span}}
\newcommand{\Spec}{\operatorname{Spec}}
\newcommand{\supp}{\operatorname{supp}}
\newcommand{\Torus}{\mathbb T}
\DeclareMathOperator{\tr}{tr}

\DeclareMathOperator{\adj}{adj}
\DeclareMathOperator{\curl}{curl}

% Calculus of variations
\DeclareMathOperator{\pp}{\mathbf p}
\DeclareMathOperator{\zz}{\mathbf z}
\DeclareMathOperator{\uu}{\mathbf u}
\DeclareMathOperator{\vv}{\mathbf v}
\DeclareMathOperator{\ww}{\mathbf w}

\DeclareMathOperator{\Olo}{\mathscr O}

% Categories
\newcommand{\Ab}{\mathbf{Ab}}
\newcommand{\Cat}{\mathbf{Cat}}
\newcommand{\Group}{\mathbf{Group}}
\newcommand{\Module}{\mathbf{Module}}
\newcommand{\Set}{\mathbf{Set}}
\DeclareMathOperator{\Fun}{Fun}
\DeclareMathOperator{\Iso}{Iso}

% Complex analysis
\renewcommand{\Re}{\operatorname{Re}}
\renewcommand{\Im}{\operatorname{Im}}

% Logic
\renewcommand{\iff}{\leftrightarrow}
\newcommand{\Henkin}{\operatorname{Henk}}
\newcommand{\PA}{\mathbf{PA}}
\DeclareMathOperator{\proves}{\vdash}

% Group
\DeclareMathOperator{\Gal}{Gal}
\DeclareMathOperator{\Fix}{Fix}
\DeclareMathOperator{\Out}{Out}

% Other symbols
\newcommand{\heart}{\ensuremath\heartsuit}

\DeclareMathOperator{\atanh}{atanh}

% Theorems
\theoremstyle{definition}
\newtheorem*{corollary}{Corollary}
\newtheorem*{falselemma}{Grader's ``Lemma"}
\newtheorem{exer}{Exercise}
\newtheorem{lemma}{Lemma}[exer]
\newtheorem{theorem}[lemma]{Theorem}

\usepackage[backend=bibtex,style=alphabetic,maxcitenames=50,maxnames=50]{biblatex}
\renewbibmacro{in:}{}
\DeclareFieldFormat{pages}{#1}

\begin{document}
\noindent
\large\textbf{Finite element methods, HW 4} \hfill \textbf{Aidan Backus} \\
% --------------------------------------------------------------
%                         Start here
% --------------------------------------------------------------\

\begin{exer}
Consider the variational formulation of the Timoshenko beam of length $\ell$.

Explain why it is reasonable to assume that $EI$ and $GA$ satisfy inequalities of the form
$$0 < \gamma \leq E(x)I(x) \leq \Gamma.$$

Prove that 
$$||\psi'||^2 + ||v' - \psi||^2 \leq 2||(v, \psi)||_V^2.$$
Conclude that the bilinear form satisfies 
$$a(u, \phi, v, \psi) \leq 2\Gamma||(u, \phi)||_V ||(v, \psi)||_V.$$

Show that 
$$||\psi||^2 + ||\psi'||^2 \leq C||\psi'||^2$$
and that 
$$||v||^2 + ||v'||^2 \leq C(||\phi'||^2 + ||v' - \phi||^2)$$
where $C > 0$ only depends on $\ell$.
Deduce that the bilinear form is coercive.

Show that the linear form is continuous under appropriate assumptions on the data, so the Timoshenko beam admits a unique solution.
\end{exer}

If $EI$ became arbitrarily small, then the beam would have no flexural rigidity and would twist or shatter by an arbitrary amount from an arbitrarily small weight; such a beam would be both unrealistic and useless.
If it became arbitrarily large, then the beam would be ``invincible'' and could never be bent or broken, which would be nice but isn't actually possible in real life.

If $GA$ became arbitrarily small or arbitrarily large, then either the shear modulus $G$ or cross-sectional area $A$ would become arbitrarily small or large. If $G$ became arbitrarily small, then the beam could be cut by an arbitrarily weak pair of scissors, and therefore would be useless; if it became arbitrarily large, then the beam could \emph{never} be cut, which would also be useless since one needs to cut a beam to insert it into an object.
If $A$ became arbitrarily small then the beam would have to be thinner than an electron which is absurd; if it became arbitrarily large then the beam would have to be infinitely wide which is similarly absurd.

To show that $a$ is continuous, we estimate  
$$||v' - \psi||^2 = ||v'||^2 - 2\langle v', \psi\rangle + ||\psi||^2$$
which by the Cauchy-Schwarz and Cauchy inequalities is 
$$\leq ||v'||^2 + ||v'||^2 + ||\psi||^2 + ||\psi||^2 = 2||v'||^2 + 2||\psi||^2.$$
Therefore 
$$||\psi'||^2 + ||v' - \psi||^2 \leq ||\psi'||^2 + 2||v'||^2 + 2||\psi||^2 \leq 2||(v, \psi)||_V^2.$$
In particular,
$$a(u, \phi, v, \psi) \leq \Gamma |\langle \phi', \psi'\rangle| + \Gamma |\langle u' - \phi, v' - \psi\rangle|$$
and from the Cauchy-Schwarz and Cauchy inequalities we obtain 
$$a(U, \phi, v, \psi) \leq \Gamma \sqrt{||\psi'||^2 + ||v' - \psi||^2} \sqrt{||\phi'||^2 + ||u' - \phi||^2} \leq 2\Gamma ||(v, \psi)||_V ||(u, \phi)||_V.$$

To show that $a$ is elliptic, we observe that by the Poincar\'e inequality, $||\psi|| \lesssim ||\psi'||$ (here and always, implied constants depend only on $\ell$).
We conclude that
$$||\psi||^2 + ||\psi'||^2 \leq ||\psi||^2 + 2|\langle \psi, \psi'\rangle| + ||\psi'||^2 = (||\psi|| + ||\psi'||)^2 \lesssim ||\psi'||^2.$$
We then estimate 
$$||v||^2 + ||v'||^2 \lesssim ||v'||^2 \leq (||v' - \phi|| + ||\phi||)^2 \lesssim ||v' - \phi||^2 + ||\phi||^2$$
where the last inequality is a consequence of the Cauchy-Schwarz inequality. Since $||\phi|| \leq ||\phi'||$ by the Poincar\'e inequality, we conclude that 
$$||v||^2 + ||v'||^2 \lesssim ||\phi'||^2 + ||v' - \phi||^2 \leq \frac{a(v, \phi, v, \phi)}{\gamma}$$
as desired.

Finally if the data $f, g \in L^2$ then we have 
$$L(v, \psi) \leq ||f|| \cdot ||v|| + ||g|| \cdot ||\psi|| + O(||v|| + ||\psi||)$$
where the bound on the point load and point moment follows from the Sobolev embedding theorem. Since $||f||, ||g|| < \infty$, this shows that $L$ is continuous. So by the Lax-Milgram theorem, we can find $(u, \phi) \in V$ such that for every $(v, \psi) \in V$,
$$a(u, \phi, v, \psi) = L(v, \psi).$$

\begin{exer}
Let $V$ be a real Hilbert space with nonlinear operator $A: V \to V$ such that there exists $M, \alpha$ with $||A(u) - A(v)|| \leq M||u - v||$ and $\langle A(u) - A(v), u - v\rangle \geq \alpha ||u - v||^2$.
Show that the equation $A(u) = f$ has a unique solution.
\end{exer}

Let $B(u) = \rho f - \rho A(u) + u$ where $\rho > 0$ is to be determined. Then if $u = B(u)$,
$$u = \rho f - \rho A(u) + u$$
which implies $A(u) = f$. So it suffices to show that $B$ has a fixed point, and in particular that $B$ is a contraction.
Actually,
$$||B(u) - B(v)||^2 = ||u - v||^2 - 2\rho\langle A(u) - A(v), u - v\rangle + \rho^2||A(u) - A(v)||^2 \leq (1 - 2\alpha\rho + C^2\rho^2) ||u - v||^2.$$
If we set $\rho = \alpha/C^2$ then we get 
$$||B(u) - B(v)||^2 \leq (1 - \alpha\rho) ||u - v||^2$$
as desired.


\end{document}
