
% --------------------------------------------------------------
% This is all preamble stuff that you don't have to worry about.
% Head down to where it says "Start here"
% --------------------------------------------------------------

\documentclass[10pt]{article}

\usepackage[margin=.7in]{geometry}
\usepackage{amsmath,amsthm,amssymb,mathrsfs}
\usepackage{enumitem}
\usepackage{tikz-cd}
\usepackage{mathtools}
\usepackage{amsfonts}
\usepackage{listings}
\usepackage{algorithm2e}
\usepackage{verse,stmaryrd}
\usepackage{fancyvrb}

% Number systems
\newcommand{\NN}{\mathbb{N}}
\newcommand{\ZZ}{\mathbb{Z}}
\newcommand{\QQ}{\mathbb{Q}}
\newcommand{\RR}{\mathbb{R}}
\newcommand{\CC}{\mathbb{C}}
\newcommand{\PP}{\mathbb P}
\newcommand{\FF}{\mathbb F}
\newcommand{\DD}{\mathbb D}
\renewcommand{\epsilon}{\varepsilon}

\newcommand{\Aut}{\operatorname{Aut}}
\newcommand{\coker}{\operatorname{coker}}
\newcommand{\CVect}{\CC\operatorname{-Vect}}
\newcommand{\Cantor}{\mathcal{C}}
\newcommand{\D}{\mathcal{D}}
\newcommand{\card}{\operatorname{card}}
\newcommand{\dbar}{\overline \partial}
\DeclareMathOperator*{\esssup}{ess\,sup}
\newcommand{\GL}{\operatorname{GL}}
\newcommand{\Hom}{\operatorname{Hom}}
\newcommand{\id}{\operatorname{id}}
\newcommand{\Ind}{\operatorname{Ind}}
\newcommand{\Inn}{\operatorname{Inn}}
\newcommand{\interior}{\operatorname{int}}
\newcommand{\lcm}{\operatorname{lcm}}
\newcommand{\mesh}{\operatorname{mesh}}
\newcommand{\LL}{\mathcal L_0}
\newcommand{\Leb}{\mathcal{L}_{\text{loc}}^2}
\newcommand{\Lip}{\operatorname{Lip}}
\newcommand{\ppGL}{\operatorname{PGL}}
\newcommand{\ppic}{\vspace{35mm}}
\newcommand{\ppset}{\mathcal{P}}
\DeclareMathOperator{\proj}{proj}
\DeclareMathOperator*{\Res}{Res}
\newcommand{\Riem}{\mathcal{R}}
\newcommand{\RVect}{\RR\operatorname{-Vect}}
\newcommand{\Sch}{\mathcal{S}}
\newcommand{\SL}{\operatorname{SL}}
\newcommand{\sgn}{\operatorname{sgn}}
\newcommand{\spn}{\operatorname{span}}
\newcommand{\Spec}{\operatorname{Spec}}
\newcommand{\supp}{\operatorname{supp}}
\newcommand{\Torus}{\mathbb T}
\DeclareMathOperator{\tr}{tr}

\DeclareMathOperator{\adj}{adj}
\DeclareMathOperator{\curl}{curl}

% Calculus of variations
\DeclareMathOperator{\pp}{\mathbf p}
\DeclareMathOperator{\zz}{\mathbf z}
\DeclareMathOperator{\uu}{\mathbf u}
\DeclareMathOperator{\vv}{\mathbf v}
\DeclareMathOperator{\ww}{\mathbf w}

\DeclareMathOperator{\Olo}{\mathscr O}

% Categories
\newcommand{\Ab}{\mathbf{Ab}}
\newcommand{\Cat}{\mathbf{Cat}}
\newcommand{\Group}{\mathbf{Group}}
\newcommand{\Module}{\mathbf{Module}}
\newcommand{\Set}{\mathbf{Set}}
\DeclareMathOperator{\Fun}{Fun}
\DeclareMathOperator{\Iso}{Iso}

% Complex analysis
\renewcommand{\Re}{\operatorname{Re}}
\renewcommand{\Im}{\operatorname{Im}}

% Logic
\renewcommand{\iff}{\leftrightarrow}
\newcommand{\Henkin}{\operatorname{Henk}}
\newcommand{\PA}{\mathbf{PA}}
\DeclareMathOperator{\proves}{\vdash}

% Group
\DeclareMathOperator{\Gal}{Gal}
\DeclareMathOperator{\Fix}{Fix}
\DeclareMathOperator{\Out}{Out}

% Other symbols
\newcommand{\heart}{\ensuremath\heartsuit}

\DeclareMathOperator{\atanh}{atanh}

% Theorems
\theoremstyle{definition}
\newtheorem*{corollary}{Corollary}
\newtheorem*{falselemma}{Grader's ``Lemma"}
\newtheorem{exer}{Exercise}
\newtheorem{lemma}{Lemma}[exer]
\newtheorem{theorem}[lemma]{Theorem}

\usepackage[backend=bibtex,style=alphabetic,maxcitenames=50,maxnames=50]{biblatex}
\renewbibmacro{in:}{}
\DeclareFieldFormat{pages}{#1}

\begin{document}
\noindent
\large\textbf{Finite element methods, HW 3} \hfill \textbf{Aidan Backus} \\
% --------------------------------------------------------------
%                         Start here
% --------------------------------------------------------------\

\begin{exer}
Let $\varepsilon(u)$ be the strain tensor of a deformation $u$.
Show that
$$\varepsilon(u):\varepsilon(v) = \sum_{i,j} \varepsilon_{ij}(u) v_{i,j}.$$
Show that if $v$ vanishes on $\partial \Omega$ then
$$\int_\Omega \varepsilon(u):\varepsilon(v) = -\int_\Omega v \cdot (\nabla \cdot \varepsilon(u)).$$
\end{exer}

We have
\begin{align*}
\varepsilon(u):\varepsilon(v) &= \frac{1}{4} \sum_{i,j} (u_{i,j} + u_{j,i})(v_{i, j} + v_{j, i}) \\
&= \frac{1}{4} \sum_{i,j} (u_{i,j} + u_{j,i})v_{i,j} + (u_{i,j} + u_{j,i})v_{j,i}\\
&= \frac{1}{2} \sum_{i,j} \varepsilon_{ij}(u) v_{i, j} + \varepsilon_{ij}(u) v_{j,i} \\
&= \sum_{i,j} \varepsilon_{ij}(u) v_{i,j}
\end{align*}
where the last line follows from the symmetry of the strain tensor, $\varepsilon_{ij}(u) = \varepsilon_{ji}(u)$.

Now we integrate by parts:
$$\int_\Omega \varepsilon(u):\varepsilon(v) = \sum_{i,j} \int_\Omega \varepsilon_{ij}(u) v_{i,j} = \sum_{i,j} \int_{\partial \Omega} v_i n_j \varepsilon_{ij}(u) - \int_\Omega v_i \varepsilon_{ij}(u)_{,j}.$$
Here $n$ is the normal vector and the boundary term vanishes because $v_i = 0$ there.
From the definition of the divergence of the strain tensor, we conclude
$$\int_\Omega \varepsilon(u):\varepsilon(v) = -\int_\Omega v \cdot (\nabla \cdot \varepsilon(u)).$$

\begin{exer}
Consider an elastic plate of thickness $t > 0$ with smooth midsurface $\Omega \subseteq \{z = 0\}$ under a load per unit area of $f(x, y)$ at $(x, y) \in \Omega$.
Suppose that the load gives a deformation of the form $(-z\theta_1, -z\theta_2, w)$ where $\theta_i, w$ are smooth functions on $\Omega$ and we have stress-strain relations $\sigma_{ij} = E\varepsilon_{ij}$ for $i,j \in \{1, 2\}$ and $\sigma_{13} = 2G\varepsilon_{13}$, $\sigma_{23} = 2G\varepsilon_{23}$.
Here $E$ is the elastic modulus and $G$ is the shear modulus.

Compute the strain tensor associated to the deformation.

Show that the elastic energy stored by the plate is given by
$$\alpha Et^3 \int_\Omega \varepsilon(\theta):\varepsilon(\theta) + \beta Gt \int_\Omega |\nabla w - \theta|^2$$
where $\alpha,\beta$ are explicit numerical constants.

Write down the potential energy functional for the load and derive a variational equation.

Show that if the deformation is smooth then $w,\theta$ satisfy a system of PDE of the form
\begin{align*}
\alpha' E \nabla \cdot \varepsilon(\theta) + Gt(\theta - \nabla w) &= 0\\
\beta' Gt \nabla \cdot(\theta - \nabla w) &= f
\end{align*}
wbere $\alpha',\beta'$ are explicit numerical constants.
\end{exer}

To compute the strain tensor we observe that
if $i, j \in \{1, 2\}$ then
$$\varepsilon(u)_{ij} = \frac{-z}{2}(\theta_{i,j} + \theta_{j,i}) -z\varepsilon(\theta)_{ij}$$
and if $i \in \{1, 2\}$ then
$$\varepsilon(u)_{i3} = \varepsilon(u)_{3i} = \frac{1}{2} (w_{,i} - \theta_i)$$
and finally $\varepsilon(u)_{33} = w_{,3} = 0$.
We write this more suggestively as
$$\varepsilon(u) = \begin{bmatrix}-z\varepsilon(\theta) & (\nabla w - \theta)/2 \\ (\nabla w - \theta)^t/2 & 0\end{bmatrix}$$
which is our equation for the strain tensor.

This notation turns out to be quite handy for computing the stress tensor, which is
$$\sigma(u) = \begin{bmatrix}-zE\varepsilon(\theta) & G(\nabla w - \theta) \\ G(\nabla w - \theta)^t & 0\end{bmatrix}.$$
The elastic energy density $dU$ is now given by
$$dU = \varepsilon(u):\sigma(u) ~dxdydz = z^2E\varepsilon(\theta):\varepsilon(\theta) ~dxdydz + G|\nabla w - \theta|^2 ~dxdydz.$$
Integrating in $P = \Omega \times [-t, t]$ and applying Fubini's theorem, we conclude
\begin{align*}
U &= \int_P dU \\
&= \int_{-t}^t Ez^2 \left[\int_\Omega \varepsilon(\theta):\varepsilon(\theta) ~dxdy + \int_\Omega |\nabla w - \theta|^2 ~dxdy\right] ~dz \\
&= \frac{2}{3}Et^3 \int_\Omega \varepsilon(\theta):\varepsilon(\theta) ~dxdy + 2Gt \int_\Omega |\nabla w - \theta|^2 ~dxdy.
\end{align*}
This gives a formula for the elastic energy.

It is reasonable to assume that the force $f$ is pushing downward and so will do work per unit area given by $fw$, thus
$$W = \int_\Omega wf ~dxdy.$$
The total potential energy is then
\begin{align*}
E &= U - W\\
&= \frac{2}{3}Et^3 \int_\Omega \varepsilon(\theta):\varepsilon(\theta) ~dxdy + 2Gt \int_\Omega |\nabla w - \theta|^2 ~dxdy - \int_\Omega wf ~dxdy.
\end{align*}
Here we know that the minus sign on $W$ is motivated by the fact that the potential energy will be minimized when the elastic energy is balanced by the work.

To derive a variational equation from the potential energy, suppose that $\varphi$ is a perturbation of $\theta$ and $v$ is a perturbation of $w$.
We assume that they are compactly supported in $\Omega$, and let $s$ be a small parameter. Then
$$\frac{\partial}{\partial s}\bigg|_{s = 0} E[\theta + s\varphi, w + sv] = 0$$
if $\theta,w$ are minimizers of $E$, but this can be rewritten as
$$\frac{2}{3} Et^3 \int_\Omega \frac{\partial}{\partial s}\bigg|_{s = 0} \varepsilon(\theta + s\varphi):\varepsilon(\theta + s\varphi)
+ 2Gt \int_\Omega \frac{\partial}{\partial s}\bigg|_{s = 0} |\nabla(w + sv) - (\theta + s\varphi)|^2 = \int_\Omega \frac{\partial}{\partial s}\bigg|_{s = 0} (w + sv)f.$$
Differentiating, we conclude the variational equation
$$\frac{4}{3} Et^3 \int_\Omega \varepsilon(\theta):\varepsilon(\varphi) + 4Gt \int_\Omega (\nabla w - \theta) \cdot (\nabla v - \varphi) = \int_\Omega vf.$$

Our next task is to derive the Euler-Lagrange equations.
If we plug in $v = 0$ into the variational equation then we obtain
$$\frac{4}{3} Et^3 \int_\Omega \varepsilon(\theta):\varepsilon(\varphi) - 4Gt \int_\Omega (\nabla w - \theta) \cdot \varphi = 0.$$
We can then integrate by parts using the first problem to conclude
$$\frac{4}{3} Et^3 \int_\Omega \varphi \cdot (\nabla \cdot \varepsilon(\theta)) + 4Gt \int_\Omega (\nabla w - \theta) \cdot \varphi = 0.$$
Since $\varphi$ was arbitrary, if $w \in C^1$ and $\theta \in C^2$ we obtain a PDE
$$\frac{4}{3} Et^3 \nabla \cdot \varepsilon(\theta) + 4Gt (\nabla w - \theta) = 0.$$
On the other hand, we can plug in $\varphi = 0$ to the variational equation to get
$$4Gt \int_\Omega (\nabla w - \theta) \cdot \nabla v = \int_\Omega vf.$$
Integrating by parts we get
$$-4Gt \int_\Omega \nabla \cdot (\nabla w - \theta) v = \int_\Omega vf$$
which gives rise to the PDE
$$4Gt \nabla \cdot (\theta - \nabla w) = f$$
at least assuming that $w \in C^2$ and $\theta \in C^1$.
Dividing our first PDE by $-4$ we obtain the desired differential system
\begin{align*}
-\frac{1}{3} Et^3 \nabla \cdot \varepsilon(\theta) + Gt (\theta - \nabla w) &= 0, \\
4Gt \nabla \cdot (\theta - \nabla w) &= f,
\end{align*}
as long as $\theta, w \in C^2$.

\end{document}
