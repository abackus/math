
% --------------------------------------------------------------
% This is all preamble stuff that you don't have to worry about.
% Head down to where it says "Start here"
% --------------------------------------------------------------

\documentclass[10pt]{article}

\usepackage[margin=.7in]{geometry}
\usepackage{amsmath,amsthm,amssymb,mathrsfs}
\usepackage{enumitem}
\usepackage{tikz-cd}
\usepackage{mathtools}
\usepackage{amsfonts}
\usepackage{listings}
\usepackage{algorithm2e}
\usepackage{verse,stmaryrd}
\usepackage{fancyvrb}

% Number systems
\newcommand{\NN}{\mathbb{N}}
\newcommand{\ZZ}{\mathbb{Z}}
\newcommand{\QQ}{\mathbb{Q}}
\newcommand{\RR}{\mathbb{R}}
\newcommand{\CC}{\mathbb{C}}
\newcommand{\PP}{\mathbb P}
\newcommand{\FF}{\mathbb F}
\newcommand{\DD}{\mathbb D}
\renewcommand{\epsilon}{\varepsilon}

\newcommand{\Aut}{\operatorname{Aut}}
\newcommand{\coker}{\operatorname{coker}}
\newcommand{\CVect}{\CC\operatorname{-Vect}}
\newcommand{\Cantor}{\mathcal{C}}
\newcommand{\D}{\mathcal{D}}
\newcommand{\card}{\operatorname{card}}
\newcommand{\dbar}{\overline \partial}
\DeclareMathOperator*{\esssup}{ess\,sup}
\newcommand{\GL}{\operatorname{GL}}
\newcommand{\Hom}{\operatorname{Hom}}
\newcommand{\id}{\operatorname{id}}
\newcommand{\Ind}{\operatorname{Ind}}
\newcommand{\Inn}{\operatorname{Inn}}
\newcommand{\interior}{\operatorname{int}}
\newcommand{\lcm}{\operatorname{lcm}}
\newcommand{\mesh}{\operatorname{mesh}}
\newcommand{\LL}{\mathcal L_0}
\newcommand{\Leb}{\mathcal{L}_{\text{loc}}^2}
\newcommand{\Lip}{\operatorname{Lip}}
\newcommand{\ppGL}{\operatorname{PGL}}
\newcommand{\ppic}{\vspace{35mm}}
\newcommand{\ppset}{\mathcal{P}}
\DeclareMathOperator{\proj}{proj}
\DeclareMathOperator*{\Res}{Res}
\newcommand{\Riem}{\mathcal{R}}
\newcommand{\RVect}{\RR\operatorname{-Vect}}
\newcommand{\Sch}{\mathcal{S}}
\newcommand{\SL}{\operatorname{SL}}
\newcommand{\sgn}{\operatorname{sgn}}
\newcommand{\spn}{\operatorname{span}}
\newcommand{\Spec}{\operatorname{Spec}}
\newcommand{\supp}{\operatorname{supp}}
\newcommand{\Torus}{\mathbb T}
\DeclareMathOperator{\tr}{tr}

\DeclareMathOperator{\adj}{adj}
\DeclareMathOperator{\curl}{curl}

% Calculus of variations
\DeclareMathOperator{\pp}{\mathbf p}
\DeclareMathOperator{\zz}{\mathbf z}
\DeclareMathOperator{\uu}{\mathbf u}
\DeclareMathOperator{\vv}{\mathbf v}
\DeclareMathOperator{\ww}{\mathbf w}

\DeclareMathOperator{\Olo}{\mathscr O}

% Categories
\newcommand{\Ab}{\mathbf{Ab}}
\newcommand{\Cat}{\mathbf{Cat}}
\newcommand{\Group}{\mathbf{Group}}
\newcommand{\Module}{\mathbf{Module}}
\newcommand{\Set}{\mathbf{Set}}
\DeclareMathOperator{\Fun}{Fun}
\DeclareMathOperator{\Iso}{Iso}

% Complex analysis
\renewcommand{\Re}{\operatorname{Re}}
\renewcommand{\Im}{\operatorname{Im}}

% Logic
\renewcommand{\iff}{\leftrightarrow}
\newcommand{\Henkin}{\operatorname{Henk}}
\newcommand{\PA}{\mathbf{PA}}
\DeclareMathOperator{\proves}{\vdash}

% Group
\DeclareMathOperator{\Gal}{Gal}
\DeclareMathOperator{\Fix}{Fix}
\DeclareMathOperator{\Out}{Out}

% Other symbols
\newcommand{\heart}{\ensuremath\heartsuit}

\DeclareMathOperator{\atanh}{atanh}

% Theorems
\theoremstyle{definition}
\newtheorem*{corollary}{Corollary}
\newtheorem*{falselemma}{Grader's ``Lemma"}
\newtheorem{exer}{Exercise}
\newtheorem{lemma}{Lemma}[exer]
\newtheorem{theorem}[lemma]{Theorem}

\usepackage[backend=bibtex,style=alphabetic,maxcitenames=50,maxnames=50]{biblatex}
\renewbibmacro{in:}{}
\DeclareFieldFormat{pages}{#1}

\begin{document}
\noindent
\large\textbf{Finite element methods, HW 4} \hfill \textbf{Aidan Backus} \\
% --------------------------------------------------------------
%                         Start here
% --------------------------------------------------------------\

\begin{exer}
Let $P_3(I)$ denote the space of cubic polynomials on $I = [a, b]$.
Show that $v \in P_3(I)$ is determined by $(v(a), v'(a), v(b), v'(b))$ and find the corresponding basis functions.
Construct a finite-dimensional subspace $W_h$ of $W = H^2_0([0, 1])$ of piecewise cubic functions and determine the natural degrees of freedom and basis functions.
Determine the corresponding linear system for a uniform partition $h$, and determine the corresponding numerical solution of the beam equation $u^{(4)} = f$ with $f$ constant where the partition only has two intervals.
Compare it to the exact solution.
\end{exer}

Recall that $P_3(I)$ is $4$-dimensional, so to show that a function in $P_3(I)$ is determined by
$$X(v) = (v(a), v(b), v'(a), v'(b)),$$
it suffices to find functions $f_0, f_1, g_0, g_1 \in P_3([0, 1])$ which are linearly independent and such that $f_i(i) = 1$ and $f_i$ annihilates all other degrees of freedom, and $g_i'(i) = 1$ and $g_i$ annihilates all other degrees of freedom. We easily compute
$$f_0(x) = 1 - 3x^2 + 2x^3$$
and hence
$$f_1(x) = 1 - 3(1 - x)^2 + 2(1 - x)^3.$$
Similarly we compute
$$g_0(x) = x - 2x^2 + x^3$$
and hence
$$g_1(x) = (1 - x) - 2(1 - x)^2 + (1 - x)^3.$$
This gives a basis of $P_3([0, 1])$ which corresponds to the desired degrees of freedom. The general case follows by scaling, namely for $[a, b]$ we get
\begin{align*}
f_{I, -}(x) &= 1 - 3(x - a)^2 + 2(x - a)^3,\\
f_{I, +}(x) &= 1 - 3(b - x)^2 + 2(b - x)^3,\\
g_{I, -}(x) &= \frac{(x - a) - 2(x - a)^2 + (x - a)^3}{b - a}, \\
g_{I, +}(x) &= \frac{(b - x) - 2(b - x)^2 + (b - x)^3}{b - a}.
\end{align*}
Given a partition $\mathcal P$ of $[0, 1]$ at points $x_0, \dots, x_{n + 1}$ where $|x_i - x_{i + 1}| \leq h$, we define $W_h$ to be the space of piecewise cubic functions in $W$ which are cubic on each interval $(x_i, x_{i + 1})$ and $C^1$ everywhere.
A basis is given by functions $f_i$, which is supported on $[x_{i - 1}, x_{i + 1}]$, with $f_i = f_{[x_{i - 1}, x_i], +}$ on $[x_{i - 1}, x_i]$ and $f_i = f_{[x_i, x_{i + 1}], -}$ on $[x_i, x_{i + 1}]$, which are $1$ at $x_i$,
and functions $g_i$ which are supported on $[x_{i - 1}, x_{i + 1}]$, with $g_i = g_{[x_i, x_{i + 1}], -}$ on $[x_i, x_{i + 1}]$ and $g_i = - g_{[x_{i - 1}, x_i], +}$ on $[x_{i - 1}, x_i]$, which have derivative $1$ at $x_i$.
These compatibility conditions ensure that $f_i,g_i$ are $C^1$.

The finite-element scheme here is to solve the equations
\begin{align*}
\int_{x_{i - 1}}^{x_{i + 1}} u^{(2)}(x) f_i^{(2)}(x) ~dx &= \int_{x_{i - 1}}^{x_{i + 1}} f(x) f_i(x) ~dx,\\
\int_{x_{i - 1}}^{x_{i + 1}} u^{(2)}(x) g_i^{(2)}(x) ~dx &= \int_{x_{i - 1}}^{x_{i + 1}} f(x) g_i(x) ~dx.
\end{align*}
I don't have time to write out all the details here, and in particular to compute the stiffness operator, so let me just observe that each $f_i$ intersects its supports with only $f_{i - 1}, f_{i + 1}, g_{i - 1}, g_i, g_{i + 1}$.
Similarly for $g_i$.
Thus the matrix that we get will be highly sparse.

The exact solution to the homogeneous Dirichlet problem for the beam equation with $u^{(4)} = f$, $f = A$ constant, is
$$u(x) = \frac{A}{24}x^4 - \frac{A}{12}x^3 + \frac{A}{24}x^2.$$
Suppose that we have a uniform mesh with just two intervals. Then our only degrees of freedom are the function at $1/2$ and its derivative at $1/2$, namely we can write
$$u = u(1/2)v + u'(1/2)w$$
where $v,w$ are both piecewise cubic,
$$v(0) = v(1) = v'(0) = v'(1) = v'(1/2) = 0,$$
$v(1/2) = 1$, $w'(1/2) = 1$, and
$$w(0) = w(1) = w'(0) = w'(1) = w(1/2) = 0.$$




\begin{exer}
Let $u: [0, 1] \to \RR$ be a smooth function and let $p$ be its cubic interpolant on $[0, 1]$.
Define
$$K(x, t) = \frac{1}{6} \begin{cases} x^2 (1 - t)^2 (3t - x - 2xt), & 0 \leq x \leq t \leq 1 \\
t^2(1 - x)^2(3x - t - 2xt), & 0 \leq t < x \leq 1.
\end{cases}$$
Show using integration by parts that
$$u(x) - p(x) = \int_0^1 K(x, t) u^{(4)}(t) ~dt.$$
Conclude that for every $k \leq 3$, there exists $C_k > 0$ such that
$$||u^{(k)} - p^{(k)}||_{L^2([0, 1])} \leq C_k ||u^{(4)}||_{L^2([0, 1])}.$$
Show that for every $a < b$, if $p$ is instead chosen as the cubic interpolant of $u$ on $[a, b]$.
$$||u^{(k)} - p^{(k)}||_{L^2([a, b])} \leq C_k (b - a)^{4 - k} ||u^{(4)}||.$$
Finally, obtain an error estimate for the above finite-element scheme for the beam equation.
\end{exer}

To show that $K$ is the Peano kernel for cubic interpolation we observe that
$$p(x) = u(0) + u'(0)x + (3u(1) - 3u(0) - 2u'(0) - u'(1))x^2 + (2u(0) - 2u(1) + u'(0) + u'(1))x^3.$$
Indeed, that $p(0) = u(0)$ and $p'(0) = u'(0)$ is clear, and we have
$$p(1) = u(0) + u'(0) + 3u(1) - 3u(0) - 2u'(0) - u'(1) + 2u(0) - 2u(1) + u'(0) + u'(1) = u(1)$$
and
$$p'(1) = u'(0) + 6u(1) - 6u(0) - 4u'(0) - 2u'(1) + 6u(0) - 6u(1) + 3u'(0) + 3u'(1) = u'(1).$$
We now integrate by parts twice:
\begin{align*}\int_0^1 K(x, t) u^{(4)}(t) ~dt &= \int_0^1 \partial_t^2 K(x, t) u^{(2)}(t) ~dt \\
&\qquad+ K(x, 1)u^{(3)}(1) + \partial_t K(x, 0)u^{(2)}(0) - K(x, 0)u^{(3)}(0) - \partial_tK(x, 1)u^{(2)}(1).
\end{align*}
This formula looks slightly ridiculous because of the large number of boundary terms that have accumulated. To fix that we compute
$$K(x, 0) = K(x, 1) = \partial_t K(x, 0) = \partial_t K(x, 1) = 0,$$
thus
\begin{align*}\int_0^1 K(x, t) u^{(4)}(t) ~dt &= \int_0^1 \partial_t^2 K(x, t) u(t) ~dt.
\end{align*}
To further simplify we compute
$$\partial_t^2 K(x, t) = \begin{cases}
(-2x^3 + 3x^2)t + (x^3 - 2x^2), & x \leq t, \\
(-2x^3 + 3x^2 - 1)t + (x^3 - 2x^2 + 1), & x \geq t.
\end{cases}$$
This function has a jump discontinuity at $t = x$ of size $x - 1$.
I think that we need to use the Dirac delta part of $\partial_t^3 K$ to get the desired result but unfortunately I don't have time.

Therefore
$$u^{(k)}(x) - p^{(k)}(x) = \int_0^1 \partial_x^k K(x, t) u^{(4)}(t) ~dt$$
which gives
$$\int_0^1 |u^{(k)}(x) - p^{(k)}(x)|^2 ~dx = \int_0^1 \left|\int_0^1 \partial_x^k K(x, t) u^{(4)}(t) ~dt\right|^2 ~dx$$
and from the triangle inequality the right-hand side is bounded by
$$\leq ||\partial_x^k K||_{L^\infty}^2 \cdot ||u^{(4)}||_{L^2}^2$$
and taking square roots of both sides we obtain
$$||u^{(k)} - p^{(k)}||_{L^2([0, 1])} \leq ||\partial_x^k K||_{L^\infty([0, 1])} ||u^{(4)}||_{L^2([0, 1])}.$$
Rescaling $[0, 1]$ to have length $b - a$ does not affect the $L^\infty$ norm but but affects the $L^2$ norm of the $k$th derivative by multiplication by $(b - a)^{-k}$.
This implies that
$$(b - a)^{-4} ||u^{(k)} - p^{(k)}||_{L^2([0, 1])} \leq (b - a)^{-k} ||\partial_x^k K||_{L^\infty([0, 1])} ||u^{(4)}||_{L^2([0, 1])}$$
which is the desired estimate.

To apply this to the beam equation we apply C\'ea's lemma to obtain a bound in energy norm
$$||u - u_h||_a = \inf_{v \in W_h} ||u - v||_a \leq ||u - p||_a = ||Tu||_a.$$
But the energy norm is $||v||_a = ||v^{(2)}||_{L^2}$, so
$$||Tu||_a \leq ||(Tu)^{(2)}||_{L^2} \lesssim \sum_{i=1}^{n} (b - a)^2 ||u^{(4)}||_{L^2([0, 1])} \lesssim h||u^{(4)}||_{L^2([0, 1])}.$$
Putting it all together we obtain
$$||u^{(2)} - u_h^{(2)}||_{L^2} \lesssim h||u^{(4)}||_{L^2}.$$
Actually we can do a little better than that, as the Poincar\'e inequality gives us bounds on $u - u_h$ and $u' - u_h'$ as well from this bound:
$$||u - u_h||_{H^2} \lesssim h ||u^{(4)}||_{L^2}.$$

\end{document}
