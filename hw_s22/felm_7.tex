
% --------------------------------------------------------------
% This is all preamble stuff that you don't have to worry about.
% Head down to where it says "Start here"
% --------------------------------------------------------------

\documentclass[10pt]{article}

\usepackage[margin=.7in]{geometry}
\usepackage{amsmath,amsthm,amssymb,mathrsfs}
\usepackage{enumitem}
\usepackage{tikz-cd}
\usepackage{mathtools}
\usepackage{amsfonts}
\usepackage{listings}
\usepackage{algorithm2e}
\usepackage{verse,stmaryrd}
\usepackage{fancyvrb}

% Number systems
\newcommand{\NN}{\mathbb{N}}
\newcommand{\ZZ}{\mathbb{Z}}
\newcommand{\QQ}{\mathbb{Q}}
\newcommand{\RR}{\mathbb{R}}
\newcommand{\CC}{\mathbb{C}}
\newcommand{\PP}{\mathbb P}
\newcommand{\FF}{\mathbb F}
\newcommand{\DD}{\mathbb D}
\renewcommand{\epsilon}{\varepsilon}

\newcommand{\Aut}{\operatorname{Aut}}
\newcommand{\coker}{\operatorname{coker}}
\newcommand{\CVect}{\CC\operatorname{-Vect}}
\newcommand{\Cantor}{\mathcal{C}}
\newcommand{\D}{\mathcal{D}}
\newcommand{\card}{\operatorname{card}}
\newcommand{\dbar}{\overline \partial}
\DeclareMathOperator*{\esssup}{ess\,sup}
\newcommand{\GL}{\operatorname{GL}}
\newcommand{\Hom}{\operatorname{Hom}}
\newcommand{\id}{\operatorname{id}}
\newcommand{\Ind}{\operatorname{Ind}}
\newcommand{\Inn}{\operatorname{Inn}}
\newcommand{\interior}{\operatorname{int}}
\newcommand{\lcm}{\operatorname{lcm}}
\newcommand{\mesh}{\operatorname{mesh}}
\newcommand{\LL}{\mathcal L_0}
\newcommand{\Leb}{\mathcal{L}_{\text{loc}}^2}
\newcommand{\Lip}{\operatorname{Lip}}
\newcommand{\ppGL}{\operatorname{PGL}}
\newcommand{\ppic}{\vspace{35mm}}
\newcommand{\ppset}{\mathcal{P}}
\DeclareMathOperator{\proj}{proj}
\DeclareMathOperator*{\Res}{Res}
\newcommand{\Riem}{\mathcal{R}}
\newcommand{\RVect}{\RR\operatorname{-Vect}}
\newcommand{\Sch}{\mathcal{S}}
\newcommand{\SL}{\operatorname{SL}}
\newcommand{\sgn}{\operatorname{sgn}}
\newcommand{\spn}{\operatorname{span}}
\newcommand{\Spec}{\operatorname{Spec}}
\newcommand{\supp}{\operatorname{supp}}
\newcommand{\Torus}{\mathbb T}
\DeclareMathOperator{\tr}{tr}

\DeclareMathOperator{\adj}{adj}
\DeclareMathOperator{\curl}{curl}

% Calculus of variations
\DeclareMathOperator{\pp}{\mathbf p}
\DeclareMathOperator{\zz}{\mathbf z}
\DeclareMathOperator{\uu}{\mathbf u}
\DeclareMathOperator{\vv}{\mathbf v}
\DeclareMathOperator{\ww}{\mathbf w}

\DeclareMathOperator{\Olo}{\mathscr O}

% Categories
\newcommand{\Ab}{\mathbf{Ab}}
\newcommand{\Cat}{\mathbf{Cat}}
\newcommand{\Group}{\mathbf{Group}}
\newcommand{\Module}{\mathbf{Module}}
\newcommand{\Set}{\mathbf{Set}}
\DeclareMathOperator{\Fun}{Fun}
\DeclareMathOperator{\Iso}{Iso}

% Complex analysis
\renewcommand{\Re}{\operatorname{Re}}
\renewcommand{\Im}{\operatorname{Im}}

% Logic
\renewcommand{\iff}{\leftrightarrow}
\newcommand{\Henkin}{\operatorname{Henk}}
\newcommand{\PA}{\mathbf{PA}}
\DeclareMathOperator{\proves}{\vdash}

% Group
\DeclareMathOperator{\Gal}{Gal}
\DeclareMathOperator{\Fix}{Fix}
\DeclareMathOperator{\Out}{Out}

% Other symbols
\newcommand{\heart}{\ensuremath\heartsuit}

\DeclareMathOperator{\atanh}{atanh}

% Theorems
\theoremstyle{definition}
\newtheorem*{corollary}{Corollary}
\newtheorem*{falselemma}{Grader's ``Lemma"}
\newtheorem{exer}{Exercise}
\newtheorem{lemma}{Lemma}[exer]
\newtheorem{theorem}[lemma]{Theorem}

\usepackage[backend=bibtex,style=alphabetic,maxcitenames=50,maxnames=50]{biblatex}
\renewbibmacro{in:}{}
\DeclareFieldFormat{pages}{#1}

\begin{document}
\noindent
\large\textbf{Finite element methods, HW 7} \hfill \textbf{Aidan Backus} \\
% --------------------------------------------------------------
%                         Start here
% --------------------------------------------------------------\

\begin{exer}
Show that $u(r, \theta) = \log \log(1/r)$ satisfies $u \in H^1(B) \setminus C(B)$.
\end{exer}

I don't think this is actually true, since the square of the derivative is 
$$|\nabla u(r, \theta)|^2 = \frac{1}{r^2 \log^2(1/r)}$$
which is not integrable with respect to the measure $r~dr$.
On the other hand, Evans says that the actual counterexample is $u(r, \theta) = \log \log(1 + 1/r)$, and this actually works.
It clearly is not continuous since
$$\log \log(1 + 1/0) = \log \log \infty = \infty.$$
The square derivative is 
$$|\nabla u(r, \theta)|^2 = \frac{1}{r^2(r + 1)^2 \log^2(1 + 1/r)}$$
and we have 
$$||u||_{H^1}^2 = \int_0^1 \left(\log^2 \log(1 + 1/r) + \frac{1}{r^2(r + 1)^2 \log^2(1 + 1/r)}\right) r~dr.$$
Setting $s = r^{-1}$, we obtain 
$$r~dr = -s^{-3}~ds$$
and hence 
$$||u||_{H^1}^2 = \int_1^\infty \left(\log^2 \log(1 + s) + \frac{s^2}{(1 + s^{-1})^2 \log^2(1 + s)}\right) s^{-3}~ds.$$
Clearly $s^{-3} \log^2 \log(1 + s)$ is integrable with respect to $ds$. For the other term, we observe that it is 
$$\leq \int_1^\infty \frac{2}{s \log^2(1 + s)}~ds$$
which is comparable to 
$$\int_2^\infty \frac{1}{s \log^2 s}~ds = -\left[\frac{1}{\log s}\right]_{s = 2}^\infty = -\frac{1}{\log 2}.$$

\begin{exer}
Let $u$ solve
$$-\varepsilon \Delta u + b \cdot \nabla u + cu = f$$
where $\varepsilon > 0$ is constant, $b,c \in L^\infty$, and $c \geq \nabla \cdot b/2$.
Show that a solution exists in $H^1_0$ with an a priori estimate which is independent of $b$.
\end{exer}

We introduce the stiffness form
$$a(u, v) = \int_\Omega \varepsilon \nabla u \cdot \nabla v + vb\cdot \nabla u + cuv.$$
(It would be more natural to define a fractional differential operator $\nabla^{1/2}$ to get a symmetric bilinear form -- and then we could even apply the Riesz representation theorem rather than the Lax-Milgram theorem -- but I'm lazy.)
A bunch of applications of the Cauchy-Schwarz inequality give
\begin{align*}
|a(u, v)| &\leq \varepsilon ||\nabla u|| \cdot ||\nabla v|| + ||b||_{L^\infty} \cdot ||\nabla u|| \cdot ||v|| + ||c||_{L^\infty} \cdot ||u|| \cdot ||v||\\
&\leq (\varepsilon + ||b||_{L^\infty} + ||c||_{L^\infty}) \cdot ||u||_{H^1} \cdot ||v||_{H^1}
\end{align*}
so $a$ is continuous with norm $\varepsilon + ||b||_{L^\infty} + ||c||_{L^\infty}$.
As usual, we also have a load form $Lv = \int_\Omega fv$, which is continuous with norm $||f||_{L^2}$.

To show ellipticity of the stiffness form we integrate by parts to compute
$$\int_\Omega ub \cdot \nabla u = -\int_\Omega ub \cdot \nabla u + u^2 \nabla \cdot b$$
which implies that
$$\int_\Omega ub \cdot \nabla u = -\frac{1}{2} \int_\Omega u^2 \nabla \cdot b.$$
It follows that
$$a(u, u) = \int_\Omega \varepsilon |\nabla u|^2 + (c - \nabla \cdot b/2)|u|^2,$$
which by hypothesis is
$$\geq \varepsilon ||\nabla u||^2 \gtrsim_\Omega ||u||_{H^1}^2$$
by the Poincar\'e inequality. In particular, by the Lax-Milgram theorem, we can, for any $v \in H^1_0$, solve the variational equation
$$a(u, v) = Lv$$
for $u \in H^1_0$ which satisfies the estimate
$$||u||_{H^1} \lesssim \frac{||f||}{\varepsilon}.$$

\end{document}
