
% --------------------------------------------------------------
% This is all preamble stuff that you don't have to worry about.
% Head down to where it says "Start here"
% --------------------------------------------------------------

\documentclass[10pt]{article}

\usepackage[margin=.7in]{geometry}
\usepackage{amsmath,amsthm,amssymb,mathrsfs}
\usepackage{enumitem}
\usepackage{tikz-cd}
\usepackage{mathtools}
\usepackage{amsfonts}
\usepackage{listings}
\usepackage{algorithm2e}
\usepackage{verse,stmaryrd}
\usepackage{fancyvrb}

% Number systems
\newcommand{\NN}{\mathbb{N}}
\newcommand{\ZZ}{\mathbb{Z}}
\newcommand{\QQ}{\mathbb{Q}}
\newcommand{\RR}{\mathbb{R}}
\newcommand{\CC}{\mathbb{C}}
\newcommand{\PP}{\mathbb P}
\newcommand{\FF}{\mathbb F}
\newcommand{\DD}{\mathbb D}
\renewcommand{\epsilon}{\varepsilon}

\newcommand{\Aut}{\operatorname{Aut}}
\newcommand{\coker}{\operatorname{coker}}
\newcommand{\CVect}{\CC\operatorname{-Vect}}
\newcommand{\Cantor}{\mathcal{C}}
\newcommand{\D}{\mathcal{D}}
\newcommand{\card}{\operatorname{card}}
\newcommand{\dbar}{\overline \partial}
\DeclareMathOperator*{\esssup}{ess\,sup}
\newcommand{\GL}{\operatorname{GL}}
\newcommand{\Hom}{\operatorname{Hom}}
\newcommand{\id}{\operatorname{id}}
\newcommand{\Ind}{\operatorname{Ind}}
\newcommand{\Inn}{\operatorname{Inn}}
\newcommand{\interior}{\operatorname{int}}
\newcommand{\lcm}{\operatorname{lcm}}
\newcommand{\mesh}{\operatorname{mesh}}
\newcommand{\LL}{\mathcal L_0}
\newcommand{\Leb}{\mathcal{L}_{\text{loc}}^2}
\newcommand{\Lip}{\operatorname{Lip}}
\newcommand{\ppGL}{\operatorname{PGL}}
\newcommand{\ppic}{\vspace{35mm}}
\newcommand{\ppset}{\mathcal{P}}
\DeclareMathOperator{\proj}{proj}
\DeclareMathOperator*{\Res}{Res}
\newcommand{\Riem}{\mathcal{R}}
\newcommand{\RVect}{\RR\operatorname{-Vect}}
\newcommand{\Sch}{\mathcal{S}}
\newcommand{\SL}{\operatorname{SL}}
\newcommand{\sgn}{\operatorname{sgn}}
\newcommand{\spn}{\operatorname{span}}
\newcommand{\Spec}{\operatorname{Spec}}
\newcommand{\supp}{\operatorname{supp}}
\newcommand{\Torus}{\mathbb T}
\DeclareMathOperator{\tr}{tr}

\DeclareMathOperator{\adj}{adj}
\DeclareMathOperator{\curl}{curl}

% Calculus of variations
\DeclareMathOperator{\pp}{\mathbf p}
\DeclareMathOperator{\zz}{\mathbf z}
\DeclareMathOperator{\uu}{\mathbf u}
\DeclareMathOperator{\vv}{\mathbf v}
\DeclareMathOperator{\ww}{\mathbf w}

\DeclareMathOperator{\Olo}{\mathscr O}

% Categories
\newcommand{\Ab}{\mathbf{Ab}}
\newcommand{\Cat}{\mathbf{Cat}}
\newcommand{\Group}{\mathbf{Group}}
\newcommand{\Module}{\mathbf{Module}}
\newcommand{\Set}{\mathbf{Set}}
\DeclareMathOperator{\Fun}{Fun}
\DeclareMathOperator{\Iso}{Iso}

% Complex analysis
\renewcommand{\Re}{\operatorname{Re}}
\renewcommand{\Im}{\operatorname{Im}}

% Logic
\renewcommand{\iff}{\leftrightarrow}
\newcommand{\Henkin}{\operatorname{Henk}}
\newcommand{\PA}{\mathbf{PA}}
\DeclareMathOperator{\proves}{\vdash}

% Group
\DeclareMathOperator{\Gal}{Gal}
\DeclareMathOperator{\Fix}{Fix}
\DeclareMathOperator{\Out}{Out}

% Other symbols
\newcommand{\heart}{\ensuremath\heartsuit}

\DeclareMathOperator{\atanh}{atanh}

% Theorems
\theoremstyle{definition}
\newtheorem*{corollary}{Corollary}
\newtheorem*{falselemma}{Grader's ``Lemma"}
\newtheorem{exer}{Exercise}
\newtheorem{lemma}{Lemma}[exer]
\newtheorem{theorem}[lemma]{Theorem}

\usepackage[backend=bibtex,style=alphabetic,maxcitenames=50,maxnames=50]{biblatex}
\renewbibmacro{in:}{}
\DeclareFieldFormat{pages}{#1}

\begin{document}
\noindent
\large\textbf{Finite element methods, HW 2} \hfill \textbf{Aidan Backus} \\
% --------------------------------------------------------------
%                         Start here
% --------------------------------------------------------------\

\begin{exer}
    Show that if $v$ is continuous on $[0, 1]$ and 
    $$\int_0^1 v(x) w(x) ~dx = 0$$
    for every piecewise $C^1$ function $w \in H^1_0([0, 1])$, then $v = 0$ on $[0, 1]$.
\end{exer}

By replacing $v$ with $|v|$ it suffices to assume $v \geq 0$.
If $v(x^*) > 0$ for some $x^* \in [0, 1]$, then by continuity, we may assume that $x^* \in (0, 1)$.
Then there exist $\varepsilon, \delta > 0$ such that if $|x - x^*| < \delta$ then $v(x) > \varepsilon$, and $\delta < d(x^*, \{0, 1\})$.
So if we let $w$ be a piecewise linear, nonnegative continuous function which is supported on $[x^* - \delta, x^* + \delta]$ such that $\int_0^1 w = 1$, then 
$$\int_0^1 v(x) w(x) ~dx \geq \varepsilon \int_{x^* - \delta}^{x^* + \delta} w(x) ~dx = \varepsilon > 0.$$

\begin{exer}
    Suppose that $-\Delta u = f$, $u|\partial \Omega = u_0$. Show that for every $v \in \dot H^1_0(\Omega)$,
    $$\int_\Omega \nabla u \cdot \nabla v = \int_\Omega fv$$
    and for every $v$ with $v|\partial \Omega = u_0$,
    $$J(u) \leq J(v),$$
    where $J$ is Dirichlet energy.
\end{exer}

To derive the variational formulation, we integrate by parts:
$$\int_\Omega \nabla u \cdot \nabla v = \int_{\partial \Omega} v\partial_n u - \int_\Omega v\Delta u = \int_\Omega fv.$$
Here $\partial_n u$ is the normal derivative of $u$, and $v\partial_n u = 0$ on $\partial \Omega$ since $v \in \dot H^1_0(\Omega)$.

To derive the minimization formulation, we decompose $v = u + w$ and then estimate 
\begin{align*}
J(v) &= \int_\Omega \frac{|\nabla v|^2}{2} - fv\\
&= \int_\Omega \frac{|\nabla u|^2}{2} - fu + \frac{|\nabla w|^2}{2} + \nabla u \cdot \nabla w - fw\\
&= J(u) + \frac{||w||_{\dot H^1}^2}{2} \geq J(u).
\end{align*}
So $u$ is a minimizer.

\end{document}
