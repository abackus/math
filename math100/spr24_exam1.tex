
% --------------------------------------------------------------
% This is all preamble stuff that you don't have to worry about.
% Head down to where it says "Start here"
% --------------------------------------------------------------

\documentclass[10pt]{article}

\usepackage[margin=.7in]{geometry}
\usepackage{amsmath,amsthm,amssymb,mathrsfs}
\usepackage{enumitem}
\usepackage{tikz-cd}
\usepackage{mathtools}
\usepackage{amsfonts}
\usepackage{listings}
\usepackage{algorithm2e}
\usepackage{verse,stmaryrd}
\usepackage{fancyvrb}

% Number systems
\newcommand{\NN}{\mathbb{N}}
\newcommand{\ZZ}{\mathbb{Z}}
\newcommand{\QQ}{\mathbb{Q}}
\newcommand{\RR}{\mathbb{R}}
\newcommand{\CC}{\mathbb{C}}
\newcommand{\PP}{\mathbb P}
\newcommand{\FF}{\mathbb F}
\newcommand{\DD}{\mathbb D}
\renewcommand{\epsilon}{\varepsilon}

\newcommand{\Aut}{\operatorname{Aut}}
\newcommand{\coker}{\operatorname{coker}}
\newcommand{\CVect}{\CC\operatorname{-Vect}}
\newcommand{\Cantor}{\mathcal{C}}
\newcommand{\D}{\mathcal{D}}
\newcommand{\card}{\operatorname{card}}
\newcommand{\dbar}{\overline \partial}
\DeclareMathOperator*{\esssup}{ess\,sup}
\newcommand{\GL}{\operatorname{GL}}
\newcommand{\Hom}{\operatorname{Hom}}
\newcommand{\id}{\operatorname{id}}
\newcommand{\Ind}{\operatorname{Ind}}
\newcommand{\Inn}{\operatorname{Inn}}
\newcommand{\interior}{\operatorname{int}}
\newcommand{\lcm}{\operatorname{lcm}}
\newcommand{\mesh}{\operatorname{mesh}}
\newcommand{\LL}{\mathcal L_0}
\newcommand{\Leb}{\mathcal{L}_{\text{loc}}^2}
\newcommand{\Lip}{\operatorname{Lip}}
\newcommand{\ppGL}{\operatorname{PGL}}
\newcommand{\ppic}{\vspace{35mm}}
\newcommand{\ppset}{\mathcal{P}}
\DeclareMathOperator{\proj}{proj}
\DeclareMathOperator*{\Res}{Res}
\newcommand{\Riem}{\mathcal{R}}
\newcommand{\RVect}{\RR\operatorname{-Vect}}
\newcommand{\Sch}{\mathcal{S}}
\newcommand{\SL}{\operatorname{SL}}
\newcommand{\sgn}{\operatorname{sgn}}
\newcommand{\spn}{\operatorname{span}}
\newcommand{\Spec}{\operatorname{Spec}}
\newcommand{\supp}{\operatorname{supp}}
\newcommand{\Torus}{\mathbb T}
\DeclareMathOperator{\tr}{tr}

\DeclareMathOperator{\adj}{adj}
\DeclareMathOperator{\curl}{curl}

% Calculus of variations
\DeclareMathOperator{\pp}{\mathbf p}
\DeclareMathOperator{\zz}{\mathbf z}
\DeclareMathOperator{\uu}{\mathbf u}
\DeclareMathOperator{\vv}{\mathbf v}
\DeclareMathOperator{\ww}{\mathbf w}

\DeclareMathOperator{\Olo}{\mathscr O}

% Categories
\newcommand{\Ab}{\mathbf{Ab}}
\newcommand{\Cat}{\mathbf{Cat}}
\newcommand{\Group}{\mathbf{Group}}
\newcommand{\Module}{\mathbf{Module}}
\newcommand{\Set}{\mathbf{Set}}
\DeclareMathOperator{\Fun}{Fun}
\DeclareMathOperator{\Iso}{Iso}

% Complex analysis
\renewcommand{\Re}{\operatorname{Re}}
\renewcommand{\Im}{\operatorname{Im}}

% Logic
\renewcommand{\iff}{\leftrightarrow}
\newcommand{\Henkin}{\operatorname{Henk}}
\newcommand{\PA}{\mathbf{PA}}
\DeclareMathOperator{\proves}{\vdash}

% Group
\DeclareMathOperator{\Gal}{Gal}
\DeclareMathOperator{\Fix}{Fix}
\DeclareMathOperator{\Out}{Out}

% Other symbols
\newcommand{\heart}{\ensuremath\heartsuit}

\DeclareMathOperator{\atanh}{atanh}

% Theorems
\theoremstyle{definition}
\newtheorem*{corollary}{Corollary}
\newtheorem*{falselemma}{Grader's ``Lemma"}
\newtheorem{exer}{Exercise}
\newtheorem{lemma}{Lemma}[exer]
\newtheorem{theorem}[lemma]{Theorem}

\usepackage[backend=bibtex,style=alphabetic,maxcitenames=50,maxnames=50]{biblatex}
\renewbibmacro{in:}{}
\DeclareFieldFormat{pages}{#1}

\begin{document}
\noindent
% --------------------------------------------------------------
%                         Start here
% --------------------------------------------------------------\

I was feeling pretty frustrated with research today, so I when I saw your email I decided to take a break and answer the exam feedback.
Here's a prospective question:

\begin{itemize}
\item[(a)] Compute the antiderivative
$$\int e^{-x} \cos x ~dx.$$
\item[(b)] Compute $$\int_0^\infty e^{-x} \cos x ~dx,$$ or explain why the improper integral diverges. If you could not solve (a), you can assume that $e^{-x} \sin x \cos x + C$ is the antiderivative of $e^{-x} \cos x$ (even though this is not the actual antiderivative).
\item[(c)] Compute $$\int_0^\infty e^x (2 + \cos x) ~dx,$$ or explain why the improper integral diverges.
\end{itemize}

The idea for (a) is to test integration by parts -- especially the ``moving the integral to the other side'' trick.
We can give close to full credit on (a) if they recognize that's what you need to do, even if they made an algebra error.

I included the error-correcting alternative for (b) out of fairness, since (b) tests a fundamentally different skill than (a). 
Since we haven't covered improper integration yet, I can't tell if students will catch on that you don't need to do l'H\^opital bashing here but can just observe the rapid decay of $e^{-x}$.
There were some students who caught onto that idea when I was reviewing limits a few weeks ago.

I actually considered replacing (a) and (b) with $\int_0^1 \ln x ~dx$, which has the same idea.
But then I realized that probably there are students who went to the trouble of memorizing the antiderivative of $\ln x$ and wouldn't do the integration by parts, and probably ``how many obscure integrals can you memorize'' isn't something we want to test.

For (c), you could do it using integration by parts and l'H\^opital, so students who didn't study direct comparison can still do the problem.
But it rewards students who *did* study, because they will see that
$$e^x (2 + \cos x) \geq e^x$$
which is easier to integrate.
I know that you asked to not have questions of the form ``show that this is true'' but it's kind of hard to have test questions about divergence without some use of the English language.
Here they don't need to say much: ``Use direct comparison with $e^x$'' would basically be enough for full credit.

I'm not actually sure about the skill level of about half of my class, who are pretty hard to get to participate (well, a lot of them don't even come to class...)
I guess they should be weaker than the participating half; if they're a lot weaker, then this problem might be too hard.
My assessment is that the participating half should be able to do the majority of this problem, since they actively followed along and participated in discussion about similar problems regarding integration by parts and limits during the lecture.
But I would be interested in the thoughts of the recitation TAs who might have a better idea of the class skill level.

\end{document}
