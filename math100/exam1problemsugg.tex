
% --------------------------------------------------------------
% This is all preamble stuff that you don't have to worry about.
% Head down to where it says "Start here"
% --------------------------------------------------------------

\documentclass[10pt]{article}

\usepackage[margin=.7in]{geometry}
\usepackage{amsmath,amsthm,amssymb,mathrsfs}
\usepackage{enumitem}
\usepackage{tikz-cd}
\usepackage{mathtools}
\usepackage{amsfonts}
\usepackage{listings}
\usepackage{algorithm2e}
\usepackage{verse,stmaryrd}
\usepackage{fancyvrb}

% Number systems
\newcommand{\NN}{\mathbb{N}}
\newcommand{\ZZ}{\mathbb{Z}}
\newcommand{\QQ}{\mathbb{Q}}
\newcommand{\RR}{\mathbb{R}}
\newcommand{\CC}{\mathbb{C}}
\newcommand{\PP}{\mathbb P}
\newcommand{\FF}{\mathbb F}
\newcommand{\DD}{\mathbb D}
\renewcommand{\epsilon}{\varepsilon}

\newcommand{\Aut}{\operatorname{Aut}}
\newcommand{\coker}{\operatorname{coker}}
\newcommand{\CVect}{\CC\operatorname{-Vect}}
\newcommand{\Cantor}{\mathcal{C}}
\newcommand{\D}{\mathcal{D}}
\newcommand{\card}{\operatorname{card}}
\newcommand{\dbar}{\overline \partial}
\DeclareMathOperator*{\esssup}{ess\,sup}
\newcommand{\GL}{\operatorname{GL}}
\newcommand{\Hom}{\operatorname{Hom}}
\newcommand{\id}{\operatorname{id}}
\newcommand{\Ind}{\operatorname{Ind}}
\newcommand{\Inn}{\operatorname{Inn}}
\newcommand{\interior}{\operatorname{int}}
\newcommand{\lcm}{\operatorname{lcm}}
\newcommand{\mesh}{\operatorname{mesh}}
\newcommand{\LL}{\mathcal L_0}
\newcommand{\Leb}{\mathcal{L}_{\text{loc}}^2}
\newcommand{\Lip}{\operatorname{Lip}}
\newcommand{\ppGL}{\operatorname{PGL}}
\newcommand{\ppic}{\vspace{35mm}}
\newcommand{\ppset}{\mathcal{P}}
\DeclareMathOperator{\proj}{proj}
\DeclareMathOperator*{\Res}{Res}
\newcommand{\Riem}{\mathcal{R}}
\newcommand{\RVect}{\RR\operatorname{-Vect}}
\newcommand{\Sch}{\mathcal{S}}
\newcommand{\SL}{\operatorname{SL}}
\newcommand{\sgn}{\operatorname{sgn}}
\newcommand{\spn}{\operatorname{span}}
\newcommand{\Spec}{\operatorname{Spec}}
\newcommand{\supp}{\operatorname{supp}}
\newcommand{\Torus}{\mathbb T}
\DeclareMathOperator{\tr}{tr}

\DeclareMathOperator{\adj}{adj}
\DeclareMathOperator{\curl}{curl}

% Calculus of variations
\DeclareMathOperator{\pp}{\mathbf p}
\DeclareMathOperator{\zz}{\mathbf z}
\DeclareMathOperator{\uu}{\mathbf u}
\DeclareMathOperator{\vv}{\mathbf v}
\DeclareMathOperator{\ww}{\mathbf w}

\DeclareMathOperator{\Olo}{\mathscr O}

% Categories
\newcommand{\Ab}{\mathbf{Ab}}
\newcommand{\Cat}{\mathbf{Cat}}
\newcommand{\Group}{\mathbf{Group}}
\newcommand{\Module}{\mathbf{Module}}
\newcommand{\Set}{\mathbf{Set}}
\DeclareMathOperator{\Fun}{Fun}
\DeclareMathOperator{\Iso}{Iso}

% Complex analysis
\renewcommand{\Re}{\operatorname{Re}}
\renewcommand{\Im}{\operatorname{Im}}

% Logic
\renewcommand{\iff}{\leftrightarrow}
\newcommand{\Henkin}{\operatorname{Henk}}
\newcommand{\PA}{\mathbf{PA}}
\DeclareMathOperator{\proves}{\vdash}

% Group
\DeclareMathOperator{\Gal}{Gal}
\DeclareMathOperator{\Fix}{Fix}
\DeclareMathOperator{\Out}{Out}

% Other symbols
\newcommand{\heart}{\ensuremath\heartsuit}

\DeclareMathOperator{\atanh}{atanh}

% Theorems
\theoremstyle{definition}
\newtheorem*{corollary}{Corollary}
\newtheorem*{falselemma}{Grader's ``Lemma"}
\newtheorem{exer}{Exercise}
\newtheorem{lemma}{Lemma}[exer]
\newtheorem{theorem}[lemma]{Theorem}

\usepackage[backend=bibtex,style=alphabetic,maxcitenames=50,maxnames=50]{biblatex}
\renewbibmacro{in:}{}
\DeclareFieldFormat{pages}{#1}

\begin{document}
\noindent
% --------------------------------------------------------------
%                         Start here
% --------------------------------------------------------------\

Consider a tightrope stretched between two cliffs, positioned at $x = -1$ and $x = 1$, and suppose that people are walking across the tightrope. The people exert a force $f(x)$ on the tightrope at position $x$, causing the tightrope to be displaced by an amount $u(x)$. Under these conditions, we have
$$\begin{cases}
-u''(x) &= f(x) \\
u(-1) &= 0\\
u(1) &= 0
\end{cases}.$$

A. Explain why every continuous function $g$ with $g(-1) = g(1) = 0$ satisfies
$$\int_{-1}^1 u'(x) g'(x) ~dx = \int_{-1}^1 f(x) g(x) ~dx.$$

B. Suppose that $f(x) = \sec x \tan x$. Determine $u$ or explain why the problem has no solution.

C. Suppose that $f(x) = \sec(\pi x) \tan(\pi x)$. Determine $u$ or explain why the problem has no solution.

\section{Solution of A}
This is because of integration by parts:
$$\int_{-1}^1 u'(x) g'(x) ~dx = u'(1)g(1) - u'(-1)g(-1) -\int_{-1}^1 u''(x) g(x) ~dx = -\int_{-1}^1 u''(x) g(x) ~dx = \int_{-1}^1 f(x) g(x) ~dx.$$

\section{Solution of B}
Since $u'$ is an antiderivative of $f$, it must be $u'(x) = C_1 + \tan x$, and since $u'(-1) = 0$, it follows that $C_1 = -\tan(-1) = \tan(-1) = \pi/4$. Therefore
$$u'(x) = \frac{\pi}{4} + \tan x.$$
So now we need to take the antiderivative of that, namely
$$\int \frac{\pi}{4} + \tan x ~dx = \frac{\pi x}{4} + \int \frac{\sin x}{\cos x} ~dx.$$
For the second term we use $v = \cos x$, $-dv = \sin x ~dx$ to get
$$\int \frac{\sin x}{\cos x} ~dx = -\int \frac{dv}{v} = C_2 -\log |v| = C_2 - \log|\cos x|$$
which means that
$$u(x) = C_2 + \frac{\pi x}{4} - \log|\cos x|.$$
Since $u(1) = 0$ we get
$$C_2 = \log|\cos 1| - \frac{\pi}{4}$$
and therefore
$$u(x) = \log|\cos 1| - \frac{\pi}{4} + \frac{\pi x}{4} - \log|\cos x|.$$

\section{Solution of C}
The integral is improper, with
$$u'(x) = C_1 + \frac{\sec(\pi x)}{\pi}.$$
Taking the limit as $x \to -1$, we see that
$$C_1 = \lim_{x \to -1} \frac{\sec(\pi x)}{\pi} = -\infty.$$
That makes no sense whatsoever so the solution does not exist.

Alternatively you can notice that the solution to this one looks like the solution to part B but with a $\log|\cos(\pi x)|$.
In case $x = 1/2$ this is $\log|\cos(\pi/2)| = \log 0 = -\infty$. Thus the rope extended to be infinitely long.
We definitely don't have an infinite amount of rope so that also makes no sense.


\end{document}
