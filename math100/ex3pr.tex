
% --------------------------------------------------------------
% This is all preamble stuff that you don't have to worry about.
% Head down to where it says "Start here"
% --------------------------------------------------------------

\documentclass[10pt]{article}

\usepackage[margin=.7in]{geometry}
\usepackage{amsmath,amsthm,amssymb,mathrsfs}
\usepackage{enumitem}
\usepackage{tikz-cd}
\usepackage{mathtools}
\usepackage{amsfonts}
\usepackage{listings}
\usepackage{algorithm2e}
\usepackage{verse,stmaryrd}
\usepackage{fancyvrb}

% Number systems
\newcommand{\NN}{\mathbb{N}}
\newcommand{\ZZ}{\mathbb{Z}}
\newcommand{\QQ}{\mathbb{Q}}
\newcommand{\RR}{\mathbb{R}}
\newcommand{\CC}{\mathbb{C}}
\newcommand{\PP}{\mathbb P}
\newcommand{\FF}{\mathbb F}
\newcommand{\DD}{\mathbb D}
\renewcommand{\epsilon}{\varepsilon}

\newcommand{\Aut}{\operatorname{Aut}}
\newcommand{\coker}{\operatorname{coker}}
\newcommand{\CVect}{\CC\operatorname{-Vect}}
\newcommand{\Cantor}{\mathcal{C}}
\newcommand{\D}{\mathcal{D}}
\newcommand{\card}{\operatorname{card}}
\newcommand{\dbar}{\overline \partial}
\DeclareMathOperator*{\esssup}{ess\,sup}
\newcommand{\GL}{\operatorname{GL}}
\newcommand{\Hom}{\operatorname{Hom}}
\newcommand{\id}{\operatorname{id}}
\newcommand{\Ind}{\operatorname{Ind}}
\newcommand{\Inn}{\operatorname{Inn}}
\newcommand{\interior}{\operatorname{int}}
\newcommand{\lcm}{\operatorname{lcm}}
\newcommand{\mesh}{\operatorname{mesh}}
\newcommand{\LL}{\mathcal L_0}
\newcommand{\Leb}{\mathcal{L}_{\text{loc}}^2}
\newcommand{\Lip}{\operatorname{Lip}}
\newcommand{\ppGL}{\operatorname{PGL}}
\newcommand{\ppic}{\vspace{35mm}}
\newcommand{\ppset}{\mathcal{P}}
\DeclareMathOperator{\proj}{proj}
\DeclareMathOperator*{\Res}{Res}
\newcommand{\Riem}{\mathcal{R}}
\newcommand{\RVect}{\RR\operatorname{-Vect}}
\newcommand{\Sch}{\mathcal{S}}
\newcommand{\SL}{\operatorname{SL}}
\newcommand{\sgn}{\operatorname{sgn}}
\newcommand{\spn}{\operatorname{span}}
\newcommand{\Spec}{\operatorname{Spec}}
\newcommand{\supp}{\operatorname{supp}}
\newcommand{\Torus}{\mathbb T}
\DeclareMathOperator{\tr}{tr}

\DeclareMathOperator{\adj}{adj}
\DeclareMathOperator{\curl}{curl}

% Calculus of variations
\DeclareMathOperator{\pp}{\mathbf p}
\DeclareMathOperator{\zz}{\mathbf z}
\DeclareMathOperator{\uu}{\mathbf u}
\DeclareMathOperator{\vv}{\mathbf v}
\DeclareMathOperator{\ww}{\mathbf w}

\DeclareMathOperator{\Olo}{\mathscr O}

% Categories
\newcommand{\Ab}{\mathbf{Ab}}
\newcommand{\Cat}{\mathbf{Cat}}
\newcommand{\Group}{\mathbf{Group}}
\newcommand{\Module}{\mathbf{Module}}
\newcommand{\Set}{\mathbf{Set}}
\DeclareMathOperator{\Fun}{Fun}
\DeclareMathOperator{\Iso}{Iso}

% Complex analysis
\renewcommand{\Re}{\operatorname{Re}}
\renewcommand{\Im}{\operatorname{Im}}

% Logic
\renewcommand{\iff}{\leftrightarrow}
\newcommand{\Henkin}{\operatorname{Henk}}
\newcommand{\PA}{\mathbf{PA}}
\DeclareMathOperator{\proves}{\vdash}

% Group
\DeclareMathOperator{\Gal}{Gal}
\DeclareMathOperator{\Fix}{Fix}
\DeclareMathOperator{\Out}{Out}

% Other symbols
\newcommand{\heart}{\ensuremath\heartsuit}

\DeclareMathOperator{\atanh}{atanh}

% Theorems
\theoremstyle{definition}
\newtheorem*{corollary}{Corollary}
\newtheorem*{falselemma}{Grader's ``Lemma"}
\newtheorem{exer}{Exercise}
\newtheorem{lemma}{Lemma}[exer]
\newtheorem{theorem}[lemma]{Theorem}

\usepackage[backend=bibtex,style=alphabetic,maxcitenames=50,maxnames=50]{biblatex}
\renewbibmacro{in:}{}
\DeclareFieldFormat{pages}{#1}

\begin{document}
\noindent
% --------------------------------------------------------------
%                         Start here
% --------------------------------------------------------------\

\section{Review problem}
In this problem we will compute
$$I = \int_0^{1/4} \frac{\sin x}{x} ~dx.$$
Compute the value of this integral to within $0.0001$ and explain why your solution is so precise.

(I threw in this problem because my experience is that students found problem 6 on the midterm2 to be the hardest, at least that's what they said when talking to me. 
I felt a bit bad that I made a problem for which they were so disappointed in themselves, so I wanted to give them a chance to redeem themselves on the final.
Of course, it helps that in my personal view, numerical integration is the most important topic in Calc 2 :-))

\section{Review solution}
The Taylor series is 
$$\frac{\sin x}{x} = 1 - \frac{x^2}{3!} + \frac{x^4}{5!} - \cdots$$
which is alternating. Now 
$$I = \int_0^{1/4} 1 - \frac{x^2}{3!} + \frac{x^4}{5!} - \cdots ~dx = \frac{1}{4} - \frac{1}{4^3 \cdot 3 \cdot 3!} + \frac{1}{4^5 \cdot 5 \cdot 5!} - \cdots.$$
By the alternating series test, we just need to keep adding terms to $I$ until we hit a term of size $< 0.0001$. Actually, $4^5 \cdot 5 \cdot 5! > 10000$ (actually it's $> 500000$) so this term is good:
$$I \approx \frac{1}{4} - \frac{1}{4^3 \cdot 3 \cdot 3!}.$$

\section{New problem}
Suppose that Alice puts $100,000$ pesos in her retirement fund every year.
We assume that her retirement fund earns $5\%$ continuously compounded interest, so that the value of her retirement fund can be modeled by 
$$\frac{dP}{dt} = 0.05P + 100$$
where $P$ is the value of the fund in thousands of pesos and $t$ is the time in years.

Suppose that Alice has no money whatsoever in her retirement fund in $2030$.
How much money will she have in her retirement fund as a function of time after $2030$?

Don't try to get a numerical answer, just get a function in terms of familiar functions.

\section{New solution}
We begin by separating the differential equation into differential forms:
$$\frac{dP}{0.05P + 100} = dt.$$
Integrating both sides we get 
$$20 \log |0.05P + 100| = t + c.$$
Taking the exponential of both sides,
$$0.05P + 100 = \exp\left(\frac{t + c}{20}\right).$$
For simplicity we rewrite the calendar so $t = 0$ is the year $2030$. Then 
$$100 = e^{c/20}$$
or in other words 
$$c = 20 \log 100.$$
Therefore 
$$0.05P + 100 = 100e^{t/20}.$$
In other words 
$$0.05P = 100(e^{t/20} - 1)$$
or 
$$P = 2000(e^{t/20} - 1)$$
where $t$ is the number of years after $2030$ and $P$ is in thousands of pesos.

\end{document}
