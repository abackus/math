
% --------------------------------------------------------------
% This is all preamble stuff that you don't have to worry about.
% Head down to where it says "Start here"
% --------------------------------------------------------------

\documentclass[10pt]{article}

\usepackage[margin=.7in]{geometry}
\usepackage{amsmath,amsthm,amssymb,mathrsfs}
\usepackage{enumitem}
\usepackage{tikz-cd}
\usepackage{mathtools}
\usepackage{amsfonts}
\usepackage{listings}
\usepackage{algorithm2e}
\usepackage{verse,stmaryrd}
\usepackage{fancyvrb}

% Number systems
\newcommand{\NN}{\mathbb{N}}
\newcommand{\ZZ}{\mathbb{Z}}
\newcommand{\QQ}{\mathbb{Q}}
\newcommand{\RR}{\mathbb{R}}
\newcommand{\CC}{\mathbb{C}}
\newcommand{\PP}{\mathbb P}
\newcommand{\FF}{\mathbb F}
\newcommand{\DD}{\mathbb D}
\renewcommand{\epsilon}{\varepsilon}

\newcommand{\Aut}{\operatorname{Aut}}
\newcommand{\coker}{\operatorname{coker}}
\newcommand{\CVect}{\CC\operatorname{-Vect}}
\newcommand{\Cantor}{\mathcal{C}}
\newcommand{\D}{\mathcal{D}}
\newcommand{\card}{\operatorname{card}}
\newcommand{\dbar}{\overline \partial}
\DeclareMathOperator*{\esssup}{ess\,sup}
\newcommand{\GL}{\operatorname{GL}}
\newcommand{\Hom}{\operatorname{Hom}}
\newcommand{\id}{\operatorname{id}}
\newcommand{\Ind}{\operatorname{Ind}}
\newcommand{\Inn}{\operatorname{Inn}}
\newcommand{\interior}{\operatorname{int}}
\newcommand{\lcm}{\operatorname{lcm}}
\newcommand{\mesh}{\operatorname{mesh}}
\newcommand{\LL}{\mathcal L_0}
\newcommand{\Leb}{\mathcal{L}_{\text{loc}}^2}
\newcommand{\Lip}{\operatorname{Lip}}
\newcommand{\ppGL}{\operatorname{PGL}}
\newcommand{\ppic}{\vspace{35mm}}
\newcommand{\ppset}{\mathcal{P}}
\DeclareMathOperator{\proj}{proj}
\DeclareMathOperator*{\Res}{Res}
\newcommand{\Riem}{\mathcal{R}}
\newcommand{\RVect}{\RR\operatorname{-Vect}}
\newcommand{\Sch}{\mathcal{S}}
\newcommand{\SL}{\operatorname{SL}}
\newcommand{\sgn}{\operatorname{sgn}}
\newcommand{\spn}{\operatorname{span}}
\newcommand{\Spec}{\operatorname{Spec}}
\newcommand{\supp}{\operatorname{supp}}
\newcommand{\Torus}{\mathbb T}
\DeclareMathOperator{\tr}{tr}

\DeclareMathOperator{\adj}{adj}
\DeclareMathOperator{\curl}{curl}

% Calculus of variations
\DeclareMathOperator{\pp}{\mathbf p}
\DeclareMathOperator{\zz}{\mathbf z}
\DeclareMathOperator{\uu}{\mathbf u}
\DeclareMathOperator{\vv}{\mathbf v}
\DeclareMathOperator{\ww}{\mathbf w}

\DeclareMathOperator{\Olo}{\mathscr O}

% Categories
\newcommand{\Ab}{\mathbf{Ab}}
\newcommand{\Cat}{\mathbf{Cat}}
\newcommand{\Group}{\mathbf{Group}}
\newcommand{\Module}{\mathbf{Module}}
\newcommand{\Set}{\mathbf{Set}}
\DeclareMathOperator{\Fun}{Fun}
\DeclareMathOperator{\Iso}{Iso}

% Complex analysis
\renewcommand{\Re}{\operatorname{Re}}
\renewcommand{\Im}{\operatorname{Im}}

% Logic
\renewcommand{\iff}{\leftrightarrow}
\newcommand{\Henkin}{\operatorname{Henk}}
\newcommand{\PA}{\mathbf{PA}}
\DeclareMathOperator{\proves}{\vdash}

% Group
\DeclareMathOperator{\Gal}{Gal}
\DeclareMathOperator{\Fix}{Fix}
\DeclareMathOperator{\Out}{Out}

% Other symbols
\newcommand{\heart}{\ensuremath\heartsuit}

\DeclareMathOperator{\atanh}{atanh}

% Theorems
\theoremstyle{definition}
\newtheorem*{corollary}{Corollary}
\newtheorem*{falselemma}{Grader's ``Lemma"}
\newtheorem{exer}{Exercise}
\newtheorem{lemma}{Lemma}[exer]
\newtheorem{theorem}[lemma]{Theorem}

\usepackage[backend=bibtex,style=alphabetic,maxcitenames=50,maxnames=50]{biblatex}
\renewbibmacro{in:}{}
\DeclareFieldFormat{pages}{#1}

\begin{document}
\noindent
% --------------------------------------------------------------
%                         Start here
% --------------------------------------------------------------\

\section{Statement of the problem}
None of the techniques you have learned so far in this course will help you obtain the exact value of the so-called Gaussian Integral $\int e^{-x^2} dx$ -- that's more a Calc 3 kind of problem.
So, instead, we are going to approximate it using Taylor series.

Part A: Choose an appropriate Taylor series of $e^{-x^2}$ for approximating the integral
$$I = \int_{-1/2}^{1/2} e^{-x^2} ~dx.$$

Part B: Compute this Taylor series and its radius of convergence.

Part C: Using a Taylor polynomial with two nonzero terms, give an approximation to $I$. 

Part D: Give an error bound on your approximation.

For Parts C and D, just leave your answer as a fraction; \textbf{don't try to turn it into a decimal}.

\section{Notes on this problem}
The goals of this problem are to test whether the student knows how to:
\begin{enumerate}
\item choose appropriate centers for Taylor series approximations,
\item paint the shed,
\item either transfer radii of convergence or use the ratio test, 
\item approximate integrals using Taylor series,
\item and use the alternating series test to compute error bounds.
\end{enumerate}

For Part A: Not quite sure of the best way to word this, the point is I just want to ask how the student would center the approximating series.

For Parts B--D: If the student makes an incorrect choice in Part A we can still grade them for these parts as if their solution for Part A was correct, so they won't be punished too harshly for getting Part A wrong. Similarly if Part B is wrong we can still assume that it's right for Parts C and D. However, if Part C is wrong, I don't think we can avoid a double jeopardy on Part D.

For Parts C and D: Not sure if it's still an issue but I know in Math 90 I had several students independently complain to me that the exams were unfair because they couldn't use calculators to turn fractions into decimals, no matter how much we tried to spell out that we don't care at all about the decimal values...

Finally, if this problem is too easy we could make it harder by instead integrating on $[1/2, 3/2]$ instead.

\section{Solution}
Part A: Since we're interested in an integral that's centered on $0$, we should also center our Taylor series at $0$.

Part B: Recall that $e^y = \sum_{n \geq 0} y^n/n!$, so plugging in $y = -x^2$, we get 
$$e^{-x^2} = \sum_{n = 0}^\infty (-1)^n \frac{x^{2n}}{n!}.$$
Since $e^y$ has an infinite radius of convergence this new series also has an infinite radius of convergence. Or you could use the ratio test here, because the ratio is 
$$\frac{\frac{(x + 1)^{2n}}{(n+1)!}}{\frac{x^{2n}}{n!}} = (n + 1)^{-1} \frac{(x + 1)^{2n}}{x^{2n}} \sim \frac{1}{n + 1} < 1.$$

Part C: The Taylor polynomial up to the $x^2$ term is
$$e^{-x^2} \approx 1 - x^2.$$
We integrate this to get 
$$I \approx \left[x - \frac{x^3}{3}\right]_{x = -1/2}^{1/2}.$$
Either you can compute this outright or realize that we have odd symmetry to obtain 
$$I \approx 2\left[x - \frac{x^3}{3}\right]_{x = 0}^{1/2} = 1 - \frac{1}{12} = \frac{11}{12}.$$
(Note: in case some poor student ignores the bolded, shrieked, instructions and tries to compute this as a decimal, it's $0.91\overline 6$.)

Part D: Since $I$ is an alternating series the error term is dominated by the next term in the series.
The next term in the series for $e^{-x^2}$ is $x^4/2$ so the next term in the series for its antiderivative is $-x^5/10$. We compute 
$$\left|2[-x^5/10]_{x = 0}^{1/2}\right| = \frac{1}{320}$$
which is the bound on the error by the alternating series test. (Note: It's $3.125 \cdot 10^{-3}$.)


\end{document}
