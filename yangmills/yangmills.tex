\documentclass[reqno,11pt]{amsart}
\usepackage[letterpaper, margin=1in]{geometry}
\RequirePackage{amsmath,amssymb,amsthm,graphicx,mathrsfs,url,slashed,subcaption}
\RequirePackage[usenames,dvipsnames]{xcolor}
\RequirePackage[colorlinks=true,linkcolor=Red,citecolor=Green]{hyperref}
\RequirePackage{amsxtra}
\usepackage{cancel}
\usepackage{tikz-cd}

% \setlength{\textheight}{9.3in} \setlength{\oddsidemargin}{-0.25in}
% \setlength{\evensidemargin}{-0.25in} \setlength{\textwidth}{7in}
% \setlength{\topmargin}{-0.25in} \setlength{\headheight}{0.18in}
% \setlength{\marginparwidth}{1.0in}
% \setlength{\abovedisplayskip}{0.2in}
% \setlength{\belowdisplayskip}{0.2in}
% \setlength{\parskip}{0.05in}
%\renewcommand{\baselinestretch}{1.05}

\title{Notes on Yang-Mills and gauge theory}
\author{Aidan Backus}
\date{October 2022}

\newcommand{\NN}{\mathbf{N}}
\newcommand{\ZZ}{\mathbf{Z}}
\newcommand{\QQ}{\mathbf{Q}}
\newcommand{\RR}{\mathbf{R}}
\newcommand{\CC}{\mathbf{C}}
\newcommand{\DD}{\mathbf{D}}
\newcommand{\PP}{\mathbf P}
\newcommand{\MM}{\mathbf M}
\newcommand{\II}{\mathbf I}
\newcommand{\Hyp}{\mathbf H}
\newcommand{\Sph}{\mathbf S}
\newcommand{\Group}{\mathbf G}
\newcommand{\GL}{\mathbf{GL}}
\newcommand{\Orth}{\mathbf{O}}
\newcommand{\SpOrth}{\mathbf{SO}}
\newcommand{\SpUn}{\mathbf{SU}}
\newcommand{\Ball}{\mathbf{B}}

\newcommand*\dif{\mathop{}\!\mathrm{d}}
\newcommand*\Dif{\mathop{}\!\mathrm{D}}

\newcommand{\dfn}[1]{\emph{#1}\index{#1}}

\renewcommand{\Re}{\operatorname{Re}}
\renewcommand{\Im}{\operatorname{Im}}

\newcommand{\Ad}{\mathrm{Ad}}
\newcommand{\ad}{\mathrm{ad}}
\newcommand{\Aut}{\mathrm{Aut}}
\newcommand{\tr}{\mathrm{tr}}

\newcommand{\loc}{\mathrm{loc}}
\newcommand{\cpt}{\mathrm{cpt}}

\def\Japan#1{\left \langle #1 \right \rangle}

\newtheorem{theorem}{Theorem}[section]
\newtheorem{badtheorem}[theorem]{``Theorem"}
\newtheorem{prop}[theorem]{Proposition}
\newtheorem{lemma}[theorem]{Lemma}
\newtheorem{sublemma}[theorem]{Sublemma}
\newtheorem{proposition}[theorem]{Proposition}
\newtheorem{corollary}[theorem]{Corollary}
\newtheorem{conjecture}[theorem]{Conjecture}
\newtheorem{axiom}[theorem]{Axiom}
\newtheorem{assumption}[theorem]{Assumption}

\newtheorem{mainthm}{Theorem}
\renewcommand{\themainthm}{\Alph{mainthm}}

% \newtheorem{claim}{Claim}[theorem]
% \renewcommand{\theclaim}{\thetheorem\Alph{claim}}
\newtheorem*{claim}{Claim}

\theoremstyle{definition}
\newtheorem{definition}[theorem]{Definition}
\newtheorem{remark}[theorem]{Remark}
\newtheorem{example}[theorem]{Example}
\newtheorem{notation}[theorem]{Notation}

\newtheorem{exercise}[theorem]{Discussion topic}
\newtheorem{homework}[theorem]{Homework}
\newtheorem{problem}[theorem]{Problem}

\makeatletter
\newcommand{\proofpart}[2]{%
  \par
  \addvspace{\medskipamount}%
  \noindent\emph{Part #1: #2.}
}
\makeatother



\numberwithin{equation}{section}


% Mean
\def\Xint#1{\mathchoice
{\XXint\displaystyle\textstyle{#1}}%
{\XXint\textstyle\scriptstyle{#1}}%
{\XXint\scriptstyle\scriptscriptstyle{#1}}%
{\XXint\scriptscriptstyle\scriptscriptstyle{#1}}%
\!\int}
\def\XXint#1#2#3{{\setbox0=\hbox{$#1{#2#3}{\int}$ }
\vcenter{\hbox{$#2#3$ }}\kern-.6\wd0}}
\def\ddashint{\Xint=}
\def\dashint{\Xint-}

\usepackage[backend=bibtex,style=numeric]{biblatex}
\renewcommand*{\bibfont}{\normalfont\footnotesize}
\addbibresource{yangmills.bib}
\renewbibmacro{in:}{}
\DeclareFieldFormat{pages}{#1}


\begin{document}
\begin{abstract}
    Here are some notes on Yang-Mills as I try to learn about it.
\end{abstract}

\maketitle

%%%%%%%%%%%%%%%%%%%%%%%%%%%%%%%%%%%%%%%%%%%%%%%%%%%%%%%

 \tableofcontents

\section{Preliminaries}
\subsection{Lie algebras}
Let $G$ be a compact semisimple Lie group over $\CC$.
In other words $G$ is the finite product of simple Lie groups, those which do not have a nontrivial Lie quotient.
(This theory works in higher generality but I want to be concrete.)
The (real) tangent space at the identity, that is, the \dfn{linearization} of $G$, is a Lie algebra $\mathfrak g$ over $\CC$.
Thus, $\chi \in G$ which is close to the identity can be written $\chi = e^A$ for some $A \in \mathfrak g$.

\begin{definition}
The \dfn{adjoint representation} $\Ad$ of $G$ on $\mathfrak g$ is the linearization of the action of $G$ on itself by conjugation, namely
$$\Ad(\chi) A := \chi A \chi^{-1}.$$
The \dfn{adjoint representation} $\ad$ of $\mathfrak g$ on itself is the linearization of the adjoint action of $G$, namely 
$$\ad(A) B := [A, B].$$
\end{definition}

In particular we have 
$$\Ad: G \to \Aut(\mathfrak g)$$
and hence $\ad$ maps $\mathfrak g$ into the Lie algebra of derivations of $\mathfrak g$, that is 
$$\ad(A) [B, C] = [B, \ad(A)C] + [\ad(A)B, C].$$
Indeed, this Lie algebra is the linearization of the Lie group $\Aut(\mathfrak g)$.
Thinking of $\ad(A)$ as a linear map, we can define:

\begin{definition}
    The \dfn{Killing form} of a Lie algebra $\mathfrak g$ acts on $A, B \in \mathfrak g$ by 
    $$(A, B) := \tr(\ad(A) \overline{\ad(B)}).$$
\end{definition}

\begin{theorem}
    Suppose that $\mathfrak g$ is semisimple. Then the Killing form is an $\Ad$-invariant inner product on $\mathfrak g$.
\end{theorem}

Since $G$ is semisimple, so is its linearization, thus its linearization carries the structure of a Hilbert space.

%%%%%%%%%%%%%%%
\subsection{Connections}
\begin{definition}
Let $M$ be a manifold, $G$ a Lie group, and $E \to M$ a vector bundle over $\CC$.
We call $E \to M$ a $G$-\dfn{bundle} if it is equipped with a representation $G \to \Aut(E)$ so that transition maps can be chosen to lie in $G$.
In this setting we call a map $M \to G$ a \dfn{gauge transformation}.
\end{definition}

If $E \to M$ is a $G$-bundle, then any gauge transformation $\chi: M \to G$ acts on the fibers of $E$:
$$\chi(x, v) = (x, \chi(x)v).$$
In particular $\chi$ acts on sections $f: M \to E$ by $\chi f(x) = \chi(x) f(x)$.

Recall that if $E \to M$ is a vector bundle and $\Dif$ is a connection on $E$, then locally we can find \dfn{gauge fields} or \dfn{Christoffel symbols}, which are $\mathfrak{gl}(E)$-valued $1$-forms 
$$A: M \to T'M \otimes \mathfrak{gl}(E)$$
such that for any section $f: M \to E$,
$$\Dif_\mu f = \partial_\mu f + A_\mu f.$$
Here $\mathfrak{gl}(E)$ is the bundle of Lie algebras of linear maps $E_x \to E_x$, $x \in M$.
On the other hand, if $\mathfrak g$ is the Lie algebra of $G$, then the linearization of the representation $G \to \Aut(E)$ is a representation $\mathfrak g \to \mathfrak{gl}(E)$.
So the next definition makes sense:

\begin{definition}
Let $G$ be a Lie group with Lie algebra $\mathfrak g$.
By a \dfn{connection} on a $G$-bundle $E$ we shall mean a connection $\Dif$ on the level of vector bundles, such that we can cover $M$ by open sets in which $\Dif$ admits gauge fields valued in $\mathfrak g$.
\end{definition}

Assuming that the representation $G \to \Aut(E)$ is faithful, we can think of the linearization $\mathfrak g$ of $G$ as a Lie algebra bundle, by identifying it with a subbundle of $\mathfrak{gl}(E)$.
Then a connection $\Dif$ acts on sections of $\mathfrak g$ using the adjoint representation.
Indeed, if $\Dif$ has gauge field $A$, then 
$$\Dif_\mu B = \partial_\mu B + [A_\mu, B].$$
In particular the curvature satisfies
$$[\Dif_\mu, \Dif_\nu] = \ad(\partial_\mu A_\nu - \partial_\nu A_\mu + [A_\mu, A_\nu]).$$
This motivates our next definition, and shows that it does not depend on a choice of gauge field.

\begin{definition}
The \dfn{curvature} $2$-form of a connection with gauge field $A$ is the $\mathfrak g$-valued $2$-form
$$F_{\mu \nu} := \partial_\mu A_\nu - \partial_\nu A_\mu + [A_\mu, A_\nu].$$
\end{definition}

Henceforth we assume that $G$ is semisimple, compact, and faithful so everything is well-defined.

The curvature form $F$ of a connection $\Dif$ is antisymmetric and its derivative is the $\mathfrak g$-valued $3$-form
$$(\Dif F)_{\mu \nu \lambda} = \Dif_\mu F_{\nu \lambda} + \Dif_\nu F_{\lambda \mu} + \Dif_\lambda F_{\mu \nu}.$$

\begin{proposition}[Bianchi identity]
The curvature $2$-form $F$ of $\Dif$ satisfies $\Dif F = 0$.
\end{proposition}
\begin{proof}
The first term is 
$$\Dif_\mu F_{\nu \lambda} = \partial_\mu (\partial_\nu A_\lambda - \partial_\lambda A_\nu + [A_\nu, A_\lambda]) + [A_\mu, F_{\nu \lambda}].$$
Note that
$$\partial_\mu (\partial_\nu A_\lambda - \partial_\lambda A_\nu) + \partial_\nu (\partial_\lambda A_\mu - \partial_\mu A_\lambda) + \partial_\lambda (\partial_\mu A_\nu - \partial_\nu A_\mu) = 0$$
by the classical Bianchi identity. So 
$$(\Dif F)_{\mu \nu \lambda} = \partial_\mu [A_\nu, A_\lambda] + \partial_\nu [A_\lambda, A_\mu] + \partial_\lambda [A_\mu, A_\nu] + [A_\mu, F_{\nu \lambda}] + [A_\nu, F_{\lambda \mu}] + [A_\lambda, F_{\mu \nu}].$$
Now observe that 
$$[A_\mu, F_{\nu \lambda}] = [A_\mu, \partial_\nu A_\lambda] - [A_\mu, \partial_\lambda A_\nu] + [A_\mu, [A_\nu, A_\lambda]].$$
By the Jacobi identity
$$[A_\mu, [A_\nu, A_\lambda]] = [A_\nu, [A_\lambda, A_\mu]] + [A_\lambda, [A_\mu, A_\nu]] = 0,$$
the double commutator terms all cancel out.
Thus 
\begin{align*}
(\Dif F)_{\mu \nu \lambda} &= \partial_\mu [A_\nu, A_\lambda] + \partial_\nu [A_\lambda, A_\mu] + \partial_\lambda [A_\mu, A_\nu] \\
&\qquad + [A_\mu, \partial_\nu A_\lambda] - [A_\mu \partial_\lambda A_\nu] + [A_\nu, \partial_\lambda A_\mu] - [A_\nu, \partial_\mu A_\lambda] + [A_\lambda, \partial_\mu A_\nu] - [A_\lambda, \partial_\nu A_\mu].
\end{align*}
By the product rule
$$\partial_\mu [A_\nu, A_\lambda] = [\partial_\mu A_\nu, A_\lambda] + [A_\nu, \partial_\mu A_\lambda]$$
all the remaining terms cancel.
\end{proof}

Now consider the action of a gauge transformation $\chi: M \to G$.
We define a gauge-transformed gauge field
$$A_\mu^{(\chi)} := \chi A_\mu \chi^{-1} - \chi_{,\mu} \chi^{-1}$$
so that for a gauge-transformed section $f^{(\chi)} := \chi f$,
$$\Dif^{(\chi)}_\mu f^{(\chi)} = \chi_{,\mu} f + \chi \Dif_\mu f - \chi_{,\mu} f = \chi \Dif_\mu f = (\Dif_\mu f)^{(\chi)}.$$
Noticing that the $\chi_{,\mu} \chi^{-1}$ terms all cancel when we take curvatures, we observe that gauge transformations act on curvature by
\begin{equation}\label{curvature transformation}
F_{\mu \nu}^{(\chi)} = \chi F_{\mu \nu} \chi^{-1}.
\end{equation}

\begin{definition}
The \dfn{Laplace} (for $M$ Riemannian) $\Delta_A$ or \dfn{wave} (for $M$ Lorentzian) $\Box_A$ operators on sections $f: M \to E$ are defined by $\Dif^\mu \Dif_\mu$.
\end{definition}

%%%%%%%%%%%%%%%%%%%%%%%%
\subsection{Characteristic classes}
\begin{definition}
Let $\Dif$ a connection on a $G$-bundle of rank $k$ with curvature $F$.
For a symmetric $\Ad$-invariant homogeneous polynomial $f: \mathfrak g \to \CC$ of degree $k$, define the cohomology class of
$$f(F_{\mu_1 \mu_2}, \dots, F_{\mu_{2k - 1} \mu_{2k}}) \dif x^{\mu_1} \wedge \cdots \wedge \dif x^{\mu_{2k}}$$
to be the \dfn{characteristic class} of $\Dif$ measured by $f$.
\end{definition}

\begin{theorem}[Chern-Weil]
Let $\Dif$ be a connection on a $G$-bundle $E \to M$ of rank $k$.
Then the characteristic class of $\Dif$ measured by a symmetric $k$-linear form $f$ is well-defined as an element of $H^{2k}(M, \CC)$, and only depends on $E$ and $f$.
\end{theorem}
\begin{proof}
Since $f$ is $\Ad$-invariant, it commutes with $\Dif$, and by the Bianchi identity, $\Dif F = 0$.
So $\Dif$ annihilates the characteristic class.
But the characteristic class is a $2k$-form with values in $\CC$, not in $\mathfrak g$, so it commutes with the gauge field $A$ of $\Dif$.
Therefore $A$ annihilates the characteristic class, so $\dif$ does as well.
In particular, the representative of the characteristic class is a closed $2k$-form, so it honestly defines an element of $H^{2k}(M, \CC)$.
Any two connections on $E$ are homotopic, so as an element of $H^{2k}(M, \CC)$ their characteristic classes are identical.
\end{proof}

This suggests that the homotopy class of the vector bundle associated to a gauge field is of great importance, since the gauge field already knows the characteristic class of the vector bundle.
Thus we make the following definition.

\begin{definition}
Let $A$ be a gauge field on a closed manifold.
The \dfn{homotopy class} of $A$ is the isomorphism class of the vector bundle which admits $A$ as a connection.
\end{definition}

If $A$ is a gauge field on $\RR^d$ with compactly supported curvature, then we can also define its homotopy class.
Applying a gauge transformation, we may assume that $A = 0$ near infinity.
We then pull back $A$ by stereographic projection to a bundle on $\Sph^d$, which is well-defined because we just set $A = 0$ at the north pole of $\Sph^d$.
Since $\Sph^d$ is a closed manifold, this gives us a gauge field on $\Sph^d$ and we define the homotopy class of $A$ to be the homotopy class of this extended gauge field.

%%%%%%%%%%%%%
\subsection{Uhlenbeck's lemma}
Let's work on a Riemannian manifold and put units on everything.
A gauge field $A$ has one down-index, so it should be in units of inverse meters.
Thus the curvature $2$-form should be in units of inverse square meters, and the $L^p$ norm of the curvature is in units of meters$^{d - 2p}$.
So the curvature naturally lives in $L^{d/2}$, and hence we naturally think of $A \in W^{1, d/2}$.

The first upshot is that the energy is critical if $d = 4$.
For $d \geq 5$ the energy is subcritical and Yang-Mills is ``easy'' to understand; if $d \leq 3$ then the energy is supercritical and we expect some kind of blowup.
This is why so many papers about Yang-Mills focus on applications to fourfolds, or equivalently five-dimensional spacetimes.

The second upshot is that we want an ellipticity bound 
$$||A||_{W^{1, d/2}} \lesssim ||F||_{L^{d/2}}.$$
Now this is obviously not true because we can consider a flat connection $A$ (thus $F = 0$) which is wildly oscillatory at infinity.
This can be easily imposed using a gauge transformation of the trivial connection.
However, the claim is even pretty unreasonable if we only require that there exists a gauge transformation $\chi$ such that $||A^{(\chi)}||_{W^{1, d/2}}$ is small, because Yang-Mills is nonlinear.
The best we can hope for is the following:

\begin{theorem}[Uhlenbeck's lemma]
Let $F$ be a curvature $2$-form such that $||F||_{L^{d/2}} \ll 1$.
Then there exists a gauge field $A$ and a gauge transformation $\chi$ with curvature $F^{(\chi)}$, such that
$$||A||_{W^{1, d/2}} \lesssim ||F||_{L^{d/2}},$$
and moreover $A$ is in Coulomb gauge.
\end{theorem}
TODO


%%%%%%%%%%%%%%%%%%%%%%%%
\section{Yang-Mills equation}
\begin{definition}
Let $M$ be a Lorentzian or Riemannian manifold, let $A$ be a gauge field over $M$, and let $F$ be the curvature of $A$.
The \dfn{Yang-Mills energy} of $A$ is 
$$E(A) := \frac{1}{2} \int_M \star (F_{\mu \nu}, F^{\mu \nu}).$$
The \dfn{Yang-Mills equation} is the Euler-Lagrange equation
$$\Dif^\mu F_{\mu \nu} = 0$$
for critical points of the Yang-Mills energy.
\end{definition}

Applying (\ref{curvature transformation}) and the fact that the Killing form $(\cdot, \cdot)$ is invariant under gauge transformations, we see that the Yang-Mills energy, hence also the Yang-Mills equation, is preserved by gauge transformations as well.
Writing out the Yang-Mills equations explicitly for $A$, they are 
\begin{equation}\label{explicit yang mills}
(\partial^\mu + \ad(A^\mu))(\partial_\mu A_\nu - \partial_\nu A_\mu + [A_\mu, A_\nu]) = 0.
\end{equation}

Henceforth we use Latin indices for Riemannian indices, Greek indices for Lorentzian indices, and $0$ for the timelike Lorentzian index.
Thus we usually write 
$$\Dif^j F_{jk} = 0$$
for the Yang-Mills equation on a Riemannian manifold, that is, the \dfn{elliptic Yang-Mills equation}.
We call solutions \dfn{Yang-Mills connections}.
It's a bit of a lie to call this equation elliptic, though, since it has many solutions: if $A$ is a solution, then so is $A^{(\chi)}$ for any gauge transformation $\chi$.

\begin{definition}
A gauge field $A$ is in \dfn{Coulomb gauge} if $\partial^j A_j = 0$.
\end{definition}

Note that in Coulomb gauge, the term $\partial^j \partial_j A_j$ is zero, so the only second-order term in (\ref{explicit yang mills}) is $\partial^\mu \partial_\mu A_\mu$.
In particular the principal Yang-Mills symbol in Coulomb gauge is $|\xi|^2$ which is elliptic.

We can do a similar trick for the \dfn{hyperbolic Yang-Mills equation} (that is, the case that $M$ is Lorentzian) using the Lorenz gauge:

\begin{definition}
A gauge field $A$ is in \dfn{Lorenz gauge} if $\partial^\mu A_\mu = 0$.
\end{definition}

We call solutions to the hyperbolic Yang-Mills equation \dfn{Yang-Mills waves}.

%%%%%%%%%%%%%%%%%%
\subsection{Ground states}
If $M$ is Riemannian and we give $M \times \RR$ a Lorentzian metric in the usual way, then Yang-Mills connections on $M$ are static Yang-Mills waves on $M \times \RR$.
On the other hand, since the ellipticity of Coulomb gauge implies that the elliptic Yang-Mills equation honestly is elliptic up to gauge transformation, we can look for an energy-minimizing connection on $M$ in each homotopy class.
Such a connection must be a Yang-Mills connection.

\begin{definition}
A nonzero Yang-Mills connection which minimizes the energy in its homotopy class is called an \dfn{instanton}.
An instanton which minimizes energy among all instantons (possibly not in its homotopy class) is called a \dfn{ground state}.
\end{definition}

If $G$ is abelian, then the Yang-Mills equations reduce to the Maxwell equations
$$\partial^\mu(\partial_\mu A_\nu - \partial_\nu A_\mu) = 0.$$
Here $A_0$ is the electric potential, $A_j$ is the magnetic potential, and the reduction occurred because all the commutator terms dropped out.
We call such an $A$ a \dfn{photon field}.
There are no finite-energy static photon fields except the trivial field, if $M$ is complete (since there are no nontrivial harmonic functions with finite Dirichlet energy), and in particular there are no ground states in the theory of electromagnetism.
Notice that in Lorenz gauge, the Maxwell equations simplify further to $\Box A_\mu = 0$, the wave equation!
Solutions are light waves, which justifies why we call $A$ a photon field.

However, if $G$ is nonabelian, then there can exist ground states. Let me take the following as a black box:

\begin{theorem}
The second Chern class $c_2$ of a $\SpUn(2)$-bundle $E \to \Sph^4$ is given by the characteristic class $\star \tr(F \wedge F)$ where $F$ is the curvature of a connection on $E$.
Moreover, $c_2$ is a complete invariant of $\SpUn(2)$-bundles over $\Sph^4$, and $c_2$ is in turn completely determined by $\int_{\Sph^4} c_2/8\pi^2 \in \ZZ$.
\end{theorem}

In particular we have a $\SpUn(2)$-bundle over $\Sph^4$ for each integer, and the ground state is the instanton corresponding to the $\SpUn(2)$-bundle for which $\int_{\Sph^4} c_2 = 8\pi^2$.

%%%%%%%%%%%%%%%
\subsection{Physical interpretation}
We already argued that for $G$ abelian, a $G$-Yang-Mills wave is nothing more than a photon field.
However, the same thing happens for $G = \mathbf U(1) \times \SpUn(2)$ or $G = \SpUn(3)$ on the Minkowski spacetime $\RR^{3, 1}$, where $\mathbf U(1) \times \SpUn(2)$ corresponds to the electroweak force and $\SpUn(3)$ corresponds to the strong force.
In particular, a $\SpUn(3)$-Yang-Mills wave is a \dfn{gluon field}.

Let's look a bit more at the strong force.
There are $8$ gluons, which correspond to the $8$ basis vectors of $\mathfrak{su}(3)$.
Basis vectors of $\mathfrak{su}(3)$ are called \dfn{Gell-Mann matrices}.
Since $\SpUn(3)$ is nonabelian, we expect there to exist a ground state gluon field, and the Yang-Mills equation are nonlinear.
The nonlinearity arises from the terms $[A^\mu, \partial_\mu A_\nu]$, $[A^\mu, \partial_\nu A_\mu]$, and $[A^\mu, [A_\mu, A_\nu]]$.
The first two are quadratic nonlinearities and correspond to two gluons interacting.
However we could also have three gluons interacting.

But gluons don't just act on each other.
In particle physics the $\SpUn(3)$-Yang-Mills equation is usually coupled to the Dirac equation
\begin{align*}
(i\cancel \Dif - m) \psi &= 0\\
\Dif^\mu F_{\mu \nu} &= \Im(\psi \overline{\Dif_\nu \psi})
\end{align*}
where $\cancel \Dif$ is the Dirac operator, i.e. the natural action of $\Dif$ on spinor fields tensored with a representation of $\SpUn(3)$, $m$ is the mass of a quark, and the spinor field $\psi$ is called the \dfn{quark field}.
Unfortunately this system seems analytically out of reach, given that we can't even handle nonabelian Yang-Mills half the time.
Since $\psi$ consists of a spinor field tensored with a representation of $\SpUn(3)$, it has 3 ``components'', each of which is a spinor field.
The components correspond to the three different ``color charges'' that a quark can have.

Similar happens in the Maxwell case, where $\psi$ carries electric charge (thus $\psi$ can be thought of as the electron and proton fields, at least to crude approximation).
There the fact that the spinor field needs to be tensored with a representation of $\mathbf U(1)$ corresponds to the fact that there is only one electric charge, unlike color charge.



\printbibliography

\end{document}
