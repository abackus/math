\documentclass[12pt]{report}
\usepackage[utf8]{inputenc}
%\usepackage[margin=1in]{geometry}
\usepackage{amsmath,amsthm,amssymb}
\usepackage{mathrsfs}

\usepackage{enumitem}
%\usepackage[shortlabels]{enumerate}
\usepackage{tikz-cd}
\usepackage{mathtools}
\usepackage{amsfonts}
\usepackage{amscd}
\usepackage{makeidx}
\usepackage{enumitem}

\title{The Breit-Wigner formula}
\author{Aidan Backus}
\date{Fall 2019}

\newcommand{\NN}{\mathbf{N}}
\newcommand{\ZZ}{\mathbf{Z}}
\newcommand{\QQ}{\mathbf{Q}}
\newcommand{\RR}{\mathbf{R}}
\newcommand{\CC}{\mathbf{C}}
\newcommand{\DD}{\mathbf{D}}

\DeclareMathOperator{\ch}{ch}
\DeclareMathOperator{\supp}{supp}

\newcommand{\dbar}{\overline \partial}

\newcommand{\pic}{\vspace{30mm}}
\newcommand{\dfn}[1]{\emph{#1}\index{#1}}

\renewcommand{\Re}{\operatorname{Re}}
\renewcommand{\Im}{\operatorname{Im}}

\newcommand{\tr}{\operatorname{tr}}

\theoremstyle{definition}
\newtheorem{theorem}{Theorem}[chapter]
\newtheorem{badtheorem}[theorem]{``Theorem"}
\newtheorem{prop}[theorem]{Proposition}
\newtheorem{lemma}[theorem]{Lemma}
\newtheorem{proposition}[theorem]{Proposition}
\newtheorem{corollary}[theorem]{Corollary}
\newtheorem{conjecture}[theorem]{Conjecture}
\newtheorem{axiom}[theorem]{Axiom}

\newtheorem{definition}[theorem]{Definition}
\newtheorem{remark}[theorem]{Remark}
\newtheorem{example}[theorem]{Example}

\newtheorem{exercise}[theorem]{Discussion topic}
\newtheorem{homework}[theorem]{Homework}
\newtheorem{problem}[theorem]{Problem}

\usepackage{color}
\usepackage{hyperref}
\hypersetup{
    colorlinks=true, % make the links colored
    linkcolor=blue, % color TOC links in blue
    urlcolor=red, % color URLs in red
    linktoc=all % 'all' will create links for everything in the TOC
    %Ning added hyperlinks to the table of contents 6/17/19
}

\usepackage[backend=bibtex]{biblatex}
\addbibresource{thesis.bib}

\makeindex
\begin{document}

\maketitle

\tableofcontents


\section*{Acknowledgments}
I would like to thank Prof. Maciej Zworski for suggesting this topic and mentoring me, as well as teaching me much of what I know about partial differential equations, harmonic analysis, and complex analysis. I would also like to thank Erik Wendt for suggesting the use of conformal mappings to better explain Green functions.



\chapter{Zeroes of Fourier transforms}
Let $\mu$ be a distribution on $\RR$, i.e. a continuous element of the dual of the space of test functions $C^\infty_{comp}(\RR)$. We will abuse notation and write $\int_E f\mu$ for the pairing of $f$ and $\mu$, whenever it is defined. If $\mu$ satisfies certain growth conditions, then the Fourier transform
$$\hat \mu(\xi) = \int_{-\infty}^\infty e^{-ix\xi} \mu(x) ~dx$$
is well-defined and is also a distribution.

For any distribution $\mu$ with bounded support $\supp \mu$, the convex hull of the support $\ch \supp \mu$ is given by a compact interval, and the Paley-Weiner theorem guarantees that $\hat \mu$ is an entire function. In particular, the number of zeroes of $\hat \mu$ lying in any compact set is necessarily finite, even when counted with multiplicity, assuming that $\mu \neq 0$.

Our goal is to prove the following theorem, which was first proven by Titchmarsh \cite{titchmarsh1926zeros}. The proof we give is based on the proof of Beurling that was first published in \cite[Chapter XVI]{hormander2004analysis}.
\begin{theorem}
    \index{Titchmarsh's theorem}
    Let $\mu$ be a distribution with $\ch \supp \mu = [a, b]$. Let $N(R)$ denote the number of zeroes $z$ of $\hat \mu$ with $|z| < R$, counted with multiplicity. Then
    $$\lim_{R \to \infty} \frac{N(R)}{R} = \frac{b-a}{\pi}.$$
\end{theorem}

\section{The Paley-Wiener theorem}
We review the proof of the Paley-Wiener theorem, proven by Schwartz. The exposition given here is based on \cite[Chapter VII]{hormander2015analysis}.

\begin{definition}
    For $\alpha,\beta$ multiindices, we define the $(\alpha,\beta)$th \dfn{Schwartz seminorm} on $\Omega \subseteq \CC^n$ by
$$||f||_{\alpha,\beta} = \sup_{x \in \Omega} |x^\alpha \partial^\beta f(x)|.$$
    The locally convex space of all $f$ for which every Schwartz seminorm $||f||_{\alpha,\beta}$ is finite is called the \dfn{Schwartz space}. A \dfn{tempered distribution} is a distribution $g$ whose pairing with every element of Schwartz space is finite.
\end{definition}
    If a distribution is tempered on $\RR^n$, then its Fourier transform is well-defined and also tempered. This can be easily proven using duality once it is shown that the Fourier transform is an automorphism of Schwartz space. Most distributions or functions which are ``not too discontinuous" and ``do not grow too fast" are tempered distributions. For example, any compactly supported distribution is tempered, as is any smooth, polynomially growing function.

If $E \subset \RR$ is bounded, then the convex hull of $E$, $\ch E$, is defined to be the intersection of all compact, convex subsets containing $E$; since $\ch E$ is connected, it must be the compact interval $[a, b]$, where $a = \inf E$ and $b = \sup E$.
\begin{definition}
    The \dfn{supporting function} $h_E$ of a bounded set $E \subset \RR$ is defined for $\xi \in \RR$ by
$$h_E(\xi) = \sup_{x \in E} x\xi.$$
\end{definition}
    Taking the closure of $E$ will not affect $h_E$. Then for $\xi > 0$, $h_E(\xi) = b\xi$, $h_E(0) = 0$, and $h_E(-\xi)     = -a\xi$. Taking the convex hull will not change $a$ or $b$, so $h_E = h_{\ch E}$.

    Let us fix some notation. Let $E_\delta$ denote the ball around $E$ of radius $\delta$; that is,
$$E_\delta = \{x \in \RR: \exists y \in E~|x - y| < \delta\}.$$
    For $\mu$ a distribution on $\RR$, we will write $||\mu|| = |\int_{-\infty}^\infty \mu(x) ~dx|$, if such an integral is indeed finite. So if $\mu$ is actually a positive function, then $||\mu|| = ||\mu||_{L^1}$.
\begin{theorem}[Paley-Wiener]
    Let $E \subset \RR$ be bounded. If $\mu$ is a distribution on $\RR$ with $\supp \mu \subseteq E$, then the Fourier transform $\hat \mu$ is an entire function satisfying the estimate
$$|\hat \mu(\zeta)| = O(e^{h_E(\Im \zeta)}).$$
    Conversely, for every entire function $f$ such that $|f(\zeta)| = O(e^{h_E(\Im \zeta)})$, there is a distribution $\mu$ on $\RR$ with $\supp \mu \subseteq \ch E$ and $\hat \mu = f$.
\end{theorem}
\begin{proof}
    Since $\mu$ is compactly supported, it lies in the dual of $C^\infty(\RR)$. So $||\mu||$ is finite, and we can differentiate $\hat \mu$ by putting all derivatives on the smooth function $e^{-ix\xi}$, using integration by parts. Therefore $\hat \mu \in C^\infty(\CC)$. In particular, if $\dbar$ is the Cauchy-Riemann operator, then we can use the fact that $\dbar e^{-ix\xi} = 0$ (since $e^{-i x\xi}$ is clearly entire). Therefore $\hat \mu$ is entire.

    Let $\Im \zeta > 0$ and $\ch E = [a, b]$. Then
$$|\hat \mu(\zeta)| \leq \int_{-\infty}^\infty |e^{-ix\zeta} \mu(x)| ~dx \leq ||\mu|| \int_a^b e^{x \Im \zeta} ~dx \leq ||\mu|| e^{h_E(\Im \zeta)}.$$

    For the converse, let $f$ be an entire function satisfying $|f(\zeta)| = O(e^{h_E(\Im \zeta)})$. Then the restriction $\tilde f = f|_\RR$ is a bounded smooth function. Therefore $\tilde f$ is tempered, so has an inverse Fourier transform $\mu$.

    Let $\varphi \in C^\infty_{comp}((-1, 1))$ be a positive function, normalized so that such that $\int_{-1}^1 \varphi = 1$, and set $\varphi_\delta(x) \varphi(x/\delta)/\delta$. Thus $\mu * \varphi_\delta$ is a mollification of $\mu$, i.e. $\mu * \varphi_\delta \in C^\infty(\RR)$ and $\lim_{\delta \to 0} \mu * \varphi_\delta = \mu$. $\widehat{\mu*\varphi_\delta} = \hat \mu  \hat \varphi_\delta = \tilde f \hat \varphi_\delta$. Taking the unique analytic continuation of $\tilde f$ to $\CC$, we extend $\tilde f \hat \varphi_\delta$ to $f \hat \varphi_\delta$.

    Since $\varphi_\delta$ is supported in $(-\delta, \delta)$, $|\hat \varphi_\delta(\zeta)| = O(\delta |\Im \zeta|)$. Therefore
$$|f(\zeta) \varphi_\delta(\zeta)| = O(\exp(h_E(\Im \zeta) + \delta|\Im \zeta|)).$$
    If we can prove the converse for $\mu \in C^\infty_{comp}(\RR)$, then we can replace $\mu$ by $\mu * \varphi_\delta$ to show that $\supp (\mu * \varphi_\delta) \subseteq \ch \supp E_\delta$. Since $\delta$ was arbitrary, it will follow that $\supp \mu \subseteq \ch \supp E$. Thus, we may assume without loss of generality that $\mu \in C^\infty_{comp}(\RR)$.

    In fact, if $\mu \in C^\infty_{comp}(\RR)$, then in particular $\mu$ lies in Schwartz space. In this case, we can find a $C_n$, independent of $\zeta = \xi + i\eta$, so that
$$|\zeta^n f(\zeta)| \leq C_n e^{h_E(\Im \zeta)}.$$
    Dividing both sides by $|\zeta|^{-n}$, we have
    $$|f(\xi + i \eta)| \leq C_n \frac{e^{h_E(\eta)}}{|\xi + i\eta|^n}.$$
    Thus for $\eta$ fixed, $f(\cdot + i\eta)$ is rapidly decreasing. Thus we can make a change of variables to see that
$$\mu(x) = \frac{1}{2\pi} \int_{-\infty}^\infty e^{ix\xi} f(\xi) ~d\xi = \frac{1}{2\pi} \int_{-\infty}^\infty e^{ix(\xi+i\eta)} f(\xi + i\eta) ~d\xi.$$
    Therefore
    $$|\mu(x)| \leq C_Ne^{-x\eta + h_E(\eta)} \int_{-\infty}^\infty \frac{d\xi}{|\xi + i\eta|^N}.$$
    We fix a sufficiently large $N$ and let $\eta \to 0$. This proves that $\mu(x) = 0$ if $x\eta \leq h_E(\eta)$, which happens if and only if $x \notin \ch E$. Therefore $\supp \mu \subseteq \ch E$.
\end{proof}
    Since the Paley-Wiener theorem is a biconditional, the estimate
    $$|\hat \mu(\zeta)| \leq ||\mu|| e^{h_E(\Im \zeta)}$$
    is sharp: we cannot replace the $a, b$ appearing in the piecewise-linear definition of $h_E$ with better constants. This precision will be important in the proof of Titchmarsh's theorem.


\section{Subharmonic functions}
Fix an open set $\Omega \subseteq \CC$. Recall that a function $u: \Omega \to [-\infty, \infty)$ which is upper-semicontinuous is called subharmonic if for each $z \in \CC$, the averages
$$M(z, r) = \frac{1}{2\pi} \int_0^{2\pi} u(z + re^{i\theta}) ~d\theta$$
are increasing in $r$. This condition is logically equivalent to assuming that $u$ satisfies a maximum principle: for every compact set $K$ with nonempty interior $U$, if $u|_K$ attains its maximum on $U$, then $u|_K$ is a constant. To avoid trivialities, we shall assume that $u$ is not identically $-\infty$, though not every author makes this assumption.

It is a well-known result that an upper-semicontinuous function $u$ is subharmonic iff the weak Laplacian $\Delta u \geq 0$; that is, for any nonnegative test function $\varphi \in C^\infty_{comp}(\Omega)$,
$$\int_\Omega u(z) \Delta \varphi(z) ~dz \geq 0.$$ If actually $\Delta u = 0$, then we call $u$ harmonic; by a typical mollification argument, if $u$ is harmonic, then $u$ is actually smooth (even real analytic), but such niceties may not be true for subharmonic functions. The reason why we refer to functions $u$ with $\Delta u \geq 0$ as subharmonic rather than superharmonic is that $-\Delta$ is a positive operator, and so we think of applying $\Delta$ as akin to ``multiplying by a negative function". Such a function lies in $L^1_{loc}(\Omega)$. (For the proofs, see \cite[Chapter 1]{hormander1973introduction}.)

By the Paley-Wiener theorem, if $\mu$ is a distribution with $\ch \supp \mu = [a, b]$, then $\hat \mu$ is an entire function which satisfies the estimate
$$\hat \mu(x + iy) \leq Ce^{h(y)}$$
for some constant $C$, where $h(y) = by$ for $y > 0$, $h(y) = ay$ for $y < 0$. In particular, in the upper half-plane $\CC_+$, we have the estimate
$$\hat \mu(x + iy) \leq Ce^{by}.$$
Moreover, since $\hat \mu$ is holomorphic, it solves the Cauchy-Riemann equation $\dbar \hat \mu = 0$. Since we can factor the Laplacian as $4\Delta = \partial \dbar$, it follows that $\Delta \hat \mu = 0$.

Let
$$E(z) = \frac{\log |z|}{2\pi}.$$
Then $E$ is the fundamental solution of the Laplacian. This means that $\Delta E = \delta_0$, where $\delta_z$ denotes the Dirac distribution centered at $z$. In particular, the solution of the equation $\Delta u = f$ is $u = E*f$.

Since $\hat \mu$ is holomorphic, if it has a zero $z \in \CC_+$ of multiplicity $m$, we can write
$$\hat \mu(\zeta) = (\zeta - z)^m g(\zeta)$$
for some holomorphic function $g$ with $g(z) \neq 0$. If $g \neq 0$ everywhere, then, because $\Delta g = 0$, it follows that
$$\Delta \log |\hat \mu| = m\log |\zeta - z| + \log |g| = 2\pi m \delta_z.$$
Using a partition of unity to sum over all zeroes in this manner, it follows that if $Z$ denotes the multiset of zeroes of $\hat \mu$ counted with multiplicity, then
$$\Delta(\log |\hat \mu|) = 2\pi \sum_{z \in Z} \delta_z \geq 0.$$
It follows that $\log |\hat \mu|$ is subharmonic and, on $\CC_+$, satisfies the estimate
$$\log |\hat \mu(x + iy)| \leq C + Dy$$
for some constants $C, D$, by the Paley-Wiener theorem.

\begin{definition}
    Let $u$ be a subharmonic function on $\CC_+$. If there are constants $C, D > 0$ so that $u(x + iy) \leq C + Dy$, then we say that $u$ is \dfn{imaginary-sublinear}.
\end{definition}
It is immediate that $\log |\hat \mu|$ is imaginary-sublinear.

\begin{lemma}
    \label{imaginary sublinear limit}
    Let $u$ be an imaginary-sublinear subharmonic function on $\CC_+$ and define
    $$\gamma = \lim_{y \to \infty} \sup_x u(x + iy).$$
    Then $\gamma$ is well-defined, and $\gamma \in (-\infty, D]$.
\end{lemma}
\begin{proof}
    We first let
    $$M(y) = \sup_x u(x + iy).$$
    Then $\gamma = \lim_y M(y)$, and $M(y) \leq C + Dy$ since $u$ is imaginary-sublinear.

    Let us prove that $M$ is convex. Let $0 < a < b$ and let $L: \RR \to \RR$ be a linear function such that $M(a) \leq L(a)$ and $M(b) \leq L(b)$. Let
    $$v(x + iy, \varepsilon) = u(x + iy) - L(y) - \varepsilon(x^2 - (y^2 - b^2)).$$
    Then $v(\cdot + iy_0, \varepsilon) \leq 0$ for $y_0 \in \{a, b\}$. Similarly, $\lim_{x \to \pm \infty} v(x + \cdot, \varepsilon) = -\infty$. Moreover, $\Delta v(\cdot, \varepsilon) \geq \Delta u \geq 0$.

    Applying the maximum principle to a sufficiently large compact subset $K$, and noting that $v(\cdot, \varepsilon) \leq 0$ on $\partial K$, $v(\cdot, \varepsilon) \leq 0$ on $K$, hence globally since $K$ was arbitrary. Taking $\varepsilon \to 0$, we see that $u(x + iy) \leq L(y)$, so maximizing over $x$, $M \leq L$ on $[a, b]$. So $M$ is convex. But the limit of a sublinear, convex function exists, so $\gamma$ is well-defined, and $\gamma > -\infty$ since $u \in L^1_{loc}(\Omega)$, hence not $-\infty$ except on a discrete set. The bound $\gamma \leq D$ follows easily.
\end{proof}

We now begin working towards a representation formula for imaginary-sublinear functions on the upper-half plane in terms of their Laplacian and their boundary values. To do this, we construct the Green function of $\Delta$ in the half plane.
\begin{definition}
    The \dfn{Green function} for $\Delta$ on $\CC_+$ is defined on $\CC_+ \times \CC_+$ by
$$G(z, w) = E(z - w) - E(z - \overline w).$$
    The \dfn{Poisson kernel} for $\Delta$ on $\CC_+$ is defined on $\CC+_ \times \RR$ by
$$P(z, x) = -\frac{\partial G(z, x + iy)}{\partial y}|_{y = 0}.$$
\end{definition}
  To motivate the definition of a Green function, suppose that we want to solve the boundary-value problem for $\Delta$ on $\CC_+$. That is, given any function $f \in C(\RR)$, we want to find a harmonic function $u$ on $\CC_+$ which continuously extends to $\RR$, such that $u|_\RR = f$. If we can find a function $G$ on $\CC_+ \times \CC_+$ such that $\Delta G(z, w) = \delta_z w$ which continuously extends to $0$ on $\RR$, then one can use Stokes's theorem to see that
$$u(z) = \int_{-\infty}^\infty f(x) P(z, x) ~dx,$$
  since by definition the Poisson kernel is the normal derivative (i.e. infinitesimal of the flux) of $G$ along $\RR$. Now $G(z, w) = E(z - w)$ would suffice as such a function, except that it is nonzero at the boundary. On the other hand, $\Delta_z E(z - \overline w) = 0$ for $z, w \in \CC_+$, and by symmetry introducing this error term will cancel out the boundary term in $E(z - w)$.

    One has $\Delta_w G(z, w) = \Delta_w E(z - w) = \delta_z$ since $\overline w \notin \CC_+$, hence $z \neq w$. Moreover, $\lim_{w \to 0} G(z, w) = 0$ for $z$ fixed. We can rewrite $G$ as
$$G(z, w) = \frac{1}{2\pi} \log\left|\frac{z - w}{z - \overline w}\right|,$$
    which is clearly homogeneous: for $t > 0$, $G(tz, tw) = G(z, w)$. Since $(z-w)/(z - \overline w) \to 1$ as $z \to \infty$, $G(z, w) \to 0$.  Moreover,
$$\partial_y G(z, x + iy) = -\frac{1}{2\pi} \left(\frac{y - \Re z}{|x + iy - z|^2} - \frac{y + \Re z}{|x + iy - \overline z|^2}\right)$$
  and setting $y = 0$ we have
$$P(z, x) = \frac{\Im z}{2\pi|z-x|^2}.$$
We have the estimate $P(z, x) = O(\Im z|z|^{-2})$ for small $x$ as $z \to \infty$. In the other direction, $P$ is a nascent Dirac mass in the sense that
$$\lim_{b \to 0} P(a + ib, x) = \delta_x(a).$$ Moreover, $\Delta G(\cdot, w) = 0$ away from $w$, so commuting $\partial_y$ and $\Delta$, we see that $P(\cdot, x)$ is harmonic on $\CC_+$.

To prove the representation formula, we will need some estimates on the Green function's order of growth.
\begin{lemma}
\label{estimate on Green function}
    For every $w \in \CC_+$ there is a constant $C > 0$ such that for every $z \in \CC_+$ such that $|z|$ is large enough,
$$\frac{\Im z}{C(1 + |z|)^2} \leq |G(w, z)| \leq \frac{C \Im z}{(1 + |z|)^2}.$$
\end{lemma}
\begin{proof}
  Let $z = x + iy$. Let us Taylor expand $G(w, z)$ in $y$ at the origin, so $G(w, z) = \sum_j c_j(w, x) y^j$. Since $G = 0$ on $\RR$, $c_j = 0$. By definition of the Poisson kernel, $c_1 + P = 0$. By homogeneity,
\begin{align*}
  G(w, z) &= G\left(\frac{w}{|z|}, \frac{z}{|z|}\right) = \sum_{j=0}^\infty c_j\left(\frac{w}{|z|}, \frac{x}{|z|}\right) y^j |z|^{-j}
  \\&= -P\left(\frac{w}{|z|}, \frac{x}{|z|}\right) \frac{y}{|z|} + o\left(\frac{y}{|z|^2} \right) = -\Theta\left(P\left(\frac{w}{|z|}, 0\right)\right)
  \\&= \Theta \left(\frac{\Im z}{|1 + z|^2}\right)
\end{align*}
  where the implied constants are allowed to depend on $w$, and we have used Knuth's big-$\Theta$ notation. Indeed, if $|z|$ is large then $x/|z|$ is small, and so does not contribute meaningfully to the long-term behavior of $P$.
\end{proof}

We shall also need a representation formula for the unit disc $\DD$. We recall that the Cayley transform, which we will denote $z \mapsto z^\flat$ (with inverse $w \mapsto w^\sharp$), conformally transforms $\CC_+$ into $\DD$, and so all that we have proven about $\CC_+$ corresponds to a fact about $\DD$. The Cayley transform is given by
$$z^\flat = \frac{z - i}{z + i}.$$
Pushing forward the Poisson kernel along the Cayley transform, we arrive at the following definition.
\begin{definition}
The \dfn{Poisson kernel} for $\Delta$ on $\DD$ is defined by
$$P^\flat(z, \theta) = P(z^\sharp, (e^{i\theta})^\sharp).$$
\end{definition}
\begin{theorem}[Poisson representation formula]
Let $f \in C(\partial \DD)$ and let
$$F(z) = \int_0^{2\pi} P^\flat(z, \theta)f(e^{i\theta}) ~d\theta.$$
Then $F$ is the unique solution to the boundary-value problem for $\Delta$ with boundary condition $f$.
\end{theorem}
\begin{proof}
Since $P$ is harmonic, it follows that $P^\flat$ is harmonic as well, and
$$\int_0^{2\pi} P^\flat(z, \theta) ~d\theta = \int_{-\infty}^\infty P(z^\sharp, x) ~dx = 1.$$
So for any $r \in (0, 1)$ and any $\varepsilon > 0$, we have the estimate
\begin{align*}
  |F(re^{i\theta}) - f(e^{i\theta})| &\leq \int_0^{2\pi} P^\flat(re^{i\theta}, e^{i\eta})|f(e^{i\theta}) - f(e^{i\eta})| ~d\eta\\
    &\leq \sup_{B_\varepsilon} |f(e^{i\theta}) - f(e^{i\eta})| + \int_{B_\varepsilon^c} P^\flat(re^{i\theta}, e^{i\eta})|f(e^{i\theta}) - f(e^{i\eta})| ~d\eta\\
    &\leq \sup_{B_\varepsilon} |f(e^{i\theta}) - f(e^{i\eta})| + \sup_{B_\varepsilon^c} P^\flat(re^{i\theta}, e^{i\eta}) \int_{B_\varepsilon^c} |f(e^{i\theta}) - f(e^{i\eta})| ~d\eta
\end{align*}
where $B_\varepsilon$ is a interval in $[0, 2\pi]$ modulo $2\pi$ of radius $\varepsilon$ centered on $e^{i\theta}$. Since $f$ is continuous,
$$\lim_{\varepsilon \to 0}\sup_{B_\varepsilon} |f(e^{i\theta}) - f(e^{i\eta})| = 0.$$
On the other hand, since $P(z^\sharp, x) \to 0$ uniformly in $x$ as $z^\sharp \to \infty$, $P^\flat(z, x^\flat) \to 0$ uniformly in $x^\flat$ as $|z| \to 1$, provided that we are away from the singularity $z = x^\flat$. Thus
$$\lim_{r \to 1} F(re^{i\theta}) = f(e^{i\theta})$$
uniformly in $\theta$. So $F|_{\partial \DD} = f$. By the maximum principle, $F$ is unique.
\end{proof}
Recall the reflection principle, which says that $h$ is a harmonic function on $B(0, R) \cap \CC_+$ such that $h|_\RR = 0$, then $h$ extends to a harmonic function on $B(x, R)$ such that
$$h(z) + h(\overline z) = 0.$$
Taking $R \to \infty$, we see that the reflection principle still holds for harmonic functions on $\CC_+$ such that $h|_\RR = 0$. Let us now use the Poisson representation formula to show that the only such functions are linear.
\begin{lemma}
\label{asymptotics for the poisson kernel}
The Poisson kernel $P^\flat$ has the asymptotic expansion
$$P^\flat(z/R, e^{i\theta}) - P^\flat(\overline z/R, e^{i\theta}) = \frac{2\Im z \sin 3\theta}{R\pi} + O(R^{-2})$$
as $R \to \infty$, where $(z, \theta)$ is fixed.
\end{lemma}
\begin{proof}
We Taylor expand $P^\flat(z/R, e^{i\theta}) - P^\flat(\overline z/R, e^{i\theta})$ in $R$ at infinity (in other words, take the Maclaurin expansion in $1/R$). Clearly the zeroth-order term is $0$, and the second order term is $O(R^{-2})$. So we must only show that
$$\lim_{R \to \infty} \partial_{R^{-1}}(P^\flat(z/R, e^{i\theta}) - P^\flat(\overline z/R, e^{i\theta})) = \frac{2}{\pi} \Im z \sin 3\theta.$$

We calculate
$$P^\flat(z, e^{i\theta}) = \frac{\Im\left(\frac{z+1}{iz-i}\right)}{2\pi\left|\frac{z+1}{iz-i} - \frac{e^{i\theta} - 1}{ie^{i\theta} -i}\right|^2}
  = \frac{-\Re\left(\frac{z+1}{z-1}\right)}{2\pi|z + 1 - e^{i\theta} - 1|^2} = \frac{(1 - |z|^2)}{2\pi|z - e^{i\theta}|^2}.$$
Now
$$\partial_t (1 - t|z|^2)|_{t=0} = 0$$
and
$$\partial_t \left(\frac{1}{|tz - e^{i\theta}|^2} - \frac{1}{|t\overline z - e^{i\theta}|^2}\right)|_{t = 0} = 4e^{i(\pi - 3\theta)}\Im z.$$
Therefore
$$\lim_{R \to \infty} \partial_{R^{-1}}(P^\flat(z/R, e^{i\theta} - P^\flat(\overline z/R, e^{i\theta}))) = \Re \frac{2}{\pi}e^{i(\pi - 3\theta)}\Im z = \frac{2}{\pi} \Im z \sin 3\theta,$$
as promised.
\end{proof}
\begin{corollary}
\label{reflected harmonics are linear}
Let $h: \CC \to \RR$ be a harmonic function such that $h(z) + h(\overline z) = 0$ for every $z \in \CC$. Then there is an $A \in \RR$ such that $h(z) = A \Im z$ for every $z \in \CC$.
\end{corollary}
\begin{proof}
  $$\int_\pi^{2\pi} P(z, e^{i\theta})h(e^{i\theta}) ~d\theta = -\int_0^\pi P(\overline z, e^{i\theta})h(e^{i\theta}) ~d\theta$$
  so by the Poisson representation formula, for any $R > 0$,
  $$h(z) = \int_0^\pi (P(z/R, e^{i\theta} - P(\overline z/R, e^{i\theta}))h(Re^{i\theta})~d\theta.$$
  By Lemma \ref{asymptotics for the poisson kernel}, we have
  $$h(z) = \left(\frac{2\Im z}{\pi} + O(R^{-1})\right)\int_0^\pi \frac{h(Re^{i\theta})}{R}\sin 3\theta ~d\theta.$$
  Since the left-hand side of the resulting asymptotic expansion of $h$ does not depend on $R$, the limit of the right-hand side as $R \to \infty$ must exist, and taking
  $$A = \lim_{R \to \infty} \frac{2}{\pi}\int_0^\pi \frac{h(Re^{i\theta})}{R}\sin 3\theta ~d\theta,$$
  we conclude that $h(z) = A\Im z$. Since $h$ is real-valued, $A \in \RR$.
\end{proof}

\section{The Riesz representation formula}
Now we are ready to prove the Riesz representation formula for $\CC_+$.
\begin{theorem}[Riesz representation formula]
    \index{Riesz representation formula}
    Let $u$ be an imaginary-sublinear subharmonic function on $\CC_+$, let $\gamma$ be as in Lemma \ref{imaginary sublinear limit}, and fix any $w \in \CC_+$. Let $\mu = \Delta v$; then
    \begin{equation}\label{estimate on mu}\int_{\CC_+} \frac{\Im z \mu(z) ~dz}{(1 + |z|)^2} < \infty.\end{equation}
    Moreover $v(\cdot + iy)$ converges to a distribution $\sigma$ on $\RR$ as $y \to 0$ such that, in the sense of distributions,
\begin{equation}\label{estimate on sigma}\int_{-\infty}^\infty \frac{|\sigma(x) ~dx|}{(1 + |x|)^2} < \infty,\end{equation}
    and
\begin{equation}\label{riesz formula}u(z) = \int_{\CC_+} G(z, w) \mu(w) ~dw + \int_{-\infty}^\infty P(z, x) \sigma(x) ~dx + \gamma \Im z.\end{equation}
\end{theorem}
\begin{proof}
    Let us replace $u(z)$ with $u(z) - C - \gamma \Im z$, where $C$ is the constant appearing the definition of an imaginary-sublinear function. Then $u(z) \leq D + \Im z$, and if we take an optimal choice of $D$, then $D \leq 0$, so $u \leq 0$. After completing the proof, we can simply add a $\gamma \Im z$ back in, the $C$ having already been absorbed into the boundary term $\sigma$.

    We first prove (\ref{estimate on mu}). Recall that we have assumed that there is a $z \in \CC_+$ so that $u(z) > -\infty$. Moreover, for any $x \in \RR$, $G(z, x) = 0$ By Lemma \ref{estimate on Green function}, for some constant $B > 0$,
\begin{align*}\int_{\CC_+} \frac{\Im w}{(1 + |w|)^2} \mu(w) ~dw &\leq -B\int_{\CC_+} G(z, w) \Delta u(w) ~dw
    \\&= B \int_{\CC_+} \nabla G(z, w) \nabla u(w) ~dw + B\int_{-\infty}^\infty G(z, x) \nabla u(x) ~dx\\
    &= -B \int_{\CC_+} \Delta G(z, w) u(w) ~dw \\&= -B\int_{\CC_+} \delta_z(w) u(w) ~dw = -Bu(w) < \infty.
  \end{align*}

    Now we decompose $u$. Fix an increasing chain $K_j$ of compact sets which cover $\CC_+$, and let $\chi_j \in C^\infty_{comp}(\CC_+)$ be an increasing chain of cutoff functions which are identically $1$ on $K_j$. Then let
    $$v_j(z) = u(z) - \int_{\CC_+} G(z, w) \chi_j(w)\mu(w) ~dw.$$
\begin{lemma}
    The functions $v_j$ are subharmonic on $\CC_+$, harmonic on $K_j$, and $\leq 0$.
\end{lemma}
\begin{proof}[Proof of lemma]
    We compute
\begin{align*}
    \Delta v_j(z) &= \mu(z) - \int_{\CC_+} \Delta_z(E(z - w) - E(z - \overline w)) \chi_j(w) \mu(w) ~dw \\&= \mu(z) - \int_{\CC_+} (\delta_w(z) - \delta_{\overline w}(z) \chi_j(w) \mu(w) ~dw \\&= \mu(z)(1 - \chi_j(z)),
\end{align*}
    the $\delta_{\overline w}$ term vanishing because $z \neq \overline w$, since $\overline w \notin \CC_+$. Since $\chi_j$ is a cutoff, $\chi_j \leq 1$, so $\Delta v_j \geq 0$. On the other hand, if $z \in K_j$, then $1 - \chi_j(z) = 0$, so $\Delta v_j(z) = 0$. This proves the first two claims.

    Let $\varepsilon > 0$; we will prove that $v_j < \varepsilon$. Since $u$ is subharmonic, $\mu \geq 0$, and $\chi_j \geq 0$ while $G \leq 0$, so
    $$\int_{\CC_+} G(z, w) \chi_j(w) \mu(w) ~d\mu(w) \leq 0,$$ and we are only integrating over $w$ close to the compact set $K_j$. So we can view $w$ as essentially fixed compared to $z$, and apply the estimate $G(z, w) = O(\Im z|z|^{-2})$ to see that
    $$\lim_{z \to \infty} \int_{\CC_+} G(z, w) \chi_j(w) \mu(w) ~d\mu(w) \to 0.$$
    In particular, $\int_{\CC_+} G(z, w) \chi_j(w) \mu(w) ~d\mu(w) > -\varepsilon$ for $|z|$ large enough. Since $u \leq 0$, $v_j(z) < \varepsilon$ for $z$ large enough, hence for any $z$ by the maximum principle. Therefore $v_j < 0$.
\end{proof}
    The $v_j$ form an increasing sequence which is bounded above, so converge to a limit $u_1 \leq 0$. Moreover, $\Delta v_j \to 0$ pointwise, so $u_1$ is harmonic. Meanwhile, $\chi_j \to 1$ pointwise, so if we let
$$u_2(z) = \int_{\CC_+} G(z, w) \mu(w) ~dw,$$
    we arrive at the decomposition $u = u_1 + u_2$, $\Delta u_1 = 0$. We will view $u_1$ as the ``boundary part" of $u$ and $u_2$ as the ``subharmonic part" of $u$.

    We now show that the subharmonic part of $u$ does not contribute to its boundary value.
\begin{lemma}
    In the sense of distributions,
$$\lim_{y \to 0} u_2(\cdot + iy) = 0.$$
\end{lemma}
\begin{proof}[Proof of lemma]
    Let $\varphi \in C^\infty_{comp}(\RR)$ be a test function, $\ch \supp \varphi = [a, b]$. We must show that the limit
$$\lim_{y \to 0} \int_{-\infty}^\infty u_2(x + iy) \varphi(x) ~dx = \lim_{y \to 0} \int_{\CC_+} \mu(w) \int_{-\infty}^\infty G(x + iy, w) \varphi(x) ~dx ~dw = 0.$$
    Now $G(x + iy, w)$ vanishes for fixed $w,x$ as $y \to 0$, and $G(x + iy, w) \varphi(x) = O(\Im w|w|^{-2})$ at infinity by Lemma \ref{estimate on Green function}. If we can show that this bound is valid on compact sets as well, then we will have
$$\int_{\CC_+} \mu(y, w) H(y, w) ~dw \leq \int_{\CC_+} \frac{\Im z \mu(z) ~dz}{(1 + |z|)^2} < \infty$$
    by (\ref{estimate on mu}), whence
$$\lim_{y \to 0} \int_{-\infty}^\infty u_2(x + iy) \varphi(x) ~dx = \lim_{y \to 0} \int_{\CC_+} \mu(y, w) H(y, w) ~dw = 0$$
    by the dominated convergence theorem.

    We now define
$$F(w) = \int_{-\infty}^\infty E(w - x)\varphi(x) ~dx.$$
    Now $E$ is continuous away from $0$, but $\Im w > 0$, so the integrand is continuous. The integral is formally taken over $\RR$, but is actually being taken over $[a, b]$, so the integrand is integrable; hence $F$ is continuous. Since $\Delta E(w - x) = 0$ for $\Im w > 0$, $F$ is harmonic. But $F$ is a convolution, so if $P$ is any linear differential operator in $\Re w$, $PF = E * P\varphi$, which is continuous since $\varphi$ is smooth. Therefore $PF$ is locally bounded. In particular,
$$\partial_{\Im w}^2 F(w) = \Delta F(w) - \partial_{\Re w}^2 F(w) = -\partial_{\Re w}^2 F(w)$$
    which is locally bounded. So $PF$ is locally bounded for any linear differential operator whatsoever. In particular, $F \in C^{Lip}_{loc}(\CC_+)$. Because $F(w)$ is bounded for a fixed $\Im w$, since $F$ is continuous and $\Re a$ ranges over the compact set $[a, b]$, we have $F(w) = O(\Im w)$. That is,
$$\int_{-\infty}^\infty G(x + iy, w) \varphi(x) ~dx = F(\overline w + iy) - F(w + iy) = O(\Im w)$$
    for bounded $y$. In particular, the integral is bounded on any compact set in $w$. Therefore
$$\int_{-\infty}^\infty G(x + iy, w) \varphi(x) ~dx = O\left(\frac{|\Im w|}{(1 + |w|^2)}\right).$$
    By the remarks at the start of this proof, the lemma follows.
\end{proof}
    We now will construct a representation formula for the boundary part.
\begin{lemma}
  \label{approximate sigma rep}
    For any $\varepsilon > 0$ and $y > \varepsilon$,
$$u_1(x + iy) = \int_{-\infty}^\infty P(x + i(y-\varepsilon), t) u_1(t + i\varepsilon) ~dt.$$
\end{lemma}
\begin{proof}[Proof of lemma]
    Let $\psi_j$ be an increasing sequence of cutoff functions on $\RR$ which are identically $1$ on $[-j, j]$. Then
$$h_j(z) = u_1(z + i\varepsilon) - \int_{-\infty}^\infty P(z, x)\psi_j(x)u_1(x + i\varepsilon) ~dx.$$
    Then for $z \in \CC_+$,
$$\Delta h_j(z) = -\Delta \int_{-\infty}^\infty P(z, x)\psi_j(x) u_1(x + i\varepsilon) ~dx = -\int_{-\infty}^\infty \Delta_z P(z, x) \psi_j(x) u_1(x + i\varepsilon) ~dx = 0$$
    since $P(\cdot, x)$ is harmonic on $\CC_+$. So the $h_j$ are harmonic on $\CC_+$. Since $P$ is a nascent Dirac mass, if $\Im z = 0$, then
$$h_j(z) = u_1(z + i\varepsilon) - \int_{-\infty}^\infty \delta_x(z) \psi_j(x) u_1(x + i\varepsilon) ~dx = u_1(z + i\varepsilon)(1 - \psi_j(x)) \leq 0$$
    since $u_1 \leq 0$. On the other hand, $P(z, x)$ is small when $|z|$ is large; so
    \begin{align*}\limsup_{z \to \infty} h_j(z) &\leq -\liminf_{z \to \infty} \int_{-\infty}^\infty P(z, x) \psi_j(x) u_1(x + i\varepsilon) ~dx
      \\&= -\int_{-\infty}^\infty \liminf_{z \to \infty} P(z, x) \psi_j(x) u_1(x + i\varepsilon) ~dx = 0.\end{align*}
    Therefore $h_j|_{\partial \CC_+} \leq 0$, so by the maximum principle, $h_j \leq 0$.

    Therefore the $h_j$ increase to a harmonic function $h$, which by the reflection principle, extends uniquely to $\CC$ and satisfies $h(z) + h(\overline z) = 0$. By Corollary \ref{reflected harmonics are linear}, we can find a $A \in \RR$ such that $h(z) = A\Im z$. By our assumption that $C = \gamma = 0$, it follows that $A = 0$, so $h = 0$. But
    $$u_1(z + i\varepsilon) = \lim_{j \to \infty} h_j(z) + \int_{-\infty}^\infty P(z, x)\psi_j(x)u_1(x + i\varepsilon)~dx = \int_{-\infty}^\infty P(z, x)u_1(x + i\varepsilon) ~dx.$$
    The lemma then follows by changing variables.
\end{proof}
    Since $\gamma = 0$ and $u \leq 0$, for every $\delta > 0$ and every $x \in \RR$, we can choose $y > 0$ so large that $u(x + iy) > -\delta y$. Since $G$ vanishes at infinity, so does $u_2$, so if $y$ is chosen even larger still, we can arrange that $u_1(x + iy) > -\delta y$. By Lemma \ref{approximate sigma rep} with a change of variable in $y$,
$$-\int_{-\infty}^\infty \frac{u_1(x + i\varepsilon)}{|x + iy - t|^2} ~dt < \frac{\pi \delta y}{y - \varepsilon}.$$
    Taking $\varepsilon$ and $R$ large enough, we can find a constant $B > 0$ such that
$$\left(\int_{-\infty}^{-R} + \int_R^\infty\right) \frac{-u_1(x + i\varepsilon)}{x^2} ~dx < B \delta.$$
    This estimate is uniform in $\varepsilon$, and $u_1(\cdot + i\varepsilon)$ is bounded on $[-R, R]$ since it is harmonic; these bounds are also uniform in $\varepsilon$. Certainly any test function on $\RR$ decays faster than $x^2$, so it follows that $u_1(\cdot + i\varepsilon)$ is uniformly bounded in $\varepsilon$ as a functional acting on $C^\infty_{comp}(\RR)$. So by the Banach-Alaoglu theorem, the space of all such distributions $u_1(\cdot + i\varepsilon)$ is weakstar compact, and so there is a weak limit $\sigma$ as $\varepsilon \to 0$. Since $\sigma$ satisfies the same bounds as $u_1(\cdot + i\varepsilon)$, (\ref{estimate on sigma}) follows, and, passing to the limit in Lemma \ref{approximate sigma rep},
$$u_1(z) = \int_{-\infty}^\infty P(z, x) \sigma(x) ~dx.$$
    Plugging the representation formulae for $u_1$ and $u_2$ back into the decomposition $u = u_1 + u_2$, we complete the proof.
\end{proof}



\section{Proof of Titchmarsh's theorem}
\begin{lemma}
    \label{gamma in mean}
    Let $u$ be an imaginary-sublinear subharmonic function on $\CC_+$, and let $\gamma$ be as in Lemma \ref{imaginary sublinear limit}. Then
$$\lim_{t \to \infty} \frac{u(t(x + iy))}{t} = \gamma y$$
    in $L^1_{loc}(\overline{\CC_+})$.
\end{lemma}
\begin{proof}
    Fix a compact set $K \subset \overline{\CC_+}$. If $u$ is a constant, then both sides of the claimed equation are $0$, so by linearity we may subtract the constant $C$ that appears in the definition of an imaginary-sublinear function from $u$, and so assume without loss of generality that $C = 0$. Then $u(t(x+iy)) \leq \gamma y$, so it suffices to prove that
$$\lim_{t \to \infty} \int_K \frac{u(tz)}{t} - \gamma \Im z ~dz = 0$$
    to show convergence in $L^1(K)$.

    By the Riesz representation formula,
$$\frac{u(tz)}{t} - \gamma \Im z = \int_{-\infty}^\infty P(z, x) \sigma(x) ~dx + \int_{\CC_+} G(z, w) \mu(w) ~dw.$$
    We now let
    $$f(t, x) = \frac{1}{t^2} \int_K P\left(t, \frac{x}{t}\right) ~dx.$$
    Since $t > 0$ and $P \geq 0$, it follows that $f \geq 0$. Moreover, by Lemma \ref{estimate on Green function}, there is a $B > 0$ which only depends on $K$ such that for any $x/t$ large enough,
    $$f(t, x) \leq \frac{1}{t^2} \sup_{x \in K} P\left(t, \frac{x}{t}\right) \leq \frac{B}{t^2(1 + |t|)^2}.$$
    Moreover, this estimate on $f$ is trivial if we bound $x/t$ from above; so we can take $B$ to be large enough that this estimate is valid for any $x/t \in \RR$. Similarly, we take
    $$g(t, w) = -\frac{1}{t} \int_K G(z, w) ~dw \leq \frac{D \Im z}{t^2(1+|z/t|)^2}.$$
    Thus $g \geq 0$, and
    $$\int_K \frac{u(tz)}{t} - \gamma \Im z ~dz = \int_{-\infty}^\infty f(t, x) \sigma(x) ~dx - \int_{\CC_+} g(t, w) \mu(w) ~dw.$$
    Since we have estimated $f$ and $g$, we can use (\ref{estimate on sigma}) and (\ref{estimate on mu}) to apply the dominated convergence theorem. Clearly the dominators of $f$ and $g$ converge to $0$ pointwise as $t \to \infty$, so the integrals against $\sigma$ and $\mu$ converge to $0$.
\end{proof}

\begin{proof}[Proof of Titchmarsh's theorem]
Fix a distribution $\mu$ on $\RR$ with $\ch \supp \mu = [a, b]$. By the Paley-Wiener theorem, $\hat \mu$ is an entire function and $\log |\hat \mu|$ is an imaginary-sublinear subharmonic function. Viewing $\log |\hat \mu|$ as a function on $\CC_+$ and taking $\gamma$ as in Lemma \ref{imaginary sublinear limit}, we have $\gamma = b$, since the estimate in the Paley-Wiener theorem is sharp. On the other hand, viewing $\log |\hat \mu|$ as a function on $\CC_-$, we have $\gamma = a$. Let $Z$ be the multiset of zeroes of $\hat \mu$ with repetition.

Let $h(t) = at$ for $t < 0$, $h(t) = bt$ for $t > 0$. Then $h'$ is a rescaled Heaviside function, so $h'' = (b-a)\delta_0$. By Lemma \ref{gamma in mean}, we have
$$\lim_{t \to \infty} \frac{\log |\hat \mu(t\zeta)|}{t} = h(\Im \zeta)$$
in $L^1_{loc}$. Taking the Laplacian of both sides, we see that
$$\lim_{t \to \infty} \frac{2\pi}{t} \sum_{z \in Z} \delta_{z/t}(w) = (b-a)\delta_0(\Im w)$$
in the sense of distributions.

Integrating both sides,
$$\int_{D(0, 1)} \lim_{t \to \infty} \frac{2\pi}{t} \sum_{z \in Z} \delta_{z/t} = \int_{D(0, 1)} (b-a)\delta_0(\Im w) ~dw = \int_{-1}^1 b - a ~dw = 2(b - a).$$
Rewriting the left-hand side, we have
$$\lim_{R \to \infty} \frac{2\pi }{R} \int_{D(0, R)} \sum_{z \in Z} \delta_z = \lim_{R \to \infty} \frac{2\pi N(R)}{R}.$$
Dividing both sides by $2\pi$, we see that
$$\lim_{R \to \infty} \frac{N(R)}{R} = \frac{b - a}{\pi}.$$
This proves Titchmarsh's theorem.
\end{proof}


\chapter{Compactly supported potentials}
\begin{quote}
  ``EVERY MORNING I WAKE UP AND OPEN PALM SLAM A BOOK ONTO THE DESK. IT'S MATHEMATICAL THEORY OF SCATTERING RESONANCES BY DYATLOV AND ZWORSKI AND RIGHT THEN AND THERE I START DOING THE MOVES ALONGSIDE WITH THE AUTHORS, DYATLOV AND ZWORSKI. I DO EVERY PROOF AND I DO EVERY PROOF HARD. MAKIN WHOOSHING SOUNDS WHEN I MEROMORPHICALLY CONTINUE THE RESOLVENT INTO THE LOWER HALF PLANE OR EVEN WHEN I MESS UP TECHNIQUE. NOT MANY CAN SAY THEY CAN BOUND THE NUMBER OF RESONANCES OF A BLOCK BOX OPERATOR. I CAN. I SAY IT AND I SAY IT OUTLOUD EVERYDAY TO PEOPLE IN MY COLLEGE CLASS AND ALL THEY DO IS PROVE PEOPLE IN COLLEGE CLASS CAN STILL BE IMMATURE JEKRS. AND IVE LEARNED ALL THE LEMMAS AND IVE LEARNED HOW TO MAKE MYSELF AND MY APARTMENT LESS LONELY BY SHOUTING EM ALL. 2 HOURS INCLUDING WIND DOWN EVERY MORNIng"

  -Analytic Memes for Estimate-Loving Teens
\end{quote}

Recall that if $A$ is a quantum observable, then the spectrum of $A$ consists of the possible values $\lambda^2$ that the observable can take when a state $\psi$ is measured. Moreover, $\psi$ is a pure state for measurement $\lambda^2$ (i.e. an observer is guaranteed to measure $\psi$ in state $\lambda^2$) if and only if $\psi$ solves the eigenvalue equation $A\psi = \lambda^2 \psi$. So we are especially interested in the eigenvalue equation $(A - \lambda^2)\psi = 0$.

Let $V \in L^\infty_{comp}(\RR \to \RR)$. We view $V$ as a potential, and define the Hamiltonian
$$H_V = D_x^2 + V$$
where $D_x = -i\partial_x$. We can think of $D_x^2$ as the one-dimensional Laplacian $-\Delta_x$, or equivalently as the momentum observable. Since $\supp V$ is compact, a solution $\psi$ of the time-independent Schrödinger equation
$$(H_V - \lambda^2)\psi = 0$$
with energy $\lambda^2$ has the form
$$\psi(x) = Ae^{i\lambda x} + Be^{-i\lambda x},$$
for $x$ far from the support of $\supp V$ -- so $\psi$ consists of plane waves of frequency $\lambda$. (This is our reason for writing the eigenvalue as $\lambda^2$.) Clearly $\psi$ is not normalizable (i.e. $\psi \notin L^2(\RR \to \CC)$), essentially because $H_V$ has continuous spectrum. On the other hand, since $V$ is bounded, it is possible for a particle to tunnel through the barrier $\supp V$. This mathematically manifests as relations between the coefficients $A,B$: scattering.

More generally, we will study the inhomogeneous Schrödinger equation $(H_V - \lambda^2)\psi = f$, for $f \in L^2_{comp}(\RR)$. By definition, the resolvent $R_V(\lambda)$ tells us how to solve the Schrödinger equation: it is an operator $L^2_{comp}(\RR) \to L^2_{loc}(\RR)$ satisfying the equation
$$(H_V - \lambda^2)R_V(\lambda)f = f.$$

\section{Meromorphic continuation of the resolvent}
Viewing the resolvent $R_V$ as a function valued in $B(L^2_{comp}(\RR) \to L^2_{loc}(\RR))$, we want to understand the relationship between the poles of the resolvent $R_V$ function and the scattering behavior of $H_V$.

We first show that $R_V(\lambda)$ is a meromorphic function in $\lambda$. To do this, we consider the \dfn{resolvent function}, $(x, y) \mapsto R_V(\lambda; x, y)$, defined by
$$R_V(\lambda)f(x) = \int_{-\infty}^\infty f(y) R_V(\lambda; x, y) ~dy.$$
Since $R_V(\lambda)$ is a right inverse to $H_V - \lambda^2$, we can appropriately view the resolvent function as the Green function of $H_V - \lambda^2$ on $\RR$.

\begin{definition}
The \dfn{free resolvent} is $R_0(\lambda)$.
\end{definition}
When we study the free resolvent, we are just solving the Schrödinger equation for a free particle, so
$$R_0(\lambda; x, y) = \frac{i}{2\lambda} e^{i\lambda|x - y|}.$$
In fact,
$$2i\lambda (H_0 - \lambda^2)R_0(\lambda; x, 0) = (\Delta_x + \lambda^2) e^{i\lambda|x|} = -\lambda^2 e^{i\lambda|x|} + \lambda^2 e^{i\lambda|x|} = -2i\delta_0(x).$$
Now if $\Im \lambda \leq 0$, then we have no decay in the plane waves $e^{i\lambda|x - y|}$ and so $R_0$ is not a bounded operator on $L^2$. The spectrum of $H_0 = -\Delta$ consists of $\RR_+$, so if $\Im \lambda > 0$, it follows from the spectral radius theorem that
$$||R_0(\lambda)||_{L^2 \to L^2} = \sup_{\mu \in \RR_+} \frac{1}{|\lambda^2 - \mu|} = \frac{1}{d(\RR_+, \lambda^2)}.$$
Therefore $R_0(\lambda)$ is bounded on $L^2$ if $\Im \lambda > 0$. In particular, $R_0$ is holomorphic on $\CC_+$.

\begin{lemma}
  Let $\rho \in C^\infty_{comp}(\RR)$ be such that $\rho V = V$. Then
$$(1 + VR_0(\lambda))^{-1}\rho = \rho(1 + VR_0(\lambda)\rho)^{-1}.$$
\end{lemma}
\begin{proof}
  We have
  $$R_V(\lambda) = R_0(\lambda)(1 + VR_0(\lambda))^{-1}.$$
  Suppose $\rho$ is supported on $[-R, R]$. Then
  $$\supp (1 + VR_0(\lambda))\rho \subseteq [-R, R]$$
  so it follows that
  $$\rho(1 + VR_0(\lambda))\rho = (1 + VR_0(\lambda))\rho.$$
  Inverting the operator $1 + VR_0(\lambda)$, the theorem follows.
\end{proof}

\begin{theorem}
\label{meromorphic continuation with compact support}
The family of operators
$$R_V: L^2_{comp} \to L^2_{loc}$$
is meromorphic on $\CC$.
\end{theorem}
\begin{proof}
By the definition of the free resolvent, we have
$$(H_V - \lambda^2)R_V(\lambda) = 1 + VR_0(\lambda).$$
If $\Im \lambda$ is very large, then $||VR_0(\lambda)||_{L^2 \to L^2}$ is very small, so the formal Neumann series computation
$$(1 + VR_0(\lambda))^{-1} = \sum_{k=0}^\infty (-VR_0(\lambda))^k$$
is valid. Thus $(1 + VR_0(\lambda))^{-1}$ is a holomorphic family of operators on $L^2$, for $\Im \lambda$ sufficiently large.

With $\rho$ a cutoff such that $\rho V = V$, $VR_0(\lambda)\rho = V\rho R_0(\lambda)\rho$ is a meromorphic family of compact operators $L^2 \to H^2_{comp}$, so $1 + VR_0(\lambda)\rho$ is a meromorphic family of Fredholm operators. It follows that
$$R_V(\lambda)\rho = R_0(\lambda)\rho(1 + VR_0(\lambda)\rho)^{-1}$$ is a meromorphic family of operators $L^2 \to L^2_{loc}$. Taking $R \to \infty$, the theorem follows.
\end{proof}

Since we have assumed that $V$ and $f$ have compact support, if $|x|$ is large, then
$$(H_0 - \lambda^2)R_V(\lambda)f(x) = 0.$$
So if we let $\psi = R_V(\lambda)f$, it follows from some calculus that we can find constants $A_+, B_-$ such that for all $x$ large enough,
$$\psi(x) = a_+e^{i\lambda x} + b_+e^{-i\lambda x}.$$
Similarly, we can find $A_-, B_+$ such that for every $x$ with $-x$ large enough,
$$\psi(x) = a_-e^{-i\lambda x} + b_-e^{i\lambda x}.$$
In case $\lambda > 0$, we can view waves of the form $C e^{i\lambda x}$ as ``moving to the right with frequency $\lambda$" and waves of the form $C e^{-i\lambda x}$ as ``moving to the left with frequency $\lambda$." Of course, this distinction makes sense for any fixed $\lambda \in \CC$, even if $\lambda$ is not a positive real number. So we make the following definition.
\begin{definition}
An \dfn{incoming wave} is a linear combination of waves of the form $e^{i\lambda x}$ for $x < 0$ and $e^{-i\lambda x}$ for $x > 0$. An \dfn{outgoing wave} is a linear combination of waves of the form $e^{-i\lambda x}$ for $x < 0$ and $e^{i\lambda x}$ for $x > 0$.
\end{definition}

Let us classify which outgoing solutions can be solutions of the eigenvalue equation
$$H_Vu = \lambda^2u.$$
Our goal is the following theorem.
\begin{theorem}
\label{real poles are zero}
Let $\lambda \in \RR$ and suppose $\lambda \neq 0$. Then $\lambda$ is not a pole of $R_V$.
\end{theorem}
\begin{lemma}
\label{laurent expansion of the resolvent}
Suppose that $R_V$ has a pole at $\lambda_0 \neq 0$. Suppose that
$$R_V(\lambda) = \frac{P_N}{(\lambda - \lambda_0)^N} + \dots + \frac{P_1}{\lambda - \lambda_0} + Q(\lambda)$$
is the Laurent expansion of $R_V$ at $\lambda_0$ (so $Q$ is holomorphic). Then for every $f \in L^2_{comp}$, $u = P_Nf$ is an outgoing solution of the eigenvalue equation $H_Vu = \lambda_0^2u$.
\end{lemma}
\begin{proof}
We have
\begin{align*}
  (\lambda - \lambda_0)^N(H_V - \lambda^2)R_V(\lambda)f &= (\lambda - \lambda_0)^N (H_V - \lambda^2)\left(\frac{u}{(\lambda - \lambda_0)^N} + O((\lambda - \lambda_0)^{1-N}f)\right)\\
  &= (H_V - \lambda^2)u + O((\lambda - \lambda_0)f)
\end{align*}
and taking $\lambda \to \lambda_0$ we see that $(H_V - \lambda_0^2)u = 0$. Therefore $u$ solves the eigenvalue equation. To see that $u$ is outgoing, choose $\rho \in C^\infty_{comp}(\RR)$ such that $\rho V = V$. Then
$$R_0(\lambda)\rho(1 + VR_0(\lambda)\rho)^{-1} = R_V(\lambda)\rho$$
so $(1 + VR_0(\lambda)\rho)^{-1}$ is meromorphic with a pole of order $\leq N$ at $\lambda_0$. Write
$$(1 + VR_0(\lambda)\rho)^{-1} = \frac{\tilde P_N}{(\lambda - \lambda_0)^N} + \dots + \frac{\tilde P_1}{\lambda - \lambda_0} + \tilde Q(\lambda)$$
for the Laurent expansion of $(1 + VR_0(\lambda)\rho)^{-1}$ at $\lambda_0$. Then
$$P_N(\rho(L^2(\RR))) = R_0(\lambda) \rho \tilde P_N(L^2(\RR)) \subseteq R_0(\lambda)(L^2_{comp}(\RR)).$$
Since $\rho$ was arbitrary, it follows that $u \in R_0(\lambda)(L^2_{comp}(\RR))$. Since $R_0$ is the outgoing free resolvent, $u$ is outgoing.
\end{proof}

\begin{lemma}
\label{no outgoing solutions, part 1}
Let $\lambda \in \RR$ and suppose $\lambda \neq 0$. Suppose $H_Vu = \lambda^2u$. If we expand $u$ as
$$u(x) = \begin{cases}
  A_+ e^{i\lambda x} + B_-e^{-i\lambda x}, &x \gg 0,\\
  A_- e^{i\lambda x} + B_+e^{-i\lambda x}, &x \ll 0,
\end{cases}$$
then $|A_+|^2 + |B_+|^2 = |A_-|^2 + |B_-|^2$.
\end{lemma}
\begin{proof}
Since $\lambda \in \RR$, $H_V - \lambda^2$ is a real operator, so $(H_V - \lambda^2) \overline u = 0$. Therefore the Wronskian
$$\begin{vmatrix}
u&\overline u\\u' & \overline u'
\end{vmatrix} = -2i\lambda \begin{cases}
|A_+|^2 + |B_-|^2, &x \gg 0,\\
|A_-|^2 + |B_+|^2, &x \ll 0,
\end{cases}$$
is constant so $|A_+|^2 + |B_-|^2 = |A_-|^2 + |B_+|^2$.
\end{proof}

We will need the following lemma to show that certain solutions are unique. It guarantees that a wave's transmission cannot be completely intercepted by a bounded potential, and so must scatter. Note that this stands in contrast to, say, the infinite potential well that is taught in introductory quantum mechanics class.
\begin{lemma}
\label{bounded potentials must scatter}
Let $u \in L^\infty(\RR)$, $W \in L^\infty(\RR)$, and $\supp u \subseteq [0, \infty)$. If $H_Wu = 0$, then $u = 0$.
\end{lemma}
\begin{proof}
Let $h \in (0, 1)$ and $v(x) = e^{-x/h}u(x)$. Since $u$ is supported in the right half-ray, it follows that $v \in L^\infty(\RR)$ and that $v$ is rapidly decaying. Therefore $v \in L^2(\RR)$. For $\xi \in \RR$, $|i - h\xi|^2 \geq 1$. Therefore
$$||v||_{L^2(\RR)}^2 = ||\hat v||_{L^2(\RR)}^2 \leq \int_{-\infty}^\infty |(h\xi - i)^2\hat v(\xi)|^2 ~d\xi.$$
The inverse Fourier transform of $(h\xi - i)^2 = h^2\xi^2 - 2ih\xi - 1$ is
$$h^2D^2_x - 2ihD_x - 1 = h^2 e^{-x/h} D^2 e^{x/h}.$$
So by the Plancherel formula,
\begin{align*}
  ||v||_{L^2(\RR)} &\leq \int_{-\infty}^\infty |e^{-x/h}(hD_x)^2 u(x)|^2 ~dx
  \\&= h^2 \int_{-\infty}^\infty |e^{-x/h}W(x)u(x)|^2 ~dx
  \\&\leq h^2 ||W||_{L^\infty(\RR)} ||v||_{L^2(\RR)}.
\end{align*}
Taking $h \to 0$, we see that $v = 0$, but $e^{-x/h} \neq 0$, so $u = 0$.
\end{proof}

\begin{corollary}
\label{no outgoing solutions, part 2}
Let $\lambda \in \RR$ and suppose $\lambda \neq 0$. If $H_Vu = \lambda^2u$, then $u$ is not outgoing.
\end{corollary}
\begin{proof}
Let $A_\pm,B_\pm$ be as in Lemma \ref{no outgoing solutions, part 1}. If $u$ is outgoing and not compactly supported, then $|A_+|^2 + |B_-|^2 > 0$ while $|A_-|^2 + |B_+|^2 = 0$. Therefore $u$ is compactly supported, but after translating the support of $u$, we may assume that $\supp u \subseteq [0, \infty)$. But then Lemma \ref{bounded potentials must scatter} implies that $u = 0$.
\end{proof}

\begin{proof}[Proof of Theorem \ref{real poles are zero}]
Suppose that $\lambda$ is a pole. By Lemma \ref{laurent expansion of the resolvent}, there is an outgoing function $u$ such that $H_Vu = \lambda^2u$. This contradicts Corollary \ref{no outgoing solutions, part 2}.
\end{proof}


\section{The transmission coefficients}
We construct the eigenfunctions for the continuous spectrum of $H_V$; namely, for $x \in \RR$, $\lambda \in \RR$, $\lambda \neq 0$,
$$e_\pm(x, \lambda) = e^{\pm i\lambda x} - R_V(\lambda)(V(x)e^{\pm i\lambda x}).$$
By Theorem \ref{real poles are zero}, $\lambda$ is not a pole of $R_V$, so this definition makes sense.
\begin{lemma}
The function $e_\pm(\cdot, \lambda)$ is the unique eigenfunction of $H_V$ with eigenvalue $\lambda^2$ which is equal to $e^{\pm i\lambda x}$ modulo outgoing terms.
\end{lemma}
\begin{proof}
Clearly $e_\pm$ is an eigenfunction of $H_V$ with eigenvalue $\lambda^2$. If $\rho V = V$, then
$$R_V(\lambda)\rho = R_0(\lambda)\rho(1 + VR_0(\lambda)\rho)^{-1}.$$
So
$$R_V(\lambda)(Ve^{i\pm x}) = R_V(\lambda)(\rho Ve^{i\pm x}) = R_0(\lambda)\rho(1 + VR_0(\lambda)\rho)^{-1} Ve^{i\pm x}$$
which lies in the image of the outgoing resolvent $R_0(\lambda)$, so is outgoing. Therefore $e_\pm$ is equal to $e^{\pm i\lambda x}$ modulo outgoing terms.

Assume that $\tilde e_\pm(x, \lambda) = e^{\pm i\lambda x} + g(x, \lambda)$ is also an eigenfunction with $g(\cdot, \lambda)$ outgoing. Then
$$\tilde e_\pm(x, \lambda) - e_\pm(x, \lambda) = g(x, \lambda) - R_V(\lambda)(V(x)e^{\pm i\lambda x})$$
is also an eigenfunction, which is outgoing. Then, by Corollary \ref{no outgoing solutions, part 2}, $g(x, \lambda) - R_V(\lambda)(V(x)e^{\pm i\lambda x}) = 0$.
\end{proof}

In the trivial case $V = 0$, we have $e_\pm(x, \lambda) = e^{\pm i\lambda x}$. In this case, the Wronskian $W(x, \lambda)$ of $e_\pm(x, \lambda)$ is given by $-2i\lambda$. We interpret this as meaning that the entirety of an incoming wave is transmitted through $\ch \supp V = \emptyset$. But even if $V$ is nonzero, $H_V$ does not have a first-order term, so it follows from Abel's Wronskian formula that the function $W(\cdot, \lambda)$ is a constant.

\begin{definition}
The \dfn{transmission coefficient} $T$ is defined by
$$T(\lambda) = \frac{iW(\lambda)}{2\lambda}.$$
\end{definition}

We now define
$$\phi_\pm(x, \lambda) = \frac{e_\pm(x, \mp \lambda)}{T(\pm \lambda)}.$$

\begin{lemma}
\label{construction of intertwining, part 1}
For every $\lambda \in \RR$, the $\phi_\pm$ are functions on $\RR$ such that $(D_x^2 + V - \lambda^2)\phi_\pm = 0$ and such that $\phi_\pm(x, \lambda) = e^{-i\lambda x}$ for $\pm x$ large enough.
\end{lemma}
\begin{proof}
To show that the $\phi_\pm$ are functions, we must show that they do not have poles.

First we rule out the possibility that $T(\pm \lambda) = 0$. If $\lambda \neq 0$, then the $e_\pm(\lambda)$ are linearly independent %TODO
so $T(\lambda) \neq 0$.

On the other hand, if the $e_\pm(0)$ are linearly dependent, then there is an $m \in \NN$ such that the $\phi_\pm$ have a pole of order of $m$ at $0$. So
$$\tilde \phi_\pm(x) = \lim_{\lambda \to 0} \lambda^m\phi_\pm(x, \lambda)$$
is a holomorphic function on a ball close to $0$, and $\tilde \phi_\pm \in \ker \tilde \phi_\pm$. Since, for $\pm x$ large enough, $\tilde \phi_\pm(x) = 0$, we conclude that $\tilde \phi_\pm = \phi_\pm$, a contradiction. So the $e_\pm(0)$ are linearly independent, and $\phi_\pm$ is defined on all of $\RR$. Clearly we have $(D_x^2 + V - \lambda^2)\phi_\pm = 0$, so we are done.
\end{proof}

We now put
$$w_\pm(x, y) = \frac{1}{2\pi} \int_{-\infty}^\infty \phi_\pm(x, \lambda)e^{i\lambda y} ~d\lambda$$
By Lemma \ref{construction of intertwining, part 1}, $A_\pm$ solves the equation
\begin{align*}
  (D_x^2 + V(x))w_\pm(x, y) &= D_y^2(x, y),\\
  w_\pm(x, y) &= \delta(x - y) &\pm x \gg 0.
\end{align*}
Rewriting the first equation as
$$(D_x^2 - D_y^2)w_\pm(x, y) = V(x)w_\pm(x, y)$$
we see that the $w_\pm$ are the unique solutions to these equations, by uniqueness for the wave equation (since $D_x^2 - D_y^2$ is the wave operator), c.f. \cite[\S2.4.3]{evans10}.

\begin{lemma}
\label{construction of intertwining, part 2}
Let $[a, b] = \ch \supp V$. Then there are distributions $X,Y$ such that if $x \gg 0$,
$$\partial_yw_-(x, y) = X(y - x) + Y(x + y),$$
such that $\supp X \subseteq [-2(b-a), 0]$ and $\supp Y \subseteq [2a, 2b]$.
\end{lemma}
\begin{proof}
For $x \gg 0$,
$$(D_x^2 - D_y^2)\partial_yA_-(x, y) = 0$$
so $\partial_yA_-(x, y)$ solves the wave equation where $x$ is the time variable and $y$ is the space variable. The lemma then follows from the causality properties of the wave equation, c.f. \cite[\S2.4.1a]{evans10}, where $x = a$ is the inital-time slice of $\RR^2$, the future is $x > a$, and the distributions $X,Y$ must have support of measure at most $2|\ch \supp V|$.
\end{proof}



\section{The scattering matrix}
We now define a matrix which encodes how a wave transforms as it passes through $\ch \supp V$.

Let $A_\pm$, $B_\pm$ be the incoming and outgoing coefficients, as in Lemma \ref{no outgoing solutions, part 1}, and let $X, Y$ be as in Lemma \ref{construction of intertwining, part 2}. The scattering matrix maps the incoming coefficients to the outgoing coefficients. Since $X$ is a compactly supported distribution, the Paley-Weiner theorem implies that the Fourier transform $\hat X$ is an entire function, and in particular $1/\hat X$ is a meromorphic function.
\begin{definition}
The \dfn{scattering matrix} is the operator $S(\lambda): \CC^2 \to \CC^2$ defined by
$$S(\lambda)\begin{bmatrix}A_-\\B_-\end{bmatrix} = \begin{bmatrix}A_+\\B_+\end{bmatrix}.$$
\end{definition}
\begin{theorem}
  The scattering matrix $S$ is a meromorphic family of operators on $\CC$ such that
  $$S(\lambda) = \begin{bmatrix}\frac{i\lambda}{\hat X(\lambda)} & \frac{\hat Y(\lambda)}{\hat X(\lambda)}\\ \frac{\hat Y(-\lambda)}{\hat X(\lambda)} & \frac{i\lambda}{\hat X(\lambda)}\end{bmatrix}.$$
  If $\lambda \in \CC_+$ is a pole of $S$ and $\Im \lambda > 0$, then $\lambda^2$ is an eigenvalue of $H_V$.
\end{theorem}
\begin{proof}
  By Lemma \ref{construction of intertwining, part 1}, we may write
\begin{align*}
  \phi_-(x, \lambda) &= \begin{cases}
  A(\lambda)e^{i\lambda x} + B(\lambda)e^{-i\lambda x}, &x \gg 0\\
  e^{-i\lambda x}, &x \ll 0,
\end{cases}\\
  \phi_+(x, \lambda) &= \begin{cases}
  e^{-i\lambda x}, &x \gg 0\\
  C(\lambda) e^{i\lambda x} + D(\lambda)e^{-i\lambda x}, &x \ll 0.
\end{cases}
\end{align*}
  By definition of the scattering matrix, we have
  $$\begin{bmatrix}
  -\frac{A(\lambda)}{B(\lambda)C(\lambda)} &\frac{A(\lambda)}{B(\lambda)}\\
  \frac{B(\lambda)D(\lambda) - 1}{B(\lambda)C(\lambda)} & \frac{1}{B(\lambda)}
  \end{bmatrix}.$$
  Now
  $$\partial_yw_-(x, y) = \frac{i}{2\pi} \int_{-\infty}^\infty \phi_=(x, \lambda) \lambda e^{i\lambda y} ~d\lambda$$
  TODO: This somehow gives the scattering matrix.
\end{proof}
  TODO: Other properties of scatmat as we need them.

We now construct an operator on $L^2(\RR)$ which is controlled by the scattering matrix $S$.

\begin{definition}
The operators $W_\pm$ defined by
$$W_\pm u = \lim_{t \to \pm \infty} e^{itH_V} e^{-itH_0} u$$
are called the \dfn{wave operators} of $V$.
\end{definition}
Recall that if $T$ is a self-adjoint operator, then for any initial data $u$, we have
$$(i\partial_t - T)e^{-itT}u = 0,$$
where $t \mapsto e^{-itT}$ is a one-parameter unitary group, which is well-defined because we can use the spectral theorem to diagonalize $T$ and then plug in the eigenvalues of $T$ into $e^{-it\cdot}$. Thus we can think of $e^{itH_V}e^{-itH_0} u$ as running the Schrödinger equation $(i\partial_t - H_V)u = 0$ back in time by $t$, and then running the equation $(i\partial_t - H_0)u = 0$ forward in time by $t$, so the wave operators approximate $u$ by a solution to the free Schrödinger equation.

\begin{definition}
The \dfn{scattering operator} is defined by
$$\mathcal S = W_+^*W_-.$$
\end{definition}
It is not obvious that such definitions even make sense, so we now prove this.
\begin{lemma}
The wave operators $W_\pm$ are well-defined as partial isometries on $L^2(\RR)$.
\end{lemma}
\begin{proof}
Let
$$U(t) = e^{itH_V}e^{-itH_0}.$$
Then $U$ is a one-parameter unitary group on $L^2(\RR)$, and it suffices to show that the limits
$$W_\pm u = \lim_{t \to \pm \infty} U(t)u$$
exist for a dense set of $u$. In fact, if
$$D = \{u \in L^2(\RR): \hat u \in C^\infty_{comp}(\RR \setminus 0)\},$$
then because $C^\infty_{comp}(\RR \setminus 0)$ is dense in $L^2(\RR)$ and the Fourier transform is a unitary operator where it is defined, $D$ is dense in $L^2(\RR)$.

Let $u \in D$. Since
$$\partial_t e^{itH_V} e^{-itH_0} u = ie^{itH_V}(H_V - H_0)e^{-itH_0}u = ie^{itH_V}Ve^{-itH_0}u,$$
the fundamental theorem of calculus implies that
$$U(s)u = u + i\int_0^s e^{itH_V}Ve^{-itH_0}u ~dt.$$
By definition of $D$, we can find $R > r > 0$ such that if $\xi \in \supp \hat u$, then $\xi$ lies in the annulus $A(r, R)$. Let $u_t = e^{-itH_0}u$. Then
$$\partial_{\xi_j}e^{ix\xi - it|\xi|^2} = i(x_j - 2t\xi_j)$$
so, integrating by parts,
\begin{align*}
  u_t(x) &= e^{-itH_0}u(x) = \frac{1}{2\pi} \int_{A(r, R)} e^{ix\xi - it|\xi|^2} \hat u(\xi) ~d\xi\\
    &= \frac{1}{2\pi} \int_{A(r, R)} \left(\frac{\partial_{\xi_j}}{i(x_j - 2t\xi_j)}\right)^2 e^{ix\xi - it|\xi|^2} \hat u(\xi) ~d\xi\\
    &= \frac{1}{2\pi} \int_{A(r, R)} e^{ix\xi - it|\xi|^2}\left(\partial_{\xi_j}\left(\frac{1}{i(x_j - 2t\xi_j)}\right)\right)^2 \hat u(\xi) ~d\xi.
\end{align*}
If $x \in \supp V$ and $\xi \in A(r, R)$ are held fixed, and $t \to \infty$, then we have
$$\left|\frac{1}{|x_j - 2t\xi_j|}\right| = O(|t|^{-1})$$ and $|\hat u(\xi)| = O(1)$ since $\hat u$ is holomorphic and restricted to the compact set $A(r, R)$, so
$$
  |u_t(x)| \leq \frac{1}{2\pi} \int_{A(r, R)} \left|\partial_{\xi_j}\left(\frac{1}{i(x_j - 2t\xi_j)}\right)\right|^2 |\hat u(\xi)| ~d\xi
    = O(|t|^{-2}).
$$
Therefore
$$\int_{-\infty}^\infty ||e^{itH_V}Ve^{-itH_0}u||_{L^2(\RR)} ~dt = \int_{-\infty}^\infty ||Vu_t||_{L^2(\RR)} ~dt = \int_{-\infty}^\infty O(|t|^{-2}) ~dt < \infty.$$
So the one-parameter unitary group $U$ remains bounded on $L^2(\RR)$ at infinity, so $W_\pm$ exist. A strong limit of unitary operators is a partial isometry, so we are done.
\end{proof}

\begin{lemma}
One has
$$W_\pm H_0 = H_VW_\pm.$$
\end{lemma}
\begin{proof}
One has
$$e^{isH_V}W_\pm e^{-isH_0} = \lim_{t \to \pm \infty} e^{i(s+t)H_V}e^{-i(s+t)H_0} = W_\pm$$
which is independent of $s$. Therefore
$$0 = \partial_s(e^{isH_V}W_\pm e^{-isH_0}) = iH_Ve^{isH_V}W_\pm e^{-isH_0} - ie^{isH_V}W_\pm e^{-isH_0}H_0 = i(H_VW_\pm - W_\pm H_0)$$
whence $H_VW_\pm = W_\pm H_0$.
\end{proof}

To characterize the scattering operator in terms of the scattering matrix, we define the isomorphism
\begin{align*}
  \Phi: L^2(\RR) &\to L^2(\RR_+) \oplus L^2(\RR_+)\\
  u &\mapsto (\hat u, \hat u(-\cdot)).
\end{align*}
Here $L^2(\RR_+) \oplus L^2(\RR_+)$ is being viewed as a Hilbert space with the direct sum inner product. Since the Fourier transform is unitary, so is $\Phi$.

\begin{lemma}
The scattering operator $\mathcal S$ satisfies
$$\mathcal S = \Phi^*S\Phi.$$
In particular, $\mathcal S$ is unitary.
\end{lemma}
\begin{proof}
Let $v \in C^\infty_{comp}(\RR)$, and let $u = \Phi^*S\Phi v$. Then
$$(\hat u(\lambda), \hat u(-\lambda)) = S(\lambda)(\hat v(\lambda), \hat v(-\lambda)).$$
We must prove $W_+u = W_-v$. To this end, let
$$w(x) = \frac{1}{2\pi} \int_0^\infty e_+(x, \lambda)\hat v(\lambda) + e_-(x, \lambda)\hat v(-\lambda) ~d\lambda.$$
Since $e_\pm(\cdot, \lambda)$ are eigenfunctions of $H_V$, they are stationary solutions of the Schrödinger equation, hence $e^{-itH_V}e_\pm(x, \lambda) = e^{-it\lambda^2}e_\pm(x, \lambda)$. Since $e^{-itH_V}$ is unitary, it commutes with integration, so
$$e^{-itH_V}w(x) = \frac{1}{2\pi} \int_0^\infty (e_+(x, \lambda) \hat v(\lambda) + e_-(x, \lambda) \hat v(-\lambda))e^{-it\lambda^2} ~d\lambda.$$
We let $f_\pm(x, \lambda) = -R_V(\lambda)(V(x)e^{\pm ix\lambda})$ be the outgoing part of $e_\pm$. Then
$$ e^{-itH_V}w(x) = \frac{1}{2\pi} \int_{-\infty}^\infty e^{i\lambda x}\hat v(\lambda) e^{-it\lambda^2} ~d\lambda + \frac{1}{2\pi} \int_0^\infty (f_+(x, \lambda) \hat v(\lambda) + f_-(x, \lambda) \hat v(-\lambda))e^{-it\lambda^2} ~d\lambda.$$
Since the free Hamiltonian $H_0 = -\Delta$ has continuous spectrum $[0, \infty)$, it can be diagonalized as the multiplier $H_0 = \mathcal F^{-1} \lambda^2 \mathcal F$, where $\mathcal F$ is the Fourier transform. Therefore
$$e^{-itH_0}v(x) = \frac{1}{2\pi} \int_{-\infty}^\infty e^{ix\lambda} \hat v(\lambda) e^{-it\lambda^2} ~d\lambda.$$
Moreover, if $Q = \{\lambda \in \CC: \Re \lambda > 0, ~\Im \lambda > 0\}$ is the first quadrant, then $Ve^{\pm i\lambda x} \in (1 + VR_0(\lambda))(L^2_{comp}(\RR))$, so since $R_V(\lambda) = R_0(\lambda)(1 + VR_0(\lambda))^{-1}$ and $R_0$ is holomorphic on $Q$, $f_\pm(\cdot, \lambda)$ exists and is holomorphic in $\lambda$ on $Q$. Since $v \in C_{comp}^\infty(\RR)$, the Paley-Weiner theorem implies that the integrand $I(t, x, \lambda) = e^{i\lambda x}\hat v(\lambda) e^{-it\lambda^2} ~d\lambda + \frac{1}{2\pi} \int_0^\infty (f_+(x, \lambda) \hat v(\lambda) + f_-(x, \lambda) \hat v(-\lambda))e^{-it\lambda^2}$ is holomorphic on $Q$ and remains holomorphic at the ``boundary" $\Re \lambda \to \infty$, since the $f_\pm$ are outgoing waves and hence bounded, and $\hat v(\lambda)$ is bounded for $\Im \lambda$ fixed by the Paley-Weiner theorem, so that $I(t, x, \lambda)$ remains bounded for $t,x,\Im \lambda$ fixed.

Letting $\gamma$ be the contour $\mu + i\mu: \mu > 0$ through $Q$, we use the Paley-Weiner to find a $C > 0$ such that
\begin{align*}
  \left|\int_0^\infty I(t, x, \lambda) ~d\lambda\right|^2 &= \left|\int_\gamma I(t, x, \lambda) ~d\lambda\right|^2 \\
    &\leq \frac{1}{2} \int_0^\infty |f_+(x, \mu + i\mu)\hat v(\mu + i\mu) + f_-(x, \mu+i\mu)\hat v(-\mu - i\mu)|^2e^{-2t\mu^2} ~d\mu\\
    &\leq \frac{1}{2} \int_0^\infty |f_+(x, \mu + i\mu) + f_-(x, \mu + i\mu)|^2 |e^{C\mu - 2t\mu^2})|^2 ~d\mu\\
    &\leq \frac{1}{2} \int_0^\infty |f_+(x, \mu + i\mu)|^2 + |f_-(x, \mu + i\mu)|^2 ~d\mu \int_0^\infty |e^{C\mu - 2t\mu^2})|^2 ~d\mu
\end{align*}
and recalling that
$$||R_V(\mu + i\mu)||_{L^2 \to L^2} = O(\mu^{-2})$$
we see that the integral of the $f_\pm$ are bounded, and that the integral of $I(t, x, \cdot)$ vanishes as $t \to \infty$. Therefore $W_-v = w$. A similar argument with $Q, \gamma$ reflected across the real axis $\Im \lambda = 0$ shows that if
$$\tilde w(x) = \frac{1}{2\pi} \int_0^\infty e_+(x, -\lambda)\hat u(-\lambda) + e_-(x, \lambda)\hat u(\lambda) ~d\lambda$$
then $W_+u = w$. TODO: This seems nontrivial since we applied a Fourier multiplier to $v$ to get $u$. Also $\hat u$ is not of exponential type but we can approximate $u$ by a compactly supported function. So these are details to fill in.

TODO: Use the functional equation for scatmat to prove $w = \tilde w$.
\end{proof}

TODO: If we need it, interpret that crazy commutative diagram to compute the images of $W_\pm$.




\section{Scattering resonances}
In what follows, we let $\Gamma_\lambda$ denote a sufficiently small (counterclockwise-oriented) circle centered on $\lambda$.

\begin{definition}
A \dfn{scattering resonance} is a pole of the resolvent family $R_V$. If $\lambda_0$ is a scattering resonance, its \dfn{multiplicity} $m_R(\lambda_0)$ is defined to be the rank of the operator
$$\frac{1}{2\pi i} \int_{\Gamma_{\lambda_0}} R_V(\lambda) ~d\lambda.$$
\end{definition}
In fact, if $\lambda_0$ is a scattering resonance, then for any $\lambda$ close to $\lambda_0$, we can express $R_V(\lambda)$ as a Laurent series
$$R_V(\lambda) = Q(\lambda - \lambda_0) + \sum_{j=1}^N \frac{P_j}{(\lambda - \lambda_0)^j}$$
for some holomorphic family of operators $Q$ defined on a small neighborhood of $0$ and some operators $P_1, \dots, P_N$. The operators $P_1, \dots, P_N$ are of finite rank by construction of $R_V(\lambda)$. Applying the Cauchy-Goursat theorem,
$$\int_{\Gamma_{\lambda_0}} Q(\lambda - \lambda_0) ~d\lambda = 0$$
so the multiplicity is entirely determined by the principal part $R_V(\lambda) - Q(\lambda - \lambda_0)$.


We put $m_S(\lambda_0)$ for the trace of the operator
$$-\frac{1}{2\pi i} \int_{\Gamma_{\lambda_0}} S(\lambda)^{-1}S'(\lambda) ~d\lambda.$$
This satisfies the equation $m_S(\lambda) = m_R(\lambda) - m_R(-\lambda)$. (For a proof, again see \cite[Chapter 2]{dyatlov2019mathematical}.)



\section{The Eisenbud-Wigner time delay formula}
We have the intuition that the wave operators should satisfy
$$e^{-itH_V}W_- \approx e^{-itH_0}$$
in some suitable sense, for $-t$ large enough, and would like to measure how good this approximation is for a typical function $f \in L^2(\RR)$.

In this section, let $\chi_r(x) = 1_{|x| < r}$ be the indicator function of the interval $(-r,  r)$.
\begin{definition}
  For every $f \in C^\infty_{comp}(\RR)$, let
  $$S_r(f) = \int_{-\infty}^\infty ||\chi_r e^{-itH_V} W_-f||_{L^2(\RR)}^2 ~dt$$
  and
  $$S_r^0(f) = \int_{-\infty}^\infty ||\chi_r e^{-itH_0} f||_{L^2(\RR)}^2 ~dt.$$
  Let $\tilde T_r$ be defined by $\langle \tilde T_rf, f\rangle = S_r(f) - S_r^0(f)$ and let $\tilde T$ be the weak limit of $\tilde T_r$ as $r \to \infty$. Then $\tilde T$ is known as the \dfn{time delay operator} for $V$.
\end{definition}
It is not obvious that the weak limit $\tilde T$ exists, but we will show that $\tilde T$ can be explicitly computed, so that it is does exist.

Recall that $\Phi$ is the isomorphism $L^2(\RR) \to L^2(\RR_+) \oplus L^2(\RR_+)$. Let
$$T(\lambda) = -2\lambda iS(\lambda)^* S'(\lambda).$$

\begin{theorem}[Eisenbud-Wigner]
\index{Eisenbud-Wigner time delay formula}
The time delay operator $\tilde T$ exists, and
$$\tilde Tf(\lambda) = \Phi^*(T(\lambda)\Phi(f))(\lambda).$$
\end{theorem}
\begin{proof}
Expanding out the definitions of $T,\Phi$, we are being asked to prove, for every $f \in C_{comp}^\infty(\RR)$ and $\lambda \in \RR_+$,
$$(\widehat{\tilde Tf}(\lambda), \widehat{\tilde Tf}(-\lambda)) = -2i\lambda S(\lambda)^* S'(\lambda)(\hat f(\lambda), \hat f(-\lambda)).$$
By the unitarity of the Fourier transform, $2\pi \langle \tilde Tf, f\rangle = \langle \widehat{\tilde Tf}, \hat f\rangle$.
TODO: The proof in TZ is just wrong lol.
\end{proof}



\section{The Breit-Wigner approximation}
TODO: Motivation . Mention that this proof was outlined by Zworski and Dyatlov...

\begin{theorem}[Breit-Wigner]
  \index{Breit-Wigner approximation}
Let $Z$ be the set of all nonzero resonances of $R_V$. Then
$$\frac{1}{2\pi i} \tr S'(\lambda_0)S(\lambda_0)^* = -\frac{1}{\pi}|\ch\supp V| - \frac{1}{\pi}\sum_{\lambda \in Z} \frac{\Im \lambda}{|\lambda - \lambda_0|^2}.$$
\end{theorem}
\begin{proof}
  TODO: Compute $S$ in terms of $X$ and $Y$.

  TODO: Show that
  $$t(\lambda) = i\frac{\lambda}{\hat X(\lambda)}.$$

  TODO: Show that
  $$S(-\lambda)J = JS(\lambda)^*.$$

  TODO: Show that
  $$\det S(\lambda) = \frac{t(\lambda)}{t(-\lambda)}.$$

  TODO: Show that
  $$\det S(\lambda) = \frac{\hat X(-\lambda)}{\hat X(\lambda)}.$$

  TODO: Show that
  $$\tr \frac{S'(\lambda)}{S(\lambda)} = \frac{(\det S)'(\lambda)}{\det S(\lambda)}.$$

  TODO: Use the argument principle and Titchmarsh's theorem.
\end{proof}



\chapter{Super-exponentially decreasing potentials}
We are interested in generalizing the Breit-Wigner formula to non-compactly supported potentials. Reasoning formally, we write
$$\frac{1}{2\pi i}\tr S'(\lambda)S(\lambda)^* = \infty - \frac{1}{\pi} \sum_{\kappa \in P} \frac{\Im \kappa}{|\kappa - \lambda|^2}.$$
If the left-hand side is to be finite, one then expects
$$\sum_{\kappa \in P} \frac{\Im \kappa}{|\kappa - \lambda|^2} = \infty.$$
Of course, our definition of the scattering matrix, or even of the set of resonances $P$, relied on the notion of an outgoing wave, which makes no sense if $\supp V$ is not compact. We thus consider a generalization of these definitions to a broader class of functions.

blah blah Richard Froese
\begin{definition}
A function $V \in L^\infty(\RR)$ is said to be \dfn{super-exponentially decreasing} if for every $N \in \NN$ we can find a $C_N > 0$ such that
$$V(x) \leq C_Ne^{-N|x|}.$$
\end{definition}
Fpr example, a Gaussian is super-exponentially decreasing. We will indicate that a function $V$ is super-exponentially decreasing by writing $V(x) = O(e^{-x\infty})$.

\section{Noncommutative sequence spaces}
We review the theory of trace-class operators, which are those operators $T$ such that $T$ has a well-defined trace and $1 + T$ has a well-defined determinant. To do this, we will need the theories of compact and Hilbert-Schmidt operators as well.

Fix a Hilbert space $H$. If $T \in B(H)$, then $T^*T$ is a positive operator, so has a unique positive square root $|T| = \sqrt{T^*T}$, which we can reasonably think of as the absolute value of $T$. If $H$ is actually finite-dimensional, then the trace of $T$ is given by $\tr T = \sum_j \langle Te_j, e_j\rangle$ for any and every orthonormal basis $(e_j)_j$. So it is reasonable to define $\tr T$ this way whenever $H$ is separable (hence has a countable orthonormal basis), though the series may not converge in that case. Henceforth we will assume that $H$ is separable.

\begin{definition}
The \dfn{trace-class norm} is defined by $||T||_1 = \tr |T|$, and the \dfn{Hilbert-Schmidt norm} is defined by $||T||_2^2 = \tr(T^*T)$. If $||T||_1$ (resp. $||T||_2$) is finite, we say that $T$ is a \dfn{trace-class operator} (resp. \dfn{Hilbert-Schmidt operator}). The space of trace-class operators is known as $B^1(H)$ and the space of Hilbert-Schmidt operators is known as $B^2(H)$.
\end{definition}
Let $X$ be a ``good" measure space. Then one can think of $B^1(L^2(X))$ as the ``noncommutative analogue" of $L^1(X)$. In this metaphor, the Hilbert space of Hilbert-Schmidt operators $B^2(L^2(X))$ corresponds to $L^2(X)$, and the space $B^\infty(L^2(X))$ of compact operators corresponds to $L^\infty(X)$. This metaphor is especially good when $X$ is a discrete space, for then we have the chains of inclusions $B^p(L^2(X)) \subseteq B^q(L^2(X))$ and $L^p(X) \subseteq L^q(X)$ for $1 \leq q \leq p \leq \infty$.

Letting $||\cdot||_\infty$ denote the usual operator norm, we observe that
\begin{align*}
||T||_\infty & \leq ||T||_1,\\
||TS||_1 &\leq ||T||_2||S||_2;
\end{align*}
the proof is the same as their ``commutative analogues" which interpolate between $\ell^1$, $\ell^2$, and $\ell^\infty$.

We now construct a wealth of Hilbert-Schmidt operators, some of which we will need later.
\begin{lemma}
Let $k \in L^2(\RR^2)$ be the integral kernel of an operator $K \in B(L^2(\RR))$. Then $||K||_2 = ||k||_2$, so $K \in B^2(L^2(\RR))$.
\end{lemma}
\begin{proof}
Fix an orthonormal basis $(e_n)_n$ of $L^2(\RR)$. This determines an orthonormal basis $(e_{nm})_{nm}$ of $L^2(\RR^2)$ by $e_{nm}(x, y) = e_n(x) \overline{e_m(y)}$. So
$$||k||_2 = \sum_{nm} |\langle k, e_{nm}\rangle|^2.$$
We therefore compute
\begin{align*}
  \langle k, e_{nm}\rangle &= \int_{-\infty}^\infty \int_{-\infty}^\infty k(x, y) \overline{e_m(x)} e_n(y) ~dx ~dy\\
  &= \int_{-\infty}^\infty \overline{e_m(x)} \int_{-\infty}^\infty k(x, y) e_n(y) ~dx ~dy\\
  &= \int_{-\infty}^\infty \overline{e_m(x)} Ke_n(y) ~dx ~dy = \langle Ke_n, e_m\rangle.
\end{align*}
Therefore
$$||k||_2 = \sum_n ||Ke_n||^2 = ||K||_2.$$
In particular, $K \in B^2(L^2(\RR))$.
\end{proof}
It can be shown that every Hilbert-Schmidt operator on $L^2(\RR)$ can be written as an integral operator whose kernel lies in $L^2(\RR^2)$; but we will not need this fact.
\begin{lemma}
Let $f, g \in L^2(\RR)$ and suppose that
$$Ku = \mathcal F^{-1}(\hat f \mathcal F(gu)).$$
Then $||K||_2 = ||f||_2||g||_2$, so $K \in B^2(L^2(\RR))$.
\end{lemma}
\begin{proof}
By an approximation argument, we may assume that $f$, $g$, and $u$ are Schwartz-class functions. Then
\begin{align*}
Ku(x) &= \frac{1}{2\pi} \int_{-\infty}^\infty e^{i\xi x} \hat f(\xi) \widehat{gu}(\xi) ~d\xi \\
  &= \frac{1}{2\pi} \int_{-\infty}^\infty \int_{-\infty}^\infty e^{i\xi(x - y)} \hat f(\xi) g(y) u(y) ~dy ~d\xi\\
  &= \frac{1}{2\pi} \int_{-\infty}^\infty g(y)u(y) \int_{-\infty}^\infty e^{i\xi(x - y)} \hat f(\xi) ~d\xi ~dy\\
  &= \int_{-\infty}^\infty f(x - y) g(y) u(y) ~d\xi~dy
\end{align*}
so $k(x, y) = f(x - y) g(y)$ is the integral kernel of $K$. Besides,
\begin{align*}
  ||k||_2^2 &= \int_{-\infty}^\infty \int_{-\infty}^\infty |g(y)|^2 |f(x - y)|^2 ~dx ~dy
  \\&= \int_{-\infty}^\infty |g(y)|^2 \int_{-\infty}^\infty |f(x - y)|^2 ~dx ~dy = ||g||_2^2||f||_2^2.
\end{align*}
Therefore $||K||_2 = ||g||_2 ||f||_2$.
\end{proof}
\begin{lemma}
  \label{partial resolvent is hilbert schmidt}
Let $\chi$ denote the indicator function of $[-1, 1]$ and let $\Im \lambda > 0$. Then the operator $T(\lambda) = (-i\partial - \lambda)^{-1}\chi$ is Hilbert-Schmidt, and
$$||T(\lambda)||_2 = \frac{1}{\sqrt{\Im \lambda}}.$$
\end{lemma}
\begin{proof}
Let $\hat f(\xi) = (\xi - \lambda)^{-1}$. Then
$$T(\lambda)u = \mathcal F^{-1}(\hat f \mathcal F(\chi u)).$$
So $||T(\lambda)||_2 = ||f||_2||\chi||_2$, and $||\chi||_2 = \sqrt 2$. Also,
$$||f||_2^2 = \frac{||\hat f||_2^2}{2\pi} = \frac{1}{2\pi} \int_{-\infty}^\infty \frac{d\xi}{|\xi - \lambda|^2}.$$
Up to a translation of $\xi$, we may assume $\Re \lambda = 0$. Then
$$|\xi - \lambda|^2 = |\xi|^2 + |\lambda|^2$$
by the Pythagorean theorem. Since $|\xi|^2 = \xi^2$ we have
$$||f||_2^2 = \frac{1}{2\pi} \int_{-\infty}^\infty \frac{d\xi}{\xi^2 + |\lambda|^2} = \frac{1}{2\Im \lambda}$$
wherefore the claim.
\end{proof}

A trace-class operator $T$ can clearly be approximated by finite-rank operators in $||\cdot||_1$, and the compact operators are those that can be approximated by finite-rank operators in $||\cdot||_\infty$. Therefore a trace-class operator is compact. In particular, its spectrum is countable and its only limit point is $0$ (provided that $H$ is infinite-dimensional). So every element of the spectrum is an eigenvalue except possibly $0$. (That is, every element of the spectrum of $1 + T$ is an eigenvalue except possibly $1$.) We recall that the multiplicity of an eigenvalue $\lambda$ of $1 + T$ is defined to be the dimension of the space of vectors annihilated by $(T - \lambda)^j$ for some $j \geq 0$.
\begin{definition}
Let $T \in B^1(H)$ and let $(\lambda_j)_j$ be an enumeration of the spectrum of $1 + T$, counting multiplicities. Then the \dfn{Fredholm determinant} of $1 + T$ is defined by
$$\det(1 + T) = \prod_j \lambda_j.$$
\end{definition}
\begin{lemma}
The infinite product in the definition of the Fredholm determinant converges absolutely, satisfying the estimate
$$|\det(1 + T)| \leq e^{||T||_1}.$$
One has $\det(1 + T) = 0$ if and only $1 + T$ is not injective. If $H$ is finite-dimensional, then the Fredholm determinant agrees with the classical determinant.
\end{lemma}
\begin{proof}
The spectrum of $T$ is given by $(\lambda_j - 1)_j$, so
$$|\det(1 + T)| \leq \exp\left(\sum_j |\lambda_j - 1|\right) = \exp(\tr|T|) = e^{||T||_1} < \infty.$$
If $1 + T$ is injective, then $\ker(1 + T) = 0$, so $0$ is not an eigenvalue of $1 + T$. Since $0$ is not a limit point of the spectrum of $1 + T$, it follows that there is a $\varepsilon > 0$ such that for every $j$, $|\lambda_j| > \varepsilon$. In particular, $\det(1 + T) \neq 0$. The converse is easy.

Finally, if $H$ is finite-dimensional, we can write $1 + T$ in a Jordan canonical form $J$, and the classical determinant of $1 + T$ will be the product of the diagonal entries of $J$, which are exactly the eigenvalues of $1 + T$, which appear as many times as their multiplicities.
\end{proof}

By the spectral theorem for compact operators, a compact operator $T \in B^\infty(H)$ admits a singular value decomposition
$$Tu = \sum_{n=1}^\infty \rho_n \langle u, e_n\rangle f_n$$
for some orthonormal sets $e_n,f_n \in H$ and singular values $\rho_n \in \RR$. Then the assumption that $T \in B^p(H)$ is equivalent to saying that $(\rho_n)_n \in \ell^p$ (and this makes sense for any $p \in [1, \infty]$, not just $p =1,2,\infty$). In particular, given $f, g \in H$, the operator
$$Tu = \langle u, f \rangle g$$
is a rank-$1$ operator (since its image is the span of $g$), and we can compute its SVD by finding a $\rho \in \RR$ such that if
$$Tu = \rho \langle u, \tilde f \rangle \tilde g$$
has $||\tilde f||_2 = ||\tilde g||_2 = 1$. Clearly then
\begin{equation}\label{b1 norm of a tensor product}||T||_1 = |\tr T| = |\rho| = ||f||_2 ||g||_2.\end{equation}
So the following linear map is well-defined.
\begin{definition}
\label{tensor products are trace class}
We define a linear map $b: H \otimes H \to B^1(H)$ by
$$b(f \otimes g)u = \langle u, f \rangle g.$$
\end{definition}


\section{Constructing the resolvent}
Let $R_V(\lambda) = (H_V - \lambda^2)^{-1}$ denote the resolvent, as in the compactly supported case. Trouble comes because $R_V$ may not extend meromorphically to $\CC$ if $V$ is not compactly supported. Indeed, the proof of Theorem \ref{meromorphic continuation with compact support} used the fact that we could find a cutoff function $\rho$ such that $\rho V = V$.

If $V$ is compactly supported, then we can meromorphically continue $R_V$ to all of $\CC$, and the following are equivalent for each $\lambda \in \CC \setminus 0$:
\begin{enumerate}
\item $\lambda$ is a resonance.
\item $\lambda$ is a pole of $R_V$.
\item $\lambda$ is a pole of $\sqrt VR_V\sqrt{|V|}$.
\end{enumerate}
Moreover, we have
$$R_V(\lambda) = R_0(\lambda)(1 + VR_V(\lambda)),$$
so if we put
$$S_V(\lambda) = \sqrt VR_0(\lambda) \sqrt{|V|},$$
it follows that
$$\sqrt V R_V(\lambda) \sqrt{|V|} = S_V(\lambda)(1 - \sqrt VR_V(\lambda)\sqrt{|V|}),$$
which can be rewritten as
$$S_V(\lambda) = (1 + S_V(\lambda))\sqrt V R_V(\lambda)\sqrt{|V|}.$$
The definition of $S_V$ makes sense even if $V$ is only assumed to be in $L^\infty(\RR)$.
\begin{theorem}
  \label{sv is b1 family}
Suppose that $V$ is super-exponentially decreasing. Then $S_V$ is holomorphic as a function $\CC \setminus 0 \to B^1(L^2(\RR))$, and has a simple pole at $0$.
\end{theorem}
Suppose that $V$ is compactly supported and $\lambda \neq 0$ is a resonance, hence a pole of $\sqrt VR_V\sqrt{|V|}$. By the lemma, $S_V(\lambda)$ exists, so we have
$$(1 + S_V(\lambda))^{-1}S_V(\lambda) = \sqrt V R_V(\lambda) \sqrt{|V|} = \infty.$$
This only makes sense if $1 + S_V(\lambda)$ is singular, and since $S_V(\lambda)$ is trace-class, this means that $\lambda$ is a zero of $\det(1 + S_V(\lambda))$.
We can therefore easily extend the definition of resonance to super-exponentially decreasing potentials.
\begin{definition}
Suppose that $V$ is super-exponentially decreasing and let
$$D(\lambda) = \det(1 + S_V(\lambda)).$$
A \dfn{scattering resonance} of $V$ is a zero of $D$.
\end{definition}
\begin{corollary}
Suppose that $V$ is super-exponentially decreasing. Then $D$ is holomorphic as a function $\CC \setminus 0 \to \CC$, and has a simple pole at $0$.
\end{corollary}
In particular, we can find an entire function $E$ such that $E(0) \neq 0$ and
\begin{equation}
  \label{determinant is an entire function}
  D(z) = \frac{E(z)}{z},
\end{equation}
so that the scattering resonances of $D$ are exactly the zeroes of $E$.

Before we can prove Theorem \ref{sv is b1 family}, we need some careful estimates on $S_V(\lambda)$.

We recall that $R_0$ admits the resolvent function
$$R_0(x, y; \lambda) = \frac{i}{2\lambda}e^{i\lambda|x-y|},$$
which is the Green function of $-\Delta$, i.e.
$$(-\Delta - \lambda^2)R_0(x, y; \lambda) = \delta(x - y).$$
Since it is true of the Green function, it follows that $R_0$ is meromorphic with a simple pole at $0$.
\begin{lemma}
Let $L > 0$, $x,z \in (-L, L)$. For every $\lambda_1,\lambda_2 \in \CC$ we have
\begin{align*}R_0(x, z; \lambda_1) - R_0(x, z; \lambda_2)
    &= (\lambda_1^2 - \lambda_2^2)\int_{-L}^L R_0(x, y; \lambda_1)R_0(y, z; \lambda_2) ~dz\\
    &\quad+\frac{i}{4}e^{L + i(\lambda_1 + \lambda_2)}\left(\frac{1}{\lambda_1} - \frac{1}{\lambda_2}\right)(e^{-i(\lambda_1x + \lambda_2z)} + e^{i(\lambda_1x + \lambda_2z)}).
\end{align*}
\end{lemma}
\begin{proof}
Adding and subtracting $\Delta_y$,
$$(\lambda_1^2 - \lambda_2^2)R_0(x, y; \lambda_1)R_0(y, z; \lambda_2) = R_0(x, y; \lambda_1)(\Delta_y + \lambda_1^2 - \Delta_y - \lambda_2^2)R_0(y, z; \lambda_2).$$
By the definition of the Green function,
$$\int_{-L}^L R_0(x, y; \lambda_1)(-\Delta_y - \lambda_2^2)R_0(y, z; \lambda_2) ~dy = R_0(x, z; \lambda_1).$$
Therefore
\begin{align*}
  (\lambda_1^2 - \lambda_2^2)\int_{-L}^L R_0(x, y; \lambda_1)R_0(y, z; \lambda_2) ~dy &= R_0(x, z; \lambda_1)\\
  &\quad-\int_{-L}^L R_0(x, y;\lambda_1)(\Delta_y - \lambda_1^2)R_0(y, z; \lambda_2) ~dy.
\end{align*}
Integrating by parts,
\begin{align*}
  -\int_{-L}^L R_0(x, y; \lambda_1)\Delta_y R_0(y, z; \lambda_2) ~dy &= -\int_{-L}^L \partial_yR_0(x, y; \lambda_1)\partial_y R_0(y, z; \lambda_2) ~dy\\
    &\quad+ [R_0(x, y; \lambda_1)\partial_y R_0(y, z; \lambda_2)]_{y=-L}^L.
\end{align*}
We observe that
$$R_0(x, y; \lambda_1)\partial_y R_0(y, z; \lambda_2) = \frac{i}{4\lambda_1}e^{i(\lambda_1|x-y|+\lambda_2|y-z|)}\frac{y-z}{|y-z|}$$
from which it follows that
$$[R_0(x, y; \lambda_1)\partial_y R_0(y, z; \lambda_2)]_{y=-L}^L = \frac{i}{4\lambda_1} e^{i\lambda_1L}(e^{-i(\lambda_1x + \lambda_2z)} + e^{i(\lambda_1x + \lambda_2z)}).$$
Integrating by parts again,
\begin{align*}
  -\int_{-L}^L \partial_yR_0(x, y; \lambda_1)\partial_y R_0(y, z; \lambda_2) ~dy &= \int_{-L}^L \Delta_yR_0(x, y; \lambda_1)R_0(y, z; \lambda_2) ~dy\\
  &\quad- [\partial_yR_0(x, y; \lambda_1) R_0(y, z; \lambda_2)]_{y=-L}^L\\
  &= R_0(x, z; \lambda_2) \\
  &\quad- \frac{i}{4\lambda_2} e^{i\lambda_1L}(e^{-i(\lambda_1x + \lambda_2z)} + e^{i(\lambda_1x + \lambda_2z)}).
\end{align*}
Putting it all together, the lemma follows.
\end{proof}
Let $T_L = S_{\chi_{[-L, L]}}$ and let $T = T_1$. Then, if $f$ decays fast enough, it makes sense to define
$$F(\lambda_1, \lambda_2)f(x) = \frac{i}{4}e^{i(\lambda_1 + \lambda_2)}\left(\frac{1}{\lambda_1} - \frac{1}{\lambda_2}\right)b(\chi e^{-i\lambda_1x} \otimes \chi e^{-i\lambda_2x} + \chi e^{i\lambda_1x} \otimes e^{i\lambda_2x}),$$
where $b$ is the operator $L^2(\RR) \otimes L^2(\RR) \to B^1(L^2(\RR))$ given by Definition \ref{tensor products are trace class}. Using (\ref{b1 norm of a tensor product}) we can estimate, for $\lambda \in \RR$,
\begin{align*}
||F(\lambda, i|\lambda|)||_1 \leq e^{-i|\lambda|}\left|\frac{1}{\lambda} - \frac{1}{i|\lambda|}\right|(||\chi||_2 ||\chi e^{|\lambda| x}||_2 + ||\chi||_2 ||\chi e^{-|\lambda|x}||_2)
\end{align*}
whence
\begin{equation}
\label{estimate on F}
||F(\lambda, i|\lambda|)||_1 = O(|\lambda|^{-1}).
\end{equation}
On the other hand, if $\lambda \notin \RR$, we have
$$||F(\lambda, -\lambda)||_1 \leq Ce^{2|\Im \lambda|}|\Im \lambda|^{-1}||\chi||_2||\chi e^{2x\Im \lambda}||_2$$
whence
\begin{equation}
  \label{imaginary estimate on F}
  ||F(\lambda, -\lambda)||_1 = O\left(\frac{e^{4 |\Im \lambda|}}{|\Im \lambda|}\right).
\end{equation}

By the above lemma, we have
\begin{equation}
\label{integrating T by parts}
  T(\lambda_1) - T(\lambda_2) = (\lambda_1^2 - \lambda_2^2)T(\lambda_1)T(\lambda_2) + F(\lambda_1, \lambda_2).
\end{equation}
\begin{lemma}
\label{T1 is b1}
Let $\lambda \neq 0$. Then $T(\lambda)$ is trace-class. If $\Im \lambda > 0$, then $||T(\lambda)||_1 \leq (\Im \lambda)^{-1}$. If $\lambda \in \RR$, then $||T(\lambda)||_1 = O(1 + |\lambda|^{-1})$. If $\Im \lambda < 0$, then
$$||T(\lambda)||_1 = O\left(\frac{e^{4|\Im \lambda|}}{-\Im \lambda}\right).$$
\end{lemma}
\begin{proof}
If $\Im \lambda > 0$, we have the factorization
\begin{align*}T(\lambda) &= \chi(-\Delta - \lambda^2)^{-1}\chi = \chi(-i\partial + \lambda)^{-1} (-i\partial - \lambda)^{-1} \chi\\ &= -((-i\partial - \lambda)^{-1} \chi)^*((-i\partial - \lambda)^{-1} \chi).\end{align*}
Since taking adjoints clearly preserves the Hilbert-Schmidt norm, we have
$$||T(\lambda)||_1 \leq ||(-i\partial - \lambda)^{-1}\chi||_2^2 = \frac{1}{\Im \lambda}$$
by Lemma \ref{partial resolvent is hilbert schmidt}.

If $\Im \lambda = 0$, we use the fact that
$$||T(\lambda)||_2 \leq ||(-\Delta - \lambda^2)^{-1}||_2 = \frac{1}{2\pi}\int_{-\infty}^\infty \frac{d\xi}{\xi^2 + \lambda^2} = \frac{1}{2|\lambda|},$$
as in the proof of Lemma \ref{partial resolvent is hilbert schmidt}. By (\ref{integrating T by parts}) and (\ref{estimate on F}),
\begin{align*}
||T(\lambda)||_1 &\leq ||T(i|\lambda|)||_1 + (\lambda^2 + |\lambda|^2)||T(\lambda)||_2||T(i|\lambda|)||_1 + ||F(\lambda, i|\lambda|)||_1\\
  &\leq \frac{1}{|\lambda|} + O\left(\frac{\lambda^2 + |\lambda|^2}{|\lambda|^2}\right) + O\left(\frac{1}{|\lambda|}\right)\\
  &= O(1 + |\lambda|^{-1}).
\end{align*}

If $\Im \lambda < 0$, we notice that by (\ref{integrating T by parts}),
\begin{align*}
  T(\lambda) &= T(-\lambda) + (\lambda^2 - (-1)^2\lambda^2)T(\lambda)T(-\lambda) + F(\lambda, -\lambda)\\
  &= T(-\lambda) + F(\lambda_1, \lambda_2).
\end{align*}
Using our estimate for $\Im \lambda > 0$ and (\ref{imaginary estimate on F}),
$$||T(\lambda)||_1 \leq ||T(-\lambda)||_1 + ||F(\lambda, -\lambda)||_1 = O\left(\frac{e^{4|\Im \lambda|}}{-\Im \lambda}\right),$$
as promised.
\end{proof}

\begin{lemma}
\label{estimate on TL}
Suppose that $\Im \lambda < 0$. Let $\chi_L$ denote the indicator function of $[-L, L]$. Then
$$||T_L(\lambda)||_1 = O\left(\frac{Le^{4L|\Im \lambda|}}{|\lambda|}\right).$$
\end{lemma}
\begin{proof}
If $U_L$ denotes the unitary dilation
$$U_Lu(x) = L^{-1/2}u(xL^{-1})$$
then it is easy to see that
$$T_L(\lambda)U_L = L^2U_LT(L\lambda).$$
Since conjugation by a unitary operator preserves trace,
$$||T_L(\lambda)||_1 = L^2 ||T(L\lambda)||_2 \leq CL^2 e^{4L|\Im \lambda|}{L|\Im \lambda|}$$
which proves the lemma.
\end{proof}


\begin{lemma}
\label{convergence in b1 topology}
Let $N > 0$. Suppose that $\sqrt{V(x)} = O(e^{-N|x|})$, $\lambda \neq 0$, and $|\Im \lambda| < N/4$. Then
$$\lim_{L \to \infty} \sqrt V T_L(\lambda) \sqrt{|V|} = S_V(\lambda)$$
in the topology of $B^1(L^2(\RR))$.
\end{lemma}
\begin{proof}
Up to a rescaling we may assume that $|\sqrt{V(x)}| \leq e^{-N|x|}$. Let $w(x) = e^{-LN}$ for $|x| \in (L-1, L]$, so $w$ dominates $\sqrt V$. Therefore
\begin{align*}
||\sqrt V T_L(\lambda) \sqrt{|V|} - S_V(\lambda)||_1 &= ||\sqrt V(\chi_L R_0(\lambda)\chi_L)\sqrt{|V|}||_1\\
  &= \leq ||w(\chi_L R_0(\lambda) \chi_L - R_0(\lambda))w||_1 \\
  &= ||wT_L(\lambda)w - S_{w^2}(\lambda)||_1
\end{align*}
so it suffices to prove the lemma when $\sqrt V = w$. Moreover,
\begin{align*}
||w(\chi_L R_0(\lambda) \chi_L - R_0(\lambda))w||_1 &= ||w(\chi_L R_0(\lambda) \chi_L - R_0(\lambda)\chi_L \\
  &\quad+ R_0(\lambda)\chi_L - R_0(\lambda))w||_1\\
  &\leq ||w((1 - \chi_L)R_0(\lambda)\chi_L)w||_1 \\
  &\quad+ ||w(R_0(\lambda)(1 - \chi_L))w||_1\\
  &\leq ||w((1 - \chi_L)R_0(\lambda))w||_1 \\
  &\quad + ||w(R_0(\lambda)(1 - \chi_L))w||_1\\
  &\leq 2||wR_0(\lambda)(1 - \chi_L)w||_1
\end{align*}
since $||\chi_Lw||_2 \leq ||w||_2$ and $||wR_0(\lambda)(1 - \chi_L)w||_1 = ||w(1 - \chi_L)R_0(\lambda)w||_1$ by adjointness.

Let $a_1 = e^{-N}$, $a_L = e^{-NL} - e^{-N(L-1)}$. Then $w = \sum_L a_L \chi_L$, and in particular $\sum_L a_L = 1$, so
$$wR_0(\lambda)(1 - \chi_L)w = \sum_\ell \sum_{m=L}^\infty a_\ell\chi_\ell R_0(\lambda) a_m \chi_m$$
whence
\begin{align*}
  ||wR_0(\lambda)(1 - \chi_L)w||_1 &\leq \sum_\ell \sum_{m=L}^\infty a_\ell a_m ||\chi_\ell R_0(\lambda) \chi_m||\\
  &\leq \sum_{m=L}^\infty a_m \left(\sum_{\ell=1}^m a_\ell ||\chi_m R_0(\lambda) \chi_m||_1 + \sum_{\ell=m+1}^\infty a_\ell ||\chi_\ell R_0(\lambda) \chi_\ell||_1 \right)\\
  &= \sum_{m=L}^\infty a_m \left(\sum_{\ell=1}^m a_\ell ||T_m(\lambda)||_1 + \sum_{\ell=m+1}^\infty a_\ell ||T_\ell(\lambda)||_1 \right).
\end{align*}
We now apply Lemma \ref{estimate on TL} and the fact that $\sum_\ell a_\ell = 1$ (and the sum converges monotonically) to see that
\begin{align*}
  \sum_{m=L}^\infty\sum_{\ell=1}^m a_m a_\ell ||T_m(\lambda)||_1 &\leq \sum_{m=L}^\infty a_m ||T_m(\lambda)||_1\\
  &= O\left(\sum_{m=L}^\infty me^{-m(N-4|\Im \lambda|)}\right).
\end{align*}
Similarly,
$$\sum_{m=L}^\infty \sum_{\ell=m+1}^\infty a_\ell ||T_\ell(\lambda)||_1 = O\left(\sum_{\ell=L}^\infty \ell e^{-\ell(N-4|\Im \lambda|)}\right).$$
Let $\delta = N - 4|\Im \lambda| > 0$. Then
\begin{align*}
  ||\sqrt V T_L(\lambda) \sqrt{|V|} - S_V(\lambda)||_1 &= O(||wR_0(\lambda)(1 - \chi_L)w||_1) \\
  &= O\left(\sum_{\ell=L}^\infty \ell e^{-\delta \ell} \right) = o(\delta L)
\end{align*}
and the theorem holds taking $L \to \infty$.
\end{proof}

\begin{proof}[Proof of Theorem \ref{sv is b1 family}]
Holomorphy follows from that of $R_0$. Since $V$ is super-exponentially decreasing, so is $\sqrt V$. For every $\lambda$, we can find a $N$ large enough that Lemma \ref{convergence in b1 topology} applies.
\end{proof}



\section{original research lol}
\begin{theorem}
If $V$ is not compactly supported, then
$$\sum_{\kappa \in P} \frac{\Im \kappa}{|\kappa - \lambda|^2} = \infty.$$
\end{theorem}
\begin{proof}
Suppose it's finite, then get a bound
$$|\ch\supp V| \leq C \sum_{\kappa \in P}\frac{\Im \kappa}{|\kappa - \lambda|^2}.$$
Then $V$ is compactly supported.
\end{proof}





\printbibliography



\printindex

\end{document}
