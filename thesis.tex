\documentclass[12pt]{report}
\usepackage[utf8]{inputenc}
%\usepackage[margin=1in]{geometry}
\usepackage{amsmath,amsthm,amssymb}
\usepackage{mathrsfs}

\usepackage{enumitem}
%\usepackage[shortlabels]{enumerate}
\usepackage{tikz-cd}
\usepackage{mathtools}
\usepackage{amsfonts}
\usepackage{amscd}
\usepackage{makeidx}
\usepackage{enumitem}

\title{The Breit-Wigner series and distribution of resonances of potentials}
\author{Aidan Backus}
\date{May 2020}

\newcommand{\NN}{\mathbf{N}}
\newcommand{\ZZ}{\mathbf{Z}}
\newcommand{\QQ}{\mathbf{Q}}
\newcommand{\RR}{\mathbf{R}}
\newcommand{\CC}{\mathbf{C}}
\newcommand{\DD}{\mathbf{D}}

\DeclareMathOperator{\card}{card}
\DeclareMathOperator{\ch}{ch}
\DeclareMathOperator{\dom}{dom}
\DeclareMathOperator{\Res}{Res}
\DeclareMathOperator{\sgn}{sgn}
\DeclareMathOperator{\singsupp}{sing~supp}
\DeclareMathOperator{\Spec}{Spec}
\DeclareMathOperator{\supp}{supp}
\newcommand{\tr}{\operatorname{tr}}

\newcommand{\dbar}{\overline \partial}

\newcommand{\pic}{\vspace{30mm}}
\newcommand{\dfn}[1]{\emph{#1}\index{#1}}

\renewcommand{\Re}{\operatorname{Re}}
\renewcommand{\Im}{\operatorname{Im}}


\newtheorem{theorem}{Theorem}[chapter]
\newtheorem{badtheorem}[theorem]{``Theorem"}
\newtheorem{prop}[theorem]{Proposition}
\newtheorem{lemma}[theorem]{Lemma}
\newtheorem{proposition}[theorem]{Proposition}
\newtheorem{corollary}[theorem]{Corollary}
\newtheorem{conjecture}[theorem]{Conjecture}
\newtheorem{axiom}[theorem]{Axiom}

\theoremstyle{definition}
\newtheorem{definition}[theorem]{Definition}
\newtheorem{remark}[theorem]{Remark}
\newtheorem{example}[theorem]{Example}

\newtheorem{exercise}[theorem]{Discussion topic}
\newtheorem{homework}[theorem]{Homework}
\newtheorem{problem}[theorem]{Problem}

\usepackage{color}
\usepackage{hyperref}
\hypersetup{
    colorlinks=true, % make the links colored
    linkcolor=blue, % color TOC links in blue
    urlcolor=red, % color URLs in red
    linktoc=all % 'all' will create links for everything in the TOC
    %Ning added hyperlinks to the table of contents 6/17/19
}

\usepackage[backend=bibtex]{biblatex}
\addbibresource{thesis.bib}

\makeindex
\begin{document}

\maketitle

\tableofcontents




\chapter{Introduction}
%\begin{quote}
%  ``EVERY MORNING I WAKE UP AND OPEN PALM SLAM A BOOK ONTO THE DESK. IT'S MATHEMATICAL THEORY OF SCATTERING RESONANCES BY DYATLOV AND ZWORSKI AND RIGHT THEN AND THERE I START DOING THE MOVES ALONGSIDE WITH THE AUTHORS, DYATLOV AND ZWORSKI. I DO EVERY PROOF AND I DO EVERY PROOF HARD. MAKIN WHOOSHING SOUNDS WHEN I MEROMORPHICALLY CONTINUE THE RESOLVENT INTO THE LOWER HALF PLANE OR EVEN WHEN I MESS UP TECHNIQUE. NOT MANY CAN SAY THEY CAN BOUND THE NUMBER OF RESONANCES OF A BLOCK BOX OPERATOR. I CAN. I SAY IT AND I SAY IT OUTLOUD EVERYDAY TO PEOPLE IN MY COLLEGE CLASS AND ALL THEY DO IS PROVE PEOPLE IN COLLEGE CLASS CAN STILL BE IMMATURE JEKRS. AND IVE LEARNED ALL THE LEMMAS AND IVE LEARNED HOW TO MAKE MYSELF AND MY APARTMENT LESS LONELY BY SHOUTING EM ALL. 2 HOURS INCLUDING WIND DOWN EVERY MORNIng"

%  -Analytic Memes for Estimate-Loving Teens
%\end{quote}



Recall that if $A$ is a quantum observable, then the spectrum of $A$ consists of the possible values $\lambda^2$ that the observable can take when a state $\psi$ is measured. Moreover, $\psi$ is a pure state for measurement $\lambda^2$ (i.e. an observer is guaranteed to measure $\psi$ in state $\lambda^2$) if and only if $\psi$ solves the eigenvalue equation $A\psi = \lambda^2 \psi$. In other words, the pure states for $\lambda^2$ are exactly the kernel of the operator $A - \lambda^2$.

The stationary (homogeneous) Schrödinger equation is the eigenvalue equation when $A$ is the Hamiltonian
$$H_V = D^2 + V$$
where $D = -i\partial$ and $V \in L^\infty(\RR \to \RR)$ is a potential. (Here we can view $D^2$ as the Laplacian $-\Delta$, or as the momentum operator.)

Suppose that we are working in one-dimensional space. If $\supp V$ is compact, then a solution $\psi$ of the stationary Schrödinger equation
with energy $\lambda^2$ has the form
\begin{equation}
\label{solutions are waves}
\psi(x) = \begin{cases}
  A_+ e^{i\lambda x} + B_-e^{-i\lambda x}, &x \gg 0,\\
  A_- e^{i\lambda x} + B_+e^{-i\lambda x}, &x \ll 0,
\end{cases}
\end{equation}
as we discuss after the proof of Theorem \ref{meromorphic continuation with compact support}.
The physical interpretation of (\ref{solutions are waves}) is that since $\psi$ is a pure state, it only has one frequency $\lambda$\footnote{This is our reason for writing the eigenvalue as $\lambda^2$.}, and if $V = 0$, it must be a superposition of an incoming and an outgoing wave of frequency $\lambda$.
Moreover, since $V$ is bounded, the wave $\psi$ can ``tunnel" through the imperfect barrier $V$; thus the amplitudes $A_+,B_-$ should affect the amplitudes $A_-,B_+$.
This phenomenon is known as ``scattering", and it can be mathematically described by a linear operator known as the scattering matrix (c.f. Definition \ref{scattering matrix definition}) $S(\lambda)$, which satisfies the equation
$$S(\lambda)\begin{bmatrix}A_-\\B_-\end{bmatrix} = \begin{bmatrix}A_+\\B_+\end{bmatrix}.$$
The function $S$ is meromorphic.

Another meromorphic function of significance is the resolvent $R_V$, which tells us how to solve the inhomogeneous stationary Schrödinger equation: it returns an operator $L^2_{comp}(\RR) \to L^2_{loc}(\RR)$ satisfying the equation
$$(H_V - \lambda^2)R_V(\lambda)f = f.$$
The poles $\lambda$ of $R_V$ are the square roots of isolated points of the spectrum of $H_V$. Such poles are known as ``scattering resonances" (c.f. Definition \ref{scattering resonance multiplicity}), and they have finite multiplicity.
We let $\Res V$ be the set of nonzero resonances of $V$ and let $m_R(\lambda)$ denote the multiplicity of a resonance $\lambda$. This satisfies the equation
$$m_R(-\lambda_0) - m_R(\lambda_0) = \frac{1}{2\pi i}\tr \oint_{\Gamma_{\lambda_0}} S'(\lambda)S(\lambda)^* ~d\lambda$$
where $\Gamma_{\lambda_0}$ is a sufficiently small contour containing $\lambda_0$.
So one is naturally interested in the trace of the log-derivative of $S$, $\tr S'S^*$.
In fact, the Breit-Wigner approximation relates $\tr S'S^*$ to a sum over resonances:
\begin{theorem}
Suppose that $V$ is compactly supported and $\lambda_0 \in \RR$. Then the series
\begin{equation}
\label{Breit-Wigner series}
-\infty < \sum_{\lambda \in \Res V \setminus 0} \frac{\Im \lambda}{|\lambda - \lambda_0|^2} < \infty
\end{equation}
is absolutely convergent, and we have
\begin{equation}
\label{Breit-Wigner formula}
\frac{1}{2\pi i} \tr S'(\lambda_0) S(\lambda_0)^* = -\frac{1}{\pi}|\ch\supp V| - \frac{1}{2\pi}\sum_{\lambda \in \Res V \setminus 0}\frac{\Im \lambda}{|\lambda - \lambda_0|^2}.
\end{equation}
\end{theorem}
Here $\ch\supp V$ is the convex hull of $\supp V$\footnote{In plain English, this means that $\ch \supp V$ is the intersection of all compact intervals $[\alpha, \beta]$, such that there is a $x \in [\alpha, \beta]$ such that $V(x) \neq 0$.} and $|\ch\supp V|$ is its length.
Our first goal is to review the proof of the Breit-Wigner approximation outlined by Dyatlov and Zworski in their book \cite{dyatlov2019mathematical}; c.f. Theorem \ref{proof of Breit-Wigner}.

Our main goal is to consider when the Breit-Wigner approximation can be reasonably generalized to non-compactly supported potentials satisfying the following condition:
\begin{definition}
A function $V \in L^\infty(\RR)$ is said to be \dfn{super-exponentially decreasing} if for every $N \in \NN$ we can find a $C_N > 0$ such that
$$V(x) \leq C_Ne^{-N|x|}.$$
\end{definition}
For example, a Gaussian is super-exponentially decreasing. We will indicate that a function $V$ is super-exponentially decreasing by writing $V(x) = O(e^{-|x|\infty})$.
One can define resonances using the Fourier-Laplace transform, which will exist for a super-exponential decreasing potential; so it is reasonable to demand this hypothesis.

\begin{definition}
The \dfn{Breit-Wigner series} for $V$ is defined by
$$B(V) = -\sum_{\lambda \in \Res V \setminus 0} \frac{\Im \lambda}{|\lambda|^2}.$$
\end{definition}
We will always assume $0 \notin \Res V$ when referring to $B(V)$, even if we do not explicitly note this.
Since $\Res V$ is a discrete set, this will give a $\delta > 0$ such that every $\lambda \in \Res V$ has $|\lambda| > \delta$, avoiding a trivial form of divergence.

By (\ref{Breit-Wigner series}), $B(V)$ converges if $V$ is compactly supported.
It is natural, then, to ask whether the Breit-Wigner approximation still holds when $V$ is not compactly supported. It is not obvious that the definition of the scattering matrix $S$ makes sense in this case, but in fact, one can generalize the definition appropriately; see the remarks after Lemma 3.2 in \cite{froese1997asymptotic}.
Under appropriate assumptions $\tr S'S^*$ remains meromorphic, yet $|\ch\supp V|$ is now infinite, since $V$ is not compactly supported. Therefore, for (\ref{Breit-Wigner formula}) to remain formally valid, (\ref{Breit-Wigner series}) must fail, and hence $B(V) = \infty$.
\begin{conjecture}
\label{strong conjecture}
Let $V(x) = O(e^{-|x|\infty})$. The Breit-Wigner series $B(V)$ converges if and only if $V$ is compactly supported.
\end{conjecture}

This conjecture is really a statement about the distribution of the discrete set $\Res V$.
``Most" of the points of $\Res V$ lie in the lower-half plane $\CC_-$, and the Breit-Wigner series will converge just in case the resonances have small imaginary part or large real part.

\section{The main conjecture}
The distribution of $\Res V$ is in general quite difficult to study.
However, Froese made a conjecture \cite[Conjecture 1.2]{froese1997asymptotic} which gives one control over the distribution of $\Res V$, and proved that a large class of potentials including Gaussians satisfy this conjecture.

To state Froese's conjecture, recall that if $V(x) = O(e^{-|x|\infty})$, then its Fourier transform $\hat V$ is an entire function.
By definition, the order $\rho$ of $\hat V$ is
\begin{equation}
\label{order of Vhat}
\rho = \limsup_{r \to \infty} \frac{\sup_\theta \log \log |\hat V(re^{i\theta})|}{\log r}.
\end{equation}
We say that $\hat V$ is of normal type provided that
$$0 < \limsup_{r \to \infty} \frac{\sup_\theta \log|\hat V(re^{i\theta})|}{r^\rho} < \infty.$$
We say that $\hat V$ is of completely regular growth provided that $\hat V$ is of normal type and that on a set $A \subseteq (0, \infty)$ of density $1$ in $(0, \infty)$, the $\limsup$ in the definition of normal type is actually a uniform limit.

Under these hypotheses, Froese's conjecture gives us a great deal of control over the distribution of $\Res V$:
\begin{conjecture}[Froese]
Suppose that $V(x) = O(e^{-|x|\infty})$ and $\hat V$ is of completely regular growth. Then in the lower-half plane $\CC_-$, the distribution  of $\Res V$ is identical to the distribution of the zeroes of
\begin{equation}
\label{associated entire function}
F(z) = \hat V(2z)\hat V(-2z) + 1.
\end{equation}
\end{conjecture}
We thus retreat to a weaker form of Conjecture \ref{strong conjecture}, that, given Froese's conjecture, is somewhat more plausible.
\begin{conjecture}
\label{weak conjecture}
Suppose that $V(x) = O(e^{-|x|\infty})$, $V$ is not compactly supported and $\hat V$ is of completely regular growth. Then either $B(V)$ diverges, or $V$ is a counterexample to Froese's conjecture.
\end{conjecture}

Aside from the heuristic evidence for Conjecture \ref{weak conjecture}, we prove a special case:
\begin{theorem}
\label{divergence of breit wigner, preliminary version}
Suppose that $V(x) = O(e^{-|x|\infty})$, $V$ is not compactly supported and $\hat V$ is of completely regular growth. Let $F$ be given by (\ref{associated entire function}) and $\rho$ be given by (\ref{order of Vhat}). Let
$$h(\theta) = \limsup_{r \to \infty} \frac{\log|F(re^{i\theta})|}{r^\rho}.$$
For $0 \leq \theta < \varphi \leq 2\pi$, let
$$s(\theta, \varphi) = h'(\theta) - h'(\varphi) + \rho^2\int_\theta^\varphi h(\xi) ~d\xi.$$
Suppose that one of the following is true:
\begin{enumerate}
\item There are $\theta < \varphi$ such that $\pi \notin (\theta, \varphi)$ and $s(\theta, \varphi) \neq 0$.
\item There are $\theta < \varphi$ such that $\pi \notin (\theta, \varphi)$ and $s(\theta, \varphi)$ is not defined.
\item One has
$$\lim_{\varphi \to \pi} s(\pi, \varphi) = 0.$$
\end{enumerate}
Then either the Breit-Wigner series $B(V)$ diverges, or $V$ is a counterexample to Froese's conjecture.
\end{theorem}
We restate and prove Theorem \ref{divergence of breit wigner, preliminary version} as Theorem \ref{divergence of breit wigner}.

\section{Some explicit examples}
Let us first note that the assumption that $V$ is super-exponentially decreasing is not sharp. In fact, the P\"oschl-Teller well, given by
$$V(x) = -\frac{2}{\cosh^2(x)},$$
has a divergent Breit-Wigner series. In fact, it has resonances of the form $-i(n+2)$, $n \in \NN$ \cite{cevik_2016}, and so
$$B(V) \geq \sum_{n=0}^\infty \frac{n+2}{|n+2|^2} = \sum_{n=2}^\infty n^{-1} = \infty.$$
Such a potential is not super-exponentially decreasing.

Let us check that the Breit-Wigner series of the Gaussian potential $V(x) = e^{-x^2}$ also diverges, without using Theorem \ref{divergence of breit wigner, preliminary version}.
Doing so will illustrate some of the techniques that we will use in the proof. First, up to an irrelevant constant, we have $\hat V(z) = e^{-z^2}$. The function $F$ given by (\ref{associated entire function}) is therefore $F(z) = e^{-8z^2} + 1$.
By Theorem \ref{Froese conjecture theorem}, a result of Froese \cite[Theorem 1.2]{froese1997asymptotic}, it follows that the asymptotic distribution of resonances of $V$ is identical to the asymptotic distribution of solutions of the equation $e^{-8z^2} = -1$ such that $\Im z \leq 0$. Such solutions are of the form
$$z = \sqrt{\frac{\pi(2n + 1)}{8}}(-1)^{1/4}$$
for $n \in \NN$ and the value of $(-1)^{1/4}$ chosen to be in the lower-half plane.
Thus we have
$$B(V) \sim \sqrt{\frac{8}{\pi}} \sum_{n=0}^\infty \frac{\sqrt{2(2n+1)}}{2n + 1} = \sqrt{2\pi} \sum_{n=0}^\infty \frac{1}{\sqrt{2n+1}}$$
which certainly diverges.

The constant $\sqrt 2$ that appears in the summand $\sqrt{2(2n+1)}(2n+1)^{-1}$ arises from the fact that $\sin((-1)^{1/4}) = 1/\sqrt 2$, but we just needed that it is nonzero.
This explains the added hypothesis in the statement of Theorem \ref{divergence of breit wigner, preliminary version} that does not already appear in Conjecture \ref{weak conjecture}.
If one could show that a non-compactly supported potential has ``enough" resonances which are ``far from $\RR$"\footnote{By ``far from $\RR$", we mean that there is a uniform $\varepsilon > 0$ such that for every resonance $re^{i\theta}$ which is ``far from $\RR$", $\sin \theta > \varepsilon$.}, one could remove this hypothesis.
For example, one may hope to use complex scaling to remove this hypothesis if $V$ is in fact holomorphic in a neighborhood of $\RR$.

In addition, while we used the fact that the zeroes of $F$ were distributed at a square-root rate, we only needed them to be distributed at most linearly. This follows essentially because the order $\rho$ of $F$ is $2$; we will show that the any Fourier-Laplace transform of any potential that we are considering has zeroes which are distributed at most linearly ($\rho \geq 1$).

\section{Outline of the thesis}
In Chapter \ref{review of complex analysis}, we review the complex-analytic preliminaries that will be used throughout this thesis.
In particular, we give a proof of Titchmarsh's theorem on the zeroes of a Fourier transform, Theorem \ref{Titchmarsh I}, which is somewhat more detailed than the proof that appears in H\"ormander's book \cite{hormander2004analysis}.
In Chapter \ref{review of scattering theory}, we review the basic theory of scattering for compactly supported potentials, roughly following Dyatlov and Zworski \cite{dyatlov2019mathematical}, and use Theorem \ref{Titchmarsh I} to give a proof of the Breit-Wigner approximation, Theorem \ref{proof of Breit-Wigner}.
Finally, in Chapter \ref{SED potentials chapter}, we discuss relevant results from Froese's paper \cite{froese1997asymptotic} and then prove Theorem \ref{divergence of breit wigner, preliminary version}.


\section{Acknowledgements}
I would like to thank Prof. Maciej Zworski for suggesting this topic and mentoring me, as well as teaching me much of what I know about partial differential equations, harmonic analysis, and complex analysis. This thesis would not have appeared without him.
I would also like to thank Erik Wendt and James Leng for various helpful discussions.

This work is dedicated to Java Darleen Villano, who has given me emotional (and mathematical!) support when I needed it the most.


\chapter{Review of complex analysis}
\label{review of complex analysis}
We review the main prerequisites of this thesis in this chapter.
In Section \ref{growth of fourier transforms} we recall the basics of distributional calculus, Fourier-Laplace transforms, and the Paley-Wiener theorem, and in Section \ref{growth of entire functions} we discuss the relationship of the growth of an entire function $f$ to the distribution of the zeroes of $f$. Both of these sections are used throughout.
In Sections \ref{subharmonic functions} and \ref{Titchmarsh section} we give Beurling's proof of Titchmarsh's theorem, which is used in the proof of the Breit-Wigner approximation.
In Section \ref{noncommutative sequence space section} we discuss the ``Schatten classes" $B^p$, which are necessary to define the Fredholm determinant used in the setup of the proof of main theorem.

\section{The Fourier-Laplace transform}
\label{growth of fourier transforms}
We recall basic facts about the Fourier-Laplace transform of a distribution. The exposition given here is based on \cite[Chapter VII]{hormander2015analysis}.

Let $\mu$ be a distribution on $\RR$, i.e. a continuous linear functional space of test functions $C^\infty_{comp}(\RR)$. We will abuse notation and write $\int_E f\mu$ for the pairing of $f1_E$ and $\mu$, whenever it is defined. If $\mu$ satisfies certain growth conditions, then the Fourier-Laplace transform
$$\hat \mu(\xi) = \int_{-\infty}^\infty e^{-ix\xi} \mu(x) ~dx$$
is well-defined and is also a distribution.
We will abuse terminology and refer to the Fourier-Laplace transform as simply the Fourier transform, since it is the analytic continuation of the Fourier transform from $\RR$ to the entire plane $\CC$, and we will rarely need to refer to the classical Fourier transform (only defined for $\xi \in \RR$).
However, we will on occasion need to talk about the Laplace transform, i.e. the function $\eta \mapsto \hat \mu(i\eta)$ on $\RR$.

\begin{definition}
    For $\alpha,\beta$ multiindices, we define the $(\alpha,\beta)$th \dfn{Schwartz seminorm} on $\Omega \subseteq \CC^n$ by
$$||f||_{\alpha,\beta} = \sup_{x \in \Omega} |x^\alpha \partial^\beta f(x)|.$$
    The locally convex space of all $f$ for which every Schwartz seminorm $||f||_{\alpha,\beta}$ is finite is called the \dfn{Schwartz space}. A \dfn{tempered distribution} is a distribution $g$ whose pairing with every element of Schwartz space is finite.
\end{definition}
    If a distribution is tempered on $\RR^n$, then its Fourier transform is well-defined and also tempered. This can be easily proven using duality once it is shown that the Fourier transform is an automorphism of Schwartz space. Most distributions or functions which are ``not too discontinuous" and ``do not grow too fast" are tempered distributions. For example, any compactly supported distribution is tempered, as is any smooth, polynomially growing function.

If $E \subset \RR$ is bounded, then the convex hull of $E$, $\ch E$, is defined to be the intersection of all compact, convex subsets containing $E$; since $\ch E$ is connected, it must be the compact interval $[a, b]$, where $a = \inf E$ and $b = \sup E$.
\begin{definition}
    The \dfn{supporting function} $h_E$ of a bounded set $E \subset \RR$ is defined for $\xi \in \RR$ by
$$h_E(\xi) = \sup_{x \in E} x\xi.$$
\end{definition}
    Taking the closure of $E$ will not affect $h_E$. Then for $\xi > 0$, $h_E(\xi) = b\xi$, $h_E(0) = 0$, and $h_E(-\xi)     = -a\xi$. Taking the convex hull will not change $a$ or $b$, so $h_E = h_{\ch E}$.

      We recall that the notation $A \lesssim_t B$ means that there is a constant $C > 0$ which depends on $t$, such that $A \leq BC$.
\begin{lemma}
\label{fourier transforms are horizontally bounded}
  Let $F$ be an entire function, and assume that there is a super-exponentially decreasing distribution $\mu$ such that $F = \hat \mu$. Then for every $\eta$,
  $$|F(\cdot + i\eta)| \lesssim_{\eta,\mu} 1$$
  and in particular $F|_\RR$ is a tempered distribution.
\end{lemma}
\begin{proof}
  Fix $\eta$; then for every $\xi$,
\begin{align*}
  |F(\xi + i\eta)| &= \left|\int_{-\infty}^\infty \mu(x)e^{-ix\xi}e^{x\eta}~dx\right| \\
  &\leq \int_{-\infty}^\infty |\mu(x)|e^{x\eta} ~dx.
\end{align*}
  Since $\mu$ is super-exponentially decreasing, $\langle \mu, e^{\eta \cdot}\rangle$ exists, and is a bound on $|F(\xi + i\eta)|$ which is uniform in $\xi$.

  In particular $F|_\RR$ is bounded and smooth, hence tempered.
\end{proof}
    Let us fix some notation. Let $E_\delta$ denote the ball around $E$ of radius $\delta$; that is,
$$E_\delta = \{x \in \RR: \exists y \in E~|x - y| < \delta\}.$$
    For $\mu$ a distribution on $\RR$, we will write $||\mu|| = |\int_{-\infty}^\infty \mu(x) ~dx|$, if such an integral is indeed finite. So if $\mu$ is actually a positive function, then $||\mu|| = ||\mu||_{L^1}$.

    \begin{definition}
    Let $f$ be a entire function. The \dfn{order} $\rho$ of $f$ is defined by
    $$\rho = \inf \{m \in \RR: f(z) = O(\exp(|z|^m))\}.$$
    \end{definition}
    It is easy to see that the order $\rho$ of $f$ is given by
    $$\rho = \limsup_{r \to \infty} \frac{\log \log \sup_{|z| = r} |f(z)|}{\log r}.$$
    This motivates the following definition.

    \begin{definition}
    The \dfn{type} $\tau$ of $f$ is
    $$\tau = \limsup_{r \to \infty} \frac{\log \sup_{|z| = r} |f(z)|}{r^\rho}.$$
    If $\tau \in (0, \infty)$, then we say that $f$ is has a \dfn{normal type}.
    If $\rho = 1$ and $f$ is of normal type, then we say that $\tau$ is the \dfn{exponential type} of $f$.
    \end{definition}

We now show that a quantitative property of a distribution (the length of its support) is related to a quantitative property of its Fourier transform, the exponential type of its Fourier transform.
\begin{theorem}[Paley-Wiener]
\label{PWS theorem}
    Let $E \subset \RR$ be bounded. If $\mu$ is a distribution on $\RR$ with $\supp \mu \subseteq E$, then the Fourier transform $\hat \mu$ is an entire function satisfying the estimate
$$|\hat \mu(\zeta)| = O(e^{h_E(\Im \zeta)}).$$
    In particular, $\hat \mu$ is of exponential type $\sup_{x \in E} |x|$.
    Conversely, for every entire function $f$ such that $|f(\zeta)| = O(e^{h_E(\Im \zeta)})$, there is a distribution $\mu$ on $\RR$ with $\supp \mu \subseteq \ch E$ and $\hat \mu = f$.
\end{theorem}
\begin{proof}
    Since $\mu$ is compactly supported, it lies in the dual of $C^\infty(\RR)$. So $||\mu||$ is finite, and we can differentiate $\hat \mu$ by putting all derivatives on the smooth function $e^{-ix\xi}$, using integration by parts. Therefore $\hat \mu \in C^\infty(\CC)$. In particular, if $\dbar$ is the Cauchy-Riemann operator, then we can use the fact that $\dbar e^{-ix\xi} = 0$ (since $e^{-i x\xi}$ is clearly entire). Therefore $\hat \mu$ is entire.

    Let $\Im \zeta > 0$ and $\ch E = [a, b]$. Then
$$|\hat \mu(\zeta)| \leq \int_{-\infty}^\infty |e^{-ix\zeta} \mu(x)| ~dx \leq ||\mu|| \int_a^b e^{x \Im \zeta} ~dx \leq ||\mu|| e^{h_E(\Im \zeta)}.$$

    For the converse, let $f$ be an entire function satisfying $|f(\zeta)| = O(e^{h_E(\Im \zeta)})$. Then the restriction $\tilde f = f|_\RR$ is tempered, so has an inverse Fourier transform $\mu$.

    Let $\varphi \in C^\infty_{comp}((-1, 1))$ be a positive function, normalized so that such that $\int_{-1}^1 \varphi = 1$, and set $\varphi_\delta(x) \varphi(x/\delta)/\delta$. (In fact, we can take $\varphi$ to be the standard mollifier.)
    Thus $\mu * \varphi_\delta$ is a mollification of $\mu$, i.e.
    $\mu * \varphi_\delta \in C^\infty(\RR)$ and $\lim_{\delta \to 0} \mu * \varphi_\delta = \mu$. $\widehat{\mu*\varphi_\delta} = \hat \mu  \hat \varphi_\delta = \tilde f \hat \varphi_\delta$.
    Taking the unique analytic continuation of $\tilde f$ to $\CC$, we extend $\tilde f \hat \varphi_\delta$ to $f \hat \varphi_\delta$.

    Since $\varphi_\delta$ is supported in $(-\delta, \delta)$, $|\hat \varphi_\delta(\zeta)| = O(\delta |\Im \zeta|)$. Therefore
$$|f(\zeta) \varphi_\delta(\zeta)| = O(\exp(h_E(\Im \zeta) + \delta|\Im \zeta|)).$$
    If we can prove the converse for $\mu \in C^\infty_{comp}(\RR)$, then we can replace $\mu$ by $\mu * \varphi_\delta$ to show that $\supp (\mu * \varphi_\delta) \subseteq \ch \supp E_\delta$. Since $\delta$ was arbitrary, it will follow that $\supp \mu \subseteq \ch \supp E$. Thus, we may assume without loss of generality that $\mu \in C^\infty_{comp}(\RR)$.

    In fact, if $\mu \in C^\infty_{comp}(\RR)$, then in particular $\mu$ lies in Schwartz space. In this case, we can find a $C_n$, independent of $\zeta = \xi + i\eta$, so that
$$|\zeta^n f(\zeta)| \leq C_n e^{h_E(\Im \zeta)}.$$
    Dividing both sides by $|\zeta|^{-n}$, we have
    $$|f(\xi + i \eta)| \leq C_n \frac{e^{h_E(\eta)}}{|\xi + i\eta|^n}.$$
    Thus for $\eta$ fixed, $f(\cdot + i\eta)$ is rapidly decreasing. Thus we can make a change of variables to see that
$$\mu(x) = \frac{1}{2\pi} \int_{-\infty}^\infty e^{ix\xi} f(\xi) ~d\xi = \frac{1}{2\pi} \int_{-\infty}^\infty e^{ix(\xi+i\eta)} f(\xi + i\eta) ~d\xi.$$
    Therefore
    $$|\mu(x)| \leq C_Ne^{-x\eta + h_E(\eta)} \int_{-\infty}^\infty \frac{d\xi}{|\xi + i\eta|^N}.$$
    We fix a sufficiently large $N$ and let $\eta \to 0$. This proves that $\mu(x) = 0$ if $x\eta \leq h_E(\eta)$, which happens if and only if $x \notin \ch E$. Therefore $\supp \mu \subseteq \ch E$.
\end{proof}
    Since the Paley-Wiener theorem is a biconditional, the estimate
    $$|\hat \mu(\zeta)| \leq ||\mu|| e^{h_E(\Im \zeta)}$$
    is sharp: we cannot replace the $a, b$ appearing in the piecewise-linear definition of $h_E$ with better constants. This precision will be important in the proof of Theorem \ref{Titchmarsh I}.

    For any distribution $\mu$ with bounded support $\supp \mu$, the Paley-Wiener theorem guarantees that $\hat \mu$ is an entire function. In particular, the number of zeroes of $\hat \mu$ lying in any compact set is necessarily finite, even when counted with multiplicity, assuming that $\mu \neq 0$; the zeroes will be counted by Theorem \ref{Titchmarsh I}.

    We will also need the following lower bound on Fourier transforms, which easily follows from the proof of the Paley-Wiener theorem.
\begin{lemma}
\label{fourier transforms are at least exponential type}
Let $\mu$ be a super-exponentially decreasing distribution, and assume that $\hat \mu$ is an entire function. If $\rho$ is the order of $\hat \mu$, then either $\rho \geq 1$, or $\rho = 0$ and $\supp \mu = \singsupp \mu = \{0\}$.
\end{lemma}
\begin{proof}
Let $F = \hat \mu$ and suppose towards contradiction that $\rho < 1$. Then, in particular, $F(\xi + i\eta) = o(e^{\eta})$.

Since $F$ is a Fourier transform, by Lemma \ref{fourier transforms are horizontally bounded}, $\mu$ is tempered.
Let $\varphi$ be the standard mollifier, so $\mu * \varphi_\delta$ is the standard mollification of $\mu$ and $\widehat{\mu * \varphi_\delta} =  F\hat \varphi_\delta$, and
$$|F(\xi + i\eta)\varphi_\delta(\xi + i\eta)| = O(e^{\delta\eta}).$$
Since $\mu * \varphi_\delta$ is compactly supported, by the Paley-Wiener theorem, $\mu*\varphi_\delta$ is supported on $[-\delta, \delta]$.
Since $\delta$ was arbitrary, it follows that $\supp \mu \subseteq \{0\}$ and the claim follows.
\end{proof}


    \section{Growth of entire functions}
    \label{growth of entire functions}
    Recall that our potential $V$ is assumed to be super-exponentially decreasing. This justifies the differentiation under the integral sign
    $$\dbar \hat V(\xi) = \int_{-\infty}^\infty \dbar_\xi V(x) e^{-ix\xi} ~dx = 0$$
    whence $\hat V$ is an entire function. We do not have the Paley-Wiener theorem in this case, but we can still recover some results about the growth of entire functions.

    The growth of an entire function is closely tied to its distribution of zeroes, as given by the angular counting function.
    \begin{definition}
    Let $R > 0$ and let $\theta,\varphi$ be angles; let $\Gamma_{R,\theta,\varphi}$ be the contour around the sector
    $$\{re^{i\xi} \in \CC: r < R \text{ and } \xi \in (\theta, \varphi)\}.$$
    Given an entire function $f$ which is not identically $0$, its \dfn{angular counting function} $N$ to be the number of zeroes of $f$ inside $\Gamma_{R,\theta,\varphi}$, counted by multiplicity.
    \end{definition}
    Note that by the argument principle, the angular counting function could also be defined by the contour integral
    $$N(R, \theta, \varphi) = \int_{\Gamma_{R,\theta,\varphi}} \frac{f'(z)}{f(z)} ~dz.$$

    We now show that we can determine whether a distribution is of compact support by considering its order.
    \begin{lemma}
    Let $\mu$ be a super-exponentially decreasing distribution which is not identically zero, and let $F = \hat \mu$. Suppose that $F$ is of normal type, and let $\rho$ be the order of $F$. Then $\rho \geq 1$, and $\rho = 1$ if and only if $\mu$ has compact support.
    \end{lemma}
    \begin{proof}
    Since $F$ is a Fourier transform, $F(\xi + i\eta)$ is bounded in $\xi$ if $\eta$ is held fixed. To see this, note that
    $$|F(\xi + i\eta)| = \left|\int_{-\infty}^\infty \mu(x)e^{-ix\xi}e^{x\eta} ~dx\right| \leq \int_{-\infty}^\infty |\mu(x)| e^{x\eta} ~dx.$$
    The last integral converges because $\mu$ is assumed super-exponentially decreasing, so $\mu(x) = O(e^{-2x\eta})$ whence $\mu(x)e^{x\eta} = O(e^{-x\eta})$.
    Therefore $F(\xi + i\eta) = O(g(\eta))$ for some function $g$ which does not depend on $\xi$, and $\rho = \inf \{m: g(\eta) = O(e^{\eta^m})\}$.

    If $\mu$ has compact support, then by the Paley-Wiener theorem, Theorem \ref{PWS theorem} for every $\varepsilon > 0$ and some $\tau > 0$ which depends on $\supp \mu$, $g(\eta) = O(e^{\tau\eta}) = O(e^{\eta^{1+\varepsilon}})$, and this estimate is sharp; therefore $\rho = 1$.
    Conversely, if $\rho = 1$, then because $F$ is of normal type there is a $\tau > 0$ such that $g(\eta) = O(e^{\tau\eta})$, and then the Paley-Wiener theorem bounds $\sup\{|x|: x \in \supp \mu\}$ in terms of $\tau$, so $\supp \mu$ is compact.

    We finally show that $\rho \geq 1$. So suppose that $\rho < 1$, and let $\chi$ be the indicator function of a compact set $K$ such that $\int_K |\mu| \neq 0$. (Such a set $K$ must exist, since $\mu$ is not identically zero.)
    Since $\rho \neq 1$, $\mu$ is not compactly supported, so there is a $\mu_0$ which does not have compact support such that $\mu = \mu\chi + \mu_0$, so that $F = \widehat{\mu\chi} + \widehat{\mu_0}$.
    Since $\mu_0$ is not compactly supported, the order $\rho_0$ of $\widehat{\mu_0}$ satisfies $\rho_0 \neq 1$.
    On the other hand, the order of $\widehat{\rho\chi}$ is $1$.

    If $\rho_0 < 1$, then $\widehat{\mu_0}$ is strictly dominated by $\widehat{\mu\chi}$ along any line $\xi + i\eta$ as $\eta \to \infty$ and $\xi$ is held fixed, so $\rho$ is given by the order of $\widehat{\mu\chi}$, and so $\rho = 1$. (Running this argument again then shows that $\rho_0 < 1$ implies a contradiction, but we did not know this a priori.)
    On the other hand, if $\rho_0 > 1$, then $\widehat{\mu\chi}$ is strictly dominated by $\widehat{\mu_0}$ on the imaginary axis, so $\rho = \rho_0 > 1$.
    Either way, $\rho \geq 1$.
    \end{proof}

    If $f$ is of normal type $\tau$, then we can understand how quickly $f$ grows along a ray $\{re^{i\theta}: r > 0\}$ by understanding the growth rate of the logarithm in the definition of $\tau$. This information is encoded by the indicator function of $f$.
    \begin{definition}
    Let $f$ be an entire function of order $\rho$ and normal type. The \dfn{indicator function} $h$ of $f$ is given by
    $$h(\theta) = \limsup_{r \to \infty} \frac{\log|f(re^{i\theta})|}{r^\rho}.$$
    Suppose that $\alpha < \beta$ are angles. If the $\limsup$ appearing in the definition of $h$ is actually a uniform limit as $r \to \infty$ along a subset of $\{re^{i\theta}: r>0\}$ of density $1$ for every $\theta \in (\alpha, \beta)$, then $f$ is said to have \dfn{completely regular growth} in $(\alpha, \beta)$. If $f$ is of completely regular growth and order $\rho$, then we define
    $$s(\theta, \varphi) = h'(\theta) - h'(\varphi) + \rho^2 \int_\theta^\varphi h(\xi) ~d\xi$$
    for every $\theta, \varphi$ such that $h'$ exists and is continuous at $\theta,\varphi$.
    \end{definition}
    We will need two facts about the indicator function of a function of completely regular growth, and we omit their proof. They can be found in Levin \cite[Chapter III]{levin1964distribution}.
    \begin{theorem}
    \label{indicator function is C1}
    For every function $f$ of completely regular growth, there is a countable set $Z \subset [0, 2\pi]$ such that the indicator function $h$ is continuously differentiable on $[0, 2\pi] \setminus Z$. In particular, for every $\theta \notin Z$, $\varphi \mapsto s(\theta, \varphi)$ is continuous on $Z$.
    \end{theorem}
    \begin{theorem}
    Let $f$ be a function of completely regular growth. Let $\alpha < \theta < \varphi < \beta$ be angles such that $s(\theta,\varphi)$ exists. Then
    $$s(\theta, \varphi) = 2\pi\rho \lim_{r\to\infty} \frac{N(r, \theta, \varphi)}{r^\rho}.$$
    \end{theorem}

    Finally, we will need a Weierstrass-type theorem that was due to Titchmarsh. We omit its proof, but refer the reader to Titchmarsh's original paper \cite[Theorem VI]{titchmarsh1926zeros}.
    \begin{theorem}
    \label{Titchmarsh II}
    Suppose that $\mu$ is a distribution and $\ch \supp \mu = [a, b]$. Let $\{z_n\}$ be an enumeration of the zeroes of $\hat \mu$ such that $|z_n| \leq |z_{n+1}|$, and suppose that $\hat \mu(0) \neq 0$. Then
    $$\frac{\hat \mu(z)}{\hat \mu(0)} = e^{-i(a+b)z/2} \prod_n\left(1 - \frac{z}{z_n}\right).$$
    \end{theorem}


    It follows from Theorem \ref{Titchmarsh I} that there are infinitely many $z_n$, and that $N(R) \sim R$, so one can show that the product in Theorem \ref{Titchmarsh II} is just conditionally convergent. It is this reason that forces us to add the condition $|z_n| \leq |z_{n+1}|$, as otherwise we could use a Riemann rearrangement to find a counterexample to Theorem \ref{Titchmarsh II}.

\section{Subharmonic functions}
\label{subharmonic functions}
We will review the proof of the Riesz representation formula for certain subharmonic functions on the upper-half plane $\CC_+$.
This material will be used in the proof of Theorem \ref{Titchmarsh I}, which in turn will be used in the proof of the Breit-Wigner approximation for compactly supported potentials, Theorem \ref{proof of Breit-Wigner}.

Fix an open set $\Omega \subseteq \CC$. Recall that a function $u: \Omega \to [-\infty, \infty)$ which is upper-semicontinuous is called subharmonic if for each $z \in \CC$, the averages
$$M(z, r) = \frac{1}{2\pi} \int_0^{2\pi} u(z + re^{i\theta}) ~d\theta$$
are increasing in $r$. This condition is logically equivalent to assuming that $u$ satisfies a maximum principle: for every compact set $K$ with nonempty interior $U$, if $u|_K$ attains its maximum on $U$, then $u|_K$ is a constant. To avoid trivialities, we shall assume that $u$ is not identically $-\infty$, though not every author makes this assumption.

It is a well-known result that an upper-semicontinuous function $u$ is subharmonic iff the weak Laplacian $\Delta u \geq 0$; that is, for any nonnegative test function $\varphi \in C^\infty_{comp}(\Omega)$,
$$\int_\Omega u(z) \Delta \varphi(z) ~dz \geq 0.$$ If actually $\Delta u = 0$, then we call $u$ harmonic; by a typical mollification argument, if $u$ is harmonic, then $u$ is actually smooth (even real analytic), but such niceties may not be true for subharmonic functions. The reason why we refer to functions $u$ with $\Delta u \geq 0$ as subharmonic rather than superharmonic is that $-\Delta$ is a positive operator, and so we think of applying $\Delta$ as akin to ``multiplying by a negative function". Such a function lies in $L^1_{loc}(\Omega)$. (For the proofs, see \cite[Chapter 1]{hormander1973introduction}.)

By the Paley-Wiener theorem, Theorem \ref{PWS theorem}, if $\mu$ is a distribution with $\ch \supp \mu = [a, b]$, then $\hat \mu$ is an entire function which satisfies the estimate
$$\hat \mu(x + iy) \leq Ce^{h(y)}$$
for some constant $C$, where $h(y) = by$ for $y > 0$, $h(y) = ay$ for $y < 0$. In particular, in the upper half-plane $\CC_+$, we have the estimate
$$\hat \mu(x + iy) \leq Ce^{by}.$$
Moreover, since $\hat \mu$ is holomorphic, it solves the Cauchy-Riemann equation $\dbar \hat \mu = 0$. Since we can factor the Laplacian as $4\Delta = \partial \dbar$, it follows that $\Delta \hat \mu = 0$.

Let
$$E(z) = \frac{\log |z|}{2\pi}.$$
Then $E$ is the fundamental solution of the Laplacian. This means that $\Delta E = \delta_0$, where $\delta_z$ denotes the Dirac distribution centered at $z$. In particular, the solution of the equation $\Delta u = f$ is $u = E*f$.

Since $\hat \mu$ is holomorphic, if it has a zero $z \in \CC_+$ of multiplicity $m$, we can write
$$\hat \mu(\zeta) = (\zeta - z)^m g(\zeta)$$
for some holomorphic function $g$ with $g(z) \neq 0$. If $g \neq 0$ everywhere, then, because $\Delta g = 0$, it follows that
$$\Delta \log |\hat \mu| = m\log |\zeta - z| + \log |g| = 2\pi m \delta_z.$$
Using a partition of unity to sum over all zeroes in this manner, it follows that if $Z$ denotes the multiset of zeroes of $\hat \mu$ counted with multiplicity, then
$$\Delta(\log |\hat \mu|) = 2\pi \sum_{z \in Z} \delta_z \geq 0.$$
It follows that $\log |\hat \mu|$ is subharmonic and, on $\CC_+$, satisfies the estimate
$$\log |\hat \mu(x + iy)| \leq C + Dy$$
for some constants $C, D$, by the Paley-Wiener theorem, Theorem \ref{PWS theorem}

\begin{definition}
    Let $u$ be a subharmonic function on $\CC_+$. If there are constants $C, D > 0$ so that $u(x + iy) \leq C + Dy$, then we say that $u$ is \dfn{imaginary-sublinear}.
\end{definition}
It is immediate that $\log |\hat \mu|$ is imaginary-sublinear.

\begin{lemma}
    \label{imaginary sublinear limit}
    Let $u$ be an imaginary-sublinear subharmonic function on $\CC_+$ and define
    $$\gamma = \lim_{y \to \infty} \sup_x u(x + iy).$$
    Then $\gamma$ is well-defined, and $\gamma \in (-\infty, D]$.
\end{lemma}
\begin{proof}
    We first let
    $$M(y) = \sup_x u(x + iy).$$
    Then $\gamma = \lim_y M(y)$, and $M(y) \leq C + Dy$ since $u$ is imaginary-sublinear.

    Let us prove that $M$ is convex. Let $0 < a < b$ and let $L: \RR \to \RR$ be a linear function such that $M(a) \leq L(a)$ and $M(b) \leq L(b)$. Let
    $$v(x + iy, \varepsilon) = u(x + iy) - L(y) - \varepsilon(x^2 - (y^2 - b^2)).$$
    Then $v(\cdot + iy_0, \varepsilon) \leq 0$ for $y_0 \in \{a, b\}$. Similarly, $\lim_{x \to \pm \infty} v(x + \cdot, \varepsilon) = -\infty$. Moreover, $\Delta v(\cdot, \varepsilon) \geq \Delta u \geq 0$.

    Applying the maximum principle to a sufficiently large compact subset $K$, and noting that $v(\cdot, \varepsilon) \leq 0$ on $\partial K$, $v(\cdot, \varepsilon) \leq 0$ on $K$, hence globally since $K$ was arbitrary. Taking $\varepsilon \to 0$, we see that $u(x + iy) \leq L(y)$, so maximizing over $x$, $M \leq L$ on $[a, b]$. So $M$ is convex. But the limit of a sublinear, convex function exists, so $\gamma$ is well-defined, and $\gamma > -\infty$ since $u \in L^1_{loc}(\Omega)$, hence not $-\infty$ except on a discrete set. The bound $\gamma \leq D$ follows easily.
\end{proof}

We now begin working towards a representation formula for imaginary-sublinear functions on the upper-half plane in terms of their Laplacian and their boundary values. To do this, we construct the Green function of $\Delta$ in the half plane.
\begin{definition}
    The \dfn{Green function} for $\Delta$ on $\CC_+$ is defined on $\CC_+ \times \CC_+$ by
$$G(z, w) = E(z - w) - E(z - \overline w).$$
    The \dfn{Poisson kernel} for $\Delta$ on $\CC_+$ is defined on $\CC+_ \times \RR$ by
$$P(z, x) = -\frac{\partial G(z, x + iy)}{\partial y}|_{y = 0}.$$
\end{definition}
  To motivate the definition of a Green function, suppose that we want to solve the boundary-value problem for $\Delta$ on $\CC_+$. That is, given any function $f \in C(\RR)$, we want to find a harmonic function $u$ on $\CC_+$ which continuously extends to $\RR$, such that $u|_\RR = f$. If we can find a function $G$ on $\CC_+ \times \CC_+$ such that $\Delta G(z, w) = \delta_z w$ which continuously extends to $0$ on $\RR$, then one can use Stokes's theorem to see that
$$u(z) = \int_{-\infty}^\infty f(x) P(z, x) ~dx,$$
  since by definition the Poisson kernel is the normal derivative (i.e. infinitesimal of the flux) of $G$ along $\RR$. Now $G(z, w) = E(z - w)$ would suffice as such a function, except that it is nonzero at the boundary. On the other hand, $\Delta_z E(z - \overline w) = 0$ for $z, w \in \CC_+$, and by symmetry introducing this error term will cancel out the boundary term in $E(z - w)$.

    One has $\Delta_w G(z, w) = \Delta_w E(z - w) = \delta_z$ since $\overline w \notin \CC_+$, hence $z \neq w$. Moreover, $\lim_{w \to 0} G(z, w) = 0$ for $z$ fixed. We can rewrite $G$ as
$$G(z, w) = \frac{1}{2\pi} \log\left|\frac{z - w}{z - \overline w}\right|,$$
    which is clearly homogeneous: for $t > 0$, $G(tz, tw) = G(z, w)$. Since $(z-w)/(z - \overline w) \to 1$ as $z \to \infty$, $G(z, w) \to 0$.  Moreover,
$$\partial_y G(z, x + iy) = -\frac{1}{2\pi} \left(\frac{y - \Re z}{|x + iy - z|^2} - \frac{y + \Re z}{|x + iy - \overline z|^2}\right)$$
  and setting $y = 0$ we have
$$P(z, x) = \frac{\Im z}{2\pi|z-x|^2}.$$
We have the estimate $P(z, x) = O(\Im z|z|^{-2})$ for small $x$ as $z \to \infty$. In the other direction, $P$ is a nascent Dirac mass in the sense that
$$\lim_{b \to 0} P(a + ib, x) = \delta_x(a).$$ Moreover, $\Delta G(\cdot, w) = 0$ away from $w$, so commuting $\partial_y$ and $\Delta$, we see that $P(\cdot, x)$ is harmonic on $\CC_+$.

To prove the representation formula, we will need some estimates on the Green function's order of growth.
\begin{lemma}
\label{estimate on Green function}
    For every $w \in \CC_+$ there is a constant $C > 0$ such that for every $z \in \CC_+$ such that $|z|$ is large enough,
$$\frac{\Im z}{C(1 + |z|)^2} \leq |G(w, z)| \leq \frac{C \Im z}{(1 + |z|)^2}.$$
\end{lemma}
\begin{proof}
  Let $z = x + iy$. Let us Taylor expand $G(w, z)$ in $y$ at the origin, so $G(w, z) = \sum_j c_j(w, x) y^j$. Since $G = 0$ on $\RR$, $c_j = 0$. By definition of the Poisson kernel, $c_1 + P = 0$. By homogeneity,
\begin{align*}
  G(w, z) &= G\left(\frac{w}{|z|}, \frac{z}{|z|}\right) = \sum_{j=0}^\infty c_j\left(\frac{w}{|z|}, \frac{x}{|z|}\right) y^j |z|^{-j}
  \\&= -P\left(\frac{w}{|z|}, \frac{x}{|z|}\right) \frac{y}{|z|} + o\left(\frac{y}{|z|^2} \right) = -\Theta\left(P\left(\frac{w}{|z|}, 0\right)\right)
  \\&= \Theta \left(\frac{\Im z}{|1 + z|^2}\right)
\end{align*}
  where the implied constants are allowed to depend on $w$, and we have used Knuth's big-$\Theta$ notation. Indeed, if $|z|$ is large then $x/|z|$ is small, and so does not contribute meaningfully to the long-term behavior of $P$.
\end{proof}

We shall also need a representation formula for the unit disc $\DD$. We recall that the Cayley transform, which we will denote $z \mapsto z^\flat$ (with inverse $w \mapsto w^\sharp$), conformally transforms $\CC_+$ into $\DD$, and so all that we have proven about $\CC_+$ corresponds to a fact about $\DD$. The Cayley transform is given by
$$z^\flat = \frac{z - i}{z + i}.$$
Pushing forward the Poisson kernel along the Cayley transform, we arrive at the following definition.
\begin{definition}
The \dfn{Poisson kernel} for $\Delta$ on $\DD$ is defined by
$$P^\flat(z, \theta) = P(z^\sharp, (e^{i\theta})^\sharp).$$
\end{definition}
\begin{theorem}[Poisson representation formula]
Let $f \in C(\partial \DD)$ and let
$$F(z) = \int_0^{2\pi} P^\flat(z, \theta)f(e^{i\theta}) ~d\theta.$$
Then $F$ is the unique solution to the boundary-value problem for $\Delta$ with boundary condition $f$.
\end{theorem}
\begin{proof}
Since $P$ is harmonic, it follows that $P^\flat$ is harmonic as well, and
$$\int_0^{2\pi} P^\flat(z, \theta) ~d\theta = \int_{-\infty}^\infty P(z^\sharp, x) ~dx = 1.$$
So for any $r \in (0, 1)$ and any $\varepsilon > 0$, we have the estimate
\begin{align*}
  |F(re^{i\theta}) - f(e^{i\theta})| &\leq \int_0^{2\pi} P^\flat(re^{i\theta}, e^{i\eta})|f(e^{i\theta}) - f(e^{i\eta})| ~d\eta\\
    &\leq \sup_{B_\varepsilon} |f(e^{i\theta}) - f(e^{i\eta})| \\
      &\quad+ \int_{B_\varepsilon^c} P^\flat(re^{i\theta}, e^{i\eta})|f(e^{i\theta}) - f(e^{i\eta})| ~d\eta\\
    &\leq \sup_{B_\varepsilon} |f(e^{i\theta}) - f(e^{i\eta})| \\
      &\quad+ \sup_{B_\varepsilon^c} P^\flat(re^{i\theta}, e^{i\eta}) \int_{B_\varepsilon^c} |f(e^{i\theta}) - f(e^{i\eta})| ~d\eta
\end{align*}
where $B_\varepsilon$ is a interval in $[0, 2\pi]$ modulo $2\pi$ of radius $\varepsilon$ centered on $e^{i\theta}$. Since $f$ is continuous,
$$\lim_{\varepsilon \to 0}\sup_{B_\varepsilon} |f(e^{i\theta}) - f(e^{i\eta})| = 0.$$
On the other hand, since $P(z^\sharp, x) \to 0$ uniformly in $x$ as $z^\sharp \to \infty$, $P^\flat(z, x^\flat) \to 0$ uniformly in $x^\flat$ as $|z| \to 1$, provided that we are away from the singularity $z = x^\flat$. Thus
$$\lim_{r \to 1} F(re^{i\theta}) = f(e^{i\theta})$$
uniformly in $\theta$. So $F|_{\partial \DD} = f$. By the maximum principle, $F$ is unique.
\end{proof}
Recall the reflection principle, which says that $h$ is a harmonic function on $B(0, R) \cap \CC_+$ such that $h|_\RR = 0$, then $h$ extends to a harmonic function on $B(x, R)$ such that
$$h(z) + h(\overline z) = 0.$$
Taking $R \to \infty$, we see that the reflection principle still holds for harmonic functions on $\CC_+$ such that $h|_\RR = 0$. Let us now use the Poisson representation formula to show that the only such functions are linear.
\begin{lemma}
\label{asymptotics for the poisson kernel}
The Poisson kernel $P^\flat$ has the asymptotic expansion
$$P^\flat(z/R, e^{i\theta}) - P^\flat(\overline z/R, e^{i\theta}) = \frac{2\Im z \sin 3\theta}{R\pi} + O(R^{-2})$$
as $R \to \infty$, where $(z, \theta)$ is fixed.
\end{lemma}
\begin{proof}
We Taylor expand $P^\flat(z/R, e^{i\theta}) - P^\flat(\overline z/R, e^{i\theta})$ in $R$ at infinity (in other words, take the Maclaurin expansion in $1/R$). Clearly the zeroth-order term is $0$, and the second order term is $O(R^{-2})$. So we must only show that
$$\lim_{R \to \infty} \partial_{R^{-1}}(P^\flat(z/R, e^{i\theta}) - P^\flat(\overline z/R, e^{i\theta})) = \frac{2}{\pi} \Im z \sin 3\theta.$$

We calculate
$$P^\flat(z, e^{i\theta}) = \frac{\Im\left(\frac{z+1}{iz-i}\right)}{2\pi\left|\frac{z+1}{iz-i} - \frac{e^{i\theta} - 1}{ie^{i\theta} -i}\right|^2}
  = \frac{-\Re\left(\frac{z+1}{z-1}\right)}{2\pi|z + 1 - e^{i\theta} - 1|^2} = \frac{(1 - |z|^2)}{2\pi|z - e^{i\theta}|^2}.$$
Now
$$\partial_t (1 - t|z|^2)|_{t=0} = 0$$
and
$$\partial_t \left(\frac{1}{|tz - e^{i\theta}|^2} - \frac{1}{|t\overline z - e^{i\theta}|^2}\right)|_{t = 0} = 4e^{i(\pi - 3\theta)}\Im z.$$
Therefore
$$\lim_{R \to \infty} \partial_{R^{-1}}(P^\flat(z/R, e^{i\theta} - P^\flat(\overline z/R, e^{i\theta}))) = \Re \frac{2}{\pi}e^{i(\pi - 3\theta)}\Im z = \frac{2}{\pi} \Im z \sin 3\theta,$$
as promised.
\end{proof}
\begin{corollary}
\label{reflected harmonics are linear}
Let $h: \CC \to \RR$ be a harmonic function such that $h(z) + h(\overline z) = 0$ for every $z \in \CC$. Then there is an $A \in \RR$ such that $h(z) = A \Im z$ for every $z \in \CC$.
\end{corollary}
\begin{proof}
  $$\int_\pi^{2\pi} P(z, e^{i\theta})h(e^{i\theta}) ~d\theta = -\int_0^\pi P(\overline z, e^{i\theta})h(e^{i\theta}) ~d\theta$$
  so by the Poisson representation formula, for any $R > 0$,
  $$h(z) = \int_0^\pi (P(z/R, e^{i\theta} - P(\overline z/R, e^{i\theta}))h(Re^{i\theta})~d\theta.$$
  By Lemma \ref{asymptotics for the poisson kernel}, we have
  $$h(z) = \left(\frac{2\Im z}{\pi} + O(R^{-1})\right)\int_0^\pi \frac{h(Re^{i\theta})}{R}\sin 3\theta ~d\theta.$$
  Since the left-hand side of the resulting asymptotic expansion of $h$ does not depend on $R$, the limit of the right-hand side as $R \to \infty$ must exist, and taking
  $$A = \lim_{R \to \infty} \frac{2}{\pi}\int_0^\pi \frac{h(Re^{i\theta})}{R}\sin 3\theta ~d\theta,$$
  we conclude that $h(z) = A\Im z$. Since $h$ is real-valued, $A \in \RR$.
\end{proof}

Now we are ready to prove the Riesz representation formula for $\CC_+$.
\begin{theorem}[Riesz representation formula]
    \index{Riesz representation formula}
    Let $u$ be an imaginary-sublinear subharmonic function on $\CC_+$, let $\gamma$ be as in Lemma \ref{imaginary sublinear limit}, and fix any $w \in \CC_+$. Let $\mu = \Delta v$; then
    \begin{equation}\label{estimate on mu}\int_{\CC_+} \frac{\Im z \mu(z) ~dz}{(1 + |z|)^2} < \infty.\end{equation}
    Moreover $v(\cdot + iy)$ converges to a distribution $\sigma$ on $\RR$ as $y \to 0$ such that, in the sense of distributions,
\begin{equation}\label{estimate on sigma}\int_{-\infty}^\infty \frac{|\sigma(x) ~dx|}{(1 + |x|)^2} < \infty,\end{equation}
    and
\begin{equation}\label{riesz formula}u(z) = \int_{\CC_+} G(z, w) \mu(w) ~dw + \int_{-\infty}^\infty P(z, x) \sigma(x) ~dx + \gamma \Im z.\end{equation}
\end{theorem}
\begin{proof}
    Let us replace $u(z)$ with $u(z) - C - \gamma \Im z$, where $C$ is the constant appearing the definition of an imaginary-sublinear function. Then $u(z) \leq D + \Im z$, and if we take an optimal choice of $D$, then $D \leq 0$, so $u \leq 0$. After completing the proof, we can simply add a $\gamma \Im z$ back in, the $C$ having already been absorbed into the boundary term $\sigma$.

    We first prove (\ref{estimate on mu}). Recall that we have assumed that there is a $z \in \CC_+$ so that $u(z) > -\infty$. Moreover, for any $x \in \RR$, $G(z, x) = 0$ By Lemma \ref{estimate on Green function}, for some constant $B > 0$,
\begin{align*}\int_{\CC_+} \frac{\Im w}{(1 + |w|)^2} \mu(w) ~dw &\leq -B\int_{\CC_+} G(z, w) \Delta u(w) ~dw
    \\&= B \int_{\CC_+} \nabla G(z, w) \nabla u(w) ~dw \\
      &\quad+ B\int_{-\infty}^\infty G(z, x) \nabla u(x) ~dx\\
    &= -B \int_{\CC_+} \Delta G(z, w) u(w) ~dw \\&= -B\int_{\CC_+} \delta_z(w) u(w) ~dw = -Bu(w) < \infty.
  \end{align*}

    Now we decompose $u$. Fix an increasing chain $K_j$ of compact sets which cover $\CC_+$, and let $\chi_j \in C^\infty_{comp}(\CC_+)$ be an increasing chain of cutoff functions which are identically $1$ on $K_j$. Then let
    $$v_j(z) = u(z) - \int_{\CC_+} G(z, w) \chi_j(w)\mu(w) ~dw.$$
\begin{lemma}
    The functions $v_j$ are subharmonic on $\CC_+$, harmonic on $K_j$, and $\leq 0$.
\end{lemma}
\begin{proof}[Proof of lemma]
    We compute
\begin{align*}
    \Delta v_j(z) &= \mu(z) - \int_{\CC_+} \Delta_z(E(z - w) - E(z - \overline w)) \chi_j(w) \mu(w) ~dw \\&= \mu(z) - \int_{\CC_+} (\delta_w(z) - \delta_{\overline w}(z) \chi_j(w) \mu(w) ~dw \\&= \mu(z)(1 - \chi_j(z)),
\end{align*}
    the $\delta_{\overline w}$ term vanishing because $z \neq \overline w$, since $\overline w \notin \CC_+$. Since $\chi_j$ is a cutoff, $\chi_j \leq 1$, so $\Delta v_j \geq 0$. On the other hand, if $z \in K_j$, then $1 - \chi_j(z) = 0$, so $\Delta v_j(z) = 0$. This proves the first two claims.

    Let $\varepsilon > 0$; we will prove that $v_j < \varepsilon$. Since $u$ is subharmonic, $\mu \geq 0$, and $\chi_j \geq 0$ while $G \leq 0$, so
    $$\int_{\CC_+} G(z, w) \chi_j(w) \mu(w) ~d\mu(w) \leq 0,$$ and we are only integrating over $w$ close to the compact set $K_j$. So we can view $w$ as essentially fixed compared to $z$, and apply the estimate $G(z, w) = O(\Im z|z|^{-2})$ to see that
    $$\lim_{z \to \infty} \int_{\CC_+} G(z, w) \chi_j(w) \mu(w) ~d\mu(w) \to 0.$$
    In particular, $\int_{\CC_+} G(z, w) \chi_j(w) \mu(w) ~d\mu(w) > -\varepsilon$ for $|z|$ large enough. Since $u \leq 0$, $v_j(z) < \varepsilon$ for $z$ large enough, hence for any $z$ by the maximum principle. Therefore $v_j < 0$.
\end{proof}
    The $v_j$ form an increasing sequence which is bounded above, so converge to a limit $u_1 \leq 0$. Moreover, $\Delta v_j \to 0$ pointwise, so $u_1$ is harmonic. Meanwhile, $\chi_j \to 1$ pointwise, so if we let
$$u_2(z) = \int_{\CC_+} G(z, w) \mu(w) ~dw,$$
    we arrive at the decomposition $u = u_1 + u_2$, $\Delta u_1 = 0$. We will view $u_1$ as the ``boundary part" of $u$ and $u_2$ as the ``subharmonic part" of $u$.

    We now show that the subharmonic part of $u$ does not contribute to its boundary value.
\begin{lemma}
    In the sense of distributions,
$$\lim_{y \to 0} u_2(\cdot + iy) = 0.$$
\end{lemma}
\begin{proof}[Proof of lemma]
    Let $\varphi \in C^\infty_{comp}(\RR)$ be a test function, $\ch \supp \varphi = [a, b]$. We must show that the limit
$$\lim_{y \to 0} \int_{-\infty}^\infty u_2(x + iy) \varphi(x) ~dx = \lim_{y \to 0} \int_{\CC_+} \mu(w) \int_{-\infty}^\infty G(x + iy, w) \varphi(x) ~dx ~dw = 0.$$
    Now $G(x + iy, w)$ vanishes for fixed $w,x$ as $y \to 0$, and $G(x + iy, w) \varphi(x) = O(\Im w|w|^{-2})$ at infinity by Lemma \ref{estimate on Green function}. If we can show that this bound is valid on compact sets as well, then we will have
$$\int_{\CC_+} \mu(y, w) H(y, w) ~dw \leq \int_{\CC_+} \frac{\Im z \mu(z) ~dz}{(1 + |z|)^2} < \infty$$
    by (\ref{estimate on mu}), whence
$$\lim_{y \to 0} \int_{-\infty}^\infty u_2(x + iy) \varphi(x) ~dx = \lim_{y \to 0} \int_{\CC_+} \mu(y, w) H(y, w) ~dw = 0$$
    by the dominated convergence theorem.

    We now define
$$F(w) = \int_{-\infty}^\infty E(w - x)\varphi(x) ~dx.$$
    Now $E$ is continuous away from $0$, but $\Im w > 0$, so the integrand is continuous. The integral is formally taken over $\RR$, but is actually being taken over $[a, b]$, so the integrand is integrable; hence $F$ is continuous. Since $\Delta E(w - x) = 0$ for $\Im w > 0$, $F$ is harmonic. But $F$ is a convolution, so if $P$ is any linear differential operator in $\Re w$, $PF = E * P\varphi$, which is continuous since $\varphi$ is smooth. Therefore $PF$ is locally bounded. In particular,
$$\partial_{\Im w}^2 F(w) = \Delta F(w) - \partial_{\Re w}^2 F(w) = -\partial_{\Re w}^2 F(w)$$
    which is locally bounded. So $PF$ is locally bounded for any linear differential operator whatsoever. In particular, $F \in C^{Lip}_{loc}(\CC_+)$. Because $F(w)$ is bounded for a fixed $\Im w$, since $F$ is continuous and $\Re a$ ranges over the compact set $[a, b]$, we have $F(w) = O(\Im w)$. That is,
$$\int_{-\infty}^\infty G(x + iy, w) \varphi(x) ~dx = F(\overline w + iy) - F(w + iy) = O(\Im w)$$
    for bounded $y$. In particular, the integral is bounded on any compact set in $w$. Therefore
$$\int_{-\infty}^\infty G(x + iy, w) \varphi(x) ~dx = O\left(\frac{|\Im w|}{(1 + |w|^2)}\right).$$
    By the remarks at the start of this proof, the lemma follows.
\end{proof}
    We now will construct a representation formula for the boundary part.
\begin{lemma}
  \label{approximate sigma rep}
    For any $\varepsilon > 0$ and $y > \varepsilon$,
$$u_1(x + iy) = \int_{-\infty}^\infty P(x + i(y-\varepsilon), t) u_1(t + i\varepsilon) ~dt.$$
\end{lemma}
\begin{proof}[Proof of lemma]
    Let $\psi_j$ be an increasing sequence of cutoff functions on $\RR$ which are identically $1$ on $[-j, j]$. Then
$$h_j(z) = u_1(z + i\varepsilon) - \int_{-\infty}^\infty P(z, x)\psi_j(x)u_1(x + i\varepsilon) ~dx.$$
    Then for $z \in \CC_+$,
\begin{align*}
  \Delta h_j(z) &= -\Delta \int_{-\infty}^\infty P(z, x)\psi_j(x) u_1(x + i\varepsilon) ~dx \\
    &= -\int_{-\infty}^\infty \Delta_z P(z, x) \psi_j(x) u_1(x + i\varepsilon) ~dx = 0
\end{align*}
    since $P(\cdot, x)$ is harmonic on $\CC_+$. So the $h_j$ are harmonic on $\CC_+$. Since $P$ is a nascent Dirac mass, if $\Im z = 0$, then
$$h_j(z) = u_1(z + i\varepsilon) - \int_{-\infty}^\infty \delta_x(z) \psi_j(x) u_1(x + i\varepsilon) ~dx = u_1(z + i\varepsilon)(1 - \psi_j(x)) \leq 0$$
    since $u_1 \leq 0$. On the other hand, $P(z, x)$ is small when $|z|$ is large; so
    \begin{align*}\limsup_{z \to \infty} h_j(z) &\leq -\liminf_{z \to \infty} \int_{-\infty}^\infty P(z, x) \psi_j(x) u_1(x + i\varepsilon) ~dx
      \\&= -\int_{-\infty}^\infty \liminf_{z \to \infty} P(z, x) \psi_j(x) u_1(x + i\varepsilon) ~dx = 0.\end{align*}
    Therefore $h_j|_{\partial \CC_+} \leq 0$, so by the maximum principle, $h_j \leq 0$.

    Therefore the $h_j$ increase to a harmonic function $h$, which by the reflection principle, extends uniquely to $\CC$ and satisfies $h(z) + h(\overline z) = 0$. By Corollary \ref{reflected harmonics are linear}, we can find a $A \in \RR$ such that $h(z) = A\Im z$. By our assumption that $C = \gamma = 0$, it follows that $A = 0$, so $h = 0$. But
  \begin{align*}u_1(z + i\varepsilon) &= \lim_{j \to \infty} h_j(z) + \int_{-\infty}^\infty P(z, x)\psi_j(x)u_1(x + i\varepsilon)~dx \\
  &= \int_{-\infty}^\infty P(z, x)u_1(x + i\varepsilon) ~dx.\end{align*}
    The lemma then follows by changing variables.
\end{proof}
    Since $\gamma = 0$ and $u \leq 0$, for every $\delta > 0$ and every $x \in \RR$, we can choose $y > 0$ so large that $u(x + iy) > -\delta y$. Since $G$ vanishes at infinity, so does $u_2$, so if $y$ is chosen even larger still, we can arrange that $u_1(x + iy) > -\delta y$. By Lemma \ref{approximate sigma rep} with a change of variable in $y$,
$$-\int_{-\infty}^\infty \frac{u_1(x + i\varepsilon)}{|x + iy - t|^2} ~dt < \frac{\pi \delta y}{y - \varepsilon}.$$
    Taking $\varepsilon$ and $R$ large enough, we can find a constant $B > 0$ such that
$$\left(\int_{-\infty}^{-R} + \int_R^\infty\right) \frac{-u_1(x + i\varepsilon)}{x^2} ~dx < B \delta.$$
    This estimate is uniform in $\varepsilon$, and $u_1(\cdot + i\varepsilon)$ is bounded on $[-R, R]$ since it is harmonic; these bounds are also uniform in $\varepsilon$. Certainly any test function on $\RR$ decays faster than $x^2$, so it follows that $u_1(\cdot + i\varepsilon)$ is uniformly bounded in $\varepsilon$ as a functional acting on $C^\infty_{comp}(\RR)$. So by the Banach-Alaoglu theorem, the space of all such distributions $u_1(\cdot + i\varepsilon)$ is weakstar compact, and so there is a weak limit $\sigma$ as $\varepsilon \to 0$. Since $\sigma$ satisfies the same bounds as $u_1(\cdot + i\varepsilon)$, (\ref{estimate on sigma}) follows, and, passing to the limit in Lemma \ref{approximate sigma rep},
$$u_1(z) = \int_{-\infty}^\infty P(z, x) \sigma(x) ~dx.$$
    Plugging the representation formulae for $u_1$ and $u_2$ back into the decomposition $u = u_1 + u_2$, we complete the proof.
\end{proof}


\section{Titchmarsh's theorem}
\label{Titchmarsh section}
Our goal is to prove the following theorem, which was first proven by Titchmarsh \cite{titchmarsh1926zeros}.
The proof we give is based on the proof of Beurling that was first published in \cite[Chapter XVI]{hormander2004analysis}.
\begin{theorem}[Titchmarsh]
    \index{Titchmarsh's theorem}
    \label{Titchmarsh I}
    Let $\mu$ be a distribution with $\ch \supp \mu = [a, b]$. Let $N(R) = N(R, 0, 2\pi)$ be the zero-counting function for $\hat \mu$. Then
    $$\lim_{R \to \infty} \frac{N(R)}{R} = \frac{b-a}{\pi}.$$
\end{theorem}
To prove Theorem \ref{Titchmarsh I}, we will need to appeal to the following application of the Riesz representation formula.
\begin{lemma}
    \label{gamma in mean}
    Let $u$ be an imaginary-sublinear subharmonic function on $\CC_+$, and let $\gamma$ be as in Lemma \ref{imaginary sublinear limit}. Then
$$\lim_{t \to \infty} \frac{u(t(x + iy))}{t} = \gamma y$$
    in $L^1_{loc}(\overline{\CC_+})$.
\end{lemma}
\begin{proof}
    Fix a compact set $K \subset \overline{\CC_+}$. If $u$ is a constant, then both sides of the claimed equation are $0$, so by linearity we may subtract the constant $C$ that appears in the definition of an imaginary-sublinear function from $u$, and so assume without loss of generality that $C = 0$. Then $u(t(x+iy)) \leq \gamma y$, so it suffices to prove that
$$\lim_{t \to \infty} \int_K \frac{u(tz)}{t} - \gamma \Im z ~dz = 0$$
    to show convergence in $L^1(K)$.

    By the Riesz representation formula,
$$\frac{u(tz)}{t} - \gamma \Im z = \int_{-\infty}^\infty P(z, x) \sigma(x) ~dx + \int_{\CC_+} G(z, w) \mu(w) ~dw.$$
    We now let
    $$f(t, x) = \frac{1}{t^2} \int_K P\left(t, \frac{x}{t}\right) ~dx.$$
    Since $t > 0$ and $P \geq 0$, it follows that $f \geq 0$. Moreover, by Lemma \ref{estimate on Green function}, there is a $B > 0$ which only depends on $K$ such that for any $x/t$ large enough,
    $$f(t, x) \leq \frac{1}{t^2} \sup_{x \in K} P\left(t, \frac{x}{t}\right) \leq \frac{B}{t^2(1 + |t|)^2}.$$
    Moreover, this estimate on $f$ is trivial if we bound $x/t$ from above; so we can take $B$ to be large enough that this estimate is valid for any $x/t \in \RR$. Similarly, we take
    $$g(t, w) = -\frac{1}{t} \int_K G(z, w) ~dw \leq \frac{D \Im z}{t^2(1+|z/t|)^2}.$$
    Thus $g \geq 0$, and
    $$\int_K \frac{u(tz)}{t} - \gamma \Im z ~dz = \int_{-\infty}^\infty f(t, x) \sigma(x) ~dx - \int_{\CC_+} g(t, w) \mu(w) ~dw.$$
    Since we have estimated $f$ and $g$, we can use (\ref{estimate on sigma}) and (\ref{estimate on mu}) to apply the dominated convergence theorem. Clearly the dominators of $f$ and $g$ converge to $0$ pointwise as $t \to \infty$, so the integrals against $\sigma$ and $\mu$ converge to $0$.
\end{proof}

\begin{proof}[Proof of Titchmarsh's theorem]
Fix a distribution $\mu$ on $\RR$ with $\ch \supp \mu = [a, b]$. By the Paley-Wiener theorem, Theorem \ref{PWS theorem}, $\hat \mu$ is an entire function and $\log |\hat \mu|$ is an imaginary-sublinear subharmonic function. Viewing $\log |\hat \mu|$ as a function on $\CC_+$ and taking $\gamma$ as in Lemma \ref{imaginary sublinear limit}, we have $\gamma = b$, since the estimate in the Paley-Wiener theorem is sharp. On the other hand, viewing $\log |\hat \mu|$ as a function on $\CC_-$, we have $\gamma = a$. Let $Z$ be the multiset of zeroes of $\hat \mu$ with repetition.

Let $h(t) = at$ for $t < 0$, $h(t) = bt$ for $t > 0$. Then $h'$ is a rescaled Heaviside function, so $h'' = (b-a)\delta_0$. By Lemma \ref{gamma in mean}, we have
$$\lim_{t \to \infty} \frac{\log |\hat \mu(t\zeta)|}{t} = h(\Im \zeta)$$
in $L^1_{loc}$. Taking the Laplacian of both sides, we see that
$$\lim_{t \to \infty} \frac{2\pi}{t} \sum_{z \in Z} \delta_{z/t}(w) = (b-a)\delta_0(\Im w)$$
in the sense of distributions.

Integrating both sides,
\begin{align*}
\int_{D(0, 1)} \lim_{t \to \infty} \frac{2\pi}{t} \sum_{z \in Z} \delta_{z/t} &= \int_{D(0, 1)} (b-a)\delta_0(\Im w) ~dw \\
&= \int_{-1}^1 b - a ~dw = 2(b - a).\end{align*}
Rewriting the left-hand side, we have
$$\lim_{R \to \infty} \frac{2\pi }{R} \int_{D(0, R)} \sum_{z \in Z} \delta_z = \lim_{R \to \infty} \frac{2\pi N(R)}{R}.$$
Dividing both sides by $2\pi$, we prove Titchmarsh's theorem.
\end{proof}

\section{The Schatten classes}
\label{noncommutative sequence space section}
We review the theory of operators of trace-class, which are those operators $T$ such that $T$ has a well-defined trace and $1 + T$ has a well-defined determinant. To do this, we will need the theories of compact and Hilbert-Schmidt operators as well.

Fix a Hilbert space $H$. If $T \in B(H)$, then $T^*T$ is a positive operator, so has a unique positive square root $|T| = \sqrt{T^*T}$, which we can reasonably think of as the absolute value of $T$. If $H$ is actually finite-dimensional, then the trace of $T$ is given by $\tr T = \sum_j \langle Te_j, e_j\rangle$ for any and every orthonormal basis $(e_j)_j$. So it is reasonable to define $\tr T$ this way whenever $H$ is separable (hence has a countable orthonormal basis), though the series may not converge in that case. Henceforth we will assume that $H$ is separable.

\begin{definition}
The \dfn{trace-class norm} is defined by $||T||_1 = \tr |T|$, and the \dfn{Hilbert-Schmidt norm} is defined by $||T||_2^2 = \tr(T^*T)$. If $||T||_1$ (resp. $||T||_2$) is finite, we say that $T$ is an \dfn{operator of trace class} (resp. \dfn{Hilbert-Schmidt operator}). The trace class is known as $B^1(H)$ and the space of Hilbert-Schmidt operators is known as $B^2(H)$.
\end{definition}
One can define spaces $B^p(H)$ for any $p \in [1, \infty]$; these are known as \dfn{Schatten classes}. However, we will only need $p=1,2,\infty$.

Letting $||\cdot||_\infty$ denote the usual operator norm, we observe that
\begin{align*}
||T||_\infty & \leq ||T||_1,\\
||TS||_1 &\leq ||T||_2||S||_2;
\end{align*}
the proof is the same as their ``commutative analogues" which interpolate between $\ell^1$, $\ell^2$, and $\ell^\infty$.

We now construct a wealth of Hilbert-Schmidt operators, some of which we will need later.
\begin{lemma}
Let $k \in L^2(\RR^2)$ be the integral kernel of an operator $K \in B(L^2(\RR))$. Then $||K||_2 = ||k||_2$, so $K \in B^2(L^2(\RR))$.
\end{lemma}
\begin{proof}
Fix an orthonormal basis $(e_n)_n$ of $L^2(\RR)$. This determines an orthonormal basis $(e_{nm})_{nm}$ of $L^2(\RR^2)$ by $e_{nm}(x, y) = e_n(x) \overline{e_m(y)}$. So
$$||k||_2 = \sum_{nm} |\langle k, e_{nm}\rangle|^2.$$
We therefore compute
\begin{align*}
  \langle k, e_{nm}\rangle &= \int_{-\infty}^\infty \int_{-\infty}^\infty k(x, y) \overline{e_m(x)} e_n(y) ~dx ~dy\\
  &= \int_{-\infty}^\infty \overline{e_m(x)} \int_{-\infty}^\infty k(x, y) e_n(y) ~dx ~dy\\
  &= \int_{-\infty}^\infty \overline{e_m(x)} Ke_n(y) ~dx ~dy = \langle Ke_n, e_m\rangle.
\end{align*}
Therefore
$$||k||_2 = \sum_n ||Ke_n||^2 = ||K||_2.$$
In particular, $K \in B^2(L^2(\RR))$.
\end{proof}
It can be shown that every Hilbert-Schmidt operator on $L^2(\RR)$ can be written as an integral operator whose kernel lies in $L^2(\RR^2)$; but we will not need this fact.
\begin{lemma}
Let $f, g \in L^2(\RR)$ and suppose that
$$Ku = \mathcal F^{-1}(\hat f \mathcal F(gu)).$$
Then $||K||_2 = ||f||_2||g||_2$, so $K \in B^2(L^2(\RR))$.
\end{lemma}
\begin{proof}
By an approximation argument, we may assume that $f$, $g$, and $u$ are Schwartz. Then
\begin{align*}
Ku(x) &= \frac{1}{2\pi} \int_{-\infty}^\infty e^{i\xi x} \hat f(\xi) \widehat{gu}(\xi) ~d\xi \\
  &= \frac{1}{2\pi} \int_{-\infty}^\infty \int_{-\infty}^\infty e^{i\xi(x - y)} \hat f(\xi) g(y) u(y) ~dy ~d\xi\\
  &= \frac{1}{2\pi} \int_{-\infty}^\infty g(y)u(y) \int_{-\infty}^\infty e^{i\xi(x - y)} \hat f(\xi) ~d\xi ~dy\\
  &= \int_{-\infty}^\infty f(x - y) g(y) u(y) ~d\xi~dy
\end{align*}
so $k(x, y) = f(x - y) g(y)$ is the integral kernel of $K$. Besides,
\begin{align*}
  ||k||_2^2 &= \int_{-\infty}^\infty \int_{-\infty}^\infty |g(y)|^2 |f(x - y)|^2 ~dx ~dy
  \\&= \int_{-\infty}^\infty |g(y)|^2 \int_{-\infty}^\infty |f(x - y)|^2 ~dx ~dy = ||g||_2^2||f||_2^2.
\end{align*}
Therefore $||K||_2 = ||g||_2 ||f||_2$.
\end{proof}
\begin{lemma}
  \label{partial resolvent is hilbert schmidt}
Let $\chi$ denote the indicator function of $[-1, 1]$ and let $\Im \lambda > 0$. Then the operator $T(\lambda) = (-i\partial - \lambda)^{-1}\chi$ is Hilbert-Schmidt, and
$$||T(\lambda)||_2 = \frac{1}{\sqrt{\Im \lambda}}.$$
\end{lemma}
\begin{proof}
Let $\hat f(\xi) = (\xi - \lambda)^{-1}$. Then
$$T(\lambda)u = \mathcal F^{-1}(\hat f \mathcal F(\chi u)).$$
So $||T(\lambda)||_2 = ||f||_2||\chi||_2$, and $||\chi||_2 = \sqrt 2$. Also,
$$||f||_2^2 = \frac{||\hat f||_2^2}{2\pi} = \frac{1}{2\pi} \int_{-\infty}^\infty \frac{d\xi}{|\xi - \lambda|^2}.$$
Up to a translation of $\xi$, we may assume $\Re \lambda = 0$. Then
$$|\xi - \lambda|^2 = |\xi|^2 + |\lambda|^2$$
by the Pythagorean theorem. Since $|\xi|^2 = \xi^2$ we have
$$||f||_2^2 = \frac{1}{2\pi} \int_{-\infty}^\infty \frac{d\xi}{\xi^2 + |\lambda|^2} = \frac{1}{2\Im \lambda}$$
wherefore the claim.
\end{proof}

An operator $T$ of trace class can clearly be approximated by finite-rank operators in $||\cdot||_1$, and the compact operators $B^\infty(H)$ are those that can be approximated by finite-rank operators in $||\cdot||_\infty$. Therefore if an operator is of trace class, then it is compact.
In particular, its spectrum is countable and its only limit point is $0$ (provided that $H$ is infinite-dimensional). So every element of the spectrum is an eigenvalue except possibly $0$. (That is, every element of the spectrum of $1 - T$ is an eigenvalue except possibly $1$.) We recall that the multiplicity of a nonzero eigenvalue $\lambda$ of $1 - T$ is defined to be the dimension of the space of vectors annihilated by $(T - \lambda)^j$ for some $j \geq 0$ large enough; for compact operators, the kernels of $(T - \lambda)^j$ stabilize as $j \to \infty$, so the multiplicity of $\lambda$ is well-defined.

We now use the fact that finite-rank operators are dense in $B^1(H)$ to define an infinite-dimensional determinant. For the proofs in what follows, see Dyatlov and Zworski \cite[Appendix B]{dyatlov2019mathematical}.
Recall that if $T$ is a finite-rank operator, then all but finitely many of the eigenvalues (counting multiplicity) of $A$ are zero, so the determinant
$$\det(1-T) = \prod_{\lambda \in \Spec T} 1 - \lambda$$
is a finite product, and therefore makes sense.
\begin{lemma}
\label{fredholm determinant is well-defined}
Define a map on finite-rank operators by $T \mapsto \det(1-T)$. This map is continuous for the trace-class norm, so extends uniquely to a continuous map $B^1(H) \to \CC$.
\end{lemma}
\begin{definition}
Let $T \in B^1(H)$. The \dfn{Fredholm determinant} of $1 - T$ is defined for finite-rank operators by $\det(1 - T)$, and for general $T$ by Lemma \ref{fredholm determinant is well-defined}.
\end{definition}
\begin{lemma}
Let $T \in B^1(H)$. Then
$$|\det(1 - T)| \leq e^{||T||_1},$$
and $\det(1 - T) = 0$ if and only $1 - T$ is not injective. If we enumerate $\Spec T$ so that $|\lambda_0| \geq |\lambda_1| \geq \cdots$, then
$$\det(1-T) = \prod_{j=0}^\infty 1 - \lambda_j.$$
\end{lemma}

By the spectral theorem for compact operators, a compact operator $T \in B^\infty(H)$ admits a singular value decomposition
$$Tu = \sum_{n=1}^\infty \rho_n \langle u, e_n\rangle f_n$$
for some orthonormal sets $e_n,f_n \in H$ and singular values $\rho_n \in \RR$. Then the assumption that $T \in B^p(H)$ is equivalent to saying that $(\rho_n)_n \in \ell^p$ (and this makes sense for any $p \in [1, \infty]$, not just $p =1,2,\infty$). In particular, given $f, g \in H$, the operator
$$Tu = \langle u, f \rangle g$$
is a rank-$1$ operator (since its image is the span of $g$), and we can compute its SVD by finding a $\rho \in \RR$ such that if
$$Tu = \rho \langle u, \tilde f \rangle \tilde g$$
has $||\tilde f||_2 = ||\tilde g||_2 = 1$. Clearly then
\begin{equation}\label{b1 norm of a tensor product}||T||_1 = |\tr T| = |\rho| = ||f||_2 ||g||_2.\end{equation}
So the following linear map is well-defined.
\begin{definition}
\label{tensor products are trace class}
We define a linear map $b: H \otimes H \to B^1(H)$ by
$$b(f \otimes g)u = \langle u, f \rangle g.$$
\end{definition}



\chapter{Review of scattering theory}
\label{review of scattering theory}
In this chapter we review potential scattering for compactly supported and super-exponentially decreasing potentials on $\RR$. The results in Sections \ref{beginning of scat review}-\ref{end of scat review} will often be used without comment in Chapter \ref{SED potentials chapter}, and many are stated without proof, but can reviewed in \cite{dyatlov2019mathematical} and \cite{tang2007potential}.
In Section \ref{BW section} we fill in the details of the proof of the Breit-Wigner formula outlined by Dyatlov and Zworski in \cite{dyatlov2019mathematical}; Chapter \ref{SED potentials chapter} is independent of this section, which relies on Theorem \ref{Titchmarsh I}.

\section{The Schr\"odinger picture}
\label{beginning of scat review}
In the Schr\"odinger picture of quantum mechanics, one wishes to solve the Schr\"odinger equation
\begin{equation}
\label{Schrodinger equation}
Hu = \partial_tu.
\end{equation}
Here $H$ is a fixed self-adjoint operator known as the Hamiltonian and a solution $u = u(t, x)$ is known as a wavefunction, especially if $u(t) \in L^2(\RR)$.
We will be interested in $H = H_V$, where $H_V = D^2_x + V$, and $V$ is multiplication by a function known as the potential.
Since $H_V$ is invariant under time-translation, we can solve (\ref{Schrodinger equation}) using separation of variables, so that
$$u(t, x) = e^{-itH_V}u_0(x).$$
Here $u_0(x)$ is an initial datum and $e^{-itH_V}$ is the one-parameter family of unitary operators defined by requiring that if $H_Vv_\lambda = \lambda^2v_\lambda$, then
$$e^{-itH_V}v_\lambda = e^{-it\lambda^2}v_\lambda.$$
By the spectral theorem for unbounded self-adjoint operators, or, equivalently, Borel functional calculus, $u_0$ can always be written as a direct integral of eigenfunctions of $H_V$,
so $e^{-itH_V}$ is well-defined and we might as well assume that $u_0 = v_\lambda$ is an eigenfunction, with eigenvalue $\lambda^2 \in \RR$.
The physical interpretation of this situation is that $v_\lambda$ is a wave with energy $\lambda^2$.
The problem of solving (\ref{Schrodinger equation}) then reduces to the problem of diagonalizing $H_V$.

By definition, the resolvent operator $R_V(\lambda) = (H_V - \lambda^2)^{-1}$ is the right inverse of the Schrödinger operator $H_V - \lambda^2$. The definition of $R_V(\lambda)$ means that it is the linear operator such that
$$(H_V - \lambda^2)R_V(\lambda) f = f$$
for all compactly supported $f \in L^2$, assuming that such an operator in fact exists and is unique.
In particular, if $\lambda^2$ is an eigenvalue of $H_V$, then $R_V(\lambda)$ cannot exist.

Viewing the resolvent $R_V$ as a function valued in $B(L^2_{comp}(\RR) \to L^2_{loc}(\RR))$, we want to understand the relationship between the poles of the resolvent and the scattering behavior of $H_V$; that is, the behavior of solutions $u$ to the eigenvalue equation
$$H_Vu = \lambda^2u$$
at infinity, assuming that $V$ is zero at infinity.

\section{Analytic continuation}
The trouble is that $R_V(\lambda)$ may not actually exist if $\Im \lambda$ is not too large and $V$ does not decay fast enough.
In fact, if $V$ is compactly supported, then $R_V$ will admit a meromorphic continuation to all of $\CC$. However, if $V$ is simply super-exponentially decreasing, we may not have this luxury. So we consider certain weighted resolvents to act as a suitable substitute.

We introduce the \dfn{resolvent function}, $(x, y) \mapsto R_V(\lambda; x, y)$, defined by
$$R_V(\lambda)f(x) = \int_{-\infty}^\infty f(y) R_V(\lambda; x, y) ~dy.$$
Since $R_V(\lambda)$ is a right inverse to $H_V - \lambda^2$, we can appropriately view the resolvent function as the Green function of $H_V - \lambda^2$ on $\RR$.
The resolvent $R_0$ is known as the free resolvent.

When we study the free resolvent, we are just solving the Schrödinger equation for a free particle, so
$$R_0(\lambda; x, y) = \frac{i}{2\lambda} e^{i\lambda|x - y|}.$$
To see this, we just have to check
\begin{align*}
2i\lambda (H_0 - \lambda^2)R_0(\lambda; x, 0) &= (\Delta_x + \lambda^2) e^{i\lambda|x|} = -\lambda^2 e^{i\lambda|x|} + \lambda^2 e^{i\lambda|x|} \\
&= -2i\delta_0(x).
\end{align*}
Thus $R_0(\lambda)$ is in fact the Green function for $H_0 - \lambda^2$.

Now if $\Im \lambda \leq 0$, then we have no decay in the plane waves $e^{i\lambda|x - y|}$, so $R_0(\lambda)$ does not carry $L^2$ to itself.
The spectrum of $H_0 = -\Delta$ consists of $\RR_+$, so if $\Im \lambda > 0$, it follows from the spectral radius theorem that
\begin{equation}
\label{bound on R_0}
  ||R_0(\lambda)||_{L^2 \to L^2} = \sup_{\mu \in \RR_+} \frac{1}{|\lambda^2 - \mu|} = \frac{1}{d(\RR_+, \lambda^2)}.
\end{equation}
Therefore $R_0(\lambda)$ is bounded on $L^2$ if $\Im \lambda > 0$. In particular, $R_0$ is holomorphic on $\CC_+$.

We put
$$W_V(\lambda) = \sqrt VR_0(\lambda)\sqrt{|V|}.$$
We view $W_V$ as a weighted version of the resolvent $R_0(\lambda)$.
We will need the fact that $W_V(\lambda)$ is holomorphic in $\lambda$ and trace-class.
\begin{theorem}
  \label{sv is b1 family}
Suppose that $V$ is super-exponentially decreasing. Then $W_V$ is holomorphic as a function $\CC \setminus 0 \to B^1(L^2(\RR))$, and has a simple pole at $0$.
\end{theorem}
Before we can prove Theorem \ref{sv is b1 family}, we need some careful estimates on $W_V(\lambda)$.
Our proof follows \cite[Lemma 3.1]{froese1997asymptotic}.

\begin{lemma}
Let $L > 0$, $x,z \in (-L, L)$. For every $\lambda_1,\lambda_2 \in \CC$ we have
\begin{align*}R_0(x, z; \lambda_1) - R_0(x, z; \lambda_2)
    &= (\lambda_1^2 - \lambda_2^2)\int_{-L}^L R_0(x, y; \lambda_1)R_0(y, z; \lambda_2) ~dz\\
    &\quad+\frac{i}{4}e^{L + i(\lambda_1 + \lambda_2)}\left(\frac{1}{\lambda_1} - \frac{1}{\lambda_2}\right)(e^{-i(\lambda_1x + \lambda_2z)} + e^{i(\lambda_1x + \lambda_2z)}).
\end{align*}
\end{lemma}
\begin{proof}
Adding and subtracting $\Delta_y$,
$$(\lambda_1^2 - \lambda_2^2)R_0(x, y; \lambda_1)R_0(y, z; \lambda_2) = R_0(x, y; \lambda_1)(\Delta_y + \lambda_1^2 - \Delta_y - \lambda_2^2)R_0(y, z; \lambda_2).$$
By the definition of the Green function,
$$\int_{-L}^L R_0(x, y; \lambda_1)(-\Delta_y - \lambda_2^2)R_0(y, z; \lambda_2) ~dy = R_0(x, z; \lambda_1).$$
Therefore
\begin{align*}
  (\lambda_1^2 - \lambda_2^2)\int_{-L}^L R_0(x, y; \lambda_1)R_0(y, z; \lambda_2) ~dy &= R_0(x, z; \lambda_1)\\
  &\quad-\int_{-L}^L R_0(x, y;\lambda_1)(\Delta_y - \lambda_1^2)R_0(y, z; \lambda_2) ~dy.
\end{align*}
Integrating by parts,
\begin{align*}
  -\int_{-L}^L R_0(x, y; \lambda_1)\Delta_y R_0(y, z; \lambda_2) ~dy &= -\int_{-L}^L \partial_yR_0(x, y; \lambda_1)\partial_y R_0(y, z; \lambda_2) ~dy\\
    &\quad+ [R_0(x, y; \lambda_1)\partial_y R_0(y, z; \lambda_2)]_{y=-L}^L.
\end{align*}
We observe that
$$R_0(x, y; \lambda_1)\partial_y R_0(y, z; \lambda_2) = \frac{i}{4\lambda_1}e^{i(\lambda_1|x-y|+\lambda_2|y-z|)}\frac{y-z}{|y-z|}$$
from which it follows that
$$[R_0(x, y; \lambda_1)\partial_y R_0(y, z; \lambda_2)]_{y=-L}^L = \frac{i}{4\lambda_1} e^{i\lambda_1L}(e^{-i(\lambda_1x + \lambda_2z)} + e^{i(\lambda_1x + \lambda_2z)}).$$
Integrating by parts again,
\begin{align*}
  -\int_{-L}^L \partial_yR_0(x, y; \lambda_1)\partial_y R_0(y, z; \lambda_2) ~dy &= \int_{-L}^L \Delta_yR_0(x, y; \lambda_1)R_0(y, z; \lambda_2) ~dy\\
  &\quad- [\partial_yR_0(x, y; \lambda_1) R_0(y, z; \lambda_2)]_{y=-L}^L\\
  &= R_0(x, z; \lambda_2) \\
  &\quad- \frac{i}{4\lambda_2} e^{i\lambda_1L}(e^{-i(\lambda_1x + \lambda_2z)} + e^{i(\lambda_1x + \lambda_2z)}).
\end{align*}
Putting it all together, the lemma follows.
\end{proof}
Let $T_L = S_{\chi_{[-L, L]}}$ and let $T = T_1$. Then, if $f$ decays fast enough, it makes sense to define
$$F(\lambda_1, \lambda_2)f(x) = \frac{i}{4}e^{i(\lambda_1 + \lambda_2)}\left(\frac{1}{\lambda_1} - \frac{1}{\lambda_2}\right)b(\chi e^{-i\lambda_1x} \otimes \chi e^{-i\lambda_2x} + \chi e^{i\lambda_1x} \otimes e^{i\lambda_2x}),$$
where $b$ is the operator $L^2(\RR) \otimes L^2(\RR) \to B^1(L^2(\RR))$ given by Definition \ref{tensor products are trace class}. Using (\ref{b1 norm of a tensor product}) we can estimate, for $\lambda \in \RR$,
\begin{align*}
||F(\lambda, i|\lambda|)||_1 \leq e^{-i|\lambda|}\left|\frac{1}{\lambda} - \frac{1}{i|\lambda|}\right|(||\chi||_2 ||\chi e^{|\lambda| x}||_2 + ||\chi||_2 ||\chi e^{-|\lambda|x}||_2)
\end{align*}
whence
\begin{equation}
\label{estimate on F}
||F(\lambda, i|\lambda|)||_1 = O(|\lambda|^{-1}).
\end{equation}
On the other hand, if $\lambda \notin \RR$, we have
$$||F(\lambda, -\lambda)||_1 \leq Ce^{2|\Im \lambda|}|\Im \lambda|^{-1}||\chi||_2||\chi e^{2x\Im \lambda}||_2$$
whence
\begin{equation}
  \label{imaginary estimate on F}
  ||F(\lambda, -\lambda)||_1 = O\left(\frac{e^{4 |\Im \lambda|}}{|\Im \lambda|}\right).
\end{equation}

By the above lemma, we have
\begin{equation}
\label{integrating T by parts}
  T(\lambda_1) - T(\lambda_2) = (\lambda_1^2 - \lambda_2^2)T(\lambda_1)T(\lambda_2) + F(\lambda_1, \lambda_2).
\end{equation}
\begin{lemma}
\label{T1 is b1}
Let $\lambda \neq 0$. Then $T(\lambda)$ is trace-class. If $\Im \lambda > 0$, then $||T(\lambda)||_1 \leq (\Im \lambda)^{-1}$. If $\lambda \in \RR$, then $||T(\lambda)||_1 = O(1 + |\lambda|^{-1})$. If $\Im \lambda < 0$, then
$$||T(\lambda)||_1 = O\left(\frac{e^{4|\Im \lambda|}}{-\Im \lambda}\right).$$
\end{lemma}
\begin{proof}
If $\Im \lambda > 0$, we have the factorization
\begin{align*}T(\lambda) &= \chi(-\Delta - \lambda^2)^{-1}\chi = \chi(-i\partial + \lambda)^{-1} (-i\partial - \lambda)^{-1} \chi\\ &= -((-i\partial - \lambda)^{-1} \chi)^*((-i\partial - \lambda)^{-1} \chi).\end{align*}
Since taking adjoints clearly preserves the Hilbert-Schmidt norm, we have
$$||T(\lambda)||_1 \leq ||(-i\partial - \lambda)^{-1}\chi||_2^2 = \frac{1}{\Im \lambda}$$
by Lemma \ref{partial resolvent is hilbert schmidt}.

If $\Im \lambda = 0$, we use the fact that
$$||T(\lambda)||_2 \leq ||(-\Delta - \lambda^2)^{-1}||_2 = \frac{1}{2\pi}\int_{-\infty}^\infty \frac{d\xi}{\xi^2 + \lambda^2} = \frac{1}{2|\lambda|},$$
as in the proof of Lemma \ref{partial resolvent is hilbert schmidt}. By (\ref{integrating T by parts}) and (\ref{estimate on F}),
\begin{align*}
||T(\lambda)||_1 &\leq ||T(i|\lambda|)||_1 + (\lambda^2 + |\lambda|^2)||T(\lambda)||_2||T(i|\lambda|)||_1 + ||F(\lambda, i|\lambda|)||_1\\
  &\leq \frac{1}{|\lambda|} + O\left(\frac{\lambda^2 + |\lambda|^2}{|\lambda|^2}\right) + O\left(\frac{1}{|\lambda|}\right)\\
  &= O(1 + |\lambda|^{-1}).
\end{align*}

If $\Im \lambda < 0$, we notice that by (\ref{integrating T by parts}),
\begin{align*}
  T(\lambda) &= T(-\lambda) + (\lambda^2 - (-1)^2\lambda^2)T(\lambda)T(-\lambda) + F(\lambda, -\lambda)\\
  &= T(-\lambda) + F(\lambda_1, \lambda_2).
\end{align*}
Using our estimate for $\Im \lambda > 0$ and (\ref{imaginary estimate on F}),
$$||T(\lambda)||_1 \leq ||T(-\lambda)||_1 + ||F(\lambda, -\lambda)||_1 = O\left(\frac{e^{4|\Im \lambda|}}{-\Im \lambda}\right),$$
as promised.
\end{proof}

\begin{lemma}
\label{estimate on TL}
Suppose that $\Im \lambda < 0$. Let $\chi_L$ denote the indicator function of $[-L, L]$. Then
$$||T_L(\lambda)||_1 = O\left(\frac{Le^{4L|\Im \lambda|}}{|\lambda|}\right).$$
\end{lemma}
\begin{proof}
If $U_L$ denotes the unitary dilation
$$U_Lu(x) = L^{-1/2}u(xL^{-1})$$
then it is easy to see that
$$T_L(\lambda)U_L = L^2U_LT(L\lambda).$$
Since conjugation by a unitary operator preserves trace,
$$||T_L(\lambda)||_1 = L^2 ||T(L\lambda)||_2 \leq CL^2 e^{4L|\Im \lambda|}{L|\Im \lambda|}$$
which proves the lemma.
\end{proof}

\begin{lemma}
\label{convergence in b1 topology}
Let $N > 0$. Suppose that $\sqrt{V(x)} = O(e^{-N|x|})$, $\lambda \neq 0$, and $|\Im \lambda| < N/4$. Then
$$\lim_{L \to \infty} \sqrt V T_L(\lambda) \sqrt{|V|} = W_V(\lambda)$$
in the topology of $B^1(L^2(\RR))$.
\end{lemma}
\begin{proof}
Up to a rescaling we may assume that $|\sqrt{V(x)}| \leq e^{-N|x|}$. Let $w(x) = e^{-LN}$ for $|x| \in (L-1, L]$, so $w$ dominates $\sqrt V$. Therefore
\begin{align*}
||\sqrt V T_L(\lambda) \sqrt{|V|} - W_V(\lambda)||_1 &= ||\sqrt V(\chi_L R_0(\lambda)\chi_L)\sqrt{|V|}||_1\\
  & \leq ||w(\chi_L R_0(\lambda) \chi_L - R_0(\lambda))w||_1 \\
  &= ||wT_L(\lambda)w - S_{w^2}(\lambda)||_1
\end{align*}
so it suffices to prove the lemma when $\sqrt V = w$. Moreover,
\begin{align*}
||w(\chi_L R_0(\lambda) \chi_L - R_0(\lambda))w||_1 &= ||w(\chi_L R_0(\lambda) \chi_L - R_0(\lambda)\chi_L \\
  &\quad+ R_0(\lambda)\chi_L - R_0(\lambda))w||_1\\
  &\leq ||w((1 - \chi_L)R_0(\lambda)\chi_L)w||_1 \\
  &\quad+ ||w(R_0(\lambda)(1 - \chi_L))w||_1\\
  &\leq ||w((1 - \chi_L)R_0(\lambda))w||_1 \\
  &\quad + ||w(R_0(\lambda)(1 - \chi_L))w||_1\\
  &\leq 2||wR_0(\lambda)(1 - \chi_L)w||_1
\end{align*}
since $||\chi_Lw||_2 \leq ||w||_2$ and $||wR_0(\lambda)(1 - \chi_L)w||_1 = ||w(1 - \chi_L)R_0(\lambda)w||_1$ by adjointness.

Let $a_1 = e^{-N}$, $a_L = e^{-NL} - e^{-N(L-1)}$. Then $w = \sum_L a_L \chi_L$, and in particular $\sum_L a_L = 1$, so
$$wR_0(\lambda)(1 - \chi_L)w = \sum_\ell \sum_{m=L}^\infty a_\ell\chi_\ell R_0(\lambda) a_m \chi_m$$
whence
\begin{align*}
  ||wR_0(\lambda)(1 - \chi_L)w||_1 &\leq \sum_\ell \sum_{m=L}^\infty a_\ell a_m ||\chi_\ell R_0(\lambda) \chi_m||\\
  &\leq \sum_{m=L}^\infty a_m \bigg(\sum_{\ell=1}^m a_\ell ||\chi_m R_0(\lambda) \chi_m||_1\\
    &\quad+ \sum_{\ell=m+1}^\infty a_\ell ||\chi_\ell R_0(\lambda) \chi_\ell||_1 \bigg)\\
  &= \sum_{m=L}^\infty a_m \left(\sum_{\ell=1}^m a_\ell ||T_m(\lambda)||_1 + \sum_{\ell=m+1}^\infty a_\ell ||T_\ell(\lambda)||_1 \right).
\end{align*}
We now apply Lemma \ref{estimate on TL} and the fact that $\sum_\ell a_\ell = 1$ (and the sum converges monotonically) to see that
\begin{align*}
  \sum_{m=L}^\infty\sum_{\ell=1}^m a_m a_\ell ||T_m(\lambda)||_1 &\leq \sum_{m=L}^\infty a_m ||T_m(\lambda)||_1\\
  &= O\left(\sum_{m=L}^\infty me^{-m(N-4|\Im \lambda|)}\right).
\end{align*}
Similarly,
$$\sum_{m=L}^\infty \sum_{\ell=m+1}^\infty a_\ell ||T_\ell(\lambda)||_1 = O\left(\sum_{\ell=L}^\infty \ell e^{-\ell(N-4|\Im \lambda|)}\right).$$
Let $\delta = N - 4|\Im \lambda| > 0$. Then
\begin{align*}
  ||\sqrt V T_L(\lambda) \sqrt{|V|} - W_V(\lambda)||_1 &= O(||wR_0(\lambda)(1 - \chi_L)w||_1) \\
  &= O\left(\sum_{\ell=L}^\infty \ell e^{-\delta \ell} \right) = o\left(\frac{1}{\delta L}\right)
\end{align*}
and the theorem holds taking $L \to \infty$.
\end{proof}

\begin{proof}[Proof of Theorem \ref{sv is b1 family}]
Holomorphy follows from that of $R_0$. Since $V$ is super-exponentially decreasing, so is $\sqrt V$. For every $\lambda$, we can find a $N$ large enough that Lemma \ref{convergence in b1 topology} applies.
\end{proof}

We now observe that if $V$ is compactly supported, then $R_V$ extends to a meromorphic family of operators.
To accomplish this, we first extend $R_0$ to a family of bounded operators on $L^2_{comp}$.
We recall that for an operator $T$ on $L^2_{comp}$ is bounded if, for every $L \in \NN$ and every test function $\rho \in C^\infty_{comp}(\RR)$ such that $\supp \rho \subseteq [-L, L]$, the operator $\rho T\rho$ is bounded on $L^2$.
\begin{lemma}
$R_0$ extends to a meromorphic family of operators on $L^2_{comp}$.
\end{lemma}
\begin{proof}
We must show that given $L, \rho$ as above and $\lambda \in \CC_-$, $\rho R_0(\lambda) \rho \in B(L^2)$. In fact,
\begin{align*}
  \int_{-\infty}^\infty |\rho(x)\rho(y)R_0(\lambda; x, y)| ~dx &\leq \frac{C}{|\lambda|} \int_{-\infty}^\infty |\rho(x)\rho(y)| e^{-\Im \lambda|x-y|}\\
  &= \frac{C_\rho}{|\lambda|} e^{2L \Im \lambda}
\end{align*}
so, by Schur's criterion,
$$||\rho R_0(\lambda) \rho||_{L^2 \to L^2} \leq C_\rho \frac{e^{2L\Im \lambda}}{|\lambda|}.$$
\end{proof}

\begin{lemma}
Let $\rho \in C^\infty_{comp}(\RR)$ be such that $\rho V = V$. Then
$$(1 + VR_0(\lambda))^{-1}\rho = \rho(1 + VR_0(\lambda)\rho)^{-1}.$$
\end{lemma}
\begin{proof}
  We have
  $$R_V(\lambda) = R_0(\lambda)(1 + VR_0(\lambda))^{-1}.$$
  Suppose $\rho$ is supported on $[-R, R]$. Then
  $$\supp (1 + VR_0(\lambda))\rho \subseteq [-R, R]$$
  so it follows that
  $$\rho(1 + VR_0(\lambda))\rho = (1 + VR_0(\lambda))\rho.$$
  Inverting the operator $1 + VR_0(\lambda)$, the theorem follows.
\end{proof}

\begin{theorem}
\label{meromorphic continuation with compact support}
If $V$ is compactly supported, then the family of operators
$$R_V: L^2_{comp} \to L^2_{loc}$$
is meromorphic on $\CC$.
\end{theorem}
\begin{proof}
By the definition of the free resolvent, we have
$$(H_V - \lambda^2)R_V(\lambda) = 1 + VR_0(\lambda).$$
If $\Im \lambda$ is very large, then $||VR_0(\lambda)||_{L^2 \to L^2}$ is very small, so the formal Neumann series computation
$$(1 + VR_0(\lambda))^{-1} = \sum_{k=0}^\infty (-VR_0(\lambda))^k$$
is valid. Thus $(1 + VR_0(\lambda))^{-1}$ is a holomorphic family of operators on $L^2$, for $\Im \lambda$ sufficiently large.

With $\rho$ a cutoff such that $\rho V = V$, $VR_0(\lambda)\rho = V\rho R_0(\lambda)\rho$ is a meromorphic family of compact operators $L^2 \to H^2_{comp}$, so $1 + VR_0(\lambda)\rho$ is a meromorphic family of Fredholm operators. It follows that
$$R_V(\lambda)\rho = R_0(\lambda)\rho(1 + VR_0(\lambda)\rho)^{-1}$$ is a meromorphic family of operators $L^2 \to L^2_{loc}$. Taking $R \to \infty$, the theorem follows.
\end{proof}

\section{The scattering matrix}
In what follows, we will assume that $V$ is compactly supported, so that we may define a matrix $S(\lambda)$ known as the ``scattering matrix," which encodes how waves are distorted when they pass through the ``barrier" $\ch \supp V$.

Since we have assumed that $V$ and $f$ have compact support, if $|x|$ is large, then
$$(H_0 - \lambda^2)R_V(\lambda)f(x) = 0.$$
So if we let $\psi = R_V(\lambda)f$, it follows from some calculus that we can find constants $A_+, B_-$ such that for all $x$ large enough,
$$\psi(x) = A_+e^{i\lambda x} + B_-e^{-i\lambda x}.$$
Similarly, we can find $A_-, B_+$ such that for every $x$ with $-x$ large enough,
$$\psi(x) = A_-e^{-i\lambda x} + B_+-e^{i\lambda x}.$$
In case $\lambda > 0$, we can view waves of the form $C e^{i\lambda x}$ as ``moving to the right with frequency $\lambda$" and waves of the form $C e^{-i\lambda x}$ as ``moving to the left with frequency $\lambda$." Of course, this distinction makes sense for any fixed $\lambda \in \CC$, even if $\lambda$ is not a positive real number. So we make the following definition.
\begin{definition}
An \dfn{incoming wave} is a linear combination of waves of the form $e^{i\lambda x}$ for $x \ll 0$ and $e^{-i\lambda x}$ for $x \gg 0$. An \dfn{outgoing wave} is a linear combination of waves of the form $e^{-i\lambda x}$ for $x \ll 0$ and $e^{i\lambda x}$ for $x \gg 0$.
\end{definition}

We can use the formalism of incoming and outgoing waves to show that $R_V$ has no nonzero real poles.
\begin{theorem}
\label{real poles are zero}
Let $\lambda \in \RR$ and suppose $\lambda \neq 0$. Then $\lambda$ is not a pole of $R_V$.
\end{theorem}
\begin{lemma}
\label{laurent expansion of the resolvent}
Suppose that $R_V$ has a pole at $\lambda_0 \neq 0$. Suppose that
$$R_V(\lambda) = \frac{P_N}{(\lambda - \lambda_0)^N} + \dots + \frac{P_1}{\lambda - \lambda_0} + Q(\lambda)$$
is the Laurent expansion of $R_V$ at $\lambda_0$ (so $Q$ is holomorphic). Then for every $f \in L^2_{comp}$, $u = P_Nf$ is an outgoing solution of the eigenvalue equation $H_Vu = \lambda_0^2u$.
\end{lemma}
\begin{proof}
We have
\begin{align*}
  (\lambda - \lambda_0)^N(H_V - \lambda^2)R_V(\lambda)f &= (\lambda - \lambda_0)^N (H_V - \lambda^2)\\
  &\quad\left(\frac{u}{(\lambda - \lambda_0)^N} + O((\lambda - \lambda_0)^{1-N}f)\right)\\
  &= (H_V - \lambda^2)u + O((\lambda - \lambda_0)f)
\end{align*}
and taking $\lambda \to \lambda_0$ we see that $(H_V - \lambda_0^2)u = 0$. Therefore $u$ solves the eigenvalue equation. To see that $u$ is outgoing, choose $\rho \in C^\infty_{comp}(\RR)$ such that $\rho V = V$. Then
$$R_0(\lambda)\rho(1 + VR_0(\lambda)\rho)^{-1} = R_V(\lambda)\rho$$
so $(1 + VR_0(\lambda)\rho)^{-1}$ is meromorphic with a pole of order $\leq N$ at $\lambda_0$. Write
$$(1 + VR_0(\lambda)\rho)^{-1} = \frac{\tilde P_N}{(\lambda - \lambda_0)^N} + \dots + \frac{\tilde P_1}{\lambda - \lambda_0} + \tilde Q(\lambda)$$
for the Laurent expansion of $(1 + VR_0(\lambda)\rho)^{-1}$ at $\lambda_0$. Then
$$P_N(\rho(L^2(\RR))) = R_0(\lambda) \rho \tilde P_N(L^2(\RR)) \subseteq R_0(\lambda)(L^2_{comp}(\RR)).$$
Since $\rho$ was arbitrary, it follows that $u \in R_0(\lambda)(L^2_{comp}(\RR))$. Since $R_0$ is the outgoing free resolvent, $u$ is outgoing.
\end{proof}

\begin{lemma}
\label{no outgoing solutions, part 1}
Let $\lambda \in \RR$ and suppose $\lambda \neq 0$. Suppose $H_Vu = \lambda^2u$. If we expand $u$ as
$$u(x) = \begin{cases}
  A_+ e^{i\lambda x} + B_-e^{-i\lambda x}, &x \gg 0,\\
  A_- e^{i\lambda x} + B_+e^{-i\lambda x}, &x \ll 0,
\end{cases}$$
then $|A_+|^2 + |B_+|^2 = |A_-|^2 + |B_-|^2$.
\end{lemma}
\begin{proof}
Since $\lambda \in \RR$, $H_V - \lambda^2$ is a real operator, so $(H_V - \lambda^2) \overline u = 0$. Therefore the Wronskian
$$\begin{vmatrix}
u&\overline u\\u' & \overline u'
\end{vmatrix} = -2i\lambda \begin{cases}
|A_+|^2 + |B_-|^2, &x \gg 0,\\
|A_-|^2 + |B_+|^2, &x \ll 0,
\end{cases}$$
is constant so $|A_+|^2 + |B_-|^2 = |A_-|^2 + |B_+|^2$.
\end{proof}

We will need the following lemma to show that certain solutions are unique. It guarantees that a wave's transmission cannot be completely intercepted by a bounded potential, and so must scatter. Note that this stands in contrast to, say, the infinite potential well that is taught in introductory quantum mechanics class.
\begin{lemma}
\label{bounded potentials must scatter}
Let $u \in L^\infty(\RR)$, $W \in L^\infty(\RR)$, and $\supp u \subseteq [0, \infty)$. If $H_Wu = 0$, then $u = 0$.
\end{lemma}
\begin{proof}
Let $h \in (0, 1)$ and $v(x) = e^{-x/h}u(x)$. Since $u$ is supported in the right half-ray, it follows that $v \in L^\infty(\RR)$ and that $v$ is rapidly decaying. Therefore $v \in L^2(\RR)$. For $\xi \in \RR$, $|i - h\xi|^2 \geq 1$. Therefore
$$||v||_{L^2(\RR)}^2 = ||\hat v||_{L^2(\RR)}^2 \leq \int_{-\infty}^\infty |(h\xi - i)^2\hat v(\xi)|^2 ~d\xi.$$
The inverse Fourier transform of $(h\xi - i)^2 = h^2\xi^2 - 2ih\xi - 1$ is
$$h^2D^2_x - 2ihD_x - 1 = h^2 e^{-x/h} D^2 e^{x/h}.$$
So by the Plancherel formula,
\begin{align*}
  ||v||_{L^2(\RR)} &\leq \int_{-\infty}^\infty |e^{-x/h}(hD_x)^2 u(x)|^2 ~dx
  \\&= h^2 \int_{-\infty}^\infty |e^{-x/h}W(x)u(x)|^2 ~dx
  \\&\leq h^2 ||W||_{L^\infty(\RR)} ||v||_{L^2(\RR)}.
\end{align*}
Taking $h \to 0$, we see that $v = 0$, but $e^{-x/h} \neq 0$, so $u = 0$.
\end{proof}

\begin{corollary}
\label{no outgoing solutions, part 2}
Let $\lambda \in \RR$ and suppose $\lambda \neq 0$. If $H_Vu = \lambda^2u$, then $u$ is not outgoing.
\end{corollary}
\begin{proof}
Let $A_\pm,B_\pm$ be as in Lemma \ref{no outgoing solutions, part 1}. If $u$ is outgoing and not compactly supported, then $|A_+|^2 + |B_-|^2 > 0$ while $|A_-|^2 + |B_+|^2 = 0$. Therefore $u$ is compactly supported, but after translating the support of $u$, we may assume that $\supp u \subseteq [0, \infty)$. But then Lemma \ref{bounded potentials must scatter} implies that $u = 0$.
\end{proof}

\begin{proof}[Proof of Theorem \ref{real poles are zero}]
Suppose that $\lambda$ is a pole. By Lemma \ref{laurent expansion of the resolvent}, there is an outgoing function $u$ such that $H_Vu = \lambda^2u$. This contradicts Corollary \ref{no outgoing solutions, part 2}.
\end{proof}

We now are in a position to construct the eigenfunctions for the continuous spectrum of $H_V$; namely, for $x \in \RR$, $\lambda \in \RR$, $\lambda \neq 0$,
$$e_\pm(x, \lambda) = e^{\pm i\lambda x} - R_V(\lambda)(V(x)e^{\pm i\lambda x}).$$
By Theorem \ref{real poles are zero}, $\lambda$ is not a pole of $R_V$, so this definition makes sense.
\begin{lemma}
The function $e_\pm(\cdot, \lambda)$ is the unique eigenfunction of $H_V$ with eigenvalue $\lambda^2$ which is equal to $e^{\pm i\lambda x}$ modulo outgoing terms.
\end{lemma}
\begin{proof}
Clearly $e_\pm$ is an eigenfunction of $H_V$ with eigenvalue $\lambda^2$. If $\rho V = V$, then
$$R_V(\lambda)\rho = R_0(\lambda)\rho(1 + VR_0(\lambda)\rho)^{-1}.$$
So
$$R_V(\lambda)(Ve^{i\pm x}) = R_V(\lambda)(\rho Ve^{i\pm x}) = R_0(\lambda)\rho(1 + VR_0(\lambda)\rho)^{-1} Ve^{i\pm x}$$
which lies in the image of the outgoing resolvent $R_0(\lambda)$, so is outgoing. Therefore $e_\pm$ is equal to $e^{\pm i\lambda x}$ modulo outgoing terms.

Assume that $\tilde e_\pm(x, \lambda) = e^{\pm i\lambda x} + g(x, \lambda)$ is also an eigenfunction with $g(\cdot, \lambda)$ outgoing. Then
$$\tilde e_\pm(x, \lambda) - e_\pm(x, \lambda) = g(x, \lambda) - R_V(\lambda)(V(x)e^{\pm i\lambda x})$$
is also an eigenfunction, which is outgoing. Then, by Corollary \ref{no outgoing solutions, part 2}, $g(x, \lambda) - R_V(\lambda)(V(x)e^{\pm i\lambda x}) = 0$.
\end{proof}

In the trivial case $V = 0$, we have $e_\pm(x, \lambda) = e^{\pm i\lambda x}$. In this case, the Wronskian $W(x, \lambda)$ of $e_\pm(x, \lambda)$ is given by $-2i\lambda$. We interpret this as meaning that the entirety of an incoming wave is transmitted through $\ch \supp V = \emptyset$. But even if $V$ is nonzero, $H_V$ does not have a first-order term, so it follows from Abel's Wronskian formula that the function $W(\cdot, \lambda)$ is a constant.

\begin{definition}
The \dfn{transmission coefficient} $T$ is defined by
$$T(\lambda) = \frac{iW(\lambda)}{2\lambda}.$$
\end{definition}

We now define
$$\phi_\pm(x, \lambda) = \frac{e_\pm(x, \mp \lambda)}{T(\pm \lambda)}.$$

\begin{lemma}
\label{construction of intertwining, part 1}
For every $\lambda \in \RR$, the $\phi_\pm$ are functions on $\RR$ such that $(D_x^2 + V - \lambda^2)\phi_\pm = 0$ and such that $\phi_\pm(x, \lambda) = e^{-i\lambda x}$ for $\pm x$ large enough.
\end{lemma}
\begin{proof}
To show that the $\phi_\pm$ are functions, we must show that they do not have poles.

First we rule out the possibility that $T(\pm \lambda) = 0$. If $\lambda \neq 0$, then the $e_\pm(\lambda)$ are linearly independent,
so $T(\lambda) \neq 0$.

On the other hand, if the $e_\pm(0)$ are linearly dependent, then there is an $m \in \NN$ such that the $\phi_\pm$ have a pole of order of $m$ at $0$. So
$$\tilde \phi_\pm(x) = \lim_{\lambda \to 0} \lambda^m\phi_\pm(x, \lambda)$$
is a holomorphic function on a ball close to $0$, and $\tilde \phi_\pm \in \ker \tilde \phi_\pm$. Since, for $\pm x$ large enough, $\tilde \phi_\pm(x) = 0$, we conclude that $\tilde \phi_\pm = \phi_\pm$, a contradiction. So the $e_\pm(0)$ are linearly independent, and $\phi_\pm$ is defined on all of $\RR$. Clearly we have $(D_x^2 + V - \lambda^2)\phi_\pm = 0$, so we are done.
\end{proof}

We now put
$$w_\pm(x, y) = \frac{1}{2\pi} \int_{-\infty}^\infty \phi_\pm(x, \lambda)e^{i\lambda y} ~d\lambda$$
By Lemma \ref{construction of intertwining, part 1}, $w_\pm$ solves the equation
\begin{align*}
  (D_x^2 + V(x))w_\pm(x, y) &= D_y^2(x, y),\\
  w_\pm(x, y) &= \delta(x - y) &\pm x \gg 0.
\end{align*}
Rewriting the first equation as
$$(D_x^2 - D_y^2)w_\pm(x, y) = V(x)w_\pm(x, y)$$
we see that the $w_\pm$ are the unique solutions to these equations, since the operator $D_x^2 - D_y^2$ conserves an energy functional. See Evans \cite[\S2.4.3]{evans10} for details.

We recall that a distribution $u$ is real if for every real test function $\varphi$, $\int u\varphi$ is real.
\begin{lemma}
\label{construction of intertwining, part 2}
Let $[a, b] = \ch \supp V$. Then there are unique distributions $X,Y$ such that if $x \gg 0$,
$$\partial_yw_-(x, y) = X(y - x) + Y(x + y),$$
such that $\supp X \subseteq [-2(b-a), 0]$ and $\supp Y \subseteq [2a, 2b]$.
Besides, $0 \in \supp X$ and $X,Y$ are real.
\end{lemma}
\begin{proof}
For $x \gg 0$,
$$(D_x^2 - D_y^2)\partial_yw_-(x, y) = 0$$
so $\partial_yw_-(x, y)$ solves the wave equation with initial data $V$ and forcing data $\delta_0$ where $x$ is the time variable and $y$ is the space variable.
The lemma then follows from the d'Alembertian formula and causality properties of the wave equation, c.f. \cite[\S2.4.1a]{evans10}, where $x = a$ is the inital-time slice of $\RR^2$, the future is $x > a$, and the distributions $X,Y$ must have support contained in an interval whose length is at most $2|\ch \supp V|$.
Moreover, since $V$ and $\delta_0$ are real, so must be $\partial_yw_-(x, y)$, and the d'Alembertian formula then implies that $X,Y$ are real.
\end{proof}
We will later show that $-2(b - a) \in \supp X$, so that
$$\ch \supp X = [-2|\ch \supp V|, 0].$$

The scattering matrix maps the incoming coefficients $A_-,B_-$ to the outgoing coefficients $A_+,B_+$. Since $X$ is a compactly supported distribution, the Paley-Wiener theorem, Theorem \ref{PWS theorem},
implies that the Fourier transform $\hat X$ is an entire function, and in particular $1/\hat X$ is a meromorphic function.
\begin{definition}
\label{scattering matrix definition}
The \dfn{scattering matrix} is the operator $S(\lambda): \CC^2 \to \CC^2$ defined by
$$S(\lambda)\begin{bmatrix}A_-\\B_-\end{bmatrix} = \begin{bmatrix}A_+\\B_+\end{bmatrix}.$$
\end{definition}
If $J = \begin{bmatrix}&1\\1\end{bmatrix}$, then
$$S(-\lambda) = JS(\lambda)^{-1}J.$$
This follows immediately from the definition of the coefficients $A_\pm,B_\pm$.
\begin{theorem}
\label{properties of scatmat}
  The scattering matrix $S$ is a meromorphic family of unitary operators on $\CC$ such that
\begin{equation}
\label{formula for S}
S(\lambda) = \frac{1}{\hat X(\lambda)}\begin{bmatrix}i\lambda & \hat Y(\lambda)\\ \hat Y(-\lambda) & i\lambda\end{bmatrix}.
\end{equation}
  If $\lambda \in \CC_+$ is a pole of $S$, then $\lambda^2$ is an eigenvalue of $H_V$.
  Moreover,
  \begin{equation}
  \label{unitarity for Xhat}
  \hat X(\lambda)\hat X(-\lambda) = \lambda^2 + \hat Y(\lambda)\hat Y(-\lambda),
  \end{equation}
  and
  \begin{equation}
  \label{determinant of scatmat}
  \det S(\lambda) = \frac{\hat X(-\lambda)}{\hat X(\lambda)}.
  \end{equation}
\end{theorem}
\begin{proof}
We consider the decompositions
\begin{align*}
\phi_-(x, \lambda) &= \begin{cases}
A(\lambda)e^{i\lambda x} + B(\lambda) e^{-i\lambda x}, &x \gg 0\\
e^{-i\lambda x}, & x \ll 0,
\end{cases}\\
\phi_+(x, \lambda) &= \begin{cases}
e^{-i\lambda x}, & x \gg 0,\\
C(\lambda) e^{i\lambda x} + D(\lambda) e^{-i\lambda x}, &x \ll 0.
\end{cases}\end{align*}
Here if $y = A,B,C,D$ then $\overline{y(\lambda)} = y(-\lambda)$. In addition, $|A(\lambda)|^2 + 1 = |B(\lambda)|^2$ and $|C(\lambda)|^2 + 1 = |D(\lambda)|^2$.
Thus
\begin{equation}
\label{unitarity}
A(\lambda)A(-\lambda) + 1 = B(\lambda)B(-\lambda)
\end{equation}
and similarly for $C,D$.

From the definition of $S$, we have
\begin{equation}
\label{ABCD S-matrix}
S(\lambda) = \frac{1}{B(\lambda)}\begin{bmatrix}
-\frac{A(\lambda)}{C(\lambda)} & A(\lambda)\\
\frac{B(\lambda)D(\lambda)-1}{C(\lambda)} & 1
\end{bmatrix}.
\end{equation}
From Lemma \ref{construction of intertwining, part 2}, if $\lambda \in \RR$, then
$$i\lambda \phi_-(x, \lambda) = \hat X(\lambda) \phi_+(x, \lambda) + \hat Y(\lambda) \phi_+(x, -\lambda).$$
For $x \gg 0$,
$$i\lambda(A(\lambda) e^{i\lambda x} + B(\lambda)e^{-i\lambda x}) = \hat X(\lambda) e^{-i\lambda x} + \hat Y(-\lambda) e^{i\lambda x}$$
and since the $e^{\pm i\lambda x}$ are linearly independent, $i\lambda A(\lambda) = \hat Y(\lambda)$ and $i\lambda B(\lambda) = \hat X(\lambda)$.
Similarly, $-i\lambda C(\lambda) = \hat Y(\lambda)$ and $-i\lambda D(\lambda) = \hat Y(\lambda)$. Plugging these formulae into (\ref{ABCD S-matrix}), we establish the formula (\ref{formula for S}).
Moreover, (\ref{unitarity}) implies (\ref{unitarity for Xhat}). Plugging (\ref{unitarity for Xhat}) into (\ref{formula for S}), we conclude that $S(\lambda)$ is unitary. Since $\hat X, \hat Y$ are entire functions, $S$ is meromorphic.
Since $S$ is unitary, (\ref{determinant of scatmat}) follows from (\ref{formula for S}).

Finally, if $S$ has a pole at $\lambda_0$, then $B(\lambda_0) = 0$, so if $x \gg 0$,
$$\phi_-(x, \lambda_0) = A(\lambda)e^{i\lambda x}.$$
Therefore $\phi_-(\lambda_0)$ has only outgoing terms away from $\ch \supp V$, and if $\Im \lambda_0 > 0$ this implies that
$$H_V\phi_- = \lambda_0^2\phi_-,$$
so $\lambda_0^2$ is an eigenvalue of $H_V$.
\end{proof}



\section{Scattering resonances}
\label{end of scat review}
In what follows, we let $\Gamma_\lambda$ denote a sufficiently small (counterclockwise-oriented) circle centered on $\lambda$.

\begin{definition}
\label{scattering resonance multiplicity}
Let $V$ be a compactly supported potential. A \dfn{scattering resonance} is a pole of the resolvent family $R_V$. If $\lambda_0$ is a scattering resonance, its \dfn{multiplicity} $m_R(\lambda_0)$ is defined to be the rank of the operator
$$\frac{1}{2\pi i} \int_{\Gamma_{\lambda_0}} R_V(\lambda) ~d\lambda.$$
\end{definition}
In fact, if $\lambda_0$ is a scattering resonance, then for any $\lambda$ close to $\lambda_0$, we can express $R_V(\lambda)$ as a Laurent series
$$R_V(\lambda) = Q(\lambda - \lambda_0) + \sum_{j=1}^N \frac{P_j}{(\lambda - \lambda_0)^j}$$
for some holomorphic family of operators $Q$ defined on a small neighborhood of $0$ and some operators $P_1, \dots, P_N$. The operators $P_1, \dots, P_N$ are of finite rank by construction of $R_V(\lambda)$. Applying the Cauchy-Goursat theorem,
$$\int_{\Gamma_{\lambda_0}} Q(\lambda - \lambda_0) ~d\lambda = 0$$
so the multiplicity is entirely determined by the principal part $R_V(\lambda) - Q(\lambda - \lambda_0)$.

The trouble is that the above definition does not make sense if $R_V$ does not admit a meromorphic continuation to $\CC$.
To justify a more general definition, note that if $V$ is compactly supported, the following are equivalent for each $\lambda \in \CC \setminus 0$:
\begin{enumerate}
\item $\lambda$ is a resonance.
\item $\lambda$ is a pole of $R_V$.
\item $\lambda$ is a pole of $\sqrt VR_V\sqrt{|V|}$.
\end{enumerate}
Moreover, we have
$$R_V(\lambda) = R_0(\lambda)(1 + VR_V(\lambda)),$$
and it follows that
$$\sqrt V R_V(\lambda) \sqrt{|V|} = W_V(\lambda)(1 - \sqrt VR_V(\lambda)\sqrt{|V|}),$$
which can be rewritten as
$$W_V(\lambda) = (1 + W_V(\lambda))\sqrt V R_V(\lambda)\sqrt{|V|}.$$
By Theorem \ref{sv is b1 family}, $W_V(\lambda)$ exists, so we have
$$(1 + W_V(\lambda))^{-1}W_V(\lambda) = \sqrt V R_V(\lambda) \sqrt{|V|} = \infty.$$
This only makes sense if $1 + W_V(\lambda)$ is singular, and since $W_V(\lambda)$ is in the trace class, this means that $\det(1 + W_V(\lambda)) = 0$.

Conversely, suppose that $\det(1 + W_V(\lambda)) = 0$. If $R_V$ is meromorphic on some open neighborhood of $\lambda$, then $\lambda$ is a pole of $R_V$, even if $V$ is not compactly supported. In fact, $\lambda$ is a pole of $(1 + W_V(\lambda))^{-1}$, hence of $\sqrt VR_V(\lambda) \sqrt{|V|}$. Since $W_V$ is holomorphic near $\lambda$, $\lambda$ must be a pole of $R_V(\lambda)$.

We can therefore easily extend the definition of resonance to super-exponentially decreasing potentials.
\begin{definition}
\label{determinant definition of resonance}
Let
$$D(\lambda) = \det(1 + W_V(\lambda)).$$
A \dfn{scattering resonance} of $V$ is a zero of $D$. The \dfn{multiplicity} $m_R(\lambda)$ of a resonance $\lambda$ is the order of vanishing of $D$.
We let $\Res V$ note the multiset of resonances of $V$, counted by multiplicity.
\end{definition}

Recall that the Breit-Wigner series is by definition the sum
$$B(V) = -\sum_{\lambda \in \Res V} \frac{\Im \lambda}{|\lambda|^2}.$$
One might fear that $B(V)$ conditionally converges, because a priori there may be infinitely many resonances in both $\CC_+$ and $\CC_-$. However, this is not the case.
\begin{lemma}
\label{no resonances of positive imaginary part}
The set $\Res V \cap \overline{\CC_+}$ is finite.
\end{lemma}
\begin{proof}
Let $\Im \lambda \geq 0$. We only have to prove that
\begin{equation}
\label{operator norm of SV}
||W_V(\lambda)||_{L^2 \to L^2} \leq \frac{||V||_{L^1}}{2|\lambda|}
\end{equation}
because then if $|\lambda| > 2/||V||_{L^1}$, we have $\Spec W_V(\lambda) \subset D(0, 1)$ and hence $D(\lambda) \neq 0$, so $\lambda$ is not a scattering resonance. Thus every scattering resonance in $\CC_+$ is in the compact disc $\overline{D(0, 2/||V||_{L^1})}$, but $\Res V$ is discrete since it is the set of zeroes of the holomorphic function $D$, so $\Res V \cap \CC_+$ is finite.

We have $e^{i\lambda|x-y|} \leq 1$, so the Cauchy-Schwarz inequality implies
\begin{align*}
||W_V(\lambda)u||_{L^2}^2 &= \int_{-\infty}^\infty\left|\int_{-\infty}^\infty \frac{i\sqrt{V(x)}}{2\lambda} e^{i\lambda|x-y|}u(y)\sqrt{|V(y)|}~dy\right|^2~dx\\
&\leq \frac{1}{4|\lambda|^2} \int_{-\infty}^\infty |V(x)| \left|\langle u, \sqrt V\rangle\right|^2 ~dx\\
&= \frac{||u||_{L^2}^2 ||V||_{L^1}^2}{4|\lambda|^2}
\end{align*}
which verifies (\ref{operator norm of SV}).
\end{proof}


\section{The Breit-Wigner formula}
\label{BW section}
We put $m_S(\lambda_0)$ for the trace of the operator
$$-\frac{1}{2\pi i} \int_{\Gamma_{\lambda_0}} S'(\lambda)S(\lambda)^* ~d\lambda.$$
This satisfies the equation $m_S(\lambda) = m_R(\lambda) - m_R(-\lambda)$. (For a proof, see \cite[Theorem 2.14]{dyatlov2019mathematical}.)

Therefore we are interested in the matrix $S'(\lambda)S(\lambda)^*$, which can reasonably be thought of as the log-derivative of $S$, since we have the identity\footnote{For all unproven matrix calculus identities, see \cite{petersen2008matrix} or similar.}
\begin{equation}
\label{derivative of log det}
(\log \det S)' = \tr(S'S^*).
\end{equation}

The Breit-Wigner approximation gives a formula for the trace of $S'(\lambda)S(\lambda)^*$ in terms of the support of $V$ and a sum over resonances. Since $P$ is an infinite set in general, one cannot hope to compute $\tr S'(\lambda)S(\lambda)^*$ exactly, but the approximation is suitable enough.
\begin{theorem}[Breit-Wigner approximation]
\label{proof of Breit-Wigner}
  \index{Breit-Wigner approximation}
For every $\lambda_0 \in \RR$,
$$\frac{1}{2\pi i} \tr S'(\lambda_0)S(\lambda_0)^* = -\frac{1}{\pi}|\ch\supp V| - \frac{1}{\pi}\sum_{\lambda \in \Res V \setminus 0} \frac{\Im \lambda}{|\lambda - \lambda_0|^2}.$$
\end{theorem}
To begin the proof of Theorem \ref{proof of Breit-Wigner}, we combine (\ref{derivative of log det}) with (\ref{determinant of scatmat}), to see that
\begin{equation}
\label{trace of log derivative}
-\tr S'(\lambda)S(\lambda)^* = \frac{\hat X'(-\lambda)}{\hat X(-\lambda)} + \frac{\hat X'(\lambda)}{\hat X(\lambda)}.
\end{equation}
We therefore expect scattering resonances to be closely related to $\hat X$. To see this in greater detail, let
$$h(x - y)^\pm= \begin{cases}
1 &\pm x > \pm y\\
0 &\text{ else}
\end{cases}$$
be the Heaviside functions.
\begin{lemma}
\label{decomposition of Green function}
Let $P(x, \partial_x) = a\partial_x^2 + b(x)$ be a differential operator, where $a$ is constant and $b \in L^\infty$. Let $E$ be the Green function of $P$, so that
$$P(x, \partial_x)E(x, y) = \delta(x - y).$$
For any two linearly independent solutions $u_1, u_2 \in \ker P$, if $W = u_1u_2' - u_2u_1'$ is the Wronskian of $u_1, u_2$, then
$$E(x, y) = \frac{h(x - y)^+ u_1(x) u_2(y) + h(x - y)^- u_1(y) u_2(x)}{aW}.$$
\end{lemma}
\begin{proof}
$E(\cdot, y)$ must be $C^2$ away from $y$, and $PE$ must be supported on the diagonal $D$, so
$$E(x, y) = \begin{cases}
f_1(y)u_1(x)&x < y\\
f_2(y)u_2(x)&x > y
\end{cases}$$
for some functions $f_1,f_2$. Moreover, $E(\cdot, y)$ must be $C^1$, so that $PE$ can be a distribution of order $1$. On the other hand, for any $\varepsilon$,
\begin{align*}\int_{y-\varepsilon}^{y + \varepsilon} a\partial_x^2E(x, y) + b(x)E(x, y) ~dx &= \int_{y-\varepsilon}^{y+\varepsilon} \delta(x - y) ~dx \\&= 1.\end{align*}
Taking $\varepsilon \to 0$ and using the fact that $bE \in L^\infty$, we see that
$$\lim_{\varepsilon \to 0} \int_{y-\varepsilon}^{y + \varepsilon} a\partial_x^2E(x, y) ~dx = 1$$
and so $\partial_xE$ must jump by $1/a$ along $D$ while $E$ is continuous there. Therefore $f_1(y)u_1(y) = f_2(y)u_2(y)$ and $a(f_1(y)u_1'(y) - f_2(y)u_2'(y)) = 1$.
Solving for $f_1,f_2$ we have $f_1 = u_2/aW$ and $f_2 = u_1/aW$, as desired.
\end{proof}
\begin{lemma}
\label{scattering resonances are Xhat zeroes}
Let $\lambda_0 \in \CC \setminus 0$. Then $m_R(\lambda_0)$ is the order of vanishing of $\lambda_0$ for $\hat X$.
\end{lemma}
\begin{proof}
By Lemma \ref{decomposition of Green function} on $R_V(\lambda)$ with $u_1 = \phi_+(\cdot, -\lambda)$ and $u_2 = \phi_-(\cdot, -\lambda)$, we see that
$$R_V(\lambda; x, y)= \frac{\phi_+(x, -\lambda)\phi_-(y, -\lambda)h(x -y)^+ + \phi_+(y, -\lambda)\phi_-(x, -\lambda)h(x-y)^-}{2\hat X(\lambda)}.$$
On the other hand, since
$$\frac{1}{2\pi} \partial_y \int_{-\infty}^\infty \phi_-(x, \lambda)e^{i\lambda y} ~d\lambda = X(y - x) + Y(y + x),$$
it follows that
$$i\lambda\phi_-(x, \lambda) = \hat X(\lambda)\phi_+(x, \lambda) + \hat Y(\lambda)\phi_+(x, -\lambda).$$
Therefore $m_R(\lambda_0)$ is the rank of the residue $R$ at $\lambda_0$ of the function
$$\lambda \mapsto \frac{\hat Y(\lambda)}{2i\lambda\hat X(\lambda)}\phi_+(\cdot, -\lambda)\otimes \phi_+(\cdot, \lambda)$$
where $\otimes$ is defined by Definition \ref{tensor products are trace class}.

If $k+1$ is the order of vanishing of $\lambda_0$ for $\hat X$, then by the Leibniz rule, there are nonzero scalars $c(j, k \ell)$ such that
\begin{align*}
R &= \frac{\partial_\lambda^k \hat Y(\lambda)(\lambda)\phi_+(\cdot, -\lambda) \otimes \phi_+(\cdot, \lambda)}{2k!i}\bigg|_{\lambda = \lambda_0}\\
&= \sum_{\ell=0}^k \partial^\ell_\lambda \phi_+(\cdot, \lambda) \otimes \sum_{j=0}^{k-\ell-j}c(j, \ell) \partial_\lambda^{k-j} \frac{\hat Y(\lambda)}{\lambda} \partial_\lambda^j \phi_+(\cdot, -\lambda)\bigg|_{\lambda = \lambda_0}.
\end{align*}
The functions
$$\partial^j_\lambda \phi_+(x, -\lambda) = (ix)^je^{i\lambda x}$$
if $x$ is large enough, so are linearly independent in $x$ if $\lambda$ is held constant.
Since $\hat Y(\lambda_0) \neq 0$ by (\ref{unitarity for Xhat}),
$$\partial_\lambda^{k-\ell-j} \frac{\hat Y(\lambda)}{\lambda}\bigg|_{\lambda = \lambda_0} \neq 0,$$
so it can be absorbed into the scalars $c(j, \ell, Y, \lambda)$. Thus
$$R = \sum_{j+\ell \leq k} c(j, \ell, Y, \lambda) \partial^\ell_\lambda \phi_+(\cdot, -\lambda) \otimes \partial^j_\lambda \phi_+(\cdot, -\lambda)$$
which has rank $k + 1$.
\end{proof}

Since we are now interested in the zeroes of $\hat X$, we must also be interested in $\ch \supp X$, by Titchmarsh's theorem.
\begin{lemma}
\label{computing ch supp X}
Let $[a, b] = \ch \supp V$. Then $\ch \supp X = [-2(b - a), 0]$.
\end{lemma}
\begin{proof}
After a change of coordinates we may assume $a = 0$ and $b = |\ch \supp V|$.
Recall Lemma \ref{construction of intertwining, part 2}.
We know that $\ch \supp X \subseteq [-2(b - a), 0]$ and $0 \in \ch \supp X$.
If the claim does not hold, there is a $c < 2b$ such that such that $\ch \supp X = [c, 0]$, and by Titchmarsh's theorem, Theorem \ref{Titchmarsh I}, the density of zeroes of $\hat X$ is $c/\pi$.
It follows by (\ref{unitarity for Xhat}) that the density of zeroes of $\hat G(\lambda) = \hat Y(\lambda)\hat Y(-\lambda) + \lambda^2$ is $2\lambda/\pi$. Now, if $R$ is the operator $Rf(x) = f(-x)$, then $G = (Y * RY) - \delta''$. Moreover,
\begin{align*}\ch \supp G &= \{x + y: x \in \ch \supp Y, ~y \in \ch \supp RY\} \\
&= \{x - y: x, y \in \ch \supp Y\}
\end{align*}
and $|\ch \supp G| = 2c$. So if $\ch \supp Y = [0, 2b]$, we would have $4b > 2c$, contradicting that $c < 2b$. Therefore $\ch \supp Y \subset [0, 2b]$.

Let $d = \sup \supp Y$ and suppose that $d < 2b$. Then
$$-\partial_y w_+(x, y) = X(x - y) + Y(x + y).$$
Let $E^+(x, y)$ be an upward cone from $(x, y)$, and suppose $d/2 < x < y$. Then $\partial_y w_+$ vanishes on $E(x, y)^+ \cap \{(x', y'): y' > 2b\}$, so by causality properties of the wave equation, $\partial_y w_+ = 0$ if $d/2 < x < y$.
Therefore $w_+$ only depends on $x$ in that region, so $w_+(x, y) = 0$ if $d/2 < x < y$.
Therefore $\delta(x - y)V(x) = 0$ there, so if $y > d/2$ then $V(y) = 0$.
But this implies that $b \notin \ch \supp V$, which is impossible.

Therefore $2b \in \ch \supp Y$. A similar argument with $w_-$ implies that $0 \in \ch \supp Y$, so $\ch \supp Y \subset \ch \supp Y$, which is absurd.
\end{proof}
We now recall the definition of a Cauchy principal value of an infinite product. If $(z_n)_n$ is a sequence of complex numbers such that $z_n \to 1$, then
$$\text{p.v.} \prod_n z_n = \lim_{\varepsilon \to 0} \prod_{|z_n - 1| > \varepsilon} z_n,$$
if the limit on the right-hand side does in fact exist. The factors on the right-hand are finite products since $(z_n)$ does not have any accumulation points on $\CC \setminus D(1, \varepsilon)$. If the absolutely convergent infinite product $\prod_n z_n$ exists, then it is equal to its Cauchy principal value.
The Cauchy principal value of an infinite sum is defined similarly.
\begin{proof}[Proof of Theorem \ref{proof of Breit-Wigner}]
By Theorem \ref{Titchmarsh II} and Lemma \ref{scattering resonances are Xhat zeroes},
$$\frac{\hat X(\lambda_0)}{\hat X(0)} = e^{i|\ch\supp V|\lambda_0} \text{p.v.}\prod_{\substack{\hat X(\lambda) = 0\\\lambda \neq 0}} 1 - \frac{\lambda_0}{\lambda}.$$

Taking logarithms and using Lemma \ref{computing ch supp X}\footnote{It is this step which required Theorem \ref{Titchmarsh I}, and its lengthy proof.} and the argument principle,
$$i\log\left(\frac{\hat X(\lambda_0)}{\hat X(0)}\right) = -|\ch \supp V|\lambda_0 + i \text{p.v.}\sum_{\lambda \in \Res V \setminus 0} \log\left(1 - \frac{\lambda_0}{\lambda}\right).$$
Therefore, after differentiating,
$$\frac{1}{\pi i}\frac{\hat X'(\lambda_0)}{\hat X(\lambda_0)} = -\frac{1}{\pi}|\ch \supp V| + \frac{1}{\pi}\text{p.v.} \sum_{\lambda \in \Res V \setminus 0} \frac{i}{\lambda_0 - \lambda}.$$

Recall that $X$ was a real distribution, so if $\lambda \in \RR$, then
$$\hat X(\lambda) = \int_{-\infty}^\infty X(s) \overline{e^{is\lambda}} ~ds = \overline{\hat X(-\lambda)}.$$
Therefore
\begin{equation}
\label{analytic continuation of symmetry}
\hat X(\lambda) = \overline{\hat X(-\overline \lambda)},
\end{equation}
and $\dbar \overline{\hat X(-\overline \lambda)} = 0$, so (\ref{analytic continuation of symmetry}) is valid for any $\lambda \in \CC$, by analytic continuation.

From (\ref{trace of log derivative}) and (\ref{analytic continuation of symmetry}), it follows that, for $\lambda_0 \in \RR$,
\begin{equation}
\label{first two terms good}
\frac{1}{2\pi i}\tr(S'(\lambda_0)S^*(\lambda_0)) = -\frac{1}{\pi}|\ch \supp V| + \frac{1}{\pi} \text{p.v.} \sum_{\lambda \in \Res V \setminus 0} \frac{i}{\lambda_0 - \lambda}.
\end{equation}
Each of the individual summands on the right-hand side are finite, since Lemma \ref{real poles are zero} implies that $\lambda_0 \notin \Res V \setminus 0$.

Lemma \ref{computing ch supp X} and (\ref{analytic continuation of symmetry}) together imply that $\Res V$ is closed under the $\ZZ/2$-action $\Phi(\lambda) = -\overline \lambda$.
We use this fact to replace the Cauchy principal value in (\ref{first two terms good}) with an absolutely convergent series.
Since $\Phi$ preserves $|\cdot|$, we may group up terms in the same orbit of $\Phi$, to conclude that
\begin{align*}
\frac{1}{2\pi i}\tr(S'(\lambda_0)S^*(\lambda_0)) = &-\frac{1}{\pi}|\ch \supp V| \\
&\quad+ \frac{1}{\pi} \sum_{\lambda \in \Res V \cap i\RR \setminus 0} \frac{i}{\lambda_0 - \lambda} \\
&\quad+ \frac{1}{\pi} \sum_{\lambda \in \Res V \cap H_+} \frac{i}{\lambda_0 - \lambda} + \frac{i}{\lambda_0 + \overline{\lambda}},
\end{align*}
where $H_+ = \{x + iy \in \CC: x > 0\}$ is the right half-plane, provided that we can justify moving the series
\begin{equation}
\label{sum over imaginary axis}
\frac{1}{\pi} \sum_{\lambda \in \Res V \cap i\RR \setminus 0} \frac{i}{\lambda_0 - \lambda} =
-\frac{1}{\pi} \sum_{\lambda \in \Res V \cap i\RR \setminus 0} \frac{\Im \lambda - i\lambda_0}{|\lambda - \lambda_0|^2}
\end{equation}
out of the sum over $H_+$. The real part of (\ref{sum over imaginary axis}) is a sum over terms of the form $\Im \lambda/|\lambda - \lambda_0|^2$.
Lemma \ref{no resonances of positive imaginary part} then implies that all but finitely many summands in the real part of (\ref{sum over imaginary axis}) are negative.
Similarly, the imaginary part of (\ref{sum over imaginary axis}) consists of terms of sign $\sgn \lambda_0$, and hence converges absolutely,
so if (\ref{sum over imaginary axis}) converges at all, it converges absolutely, and we can pull it out of (\ref{first two terms good}).
Since we have already proven conditional convergence, decomposition of (\ref{first two terms good}) into $\Phi$-orbits is valid.

As for the sum over $H_+$, write $\lambda = \alpha - i\beta$ to see
\begin{align*}
\frac{1}{\pi} \sum_{\lambda \in \Res V \cap H_+} \frac{i}{\lambda_0 - \lambda} + \frac{i}{\lambda_0 + \overline{\lambda}}
&= \frac{1}{\pi} \sum_{\lambda \in \Res V \cap H_+} \frac{i}{\lambda_0 - \alpha - i\beta} + \frac{i}{\lambda_0 + \alpha - i\beta}\\
&= \frac{2}{\pi} \sum_{\lambda \in \Res V \cap H_+} i\frac{\lambda_0 - i\beta}{|\lambda_0 - \lambda|^2}\\
&= \frac{2}{\pi} \sum_{\lambda \in \Res V \cap H_+} \frac{i\lambda_0 + \beta}{|\lambda_0 - \lambda|^2}
\end{align*}
and hence conclude
\begin{equation}
\label{sum over right half plane}
\frac{1}{\pi} \sum_{\lambda \in \Res V \cap H_+} \frac{i}{\lambda_0 - \lambda} + \frac{i}{\lambda_0 + \overline{\lambda}}
= -\frac{1}{\pi} \sum_{\lambda \in \Res V \setminus i\RR} \frac{\Im \lambda - i\lambda_0}{|\lambda - \lambda_0|^2}
\end{equation}
where we used the symmetry $\Phi$ to replace the factor $2$ with a sum over the left half-plane $H_-$.
Lemma \ref{no resonances of positive imaginary part} again implies that (\ref{sum over right half plane}) converges absolutely.

Combining (\ref{first two terms good}), (\ref{sum over imaginary axis}), and (\ref{sum over right half plane}), we have
\begin{equation}
\label{eliminate imaginary part}
\frac{1}{2\pi i}\tr(S'(\lambda_0)S^*(\lambda_0)) = -\frac{1}{\pi}|\ch \supp V| -\frac{1}{\pi} \sum_{\lambda \in \Res V \setminus 0} \frac{\Im \lambda - i\lambda_0}{|\lambda - \lambda_0|^2}.
   \end{equation}
But recall (\ref{derivative of log det}); $S$ is unitary, so $\det S$ carries $\CC$ into the unit circle, and hence $\log \det S$ carries $\CC$ into $i\RR$.
So $\tr S'(\lambda_0)S^*(\lambda_0)$ is imaginary, so taking the real part of both sides of (\ref{eliminate imaginary part}), we complete the proof.
\end{proof}


\chapter{The Breit-Wigner series}
\label{SED potentials chapter}
In this chapter we prove Theorem \ref{divergence of breit wigner, preliminary version}, which gives conditions for the Breit-Wigner series $B(V)$ of a super-exponentially decreasing potential to diverge.

\section{Froese's conjecture}
Froese \cite{froese1997asymptotic} conjectured an asymptotic formula for the distribution of resonances of $V$, provided of course that $V$ is super-exponentially decreasing. The rest of the work that we will do will be largely under the assumption of this conjecture.
\begin{definition}
The \dfn{associated entire function} $F$ to the potential $V$ is the function
$$F(z) = \hat V(2z)\hat V(-2z) + 1.$$
\end{definition}
When we refer to the indicator function $h$, the counting function $N$, and the auxiliary function $s$ that relates $h$ and $N$, we will always do so in the context of the associated entire function $F$. So
$$h(\theta) = \limsup_{r \to \infty} \frac{\log|F(re^{i\theta})|}{r^\rho},$$
$N$ counts the zeroes of $F$ in a sector, and
\begin{align*}
s(\theta, \varphi) &= 2\pi\rho \lim_{r \to \infty} \frac{N(r, \theta, \varphi)}{r^\rho}\\
&= h'(\theta) - h'(\varphi) + \rho^2\int_\theta^\varphi h(\xi) ~d\xi
\end{align*}
where $\rho$ is the order of $F$.

In addition to the zero-counting function $N$, we will need a resonance-counting function, so let $n(R, \theta, \varphi)$ be the number of resonances $re^{i\xi}$ of $V$ such that $r < R$ and $\xi \in [\theta, \varphi]$, assuming that $\pi \leq \theta \leq \varphi \leq 2\pi$.
This assumption is essential, because the Breit-Wigner series summed over $\Res V \cap \CC_+$ is always finite, by Lemma \ref{no resonances of positive imaginary part}.
\begin{conjecture}[Froese's conjecture]
Let $F$ be the associated entire function to $V$. Suppose that $F$ is of order $\rho$ and has completely regular growth. If $\pi \leq \theta \leq \varphi \leq 2\pi$, then the asymptotic distribution of resonances of $V$ and zeroes of $F$ is identical. So, in particular,
$$|N(R, \theta, \varphi) - n(R, \theta, \varphi)| = o(R^\rho).$$
\end{conjecture}

Froese was able to prove his conjecture for potentials which meet the following hypothesis. (See \cite[Theorem 1.3]{froese1997asymptotic} for a sharper statement and the proof.)
\begin{theorem}[Froese]
\label{Froese conjecture theorem}
Suppose that $\hat V$ has completely regular growth, and either:
\begin{enumerate}
\item $V$ is compactly supported;
\item or, $\rho > 1$, $V \geq 0$, and there is a $C > 0$ such that for every $re^{i\theta} \in \CC$ such that
$$2|\theta| \leq \pi - \frac{\pi}{\rho},$$
we have
$$|\hat V(\lambda)| + |\hat V'(\lambda)| + |\hat V^{(2)}(\lambda)| \leq e^{C|\Im \lambda|}.$$
\end{enumerate}
Then $V$ satisfies Froese's conjecture.
\end{theorem}
Therefore, when we assume that $V$ satisfies Froese's conjecture in the sequel, it will suffice that $V$ meets the hypotheses of Theorem \ref{Froese conjecture theorem}.

\section{Proof of Theorem \ref{divergence of breit wigner, preliminary version}}
There were three cases in the statement of Theorem \ref{divergence of breit wigner, preliminary version}.
The first two are treated by Theorem \ref{divergence of breit wigner}; the last is treated by Theorem \ref{third case of divergence}.

\begin{theorem}
\label{divergence of breit wigner}
Suppose that $V$ satisfies Froese's conjecture, $F$ is the associated entire function to $V$, and $F$ has completely regular growth.
If there are $0 \leq \theta < \varphi \leq 2\pi$ such that $\pi \notin (\theta, \varphi)$ and either $s(\theta, \varphi) \neq 0$ or $s(\theta, \varphi)$ does not exist, then $B(V)$ diverges.
\end{theorem}

We first observe that $F$ has a great deal of symmetry;
$$\overline{\hat V(2z)} = \int_{-\infty}^\infty \overline{V(x)e^{-2ixz}} ~dx = \int_{-\infty}^\infty V(x) e^{2ixz} ~dx = \hat V(-2z)$$
so $z$ satisfies $\hat V(2z) \hat V(-2z) = -1$ iff $\overline z$ and $-z$ do; this implies $F$ is preserved by reflection about a real or imaginary axis, or about the origin.

Now let $\rho$ be the order of $\hat V$. By Lemma \ref{fourier transforms are at least exponential type}, either $\rho \geq 1$ or $\rho = 0$ and $V$ is supported at $0$.
In the latter case, this implies that $V = \sum_{n=1}^N c_n\delta_0^{(n)}$ for some $c_n \in \RR$, which contradicts that $V \in L^\infty$.
So $\rho \geq 1$.

\begin{lemma}
\label{order of F}
The order of $F$ is $\rho$.
\end{lemma}
\begin{proof}
We have
\begin{align*}\log \log \sup_{|z| = r} |F(z)| &= \log \log \sup_{|z| = r} |\hat V(2z)\hat V(-2z) + 1| \\
  &\sim \sup_{|z| = r} \log(\log |\hat V(2z)| + \log |\hat V(-2z)|)\\
  &\sim \sup_{|z| = r} \log \max_{\sigma \in \{1,-1\}} \log |\hat V(2\sigma z)|\\
  &= \sup_{|z| = r} \log \log |\hat V(2z)|\\
  &= \log \log \sup_{|z| = r} |\hat V(z)|
\end{align*}
where we used the fact that $\rho \geq 1$ to remove the dependency on the summand $1$ and took $r$ large enough that $\log(A + B) \sim \log(\max(A, B))$. Precomposing with a constant factor does not affect order, so we are done.
\end{proof}

As a consequence, we obtain a linear lower bound on the growth of the resonance-counting function $n$.
\begin{lemma}
\label{linear lower bound on resonances}
There is an angle $\theta \in [\pi, 2\pi]$ such that $s(\pi, \theta)$ exists and is nonzero. In particular, $n(\cdot, 0, 2\pi)$ grows at least linearly.
\end{lemma}
\begin{proof}
We first show that $s(0, \cdot)$ is not identically zero. If it is, then so is $s'(0, \cdot)$, so $0 = -h^{(2)} + \rho^2 h$,
and solving this ODE we see that
$$h(\xi) = c_+e^{\rho\xi} + c_-e^{-\rho\xi}.$$
If $h$ is identically zero, then $F$ is of zero type, yet $F$ is of normal type since $F$ is of completely regular growth, so this is a contradiction. Therefore $h$ is not identically zero.

Since $F$ is of completely regular growth, $h(0) = h(2\pi)$. Moreover, $F(z) = F(-z)$, so $h(0) = h(\pi)$. Moreover, $\rho \neq 0$ and $h$ is analytic, so $h'$ must have two real zeroes since $h$ maps three to one. However, the only real zero $\alpha$ of $h'$ is
$$\alpha = \frac{\log c_- - \log c_+}{2\rho},$$
unless one of $c_\pm = 0$ (but then $h$ is injective, which is impossible).
This is a contradiction. So there is a $\theta \in [0, 2\pi]$ such that $s(0, \theta) \neq 0$ or $s(0, \theta)$ does not exist. Using the fact that $F(z) = F(-z)$, we may replace $\theta$ with a $\theta \in [\pi, 2\pi]$ such that $s(\pi, \theta) \neq 0$ or $s(\pi, \theta)$ does not exist, if necessary.

Let $Z$ be the countable set of points $\xi$ such that $s(0, \xi)$ does not exist. If $\theta \in Z$, then since $s(0, \cdot)$ is monotone, it must have a jump discontinuity at $\theta$, and since $Z$ is countable, for every $\varepsilon > 0$ we can find a $\delta < \varepsilon$ such that $s(0, \theta + \delta)$ exists and is nonzero. We may then replace $\theta$ with $\theta + \delta$ if $\varepsilon$ was chosen small enough.

This proves the first claim. The second follows immediately from the first claim, Froese's conjecture, and the fact that $\rho \geq 1$.
\end{proof}

In what follows, let
$$B(V, \theta, \varphi) = -\sum_{\substack{\lambda \in \Res V\\\arg \lambda \in [\theta, \varphi]}} \frac{\Im \lambda}{|\lambda|^2}.$$
So $B(V) = B(V, 0, 2\pi)$.

\begin{lemma}
\label{divergence of angular series}
Suppose that $\pi < \theta \leq \varphi < 2\pi$. If $N(r, \theta, \varphi) \gtrsim r$, then $B(V, \theta, \varphi)$ diverges.
\end{lemma}
\begin{proof}
For convenience let
$$k_j = n(j, \theta, \varphi).$$
Without loss of generality assume that $e^{i\theta}$ is closer to the real axis than $e^{i\varphi}$.
Let $\Res^* V$ be the set of resonances $re^{i\xi}$ such that $\theta \leq \xi \leq \varphi$.

We readily check
\begin{align*}
B(V, \theta, \varphi) &= -\sum_{\lambda \in \Res^* V} \frac{\Im \lambda}{|\lambda|^2} = -\sum_{\lambda \in \Res^* V} \frac{\sin \arg \lambda}{|\lambda|}\\
  &\geq \sin \theta \sum_{\lambda \in \Res^* V} |\lambda|^{-1} \geq \sin \theta \sum_{j=0}^\infty \sum_{\substack{\lambda \in \Res^* V\\|\lambda| \in [j, j+1)}} (j+1)^{-1}\\
  &= \sin \theta \sum_{j=1}^\infty \frac{k_j - k_{j-1}}{j} = \sin \theta \sum_{j=1}^\infty k_j\left(\frac{1}{j} - \frac{1}{j-1}\right) \\
  &= \sin \theta\left(k_1 + \sum_{j=2}^\infty \frac{k_j}{j(j+1)}\right).
\end{align*}
By Froese's conjecture, $k_j \sim N(j, \theta, \varphi)$, and by assumption this implies that $k_j \gtrsim j$. So
$$B(V, \theta, \varphi) \gtrsim 1 + \sum_{j=2}^\infty \frac{j}{j(j+1)} = 1 + \sum_{j=3}^\infty j^{-1}$$
which clearly diverges.
\end{proof}
At this point of the proof, we have shown that the only way that $B(V)$ could possibly converge is if ``most of the resonances of $V$ are close to the real axis" in the sense that they are clustered in arbitrarily small sectors about $\RR$, except possibly for an exceptional set of resonances that grows sublinearly.
For compactly supported potentials, this was already proven by Zworski \cite{zworski1987distribution}.
This explains the somewhat unnatural hypotheses in Theorem \ref{divergence of breit wigner, preliminary version}: they are sufficient conditions that resonances of $V$ cannot be contained in arbitrarily small sectors about $\RR$.

\begin{proof}[Proof of Theorem \ref{divergence of breit wigner}]
Suppose first that $s(\theta, \varphi)$ does not exist. Then neither does $\lim_{r \to \infty} N(r, \theta, \varphi)/r^\rho$, so there is a sequence of $r_k$ such that
$$\lim_{k \to \infty} \frac{N(r_k, \theta, \varphi)}{r_k^\rho} = \infty;$$
since $N(\cdot, \theta, \varphi)$ is a monotone function, it follows that $N(r, \theta, \varphi) \gtrsim r^\rho$.
On the other hand, if $s(\theta, \varphi)$ exists and is nonzero, then $N(r, \theta, \varphi) \gtrsim r^\rho$ is clear.

By symmetry properties of $F$ and the assumption that $\pi \notin [\theta, \varphi]$, we may reflect the angle $[\theta, \varphi]$ around the real axis and so assume that $[\theta, \varphi] \subset [\pi, 2\pi]$.
Under this assumption, Froese's conjecture implies that
$$N(r, \theta, \varphi) \gtrsim r^\rho \geq r.$$
By Lemma \ref{no resonances of positive imaginary part}, there are only finitely many resonances in $\overline{\CC_+}$, which we may remove without affecting convergence properties of $B(V)$.
After doing so, all summands in $B(V)$ have the same sign, so $B(V) \geq B(V, \theta, \varphi)$, and by Lemma \ref{divergence of angular series}, $B(V, \theta, \varphi) = \infty$.
\end{proof}

\begin{theorem}
\label{third case of divergence}
Suppose that $V$ satisfies Froese's conjecture, $F$ is the associated entire function to $V$, and $F$ has completely regular growth.
If
$$\lim_{\varphi \to \pi} s(\pi, \varphi) = 0$$
then $B(V)$ diverges.
\end{theorem}
\begin{proof}
If $s(0, \pi)$ does not exist, then there is a jump discontinuity in $\varphi \mapsto s(0, \varphi)$ at $\pi$, say by $\eta > 0$, and then $s(0, \pi + \varepsilon) \geq \eta$ for any $\varepsilon > 0$, contradicting the hypothesis. So $s(0, \pi)$ exists.

By Lemma \ref{linear lower bound on resonances}, there is a $\xi \in [\pi, 2\pi]$ such that $s(\pi, \xi)$ exists and is nonzero.
Therefore $s(0, \xi) = s(0, \pi - \xi) + s(\pi - \xi, \pi) + s(\pi, \xi)$.
By symmetry of $F$, $s(0, \pi - \xi) = s(\pi, \xi)$ and $s(\pi - \xi, \pi) = s(2\pi - \xi, 2\pi)$, so
$$0 \neq 2s(\pi, \xi) + s(2\pi - \xi, 2\pi).$$
The terms on the right-hand side are nonnegative, so one of them is positive.

Suppose that $s(\pi, \xi) > 0$; the other case is similar. Since there is not a jump discontinuity at $\pi$, there must be a $\theta > \pi$ such that $s(\theta, \xi) > 0$. Thus this case reduces to Theorem \ref{divergence of breit wigner}.
\end{proof}




\printbibliography



\printindex

\end{document}
