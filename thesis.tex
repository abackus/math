\documentclass[12pt]{report}
\usepackage[utf8]{inputenc}
\usepackage[margin=1in]{geometry} 
\usepackage{amsmath,amsthm,amssymb}
\usepackage{mathrsfs}

\usepackage{enumitem}
%\usepackage[shortlabels]{enumerate}
\usepackage{tikz-cd}
\usepackage{mathtools}
\usepackage{amsfonts}
\usepackage{amscd}
\usepackage{makeidx}
\usepackage{enumitem}

\title{The Breit-Wigner formula}
\author{Aidan Backus}
\date{Fall 2019}

\newcommand{\NN}{\mathbf{N}}
\newcommand{\ZZ}{\mathbf{Z}}
\newcommand{\QQ}{\mathbf{Q}}
\newcommand{\RR}{\mathbf{R}}
\newcommand{\CC}{\mathbf{C}}

\DeclareMathOperator{\ch}{ch}
\DeclareMathOperator{\supp}{supp}

\newcommand{\dbar}{\overline \partial}

\newcommand{\pic}{\vspace{30mm}}
\newcommand{\dfn}[1]{\emph{#1}\index{#1}}

\renewcommand{\Re}{\operatorname{Re}}
\renewcommand{\Im}{\operatorname{Im}}

\newtheorem{theorem}{Theorem}[chapter]
\newtheorem{badtheorem}[theorem]{``Theorem"}
\newtheorem{prop}[theorem]{Proposition}
\newtheorem{lemma}[theorem]{Lemma}
\newtheorem{proposition}[theorem]{Proposition}
\newtheorem{corollary}[theorem]{Corollary}
\newtheorem{conjecture}[theorem]{Conjecture}
\newtheorem{axiom}[theorem]{Axiom}

\theoremstyle{definition}
\newtheorem{definition}[theorem]{Definition}
\newtheorem{remark}[theorem]{Remark}
\newtheorem{example}[theorem]{Example}

\theoremstyle{remark}
\newtheorem{exercise}[theorem]{Discussion topic}
\newtheorem{homework}[theorem]{Homework}
\newtheorem{problem}[theorem]{Problem}

\usepackage{color}
\usepackage{hyperref}
\hypersetup{
    colorlinks=true, % make the links colored
    linkcolor=blue, % color TOC links in blue
    urlcolor=red, % color URLs in red
    linktoc=all % 'all' will create links for everything in the TOC
    %Ning added hyperlinks to the table of contents 6/17/19
}

\makeindex
\begin{document}

\maketitle

\tableofcontents

\newpage

\section*{Acknowledgments}
I would like to thank Prof. Maciej Zworski for mentoring me, ..., and Erik Wendt for helpful comments.

\chapter{Zeroes of Fourier Transforms}
Let $\mu$ be a distribution on $\RR$, i.e. a continuous element of the dual of the space of test functions $C^\infty_{comp}(\RR)$. We will abuse notation and write $\int_E f\mu$ for the pairing of $f$ and $\mu$, whenever it is defined. If $\mu$ satisfies certain growth conditions, then the Fourier transform
$$\hat \mu(\xi) = \int_{-\infty}^\infty e^{-ix\xi} \mu(x) ~dx$$
is well-defined and is also a distribution.

For any distribution $\mu$ with bounded support $\supp \mu$, the convex hull of the support $\ch \supp \mu$ is given by a compact interval, and the Paley-Weiner theorem guarantees that $\hat \mu$ is an entire function. In particular, the number of zeroes of $\hat \mu$ lying in any compact set is necessarily finite, even when counted with multiplicity, assuming that $\mu \neq 0$.

Our goal is to prove the following theorem:
\begin{theorem}[Titchmarsh]
    \index{Titchmarsh's theorem}
    Let $\mu$ be a distribution with $\ch \supp \mu = [a, b]$. Let $N(R)$ denote the number of zeroes $z$ of $\hat \mu$ with $|z| < R$, counted with multiplicity. Then
    $$\lim_{R \to \infty} \frac{N(R)}{R} = \frac{b-a}{\pi}.$$
\end{theorem}

\section{The Paley-Weiner theorem}
We review the proof of the Paley-Weiner theorem, proven by Schwartz.

\begin{definition}
    For $\alpha,\beta$ multiindices, we define the $(\alpha,\beta)$th \dfn{Schwartz seminorm} on $\Omega \subseteq \CC^n$ by
$$||f||_{\alpha,\beta} = \sup_{x \in \Omega} |x^\alpha \partial^\beta f(x)|.$$
    The locally convex space of all $f$ for which every Schwartz seminorm $||f||_{\alpha,\beta}$ is finite is called the \dfn{Schwartz space}. A \dfn{tempered distribution} is a distribution $g$ whose pairing with every element of Schwartz space is finite.
\end{definition}
    If a distribution is tempered on $\RR^n$, then its Fourier transform is well-defined and also tempered. This can be easily proven using duality once it is shown that the Fourier transform is an automorphism of Schwartz space. Most distributions or functions which are ``not too discontinuous" and ``do not grow too fast" are tempered distributions. For example, any compactly supported distribution is tempered, as is any smooth, polynomially growing function.

If $E \subset \RR$ is bounded, then the convex hull of $E$, $\ch E$, is defined to be the intersection of all compact, convex subsets containing $E$; since $\ch E$ is connected, it must be the compact interval $[a, b]$, where $a = \inf E$ and $b = \sup E$. 
\begin{definition}
    The \dfn{supporting function} $h_E$ of a bounded set $E \subset \RR$ is defined for $\xi \in \RR$ by
$$h_E(\xi) = \sup_{x \in E} x\xi.$$
\end{definition}
    Taking the closure of $E$ will not affect $h_E$. Then for $\xi > 0$, $h_E(\xi) = b\xi$, $h_E(0) = 0$, and $h_E(-\xi)     = -a\xi$. Taking the convex hull will not change $a$ or $b$, so $h_E = h_{\ch E}$.
    
    Let us fix some notation. Let $E_\delta$ denote the ball around $E$ of radius $\delta$; that is,
$$E_\delta = \{x \in \RR: \exists y \in E~|x - y| < \delta\}.$$
    For $\mu$ a distribution on $\RR$, we will write $||\mu|| = |\int_{-\infty}^\infty \mu(x) ~dx|$, if such an integral is indeed finite. So if $\mu$ is actually a positive function, then $||\mu|| = ||\mu||_{L^1}$.
\begin{theorem}[Paley-Wiener]
    Let $E \subset \RR$ be bounded. If $\mu$ is a distribution on $\RR$ with $\supp \mu \subseteq E$, then the Fourier transform $\hat \mu$ is an entire function satisfying the estimate
$$|\hat \mu(\zeta)| = O(e^{h_E(\Im \zeta)}).$$
    Conversely, for every entire function $f$ such that $|f(\zeta)| = O(e^{h_E(\Im \zeta)})$, there is a distribution $\mu$ on $\RR$ with $\supp \mu \subseteq \ch E$ and $\hat \mu = f$.
\end{theorem}
\begin{proof}
    Since $\mu$ is compactly supported, it lies in the dual of $C^\infty(\RR)$. So $||\mu||$ is finite, and we can differentiate $\hat \mu$ by putting all derivatives on the smooth function $e^{-ix\xi}$, using integration by parts. Therefore $\hat \mu \in C^\infty(\CC)$. In particular, if $\dbar$ is the Cauchy-Riemann operator, then we can use the fact that $\dbar e^{-ix\xi} = 0$ (since $e^{-i x\xi}$ is clearly entire). Therefore $\hat \mu$ is entire.
    
    Let $\Im \zeta > 0$ and $\ch E = [a, b]$. Then
$$|\hat \mu(\zeta)| \leq \int_{-\infty}^\infty |e^{-ix\zeta} \mu(x)| ~dx \leq ||\mu|| \int_a^b e^{x \Im \zeta} ~dx \leq ||\mu|| e^{h_E(\Im \zeta)}.$$
    
    For the converse, let $f$ be an entire function satisfying $|f(\zeta)| = O(e^{h_E(\Im \zeta)})$. Then the restriction $\tilde f = f|_\RR$ is a bounded smooth function. Therefore $\tilde f$ is tempered, so has an inverse Fourier transform $\mu$.
    
    Let $\varphi \in C^\infty_{comp}((-1, 1))$ be a positive function, normalized so that such that $\int_{-1}^1 \varphi = 1$, and set $\varphi_\delta(x) \varphi(x/\delta)/\delta$. Thus $\mu * \varphi_\delta$ is a mollification of $\mu$, i.e. $\mu * \varphi_\delta \in C^\infty(\RR)$ and $\lim_{\delta \to 0} \mu * \varphi_\delta = \mu$. $\widehat{\mu*\varphi_\delta} = \hat \mu  \hat \varphi_\delta = \tilde f \hat \varphi_\delta$. Taking the unique analytic continuation of $\tilde f$ to $\CC$, we extend $\tilde f \hat \varphi_\delta$ to $f \hat \varphi_\delta$.
    
    Since $\varphi_\delta$ is supported in $(-\delta, \delta)$, $|\hat \varphi_\delta(\zeta)| = O(\delta |\Im \zeta|)$. Therefore
$$|f(\zeta) \varphi_\delta(\zeta)| = O(\exp(h_E(\Im \zeta) + \delta|\Im \zeta|)).$$
    If we can prove the converse for $\mu \in C^\infty_{comp}(\RR)$, then we can replace $\mu$ by $\mu * \varphi_\delta$ to show that $\supp (\mu * \varphi_\delta) \subseteq \ch \supp E_\delta$. Since $\delta$ was arbitrary, it will follow that $\supp \mu \subseteq \ch \supp E$. Thus, we may assume without loss of generality that $\mu \in C^\infty_{comp}(\RR)$.

    In fact, if $\mu \in C^\infty_{comp}(\RR)$, then in particular $\mu$ lies in Schwartz space. In this case, we can find a $C_n$, independent of $\zeta = \xi + i\eta$, so that
$$|\zeta^n f(\zeta)| \leq C_n e^{h_E(\Im \zeta)}.$$
    Dividing both sides by $|\zeta|^{-n}$, we have
    $$|f(\xi + i \eta)| \leq C_n \frac{e^{h_E(\eta)}}{|\xi + i\eta|^n}.$$
    Thus for $\eta$ fixed, $f(\cdot + i\eta)$ is rapidly decreasing. Thus we can make a change of variables to see that
$$\mu(x) = \frac{1}{2\pi} \int_{-\infty}^\infty e^{ix\xi} f(\xi) ~d\xi = \frac{1}{2\pi} \int_{-\infty}^\infty e^{ix(\xi+i\eta)} f(\xi + i\eta) ~d\xi.$$
    Therefore
    $$|\mu(x)| \leq C_Ne^{-x\eta + h_E(\eta)} \int_{-\infty}^\infty \frac{d\xi}{|\xi + i\eta|^N}.$$
    We fix a sufficiently large $N$ and let $\eta \to 0$. This proves that $\mu(x) = 0$ if $x\eta \leq h_E(\eta)$, which happens if and only if $x \notin \ch E$. Therefore $\supp \mu \subseteq \ch E$. 
\end{proof}
    Since the Paley-Wiener theorem is a biconditional, the estimate
    $$|\hat \mu(\zeta)| \leq ||\mu|| e^{h_E(\Im \zeta)}$$
    is sharp: we cannot replace the $a, b$ appearing in the piecewise-linear definition of $h_E$ with better constants. This precision will be important in the proof of Titchmarsh's theorem.


\section{Subharmonic functions}
Fix an open set $\Omega \subseteq \CC$. Recall that a function $u: \Omega \to [-\infty, \infty)$ which is upper-semicontinuous and not identically $-\infty$ is called subharmonic if for each $z \in \CC$, the averages
$$M(z, r) = \frac{1}{2\pi} \int_0^{2\pi} u(z + re^{i\theta}) ~d\theta$$
are increasing in $r$. In particular, an upper-semicontinuous function $u$ is subharmonic iff the weak Laplacian $\Delta u \geq 0$; that is, for any nonnegative test function $\varphi \in C^\infty_c(\Omega)$,
$$\int_\Omega u(z) \Delta \varphi(z) ~dz \geq 0.$$ If actually $\Delta u = 0$, then we call $u$ harmonic. The reason why we refer to such functions as subharmonic rather than superharmonic is that $-\Delta$ is a positive operator, and so we think of applying $\Delta$ as akin to ``multiplying by a negative function". Such a function lies in $L^1_{loc}(\Omega)$.

By the Paley-Weiner theorem, if $\mu$ is a distribution with $\ch \supp \mu = [a, b]$, then $\hat \mu$ is an entire function which satisfies the estimate
$$\hat \mu(x + iy) \leq Ce^{h(y)}$$
for some constant $C$, where $h(y) = by$ for $y > 0$, $h(y) = ay$ for $y < 0$. In particular, in the upper half-plane $\CC_+$, we have the estimate
$$\hat \mu(x + iy) \leq Ce^{by}.$$
Moreover, since $\hat \mu$ is holomorphic, it solves the Cauchy-Riemann equation $\dbar \hat \mu = 0$. Since we can factor the Laplacian as $4\Delta = \partial \dbar$, it follows that $\Delta \hat \mu = 0$.

Let
$$E(z) = \frac{\log |z|}{2\pi}.$$
Then $E$ is the fundamental solution of the Laplacian. This means that $\Delta E = \delta_0$, where $\delta_z$ denotes the Dirac distribution centered at $z$. In particular, the solution of the equation $\Delta u = f$ is $u = E*f$.

Since $\hat \mu$ is holomorphic, if it has a zero $z \in \CC_+$ of multiplicity $m$, we can write
$$\hat \mu(\zeta) = (\zeta - z)^m g(\zeta)$$
for some holomorphic function $g$ with $g(z) \neq 0$. If $g \neq 0$ everywhere, then, because $\Delta g = 0$, it follows that
$$\Delta \log |\hat \mu| = m\log |\zeta - z| + \log |g| = 2\pi m \delta_z.$$
Using a partition of unity to sum over all zeroes in this manner, it follows that if $Z$ denotes the multiset of zeroes of $\hat \mu$ counted with multiplicity, then
$$\Delta(\log |\hat \mu|) = 2\pi \sum_{z \in Z} \delta_z \geq 0.$$
It follows that $\log |\hat \mu|$ is subharmonic and, on $\CC_+$, satisfies the estimate
$$\log |\hat \mu(x + iy)| \leq C + Dy$$
for some constants $C, D$, by the Paley-Wiener theorem.

\begin{definition}
    Let $u$ be a subharmonic function on $\CC_+$. If there are constants $C, D > 0$ so that $u(x + iy) \leq C + Dy$, then we say that $u$ is \dfn{imaginary-sublinear}.
\end{definition}
It is immediate that $\log |\hat \mu|$ is imaginary-sublinear.

\begin{lemma}
    Let $u$ be an imaginary-sublinear subharmonic function on $\CC_+$ and define
    $$\gamma = \lim_{y \to \infty} \sup_x u(x + iy).$$
    Then $\gamma$ is well-defined, and $\gamma \in (-\infty, D]$.
\end{lemma}
\begin{proof}
    We first let
    $$M(y) = \sup_x u(x + iy).$$
    Then $\gamma = \lim_y M(y)$, and $M(y) \leq C + Dy$ since $u$ is imaginary-sublinear.
    
    Let us prove that $M$ is convex. Let $0 < a < b$ and let $L: \RR \to \RR$ be a linear function such that $M(a) \leq L(a)$ and $M(b) \leq L(b)$. Let
    $$v(x + iy, \varepsilon) = u(x + iy) - L(y) - \varepsilon(x^2 - (y^2 - b^2)).$$
    Then $v(\cdot + iy_0, \varepsilon) \leq 0$ for $y_0 \in \{a, b\}$. Similarly, $\lim_{x \to \pm \infty} v(x + \cdot, \varepsilon) = -\infty$. Moreover, $\Delta v(\cdot, \varepsilon) \geq \Delta u \geq 0$.
    
    Applying the maximum principle to a sufficiently large compact subset $K$, and noting that $v(\cdot, \varepsilon) \leq 0$ on $\partial K$, $v(\cdot, \varepsilon) \leq 0$ on $K$, hence globally since $K$ was arbitrary. Taking $\varepsilon \to 0$, we see that $u(x + iy) \leq L(y)$, so maximizing over $x$, $M \leq L$ on $[a, b]$. So $M$ is convex. But the limit of a sublinear, convex function exists, so $\gamma$ is well-defined, and $\gamma > -\infty$ since $u \in L^1_{loc}(\Omega)$, hence not $-\infty$ except on a discrete set. The bound $\gamma \leq D$ follows easily.
\end{proof}

We now give a representation formula for functions on the upper-half plane in terms of their Laplacian and their boundary values.
\begin{definition}
    The \dfn{Green function} for $\Delta$ on $\CC_+$ is defined on $\CC_+ \times \CC_+$ by
$$G(z, w) = E(z - w) - E(z - \overline w).$$
    The \dfn{Poisson kernel} for $\Delta$ on $\CC_+$ is defined on $\CC+_ \times \RR$ by
$$P(z, x) = -\frac{\partial G(z, x + iy)}{\partial y}|_{y = 0}.$$
\end{definition}
    Then $\Delta_w G(z, w) = \Delta_w E(z - w) = \delta_z$ since $\overline w \notin \CC_+$, hence $z \neq w$. Moreover, $\lim_{w \to 0} G(z, w) = 0$ for $z$ fixed. We can rewrite $G$ as
$$G(z, w) = \frac{1}{2\pi} \log\left(\frac{z - w}{z - \overline w}\right),$$
    which is clearly homogeneous: for $t > 0$, $G(tz, tw) = G(z, w)$. Since $(z-w)/(z - \overline w) \to 1$ as $z \to \infty$, $G(z, w) \to 0$.
    
\begin{lemma}
\label{estimate on Green function}
    Fix $w \in \CC_+$. Then there is a constant $C$ such that
$$\frac{\Im z}{C(1 + |z|)^2} \leq |G(z, w)| \leq \frac{C \Im z}{(1 + |z|)^2}.$$ 
\end{lemma}
\begin{proof}
    Clearly $\Im z(1 + |z|)^{-2} = O(|z|^{-1})$, so $\Im z \leq (1 + |z|^2) \leq |z|^{-1}$ off of a compact set, with bounded error on that compact set. wait this doesn't work because i ineq'd the log int he wrong direction.
\end{proof}

\begin{theorem}[Green representation formula]
    \index{Green representation formula}
    For any function $u \in C_{comp}^\infty(\CC)$ and $z \in \CC_+$, one has
    $$u(z) = \int_{-\infty}^\infty P(z, x) u(x) ~dx + \int_{\CC_+} G(z, w) \Delta u(w) ~dw.$$
\end{theorem}
In particular, a harmonic function can be recovered by integrating it against the Poisson kernel. In light of the Riemann mapping theorem, one could deform the Green function on $\CC_+$ into a Green function on $U$ for any open set $U \subseteq \CC$ with trivial fundamental group.

To prove the Green representation formula, we need Green's identity.
\begin{lemma}[Green's identity]
    \index{Green's identity}
    If $u,v \in C^2(\overline{\CC_+} \to \RR)$, we have
$$\int_{\CC_+} u \Delta v - v \Delta u = \int_{-\infty}^\infty u \partial_y v - v \partial_y u.$$
\end{lemma}
\begin{proof}
It follows immediately from Stokes' theorem for manifolds that for a vector field $F = (F_x, F_y): \CC \to \CC$,
$$\int_{\CC_+} \nabla \cdot F = \int_{-\infty}^\infty F_y.$$
In case $F = f\nabla g$, we prove
$$\int_{\CC_+} \nabla f \cdot \nabla g + \int_{\CC_+} f\Delta g = \int_{-\infty}^\infty f\partial_y g.$$
Swapping $f$ and $g$ in the above formula and subtracting the two claims, the lemma follows.
\end{proof}
\begin{proof}[Proof of Green representation formula]
We consider Green's identity with $v = G$, $z = x + iy$,
\begin{align*}
    u(w) - \int_{\CC_+} G(z, w) \Delta u(z) ~dz 
        &= \int_{\CC_+} u(z) \delta_w(z) - G(z, w) \Delta u(z) ~dz\\
        &= \int_{-\infty}^\infty u(z) \partial_y G(z, w)|_{y = 0} - G(0, w) \partial_y u(z) ~dz\\
        &= \int_{-\infty}^\infty u(z) P(w, x) ~dz.
\end{align*}
This proves the claim.
\end{proof}

The Green representation formula generalizes to a representation theorem due to Riesz, not to be confused with his other two representation theorems.
\begin{theorem}[Riesz representation formula]
    \index{Riesz representation formula}
    Let $u$ be an imaginary-sublinear subharmonic function on $\CC_+$ and let $\gamma$ be as above. Then $v(\cdot + iy)$ converges to a distribution $\sigma$ on $\RR$ as $y \to 0$ such that
$$\int_{-\infty}^\infty \frac{|\sigma(x) ~dx|}{(1 + |x|)^2} < \infty.$$
    Let $\mu = \Delta v$; then for any $w \in \CC_+$,
    $$\left|\int_{\CC_+} G(z, w) \mu(z)\right| < \infty$$
    and moreover
$$u(z) = \int_{\CC_+} G(z, w) \mu(w) ~dw + \int_{-\infty}^\infty P(z, x) \sigma(x) ~dx + \gamma \Im z.$$
\end{theorem}
\begin{proof}
    Let us replace $u(z)$ with $u(z) - C - \gamma \Im z$. Then $u(z) \leq D + \Im z$, and if we take an optimal choice of $D$, then $D \leq 0$, so $u \leq 0$. After completing the proof, we can simply add a $\gamma \Im z$ back in, the $C$ having already been absorbed into the boundary term $\sigma$.
    
    We first decompose $u$. Fix an increasing chain $K_j$ of compact sets which cover $\CC_+$, and let $\chi_j \in C^\infty_{comp}(\CC_+)$ be an increasing chain of cutoff functions which are identically $1$ on $K_j$. Then let
    $$v_j(z) = u(z) - \int_{\CC_+} G(z, w) \chi_j(w)\mu(w) ~dw.$$
\begin{lemma}
    The functions $v_j$ are subharmonic on $\CC_+$, harmonic on $K_j$, and $\leq 0$.
\end{lemma}
\begin{proof}[Proof of lemma]
    We compute
\begin{align*}
    \Delta v_j(z) &= \mu(z) - \int_{\CC_+} \Delta_z(E(z - w) - E(z - \overline w)) \chi_j(w) \mu(w) ~dw \\&= \mu(z) - \int_{\CC_+} (\delta_w(z) - \delta_{\overline w}(z) \chi_j(w) \mu(w) ~dw \\&= \mu(z)(1 - \chi_j(z)),
\end{align*}
    the $\delta_{\overline w}$ term vanishing because $z \neq \overline w$, since $\overline w \notin \CC_+$. Since $\chi_j$ is a cutoff, $\chi_j \leq 1$, so $\Delta v_j \geq 0$. On the other hand, if $z \in K_j$, then $1 - \chi_j(z) = 0$, so $\Delta v_j(z) = 0$. This proves the first two claims.
    
    Let $\varepsilon > 0$; we will prove that $v_j < \varepsilon$. Since $u$ is subharmonic, $\mu \geq 0$, and $\chi_j \geq 0$ while $G \leq 0$, so the integral $\int_{\CC_+} G(z, w) \chi_j(w) \mu(w) ~d\mu(w) \leq 0$, and we are only integrating over $w$ close to the compact set $K_j$. So we can view $w$ as essentially fixed compared to $z$, and apply the estimate $G(z, w) = O(\Im z|z|^{-2})$ to see that $\int_{\CC_+} G(z, w) \chi_j(w) \mu(w) ~d\mu(w) \to 0$ as $z \to \infty$. In particular, $\int_{\CC_+} G(z, w) \chi_j(w) \mu(w) ~d\mu(w) > \varepsilon$ for $|z|$ large enough. Since $u \leq 0$, $v_j(z) < \varepsilon$ for $z$ large enough, hence for any $z$ by the maximum principle. Therefore $v_j < 0$.
\end{proof}
    The $v_j$ form an increasing sequence which is bounded above, so converge to a limit $u_1 \leq 0$. Moreover, $\Delta v_j \to 0$ pointwise, so $u_1$ is harmonic. Meanwhile, $\chi_j \to 1$ pointwise, so if we let
$$u_2(z) = \int_{\CC_+} G(z, w) \mu(w) ~dw,$$
    we arrive at the decomposition $u = u_1 + u_2$, $\Delta u_1 = 0$.
    
    Since $u_1$ is harmonic, it is finite. There is a $z \in \CC_+$ so that $u(z) > -\infty$, and so $u_2(z) > -\infty$ also. This proves the claimed estimate on $\mu$. (TODO: This AND MUST BE be sharpened somehow?)
\begin{lemma}
    In the sense of distributions,
$$\lim_{y \to 0} u_2(\cdot + iy) = 0.$$
\end{lemma}
\begin{proof}[Proof of lemma]
    Let $\varphi \in C^\infty_{comp}(\RR)$ be a test function, $\ch \supp \varphi = [a, b]$. We must show that the limit
$$\lim_{y \to 0} \int_{-\infty}^\infty u_2(x + iy) \varphi(x) ~dx = \lim_{y \to 0} \int_{\CC_+} \mu(w) \int_{-\infty}^\infty G(x + iy, w) \varphi(x) ~dx ~dw = 0.$$
    By symmetry, $G(z, w) = G(w, z)$, so $G(x + iy, w)$ vanishes for fixed $w,x$ as $y \to 0$. In particular, for fixed $w$,
$$\lim_{y \to 0} \int_{-\infty}^\infty G(x + iy, w) \varphi(x) ~dx = 0.$$
    By Lemma \ref{estimate on Green function}, $G(x + iy, w) = O(\Im w|w|^{-2})$. (TODO: Actually the whole integral is bounded by this.)
    
    We now define
$$F(w) = \int_{-\infty}^\infty E(w - x)\varphi(x) ~dx.$$
    Now $E$ is continuous away from $0$, but $\Im w > 0$, so the integrand is continuous. The integral is formally taken over $\RR$, but is actually being taken over $[a, b]$, so the integrand is integrable; hence $F$ is continuous. Since $\Delta E(w - x) = 0$ for $\Im w > 0$, $F$ is harmonic. But $F$ is a convolution, so if $P$ is any linear differential operator in $\Re w$, $PF = E * P\varphi$, which is continuous since $\varphi$ is smooth. Therefore $PF$ is locally bounded. In particular,
$$\partial_{\Im w}^2 F(w) = \Delta F(w) - \partial_{\Re w}^2 F(w) = -\partial_{\Re w}^2 F(w)$$
    which is locally bounded. So $PF$ is locally bounded for any linear differential operator whatsoever. In particular, $F \in C^{Lip}_{loc}(\CC_+)$. Because $F(w)$ is bounded for a fixed $\Im w$, since $F$ is continuous and $\Re a$ ranges over the compact set $[a, b]$, we have $F(w) = O(\Im w)$. That is,
$$\int_{-\infty}^\infty G(x + iy, w) \varphi(x) ~dx = F(\overline w + iy) - F(w + iy) = O(\Im w)$$
    for bounded $y$. In particular, the integral is bounded on any compact set in $w$. Therefore
$$\int_{-\infty}^\infty G(x + iy, w) \varphi(x) ~dx = O\left(\frac{|\Im w|}{(1 + |w|^2)}\right).$$
    
    We now set
$$H(y, w) = \int_{-\infty}^\infty G(x + iy, w)\varphi(x) ~dx.$$
    Then $H(y, \cdot)$ is dominated
\end{proof}
\begin{lemma}
    For any $\varepsilon > 0$ and $y > \varepsilon$,
$$u_1(x + iy) = \int_{-\infty}^\infty P(x + i(y-\varepsilon), t) u_1(t + i\varepsilon) ~dt.$$
\end{lemma}
\begin{proof}
    TODO
\end{proof}
    Since $\gamma = 0$, for any $\delta > 0$ we can find $y$ so large that $u(x + iy) > -\delta y$. So TODO, thus if $M_\delta$ is large enough, TODO, so the distributions $u_1(\cdot, \varepsilon)$ are uniformly bounded in $\varepsilon$. The Banach-Alaoglu theorem furnishes a weak limit $\sigma$. TODO: $\sigma$ has the properties we want.
\end{proof}

\begin{corollary}
    \label{gamma in mean}
    Let $u$ be an imaginary-sublinear subharmonic function on $\CC_+$, and let $\gamma = \lim_{y \to \infty} \sup_x u(x + iy)/y$, as above. Then
$$\lim_{t \to \infty} \frac{u(t(x + iy))}{t} = \gamma y$$
    in $L^1_{loc}(\overline{\CC_+})$.
\end{corollary}
\begin{proof}
    Fix a compact set $K \subset \overline{\CC_+}$. If $u$ is a constant, then both sides of the claimed equation are $0$, so by linearity we may subtract $C$ from $u$, and so assume without loss of generality that $C = 0$. Then $u(t(x+iy)) \leq \gamma y$, so it suffices to prove that
$$\lim_{t \to \infty} \int_K \frac{u(tz)}{t} - \gamma \Im z ~dz = 0$$
    to show convergence in $L^1(K)$.
    
    Let

\end{proof}



\section{Proof of Titchmarsh's theorem}
Fix a distribution $\mu$ on $\RR$ with $\ch \supp \mu = [a, b]$. By the Paley-Weiner theorem, $\hat \mu$ is an entire function and $\log |\hat \mu|$ is an imaginary-sublinear subharmonic function. In fact, viewing $\log |\hat \mu|$ as a function on $\CC_+$ and taking $\gamma$ as in Corollary \ref{gamma in mean}, we have $\gamma = b$. On the other hand, viewing $\log |\hat \mu|$ as a function on $\CC_-$, we have $\gamma = a$. Let $Z$ be the multiset of zeroes of $\hat \mu$ with repetition.

Let $h(t) = at$ for $t < 0$, $h(t) = bt$ for $t > 0$. Then $h'$ is a rescaled Heaviside function, so $h'' = (b-a)\delta_0$. Using Corollary \ref{gamma in mean}, we have
$$\lim_{t \to \infty} \frac{\log |\hat \mu(t\zeta)|}{t} = h(\Im \zeta)$$
in $L^1_{loc}$. Taking the Laplacian of both sides, we see that
$$\lim_{t \to \infty} \frac{2\pi}{t} \sum_{z \in Z} \delta_{z/t}(w) = (b-a)\delta_0(\Im w)$$
in the sense of distributions.

Integrating both sides,
$$\int_{D(0, 1)} \lim_{t \to \infty} \frac{2\pi}{t} \sum_{z \in Z} \delta_{z/t} = \int_{D(0, 1)} (b-a)\delta_0(\Im w) ~dw = \int_{-1}^1 b - a ~dw = 2(b - a).$$
Rewriting the left-hand side, we have
$$\lim_{R \to \infty} \frac{2\pi }{R} \int_{D(0, R)} \sum_{z \in Z} \delta_z = \lim_{R \to \infty} \frac{2\pi N(R)}{R}.$$
Dividing both sides by $2\pi$, we see that
$$\lim_{R \to \infty} \frac{N(R)}{R} = \frac{b - a}{\pi}.$$
This proves Titchmarsh's theorem.








\printindex

\end{document}
