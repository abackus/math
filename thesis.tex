\documentclass[12pt]{report}
\usepackage[utf8]{inputenc}
\usepackage[margin=1in]{geometry}
\usepackage{amsmath,amsthm,amssymb}
\usepackage{mathrsfs}

\usepackage{enumitem}
%\usepackage[shortlabels]{enumerate}
\usepackage{tikz-cd}
\usepackage{mathtools}
\usepackage{amsfonts}
\usepackage{amscd}
\usepackage{makeidx}
\usepackage{enumitem}

\title{The Breit-Wigner formula}
\author{Aidan Backus}
\date{Fall 2019}

\newcommand{\NN}{\mathbf{N}}
\newcommand{\ZZ}{\mathbf{Z}}
\newcommand{\QQ}{\mathbf{Q}}
\newcommand{\RR}{\mathbf{R}}
\newcommand{\CC}{\mathbf{C}}
\newcommand{\DD}{\mathbf{D}}

\DeclareMathOperator{\ch}{ch}
\DeclareMathOperator{\supp}{supp}

\newcommand{\dbar}{\overline \partial}

\newcommand{\pic}{\vspace{30mm}}
\newcommand{\dfn}[1]{\emph{#1}\index{#1}}

\renewcommand{\Re}{\operatorname{Re}}
\renewcommand{\Im}{\operatorname{Im}}

\newtheorem{theorem}{Theorem}[chapter]
\newtheorem{badtheorem}[theorem]{``Theorem"}
\newtheorem{prop}[theorem]{Proposition}
\newtheorem{lemma}[theorem]{Lemma}
\newtheorem{proposition}[theorem]{Proposition}
\newtheorem{corollary}[theorem]{Corollary}
\newtheorem{conjecture}[theorem]{Conjecture}
\newtheorem{axiom}[theorem]{Axiom}

\theoremstyle{definition}
\newtheorem{definition}[theorem]{Definition}
\newtheorem{remark}[theorem]{Remark}
\newtheorem{example}[theorem]{Example}

\theoremstyle{remark}
\newtheorem{exercise}[theorem]{Discussion topic}
\newtheorem{homework}[theorem]{Homework}
\newtheorem{problem}[theorem]{Problem}

\usepackage{color}
\usepackage{hyperref}
\hypersetup{
    colorlinks=true, % make the links colored
    linkcolor=blue, % color TOC links in blue
    urlcolor=red, % color URLs in red
    linktoc=all % 'all' will create links for everything in the TOC
    %Ning added hyperlinks to the table of contents 6/17/19
}

\usepackage[backend=bibtex]{biblatex}
\addbibresource{thesis.bib}

\makeindex
\begin{document}

\maketitle

\tableofcontents


\section*{Acknowledgments}
I would like to thank Prof. Maciej Zworski for suggesting this topic and mentoring me, as well as teaching me much of what I know about partial differential equations, harmonic analysis, and complex analysis. I would also like to thank Erik Wendt for suggesting the use of conformal mappings to better explain Green functions.



\chapter{Zeroes of Fourier Transforms}
Let $\mu$ be a distribution on $\RR$, i.e. a continuous element of the dual of the space of test functions $C^\infty_{comp}(\RR)$. We will abuse notation and write $\int_E f\mu$ for the pairing of $f$ and $\mu$, whenever it is defined. If $\mu$ satisfies certain growth conditions, then the Fourier transform
$$\hat \mu(\xi) = \int_{-\infty}^\infty e^{-ix\xi} \mu(x) ~dx$$
is well-defined and is also a distribution.

For any distribution $\mu$ with bounded support $\supp \mu$, the convex hull of the support $\ch \supp \mu$ is given by a compact interval, and the Paley-Weiner theorem guarantees that $\hat \mu$ is an entire function. In particular, the number of zeroes of $\hat \mu$ lying in any compact set is necessarily finite, even when counted with multiplicity, assuming that $\mu \neq 0$.

Our goal is to prove the following theorem, which was first proven by Titchmarsh \cite{titchmarsh1926zeros}. The proof we give is based on the proof of Beurling that was first published in \cite[Chapter XVI]{hormander2004analysis}.
\begin{theorem}
    \index{Titchmarsh's theorem}
    Let $\mu$ be a distribution with $\ch \supp \mu = [a, b]$. Let $N(R)$ denote the number of zeroes $z$ of $\hat \mu$ with $|z| < R$, counted with multiplicity. Then
    $$\lim_{R \to \infty} \frac{N(R)}{R} = \frac{b-a}{\pi}.$$
\end{theorem}

\section{The Paley-Weiner theorem}
We review the proof of the Paley-Weiner theorem, proven by Schwartz. The exposition given here is based on \cite[Chapter VII]{hormander2015analysis}.

\begin{definition}
    For $\alpha,\beta$ multiindices, we define the $(\alpha,\beta)$th \dfn{Schwartz seminorm} on $\Omega \subseteq \CC^n$ by
$$||f||_{\alpha,\beta} = \sup_{x \in \Omega} |x^\alpha \partial^\beta f(x)|.$$
    The locally convex space of all $f$ for which every Schwartz seminorm $||f||_{\alpha,\beta}$ is finite is called the \dfn{Schwartz space}. A \dfn{tempered distribution} is a distribution $g$ whose pairing with every element of Schwartz space is finite.
\end{definition}
    If a distribution is tempered on $\RR^n$, then its Fourier transform is well-defined and also tempered. This can be easily proven using duality once it is shown that the Fourier transform is an automorphism of Schwartz space. Most distributions or functions which are ``not too discontinuous" and ``do not grow too fast" are tempered distributions. For example, any compactly supported distribution is tempered, as is any smooth, polynomially growing function.

If $E \subset \RR$ is bounded, then the convex hull of $E$, $\ch E$, is defined to be the intersection of all compact, convex subsets containing $E$; since $\ch E$ is connected, it must be the compact interval $[a, b]$, where $a = \inf E$ and $b = \sup E$.
\begin{definition}
    The \dfn{supporting function} $h_E$ of a bounded set $E \subset \RR$ is defined for $\xi \in \RR$ by
$$h_E(\xi) = \sup_{x \in E} x\xi.$$
\end{definition}
    Taking the closure of $E$ will not affect $h_E$. Then for $\xi > 0$, $h_E(\xi) = b\xi$, $h_E(0) = 0$, and $h_E(-\xi)     = -a\xi$. Taking the convex hull will not change $a$ or $b$, so $h_E = h_{\ch E}$.

    Let us fix some notation. Let $E_\delta$ denote the ball around $E$ of radius $\delta$; that is,
$$E_\delta = \{x \in \RR: \exists y \in E~|x - y| < \delta\}.$$
    For $\mu$ a distribution on $\RR$, we will write $||\mu|| = |\int_{-\infty}^\infty \mu(x) ~dx|$, if such an integral is indeed finite. So if $\mu$ is actually a positive function, then $||\mu|| = ||\mu||_{L^1}$.
\begin{theorem}[Paley-Wiener]
    Let $E \subset \RR$ be bounded. If $\mu$ is a distribution on $\RR$ with $\supp \mu \subseteq E$, then the Fourier transform $\hat \mu$ is an entire function satisfying the estimate
$$|\hat \mu(\zeta)| = O(e^{h_E(\Im \zeta)}).$$
    Conversely, for every entire function $f$ such that $|f(\zeta)| = O(e^{h_E(\Im \zeta)})$, there is a distribution $\mu$ on $\RR$ with $\supp \mu \subseteq \ch E$ and $\hat \mu = f$.
\end{theorem}
\begin{proof}
    Since $\mu$ is compactly supported, it lies in the dual of $C^\infty(\RR)$. So $||\mu||$ is finite, and we can differentiate $\hat \mu$ by putting all derivatives on the smooth function $e^{-ix\xi}$, using integration by parts. Therefore $\hat \mu \in C^\infty(\CC)$. In particular, if $\dbar$ is the Cauchy-Riemann operator, then we can use the fact that $\dbar e^{-ix\xi} = 0$ (since $e^{-i x\xi}$ is clearly entire). Therefore $\hat \mu$ is entire.

    Let $\Im \zeta > 0$ and $\ch E = [a, b]$. Then
$$|\hat \mu(\zeta)| \leq \int_{-\infty}^\infty |e^{-ix\zeta} \mu(x)| ~dx \leq ||\mu|| \int_a^b e^{x \Im \zeta} ~dx \leq ||\mu|| e^{h_E(\Im \zeta)}.$$

    For the converse, let $f$ be an entire function satisfying $|f(\zeta)| = O(e^{h_E(\Im \zeta)})$. Then the restriction $\tilde f = f|_\RR$ is a bounded smooth function. Therefore $\tilde f$ is tempered, so has an inverse Fourier transform $\mu$.

    Let $\varphi \in C^\infty_{comp}((-1, 1))$ be a positive function, normalized so that such that $\int_{-1}^1 \varphi = 1$, and set $\varphi_\delta(x) \varphi(x/\delta)/\delta$. Thus $\mu * \varphi_\delta$ is a mollification of $\mu$, i.e. $\mu * \varphi_\delta \in C^\infty(\RR)$ and $\lim_{\delta \to 0} \mu * \varphi_\delta = \mu$. $\widehat{\mu*\varphi_\delta} = \hat \mu  \hat \varphi_\delta = \tilde f \hat \varphi_\delta$. Taking the unique analytic continuation of $\tilde f$ to $\CC$, we extend $\tilde f \hat \varphi_\delta$ to $f \hat \varphi_\delta$.

    Since $\varphi_\delta$ is supported in $(-\delta, \delta)$, $|\hat \varphi_\delta(\zeta)| = O(\delta |\Im \zeta|)$. Therefore
$$|f(\zeta) \varphi_\delta(\zeta)| = O(\exp(h_E(\Im \zeta) + \delta|\Im \zeta|)).$$
    If we can prove the converse for $\mu \in C^\infty_{comp}(\RR)$, then we can replace $\mu$ by $\mu * \varphi_\delta$ to show that $\supp (\mu * \varphi_\delta) \subseteq \ch \supp E_\delta$. Since $\delta$ was arbitrary, it will follow that $\supp \mu \subseteq \ch \supp E$. Thus, we may assume without loss of generality that $\mu \in C^\infty_{comp}(\RR)$.

    In fact, if $\mu \in C^\infty_{comp}(\RR)$, then in particular $\mu$ lies in Schwartz space. In this case, we can find a $C_n$, independent of $\zeta = \xi + i\eta$, so that
$$|\zeta^n f(\zeta)| \leq C_n e^{h_E(\Im \zeta)}.$$
    Dividing both sides by $|\zeta|^{-n}$, we have
    $$|f(\xi + i \eta)| \leq C_n \frac{e^{h_E(\eta)}}{|\xi + i\eta|^n}.$$
    Thus for $\eta$ fixed, $f(\cdot + i\eta)$ is rapidly decreasing. Thus we can make a change of variables to see that
$$\mu(x) = \frac{1}{2\pi} \int_{-\infty}^\infty e^{ix\xi} f(\xi) ~d\xi = \frac{1}{2\pi} \int_{-\infty}^\infty e^{ix(\xi+i\eta)} f(\xi + i\eta) ~d\xi.$$
    Therefore
    $$|\mu(x)| \leq C_Ne^{-x\eta + h_E(\eta)} \int_{-\infty}^\infty \frac{d\xi}{|\xi + i\eta|^N}.$$
    We fix a sufficiently large $N$ and let $\eta \to 0$. This proves that $\mu(x) = 0$ if $x\eta \leq h_E(\eta)$, which happens if and only if $x \notin \ch E$. Therefore $\supp \mu \subseteq \ch E$.
\end{proof}
    Since the Paley-Wiener theorem is a biconditional, the estimate
    $$|\hat \mu(\zeta)| \leq ||\mu|| e^{h_E(\Im \zeta)}$$
    is sharp: we cannot replace the $a, b$ appearing in the piecewise-linear definition of $h_E$ with better constants. This precision will be important in the proof of Titchmarsh's theorem.


\section{Subharmonic functions}
Fix an open set $\Omega \subseteq \CC$. Recall that a function $u: \Omega \to [-\infty, \infty)$ which is upper-semicontinuous is called subharmonic if for each $z \in \CC$, the averages
$$M(z, r) = \frac{1}{2\pi} \int_0^{2\pi} u(z + re^{i\theta}) ~d\theta$$
are increasing in $r$. This condition is logically equivalent to assuming that $u$ satisfies a maximum principle: for every compact set $K$ with nonempty interior $U$, if $u|_K$ attains its maximum on $U$, then $u|_K$ is a constant. To avoid trivialities, we shall assume that $u$ is not identically $-\infty$, though not every author makes this assumption.

It is a well-known result that an upper-semicontinuous function $u$ is subharmonic iff the weak Laplacian $\Delta u \geq 0$; that is, for any nonnegative test function $\varphi \in C^\infty_{comp}(\Omega)$,
$$\int_\Omega u(z) \Delta \varphi(z) ~dz \geq 0.$$ If actually $\Delta u = 0$, then we call $u$ harmonic; by a typical mollification argument, if $u$ is harmonic, then $u$ is actually smooth (even real analytic), but such niceties may not be true for subharmonic functions. The reason why we refer to functions $u$ with $\Delta u \geq 0$ as subharmonic rather than superharmonic is that $-\Delta$ is a positive operator, and so we think of applying $\Delta$ as akin to ``multiplying by a negative function". Such a function lies in $L^1_{loc}(\Omega)$. (For the proofs, see \cite[Chapter 1]{hormander1973introduction}.)

By the Paley-Weiner theorem, if $\mu$ is a distribution with $\ch \supp \mu = [a, b]$, then $\hat \mu$ is an entire function which satisfies the estimate
$$\hat \mu(x + iy) \leq Ce^{h(y)}$$
for some constant $C$, where $h(y) = by$ for $y > 0$, $h(y) = ay$ for $y < 0$. In particular, in the upper half-plane $\CC_+$, we have the estimate
$$\hat \mu(x + iy) \leq Ce^{by}.$$
Moreover, since $\hat \mu$ is holomorphic, it solves the Cauchy-Riemann equation $\dbar \hat \mu = 0$. Since we can factor the Laplacian as $4\Delta = \partial \dbar$, it follows that $\Delta \hat \mu = 0$.

Let
$$E(z) = \frac{\log |z|}{2\pi}.$$
Then $E$ is the fundamental solution of the Laplacian. This means that $\Delta E = \delta_0$, where $\delta_z$ denotes the Dirac distribution centered at $z$. In particular, the solution of the equation $\Delta u = f$ is $u = E*f$.

Since $\hat \mu$ is holomorphic, if it has a zero $z \in \CC_+$ of multiplicity $m$, we can write
$$\hat \mu(\zeta) = (\zeta - z)^m g(\zeta)$$
for some holomorphic function $g$ with $g(z) \neq 0$. If $g \neq 0$ everywhere, then, because $\Delta g = 0$, it follows that
$$\Delta \log |\hat \mu| = m\log |\zeta - z| + \log |g| = 2\pi m \delta_z.$$
Using a partition of unity to sum over all zeroes in this manner, it follows that if $Z$ denotes the multiset of zeroes of $\hat \mu$ counted with multiplicity, then
$$\Delta(\log |\hat \mu|) = 2\pi \sum_{z \in Z} \delta_z \geq 0.$$
It follows that $\log |\hat \mu|$ is subharmonic and, on $\CC_+$, satisfies the estimate
$$\log |\hat \mu(x + iy)| \leq C + Dy$$
for some constants $C, D$, by the Paley-Wiener theorem.

\begin{definition}
    Let $u$ be a subharmonic function on $\CC_+$. If there are constants $C, D > 0$ so that $u(x + iy) \leq C + Dy$, then we say that $u$ is \dfn{imaginary-sublinear}.
\end{definition}
It is immediate that $\log |\hat \mu|$ is imaginary-sublinear.

\begin{lemma}
    \label{imaginary sublinear limit}
    Let $u$ be an imaginary-sublinear subharmonic function on $\CC_+$ and define
    $$\gamma = \lim_{y \to \infty} \sup_x u(x + iy).$$
    Then $\gamma$ is well-defined, and $\gamma \in (-\infty, D]$.
\end{lemma}
\begin{proof}
    We first let
    $$M(y) = \sup_x u(x + iy).$$
    Then $\gamma = \lim_y M(y)$, and $M(y) \leq C + Dy$ since $u$ is imaginary-sublinear.

    Let us prove that $M$ is convex. Let $0 < a < b$ and let $L: \RR \to \RR$ be a linear function such that $M(a) \leq L(a)$ and $M(b) \leq L(b)$. Let
    $$v(x + iy, \varepsilon) = u(x + iy) - L(y) - \varepsilon(x^2 - (y^2 - b^2)).$$
    Then $v(\cdot + iy_0, \varepsilon) \leq 0$ for $y_0 \in \{a, b\}$. Similarly, $\lim_{x \to \pm \infty} v(x + \cdot, \varepsilon) = -\infty$. Moreover, $\Delta v(\cdot, \varepsilon) \geq \Delta u \geq 0$.

    Applying the maximum principle to a sufficiently large compact subset $K$, and noting that $v(\cdot, \varepsilon) \leq 0$ on $\partial K$, $v(\cdot, \varepsilon) \leq 0$ on $K$, hence globally since $K$ was arbitrary. Taking $\varepsilon \to 0$, we see that $u(x + iy) \leq L(y)$, so maximizing over $x$, $M \leq L$ on $[a, b]$. So $M$ is convex. But the limit of a sublinear, convex function exists, so $\gamma$ is well-defined, and $\gamma > -\infty$ since $u \in L^1_{loc}(\Omega)$, hence not $-\infty$ except on a discrete set. The bound $\gamma \leq D$ follows easily.
\end{proof}

The main result of this section is a representation formula for functions on the upper-half plane in terms of their Laplacian and their boundary values. To do this, we construct the Green function of $\Delta$ in the half plane.
\begin{definition}
    The \dfn{Green function} for $\Delta$ on $\CC_+$ is defined on $\CC_+ \times \CC_+$ by
$$G(z, w) = E(z - w) - E(z - \overline w).$$
    The \dfn{Poisson kernel} for $\Delta$ on $\CC_+$ is defined on $\CC+_ \times \RR$ by
$$P(z, x) = -\frac{\partial G(z, x + iy)}{\partial y}|_{y = 0}.$$
\end{definition}
  To motivate the definition of a Green function, suppose that we want to solve the boundary-value problem for $\Delta$ on $\CC_+$. That is, given any function $f \in C(\RR)$, we want to find a harmonic function $u$ on $\CC_+$ which continuously extends to $\RR$, such that $u|_\RR = f$. If we can find a function $G$ on $\CC_+ \times \CC_+$ such that $\Delta G(z, w) = \delta_z w$ which continuously extends to $0$ on $\RR$, then one can use Stokes's theorem to see that
$$u(z) = \int_{-\infty}^\infty f(x) P(z, x) ~dx,$$
  since by definition the Poisson kernel is the normal derivative (i.e. infinitesimal of the flux) of $G$ along $\RR$. Now $G(z, w) = E(z - w)$ would suffice as such a function, except that it is nonzero at the boundary. On the other hand, $\Delta_z E(z - \overline w) = 0$ for $z, w \in \CC_+$, and by symmetry introducing this error term will cancel out the boundary term in $E(z - w)$.

    One has $\Delta_w G(z, w) = \Delta_w E(z - w) = \delta_z$ since $\overline w \notin \CC_+$, hence $z \neq w$. Moreover, $\lim_{w \to 0} G(z, w) = 0$ for $z$ fixed. We can rewrite $G$ as
$$G(z, w) = \frac{1}{2\pi} \log\left|\frac{z - w}{z - \overline w}\right|,$$
    which is clearly homogeneous: for $t > 0$, $G(tz, tw) = G(z, w)$. Since $(z-w)/(z - \overline w) \to 1$ as $z \to \infty$, $G(z, w) \to 0$.  Moreover,
$$\partial_y G(z, x + iy) = -\frac{1}{2\pi} \left(\frac{y - \Re z}{|x + iy - z|^2} - \frac{y + \Re z}{|x + iy - \overline z|^2}\right)$$
  and setting $y = 0$ we have
$$P(z, x) = \frac{\Im z}{2\pi|z-x|^2}.$$
We have the estimate $P(z, x) = O(\Im z|z|^{-2})$ for small $x$ as $z \to \infty$. In the other direction, $P$ is a nascent Dirac mass in the sense that
$$\lim_{b \to 0} P(a + ib, x) = \delta_x(a).$$ Moreover, $\Delta G(\cdot, w) = 0$ away from $w$, so commuting $\partial_y$ and $\Delta$, we see that $P(\cdot, x)$ is harmonic on $\CC_+$.

To prove the representation formula, we will need some estimates on the Green function's order of growth.
\begin{lemma}
\label{estimate on Green function}
    For every $w \in \CC_+$ there is a constant $C > 0$ such that for every $z \in \CC_+$ such that $|z|$ is large enough,
$$\frac{\Im z}{C(1 + |z|)^2} \leq |G(w, z)| \leq \frac{C \Im z}{(1 + |z|)^2}.$$
\end{lemma}
\begin{proof}
  Let $z = x + iy$. Let us Taylor expand $G(w, z)$ in $y$ at the origin, so $G(w, z) = \sum_j c_j(w, x) y^j$. Since $G = 0$ on $\RR$, $c_j = 0$. By definition of the Poisson kernel, $c_1 + P = 0$. By homogeneity,
\begin{align*}
  G(w, z) &= G\left(\frac{w}{|z|}, \frac{z}{|z|}\right) = \sum_{j=0}^\infty c_j\left(\frac{w}{|z|}, \frac{x}{|z|}\right) y^j |z|^{-j}
  \\&= -P\left(\frac{w}{|z|}, \frac{x}{|z|}\right) \frac{y}{|z|} + o\left(\frac{y}{|z|^2} \right) = -\Theta\left(P\left(\frac{w}{|z|}, 0\right)\right)
  \\&= \Theta \left(\frac{\Im z}{|1 + z|^2}\right)
\end{align*}
  where the implied constants are allowed to depend on $w$, and we have used Knuth's big-$\Theta$ notation. Indeed, if $|z|$ is large then $x/|z|$ is small, and so does not contribute meaningfully to the long-term behavior of $P$.
\end{proof}

We shall also need a representation formula for the unit disc $\DD$. We recall that the Cayley transform, which we will denote $z \mapsto z^\flat$ (with inverse $w \mapsto w^\sharp$), conformally transforms $\CC_+$ into $\DD$, and so all that we have proven about $\CC_+$ corresponds to a fact about $\DD$. The Cayley transform is given by
$$z^\flat = \frac{z - i}{z + i}.$$
Pushing forward the Poisson kernel along the Cayley transform, we arrive at the following definition.
\begin{definition}
The \dfn{Poisson kernel} for $\Delta$ on $\DD$ is defined by
$$P^\flat(z, \theta) = P(z^\sharp, (e^{i\theta})^\sharp).$$
\end{definition}
\begin{lemma}[Poisson representation formula]
Let $f \in C(\partial \DD)$ and let
$$F(z) = \int_0^{2\pi} P^\flat(z, \theta)f(e^{i\theta}) ~d\theta.$$
Then $F$ is the unique solution to the boundary-value problem for $\Delta$ with boundary condition $f$.
\end{lemma}
\begin{proof}
Since $P$ is harmonic, it follows that $P^\flat$ is harmonic as well, and
$$\int_0^{2\pi} P^\flat(z, \theta) ~d\theta = \int_{-\infty}^\infty P(z^\sharp, x) ~dx = 1.$$
So for any $r \in (0, 1)$ and any $\varepsilon > 0$, we have the estimate
\begin{align*}
  |F(re^{i\theta}) - f(e^{i\theta})| &\leq \int_0^{2\pi} P^\flat(re^{i\theta}, e^{i\eta})|f(e^{i\theta}) - f(e^{i\eta})| ~d\eta\\
    &\leq \sup_{B_\varepsilon} |f(e^{i\theta}) - f(e^{i\eta})| + \int_{B_\varepsilon^c} P^\flat(re^{i\theta}, e^{i\eta})|f(e^{i\theta}) - f(e^{i\eta})| ~d\eta\\
    &\leq \sup_{B_\varepsilon} |f(e^{i\theta}) - f(e^{i\eta})| + \sup_{B_\varepsilon^c} P^\flat(re^{i\theta}, e^{i\eta}) \int_{B_\varepsilon^c} |f(e^{i\theta}) - f(e^{i\eta})| ~d\eta
\end{align*}
where $B_\varepsilon$ is a interval in $[0, 2\pi]$ modulo $2\pi$ of radius $\varepsilon$ centered on $e^{i\theta}$. Since $f$ is continuous,
$$\lim_{\varepsilon \to 0}\sup_{B_\varepsilon} |f(e^{i\theta}) - f(e^{i\eta})| = 0.$$
On the other hand, since $P(z^\sharp, x) \to 0$ uniformly in $x$ as $z^\sharp \to \infty$, $P^\flat(z, x^\flat) \to 0$ uniformly in $x^\flat$ as $|z| \to 1$, provided that we are away from the singularity $z = x^\flat$. Thus
$$\lim_{r \to 1} F(re^{i\theta}) = f(e^{i\theta})$$
uniformly in $\theta$. So $F|_{\partial \DD} = f$. By the maximum principle, $F$ is unique.
\end{proof}

Now we are ready to prove the Riesz representation formula for $\CC_+$.
\begin{theorem}[Riesz representation formula]
    \index{Riesz representation formula}
    Let $u$ be an imaginary-sublinear subharmonic function on $\CC_+$, let $\gamma$ be as in Lemma \ref{imaginary sublinear limit}, and fix any $w \in \CC_+$. Let $\mu = \Delta v$; then
    \begin{equation}\label{estimate on mu}\int_{\CC_+} \frac{\Im z \mu(z) ~dz}{(1 + |z|)^2} < \infty.\end{equation}
    Moreover $v(\cdot + iy)$ converges to a distribution $\sigma$ on $\RR$ as $y \to 0$ such that, in the sense of distributions,
\begin{equation}\label{estimate on sigma}\int_{-\infty}^\infty \frac{|\sigma(x) ~dx|}{(1 + |x|)^2} < \infty,\end{equation}
    and
\begin{equation}\label{riesz formula}u(z) = \int_{\CC_+} G(z, w) \mu(w) ~dw + \int_{-\infty}^\infty P(z, x) \sigma(x) ~dx + \gamma \Im z.\end{equation}
\end{theorem}
\begin{proof}
    Let us replace $u(z)$ with $u(z) - C - \gamma \Im z$, where $C$ is the constant appearing the definition of an imaginary-sublinear function. Then $u(z) \leq D + \Im z$, and if we take an optimal choice of $D$, then $D \leq 0$, so $u \leq 0$. After completing the proof, we can simply add a $\gamma \Im z$ back in, the $C$ having already been absorbed into the boundary term $\sigma$.

    We first prove (\ref{estimate on mu}). Recall that we have assumed that there is a $z \in \CC_+$ so that $u(z) > -\infty$. Moreover, for any $x \in \RR$, $G(z, x) = 0$ By Lemma \ref{estimate on Green function}, for some constant $B > 0$,
\begin{align*}\int_{\CC_+} \frac{\Im w}{(1 + |w|)^2} \mu(w) ~dw &\leq -B\int_{\CC_+} G(z, w) \Delta u(w) ~dw
    \\&= B \int_{\CC_+} \nabla G(z, w) \nabla u(w) ~dw + B\int_{-\infty}^\infty G(z, x) \nabla u(x) ~dx\\
    &= -B \int_{\CC_+} \Delta G(z, w) u(w) ~dw \\&= -B\int_{\CC_+} \delta_z(w) u(w) ~dw = -Bu(w) < \infty.
  \end{align*}

    Now we decompose $u$. Fix an increasing chain $K_j$ of compact sets which cover $\CC_+$, and let $\chi_j \in C^\infty_{comp}(\CC_+)$ be an increasing chain of cutoff functions which are identically $1$ on $K_j$. Then let
    $$v_j(z) = u(z) - \int_{\CC_+} G(z, w) \chi_j(w)\mu(w) ~dw.$$
\begin{lemma}
    The functions $v_j$ are subharmonic on $\CC_+$, harmonic on $K_j$, and $\leq 0$.
\end{lemma}
\begin{proof}[Proof of lemma]
    We compute
\begin{align*}
    \Delta v_j(z) &= \mu(z) - \int_{\CC_+} \Delta_z(E(z - w) - E(z - \overline w)) \chi_j(w) \mu(w) ~dw \\&= \mu(z) - \int_{\CC_+} (\delta_w(z) - \delta_{\overline w}(z) \chi_j(w) \mu(w) ~dw \\&= \mu(z)(1 - \chi_j(z)),
\end{align*}
    the $\delta_{\overline w}$ term vanishing because $z \neq \overline w$, since $\overline w \notin \CC_+$. Since $\chi_j$ is a cutoff, $\chi_j \leq 1$, so $\Delta v_j \geq 0$. On the other hand, if $z \in K_j$, then $1 - \chi_j(z) = 0$, so $\Delta v_j(z) = 0$. This proves the first two claims.

    Let $\varepsilon > 0$; we will prove that $v_j < \varepsilon$. Since $u$ is subharmonic, $\mu \geq 0$, and $\chi_j \geq 0$ while $G \leq 0$, so
    $$\int_{\CC_+} G(z, w) \chi_j(w) \mu(w) ~d\mu(w) \leq 0,$$ and we are only integrating over $w$ close to the compact set $K_j$. So we can view $w$ as essentially fixed compared to $z$, and apply the estimate $G(z, w) = O(\Im z|z|^{-2})$ to see that
    $$\lim_{z \to \infty} \int_{\CC_+} G(z, w) \chi_j(w) \mu(w) ~d\mu(w) \to 0.$$
    In particular, $\int_{\CC_+} G(z, w) \chi_j(w) \mu(w) ~d\mu(w) > -\varepsilon$ for $|z|$ large enough. Since $u \leq 0$, $v_j(z) < \varepsilon$ for $z$ large enough, hence for any $z$ by the maximum principle. Therefore $v_j < 0$.
\end{proof}
    The $v_j$ form an increasing sequence which is bounded above, so converge to a limit $u_1 \leq 0$. Moreover, $\Delta v_j \to 0$ pointwise, so $u_1$ is harmonic. Meanwhile, $\chi_j \to 1$ pointwise, so if we let
$$u_2(z) = \int_{\CC_+} G(z, w) \mu(w) ~dw,$$
    we arrive at the decomposition $u = u_1 + u_2$, $\Delta u_1 = 0$. We will view $u_1$ as the ``boundary part" of $u$ and $u_2$ as the ``subharmonic part" of $u$.

    We now show that the subharmonic part of $u$ does not contribute to its boundary value.
\begin{lemma}
    In the sense of distributions,
$$\lim_{y \to 0} u_2(\cdot + iy) = 0.$$
\end{lemma}
\begin{proof}[Proof of lemma]
    Let $\varphi \in C^\infty_{comp}(\RR)$ be a test function, $\ch \supp \varphi = [a, b]$. We must show that the limit
$$\lim_{y \to 0} \int_{-\infty}^\infty u_2(x + iy) \varphi(x) ~dx = \lim_{y \to 0} \int_{\CC_+} \mu(w) \int_{-\infty}^\infty G(x + iy, w) \varphi(x) ~dx ~dw = 0.$$
    Now $G(x + iy, w)$ vanishes for fixed $w,x$ as $y \to 0$, and $G(x + iy, w) \varphi(x) = O(\Im w|w|^{-2})$ at infinity by Lemma \ref{estimate on Green function}. If we can show that this bound is valid on compact sets as well, then we will have
$$\int_{\CC_+} \mu(y, w) H(y, w) ~dw \leq \int_{\CC_+} \frac{\Im z \mu(z) ~dz}{(1 + |z|)^2} < \infty$$
    by (\ref{estimate on mu}), whence
$$\lim_{y \to 0} \int_{-\infty}^\infty u_2(x + iy) \varphi(x) ~dx = \lim_{y \to 0} \int_{\CC_+} \mu(y, w) H(y, w) ~dw = 0$$
    by the dominated convergence theorem.

    We now define
$$F(w) = \int_{-\infty}^\infty E(w - x)\varphi(x) ~dx.$$
    Now $E$ is continuous away from $0$, but $\Im w > 0$, so the integrand is continuous. The integral is formally taken over $\RR$, but is actually being taken over $[a, b]$, so the integrand is integrable; hence $F$ is continuous. Since $\Delta E(w - x) = 0$ for $\Im w > 0$, $F$ is harmonic. But $F$ is a convolution, so if $P$ is any linear differential operator in $\Re w$, $PF = E * P\varphi$, which is continuous since $\varphi$ is smooth. Therefore $PF$ is locally bounded. In particular,
$$\partial_{\Im w}^2 F(w) = \Delta F(w) - \partial_{\Re w}^2 F(w) = -\partial_{\Re w}^2 F(w)$$
    which is locally bounded. So $PF$ is locally bounded for any linear differential operator whatsoever. In particular, $F \in C^{Lip}_{loc}(\CC_+)$. Because $F(w)$ is bounded for a fixed $\Im w$, since $F$ is continuous and $\Re a$ ranges over the compact set $[a, b]$, we have $F(w) = O(\Im w)$. That is,
$$\int_{-\infty}^\infty G(x + iy, w) \varphi(x) ~dx = F(\overline w + iy) - F(w + iy) = O(\Im w)$$
    for bounded $y$. In particular, the integral is bounded on any compact set in $w$. Therefore
$$\int_{-\infty}^\infty G(x + iy, w) \varphi(x) ~dx = O\left(\frac{|\Im w|}{(1 + |w|^2)}\right).$$
    By the remarks at the start of this proof, the lemma follows.
\end{proof}
    We now will construct a representation formula for the boundary part.
\begin{lemma}
  \label{approximate sigma rep}
    For any $\varepsilon > 0$ and $y > \varepsilon$,
$$u_1(x + iy) = \int_{-\infty}^\infty P(x + i(y-\varepsilon), t) u_1(t + i\varepsilon) ~dt.$$
\end{lemma}
\begin{proof}
    Let $\psi_j$ be an increasing sequence of cutoff functions on $\RR$ which are identically $1$ on $[-j, j]$. Then
$$w_j(z) = u_1(z + i\varepsilon) - \int_{-\infty}^\infty P(z, x)\psi_j(x)u_1(x + i\varepsilon) ~dx.$$
    Then for $z \in \CC_+$,
$$\Delta w_j(z) = -\Delta \int_{-\infty}^\infty P(z, x)\psi_j(x) u_1(x + i\varepsilon) ~dx = -\int_{-\infty}^\infty \Delta_z P(z, x) \psi_j(x) u_1(x + i\varepsilon) ~dx = 0$$
    since $P(\cdot, x)$ is harmonic on $\CC_+$. So the $w_j$ are harmonic on $\CC_+$. Since $P$ is a nascent Dirac mass, if $\Im z = 0$, then
$$w_j(z) = u_1(z + i\varepsilon) - \int_{-\infty}^\infty \delta_x(z) \psi_j(x) u_1(x + i\varepsilon) ~dx = u_1(z + i\varepsilon)(1 - \psi_j(x)) \leq 0$$\
    since $u_1 \leq 0$. On the other hand, $P(z, x)$ is small when $|z|$ is large; so
    \begin{align*}\limsup_{z \to \infty} w_j(z) &\leq -\liminf_{z \to \infty} \int_{-\infty}^\infty P(z, x) \psi_j(x) u_1(x + i\varepsilon) ~dx
      \\&= -\int_{-\infty}^\infty \liminf_{z \to \infty} P(z, x) \psi_j(x) u_1(x + i\varepsilon) ~dx = 0.\end{align*}
    Therefore $h_j|_{\partial \CC_+} \leq 0$, so by the maximum principle, $h_j \leq 0$.

    By the Schwarz reflection principle, we can extend $h_j$ by $h_j(\overline z) = -h(z)$, and the resulting function is harmonic on the strip $\{z \in \CC: |\Re z| < j\}$. Taking the limit as $j \to \infty$, we construct a harmonic function $h$ on all of $\CC$, such that if $z \in \CC_+$,
\begin{equation}\label{sigma rep with h}u_1(z + i\varepsilon) = h(z) + \int_{-\infty}^\infty P(z, x)u_1(x + i\varepsilon) ~dx.\end{equation}
    In particular, $h$ restricts to a harmonic function on the rescaled disc $R\DD$ for any $R > 0$, so by the Poisson representation formula, for any $z < R$,
$$h(z) = \int_0^{2\pi} P^\flat(z/R, \theta) h(Re^{i\theta}) ~d\theta = \int_0^\pi (P^\flat(z/R, \theta) - P^\flat(\overline z/R, \theta)) h(Re^{i\theta}) ~d\theta.$$
    We set $x^\flat = e^{i\theta}$ and use the fact that the Cayley transform and complex conjugation commute (and so $|(z/R)^\sharp - x| = |\overline{(z/R)^\sharp} - x|$) to see
\begin{align*}
  P^\flat(z/R, \theta) - P^\flat(\overline z/R, \theta) &= P((z/R)^\sharp, x) - P\left(\overline{(z/R)^\sharp}, x\right)\\
    &= \frac{\Im (z/R)^\sharp - \Im \overline{(z/R)^\sharp}}{2\pi |(z/R)^\sharp - x|^2}
    = \frac{\Im (z/R)^\sharp}{\pi|(z/R)^\sharp - x|^2}.
\end{align*}
    If $z = \alpha + i\beta$, then if $R$ is large and $|z|$ is fixed,
$$\Im (z/R)^\sharp = \frac{1 - |z|^2}{R^2((\alpha-R^{-1})^2 + \beta^2)} = \frac{1 - |z|^2}{R^2|z|^2 - 2\alpha R + 1} = O(R^{-2}).$$
    Meanwhile, $|(z/R)^\sharp - x|^2 = |x + i|^2 = O(1)$. Taking the limit as $R \to \infty$, we see that $h = 0$. Plugging $h$ back into (\ref{sigma rep with h}), we complete the proof.
\end{proof}
    Since $\gamma = 0$ and $u \leq 0$, for every $\delta > 0$ and every $x \in \RR$, we can choose $y > 0$ so large that $u(x + iy) > -\delta y$. Since $G$ vanishes at infinity, so does $u_2$, so if $y$ is chosen even larger still, we can arrange that $u_1(x + iy) > -\delta y$. By Lemma \ref{approximate sigma rep} with a change of variable in $y$,
$$-\int_{-\infty}^\infty \frac{u_1(x + i\varepsilon)}{|x + iy - t|^2} ~dt < \frac{\pi \delta y}{y - \varepsilon}.$$
    Taking $\varepsilon$ and $R$ large enough, we can find a constant $B > 0$ such that
$$\left(\int_{-\infty}^{-R} + \int_R^\infty\right) \frac{-u_1(x + i\varepsilon)}{x^2} ~dx < B \delta.$$
    This estimate is uniform in $\varepsilon$, and $u_1(\cdot + i\varepsilon)$ is bounded on $[-R, R]$ since it is harmonic; these bounds are also uniform in $\varepsilon$. Certainly any test function on $\RR$ decays faster than $x^2$, so it follows that $u_1(\cdot + i\varepsilon)$ is uniformly bounded in $\varepsilon$ as a functional acting on $C^\infty_{comp}(\RR)$. So by the Banach-Alaoglu theorem, the space of all such distributions $u_1(\cdot + i\varepsilon)$ is weakstar compact, and so there is a weak limit $\sigma$ as $\varepsilon \to 0$. Since $\sigma$ satisfies the same bounds as $u_1(\cdot + i\varepsilon)$, (\ref{estimate on sigma}) follows, and, passing to the limit in Lemma \ref{approximate sigma rep},
$$u_1(z) = \int_{-\infty}^\infty P(z, x) \sigma(x) ~dx.$$
    Plugging the representation formula for $u_1$ and $u_2$ back into the decomposition $u = u_1 + u_2$, we complete the proof.
\end{proof}

\begin{corollary}
    \label{gamma in mean}
    Let $u$ be an imaginary-sublinear subharmonic function on $\CC_+$, and let $\gamma$ be as in Lemma \ref{imaginary sublinear limit}. Then
$$\lim_{t \to \infty} \frac{u(t(x + iy))}{t} = \gamma y$$
    in $L^1_{loc}(\overline{\CC_+})$.
\end{corollary}
\begin{proof}
    Fix a compact set $K \subset \overline{\CC_+}$. If $u$ is a constant, then both sides of the claimed equation are $0$, so by linearity we may subtract the constant $C$ that appears in the definition of an imaginary-sublinear function from $u$, and so assume without loss of generality that $C = 0$. Then $u(t(x+iy)) \leq \gamma y$, so it suffices to prove that
$$\lim_{t \to \infty} \int_K \frac{u(tz)}{t} - \gamma \Im z ~dz = 0$$
    to show convergence in $L^1(K)$.

    By the Riesz representation formula,
$$\frac{u(tz)}{t} - \gamma \Im z = \int_{-\infty}^\infty P(z, x) \sigma(x) ~dx + \int_{\CC_+} G(z, w) \mu(w) ~dw.$$
    We now let
    $$f(t, x) = \frac{1}{t^2} \int_K P\left(t, \frac{x}{t}\right) ~dx.$$
    Since $t > 0$ and $P \geq 0$, it follows that $f \geq 0$. Moreover, by Lemma \ref{estimate on Green function}, there is a $B > 0$ which only depends on $K$ such that for any $x/t$ large enough,
    $$f(t, x) \leq \frac{1}{t^2} \sup_{x \in K} P\left(t, \frac{x}{t}\right) \leq \frac{B}{t^2(1 + |t|)^2}.$$
    Moreover, this estimate on $f$ is trivial if we bound $x/t$ from above; so we can take $B$ to be large enough that this estimate is valid for any $x/t \in \RR$. Similarly, we take
    $$g(t, w) = -\frac{1}{t} \int_K G(z, w) ~dw \leq \frac{D \Im z}{t^2(1+|z/t|)^2}.$$
    Thus $g \geq 0$, and
    $$\int_K \frac{u(tz)}{t} - \gamma \Im z ~dz = \int_{-\infty}^\infty f(t, x) \sigma(x) ~dx - \int_{\CC_+} g(t, w) \mu(w) ~dw.$$
    Since we have estimated $f$ and $g$, we can use (\ref{estimate on sigma}) and (\ref{estimate on mu}) to apply the dominated convergence theorem. Clearly the dominators of $f$ and $g$ converge to $0$ pointwise as $t \to \infty$, so the integrals against $\sigma$ and $\mu$ converge to $0$.
\end{proof}



\section{Proof of Titchmarsh's theorem}
Fix a distribution $\mu$ on $\RR$ with $\ch \supp \mu = [a, b]$. By the Paley-Weiner theorem, $\hat \mu$ is an entire function and $\log |\hat \mu|$ is an imaginary-sublinear subharmonic function. In fact, viewing $\log |\hat \mu|$ as a function on $\CC_+$ and taking $\gamma$ as in Lemma \ref{imaginary sublinear limit}, we have $\gamma = b$. On the other hand, viewing $\log |\hat \mu|$ as a function on $\CC_-$, we have $\gamma = a$. Let $Z$ be the multiset of zeroes of $\hat \mu$ with repetition.

Let $h(t) = at$ for $t < 0$, $h(t) = bt$ for $t > 0$. Then $h'$ is a rescaled Heaviside function, so $h'' = (b-a)\delta_0$. Using Corollary \ref{gamma in mean}, we have
$$\lim_{t \to \infty} \frac{\log |\hat \mu(t\zeta)|}{t} = h(\Im \zeta)$$
in $L^1_{loc}$. Taking the Laplacian of both sides, we see that
$$\lim_{t \to \infty} \frac{2\pi}{t} \sum_{z \in Z} \delta_{z/t}(w) = (b-a)\delta_0(\Im w)$$
in the sense of distributions.

Integrating both sides,
$$\int_{D(0, 1)} \lim_{t \to \infty} \frac{2\pi}{t} \sum_{z \in Z} \delta_{z/t} = \int_{D(0, 1)} (b-a)\delta_0(\Im w) ~dw = \int_{-1}^1 b - a ~dw = 2(b - a).$$
Rewriting the left-hand side, we have
$$\lim_{R \to \infty} \frac{2\pi }{R} \int_{D(0, R)} \sum_{z \in Z} \delta_z = \lim_{R \to \infty} \frac{2\pi N(R)}{R}.$$
Dividing both sides by $2\pi$, we see that
$$\lim_{R \to \infty} \frac{N(R)}{R} = \frac{b - a}{\pi}.$$
This proves Titchmarsh's theorem.



\printbibliography



\printindex

\end{document}
