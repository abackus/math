\documentclass[reqno,11pt]{amsart}
\usepackage[letterpaper, margin=1in]{geometry}
\RequirePackage{amsmath,amssymb,amsthm,graphicx,mathrsfs,url,slashed,subcaption}
\RequirePackage[usenames,dvipsnames]{xcolor}
\RequirePackage[colorlinks=true,linkcolor=Red,citecolor=Green]{hyperref}
\RequirePackage{amsxtra}
\usepackage{cancel}
\usepackage{tikz-cd}
%\usepackage[T1]{fontenc}

% \setlength{\textheight}{9.3in} \setlength{\oddsidemargin}{-0.25in}
% \setlength{\evensidemargin}{-0.25in} \setlength{\textwidth}{7in}
% \setlength{\topmargin}{-0.25in} \setlength{\headheight}{0.18in}
% \setlength{\marginparwidth}{1.0in}
% \setlength{\abovedisplayskip}{0.2in}
% \setlength{\belowdisplayskip}{0.2in}
% \setlength{\parskip}{0.05in}
%\renewcommand{\baselinestretch}{1.05}

\title{Some interesting open problems}
\author{Aidan Backus}
\address{Department of Mathematics, Brown University}
\email{aidan\_backus@brown.edu}
\date{\today}

\newcommand{\NN}{\mathbf{N}}
\newcommand{\ZZ}{\mathbf{Z}}
\newcommand{\QQ}{\mathbf{Q}}
\newcommand{\RR}{\mathbf{R}}
\newcommand{\CC}{\mathbf{C}}
\newcommand{\DD}{\mathbf{D}}
\newcommand{\PP}{\mathbf P}
\newcommand{\MM}{\mathbf M}
\newcommand{\II}{\mathbf I}
\newcommand{\Hyp}{\mathbf H}
\newcommand{\Sph}{\mathbf S}
\newcommand{\Group}{\mathbf G}
\newcommand{\GL}{\mathbf{GL}}
\newcommand{\Orth}{\mathbf{O}}
\newcommand{\SpOrth}{\mathbf{SO}}
\newcommand{\Ball}{\mathbf{B}}

\newcommand*\dif{\mathop{}\!\mathrm{d}}

\DeclareMathOperator{\card}{card}
\DeclareMathOperator{\dist}{dist}
\DeclareMathOperator{\id}{id}
\DeclareMathOperator{\supp}{supp}
\DeclareMathOperator{\Teich}{Teich}
\DeclareMathOperator{\tr}{tr}

\newcommand{\Leaves}{\mathscr L}
\newcommand{\Lagrange}{\mathcal L}
\newcommand{\Hypspace}{\mathscr H}

\newcommand{\Chain}{\underline C}

\newcommand{\Two}{\mathrm{I\!I}}

\newcommand{\normal}{\mathbf n}
\newcommand{\radial}{\mathbf r}
\newcommand{\evect}{\mathbf e}
\newcommand{\vol}{\mathrm{vol}}

\newcommand{\diam}{\mathrm{diam}}
\newcommand{\Ell}{\mathrm{Ell}}
\newcommand{\inj}{\mathrm{inj}}
\newcommand{\Lip}{\mathrm{Lip}}
\newcommand{\Riem}{\mathrm{Riem}}

\newcommand{\Min}{\mathrm{Min}}
\newcommand{\Max}{\mathrm{Max}}

\newcommand{\dfn}[1]{\emph{#1}\index{#1}}

\renewcommand{\Re}{\operatorname{Re}}
\renewcommand{\Im}{\operatorname{Im}}

\newcommand{\loc}{\mathrm{loc}}
\newcommand{\cpt}{\mathrm{cpt}}

\def\Japan#1{\left \langle #1 \right \rangle}

\newtheorem{theorem}{Theorem}[section]
\newtheorem{badtheorem}[theorem]{``Theorem"}
\newtheorem{prop}[theorem]{Proposition}
\newtheorem{lemma}[theorem]{Lemma}
\newtheorem{sublemma}[theorem]{Sublemma}
\newtheorem{proposition}[theorem]{Proposition}
\newtheorem{corollary}[theorem]{Corollary}
\newtheorem{conjecture}[theorem]{Conjecture}
\newtheorem{axiom}[theorem]{Axiom}
\newtheorem{assumption}[theorem]{Assumption}

\newtheorem{mainthm}{Theorem}
\renewcommand{\themainthm}{\Alph{mainthm}}

\newtheorem{claim}{Claim}[theorem]
\renewcommand{\theclaim}{\thetheorem\Alph{claim}}
% \newtheorem*{claim}{Claim}

\theoremstyle{definition}
\newtheorem{definition}[theorem]{Definition}
\newtheorem{remark}[theorem]{Remark}
\newtheorem{example}[theorem]{Example}
\newtheorem{notation}[theorem]{Notation}

\newtheorem{exercise}[theorem]{Discussion topic}
\newtheorem{homework}[theorem]{Homework}
\newtheorem{problem}[theorem]{Problem}

\makeatletter
\newcommand{\proofpart}[2]{%
  \par
  \addvspace{\medskipamount}%
  \noindent\emph{Part #1: #2.}
}
\makeatother



\numberwithin{equation}{section}


% Mean
\def\Xint#1{\mathchoice
{\XXint\displaystyle\textstyle{#1}}%
{\XXint\textstyle\scriptstyle{#1}}%
{\XXint\scriptstyle\scriptscriptstyle{#1}}%
{\XXint\scriptscriptstyle\scriptscriptstyle{#1}}%
\!\int}
\def\XXint#1#2#3{{\setbox0=\hbox{$#1{#2#3}{\int}$ }
\vcenter{\hbox{$#2#3$ }}\kern-.6\wd0}}
\def\ddashint{\Xint=}
\def\dashint{\Xint-}

\usepackage[backend=bibtex,style=alphabetic,giveninits=true]{biblatex}
\renewcommand*{\bibfont}{\normalfont\footnotesize}
\addbibresource{problems.bib}
\renewbibmacro{in:}{}
\DeclareFieldFormat{pages}{#1}

\newcommand\todo[1]{\textcolor{red}{TODO: #1}}


\begin{document}

\maketitle

\section{\texorpdfstring{$L^\infty$}{L-infinity} calculus of variations}
\begin{problem}
Show that spectral $\infty$-harmonic maps from surfaces are AML on simply connected sets.
\end{problem}

\begin{problem}
Suppose that $q \in (1, 2]$ and $\beta_q$ is a closed $d - 1$-form such that 
$$\dif^*(|\beta_q|^{q - 2} \beta_q) = 0.$$
Show that, if $B_2$ is contractible, then we have the \dfn{gauge Caccioppoli inequality}
$$\int_{B_1} |\beta_q|^q \lesssim \int_{B_2 \setminus B_1} |\beta_q|^q.$$
If we do, conclude that spectral $\infty$-harmonic maps (possibly not from surfaces) are AML on contractible sets.
\end{problem}

\begin{problem}
Show that every totally nonconformal spectral $\infty$-harmonic map $u$ from a surface is an $\infty$-harmonic fan.
We know that if $u$ is smooth, then this holds \cite[Proposition 3.5]{Sheffield12}.
\end{problem}

\begin{problem}
Is the stretch set of a totally nonconformal spectral $\infty$-harmonic map from a surface a geodesic lamination?
\end{problem}

\begin{problem}
Formulate a vector-valued tug of war process. Show that spectral $\infty$-harmonic maps are value functions for the vectorial tug of war process.
The scalar-valued case is \cite{Peres_2008}.
\end{problem}

\begin{problem}
A \dfn{contact viscosity solution} is a generalization of viscosity solutions for degenerations of elliptic systems \cite{katzourakis2022vectorvalued}.
Interpret spectral $\infty$-harmonic maps as contact viscosity solutions of the spectral $\infty$-Laplacian.
\end{problem}

\begin{problem}
Consider the scalar $\infty$-elliptic equation 
$$A^{ijk\ell} \partial_i \partial_j u \partial_k u \partial_\ell u = F(u)$$
where $F(u)$ is a first-order nonlinearity and $A$ is an elliptic tensor (in the sense of Legendre--Hadamard?)
This is the form that the $\infty$-Laplacian takes on a Riemannian manifold.
Show that the Evans-Savin theorem \cite{Evans08} holds: if $u$ is defined on $\Ball^2$, then $u \in C^{1 + \alpha}$ for some $\alpha > 0$.
\end{problem}

\begin{problem}
Let $u$ be the dual $1$-harmonic function to a tight $2$-form $F$ on a closed hyperbolic $3$-fold.
Show that $F, u$ are (non)unique in their cohomology classes.
\end{problem}

\begin{problem}[Karen Uhlenbeck]
Study solutions to the eikonal equation on forms 
$$|\dif A| = 1$$
where $A$ is a $1$-form on a domain in $\Hyp^3$.
Show that if $\dif A$ calibrates a foliation $\lambda$, then $\lambda$ is totally geodesic.
Conclude that if $M$ is a closed hyperbolic $3$-fold and $f: M \to \Sph^1$ is a fibration, then there does not exist a realization of $f$ as a fiber bundle whose fibers are minimal surfaces.
\end{problem}

\section{\texorpdfstring{$L^1$}{L-1} calculus of variations}
\begin{problem}
Let $\lambda$ be a measured oriented lamination.
Show that $\lambda$ is ``weak homotopy equivalent'' to a minimal such lamination in some sense.
For example, it would suffice to show that if $u$ is a $BV$ function with leaf lamination $\lambda$, then the level set flow acts on the leaves of $\lambda$ by mean curvature flow without singularities.
\end{problem}

\begin{problem}
Prove well-posedness or blowup of the total variation flow 
$$\partial_t u = \dif^* \frac{\dif u}{|\dif u|}$$
and use this to solve the denoising equation (possibly in the low-noise limit $\lambda \to 0$), in the case that $u$ is a vector-valued map and $|\cdot|$ is a matrix norm.
See \cite{Novaga04} for the definitions here.
\end{problem}

\begin{problem}
Give examples of maps of tracial least gradient.
In particular, classify the rotational symmetric examples and the homogeneous examples, $\RR^2 \to \RR^2$.
\end{problem}

\begin{problem}
What are the maps of tracial least gradient with image a circle in $\RR^2$?
\end{problem}

\begin{problem}
Construct tangent maps of maps of $sv^p$-least gradient, when $p \in [1, 2]$.
\end{problem}

\begin{problem}
Show that direct sums of maps of $sv^p$-least gradient have $sv^p$-least gradient iff $p = 1$.
\end{problem}

\section{Computational geometry}
\begin{problem}
Give a numerical FEM scheme for computing minimal surfaces using the equivalence of $1$-harmonic functions and minimal laminations, and Loisel's algorithm \cite{Loisel20}.
Rigorously justify the convergence in Gromov-Prokhorov (and possibly stronger) sense, and runtime.
\end{problem}

\begin{problem}
Construct minimal surfaces which cannot be realized, or can only be realized by specialized initial configurations, as limits of the level set flow, in violation of the theory of \cite{CHOPP199377}.
\end{problem}

\begin{problem}[{\cite[Conjecture 5.3]{naff2022prescribed}}]
By randomly generating highly asymmetric examples, construct a minimal hypersurface $\Sigma$ in the hemisphere $\Sph^n_+$ with $\partial \Sigma \subset \partial \Sph^n_+ = \Sph^{n - 1}$ but
$$|\Sigma| < \frac{|\Sph^{d - 1}|}{2}.$$
\end{problem}

\begin{problem}
Show that the continuous max flow min cut theorem follows from the discrete version using the finite element method.
\end{problem}

\section{Microlocal and harmonic analysis}
\begin{problem}[UW Madison summer school, 2023]
For which H\"older exponents $p, q$ and dimensions $\delta, \delta'$ does the fractal uncertainty principle hold?
\end{problem}

In the next few problems, let $\beta^\sharp(X, Y)$ denote the \dfn{sharp FUP exponent} of two fractals $X, Y$, thus
$$\beta^\sharp(X, Y) := \sup \{\beta: \|1_{X_h} \mathscr F_h 1_{Y_h}\|_{L^2 \to L^2} \lesssim_\beta h^\beta\}.$$

\begin{problem}[{\cite[Conjecture 4.4]{Dyatlov_2019}}]
For each $\delta \in (0, 1)$, construct arithmetic Cantor sets $X_j$ of Hausdorff dimension $\delta$ such that $\beta^\sharp(X_j, X_j) \to 0$.
\end{problem}

\begin{problem}[{\cite[Conjecture 4.5]{Dyatlov_2019}}]
Show that if $X$ is an arithmetic Cantor set of Hausdorff dimension $\delta \in (0, 1)$, and $\alpha \in \RR_+$ is a generic scalar, then $\beta^\sharp(X, \alpha X) \geq \beta^\sharp_\delta$ where $\beta^\sharp_\delta > 0$ only depends on $\delta$.
\end{problem}

\begin{problem}[{\cite[\S4]{eswarathasan2021fractal}}]
Draw a sequence of alphabets $\mathcal A_j \subset \{1, \dots, M\}^d$ of length $k$ (uniformly?) at random, and consider the \dfn{random Cantor set} that corresponds to removing the cubes specified by $\mathcal A_j$ at stage $j$.
Show that if $X, Y$ are random Cantor sets, then with overwhelming probability, $\beta^\sharp(X, Y)$ significantly improves over deterministic estimates.
\end{problem}

\begin{problem}
Construct arithmetic Cantor sets $X_j$ of Hausdorff dimension $\delta_j \in (1 - 1/j, 1)$, such that the Ahlfors-David regularity constant of $X_j$ is bounded, but $\beta^\sharp(X_j, X_j)$ is exponentially small as $j \to \infty$.
\end{problem}

\begin{problem}[{\cite{Solomon11}}]
Let $\alpha$ be a differential form on $\RR^d$ with decay $O(|x|^{-d-\varepsilon})$ (realistically, the correct hypothesis here is $\alpha \in L^1$).
Prove the inversion formula for the X-ray transform of $\alpha$.
\end{problem}


\section{Logic}
\begin{problem}
What low-hanging fruit in the study of the Kakeya and Falconer problems can be picked using the point-to-set principle \cite{Lutz18}?
\end{problem}

\begin{problem}
One can build a logic gate out of a system of ODE \cite{Tao16}.
Show that for each $n \in \NN$, one can build a system of ODE which can simulate a register machine (eg, one which can increment a register, decrement a register, or jump if a register is $0$) with $n$ registers.
Conclude that one can build a PDE $Pu = 0$ which can simulate a Turing complete register machine, so the dynamics of $P$ are independent of ZFC.
\end{problem}



\printbibliography

\end{document}
