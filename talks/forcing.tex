% --------------------------------------------------------------
% This is all preamble stuff that you don't have to worry about.
% Head down to where it says "Start here"
% --------------------------------------------------------------

\documentclass[10pt]{beamer}

% \usepackage[margin=1in]{geometry}
\usepackage{amsmath,amsthm,amssymb,mathrsfs}
\usepackage{tikz}
\usepackage{wrapfig}
\usepackage{cjhebrew}
% \usepackage[shortlabels]{enumerate}

\usepackage[utf8]{inputenc}
\usepackage[english]{babel}

\usetikzlibrary{calc}
\usetikzlibrary{arrows.meta}
\usetikzlibrary{arrows}

\title{The independence of the continuum hypothesis}
\author{Aidan Backus}

\newcommand{\NN}{\mathbb{N}}
\newcommand{\ZZ}{\mathbb{Z}}
\newcommand{\QQ}{\mathbb{Q}}
\newcommand{\RR}{\mathbb{R}}
\newcommand{\CC}{\mathbb{C}}


\newcommand*\dif{\mathop{}\!\mathrm{d}}
\DeclareMathOperator*{\argmin}{argmin}
\DeclareMathOperator{\dist}{dist}
\DeclareMathOperator{\supp}{supp}
\DeclareMathOperator{\Var}{Var}
\DeclareMathOperator{\Exc}{Exc}
\DeclareMathOperator*{\Expect}{\mathbf E}
\newcommand{\normal}{\vec n}
\newcommand{\norm}[1]{\left\lVert#1\right\rVert}

\newcommand{\ZFC}{\mathrm{ZFC}}

\newcommand{\Spec}{\operatorname{Spec}}

\renewcommand{\Re}{\operatorname{Re}}
\renewcommand{\Im}{\operatorname{Im}}

\newtheorem{theoremconjecture}{Theorem-Conjecture}
\newtheorem{conjecture}{Conjecture}
\newtheorem{proposition}{Proposition}
\newtheorem{question}{Question}

\usepackage[backend=bibtex,style=alphabetic,maxcitenames=50,maxnames=50]{biblatex}
\addbibresource{forcing.bib}
\renewbibmacro{in:}{}
\DeclareFieldFormat{pages}{#1}

\usetheme{AnnArbor}
\usecolortheme{dove}

% \setbeamertemplate{itemize item}{\usebeamerfont*{itemize item}\raise1.25pt\hbox{\donotcoloroutermaths$\bullet$}}
% \setbeamertemplate{itemize subitem}[triangle]
% \setbeamertemplate{itemize subsubitem}[square]

\begin{document}
% --------------------------------------------------------------
%                         Start here
% --------------------------------------------------------------
\begin{frame}
    \titlepage
\end{frame}

\begin{frame}{The axioms of set theory}
\begin{definition}
$\ZFC$ (short for \emph{Zermelo-Frankael axioms, and the axiom of choice}) is the set of axioms for mathematics expressed in the first-order language of set theory.
Translated to plain English:
\begin{itemize}
\item Extensionality: sets are determined by their elements.
\item Existence of $\emptyset$.
\item Existence of an infinite set.
\item Existence of unordered pairs.
\item Existence of unions.
\item Existence of power sets.
\item Choice: surjective functions have right inverses.
\item Replacement scheme: definable class-functions have images.
\item Foundation: existence of the rank function.
\end{itemize}
\end{definition}
\end{frame}

\end{document}
