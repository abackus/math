% --------------------------------------------------------------
% This is all preamble stuff that you don't have to worry about.
% Head down to where it says "Start here"
% --------------------------------------------------------------

\documentclass[10pt]{beamer}

% \usepackage[margin=1in]{geometry}
\usepackage{amsmath,amsthm,amssymb}
\usepackage{enumitem}
\usepackage{tikz}
\usepackage{wrapfig}
\usepackage{cjhebrew}
% \usepackage[shortlabels]{enumerate}

\usepackage[utf8]{inputenc}
\usepackage[english]{babel}

\usetikzlibrary{calc}
\usetikzlibrary{arrows.meta}
\usetikzlibrary{arrows}

\newcommand{\NN}{\mathbb{N}}
\newcommand{\ZZ}{\mathbb{Z}}
\newcommand{\QQ}{\mathbb{Q}}
\newcommand{\RR}{\mathbb{R}}
\newcommand{\CC}{\mathbb{C}}

\DeclareMathOperator*{\argmin}{argmin}
\DeclareMathOperator{\supp}{supp}
\DeclareMathOperator{\Exc}{Exc}
\newcommand{\normal}{\vec n}
\newcommand{\norm}[1]{\left\lVert#1\right\rVert}

\newcommand{\Spec}{\operatorname{Spec}}

\renewcommand{\Re}{\operatorname{Re}}
\renewcommand{\Im}{\operatorname{Im}}

\newtheorem{theoremconjecture}{Theorem-Conjecture}
\newtheorem{conjecture}{Conjecture}
\newtheorem{question}{Question}

\usepackage[backend=bibtex,style=alphabetic,maxcitenames=50,maxnames=50]{biblatex}
\addbibresource{ultrapowerful.bib}
\renewbibmacro{in:}{}
\DeclareFieldFormat{pages}{#1}

\usetheme{AnnArbor}
\usecolortheme{dove}

\begin{document}
\title{Functions of least gradient and zero-mean-curvature laminations}
\author{Catuse from the Analyst Lair}
% --------------------------------------------------------------
%                         Start here
% --------------------------------------------------------------
\begin{frame}
    \titlepage
\end{frame}

\begin{frame}{The maximum principle for harmonic functions}
\begin{theorem}[maximum principle]
Assume $u: M \to \RR$ is nonconstant and $\Delta u := d^*d u = 0$.
Then $u$ has no maximum on (the interior of) $M$.
\end{theorem}

\pause

\begin{corollary}
If $u: M \to \RR$ is nonconstant, $\Delta u = 0$, and a level set $\{u = y\}$ is a cycle, then $\{u = y\}$ is not a boundary (and in particular $M$ has nontrivial homology).
\end{corollary}

\pause

Proof: If $[\{u = y\}] = 0$, then $\{u = y\}$ is a boundary. Suppose without loss of generality that one of the components of the cell bounded by $\{u = y\}$ is $\{u > y\}$. But then $u = y$ on $\{u > y\}$ by the maximum principle, a contradiction.

\pause

\begin{example}
If $u(x) = -\log |x|$ on $\RR^2 \setminus \{0\}$, then the level sets of $u$ are cycles $\{|x| = r\}$ which are not boundaries.
\end{example}

\end{frame}

\begin{frame}{Abstract maximum principles}
G\'orny '18: A maximum principle gives information about the geometry and topology of level sets of the solutions to an elliptic PDE.

\pause

Let's try to come up with a particularly strong maximum principle in this sense. It should assert that the solutions to some elliptic PDE have zero mean curvature, that is, are locally area-minimizing.
(So, if $M$ is flat then level cycles are not boundaries; if $d = 2$ then the level sets are geodesics.)

\pause

\begin{definition}
The $1$-Laplacian is the operator $\Delta_1 u := d^*(du/|du|)$.
\end{definition}

\pause

Suppose $u$ is a $C^2$ submersion and $u(x) = y$.
Then $du(x)$ is conormal to the level set $\{u = y\}$, so $\Delta_1 u(x)$ is the divergence of the normal vector, thus the mean curvature, of $\{u = y\}$.
So $\Delta_1 u = 0$ implies that $\{u = y\}$ has zero mean curvature.
\end{frame}

\begin{frame}{Laminations}
\begin{definition}
A lamination $\lambda$ in $M$ is a closed set $|\lambda|$ equipped with charts which identify $|\lambda|$ with $K \times \RR^{d - 1}$ for some compact set $K \subset \RR$.
The sets that are locally identified with $\{k\} \times \RR^{d - 1}$, $k \in K$, are called leaves.
\end{definition}

\pause

\begin{definition}
A zero-mean-curvature lamination is a lamination whose leaves are all $C^2$ hypersurfaces with zero mean curvature.
If $d = 2$ we also call it a geodesic lamination.
\end{definition}

\pause

\begin{theorem}[maximum principle for $\Delta_1$]
If $\Delta_1 u = 0$ and $u$ is a $C^2$ submersion then the level sets $\{u = y\}$ are the leaves of a zero-mean-curvature lamination $\lambda$ with $|\lambda| = M$.
\end{theorem}
\end{frame}

\begin{frame}{Weak solutions of $1$-Laplacian}
\begin{example}
Let $\lambda$ be a geodesic lamination in a closed Riemann surface $M$ of genus $\geq 2$. Then $|\lambda|$ has measure zero. \pause
So there are no $C^2$ submersions $u$ on $M$ with $\Delta_1 u = 0$.
\end{example}

\pause

On the other hand, $\Delta_1 u = d^*(du/|du|)$ makes no sense near the critical points of $u$.
So we need a notion of weak solutions for $\Delta_1$.

\pause

\begin{definition}
A function of least gradient is a minimizer in $BV$ of $\int |du|dV$. \pause

A weak solution of $\Delta_1 u = 0$ is $u \in BV$ such that there exists a $L^\infty(|du|dV)$ vector field $X$ with $(du, X) = |du|$ and zero divergence.
\end{definition}

\pause

\begin{theorem}[Maz\'on, Rossi, and de Le\'on, '14]
$u \in BV$ is a function of least gradient iff it is a weak solution of $\Delta_1 u = 0$.
\end{theorem}
\end{frame}

\begin{frame}{The maximum principle for weak solutions}
\begin{theorem}[maximum principle]
Let $2 \leq d \leq 7$ and assume ???. \pause

If $u$ is a function of least gradient, then the level sets $\partial \{u > y\}$ are smooth and form a zero-mean-curvature lamination. \pause

Conversely, if $\lambda$ is a zero-mean-curvature lamination, then there is a function of least gradient whose level sets are the leaves of $\lambda$.
\end{theorem} \pause

History of Assumption ???: \pause
\begin{itemize}
\item $M$ open in $\RR^d$: essentially Miranda '66, codified by G\'orny '18\pause
\item $M$ a closed Riemann surface of genus $\geq 2$ and $u$ the conjugate harmonic to a solution of $\Delta_\infty v := u^{;\mu\nu} u_{;\mu} u_{;\nu} = 0$: Daskalopolous and Uhlenbeck '20\pause
\item $M$ constant sectional curvature: -- '22\pause
\item Cotton ($d \leq 3$) or Weyl ($d \geq 4$) tensor of $M$ vanishes: You?? 'XX\pause
\item No assumption: You?? 'XX 
\end{itemize}
\end{frame}

\begin{frame}{Corollaries of the maximum principle}
\begin{corollary}[Daskalopolous and Uhlenbeck '21]
Let $M$ be a closed Riemann surface of genus $\geq 2$ and $\Delta_\infty v = 0$. Then $\{|dv| = ||dv||_{L^\infty}\}$ contains a geodesic lamination.
\end{corollary}

\pause

Proof sketch: Let $u, v$ be conjugate harmonics. Computation shows that $du = 0$ away from $\{|dv| = ||dv||_{L^\infty}\}$.\pause

\begin{corollary}[G\'orny '18]
If $u$ has least gradient, then $u = u_c + u_j$ where $u_c$ is continuous with least gradient and $u_j$ is the jump part with least gradient.
\end{corollary}\pause
    
In particular, $u$ doesn't look like $z^{1/2}$ at the origin of $\CC$.
\end{frame}

\begin{frame}{Sets of least perimeter}
\begin{definition}
$U$ has least perimeter if $1_U$ has least gradient.
\end{definition}\pause

\begin{theorem}[Bombieri, de Giorgi, and Giusti '69]
    If $u$ has least gradient then $\partial \{u > y\}$ has least perimeter.
\end{theorem}\pause

Proof sketch for BdGG: just use the coarea formula 
$$\int_E |du|dV = \int_{-\infty}^\infty |\partial \{u > y\} \cap E| dy.$$
\pause

By BdGG: want to show that $U$ sets of least perimeter are bounded by smooth sets. We'll do this for $M = \mathbb H^d$ for simplicity.
\end{frame}

\begin{frame}{The gauge group}
    The metric is $g_{\mu\nu} = (1 - |x|^2/4)^{-2} \delta_{\mu\nu}$ centered on $O = (0, \dots, 0)$.
To study the behavior of $\partial U$ near $P$ instead of $O$, we use an isometry $\Phi^P: M \to M$ such that $\Phi^P(O) = P$.\pause

Let $\partial_\mu^P := \Phi^P_* \partial_\mu$ and $x_P^\mu := x^\mu \circ \Phi^P$.\pause

This construction is well-defined up to a choice of ``gauge transformation'', ie a section $\chi$ of $SO(TM) \to M$, which transforms the coordinates by $\Phi^P \mapsto \chi^P \circ \Phi^P$.\pause

Anything we define from now on should be gauge-invariant, ie, if we rotate it around $P$ it remains invariant.
\end{frame}

\begin{frame}{The approximate conormal $1$-form}
\begin{theorem}[monotonicity formula; essentially Miranda '66]
    There exists $A \geq 0$ depending only on $M$ such that for $P \in M$, $r > 0$ small enough depending on $M$, and $u$ a function of least gradient,
    $$\frac{d}{dr} e^{Ar^2} r^{1 - d} \int_{B(P, r)} |du|dV \geq 0.$$
\end{theorem}\pause

From the monotonicity formula we know that $|\partial U \cap B(P, r)| \sim r^{d - 1}$, so it's reasonable to define the approximate conormal $1$-form 
$$n(P, r) := |\partial U \cap B(P, r)|^{-1} \left[\int_{B(P, r)} \partial_\mu^P 1_U dV\right] dx^\mu_P(P) \in T'_P M.$$\pause
It converges $|du|dV$-almost everywhere as $r \to 0$ to the conormal $1$-form $n(P)$ and $P \mapsto n(P, r)$ is continuous.

\pause Goal: Show that $r \mapsto n(P, r)$ is locally uniformly Cauchy, so $n(P)$ is continuous in $P$, so $\partial U$ is $C^1$ and has zero mean curvature, hence is smooth by bootstrapping!!
\end{frame}

\begin{frame}{The excess}
\begin{definition}
The excess of set $U$ with locally finite perimeter, in a Borel set $A$ at $P \in A$, is 
$$\gamma_A(U, P) := |\partial U \cap A| - \left|\left[\int_A \partial^P_\mu 1_U dV\right] dx_P^\mu(P)\right|.$$
We write $\gamma_r(U, P) := \gamma_{B(P, r)}(U, P)$.
\end{definition}\pause

The idea is that if the conormal is ``very close to constant'' in $A$, as measured using the coordinates based at $P$, then the second term in $\gamma_A(U, P)$ is basically
$$\int_A |d1_U|dV = |\partial U \cap A|,$$
thus $\gamma_A(U, P)$ is basically $0$. But if the conormal oscillates wildly it experiences cancellation and $\gamma_A(U, P)$ is huge.\pause

From the definitions and a long computation:
$$|n_U(P, r) - n_U(P, s)|^2 \lesssim r^{1 - d} \gamma_r(U, P) + r^2.$$
\end{frame}

\begin{frame}{Symmetries of the excess}
Recall 
$$\gamma_A(U, P) := |\partial U \cap A| - \left|\left[\int_A \partial^P_\mu 1_U dV\right] dx_P^\mu(P)\right|.$$
For $M = \RR^d$ we get $\gamma_A(U, P)$ is independent of $P$ since in that case $(\partial^P_\mu)$ is just a rotated version of the coordinate frame $(\partial_\mu)$.
Excess is gauge-invariant so this rotation is no-problemo.\pause

On hyperbolic space, translating the coordinate frame distorts it quite a bit! So $\gamma_A(U, P)$ depends on $P$.
Need to bound the error this makes:

\begin{lemma}
For $P, Q \in B$ where $B$ is a ball of radius $r$,
$$|\gamma_B(U, P) - \gamma_B(U, Q)| \lesssim r^{d + 1}.$$
\end{lemma}
\end{frame}

\begin{frame}{Symmetries of the excess II}{Relativistic boogaloo}
    Trying to prove 
    $$|\gamma_{B_r}(U, P) - \gamma_{B_r}(U, Q)| \lesssim r^{d + 1}$$
    taking a leaf out of Daskalopolous and Uhlenbeck's '22 paper on the $p$-Schatten-von Neumann-Laplace equation: embed $\mathbb H^d$ in the Minkowski spacetime $\RR^{1, d}$:
    $$\Psi: \mathbb H^d \to \{t > 0\} \cap \{|x|^2 = t^2\}.$$\pause
    A $1$-form on $\mathbb H^d$ is in particular a smooth map into $\RR^{1, d}$ since we have canonical isomorphisms
    $$\RR^{1, d} = T_P' \mathbb H^d \otimes N_P' \mathbb H^d.$$\pause
    Projecting a covector in $T_P' \mathbb H^d \subset \RR^{1, d}$ into $T_Q' \mathbb H^d$ incurs an error term of size $O(r^2)$.
\pause

I think I'll get cancelled if I do this diagram chase in the Analyst Lair, but geometrically, it's like how we can add a tangent vector to the north pole of $\mathbb S^2$ to a tangent vector to any other point, since both tangent vectors are in $\RR^3$ anyways.
\end{frame}

\begin{frame}{Symmetries of the excess III}{Revenge of Ted Woosley}
    On small scales (the ``nonrelativistic limit'') $\mathbb H^d$ looks like $\{t = 1\}$ to second order.
    That's why we get an $r^2$ gain over the trivial bound $\lesssim r^{d - 1}$.

    The actual proof is a computation involving more (non)commutative diagrams, Lorentz boosts, and lots of $O(r^2)$.
\end{frame}

\begin{frame}{de Giorgi lemma}
\begin{theorem}
There exist $c, C > 0$ depending only on $M$ such that if $U$ is a set of least perimeter and $\gamma_\rho(U, P) \leq c\rho^{d - 1}$, then 
$$\gamma_{\rho/2}(U, P) \leq 2^{-d} \gamma_\rho(U, P) + C\rho^{d + 1}.$$
\end{theorem}\pause

The $C\rho^{d + 1}$ isn't there in the euclidean case. It's an artifact of Taylor expanding the metric, and of the translation errors.\pause
It's harmless though, since by induction,
$$\gamma_{\rho/2^n}(U, P) \leq \frac{c\rho^{d - 1}}{2^{nd}} + C\sum_{k=0}^n \frac{1}{2^{k(d + 1) + (n - k)d}} \leq \frac{c\rho^{d - 1} + C}{2^{nd}}.$$\pause
In other words at small scales we have $\gamma_\rho(U, P) \sim \rho^d$, a gain over $\rho^{d - 1}$!!
\end{frame}

\begin{frame}{Geometric Measure Theory Spins Me Right Round}
    In the reduced boundary $\partial^* U$, $\partial U$ has a possibly nonunique tangent cone. \pause (Why nonunique?) \pause
    This tangent cone is area-minimizing in $\RR^d \subseteq \RR^7$.

    \begin{theorem}[Fleming? '60s]
    Every area-minimizing cone in $\RR^7$ is a vector space.
    \end{theorem}\pause
    
    So $\partial U$ has a (possibly nonunique) tangent SPACE at $P \in \partial^* U$.
    This implies $\gamma_\rho(U, P) \ll \rho^{d - 1}$ for $P \in \partial^* U$.\pause

    If dGL hypothesis is met on a dense set, it's met everywhere, and locally uniformly.
\end{frame}

\begin{frame}{de Giorgi lemma proof}
    Miranda '66 and the monotoncity formula imply: to prove dGL, wlog $U$ has $C^1$ boundary. \pause
    Rotate the explicit coordinates $(x^\mu_P)$, where $y_P := x^0_P$, using gauge invariance: wlog $\partial U$ is (the pushforward by $\Phi^P$ of) the graph of a $C^1$ function $w: \RR^{d - 1} \to \RR$ with $dw$ small. \pause Minkowski spacetime trick: wlog $w(0) = 0$.

    Surface area of $\partial U$ is $\int L(w, dw)$ where 
    $$L(y, \xi) := \sqrt{1 + |\xi|^2}{(1 - (|x|^2 - y^2)/4)^{d - 1}} dx^1 \wedge \cdots \wedge dx^{d - 1}.$$\pause
    Since $w(0) = 0$ and $dw \approx 0$, this Taylor expands to $L(w, dw) \approx 1 + |dw|^2/2$, the Dirichlet energy of $w$.
    So $\Delta w$ is basically $0$.
\end{frame}

\begin{frame}{de Giorgi lemma proof}
\begin{lemma}
There exists $c > 0$ such that for $w: \RR^{d - 1} \to \RR$ with $w(0) = 0$ and $||dw||_{C^0} \leq c$, if $\beta > 0$ satisfies
$$\int_{B_\rho} L(w, dw) - L(w, avg_\rho dw) \leq \beta,$$
and $\int_{B_\rho} L(w, dw)$ is less than the area of a minimizer plus $c \beta$, \pause then 
$$\int_{B_{\rho/2}} L(w, dw) - L(w, avg_{\rho/2} dw) \leq 2^{-d} \beta + C\rho^{d + 1}.$$ 
\end{lemma}

Euclidean case appeared in de Giorgi '61.
Replace $w$ with its harmonic competitor $u$, then use the fact that this is true for a harmonic polynomial (and $u$ splits into harmonic polynomials).\pause 

OTOH, $\int L(w, dw) - L(w, avg_\rho dw)$ is basically just $\gamma_\rho(U, P)$ so we win.
\end{frame}

\end{document}
