% --------------------------------------------------------------
% This is all preamble stuff that you don't have to worry about.
% Head down to where it says "Start here"
% --------------------------------------------------------------

\documentclass[10pt]{beamer}

% \usepackage[margin=1in]{geometry}
\usepackage{amsmath,amsthm,amssymb}
\usepackage{enumitem}
\usepackage{tikz}
\usepackage{wrapfig}
\usepackage{cjhebrew}
% \usepackage[shortlabels]{enumerate}

\usepackage[utf8]{inputenc}
\usepackage[english]{babel}

\usetikzlibrary{calc}
\usetikzlibrary{arrows.meta}
\usetikzlibrary{arrows}

\newcommand{\NN}{\mathbb{N}}
\newcommand{\ZZ}{\mathbb{Z}}
\newcommand{\QQ}{\mathbb{Q}}
\newcommand{\RR}{\mathbb{R}}
\newcommand{\CC}{\mathbb{C}}

\DeclareMathOperator*{\argmin}{argmin}
\DeclareMathOperator{\supp}{supp}
\DeclareMathOperator{\Exc}{Exc}
\newcommand{\normal}{\vec n}
\newcommand{\norm}[1]{\left\lVert#1\right\rVert}

\newcommand{\Spec}{\operatorname{Spec}}

\renewcommand{\Re}{\operatorname{Re}}
\renewcommand{\Im}{\operatorname{Im}}

\newtheorem{theoremconjecture}{Theorem-Conjecture}
\newtheorem{conjecture}{Conjecture}
\newtheorem{question}{Question}

\usepackage[backend=bibtex,style=alphabetic,maxcitenames=50,maxnames=50]{biblatex}
\addbibresource{ultrapowerful.bib}
\renewbibmacro{in:}{}
\DeclareFieldFormat{pages}{#1}

\usetheme{AnnArbor}
\usecolortheme{dove}

\begin{document}
\title{Functions of least gradient and minimal laminations}
\author{Aidan Backus}
% --------------------------------------------------------------
%                         Start here
% --------------------------------------------------------------
\begin{frame}
    \titlepage
\end{frame}

\begin{frame}{The maximum principle for harmonic functions}
\begin{theorem}[maximum principle]
    Let $u$ be a nonconstant harmonic function.
    Then every maximum of $u$ lies on the boundary.
\end{theorem}

Another way of phrasing the same result involves level sets $\partial\{u > y\}$.

\begin{theorem}[maximum principle, reprise]
    Let $u$ be a harmonic function.
    Then every level set of $u$ either has nontrivial homology class, or intersects the boundary.
\end{theorem}

So, maximum principles tell us something about level sets.
What if we cook up a PDE whose solutions have level sets which satisfy a really strong maximum principle?
\end{frame}

\begin{frame}{Functions of least gradient}
Recall that a function $u \in W^{1, p}$ is $p$-\emph{harmonic} if it minimizes $\int |du|^p$, $p > 1$.

\begin{definition}
    A function $u \in BV$ has \emph{least gradient} if it minimizes $\int |du|$ among all functions satisfying the same boundary condition.
\end{definition}

Here as usual 
$$\int |du| = -\inf_{||X||_{C^0} \leq 1} \int u \cdot d^*X$$
where the infimum is taken over compactly supported vector fields $X$.
These functions are degenerations of $p$-harmonic functions as $p \to 1$.

We do not assume $u \in W^{1, 1}$ since functions in $W^{1, 1}$ cannot have jump discontinuities but we want any monotone function, including those with jump discontinuities, on $(0, 1)$ to have least gradient.
\end{frame}

\begin{frame}{Minimal laminations}
\begin{definition}
We say that a hypersurface $N \subset M$ of a Riemannian manifold is \emph{minimal} if for any small ball $B \subseteq M$, $B \cap N$ has minimal area among all intersections of hypersurfaces with $N$.
\end{definition}

Equivalently, $N$ has zero mean curvature. A minimal curve in $\RR^2$ is the same thing as a straight line (more generally, a minimal curve is a geodesic.)
A minimal surface in $\RR^3$ is anything which is ``held taut:'' a plane, soap film, etc.

\begin{definition}
A \emph{minimal lamination} $\lambda$ in a Riemannian manifold $M$ is a closed subset of $M$ equipped with a partition of $\lambda$ into minimal hypersurfaces.
\end{definition}
\end{frame}

\begin{frame}{The maximum principle for functions of least gradient}
\begin{theorem}[essentially Miranda '67]
Let $u$ be a function of least gradient on an open set $U \subseteq \RR^d$, $2 \leq d \leq 7$.
Then the union of level sets $\bigcup_{y \in \RR} \partial\{u > y\}$ is a minimal lamination in $U$.
\end{theorem}

\begin{example}
Let $u$ be a function of least gradient on $\RR^2$. Then the level sets of $u$ are a union of parallel lines (since any two disjoint lines must be parallel).
So there exists a monotone function $v: \RR \to \RR$ such that, after rotating appropriately, $u(x, y) = v(x)$.
\end{example}

\begin{theorem}[-- '22]
Let $M$ be a manifold of constant sectional curvature and dimension $2 \leq d \leq 7$.
Then the union of level sets $\bigcup_{y \in \RR} \partial\{u > y\}$ is a minimal lamination in $U$.
\end{theorem}

\end{frame}

\begin{frame}{Applications}
    
\end{frame}

\begin{frame}{Proof}
    
\end{frame}

\end{document}
