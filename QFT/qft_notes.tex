\documentclass[reqno,12pt,letterpaper]{amsart}
\RequirePackage{amsmath,amssymb,amsthm,graphicx,mathrsfs,url}
\RequirePackage[usenames,dvipsnames]{color}
\RequirePackage[colorlinks=true,linkcolor=Red,citecolor=Green]{hyperref}
\RequirePackage{amsxtra}
\usepackage{cancel}
\usepackage{tikz-cd}

\setlength{\textheight}{8.50in} \setlength{\oddsidemargin}{0.00in}
\setlength{\evensidemargin}{0.00in} \setlength{\textwidth}{6.08in}
\setlength{\topmargin}{0.00in} \setlength{\headheight}{0.18in}
\setlength{\marginparwidth}{1.0in}
\setlength{\abovedisplayskip}{0.2in}
\setlength{\belowdisplayskip}{0.2in}
\setlength{\parskip}{0.05in}
\renewcommand{\baselinestretch}{1.10}

\title{QFT notes}
\author{Aidan Backus}
\date{July 2021}

\newcommand{\NN}{\mathbf{N}}
\newcommand{\ZZ}{\mathbf{Z}}
\newcommand{\QQ}{\mathbf{Q}}
\newcommand{\RR}{\mathbf{R}}
\newcommand{\CC}{\mathbf{C}}
\newcommand{\DD}{\mathbf{D}}
\newcommand{\PP}{\mathbf P}
\newcommand{\MM}{\mathbf M}
\newcommand{\II}{\mathbf I}

\DeclareMathOperator{\card}{card}
\DeclareMathOperator{\ch}{ch}
\DeclareMathOperator{\codim}{codim}
\DeclareMathOperator{\diag}{diag}
\DeclareMathOperator{\diam}{diam}
\DeclareMathOperator{\dom}{dom}
\DeclareMathOperator{\Gal}{Gal}
\DeclareMathOperator{\Hom}{Hom}
\DeclareMathOperator{\Jac}{Jac}
\DeclareMathOperator{\Lip}{Lip}
\DeclareMathOperator{\Met}{Met}
\DeclareMathOperator{\id}{id}
\DeclareMathOperator{\rad}{rad}
\DeclareMathOperator{\rank}{rank}
\DeclareMathOperator{\Radon}{Radon}
\DeclareMathOperator*{\Res}{Res}
\DeclareMathOperator{\Riem}{Riem}
\DeclareMathOperator{\sgn}{sgn}
\DeclareMathOperator{\singsupp}{sing~supp}
\DeclareMathOperator{\Spec}{Spec}
\DeclareMathOperator{\supp}{supp}
\DeclareMathOperator{\Tan}{Tan}
\DeclareMathOperator{\vol}{vol}
\newcommand{\tr}{\operatorname{tr}}

\newcommand{\dbar}{\overline \partial}

\DeclareMathOperator{\atanh}{atanh}
\DeclareMathOperator{\csch}{csch}
\DeclareMathOperator{\sech}{sech}

\DeclareMathOperator{\Div}{div}
\DeclareMathOperator{\grad}{grad}
\DeclareMathOperator{\Ell}{Ell}
\DeclareMathOperator{\WF}{WF}

\newcommand{\Hilb}{\mathcal H}
\newcommand{\Poin}{\mathscr P}

\newcommand{\pic}{\vspace{30mm}}
\newcommand{\dfn}[1]{\emph{#1}\index{#1}}

\renewcommand{\Re}{\operatorname{Re}}
\renewcommand{\Im}{\operatorname{Im}}


\newtheorem{theorem}{Theorem}[section]
\newtheorem{badtheorem}[theorem]{``Theorem"}
\newtheorem{prop}[theorem]{Proposition}
\newtheorem{lemma}[theorem]{Lemma}
\newtheorem{proposition}[theorem]{Proposition}
\newtheorem{corollary}[theorem]{Corollary}
\newtheorem{conjecture}[theorem]{Conjecture}
\newtheorem{axiom}[theorem]{Axiom}

\theoremstyle{definition}
\newtheorem{definition}[theorem]{Definition}
\newtheorem{remark}[theorem]{Remark}
\newtheorem{example}[theorem]{Example}

\newtheorem{exercise}[theorem]{Discussion topic}
\newtheorem{homework}[theorem]{Homework}
\newtheorem{problem}[theorem]{Problem}

\newtheorem{ack}{Acknowledgements}
\newtheorem{notate}{Notation}

\numberwithin{equation}{section}


% Mean
\def\Xint#1{\mathchoice
{\XXint\displaystyle\textstyle{#1}}%
{\XXint\textstyle\scriptstyle{#1}}%
{\XXint\scriptstyle\scriptscriptstyle{#1}}%
{\XXint\scriptscriptstyle\scriptscriptstyle{#1}}%
\!\int}
\def\XXint#1#2#3{{\setbox0=\hbox{$#1{#2#3}{\int}$ }
\vcenter{\hbox{$#2#3$ }}\kern-.6\wd0}}
\def\ddashint{\Xint=}
\def\dashint{\Xint-}

%\usepackage{color}
%\hypersetup{%
%    colorlinks=true, % make the links colored%
%    linkcolor=blue, % color TOC links in blue
%    urlcolor=red, % color URLs in red
%    linktoc=all % 'all' will create links for everything in the TOC
%Ning added hyperlinks to the table of contents 6/17/19
%}

% style=alphabetic
\usepackage[backend=bibtex,maxcitenames=50,maxnames=50]{biblatex}
%\addbibresource{topics.bib}
\renewbibmacro{in:}{}
\DeclareFieldFormat{pages}{#1}

\begin{document}

\maketitle

%\tableofcontents

\section{Introduction}
Recall the foundations of quantum mechanics.
We are given a separable Hilbert space $\Hilb$.
States of the physical system are vectors $\psi$ modulo scalars $\lambda$, i.e. points in the projective space given by $\Hilb$.
There is a distinguished densely-defined self-adjoint operator $H$ on $\Hilb$ called the \dfn{Hamiltonian}, which is a second-order differential operator.
An eigenstate $\psi$ of $H$ of eigenvalue $E$ has energy $E$, and the energy of a more general state is ill-defined.
The time-evolution of the system is given by the Schr\"odinger equation
$$i\partial_t \psi = H\psi$$
which clearly is not relativistic in general because the right-hand side is second-order in space, while the left-hand side is first-order in time.

Let us consider the spectrum $S$ of $H$.
If $S$ is unbounded from below then we can repeatedly apply annihilation operators to disperse energy from an eigenstate $\psi$.
Thus it is possible for the universe to lose all its energy.
Since this isn't possible we assume that $S$ is unbounded from below.
Once that's done we might as well add a constant to $H$ to impose that $S \subseteq \RR^+$.
This is the \dfn{positive-energy condition} and we will always assume it.

Let $\Poin$ be the Poincar\'e group.
Any relativistic theory must be invariant under $\Poin$.
What does that mean for quantum mechanics?
We want $\Poin$ to act on $\Hilb$ like a unitary representation, but we need to allow $\Poin$ to be twisted by the gauge freedom inherent in $\Hilb$.

To be precise: Recall that the circle group $S^1$ acts on $\Hilb$ by scalars, namely
$$\theta \mapsto (\psi \mapsto e^{i\theta}\psi).$$
Moreover, while we can demand that $||\psi|| = 1$ to normalize away the gauge freedom given by the action of $\RR^*$ on $\Hilb$, there's no good way to normalize away the action of $S^1$.
No physical experiment can distinguish between two wavefunctions in the same orbit of $S^1$!

\begin{definition}
Let $G$ be a Lie group.
A smooth action $\varphi$ of $G$ on $\Hilb$ is a \dfn{projective unitary representation} of $G$ if $G$ acts by unitary operators and for every $g_1,g_2 \in G$ there exists $\omega \in S^1$ such that
$$\varphi(g_1g_2) = \omega \varphi(g_1) \varphi(g_2).$$
\end{definition}

Unfortunately, projective unitary representations are really hard to construct.

\begin{theorem}
There are no faithful finite-dimensional projective unitary representations of $\Poin$.
\end{theorem}

So we need to consider $\Hilb$ infinite-dimensional only.

As a starting point, let's consider representations $G$ of the subgroup of $\Poin$ given by time-translations.
Since Schr\"odinger propagators are unitary, giving a representation gives a solution of a Schr\"odinger equation for some Hamiltonian $H$.

\begin{definition}
A \dfn{quantum field} is a separable projective unitary representation of $\Poin$ which obeys the positive-energy condition in the sense that, if we view the action of the time-translation subgroup of $\Poin$ as by Schr\"odinger propagators, we obtain a Hamiltonian $H$ such that $\Spec H \subseteq \RR^+$.
\end{definition}

Wigner gave a full classification of the irreducible projective unitary representations of $\Poin$.
These representations are the simplest possible quantum systems, thus we define:

\begin{definition}
An \dfn{elementary particle} is a nontrivial irreducible quantum field.
\end{definition}

\section{Classical fields}
In the low-energy, large-scale, noisy regime, we can approximate quantum fields by classical fields.

Let $M$ be a Lorentz manifold, which will almost always be Minkowski spacetime.

\begin{definition}
A \dfn{classical field} is a $C^2$ section of a fiber bundle on $M$.
\end{definition}

For now let's assume that all fiber bundles are trivial.
Gauge fields will NOT have this property however.

\begin{definition}
A \dfn{$\sigma$-model} is a classical field valued in $S^2$.
\end{definition}

We will mainly be interested in classical fields which evolve according to Lagrangian mechanics.
Nobody knows why this is the physically correct thing to do but it is.
We will always assume that the action of a family of scalar fields $(\phi_a)$ has the form
$$S(\Omega) = \int_\Omega \mathcal L(x, \phi_a(x), \partial_\mu \phi_a(x)) ~dx$$
where $\Omega$ ranges over Borel subsets of $M$.
(More precisely, we define $\mathcal L$ to be a \emph{density} on $M$, and restrict to $\Omega$-compactly supported perturbations of $\mathcal L$ when applying variational calculus.)
In relativistic situations we will demand that $\mathcal L$ is invariant under the action of $\Poin$, which in particular implies that the classical field will behave relativistically.

\begin{example}
If
$$\mathcal L = \phi_{,\mu} \phi^{,\mu} - m^2 \phi^2$$
then $\phi$ satisfies the Klein-Gordon equation
$$(\Box + mu^2)\phi = 0$$
which is the hyperbolic equation which governs a wavefunction of a relativistic quantum particle with no spin (equivalently scalar field) $\psi$.
\end{example}








\printbibliography


\end{document}
