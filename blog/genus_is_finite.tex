\documentclass[12pt]{article}
\usepackage[pdftex,pagebackref,letterpaper=true,colorlinks=true,pdfpagemode=none,urlcolor=blue,linkcolor=blue,citecolor=blue,pdfstartview=FitH]{hyperref}

\usepackage{amsmath,amsfonts}
\usepackage{graphicx}
\usepackage{color}

% Number systems
\newcommand{\NN}{\mathbb{N}}
\newcommand{\ZZ}{\mathbb{Z}}
\newcommand{\QQ}{\mathbb{Q}}
\newcommand{\RR}{\mathbb{R}}
\newcommand{\CC}{\mathbb{C}}
\newcommand{\PP}{\mathbb P}
\newcommand{\FF}{\mathbb F}
\newcommand{\DD}{\mathbb D}
\newcommand{\EE}{\mathbb E}
\renewcommand{\epsilon}{\varepsilon}

\newcommand{\Aut}{\operatorname{Aut}}
\newcommand{\cl}{\operatorname{cl}}
\newcommand{\ch}{\operatorname{ch}}
\newcommand{\Con}{\operatorname{Con}}
\newcommand{\coker}{\operatorname{coker}}
\newcommand{\CVect}{\CC\operatorname{-Vect}}
\newcommand{\Cantor}{\mathcal{C}}
\newcommand{\D}{\mathcal{D}}
\newcommand{\card}{\operatorname{card}}
\newcommand{\dbar}{\overline \partial}
\newcommand{\diam}{\operatorname{diam}}
\newcommand{\dom}{\operatorname{dom}}
\newcommand{\End}{\operatorname{End}}
\DeclareMathOperator*{\esssup}{ess\,sup}
\newcommand{\Hess}{\operatorname{Hess}}
\newcommand{\Hom}{\operatorname{Hom}}
\newcommand{\id}{\operatorname{id}}
\newcommand{\Ind}{\operatorname{Ind}}
\newcommand{\Inn}{\operatorname{Inn}}
\newcommand{\interior}{\operatorname{int}}
\newcommand{\lcm}{\operatorname{lcm}}
\newcommand{\mesh}{\operatorname{mesh}}
\newcommand{\LL}{\mathcal L_0}
\newcommand{\Leb}{\mathcal{L}_{\text{loc}}^2}
\newcommand{\Lip}{\operatorname{Lip}}
\newcommand{\ppic}{\vspace{35mm}}
\newcommand{\ppset}{\mathcal{P}}
\DeclareMathOperator{\proj}{proj}
\DeclareMathOperator*{\Res}{Res}
\newcommand{\Riem}{\mathcal{R}}
\newcommand{\RVect}{\RR\operatorname{-Vect}}
\newcommand{\Sch}{\mathcal{S}}
\newcommand{\sgn}{\operatorname{sgn}}
\newcommand{\spn}{\operatorname{span}}
\newcommand{\Spec}{\operatorname{Spec}}
\newcommand{\supp}{\operatorname{supp}}
\newcommand{\TT}{\mathcal T}
\DeclareMathOperator{\tr}{tr}

% Calculus of variations
\DeclareMathOperator{\pp}{\mathbf p}
\DeclareMathOperator{\zz}{\mathbf z}
\DeclareMathOperator{\uu}{\mathbf u}
\DeclareMathOperator{\vv}{\mathbf v}
\DeclareMathOperator{\ww}{\mathbf w}

% Categories
\newcommand{\Ab}{\mathbf{Ab}}
\newcommand{\Cat}{\mathbf{Cat}}
\newcommand{\Group}{\mathbf{Group}}
\newcommand{\Module}{\mathbf{Module}}
\newcommand{\Set}{\mathbf{Set}}
\DeclareMathOperator{\Fun}{Fun}
\DeclareMathOperator{\Iso}{Iso}

% Complex analysis
\renewcommand{\Re}{\operatorname{Re}}
\renewcommand{\Im}{\operatorname{Im}}

% Logic
\renewcommand{\iff}{\leftrightarrow}
\newcommand{\Henkin}{\operatorname{Henk}}
\newcommand{\PA}{\mathbf{PA}}
\DeclareMathOperator{\proves}{\vdash}

% Group
\DeclareMathOperator{\Gal}{Gal}
\DeclareMathOperator{\Fix}{Fix}
\DeclareMathOperator{\Lie}{Lie}
\DeclareMathOperator{\Out}{Out}

\DeclareMathOperator{\Diffeo}{Diffeo}

\newcommand{\GL}{\operatorname{GL}}
\newcommand{\ppGL}{\operatorname{PGL}}
\newcommand{\SL}{\operatorname{SL}}
\newcommand{\SO}{\operatorname{SO}}

% Other symbols
\newcommand{\heart}{\ensuremath\heartsuit}
\newcommand{\club}{\ensuremath\clubsuit}

\DeclareMathOperator{\atanh}{atanh}

\def\Xint#1{\mathchoice
{\XXint\displaystyle\textstyle{#1}}%
{\XXint\textstyle\scriptstyle{#1}}%
{\XXint\scriptstyle\scriptscriptstyle{#1}}%
{\XXint\scriptscriptstyle\scriptscriptstyle{#1}}%
\!\int}
\def\XXint#1#2#3{{\setbox0=\hbox{$#1{#2#3}{\int}$ }
\vcenter{\hbox{$#2#3$ }}\kern-.6\wd0}}
\def\ddashint{\Xint=}
\def\dashint{\Xint-}

\setlength{\oddsidemargin}{0pt}
\setlength{\evensidemargin}{0pt}
\setlength{\textwidth}{6.0in}
\setlength{\topmargin}{0in}
\setlength{\textheight}{8.5in}

\setlength{\parindent}{0in}
\setlength{\parskip}{5px}

%%%%%%%%% For wordpress conversion

\def\more{}

\newif\ifblog
\newif\iftex
\blogfalse
\textrue


\usepackage{ulem}
\def\em{\it}
\def\emph#1{\textit{#1}}

\def\image#1#2#3{\begin{center}\includegraphics[#1pt]{#3}\end{center}}

\let\hrefnosnap=\href

\newenvironment{btabular}[1]{\begin{tabular} {#1}}{\end{tabular}}

\newenvironment{red}{\color{red}}{}
\newenvironment{green}{\color{green}}{}
\newenvironment{blue}{\color{blue}}{}

%%%%%%%%% Typesetting shortcuts

\def\B{\{0,1\}}
\def\xor{\oplus}

\def\P{{\mathbb P}}
\def\E{{\mathbb E}}
\def\var{{\bf Var}}

\def\N{{\mathbb N}}
\def\Z{{\mathbb Z}}
\def\R{{\mathbb R}}
\def\C{{\mathbb C}}
\def\Q{{\mathbb Q}}
\def\eps{{\epsilon}}

\def\bz{{\bf z}}

\def\true{{\tt true}}
\def\false{{\tt false}}

%%%%%%%%% Theorems and proofs

\newtheorem{exercise}{Exercise}
\newtheorem{theorem}{Theorem}
\newtheorem{lemma}[theorem]{Lemma}
\newtheorem{definition}[theorem]{Definition}
\newtheorem{corollary}[theorem]{Corollary}
\newtheorem{proposition}[theorem]{Proposition}
\newtheorem{example}{Example}
\newtheorem{remark}[theorem]{Remark}
\newenvironment{proof}{\noindent {\sc Proof:}}{$\Box$ \medskip} 


\begin{document}

% --------------------------------------------------------------
%                         Start here
% --------------------------------------------------------------\

A few years ago I took a PDE course. We were learning about something to do with elliptic pseudodifferential operators and the speaker drew a commutative diagram on the board and said, "You see, this comes from a short exact sequence --" and the whole room started laughing in discomfort.
The speaker then remarked that Craig Evans himself would ban him from teaching analysis if word of the incident ever leaked, which might have something to do with why I have not disclosed the speaker's name
%🥵

Before recently, I found topology to be quite a scary area of math. It is still very much my weakest suit, but I should like to have some amount of competency with it.
I have since come around to the viewpoint that cohomology is just a clever gadget for counting solutions of PDE.
This has made the pill a little easier to swallow, and makes the previous anecdote all the more awkward.

As part of my ventures into trying to learn topology, in this post I will give a proof that the genus of any compact Riemann surface is finite.
I am confident that this proof is not original, because it's sort of the obvious proof if an analyst trying to prove this fact just followed their nose, but it seems a lot more natural to me than the proof in Forster, so let's do this.

Let us start with some generalities.
Fix a compact Riemann surface $X$, references to which we will suppress when possible.
Let
$$0 \to A \to B \to C \to 0$$
be a short exact sequence of sheaves.
In our case, the sheaves will be sheaves of Fréchet spaces on $X$, which might not be homologically kosher, but that won't cause any real issues.
Then we get a long exact sequence in cohomology
$$0 \to H^0(A) \to H^0(B) \to H^0(C) \to H^1(A) \to H^1(B) \to H^1(C) \to \cdots.$$
If every germ $u$ of $B$ extends to a global section $\tilde u \in B(X)$ of $B$, then $H^1(B) = 0$ and the long exact sequence collapses to the exact sequence
$$0 \to H^0(A) \to B(X) \to C(X) \to H^1(A) \to 0.$$
In particular, the morphism of sheaves $B \to C$ induces a bounded linear map $T: B(X) \to C(X)$ such that $H^0(A)$ is the kernel of $T$ and $H^1(A)$ is the cokernel of $T$.
Now, if $T$ is a Fredholm operator, then its index $k$ satisfies
$$k = \operatorname{dim} H^0(A) - \operatorname{dim} H^1(A).$$

Let $\mathcal O$ denote the sheaf of holomorphic functions on $X$ and $\overline \partial$ the Cauchy-Riemann operator.
Let $\mathcal E$ denote the sheaf of smooth functions on $X$; since $X$ has enough partitions of unity, every smooth germ on $X$ extends to a smooth function on $X$.
The maps $\overline \partial: \mathcal E(U) \to \mathcal E(U)$, for $U \subseteq X$ open, induces a short exact sequence of sheaves of Fréchet spaces
$$0 \to \mathcal O \to \mathcal E \to \mathcal E \to 0$$
and hence an exact sequence in cohomology
$$0 \to \mathbf C \to \mathcal E(X) \to \mathcal E(X) \to H^1(\mathcal O) \to 0.$$
Here we used Liouville's theorem.
On the other hand, the dimension of $H^1(\mathcal O)$ is by definition the genus $g$ of $X$.
Therefore, if $k$ is the Fredholm index of $\overline \partial$, then
$$g = 1 - k.$$

It remains to show that $k$ is well-defined and finite; that is, $\overline \partial$ is Fredholm.
This is a standard elliptic regularity argument, which I will now recall.
We first fix a volume form $dV$ on $X$, which exists since $X$ is an orientable surface.
This induces an $L^2$ norm on $X$, namely
$$||u||_{L^2} = \int_X |u|^2 ~dV.$$
Unfortunately the usual Sobolev notation $H^s$ clashes with the notation for cohomology, so let me use $W^s$ to denote the completion of $\mathcal E$ under the norm
$$||u||_s = \sum_{|\alpha| \leq s} ||\partial^\alpha u||_{L^2}$$
where $\alpha$ ranges over multiindices.
Then $W^0 = L^2$ and $\overline \partial$ maps $W^1 \to W^0$.
The kernel of $\overline \partial$ is finite-dimensional (since it is isomorphic to $\mathbf C$, by Liouville's theorem and Weyl's lemma), so to deduce that $\overline \partial$ is Fredholm as an operator $W^1 \to W^0$ it suffices to show that the image of $\overline \partial$ is closed in $W^0$.

We first claim the elliptic regularity estimate
$$||u||_1 \leq C ||f||_0 + C ||u||_0.$$
By definition of the Sobolev norm, we have
$$||u||_1 = ||u||_0 + ||u'||_0 + ||f||_0.$$
Without loss of generality, we may assume that $u$ is smooth.
Then we can write $u = v + w$ where $v$ and $\overline w$ are holomorphic.
In particular, $u' = v'$ and $f = \overline \partial w$, so
$$||u||_1 = ||u||_0 + ||v'||_0 + ||f||_0.$$
The only troublesome term here is $v'$.
Taking a Cauchy estimate, we see that
$$|v'(z)| \leq ||v||_{L^\infty} \leq C||v||_{L^2} = C||v||_0.$$
But $X$ is compact, so has finite volume; therefore
$$||v'||_0 = ||v'||_{L^2} \leq C||v||_{L^\infty} \leq C||v||_0 \leq C||u||_0.$$
This gives the desired bound.

Let $u_n$ be a sequence in $W^1$ with $f_n = \overline \partial u_n \in W^0$, and assume that the $f_n$ are Cauchy in $W^0$.
Without loss of generality we may assume that $u_n \in K^\perp$ where $K$ is the kernel of $\overline \partial$.
If the $u_n$ are not bounded in $W^1$, we may replace them with $u_n/||u_n||_1$, and thus assume that they are in fact bounded.
By the Rellich-Kondrachov theorem (which says that the natural map $W^1 \to W^0$ is compact), we may therefore assume that the $u_n$ are Cauchy in $W^0$.
But then
$$||u_n - u_m||_1 \leq C ||f_n - f_m||_0 + C ||u_n - u_m||_0$$
so the $u_n$ are Cauchy in $W^1$.
Therefore the $u_n$ converge in $K^\perp$, hence the $f_n$ converge in the image $Z$ of $\overline \partial$, since $\overline \partial$ gives an isomorphism $K^\perp \to Z$.
Therefore $Z$ is closed, so $\overline \partial$ is Fredholm. Therefore $k$ and hence $g$ is finite.



\end{document}
