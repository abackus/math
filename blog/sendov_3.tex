\documentclass[12pt]{article}
\usepackage[pdftex,pagebackref,letterpaper=true,colorlinks=true,pdfpagemode=none,urlcolor=blue,linkcolor=blue,citecolor=blue,pdfstartview=FitH]{hyperref}

\usepackage{amsmath,amsfonts}
\usepackage{graphicx}
\usepackage{color}

% Number systems
\newcommand{\NN}{\mathbb{N}}
\newcommand{\ZZ}{\mathbb{Z}}
\newcommand{\QQ}{\mathbb{Q}}
\newcommand{\RR}{\mathbb{R}}
\newcommand{\CC}{\mathbb{C}}
\newcommand{\PP}{\mathbb P}
\newcommand{\FF}{\mathbb F}
\newcommand{\DD}{\mathbb D}
\newcommand{\EE}{\mathbb E}
\renewcommand{\epsilon}{\varepsilon}

\newcommand{\Aut}{\operatorname{Aut}}
\newcommand{\cl}{\operatorname{cl}}
\newcommand{\ch}{\operatorname{ch}}
\newcommand{\Con}{\operatorname{Con}}
\newcommand{\coker}{\operatorname{coker}}
\newcommand{\CVect}{\CC\operatorname{-Vect}}
\newcommand{\Cantor}{\mathcal{C}}
\newcommand{\D}{\mathcal{D}}
\newcommand{\card}{\operatorname{card}}
\newcommand{\dbar}{\overline \partial}
\newcommand{\diam}{\operatorname{diam}}
\newcommand{\dom}{\operatorname{dom}}
\newcommand{\End}{\operatorname{End}}
\DeclareMathOperator*{\esssup}{ess\,sup}
\newcommand{\Hess}{\operatorname{Hess}}
\newcommand{\Hom}{\operatorname{Hom}}
\newcommand{\id}{\operatorname{id}}
\newcommand{\Ind}{\operatorname{Ind}}
\newcommand{\Inn}{\operatorname{Inn}}
\newcommand{\interior}{\operatorname{int}}
\newcommand{\lcm}{\operatorname{lcm}}
\newcommand{\mesh}{\operatorname{mesh}}
\newcommand{\LL}{\mathcal L_0}
\newcommand{\Leb}{\mathcal{L}_{\text{loc}}^2}
\newcommand{\Lip}{\operatorname{Lip}}
\newcommand{\ppic}{\vspace{35mm}}
\newcommand{\ppset}{\mathcal{P}}
\DeclareMathOperator{\proj}{proj}
\DeclareMathOperator*{\Res}{Res}
\newcommand{\Riem}{\mathcal{R}}
\newcommand{\RVect}{\RR\operatorname{-Vect}}
\newcommand{\Sch}{\mathcal{S}}
\newcommand{\sgn}{\operatorname{sgn}}
\newcommand{\spn}{\operatorname{span}}
\newcommand{\Spec}{\operatorname{Spec}}
\newcommand{\supp}{\operatorname{supp}}
\newcommand{\TT}{\mathcal T}
\DeclareMathOperator{\tr}{tr}

% Calculus of variations
\DeclareMathOperator{\pp}{\mathbf p}
\DeclareMathOperator{\zz}{\mathbf z}
\DeclareMathOperator{\uu}{\mathbf u}
\DeclareMathOperator{\vv}{\mathbf v}
\DeclareMathOperator{\ww}{\mathbf w}

% Categories
\newcommand{\Ab}{\mathbf{Ab}}
\newcommand{\Cat}{\mathbf{Cat}}
\newcommand{\Group}{\mathbf{Group}}
\newcommand{\Module}{\mathbf{Module}}
\newcommand{\Set}{\mathbf{Set}}
\DeclareMathOperator{\Fun}{Fun}
\DeclareMathOperator{\Iso}{Iso}

% Complex analysis
\renewcommand{\Re}{\operatorname{Re}}
\renewcommand{\Im}{\operatorname{Im}}

% Logic
\renewcommand{\iff}{\leftrightarrow}
\newcommand{\Henkin}{\operatorname{Henk}}
\newcommand{\PA}{\mathbf{PA}}
\DeclareMathOperator{\proves}{\vdash}

% Group
\DeclareMathOperator{\Gal}{Gal}
\DeclareMathOperator{\Fix}{Fix}
\DeclareMathOperator{\Lie}{Lie}
\DeclareMathOperator{\Out}{Out}

\DeclareMathOperator{\Diffeo}{Diffeo}

\newcommand{\GL}{\operatorname{GL}}
\newcommand{\ppGL}{\operatorname{PGL}}
\newcommand{\SL}{\operatorname{SL}}
\newcommand{\SO}{\operatorname{SO}}

% Other symbols
\newcommand{\heart}{\ensuremath\heartsuit}
\newcommand{\club}{\ensuremath\clubsuit}

\DeclareMathOperator{\atanh}{atanh}

\def\Xint#1{\mathchoice
{\XXint\displaystyle\textstyle{#1}}%
{\XXint\textstyle\scriptstyle{#1}}%
{\XXint\scriptstyle\scriptscriptstyle{#1}}%
{\XXint\scriptscriptstyle\scriptscriptstyle{#1}}%
\!\int}
\def\XXint#1#2#3{{\setbox0=\hbox{$#1{#2#3}{\int}$ }
\vcenter{\hbox{$#2#3$ }}\kern-.6\wd0}}
\def\ddashint{\Xint=}
\def\dashint{\Xint-}

\setlength{\oddsidemargin}{0pt}
\setlength{\evensidemargin}{0pt}
\setlength{\textwidth}{6.0in}
\setlength{\topmargin}{0in}
\setlength{\textheight}{8.5in}

\setlength{\parindent}{0in}
\setlength{\parskip}{5px}

\input{macrosblog}

\begin{document}

% --------------------------------------------------------------
%                         Start here
% --------------------------------------------------------------\

In this post we're going to complete the proof of case zero and continue the proof of case one. In the last two posts we managed to prove:

\begin{theorem}
Let $n$ be a nonstandard natural, and let $f$ be a monic polynomial of degree $n$ on $\mathbf C$ with all zeroes in $\overline{D(0, 1)}$.
Suppose that $a$ is a zero of $f$ such that:
\begin{enumerate}
\item Either $a$ or $a - 1$ is infinitesimal, and
\item $f$ has no critical points on $\overline{D(a, 1)}$.
\end{enumerate}
Let $\lambda$ be a random zero of $f$ and $\zeta$ a random critical point of $f$.
Let $\mu$ be the expected value of $\lambda$.
Let $z$ be a complex number outside some measure zero set and let $\gamma$ be a contour that misses the zeroes of $f$,$f'$. Then:
\begin{enumerate}
\item $\zeta \in \overline{D(0, 1)} \setminus \overline{D(a, 1)}$.
\item $\mu = \mathbf E \zeta$.
\item One has
$$U_\lambda(z) = -\frac{1}{n} \log |f(z)|$$
and
$$U_\zeta(z) = -\frac{\log n}{n - 1} - \frac{1}{n-1} \log |f'(z)|.$$
\item One has
$$s_\lambda(z) = \frac{1}{n} \frac{f'(z)}{f(z)}$$
and
$$s_\zeta(z) = \frac{1}{n - 1} \frac{f''(z)}{f'(z)}.$$
\item One has
$$U_\lambda(z) - \frac{n - 1}{n} U_\zeta(z) = \frac{1}{n} \log |s_\lambda(z)|$$
and
$$s_\lambda(z) - \frac{n - 1}{n} s_\zeta(z) = -\frac{1}{n} \frac{s_\lambda'(z)}{s_\lambda(z)}.$$
\item One has
$$f(\gamma(1)) = f(\gamma(0)) \exp \left(n \int_\gamma s_\lambda(z) ~dz\right)$$
and
$$f'(\gamma(1)) = f(\gamma(0)) \exp\left((n-1) \int_\gamma s_\zeta(z) ~dz\right).$$
\end{enumerate}
Moreover,
\begin{enumerate}
\item If $a$ is infinitesimal (case zero), then $\lambda^{(\infty)},\zeta^{(\infty)}$ are identically distributed and almost surely lie in
$$C = \{e^{i\theta}: 2\theta \in [\pi, 3\pi]\}.$$
Moreover, if $K$ is any compact set which misses $C$, then
$$\mathbf P(\lambda \in K) = O\left(a + \frac{\log n}{n^{1/3}}\right),$$
so $d(\lambda, C)$ is infinitesimal in probability.
\item If $a - 1$ is infinitesimal (case one), then $\lambda^{(\infty)}$ is uniformly distributed on $\partial D(0, 1)$ and $\zeta^{(\infty)}$ is almost surely zero.
Moreover,
$$\mathbf E \log \frac{1}{|\lambda|}, \mathbf E \log |\zeta - a| = O(n^{-1}).$$
\end{enumerate}
\end{theorem}

We also saw that Sendov's conjecture in high degree was equivalent to the following result, that we will now prove.
\begin{lemma}
\label{main conjecture}
Let $n$ be a nonstandard natural, and let $f$ be a monic polynomial of degree $n$ on $\mathbf C$ with all zeroes in $\overline{D(0, 1)}$.
Let $a$ be a zero of $f$ such that:
\begin{enumerate}
\item Either $a \log n$ is infinitesimal (case zero), or
\item There is a standard $\varepsilon_0 > 0$ such that
$$1 - o(1) \leq a \leq 1 - \varepsilon_0^n$$
(case one).
\end{enumerate}
If there are no critical points of $f$ in $\overline{D(a, 1)}$, then $0 = 1$.
\end{lemma}

\section{Case zero}
Now we prove case zero -- the easy case -- of Lemma \ref{main conjecture}.

Suppose that $a \log n$ is infinitesimal.
In this case, $\lambda^{(\infty)}, \zeta^{(\infty)}$ are identically distributed and almost surely are $\in C$.

\begin{lemma}
There are $0 < r_1 < r_2 < 1/2$ such that for every $|z| \in [r_1, r_2]$,
$$|s_{\lambda^{(\infty)}}(z)| \sim 1$$
uniformly.
\end{lemma}
\begin{proof}
Since $\lambda^{(\infty)}$ is supported in $C$, $s_{\lambda^{(\infty)}}$ is holomorphic away from $C$.
Since $\lambda^{(\infty)}$ is bounded, if $z$ is near $\infty$ then
$$s_{\lambda^{(\infty)}}(z) = \mathbf E\frac{1}{z - \lambda^{(\infty)}} \sim \mathbf E \frac{1}{z} = \frac{1}{z}$$
which is nonzero near $\infty$.
So the variety $s_{\lambda^{(\infty)}} = 0$ is discrete, so there are $0 < r_1 < r_2 < 1/2$ such that $s_{\lambda^{(\infty)}}(re^{i\theta}) \neq 0$ whenever $r \in [r_1, r_2]$.
To see this, suppose not; then for every $r_1 < r_2$ we can find $r \in [r_1, r_2]$ and $\theta$ with $s_{\lambda^{(\infty)}}(re^{i\theta}) = 0$, so $s_{\lambda^{(\infty)}}$ has infinitely many zeroes in the compact set $\overline{D(0, 1/2)}$.
Since this is definitely not true, the claim follows by continuity of $s_{\lambda^{(\infty)}}$.
\end{proof}

Let $m$ be the number of zeroes of $s_{\lambda^{(\infty)}}$ in $D(0, r_1)$, so $m$ is a nonnegative standard natural since $s_{\lambda^{(\infty)}}$ is standard and $\overline{D(0, 1/2)}$ is compact.
Let $\gamma(\theta) = re^{i\theta}$ where $r \in (r_1, r_2)$; then
$$\int_\gamma \frac{s_{\lambda^{(\infty)}}'(z)}{s_{\lambda^{(\infty)}}(z)} = m,$$
by the argument principle.

We claim that in fact $m \leq -1$, which contradicts that $m$ is nonnegative.
This will be proven in the rest of this section.

Here something really strange happens in Tao's paper. He proves this:
\begin{lemma}
One has
$$\left|\frac{1}{n} \frac{f'(z)}{f(z)} - s_{\lambda^{(\infty)}(z)}\right| = o(1)$$
in $L^1_{loc}$.
\end{lemma}
We now need to show that the convergence in $L^1_{loc}$ above commutes with the use of the argument principle so that
$$\int_\gamma \frac{(f'/f)'(z)}{f'(z)/f(z)} = m;$$
this will be good because we have control on the zeroes and critical points of $f$ using our contradiction assumption.
What's curious to me is that Tao seems to substitute this with convergence in $L^\infty_{loc}$ on an annulus.
Indeed, convergence in $L^\infty_{loc}$ does commute with use of the argument principle, but at no point of the proof does it seem like he uses the convergence in $L^1_{loc}$. So I include the proof of the latter in the next section as a curiosity item, but I think it can be omitted entirely.
Tell me in the comments if I've made a mistake here.

If $\chi$ is a smooth cutoff supported on $\overline{D(0, 1/2)}$ and identically one on $\overline{D(0, r_3)}$ (where $r_2 < r_3 < 1/2$), one has
$$\frac{1}{n} \frac{f'(z)}{f(z)} = \mathbf E \frac{1}{z - \lambda} = \mathbf E \frac{1 - \chi(\lambda)}{z - \lambda} + \mathbf E \frac{\chi(\lambda)}{z - \lambda}.$$

The $1 - \chi$ term is easy to deal with, since for every $z$, $(1 - \chi(\lambda))/(z - \lambda)$ is a bounded continuous function of $\lambda$ whenever $|z| < r_2$ (so $|\lambda - z| \geq r_3 - r_2 > 0$).
By the definition of being infinitesimal in distribution we have
$$\left|\mathbf E\frac{1 - \chi(\lambda)}{z - \lambda} - \frac{1}{z - \lambda^{(\infty)}}\right| = o(1).$$
Therefore
$$\mathbf E \frac{1 - \chi(\lambda)}{z - \lambda} - s_{\lambda^{(\infty)}}(z)$$
is uniformly infinitesimal.

Now we treat the $\chi$ term. Interestingly, this is the main point of the argument where we use that $a \log n$ is infinitesimal, and the rest of the argument seems to mainly go through with much weaker assumptions on $a$.
\begin{lemma}
There is an $r \in [r_1, r_2]$ such that if $|z| = r$ then
$$\left|\mathbf E \frac{\chi(\lambda)}{z - \lambda}\right| = o(1).$$
\end{lemma}
\begin{proof}
By the triangle inequality and its reverse, if $|z| = r$ then
$$\left|\mathbf E \frac{\chi(\lambda)}{z - \lambda}\right| \leq \mathbf E \frac{\chi(\lambda)}{|r - |\lambda||}.$$
Here $r \in [r_1, r_2]$ is to be chosen.

Since we have
$$\mathbf P(\lambda \in K) = O\left(a + \frac{\log n}{n^{1/3}}\right)$$
whenever $K$ is a compact set which misses $C$, this in particular holds when $K = \overline{B(0, 1/2)}$.
Since $a = o(1/\log n)$ and $n^{-1/3}\log n = o(1/\log n)$ it follows that
$$\mathbf P(|\lambda| \leq 1/2) = o(1/\log n).$$
In particular,
$$\mathbf E\chi(\lambda) = o(1/\log n).$$

We now claim
$$\int_{r_1}^{r_2} \frac{dr}{\max(|r - |\lambda||, n^{-10})} = O(\log n).$$
By splitting the integrand we first bound
$$\int_{\substack{[r_1,r_2]\\|r-|\lambda|| \leq n^{-10}}} \frac{dr}{\max(|r - |\lambda||, n^{-10})} \leq 2n^{-10}n^{10} = 2 = O(\log n)$$
since $\log n$ is nonstandard and the domain of integration has measure at most $2n^{-10}$.
On the other hand, the other term
$$\int_{\substack{[r_1,r_2]\\|r-|\lambda|| \geq n^{-10}}} \frac{dr}{\max(|r - |\lambda||, n^{-10})} \leq \int_{n^{-10}}^{r_2 - r_1} \frac{dr}{r} = \log(r_2 - r_1) - 10 \log n = O(\log n)$$
since $\log(r_2 - r_1)$ is standard while $\log n$ is nonstandard.
This proves the claim.

Putting the above two paragraphs together and using Fubini's theorem,
$$\int_{r_1}^{r_2} \mathbf E \frac{\chi(\lambda)}{\max(|r - |\lambda||, n^{-10})} ~dr = \mathbf E\chi(\lambda) \int_{r_1}^{r_2} \frac{1}{\max(|r - |\lambda||, n^{-10})} ~dr = O(\log n) \mathbf E\chi(\lambda)$$
is infinitesimal.
So outside of a set of infinitesimal measure, $r \in [r_1, r_2]$ satisfies
$$\mathbf E \frac{\chi(\lambda)}{\max(|r - |\lambda||, n^{-10})} = o(1).$$

If $|r - |\lambda|| \leq n^{-10}$ then there is a (deterministic) zero $\lambda_0$ such that $|r - |\lambda_0|| \leq n^{-10}$, thus $r$ lies in a set of measure $2n^{-10}$. There are $\leq n$ such sets since there are $n$ zeroes of $f$, so their union has measure $2n^{-9}$, which is infinitesimal.
Therefore
$$\mathbf E \frac{\chi(\lambda)}{|r - |\lambda||} = o(1)$$
which implies the claim.
\end{proof}

Summing up, we have
$$\frac{f'}{nf} = s_{\lambda^{(\infty)}} + o(1)$$
in $L^\infty(B(0, r))$, where $r$ is as in the previous lemma.
Pulling out the factor of $1/n$, which is harmless, we can use the argument principle to deduce that $m$ is the number of zeroes minus poles of $f'/f$; that is, the number of critical points minus zeroes of $f$.
Indeed, convergence in $L^\infty$ does commute with the argument principle, so we can throw out the infinitesimal $o(1)$.

But $a$ is infinitesimal, and we assumed that $f$ had no critical points in $\overline{D(a, 1)}$, which contains $D(0, r)$.
So $f$ has no critical points, but has a zero $a$; therefore $m \leq -1$.

In a way this part of the proof was very easy: the only tricky bit was using the cutoff to get convergence in $L^\infty$ like we needed.
The hint that we could use the argument principle was the fact that $a$ was infinitesimal, so we had control of the critical points near the origin.

\section{Convergence in $L^1_{loc}$}
Let $\nu$ be the distribution of $\lambda$ and $\nu^{(\infty)}$ of $\lambda^{(\infty)}$.
Since $\lambda - \lambda^{(\infty)}$ is infinitesimal in distribution, $\nu^{(\infty)} - \nu$ is infinitesimal in the weak topology of measures; that is, for every continuous function $g$ and compact set $K$,
$$\int_K g ~d(\nu^{(\infty)} - \nu) = o(1).$$
Now
$$s_{\lambda^{(\infty)}}(z) - s_\lambda(z) = \int_{D(0, 1)} \frac{d\nu^{(\infty)}(w)}{z - w} - \int_{D(0, 1)} \frac{d\nu(w)}{z - w}.$$
If $K$ is a compact set and $\rho$ is Lebesgue measure then
$$\int_K s_{\lambda^{(\infty)}} - s_\lambda ~d\rho = \int_K \int_{D(0, 1)} \frac{d(\nu^{(\infty)} - \nu)(w)}{z - w} ~d\rho(z).$$
By Tonelli's theorem
$$\int_K \int_{D(0, 1)} \frac{d(\nu^{(\infty)} - \nu)(w)}{|z - w|} ~d\rho(z) = \int_{D(0, 1)} \int_K \frac{d\rho(z)}{|z - w|} d(\nu^{(\infty)} - \nu)(w)$$
and the inner integral is finite since $1/|z|$ is Lebesgue integrable in codimension $2$.
So the outer integrand is a bounded continuous function, which implies that
$$\int_K \int_{D(0, 1)} \frac{d(\nu^{(\infty)} - \nu)(w)}{|z - w|} ~d\rho(z) = o(1)$$
which gives what we want when we recall
$$s_\lambda(z) = \frac{1}{n} \frac{f'(z)}{f(z)}$$
and we plug in $s_\lambda$.

\section{Case one: Outlining the proof}
The proof for case one is much longer, and is motivated by the pseudo-counterexample
$$f(z) = z^n - 1.$$
Here $a$ is an $n$th root of unity, and $f$ has no critical points on $D(a, 1)$, but does have $n - 1$ critical points at $0 \in \partial D(a, 1)$.
Similar pseudo-counterexamples hold for
$$1 - o(1) \leq a \leq 1 - \varepsilon_0^n$$
where $\varepsilon_0 > 0$ is standard. We will seek to control these examples by controlling $\zeta$ up to an error of size $O(\sigma^2) + o(1)^n$; here $\sigma^2$ is the variance of $\zeta$ and $o(1)^n$ is an infinitesimal raised to the power of $n$, thus is very small, and forces us to balance out everything in terms of $a$.

As discussed in the introduction of this post, $\zeta$ is infinitesimal in probability (and, in particular, its expected value $\mu$ is infinitesimal); thus, with overwhelming probability, the critical points of $f$ are all infinitesimals.
Combining this with the fact that $\lambda^{(\infty)}$ is uniformly distributed on $\partial D(0, 1)$, it follows that $f$ sort of looks like $f(z) = z^n - 1$.

We start with some nice bounds:
\begin{lemma}[preliminary bounds]
For any standard compact set $K \subset \mathbf C$, one has
$$f(z) = f(0) + O((|z| + o(1))^n)$$
and
$$f'(z) = O((|z| + o(1))^n)$$
uniformly in $z \in K$.
\end{lemma}
In other words, $f$ sort of grows like the translate of a homogeneous polynomial of degree $n$.

It would be nice if $\zeta$ was infinitesimal in $L^\infty$, but this isn't quite true; the following lemma is the best we can do.
\begin{lemma}[uniform convergence of $\zeta$]
There is a standard compact set
$$S = (\overline{D(0, 1)} \cap \partial D(1, 1)) \cup T$$
where $T$ is countable, standard, does not meet $\overline{D(1, 1)}$, and consists of isolated points of $S$, such that $d(\zeta, S)$ is infinitesimal in $L^\infty$.
\end{lemma}
So we think of $S$ as some sort of generalization of $0$.
Away from $S$ we have good bounds on $f,f'$:
\begin{lemma}[approximating $f,f'$ outside $S$]
For any standard compact set $K \subset \mathbf C \setminus S$:
\begin{enumerate}
\item Uniformly in $z, w \in K$,
$$f'(w) = (1 + O(n|z - w|\sigma^2|e^{o(n|z -w|)})) \frac{f'(z)}{(z - \mu)^{n-1}} (w - \mu)^{n-1}.$$
\item For every standard $\varepsilon > 0$ and uniformly in $z \in K$,
$$f(z) = f(0) + \frac{1 + O(\sigma^2)}{n} f'(z) (z - \mu) + O((\varepsilon + o(1))^n).$$
\end{enumerate}
\end{lemma}
As a consequence, we can show that every zero of $f$ which is far from $S$ is close to the level set
$$U_\zeta = \frac{1}{n} \log\frac{1}{|f(0)|}.$$
This in particular holds for $a$, since the standard part of $a$ is $1$, and $T$ does not come close to $1$ (so neither does $S$).
In fact the error term is infinitesimal:
\begin{lemma}[zeroes away from $S$]
For any standard compact set $K \subset \mathbf C \setminus S$, any standard $\varepsilon > 0$, and any zero $\lambda_0 \in K$,
$$U_\zeta(\lambda_0) = \frac{1}{n} \log \frac{1}{|f(0)|} + O(n^{-1}\sigma^2) + O((\varepsilon + o(1))^n)$$
uniformly in $\lambda_0$.
\end{lemma}
Since $a$ satisfies the hypotheses of the above lemma,
$$U_\zeta(\lambda) - U_\zeta(a) = O(n^{-1}\sigma^2 + (\varepsilon + o(1))^n)$$
is infinitesimal.
This gives us some more bounds:
\begin{lemma}[fine control]
For every standard $\varepsilon > 0$:
\begin{enumerate}
\item One has
$$\mu, 1 - a = O(\sigma^2 + (\varepsilon + o(1))^n).$$
\item For every compact set $I \subseteq \partial D(0, 1) \setminus S$ and $e^{i\theta} \in I$,
$$U_\zeta(a) - U_\zeta(e^{i\theta}) -o(\sigma^2) - o(1)^n.$$
\item For every standard smooth function $\varphi: \partial D(0, 1) \to \mathbf C$,
$$\int_0^{2\pi} \varphi(e^{i\theta}) U_\zeta(e^{i\theta}) ~d\theta = o(\sigma^2) + o(1)^n.$$
\item One has
$$U_\zeta(a) = o(\sigma)^2 + o(1)^n.$$
\end{enumerate}
\end{lemma}

Here Tao claims
$$\int_0^{2\pi} e^{-2i\theta} \log \frac{1}{|e^{i\theta} - \zeta|} ~d\theta = \frac{\pi}{2}\zeta^2.$$
Apparently this follows from Fourier inversion but I don't see it.
In any case if we take the expected value of the left-hand side we get
$$\int_0^{2\pi} \mathbf Ee^{-2i\theta} \log \frac{1}{|e^{i\theta} - \zeta|} ~d\theta = \int_0^{2\pi} e^{-2i\theta} U_\zeta(e^{i\theta}) = o(\sigma^2) + o(1)^n$$
by the fine control lemma, so
$$\mathbf E \zeta^2 = o(\sigma^2) + o(1)^n.$$
In particular this holds for the real part of $\zeta$.
Since $\sigma^2$ is infinitesimal, so are the first two moments of the real part of $\zeta$.

Since $|a - \zeta| \in [1, 2]$, one has
$$|a - \zeta| - 1 \sim \log |a - \zeta|.$$
This is true since for any $s \in [1, 2]$ one has $\log s \sim s - 1$ (which follows by Taylor expansion).
In particular,
$$|1 - \zeta| \leq |1 - a| + |a - \zeta| = 1 + O((1 - a) + \log |a - \zeta|).$$
Let $\tilde \zeta$ be the best approximation of $\zeta$ on the arc $\partial D(1, 1) \cap \overline{D(0, 1)}$, which exists since that arc is compact; then
$$|\zeta - \tilde \zeta| = O((1 - a) + \log|a - \zeta|).$$
Since $\tilde \zeta \in \partial D(1, 1)$, it has the useful property that
$$\text{arg }\zeta \in [\pi/3,\pi/2] \cup [-\pi/2, -\pi/3];$$
therefore
$$\text{Re } \tilde \zeta^2 \leq -\frac{1}{2} |\tilde \zeta|^2.$$
Plugging in the expansion for $\tilde \zeta$ we have
$$\text{Re } \zeta^2 \leq -\frac{1}{2} |\zeta|^2 + O(|\zeta|((1-a) + \log|a -\zeta|) + ((1 - a) + \log|a - \zeta|)^2).$$
We now use the inequality $2|zw| \leq |z|^2 + |w|^2$ several times. First we bound
$$\frac{1}{2} |\zeta|((1-a) + \log|a -\zeta|) \leq \frac{1}{4}|\zeta|^2 + O(\log^2 |a - \zeta|).$$
I had to think a bit about why this is legal; the point is that you can absorb the implied constant on $\zeta$ into the implied constant on $\log |a - \zeta|$ before applying the inequality.
Now we bound
$$((1 - a) + \log|a - \zeta|)^2 = (1 - a)^2 + 2(1 - a)\log|a - \zeta| + \log^2 |a - \zeta| = O((1-a)^2 + \log^2 |a - \zeta|)$$
by similar reasoning.

Thus we conclude the bound
$$\text{Re } \zeta^2 \leq - \frac{1}{4} |\zeta|^2 + O((1-a)^2 + \log^2 |a - \zeta|),$$
or in other words,
$$\mathbf E \text{Re }\zeta^2 \leq -\frac{1}{4} \mathbf E |\zeta|^2 + O((1-a)^2+ \mathbf E \log^2 |a - \zeta|).$$
Applying the fine control lemma, or more precisely the result
$$1 - a = O(\sigma^2 + (\varepsilon + o(1))^n),$$
as well as the fact that $1 - a$ is infinitesimal, we have
$$(1-a)^2 = (1 - a) O(\sigma^2 + (\varepsilon + o(1))^n) = o(\sigma^2) + o(\varepsilon + o(1))^n)$$
for every standard $\varepsilon > 0$, hence by underspill
$$(1-a)^2 = o(\sigma^2) + o(1)^n.$$

By the fine control lemma,
$$U_\zeta(a) = o(\sigma^2) + o(1)^n.$$
Thus we bound
$$\mathbf E \log^2 |a - \zeta| \leq -\mathbf E \log |a - \zeta| = U_\zeta(a) = o(\sigma^2) + o(1)^n$$
owing to the fact that $|a - \zeta| \in [1, 2]$ so that $\log |a - \zeta| \in [0, 1]$.

Plugging in the above bounds,
$$\mathbf E \text{Re }\zeta^2 \leq -\frac{1}{4} \mathbf E|\zeta|^2 + o(\sigma^2) + o(1)$$
By definition of variance we have
$$\mathbf E |\zeta|^2 - |\mu|^2 = \sigma^2$$
and $\mu$ is infinitesimal so we can spend the $o(\sigma^2)$ term as
$$\mathbf E\text{Re }\zeta^2 \leq -\frac{1+o(1)}{4} \mathbf E |\zeta|^2 + o(1)^n.$$
But the fine control lemma said $$\mathbf E\text{Re }\zeta^2 = o(\sigma^2) + o(1)^n.$$
So
$$|\mu|^2 + \sigma^2 = o(1)^n.$$
In particular,
$$o(\sigma^2) = o(1)^n$$
since $\mu$ is infinitesimal.

We used underspill to show
$$(1 - a)^2 = o(\sigma^2) + o(1)^n = o(1)^n$$
so
$$1 - \varepsilon_0^n \geq a = 1 - o(1)^n > 1 - \varepsilon_0^n$$
since $\varepsilon_0$ was standard, which implies $0 = 1$.

Next time, we'll go back and fill in all the lemmata that we skipped in the proof for case one.
This is a tricky bit -- pages 25 through 34 of Tao's paper. (For comparison, we covered pages 19 through 21, some of the exposition in pages 24 through 34, and pages 34 through 36 this time). Next time, then.


\end{document}
