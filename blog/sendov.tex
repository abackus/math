\documentclass[12pt]{article}
\usepackage[pdftex,pagebackref,letterpaper=true,colorlinks=true,pdfpagemode=none,urlcolor=blue,linkcolor=blue,citecolor=blue,pdfstartview=FitH]{hyperref}

\usepackage{amsmath,amsfonts}
\usepackage{graphicx}
\usepackage{color}

% Number systems
\newcommand{\NN}{\mathbb{N}}
\newcommand{\ZZ}{\mathbb{Z}}
\newcommand{\QQ}{\mathbb{Q}}
\newcommand{\RR}{\mathbb{R}}
\newcommand{\CC}{\mathbb{C}}
\newcommand{\PP}{\mathbb P}
\newcommand{\FF}{\mathbb F}
\newcommand{\DD}{\mathbb D}
\newcommand{\EE}{\mathbb E}
\renewcommand{\epsilon}{\varepsilon}

\newcommand{\Aut}{\operatorname{Aut}}
\newcommand{\cl}{\operatorname{cl}}
\newcommand{\ch}{\operatorname{ch}}
\newcommand{\Con}{\operatorname{Con}}
\newcommand{\coker}{\operatorname{coker}}
\newcommand{\CVect}{\CC\operatorname{-Vect}}
\newcommand{\Cantor}{\mathcal{C}}
\newcommand{\D}{\mathcal{D}}
\newcommand{\card}{\operatorname{card}}
\newcommand{\dbar}{\overline \partial}
\newcommand{\diam}{\operatorname{diam}}
\newcommand{\dom}{\operatorname{dom}}
\newcommand{\End}{\operatorname{End}}
\DeclareMathOperator*{\esssup}{ess\,sup}
\newcommand{\Hess}{\operatorname{Hess}}
\newcommand{\Hom}{\operatorname{Hom}}
\newcommand{\id}{\operatorname{id}}
\newcommand{\Ind}{\operatorname{Ind}}
\newcommand{\Inn}{\operatorname{Inn}}
\newcommand{\interior}{\operatorname{int}}
\newcommand{\lcm}{\operatorname{lcm}}
\newcommand{\mesh}{\operatorname{mesh}}
\newcommand{\LL}{\mathcal L_0}
\newcommand{\Leb}{\mathcal{L}_{\text{loc}}^2}
\newcommand{\Lip}{\operatorname{Lip}}
\newcommand{\ppic}{\vspace{35mm}}
\newcommand{\ppset}{\mathcal{P}}
\DeclareMathOperator{\proj}{proj}
\DeclareMathOperator*{\Res}{Res}
\newcommand{\Riem}{\mathcal{R}}
\newcommand{\RVect}{\RR\operatorname{-Vect}}
\newcommand{\Sch}{\mathcal{S}}
\newcommand{\sgn}{\operatorname{sgn}}
\newcommand{\spn}{\operatorname{span}}
\newcommand{\Spec}{\operatorname{Spec}}
\newcommand{\supp}{\operatorname{supp}}
\newcommand{\TT}{\mathcal T}
\DeclareMathOperator{\tr}{tr}

% Calculus of variations
\DeclareMathOperator{\pp}{\mathbf p}
\DeclareMathOperator{\zz}{\mathbf z}
\DeclareMathOperator{\uu}{\mathbf u}
\DeclareMathOperator{\vv}{\mathbf v}
\DeclareMathOperator{\ww}{\mathbf w}

% Categories
\newcommand{\Ab}{\mathbf{Ab}}
\newcommand{\Cat}{\mathbf{Cat}}
\newcommand{\Group}{\mathbf{Group}}
\newcommand{\Module}{\mathbf{Module}}
\newcommand{\Set}{\mathbf{Set}}
\DeclareMathOperator{\Fun}{Fun}
\DeclareMathOperator{\Iso}{Iso}

% Complex analysis
\renewcommand{\Re}{\operatorname{Re}}
\renewcommand{\Im}{\operatorname{Im}}

% Logic
\renewcommand{\iff}{\leftrightarrow}
\newcommand{\Henkin}{\operatorname{Henk}}
\newcommand{\PA}{\mathbf{PA}}
\DeclareMathOperator{\proves}{\vdash}

% Group
\DeclareMathOperator{\Gal}{Gal}
\DeclareMathOperator{\Fix}{Fix}
\DeclareMathOperator{\Lie}{Lie}
\DeclareMathOperator{\Out}{Out}

\DeclareMathOperator{\Diffeo}{Diffeo}

\newcommand{\GL}{\operatorname{GL}}
\newcommand{\ppGL}{\operatorname{PGL}}
\newcommand{\SL}{\operatorname{SL}}
\newcommand{\SO}{\operatorname{SO}}

% Other symbols
\newcommand{\heart}{\ensuremath\heartsuit}
\newcommand{\club}{\ensuremath\clubsuit}

\DeclareMathOperator{\atanh}{atanh}

\def\Xint#1{\mathchoice
{\XXint\displaystyle\textstyle{#1}}%
{\XXint\textstyle\scriptstyle{#1}}%
{\XXint\scriptstyle\scriptscriptstyle{#1}}%
{\XXint\scriptscriptstyle\scriptscriptstyle{#1}}%
\!\int}
\def\XXint#1#2#3{{\setbox0=\hbox{$#1{#2#3}{\int}$ }
\vcenter{\hbox{$#2#3$ }}\kern-.6\wd0}}
\def\ddashint{\Xint=}
\def\dashint{\Xint-}

\setlength{\oddsidemargin}{0pt}
\setlength{\evensidemargin}{0pt}
\setlength{\textwidth}{6.0in}
\setlength{\topmargin}{0in}
\setlength{\textheight}{8.5in}

\setlength{\parindent}{0in}
\setlength{\parskip}{5px}

%%%%%%%%% For wordpress conversion

\def\more{}

\newif\ifblog
\newif\iftex
\blogfalse
\textrue


\usepackage{ulem}
\def\em{\it}
\def\emph#1{\textit{#1}}

\def\image#1#2#3{\begin{center}\includegraphics[#1pt]{#3}\end{center}}

\let\hrefnosnap=\href

\newenvironment{btabular}[1]{\begin{tabular} {#1}}{\end{tabular}}

\newenvironment{red}{\color{red}}{}
\newenvironment{green}{\color{green}}{}
\newenvironment{blue}{\color{blue}}{}

%%%%%%%%% Typesetting shortcuts

\def\B{\{0,1\}}
\def\xor{\oplus}

\def\P{{\mathbb P}}
\def\E{{\mathbb E}}
\def\var{{\bf Var}}

\def\N{{\mathbb N}}
\def\Z{{\mathbb Z}}
\def\R{{\mathbb R}}
\def\C{{\mathbb C}}
\def\Q{{\mathbb Q}}
\def\eps{{\epsilon}}

\def\bz{{\bf z}}

\def\true{{\tt true}}
\def\false{{\tt false}}

%%%%%%%%% Theorems and proofs

\newtheorem{exercise}{Exercise}
\newtheorem{theorem}{Theorem}
\newtheorem{lemma}[theorem]{Lemma}
\newtheorem{definition}[theorem]{Definition}
\newtheorem{corollary}[theorem]{Corollary}
\newtheorem{proposition}[theorem]{Proposition}
\newtheorem{example}{Example}
\newtheorem{remark}[theorem]{Remark}
\newenvironment{proof}{\noindent {\sc Proof:}}{$\Box$ \medskip} 


\begin{document}

% --------------------------------------------------------------
%                         Start here
% --------------------------------------------------------------\

Having some more free time than usual, I figured I would read a recent paper that looked interesting.
However, I'm something of a noob at math, so I figure it's worth it to take it slowly and painstakingly think through the details.
This will be a sort of stream-of-conciousness post where I do just that.

The paper I'll be reading will be Terry Tao's ``Sendov's conjecture for sufficiently high degree polynomials."
This paper looks interesting to me because it applies ``cheap nonstandard" methods to prove a result in complex analysis.
In addition it uses probability-theoretic methods, which I'm learning a lot of right now.
Sendov's conjecture is the following:

\begin{conjecture}
Let $f: \CC \to \CC$ be a polynomial of degree $n \geq 2$ that has all zeroes $\leq 1$. If $a$ is a zero then $f'$ has a zero in $\overline{D(a, 1)}$.
\end{conjecture}

Without loss of generality, we may assume that $f$ is monic and that $a \in [0, 1]$.
Indeed, if $f$ is a polynomial in the variable $z$ we divide by the argument of $z$ and then rescale by the top coefficient.

Tao notes that by Tarski's theorem, in principle it suffices to get an upper bound $n_0$ on the degree of a counterexample and then use a computer-assisted proof to complete the argument.
I think that for every $n \leq n_0$ you'd get a formula in the theory of real closed fields that could be decided in $O(n^r)$ time, where $r$ is an absolute constant (which, unfortunately, is exponential in the number of variables of the formula, and so is probably quite large).
Worse, Tao is going to use a compactness argument and so is going to get an astronomical bound $n_0$.
Still, something to keep in mind -- computer-assisted proofs seem like the future in analysis.

More precisely, Tao proves the following:

\begin{proposition}
\label{main proposition}
For every $n$ in a monotone sequence in $\NN$, let $f$ be a monic polynomial of degree $n$ with all zeroes in $\overline{D(0, 1)}$ and let $a \in [0, 1]$ satisfy $f(a) = 0$.
If for every $n$, $f'$ has no zeroes in $\overline{D(a, 1)}$, then $0 = 1$.
\end{proposition}

It's now pretty natural to see how ``cheap nonstandard" methods apply.
One can pass to a subsequence countably many times and still preserve the hypotheses of the proposition, so by diagonalization and compactness, we may assume good (but ineffective) convergence properties.
For example, we can assume that $a = a^{(\infty)} + o(1)$, where $a^{(\infty)}$ does not depend on $n$ and $o(1)$ is with respect to $n$.

So one can view the proposition model-theoretically: it says that if $n$ is a nonstandard natural number, $f$ a monic polynomial of degree $n$ with a zero $a \in [0, 1]$, and there are no zeroes of $f'$ in $\overline{D(a, 1)}$, then $0 = 1$.
Tao never fully takes this POV, but frequently appeals to results like the following:
\begin{proposition}
Let $P$ be a first-order predicate. Then:
\begin{enumerate}
\item (Overspill) If for every sufficiently small standard $\varepsilon$, $P(\varepsilon)$, then there is an infinitesimal $\delta$ such that $P(\delta)$.
\item (Underspill) If for every infinitesimal $\delta$, $P(\delta)$, then there are arbitrarily small $\varepsilon$ such that $P(\varepsilon)$.
\item ($\aleph_0$-saturation) If $P$ is $\forall z \in K(f(z) = O(1))$ where $K \subseteq \CC$ is compact, then the implied constant in the statement of $P$ is independent of $z$.
\end{enumerate}
\end{proposition}

Henceforth we will use asymptotic notation in the nonstandard sense; for example, a quantity is $o(1)$ if it is infinitesimal. This is equivalent to the cheap nonstandard perspective where a quantity is $o(1)$ iff it is with respect to $n$, where $n$ is ranging over some monotone sequence of naturals.
I think the model-theoretic perspective is helpful here because we are going to pass to subsequences a lot, and at least in the presence of the boolean prime ideal theorem, letting $n$ be a fixed nonstandard natural number is equivalent to choosing a nonprincipal ultrafilter on $\NN$ that picks out the subsequences we are going to pass to, in the perspective where $n$ ranges over a monotone sequence of standard naturals.

This follows because the Stone-Cech compactification $\beta \NN$ is exactly the space of ultrafilters on $\NN$.
Indeed, if $n$ ranges over a monotone sequence of standard naturals, then in $\beta \NN$, $n$ converges to an element $U$ of $\beta \NN \setminus \NN$, which then is a nonprincipal ultrafilter.
If $\NN^\omega/U$ denotes the ultrapower of $\NN$ with respect to $U$, then I think the equivalence class of the sequence $\{1, 2, \dots\}$ in $\NN^\omega/U$ is exactly the limit of $n$.
Conversely, once a nonprincipal ultrafilter $U \in \beta \NN \setminus \NN$ has been fixed, we have a canonical way to pass to subsequences: only pass to a subsequence which converges to $U$.
This is possible since $\beta \NN$ is compact.

I think it will be at times convenient to go back to the ``monotone sequence of standard naturals" perspective, especially when we're doing computations, so I reserve the right to go between the two.
We'll call the monotone sequence perspective the ``cheap perspective" and the model-theoretic perspective the ``expensive perspective".

I'm not familiar with the literature on Sendov's conjecture, so I'm going to blackbox the reduction that Tao carries out.
The reduction says that, due to the Gauss-Lucas theorem and previously existing partial results on Sendov's conjecture, to prove Proposition \ref{main proposition}, it suffices to show:

\begin{proposition}
\label{main proposition 2}
Let $n$ be a nonstandard natural, let $f$ be a monic polynomial of degree $n$ with all zeroes in $\overline{D(0, 1)}$ and let $a \in [0, 1]$ satisfy $f(a) = 0$.
Suppose that $f'$ has no zeroes in $\overline{D(0, 1)}$ and
\begin{enumerate}
\item (Theorem 3.1 in Tao) either $a = o(1/\log n)$, or
\item (Theorem 5.1 in Tao) there is a standard $\varepsilon_0 > 0$ such that
$$1 - o(1) \leq a \leq 1 - \varepsilon_0^n.$$
\end{enumerate}
Then $0 = 1$.
\end{proposition}

In the former case we have $a = o(1)$ and in the latter we have $|a - 1| = o(1)$.
We'll call the former ``case zero" and the latter ``case one."

Tao gives a probabilistic proof of Proposition \ref{main proposition 2}, and that's going to be the bulk of this post and its sequels.
Let $\zeta$ be a random zero of $f'$, drawn uniformly from the finite set of zeroes.
Le $\lambda$ denote a random zero of $f$ chosen independently of $\zeta$.

In the cheap perspective, $\zeta$ depends on $n$, we are going to study properties of the convergence of $\zeta$ as $n \to \infty$, by using our chosen ultrafilter to repeatedly pass to subsequences to make $\zeta$ converge in some suitable topology.
The probability spaces that $\zeta$ lives in depend on $n$, but as long as we are interested in a deterministic limit $c$ that does not depend on $n$, this is no problem.
Indeed $\zeta$ will converge to $c$ uniformly (resp. in probability) provided that for every $\varepsilon > 0$, $|\zeta - c| \leq \varepsilon$ almost surely (resp. $\PP(|\zeta - c| \leq \varepsilon) = 1 - o(1)$), and this makes sense even though the probability space we are studying depends on $n$.
The usual definition of convergence in distribution still makes sense even for random variables $\zeta$ converges to a random variable $X$ that deos not depend on $n$ provided that their distributions $\zeta_*\PP$ converge vaguely to $X_*\PP$.

Okay, it's pretty obvious what being infinitesimally close to a deterministic standard real is in the uniform or probabilistic sense.
Expanding out the definition of the vague topology of measures, a nonstandard measure $\mu$ on a locally compact Hausdorff space $Q$ is infinitesimally close to a standard measure $\nu$ provided that for every continuous function $f$ with compact support,
$$\left|\int_Q f~d\mu - \int_Q f~d\nu\right| = o(1).$$
This induces a definition of being infinitesimally close in distribution.

Okay, no more model-theoretic games, it's time to start the actual proof.

\begin{definition}
Let $\eta$ be a bounded complex random variable. The \emph{logarithmic potential} of $\eta$ is
$$U_\eta(z) = \EE \log \frac{1}{|z - \zeta|}.$$
\end{definition}
Here $\EE$ denotes expected value.
Tao claims but does not prove that this definition makes sense for almost every $z \in \CC$.
Clearly the latter claim implies the former.
To check this, let $K$ be a compact set $\subseteq \CC$ equipped with Lebesgue measure $\mu$ and let $\nu$ be the distribution of $\zeta$. Then
$$\int_K U_\eta(z) ~d\mu(z) = \int_K \int_\CC \log \frac{1}{|z - \omega|} ~d\nu(\omega) ~d\mu(z)$$
and the integrand is singular along the set $\{z = \omega\}$, which has real codimension $2$ in $K \times \supp \nu$.
The double integral of a logarithm makes sense almost surely provided that the logarithm blows up with real codimension $2$ (to see this, check the double integral of log $1/x$ on $\RR^2$) so this looks good.

\begin{definition}
Let $\eta$ be a bounded complex random variable. The \emph{Stieltjes transform} of $\eta$ is
$$s_\eta(z) = \EE \frac{1}{z - \eta}.$$
\end{definition}
Then $s_\eta$ is ``less singular" than $U_\eta$, so this definition is inoffensive almost everywhere.

Henceforth Tao lets $\mu_\eta$ denote the distribution of $\eta$, where $\eta$ is any bounded complex random variable.
Then $s_\eta = -\partial U_\eta$ and $\mu_\eta = \overline \partial s_\eta/2\pi$ where $\partial,\overline \partial$ are the complex derivative and Cauchy-Riemann operators respectively.
Since $\partial \overline \partial = \Delta$ we have
$$2\pi\mu_n = -\Delta s_\eta.$$
This just follows straight from the definitions.
Of course $\mu_\eta$ might not be an absolutely continuous measure, so this only makes sense if we use the calculus of distributions.

Does this make sense if $\eta$ is deterministic, say $\eta = 0$ almost surely? In that case $\mu_\eta$ is a Dirac measure at $0$ and $s_\eta(z) = 1/z$.
Everything looks good, since $U_\eta(z) = -\log |z|$.

For the next claims I need the Gauss-Lucas theorem:
\begin{theorem}[Gauss-Lucas]
If $P$ is a polynomial on $\CC$, then all zeroes of $P'$ belong to the convex hull of the variety $P = 0$.
\end{theorem}

\begin{lemma}[Lemma 1.6i in Tao]
$\lambda$ surely lies in $\overline{D(0, 1)}$ and $\zeta$ surely lies in $\overline{D(0, 1)} \setminus \overline{D(a, 1)}$.
\end{lemma}
\begin{proof}
The first claim is just a tautology.
For the second, by assumption on $f$ all zeroes of $f$ lie in the convex set $\overline{D(0, 1)}$, so so does their convex hull.
In particular $\zeta$ lies in $\overline{D(0, 1)}$ almost surely by the Gauss-Lucas theorem.
Our contradiction hypothesis says that $\zeta \notin \overline{D(a, 1)}$.
\end{proof}

\begin{lemma}[Lemma 1.6ii-iv in Tao]
One has $\EE\lambda = \EE\zeta$. For almost every $z \in \CC$,
$$U_\lambda(z) = -\frac{1}{n} \log |f(z)|$$
and
$$U_\zeta(z) = -\frac{\log n}{n - 1} - \frac{1}{n - 1} \log |f'(z)|.$$
Moreover,
$$s_\lambda(z) = \frac{1}{n} \frac{f'(z)}{f(z)}$$
and
$$s_\zeta(z) = \frac{1}{n - 1} \frac{f''(z)}{f'(z)}.$$
Moreover,
$$U_\lambda(z) - \frac{n-1}{n}U_\zeta(z) = \frac{1}{n} \log |s_\lambda(z)|$$
and
$$s_\lambda(z) - \frac{n - 1}{n} s_\zeta(z) = -\frac{1}{n} \frac{s'_\lambda(z)}{s_\lambda(z)}.$$
\end{lemma}
\begin{proof}
We pass to the cheap perspective, so $n$ is a large standard natural.
Since $n$ is large, in particular $n \geq 2$, if $b_{n-1}$ is the coefficient of $z^{n-1}$ in $f$ then the roots of $f$ sum to $-b_{n-1}$.
The roots of $f'$ sum to $-(n-1)b_{n-1}/n$ by calculus. So $\EE\lambda = \EE\zeta$.

We write
$$f(z) = \prod_{j=1}^n z - \lambda_j$$
and
$$f'(z) = n\prod_{j=1}^{n-1} z - \zeta_j.$$
Taking $-\log|\cdot|$ of both sides and then dividing by $n$ we immediately get $U_\lambda$ and $U_\zeta$.
Then we take the complex derivative of both sides of the $U_\lambda$ and $U_\zeta$ formulae to get the formulae for $s_\lambda$ and $s_\zeta$.

Now the formula for $\log |s_\lambda|$ follows by subtracting the above formulae, as does the formula for $s_\lambda'/s_\lambda$.
\end{proof}

Since the distributions of $\lambda$ and $\zeta$ have bounded support (it's contained in $\overline{D(0, 1)}$) by Prokhorov's theorem we can find standard random variables $\lambda^{(\infty)}$ and $\zeta^{(\infty)}$ such that $\lambda - \lambda^{(\infty)}$ is infinitesimal in distribution and similarly for $\zeta$.
The point is that $\lambda^{(\infty)}$ and $\zeta^{(\infty)}$ give, up to an infinitesimal error, information about the behavior of $f$, $f'$, and $f''$ by the above lemma and the following proposition.

\begin{proposition}[Theorem 1.10 in Tao]
One has:
\begin{enumerate}
\item In case zero, $\lambda^{(\infty)}$ and $\zeta^{(\infty)}$ are identically distributed and almost surely lie in $C = \{e^{i\theta}: 2\theta \in [\pi, 3\pi]\}$, so $d(\lambda, C)$ is infinitesimal in probability. Moreover, for every compact set $K \subseteq \overline{D(0, 1)} \setminus C$,
$$\PP(\lambda \in K) = O\left(a + \frac{\log n}{n^{1/3}}\right).$$
\item In case one, $\lambda^{(\infty)}$ is uniformly distributed on $\partial D(0, 1)$ and $\zeta^{(\infty)}$ is almost surely zero. Moreover,
$$\EE \log \frac{1}{|\lambda|}, \EE\log |\zeta - \eta| = O\left(\frac{1}{n}\right).$$
\end{enumerate}
\end{proposition}

The proposition gives quantitative bounds that force the zeroes to all be in certain locations.
Looking ahead in the paper, it follows that:
\begin{enumerate}
\item In case zero, the Stieltjes transform of $\lambda^{(\infty)}$ is infinitesimally close to $f'/nf$, so by a stability-of-zeroes argument to show that $f$ has no zeroes near the origin, even though $a = o(1)$.
\item In case one, if $\sigma$ is the standard deviation of $\zeta$, then we have control on the zeroes of $f$ up to an error of size $o(\sigma^2) + o(1)^n$, which we can then use to deduce a contradiction.
\end{enumerate}

Next time I'll start the proof of this proposition. Its proof apparently follows from the theory of Newtonian potentials, which is not too surprising since $-\Delta \log x = \delta(x)$ if $\Delta$ is the Laplacian of $\RR^2$.
It needs the following lemma:

\begin{lemma}[Lemma 1.6vi in Tao]
If $\gamma$ is a curve in $\CC$ that misses the zeroes of $f$ and $f'$ then
$$f(\gamma(1)) = f(\gamma(0)) \exp\left(n \int_\gamma s_\lambda(z) ~dz\right)$$
and
$$f'(\gamma(1)) = f'(\gamma(0)) \exp\left((n-1) \int_\gamma s_\zeta(z) ~dz\right).$$
\end{lemma}
\begin{proof}
One has
$$ns_\lambda(z) = \frac{f'(z)}{f(z)}$$
by the previous lemma. Breaking up $\gamma$ into finitely many parts one can assume that $\gamma$ is a contractible curve in a simply connected set, in which case we have a branch of the logarithm along $\gamma$.
Now apply the fundamental theorem.
The case for $s_\zeta$ is the same.
\end{proof}






\end{document}
