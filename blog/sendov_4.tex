\documentclass[12pt]{article}
\usepackage[pdftex,pagebackref,letterpaper=true,colorlinks=true,pdfpagemode=none,urlcolor=blue,linkcolor=blue,citecolor=blue,pdfstartview=FitH]{hyperref}

\usepackage{amsmath,amsfonts}
\usepackage{graphicx}
\usepackage{color}

% Number systems
\newcommand{\NN}{\mathbb{N}}
\newcommand{\ZZ}{\mathbb{Z}}
\newcommand{\QQ}{\mathbb{Q}}
\newcommand{\RR}{\mathbb{R}}
\newcommand{\CC}{\mathbb{C}}
\newcommand{\PP}{\mathbb P}
\newcommand{\FF}{\mathbb F}
\newcommand{\DD}{\mathbb D}
\newcommand{\EE}{\mathbb E}
\renewcommand{\epsilon}{\varepsilon}

\newcommand{\Aut}{\operatorname{Aut}}
\newcommand{\cl}{\operatorname{cl}}
\newcommand{\ch}{\operatorname{ch}}
\newcommand{\Con}{\operatorname{Con}}
\newcommand{\coker}{\operatorname{coker}}
\newcommand{\CVect}{\CC\operatorname{-Vect}}
\newcommand{\Cantor}{\mathcal{C}}
\newcommand{\D}{\mathcal{D}}
\newcommand{\card}{\operatorname{card}}
\newcommand{\dbar}{\overline \partial}
\newcommand{\diam}{\operatorname{diam}}
\newcommand{\dom}{\operatorname{dom}}
\newcommand{\End}{\operatorname{End}}
\DeclareMathOperator*{\esssup}{ess\,sup}
\newcommand{\Hess}{\operatorname{Hess}}
\newcommand{\Hom}{\operatorname{Hom}}
\newcommand{\id}{\operatorname{id}}
\newcommand{\Ind}{\operatorname{Ind}}
\newcommand{\Inn}{\operatorname{Inn}}
\newcommand{\interior}{\operatorname{int}}
\newcommand{\lcm}{\operatorname{lcm}}
\newcommand{\mesh}{\operatorname{mesh}}
\newcommand{\LL}{\mathcal L_0}
\newcommand{\Leb}{\mathcal{L}_{\text{loc}}^2}
\newcommand{\Lip}{\operatorname{Lip}}
\newcommand{\ppic}{\vspace{35mm}}
\newcommand{\ppset}{\mathcal{P}}
\DeclareMathOperator{\proj}{proj}
\DeclareMathOperator*{\Res}{Res}
\newcommand{\Riem}{\mathcal{R}}
\newcommand{\RVect}{\RR\operatorname{-Vect}}
\newcommand{\Sch}{\mathcal{S}}
\newcommand{\sgn}{\operatorname{sgn}}
\newcommand{\spn}{\operatorname{span}}
\newcommand{\Spec}{\operatorname{Spec}}
\newcommand{\supp}{\operatorname{supp}}
\newcommand{\TT}{\mathcal T}
\DeclareMathOperator{\tr}{tr}

% Calculus of variations
\DeclareMathOperator{\pp}{\mathbf p}
\DeclareMathOperator{\zz}{\mathbf z}
\DeclareMathOperator{\uu}{\mathbf u}
\DeclareMathOperator{\vv}{\mathbf v}
\DeclareMathOperator{\ww}{\mathbf w}

% Categories
\newcommand{\Ab}{\mathbf{Ab}}
\newcommand{\Cat}{\mathbf{Cat}}
\newcommand{\Group}{\mathbf{Group}}
\newcommand{\Module}{\mathbf{Module}}
\newcommand{\Set}{\mathbf{Set}}
\DeclareMathOperator{\Fun}{Fun}
\DeclareMathOperator{\Iso}{Iso}

% Complex analysis
\renewcommand{\Re}{\operatorname{Re}}
\renewcommand{\Im}{\operatorname{Im}}

% Logic
\renewcommand{\iff}{\leftrightarrow}
\newcommand{\Henkin}{\operatorname{Henk}}
\newcommand{\PA}{\mathbf{PA}}
\DeclareMathOperator{\proves}{\vdash}

% Group
\DeclareMathOperator{\Gal}{Gal}
\DeclareMathOperator{\Fix}{Fix}
\DeclareMathOperator{\Lie}{Lie}
\DeclareMathOperator{\Out}{Out}

\DeclareMathOperator{\Diffeo}{Diffeo}

\newcommand{\GL}{\operatorname{GL}}
\newcommand{\ppGL}{\operatorname{PGL}}
\newcommand{\SL}{\operatorname{SL}}
\newcommand{\SO}{\operatorname{SO}}

% Other symbols
\newcommand{\heart}{\ensuremath\heartsuit}
\newcommand{\club}{\ensuremath\clubsuit}

\DeclareMathOperator{\atanh}{atanh}

\def\Xint#1{\mathchoice
{\XXint\displaystyle\textstyle{#1}}%
{\XXint\textstyle\scriptstyle{#1}}%
{\XXint\scriptstyle\scriptscriptstyle{#1}}%
{\XXint\scriptscriptstyle\scriptscriptstyle{#1}}%
\!\int}
\def\XXint#1#2#3{{\setbox0=\hbox{$#1{#2#3}{\int}$ }
\vcenter{\hbox{$#2#3$ }}\kern-.6\wd0}}
\def\ddashint{\Xint=}
\def\dashint{\Xint-}

\setlength{\oddsidemargin}{0pt}
\setlength{\evensidemargin}{0pt}
\setlength{\textwidth}{6.0in}
\setlength{\topmargin}{0in}
\setlength{\textheight}{8.5in}

\setlength{\parindent}{0in}
\setlength{\parskip}{5px}

\input{macrosblog}

\begin{document}

% --------------------------------------------------------------
%                         Start here
% --------------------------------------------------------------\

In this proof we (finally!) finish the proof of case one.

As usual, we throughout fix a nonstandard natural $n$ and a complex polynomial of degree $n$ whose zeroes are all in $\overline{D(0, 1)}$.
We assume that $a$ is a zero of $f$ whose standard part is $1$, and assume that $f$ has no critical points in $\overline{D(a, 1)}$.
Let $\lambda$ be a random zero of $f$ and $\zeta$ a random critical point.
Under these circumstances, $\lambda^{(\infty)}$ is uniformly distributed on $\partial D(0, 1)$ and $\zeta^{(\infty)}$ is almost surely zero. In particular,
$$\mathbf E \log\frac{1}{|\lambda|}, \mathbf E \log |\zeta - a| = O(n^{-1})$$
and $\zeta$ is infinitesimal in probability, hence infinitesimal in distribution.
Let $\mu$ be the expected value of $\zeta$ (thus also of $\lambda$) and $\sigma^2$ its variance.
I think we won't need the nonstandard-exponential bound $\varepsilon_0^n$ this time, as its purpose was fulfilled last time.

Last time we reduced the proof of case one to a sequence of lemmata. We now prove them.

\section{Preliminary bounds}
\begin{lemma}
Let $K \subseteq \mathbf C$ be a compact set. Then
$$f(z) - f(0), ~f'(z) = O((|z| + o(1))^n)$$
uniformly for $z \in K$.
\end{lemma}
\begin{proof}
It suffices to prove this for a compact exhaustion, and thus it suffices to assume
$$K = \overline{D(0, R)}.$$
By underspill, it suffices to show that for every standard $\varepsilon > 0$ we have
$$|f(z) - f(0)|, ~|f'(z)| \leq C(|z| + \varepsilon)^n.$$
We first give the proof for $f'$.

First suppose that $\varepsilon < |z| \leq R$.
Since $\zeta$ is infinitesimal in distribution,
$$\mathbf E \log |z - \zeta| \leq \mathbf E \log \max(|z - \zeta|, \varepsilon/2) \leq \log \max(|z|, \varepsilon/2) + o(1);$$
here we need the $\varepsilon/2$ and the $R$ since $\log |z - \zeta|$ is not a bounded continuous function of $\zeta$.
Since $\varepsilon < |z|$ we have
$$\mathbf E \log |z - \zeta| \leq \log |z| + o(1)$$
but we know that
$$-\frac{\log n}{n - 1} - \frac{1}{n - 1} \log |f'(z)| = U_\zeta(z) = -\mathbf E \log |z - \zeta|$$
so, solving for $\log |f'(z)|$, we get
$$\log |f'(z)| \leq (n - 1) \log |z| + o(n);$$
we absorbed a $\log n$ into the $o(n)$. That gives
$$|f'(z)| \leq e^{o(n)} |z|^{n-1}.$$
Since $f'$ is a polynomial of degee $n - 1$ and $f$ is monic (so the top coefficient of $f'$ is $n$) this gives a bound
$$|f'(z)| \leq e^{o(n)} (|z| + \varepsilon)^{n - 1}$$
even for $|z| \leq \varepsilon$.

Now for $f$, we use the bound
$$|f(z) - f(0)| \leq \max_{|w| < |z|} |f'(w)|$$
to transfer the above argument.
\end{proof}

\section{Uniform convergence of $\zeta$}
\begin{lemma}
There is a standard compact set $S \subseteq \overline{D(0, 1)}$ and a standard countable set $T \subseteq \overline{D(0, 1)} \setminus \overline{D(1, 1)}$ such that
$$S = (\overline{D(0, 1)} \cap \partial D(1, 1)) \cup T,$$
all elements of $T$ are isolated in $S$, and $||\zeta - S||_{L^\infty}$ is infinitesimal.
\end{lemma}
Tao claims
$$\mathbf P(|\zeta - a| \geq \frac{1}{2m}) = O(n^{-1})$$
where $m$ is a large standard natural, which makes no sense since the left-hand side should be large (and in particular, have positive standard part).
I think this is just a typo though.

\begin{proof}
Since $\zeta$ was assumed far from $a = 1 - o(1)$ we have
$$\zeta \in \overline{D(0, 1)} \setminus D(1, 1 - o(1)).$$
We also have
$$\mathbf E \log |\zeta - a| = O(n^{-1})$$
so for every standard natural $m$ there is a standard natural $k_m$ such that
$$\mathbf P(\log |\zeta - a| \geq \frac{1}{2m}) \leq \frac{k_m}{n}.$$
Multiplying both sides by $n$ we see that
$$\text{card } Z \cap K_m = \text{card } Z \cap \{\zeta_0 \in \overline{D(0, 1)}: \log |\zeta_0 - a| \geq \frac{1}{2m}\} \leq k_m$$
where $Z$ is the variety of critical points $f' = 0$.
Let $T_m$ be the set of standard parts of zeroes in $K_m$; then $T_m$ has cardinality $\leq k_m$ and so is finite.
For every zero $\zeta_0 \in Z$, either
\begin{enumerate}
\item For every $m$,
$$|\zeta_0 - a| < \exp\left(\frac{1}{2m}\right)$$
so the standard part of $|\zeta_0 - a|$ is $1$, or
\item There is an $m$ such that $d(\zeta_0, T_m)$ is infinitesimal.
\end{enumerate}
So we may set $T = \bigcup_m T_m$; then $T$ is standard and countable, and does not converge to a point in $\partial D(1, 1)$, so $S$ is standard and $||\zeta - S||_{L^\infty}$ is infinitesimal.

I was a little stumped on why $S$ is compact; Tao doesn't prove this.
It turns out it's obvious, I was just too clueless to see it.
The construction of $T$ forces that for any $\varepsilon > 0$, there are only finitely many $z \in T$ with $|z - \partial D(1, 1)| \geq \varepsilon$, so if $T$ clusters anywhere, then it can only cluster on $\partial D(1, 1)$.
This gives the desired compactness.
\end{proof}

 The above proof is basically just the proof of Ascoli's compactness theorem adopted to this setting and rephrased to replace the diagonal argument (or 👏 KEEP 👏 PASSING 👏 TO 👏 SUBSEQUENCES 👏) with the choice of a nonstandard natural.
I think the point is that, once we have chosen a nontrivial ultrafilter on $\mathbf N$, a nonstandard function is the same thing as sequence of functions, and the ultrafilter tells us which subsequences of reals to pass to.

\section{Approximating $f,f'$ outside of $S$}
We break up the approximation lemma into multiple parts.
Let $K$ be a standard compact set which does not meet $S$.
Given a curve $\gamma$ we denote its arc length by $|\gamma|$; we always assume that an arc length does exist.

A point which stumped me for a humiliatingly long time is the following:
\begin{lemma}
Let $z, w \in K$. Then there is a curve $\gamma$ from $z$ to $w$ which misses $S$ and satisfies the uniform estimate
$$|z - w| \sim |\gamma|.$$
\end{lemma}
\begin{proof}
We use the decomposition of $S$ into the arc
$$S_0 = \partial D(1, 1) \cap \overline{D(0, 1)}$$
and the discrete set $T$. We try to set $\gamma$ to be the line segment $[z, w]$ but there are two things that could go wrong.
If $[z, w]$ hits a point of $T$ we can just perturb it slightly by an error which is negligible compared to $[z, w]$.
Otherwise we might hit a point of $S_0$ in which case we need to go the long way around.
However, $S_0$ and $K$ are compact, so we have a uniform bound
$$\max(\frac{1}{|z - S_0|}, \frac{1}{|w - S_0|}) = O(1).$$
Therefore we can instead consider a curve $\gamma$ which goes all the way around $S_0$, leaving $D(0, 1)$.
This curve has length $O(1)$ for $z, w$ close to $S_0$ (and if $z, w$ are far from $S_0$ we can just perturb a line segment without generating too much error).
Using our uniform max bound above we see that this choice of $\gamma$ is valid.
\end{proof}

Recall that the moments $\mu,\sigma$ of $\zeta$ are infinitesimal.

Since $||\zeta - S||_{L^\infty}$ is infinitesimal, and $K$ is a positive distance from any infinitesimals (since it is standard compact), we have
$$|z - \zeta|, |z - \mu| \sim 1$$
uniformly in $z$.
Therefore $f$ has no critical points near $K$ and so $f''/f'$ is holomorphic on $K$.

We first need a version of the fundamental theorem.

\begin{lemma}
Let $\gamma$ be a contour in $K$ of length $|\gamma|$. Then
$$f'(\gamma(1)) = f'(\gamma(0)) \left(\frac{\gamma(1) - \mu}{\gamma(0) - \mu}\right)^{n - 1} e^{O(n) |\gamma| \sigma^2}.$$
\end{lemma}
\begin{proof}
Our bounds on $|z - \zeta|$ imply that we can take the Taylor expansion
$$\frac{1}{z - \zeta} = \frac{1}{z - \mu} + \frac{\zeta - \mu}{(z - \mu)^2} + O(|\zeta - \mu|^2)$$
of $\zeta$ in terms of $\mu$, which is uniform in $\zeta$.
Taking expectations preserves the constant term (since it doesn't depend on $\zeta$), kills the linear term, and replaces the quadratic term with a $\sigma^2$, thus
$$s_\zeta(z) = \frac{1}{z - \mu} + O(\sigma^2).$$
At the start of this series we showed
$$f'(\gamma(1)) = f'(\gamma(0)) \exp\left((n-1)\int_\gamma s_\zeta(z) ~dz\right).$$
Plugging in the Taylor expansion of $s_\zeta$ we get
$$f'(\gamma(1)) = f'(\gamma(0)) \exp\left((n-1)\int_\gamma \frac{dz}{z - \zeta}\right) e^{O(n) |\gamma| \sigma^2}.$$
Simplifying the integral we get
$$\exp\left((n-1)\int_\gamma \frac{dz}{z - \zeta}\right) = \left(\frac{\gamma(1) - \mu}{\gamma(0) - \mu}\right)^{n - 1}$$
whence the claim.
\end{proof}

\begin{lemma}
Uniformly for $z,w \in K$ one has
$$f'(w) = (1 + O(n|z - w|\sigma^2 e^{o(n|z - w|)})) \frac{(w - \mu)^{n-1}}{(z - \mu)^{n - 1}}f'(z).$$
\end{lemma}
\begin{proof}
Applying the previous two lemmata we get
$$f'(w) = e^{O(n|z - w|\sigma^2)} \frac{(w - \mu)^{n-1}}{(z - \mu)^{n - 1}}f'(z).$$
It remains to simplify
$$e^{O(n|z - w|\sigma^2)} = 1 + O(n|z - w|\sigma^2 e^{o(n|z - w|)}).$$
Taylor expanding $\exp$ and using the self-similarity of the Taylor expansion we get
$$e^z = 1 + O(|z| e^{|z|})$$
which gives that bound.
\end{proof}

\begin{lemma}
Let $\varepsilon > 0$. Then
$$f(z) = f(0) + \frac{1 + O(\sigma^2)}{n} f'(z) (z - \mu) + O((\varepsilon + o(1))^n).$$
uniformly in $z \in K$.
\end{lemma}
\begin{proof}
We may assume that $\varepsilon$ is small enough depending on $K$, since the constant in the big-$O$ notation can depend on $K$ as well, and $\varepsilon$ only appears next to implied constants.
Now given $z$ we can find $\gamma$ from $z$ to $\partial B(0, \varepsilon)$ which is always moving at a speed which is uniformly bounded from below and always moving in a direction towards the origin.
Indeed, we can take $\gamma$ to be a line segment which has been perturbed to miss the discrete set $T$, and possibly arced to miss $S_0$ (say if $z$ is far from $D(0, 1)$).
By compactness of $K$ we can choose the bounds on $\gamma$ to be not just uniform in time but also in space (i.e. in $K$), and besides that $\gamma$ is a curve through a compact set $K'$ which misses $S$.
Indeed, one can take $K'$ to be a closed ball containing $K$, and then cut out small holes in $K'$ around $T$ and $S_0$, whose radii are bounded below since $K$ is compact.
Since the moments of $\zeta$ are infinitesimal one has
$$\int_\gamma (w - \mu)^{n-1} ~dw = \frac{(z - \mu)^n}{n} - \frac{\varepsilon^n e^{in\theta}}{n} = \frac{(z - \mu)^n}{n} - O((\varepsilon + o(1))^n).$$
Here we used $\varepsilon < 1$ to enforce
$$\varepsilon^n/n = O(\varepsilon^n).$$

By the previous lemma,
$$f'(w) = (1 + O(n|z - w|\sigma^2 e^{o(n|z - w|)})) \frac{(w - \mu)^{n-1}}{(z - \mu)^{n - 1}}f'(z).$$
Integrating this result along $\gamma$ we get
$$f(\gamma(0)) = f(\gamma(1)) - \frac{f'(\gamma(0))}{(\gamma(0) - \mu)^{n-1}} \left(\int_\gamma (w - \mu)^{n-1} ~dw + O\left(n\sigma^2 \int_\gamma|\gamma(0) - w| e^{o(n|\gamma(0) - w|)}|w - \mu|^{n-1}~dw \right) \right).$$
Applying our preliminary bound, the previous paragraph, and the fact that $|\gamma(1)| = \varepsilon$, thus
$$f(\gamma(1)) = f(0) + O((\varepsilon + o(1))^n),$$
we get
$$f(z) = f(0) + O((\varepsilon + o(1))^n) - \frac{f'(z)}{(z - \mu)^{n-1}} \left(\frac{(z - \mu)^n}{n} - O((\varepsilon + o(1))^n) + O\left(n\sigma^2 \int_\gamma|z - w| e^{o(n|z - w|)}|w - \mu|^{n-1}~dw \right)\right).$$
We treat the first term first:
$$\frac{f'(z)}{(z - \mu)^{n-1}} \frac{(z - \mu)^n}{n} = \frac{1}{n} f'(z) (z - \mu).$$
For the second term, $z \in K$ while $\mu^{(\infty)} \in K$, so $|z - \mu|$ is bounded from below, whence
$$\frac{f'(z)}{(z - \mu)^{n-1}} O((\varepsilon + o(1))^n) = O((\varepsilon + o(1))^n).$$
Thus we simplify
$$f(z) = f(0) + O((\varepsilon + o(1))^n) + \frac{1}{n} f'(z) (z - \mu) + \frac{f'(z)}{(z - \mu)^{n-1}} O\left(n\sigma^2 \int_\gamma|z - w| e^{o(n|z - w|)}|w - \mu|^{n-1}~dw \right).$$
It will be convenient to instead write this as
$$f(z) = f(0) + O((\varepsilon + o(1))^n) + \frac{1}{n} f'(z) (z - \mu) + O\left(n|f'(z)|\sigma^2 \int_\gamma|z - w| e^{o(n|z - w|)} \left|\frac{w - \mu}{z - \mu}\right|^{n-1}~dw \right).$$

Now we deal with the pesky integral.
Since $\gamma$ is moving towards $\partial B(0, \varepsilon)$ at a speed which is bounded from below uniformly in ``spacetime" (that is, $K \times [0, 1]$), there is a standard $c > 0$ such that if $w = \gamma(t)$ then
$$|w - \mu| \leq |z - \mu| - ct$$
since $\gamma$ is going towards $\mu$.
(Tao's argument puzzles me a bit here because he claims that the real inner product $\langle z - w, z\rangle$ is uniformly bounded from below in spacetime, which seems impossible if $w = z$. I agree with its conclusion though.)
Exponentiating both sides we get
$$\left|\frac{w - \mu}{z - \mu}\right|^{n-1} = O(e^{-nct})$$
which bounds
$$f(z) = f(0) + O((\varepsilon + o(1))^n) + \frac{1}{n} f'(z) (z - \mu) + O\left(n|f'(z)|\sigma^2 \int_0^1 te^{-(c-o(1))nt} ~dt\right).$$
Since $c$ is standard, it dominates the infinitesimal $o(1)$, so after shrinking $c$ a little we get a new bound
$$f(z) = f(0) + O((\varepsilon + o(1))^n) + \frac{1}{n} f'(z) (z - \mu) + O\left(n|f'(z)|\sigma^2 \int_0^1 te^{-cnt} ~dt\right).$$
Since $n\int_0^1 te^{-cnt} ~dt$ is exponentially small in $n$, in particular it is smaller than $O(n^{-1})$.
Plugging in everything we get the claim.
\end{proof}

\section{Control on zeroes away from $S$}
After the gargantuan previous section, we can now show the ``approximate level set" property that we discussed last time.

\begin{lemma}
Let $K$ be a standard compact set which misses $S$ and $\varepsilon > 0$ standard. Then for every zero $\lambda_0 \in K$ of $f$,
$$U_\zeta(\lambda) = \frac{1}{n} \log \frac{1}{|f(0)|} + O(n^{-1}\sigma^2 + (\varepsilon + o(1))^n).$$
\end{lemma}
Last time we showed that this implies
$$U_\zeta(\lambda_0) = U_\zeta(a) + O(n^{-1}\sigma^2 + (\varepsilon + o(1))^n).$$
Thus all the zeroes of $f$ either live in $S$ or a neighborhood of a level set of $U_\zeta$.
\begin{proof}
Plugging in $z = \lambda_0$ in the approximation
$$f(z) = f(0) + \frac{1 + O(\sigma^2)}{n} f'(z) (z - \mu) + O((\varepsilon + o(1))^n)$$
we get
$$f(0) + \frac{1 + O(\sigma^2)}{n} f'(\lambda_0) (\lambda_0 - \mu) = O((\varepsilon + o(1))^n).$$
Several posts ago, we proved $|f(0)| \sim 1$ as a consequence of Grace's theorem, so $f(0)O((\varepsilon + o(1))^n) = O((\varepsilon + o(1))^n)$.
In particular, if we solve for $f'(\lambda_0)$ we get
$$\frac{|f'(\lambda_0)}{n} |\lambda_0 - \mu| = |f(0)| (1 + O(\sigma^2 + (\varepsilon + o(1))^n).$$
Using
$$U_\zeta(z) = -\frac{\log n}{n - 1} - \frac{1}{n - 1} \log |f'(z)|,$$
plugging in $z = \lambda_0$, and taking logarithms, we get
$$-\frac{n - 1}{n} U_\zeta(\lambda_0) + \frac{1}{n} \log | \lambda_0 - \mu| = \frac{1}{n} \log |f(0)| + O(n^{-1}\sigma^2 + (\varepsilon + o(1))^n).$$
Now $\lambda_0 \in K$ and $K$ misses the standard compact set $S$, so since $0 \in S$ we have
$$|\lambda - \zeta|, |\lambda - \mu| \sim 1$$
(since $\zeta^{(\infty)} \in S$ and $\mu$ is infinitesimal).
So we can Taylor expand in $\zeta$ about $\mu$:
$$\log |\lambda_0 - \zeta| = \log |\lambda_0 - \mu| - \text{Re }\frac{\zeta - \mu}{\lambda_0 - \mu} + O(\sigma^2).$$
Taking expectations and using $\mathbf E \zeta - \mu$,
$$-U_\zeta(\lambda_0) = \log |\lambda_0 - \mu| + O(\sigma^2).$$
Plugging in $\log |\lambda_0 - \mu|$ we see the claim.
\end{proof}

I'm not sure who originally came up with the idea to reason like this; I think Tao credits M. J. Miller.
Whoever it was had an interesting idea, I think: $f = 0$ is a level set of $f$, but one that a priori doesn't tell us much about $f'$.
We have just replaced it with a level set of $U_\zeta$, a function that is explicitly closely related to $f'$, but at the price of an error term.

\section{Fine control}
We finish this series.
If you want, you can let $\varepsilon > 0$ be a standard real.
I think, however, that it will be easier to think of $\varepsilon$ as ``infinitesimal, but not as infinitesimal as the term of the form o(1)".
In other words, $1/n$ is smaller than any positive element of the ordered field $\mathbf R(\varepsilon)$; briefly, $1/n$ is infinitesimal with respect to $\mathbf R(\varepsilon)$.
We still reserve $o(1)$ to mean an infinitesimal with respect to $\mathbf R(\varepsilon)$.
Now $\varepsilon^n = o(1)$ by underspill, since this is already true if $\varepsilon$ is standard and $0 < \varepsilon < 1$.
Underspill can also be used to transfer facts at scale $\varepsilon$ to scale $1/n$.
I think you can formalize this notion of ``iterated infinitesimals" by taking an \emph{iterated} ultrapower of $\mathbf R$ in the theory of ordered rings.

Let us first bound $\log |a|$. Recall that $|a| \leq 1$ so $\log |a| \leq 0$ but in fact we can get a sharper bound.
Since $T$ is discrete we can get $e^{-i\theta}$ arbitrarily close to whatever we want, say $-1$ or $i$.
This will give us bounds on $1 - a$ when we take the Taylor expansion
$$\log|a| = -(1 - a)(1 + o(1)).$$
\begin{lemma}
Let $e^{i\theta} \in \partial D(0, 1) \setminus S$ be standard. Then
$$\log |a| \leq \text{Re } ((1 - e^{-i\theta} + o(1))\mu) - O(|\mu|^2 + \sigma^2 + (\varepsilon + o(1))^n).$$
\end{lemma}
\begin{proof}
Let $K$ be a standard compact set which misses $S$ and $\lambda_0 \in K$ a zero of $f$.
Since $\zeta \notin K$ (since $S$ is close to $\zeta$) and $|a-\zeta|$ has positive standard part (since $d(a, S) = 1$) we can take Taylor expansions
$$-\log |\lambda_0 - \zeta| = -\log |\lambda_0| + \text{Re } \frac{\zeta}{\lambda_0} + O(|\zeta|^2)$$
and
$$-\log |a - \zeta| = -\log|a| + \text{Re } \frac{\zeta}{a} + O(|\zeta|^2)$$
in $\zeta$ about $0$. Taking expectations we have
$$U_\zeta(\lambda_0) = -\log |\lambda_0| + \text{Re } \frac{\mu}{\lambda_0} + O(\mathbf E |\zeta|^2)$$
and similarly for $a$. Thus
$$-\log |a| + \text{Re } \frac{\mu}{a} = -\log |\lambda_0| + \text{Re } \frac{\mu}{\lambda_0} + O(\mathbf E |\zeta|^2 + n^{-1}\sigma^2 + (\varepsilon + o(1))^n)$$
since
$$U_\zeta(\lambda_0) - U_\zeta(a) = O(n^{-1}\sigma^2 + (\varepsilon + o(1))^n).$$
Since
$$\mathbf E|\zeta|^2 = |\mu|^2 + \sigma^2$$
we have
$$-\log|\lambda_0| + \text{Re } \left(\frac{1}{\lambda_0} - \frac{1}{a}\right)\mu = -\log|a| + O(|\mu|^2 + \sigma^2 + (\varepsilon + o(1))^n).$$
Now $|\lambda_0| \leq 1$ so $-\log |\lambda_0| \geq 0$, whence
$$\text{Re } \left(\frac{1}{\lambda_0} - \frac{1}{a}\right)\mu \geq -\log|a| + O(|\mu|^2 + \sigma^2 + (\varepsilon + o(1))^n).$$
Now recall that $\lambda^{(\infty)}$ is uniformly distributed on $\partial D(0, 1)$, so we can choose $\lambda_0$ so that
$$|\lambda_0 - e^{i\theta}| = o(1).$$
Thus
$$\frac{1}{\lambda_0} - \frac{1}{a} = 1 - e^{-i\theta} + o(1)$$
which we can plug in to get the claim.
\end{proof}

Now we prove the first part of the fine control lemma.
\begin{lemma}
One has
$$\mu, 1 - a = O(\sigma^2 + (\varepsilon + o(1))^n).$$
\end{lemma}
\begin{proof}
Let $\theta_+ \in [0.98\pi, 0.99\pi],\theta_- \in [1.01\pi, 1.02\pi]$ be standard reals such that $e^{i\theta_\pm} \notin S$.
I don't think the constants here actually matter; we just need $0 < 0.01 < 0.02 < \pi/8$ or something.
Anyways, summing up two copies of the inequality from the previous lemma with $\theta = \theta_\pm$ we have
$$1.9 \text{Re } \mu \geq \text{Re } ((1 + e^{-i\theta_+} + 1 + e^{-i\theta_-} + o(1))\mu) \geq \log |a| + O(|\mu|^2 + \sigma^2 + (\varepsilon + o(1))^n)$$
since
$$2 + e^{-i\theta_+} + e^{-i\theta_-} + o(1) \leq 1.9.$$
That is,
$$\text{Re } \mu \geq \frac{\log|a|}{1.9} + O(|\mu|^2 + \sigma^2 + (\varepsilon + o(1))^n).$$
Indeed,
$$-\log |a| = (1 - a)(1 + o(1)),$$
so
$$\text{Re }\mu \geq -\frac{1 - a}{1.9 + o(1)} + O(|\mu|^2 + \sigma^2 + (\varepsilon + o(1))^n).$$

If we square the tautology $|\zeta - a| \geq 1$ then we get
$$|\zeta|^2 - 2a \text{Re }\zeta + a^2 \geq 1.$$
Taking expected values we get
$$|\mu|^2 + \sigma^2 - 2a \text{Re }\mu + a^2 \geq 1$$
or in other words
$$\text{Re }\mu \leq -\frac{1 - a^2}{2a} + O(|\mu|^2 + \sigma^2) = -(1 - a)(1 + o(1)) + O(|\mu|^2 + \sigma^2)$$
where we used the Taylor expansion
$$\frac{1 - a^2}{2a} = (1 - a)(1 + o(1))$$
obtained by Taylor expanding $1/a$ about $1$ and applying $1 - a = o(1)$.
Using
$$\text{Re }\mu \geq -\frac{1 - a}{1.9 + o(1)} + O(|\mu|^2 + \sigma^2 + (\varepsilon + o(1))^n)$$
we get
$$-\frac{1 - a}{1.9 + o(1)} + O(|\mu|^2 + \sigma^2 + (\varepsilon + o(1))^n) \leq \text{Re }\mu \leq -(1 - a)(1 + o(1)) + O(|\mu|^2 + \sigma^2)$$
Thus
$$(1 - a)\left(1 + \frac{1}{1.9 + o(1)} + o(1)\right) = O(|\mu|^2 + \sigma^2 + (\varepsilon + o(1))^n).$$
Dividing both sides by $1 + \frac{1}{1.9 + o(1)} + o(1) \in [1, 2]$ we have
$$1 - a = O(|\mu|^2 + \sigma^2 + (\varepsilon + o(1))^n).$$
In particular
$$\text{Re }\mu = O(|\mu|^2 + \sigma^2 + (\varepsilon + o(1))^n)(1 + o(1)) + O(|\mu|^2 + \sigma^2) = O(|\mu|^2 + \sigma^2 + (\varepsilon + o(1))^n).$$

Now we treat the imaginary part of $\text{Im } \mu$.
The previous lemma gave
$$\text{Re } ((1 - e^{-i\theta} + o(1))\mu) - \log |a| = O(|\mu|^2 + \sigma^2 + (\varepsilon + o(1))^n).$$
Writing everything in terms of real and imaginary parts we can expand out
$$\text{Re } ((1 - e^{-i\theta} + o(1))\mu) = (\sin \theta + o(1))\text{Re } \mu + (1 - \cos \theta + o(1))\text{Re }\mu.$$
Using the bounds
$$(1 - \cos \theta + o(1))\text{Re }\mu, ~\log |a| = O(|\mu|^2 + \sigma^2 + (\varepsilon + o(1))^n)$$
(Which follow from the previous paragraph and the bound $\log |a| = O(1 - a)$), we have
$$(\sin \theta + o(1))\text{Im } \mu = O(|\mu|^2 + \sigma^2 + (\varepsilon + o(1))^n).$$
Since $T$ is discrete we can find $\theta$ arbitrarily close to $\pm \pi/2$ which meets the hypotheses of the above equation.
Therefore
$$\text{Im } \mu = O(|\mu|^2 + \sigma^2 + (\varepsilon + o(1))^n).$$
Pkugging everything in, we get
$$1 - a \sim \mu = O(|\mu|^2 + \sigma^2 + (\varepsilon + o(1))^n).$$
Now $|\mu|^2 = o(|\mu|)$ since $\mu$ is infinitesimal; therefore we can discard that term.
\end{proof}

Now we are ready to prove the second part.
The point is that we are ready to dispose of the semi-infinitesimal $\varepsilon$.
Doing so puts a lower bound on $U_\zeta(a)$.
\begin{lemma}
Let $I \subseteq \partial D(0, 1) \setminus S$ be a standard compact set. Then for every $e^{i\theta} \in I$,
$$U_\zeta(a) - U_\zeta(e^{i\theta}) \geq -o(\sigma^2) - o(1)^n.$$
\end{lemma}
\begin{proof}
Since $\lambda^{(\infty)}$ is uniformly distributed on $\partial D(0, 1)$, there is a zero $\lambda_0$ of $f$ with $|\lambda_0 - e^{i\theta}| = o(1)$.
Since $|\lambda_0| \leq 1$, we can find an infinitesimal $\eta$ such that
$$\lambda_0 = e^{i\theta}(1 - \eta)$$
and $|1 - \eta| \leq 1$.
In the previous section we proved
$$U_\zeta(a) - U_\zeta(\lambda_0) = O(n^{-1}\sigma^2) + (\varepsilon + o(1))^n).$$
Using $n^{-1} = o(1)$ and plugging in $\lambda_0$ we have
$$U_\zeta(a) - U_\zeta(e^{i\theta}(1 - \eta)) = o(\sigma^2) + O((\varepsilon + o(1))^n).$$
Now
$$\text{Re } \eta \int_0^1 \frac{dt}{1 - t\eta + e^{-i\theta}\zeta} = \log |1 - e^{-i\theta}\zeta| - \log|1 - \eta - e^{-i\theta}\zeta| = \log|e^{i\theta} - \zeta| - \log|e^{i\theta} - e^{i\theta}\eta - \zeta|.$$
Taking expectations,
$$\text{Re }\eta \mathbf E\int_0^1 \frac{dt}{1 - t\eta + e^{-i\theta}\zeta} = U_\zeta(e^{i\theta}(1 - \eta)) - U_\zeta(e^{i\theta}).$$
Taking a Taylor expansion,
$$\frac{1}{1 - t\eta - e^{-i\theta}\zeta} = \frac{1}{1 - t\eta} + \frac{e^{-i\theta}\zeta}{(1 - t\eta)^2} + O(|\zeta|^2)$$
so by Fubini's theorem
$$\mathbf E\int_0^1 \frac{dt}{1 - t\eta + e^{-i\theta}\zeta} = \int_0^1 \left(\frac{1}{1 - t\eta} + \frac{e^{-i\theta}}{(1 - t\eta)^2}\mu + O(|\mu|^2 + \sigma^2)\right)~dt;$$
using the previous lemma and $\eta = o(1)$ we get
$$ U_\zeta(e^{i\theta}(1 - \eta)) - U_\zeta(e^{i\theta}) = \text{Re }\eta \int_0^1 \frac{dt}{1 - t\eta} + o(\sigma^2) + O((\varepsilon + o(1))^n).$$
We also have
$$\text{Re } \eta \int_0^1 \frac{dt}{1 - t\eta} = -\log \frac{1}{e^{i\theta} - e^{i\theta}\eta} = U_0(1 - \eta)$$
since $0$ is deterministic (and $U_0(e^{i\theta} z) = U_0(z)$, and $U_0(1) = 0$; very easy to check!)
I think Tao makes a typo here, referring to $U_i(e^{i\theta}(1 - \eta))$, which seems irrelevant. We do have
$$U_0(1 - \eta) = -\log|1 - \eta| \geq 0$$
since $|1 - \eta| \leq 0$. Plugging in
$$\text{Re } \eta \int_0^1 \frac{dt}{1 - t\eta} \geq 0$$
we get
$$U_\zeta(e^{i\theta} - e^{i\theta}\eta) - U_\zeta(e^{i\theta}) \geq -o(\sigma^2) - O((\varepsilon + o(1))^n).$$
I think Tao makes another typo, dropping the Big O, but anyways,
$$U_\zeta(a) - U_\zeta(e^{i\theta} - e^{i\theta}\eta) = o(\sigma^2) - O((\varepsilon + o(1))^n)$$
so by the triangle inequality
$$U_\zeta(a) - U_\zeta(e^{i\theta}) \geq -o(\sigma^2) - O((\varepsilon + o(1))^n).$$
By underspill, then, we can take $\varepsilon \to 0$.
\end{proof}

We need a result from complex analysis called Jensen's formula which I hadn't heard of before.
\begin{theorem}[Jensen's formula]
Let $g: D(0, 1) \to \mathbf C$ be a holomorphic function with zeroes $a_1, \dots, a_n \in D(0, 1)$ and $g(0) \neq 0$. Then
$$\log |g(0)| = \sum_{j=1}^n \log |a_j| + \frac{1}{2\pi} \int_0^{2\pi} \log |g(e^{i\theta})| ~d\theta.$$
\end{theorem}
In hindsight this is kinda trivial but I never realized it. In fact $\log |g|$ is subharmonic and in fact its Laplacian is exactly a linear combination of delta functions at each of the zeroes of $g$. If you subtract those away then this is just the mean-value property
$$\log |g(0)| = \frac{1}{2\pi} \int_0^{2\pi} \log |g(e^{i\theta})| ~d\theta.$$

Let us finally prove the final part.
In what follows, implied constants are allowed to depend on $\varphi$ but not on $\delta$.

\begin{lemma}
For any standard $\varphi \in C^\infty(\partial D(0, 1))$,
$$\int_0^{2\pi} \varphi(e^{i\theta}) U_\zeta(e^{i\theta}) ~d\theta = o(\sigma^2) + o(1)^n.$$
Besides,
$$U_\zeta(a) = o(\sigma^2) + o(1)^n.$$
\end{lemma}
\begin{proof}
Let $m$ be the Haar measure on $\partial D(0, 1)$.
We first prove this when $\varphi \geq 0$.
Since $T$ is discrete and $\partial D(0, 1)$ is compact, for any standard (or semi-infinitesimal) $\delta > 0$, there is a standard compact set
$$I \subseteq \partial D(0, 1) \setminus S$$
such that
$$m(\partial D(0, 1) \setminus I) < \delta.$$
By the previous lemma, if $e^{i\theta} \in I$ then
$$\varphi(e^{i\theta}) U_\zeta(a) - \varphi(e^{i\theta}) U_\zeta(e^{i\theta}) \geq -o(\sigma^2) - o(1)^n$$
and the same holds when we average in Haar measure:
$$ U_\zeta(a)\int_I \varphi~dm - \int_I \varphi(e^{i\theta}) U_\zeta(e^{i\theta})~dm(e^{i\theta}) \geq -o(\sigma^2) - o(1)^n.$$

We have
$$|\log |e^{i\theta} - \zeta| + \text{Re } e^{-i\theta}\zeta| \leq |\log|3 - \zeta| + 3\text{Re } \zeta| \in L^2(dm(e^{i\theta}))$$
so, using the Cauchy-Schwarz inequality, one has
$$\int_{\partial D(0, 1) \setminus I} \varphi(e^{i\theta}) (\log |e^{i\theta} - \zeta| + \text{Re } e^{-i\theta}\zeta) ~dm(e^{i\theta}) = \sqrt{\int_I |\log |e^{i\theta} - \zeta| + \text{Re } e^{-i\theta}\zeta|} = O(\delta^{1/2}).$$
Meanwhile, if $|\zeta| \leq 1/2$ then the fact that
$$\log |e^{i\theta} - \zeta| = \text{Re }-\frac{\zeta}{e^{i\theta}} + O(|\zeta|^2)$$
implies
$$\log |e^{i\theta} - \zeta| + \text{Re } \frac{\zeta}{e^{i\theta}} = O(|\zeta|^2)$$
and hence
$$\int_{\partial D(0, 1) \setminus I} \varphi(e^{i\theta}) (\log |e^{i\theta} - \zeta| + \text{Re } e^{-i\theta}\zeta) ~dm(e^{i\theta}) = O(\delta|\zeta|^2).$$
We combine these into the unified estimate
$$\int_{\partial D(0, 1) \setminus I} \varphi(e^{i\theta}) (\log |e^{i\theta} - \zeta| + \text{Re } e^{-i\theta}\zeta) ~dm(e^{i\theta}) = O(\delta^{1/2}|\zeta|^2)$$
valid for all $|\zeta| \leq 1$, hence almost surely.
Taking expected values we get
$$\int_{\partial D(0, 1) \setminus I} \varphi(e^{i\theta})U_\zeta(e^{i\theta}) + \varphi(e^{i\theta}) \text{Re }e^{-i\theta}\mu ~dm(e^{i\theta}) = O(\delta^{1/2}(|\mu|^2 + \sigma^2)) + o(\sigma^2) + o(1)^n.$$
In the last lemma we bounded $|\mu|$ so we can absorb all the terms with $\mu$ in them to get
$$\int_{\partial D(0, 1) \setminus I} \varphi(e^{i\theta})U_\zeta(e^{i\theta}) ~dm(e^{i\theta}) = O(\delta^{1/2}\sigma^2) + o(\sigma^2) + o(1)^n.$$
We also have
$$\int_{\partial D(0, 1) \setminus I} \varphi ~dm = O(\delta)$$
(here Tao refers to a mysterious undefined measure $\sigma$ but I'm pretty sure he means $m$).
Putting these integrals together with the integrals over $I$,
$$\ U_\zeta(a)int_{\partial D(0, 1)} \varphi ~dm - \int_{\partial D(0, 1)} \varphi(e^{i\theta}) U_\zeta(e^{i\theta}) ~dm(e^{i\theta}) \geq -O(\delta^{1/2}\sigma^2) - o(\sigma^2) - o(1)^n.$$
By underspill we can delete $\delta$, thus
$$ U_\zeta(a)\int_{\partial D(0, 1)} \varphi ~dm - \int_{\partial D(0, 1)} \varphi(e^{i\theta}) U_\zeta(e^{i\theta}) ~dm(e^{i\theta}) \geq - o(\sigma^2) - o(1)^n.$$

We now consider the specific case $\varphi = 1$. Then
$$U_\zeta(a) - \int_{\partial D(0, 1)} U_\zeta ~dm \geq -o(\sigma^2) - o(1)^n.$$
Now Tao claims and doesn't prove
$$\int_{\partial D(0, 1)} U_\zeta ~dm = 0.$$
To see this, we expand as
$$\int_{\partial D(0, 1)} U_\zeta ~dm = -\mathbf E \frac{1}{2\pi} \int_0^{2\pi} \log|\zeta - e^{i\theta}| ~d\theta$$
using Fubini's theorem. Now we use Jensen's formula with $g(z) = \zeta - z$, which has a zero exactly at $\zeta$.
This seems problematic if $\zeta = 0$, but we can condition on $|\zeta| > 0$. Indeed, if $\zeta = 0$ then we have
$$ \int_0^{2\pi} \log|\zeta - e^{i\theta}| ~d\theta = \int_0^{2\pi} \log 1 ~d\theta = 0$$
which already gives us what we want. Anyways, if $|\zeta| > 0$, then by Jensen's formula,
$$\frac{1}{2\pi} \int_0^{2\pi} \log|\zeta - e^{i\theta}| ~d\theta = \log |\zeta| - \log |\zeta| = 0.$$
So that's how it is.
Thus we have
$$-U_\zeta(a) \leq o(\sigma^2) + o(1)^n.$$
Since $|a - \zeta| \geq 1$, $\log |a - \zeta| \geq 0$, so the same is true of its expected value $-U_\zeta(a)$.
This gives the desired bound
$$U_\zeta(a) = o(\sigma^2) + o(1)^n.$$

We can use that bound to discard $U_\zeta(a)$ from the average
$$ U_\zeta(a)\int_{\partial D(0, 1)} \varphi ~dm - \int_{\partial D(0, 1)} \varphi(e^{i\theta}) U_\zeta(e^{i\theta}) ~dm(e^{i\theta}) \geq - o(\sigma^2) - o(1)^n,$$
thus
$$\int_{\partial D(0, 1)} \varphi(e^{i\theta}) U_\zeta(e^{i\theta}) ~dm(e^{i\theta})= o(\sigma^2) + o(1)^n.$$
Repeating the Jensen's formula argument from above we see that we can replace $\varphi$ with $\varphi - k$ for any $k \geq 0$.
So this holds even if $\varphi$ is not necessarily nonnegative.
\end{proof}



\end{document}
