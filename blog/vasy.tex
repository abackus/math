\documentclass[12pt]{article}
\usepackage[pdftex,pagebackref,letterpaper=true,colorlinks=true,pdfpagemode=none,urlcolor=blue,linkcolor=blue,citecolor=blue,pdfstartview=FitH]{hyperref}

\usepackage{amsmath,amsfonts}
\usepackage{graphicx}
\usepackage{color}

% Number systems
\newcommand{\NN}{\mathbb{N}}
\newcommand{\ZZ}{\mathbb{Z}}
\newcommand{\QQ}{\mathbb{Q}}
\newcommand{\RR}{\mathbb{R}}
\newcommand{\CC}{\mathbb{C}}
\newcommand{\PP}{\mathbb P}
\newcommand{\FF}{\mathbb F}
\newcommand{\DD}{\mathbb D}
\newcommand{\EE}{\mathbb E}
\renewcommand{\epsilon}{\varepsilon}

\newcommand{\Aut}{\operatorname{Aut}}
\newcommand{\cl}{\operatorname{cl}}
\newcommand{\ch}{\operatorname{ch}}
\newcommand{\Con}{\operatorname{Con}}
\newcommand{\coker}{\operatorname{coker}}
\newcommand{\CVect}{\CC\operatorname{-Vect}}
\newcommand{\Cantor}{\mathcal{C}}
\newcommand{\D}{\mathcal{D}}
\newcommand{\card}{\operatorname{card}}
\newcommand{\dbar}{\overline \partial}
\newcommand{\diam}{\operatorname{diam}}
\newcommand{\dom}{\operatorname{dom}}
\newcommand{\End}{\operatorname{End}}
\DeclareMathOperator*{\esssup}{ess\,sup}
\newcommand{\Hess}{\operatorname{Hess}}
\newcommand{\Hom}{\operatorname{Hom}}
\newcommand{\id}{\operatorname{id}}
\newcommand{\Ind}{\operatorname{Ind}}
\newcommand{\Inn}{\operatorname{Inn}}
\newcommand{\interior}{\operatorname{int}}
\newcommand{\lcm}{\operatorname{lcm}}
\newcommand{\mesh}{\operatorname{mesh}}
\newcommand{\LL}{\mathcal L_0}
\newcommand{\Leb}{\mathcal{L}_{\text{loc}}^2}
\newcommand{\Lip}{\operatorname{Lip}}
\newcommand{\ppic}{\vspace{35mm}}
\newcommand{\ppset}{\mathcal{P}}
\DeclareMathOperator{\proj}{proj}
\DeclareMathOperator*{\Res}{Res}
\newcommand{\Riem}{\mathcal{R}}
\newcommand{\RVect}{\RR\operatorname{-Vect}}
\newcommand{\Sch}{\mathcal{S}}
\newcommand{\sgn}{\operatorname{sgn}}
\newcommand{\spn}{\operatorname{span}}
\newcommand{\Spec}{\operatorname{Spec}}
\newcommand{\supp}{\operatorname{supp}}
\newcommand{\TT}{\mathcal T}
\DeclareMathOperator{\tr}{tr}

% Calculus of variations
\DeclareMathOperator{\pp}{\mathbf p}
\DeclareMathOperator{\zz}{\mathbf z}
\DeclareMathOperator{\uu}{\mathbf u}
\DeclareMathOperator{\vv}{\mathbf v}
\DeclareMathOperator{\ww}{\mathbf w}

% Categories
\newcommand{\Ab}{\mathbf{Ab}}
\newcommand{\Cat}{\mathbf{Cat}}
\newcommand{\Group}{\mathbf{Group}}
\newcommand{\Module}{\mathbf{Module}}
\newcommand{\Set}{\mathbf{Set}}
\DeclareMathOperator{\Fun}{Fun}
\DeclareMathOperator{\Iso}{Iso}

% Complex analysis
\renewcommand{\Re}{\operatorname{Re}}
\renewcommand{\Im}{\operatorname{Im}}

% Logic
\renewcommand{\iff}{\leftrightarrow}
\newcommand{\Henkin}{\operatorname{Henk}}
\newcommand{\PA}{\mathbf{PA}}
\DeclareMathOperator{\proves}{\vdash}

% Group
\DeclareMathOperator{\Gal}{Gal}
\DeclareMathOperator{\Fix}{Fix}
\DeclareMathOperator{\Lie}{Lie}
\DeclareMathOperator{\Out}{Out}

\DeclareMathOperator{\Diffeo}{Diffeo}

\newcommand{\GL}{\operatorname{GL}}
\newcommand{\ppGL}{\operatorname{PGL}}
\newcommand{\SL}{\operatorname{SL}}
\newcommand{\SO}{\operatorname{SO}}

% Other symbols
\newcommand{\heart}{\ensuremath\heartsuit}
\newcommand{\club}{\ensuremath\clubsuit}

\DeclareMathOperator{\atanh}{atanh}

\def\Xint#1{\mathchoice
{\XXint\displaystyle\textstyle{#1}}%
{\XXint\textstyle\scriptstyle{#1}}%
{\XXint\scriptstyle\scriptscriptstyle{#1}}%
{\XXint\scriptscriptstyle\scriptscriptstyle{#1}}%
\!\int}
\def\XXint#1#2#3{{\setbox0=\hbox{$#1{#2#3}{\int}$ }
\vcenter{\hbox{$#2#3$ }}\kern-.6\wd0}}
\def\ddashint{\Xint=}
\def\dashint{\Xint-}

\setlength{\oddsidemargin}{0pt}
\setlength{\evensidemargin}{0pt}
\setlength{\textwidth}{6.0in}
\setlength{\topmargin}{0in}
\setlength{\textheight}{8.5in}

\setlength{\parindent}{0in}
\setlength{\parskip}{5px}

%%%%%%%%% For wordpress conversion

\def\more{}

\newif\ifblog
\newif\iftex
\blogfalse
\textrue


\usepackage{ulem}
\def\em{\it}
\def\emph#1{\textit{#1}}

\def\image#1#2#3{\begin{center}\includegraphics[#1pt]{#3}\end{center}}

\let\hrefnosnap=\href

\newenvironment{btabular}[1]{\begin{tabular} {#1}}{\end{tabular}}

\newenvironment{red}{\color{red}}{}
\newenvironment{green}{\color{green}}{}
\newenvironment{blue}{\color{blue}}{}

%%%%%%%%% Typesetting shortcuts

\def\B{\{0,1\}}
\def\xor{\oplus}

\def\P{{\mathbb P}}
\def\E{{\mathbb E}}
\def\var{{\bf Var}}

\def\N{{\mathbb N}}
\def\Z{{\mathbb Z}}
\def\R{{\mathbb R}}
\def\C{{\mathbb C}}
\def\Q{{\mathbb Q}}
\def\eps{{\epsilon}}

\def\bz{{\bf z}}

\def\true{{\tt true}}
\def\false{{\tt false}}

%%%%%%%%% Theorems and proofs

\newtheorem{exercise}{Exercise}
\newtheorem{theorem}{Theorem}
\newtheorem{lemma}[theorem]{Lemma}
\newtheorem{definition}[theorem]{Definition}
\newtheorem{corollary}[theorem]{Corollary}
\newtheorem{proposition}[theorem]{Proposition}
\newtheorem{example}{Example}
\newtheorem{remark}[theorem]{Remark}
\newenvironment{proof}{\noindent {\sc Proof:}}{$\Box$ \medskip} 


\begin{document}

% --------------------------------------------------------------
%                         Start here
% --------------------------------------------------------------\

In this post I would like to walk through a paper of Zworski, ``Resonances for asymptotically hyperbolic manifolds: Vasy's method revisited", which discusses a proof technique of Vasy for showing that the resolvent of a Laplace-Beltrami operator extends by meromorphic continuation to the entirety of $\mathbf C$.

\begin{definition}
An \emph{even asymptotically hyperbolic manifold} is a Riemannian manifold $(M, g)$ is the interior of a compact manifold $\overline M$ with nonempty boundary such that there is a neighborhood $U$ of $\partial M$ and a diffeomorphism
$$y: U \to [0, 1] \times \partial M$$
such that $\partial M = \{y = 0\}$, $dy_1$ is nonzero on $\partial M$, and there are Riemannian metrics $h(y_1^2)$ on $\partial M$ such that
$$g = \frac{dy_1^2 + h(y_1^2)}{y_1^2}$$
on $U$.
\end{definition}

Suppose $(M, g)$ is a Riemannian manifold.
We write $|\cdot|$ to mean the norm induced on the tangent and cotangent spaces of $M$ by $g$.
Moreover, $g$ induces a volume form and hence a measure on $M$.
We denote by $L^p(g)$ and $H^s(g)$ the spaces one obtains from the measure induced by $g$.

The point of the above definition is that near $\partial M$, the manifold looks like a perturbation of hyperbolic space. Indeed, in the coordinates given by the upper half-space model of hyperbolic space, one has
$$g = \frac{dy_1^2 + (dy')^2}{y_1^2}$$
where $y = (y_1, y') \in [0, \infty) \times \RR^n$.
Moreover, since $dy_1$ is nonzero on $\partial M$, $y$ can be chosen so that $|dy_1| = 1$ on $\partial M$, i.e. $\nabla y_1$ is something like the inner unit normal field of $\overline M$.

\begin{definition}
Let $(M, g)$ be a Riemannian manifold. The \emph{Laplace-Beltrami operator} $-\Delta_g$ of $(M, g)$ is the pseudodifferential operator on $M$ with symbol $|\xi|^2$.
\end{definition}

Suppose that the dimension of $\partial M$ is $n$.
The spectrum of $-\Delta_g$ is contained in $[0, \infty)$, so $-\Delta_g - \zeta(n - \zeta)$ is invertible on $H^2(g)$ if $\text{Re } \zeta > n$, so there exists a resolvent $R(\zeta): L^2(g) \to H^2(g)$.
Let $\overline C^\infty(M)$ denote the space of smooth functions which extend across $\partial M$ and $\dot C^\infty(M)$ denote the space of smooth functions which extend across $\partial M$ to a function with support in $M$.
Then by elliptic regularity, $R(\zeta)$ maps $\dot C^\infty(M) \to C^\infty(M)$.

Expanding out $-\Delta_g$ in coordinates, we can compute that if
$$P(\lambda) = 4(x_1D_{x_1}^2 - (\lambda + i)D_{x_1}) - \Delta_h +i\gamma(x)(2x_1D_{x_1} - \lambda - i(n-1)/2)$$
where $x = (y_1^2, y')$, $\lambda = i(\zeta - n/2)$, and $\gamma(y) = -\partial_{y_1}(\det h(y))/\det h(y)$, then
$$y_1^{-\zeta}(-\Delta_g - \zeta(n-\zeta))y_1^\zeta = x_1 P(\lambda).$$
The motivation for doing this change of coordinates is that we will extend $\overline M$ to a new compact manifold $X = M \cup \{x_1 \leq 0\}$ where $x$ is a coordinate valued in $[-1, 1] \times \partial M$ rather than $[0, 1] \times \partial M$ as $y$ was.

We set $d\mu = \overline h(x) ~dx$ where $\overline h = \det h$.
Then $\mu$ is a measure on $X$ which defines $L^2(X)$ and $H^s(X)$.
With respect to $\mu$, $P(\lambda)^* = P(\overline \lambda)$.

Now we introduce the function space $\mathcal Y^s$, the space of restrictions of functions in $H^s(Z)$, where $Z$ is any compact extension of $X$, to the interior of $X$, and $\mathcal X^s$, the space of $u \in \mathcal Y^{s+1}$ such that $P(0)u \in \mathcal Y^s$.
Note that the dependency on $\lambda$ is only in lower-order terms of $P(\lambda)u$, so $P(0)u \in \mathcal Y^s$ iff $P(\lambda)u \in \mathcal Y^s$.

\begin{theorem}
If $\text{Im } \lambda > - s - 1/2$ then $P(\lambda)$ is Fredholm $\mathcal X^s \to \mathcal Y^s$.
\end{theorem}

\begin{theorem}
If $\text{Im } \lambda > 0$, $\lambda^2 + n^2/4$ is not an eigenvalue of $-\Delta_g$, and $s > -\text{Im } \lambda - 1/2$, then $P(\lambda)$ is invertible $\mathcal X^s \to \mathcal Y^s$.
\end{theorem}

\section{Preliminaries}
Let $X$ be a compact manifold with boundary.
Let $\Psi^m(X)$ be the space of pseudodifferential operators on $X$ and let $\sigma$ be the principal symbol map; that is, $\sigma$ assigns $P \in \Psi^m(X)$ the unique homogeneous smooth function of order $m$ which, modulo operators of order $m - 1$, quantizes to $P$.
If $P$ is actually a differential operator then $\sigma(P)$ is a polynomial of order $m$.

Appendix B of H\"ormander defines $\overline H^s(\RR^d_+)$ to be the space of restrictions of elements of $H^s(\RR^d)$ to $\RR_+^d$ with the quotient topology, and $\dot H^s(\RR^d_+)$ to be the space of elements of $H^s(\RR^d)$ with support in $\RR_+^d$.
Since $X$ is locally modelled on $\RR_+^d$, this defines the spaces $\overline H^s(X)$ and $\dot H^s(X)$, which satisfy the duality relation
$$(\overline H^s(X))^* = \dot H^{-s}(X)$$
given by the $L^2$ inner product of any top form on $X$.

\begin{definition}
Let $N$ be a compact manifold and $P \in \Psi^m([0, T] \times N)$, $p = \sigma(P)$.
We say that $P$ is a \emph{hyperbolic pseudodifferential operator} with respect to time if for every $(t, y, \eta) \in [0, T] \times T^*N$ with $\eta \neq 0$, the equation
$$p(t, y, \tau, \eta) = 0,$$
has exactly $m$ distinct solutions $\tau \in T_t^*[0, T]$.
\end{definition}

Here $T_t^*[0, T]$ denotes the cotangent space of $[0, T]$ at $t$.

Let
$$P = D_t^2 + P_1(t, x, D_x)D_t + P_0(t, x, D_x)$$
act on $\RR \times N$ where $N$ is a closed manifold and
$$P_j \in C^\infty([0, T) \to \Psi^{2-j}(N))$$
is a strictly hyperbolic on $N$ with respect to $t$. Then for every $f \in A_s = \overline H^s([0, T) \times N)$, there is a unique $u \in B_s = \overline H^{s+1}([0, T) \times N)$ with $Pu = f$; in particular,
$$||u||_{\overline H^{s+1}([0, T) \times N)} \leq C ||f||_{\overline H^s([0, T) \times N)}.$$
In particular, as a map $A_s \to B_s$, $P$ is an isomorphism.
It follows that
$$||u||_{\overline H^{s+1}((0, T) \times N)} \leq C ||f||_{\overline H^s((0, T) \times N)} + C ||u||_{\overline H^{s+1}((0, \delta) \times N)}$$
whenever $\delta > 0$; the point is that $u$ may be quite singular at $0$, since it is a boundary point of $[0, T) \times N$.

Now the operator $P(\lambda)$ is of the form $P$ where $t = 1 + x_1$ and $T = 1 - \varepsilon$, at least when $x_1 < 0$ (and if $x_1 > 0$ then we can instead take $t = -\varepsilon-x_1$).
Indeed, we can factor out an $x_1$ and get
$$P(\lambda) = x_1(D_{x_1}^2 - P_1(x)D_{x_1} + P_0(x, D_{x'}))$$
where $P_1 \in \Psi^0$ is smooth and $P_0 \in \Psi^2$ is elliptic away from $x_1 \neq 0$.
The trouble is at the boundary, $x_1 = 0$.

We shall also need a uniqueness result for $P(\lambda)$: if $u \in \dot C^\infty(X \cap \{x_1 \leq 0\})$ and $P(\lambda)u = 0$ then $u = 0$.

The proofs of the above results can mainly be found in H\"ormander's book, but the uniqueness result is in Zworski; I will omit the proof.

\section{Propagation of singularities}
We first review propagation of singularities, a famous theorem of H\"ormander.
As usual let $X$ be a smooth compact manifold with boundary.

If $p \in S^m(T^*X)$ is a symbol we let $H_p$ be the Hamiltonian vector field associated to $p$.
I think the intuition is that if $P$ is the quantization of $p$, we are supposed to think of $\partial_t - P$ as a sort of Schr\"odinger operator, which acts on $L^2(X)$ in a way that is the ``quantum version" of how $H_p$ acts on $T^*X$.
A particle with position $x$ and momentum $\xi$ which moves tangent to $H_p$ would, in a quantum world, start as a function microlocalized to $(x, \xi)$ in $L^2(X)$ and then evolve according to $\partial_t - P$.

\begin{definition}
Let $P \in \Psi^m(X)$, $p = \sigma(P)$.
Suppose that $p(x, \xi) = 0$, and $H_p$ and $\xi \partial_\xi$ are linearly dependent at $(x, \xi)$.
Then we say that $(x, \xi) \in T^*X$ is \emph{radial} for $P$.
\end{definition}

The intuition is that $\xi \partial_\xi$ is the radial vector field for each cotangent space of $X$, so the momentum should ``escape to infinity" if the particle in question is travelling on a (scalar multiple of) $\xi \partial_\xi$.

\begin{definition}
Let $P \in \Psi^m(X)$, $U, V, W \subseteq T^*X$ open sets.
We say that $U$ is \emph{forward controlled} by $V$ in $W$ if for every $(x, \xi) \in U$ there is $T > 0$ such that for every $t$, $e^{-tH_p}(x, \xi) \in W$, and $e^{-TH_p}(x, \xi) \in V$.
\end{definition}

In other words, if $U$ is forward controlled by $V$ in $W$, then for every particle $(x, \xi) \in U$, we can travel backwards in time along the classical trajectory of $(x, \xi)$ without leaving $W$ and eventually find ourselves in $V$.

Let $A \in \Psi^m(X)$ be a pseudodifferential operator.
Recall that its wavefront set $WF(A) \subseteq T^*X$ is the largest conic set outside of which $A$ is the quantization of a Schwartz symbol.
We think of $A$ as being ``supported in $WF(A)$" modulo a harmless error term.
We let $Ell(A) \subseteq T^*X$ be the conic set of points at which $A$ is elliptic, i.e. if $(x, \xi) \in Ell(A)$ then for every $c > 0$ large enough, then $\sigma(A)(x, c\xi) \sim c^m$.

\begin{theorem}[H\"ormander's propagation of singularities]
Let $P \in \Psi^m(X)$ and let $U \subseteq T^*X$ be a conic open set of nonradial points for $P$.
Suppose that $A,B,B_1 \in \Psi^0(X)$, $WF(1 - B_1) \cap U$ is empty, and $WF(A) \cup WF(B) \subseteq U$.
If $WF(A)$ is forward controlled by $Ell(B)$ in $U$ then for every $(N, s) \in \RR^2$ there is $C > 0$ such that
$$||Au||_{H^{s+m-1}} \leq C||B_1Pu||_{H^s} + C||Bu||_{H^{s+m-1}} + C||u||_{H^{-N}}.$$
\end{theorem}
To get the intuition for propagation of singularities, we can assume $B_1$ is the identity and $U = T^*X$ is all of phase space.
We can more or less ignore the term $||u||_{H^{-N}}$ since pseudodifferential calculus always picks up error terms like that, so more or less the propagation of singularities says
$$||Au||_{H^{s+m-1}} \leq C||Pu||_{H^s} + C||Bu||_{H^{s+m-1}}.$$
Then $A, B$ are like cutoffs to regions $WF(A),WF(B)$ of phase space, which act on the wavefunction $u$ by microlocalizing it to those regions.
Forward control means that a classical particle $(x, \xi) \in WF(A)$ was previously in $Ell(B)$.
The quantities $s + m - 1$ suggest that we gain $m - 1$ derivatives when we solve $Pu = f$, which seems to be about the best gain you can get in general; I'm not too sure why this is but the point is that $P$ is definitely not elliptic in general, and you need something like elliptic regularity to gain $m$ derivatives.
Meanwhile, $B$ can be viewed as being elliptic near its wavefront set since we have forward control.
Essentially what's going on here is that inverting $P$ cannot create new singularities, at least at these nonradial points.
I don't have a great intuition for why this is true but maybe I should review the proof in H\"ormander when this is all over.

Okay so that's classical propagation of singularities. Now we need a radial version.

Let $X$ be as in the previous sections, so we have coordinates $x$ with $Y = \partial M = \{x_1 = 0\}$.
Near $\partial M$, if $q(x, \xi') = |\xi'|^2_{h(x)}$ is the symbol of the Laplace-Beltrami operator on $(\partial M, h(x))$, let
$$p(x, \xi) = x_1\xi_1^2 + q(x, \xi').$$
Assume that $p = \sigma(P)$ where $P$ is a differential operator of order $2$.
We write $q(x_1)(x', \xi')$ for $q(x_1, x', \xi')$. Then
$$H_p = \xi_1(2x_1 \partial_{x_1} - \xi_1 \partial_{\xi_1}) + \partial_{x_1} q(x, \xi') \partial_{\xi_1} + H_{q(x_1)}.$$
Therefore the set of radial points of $P$ is exactly the conormal bundle $N^*Y = \{(0, x', \xi_1, 0): x' \in Y, \xi_1 \in \RR\}$,
since
$$H_p|N^*Y = -\xi_1(\xi\partial_\xi)$$
(since $\xi = \xi_1$ there).

Now $N^*Y = p^{-1}(0) \cap \psi^{-1}(Y)$ where $\psi: T^*X \to X$ is the natural projection, since $p(x, \xi) = 0$ iff $\xi' = 0$ and $x_1 = 0$ or $\xi_1 = 0$.
Let $\Sigma_+$ be the source of the flow of $H_p$ in $p^{-1}(0)$ and $\Sigma_-$ the sink.
Let $\Gamma_\pm = \Sigma_\pm \cap \psi^{-1}(Y)$.

\begin{theorem}[radial propagation of singularities]
Write $P = P_0 + iQ$ where $P_0,Q$ are real self-adjoint in $L^2(dx_1 \otimes dh)$.
Let
$$s_\pm = \sup_{(x', \xi_1) \in \Gamma_\pm} |\xi_1|^{-1} \sigma(Q)(0, x', \xi_1, 0) - \frac{1}{2}.$$
If $s > s_+$ then for every $B_1 \in \Psi^0(X)$ with $WF(1 - B_1) \cap \Gamma_+$ empty, then there exists $A \in \Psi^0(X)$ with $\Gamma_\pm \subseteq Ell(A)$ such that:
\begin{enumerate}
\item For every $u \in C^\infty_c(X)$,
$$||Au||_{H^{s+1}} \leq C||B_1Pu||_{H^s} + C||u||_{H^{-N}}.$$
\item There is $B \in \Psi^0(A)$ such that $WF(B) \cap \Gamma_-$ is empty and for every $u \in C^\infty_c(X)$,
$$||Au||_{H^{-s}} \leq C||B_1Pu||_{H^{-s-1}} + C||Bu||_{H^{-s}} + C||u||_{H^{-N}}.$$
\end{enumerate}
\end{theorem}

The idea behind the proof for $s_+$ is to construct $F_s \in \Psi^{s+1/2}(X)$ if $s > s_+$ then
$$||F_su||_{H^{1/2}}^2 \leq C||B_1Pu||_{H^s} ||F_su||_{H^{1/2}} + C||B_1u||_{H^{s+1/2}}^2 + C||u||_{H^{-N}}^2.$$
In fact we can take
$$F_s(x, D) = \psi_1(x_1) \psi_1(-\Delta_h/D_{x_1}^2) \psi_2(D_{x_1})D_{x_1}^{s+1/2}$$
where $\psi_1$ is a cutoff to a small neighborhood of $0$, $\psi_2$ is a cutoff to $\{t: t \geq 2\}$, and $\psi_1(-\Delta_h/D_{x_1}^2)$ is the quantization of $\psi_1(-\xi_1^{-2}q(x, \xi))$,
which makes sense since $\psi_2(D_{x_1})$ localizes to $\xi_1$ large and the singularity $\xi_1^{-2}$ is only at the frequency $\xi_1 = 0$.
This operator is elliptic away from $\{\xi_1 = 0\}$, at least assuming $x_1$ is small enough.
Thus there is a cutoff $A$ such that $WF(A)$ is contained in $Ell(F_s)$.








\end{document}
