
% --------------------------------------------------------------
% This is all preamble stuff that you don't have to worry about.
% Head down to where it says "Start here"
% --------------------------------------------------------------

\documentclass[10pt]{article}

\usepackage[margin=.7in]{geometry}
\usepackage{amsmath,amsthm,amssymb,mathrsfs}
\usepackage{enumitem}
\usepackage{tikz-cd}
\usepackage{mathtools}
\usepackage{amsfonts}
\usepackage{listings}
\usepackage{algorithm2e}
\usepackage{verse,stmaryrd}
\usepackage{fancyvrb}

% Number systems
\newcommand{\NN}{\mathbb{N}}
\newcommand{\ZZ}{\mathbb{Z}}
\newcommand{\QQ}{\mathbb{Q}}
\newcommand{\RR}{\mathbb{R}}
\newcommand{\CC}{\mathbb{C}}
\newcommand{\PP}{\mathbb P}
\newcommand{\FF}{\mathbb F}
\newcommand{\DD}{\mathbb D}
\renewcommand{\epsilon}{\varepsilon}

\newcommand*\dif{\mathop{}\!\mathrm{d}}

\newcommand{\card}{\operatorname{card}}
\newcommand{\dbar}{\overline \partial}
\DeclareMathOperator*{\esssup}{ess\,sup}
\newcommand{\GL}{\operatorname{GL}}
\newcommand{\Hom}{\operatorname{Hom}}
\newcommand{\id}{\operatorname{id}}
\newcommand{\Ind}{\operatorname{Ind}}
\newcommand{\Inn}{\operatorname{Inn}}
\newcommand{\interior}{\operatorname{int}}
\newcommand{\lcm}{\operatorname{lcm}}
\newcommand{\mesh}{\operatorname{mesh}}
\newcommand{\LL}{\mathcal L_0}
\newcommand{\Leb}{\mathcal{L}_{\text{loc}}^2}
\newcommand{\Lip}{\operatorname{Lip}}
\newcommand{\ppGL}{\operatorname{PGL}}
\newcommand{\ppic}{\vspace{35mm}}
\newcommand{\ppset}{\mathcal{P}}
\DeclareMathOperator{\proj}{proj}
\DeclareMathOperator*{\Res}{Res}
\newcommand{\Riem}{\mathcal{R}}
\newcommand{\RVect}{\RR\operatorname{-Vect}}
\newcommand{\Sch}{\mathcal{S}}
\newcommand{\SL}{\operatorname{SL}}
\newcommand{\sgn}{\operatorname{sgn}}
\newcommand{\spn}{\operatorname{span}}
\newcommand{\Spec}{\operatorname{Spec}}
\newcommand{\supp}{\operatorname{supp}}
\newcommand{\Torus}{\mathbb T}
\DeclareMathOperator{\tr}{tr}

\DeclareMathOperator{\adj}{adj}
\DeclareMathOperator{\curl}{curl}

% Calculus of variations
\DeclareMathOperator{\pp}{\mathbf p}
\DeclareMathOperator{\zz}{\mathbf z}
\DeclareMathOperator{\uu}{\mathbf u}
\DeclareMathOperator{\vv}{\mathbf v}
\DeclareMathOperator{\ww}{\mathbf w}

\DeclareMathOperator{\Olo}{\mathscr O}

% Categories
\newcommand{\Ab}{\mathbf{Ab}}
\newcommand{\Cat}{\mathbf{Cat}}
\newcommand{\Group}{\mathbf{Group}}
\newcommand{\Module}{\mathbf{Module}}
\newcommand{\Set}{\mathbf{Set}}
\DeclareMathOperator{\Fun}{Fun}
\DeclareMathOperator{\Iso}{Iso}

% Complex analysis
\renewcommand{\Re}{\operatorname{Re}}
\renewcommand{\Im}{\operatorname{Im}}

% Logic
\renewcommand{\iff}{\leftrightarrow}
\newcommand{\Henkin}{\operatorname{Henk}}
\newcommand{\PA}{\mathbf{PA}}
\DeclareMathOperator{\proves}{\vdash}

% Group
\DeclareMathOperator{\Gal}{Gal}
\DeclareMathOperator{\Fix}{Fix}
\DeclareMathOperator{\Out}{Out}

% Other symbols
\newcommand{\heart}{\ensuremath\heartsuit}

\DeclareMathOperator{\atanh}{atanh}

% Theorems
\theoremstyle{definition}
\newtheorem*{corollary}{Corollary}
\newtheorem*{falselemma}{Grader's ``Lemma"}
\newtheorem{exer}{Exercise}
\newtheorem{lemma}{Lemma}[exer]
\newtheorem{theorem}[lemma]{Theorem}

\usepackage[backend=bibtex,style=alphabetic,maxcitenames=50,maxnames=50]{biblatex}
\renewbibmacro{in:}{}
\DeclareFieldFormat{pages}{#1}

\begin{document}
\noindent
\large\textbf{FEM 2, HW 6} \hfill \textbf{Aidan Backus} \\
% --------------------------------------------------------------
%                         Start here
% --------------------------------------------------------------\

\begin{exer}
    Consider $u - \Delta u = f$ on $\Omega$ where $u|_{\partial \Omega} = 0$.
    Setting $\sigma := \nabla u$ show that the problem can be reformulated in the primal mixed form 
    \begin{align*}
        a(\sigma, \tau) + b(\tau, u) &= 0\\
        b(\sigma, v) - (u, v) &= -(f, v)
    \end{align*}
    for all $(\tau, v) \in H^1(\Omega) \times H^1_0(\Omega)$ and some bilinear forms $a, b$.
    Show that the primal mixed form satisfies the Babuška-Brezzi conditions and so is well-posed.
\end{exer}

We set $a(\sigma, \tau) := (\sigma, \tau)$ and
$$b(\tau, v) := -\int_\Omega \tau(x) \cdot \nabla v(x) \dif x.$$
The first equation becomes
$$0 = (\sigma, \tau) - \int_\Omega \tau \cdot \nabla u(x) \dif x$$
and since $\tau$ is arbitrary, this is equivalent to the strong formulation $\sigma = \nabla u$.
The second equation becomes 
$$\int_\Omega \sigma(x) \cdot \nabla v(x) \dif x + (u, v) = (f, v).$$
Integrating by parts using the fact that $v \in H^1_0$, we rewrite this as 
$$-(\nabla \cdot \sigma, v) + (u, v) = (f, v)$$
which in strong form reads $-\nabla \cdot \sigma + u = f$.
Combining the two strong equations we get $-\Delta u + u = f$ which was desired.

TODO or not...

\begin{exer}
    Let $P_h$ be a triangulation of $\Omega$ and $V_h \subseteq L^2(\Omega)^d$ the space of discontinuous piecewise polynomial maps of degree $n$, $M_h \subseteq H^1_0(\Omega)$ the space of continuous piecewise polynomials of degree $n + 1$ on $P_h$.
    Show that there is a unique solution $(\sigma_h, u_h)$ of the discretization of the primal mixed form problem in $V_h \times M_h$ such that if $u \in H^{n + 1}(\Omega)$ then 
    $$||\sigma - \sigma_h||_{L^2} + ||u - u_h||_{H^1_0} \lesssim h^n ||u||_{H^{n + 1}}.$$
\end{exer}

TODO more Babuška-Brezzi...
Finally the claimed estimate follows by the Aziz-Babuška inequality and the usual estimates from approximation theory.

\begin{exer}
    Show that $(\sigma_h, u_h)$ can be obtained by solving a system of linear equations 
    \begin{align*}
        A_h \sigma_h + B_h u_h &= 0 \\
        B_h^t \sigma_h - C_h u_h &= -f,
    \end{align*}
    where $(A_h)_{ij} = a(\tau_i, \tau_j)$ for some basis $(\tau_i)$ of $V_h$, and $B_h$ is an appropriate matrix.
    Deduce that $u_h$ can be computed by solving a system
    $$(S_h + C_h) u_h = f.$$
\end{exer}

Warning: I changed a $\sigma$ to a $\tau$ in the problem statement because I wanted to talk about components of $\sigma_h$ and was starting to feel like the notation was a little overloaded.

The discretization is 
\begin{align*}
    (\sigma_h, \tau_i) - (\nabla u_h, \tau_i) &= 0 \\
    (\sigma_h, \nabla v_\mu) + (u_h, v_\mu) &= (f, v_\mu)
\end{align*}
where $(\tau_i)$ is a basis of $V_h$ and $(v_\mu)$ is a basis of $M_h$.
Writing $\sigma_h = \sum_i \sigma^i \tau_i$ and $u_h = \sum_\mu u^\mu v_\mu$ this leads us to
\begin{align*}
    \sigma^i (\tau_i, \tau_j) - u^\mu (\nabla v_\mu, \tau_j) &= 0\\
    -\sigma^i (\tau_i, \nabla v_\nu) - u^\mu (v_\mu, v_\nu) &= -(f, v_\nu).
\end{align*}
Setting $(A_h)_{ij} := (\tau_i, \tau_j)$, $(B_h)_{\mu i} := (\nabla v_\mu, \tau_i)$, $(C_h)_{\mu \nu} = (v_\mu, v_\nu)$, we obtain the desired matrix formula for the discretization.
Here $A_h$ is invertible because it is the inner product when expressed in the basis $(\tau_i)$.
Now if we set 
$$S_h := B_h^t A_h^{-1} B_h,$$
then since $\sigma_h = A_h^{-1} B_h u_h$, $(S_h + C_h) u_h = f$.

\begin{exer}
    Let $P: L^2(\Omega)^d \to V_h$ be the orthogonal projection with respect to $a$.
    Show that if $v_h \in M_h$ then $P \nabla v_h = \sum_i c(v_h)^i \tau_i$, where 
    $$c(v_h) := A_h^{-1} B_h v.$$
\end{exer}

To ease the notation let's assume that $(\tau_i)$ is an orthonormal basis of $V_h$.
We can always assume that, thanks to Gram-Schmidt, since we don't care about the sparsity of the matrix $B_h$ in this problem.

Under the above assumption, $A_h$ is the identity, $\sum_i c(v_h)^i \tau_i = c(v_h)$, and the assertion is that $P\nabla = B_h$.
But 
$$(B_hv_h)^i = (\nabla v_h, \tau_i) = (\nabla v_h, P\tau_i) = (P\nabla v_h, \tau_i)$$
where the second inequality is $\tau_i \in V_h$ and the third is because $P$ is self-adjoint, since it is an orthogonal projection.
Since $i$ was arbitrary and $(\tau_i)$ is orthonormal it follows that $B_h v = P \nabla v$.

\begin{exer}
    Deduce that $S_h$ is precisely the same matrix as would be obtained if one were to discretize the Poisson problem discretly using $M_h$.
    Why is the primal mixed form problem of the Poisson problem not used in practice?
\end{exer}

Since $A_h^{-1}$ is the inner product on $V_h$, we can again suppress it by taking $(\tau_i)$ to be orthonormal.
This is fine because $S_h: M_h \to (M_h)'$ only cares about what basis $(v_\mu)$ we put on $M_h$ (and then the dual basis will be put on $(M_h')$) and not the basis $(\tau_i)$, so $(\tau_i)$ could even be a nonlocal basis if we wanted it to be.

Let $Q$ be the orthogonal projection onto $M_h$ using the inner product on $H^1_0$.
Then since $Q$ is the identity on $M_h$,
$$B_h = P\nabla = P\nabla Q$$
and hence
$$S_h = B_h^t B_h = Q\nabla^t PP\nabla Q = Q \nabla^t P \nabla.$$
Integration by parts shows that $\nabla^t \tau = - \nabla \cdot \tau$.
Moreover, $\nabla$ carries $M_h$ into $V_h$ by construction of these two spaces, so $P\nabla|_{M_h} = \nabla$.
Thus 
$$S_h = -Q \nabla \cdot \nabla = -Q\Delta$$
and in particular for $w \in M_h$,
$$(S_h w, v_\mu) = -(Q\Delta w, v_\mu) = -(\Delta w, Qv_\mu) = -(\Delta w, v_\mu).$$
(These pairings of course need to be understood as the pairing $H^1_0 \times H^1 \to \RR$.)
However, this is exactly the definition of the usual discretization of the Laplacian we learned about last year.

In short, the discretization of the primal mixed form of the Poisson problem can be easily rewritten as the discretization of the usual Poisson problem.
So it has no advantages over the classical discretization, unless one happened to have a solver for Brezzi-type problems lying around that they were using as a black box.

\end{document}
