
% --------------------------------------------------------------
% This is all preamble stuff that you don't have to worry about.
% Head down to where it says "Start here"
% --------------------------------------------------------------

\documentclass[10pt]{article}

\usepackage[margin=.7in]{geometry}
\usepackage{amsmath,amsthm,amssymb,mathrsfs}
\usepackage{enumitem}
\usepackage{tikz-cd}
\usepackage{mathtools}
\usepackage{amsfonts}
\usepackage{listings}
\usepackage{algorithm2e}
\usepackage{verse,stmaryrd}
\usepackage{fancyvrb}

% Number systems
\newcommand{\NN}{\mathbb{N}}
\newcommand{\ZZ}{\mathbb{Z}}
\newcommand{\QQ}{\mathbb{Q}}
\newcommand{\RR}{\mathbb{R}}
\newcommand{\CC}{\mathbb{C}}
\newcommand{\PP}{\mathbb P}
\newcommand{\FF}{\mathbb F}
\newcommand{\DD}{\mathbb D}
\renewcommand{\epsilon}{\varepsilon}

\newcommand*\dif{\mathop{}\!\mathrm{d}}

\newcommand{\card}{\operatorname{card}}
\newcommand{\dbar}{\overline \partial}
\DeclareMathOperator*{\esssup}{ess\,sup}
\newcommand{\GL}{\operatorname{GL}}
\newcommand{\Hom}{\operatorname{Hom}}
\newcommand{\id}{\operatorname{id}}
\newcommand{\Ind}{\operatorname{Ind}}
\newcommand{\Inn}{\operatorname{Inn}}
\newcommand{\interior}{\operatorname{int}}
\newcommand{\lcm}{\operatorname{lcm}}
\newcommand{\mesh}{\operatorname{mesh}}
\newcommand{\LL}{\mathcal L_0}
\newcommand{\Leb}{\mathcal{L}_{\text{loc}}^2}
\newcommand{\Lip}{\operatorname{Lip}}
\newcommand{\ppGL}{\operatorname{PGL}}
\newcommand{\ppic}{\vspace{35mm}}
\newcommand{\ppset}{\mathcal{P}}
\DeclareMathOperator{\proj}{proj}
\DeclareMathOperator*{\Res}{Res}
\newcommand{\Riem}{\mathcal{R}}
\newcommand{\RVect}{\RR\operatorname{-Vect}}
\newcommand{\Sch}{\mathcal{S}}
\newcommand{\SL}{\operatorname{SL}}
\newcommand{\sgn}{\operatorname{sgn}}
\newcommand{\spn}{\operatorname{span}}
\newcommand{\Spec}{\operatorname{Spec}}
\newcommand{\supp}{\operatorname{supp}}
\newcommand{\Torus}{\mathbb T}
\DeclareMathOperator{\tr}{tr}

\DeclareMathOperator{\adj}{adj}
\DeclareMathOperator{\curl}{curl}

% Calculus of variations
\DeclareMathOperator{\pp}{\mathbf p}
\DeclareMathOperator{\zz}{\mathbf z}
\DeclareMathOperator{\uu}{\mathbf u}
\DeclareMathOperator{\vv}{\mathbf v}
\DeclareMathOperator{\ww}{\mathbf w}

\DeclareMathOperator{\Olo}{\mathscr O}

% Categories
\newcommand{\Ab}{\mathbf{Ab}}
\newcommand{\Cat}{\mathbf{Cat}}
\newcommand{\Group}{\mathbf{Group}}
\newcommand{\Module}{\mathbf{Module}}
\newcommand{\Set}{\mathbf{Set}}
\DeclareMathOperator{\Fun}{Fun}
\DeclareMathOperator{\Iso}{Iso}

% Complex analysis
\renewcommand{\Re}{\operatorname{Re}}
\renewcommand{\Im}{\operatorname{Im}}

% Logic
\renewcommand{\iff}{\leftrightarrow}
\newcommand{\Henkin}{\operatorname{Henk}}
\newcommand{\PA}{\mathbf{PA}}
\DeclareMathOperator{\proves}{\vdash}

% Group
\DeclareMathOperator{\Gal}{Gal}
\DeclareMathOperator{\Fix}{Fix}
\DeclareMathOperator{\Out}{Out}

% Other symbols
\newcommand{\heart}{\ensuremath\heartsuit}

\DeclareMathOperator{\atanh}{atanh}

% Theorems
\theoremstyle{definition}
\newtheorem*{corollary}{Corollary}
\newtheorem*{falselemma}{Grader's ``Lemma"}
\newtheorem{exer}{Exercise}
\newtheorem{lemma}{Lemma}[exer]
\newtheorem{theorem}[lemma]{Theorem}

\usepackage[backend=bibtex,style=alphabetic,maxcitenames=50,maxnames=50]{biblatex}
\renewbibmacro{in:}{}
\DeclareFieldFormat{pages}{#1}

\begin{document}
\noindent
\large\textbf{FEM 2, HW 2} \hfill \textbf{Aidan Backus} \\
% --------------------------------------------------------------
%                         Start here
% --------------------------------------------------------------\

\begin{exer}
    Consider the problem $-u'' = f$ on $(0, 1)$ subject to $u(0) = u'(1) = 0$, $f \in L^2$.
    Show that the problem may be expressed as $\sigma' = f$, $u' + \sigma = 0$ subject to $u(0) = \sigma(1) = 0$.
\end{exer}

Suppose that $-u'' = f$ holds with the desired data. We solve $u' + \sigma = 0$, thus $\sigma = -u'$ and $\sigma(1) = 0$.
Moreover, $\sigma' = -u'' = f$.

\begin{exer}
    Let $X := L^2 \times Z$ where $Z := \{\tau \in H^1: \tau(1) = 0\}$.
    Show that $||(v, \tau)||_X := ||v|| + ||\tau'||$ is a norm on $X$.
\end{exer}

The scaling property is immediate and the triangle inequality is 
$$||(u + v, \tau + \rho)||_X = ||u + v|| + ||\tau' + \rho'|| \leq ||u|| + ||v|| + ||\tau'|| + ||\rho'|| = ||(u, \tau)||_X + ||(v, \rho)||_X.$$
So we just need to show the positivity.
In fact if $||(v, \tau)||_X = 0$ then $||v|| = 0$ so $v = 0$, and $||\tau'|| = 0$ so $\tau' = 0$.
But $\tau(1) = 0$ as well, so $\tau = 0$ in that case.

\begin{exer}
    Show that the problem may be expressed in the variational form 
    $$(u, \sigma) \in X: B((u, \sigma), (v, \tau)) = (f, v) ~\forall (v, \tau) \in X.$$
\end{exer}

We set 
$$B((u, \sigma), (v, \tau)) := (\sigma', v) + (\sigma, \tau) - (u, \tau')$$
which is evidently a bilinear form.
If $B((u, \sigma), (v, \tau)) = (f, v)$ for every $(v, \tau)$ and $(u, \sigma)$ has the requisite smoothness, then integrating by parts using the boundary conditions, we get 
$$(f, v) = B((u, \sigma), (v, \tau)) = (\sigma', v) + (\sigma, \tau) + (u', \tau) = (\sigma', v) + (u' + \sigma, \tau).$$
This now decouples since the first term was paired with $v$ and does not involve $\tau$, and vice versa for the second term, thus we have 
\begin{align*}
(\sigma', v) &= (f, v) \\
(u' + \sigma, \tau) &= 0.
\end{align*}
Since $u, \sigma$ were assumed smooth and $v, \tau$ were arbitrary we conclude $\sigma' = f$ and $u' + \sigma = 0$.

\begin{exer}
    Show that $B$ is continuous but not coercive.
\end{exer}

We first bound 
\begin{align*}
    |B((u, \sigma), (v, \tau))| &\leq |(\sigma', v)| + |(\sigma, \tau)| + |(u, \tau')| \\
    &\leq ||\sigma'|| \cdot ||v|| + ||\sigma|| \cdot ||\tau|| + ||u|| \cdot ||\tau'||.
\end{align*}
By Poincar\'e's inequality and the fact that $\sigma(1) = \tau(1) = 0$ the second term is $\leq ||\sigma'|| \cdot ||\tau'||$.
Thus 
$$|B((u, \sigma), (v, \tau))| \leq 3 ||(u, \sigma)|| \cdot ||(v, \tau)||$$
which implies continuity of $B$.

As for coercivity we just observe that $B((1, 0), (1, 0)) = 0$ so $B$ is not even positive-definite.

\begin{exer}
    Let $(u, \sigma) \in X$ be nonzero and let 
\begin{align*}
    \tau(x) &:= \int_x^1 u(s) \dif s, \\
    v(x) &:= \sigma'(x) + \int_0^x \tau(s) \dif s.
\end{align*}
    Show that $(v, \tau) \in X$, 
    $$B((u, \sigma), (v, \tau)) = ||u||^2 + ||\sigma'||^2,$$
    and $||(v, \tau)||_X \lesssim ||(u, \sigma)||_X$.
\end{exer}

The third condition clearly implies the first, which in turn is needed to make sense of the second.
So we prove the third condition first.
First, 
$$||(v, \tau)||_X = ||v|| + ||\tau'||$$
and by construction, $\tau' = -u$ so $||\tau'|| = ||u||$.
Moreover, by Minkowski's and Poincar\'e's inequalities,
\begin{align*}
    ||v||^2
    &= 2\left[||\sigma'||^2 + \int_0^1 \left|\int_0^x \tau(s) \dif s\right|^2 \dif x\right] \\
    &\leq 2||\sigma'||^2 + 2 \left|\int_0^1 ||1_{[0, x]} \tau||_{L^2} \dif x\right|^2 \\
    &\leq 2||\sigma'||^2 + 2 ||\tau||^2 \\
    &\leq 2||\sigma'||^2 + 2 ||u||^2.
\end{align*}
In conclusion, $||v|| \lesssim ||(u, \sigma)||_X$, which is what we were meant to show.

Now let
$$w(x) := \int_0^x \int_y^1 u(z) \dif z \dif y.$$
Then
\begin{align*}
    B((u, \sigma), (v, \tau)) 
    &= (\sigma', v) + (\sigma, \tau) - (u, \tau') \\
    &= (\sigma', \sigma') + (\sigma', w) + (\sigma, w') - (u, -u) \\
    &= (\sigma', \sigma') + (\sigma', w) - (\sigma', w) + (u, u) \\
    &= ||\sigma'||^2 + ||u||^2
\end{align*}
which was to be shown.

\begin{exer}
    Show that the Babuška-Brezzi and sup conditions are met.
\end{exer}

First we correct $X$ to be a Hilbert space by redefining the norm to be 
$$||(v, \tau)||_X^2 := ||v||^2 + ||\tau'||^2.$$
Clearly this norm is comparable to the norm we've been using above so this is no great loss.

Now we check the Babuška-Brezzi condition.
Given $(u, \sigma) \in X$ we let $(v, \tau)$ be as in the previous problem. Then
$$B((u, \sigma), (v, \tau)) = ||(u, \sigma)||_X^2 \gtrsim ||(u, \sigma)||_X ||(v, \tau)||_X.$$

Finally we check the sup condition.
Given $(v, \tau) \in X$ we let $u = -\tau'$ and 
$$\sigma(x) = \int_x^1 v(t) - \int_0^t \tau(s) \dif s \dif t.$$
If we let 
$$\kappa(x) := \int_0^x \int_y^1 \tau(z) \dif z \dif y$$
then $(\sigma', v) = (v, v) - (\kappa, v)$, $(\sigma, \tau) = (\kappa, v) + (\kappa, \kappa)$, and $(u, \tau') = -(\tau', \tau')$.
So
$$B((u, \sigma), (v, \tau)) = (v, v) - (\kappa, v) + (\kappa, v) + (\kappa, \kappa) + (\tau', \tau').$$
Cancelling the $(\kappa, v)$ we see that this is positive for $(v, \tau) \in X$ nonzero, as desired.

\begin{exer}
    Show that the problem is well-posed.
\end{exer}

This is immediate from the above claims and the Babuška theorem.




\end{document}
