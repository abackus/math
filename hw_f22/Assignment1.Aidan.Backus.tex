
% --------------------------------------------------------------
% This is all preamble stuff that you don't have to worry about.
% Head down to where it says "Start here"
% --------------------------------------------------------------

\documentclass[10pt]{article}

\usepackage[margin=.7in]{geometry}
\usepackage{amsmath,amsthm,amssymb,mathrsfs}
\usepackage{enumitem}
\usepackage{tikz-cd}
\usepackage{mathtools}
\usepackage{amsfonts}
\usepackage{listings}
\usepackage{algorithm2e}
\usepackage{verse,stmaryrd}
\usepackage{fancyvrb}

% Number systems
\newcommand{\NN}{\mathbb{N}}
\newcommand{\ZZ}{\mathbb{Z}}
\newcommand{\QQ}{\mathbb{Q}}
\newcommand{\RR}{\mathbb{R}}
\newcommand{\CC}{\mathbb{C}}
\newcommand{\PP}{\mathbb P}
\newcommand{\FF}{\mathbb F}
\newcommand{\DD}{\mathbb D}
\renewcommand{\epsilon}{\varepsilon}

\newcommand{\Aut}{\operatorname{Aut}}
\newcommand{\coker}{\operatorname{coker}}
\newcommand{\CVect}{\CC\operatorname{-Vect}}
\newcommand{\Cantor}{\mathcal{C}}
\newcommand{\D}{\mathcal{D}}
\newcommand{\card}{\operatorname{card}}
\newcommand{\dbar}{\overline \partial}
\DeclareMathOperator*{\esssup}{ess\,sup}
\newcommand{\GL}{\operatorname{GL}}
\newcommand{\Hom}{\operatorname{Hom}}
\newcommand{\id}{\operatorname{id}}
\newcommand{\Ind}{\operatorname{Ind}}
\newcommand{\Inn}{\operatorname{Inn}}
\newcommand{\interior}{\operatorname{int}}
\newcommand{\lcm}{\operatorname{lcm}}
\newcommand{\mesh}{\operatorname{mesh}}
\newcommand{\LL}{\mathcal L_0}
\newcommand{\Leb}{\mathcal{L}_{\text{loc}}^2}
\newcommand{\Lip}{\operatorname{Lip}}
\newcommand{\ppGL}{\operatorname{PGL}}
\newcommand{\ppic}{\vspace{35mm}}
\newcommand{\ppset}{\mathcal{P}}
\DeclareMathOperator{\proj}{proj}
\DeclareMathOperator*{\Res}{Res}
\newcommand{\Riem}{\mathcal{R}}
\newcommand{\RVect}{\RR\operatorname{-Vect}}
\newcommand{\Sch}{\mathcal{S}}
\newcommand{\SL}{\operatorname{SL}}
\newcommand{\sgn}{\operatorname{sgn}}
\newcommand{\spn}{\operatorname{span}}
\newcommand{\Spec}{\operatorname{Spec}}
\newcommand{\supp}{\operatorname{supp}}
\newcommand{\Torus}{\mathbb T}
\DeclareMathOperator{\tr}{tr}

\DeclareMathOperator{\adj}{adj}
\DeclareMathOperator{\curl}{curl}

% Calculus of variations
\DeclareMathOperator{\pp}{\mathbf p}
\DeclareMathOperator{\zz}{\mathbf z}
\DeclareMathOperator{\uu}{\mathbf u}
\DeclareMathOperator{\vv}{\mathbf v}
\DeclareMathOperator{\ww}{\mathbf w}

\DeclareMathOperator{\Olo}{\mathscr O}

% Categories
\newcommand{\Ab}{\mathbf{Ab}}
\newcommand{\Cat}{\mathbf{Cat}}
\newcommand{\Group}{\mathbf{Group}}
\newcommand{\Module}{\mathbf{Module}}
\newcommand{\Set}{\mathbf{Set}}
\DeclareMathOperator{\Fun}{Fun}
\DeclareMathOperator{\Iso}{Iso}

% Complex analysis
\renewcommand{\Re}{\operatorname{Re}}
\renewcommand{\Im}{\operatorname{Im}}

% Logic
\renewcommand{\iff}{\leftrightarrow}
\newcommand{\Henkin}{\operatorname{Henk}}
\newcommand{\PA}{\mathbf{PA}}
\DeclareMathOperator{\proves}{\vdash}

% Group
\DeclareMathOperator{\Gal}{Gal}
\DeclareMathOperator{\Fix}{Fix}
\DeclareMathOperator{\Out}{Out}

% Other symbols
\newcommand{\heart}{\ensuremath\heartsuit}

\DeclareMathOperator{\atanh}{atanh}

% Theorems
\theoremstyle{definition}
\newtheorem*{corollary}{Corollary}
\newtheorem*{falselemma}{Grader's ``Lemma"}
\newtheorem{exer}{Exercise}
\newtheorem{lemma}{Lemma}[exer]
\newtheorem{theorem}[lemma]{Theorem}

\usepackage[backend=bibtex,style=alphabetic,maxcitenames=50,maxnames=50]{biblatex}
\renewbibmacro{in:}{}
\DeclareFieldFormat{pages}{#1}

\begin{document}
\noindent
\large\textbf{FEM 2, HW 1} \hfill \textbf{Aidan Backus} \\
% --------------------------------------------------------------
%                         Start here
% --------------------------------------------------------------\

\begin{exer}
Consider the problem $u - u' = f$ subject to $u(0) = u_0$ and $f \in H^{-1}$.
Show that the problem can be expressed in the form $B(u, v) = Lv$ for all $v \in Y$ where
$$B(u, v) = (u, v') + (u, v),$$
$L \in Y^*$, and $X, Y$ are explicit Hilbert spaces.
\end{exer}

Let $X := L^2$, $Y := \{v \in H^1: v(1) = 0\}$, and
$$Lv = (f, v) - u_0v(0).$$
We claim that if $u$ solves this problem, and has higher regularity, then $u - u' = f$ with boundary condition $u(0) = u_0$.
That is, this is the appropriate weak formulation.

To see why, supppose that $u$ is a weak solution in this candidate sense and has higher regularity, thus for every $v \in H^1$ such that $v(1) = 0$,
$$(u, v') + (u, v) = (f, v) - u_0v(0).$$
Integrating the left-hand side by parts,
$$(f, v) - u_0v(0) = (u - u', v) + u(1)v(1) - u(0)v(0).$$
That is, $u - u' = f$ since we could take $v$ to approximate a Dirac delta, $u(1)v(1) = 0$, and we have the Dirichlet data $u(0) = u_0$.

\begin{exer}
Show that $B, L$ are continuous and linear in each argument.
\end{exer}

$L$ is the Riesz representation of $f$, which is continuous and linear, plus the evaluation map at $1$ (the Riesz representation of $\delta_1$), which is also continuous and linear.

$B$ is the $L^2$ pairing of $u, v'$ (and $v' \in L^2$ is continuous and linear in $v$ since $v \in H^1$) plus the $L^2$ pairing of $u, v$.
Thus the continuity follows by Cauchy-Schwarz and the bilinearity is by definition.

\begin{exer}
Let $v \in Y$ be nonzero. Show that there exists $u \in X$ such that $B(u, v) > 0$.
\end{exer}

Since $v$ is continuous, we can solve the equation $u - u' = v$ with Dirichlet data $u(0) = -v(0)$. Then 
$$B(u, v) = (u, v + v') = (u - u', v) - u(0)v(0) = ||v||^2 + |v(0)|^2 > 0.$$

\begin{exer}
Let $u \in X$ be nonzero. Show that there exists $v \in Y$ such that 
$$B(u, v) \geq \alpha ||u||_X ||v||_Y.$$
\end{exer}

We solve the ODE $v + v' = u$ with Dirichlet data $v(1) = 0$ using the method of integrating factors:
$$v(x) = -e^{-x} \int_x^1 u(y) e^y ~d y.$$
Then it is clear that $||v|| \lesssim ||u||$, and moreover
$$|v'(x)| \leq |v(x)| + |u(x)|,$$
where the first term comes from differentiating the $e^{-x}$ and the second term comes from the fundamental theorem of calculus.
Therefore $||v||_Y \lesssim ||u||$ and 
$$B(u, v) = (u, v + v') = ||u||^2 \gtrsim ||u|| \cdot ||v||_Y.$$
Here $||u||_X = ||u||$.

\begin{exer}
Show that the problem has a unique solution $u \in X$ such that $||u||_X \leq C/\alpha$ for some $C = C(u_0, f)$.
\end{exer}

The problem satisfies the hypotheses of the Babuška et al. theorem and hence $||u||_X \lesssim ||L||_{Y^*}/\alpha$.
Moreover, $||L||_{Y^*} \leq C^*(f) + |u_0|$.


\end{document}
