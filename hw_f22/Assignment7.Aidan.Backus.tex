
% --------------------------------------------------------------
% This is all preamble stuff that you don't have to worry about.
% Head down to where it says "Start here"
% --------------------------------------------------------------

\documentclass[10pt]{article}

\usepackage[margin=.7in]{geometry}
\usepackage{amsmath,amsthm,amssymb,mathrsfs}
\usepackage{enumitem}
\usepackage{tikz-cd}
\usepackage{mathtools}
\usepackage{amsfonts}
\usepackage{listings}
\usepackage{algorithm2e}
\usepackage{verse,stmaryrd}
\usepackage{fancyvrb}

% Number systems
\newcommand{\NN}{\mathbb{N}}
\newcommand{\ZZ}{\mathbb{Z}}
\newcommand{\QQ}{\mathbb{Q}}
\newcommand{\RR}{\mathbb{R}}
\newcommand{\CC}{\mathbb{C}}
\newcommand{\PP}{\mathbb P}
\newcommand{\FF}{\mathbb F}
\newcommand{\DD}{\mathbb D}
\renewcommand{\epsilon}{\varepsilon}

\newcommand*\dif{\mathop{}\!\mathrm{d}}

\newcommand{\card}{\operatorname{card}}
\newcommand{\dbar}{\overline \partial}
\DeclareMathOperator*{\esssup}{ess\,sup}
\newcommand{\GL}{\operatorname{GL}}
\newcommand{\Hom}{\operatorname{Hom}}
\newcommand{\id}{\operatorname{id}}
\newcommand{\Ind}{\operatorname{Ind}}
\newcommand{\Inn}{\operatorname{Inn}}
\newcommand{\interior}{\operatorname{int}}
\newcommand{\lcm}{\operatorname{lcm}}
\newcommand{\mesh}{\operatorname{mesh}}
\newcommand{\LL}{\mathcal L_0}
\newcommand{\Leb}{\mathcal{L}_{\text{loc}}^2}
\newcommand{\Lip}{\operatorname{Lip}}
\newcommand{\ppGL}{\operatorname{PGL}}
\newcommand{\ppic}{\vspace{35mm}}
\newcommand{\ppset}{\mathcal{P}}
\DeclareMathOperator{\proj}{proj}
\DeclareMathOperator*{\Res}{Res}
\newcommand{\Riem}{\mathcal{R}}
\newcommand{\RVect}{\RR\operatorname{-Vect}}
\newcommand{\Sch}{\mathcal{S}}
\newcommand{\SL}{\operatorname{SL}}
\newcommand{\sgn}{\operatorname{sgn}}
\newcommand{\spn}{\operatorname{span}}
\newcommand{\Spec}{\operatorname{Spec}}
\newcommand{\supp}{\operatorname{supp}}
\newcommand{\Torus}{\mathbb T}
\DeclareMathOperator{\tr}{tr}

\DeclareMathOperator{\adj}{adj}
\DeclareMathOperator{\curl}{curl}
\DeclareMathOperator{\Div}{div}

% Calculus of variations
\DeclareMathOperator{\pp}{\mathbf p}
\DeclareMathOperator{\zz}{\mathbf z}
\DeclareMathOperator{\uu}{\mathbf u}
\DeclareMathOperator{\vv}{\mathbf v}
\DeclareMathOperator{\ww}{\mathbf w}

\DeclareMathOperator{\Olo}{\mathscr O}

% Categories
\newcommand{\Ab}{\mathbf{Ab}}
\newcommand{\Cat}{\mathbf{Cat}}
\newcommand{\Group}{\mathbf{Group}}
\newcommand{\Module}{\mathbf{Module}}
\newcommand{\Set}{\mathbf{Set}}
\DeclareMathOperator{\Fun}{Fun}
\DeclareMathOperator{\Iso}{Iso}

% Complex analysis
\renewcommand{\Re}{\operatorname{Re}}
\renewcommand{\Im}{\operatorname{Im}}

% Logic
\renewcommand{\iff}{\leftrightarrow}
\newcommand{\Henkin}{\operatorname{Henk}}
\newcommand{\PA}{\mathbf{PA}}
\DeclareMathOperator{\proves}{\vdash}

% Group
\DeclareMathOperator{\Gal}{Gal}
\DeclareMathOperator{\Fix}{Fix}
\DeclareMathOperator{\Out}{Out}

% Other symbols
\newcommand{\heart}{\ensuremath\heartsuit}

\DeclareMathOperator{\atanh}{atanh}

% Theorems
\theoremstyle{definition}
\newtheorem*{corollary}{Corollary}
\newtheorem*{falselemma}{Grader's ``Lemma"}
\newtheorem{exer}{Exercise}
\newtheorem{lemma}{Lemma}[exer]
\newtheorem{theorem}[lemma]{Theorem}

\usepackage[backend=bibtex,style=alphabetic,maxcitenames=50,maxnames=50]{biblatex}
\renewbibmacro{in:}{}
\DeclareFieldFormat{pages}{#1}

\begin{document}
\noindent
\large\textbf{FEM 2, HW 7} \hfill \textbf{Aidan Backus} \\
% --------------------------------------------------------------
%                         Start here
% --------------------------------------------------------------\

\begin{exer}
    We construct $H(\Div)$-conforming finite elements, as follows. Let $K = (0, h)^2$, $P = \spn((1, 0), (0, 1), (x, 0), (0, y))$, and $\Sigma = \{\sigma_\gamma: \gamma \in E\}$ where $E$ is the set of edges of $K$ and
    $$\sigma_\gamma(\tau) = \int_\gamma n \cdot \tau \dif s.$$
    Show that $(K, P, \Sigma)$ is unisolvent, and deduce that it is a finite element.
\end{exer}

    We do this at the same time as solving the next problem, constructing a Lagrange basis.
    Indeed, if we have a $\Sigma$-Lagrange basis of $P$, then $(P, \Sigma)$ is unisolvent and hence $(K, P, \Sigma)$ is a finite element.
    Let $\gamma_{11} := \{x = 0\}$, $\gamma_{12} := \{x = h\}$, $\gamma_{21} := \{y = 0\}$, and $\gamma_{22} := \{y = h\}$.
    Let $\varphi_{11} := h^{-1} (0, h - x)$, $\varphi_{12} := h^{-1} (0, x)$, $\varphi_{21} := h^{-1} (h - y, 0)$, and $\varphi_{22} := h^{-1} (y, 0)$.
    Then $\varphi_{ij}$ is orthogonal to $n_{i'j'}$ unless $i = i'$ (in which case it is parallel), and the restriction of $\varphi_{ij}$ to $\gamma_{i'j'}$ is $0$ unless $j = j'$ (in which case it is $h^{-1}$).
    Since each edge has length $h$ the claim follows, and the $\varphi_{ij}$ are clearly in $P$ since they are piecewise linear, the claim follows.

\begin{exer}
    Construct a Lagrange basis $\{\varphi_\gamma: \gamma \in E\}$.
\end{exer}

    See above.

\begin{exer}
    Write down an explicit formula for the operator $\Pi: C(K)^2 \to P$ defined by 
\begin{equation}\label{orthoprojection}
    \sigma_\gamma \cdot \Pi = \sigma_\gamma,
\end{equation}
    and show that it satisfies the commutative property 
\begin{equation}\label{commuting}
    \Div \circ \Pi = \Pi_0 \circ \Div
\end{equation}
    where you should define $\Pi_0$.
\end{exer}

    As usual, $\Pi$ is given by 
    $$\Pi(\tau) = \sum_{\gamma \in E} \sigma_\gamma(\tau) \varphi_\gamma.$$
    Then 
\begin{align*}
    \sigma_\gamma(\Pi(\tau)) &= \sum_{\kappa \in E} \sigma_\kappa(\tau) \sigma_\gamma(\varphi_\kappa) = \sum_{\kappa \in E} \sigma_\kappa(\tau) \delta_{\kappa \gamma} \\
    &= \sigma_\gamma(\tau)
\end{align*}
    which implies (\ref{orthoprojection}). Now we let $\Pi_0 u$ be the mean of $u$ on $(0, h)^2$ like usual.
    Observing that $\varphi_\gamma$ was chosen to be the outwards normal to $\gamma$ of length $h^{-1}$, and hence has divergence $\pm h^{-1}$ depending on the orientation, we get 
\begin{align*}
    \Pi_0(\Div \tau) &= h^{-2} \int_K \Div \tau \dif A = h^{-2} \int_{\partial K} n \cdot \tau \dif s = \sum_{\gamma \in E} \Div \varphi_\gamma \int_\gamma \varphi_\gamma \cdot \tau \dif s \\
    &= \Div(\Pi \tau).
\end{align*}
    As usual, the orientation signs canceled out with the signs arising from the orientation of the boundary. We conclude (\ref{commuting}).

\begin{exer}
    Show that $\Pi c = c$ for every $c \in \RR^2$.
\end{exer}

    If we choose an inner product on $C(K)^2$ so that $\{\varphi_\gamma: \gamma \in E\}$ becomes an orthonormal set, then $\Pi$ is the orthogonal projection with respect to that inner product.
    In particular $\Pi$ fixes $P$, and hence $c \in P$.

\begin{exer}
    Show that 
\begin{equation}\label{L2 estimate}
    ||\Pi \sigma||_K \lesssim ||\sigma||_K + h ||\nabla \sigma||_K
\end{equation}
    and 
    $$||\Div(\Pi \sigma)||_K \leq ||\Div \sigma||_K.$$
    Conclude that 
    $$\Pi: H^1(K)^2 \to H(\Div, K)$$
    is a bounded linear map.
\end{exer}

    We compute 
    $$||\Pi \tau||_K^2 = \sum_{\gamma, \kappa \in E} \int_\gamma n \cdot \tau \dif s \int_\kappa n \cdot \tau \dif s \int_K \varphi_\gamma \cdot \varphi_\kappa \dif s.$$
    We then can bound 
    $$\left|\int_K \varphi_\gamma \cdot \varphi_\kappa \dif s\right| \lesssim 1$$
    owing to the fact that we're integrating over a region of area $h^2$ two functions that are each $\lesssim h^{-1}$. 
    The expression $\int_\gamma n \cdot \tau \dif s$ appeared already when were building a finite element basis for simplices, where we bounded 
    $$\left|\int_\gamma n \cdot \tau \dif s\right| \lesssim ||\tau||_K + h ||\nabla \tau||_K.$$
    The $L^2$ estimate (\ref{L2 estimate}) follows.
    Since $\Pi_0$ is the orthogonal projection $L^2(K) \to \RR$,
    $$||\Div(\Pi \tau)||_K = ||\Pi(\Div \tau)||_K \leq ||\Div \tau||_K.$$
    Then 
    $$||\Pi \tau||_{H(\Div, K)}^2 = ||\Pi \tau||_K^2 + ||\Div \tau||_K^2 \lesssim ||\tau||_K^2 + h^2 ||\nabla \sigma||_K^2 + ||\Div \tau||_K^2 \lesssim ||\tau||_{H^1(K)}^2.$$
    Thus $\Pi$ carrries $H^1(K)^2$ to $H(\Div, K)$ boundedly.

\begin{exer}
    Show that if $\sigma \in H^1(K)^2$ then 
    $$||(1 - \Pi)\sigma||_K \lesssim h ||\sigma||_{H^1(K)}$$
    and if $\Div \sigma \in H^1(K)^2$ then 
    $$||\Div(1 - \Pi)\sigma||_K \lesssim h ||\Div \sigma||_{H^1(K)}.$$
\end{exer}

    Let $\tau$ be the best approximation of $\sigma$ by constant vector fields for the $L^2$ norm. Then
    $$|(1 - \Pi)\sigma||_K \leq ||(1 - \Pi)(\sigma - \tau)||_K + ||(1 - \Pi) \tau||_K.$$
    Since $\tau \in P$ the second term vanishes, and moreover
    $$||(1 - \Pi)(\sigma - \tau)||_K \leq ||\sigma - \tau||_K + ||\Pi (\sigma - \tau)||_K.$$
    We bound the first term using the Bramble-Hilbert lemma as
\begin{equation}\label{Bramble Hilbert}
    ||\sigma - \tau||_K \lesssim h ||\sigma||_{H^1(K)}.
\end{equation}
    For the second, we apply (\ref{L2 estimate}) to obtain
    $$||\Pi (\sigma - \tau)||_K \lesssim ||\sigma - \tau||_K + h ||\nabla (\sigma - \tau)||_K.$$
    The first term here is harmless by (\ref{Bramble Hilbert}) and the second is $h ||\nabla \sigma||_K$ since $\tau$ was assumed constant, hence
    $$h ||\nabla (\sigma - \tau)||_K \lesssim h ||\sigma||_{H^1(K)}.$$

    Now we use the fact that $||1 - \Pi_0||_{H^1(K) \to L^2(K)} \lesssim h$ to estimate 
    $$||\Div(1 - \Pi)\sigma||_K = ||(1 - \Pi_0) \Div \sigma||_K \lesssim h ||\Div \sigma||_{H^1(K)}.$$

\begin{exer}
    Let $P_h$ be a partition of $\Omega$ into squares of length $h$, let
    $$M_h = \{v \in L^2(\Omega): v|_K \in \mathbb P_0 \forall K \in P_h\}$$
    and let 
    $$V_h = \{\sigma \in H(\Div, \Omega): \sigma|_K \in P(K) \forall K \in P_h\}.$$
    Show that the dual finite element approximation $(\sigma_h, u_h)$ of the Poisson problem with data $f$ based on $V_h \times M_h$ is well-defined, uniformly stable, and satisfies 
    $$||\sigma - \sigma_h||_{H(\Div, \Omega)} + ||u - u_h|| \lesssim h ||f||_{H^1(\Omega)}.$$
\end{exer}

    We need to verify the Babuška-Brezzi conditions for $a(\sigma, \tau) = (\sigma, \tau)$ and 
    $$b(u, \sigma) = (u, \Div \sigma)$$
    on $V_h, M_h$. Clearly $a$ is elliptic (and in particular is elliptic on the left kernel of $b$).
    Let $\sigma \in V_h$; we must find $u \in M_h$ so that 
    $$(u, \Div \sigma) \gtrsim ||u||_{L^2} \cdot ||\Div \sigma||_{H(\Div)}.$$
    In fact, the divergences of elements of $V_h$ were chosen to be piecewise constant and hence they lie in $M_h$, so we can take $u = \Div \sigma$ with implied constant $1$ (since $||\Div \sigma||_{H(\Div)} \geq ||\Div \sigma||_{L^2}$).
    
    Since the constants in the Babuška-Brezzi conditions are both $1$, in particular they are independent of $h$.
    So the dual approximations is uniformly stable by Brezzi's theorem. So, since projection commutes with divergence,
    $$||\sigma - \sigma_h||_{H(\Div, \Omega)} + ||u - u_h|| \lesssim ||(1 - \Pi) \sigma||_{H(\Div, \Omega)} + ||(1 - \Pi_0) u||.$$
    Then, by the previous problem and the Bramble-Hilbert lemma,
    $$||(1 - \Pi) \sigma||_{H(\Div, \Omega)} + ||(1 - \Pi_0) u|| \lesssim h\left(||\sigma||_{H^1(\Omega)} + ||\Div \sigma||_{H^1(\Omega)} + ||u||_{H^1(\Omega)}\right).$$
    To bound these we first observe that
    $$\nabla \times \sigma = \nabla \times \nabla u = 0,$$
    so by elliptic regularity of the div-curl system,
    $$||\sigma||_{H^1(\Omega)} \lesssim ||\Div \sigma||,$$
    and since $\Div \sigma = f$, we obtain 
    $$||\sigma||_{H^1(\Omega)} + ||\Div \sigma||_{H^1(\Omega)} \lesssim ||f||_{H^1(\Omega)}.$$
    Finally, we use the elliptic regularity of the Laplacian to bound 
    $$||u||_{H^1(\Omega)} \lesssim ||f||_{H^{-1}(\Omega)} \leq ||f||_{H^1(\Omega)}.$$


\end{document}
