
% --------------------------------------------------------------
% This is all preamble stuff that you don't have to worry about.
% Head down to where it says "Start here"
% --------------------------------------------------------------

\documentclass[10pt]{article}

\usepackage[margin=.7in]{geometry}
\usepackage{amsmath,amsthm,amssymb,mathrsfs}
\usepackage{enumitem}
\usepackage{tikz-cd}
\usepackage{mathtools}
\usepackage{amsfonts}
\usepackage{listings}
\usepackage{algorithm2e}
\usepackage{verse,stmaryrd}
\usepackage{fancyvrb}

% Number systems
\newcommand{\NN}{\mathbb{N}}
\newcommand{\ZZ}{\mathbb{Z}}
\newcommand{\QQ}{\mathbb{Q}}
\newcommand{\RR}{\mathbb{R}}
\newcommand{\CC}{\mathbb{C}}
\newcommand{\PP}{\mathbb P}
\newcommand{\FF}{\mathbb F}
\newcommand{\DD}{\mathbb D}
\renewcommand{\epsilon}{\varepsilon}

\newcommand*\dif{\mathop{}\!\mathrm{d}}

\newcommand{\card}{\operatorname{card}}
\newcommand{\dbar}{\overline \partial}
\DeclareMathOperator*{\esssup}{ess\,sup}
\newcommand{\GL}{\operatorname{GL}}
\newcommand{\Hom}{\operatorname{Hom}}
\newcommand{\id}{\operatorname{id}}
\newcommand{\Ind}{\operatorname{Ind}}
\newcommand{\Inn}{\operatorname{Inn}}
\newcommand{\interior}{\operatorname{int}}
\newcommand{\lcm}{\operatorname{lcm}}
\newcommand{\mesh}{\operatorname{mesh}}
\newcommand{\LL}{\mathcal L_0}
\newcommand{\Leb}{\mathcal{L}_{\text{loc}}^2}
\newcommand{\Lip}{\operatorname{Lip}}
\newcommand{\ppGL}{\operatorname{PGL}}
\newcommand{\ppic}{\vspace{35mm}}
\newcommand{\ppset}{\mathcal{P}}
\DeclareMathOperator{\proj}{proj}
\DeclareMathOperator*{\Res}{Res}
\newcommand{\Riem}{\mathcal{R}}
\newcommand{\RVect}{\RR\operatorname{-Vect}}
\newcommand{\Sch}{\mathcal{S}}
\newcommand{\SL}{\operatorname{SL}}
\newcommand{\sgn}{\operatorname{sgn}}
\newcommand{\spn}{\operatorname{span}}
\newcommand{\Spec}{\operatorname{Spec}}
\newcommand{\supp}{\operatorname{supp}}
\newcommand{\Torus}{\mathbb T}
\DeclareMathOperator{\tr}{tr}

\DeclareMathOperator{\adj}{adj}
\DeclareMathOperator{\curl}{curl}

% Calculus of variations
\DeclareMathOperator{\pp}{\mathbf p}
\DeclareMathOperator{\zz}{\mathbf z}
\DeclareMathOperator{\uu}{\mathbf u}
\DeclareMathOperator{\vv}{\mathbf v}
\DeclareMathOperator{\ww}{\mathbf w}

\DeclareMathOperator{\Olo}{\mathscr O}

% Categories
\newcommand{\Ab}{\mathbf{Ab}}
\newcommand{\Cat}{\mathbf{Cat}}
\newcommand{\Group}{\mathbf{Group}}
\newcommand{\Module}{\mathbf{Module}}
\newcommand{\Set}{\mathbf{Set}}
\DeclareMathOperator{\Fun}{Fun}
\DeclareMathOperator{\Iso}{Iso}

% Complex analysis
\renewcommand{\Re}{\operatorname{Re}}
\renewcommand{\Im}{\operatorname{Im}}

% Logic
\renewcommand{\iff}{\leftrightarrow}
\newcommand{\Henkin}{\operatorname{Henk}}
\newcommand{\PA}{\mathbf{PA}}
\DeclareMathOperator{\proves}{\vdash}

% Group
\DeclareMathOperator{\Gal}{Gal}
\DeclareMathOperator{\Fix}{Fix}
\DeclareMathOperator{\Out}{Out}

% Other symbols
\newcommand{\heart}{\ensuremath\heartsuit}

\DeclareMathOperator{\atanh}{atanh}

% Theorems
\theoremstyle{definition}
\newtheorem*{corollary}{Corollary}
\newtheorem*{falselemma}{Grader's ``Lemma"}
\newtheorem{exer}{Exercise}
\newtheorem{lemma}{Lemma}[exer]
\newtheorem{theorem}[lemma]{Theorem}

\usepackage[backend=bibtex,style=alphabetic,maxcitenames=50,maxnames=50]{biblatex}
\renewbibmacro{in:}{}
\DeclareFieldFormat{pages}{#1}

\begin{document}
\noindent
\large\textbf{FEM 2, HW 5} \hfill \textbf{Aidan Backus} \\
% --------------------------------------------------------------
%                         Start here
% --------------------------------------------------------------\

\begin{exer}
    Consider the biharmonic equation $\Delta^2 u = f$ with $u = \partial_n u = 0$ on $\partial \Omega$.
    Introduce $\sigma = \Delta u$ and show that the problem can be formulated in saddle-point form
\begin{align*}
    a(\sigma, \tau) + b(\tau, u) &= 0 \\
    b(\sigma, v) &= -(f, v)
\end{align*}
    for all $(\tau, v) \in H^1(\Omega) \times H^1_0(\Omega)$.
\end{exer}

Let $a(\psi, \tau) = -(\psi, \tau)$ and $b(\tau, v) = (\nabla \tau, \nabla v)$.
Morally, the first equation is equivalent to $-\sigma + \Delta u = 0$, while the second is just the usual Laplace equation $-\Delta \sigma = -f$.
However, we have to check the boundary terms from integration by parts:
$$b(\tau, u) = (\nabla \tau, \nabla u) = -(\tau, \Delta u) + \int_{\partial \Omega} \tau \partial_n u \dif S$$
and 
$$b(\sigma, v) = (\nabla \sigma, \nabla v) = -(\Delta \sigma, v) + \int_{\partial \Omega} v \partial_n \sigma \dif S = -(\Delta \sigma, v)$$
where the second boundary term drops out because $v \in H^1_0(\Omega)$.
In particular, the saddle-point problem is equivalent to
\begin{align*}
(\psi, \tau) + (\tau, \Delta u) &= \int_{\partial \Omega} \tau \partial_n u \dif S \\
(\Delta \sigma, v) &= (f, v) \\
u|_{\partial \Omega} &= 0
\end{align*}
for all $\tau \in H^1(\Omega)$ and $v \in H^1_0(\Omega)$ (where the Dirichlet condition arises from the fact that $u \in H^1_0(\Omega)$).
Now if we assume the saddle-point formulation then we can take $\tau$ supported arbitrarily close to the boundary to make the left-hand side of the first equation arbitrarily small, which means that the right-hand side must be zero, that is, $\partial_n u = 0$.
Thus if the saddle-point formulation holds, then so does the formulation 
\begin{align*}
(\psi, \tau) + (\tau, \Delta u) &= \int_{\partial \Omega} \tau \partial_n u \dif S \\
(\Delta \sigma, v) &= (f, v) \\
u|_{\partial \Omega} &= 0\\
\partial_n u|_{\partial \Omega} &= 0
\end{align*}
for all $\tau \in H^1_0(\Omega)$ and $v \in H^1_0(\Omega)$.
Clearly this formulation implies the other saddle-point formulation; it is also manifestly equivalent to the original formulation in terms of the biharmonic equation.

\begin{exer}
Show that the saddle-point problem satisfies Brezzi's theorem and is well-posed.
\end{exer}

Since $a$ is an inner product, it is elliptic everywhere, in particular on the left kernel of $b$.
Moreover $a, b$ are clearly continuous bilinear forms.
As for the Babuška-Brezzi condition, observe that given $v \in H^1_0(\Omega)$ nonzero, we can take $\tau = v$, and then 
$$b(\tau, v) = (\nabla \tau, \nabla v) \gtrsim ||\tau||_{H^1} \cdot ||v||_{H^1_0}$$
by Poincar\'e's inequality.
So by Brezzi's theorem, the saddle-point problem is well-posed.

\begin{exer}
Let $P_h$ be a triangulation of $\Omega$ and $V_h \subseteq H^1(\Omega)$, $M_h \subseteq H^1_0(\Omega)$ consist of continuous piecewise linear functions.
Show that there is a unique solution $(\sigma_h, u_h)$ of the saddle-point problem discretized in $V_h \times M_h$.
\end{exer}

Since $a$ is elliptic everywhere, this is in particular true on the left kernel of $b|_{V_h \times M_h}$.
Also the Babuška-Brezzi condition stil holds: for $v \in M_h$, $\tau = v$ is an element of $V_h$ and so we can use Poincar\'e's inequality.
Then the existence and uniqueness is immediate from Brezzi's theorem.

For the later parts of this problem set let me record that the stability constant is the maximum of $1$ (given by the ellipticity of the inner product $a$) and the reciprocal of the Poincar\'e constant of $\Omega$.
Neither of these constants have anything to do with $P_h$, so these discretizations are actually uniformly well-posed.

\begin{exer}
Show that 
$$||\sigma - \sigma_h||_{H^1} + ||u - u_h||_{H^1_0} \lesssim h||u||_{H^4}.$$
\end{exer}

By the Aziz-Babuška inequality and the fact that the discretizations are uniformly well-posed as $h \to 0$,
$$||\sigma - \sigma_h||_{H^1} + ||u - u_h||_{H^1_0} \lesssim ||\sigma - \tilde \sigma_h||_{H^1} + ||u - \tilde u_h||_{H^1_0}$$
where $\tilde \sigma_h, \tilde u_h$ are the orthogonal projections of $\sigma, u$ to $V_h, M_h$ respectively.
Since we are using piecewise linear elements,
$$||\sigma - \tilde \sigma_h||_{H^1} \lesssim h ||\sigma||_{H^2} \leq h ||u||_{H^4}$$
and 
$$||u - \tilde u_h||_{H^1_0} \lesssim h ||u||_{H^2} \leq h ||u||_{H^4}.$$

\begin{exer}
Show that one can obtain $(\sigma_h, u_h)$ by solving a system of linear equations of the form 
\begin{align*}
    A_h \sigma + B_h^t u &= 0 \\
    B_h \sigma &= -f
\end{align*}
where $A_h$ is the mass matrix of the $L^2$ inner product and $B_h$ is the stiffness matrix of the Poisson problem for $P_h$.
Explain why this scheme is more attractive in practice than the usual disretization of the biharmonic equation.
\end{exer}

Let $(\tau_i)$ and $(v_\alpha)$ be bases of $V_h$ and $M_h$ respectively. The discrete saddle-point problem is 
\begin{align*}
-(\sigma, \tau_i) + (\nabla \tau_i, \nabla u) &= 0 \\
(\nabla \sigma, \nabla v_\alpha) &= -(f, v_\alpha)
\end{align*}
for every $i, \alpha$.
Since the matrix with rows $(\cdot, \tau_i)$ is the $L^2$ inner product on $V_h$, and $-(\nabla \bullet, \nabla \tau_i)$ exactly discretizes to the $i$th column of the stiffness matrix of the Poisson problem, the first row is
$$-A_h \sigma - B_h^t u = 0.$$
By the same argument, $B_h \sigma = -f$.

This is a very convenient scheme, because every finite element package out there has a routine to generate the basis functions for the piecewise linear spaces $V_h, M_h$, and how to solve the Poisson problem on such a discretization.
The $L^2$ mass matrix $A_h$ is also pretty straightforward to generate, as one can just compute what the pairings of any two basis functions are in $L^2$.

Solving the biharmonic equation directly would instead require one to either have access to a package, or write a package, with $H^2$ conforming elements, such as piecewise cubics. Aside from being extra work, since piecewise cubics have more degrees of freedom, such a scheme would require one to invert matrices $A_h, B_h$ which are much larger.

\end{document}
