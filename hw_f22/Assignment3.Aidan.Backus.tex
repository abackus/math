
% --------------------------------------------------------------
% This is all preamble stuff that you don't have to worry about.
% Head down to where it says "Start here"
% --------------------------------------------------------------

\documentclass[10pt]{article}

\usepackage[margin=.7in]{geometry}
\usepackage{amsmath,amsthm,amssymb,mathrsfs}
\usepackage{enumitem}
\usepackage{tikz-cd}
\usepackage{mathtools}
\usepackage{amsfonts}
\usepackage{listings}
\usepackage{algorithm2e}
\usepackage{verse,stmaryrd}
\usepackage{fancyvrb}

% Number systems
\newcommand{\NN}{\mathbb{N}}
\newcommand{\ZZ}{\mathbb{Z}}
\newcommand{\QQ}{\mathbb{Q}}
\newcommand{\RR}{\mathbb{R}}
\newcommand{\CC}{\mathbb{C}}
\newcommand{\PP}{\mathbb P}
\newcommand{\FF}{\mathbb F}
\newcommand{\DD}{\mathbb D}
\renewcommand{\epsilon}{\varepsilon}

\newcommand*\dif{\mathop{}\!\mathrm{d}}

\newcommand{\card}{\operatorname{card}}
\newcommand{\dbar}{\overline \partial}
\DeclareMathOperator*{\esssup}{ess\,sup}
\newcommand{\GL}{\operatorname{GL}}
\newcommand{\Hom}{\operatorname{Hom}}
\newcommand{\id}{\operatorname{id}}
\newcommand{\Ind}{\operatorname{Ind}}
\newcommand{\Inn}{\operatorname{Inn}}
\newcommand{\interior}{\operatorname{int}}
\newcommand{\lcm}{\operatorname{lcm}}
\newcommand{\mesh}{\operatorname{mesh}}
\newcommand{\LL}{\mathcal L_0}
\newcommand{\Leb}{\mathcal{L}_{\text{loc}}^2}
\newcommand{\Lip}{\operatorname{Lip}}
\newcommand{\ppGL}{\operatorname{PGL}}
\newcommand{\ppic}{\vspace{35mm}}
\newcommand{\ppset}{\mathcal{P}}
\DeclareMathOperator{\proj}{proj}
\DeclareMathOperator*{\Res}{Res}
\newcommand{\Riem}{\mathcal{R}}
\newcommand{\RVect}{\RR\operatorname{-Vect}}
\newcommand{\Sch}{\mathcal{S}}
\newcommand{\SL}{\operatorname{SL}}
\newcommand{\sgn}{\operatorname{sgn}}
\newcommand{\spn}{\operatorname{span}}
\newcommand{\Spec}{\operatorname{Spec}}
\newcommand{\supp}{\operatorname{supp}}
\newcommand{\Torus}{\mathbb T}
\DeclareMathOperator{\tr}{tr}

\DeclareMathOperator{\adj}{adj}
\DeclareMathOperator{\curl}{curl}

% Calculus of variations
\DeclareMathOperator{\pp}{\mathbf p}
\DeclareMathOperator{\zz}{\mathbf z}
\DeclareMathOperator{\uu}{\mathbf u}
\DeclareMathOperator{\vv}{\mathbf v}
\DeclareMathOperator{\ww}{\mathbf w}

\DeclareMathOperator{\Olo}{\mathscr O}

% Categories
\newcommand{\Ab}{\mathbf{Ab}}
\newcommand{\Cat}{\mathbf{Cat}}
\newcommand{\Group}{\mathbf{Group}}
\newcommand{\Module}{\mathbf{Module}}
\newcommand{\Set}{\mathbf{Set}}
\DeclareMathOperator{\Fun}{Fun}
\DeclareMathOperator{\Iso}{Iso}

% Complex analysis
\renewcommand{\Re}{\operatorname{Re}}
\renewcommand{\Im}{\operatorname{Im}}

% Logic
\renewcommand{\iff}{\leftrightarrow}
\newcommand{\Henkin}{\operatorname{Henk}}
\newcommand{\PA}{\mathbf{PA}}
\DeclareMathOperator{\proves}{\vdash}

% Group
\DeclareMathOperator{\Gal}{Gal}
\DeclareMathOperator{\Fix}{Fix}
\DeclareMathOperator{\Out}{Out}

% Other symbols
\newcommand{\heart}{\ensuremath\heartsuit}

\DeclareMathOperator{\atanh}{atanh}

% Theorems
\theoremstyle{definition}
\newtheorem*{corollary}{Corollary}
\newtheorem*{falselemma}{Grader's ``Lemma"}
\newtheorem{exer}{Exercise}
\newtheorem{lemma}{Lemma}[exer]
\newtheorem{theorem}[lemma]{Theorem}

\usepackage[backend=bibtex,style=alphabetic,maxcitenames=50,maxnames=50]{biblatex}
\renewbibmacro{in:}{}
\DeclareFieldFormat{pages}{#1}

\begin{document}
\noindent
\large\textbf{FEM 2, HW 3} \hfill \textbf{Aidan Backus} \\
% --------------------------------------------------------------
%                         Start here
% --------------------------------------------------------------\

\begin{exer}
    Let $A \in \RR^{n \times n}$, $B \in \RR^{m \times n}$, $f \in \RR^n$, and $g \in \RR^m$, where $A$ is symmetric.
    Let
    $$J(v) := \frac{1}{2} v^tAv - f^tv.$$
    Consider the problem of minimizing $J(v)$ subject to $Bv = g$.
    By introducing a Lagrange multiplier $\lambda \in \RR^m$, show that the optimization problem is equivalent to 
    $$\begin{bmatrix}A & B^t \\ B\end{bmatrix} \begin{bmatrix}u \\ v \end{bmatrix} = \begin{bmatrix}f \\ g\end{bmatrix}.$$
\end{exer}

Assuming that $A$ is positive-semidefinite, we are searching for critical points of the restriction of $J$ to the hyperspace $\{v: Bv = g\}$.
(Otherwise solutions of that matrix equation could define saddle points of $J$!)
Thus, we want $\partial_t J(v + tw)|_{t = 0} = 0$ for any $w \in \ker B$.
$$\partial_t J(v + tw)|_{t = 0} = v^t Aw - f^t w$$
thus (using the symmetry of $A$) the weak Euler-Lagrange equation is 
$$w^t Av = f ~\forall w \in \ker B.$$
Since $(\ker B)^\perp$ is the image $V$ of $B^t$, we can rewrite this in the strong form
$$Av \cong f \text{ modulo } V.$$
Equivalently, there exists $\lambda \in \RR^m$ such that $Av + B'\lambda = f$.
This is coupled to the constraint equation $Bv = g$.
These two equations can be thus rewritten as the desired matrix equation.

\begin{exer}
    Suppose that $A$ is positive-definite.
    Show that the matrix equation is well-posed iff $BA^{-1}B^t$ is nonsingular.
\end{exer}

We rewrite the matrix as 
$$\begin{bmatrix}A & B^t \\ B\end{bmatrix} = \begin{bmatrix}A \\ & I\end{bmatrix} \begin{bmatrix}I & A^{-1}B^t \\ B\end{bmatrix}$$
where we multiplied by the diagonal of $(A, I)$, and its inverse.
Then we have the determinant formula 
$$\det \begin{bmatrix}A & B^t \\ B\end{bmatrix} = \det \begin{bmatrix}A \\ & I\end{bmatrix} \cdot \det \begin{bmatrix}I & A^{-1}B^t \\ B\end{bmatrix}.$$
The first determinant is $\det A$ which is nonzero. So the problem is well-posed iff 
$$0 \neq \det \begin{bmatrix}I & A^{-1}B^t \\ B\end{bmatrix} = \det(BA^{-1}B^t).$$


\end{document}
