
% --------------------------------------------------------------
% This is all preamble stuff that you don't have to worry about.
% Head down to where it says "Start here"
% --------------------------------------------------------------

\documentclass[10pt]{article}

\usepackage[margin=.7in]{geometry}
\usepackage{amsmath,amsthm,amssymb,mathrsfs}
\usepackage{enumitem}
\usepackage{tikz-cd}
\usepackage{mathtools}
\usepackage{amsfonts}
\usepackage{listings}
\usepackage{algorithm2e}
\usepackage{verse,stmaryrd}
\usepackage{fancyvrb}

% Number systems
\newcommand{\NN}{\mathbb{N}}
\newcommand{\ZZ}{\mathbb{Z}}
\newcommand{\QQ}{\mathbb{Q}}
\newcommand{\RR}{\mathbb{R}}
\newcommand{\CC}{\mathbb{C}}
\newcommand{\PP}{\mathbb P}
\newcommand{\FF}{\mathbb F}
\newcommand{\DD}{\mathbb D}
\renewcommand{\epsilon}{\varepsilon}

\newcommand*\dif{\mathop{}\!\mathrm{d}}

\newcommand{\card}{\operatorname{card}}
\newcommand{\dbar}{\overline \partial}
\DeclareMathOperator*{\esssup}{ess\,sup}
\newcommand{\GL}{\operatorname{GL}}
\newcommand{\Hom}{\operatorname{Hom}}
\newcommand{\id}{\operatorname{id}}
\newcommand{\Ind}{\operatorname{Ind}}
\newcommand{\Inn}{\operatorname{Inn}}
\newcommand{\interior}{\operatorname{int}}
\newcommand{\lcm}{\operatorname{lcm}}
\newcommand{\mesh}{\operatorname{mesh}}
\newcommand{\LL}{\mathcal L_0}
\newcommand{\Leb}{\mathcal{L}_{\text{loc}}^2}
\newcommand{\Lip}{\operatorname{Lip}}
\newcommand{\ppGL}{\operatorname{PGL}}
\newcommand{\ppic}{\vspace{35mm}}
\newcommand{\ppset}{\mathcal{P}}
\DeclareMathOperator{\proj}{proj}
\DeclareMathOperator*{\Res}{Res}
\newcommand{\Riem}{\mathcal{R}}
\newcommand{\RVect}{\RR\operatorname{-Vect}}
\newcommand{\Sch}{\mathcal{S}}
\newcommand{\SL}{\operatorname{SL}}
\newcommand{\sgn}{\operatorname{sgn}}
\newcommand{\spn}{\operatorname{span}}
\newcommand{\Spec}{\operatorname{Spec}}
\newcommand{\supp}{\operatorname{supp}}
\newcommand{\Torus}{\mathbb T}
\DeclareMathOperator{\tr}{tr}

\DeclareMathOperator{\adj}{adj}
\DeclareMathOperator{\curl}{curl}

% Calculus of variations
\DeclareMathOperator{\pp}{\mathbf p}
\DeclareMathOperator{\zz}{\mathbf z}
\DeclareMathOperator{\uu}{\mathbf u}
\DeclareMathOperator{\vv}{\mathbf v}
\DeclareMathOperator{\ww}{\mathbf w}

\DeclareMathOperator{\Olo}{\mathscr O}

% Categories
\newcommand{\Ab}{\mathbf{Ab}}
\newcommand{\Cat}{\mathbf{Cat}}
\newcommand{\Group}{\mathbf{Group}}
\newcommand{\Module}{\mathbf{Module}}
\newcommand{\Set}{\mathbf{Set}}
\DeclareMathOperator{\Fun}{Fun}
\DeclareMathOperator{\Iso}{Iso}

% Complex analysis
\renewcommand{\Re}{\operatorname{Re}}
\renewcommand{\Im}{\operatorname{Im}}

% Logic
\renewcommand{\iff}{\leftrightarrow}
\newcommand{\Henkin}{\operatorname{Henk}}
\newcommand{\PA}{\mathbf{PA}}
\DeclareMathOperator{\proves}{\vdash}

% Group
\DeclareMathOperator{\Gal}{Gal}
\DeclareMathOperator{\Fix}{Fix}
\DeclareMathOperator{\Out}{Out}

% Other symbols
\newcommand{\heart}{\ensuremath\heartsuit}

\DeclareMathOperator{\atanh}{atanh}

% Theorems
\theoremstyle{definition}
\newtheorem*{corollary}{Corollary}
\newtheorem*{falselemma}{Grader's ``Lemma"}
\newtheorem{exer}{Exercise}
\newtheorem{lemma}{Lemma}[exer]
\newtheorem{theorem}[lemma]{Theorem}

\usepackage[backend=bibtex,style=alphabetic,maxcitenames=50,maxnames=50]{biblatex}
\renewbibmacro{in:}{}
\DeclareFieldFormat{pages}{#1}

\begin{document}
\noindent
\large\textbf{FEM 2, HW 4} \hfill \textbf{Aidan Backus} \\
% --------------------------------------------------------------
%                         Start here
% --------------------------------------------------------------\

\begin{exer}
Consider the pure Neumann problem $-\Delta u = f$, $\partial_n u = g$.
Reformulate it in variational form
$$a(u, v) = Lv ~\forall v \in X$$
where $X = \{v \in H^1(\Omega): \int_\Omega v = 0\}$ and $L(1) = 0$.
\end{exer}

We let
$$a(u, v) := (\nabla u, \nabla v)$$
and
$$Lv := (g, v)_{L^2(\partial \Omega)} - (f, v).$$

\begin{exer}
Show that the problem can be reformulated as minimizing $J(v)$ subject to $\int v = 0$ where
$$J(v) := \frac{1}{2} a(v, v) - Lv.$$
\end{exer}

Since $J$ is a quadratic functional on $X$ its Euler-Lagrange equation is
$$a(v, w) = Lw ~\forall w \in X.$$
We also claim that $a$ is positive-definite on $X$, so $J$ is strictly convex on $X$ (so minimizers of $J$ are exactly solutions of the Euler-Lagrange equations).
In fact, if $v \in X$ then clearly $a(v, v) \geq 0$; also if $a(v, v) = 0$ then $\nabla v = 0$, so by Poincar\'e's inequality (valid on $X$ by definition of $X$), $v = 0$.

\begin{exer}
By enforcing the constraint as a Lagrange multiplier, show that the constrained minimization problem is equivalent to the saddle point problem
\begin{align*}
a(u, v) + b(v, \lambda) &= Lv ~\forall v \in H^1\\
b(u, \mu) &= 0 ~\forall \mu \in \RR
\end{align*}
where $b(v, \mu) := \mu \int_\Omega v$.
\end{exer}

Suppose that $u$ solves the constrained minimization problem. Then $b(u, \cdot) = 0$ since $u \in X$, and besides, if we write $v = v_0 + v^\perp$ where $v_0 \in X$ and $v^\perp \in X^\perp \cong \RR$, then $a(u, v^\perp) = 0$ since $v^\perp$ is constant, and $b(v_0, \lambda) = 0$ since $v_0 \in X$.
So
$$a(u, v) + b(v, \lambda) = a(u, v_0) + b(v^\perp, \lambda) = a(u, v_0) + \lambda v^\perp |\Omega|.$$
On the other hand,
$$Lv = Lv_0 + Lv^\perp = a(u, v_0) + v^\perp L(1).$$
So if we set $\lambda := L(1)/|\Omega|$ then $(u, \lambda)$ solves the saddle point problem.

\begin{exer}
Show that the saddle point problem satisfies the hypotheses of Brezzi's theorem.
Conclude that the problem is well-posed.
\end{exer}

We already argued that $a$ is positive-definite on the left kernel $X$ of $b$.
In fact we can upgrade this positive-definiteness to ellipticity: the elliptic constant is given by the Poincar\'e constant of $\Omega$.
It remains to check the Babuška-Brezzi condition, namely that for every $\mu \in \RR$ there exists $v \in H^1$ such that
$$b(v, \mu) \gtrsim |\mu| \cdot ||v||_{H^1}.$$
In fact, if we take $v := 1$, then
$$b(v, \mu) = \mu \int_\Omega 1 = \mu |\Omega| \gtrsim \mu \cdot 1.$$
So by Brezzi's theorem, the saddle point problem is well-posed.

\begin{exer}
Show that if $(u, \lambda)$ solves the saddle point problem, then $u$ solves the pure Neumann problem, and write $\lambda$ explicitly.
\end{exer}

If $u$ solves the saddle point problem, then $u$ lies in the left kernel $X$ of $b$.
In particular $\int_\Omega u = 0$.
Taking $v = 1$, we see that
$$\lambda = \frac{1}{|\Omega|} \left[\int_{\partial \Omega} g - \int_\Omega f\right].$$
Taking $v \in X$,
$$\int_{\partial \Omega} v\partial_n u - \int_\Omega v\Delta u = \int_\Omega \nabla u \cdot \nabla v = \int_{\partial \Omega} gv - \int_\Omega fv.$$
Testing against functions with zero mean is enough to determine a function pointwise, so $\partial_n u = g$ and $\Delta u = f$.


\end{document}
